%File: anonymous-submission-latex-2025.tex
\documentclass[letterpaper]{article} % DO NOT CHANGE THIS
\usepackage[submission]{aaai25}  % DO NOT CHANGE THIS
\usepackage{times}  % DO NOT CHANGE THIS
\usepackage{helvet}  % DO NOT CHANGE THIS
\usepackage{courier}  % DO NOT CHANGE THIS
\usepackage[hyphens]{url}  % DO NOT CHANGE THIS
\usepackage{graphicx} % DO NOT CHANGE THIS
\urlstyle{rm} % DO NOT CHANGE THIS
\def\UrlFont{\rm}  % DO NOT CHANGE THIS
\usepackage{natbib}  % DO NOT CHANGE THIS AND DO NOT ADD ANY OPTIONS TO IT
\usepackage{caption} % DO NOT CHANGE THIS AND DO NOT ADD ANY OPTIONS TO IT
\frenchspacing  % DO NOT CHANGE THIS
\setlength{\pdfpagewidth}{8.5in} % DO NOT CHANGE THIS
\setlength{\pdfpageheight}{11in} % DO NOT CHANGE THIS
%
% These are recommended to typeset algorithms but not required. See the subsubsection on algorithms. Remove them if you don't have algorithms in your paper.
\usepackage{algorithm}
\usepackage{algorithmic}
\usepackage{amsmath}
\usepackage{booktabs}
\usepackage{multicol}
\usepackage{lipsum}
\usepackage{listings}
\DeclareCaptionStyle{ruled}{labelfont=normalfont,labelsep=colon,strut=off}
\lstset{%
    basicstyle={\footnotesize\ttfamily},% footnotesize acceptable for monospace
	numbers=left,numberstyle=\footnotesize,xleftmargin=2em,% show line numbers, remove this entire line if you don't want the numbers.
    aboveskip=0pt,
    belowskip=0pt,
    showstringspaces=false,
    tabsize=2,
    breaklines=true}
\usepackage{xcolor}

% Resolve problems with .eps figures
\usepackage{graphicx}
\usepackage[outdir=./]{epstopdf}
\graphicspath{{./figures/}} %Where the figures folder is located

% Include check and cross marks for qualitative comparison table
\usepackage{pifont}% http://ctan.org/pkg/pifont
\newcommand{\cmark}{\ding{51}} % Checkmark
\newcommand{\xmark}{\ding{55}} % X-mark
\newcommand{\xmarkred}{{\color{magenta}\xmark}} % Red colored X-mark

% Maths
\newcommand{\cdotx}{\,\cdot\,} % Function argument
\renewcommand{\S}{\boldsymbol{\mathcal{S}}} % System
\newcommand{\T}{^\top}

% Tool names
\newcommand{\deepsplit}{\textsc{DEEPSPLIT}}
\newcommand{\fossil}{\textsc{Fossil}}
\newcommand{\lucid}{\textsc{Lucid}}
\newcommand{\npinterval}{\textsc{npinterval}}
\newcommand{\reluplex}{\textsc{Reluplex}}
\newcommand{\trust}{\textsc{TRUST}}

% Styling
\input{iPython}

% Other
\newcommand{\todo}[1]{\textcolor{purple}{\textbf{TODO}: \textit{#1}}}
\newcommand{\new}[1]{#1}
\newcommand{\OS}[1]{{\color{blue}[Oliver]: #1}}

%
% These are are recommended to typeset listings but not required. See the subsubsection on listing. Remove this block if you don't have listings in your paper.
\usepackage{newfloat}
\usepackage{listings}
\DeclareCaptionStyle{ruled}{labelfont=normalfont,labelsep=colon,strut=off} % DO NOT CHANGE THIS
\lstset{%
	basicstyle={\footnotesize\ttfamily},% footnotesize acceptable for monospace
	numbers=left,numberstyle=\footnotesize,xleftmargin=2em,% show line numbers, remove this entire line if you don't want the numbers.
	aboveskip=0pt,belowskip=0pt,%
	showstringspaces=false,tabsize=2,breaklines=true}
\floatstyle{ruled}
\newfloat{listing}{tb}{lst}{}
\floatname{listing}{Listing}
%
% Keep the \pdfinfo as shown here. There's no need
% for you to add the /Title and /Author tags.
\pdfinfo{
/TemplateVersion (2025.1)
}

% DISALLOWED PACKAGES
% \usepackage{authblk} -- This package is specifically forbidden
% \usepackage{balance} -- This package is specifically forbidden
% \usepackage{color (if used in text)
% \usepackage{CJK} -- This package is specifically forbidden
% \usepackage{float} -- This package is specifically forbidden
% \usepackage{flushend} -- This package is specifically forbidden
% \usepackage{fontenc} -- This package is specifically forbidden
% \usepackage{fullpage} -- This package is specifically forbidden
% \usepackage{geometry} -- This package is specifically forbidden
% \usepackage{grffile} -- This package is specifically forbidden
% \usepackage{hyperref} -- This package is specifically forbidden
% \usepackage{navigator} -- This package is specifically forbidden
% (or any other package that embeds links such as navigator or hyperref)
% \indentfirst} -- This package is specifically forbidden
% \layout} -- This package is specifically forbidden
% \multicol} -- This package is specifically forbidden
% \nameref} -- This package is specifically forbidden
% \usepackage{savetrees} -- This package is specifically forbidden
% \usepackage{setspace} -- This package is specifically forbidden
% \usepackage{stfloats} -- This package is specifically forbidden
% \usepackage{tabu} -- This package is specifically forbidden
% \usepackage{titlesec} -- This package is specifically forbidden
% \usepackage{tocbibind} -- This package is specifically forbidden
% \usepackage{ulem} -- This package is specifically forbidden
% \usepackage{wrapfig} -- This package is specifically forbidden
% DISALLOWED COMMANDS
% \nocopyright -- Your paper will not be published if you use this command
% \addtolength -- This command may not be used
% \balance -- This command may not be used
% \baselinestretch -- Your paper will not be published if you use this command
% \clearpage -- No page breaks of any kind may be used for the final version of your paper
% \columnsep -- This command may not be used
% \newpage -- No page breaks of any kind may be used for the final version of your paper
% \pagebreak -- No page breaks of any kind may be used for the final version of your paperr
% \pagestyle -- This command may not be used
% \tiny -- This is not an acceptable font size.
% \vspace{- -- No negative value may be used in proximity of a caption, figure, table, section, subsection, subsubsection, or reference
% \vskip{- -- No negative value may be used to alter spacing above or below a caption, figure, table, section, subsection, subsubsection, or reference

\setcounter{secnumdepth}{0} %May be changed to 1 or 2 if section numbers are desired.

% The file aaai25.sty is the style file for AAAI Press
% proceedings, working notes, and technical reports.
%

% Title

% Your title must be in mixed case, not sentence case.
% That means all verbs (including short verbs like be, is, using,and go),
% nouns, adverbs, adjectives should be capitalized, including both words in hyphenated terms, while
% articles, conjunctions, and prepositions are lower case unless they
% directly follow a colon or long dash
\title{Lucid -- Learning-enabled Uncertainty-aware Certification of Invariant Dynamics}
% \author {
%     % Authors
    % Ernesto Casablanca\textsuperscript{\rm 1},
    % Oliver Sch\"on\textsuperscript{\rm 1}%,
    % Third Author Name\textsuperscript{\rm 2}
% }
% \affiliations {
    % Affiliations
    % \textsuperscript{\rm 1}Newcastle University, United Kingdom\\
    % \textsuperscript{\rm 2}Affiliation 2\\
    % e.casablanca2@ncl.ac.uk, o.schoen2@ncl.ac.uk%, thirdAuthor@affiliation1.com
% }


% REMOVE THIS: bibentry
% This is only needed to show inline citations in the guidelines document. You should not need it and can safely delete it.
\usepackage{bibentry}
% END REMOVE bibentry

\begin{document}

\maketitle

\begin{abstract}
Ensuring the safety of AI-enabled systems, particularly in high-stakes domains like autonomous driving and healthcare, has become increasingly critical.
Traditional formal verification tools fall short when faced with the opaque, black-box nature of AI components and the sheer scale of modern applications.
To address these challenges, we introduce (Learning-enabled Uncertainty-aware Certification of Invariant Dynamics), a verification engine for certifying safety of black-box stochastic dynamical systems from a finite dataset of random state transitions.

Lucid employs a data-driven methodology rooted in control barrier certificates, which are learned directly from system trajectory data to ensure formal safety guarantees.
We use conditional mean embeddings to embed data into a reproducing kernel Hilbert space (RKHS), where it constructs an RKHS ambiguity set that can be inflated to robustify the result to out-of-distribution behavior.
To generalize beyond safety, we provide theoretical foundations for applying Lucid to broader classes of temporal logic specifications.

A key innovation within Lucid is its use of a finite Fourier expansion to reformulate a semi-infinite non-linear optimization problem into a tractable linear program.
The resulting spectral barrier allows us to leverage the fast Fourier transform to generate the relaxed problem efficiently, offering a scalable yet distributionally robust framework for verifying safety.
Lucid thus offers a robust and efficient verification framework, able to handle the complexities of modern black-box systems while providing formal guarantees of safety.

\todo{Double-check}
\end{abstract}

% Uncomment the following to link to your code, datasets, an extended version or similar.
%
\begin{links}
    \link{Anonymized Code}{https://google.com}
    % \link{Code}{https://aaai.org/example/code}
    % \link{Datasets}{https://aaai.org/example/datasets}
    % \link{Extended version}{https://aaai.org/example/extended-version}
\end{links}
%%%%%%%%%%%%%%%%%%%%%%%%%%%%%%%%%%%%%%%%%%%%
%%%%%%%%%%%%%%%%%%%%%%%%%%%%%%%%%%%%%%%%%%%%
%%%%%%%%%%%%%%%%%%%%%%%%%%%%%%%%%%%%%%%%%%%%

\section{Introduction}
\todo{Motivation and background ...}

Trustworthy embodied AI: AI agency in physical world

\lipsum[1-6]


\begin{table}
    \centering
    \begin{tabular}{cccccc}
        \toprule
        \textbf{Tool} & \multicolumn{5}{c}{\textbf{Supported Features}} \\[.4em]
          & \rotatebox{90}{Data driven} & \rotatebox{90}{Stochastic D.} & \rotatebox{90}{Non-Poly. D.} & \rotatebox{90}{Stat. Correct.} & \rotatebox{90}{Closed Loop} \\
        \midrule
        \textbf{\lucid} & \cmark & \cmark & \cmark & ? & \cmark \\
        \trust~\shortcite{gardner2025trust} & \cmark & \xmarkred & \xmarkred & \cmark\footnotemark & \cmark \\
        \fossil~\shortcite{edwards2024fossil} & \xmarkred & \xmarkred & \cmark & NA & \cmark \\
        \npinterval~\shortcite{harapanahalli2023forward} & \xmarkred & \cmark/\xmarkred\footnotemark & ? & NA & \cmark \\
        % \deepsplit~\shortcite{henriksen2021deepsplit} & \xmarkred & \xmarkred & \cmark & \xmarkred& \xmarkred \\
        % \reluplex~\shortcite{katz2017reluplex} & \xmarkred & \xmarkred & \cmark & \xmarkred & \xmarkred \\
        tool3 & & & & \\
        tool4 & & & & \\
        \bottomrule
    \end{tabular}
    \caption{Qualitative comparison with existing tools based on their supported features: data driven, stochastic dynamics, non-polynomial dynamics, and statistical correctness guarantees of the learned model (only applicable if data driven).}
    \label{tab:qualitative_comparison}
\end{table}
\footnotetext{Assuming the data satisfies persistence of excitation.} % TRUST
\footnotetext{Accepts non-deterministic bounded disturbances.} % npinterval

\paragraph{Related Work (optionally)}
Different categories for verification in annual friendly competitions~\cite{abate2024arch}.
There exist a lot of tools for verifying the input--output behavior of neural networks~\cite{liu2021algorithms}.
For closed-loop systems, ... [see literature in \cite{harapanahalli2023forward}]
Black-box verification of stochastic dynamical systems very sparsely populated.
Comparison with the most related existing tools in Table~\ref{tab:qualitative_comparison}: Focus on data-driven tools, with few exceptions if noteworthy because very related.
Result: Few data-driven tools; No data-driven tools for stochastic systems.

What are the working principles used.
No other kernel-based tools?
Not tools: What works have achieved related things.
Sampling-based reachability analysis~\cite{lew2021sampling}.

\lipsum[1]

\paragraph{Organization}
\lipsum[2]

\section{Theoretical Working Principles}\label{sec:theory}
\subsection{Safety of Black-Box Dynamical Systems}
Due to their complexity and opacity, many such systems can be cast as black-box systems with Markovian discrete-time stochastic dynamics of the form
\begin{equation}
  \label{eq:model}
  \S\colon\left\{\begin{array}{ll}
    x_{t+1} & = f(x_t,a_t,w_t),\quad w_t\sim p_w,
  \end{array} \right.
\end{equation}
with state $x_t$, action $a_t$, and independent and identically distributed (i.i.d.) noise $w_t\sim p_w$.
\begin{figure}
    \centering
    \includegraphics[width=\linewidth]{placeholder.jpeg}
    \caption{Illustrative safety problem: Neural-network controlled robotic system.}
    % \label{fig:enter-label}
\end{figure}

\subsection{Control Barrier Certificates}


\subsection{Conditional Mean Embeddings}


\subsection{Data-Driven Spectral Barriers}
\begin{figure}
    \centering
    \includegraphics[width=\linewidth]{fourier_series.pdf}
    \caption{Spectral barrier certificate $B=b\T\phi(x)$}
    % \label{fig:enter-label}
\end{figure}


\section{Tool Use and Functionality}
Simple use case:\OS{Insert linear example here}
\begin{equation*}
  \S\colon\left\{\begin{array}{ll}
    x_{t+1} & = ...,\quad w_t\sim p_w,
  \end{array} \right.
\end{equation*}

\subsection{Command Line Interface}
Works based on following YAML configuration:
\begin{lstlisting}[language=iPython]
<...>
\end{lstlisting}
Thus can run quick example as such:
\begin{lstlisting}[language=iPython]
<...>
\end{lstlisting}

\subsection{Graphical User Interface}
\begin{figure}
    \centering
    \includegraphics[width=\linewidth]{placeholder.jpeg}
    \caption{Graphical user interface}
    % \label{fig:enter-label}
\end{figure}


\section{Tool Structure and Components}

\lucid is implemented in C++ and it is meant to be used as a library.
It provides a set of interfaces that allow users to have an high level of control and flexibility over the verification process.
We also provide a Python wrapper, called PyLucid, to facilitate the integration of the tool into existing workflows and effortlessly leverage well-established libraries such as NumPy~\cite{numpy} and SciPy~\cite{scipy}.
% \lucid is available as an open-source project on GitHub at \url{https://github.com/TendTo/lucid}.
More technical details about the implementation can be found in the online documentation at \url{https://tendto.github.io/lucid/}, but at an high level, the architecture of \lucid is illustrated in~\ref{fig:architecture}.

Map the theoretical modules from Section~\ref{sec:theory} to components/modules of the code (see Figure~\ref{fig:lucid_architecture}).
\begin{figure}
    \centering
    \includegraphics[width=\linewidth]{placeholder.jpeg}
    \caption{Architecture of \lucid: ...}
    \label{fig:lucid_architecture}
\end{figure}


\section{Experimental Evaluation}
Overview of the experimental results in Table~\ref{tbl:benchmarks_results}.
\begin{table}[tb]
    \centering
    \begin{tabular}{ccccc}
        \toprule
        \textbf{Benchmark} & \textbf{Dim} & \textbf{\#LatP} & \textbf{Runtime} & \textbf{Safety Prob.} \\
         & & & [mm:ss] & [\%] \\ % Units
        \midrule
        Linear & 1 & & X:XX & XX \\
        Barr3 & 2 & $30^2$ & X:XX & XX \\
         & & $40^2$ & X:XX & XX \\
        Overtaking & 3 & & X:XX & XX \\
        name2 & 7 & & X:XX & XX \\
        \bottomrule
    \end{tabular}
    \caption{Computational benchmarks: ...}
    \label{tbl:benchmarks_results}
\end{table}

\begin{figure}
    \centering
    \includegraphics[width=\linewidth]{placeholder.jpeg}
    \caption{Barrier}
    \label{fig:barr3_barrier}
\end{figure}

\section{Conclusion}


%%%%%%%%%%%%%%%%%%%%%%%%%%%%%%%%%%%%%%%%%%%%
%%%%%%%%%%%%%%%%%%%%%%%%%%%%%%%%%%%%%%%%%%%%
%%%%%%%%%%%%%%%%%%%%%%%%%%%%%%%%%%%%%%%%%%%%
\bibliography{aaai25}

\end{document}
