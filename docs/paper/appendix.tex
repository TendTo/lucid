% !TEX root =  main.tex
\appendix

% \section{Additional Theoretical Details}
% \paragraph{RKHS basics.}
% A symmetric function $k_\X:\X\times\X\rightarrow\R$ is called a (positive definite) \emph{kernel} (note the distinction from \emph{probability kernels}) if for all $N\in\N_{>0}$ we have $\sum_{i=1}^{N}\sum_{j=1}^{N}a_i a_j\allowbreak k_\X(x_i,x_j) \geq 0$ for $x_1,\ldots,x_N\in\X\subset{\R^n}$ and $a_1,\ldots,a_N\in\R$.
% A prominent example is the \emph{squared exponential} (SQExp) kernel \shortcite{Rasmussen2005GP,Kanagawa2018GPvsKernel}:
% \begin{equation}
%     k_\X(x,x') := \sigma_f^2 \exp\left( -\frac{1}{2} (x-x')\T \Sigma\, (x-x') \right),\quad \Sigma:=\mathrm{diag}(\sigma_l)^{-2},\label{eq:sqexp_kernel}
% \end{equation}
% with amplitude $\sigma_f^2\geq0$ and lengthscale coefficients $\sigma_l\in\R^n$.
% \new{For this work,} we assume that all kernels are bounded on their domain, i.e., $\E_{}[k_\X(x,x)]<\infty$, $x\in\X$.
% Given a kernel $k_\X$ on a non-empty set $\X$, there exists a unique corresponding
% \emph{reproducing kernel Hilbert space} (RKHS) $\Hilbert_{k_\X}$
% of functions $f:\X\rightarrow\R$ equipped with an inner product $\innerH{\cdotx}{\cdotx}{\Hilbert_{k_\X}}$
% with the celebrated \emph{reproducing property} such that for any function $f\in\Hilbert_{k_\X}$ and $x\in\X$ we have $f(x)=\innerH{f}{k_\X(\cdotx,x)}{\Hilbert_{k_\X}}$.
% Note that $k_\X(\cdotx,x):\X\rightarrow\Hilbert_{k_\X}$ is a real-valued function,
% which is also called an implicit \emph{canonical} \emph{embedding} or \emph{feature map} $\phi_\X$
% such that $k_\X(x,x')=\innerH{\phi_\X(x)}{\phi_\X(x')}{\Hilbert_{k_\X}}$ for all $x,x'\in\X$.
% For an RKHS $\Hilbert_{k_\X}$, we use the associated feature map $\phi_\X$ and kernel $k_\X$ interchangeably for ease of notation and comprehensibility.
% The inner product induces the norm $\norm{f}_{\Hilbert_{k_\X}}\!\!\!\!:=\!\!\sqrt{\smash[b]{\innerH{f}{f}{\Hilbert_{k_\X}}}}$ of the RKHS. 
% Throughout this paper, we assume that all RKHSs are \emph{separable}.
% Refer to the monograph by \citet{Berlinet2004RKHSProbStat} for a comprehensive study on RKHSs.
% Given $N$ i.i.d. samples $\hat X_N:=[\hat{x}_i]_{i=1}^N$ with $\hat{x}_i\in\X$, the
% \emph{Gram matrix} of $k_\X$ is given by
% $\new{K_{\hat{X}}^N}:=[k_\X(\hat{x}_i,\hat{x}_j)]_{i,j=1}^N.$
% Furthermore, we define the vector-valued function 
% $\new{k_{\hat{X}}^N}(x)  := [k_\X(x,\hat{x}_i)]_{i=1}^N.$

% \begin{definition}[Conditional Mean Embedding (CME)]\label{def:condMeanEmbed}
%     Given two RKHSs $\Hilbert_{k_\X}$ and $\Hilbert_{k_\Y}$ with the associated kernels $k_\X\colon\X\times\X\to\R$ and $k_\Y\colon\X\times\X\to\R$, the \emph{CME} of a probability kernel $\Tr\colon\X\times\borel{\Y}\rightarrow[0,1]$ is an $X$-measurable random variable taking values in $\Hilbert_{k_\Y}$, given by
%     \begin{equation*}
%         \cme_{k_\Y|k_\X}(\Tr)(\cdotx) := \E_{\Tr}[\phi_\Y(Y)\mid X=\cdotx].
%     \end{equation*}
% \end{definition}

\section{Linear Program}
The finitely-constrained LP discussed in Section~\ref{sec:ddbarriers} is presented. For this, the lattices $X_{{N}}\subset\X$, $\smash{\{ x_0^{1}, \ldots, x_0^{{N}_0} \}} \subset \X_0$, and $\smash{\{ x_u^{1}, \ldots, x_u^{{N}_u} \}} \subset \X_u$ of cardinality ${N}_0\in\N$ and ${N}_u\in\N$, respectively, are formed.
%
For given values of $\overline{\B}$, $\gamma$, and robustness radius $\mathcal{R}\geq0$, the following LP is obtained:
\begin{equation*}
      \begin{alignedat}{3}
                  & \min_{\stackrel{b, c, \eta}{\Bmin_{{N}}^{\X_0}, \Bmax_{{N}}^{\X_u},\Bmax_{{N}}^\X,\Bmin_\Delta}}\hspace{-1.5em} &                                          & \eta + cT,                                    &                   &
            %\label{eq:blackbox:linear_prog_objective}
            \\
                  & \text{subject to}
                  &                                                                                                                 & \phi_M(x_0^{i})\T b\leq\hat{\eta}, \quad &                                               & i=1,\ldots,{N}_0,
            %\label{eq:blackbox:linear_prog_initial}
            \\
                  &                                                                                                                 &                                          & \phi_M(x_u^{i})\T b\geq\hat{\gamma},
            \quad &                                                                                                                 & i=1,\ldots,{N}_u,
            %\label{eq:blackbox:linear_prog_unsafe}
            \\
                  &                                                                                                                 &                                          & \phi_M(x^{i})\T(Hb - b) \leq \hat{\Delta},
            \quad &                                                                                                                 & i=1,\ldots,{N},
            %\label{eq:blackbox:linear_prog_kushner}
            \\
                  &                                                                                                                 &                                          & \phi_M(x^{i})\T b\geq \hat{\xi},
            \quad &                                                                                                                 & i=1,\ldots,{N},
            %\label{eq:blackbox:linear_prog_positive}
            \\
                  &                                                                                                                 &                                          & \Bmin_{{N}}^{\X_0}\leq\phi_M(x_0^{i})\T b,
            \quad &                                                                                                                 & i=1,\ldots,{N}_0,                                                                                              \\
                  &                                                                                                                 &                                          & \Bmax_{{N}}^{\X_u}\geq\phi_M(x_u^{i})\T b,
            \quad &                                                                                                                 & i=1,\ldots,{N}_u,                                                                                              \\
                  &                                                                                                                 &                                          & \Bmax_{{N}}^\X\geq\phi_M(x^{i})\T b,
            \quad &                                                                                                                 & i=1,\ldots,{N},                                                                                                \\
                  &                                                                                                                 &                                          & \Bmin_\Delta\leq\phi_M(x^{i})\T (Hb - b),
            \quad &                                                                                                                 & i=1,\ldots,{N},                                                                                                \\
                  &                                                                                                                 &                                          & c\geq 0,\,\gamma>\eta\geq 0,\, b\in\R^{2M+1}, &                   &
            \label{eq:blackbox:linear_prog}%
      \end{alignedat}
\end{equation*}
with $\overline{\kappa}\geq\sigma_f$, $\overline{\B}\geq\norm{b}_2$, and constraint-tightening coefficients
\begin{align*}
      \hat{\eta}   & := \frac{2}{C_{{N}}+1}\eta + \frac{C_{{N}}-1}{C_{{N}}+1}\Bmin_{{N}}^{\X_0},
      \\                                                                                                                                    & \hat{\gamma} := \frac{2}{C_{{N}}+1}\gamma + \frac{C_{{N}}-1}{C_{{N}}+1}\Bmax_{{N}}^{\X_u}, \\
      \hat{\Delta} & := \frac{2}{C_{{N}}+1}\left(c - \mathcal{R}\overline{\B}\overline{\kappa}\right) + \frac{C_{{N}}-1}{C_{{N}}+1}\Bmin_\Delta,
      \\                                                                                                                                     & \hat{\xi} := \frac{C_{{N}}-1}{C_{{N}}+1}\Bmax_{{N}}^\X.
\end{align*}


% \section{Proofs}
% We have collected a series of proofs of the claims we made in the main text.

\section{Tool technical details}

\subsection{Installation}
\OS{Ernesto, can we organise this in a better order?}
The recommended way to install \pylucid is via the \texttt{pip} package manager, using one of the pre-built wheels,
hence avoiding the time-consuming need to compile the code from source.
Provided \texttt{Python>=3.8} is already present, installing \pylucid requires only the command
\begin{lstlisting}[language=bash,numbers=none,xleftmargin=0em]
pip install "pylucid[gui,plot]" --index-url \
  https://gitlab.com/api/v4/projects/71977529/packages/pypi/simple
\end{lstlisting}
If the desired platform's wheel is not available, \pylucid can be installed from source running the following commands:
\begin{lstlisting}[language=bash,numbers=none,xleftmargin=0em]
git clone https://gitlab.com/lucidtoolsource/lucid.git
cd lucid
pip install .[gui,plot]
\end{lstlisting}
Note that this requires \texttt{Python>=3.8} and \texttt{Bazel>=8.0} to be installed on the system.
For more details on the installation process, please refer to the online documentation at \url{https://lucidtoolsource.gitlab.io/lucid/md_docs_2Pylucid.html}.

\subsection{Platform support}
Some dependencies \pylucid includes do not support all platforms, which may preclude the ability of using certain features.
Formal verification of the barrier requires the \dreal SMT solver, which can be installed with \lstinline|pip install dreal|, but only supports Linux and non-ARM macOS.
The \highs \gls{lp} solver cannot be used on Windows at the moment.
The \gurobi solver must be installed separately beforehand and requires a licence.

\subsection{Configuration formats}
\pylucid provides extensive support for configuration, which can be specified as a \texttt{.yaml} file (recommended), \texttt{.json} file,
with a \texttt{.py} file returning a \texttt{Configuration} object, or directly via command line arguments.
Moreover, sampled transitions can be embedded in the configuration file, or be provided separately with \texttt{.csv}, \texttt{.mat}, \texttt{.npy}, or \texttt{.npz} file.
For example, following configurations will be equivalent:
\lstinputlisting[language=yaml,caption={Yaml configuration.},captionpos=b,label={lst:configuration.yaml}]{code/configuration.yaml}
\lstinputlisting[language=json,caption={Json configuration.},captionpos=b,label={lst:configuration.json}]{code/configuration.json}
\lstinputlisting[language=iPython,style={nonumbers},caption={Python configuration.},captionpos=b,label={lst:configuration.py}]{code/configuration.py}
\lstinputlisting[language=bash,style={nonumbers},caption={Command line configuration. Assumes that the transition samples have been stored as \texttt{.csv} files.},captionpos=b,label={lst:configuration.sh}]{code/configuration.sh}

\subsection{Graphical User Interface}

\pylucid provides a \gls{gui} to guide the user in the creation of the scenario configuration and presenting the results in an intuitive way.
Running \lstinline|pylucid-gui| will open a browser tab containing the interface shown in Figures~\ref{fig:gui-left}--\ref{fig:gui-right}, while a local server will listen for requests coming from the \gls{gui}, computing and returning the results.
Following the same numeration as the arrows in the figures, we provide a list of the main components of the \gls{gui}:
\begin{enumerate}
      \item Only a subsection of the configuration options is shown at any given moment. More options can be accessed by clicking on the \textit{Advanced} button.
      \item It is easy to import or export the configuration using the Json format. The experiments discussed in this paper are also available in the \textit{Import} menu.
      \item Clicking the \textit{Preview} button produces a visualization of the system function described by the transition samples.
      \item The dimension of the system can be specified in the \textit{Model} tab.
      \item You can either provide the transition samples yourself in the \textit{Data} tab, or specify a transition function in the \textit{Model} tab.
      \item The $x^1,\ldots,x^N$ samples, stored in \texttt{csv} format.
      \item The $x^1_+,\ldots,x^N_+$ samples, stored in \texttt{csv} format.
      \item The bounding box of the state space.
      \item The initial set $\X_0$.
      \item The unsafe set $\X_U$.
      \item Clicking the \textit{Submit} button will start the computation of the barrier function with the current configuration.
      \item Plot of the transition function or the computed barrier function.
      \item Numerical results of the computation.
      \item Logs of the computation.
\end{enumerate}
\begin{figure}
      \centering
      \includegraphics[trim={0 0 17cm 0},clip,width=.7\linewidth]{figures/frontend-left.png}
      \caption{The left half of the page allows the user to provide the transition samples and the safety specification to verify.
      }
      \label{fig:gui-left}
\end{figure}
\begin{figure}
      \centering
      \includegraphics[trim={13.5cm 12.4cm 0 0},clip,width=.7\linewidth]{figures/frontend-right.png}
      \caption{The right half of the page shows the results produced by \lucid, including a plot of the computed barrier function and detailed logs.}
      \label{fig:gui-right}
\end{figure}

\section{Benchmarks}
\label{app:benchmarks}
Complete list of results for the benchmarks presented in Section~\ref{sec:experiments}.
The legend is as follows:
\textit{Output} indicates the result produced by the solver (i.e., optimal, unbounded, infeasible or unspecified),
\textit{Prec} indicates the number of bits used in the last floating-point number representation,
\textit{Ref} indicates the number of refinements,
and \textit{time} is the time in seconds.

All benchmarks were run on a Windows 10 machine with an AMD Ryzen 9 5950X 16-Core Processor @ 3.40 GHz, and 64 GB of RAM, and all runs had the random seed set to $42$ to ensure reproducibility.
The scripts responsible for running the benchmarks and \hp tuning are available in the \texttt{benchmarks/integration} folder of the repository.
Note that a version of \texttt{Python>=3.9} is required to run the scripts, as some dependencies used to track the benchmarks' metrics across multiple runs are not compatible with \texttt{Python 3.8}.
The \hp tuning of the \estimator was performed with the \texttt{LbfgsTuner}, bounding the value of $\sigma_l$ between $[10^{-5}, 10^5]$, and refined with the \texttt{GridSearchTuner}.
The implementation can be found in the \texttt{benchmarks/integration/hp\_tuning.py} script.
Note that we used the \gurobi optimiser.
Other optimizers, such as \alglib or \highs, can be used instead, but may yield different results, especially in terms of performance.

\subsection{Linear}

We consider the following system:
\begin{equation*}
      \begin{bmatrix}
            {x}_{t+1}
      \end{bmatrix}
      = \begin{bmatrix}
            0.5 {x}_{t}
      \end{bmatrix} + w_t,
\end{equation*}
where $w_t\sim\mathcal{N}(\cdotx\vert 0,0.01I_1)$.
Given
\begin{align*}
       & \X = [-1, 1]                     \\
       & \X_0 = [-0.5, 0.5]               \\
       & \X_U = [-1, -0.9] \cup [0.9, 1],
\end{align*}
we want to ensure that the system, starting in $\X_0$, does not enter the unsafe regions $\X_U$ within $T=15$ time steps.
The complete configuration for the linear example benchmark is shown in Listing~\ref{lst:linear}.
\lstinputlisting[language=yaml,caption={Configuration for linear example},captionpos=b,label={lst:linear}]{code/linear.yaml}

\subsection{Barrier 2}

We consider the following system:
\begin{equation*}
      \begin{bmatrix}
            {x}_{1, t+1} \\
            {x}_{2, t+1}
      \end{bmatrix}
      = \begin{bmatrix}
            {x}_{2, t} - 1 + e^{-x_{1, t}} \\
            -\sin^2(x_{1, t})
      \end{bmatrix} + w_t,
\end{equation*}
where $w_t\sim\mathcal{N}(\cdotx\vert 0,0.01I_2)$.
Given
\begin{align*}
       & \X = [ -2, 2 ] \times [ -2 , 2 ]                                  \\
       & \X_0 = \{ [x_1, x_2] : (x_1 + 0.5)^2 + (x_2 - 0.5)^2 \leq 0.4 \}  \\
       & \X_U = \{ [x_1, x_2] : (x_1 - 0.7)^2 + (x_2 + 0.7)^2 \leq 0.3 \},
\end{align*}
we want to ensure that the system, starting in $\X_0$, does not enter the unsafe regions $\X_U$ within $T=5$ time steps.
The kernel \hp set to be $\sigma_f=1$ and $\sigma_l=[1.0, 5.39334]$.
Using a spectral basis of $M=15^2-1$ frequencies and a lattice density of $\hat{N} = 350^2$,
we were able to generate a \gls{cbc} with $\eta \approx 5.032$, $\gamma = 100$, and $c \approx 3.926$ in $14.7$ minutes, achieving a lower bound on the safety probability of $P_{\text{safe}}(\S^\pi) \geq 75.34\%$.
The result is shown in Figure~\ref{fig:CSBarr2}.
The complete configuration for the \barrII benchmark is shown in Listing~\ref{lst:barrier2}.
\lstinputlisting[language=yaml,caption={Configuration for \barrII},captionpos=b,label={lst:barrier2}]{code/barrier2.yaml}

\subsection{Barrier 3}

We consider the following system:
\begin{equation*}
      \begin{bmatrix}
            {x}_{1, t+1} \\
            {x}_{2, t+1}
      \end{bmatrix}
      = \begin{bmatrix}
            {x}_{2, t} \\
            \frac{1}{3} {x}^3_{1, t} - {x}_{1,t} - {x}_{2,t}
      \end{bmatrix} + w_t,
\end{equation*}
where $w_t\sim\mathcal{N}(\cdotx\vert 0,0.01I_2)$.
Given
\begin{align*}
       & \X = [ -3, 2.5 ] \times [ -2 , 1 ]                                                 \\
       & \X_0 = [ 1 , 2 ] \times [ -0.7 , 0.3 ] \cup [ -1.8 , -1.4 ] \times [ -0.1 , 0.1 ]  \\
       & \qquad \cup [-1.4, -1.2] \times [-0.5 , 0.1]                                       \\
       & \X_U = [ 0.4 , 0.6 ] \times [ 0.2 , 0.6 ] \cup [ 0.6 , 0.7 ] \times [ 0.2 , 0.4 ],
\end{align*}
we want to ensure that the system, starting in $\X_0$, does not enter the unsafe regions $\X_U$ within $T=5$ time steps.
The kernel \hp were set to $\sigma_f=1$ and $\sigma_l=[1.0, 1.0, 1.0]$.
Hence, we need a rather high number of frequencies to achieve the required expressivity in the barrier.
Using a spectral basis of $M=15^3-1$ frequencies and a lattice density of $\hat{N} = 30^2$,
we were able to generate a \gls{cbc} with $\eta \approx 0.1$, $\gamma = 100$, and $c \approx 0.1$ in $XX:XX$ minutes, achieving a lower bound on the safety probability of $P_{\text{safe}}(\S^\pi) \geq XX\%$.
The result is shown in Figure~\ref{fig:CSBarr3}.
The complete configuration for the \barrIII benchmark is shown in Listing~\ref{lst:barrier3}.
\lstinputlisting[language=yaml,caption={Configuration for \barrIII},captionpos=b,label={lst:barrier3}]{code/barrier3.yaml}

\subsection{Overtaking}

We consider a scenario where an autonomous vehicle (AV) controlled by a \gls{nn} is overtaking another vehicle.
The dynamics of the ego vehicle are given by Dubin's car model by adding the noise vector $w = \begin{bmatrix}w_t^1 & w_t^2 & w_t^3\end{bmatrix}$ where each component is drawn from a zero-mean Gaussian with standard deviation $0.01$, $0.01$, and $0.001$, respectively.
The steering wheel angle is supplied by the \gls{nn} controller and we travel at a fixed velocity.
Given
\begin{align*}
       & \X = [ 1, 90 ] \times [ -7, 19 ] \times [ -\pi, \pi ]         \\
       & \X_0 = [ 1, 3 ] \times [ -1, 1 ] \times [ -0.5, 0.5 ]         \\
       & \X_U = [ 1, 90 ] \times [ -7, -6 ] \times [ -\pi, \pi ]       \\
       & \qquad \cup [ 1, 90 ] \times [ 18, 19 ] \times [ -\pi, \pi ]  \\
       & \qquad \cup [ 40, 45 ] \times [ -6, 6 ] \times [ -\pi, \pi ],
\end{align*}
we want to ensure that the system, starting in $\X_0$, does not enter the unsafe regions $\X_U$ within $T=5$ time steps.

The complete configuration for the \overtaking benchmark is shown in Listing~\ref{lst:overtaking}.
\lstinputlisting[language=yaml,caption={Configuration for \overtaking},captionpos=b,label={lst:overtaking}]{code/overtaking.yaml}


% Linear
% Success: 93.74%, c 0.0032813476604243406, eta 0.013421457997340467, lambda 1e-05, num_frequencies 5, num_oversample 704, oversample_factor 2.0, sigma_l 0.034, sigma_f 18.0, T 15
% \begin{tabular}{rlrrrrrrrrl}
%   \toprule
%   sigma_f   & sigma_l & lambda_  & num_frequencies & num_oversample & T  & gamma    & eta      & c        & percentage & time \\
%   \midrule
%   18.000000 & 0.034   & 0.000010 & 5               & 704            & 15 & 1.000000 & 0.013421 & 0.003281 & 93.735833  & 0:01 \\
%   \bottomrule
% \end{tabular}

% Barrier 2
% \begin{tabular}{rlrrrrrrrrl}
%     \toprule
%     sigma_f  & sigma_l           & lambda_  & num_frequencies & num_oversample & T & gamma      & eta      & c        & percentage & time  \\
%     \midrule
%     1.000000 & [2.50304 3.77779] & 0.000001 & 15              & 400            & 5 & 2.000000   & 0.155993 & 0.141315 & 56.871502  & 10:01 \\
%     1.000000 & [2.50304 3.77779] & 0.000001 & 9               & 350            & 5 & 100.000000 & 8.116091 & 7.537732 & 54.195247  & 2:02  \\
%     1.000000 & [2.50304 3.77779] & 0.000001 & 11              & 300            & 5 & 2.000000   & 0.183647 & 0.150238 & 53.258213  & 2:25  \\
%     1.000000 & [2.50304 3.77779] & 0.000001 & 13              & 300            & 5 & 2.000000   & 0.182542 & 0.150821 & 53.167620  & 3:53  \\
%     1.000000 & [2.50304 3.77779] & 0.000001 & 9               & 300            & 5 & 2.000000   & 0.176700 & 0.153188 & 52.868065  & 1:28  \\
%     1.000000 & [2.50304 3.77779] & 0.000001 & 9               & 300            & 5 & 100.000000 & 8.835006 & 7.659386 & 52.868065  & 1:33  \\
%     1.000000 & [2.50304 3.77779] & 0.000001 & 20              & 300            & 5 & 2.000000   & 0.223455 & 0.147586 & 51.930650  & 16:48 \\
%     1.000000 & [2.50304 3.77779] & 0.000001 & 9               & 250            & 5 & 100.000000 & 9.875340 & 7.785029 & 51.199515  & 1:10  \\
%     \bottomrule
% \end{tabular}


% Barrier3
% Success: 42.25%, c 0.11928419647787443, eta 0.5584972599863635, lambda 1e-08, num_frequencies 15, num_oversample 800, oversample_factor 1.0, sigma_l [2.99266 4.62946], sigma_f 1.0, T 5
% \begin{tabular}{rlrrrrrrrrl}
%     \toprule
%     sigma_f  & sigma_l           & lambda_  & num_frequencies & num_oversample & T & gamma    & eta      & c        & percentage & time  \\
%     \midrule
%     1.000000 & [2.99266 4.62946] & 0.000000 & 15              & 800            & 5 & 2.000000 & 0.558497 & 0.119284 & 42.254088  & 64:20 \\
%     1.000000 & [2.99266 4.62946] & 0.000000 & 15              & 600            & 5 & 2.000000 & 0.598432 & 0.122033 & 39.570103  & 25:12 \\
%     1.000000 & [2.99266 4.62946] & 0.000000 & 15              & 450            & 5 & 2.000000 & 0.638667 & 0.127644 & 36.155716  & 12:07 \\
%     1.000000 & [2.99266 4.62946] & 0.000000 & 14              & 450            & 5 & 2.000000 & 0.638151 & 0.128054 & 36.078977  & 9:58  \\
%     1.000000 & [2.99266 4.62946] & 0.000000 & 14              & 400            & 5 & 2.000000 & 0.651139 & 0.130655 & 34.779342  & 7:50  \\
%     1.000000 & [2.99266 4.62946] & 0.000000 & 14              & 370            & 5 & 2.000000 & 0.662921 & 0.133060 & 33.589019  & 6:49  \\
%     1.000000 & [2.99266 4.62946] & 0.000000 & 16              & 370            & 5 & 2.000000 & 0.646389 & 0.136513 & 33.552259  & 11:35 \\
%     1.000000 & [2.99266 4.62946] & 0.000000 & 12              & 370            & 5 & 2.000000 & 0.710767 & 0.127533 & 32.578438  & 4:38  \\
%     1.000000 & [2.99266 4.62946] & 0.000000 & 11              & 350            & 5 & 2.000000 & 0.755459 & 0.128623 & 30.071348  & 3:14  \\
%     1.000000 & [2.99266 4.62946] & 0.000000 & 10              & 350            & 5 & 2.000000 & 0.812608 & 0.125781 & 27.924451  & 2:34  \\
%     1.000000 & [2.99266 4.62946] & 0.000000 & 9               & 350            & 5 & 2.000000 & 0.836442 & 0.128851 & 25.965203  & 2:02  \\
%     \bottomrule
% \end{tabular}



% Overtaking
% Success: 30.32%, c 0.05031756638044985, eta 0.09681458793205037, lambda 1e-05, num_frequencies 5, num_oversample -1, oversample_factor 9.0, sigma_l [10.  7.  5.], sigma_f 7.0, T 5
% \begin{tabular}{rlrrrrrrrrl}
%     \toprule
%     sigma_f  & sigma_l       & lambda_  & num_frequencies & num_oversample & T & gamma    & eta      & c        & percentage & time  \\
%     \midrule
%     7.000000 & [10.  7.  5.] & 0.000010 & 5               & 99             & 5 & 0.500000 & 0.096815 & 0.050318 & 30.319516  & 46:33 \\
%     7.000000 & [10.  7.  5.] & 0.000010 & 5               & 100            & 5 & 0.500000 & 0.100419 & 0.050519 & 29.396944  & 49:21 \\
%     7.000000 & [10.  7.  5.] & 0.000010 & 5               & 88             & 5 & 0.500000 & 0.093582 & 0.052546 & 28.737764  & 25:52 \\
%     7.000000 & [10.  7.  5.] & 0.000010 & 5               & 77             & 5 & 0.500000 & 0.103997 & 0.054252 & 24.948829  & 13:49 \\
%     7.000000 & [10.  7.  5.] & 0.000010 & 5               & 66             & 5 & 0.500000 & 0.124539 & 0.058476 & 16.616154  & 8:16  \\
%     7.000000 & [10.  7.  5.] & 0.000010 & 4               & 90             & 5 & 0.500000 & 0.101765 & 0.067857 & 11.790251  & 11:21 \\
%     7.000000 & [10.  7.  5.] & 0.000010 & 4               & 81             & 5 & 0.500000 & 0.109691 & 0.068535 & 9.527199   & 8:08  \\
%     7.000000 & [10.  7.  5.] & 0.000010 & 5               & 55             & 5 & 0.500000 & 0.135738 & 0.063642 & 9.210252   & 4:15  \\
%     7.000000 & [10.  7.  5.] & 0.000010 & 4               & 72             & 5 & 0.500000 & 0.109150 & 0.069704 & 8.465822   & 5:10  \\
%     7.000000 & [10.  7.  5.] & 0.000010 & 5               & 44             & 5 & 0.500000 & 0.160935 & 0.062971 & 4.842390   & 2:06  \\
%     7.000000 & [10.  7.  5.] & 0.000010 & 4               & 63             & 5 & 0.500000 & 0.120951 & 0.071611 & 4.198906   & 3:25  \\
%     \bottomrule
% \end{tabular}
