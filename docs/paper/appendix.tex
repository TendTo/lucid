% !TEX root =  main.tex
\appendix
\onecolumn

% \section{Additional Theoretical Details}
% \paragraph{RKHS basics.}
% A symmetric function $k_\X:\X\times\X\rightarrow\R$ is called a (positive definite) \emph{kernel} (note the distinction from \emph{probability kernels}) if for all $N\in\N_{>0}$ we have $\sum_{i=1}^{N}\sum_{j=1}^{N}a_i a_j\allowbreak k_\X(x_i,x_j) \geq 0$ for $x_1,\ldots,x_N\in\X\subset{\R^n}$ and $a_1,\ldots,a_N\in\R$.
% A prominent example is the \emph{squared exponential} (SQExp) kernel \shortcite{Rasmussen2005GP,Kanagawa2018GPvsKernel}:
% \begin{equation}
%     k_\X(x,x') := \sigma_f^2 \exp\left( -\frac{1}{2} (x-x')\T \Sigma\, (x-x') \right),\quad \Sigma:=\mathrm{diag}(\sigma_l)^{-2},\label{eq:sqexp_kernel}
% \end{equation}
% with amplitude $\sigma_f^2\geq0$ and lengthscale coefficients $\sigma_l\in\R^n$.
% \new{For this work,} we assume that all kernels are bounded on their domain, i.e., $\E_{}[k_\X(x,x)]<\infty$, $x\in\X$.
% Given a kernel $k_\X$ on a non-empty set $\X$, there exists a unique corresponding
% \emph{reproducing kernel Hilbert space} (RKHS) $\Hilbert_{k_\X}$
% of functions $f:\X\rightarrow\R$ equipped with an inner product $\innerH{\cdotx}{\cdotx}{\Hilbert_{k_\X}}$
% with the celebrated \emph{reproducing property} such that for any function $f\in\Hilbert_{k_\X}$ and $x\in\X$ we have $f(x)=\innerH{f}{k_\X(\cdotx,x)}{\Hilbert_{k_\X}}$.
% Note that $k_\X(\cdotx,x):\X\rightarrow\Hilbert_{k_\X}$ is a real-valued function,
% which is also called an implicit \emph{canonical} \emph{embedding} or \emph{feature map} $\phi_\X$
% such that $k_\X(x,x')=\innerH{\phi_\X(x)}{\phi_\X(x')}{\Hilbert_{k_\X}}$ for all $x,x'\in\X$.
% For an RKHS $\Hilbert_{k_\X}$, we use the associated feature map $\phi_\X$ and kernel $k_\X$ interchangeably for ease of notation and comprehensibility.
% The inner product induces the norm $\norm{f}_{\Hilbert_{k_\X}}\!\!\!\!:=\!\!\sqrt{\smash[b]{\innerH{f}{f}{\Hilbert_{k_\X}}}}$ of the RKHS. 
% Throughout this paper, we assume that all RKHSs are \emph{separable}.
% Refer to the monograph by \citet{Berlinet2004RKHSProbStat} for a comprehensive study on RKHSs.
% Given $N$ i.i.d. samples $\hat X_N:=[\hat{x}_i]_{i=1}^N$ with $\hat{x}_i\in\X$, the
% \emph{Gram matrix} of $k_\X$ is given by
% $\new{K_{\hat{X}}^N}:=[k_\X(\hat{x}_i,\hat{x}_j)]_{i,j=1}^N.$
% Furthermore, we define the vector-valued function 
% $\new{k_{\hat{X}}^N}(x)  := [k_\X(x,\hat{x}_i)]_{i=1}^N.$

% \begin{definition}[Conditional Mean Embedding (CME)]\label{def:condMeanEmbed}
%     Given two RKHSs $\Hilbert_{k_\X}$ and $\Hilbert_{k_\Y}$ with the associated kernels $k_\X\colon\X\times\X\to\R$ and $k_\Y\colon\X\times\X\to\R$, the \emph{CME} of a probability kernel $\Tr\colon\X\times\borel{\Y}\rightarrow[0,1]$ is an $X$-measurable random variable taking values in $\Hilbert_{k_\Y}$, given by
%     \begin{equation*}
%         \cme_{k_\Y|k_\X}(\Tr)(\cdotx) := \E_{\Tr}[\phi_\Y(Y)\mid X=\cdotx].
%     \end{equation*}
% \end{definition}

\section{Linear Program}
The finitely-constrained LP discussed in Section~\ref{sec:ddbarriers} is presented. For this, the lattices $\smash{\{ x^{1}, \ldots, x^{{N}} \}}\subset\X$, $\smash{\{ x_0^{1}, \ldots, x_0^{{N}_0} \}} \subset \X_0$, and $\smash{\{ x_U^{1}, \ldots, x_U^{N_U} \}} \subset \X_U$ of cardinality $N\in\N$, ${N}_0\in\N$, and $N_U\in\N$, respectively, are formed.
%
For given values of $\overline{\B}$, $\gamma$, and robustness radius $\mathcal{R}\geq0$, the following LP is obtained:
\begin{equation*}
      \begin{alignedat}{3}
            & \min_{\stackrel{b, c, \eta}{\Bmin_{{N}}^{\X_0}, \Bmax_{{N}}^{\X_U},\Bmax_{{N}}^\X,\Bmin_\Delta}} &                                          & \eta + cT,                                    &                   &
            %\label{eq:blackbox:linear_prog_objective}
            \\
            & \text{subject to}
            &                                                                                                  & \phi_M(x_0^{i})\T b\leq\hat{\eta}, \quad &                                               & i=1,\ldots,{N}_0,
            %\label{eq:blackbox:linear_prog_initial}
            \\
            &                                                                                                  &                                          & \phi_M(x_U^{i})\T b\geq\hat{\gamma},
            \quad &                                                                                                  & i=1,\ldots,N_U,
            %\label{eq:blackbox:linear_prog_unsafe}
            \\
            &                                                                                                  &                                          & \phi_M(x^{i})\T(Hb - b) \leq \hat{\Delta},
            \quad &                                                                                                  & i=1,\ldots,{N},
            %\label{eq:blackbox:linear_prog_kushner}
            \\
            &                                                                                                  &                                          & \phi_M(x^{i})\T b\geq \hat{\xi},
            \quad &                                                                                                  & i=1,\ldots,{N},
            %\label{eq:blackbox:linear_prog_positive}
            \\
            &                                                                                                  &                                          & \Bmin_{{N}}^{\X_0}\leq\phi_M(x_0^{i})\T b,
            \quad &                                                                                                  & i=1,\ldots,{N}_0,                                                                                              \\
            &                                                                                                  &                                          & \Bmax_{{N}}^{\X_U}\geq\phi_M(x_U^{i})\T b,
            \quad &                                                                                                  & i=1,\ldots,N_U,                                                                                              \\
            &                                                                                                  &                                          & \Bmax_{{N}}^\X\geq\phi_M(x^{i})\T b,
            \quad &                                                                                                  & i=1,\ldots,{N},                                                                                                \\
            &                                                                                                  &                                          & \Bmin_\Delta\leq\phi_M(x^{i})\T (Hb - b),
            \quad &                                                                                                  & i=1,\ldots,{N},                                                                                                \\
            &                                                                                                  &                                          & c\geq 0,\,\gamma>\eta\geq 0,\, b\in\R^{2M+1}, &                   &
            \label{eq:blackbox:linear_prog}%
      \end{alignedat}
\end{equation*}
with $\overline{\kappa}\geq\sigma_f$, $\overline{\B}\geq\norm{b}_2$, and constraint-tightening coefficients
\begin{align*}
      \hat{\eta}   & := \frac{2}{C_{{N}}+1}\eta + \frac{C_{{N}}-1}{C_{{N}}+1}\Bmin_{{N}}^{\X_0},
                   &                                                                                                                             & \hat{\gamma} := \frac{2}{C_{{N}}+1}\gamma + \frac{C_{{N}}-1}{C_{{N}}+1}\Bmax_{{N}}^{\X_U}, \\
      \hat{\Delta} & := \frac{2}{C_{{N}}+1}\left(c - \mathcal{R}\overline{\B}\overline{\kappa}\right) + \frac{C_{{N}}-1}{C_{{N}}+1}\Bmin_\Delta,
                   &                                                                                                                             & \hat{\xi} := \frac{C_{{N}}-1}{C_{{N}}+1}\Bmax_{{N}}^\X.
\end{align*}


% \section{Proofs}
% We have collected a series of proofs of the claims we made in the main text.

\section{Tool Technical Details}

\subsection{Installation}
The recommended way to install \pylucid is via the \texttt{pip} package manager, using one of the pre-built wheels,
hence avoiding the time-consuming need to compile the code from source.
Provided \texttt{Python>=3.8} is already present on the system, \pylucid is installed using a single command:
%
\begin{lstlisting}[language=bash,numbers=none,xleftmargin=0em,backgroundcolor=\color{ipython_bg}]
pip install "pylucid[gui,plot]" --index-url https://gitlab.com/api/v4/projects/71977529/packages/pypi/simple
\end{lstlisting}
%
Note that installing \pylucid this way requires \gurobi to be installed beforehand, which can be obtained for free from the official website.
A valid license is not required unless the \gurobi solver is used.
If no wheel is available for the desired platform, \pylucid can be installed from source by running the following commands:
%
\begin{lstlisting}[language=bash,numbers=none,xleftmargin=0em,backgroundcolor=\color{ipython_bg}]
git clone https://gitlab.com/lucidtoolsource/lucid.git
cd lucid
pip install ".[gui,plot]"
\end{lstlisting}
%
Note that this requires \texttt{Python>=3.8} and \texttt{Bazel>=8.0} to be installed on the system.
For more details on the installation process, please refer to the online documentation at \url{https://lucidtoolsource.gitlab.io/lucid/md_docs_2Pylucid.html}.

\subsection{Containerized Installation}

We also provide a pre-built Docker image that can be used to run \pylucid in a containerized environment, making Docker the only dependency.
To run the image with the desired configuration, the following command can be used:
\begin{lstlisting}[language=bash,numbers=none,xleftmargin=0em,backgroundcolor=\color{ipython_bg}]
docker run --name lucid -it --rm \
  -v/path/to/conf.py:/config \
  registry.gitlab.com/lucidtoolsource/lucid:latest /config/conf.py
\end{lstlisting}
Alternatively, the container can also run the GUI by changing the command to
\begin{lstlisting}[language=bash,numbers=none,xleftmargin=0em,backgroundcolor=\color{ipython_bg}]
docker run --name lucid -it --rm \
  -p 3661:3661 --entrypoint pylucid-gui \
  registry.gitlab.com/lucidtoolsource/lucid:latest
\end{lstlisting}
The GUI will be accessible at \url{http://localhost:3661}.
In both cases, to use the \gurobi solver, a \gurobi Web License Service (WLS) license\footnote{See \url{https://www.gurobi.com/features/web-license-service/}} will have to be mounted in the container.

\subsection{Platform Support}
Some of \pylucid's dependencies are not supported on all platforms, which may preclude the ability of using certain features.
For instance, optional formal verification of the barrier requires the \dreal SMT solver, which can be installed only on Linux and non-ARM macOS.
The \highs \gls{lp} solver cannot be used on Windows at the moment.
The \gurobi solver must be installed separately beforehand and requires a licence, which is free for academic use.
Using the Docker container sidesteps most of these issues, with the only requirement being a CPU with x86-64 architecture.

\subsection{Configuration Formats}
\pylucid provides extensive support for various ways of specifying configurations, which can be defined as a \texttt{.yaml} file (recommended), \texttt{.json} file,
generated via a \texttt{.py} file returning a \texttt{Configuration} object, or provided directly via command line arguments.
Moreover, system transition data can be either embedded directly in the configuration file or provided separately in the form of \texttt{.csv}, \texttt{.mat}, \texttt{.npy}, or \texttt{.npz} files.
For example, the following configurations are equivalent:

\begin{minipage}{.45\columnwidth}
      \lstinputlisting[language=yaml,style={bgnonumbers},caption={YAML configuration.},captionpos=b,label={lst:configuration.yaml},backgroundcolor=\color{ipython_bg}]{code/configuration.yaml}
\end{minipage}\hfill
\begin{minipage}{.45\columnwidth}
      \lstinputlisting[language=json,style={bgnonumbers},caption={JSON configuration.},captionpos=b,label={lst:configuration.json}]{code/configuration.json}
\end{minipage}
\begin{minipage}{.45\columnwidth}
      \lstinputlisting[language=iPython,style={bgnonumbers},caption={Python configuration.},captionpos=b,label={lst:configuration.py}]{code/configuration.py}
\end{minipage}\hfill
\begin{minipage}{.45\columnwidth}
      \lstinputlisting[language=bash,style={bgnonumbers},caption={Command line configuration assuming that the transition samples have been stored as \texttt{.csv} files.},captionpos=b,label={lst:configuration.sh}]{code/configuration.sh}
\end{minipage}


\subsection{Graphical User Interface}

\pylucid provides a \gls{gui} to guide the user in the creation of the scenario configuration and presenting the results in an intuitive format.
Running \lstinline|pylucid-gui| will open a browser tab with the interface shown in
%Figures~\ref{fig:gui-left}--\ref{fig:gui-right}, 
Figure~\ref{fig:gui-full},
while a local server will listen for requests coming from the \gls{gui}, computing and returning the results.
Following the same enumeration as the arrows in the figure, we provide a list of the main components of the \gls{gui}:

\begin{figure}
      \centering
      \shadowimage[width=16.5cm]{figures/frontend-highlight.png}
      \caption{
            \pylucid's \gls{gui} as presented in the browser.
            The left half of the page allows the user to provide the transition samples and the safety specification to verify.
            The right half of the page shows the results produced by \lucid, including a plot of the computed barrier function and detailed logs.}
      \label{fig:gui-full}
\end{figure}

\begin{enumerate}
      \item It is easy to import or export configurations in JSON format.
            In the \textit{Import} menu, the user can paste the configuration or load it from a file.
            The experiments discussed in this paper are also available and can be selected from a dropdown list.
            The \textit{Export} menu can be used to copy the current configuration to the clipboard or download it locally.
      \item The dimension $n$ of the system is specified. Here, $n=1$.
      \item  The user can provide transition samples directly by copy--pasting them into the text fields in the Data tab.
            Alternatively, if a model describing the system’s expected behavior is available, the user can specify it in the Model tab.
            Lucid will then automatically generate a set of transitions from the model to be used in the computation of the barrier function.
            Note that this model-based generation is mainly intended for reference and convenience.
      \item The $x^1,\ldots,x^N$ samples are specified in CSV format. They can be edited manually or loaded from a file.
      \item The $x^1_+,\ldots,x^N_+$ samples are specified analogously.
      \item The state space $\X$ is defined by a \texttt{RectSet}. Here, $\X=[-1,1]$.
      \item The initial set $\X_0$ is defined from one of the available type options. A \texttt{MultiSet} can be defined via the \textit{+} button. Here, $\X_0=[-0.5,0.5]$.
      \item The unsafe set $\X_U$ is defined analogously. Here, $\X_U=[-1, -0.9] \cup [0.9, 1]$.
      \item The \textit{Preview} button visualizes the expected stochastic behavior of the black-box system, predicted from the transition samples or alternatively as specified in the \textit{Model} tab,
            alongside the bounding boxes of the state space $\X$, the initial set $\X_0$, and the unsafe set $\X_U$.
      \item The \textit{Submit} button starts the computation of the barrier function with the current configuration.
      \item The plot of the transition function or the computed barrier function.
      \item The numerical results of the computation, including the lower bound on the safety probability, values of $\eta$ and $c$, and the computation time.
            If a model of the expected system behavior is provided, it is also possible to formally verify the computed barrier function via \dreal.
      \item The logs provide information on the progress of the computation, updated in real-time.
            The verbosity of the logs can be adjusted in the \textit{Advanced} options.
      \item The main \gls{gui} displays the most important configuration options.
            More advanced options can be accessed via the \textit{Advanced} button.
            This includes the kernel \hp, time horizon, number of frequencies and lattice size, \gls{lp} solver selection, etc.
\end{enumerate}



\section{Benchmarks}
\label{app:benchmarks}
We provide a more detailed description of the benchmarks presented in Section~\ref{sec:experiments}, including the corresponding configuration files and an extended list of experiments.
All benchmarks were run on a Windows 10 machine with an AMD Ryzen 9 5950X 16-Core Processor @ 3.40 GHz, NVIDIA GeForce RTX 3090 GPU, and 64 GB of RAM.
To ensure reproducibility, all runs had the random seed set to $42$.
We ascertained that the variance of the safety probability is low across different seeds, and can be reduced further by increasing the number of samples used to fit the \estimator.
The scripts for running the benchmarks and tuning the \hp are available in the \texttt{benchmarks/integration/} folder of the repository.
Note that \texttt{Python>=3.9} is required to run the scripts, as some dependencies used to track the benchmarks' metrics across multiple runs are not compatible with Python 3.8.
The hyperparameter tuning of the \estimator was performed with the \texttt{LbfgsTuner}, bounding the value of $\sigma_l$ between $[10^{-5}, 10^5]$, and refined with the \texttt{GridSearchTuner}.
We use $N=1000$ sample transitions and the \gurobi solver for all experiments;
Other optimizers, such as \alglib or \highs, can be used instead, but may yield different results, especially in terms of performance.

\subsection{Linear}

We consider a black-box system with the following dynamics:
\begin{equation*}
      {x}_{t+1} = 0.5 {x}_{t} + w_t,
\end{equation*}
where $w_t\sim\mathcal{N}(\cdotx\vert 0,0.01I_1)$.
The data sampled from this black-box system is used as input for \lucid.
Given $\X = [-1, 1]$, $\X_0 = [-0.5, 0.5]$, and $\X_U = [-1, -0.9] \cup [0.9, 1]$,
we want to ensure that the system, starting in $\X_0$, does not enter the unsafe regions $\X_U$ within $T=15$ time steps.
We set the kernel \hp to $\sigma_f=15$, $\sigma_l=1.2$, and $\lambda=0.00001$.
The complete configuration for the linear example benchmark is shown in Listing~\ref{lst:linear}, with a satisfying barrier plotted in Figure~\ref{fig:example}.
\lstinputlisting[language=yaml,style={bgnonumbers},caption={Configuration for the linear example.},captionpos=b,label={lst:linear},backgroundcolor=\color{ipython_bg}]{code/linear.yaml}

\subsection{Barrier 2}

We consider a black-box system with the following nonlinear dynamics:
\begin{equation*}
      \begin{bmatrix}
            {x}_{1, t+1} \\
            {x}_{2, t+1}
      \end{bmatrix}
      = \begin{bmatrix}
            {x}_{1, t} \\
            {x}_{2, t}
      \end{bmatrix} + 0.1 \begin{bmatrix}
            {x}_{2, t} - 1 + e^{-x_{1, t}} \\
            -\sin^2(x_{1, t})
      \end{bmatrix} + w_t,
\end{equation*}
where $w_t\sim\mathcal{N}(\cdotx\vert 0,0.01I_2)$.
The data sampled from this black-box system is used as input for \lucid.
Given
\begin{align*}
       & \X = [ -2, 2 ] \times [ -2 , 2 ]   ,                              \\
       & \X_0 = \{ [x_1, x_2] : (x_1 + 0.5)^2 + (x_2 - 0.5)^2 \leq 0.4 \}, \\
       & \X_U = \{ [x_1, x_2] : (x_1 - 0.7)^2 + (x_2 + 0.7)^2 \leq 0.3 \},
\end{align*}
we want to ensure that the system, starting in $\X_0$, does not enter the unsafe regions $\X_U$ within $T=5$ time steps.
The system dynamics and safety specification are visualized in Figure~\ref{fig:model-barrier2}.
The kernel \hp are set to $\sigma_f=1$, $\sigma_l=[2.50, 3.78]$, and $\lambda=\smash{10^{-6}}$.
The complete configuration for the \barrII benchmark is shown in Listing~\ref{lst:barrier2}, with a satisfying barrier is plotted in Figure~\ref{fig:CSBarr2}.
A list of experiments using different combinations of frequencies and lattice sizes, showing their impact on performance and final result, is presented in Table~\ref{tab:results-barrier2}.
\lstinputlisting[language=yaml,style={bgnonumbers},caption={Configuration for \barrII.},captionpos=b,label={lst:barrier2}]{code/barrier2.yaml}

\begin{figure}[ht]
      \centering
      %% Creator: Matplotlib, PGF backend
%%
%% To include the figure in your LaTeX document, write
%%   \input{<filename>.pgf}
%%
%% Make sure the required packages are loaded in your preamble
%%   \usepackage{pgf}
%%
%% Also ensure that all the required font packages are loaded; for instance,
%% the lmodern package is sometimes necessary when using math font.
%%   \usepackage{lmodern}
%%
%% Figures using additional raster images can only be included by \input if
%% they are in the same directory as the main LaTeX file. For loading figures
%% from other directories you can use the `import` package
%%   \usepackage{import}
%%
%% and then include the figures with
%%   \import{<path to file>}{<filename>.pgf}
%%
%% Matplotlib used the following preamble
%%   \def\mathdefault#1{#1}
%%   \everymath=\expandafter{\the\everymath\displaystyle}
%%   \IfFileExists{scrextend.sty}{
%%     \usepackage[fontsize=10.000000pt]{scrextend}
%%   }{
%%     \renewcommand{\normalsize}{\fontsize{10.000000}{12.000000}\selectfont}
%%     \normalsize
%%   }
%%   
%%   \ifdefined\pdftexversion\else  % non-pdftex case.
%%     \usepackage{fontspec}
%%     \setmainfont{DejaVuSerif.ttf}[Path=\detokenize{/home/campus.ncl.ac.uk/c3054737/miniconda3/envs/pylucid/lib/python3.11/site-packages/matplotlib/mpl-data/fonts/ttf/}]
%%     \setsansfont{DejaVuSans.ttf}[Path=\detokenize{/home/campus.ncl.ac.uk/c3054737/miniconda3/envs/pylucid/lib/python3.11/site-packages/matplotlib/mpl-data/fonts/ttf/}]
%%     \setmonofont{DejaVuSansMono.ttf}[Path=\detokenize{/home/campus.ncl.ac.uk/c3054737/miniconda3/envs/pylucid/lib/python3.11/site-packages/matplotlib/mpl-data/fonts/ttf/}]
%%   \fi
%%   \makeatletter\@ifpackageloaded{underscore}{}{\usepackage[strings]{underscore}}\makeatother
%%
\begingroup%
\makeatletter%
\begin{pgfpicture}%
\pgfpathrectangle{\pgfpointorigin}{\pgfqpoint{3.861960in}{3.719959in}}%
\pgfusepath{use as bounding box, clip}%
\begin{pgfscope}%
\pgfsetbuttcap%
\pgfsetmiterjoin%
\definecolor{currentfill}{rgb}{1.000000,1.000000,1.000000}%
\pgfsetfillcolor{currentfill}%
\pgfsetlinewidth{0.000000pt}%
\definecolor{currentstroke}{rgb}{1.000000,1.000000,1.000000}%
\pgfsetstrokecolor{currentstroke}%
\pgfsetdash{}{0pt}%
\pgfpathmoveto{\pgfqpoint{0.000000in}{0.000000in}}%
\pgfpathlineto{\pgfqpoint{3.861960in}{0.000000in}}%
\pgfpathlineto{\pgfqpoint{3.861960in}{3.719959in}}%
\pgfpathlineto{\pgfqpoint{0.000000in}{3.719959in}}%
\pgfpathlineto{\pgfqpoint{0.000000in}{0.000000in}}%
\pgfpathclose%
\pgfusepath{fill}%
\end{pgfscope}%
\begin{pgfscope}%
\pgfsetbuttcap%
\pgfsetmiterjoin%
\definecolor{currentfill}{rgb}{1.000000,1.000000,1.000000}%
\pgfsetfillcolor{currentfill}%
\pgfsetlinewidth{0.000000pt}%
\definecolor{currentstroke}{rgb}{0.000000,0.000000,0.000000}%
\pgfsetstrokecolor{currentstroke}%
\pgfsetstrokeopacity{0.000000}%
\pgfsetdash{}{0pt}%
\pgfpathmoveto{\pgfqpoint{0.647939in}{0.492442in}}%
\pgfpathlineto{\pgfqpoint{3.727238in}{0.492442in}}%
\pgfpathlineto{\pgfqpoint{3.727238in}{3.571741in}}%
\pgfpathlineto{\pgfqpoint{0.647939in}{3.571741in}}%
\pgfpathlineto{\pgfqpoint{0.647939in}{0.492442in}}%
\pgfpathclose%
\pgfusepath{fill}%
\end{pgfscope}%
\begin{pgfscope}%
\pgfpathrectangle{\pgfqpoint{0.647939in}{0.492442in}}{\pgfqpoint{3.079299in}{3.079299in}}%
\pgfusepath{clip}%
\pgfsetbuttcap%
\pgfsetmiterjoin%
\pgfsetlinewidth{2.007500pt}%
\definecolor{currentstroke}{rgb}{0.000000,0.000000,1.000000}%
\pgfsetstrokecolor{currentstroke}%
\pgfsetdash{}{0pt}%
\pgfpathmoveto{\pgfqpoint{1.802676in}{2.109074in}}%
\pgfpathcurveto{\pgfqpoint{1.884340in}{2.109074in}}{\pgfqpoint{1.962670in}{2.141520in}}{\pgfqpoint{2.020416in}{2.199265in}}%
\pgfpathcurveto{\pgfqpoint{2.078161in}{2.257010in}}{\pgfqpoint{2.110606in}{2.335340in}}{\pgfqpoint{2.110606in}{2.417004in}}%
\pgfpathcurveto{\pgfqpoint{2.110606in}{2.498668in}}{\pgfqpoint{2.078161in}{2.576998in}}{\pgfqpoint{2.020416in}{2.634743in}}%
\pgfpathcurveto{\pgfqpoint{1.962670in}{2.692489in}}{\pgfqpoint{1.884340in}{2.724934in}}{\pgfqpoint{1.802676in}{2.724934in}}%
\pgfpathcurveto{\pgfqpoint{1.721012in}{2.724934in}}{\pgfqpoint{1.642682in}{2.692489in}}{\pgfqpoint{1.584937in}{2.634743in}}%
\pgfpathcurveto{\pgfqpoint{1.527192in}{2.576998in}}{\pgfqpoint{1.494746in}{2.498668in}}{\pgfqpoint{1.494746in}{2.417004in}}%
\pgfpathcurveto{\pgfqpoint{1.494746in}{2.335340in}}{\pgfqpoint{1.527192in}{2.257010in}}{\pgfqpoint{1.584937in}{2.199265in}}%
\pgfpathcurveto{\pgfqpoint{1.642682in}{2.141520in}}{\pgfqpoint{1.721012in}{2.109074in}}{\pgfqpoint{1.802676in}{2.109074in}}%
\pgfpathlineto{\pgfqpoint{1.802676in}{2.109074in}}%
\pgfpathclose%
\pgfusepath{stroke}%
\end{pgfscope}%
\begin{pgfscope}%
\pgfpathrectangle{\pgfqpoint{0.647939in}{0.492442in}}{\pgfqpoint{3.079299in}{3.079299in}}%
\pgfusepath{clip}%
\pgfsetbuttcap%
\pgfsetmiterjoin%
\pgfsetlinewidth{2.007500pt}%
\definecolor{currentstroke}{rgb}{1.000000,0.000000,0.000000}%
\pgfsetstrokecolor{currentstroke}%
\pgfsetdash{}{0pt}%
\pgfpathmoveto{\pgfqpoint{2.726466in}{1.262267in}}%
\pgfpathcurveto{\pgfqpoint{2.787714in}{1.262267in}}{\pgfqpoint{2.846462in}{1.286601in}}{\pgfqpoint{2.889770in}{1.329910in}}%
\pgfpathcurveto{\pgfqpoint{2.933079in}{1.373219in}}{\pgfqpoint{2.957413in}{1.431966in}}{\pgfqpoint{2.957413in}{1.493214in}}%
\pgfpathcurveto{\pgfqpoint{2.957413in}{1.554462in}}{\pgfqpoint{2.933079in}{1.613210in}}{\pgfqpoint{2.889770in}{1.656519in}}%
\pgfpathcurveto{\pgfqpoint{2.846462in}{1.699828in}}{\pgfqpoint{2.787714in}{1.724162in}}{\pgfqpoint{2.726466in}{1.724162in}}%
\pgfpathcurveto{\pgfqpoint{2.665218in}{1.724162in}}{\pgfqpoint{2.606470in}{1.699828in}}{\pgfqpoint{2.563161in}{1.656519in}}%
\pgfpathcurveto{\pgfqpoint{2.519853in}{1.613210in}}{\pgfqpoint{2.495519in}{1.554462in}}{\pgfqpoint{2.495519in}{1.493214in}}%
\pgfpathcurveto{\pgfqpoint{2.495519in}{1.431966in}}{\pgfqpoint{2.519853in}{1.373219in}}{\pgfqpoint{2.563161in}{1.329910in}}%
\pgfpathcurveto{\pgfqpoint{2.606470in}{1.286601in}}{\pgfqpoint{2.665218in}{1.262267in}}{\pgfqpoint{2.726466in}{1.262267in}}%
\pgfpathlineto{\pgfqpoint{2.726466in}{1.262267in}}%
\pgfpathclose%
\pgfusepath{stroke}%
\end{pgfscope}%
\begin{pgfscope}%
\pgfsetbuttcap%
\pgfsetroundjoin%
\definecolor{currentfill}{rgb}{0.000000,0.000000,0.000000}%
\pgfsetfillcolor{currentfill}%
\pgfsetlinewidth{0.803000pt}%
\definecolor{currentstroke}{rgb}{0.000000,0.000000,0.000000}%
\pgfsetstrokecolor{currentstroke}%
\pgfsetdash{}{0pt}%
\pgfsys@defobject{currentmarker}{\pgfqpoint{0.000000in}{-0.048611in}}{\pgfqpoint{0.000000in}{0.000000in}}{%
\pgfpathmoveto{\pgfqpoint{0.000000in}{0.000000in}}%
\pgfpathlineto{\pgfqpoint{0.000000in}{-0.048611in}}%
\pgfusepath{stroke,fill}%
}%
\begin{pgfscope}%
\pgfsys@transformshift{0.647939in}{0.492442in}%
\pgfsys@useobject{currentmarker}{}%
\end{pgfscope}%
\end{pgfscope}%
\begin{pgfscope}%
\definecolor{textcolor}{rgb}{0.000000,0.000000,0.000000}%
\pgfsetstrokecolor{textcolor}%
\pgfsetfillcolor{textcolor}%
\pgftext[x=0.647939in,y=0.395220in,,top]{\color{textcolor}{\ifdefined\pdftexversion\else\setmainfont{Times New Roman}\rmfamily\fi\fontsize{10.000000}{12.000000}\selectfont\catcode`\^=\active\def^{\ifmmode\sp\else\^{}\fi}\catcode`\%=\active\def%{\%}\ensuremath{-}2}}%
\end{pgfscope}%
\begin{pgfscope}%
\pgfsetbuttcap%
\pgfsetroundjoin%
\definecolor{currentfill}{rgb}{0.000000,0.000000,0.000000}%
\pgfsetfillcolor{currentfill}%
\pgfsetlinewidth{0.803000pt}%
\definecolor{currentstroke}{rgb}{0.000000,0.000000,0.000000}%
\pgfsetstrokecolor{currentstroke}%
\pgfsetdash{}{0pt}%
\pgfsys@defobject{currentmarker}{\pgfqpoint{0.000000in}{-0.048611in}}{\pgfqpoint{0.000000in}{0.000000in}}{%
\pgfpathmoveto{\pgfqpoint{0.000000in}{0.000000in}}%
\pgfpathlineto{\pgfqpoint{0.000000in}{-0.048611in}}%
\pgfusepath{stroke,fill}%
}%
\begin{pgfscope}%
\pgfsys@transformshift{1.417764in}{0.492442in}%
\pgfsys@useobject{currentmarker}{}%
\end{pgfscope}%
\end{pgfscope}%
\begin{pgfscope}%
\definecolor{textcolor}{rgb}{0.000000,0.000000,0.000000}%
\pgfsetstrokecolor{textcolor}%
\pgfsetfillcolor{textcolor}%
\pgftext[x=1.417764in,y=0.395220in,,top]{\color{textcolor}{\ifdefined\pdftexversion\else\setmainfont{Times New Roman}\rmfamily\fi\fontsize{10.000000}{12.000000}\selectfont\catcode`\^=\active\def^{\ifmmode\sp\else\^{}\fi}\catcode`\%=\active\def%{\%}\ensuremath{-}1}}%
\end{pgfscope}%
\begin{pgfscope}%
\pgfsetbuttcap%
\pgfsetroundjoin%
\definecolor{currentfill}{rgb}{0.000000,0.000000,0.000000}%
\pgfsetfillcolor{currentfill}%
\pgfsetlinewidth{0.803000pt}%
\definecolor{currentstroke}{rgb}{0.000000,0.000000,0.000000}%
\pgfsetstrokecolor{currentstroke}%
\pgfsetdash{}{0pt}%
\pgfsys@defobject{currentmarker}{\pgfqpoint{0.000000in}{-0.048611in}}{\pgfqpoint{0.000000in}{0.000000in}}{%
\pgfpathmoveto{\pgfqpoint{0.000000in}{0.000000in}}%
\pgfpathlineto{\pgfqpoint{0.000000in}{-0.048611in}}%
\pgfusepath{stroke,fill}%
}%
\begin{pgfscope}%
\pgfsys@transformshift{2.187589in}{0.492442in}%
\pgfsys@useobject{currentmarker}{}%
\end{pgfscope}%
\end{pgfscope}%
\begin{pgfscope}%
\definecolor{textcolor}{rgb}{0.000000,0.000000,0.000000}%
\pgfsetstrokecolor{textcolor}%
\pgfsetfillcolor{textcolor}%
\pgftext[x=2.187589in,y=0.395220in,,top]{\color{textcolor}{\ifdefined\pdftexversion\else\setmainfont{Times New Roman}\rmfamily\fi\fontsize{10.000000}{12.000000}\selectfont\catcode`\^=\active\def^{\ifmmode\sp\else\^{}\fi}\catcode`\%=\active\def%{\%}0}}%
\end{pgfscope}%
\begin{pgfscope}%
\pgfsetbuttcap%
\pgfsetroundjoin%
\definecolor{currentfill}{rgb}{0.000000,0.000000,0.000000}%
\pgfsetfillcolor{currentfill}%
\pgfsetlinewidth{0.803000pt}%
\definecolor{currentstroke}{rgb}{0.000000,0.000000,0.000000}%
\pgfsetstrokecolor{currentstroke}%
\pgfsetdash{}{0pt}%
\pgfsys@defobject{currentmarker}{\pgfqpoint{0.000000in}{-0.048611in}}{\pgfqpoint{0.000000in}{0.000000in}}{%
\pgfpathmoveto{\pgfqpoint{0.000000in}{0.000000in}}%
\pgfpathlineto{\pgfqpoint{0.000000in}{-0.048611in}}%
\pgfusepath{stroke,fill}%
}%
\begin{pgfscope}%
\pgfsys@transformshift{2.957413in}{0.492442in}%
\pgfsys@useobject{currentmarker}{}%
\end{pgfscope}%
\end{pgfscope}%
\begin{pgfscope}%
\definecolor{textcolor}{rgb}{0.000000,0.000000,0.000000}%
\pgfsetstrokecolor{textcolor}%
\pgfsetfillcolor{textcolor}%
\pgftext[x=2.957413in,y=0.395220in,,top]{\color{textcolor}{\ifdefined\pdftexversion\else\setmainfont{Times New Roman}\rmfamily\fi\fontsize{10.000000}{12.000000}\selectfont\catcode`\^=\active\def^{\ifmmode\sp\else\^{}\fi}\catcode`\%=\active\def%{\%}1}}%
\end{pgfscope}%
\begin{pgfscope}%
\pgfsetbuttcap%
\pgfsetroundjoin%
\definecolor{currentfill}{rgb}{0.000000,0.000000,0.000000}%
\pgfsetfillcolor{currentfill}%
\pgfsetlinewidth{0.803000pt}%
\definecolor{currentstroke}{rgb}{0.000000,0.000000,0.000000}%
\pgfsetstrokecolor{currentstroke}%
\pgfsetdash{}{0pt}%
\pgfsys@defobject{currentmarker}{\pgfqpoint{0.000000in}{-0.048611in}}{\pgfqpoint{0.000000in}{0.000000in}}{%
\pgfpathmoveto{\pgfqpoint{0.000000in}{0.000000in}}%
\pgfpathlineto{\pgfqpoint{0.000000in}{-0.048611in}}%
\pgfusepath{stroke,fill}%
}%
\begin{pgfscope}%
\pgfsys@transformshift{3.727238in}{0.492442in}%
\pgfsys@useobject{currentmarker}{}%
\end{pgfscope}%
\end{pgfscope}%
\begin{pgfscope}%
\definecolor{textcolor}{rgb}{0.000000,0.000000,0.000000}%
\pgfsetstrokecolor{textcolor}%
\pgfsetfillcolor{textcolor}%
\pgftext[x=3.727238in,y=0.395220in,,top]{\color{textcolor}{\ifdefined\pdftexversion\else\setmainfont{Times New Roman}\rmfamily\fi\fontsize{10.000000}{12.000000}\selectfont\catcode`\^=\active\def^{\ifmmode\sp\else\^{}\fi}\catcode`\%=\active\def%{\%}2}}%
\end{pgfscope}%
\begin{pgfscope}%
\definecolor{textcolor}{rgb}{0.000000,0.000000,0.000000}%
\pgfsetstrokecolor{textcolor}%
\pgfsetfillcolor{textcolor}%
\pgftext[x=2.187589in,y=0.213525in,,top]{\color{textcolor}{\ifdefined\pdftexversion\else\setmainfont{Times New Roman}\rmfamily\fi\fontsize{9.000000}{10.800000}\selectfont\catcode`\^=\active\def^{\ifmmode\sp\else\^{}\fi}\catcode`\%=\active\def%{\%}$x_1$}}%
\end{pgfscope}%
\begin{pgfscope}%
\pgfsetbuttcap%
\pgfsetroundjoin%
\definecolor{currentfill}{rgb}{0.000000,0.000000,0.000000}%
\pgfsetfillcolor{currentfill}%
\pgfsetlinewidth{0.803000pt}%
\definecolor{currentstroke}{rgb}{0.000000,0.000000,0.000000}%
\pgfsetstrokecolor{currentstroke}%
\pgfsetdash{}{0pt}%
\pgfsys@defobject{currentmarker}{\pgfqpoint{-0.048611in}{0.000000in}}{\pgfqpoint{-0.000000in}{0.000000in}}{%
\pgfpathmoveto{\pgfqpoint{-0.000000in}{0.000000in}}%
\pgfpathlineto{\pgfqpoint{-0.048611in}{0.000000in}}%
\pgfusepath{stroke,fill}%
}%
\begin{pgfscope}%
\pgfsys@transformshift{0.647939in}{0.492442in}%
\pgfsys@useobject{currentmarker}{}%
\end{pgfscope}%
\end{pgfscope}%
\begin{pgfscope}%
\definecolor{textcolor}{rgb}{0.000000,0.000000,0.000000}%
\pgfsetstrokecolor{textcolor}%
\pgfsetfillcolor{textcolor}%
\pgftext[x=0.269081in, y=0.444224in, left, base]{\color{textcolor}{\ifdefined\pdftexversion\else\setmainfont{Times New Roman}\rmfamily\fi\fontsize{10.000000}{12.000000}\selectfont\catcode`\^=\active\def^{\ifmmode\sp\else\^{}\fi}\catcode`\%=\active\def%{\%}\ensuremath{-}2.0}}%
\end{pgfscope}%
\begin{pgfscope}%
\pgfsetbuttcap%
\pgfsetroundjoin%
\definecolor{currentfill}{rgb}{0.000000,0.000000,0.000000}%
\pgfsetfillcolor{currentfill}%
\pgfsetlinewidth{0.803000pt}%
\definecolor{currentstroke}{rgb}{0.000000,0.000000,0.000000}%
\pgfsetstrokecolor{currentstroke}%
\pgfsetdash{}{0pt}%
\pgfsys@defobject{currentmarker}{\pgfqpoint{-0.048611in}{0.000000in}}{\pgfqpoint{-0.000000in}{0.000000in}}{%
\pgfpathmoveto{\pgfqpoint{-0.000000in}{0.000000in}}%
\pgfpathlineto{\pgfqpoint{-0.048611in}{0.000000in}}%
\pgfusepath{stroke,fill}%
}%
\begin{pgfscope}%
\pgfsys@transformshift{0.647939in}{0.877354in}%
\pgfsys@useobject{currentmarker}{}%
\end{pgfscope}%
\end{pgfscope}%
\begin{pgfscope}%
\definecolor{textcolor}{rgb}{0.000000,0.000000,0.000000}%
\pgfsetstrokecolor{textcolor}%
\pgfsetfillcolor{textcolor}%
\pgftext[x=0.269081in, y=0.829137in, left, base]{\color{textcolor}{\ifdefined\pdftexversion\else\setmainfont{Times New Roman}\rmfamily\fi\fontsize{10.000000}{12.000000}\selectfont\catcode`\^=\active\def^{\ifmmode\sp\else\^{}\fi}\catcode`\%=\active\def%{\%}\ensuremath{-}1.5}}%
\end{pgfscope}%
\begin{pgfscope}%
\pgfsetbuttcap%
\pgfsetroundjoin%
\definecolor{currentfill}{rgb}{0.000000,0.000000,0.000000}%
\pgfsetfillcolor{currentfill}%
\pgfsetlinewidth{0.803000pt}%
\definecolor{currentstroke}{rgb}{0.000000,0.000000,0.000000}%
\pgfsetstrokecolor{currentstroke}%
\pgfsetdash{}{0pt}%
\pgfsys@defobject{currentmarker}{\pgfqpoint{-0.048611in}{0.000000in}}{\pgfqpoint{-0.000000in}{0.000000in}}{%
\pgfpathmoveto{\pgfqpoint{-0.000000in}{0.000000in}}%
\pgfpathlineto{\pgfqpoint{-0.048611in}{0.000000in}}%
\pgfusepath{stroke,fill}%
}%
\begin{pgfscope}%
\pgfsys@transformshift{0.647939in}{1.262267in}%
\pgfsys@useobject{currentmarker}{}%
\end{pgfscope}%
\end{pgfscope}%
\begin{pgfscope}%
\definecolor{textcolor}{rgb}{0.000000,0.000000,0.000000}%
\pgfsetstrokecolor{textcolor}%
\pgfsetfillcolor{textcolor}%
\pgftext[x=0.269081in, y=1.214049in, left, base]{\color{textcolor}{\ifdefined\pdftexversion\else\setmainfont{Times New Roman}\rmfamily\fi\fontsize{10.000000}{12.000000}\selectfont\catcode`\^=\active\def^{\ifmmode\sp\else\^{}\fi}\catcode`\%=\active\def%{\%}\ensuremath{-}1.0}}%
\end{pgfscope}%
\begin{pgfscope}%
\pgfsetbuttcap%
\pgfsetroundjoin%
\definecolor{currentfill}{rgb}{0.000000,0.000000,0.000000}%
\pgfsetfillcolor{currentfill}%
\pgfsetlinewidth{0.803000pt}%
\definecolor{currentstroke}{rgb}{0.000000,0.000000,0.000000}%
\pgfsetstrokecolor{currentstroke}%
\pgfsetdash{}{0pt}%
\pgfsys@defobject{currentmarker}{\pgfqpoint{-0.048611in}{0.000000in}}{\pgfqpoint{-0.000000in}{0.000000in}}{%
\pgfpathmoveto{\pgfqpoint{-0.000000in}{0.000000in}}%
\pgfpathlineto{\pgfqpoint{-0.048611in}{0.000000in}}%
\pgfusepath{stroke,fill}%
}%
\begin{pgfscope}%
\pgfsys@transformshift{0.647939in}{1.647179in}%
\pgfsys@useobject{currentmarker}{}%
\end{pgfscope}%
\end{pgfscope}%
\begin{pgfscope}%
\definecolor{textcolor}{rgb}{0.000000,0.000000,0.000000}%
\pgfsetstrokecolor{textcolor}%
\pgfsetfillcolor{textcolor}%
\pgftext[x=0.269081in, y=1.598962in, left, base]{\color{textcolor}{\ifdefined\pdftexversion\else\setmainfont{Times New Roman}\rmfamily\fi\fontsize{10.000000}{12.000000}\selectfont\catcode`\^=\active\def^{\ifmmode\sp\else\^{}\fi}\catcode`\%=\active\def%{\%}\ensuremath{-}0.5}}%
\end{pgfscope}%
\begin{pgfscope}%
\pgfsetbuttcap%
\pgfsetroundjoin%
\definecolor{currentfill}{rgb}{0.000000,0.000000,0.000000}%
\pgfsetfillcolor{currentfill}%
\pgfsetlinewidth{0.803000pt}%
\definecolor{currentstroke}{rgb}{0.000000,0.000000,0.000000}%
\pgfsetstrokecolor{currentstroke}%
\pgfsetdash{}{0pt}%
\pgfsys@defobject{currentmarker}{\pgfqpoint{-0.048611in}{0.000000in}}{\pgfqpoint{-0.000000in}{0.000000in}}{%
\pgfpathmoveto{\pgfqpoint{-0.000000in}{0.000000in}}%
\pgfpathlineto{\pgfqpoint{-0.048611in}{0.000000in}}%
\pgfusepath{stroke,fill}%
}%
\begin{pgfscope}%
\pgfsys@transformshift{0.647939in}{2.032092in}%
\pgfsys@useobject{currentmarker}{}%
\end{pgfscope}%
\end{pgfscope}%
\begin{pgfscope}%
\definecolor{textcolor}{rgb}{0.000000,0.000000,0.000000}%
\pgfsetstrokecolor{textcolor}%
\pgfsetfillcolor{textcolor}%
\pgftext[x=0.377106in, y=1.983874in, left, base]{\color{textcolor}{\ifdefined\pdftexversion\else\setmainfont{Times New Roman}\rmfamily\fi\fontsize{10.000000}{12.000000}\selectfont\catcode`\^=\active\def^{\ifmmode\sp\else\^{}\fi}\catcode`\%=\active\def%{\%}0.0}}%
\end{pgfscope}%
\begin{pgfscope}%
\pgfsetbuttcap%
\pgfsetroundjoin%
\definecolor{currentfill}{rgb}{0.000000,0.000000,0.000000}%
\pgfsetfillcolor{currentfill}%
\pgfsetlinewidth{0.803000pt}%
\definecolor{currentstroke}{rgb}{0.000000,0.000000,0.000000}%
\pgfsetstrokecolor{currentstroke}%
\pgfsetdash{}{0pt}%
\pgfsys@defobject{currentmarker}{\pgfqpoint{-0.048611in}{0.000000in}}{\pgfqpoint{-0.000000in}{0.000000in}}{%
\pgfpathmoveto{\pgfqpoint{-0.000000in}{0.000000in}}%
\pgfpathlineto{\pgfqpoint{-0.048611in}{0.000000in}}%
\pgfusepath{stroke,fill}%
}%
\begin{pgfscope}%
\pgfsys@transformshift{0.647939in}{2.417004in}%
\pgfsys@useobject{currentmarker}{}%
\end{pgfscope}%
\end{pgfscope}%
\begin{pgfscope}%
\definecolor{textcolor}{rgb}{0.000000,0.000000,0.000000}%
\pgfsetstrokecolor{textcolor}%
\pgfsetfillcolor{textcolor}%
\pgftext[x=0.377106in, y=2.368786in, left, base]{\color{textcolor}{\ifdefined\pdftexversion\else\setmainfont{Times New Roman}\rmfamily\fi\fontsize{10.000000}{12.000000}\selectfont\catcode`\^=\active\def^{\ifmmode\sp\else\^{}\fi}\catcode`\%=\active\def%{\%}0.5}}%
\end{pgfscope}%
\begin{pgfscope}%
\pgfsetbuttcap%
\pgfsetroundjoin%
\definecolor{currentfill}{rgb}{0.000000,0.000000,0.000000}%
\pgfsetfillcolor{currentfill}%
\pgfsetlinewidth{0.803000pt}%
\definecolor{currentstroke}{rgb}{0.000000,0.000000,0.000000}%
\pgfsetstrokecolor{currentstroke}%
\pgfsetdash{}{0pt}%
\pgfsys@defobject{currentmarker}{\pgfqpoint{-0.048611in}{0.000000in}}{\pgfqpoint{-0.000000in}{0.000000in}}{%
\pgfpathmoveto{\pgfqpoint{-0.000000in}{0.000000in}}%
\pgfpathlineto{\pgfqpoint{-0.048611in}{0.000000in}}%
\pgfusepath{stroke,fill}%
}%
\begin{pgfscope}%
\pgfsys@transformshift{0.647939in}{2.801916in}%
\pgfsys@useobject{currentmarker}{}%
\end{pgfscope}%
\end{pgfscope}%
\begin{pgfscope}%
\definecolor{textcolor}{rgb}{0.000000,0.000000,0.000000}%
\pgfsetstrokecolor{textcolor}%
\pgfsetfillcolor{textcolor}%
\pgftext[x=0.377106in, y=2.753699in, left, base]{\color{textcolor}{\ifdefined\pdftexversion\else\setmainfont{Times New Roman}\rmfamily\fi\fontsize{10.000000}{12.000000}\selectfont\catcode`\^=\active\def^{\ifmmode\sp\else\^{}\fi}\catcode`\%=\active\def%{\%}1.0}}%
\end{pgfscope}%
\begin{pgfscope}%
\pgfsetbuttcap%
\pgfsetroundjoin%
\definecolor{currentfill}{rgb}{0.000000,0.000000,0.000000}%
\pgfsetfillcolor{currentfill}%
\pgfsetlinewidth{0.803000pt}%
\definecolor{currentstroke}{rgb}{0.000000,0.000000,0.000000}%
\pgfsetstrokecolor{currentstroke}%
\pgfsetdash{}{0pt}%
\pgfsys@defobject{currentmarker}{\pgfqpoint{-0.048611in}{0.000000in}}{\pgfqpoint{-0.000000in}{0.000000in}}{%
\pgfpathmoveto{\pgfqpoint{-0.000000in}{0.000000in}}%
\pgfpathlineto{\pgfqpoint{-0.048611in}{0.000000in}}%
\pgfusepath{stroke,fill}%
}%
\begin{pgfscope}%
\pgfsys@transformshift{0.647939in}{3.186829in}%
\pgfsys@useobject{currentmarker}{}%
\end{pgfscope}%
\end{pgfscope}%
\begin{pgfscope}%
\definecolor{textcolor}{rgb}{0.000000,0.000000,0.000000}%
\pgfsetstrokecolor{textcolor}%
\pgfsetfillcolor{textcolor}%
\pgftext[x=0.377106in, y=3.138611in, left, base]{\color{textcolor}{\ifdefined\pdftexversion\else\setmainfont{Times New Roman}\rmfamily\fi\fontsize{10.000000}{12.000000}\selectfont\catcode`\^=\active\def^{\ifmmode\sp\else\^{}\fi}\catcode`\%=\active\def%{\%}1.5}}%
\end{pgfscope}%
\begin{pgfscope}%
\pgfsetbuttcap%
\pgfsetroundjoin%
\definecolor{currentfill}{rgb}{0.000000,0.000000,0.000000}%
\pgfsetfillcolor{currentfill}%
\pgfsetlinewidth{0.803000pt}%
\definecolor{currentstroke}{rgb}{0.000000,0.000000,0.000000}%
\pgfsetstrokecolor{currentstroke}%
\pgfsetdash{}{0pt}%
\pgfsys@defobject{currentmarker}{\pgfqpoint{-0.048611in}{0.000000in}}{\pgfqpoint{-0.000000in}{0.000000in}}{%
\pgfpathmoveto{\pgfqpoint{-0.000000in}{0.000000in}}%
\pgfpathlineto{\pgfqpoint{-0.048611in}{0.000000in}}%
\pgfusepath{stroke,fill}%
}%
\begin{pgfscope}%
\pgfsys@transformshift{0.647939in}{3.571741in}%
\pgfsys@useobject{currentmarker}{}%
\end{pgfscope}%
\end{pgfscope}%
\begin{pgfscope}%
\definecolor{textcolor}{rgb}{0.000000,0.000000,0.000000}%
\pgfsetstrokecolor{textcolor}%
\pgfsetfillcolor{textcolor}%
\pgftext[x=0.377106in, y=3.523524in, left, base]{\color{textcolor}{\ifdefined\pdftexversion\else\setmainfont{Times New Roman}\rmfamily\fi\fontsize{10.000000}{12.000000}\selectfont\catcode`\^=\active\def^{\ifmmode\sp\else\^{}\fi}\catcode`\%=\active\def%{\%}2.0}}%
\end{pgfscope}%
\begin{pgfscope}%
\definecolor{textcolor}{rgb}{0.000000,0.000000,0.000000}%
\pgfsetstrokecolor{textcolor}%
\pgfsetfillcolor{textcolor}%
\pgftext[x=0.213525in,y=2.032092in,,bottom,rotate=90.000000]{\color{textcolor}{\ifdefined\pdftexversion\else\setmainfont{Times New Roman}\rmfamily\fi\fontsize{9.000000}{10.800000}\selectfont\catcode`\^=\active\def^{\ifmmode\sp\else\^{}\fi}\catcode`\%=\active\def%{\%}$x_2$}}%
\end{pgfscope}%
\begin{pgfscope}%
\pgfpathrectangle{\pgfqpoint{0.647939in}{0.492442in}}{\pgfqpoint{3.079299in}{3.079299in}}%
\pgfusepath{clip}%
\pgfsetbuttcap%
\pgfsetroundjoin%
\pgfsetlinewidth{0.803000pt}%
\definecolor{currentstroke}{rgb}{0.501961,0.501961,0.501961}%
\pgfsetstrokecolor{currentstroke}%
\pgfsetdash{}{0pt}%
\pgfpathmoveto{\pgfqpoint{0.647939in}{0.492442in}}%
\pgfpathlineto{\pgfqpoint{0.714873in}{0.506580in}}%
\pgfpathlineto{\pgfqpoint{0.781065in}{0.523833in}}%
\pgfpathlineto{\pgfqpoint{0.846238in}{0.544580in}}%
\pgfpathlineto{\pgfqpoint{0.910052in}{0.569161in}}%
\pgfpathlineto{\pgfqpoint{0.972117in}{0.597852in}}%
\pgfpathlineto{\pgfqpoint{1.032006in}{0.630823in}}%
\pgfpathlineto{\pgfqpoint{1.089302in}{0.668108in}}%
\pgfpathlineto{\pgfqpoint{1.143645in}{0.709591in}}%
\pgfpathlineto{\pgfqpoint{1.194792in}{0.754943in}}%
\pgfpathlineto{\pgfqpoint{1.242655in}{0.803739in}}%
\pgfpathlineto{\pgfqpoint{1.287330in}{0.855477in}}%
\pgfpathlineto{\pgfqpoint{1.329069in}{0.909621in}}%
\pgfpathlineto{\pgfqpoint{1.368259in}{0.965647in}}%
\pgfpathlineto{\pgfqpoint{1.405358in}{1.023069in}}%
\pgfpathlineto{\pgfqpoint{1.440847in}{1.081485in}}%
\pgfpathlineto{\pgfqpoint{1.508880in}{1.200041in}}%
\pgfpathlineto{\pgfqpoint{1.609815in}{1.378530in}}%
\pgfpathlineto{\pgfqpoint{1.644597in}{1.437394in}}%
\pgfpathlineto{\pgfqpoint{1.680469in}{1.495573in}}%
\pgfpathlineto{\pgfqpoint{1.717719in}{1.552840in}}%
\pgfpathlineto{\pgfqpoint{1.756637in}{1.608977in}}%
\pgfpathlineto{\pgfqpoint{1.797496in}{1.663707in}}%
\pgfpathlineto{\pgfqpoint{1.840559in}{1.716647in}}%
\pgfpathlineto{\pgfqpoint{1.886120in}{1.767439in}}%
\pgfpathlineto{\pgfqpoint{1.934303in}{1.815605in}}%
\pgfpathlineto{\pgfqpoint{1.985178in}{1.860815in}}%
\pgfpathlineto{\pgfqpoint{2.123887in}{1.972815in}}%
\pgfpathlineto{\pgfqpoint{2.133219in}{1.980825in}}%
\pgfpathlineto{\pgfqpoint{2.133219in}{1.980825in}}%
\pgfpathlineto{\pgfqpoint{2.147810in}{1.995011in}}%
\pgfpathlineto{\pgfqpoint{2.155386in}{2.001765in}}%
\pgfpathlineto{\pgfqpoint{2.157083in}{2.006656in}}%
\pgfpathlineto{\pgfqpoint{2.157083in}{2.006656in}}%
\pgfpathlineto{\pgfqpoint{2.160586in}{2.008591in}}%
\pgfpathlineto{\pgfqpoint{2.161686in}{2.010768in}}%
\pgfpathlineto{\pgfqpoint{2.161686in}{2.010768in}}%
\pgfpathlineto{\pgfqpoint{2.166157in}{2.013683in}}%
\pgfpathlineto{\pgfqpoint{2.166157in}{2.013683in}}%
\pgfpathlineto{\pgfqpoint{2.163667in}{2.015541in}}%
\pgfpathlineto{\pgfqpoint{2.163667in}{2.015541in}}%
\pgfpathlineto{\pgfqpoint{2.166424in}{2.017629in}}%
\pgfpathlineto{\pgfqpoint{2.175030in}{2.020342in}}%
\pgfpathlineto{\pgfqpoint{2.175375in}{2.021176in}}%
\pgfpathlineto{\pgfqpoint{2.174011in}{2.022262in}}%
\pgfpathlineto{\pgfqpoint{2.174011in}{2.022262in}}%
\pgfpathlineto{\pgfqpoint{2.177951in}{2.024701in}}%
\pgfpathlineto{\pgfqpoint{2.177077in}{2.025538in}}%
\pgfpathlineto{\pgfqpoint{2.177077in}{2.025538in}}%
\pgfpathlineto{\pgfqpoint{2.180656in}{2.027957in}}%
\pgfpathlineto{\pgfqpoint{2.180057in}{2.028722in}}%
\pgfpathlineto{\pgfqpoint{2.180057in}{2.028722in}}%
\pgfpathlineto{\pgfqpoint{2.183400in}{2.031156in}}%
\pgfpathlineto{\pgfqpoint{2.183279in}{2.032539in}}%
\pgfpathlineto{\pgfqpoint{2.186861in}{2.034960in}}%
\pgfpathlineto{\pgfqpoint{2.186652in}{2.036345in}}%
\pgfpathlineto{\pgfqpoint{2.190286in}{2.038096in}}%
\pgfpathlineto{\pgfqpoint{2.189991in}{2.038822in}}%
\pgfpathlineto{\pgfqpoint{2.189991in}{2.038822in}}%
\pgfpathlineto{\pgfqpoint{2.191508in}{2.040789in}}%
\pgfpathlineto{\pgfqpoint{2.192369in}{2.042165in}}%
\pgfpathlineto{\pgfqpoint{2.193677in}{2.043562in}}%
\pgfpathlineto{\pgfqpoint{2.193837in}{2.046291in}}%
\pgfpathlineto{\pgfqpoint{2.195072in}{2.047804in}}%
\pgfpathlineto{\pgfqpoint{2.208145in}{2.053944in}}%
\pgfpathlineto{\pgfqpoint{2.210860in}{2.059279in}}%
\pgfpathlineto{\pgfqpoint{2.219256in}{2.067853in}}%
\pgfpathlineto{\pgfqpoint{2.221706in}{2.073493in}}%
\pgfpathlineto{\pgfqpoint{2.221706in}{2.073493in}}%
\pgfpathlineto{\pgfqpoint{2.244990in}{2.092653in}}%
\pgfpathlineto{\pgfqpoint{2.254731in}{2.103216in}}%
\pgfpathlineto{\pgfqpoint{2.276864in}{2.131087in}}%
\pgfpathlineto{\pgfqpoint{2.276864in}{2.131087in}}%
\pgfpathlineto{\pgfqpoint{2.326565in}{2.184899in}}%
\pgfpathlineto{\pgfqpoint{2.358367in}{2.219740in}}%
\pgfpathlineto{\pgfqpoint{2.445681in}{2.321559in}}%
\pgfpathlineto{\pgfqpoint{2.531969in}{2.426563in}}%
\pgfpathlineto{\pgfqpoint{2.659764in}{2.586656in}}%
\pgfpathlineto{\pgfqpoint{2.914488in}{2.908356in}}%
\pgfpathlineto{\pgfqpoint{3.000817in}{3.014506in}}%
\pgfpathlineto{\pgfqpoint{3.088670in}{3.119416in}}%
\pgfpathlineto{\pgfqpoint{3.178596in}{3.222560in}}%
\pgfpathlineto{\pgfqpoint{3.271194in}{3.323312in}}%
\pgfpathlineto{\pgfqpoint{3.318702in}{3.372554in}}%
\pgfpathlineto{\pgfqpoint{3.367133in}{3.420886in}}%
\pgfpathlineto{\pgfqpoint{3.416580in}{3.468179in}}%
\pgfpathlineto{\pgfqpoint{3.467142in}{3.514275in}}%
\pgfpathlineto{\pgfqpoint{3.534099in}{3.571741in}}%
\pgfpathlineto{\pgfqpoint{3.534099in}{3.571741in}}%
\pgfusepath{stroke}%
\end{pgfscope}%
\begin{pgfscope}%
\pgfpathrectangle{\pgfqpoint{0.647939in}{0.492442in}}{\pgfqpoint{3.079299in}{3.079299in}}%
\pgfusepath{clip}%
\pgfsetbuttcap%
\pgfsetroundjoin%
\pgfsetlinewidth{0.803000pt}%
\definecolor{currentstroke}{rgb}{0.501961,0.501961,0.501961}%
\pgfsetstrokecolor{currentstroke}%
\pgfsetdash{}{0pt}%
\pgfpathmoveto{\pgfqpoint{0.857891in}{0.492442in}}%
\pgfpathlineto{\pgfqpoint{0.857891in}{0.492442in}}%
\pgfpathlineto{\pgfqpoint{0.921153in}{0.518406in}}%
\pgfpathlineto{\pgfqpoint{0.982456in}{0.548682in}}%
\pgfpathlineto{\pgfqpoint{1.041337in}{0.583413in}}%
\pgfpathlineto{\pgfqpoint{1.097349in}{0.622580in}}%
\pgfpathlineto{\pgfqpoint{1.150134in}{0.665993in}}%
\pgfpathlineto{\pgfqpoint{1.199487in}{0.713282in}}%
\pgfpathlineto{\pgfqpoint{1.245390in}{0.763916in}}%
\pgfusepath{stroke}%
\end{pgfscope}%
\begin{pgfscope}%
\pgfpathrectangle{\pgfqpoint{0.647939in}{0.492442in}}{\pgfqpoint{3.079299in}{3.079299in}}%
\pgfusepath{clip}%
\pgfsetbuttcap%
\pgfsetroundjoin%
\pgfsetlinewidth{0.803000pt}%
\definecolor{currentstroke}{rgb}{0.501961,0.501961,0.501961}%
\pgfsetstrokecolor{currentstroke}%
\pgfsetdash{}{0pt}%
\pgfpathmoveto{\pgfqpoint{1.137828in}{0.492442in}}%
\pgfpathlineto{\pgfqpoint{1.137828in}{0.492442in}}%
\pgfpathlineto{\pgfqpoint{1.182747in}{0.543916in}}%
\pgfpathlineto{\pgfqpoint{1.223785in}{0.598535in}}%
\pgfpathlineto{\pgfqpoint{1.261369in}{0.655592in}}%
\pgfpathlineto{\pgfqpoint{1.296063in}{0.714484in}}%
\pgfpathlineto{\pgfqpoint{1.328474in}{0.774689in}}%
\pgfpathlineto{\pgfqpoint{1.359185in}{0.835787in}}%
\pgfpathlineto{\pgfqpoint{1.388723in}{0.897454in}}%
\pgfpathlineto{\pgfqpoint{1.417542in}{0.959444in}}%
\pgfusepath{stroke}%
\end{pgfscope}%
\begin{pgfscope}%
\pgfpathrectangle{\pgfqpoint{0.647939in}{0.492442in}}{\pgfqpoint{3.079299in}{3.079299in}}%
\pgfusepath{clip}%
\pgfsetbuttcap%
\pgfsetroundjoin%
\pgfsetlinewidth{0.803000pt}%
\definecolor{currentstroke}{rgb}{0.501961,0.501961,0.501961}%
\pgfsetstrokecolor{currentstroke}%
\pgfsetdash{}{0pt}%
\pgfpathmoveto{\pgfqpoint{1.347780in}{0.492442in}}%
\pgfpathlineto{\pgfqpoint{1.347780in}{0.492442in}}%
\pgfpathlineto{\pgfqpoint{1.349982in}{0.560510in}}%
\pgfpathlineto{\pgfqpoint{1.357952in}{0.628228in}}%
\pgfpathlineto{\pgfqpoint{1.370127in}{0.695415in}}%
\pgfpathlineto{\pgfqpoint{1.385462in}{0.761934in}}%
\pgfpathlineto{\pgfqpoint{1.403239in}{0.827862in}}%
\pgfusepath{stroke}%
\end{pgfscope}%
\begin{pgfscope}%
\pgfpathrectangle{\pgfqpoint{0.647939in}{0.492442in}}{\pgfqpoint{3.079299in}{3.079299in}}%
\pgfusepath{clip}%
\pgfsetbuttcap%
\pgfsetroundjoin%
\pgfsetlinewidth{0.803000pt}%
\definecolor{currentstroke}{rgb}{0.501961,0.501961,0.501961}%
\pgfsetstrokecolor{currentstroke}%
\pgfsetdash{}{0pt}%
\pgfpathmoveto{\pgfqpoint{1.627716in}{0.492442in}}%
\pgfpathlineto{\pgfqpoint{1.627716in}{0.492442in}}%
\pgfpathlineto{\pgfqpoint{1.569179in}{0.527197in}}%
\pgfpathlineto{\pgfqpoint{1.519242in}{0.573120in}}%
\pgfpathlineto{\pgfqpoint{1.484975in}{0.623570in}}%
\pgfpathlineto{\pgfqpoint{1.463667in}{0.676411in}}%
\pgfpathlineto{\pgfqpoint{1.452129in}{0.733275in}}%
\pgfusepath{stroke}%
\end{pgfscope}%
\begin{pgfscope}%
\pgfpathrectangle{\pgfqpoint{0.647939in}{0.492442in}}{\pgfqpoint{3.079299in}{3.079299in}}%
\pgfusepath{clip}%
\pgfsetbuttcap%
\pgfsetroundjoin%
\pgfsetlinewidth{0.803000pt}%
\definecolor{currentstroke}{rgb}{0.501961,0.501961,0.501961}%
\pgfsetstrokecolor{currentstroke}%
\pgfsetdash{}{0pt}%
\pgfpathmoveto{\pgfqpoint{1.837668in}{0.492442in}}%
\pgfpathlineto{\pgfqpoint{1.837668in}{0.492442in}}%
\pgfpathlineto{\pgfqpoint{1.770286in}{0.504080in}}%
\pgfpathlineto{\pgfqpoint{1.704200in}{0.521476in}}%
\pgfpathlineto{\pgfqpoint{1.640720in}{0.546547in}}%
\pgfpathlineto{\pgfqpoint{1.582279in}{0.581499in}}%
\pgfpathlineto{\pgfqpoint{1.532862in}{0.627947in}}%
\pgfusepath{stroke}%
\end{pgfscope}%
\begin{pgfscope}%
\pgfpathrectangle{\pgfqpoint{0.647939in}{0.492442in}}{\pgfqpoint{3.079299in}{3.079299in}}%
\pgfusepath{clip}%
\pgfsetbuttcap%
\pgfsetroundjoin%
\pgfsetlinewidth{0.803000pt}%
\definecolor{currentstroke}{rgb}{0.501961,0.501961,0.501961}%
\pgfsetstrokecolor{currentstroke}%
\pgfsetdash{}{0pt}%
\pgfpathmoveto{\pgfqpoint{2.257573in}{0.492442in}}%
\pgfpathlineto{\pgfqpoint{2.257573in}{0.492442in}}%
\pgfpathlineto{\pgfqpoint{2.189144in}{0.492589in}}%
\pgfpathlineto{\pgfqpoint{2.120716in}{0.492738in}}%
\pgfpathlineto{\pgfqpoint{2.052292in}{0.493462in}}%
\pgfpathlineto{\pgfqpoint{1.983896in}{0.495426in}}%
\pgfpathlineto{\pgfqpoint{1.915597in}{0.499419in}}%
\pgfusepath{stroke}%
\end{pgfscope}%
\begin{pgfscope}%
\pgfpathrectangle{\pgfqpoint{0.647939in}{0.492442in}}{\pgfqpoint{3.079299in}{3.079299in}}%
\pgfusepath{clip}%
\pgfsetbuttcap%
\pgfsetroundjoin%
\pgfsetlinewidth{0.803000pt}%
\definecolor{currentstroke}{rgb}{0.501961,0.501961,0.501961}%
\pgfsetstrokecolor{currentstroke}%
\pgfsetdash{}{0pt}%
\pgfpathmoveto{\pgfqpoint{2.677477in}{0.492442in}}%
\pgfpathlineto{\pgfqpoint{2.677477in}{0.492442in}}%
\pgfpathlineto{\pgfqpoint{2.609609in}{0.501120in}}%
\pgfpathlineto{\pgfqpoint{2.541516in}{0.507814in}}%
\pgfpathlineto{\pgfqpoint{2.473264in}{0.512646in}}%
\pgfpathlineto{\pgfqpoint{2.404914in}{0.515821in}}%
\pgfpathlineto{\pgfqpoint{2.336512in}{0.517619in}}%
\pgfusepath{stroke}%
\end{pgfscope}%
\begin{pgfscope}%
\pgfpathrectangle{\pgfqpoint{0.647939in}{0.492442in}}{\pgfqpoint{3.079299in}{3.079299in}}%
\pgfusepath{clip}%
\pgfsetbuttcap%
\pgfsetroundjoin%
\pgfsetlinewidth{0.803000pt}%
\definecolor{currentstroke}{rgb}{0.501961,0.501961,0.501961}%
\pgfsetstrokecolor{currentstroke}%
\pgfsetdash{}{0pt}%
\pgfpathmoveto{\pgfqpoint{2.887429in}{0.492442in}}%
\pgfpathlineto{\pgfqpoint{2.887429in}{0.492442in}}%
\pgfpathlineto{\pgfqpoint{2.820684in}{0.507496in}}%
\pgfpathlineto{\pgfqpoint{2.753544in}{0.520673in}}%
\pgfpathlineto{\pgfqpoint{2.686044in}{0.531854in}}%
\pgfpathlineto{\pgfqpoint{2.618237in}{0.540983in}}%
\pgfpathlineto{\pgfqpoint{2.550186in}{0.548082in}}%
\pgfpathlineto{\pgfqpoint{2.481959in}{0.553257in}}%
\pgfpathlineto{\pgfqpoint{2.413623in}{0.556705in}}%
\pgfpathlineto{\pgfqpoint{2.345227in}{0.558709in}}%
\pgfpathlineto{\pgfqpoint{2.276806in}{0.559621in}}%
\pgfpathlineto{\pgfqpoint{2.208378in}{0.559871in}}%
\pgfpathlineto{\pgfqpoint{2.139949in}{0.559959in}}%
\pgfpathlineto{\pgfqpoint{2.071523in}{0.560478in}}%
\pgfpathlineto{\pgfqpoint{2.003118in}{0.562107in}}%
\pgfpathlineto{\pgfqpoint{1.934793in}{0.565637in}}%
\pgfpathlineto{\pgfqpoint{1.866687in}{0.572026in}}%
\pgfpathlineto{\pgfqpoint{1.799104in}{0.582472in}}%
\pgfpathlineto{\pgfqpoint{1.732677in}{0.598531in}}%
\pgfpathlineto{\pgfqpoint{1.668672in}{0.622211in}}%
\pgfpathlineto{\pgfqpoint{1.609518in}{0.655890in}}%
\pgfpathlineto{\pgfqpoint{1.559321in}{0.701474in}}%
\pgfpathlineto{\pgfqpoint{1.526389in}{0.750446in}}%
\pgfpathlineto{\pgfqpoint{1.506351in}{0.801795in}}%
\pgfpathlineto{\pgfqpoint{1.495942in}{0.857255in}}%
\pgfpathlineto{\pgfqpoint{1.494037in}{0.919074in}}%
\pgfpathlineto{\pgfqpoint{1.500411in}{0.986885in}}%
\pgfpathlineto{\pgfqpoint{1.513066in}{1.053876in}}%
\pgfpathlineto{\pgfqpoint{1.530341in}{1.119906in}}%
\pgfpathlineto{\pgfqpoint{1.551178in}{1.184945in}}%
\pgfusepath{stroke}%
\end{pgfscope}%
\begin{pgfscope}%
\pgfpathrectangle{\pgfqpoint{0.647939in}{0.492442in}}{\pgfqpoint{3.079299in}{3.079299in}}%
\pgfusepath{clip}%
\pgfsetbuttcap%
\pgfsetroundjoin%
\pgfsetlinewidth{0.803000pt}%
\definecolor{currentstroke}{rgb}{0.501961,0.501961,0.501961}%
\pgfsetstrokecolor{currentstroke}%
\pgfsetdash{}{0pt}%
\pgfpathmoveto{\pgfqpoint{3.097382in}{0.492442in}}%
\pgfpathlineto{\pgfqpoint{3.097382in}{0.492442in}}%
\pgfpathlineto{\pgfqpoint{3.032003in}{0.512636in}}%
\pgfpathlineto{\pgfqpoint{2.966267in}{0.531626in}}%
\pgfpathlineto{\pgfqpoint{2.900129in}{0.549160in}}%
\pgfpathlineto{\pgfqpoint{2.833569in}{0.565008in}}%
\pgfpathlineto{\pgfqpoint{2.766589in}{0.578976in}}%
\pgfusepath{stroke}%
\end{pgfscope}%
\begin{pgfscope}%
\pgfpathrectangle{\pgfqpoint{0.647939in}{0.492442in}}{\pgfqpoint{3.079299in}{3.079299in}}%
\pgfusepath{clip}%
\pgfsetbuttcap%
\pgfsetroundjoin%
\pgfsetlinewidth{0.803000pt}%
\definecolor{currentstroke}{rgb}{0.501961,0.501961,0.501961}%
\pgfsetstrokecolor{currentstroke}%
\pgfsetdash{}{0pt}%
\pgfpathmoveto{\pgfqpoint{3.307334in}{0.492442in}}%
\pgfpathlineto{\pgfqpoint{3.307334in}{0.492442in}}%
\pgfpathlineto{\pgfqpoint{3.242848in}{0.515333in}}%
\pgfpathlineto{\pgfqpoint{3.178253in}{0.537917in}}%
\pgfpathlineto{\pgfqpoint{3.113458in}{0.559914in}}%
\pgfpathlineto{\pgfqpoint{3.048377in}{0.581045in}}%
\pgfpathlineto{\pgfqpoint{2.982936in}{0.601035in}}%
\pgfpathlineto{\pgfqpoint{2.917083in}{0.619615in}}%
\pgfpathlineto{\pgfqpoint{2.850787in}{0.636536in}}%
\pgfpathlineto{\pgfqpoint{2.784041in}{0.651581in}}%
\pgfpathlineto{\pgfqpoint{2.716869in}{0.664583in}}%
\pgfpathlineto{\pgfqpoint{2.649318in}{0.675436in}}%
\pgfpathlineto{\pgfqpoint{2.581451in}{0.684108in}}%
\pgfpathlineto{\pgfqpoint{2.513344in}{0.690656in}}%
\pgfpathlineto{\pgfqpoint{2.445076in}{0.695239in}}%
\pgfpathlineto{\pgfqpoint{2.376712in}{0.698109in}}%
\pgfpathlineto{\pgfqpoint{2.308303in}{0.699604in}}%
\pgfpathlineto{\pgfqpoint{2.239877in}{0.700148in}}%
\pgfpathlineto{\pgfqpoint{2.171449in}{0.700250in}}%
\pgfpathlineto{\pgfqpoint{2.103021in}{0.700520in}}%
\pgfpathlineto{\pgfqpoint{2.034605in}{0.701672in}}%
\pgfpathlineto{\pgfqpoint{1.966247in}{0.704546in}}%
\pgfpathlineto{\pgfqpoint{1.898073in}{0.710171in}}%
\pgfpathlineto{\pgfqpoint{1.830381in}{0.719869in}}%
\pgfpathlineto{\pgfqpoint{1.763838in}{0.735397in}}%
\pgfpathlineto{\pgfqpoint{1.699874in}{0.759092in}}%
\pgfpathlineto{\pgfqpoint{1.641396in}{0.793750in}}%
\pgfpathlineto{\pgfqpoint{1.595179in}{0.838874in}}%
\pgfpathlineto{\pgfqpoint{1.566266in}{0.886387in}}%
\pgfpathlineto{\pgfqpoint{1.549377in}{0.936493in}}%
\pgfpathlineto{\pgfqpoint{1.541729in}{0.991140in}}%
\pgfpathlineto{\pgfqpoint{1.542576in}{1.051951in}}%
\pgfusepath{stroke}%
\end{pgfscope}%
\begin{pgfscope}%
\pgfpathrectangle{\pgfqpoint{0.647939in}{0.492442in}}{\pgfqpoint{3.079299in}{3.079299in}}%
\pgfusepath{clip}%
\pgfsetbuttcap%
\pgfsetroundjoin%
\pgfsetlinewidth{0.803000pt}%
\definecolor{currentstroke}{rgb}{0.501961,0.501961,0.501961}%
\pgfsetstrokecolor{currentstroke}%
\pgfsetdash{}{0pt}%
\pgfpathmoveto{\pgfqpoint{3.517286in}{0.492442in}}%
\pgfpathlineto{\pgfqpoint{3.517286in}{0.492442in}}%
\pgfpathlineto{\pgfqpoint{3.452737in}{0.515152in}}%
\pgfpathlineto{\pgfqpoint{3.388397in}{0.538451in}}%
\pgfpathlineto{\pgfqpoint{3.324177in}{0.562078in}}%
\pgfpathlineto{\pgfqpoint{3.259979in}{0.585765in}}%
\pgfpathlineto{\pgfqpoint{3.195701in}{0.609234in}}%
\pgfpathlineto{\pgfqpoint{3.131243in}{0.632203in}}%
\pgfpathlineto{\pgfqpoint{3.066513in}{0.654388in}}%
\pgfpathlineto{\pgfqpoint{3.001427in}{0.675504in}}%
\pgfpathlineto{\pgfqpoint{2.935922in}{0.695271in}}%
\pgfpathlineto{\pgfqpoint{2.869952in}{0.713423in}}%
\pgfpathlineto{\pgfqpoint{2.803502in}{0.729721in}}%
\pgfpathlineto{\pgfqpoint{2.736582in}{0.743965in}}%
\pgfpathlineto{\pgfqpoint{2.669233in}{0.756012in}}%
\pgfpathlineto{\pgfqpoint{2.601517in}{0.765797in}}%
\pgfpathlineto{\pgfqpoint{2.533517in}{0.773346in}}%
\pgfpathlineto{\pgfqpoint{2.465313in}{0.778780in}}%
\pgfpathlineto{\pgfqpoint{2.396982in}{0.782320in}}%
\pgfpathlineto{\pgfqpoint{2.328586in}{0.784286in}}%
\pgfpathlineto{\pgfqpoint{2.260163in}{0.785104in}}%
\pgfpathlineto{\pgfqpoint{2.191735in}{0.785294in}}%
\pgfpathlineto{\pgfqpoint{2.123306in}{0.785467in}}%
\pgfpathlineto{\pgfqpoint{2.054886in}{0.786341in}}%
\pgfpathlineto{\pgfqpoint{1.986508in}{0.788783in}}%
\pgfpathlineto{\pgfqpoint{1.918291in}{0.793880in}}%
\pgfpathlineto{\pgfqpoint{1.850530in}{0.803040in}}%
\pgfpathlineto{\pgfqpoint{1.783908in}{0.818147in}}%
\pgfpathlineto{\pgfqpoint{1.719951in}{0.841761in}}%
\pgfusepath{stroke}%
\end{pgfscope}%
\begin{pgfscope}%
\pgfpathrectangle{\pgfqpoint{0.647939in}{0.492442in}}{\pgfqpoint{3.079299in}{3.079299in}}%
\pgfusepath{clip}%
\pgfsetbuttcap%
\pgfsetroundjoin%
\pgfsetlinewidth{0.803000pt}%
\definecolor{currentstroke}{rgb}{0.501961,0.501961,0.501961}%
\pgfsetstrokecolor{currentstroke}%
\pgfsetdash{}{0pt}%
\pgfpathmoveto{\pgfqpoint{3.727238in}{0.492442in}}%
\pgfpathlineto{\pgfqpoint{3.727238in}{0.492442in}}%
\pgfpathlineto{\pgfqpoint{3.661717in}{0.512162in}}%
\pgfpathlineto{\pgfqpoint{3.596628in}{0.533266in}}%
\pgfpathlineto{\pgfqpoint{3.531926in}{0.555533in}}%
\pgfpathlineto{\pgfqpoint{3.467549in}{0.578726in}}%
\pgfpathlineto{\pgfqpoint{3.403419in}{0.602594in}}%
\pgfpathlineto{\pgfqpoint{3.339444in}{0.626878in}}%
\pgfpathlineto{\pgfqpoint{3.275525in}{0.651309in}}%
\pgfpathlineto{\pgfqpoint{3.211558in}{0.675612in}}%
\pgfpathlineto{\pgfqpoint{3.147436in}{0.699504in}}%
\pgfpathlineto{\pgfqpoint{3.083060in}{0.722697in}}%
\pgfpathlineto{\pgfqpoint{3.018337in}{0.744900in}}%
\pgfpathlineto{\pgfqpoint{2.953191in}{0.765823in}}%
\pgfpathlineto{\pgfqpoint{2.887564in}{0.785183in}}%
\pgfpathlineto{\pgfqpoint{2.821428in}{0.802718in}}%
\pgfpathlineto{\pgfqpoint{2.754784in}{0.818200in}}%
\pgfpathlineto{\pgfqpoint{2.687664in}{0.831456in}}%
\pgfpathlineto{\pgfqpoint{2.620125in}{0.842381in}}%
\pgfpathlineto{\pgfqpoint{2.552249in}{0.850962in}}%
\pgfpathlineto{\pgfqpoint{2.484122in}{0.857281in}}%
\pgfpathlineto{\pgfqpoint{2.415833in}{0.861538in}}%
\pgfpathlineto{\pgfqpoint{2.347455in}{0.864044in}}%
\pgfpathlineto{\pgfqpoint{2.279039in}{0.865208in}}%
\pgfpathlineto{\pgfqpoint{2.210611in}{0.865537in}}%
\pgfpathlineto{\pgfqpoint{2.142183in}{0.865647in}}%
\pgfpathlineto{\pgfqpoint{2.073759in}{0.866283in}}%
\pgfpathlineto{\pgfqpoint{2.005369in}{0.868355in}}%
\pgfpathlineto{\pgfqpoint{1.937120in}{0.872988in}}%
\pgfpathlineto{\pgfqpoint{1.869295in}{0.881655in}}%
\pgfpathlineto{\pgfqpoint{1.802584in}{0.896387in}}%
\pgfpathlineto{\pgfqpoint{1.738625in}{0.920026in}}%
\pgfpathlineto{\pgfqpoint{1.681124in}{0.956065in}}%
\pgfpathlineto{\pgfqpoint{1.640541in}{0.999920in}}%
\pgfpathlineto{\pgfqpoint{1.616441in}{1.045653in}}%
\pgfpathlineto{\pgfqpoint{1.603590in}{1.093914in}}%
\pgfpathlineto{\pgfqpoint{1.599589in}{1.146944in}}%
\pgfpathlineto{\pgfqpoint{1.604154in}{1.206866in}}%
\pgfpathlineto{\pgfqpoint{1.617563in}{1.273690in}}%
\pgfpathlineto{\pgfqpoint{1.637146in}{1.338996in}}%
\pgfusepath{stroke}%
\end{pgfscope}%
\begin{pgfscope}%
\pgfpathrectangle{\pgfqpoint{0.647939in}{0.492442in}}{\pgfqpoint{3.079299in}{3.079299in}}%
\pgfusepath{clip}%
\pgfsetbuttcap%
\pgfsetroundjoin%
\pgfsetlinewidth{0.803000pt}%
\definecolor{currentstroke}{rgb}{0.501961,0.501961,0.501961}%
\pgfsetstrokecolor{currentstroke}%
\pgfsetdash{}{0pt}%
\pgfpathmoveto{\pgfqpoint{3.727238in}{0.562426in}}%
\pgfpathlineto{\pgfqpoint{3.727238in}{0.562426in}}%
\pgfpathlineto{\pgfqpoint{3.661900in}{0.582741in}}%
\pgfpathlineto{\pgfqpoint{3.597021in}{0.604485in}}%
\pgfpathlineto{\pgfqpoint{3.532559in}{0.627435in}}%
\pgfpathlineto{\pgfqpoint{3.468447in}{0.651350in}}%
\pgfpathlineto{\pgfqpoint{3.404605in}{0.675979in}}%
\pgfpathlineto{\pgfqpoint{3.340939in}{0.701061in}}%
\pgfpathlineto{\pgfqpoint{3.277345in}{0.726326in}}%
\pgfpathlineto{\pgfqpoint{3.213715in}{0.751497in}}%
\pgfpathlineto{\pgfqpoint{3.149936in}{0.776290in}}%
\pgfpathlineto{\pgfqpoint{3.085902in}{0.800412in}}%
\pgfpathlineto{\pgfqpoint{3.021514in}{0.823567in}}%
\pgfpathlineto{\pgfqpoint{2.956685in}{0.845455in}}%
\pgfpathlineto{\pgfqpoint{2.891353in}{0.865782in}}%
\pgfpathlineto{\pgfqpoint{2.825477in}{0.884270in}}%
\pgfpathlineto{\pgfqpoint{2.759053in}{0.900670in}}%
\pgfpathlineto{\pgfqpoint{2.692108in}{0.914785in}}%
\pgfpathlineto{\pgfqpoint{2.624701in}{0.926488in}}%
\pgfpathlineto{\pgfqpoint{2.556914in}{0.935743in}}%
\pgfpathlineto{\pgfqpoint{2.488843in}{0.942617in}}%
\pgfpathlineto{\pgfqpoint{2.420583in}{0.947292in}}%
\pgfpathlineto{\pgfqpoint{2.352216in}{0.950080in}}%
\pgfpathlineto{\pgfqpoint{2.283803in}{0.951408in}}%
\pgfpathlineto{\pgfqpoint{2.215377in}{0.951806in}}%
\pgfpathlineto{\pgfqpoint{2.146948in}{0.951923in}}%
\pgfpathlineto{\pgfqpoint{2.078523in}{0.952546in}}%
\pgfpathlineto{\pgfqpoint{2.010136in}{0.954663in}}%
\pgfpathlineto{\pgfqpoint{1.941908in}{0.959537in}}%
\pgfpathlineto{\pgfqpoint{1.874184in}{0.968859in}}%
\pgfpathlineto{\pgfqpoint{1.807854in}{0.985032in}}%
\pgfpathlineto{\pgfqpoint{1.745147in}{1.011474in}}%
\pgfpathlineto{\pgfqpoint{1.745147in}{1.011474in}}%
\pgfpathlineto{\pgfqpoint{1.700243in}{1.042878in}}%
\pgfpathlineto{\pgfqpoint{1.664378in}{1.085368in}}%
\pgfusepath{stroke}%
\end{pgfscope}%
\begin{pgfscope}%
\pgfpathrectangle{\pgfqpoint{0.647939in}{0.492442in}}{\pgfqpoint{3.079299in}{3.079299in}}%
\pgfusepath{clip}%
\pgfsetbuttcap%
\pgfsetroundjoin%
\pgfsetlinewidth{0.803000pt}%
\definecolor{currentstroke}{rgb}{0.501961,0.501961,0.501961}%
\pgfsetstrokecolor{currentstroke}%
\pgfsetdash{}{0pt}%
\pgfpathmoveto{\pgfqpoint{3.727238in}{0.632410in}}%
\pgfpathlineto{\pgfqpoint{3.727238in}{0.632410in}}%
\pgfpathlineto{\pgfqpoint{3.662099in}{0.653355in}}%
\pgfpathlineto{\pgfqpoint{3.597453in}{0.675777in}}%
\pgfpathlineto{\pgfqpoint{3.533252in}{0.699450in}}%
\pgfpathlineto{\pgfqpoint{3.469432in}{0.724132in}}%
\pgfpathlineto{\pgfqpoint{3.405908in}{0.749570in}}%
\pgfpathlineto{\pgfqpoint{3.342584in}{0.775503in}}%
\pgfpathlineto{\pgfqpoint{3.279351in}{0.801659in}}%
\pgfpathlineto{\pgfqpoint{3.216097in}{0.827762in}}%
\pgfpathlineto{\pgfqpoint{3.152704in}{0.853525in}}%
\pgfpathlineto{\pgfqpoint{3.089059in}{0.878653in}}%
\pgfpathlineto{\pgfqpoint{3.025053in}{0.902845in}}%
\pgfpathlineto{\pgfqpoint{2.960593in}{0.925794in}}%
\pgfpathlineto{\pgfqpoint{2.895603in}{0.947193in}}%
\pgfpathlineto{\pgfqpoint{2.830037in}{0.966746in}}%
\pgfpathlineto{\pgfqpoint{2.763878in}{0.984183in}}%
\pgfpathlineto{\pgfqpoint{2.697148in}{0.999279in}}%
\pgfpathlineto{\pgfqpoint{2.629904in}{1.011881in}}%
\pgfpathlineto{\pgfqpoint{2.562230in}{1.021923in}}%
\pgfpathlineto{\pgfqpoint{2.494231in}{1.029454in}}%
\pgfpathlineto{\pgfqpoint{2.426009in}{1.034641in}}%
\pgfpathlineto{\pgfqpoint{2.357659in}{1.037780in}}%
\pgfpathlineto{\pgfqpoint{2.289251in}{1.039311in}}%
\pgfpathlineto{\pgfqpoint{2.220825in}{1.039795in}}%
\pgfpathlineto{\pgfqpoint{2.152396in}{1.039921in}}%
\pgfpathlineto{\pgfqpoint{2.083972in}{1.040534in}}%
\pgfpathlineto{\pgfqpoint{2.015587in}{1.042701in}}%
\pgfpathlineto{\pgfqpoint{1.947381in}{1.047844in}}%
\pgfpathlineto{\pgfqpoint{1.879788in}{1.057948in}}%
\pgfpathlineto{\pgfqpoint{1.813992in}{1.075923in}}%
\pgfpathlineto{\pgfqpoint{1.753161in}{1.105947in}}%
\pgfpathlineto{\pgfqpoint{1.753161in}{1.105947in}}%
\pgfpathlineto{\pgfqpoint{1.714879in}{1.138203in}}%
\pgfpathlineto{\pgfqpoint{1.686294in}{1.180062in}}%
\pgfpathlineto{\pgfqpoint{1.670933in}{1.223771in}}%
\pgfpathlineto{\pgfqpoint{1.665078in}{1.271103in}}%
\pgfpathlineto{\pgfqpoint{1.667777in}{1.323861in}}%
\pgfpathlineto{\pgfqpoint{1.679551in}{1.384187in}}%
\pgfusepath{stroke}%
\end{pgfscope}%
\begin{pgfscope}%
\pgfpathrectangle{\pgfqpoint{0.647939in}{0.492442in}}{\pgfqpoint{3.079299in}{3.079299in}}%
\pgfusepath{clip}%
\pgfsetbuttcap%
\pgfsetroundjoin%
\pgfsetlinewidth{0.803000pt}%
\definecolor{currentstroke}{rgb}{0.501961,0.501961,0.501961}%
\pgfsetstrokecolor{currentstroke}%
\pgfsetdash{}{0pt}%
\pgfpathmoveto{\pgfqpoint{3.727238in}{0.702394in}}%
\pgfpathlineto{\pgfqpoint{3.727238in}{0.702394in}}%
\pgfpathlineto{\pgfqpoint{3.662318in}{0.724008in}}%
\pgfpathlineto{\pgfqpoint{3.597926in}{0.747149in}}%
\pgfpathlineto{\pgfqpoint{3.534014in}{0.771590in}}%
\pgfpathlineto{\pgfqpoint{3.470515in}{0.797087in}}%
\pgfpathlineto{\pgfqpoint{3.407344in}{0.823387in}}%
\pgfpathlineto{\pgfqpoint{3.344399in}{0.850226in}}%
\pgfpathlineto{\pgfqpoint{3.281570in}{0.877337in}}%
\pgfpathlineto{\pgfqpoint{3.218738in}{0.904441in}}%
\pgfpathlineto{\pgfqpoint{3.155781in}{0.931252in}}%
\pgfpathlineto{\pgfqpoint{3.092578in}{0.957474in}}%
\pgfpathlineto{\pgfqpoint{3.029013in}{0.982801in}}%
\pgfpathlineto{\pgfqpoint{2.964981in}{1.006919in}}%
\pgfpathlineto{\pgfqpoint{2.900397in}{1.029510in}}%
\pgfpathlineto{\pgfqpoint{2.835201in}{1.050262in}}%
\pgfpathlineto{\pgfqpoint{2.769366in}{1.068879in}}%
\pgfpathlineto{\pgfqpoint{2.702904in}{1.085109in}}%
\pgfpathlineto{\pgfqpoint{2.635865in}{1.098759in}}%
\pgfpathlineto{\pgfqpoint{2.568337in}{1.109734in}}%
\pgfpathlineto{\pgfqpoint{2.500431in}{1.118052in}}%
\pgfpathlineto{\pgfqpoint{2.432262in}{1.123862in}}%
\pgfpathlineto{\pgfqpoint{2.363936in}{1.127450in}}%
\pgfpathlineto{\pgfqpoint{2.295535in}{1.129247in}}%
\pgfpathlineto{\pgfqpoint{2.227110in}{1.129842in}}%
\pgfpathlineto{\pgfqpoint{2.158682in}{1.129979in}}%
\pgfpathlineto{\pgfqpoint{2.090258in}{1.130570in}}%
\pgfpathlineto{\pgfqpoint{2.021876in}{1.132794in}}%
\pgfpathlineto{\pgfqpoint{1.953699in}{1.138253in}}%
\pgfpathlineto{\pgfqpoint{1.886275in}{1.149316in}}%
\pgfpathlineto{\pgfqpoint{1.821265in}{1.169633in}}%
\pgfpathlineto{\pgfqpoint{1.821265in}{1.169633in}}%
\pgfpathlineto{\pgfqpoint{1.773265in}{1.196169in}}%
\pgfpathlineto{\pgfqpoint{1.773265in}{1.196169in}}%
\pgfpathlineto{\pgfqpoint{1.739136in}{1.228094in}}%
\pgfusepath{stroke}%
\end{pgfscope}%
\begin{pgfscope}%
\pgfpathrectangle{\pgfqpoint{0.647939in}{0.492442in}}{\pgfqpoint{3.079299in}{3.079299in}}%
\pgfusepath{clip}%
\pgfsetbuttcap%
\pgfsetroundjoin%
\pgfsetlinewidth{0.803000pt}%
\definecolor{currentstroke}{rgb}{0.501961,0.501961,0.501961}%
\pgfsetstrokecolor{currentstroke}%
\pgfsetdash{}{0pt}%
\pgfpathmoveto{\pgfqpoint{3.727238in}{0.772378in}}%
\pgfpathlineto{\pgfqpoint{3.727238in}{0.772378in}}%
\pgfpathlineto{\pgfqpoint{3.662560in}{0.794702in}}%
\pgfpathlineto{\pgfqpoint{3.598448in}{0.818608in}}%
\pgfpathlineto{\pgfqpoint{3.534855in}{0.843865in}}%
\pgfpathlineto{\pgfqpoint{3.471711in}{0.870230in}}%
\pgfpathlineto{\pgfqpoint{3.408929in}{0.897447in}}%
\pgfpathlineto{\pgfqpoint{3.346407in}{0.925257in}}%
\pgfpathlineto{\pgfqpoint{3.284030in}{0.953392in}}%
\pgfpathlineto{\pgfqpoint{3.221674in}{0.981574in}}%
\pgfpathlineto{\pgfqpoint{3.159213in}{1.009520in}}%
\pgfpathlineto{\pgfqpoint{3.096518in}{1.036934in}}%
\pgfpathlineto{\pgfqpoint{3.033463in}{1.063507in}}%
\pgfpathlineto{\pgfqpoint{2.969934in}{1.088921in}}%
\pgfpathlineto{\pgfqpoint{2.905832in}{1.112846in}}%
\pgfpathlineto{\pgfqpoint{2.841084in}{1.134953in}}%
\pgfpathlineto{\pgfqpoint{2.775648in}{1.154925in}}%
\pgfpathlineto{\pgfqpoint{2.709524in}{1.172474in}}%
\pgfpathlineto{\pgfqpoint{2.642753in}{1.187368in}}%
\pgfpathlineto{\pgfqpoint{2.575418in}{1.199464in}}%
\pgfpathlineto{\pgfqpoint{2.507637in}{1.208742in}}%
\pgfpathlineto{\pgfqpoint{2.439540in}{1.215324in}}%
\pgfpathlineto{\pgfqpoint{2.371248in}{1.219480in}}%
\pgfpathlineto{\pgfqpoint{2.302860in}{1.221643in}}%
\pgfpathlineto{\pgfqpoint{2.234438in}{1.222403in}}%
\pgfpathlineto{\pgfqpoint{2.166009in}{1.222552in}}%
\pgfpathlineto{\pgfqpoint{2.097585in}{1.223103in}}%
\pgfpathlineto{\pgfqpoint{2.029207in}{1.225372in}}%
\pgfpathlineto{\pgfqpoint{1.961076in}{1.231223in}}%
\pgfpathlineto{\pgfqpoint{1.893901in}{1.243535in}}%
\pgfpathlineto{\pgfqpoint{1.830123in}{1.267059in}}%
\pgfpathlineto{\pgfqpoint{1.830123in}{1.267059in}}%
\pgfusepath{stroke}%
\end{pgfscope}%
\begin{pgfscope}%
\pgfpathrectangle{\pgfqpoint{0.647939in}{0.492442in}}{\pgfqpoint{3.079299in}{3.079299in}}%
\pgfusepath{clip}%
\pgfsetbuttcap%
\pgfsetroundjoin%
\pgfsetlinewidth{0.803000pt}%
\definecolor{currentstroke}{rgb}{0.501961,0.501961,0.501961}%
\pgfsetstrokecolor{currentstroke}%
\pgfsetdash{}{0pt}%
\pgfpathmoveto{\pgfqpoint{3.727238in}{0.842362in}}%
\pgfpathlineto{\pgfqpoint{3.727238in}{0.842362in}}%
\pgfpathlineto{\pgfqpoint{3.662826in}{0.865442in}}%
\pgfpathlineto{\pgfqpoint{3.599023in}{0.890161in}}%
\pgfpathlineto{\pgfqpoint{3.535783in}{0.916288in}}%
\pgfpathlineto{\pgfqpoint{3.473034in}{0.943577in}}%
\pgfpathlineto{\pgfqpoint{3.410687in}{0.971775in}}%
\pgfpathlineto{\pgfqpoint{3.348638in}{1.000624in}}%
\pgfpathlineto{\pgfqpoint{3.286769in}{1.029858in}}%
\pgfpathlineto{\pgfqpoint{3.224953in}{1.059205in}}%
\pgfpathlineto{\pgfqpoint{3.163057in}{1.088383in}}%
\pgfpathlineto{\pgfqpoint{3.100947in}{1.117099in}}%
\pgfpathlineto{\pgfqpoint{3.038488in}{1.145046in}}%
\pgfpathlineto{\pgfqpoint{2.975553in}{1.171901in}}%
\pgfpathlineto{\pgfqpoint{2.912031in}{1.197327in}}%
\pgfpathlineto{\pgfqpoint{2.847831in}{1.220980in}}%
\pgfpathlineto{\pgfqpoint{2.782893in}{1.242516in}}%
\pgfpathlineto{\pgfqpoint{2.717201in}{1.261615in}}%
\pgfpathlineto{\pgfqpoint{2.650783in}{1.278000in}}%
\pgfpathlineto{\pgfqpoint{2.583714in}{1.291475in}}%
\pgfpathlineto{\pgfqpoint{2.516112in}{1.301957in}}%
\pgfpathlineto{\pgfqpoint{2.448118in}{1.309520in}}%
\pgfpathlineto{\pgfqpoint{2.379877in}{1.314412in}}%
\pgfpathlineto{\pgfqpoint{2.311508in}{1.317061in}}%
\pgfpathlineto{\pgfqpoint{2.243091in}{1.318078in}}%
\pgfpathlineto{\pgfqpoint{2.174662in}{1.318283in}}%
\pgfpathlineto{\pgfqpoint{2.106237in}{1.318772in}}%
\pgfpathlineto{\pgfqpoint{2.037863in}{1.321058in}}%
\pgfpathlineto{\pgfqpoint{1.969792in}{1.327374in}}%
\pgfpathlineto{\pgfqpoint{1.903053in}{1.341406in}}%
\pgfpathlineto{\pgfqpoint{1.903053in}{1.341406in}}%
\pgfpathlineto{\pgfqpoint{1.851993in}{1.362707in}}%
\pgfpathlineto{\pgfqpoint{1.851993in}{1.362707in}}%
\pgfpathlineto{\pgfqpoint{1.817285in}{1.388995in}}%
\pgfpathlineto{\pgfqpoint{1.792310in}{1.425577in}}%
\pgfpathlineto{\pgfqpoint{1.781258in}{1.463060in}}%
\pgfpathlineto{\pgfqpoint{1.779605in}{1.502917in}}%
\pgfpathlineto{\pgfqpoint{1.786496in}{1.547986in}}%
\pgfpathlineto{\pgfqpoint{1.802821in}{1.599409in}}%
\pgfusepath{stroke}%
\end{pgfscope}%
\begin{pgfscope}%
\pgfpathrectangle{\pgfqpoint{0.647939in}{0.492442in}}{\pgfqpoint{3.079299in}{3.079299in}}%
\pgfusepath{clip}%
\pgfsetbuttcap%
\pgfsetroundjoin%
\pgfsetlinewidth{0.803000pt}%
\definecolor{currentstroke}{rgb}{0.501961,0.501961,0.501961}%
\pgfsetstrokecolor{currentstroke}%
\pgfsetdash{}{0pt}%
\pgfpathmoveto{\pgfqpoint{3.727238in}{0.912347in}}%
\pgfpathlineto{\pgfqpoint{3.727238in}{0.912347in}}%
\pgfpathlineto{\pgfqpoint{3.663121in}{0.936231in}}%
\pgfpathlineto{\pgfqpoint{3.599661in}{0.961817in}}%
\pgfpathlineto{\pgfqpoint{3.536812in}{0.988870in}}%
\pgfpathlineto{\pgfqpoint{3.474502in}{1.017147in}}%
\pgfpathlineto{\pgfqpoint{3.412641in}{1.046395in}}%
\pgfpathlineto{\pgfqpoint{3.351122in}{1.076359in}}%
\pgfpathlineto{\pgfqpoint{3.289827in}{1.106778in}}%
\pgfpathlineto{\pgfqpoint{3.228625in}{1.137384in}}%
\pgfpathlineto{\pgfqpoint{3.167379in}{1.167903in}}%
\pgfpathlineto{\pgfqpoint{3.105948in}{1.198046in}}%
\pgfpathlineto{\pgfqpoint{3.044189in}{1.227508in}}%
\pgfpathlineto{\pgfqpoint{2.981965in}{1.255970in}}%
\pgfpathlineto{\pgfqpoint{2.919147in}{1.283092in}}%
\pgfpathlineto{\pgfqpoint{2.855625in}{1.308517in}}%
\pgfpathlineto{\pgfqpoint{2.791320in}{1.331879in}}%
\pgfpathlineto{\pgfqpoint{2.726189in}{1.352819in}}%
\pgfpathlineto{\pgfqpoint{2.660243in}{1.371011in}}%
\pgfpathlineto{\pgfqpoint{2.593544in}{1.386197in}}%
\pgfpathlineto{\pgfqpoint{2.526204in}{1.398224in}}%
\pgfpathlineto{\pgfqpoint{2.458374in}{1.407092in}}%
\pgfpathlineto{\pgfqpoint{2.390215in}{1.412980in}}%
\pgfpathlineto{\pgfqpoint{2.321877in}{1.416296in}}%
\pgfpathlineto{\pgfqpoint{2.253467in}{1.417677in}}%
\pgfpathlineto{\pgfqpoint{2.185039in}{1.417993in}}%
\pgfpathlineto{\pgfqpoint{2.116613in}{1.418438in}}%
\pgfpathlineto{\pgfqpoint{2.048238in}{1.420721in}}%
\pgfpathlineto{\pgfqpoint{1.980241in}{1.427608in}}%
\pgfpathlineto{\pgfqpoint{1.914260in}{1.444159in}}%
\pgfpathlineto{\pgfqpoint{1.914260in}{1.444159in}}%
\pgfpathlineto{\pgfqpoint{1.873592in}{1.464922in}}%
\pgfpathlineto{\pgfqpoint{1.873592in}{1.464922in}}%
\pgfpathlineto{\pgfqpoint{1.846185in}{1.491050in}}%
\pgfpathlineto{\pgfqpoint{1.829159in}{1.525399in}}%
\pgfpathlineto{\pgfqpoint{1.823845in}{1.560511in}}%
\pgfusepath{stroke}%
\end{pgfscope}%
\begin{pgfscope}%
\pgfpathrectangle{\pgfqpoint{0.647939in}{0.492442in}}{\pgfqpoint{3.079299in}{3.079299in}}%
\pgfusepath{clip}%
\pgfsetbuttcap%
\pgfsetroundjoin%
\pgfsetlinewidth{0.803000pt}%
\definecolor{currentstroke}{rgb}{0.501961,0.501961,0.501961}%
\pgfsetstrokecolor{currentstroke}%
\pgfsetdash{}{0pt}%
\pgfpathmoveto{\pgfqpoint{3.727238in}{0.982331in}}%
\pgfpathlineto{\pgfqpoint{3.727238in}{0.982331in}}%
\pgfpathlineto{\pgfqpoint{3.663448in}{1.007074in}}%
\pgfpathlineto{\pgfqpoint{3.600370in}{1.033585in}}%
\pgfpathlineto{\pgfqpoint{3.537957in}{1.061628in}}%
\pgfpathlineto{\pgfqpoint{3.476137in}{1.090961in}}%
\pgfpathlineto{\pgfqpoint{3.414821in}{1.121334in}}%
\pgfpathlineto{\pgfqpoint{3.353901in}{1.152496in}}%
\pgfpathlineto{\pgfqpoint{3.293256in}{1.184191in}}%
\pgfpathlineto{\pgfqpoint{3.232754in}{1.216158in}}%
\pgfpathlineto{\pgfqpoint{3.172254in}{1.248130in}}%
\pgfpathlineto{\pgfqpoint{3.111613in}{1.279830in}}%
\pgfpathlineto{\pgfqpoint{3.050680in}{1.310964in}}%
\pgfpathlineto{\pgfqpoint{2.989307in}{1.341219in}}%
\pgfpathlineto{\pgfqpoint{2.927353in}{1.370258in}}%
\pgfpathlineto{\pgfqpoint{2.864687in}{1.397721in}}%
\pgfpathlineto{\pgfqpoint{2.801203in}{1.423226in}}%
\pgfpathlineto{\pgfqpoint{2.736827in}{1.446380in}}%
\pgfpathlineto{\pgfqpoint{2.671537in}{1.466800in}}%
\pgfpathlineto{\pgfqpoint{2.605365in}{1.484149in}}%
\pgfpathlineto{\pgfqpoint{2.538416in}{1.498187in}}%
\pgfpathlineto{\pgfqpoint{2.470844in}{1.508816in}}%
\pgfpathlineto{\pgfqpoint{2.402830in}{1.516134in}}%
\pgfpathlineto{\pgfqpoint{2.334556in}{1.520471in}}%
\pgfpathlineto{\pgfqpoint{2.266163in}{1.522421in}}%
\pgfpathlineto{\pgfqpoint{2.197737in}{1.522912in}}%
\pgfpathlineto{\pgfqpoint{2.129311in}{1.523291in}}%
\pgfpathlineto{\pgfqpoint{2.060944in}{1.525583in}}%
\pgfpathlineto{\pgfqpoint{1.993091in}{1.533349in}}%
\pgfpathlineto{\pgfqpoint{1.993091in}{1.533349in}}%
\pgfpathlineto{\pgfqpoint{1.938831in}{1.548983in}}%
\pgfpathlineto{\pgfqpoint{1.938831in}{1.548983in}}%
\pgfpathlineto{\pgfqpoint{1.905571in}{1.568869in}}%
\pgfpathlineto{\pgfqpoint{1.905571in}{1.568869in}}%
\pgfpathlineto{\pgfqpoint{1.884789in}{1.593775in}}%
\pgfpathlineto{\pgfqpoint{1.874550in}{1.624894in}}%
\pgfpathlineto{\pgfqpoint{1.874262in}{1.656560in}}%
\pgfpathlineto{\pgfqpoint{1.882233in}{1.692117in}}%
\pgfusepath{stroke}%
\end{pgfscope}%
\begin{pgfscope}%
\pgfpathrectangle{\pgfqpoint{0.647939in}{0.492442in}}{\pgfqpoint{3.079299in}{3.079299in}}%
\pgfusepath{clip}%
\pgfsetbuttcap%
\pgfsetroundjoin%
\pgfsetlinewidth{0.803000pt}%
\definecolor{currentstroke}{rgb}{0.501961,0.501961,0.501961}%
\pgfsetstrokecolor{currentstroke}%
\pgfsetdash{}{0pt}%
\pgfpathmoveto{\pgfqpoint{3.727238in}{1.052315in}}%
\pgfpathlineto{\pgfqpoint{3.727238in}{1.052315in}}%
\pgfpathlineto{\pgfqpoint{3.663812in}{1.077977in}}%
\pgfpathlineto{\pgfqpoint{3.601160in}{1.105476in}}%
\pgfpathlineto{\pgfqpoint{3.539234in}{1.134578in}}%
\pgfpathlineto{\pgfqpoint{3.477964in}{1.165042in}}%
\pgfpathlineto{\pgfqpoint{3.417261in}{1.196621in}}%
\pgfpathlineto{\pgfqpoint{3.357016in}{1.229067in}}%
\pgfpathlineto{\pgfqpoint{3.297109in}{1.262135in}}%
\pgfpathlineto{\pgfqpoint{3.237409in}{1.295576in}}%
\pgfpathlineto{\pgfqpoint{3.177775in}{1.329135in}}%
\pgfpathlineto{\pgfqpoint{3.118060in}{1.362549in}}%
\pgfpathlineto{\pgfqpoint{3.058112in}{1.395541in}}%
\pgfpathlineto{\pgfqpoint{2.997775in}{1.427813in}}%
\pgfpathlineto{\pgfqpoint{2.936894in}{1.459041in}}%
\pgfpathlineto{\pgfqpoint{2.875318in}{1.488871in}}%
\pgfpathlineto{\pgfqpoint{2.812914in}{1.516916in}}%
\pgfpathlineto{\pgfqpoint{2.749573in}{1.542764in}}%
\pgfpathlineto{\pgfqpoint{2.685229in}{1.565986in}}%
\pgfpathlineto{\pgfqpoint{2.619871in}{1.586164in}}%
\pgfpathlineto{\pgfqpoint{2.553561in}{1.602937in}}%
\pgfpathlineto{\pgfqpoint{2.486432in}{1.616057in}}%
\pgfpathlineto{\pgfqpoint{2.418681in}{1.625467in}}%
\pgfpathlineto{\pgfqpoint{2.350530in}{1.631389in}}%
\pgfpathlineto{\pgfqpoint{2.282180in}{1.634359in}}%
\pgfpathlineto{\pgfqpoint{2.213763in}{1.635289in}}%
\pgfpathlineto{\pgfqpoint{2.145335in}{1.635623in}}%
\pgfpathlineto{\pgfqpoint{2.076965in}{1.637819in}}%
\pgfpathlineto{\pgfqpoint{2.009426in}{1.647048in}}%
\pgfpathlineto{\pgfqpoint{2.009426in}{1.647048in}}%
\pgfpathlineto{\pgfqpoint{1.970275in}{1.660877in}}%
\pgfpathlineto{\pgfqpoint{1.970275in}{1.660877in}}%
\pgfpathlineto{\pgfqpoint{1.945873in}{1.679474in}}%
\pgfpathlineto{\pgfqpoint{1.945873in}{1.679474in}}%
\pgfusepath{stroke}%
\end{pgfscope}%
\begin{pgfscope}%
\pgfpathrectangle{\pgfqpoint{0.647939in}{0.492442in}}{\pgfqpoint{3.079299in}{3.079299in}}%
\pgfusepath{clip}%
\pgfsetbuttcap%
\pgfsetroundjoin%
\pgfsetlinewidth{0.803000pt}%
\definecolor{currentstroke}{rgb}{0.501961,0.501961,0.501961}%
\pgfsetstrokecolor{currentstroke}%
\pgfsetdash{}{0pt}%
\pgfpathmoveto{\pgfqpoint{3.727238in}{1.122299in}}%
\pgfpathlineto{\pgfqpoint{3.727238in}{1.122299in}}%
\pgfpathlineto{\pgfqpoint{3.664220in}{1.148944in}}%
\pgfpathlineto{\pgfqpoint{3.602043in}{1.177502in}}%
\pgfpathlineto{\pgfqpoint{3.540664in}{1.207739in}}%
\pgfpathlineto{\pgfqpoint{3.480012in}{1.239414in}}%
\pgfpathlineto{\pgfqpoint{3.419998in}{1.272285in}}%
\pgfpathlineto{\pgfqpoint{3.360519in}{1.306115in}}%
\pgfpathlineto{\pgfqpoint{3.301456in}{1.340668in}}%
\pgfpathlineto{\pgfqpoint{3.242682in}{1.375710in}}%
\pgfpathlineto{\pgfqpoint{3.184060in}{1.411005in}}%
\pgfpathlineto{\pgfqpoint{3.125445in}{1.446311in}}%
\pgfpathlineto{\pgfqpoint{3.066684in}{1.481374in}}%
\pgfpathlineto{\pgfqpoint{3.007619in}{1.515921in}}%
\pgfpathlineto{\pgfqpoint{2.948087in}{1.549655in}}%
\pgfpathlineto{\pgfqpoint{2.887923in}{1.582246in}}%
\pgfpathlineto{\pgfqpoint{2.826967in}{1.613325in}}%
\pgfpathlineto{\pgfqpoint{2.765075in}{1.642484in}}%
\pgfpathlineto{\pgfqpoint{2.702124in}{1.669271in}}%
\pgfpathlineto{\pgfqpoint{2.638039in}{1.693203in}}%
\pgfpathlineto{\pgfqpoint{2.572812in}{1.713802in}}%
\pgfpathlineto{\pgfqpoint{2.506524in}{1.730645in}}%
\pgfpathlineto{\pgfqpoint{2.439342in}{1.743444in}}%
\pgfpathlineto{\pgfqpoint{2.371506in}{1.752147in}}%
\pgfpathlineto{\pgfqpoint{2.303281in}{1.757049in}}%
\pgfpathlineto{\pgfqpoint{2.234890in}{1.758926in}}%
\pgfpathlineto{\pgfqpoint{2.166463in}{1.759344in}}%
\pgfpathlineto{\pgfqpoint{2.098109in}{1.761499in}}%
\pgfpathlineto{\pgfqpoint{2.098109in}{1.761499in}}%
\pgfpathlineto{\pgfqpoint{2.041785in}{1.770559in}}%
\pgfpathlineto{\pgfqpoint{2.041785in}{1.770559in}}%
\pgfpathlineto{\pgfqpoint{2.016465in}{1.781725in}}%
\pgfpathlineto{\pgfqpoint{2.016465in}{1.781725in}}%
\pgfpathlineto{\pgfqpoint{2.002409in}{1.797190in}}%
\pgfpathlineto{\pgfqpoint{1.998627in}{1.817621in}}%
\pgfpathlineto{\pgfqpoint{2.002850in}{1.837357in}}%
\pgfusepath{stroke}%
\end{pgfscope}%
\begin{pgfscope}%
\pgfpathrectangle{\pgfqpoint{0.647939in}{0.492442in}}{\pgfqpoint{3.079299in}{3.079299in}}%
\pgfusepath{clip}%
\pgfsetbuttcap%
\pgfsetroundjoin%
\pgfsetlinewidth{0.803000pt}%
\definecolor{currentstroke}{rgb}{0.501961,0.501961,0.501961}%
\pgfsetstrokecolor{currentstroke}%
\pgfsetdash{}{0pt}%
\pgfpathmoveto{\pgfqpoint{3.727238in}{1.192283in}}%
\pgfpathlineto{\pgfqpoint{3.727238in}{1.192283in}}%
\pgfpathlineto{\pgfqpoint{3.664678in}{1.219982in}}%
\pgfpathlineto{\pgfqpoint{3.603036in}{1.249675in}}%
\pgfpathlineto{\pgfqpoint{3.542271in}{1.281127in}}%
\pgfpathlineto{\pgfqpoint{3.482317in}{1.314100in}}%
\pgfpathlineto{\pgfqpoint{3.423087in}{1.348360in}}%
\pgfpathlineto{\pgfqpoint{3.364483in}{1.383682in}}%
\pgfpathlineto{\pgfqpoint{3.306392in}{1.419843in}}%
\pgfpathlineto{\pgfqpoint{3.248694in}{1.456630in}}%
\pgfpathlineto{\pgfqpoint{3.191260in}{1.493829in}}%
\pgfpathlineto{\pgfqpoint{3.133955in}{1.531225in}}%
\pgfpathlineto{\pgfqpoint{3.076633in}{1.568594in}}%
\pgfpathlineto{\pgfqpoint{3.019142in}{1.605701in}}%
\pgfpathlineto{\pgfqpoint{2.961323in}{1.642293in}}%
\pgfpathlineto{\pgfqpoint{2.903015in}{1.678096in}}%
\pgfpathlineto{\pgfqpoint{2.844049in}{1.712800in}}%
\pgfpathlineto{\pgfqpoint{2.784256in}{1.746050in}}%
\pgfpathlineto{\pgfqpoint{2.723472in}{1.777440in}}%
\pgfpathlineto{\pgfqpoint{2.661546in}{1.806502in}}%
\pgfpathlineto{\pgfqpoint{2.598366in}{1.832709in}}%
\pgfpathlineto{\pgfqpoint{2.533879in}{1.855488in}}%
\pgfpathlineto{\pgfqpoint{2.468124in}{1.874260in}}%
\pgfpathlineto{\pgfqpoint{2.401254in}{1.888527in}}%
\pgfpathlineto{\pgfqpoint{2.333536in}{1.898005in}}%
\pgfpathlineto{\pgfqpoint{2.265318in}{1.902849in}}%
\pgfpathlineto{\pgfqpoint{2.196915in}{1.904244in}}%
\pgfpathlineto{\pgfqpoint{2.128658in}{1.906809in}}%
\pgfpathlineto{\pgfqpoint{2.128658in}{1.906809in}}%
\pgfpathlineto{\pgfqpoint{2.105933in}{1.910395in}}%
\pgfpathlineto{\pgfqpoint{2.105933in}{1.910395in}}%
\pgfpathlineto{\pgfqpoint{2.092936in}{1.917371in}}%
\pgfpathlineto{\pgfqpoint{2.092936in}{1.917371in}}%
\pgfusepath{stroke}%
\end{pgfscope}%
\begin{pgfscope}%
\pgfpathrectangle{\pgfqpoint{0.647939in}{0.492442in}}{\pgfqpoint{3.079299in}{3.079299in}}%
\pgfusepath{clip}%
\pgfsetbuttcap%
\pgfsetroundjoin%
\pgfsetlinewidth{0.803000pt}%
\definecolor{currentstroke}{rgb}{0.501961,0.501961,0.501961}%
\pgfsetstrokecolor{currentstroke}%
\pgfsetdash{}{0pt}%
\pgfpathmoveto{\pgfqpoint{3.727238in}{1.332251in}}%
\pgfpathlineto{\pgfqpoint{3.727238in}{1.332251in}}%
\pgfpathlineto{\pgfqpoint{3.665774in}{1.362302in}}%
\pgfpathlineto{\pgfqpoint{3.605416in}{1.394521in}}%
\pgfpathlineto{\pgfqpoint{3.546132in}{1.428678in}}%
\pgfpathlineto{\pgfqpoint{3.487867in}{1.464549in}}%
\pgfpathlineto{\pgfqpoint{3.430552in}{1.501920in}}%
\pgfpathlineto{\pgfqpoint{3.374107in}{1.540596in}}%
\pgfpathlineto{\pgfqpoint{3.318447in}{1.580394in}}%
\pgfpathlineto{\pgfqpoint{3.263482in}{1.621149in}}%
\pgfpathlineto{\pgfqpoint{3.209119in}{1.662702in}}%
\pgfpathlineto{\pgfqpoint{3.155265in}{1.704916in}}%
\pgfpathlineto{\pgfqpoint{3.101842in}{1.747672in}}%
\pgfpathlineto{\pgfqpoint{3.048770in}{1.790862in}}%
\pgfpathlineto{\pgfqpoint{2.995980in}{1.834395in}}%
\pgfpathlineto{\pgfqpoint{2.943420in}{1.878205in}}%
\pgfpathlineto{\pgfqpoint{2.891064in}{1.922258in}}%
\pgfpathlineto{\pgfqpoint{2.838923in}{1.966563in}}%
\pgfpathlineto{\pgfqpoint{2.787079in}{2.011207in}}%
\pgfpathlineto{\pgfqpoint{2.735727in}{2.056414in}}%
\pgfpathlineto{\pgfqpoint{2.685318in}{2.102653in}}%
\pgfpathlineto{\pgfqpoint{2.636838in}{2.150872in}}%
\pgfpathlineto{\pgfqpoint{2.592716in}{2.202993in}}%
\pgfpathlineto{\pgfqpoint{2.559625in}{2.262148in}}%
\pgfpathlineto{\pgfqpoint{2.559625in}{2.262148in}}%
\pgfpathlineto{\pgfqpoint{2.549589in}{2.305279in}}%
\pgfpathlineto{\pgfqpoint{2.552074in}{2.350344in}}%
\pgfpathlineto{\pgfqpoint{2.564280in}{2.393056in}}%
\pgfusepath{stroke}%
\end{pgfscope}%
\begin{pgfscope}%
\pgfpathrectangle{\pgfqpoint{0.647939in}{0.492442in}}{\pgfqpoint{3.079299in}{3.079299in}}%
\pgfusepath{clip}%
\pgfsetbuttcap%
\pgfsetroundjoin%
\pgfsetlinewidth{0.803000pt}%
\definecolor{currentstroke}{rgb}{0.501961,0.501961,0.501961}%
\pgfsetstrokecolor{currentstroke}%
\pgfsetdash{}{0pt}%
\pgfpathmoveto{\pgfqpoint{3.727238in}{1.472219in}}%
\pgfpathlineto{\pgfqpoint{3.727238in}{1.472219in}}%
\pgfpathlineto{\pgfqpoint{3.667188in}{1.504996in}}%
\pgfpathlineto{\pgfqpoint{3.608490in}{1.540141in}}%
\pgfpathlineto{\pgfqpoint{3.551130in}{1.577432in}}%
\pgfpathlineto{\pgfqpoint{3.495077in}{1.616663in}}%
\pgfpathlineto{\pgfqpoint{3.440294in}{1.657652in}}%
\pgfpathlineto{\pgfqpoint{3.386745in}{1.700242in}}%
\pgfpathlineto{\pgfqpoint{3.334391in}{1.744292in}}%
\pgfpathlineto{\pgfqpoint{3.283207in}{1.789698in}}%
\pgfpathlineto{\pgfqpoint{3.233198in}{1.836395in}}%
\pgfpathlineto{\pgfqpoint{3.184387in}{1.884342in}}%
\pgfpathlineto{\pgfqpoint{3.136849in}{1.933548in}}%
\pgfpathlineto{\pgfqpoint{3.090710in}{1.984068in}}%
\pgfpathlineto{\pgfqpoint{3.046189in}{2.036015in}}%
\pgfpathlineto{\pgfqpoint{3.003636in}{2.089577in}}%
\pgfpathlineto{\pgfqpoint{2.963586in}{2.145025in}}%
\pgfpathlineto{\pgfqpoint{2.926863in}{2.202718in}}%
\pgfpathlineto{\pgfqpoint{2.894702in}{2.263041in}}%
\pgfpathlineto{\pgfqpoint{2.868831in}{2.326265in}}%
\pgfpathlineto{\pgfqpoint{2.851333in}{2.392236in}}%
\pgfpathlineto{\pgfqpoint{2.843982in}{2.460020in}}%
\pgfpathlineto{\pgfqpoint{2.847253in}{2.528108in}}%
\pgfpathlineto{\pgfqpoint{2.860017in}{2.595130in}}%
\pgfpathlineto{\pgfqpoint{2.880318in}{2.660334in}}%
\pgfpathlineto{\pgfqpoint{2.906224in}{2.723571in}}%
\pgfpathlineto{\pgfqpoint{2.936230in}{2.785000in}}%
\pgfpathlineto{\pgfqpoint{2.969276in}{2.844868in}}%
\pgfpathlineto{\pgfqpoint{3.004649in}{2.903405in}}%
\pgfpathlineto{\pgfqpoint{3.041874in}{2.960795in}}%
\pgfpathlineto{\pgfqpoint{3.080641in}{3.017164in}}%
\pgfpathlineto{\pgfqpoint{3.120756in}{3.072586in}}%
\pgfusepath{stroke}%
\end{pgfscope}%
\begin{pgfscope}%
\pgfpathrectangle{\pgfqpoint{0.647939in}{0.492442in}}{\pgfqpoint{3.079299in}{3.079299in}}%
\pgfusepath{clip}%
\pgfsetbuttcap%
\pgfsetroundjoin%
\pgfsetlinewidth{0.803000pt}%
\definecolor{currentstroke}{rgb}{0.501961,0.501961,0.501961}%
\pgfsetstrokecolor{currentstroke}%
\pgfsetdash{}{0pt}%
\pgfpathmoveto{\pgfqpoint{3.727238in}{1.542203in}}%
\pgfpathlineto{\pgfqpoint{3.727238in}{1.542203in}}%
\pgfpathlineto{\pgfqpoint{3.668050in}{1.576509in}}%
\pgfpathlineto{\pgfqpoint{3.610365in}{1.613290in}}%
\pgfpathlineto{\pgfqpoint{3.554184in}{1.652331in}}%
\pgfpathlineto{\pgfqpoint{3.499496in}{1.693439in}}%
\pgfpathlineto{\pgfqpoint{3.446288in}{1.736450in}}%
\pgfpathlineto{\pgfqpoint{3.394550in}{1.781219in}}%
\pgfpathlineto{\pgfqpoint{3.344285in}{1.827636in}}%
\pgfpathlineto{\pgfqpoint{3.295531in}{1.875637in}}%
\pgfpathlineto{\pgfqpoint{3.248353in}{1.925187in}}%
\pgfpathlineto{\pgfqpoint{3.202872in}{1.976297in}}%
\pgfpathlineto{\pgfqpoint{3.159284in}{2.029025in}}%
\pgfpathlineto{\pgfqpoint{3.117876in}{2.083478in}}%
\pgfpathlineto{\pgfqpoint{3.079079in}{2.139816in}}%
\pgfpathlineto{\pgfqpoint{3.043511in}{2.198233in}}%
\pgfpathlineto{\pgfqpoint{3.012024in}{2.258925in}}%
\pgfpathlineto{\pgfqpoint{2.985732in}{2.322013in}}%
\pgfpathlineto{\pgfqpoint{2.965951in}{2.387404in}}%
\pgfpathlineto{\pgfqpoint{2.953953in}{2.454632in}}%
\pgfpathlineto{\pgfqpoint{2.950534in}{2.522813in}}%
\pgfpathlineto{\pgfqpoint{2.955649in}{2.590890in}}%
\pgfpathlineto{\pgfqpoint{2.968457in}{2.657972in}}%
\pgfpathlineto{\pgfqpoint{2.987684in}{2.723542in}}%
\pgfpathlineto{\pgfqpoint{3.012043in}{2.787419in}}%
\pgfusepath{stroke}%
\end{pgfscope}%
\begin{pgfscope}%
\pgfpathrectangle{\pgfqpoint{0.647939in}{0.492442in}}{\pgfqpoint{3.079299in}{3.079299in}}%
\pgfusepath{clip}%
\pgfsetbuttcap%
\pgfsetroundjoin%
\pgfsetlinewidth{0.803000pt}%
\definecolor{currentstroke}{rgb}{0.501961,0.501961,0.501961}%
\pgfsetstrokecolor{currentstroke}%
\pgfsetdash{}{0pt}%
\pgfpathmoveto{\pgfqpoint{3.727238in}{1.682171in}}%
\pgfpathlineto{\pgfqpoint{3.727238in}{1.682171in}}%
\pgfpathlineto{\pgfqpoint{3.670186in}{1.719913in}}%
\pgfpathlineto{\pgfqpoint{3.615017in}{1.760360in}}%
\pgfpathlineto{\pgfqpoint{3.561774in}{1.803312in}}%
\pgfpathlineto{\pgfqpoint{3.510504in}{1.848603in}}%
\pgfpathlineto{\pgfqpoint{3.461257in}{1.896088in}}%
\pgfpathlineto{\pgfqpoint{3.414118in}{1.945665in}}%
\pgfpathlineto{\pgfqpoint{3.369214in}{1.997272in}}%
\pgfpathlineto{\pgfqpoint{3.326724in}{2.050883in}}%
\pgfpathlineto{\pgfqpoint{3.286911in}{2.106508in}}%
\pgfpathlineto{\pgfqpoint{3.250125in}{2.164172in}}%
\pgfpathlineto{\pgfqpoint{3.216835in}{2.223911in}}%
\pgfpathlineto{\pgfqpoint{3.187634in}{2.285739in}}%
\pgfpathlineto{\pgfqpoint{3.163228in}{2.349597in}}%
\pgfpathlineto{\pgfqpoint{3.144380in}{2.415296in}}%
\pgfpathlineto{\pgfqpoint{3.131796in}{2.482469in}}%
\pgfpathlineto{\pgfqpoint{3.125976in}{2.550553in}}%
\pgfpathlineto{\pgfqpoint{3.127060in}{2.618870in}}%
\pgfpathlineto{\pgfqpoint{3.134783in}{2.686763in}}%
\pgfpathlineto{\pgfqpoint{3.148572in}{2.753704in}}%
\pgfpathlineto{\pgfqpoint{3.167688in}{2.819343in}}%
\pgfpathlineto{\pgfqpoint{3.191375in}{2.883491in}}%
\pgfpathlineto{\pgfqpoint{3.218958in}{2.946074in}}%
\pgfpathlineto{\pgfqpoint{3.249868in}{3.007091in}}%
\pgfpathlineto{\pgfqpoint{3.283659in}{3.066565in}}%
\pgfpathlineto{\pgfqpoint{3.319995in}{3.124524in}}%
\pgfpathlineto{\pgfqpoint{3.358630in}{3.180978in}}%
\pgfpathlineto{\pgfqpoint{3.399404in}{3.235910in}}%
\pgfpathlineto{\pgfqpoint{3.442203in}{3.289284in}}%
\pgfpathlineto{\pgfqpoint{3.486959in}{3.341027in}}%
\pgfpathlineto{\pgfqpoint{3.533647in}{3.391031in}}%
\pgfpathlineto{\pgfqpoint{3.582267in}{3.439159in}}%
\pgfpathlineto{\pgfqpoint{3.632834in}{3.485234in}}%
\pgfpathlineto{\pgfqpoint{3.685376in}{3.529041in}}%
\pgfpathlineto{\pgfqpoint{3.727238in}{3.562366in}}%
\pgfusepath{stroke}%
\end{pgfscope}%
\begin{pgfscope}%
\pgfpathrectangle{\pgfqpoint{0.647939in}{0.492442in}}{\pgfqpoint{3.079299in}{3.079299in}}%
\pgfusepath{clip}%
\pgfsetbuttcap%
\pgfsetroundjoin%
\pgfsetlinewidth{0.803000pt}%
\definecolor{currentstroke}{rgb}{0.501961,0.501961,0.501961}%
\pgfsetstrokecolor{currentstroke}%
\pgfsetdash{}{0pt}%
\pgfpathmoveto{\pgfqpoint{3.727238in}{1.822139in}}%
\pgfpathlineto{\pgfqpoint{3.727238in}{1.822139in}}%
\pgfpathlineto{\pgfqpoint{3.673059in}{1.863891in}}%
\pgfpathlineto{\pgfqpoint{3.621279in}{1.908582in}}%
\pgfpathlineto{\pgfqpoint{3.572008in}{1.956027in}}%
\pgfpathlineto{\pgfqpoint{3.525369in}{2.006063in}}%
\pgfpathlineto{\pgfqpoint{3.481524in}{2.058561in}}%
\pgfpathlineto{\pgfqpoint{3.440687in}{2.113429in}}%
\pgfpathlineto{\pgfqpoint{3.403134in}{2.170594in}}%
\pgfpathlineto{\pgfqpoint{3.369224in}{2.229986in}}%
\pgfpathlineto{\pgfqpoint{3.339398in}{2.291521in}}%
\pgfpathlineto{\pgfqpoint{3.314163in}{2.355068in}}%
\pgfpathlineto{\pgfqpoint{3.294064in}{2.420414in}}%
\pgfpathlineto{\pgfqpoint{3.279614in}{2.487229in}}%
\pgfpathlineto{\pgfqpoint{3.271213in}{2.555058in}}%
\pgfpathlineto{\pgfqpoint{3.269060in}{2.623365in}}%
\pgfpathlineto{\pgfqpoint{3.273105in}{2.691590in}}%
\pgfpathlineto{\pgfqpoint{3.283061in}{2.759217in}}%
\pgfpathlineto{\pgfqpoint{3.298468in}{2.825828in}}%
\pgfpathlineto{\pgfqpoint{3.318788in}{2.891119in}}%
\pgfpathlineto{\pgfqpoint{3.343476in}{2.954893in}}%
\pgfpathlineto{\pgfqpoint{3.372040in}{3.017034in}}%
\pgfpathlineto{\pgfqpoint{3.404061in}{3.077473in}}%
\pgfpathlineto{\pgfqpoint{3.439203in}{3.136160in}}%
\pgfpathlineto{\pgfqpoint{3.477206in}{3.193039in}}%
\pgfpathlineto{\pgfqpoint{3.517884in}{3.248034in}}%
\pgfpathlineto{\pgfqpoint{3.561114in}{3.301046in}}%
\pgfpathlineto{\pgfqpoint{3.606808in}{3.351953in}}%
\pgfpathlineto{\pgfqpoint{3.654919in}{3.400579in}}%
\pgfpathlineto{\pgfqpoint{3.705420in}{3.446715in}}%
\pgfpathlineto{\pgfqpoint{3.727238in}{3.465644in}}%
\pgfusepath{stroke}%
\end{pgfscope}%
\begin{pgfscope}%
\pgfpathrectangle{\pgfqpoint{0.647939in}{0.492442in}}{\pgfqpoint{3.079299in}{3.079299in}}%
\pgfusepath{clip}%
\pgfsetbuttcap%
\pgfsetroundjoin%
\pgfsetlinewidth{0.803000pt}%
\definecolor{currentstroke}{rgb}{0.501961,0.501961,0.501961}%
\pgfsetstrokecolor{currentstroke}%
\pgfsetdash{}{0pt}%
\pgfpathmoveto{\pgfqpoint{3.727238in}{1.892124in}}%
\pgfpathlineto{\pgfqpoint{3.727238in}{1.892124in}}%
\pgfpathlineto{\pgfqpoint{3.674867in}{1.936114in}}%
\pgfpathlineto{\pgfqpoint{3.625217in}{1.983153in}}%
\pgfpathlineto{\pgfqpoint{3.578440in}{2.033051in}}%
\pgfpathlineto{\pgfqpoint{3.534713in}{2.085640in}}%
\pgfpathlineto{\pgfqpoint{3.494258in}{2.140786in}}%
\pgfpathlineto{\pgfqpoint{3.457361in}{2.198370in}}%
\pgfpathlineto{\pgfqpoint{3.424377in}{2.258273in}}%
\pgfpathlineto{\pgfqpoint{3.395722in}{2.320355in}}%
\pgfpathlineto{\pgfqpoint{3.371862in}{2.384426in}}%
\pgfpathlineto{\pgfqpoint{3.353278in}{2.450216in}}%
\pgfusepath{stroke}%
\end{pgfscope}%
\begin{pgfscope}%
\pgfpathrectangle{\pgfqpoint{0.647939in}{0.492442in}}{\pgfqpoint{3.079299in}{3.079299in}}%
\pgfusepath{clip}%
\pgfsetbuttcap%
\pgfsetroundjoin%
\pgfsetlinewidth{0.803000pt}%
\definecolor{currentstroke}{rgb}{0.501961,0.501961,0.501961}%
\pgfsetstrokecolor{currentstroke}%
\pgfsetdash{}{0pt}%
\pgfpathmoveto{\pgfqpoint{3.727238in}{2.032092in}}%
\pgfpathlineto{\pgfqpoint{3.727238in}{2.032092in}}%
\pgfpathlineto{\pgfqpoint{3.679478in}{2.081034in}}%
\pgfpathlineto{\pgfqpoint{3.635243in}{2.133182in}}%
\pgfpathlineto{\pgfqpoint{3.594784in}{2.188307in}}%
\pgfpathlineto{\pgfqpoint{3.558389in}{2.246191in}}%
\pgfpathlineto{\pgfqpoint{3.526402in}{2.306618in}}%
\pgfpathlineto{\pgfqpoint{3.499208in}{2.369342in}}%
\pgfpathlineto{\pgfqpoint{3.477214in}{2.434066in}}%
\pgfpathlineto{\pgfqpoint{3.460803in}{2.500419in}}%
\pgfpathlineto{\pgfqpoint{3.450287in}{2.567953in}}%
\pgfpathlineto{\pgfqpoint{3.445852in}{2.636160in}}%
\pgfpathlineto{\pgfqpoint{3.447519in}{2.704495in}}%
\pgfpathlineto{\pgfqpoint{3.455141in}{2.772425in}}%
\pgfpathlineto{\pgfqpoint{3.468428in}{2.839479in}}%
\pgfpathlineto{\pgfqpoint{3.486998in}{2.905272in}}%
\pgfpathlineto{\pgfqpoint{3.510435in}{2.969502in}}%
\pgfpathlineto{\pgfqpoint{3.538340in}{3.031927in}}%
\pgfpathlineto{\pgfqpoint{3.570357in}{3.092351in}}%
\pgfpathlineto{\pgfqpoint{3.606184in}{3.150601in}}%
\pgfpathlineto{\pgfqpoint{3.645589in}{3.206495in}}%
\pgfpathlineto{\pgfqpoint{3.688388in}{3.259840in}}%
\pgfpathlineto{\pgfqpoint{3.727238in}{3.305322in}}%
\pgfusepath{stroke}%
\end{pgfscope}%
\begin{pgfscope}%
\pgfpathrectangle{\pgfqpoint{0.647939in}{0.492442in}}{\pgfqpoint{3.079299in}{3.079299in}}%
\pgfusepath{clip}%
\pgfsetbuttcap%
\pgfsetroundjoin%
\pgfsetlinewidth{0.803000pt}%
\definecolor{currentstroke}{rgb}{0.501961,0.501961,0.501961}%
\pgfsetstrokecolor{currentstroke}%
\pgfsetdash{}{0pt}%
\pgfpathmoveto{\pgfqpoint{3.727238in}{2.172060in}}%
\pgfpathlineto{\pgfqpoint{3.727238in}{2.172060in}}%
\pgfpathlineto{\pgfqpoint{3.685847in}{2.226476in}}%
\pgfpathlineto{\pgfqpoint{3.649002in}{2.284059in}}%
\pgfpathlineto{\pgfqpoint{3.617049in}{2.344488in}}%
\pgfpathlineto{\pgfqpoint{3.590352in}{2.407411in}}%
\pgfpathlineto{\pgfqpoint{3.569278in}{2.472429in}}%
\pgfpathlineto{\pgfqpoint{3.554155in}{2.539081in}}%
\pgfpathlineto{\pgfqpoint{3.545224in}{2.606844in}}%
\pgfpathlineto{\pgfqpoint{3.542603in}{2.675143in}}%
\pgfpathlineto{\pgfqpoint{3.546257in}{2.743391in}}%
\pgfpathlineto{\pgfqpoint{3.556007in}{2.811039in}}%
\pgfpathlineto{\pgfqpoint{3.571557in}{2.877600in}}%
\pgfpathlineto{\pgfqpoint{3.592553in}{2.942658in}}%
\pgfpathlineto{\pgfqpoint{3.618624in}{3.005861in}}%
\pgfpathlineto{\pgfqpoint{3.649411in}{3.066911in}}%
\pgfpathlineto{\pgfqpoint{3.684602in}{3.125534in}}%
\pgfpathlineto{\pgfqpoint{3.723932in}{3.181466in}}%
\pgfpathlineto{\pgfqpoint{3.727238in}{3.185829in}}%
\pgfusepath{stroke}%
\end{pgfscope}%
\begin{pgfscope}%
\pgfpathrectangle{\pgfqpoint{0.647939in}{0.492442in}}{\pgfqpoint{3.079299in}{3.079299in}}%
\pgfusepath{clip}%
\pgfsetbuttcap%
\pgfsetroundjoin%
\pgfsetlinewidth{0.803000pt}%
\definecolor{currentstroke}{rgb}{0.501961,0.501961,0.501961}%
\pgfsetstrokecolor{currentstroke}%
\pgfsetdash{}{0pt}%
\pgfpathmoveto{\pgfqpoint{3.727238in}{2.312028in}}%
\pgfpathlineto{\pgfqpoint{3.727238in}{2.312028in}}%
\pgfpathlineto{\pgfqpoint{3.694539in}{2.372043in}}%
\pgfpathlineto{\pgfqpoint{3.667557in}{2.434831in}}%
\pgfpathlineto{\pgfqpoint{3.646639in}{2.499887in}}%
\pgfpathlineto{\pgfqpoint{3.632086in}{2.566651in}}%
\pgfpathlineto{\pgfqpoint{3.624104in}{2.634519in}}%
\pgfpathlineto{\pgfqpoint{3.622769in}{2.702847in}}%
\pgfpathlineto{\pgfqpoint{3.628013in}{2.770987in}}%
\pgfpathlineto{\pgfqpoint{3.639630in}{2.838333in}}%
\pgfpathlineto{\pgfqpoint{3.657329in}{2.904346in}}%
\pgfpathlineto{\pgfqpoint{3.680758in}{2.968555in}}%
\pgfpathlineto{\pgfqpoint{3.709563in}{3.030544in}}%
\pgfpathlineto{\pgfqpoint{3.727238in}{3.064782in}}%
\pgfusepath{stroke}%
\end{pgfscope}%
\begin{pgfscope}%
\pgfpathrectangle{\pgfqpoint{0.647939in}{0.492442in}}{\pgfqpoint{3.079299in}{3.079299in}}%
\pgfusepath{clip}%
\pgfsetbuttcap%
\pgfsetroundjoin%
\pgfsetlinewidth{0.803000pt}%
\definecolor{currentstroke}{rgb}{0.501961,0.501961,0.501961}%
\pgfsetstrokecolor{currentstroke}%
\pgfsetdash{}{0pt}%
\pgfpathmoveto{\pgfqpoint{3.727238in}{2.521980in}}%
\pgfpathlineto{\pgfqpoint{3.727238in}{2.521980in}}%
\pgfpathlineto{\pgfqpoint{3.712561in}{2.588702in}}%
\pgfpathlineto{\pgfqpoint{3.705004in}{2.656597in}}%
\pgfpathlineto{\pgfqpoint{3.704616in}{2.724913in}}%
\pgfpathlineto{\pgfqpoint{3.711308in}{2.792914in}}%
\pgfpathlineto{\pgfqpoint{3.724858in}{2.859894in}}%
\pgfpathlineto{\pgfqpoint{3.727238in}{2.869245in}}%
\pgfusepath{stroke}%
\end{pgfscope}%
\begin{pgfscope}%
\pgfpathrectangle{\pgfqpoint{0.647939in}{0.492442in}}{\pgfqpoint{3.079299in}{3.079299in}}%
\pgfusepath{clip}%
\pgfsetbuttcap%
\pgfsetroundjoin%
\pgfsetlinewidth{0.803000pt}%
\definecolor{currentstroke}{rgb}{0.501961,0.501961,0.501961}%
\pgfsetstrokecolor{currentstroke}%
\pgfsetdash{}{0pt}%
\pgfpathmoveto{\pgfqpoint{3.319279in}{3.233947in}}%
\pgfpathlineto{\pgfqpoint{3.362804in}{3.286736in}}%
\pgfpathlineto{\pgfqpoint{3.407880in}{3.338206in}}%
\pgfpathlineto{\pgfqpoint{3.454509in}{3.388274in}}%
\pgfpathlineto{\pgfqpoint{3.502714in}{3.436825in}}%
\pgfpathlineto{\pgfqpoint{3.552540in}{3.483710in}}%
\pgfpathlineto{\pgfqpoint{3.604034in}{3.528753in}}%
\pgfpathlineto{\pgfqpoint{3.657254in}{3.571741in}}%
\pgfpathlineto{\pgfqpoint{3.657254in}{3.571741in}}%
\pgfusepath{stroke}%
\end{pgfscope}%
\begin{pgfscope}%
\pgfpathrectangle{\pgfqpoint{0.647939in}{0.492442in}}{\pgfqpoint{3.079299in}{3.079299in}}%
\pgfusepath{clip}%
\pgfsetbuttcap%
\pgfsetroundjoin%
\pgfsetlinewidth{0.803000pt}%
\definecolor{currentstroke}{rgb}{0.501961,0.501961,0.501961}%
\pgfsetstrokecolor{currentstroke}%
\pgfsetdash{}{0pt}%
\pgfpathmoveto{\pgfqpoint{0.647939in}{2.544280in}}%
\pgfpathlineto{\pgfqpoint{0.661317in}{2.545900in}}%
\pgfpathlineto{\pgfqpoint{0.729157in}{2.554825in}}%
\pgfpathlineto{\pgfqpoint{0.796797in}{2.565154in}}%
\pgfpathlineto{\pgfqpoint{0.864207in}{2.576898in}}%
\pgfpathlineto{\pgfqpoint{0.931361in}{2.590024in}}%
\pgfpathlineto{\pgfqpoint{0.998249in}{2.604449in}}%
\pgfpathlineto{\pgfqpoint{1.064872in}{2.620050in}}%
\pgfpathlineto{\pgfqpoint{1.131252in}{2.636661in}}%
\pgfpathlineto{\pgfqpoint{1.197428in}{2.654074in}}%
\pgfpathlineto{\pgfqpoint{1.263455in}{2.672040in}}%
\pgfpathlineto{\pgfqpoint{1.329409in}{2.690278in}}%
\pgfpathlineto{\pgfqpoint{1.395374in}{2.708475in}}%
\pgfpathlineto{\pgfqpoint{1.461441in}{2.726293in}}%
\pgfpathlineto{\pgfqpoint{1.527701in}{2.743379in}}%
\pgfpathlineto{\pgfqpoint{1.594230in}{2.759375in}}%
\pgfpathlineto{\pgfqpoint{1.661086in}{2.773933in}}%
\pgfpathlineto{\pgfqpoint{1.728299in}{2.786733in}}%
\pgfpathlineto{\pgfqpoint{1.795865in}{2.797504in}}%
\pgfpathlineto{\pgfqpoint{1.863746in}{2.806065in}}%
\pgfpathlineto{\pgfqpoint{1.931874in}{2.812360in}}%
\pgfpathlineto{\pgfqpoint{2.000169in}{2.816485in}}%
\pgfpathlineto{\pgfqpoint{2.068555in}{2.818720in}}%
\pgfpathlineto{\pgfqpoint{2.136977in}{2.819552in}}%
\pgfpathlineto{\pgfqpoint{2.205405in}{2.819715in}}%
\pgfpathlineto{\pgfqpoint{2.273830in}{2.820215in}}%
\pgfpathlineto{\pgfqpoint{2.342215in}{2.822319in}}%
\pgfpathlineto{\pgfqpoint{2.410413in}{2.827511in}}%
\pgfpathlineto{\pgfqpoint{2.478070in}{2.837344in}}%
\pgfpathlineto{\pgfqpoint{2.544572in}{2.853119in}}%
\pgfpathlineto{\pgfqpoint{2.609156in}{2.875461in}}%
\pgfpathlineto{\pgfqpoint{2.671206in}{2.904111in}}%
\pgfpathlineto{\pgfqpoint{2.730500in}{2.938142in}}%
\pgfpathlineto{\pgfqpoint{2.787190in}{2.976387in}}%
\pgfpathlineto{\pgfqpoint{2.841650in}{3.017755in}}%
\pgfpathlineto{\pgfqpoint{2.894337in}{3.061374in}}%
\pgfpathlineto{\pgfqpoint{2.945684in}{3.106582in}}%
\pgfpathlineto{\pgfqpoint{2.996070in}{3.152864in}}%
\pgfpathlineto{\pgfqpoint{3.045817in}{3.199837in}}%
\pgfpathlineto{\pgfqpoint{3.095200in}{3.247198in}}%
\pgfpathlineto{\pgfqpoint{3.144453in}{3.294693in}}%
\pgfpathlineto{\pgfqpoint{3.193777in}{3.342118in}}%
\pgfpathlineto{\pgfqpoint{3.243354in}{3.389277in}}%
\pgfpathlineto{\pgfqpoint{3.293346in}{3.435997in}}%
\pgfpathlineto{\pgfqpoint{3.343907in}{3.482101in}}%
\pgfpathlineto{\pgfqpoint{3.395183in}{3.527408in}}%
\pgfpathlineto{\pgfqpoint{3.447302in}{3.571741in}}%
\pgfpathlineto{\pgfqpoint{3.447302in}{3.571741in}}%
\pgfusepath{stroke}%
\end{pgfscope}%
\begin{pgfscope}%
\pgfpathrectangle{\pgfqpoint{0.647939in}{0.492442in}}{\pgfqpoint{3.079299in}{3.079299in}}%
\pgfusepath{clip}%
\pgfsetbuttcap%
\pgfsetroundjoin%
\pgfsetlinewidth{0.803000pt}%
\definecolor{currentstroke}{rgb}{0.501961,0.501961,0.501961}%
\pgfsetstrokecolor{currentstroke}%
\pgfsetdash{}{0pt}%
\pgfpathmoveto{\pgfqpoint{0.647939in}{2.842926in}}%
\pgfpathlineto{\pgfqpoint{0.665787in}{2.844996in}}%
\pgfpathlineto{\pgfqpoint{0.733679in}{2.853517in}}%
\pgfpathlineto{\pgfqpoint{0.801398in}{2.863325in}}%
\pgfpathlineto{\pgfqpoint{0.868920in}{2.874409in}}%
\pgfpathlineto{\pgfqpoint{0.936230in}{2.886717in}}%
\pgfpathlineto{\pgfqpoint{1.003323in}{2.900158in}}%
\pgfpathlineto{\pgfqpoint{1.070209in}{2.914599in}}%
\pgfpathlineto{\pgfqpoint{1.136911in}{2.929867in}}%
\pgfpathlineto{\pgfqpoint{1.203470in}{2.945753in}}%
\pgfpathlineto{\pgfqpoint{1.269940in}{2.962011in}}%
\pgfpathlineto{\pgfqpoint{1.336386in}{2.978364in}}%
\pgfpathlineto{\pgfqpoint{1.402881in}{2.994512in}}%
\pgfpathlineto{\pgfqpoint{1.469499in}{3.010147in}}%
\pgfpathlineto{\pgfqpoint{1.536305in}{3.024950in}}%
\pgfpathlineto{\pgfqpoint{1.603353in}{3.038613in}}%
\pgfpathlineto{\pgfqpoint{1.670674in}{3.050846in}}%
\pgfpathlineto{\pgfqpoint{1.738277in}{3.061400in}}%
\pgfpathlineto{\pgfqpoint{1.806144in}{3.070095in}}%
\pgfpathlineto{\pgfqpoint{1.874232in}{3.076837in}}%
\pgfpathlineto{\pgfqpoint{1.942483in}{3.081645in}}%
\pgfpathlineto{\pgfqpoint{2.010839in}{3.084670in}}%
\pgfpathlineto{\pgfqpoint{2.079248in}{3.086207in}}%
\pgfpathlineto{\pgfqpoint{2.147673in}{3.086722in}}%
\pgfpathlineto{\pgfqpoint{2.216102in}{3.086838in}}%
\pgfpathlineto{\pgfqpoint{2.284528in}{3.087325in}}%
\pgfpathlineto{\pgfqpoint{2.352927in}{3.089089in}}%
\pgfpathlineto{\pgfqpoint{2.421221in}{3.093144in}}%
\pgfpathlineto{\pgfqpoint{2.489219in}{3.100535in}}%
\pgfpathlineto{\pgfqpoint{2.556600in}{3.112188in}}%
\pgfpathlineto{\pgfqpoint{2.622940in}{3.128719in}}%
\pgfpathlineto{\pgfqpoint{2.687802in}{3.150319in}}%
\pgfpathlineto{\pgfqpoint{2.750860in}{3.176743in}}%
\pgfpathlineto{\pgfqpoint{2.811974in}{3.207419in}}%
\pgfpathlineto{\pgfqpoint{2.871197in}{3.241625in}}%
\pgfpathlineto{\pgfqpoint{2.928724in}{3.278630in}}%
\pgfpathlineto{\pgfqpoint{2.984828in}{3.317779in}}%
\pgfpathlineto{\pgfqpoint{3.039803in}{3.358509in}}%
\pgfpathlineto{\pgfqpoint{3.093945in}{3.400345in}}%
\pgfpathlineto{\pgfqpoint{3.147528in}{3.442895in}}%
\pgfpathlineto{\pgfqpoint{3.200804in}{3.485834in}}%
\pgfpathlineto{\pgfqpoint{3.254001in}{3.528872in}}%
\pgfpathlineto{\pgfqpoint{3.307334in}{3.571741in}}%
\pgfpathlineto{\pgfqpoint{3.307334in}{3.571741in}}%
\pgfusepath{stroke}%
\end{pgfscope}%
\begin{pgfscope}%
\pgfpathrectangle{\pgfqpoint{0.647939in}{0.492442in}}{\pgfqpoint{3.079299in}{3.079299in}}%
\pgfusepath{clip}%
\pgfsetbuttcap%
\pgfsetroundjoin%
\pgfsetlinewidth{0.803000pt}%
\definecolor{currentstroke}{rgb}{0.501961,0.501961,0.501961}%
\pgfsetstrokecolor{currentstroke}%
\pgfsetdash{}{0pt}%
\pgfpathmoveto{\pgfqpoint{0.647939in}{3.027379in}}%
\pgfpathlineto{\pgfqpoint{0.663101in}{3.029070in}}%
\pgfpathlineto{\pgfqpoint{0.731035in}{3.037264in}}%
\pgfpathlineto{\pgfqpoint{0.798810in}{3.046677in}}%
\pgfpathlineto{\pgfqpoint{0.866407in}{3.057294in}}%
\pgfpathlineto{\pgfqpoint{0.933815in}{3.069059in}}%
\pgfpathlineto{\pgfqpoint{1.001030in}{3.081877in}}%
\pgfpathlineto{\pgfqpoint{1.068065in}{3.095612in}}%
\pgfpathlineto{\pgfqpoint{1.134942in}{3.110095in}}%
\pgfpathlineto{\pgfqpoint{1.201701in}{3.125117in}}%
\pgfpathlineto{\pgfqpoint{1.268392in}{3.140441in}}%
\pgfpathlineto{\pgfqpoint{1.335075in}{3.155799in}}%
\pgfpathlineto{\pgfqpoint{1.401815in}{3.170906in}}%
\pgfpathlineto{\pgfqpoint{1.468676in}{3.185467in}}%
\pgfpathlineto{\pgfqpoint{1.535713in}{3.199186in}}%
\pgfpathlineto{\pgfqpoint{1.602970in}{3.211779in}}%
\pgfpathlineto{\pgfqpoint{1.670469in}{3.222989in}}%
\pgfpathlineto{\pgfqpoint{1.738214in}{3.232599in}}%
\pgfpathlineto{\pgfqpoint{1.806183in}{3.240463in}}%
\pgfpathlineto{\pgfqpoint{1.874336in}{3.246517in}}%
\pgfpathlineto{\pgfqpoint{1.942624in}{3.250804in}}%
\pgfpathlineto{\pgfqpoint{2.010996in}{3.253480in}}%
\pgfpathlineto{\pgfqpoint{2.079409in}{3.254829in}}%
\pgfpathlineto{\pgfqpoint{2.147836in}{3.255277in}}%
\pgfpathlineto{\pgfqpoint{2.216264in}{3.255377in}}%
\pgfpathlineto{\pgfqpoint{2.284691in}{3.255793in}}%
\pgfpathlineto{\pgfqpoint{2.353099in}{3.257289in}}%
\pgfpathlineto{\pgfqpoint{2.421432in}{3.260700in}}%
\pgfpathlineto{\pgfqpoint{2.489560in}{3.266883in}}%
\pgfpathlineto{\pgfqpoint{2.557260in}{3.276615in}}%
\pgfpathlineto{\pgfqpoint{2.624229in}{3.290464in}}%
\pgfpathlineto{\pgfqpoint{2.690129in}{3.308716in}}%
\pgfpathlineto{\pgfqpoint{2.754666in}{3.331328in}}%
\pgfpathlineto{\pgfqpoint{2.817654in}{3.357972in}}%
\pgfpathlineto{\pgfqpoint{2.879044in}{3.388132in}}%
\pgfpathlineto{\pgfqpoint{2.938919in}{3.421213in}}%
\pgfpathlineto{\pgfqpoint{2.997454in}{3.456624in}}%
\pgfpathlineto{\pgfqpoint{3.054874in}{3.493822in}}%
\pgfpathlineto{\pgfqpoint{3.111428in}{3.532332in}}%
\pgfpathlineto{\pgfqpoint{3.167366in}{3.571741in}}%
\pgfpathlineto{\pgfqpoint{3.167366in}{3.571741in}}%
\pgfusepath{stroke}%
\end{pgfscope}%
\begin{pgfscope}%
\pgfpathrectangle{\pgfqpoint{0.647939in}{0.492442in}}{\pgfqpoint{3.079299in}{3.079299in}}%
\pgfusepath{clip}%
\pgfsetbuttcap%
\pgfsetroundjoin%
\pgfsetlinewidth{0.803000pt}%
\definecolor{currentstroke}{rgb}{0.501961,0.501961,0.501961}%
\pgfsetstrokecolor{currentstroke}%
\pgfsetdash{}{0pt}%
\pgfpathmoveto{\pgfqpoint{0.647939in}{3.156105in}}%
\pgfpathlineto{\pgfqpoint{0.688268in}{3.160766in}}%
\pgfpathlineto{\pgfqpoint{0.756171in}{3.169211in}}%
\pgfpathlineto{\pgfqpoint{0.823918in}{3.178828in}}%
\pgfpathlineto{\pgfqpoint{0.891493in}{3.189583in}}%
\pgfpathlineto{\pgfqpoint{0.958890in}{3.201409in}}%
\pgfpathlineto{\pgfqpoint{1.026113in}{3.214189in}}%
\pgfpathlineto{\pgfqpoint{1.093180in}{3.227772in}}%
\pgfpathlineto{\pgfqpoint{1.160118in}{3.241972in}}%
\pgfpathlineto{\pgfqpoint{1.226971in}{3.256574in}}%
\pgfpathlineto{\pgfqpoint{1.293789in}{3.271334in}}%
\pgfpathlineto{\pgfqpoint{1.360630in}{3.285987in}}%
\pgfpathlineto{\pgfqpoint{1.427555in}{3.300253in}}%
\pgfpathlineto{\pgfqpoint{1.494619in}{3.313846in}}%
\pgfpathlineto{\pgfqpoint{1.561867in}{3.326487in}}%
\pgfpathlineto{\pgfqpoint{1.629330in}{3.337917in}}%
\pgfpathlineto{\pgfqpoint{1.697020in}{3.347915in}}%
\pgfpathlineto{\pgfqpoint{1.764926in}{3.356308in}}%
\pgfpathlineto{\pgfqpoint{1.833021in}{3.362997in}}%
\pgfpathlineto{\pgfqpoint{1.901263in}{3.367975in}}%
\pgfpathlineto{\pgfqpoint{1.969604in}{3.371339in}}%
\pgfpathlineto{\pgfqpoint{2.038001in}{3.373292in}}%
\pgfpathlineto{\pgfqpoint{2.106423in}{3.374157in}}%
\pgfpathlineto{\pgfqpoint{2.174851in}{3.374372in}}%
\pgfpathlineto{\pgfqpoint{2.243280in}{3.374493in}}%
\pgfpathlineto{\pgfqpoint{2.311704in}{3.375168in}}%
\pgfpathlineto{\pgfqpoint{2.380099in}{3.377108in}}%
\pgfpathlineto{\pgfqpoint{2.448402in}{3.381064in}}%
\pgfpathlineto{\pgfqpoint{2.516482in}{3.387774in}}%
\pgfpathlineto{\pgfqpoint{2.584132in}{3.397879in}}%
\pgfpathlineto{\pgfqpoint{2.651086in}{3.411834in}}%
\pgfpathlineto{\pgfqpoint{2.717063in}{3.429835in}}%
\pgfpathlineto{\pgfqpoint{2.781825in}{3.451815in}}%
\pgfpathlineto{\pgfqpoint{2.845226in}{3.477475in}}%
\pgfpathlineto{\pgfqpoint{2.907228in}{3.506365in}}%
\pgfpathlineto{\pgfqpoint{2.967902in}{3.537963in}}%
\pgfpathlineto{\pgfqpoint{3.027398in}{3.571741in}}%
\pgfpathlineto{\pgfqpoint{3.027398in}{3.571741in}}%
\pgfusepath{stroke}%
\end{pgfscope}%
\begin{pgfscope}%
\pgfpathrectangle{\pgfqpoint{0.647939in}{0.492442in}}{\pgfqpoint{3.079299in}{3.079299in}}%
\pgfusepath{clip}%
\pgfsetbuttcap%
\pgfsetroundjoin%
\pgfsetlinewidth{0.803000pt}%
\definecolor{currentstroke}{rgb}{0.501961,0.501961,0.501961}%
\pgfsetstrokecolor{currentstroke}%
\pgfsetdash{}{0pt}%
\pgfpathmoveto{\pgfqpoint{0.647939in}{3.246383in}}%
\pgfpathlineto{\pgfqpoint{0.662910in}{3.247992in}}%
\pgfpathlineto{\pgfqpoint{0.730880in}{3.255876in}}%
\pgfpathlineto{\pgfqpoint{0.798708in}{3.264907in}}%
\pgfpathlineto{\pgfqpoint{0.866376in}{3.275064in}}%
\pgfpathlineto{\pgfqpoint{0.933877in}{3.286285in}}%
\pgfpathlineto{\pgfqpoint{1.001210in}{3.298470in}}%
\pgfpathlineto{\pgfqpoint{1.068389in}{3.311482in}}%
\pgfpathlineto{\pgfqpoint{1.135438in}{3.325153in}}%
\pgfpathlineto{\pgfqpoint{1.202393in}{3.339278in}}%
\pgfpathlineto{\pgfqpoint{1.269300in}{3.353626in}}%
\pgfpathlineto{\pgfqpoint{1.336215in}{3.367941in}}%
\pgfpathlineto{\pgfqpoint{1.403194in}{3.381950in}}%
\pgfpathlineto{\pgfqpoint{1.470291in}{3.395379in}}%
\pgfpathlineto{\pgfqpoint{1.537552in}{3.407956in}}%
\pgfpathlineto{\pgfqpoint{1.605009in}{3.419426in}}%
\pgfpathlineto{\pgfqpoint{1.672679in}{3.429562in}}%
\pgfpathlineto{\pgfqpoint{1.740557in}{3.438185in}}%
\pgfpathlineto{\pgfqpoint{1.808622in}{3.445180in}}%
\pgfpathlineto{\pgfqpoint{1.876836in}{3.450517in}}%
\pgfpathlineto{\pgfqpoint{1.945158in}{3.454254in}}%
\pgfpathlineto{\pgfqpoint{2.013544in}{3.456556in}}%
\pgfpathlineto{\pgfqpoint{2.081962in}{3.457697in}}%
\pgfpathlineto{\pgfqpoint{2.150389in}{3.458068in}}%
\pgfpathlineto{\pgfqpoint{2.218818in}{3.458155in}}%
\pgfpathlineto{\pgfqpoint{2.287245in}{3.458529in}}%
\pgfpathlineto{\pgfqpoint{2.355658in}{3.459835in}}%
\pgfpathlineto{\pgfqpoint{2.424016in}{3.462767in}}%
\pgfpathlineto{\pgfqpoint{2.492228in}{3.468026in}}%
\pgfpathlineto{\pgfqpoint{2.560138in}{3.476246in}}%
\pgfpathlineto{\pgfqpoint{2.627533in}{3.487921in}}%
\pgfpathlineto{\pgfqpoint{2.694165in}{3.503352in}}%
\pgfpathlineto{\pgfqpoint{2.759797in}{3.522603in}}%
\pgfpathlineto{\pgfqpoint{2.824249in}{3.545513in}}%
\pgfpathlineto{\pgfqpoint{2.887429in}{3.571741in}}%
\pgfpathlineto{\pgfqpoint{2.887429in}{3.571741in}}%
\pgfusepath{stroke}%
\end{pgfscope}%
\begin{pgfscope}%
\pgfpathrectangle{\pgfqpoint{0.647939in}{0.492442in}}{\pgfqpoint{3.079299in}{3.079299in}}%
\pgfusepath{clip}%
\pgfsetbuttcap%
\pgfsetroundjoin%
\pgfsetlinewidth{0.803000pt}%
\definecolor{currentstroke}{rgb}{0.501961,0.501961,0.501961}%
\pgfsetstrokecolor{currentstroke}%
\pgfsetdash{}{0pt}%
\pgfpathmoveto{\pgfqpoint{0.647939in}{3.326368in}}%
\pgfpathlineto{\pgfqpoint{0.712562in}{3.334012in}}%
\pgfpathlineto{\pgfqpoint{0.780446in}{3.342610in}}%
\pgfpathlineto{\pgfqpoint{0.848180in}{3.352314in}}%
\pgfpathlineto{\pgfqpoint{0.915756in}{3.363073in}}%
\pgfpathlineto{\pgfqpoint{0.983170in}{3.374800in}}%
\pgfpathlineto{\pgfqpoint{1.050434in}{3.387369in}}%
\pgfpathlineto{\pgfqpoint{1.117567in}{3.400617in}}%
\pgfpathlineto{\pgfqpoint{1.184604in}{3.414346in}}%
\pgfpathlineto{\pgfqpoint{1.251589in}{3.428332in}}%
\pgfpathlineto{\pgfqpoint{1.318570in}{3.442332in}}%
\pgfpathlineto{\pgfqpoint{1.385602in}{3.456085in}}%
\pgfpathlineto{\pgfqpoint{1.452737in}{3.469324in}}%
\pgfpathlineto{\pgfqpoint{1.520021in}{3.481782in}}%
\pgfpathlineto{\pgfqpoint{1.587487in}{3.493204in}}%
\pgfpathlineto{\pgfqpoint{1.655153in}{3.503366in}}%
\pgfpathlineto{\pgfqpoint{1.723019in}{3.512085in}}%
\pgfpathlineto{\pgfqpoint{1.791067in}{3.519238in}}%
\pgfpathlineto{\pgfqpoint{1.859266in}{3.524777in}}%
\pgfpathlineto{\pgfqpoint{1.927576in}{3.528743in}}%
\pgfpathlineto{\pgfqpoint{1.995954in}{3.531280in}}%
\pgfpathlineto{\pgfqpoint{2.064367in}{3.532631in}}%
\pgfpathlineto{\pgfqpoint{2.132793in}{3.533137in}}%
\pgfpathlineto{\pgfqpoint{2.201222in}{3.533235in}}%
\pgfpathlineto{\pgfqpoint{2.269650in}{3.533455in}}%
\pgfpathlineto{\pgfqpoint{2.338070in}{3.534402in}}%
\pgfpathlineto{\pgfqpoint{2.406453in}{3.536719in}}%
\pgfpathlineto{\pgfqpoint{2.474733in}{3.541055in}}%
\pgfpathlineto{\pgfqpoint{2.542789in}{3.548021in}}%
\pgfpathlineto{\pgfqpoint{2.610444in}{3.558132in}}%
\pgfpathlineto{\pgfqpoint{2.677477in}{3.571741in}}%
\pgfpathlineto{\pgfqpoint{2.677477in}{3.571741in}}%
\pgfusepath{stroke}%
\end{pgfscope}%
\begin{pgfscope}%
\pgfpathrectangle{\pgfqpoint{0.647939in}{0.492442in}}{\pgfqpoint{3.079299in}{3.079299in}}%
\pgfusepath{clip}%
\pgfsetbuttcap%
\pgfsetroundjoin%
\pgfsetlinewidth{0.803000pt}%
\definecolor{currentstroke}{rgb}{0.501961,0.501961,0.501961}%
\pgfsetstrokecolor{currentstroke}%
\pgfsetdash{}{0pt}%
\pgfpathmoveto{\pgfqpoint{1.646268in}{3.538166in}}%
\pgfpathlineto{\pgfqpoint{1.714127in}{3.546941in}}%
\pgfpathlineto{\pgfqpoint{1.782166in}{3.554181in}}%
\pgfpathlineto{\pgfqpoint{1.850356in}{3.559831in}}%
\pgfpathlineto{\pgfqpoint{1.918657in}{3.563921in}}%
\pgfpathlineto{\pgfqpoint{1.987031in}{3.566577in}}%
\pgfpathlineto{\pgfqpoint{2.055442in}{3.568030in}}%
\pgfpathlineto{\pgfqpoint{2.123868in}{3.568611in}}%
\pgfpathlineto{\pgfqpoint{2.192296in}{3.568737in}}%
\pgfpathlineto{\pgfqpoint{2.260725in}{3.568909in}}%
\pgfpathlineto{\pgfqpoint{2.329148in}{3.569701in}}%
\pgfpathlineto{\pgfqpoint{2.397541in}{3.571741in}}%
\pgfpathlineto{\pgfqpoint{2.397541in}{3.571741in}}%
\pgfusepath{stroke}%
\end{pgfscope}%
\begin{pgfscope}%
\pgfpathrectangle{\pgfqpoint{0.647939in}{0.492442in}}{\pgfqpoint{3.079299in}{3.079299in}}%
\pgfusepath{clip}%
\pgfsetbuttcap%
\pgfsetroundjoin%
\pgfsetlinewidth{0.803000pt}%
\definecolor{currentstroke}{rgb}{0.501961,0.501961,0.501961}%
\pgfsetstrokecolor{currentstroke}%
\pgfsetdash{}{0pt}%
\pgfpathmoveto{\pgfqpoint{0.647939in}{3.413966in}}%
\pgfpathlineto{\pgfqpoint{0.681923in}{3.417675in}}%
\pgfpathlineto{\pgfqpoint{0.749884in}{3.425648in}}%
\pgfpathlineto{\pgfqpoint{0.817707in}{3.434711in}}%
\pgfpathlineto{\pgfqpoint{0.885381in}{3.444829in}}%
\pgfpathlineto{\pgfqpoint{0.952902in}{3.455929in}}%
\pgfpathlineto{\pgfqpoint{1.020274in}{3.467903in}}%
\pgfpathlineto{\pgfqpoint{1.087513in}{3.480602in}}%
\pgfpathlineto{\pgfqpoint{1.154648in}{3.493843in}}%
\pgfpathlineto{\pgfqpoint{1.221717in}{3.507417in}}%
\pgfpathlineto{\pgfqpoint{1.288765in}{3.521094in}}%
\pgfpathlineto{\pgfqpoint{1.355843in}{3.534625in}}%
\pgfpathlineto{\pgfqpoint{1.423001in}{3.547749in}}%
\pgfpathlineto{\pgfqpoint{1.490285in}{3.560206in}}%
\pgfpathlineto{\pgfqpoint{1.557732in}{3.571741in}}%
\pgfpathlineto{\pgfqpoint{1.557732in}{3.571741in}}%
\pgfusepath{stroke}%
\end{pgfscope}%
\begin{pgfscope}%
\pgfpathrectangle{\pgfqpoint{0.647939in}{0.492442in}}{\pgfqpoint{3.079299in}{3.079299in}}%
\pgfusepath{clip}%
\pgfsetbuttcap%
\pgfsetroundjoin%
\pgfsetlinewidth{0.803000pt}%
\definecolor{currentstroke}{rgb}{0.501961,0.501961,0.501961}%
\pgfsetstrokecolor{currentstroke}%
\pgfsetdash{}{0pt}%
\pgfpathmoveto{\pgfqpoint{0.647939in}{3.496669in}}%
\pgfpathlineto{\pgfqpoint{0.664655in}{3.498400in}}%
\pgfpathlineto{\pgfqpoint{0.732659in}{3.505989in}}%
\pgfpathlineto{\pgfqpoint{0.800535in}{3.514652in}}%
\pgfpathlineto{\pgfqpoint{0.868270in}{3.524358in}}%
\pgfpathlineto{\pgfqpoint{0.935858in}{3.535040in}}%
\pgfpathlineto{\pgfqpoint{1.003303in}{3.546594in}}%
\pgfpathlineto{\pgfqpoint{1.070618in}{3.558884in}}%
\pgfpathlineto{\pgfqpoint{1.137828in}{3.571741in}}%
\pgfpathlineto{\pgfqpoint{1.137828in}{3.571741in}}%
\pgfusepath{stroke}%
\end{pgfscope}%
\begin{pgfscope}%
\pgfpathrectangle{\pgfqpoint{0.647939in}{0.492442in}}{\pgfqpoint{3.079299in}{3.079299in}}%
\pgfusepath{clip}%
\pgfsetbuttcap%
\pgfsetroundjoin%
\pgfsetlinewidth{0.803000pt}%
\definecolor{currentstroke}{rgb}{0.501961,0.501961,0.501961}%
\pgfsetstrokecolor{currentstroke}%
\pgfsetdash{}{0pt}%
\pgfpathmoveto{\pgfqpoint{0.647939in}{2.941885in}}%
\pgfpathlineto{\pgfqpoint{0.647939in}{2.941885in}}%
\pgfpathlineto{\pgfqpoint{0.715890in}{2.949932in}}%
\pgfpathlineto{\pgfqpoint{0.783682in}{2.959221in}}%
\pgfpathlineto{\pgfqpoint{0.851292in}{2.969751in}}%
\pgfpathlineto{\pgfqpoint{0.918705in}{2.981481in}}%
\pgfpathlineto{\pgfqpoint{0.985914in}{2.994330in}}%
\pgfpathlineto{\pgfqpoint{1.052927in}{3.008173in}}%
\pgfpathlineto{\pgfqpoint{1.119763in}{3.022845in}}%
\pgfpathlineto{\pgfqpoint{1.186460in}{3.038139in}}%
\pgfpathlineto{\pgfqpoint{1.253067in}{3.053822in}}%
\pgfpathlineto{\pgfqpoint{1.319645in}{3.069632in}}%
\pgfpathlineto{\pgfqpoint{1.386259in}{3.085286in}}%
\pgfpathlineto{\pgfqpoint{1.452978in}{3.100483in}}%
\pgfpathlineto{\pgfqpoint{1.519865in}{3.114918in}}%
\pgfpathlineto{\pgfqpoint{1.586972in}{3.128288in}}%
\pgfpathlineto{\pgfqpoint{1.654331in}{3.140313in}}%
\pgfpathlineto{\pgfqpoint{1.721953in}{3.150753in}}%
\pgfpathlineto{\pgfqpoint{1.789822in}{3.159426in}}%
\pgfpathlineto{\pgfqpoint{1.857904in}{3.166233in}}%
\pgfpathlineto{\pgfqpoint{1.926147in}{3.171175in}}%
\pgfpathlineto{\pgfqpoint{1.994496in}{3.174385in}}%
\pgfpathlineto{\pgfqpoint{2.062899in}{3.176125in}}%
\pgfpathlineto{\pgfqpoint{2.131323in}{3.176792in}}%
\pgfpathlineto{\pgfqpoint{2.199751in}{3.176926in}}%
\pgfpathlineto{\pgfqpoint{2.268179in}{3.177208in}}%
\pgfpathlineto{\pgfqpoint{2.336592in}{3.178452in}}%
\pgfpathlineto{\pgfqpoint{2.404938in}{3.181562in}}%
\pgfpathlineto{\pgfqpoint{2.473089in}{3.187464in}}%
\pgfpathlineto{\pgfqpoint{2.540811in}{3.197024in}}%
\pgfpathlineto{\pgfqpoint{2.607771in}{3.210909in}}%
\pgfpathlineto{\pgfqpoint{2.673582in}{3.229471in}}%
\pgfpathlineto{\pgfqpoint{2.737902in}{3.252673in}}%
\pgfpathlineto{\pgfqpoint{2.800521in}{3.280148in}}%
\pgfpathlineto{\pgfqpoint{2.861395in}{3.311321in}}%
\pgfusepath{stroke}%
\end{pgfscope}%
\begin{pgfscope}%
\pgfpathrectangle{\pgfqpoint{0.647939in}{0.492442in}}{\pgfqpoint{3.079299in}{3.079299in}}%
\pgfusepath{clip}%
\pgfsetbuttcap%
\pgfsetroundjoin%
\pgfsetlinewidth{0.803000pt}%
\definecolor{currentstroke}{rgb}{0.501961,0.501961,0.501961}%
\pgfsetstrokecolor{currentstroke}%
\pgfsetdash{}{0pt}%
\pgfpathmoveto{\pgfqpoint{0.647939in}{2.731932in}}%
\pgfpathlineto{\pgfqpoint{0.647939in}{2.731932in}}%
\pgfpathlineto{\pgfqpoint{0.715852in}{2.740290in}}%
\pgfpathlineto{\pgfqpoint{0.783589in}{2.749967in}}%
\pgfpathlineto{\pgfqpoint{0.851124in}{2.760971in}}%
\pgfpathlineto{\pgfqpoint{0.918435in}{2.773269in}}%
\pgfpathlineto{\pgfqpoint{0.985512in}{2.786787in}}%
\pgfpathlineto{\pgfqpoint{1.052358in}{2.801408in}}%
\pgfpathlineto{\pgfqpoint{1.118994in}{2.816963in}}%
\pgfpathlineto{\pgfqpoint{1.185456in}{2.833248in}}%
\pgfpathlineto{\pgfqpoint{1.251796in}{2.850023in}}%
\pgfpathlineto{\pgfqpoint{1.318081in}{2.867021in}}%
\pgfpathlineto{\pgfqpoint{1.384384in}{2.883944in}}%
\pgfpathlineto{\pgfqpoint{1.450785in}{2.900476in}}%
\pgfpathlineto{\pgfqpoint{1.517360in}{2.916288in}}%
\pgfpathlineto{\pgfqpoint{1.584175in}{2.931048in}}%
\pgfpathlineto{\pgfqpoint{1.651276in}{2.944440in}}%
\pgfpathlineto{\pgfqpoint{1.718683in}{2.956178in}}%
\pgfpathlineto{\pgfqpoint{1.786389in}{2.966037in}}%
\pgfpathlineto{\pgfqpoint{1.854359in}{2.973869in}}%
\pgfpathlineto{\pgfqpoint{1.922536in}{2.979634in}}%
\pgfpathlineto{\pgfqpoint{1.990852in}{2.983438in}}%
\pgfpathlineto{\pgfqpoint{2.059244in}{2.985544in}}%
\pgfpathlineto{\pgfqpoint{2.127665in}{2.986383in}}%
\pgfpathlineto{\pgfqpoint{2.196094in}{2.986561in}}%
\pgfpathlineto{\pgfqpoint{2.264521in}{2.986872in}}%
\pgfpathlineto{\pgfqpoint{2.332929in}{2.988304in}}%
\pgfpathlineto{\pgfqpoint{2.401240in}{2.992003in}}%
\pgfpathlineto{\pgfqpoint{2.469256in}{2.999162in}}%
\pgfpathlineto{\pgfqpoint{2.536616in}{3.010871in}}%
\pgfpathlineto{\pgfqpoint{2.602819in}{3.027891in}}%
\pgfpathlineto{\pgfqpoint{2.667343in}{3.050457in}}%
\pgfpathlineto{\pgfqpoint{2.729802in}{3.078255in}}%
\pgfpathlineto{\pgfqpoint{2.790056in}{3.110583in}}%
\pgfpathlineto{\pgfqpoint{2.848202in}{3.146582in}}%
\pgfpathlineto{\pgfqpoint{2.904498in}{3.185424in}}%
\pgfpathlineto{\pgfqpoint{2.959275in}{3.226399in}}%
\pgfpathlineto{\pgfqpoint{3.012869in}{3.268925in}}%
\pgfpathlineto{\pgfqpoint{3.065601in}{3.312523in}}%
\pgfpathlineto{\pgfqpoint{3.117763in}{3.356803in}}%
\pgfusepath{stroke}%
\end{pgfscope}%
\begin{pgfscope}%
\pgfpathrectangle{\pgfqpoint{0.647939in}{0.492442in}}{\pgfqpoint{3.079299in}{3.079299in}}%
\pgfusepath{clip}%
\pgfsetbuttcap%
\pgfsetroundjoin%
\pgfsetlinewidth{0.803000pt}%
\definecolor{currentstroke}{rgb}{0.501961,0.501961,0.501961}%
\pgfsetstrokecolor{currentstroke}%
\pgfsetdash{}{0pt}%
\pgfpathmoveto{\pgfqpoint{0.647939in}{2.661948in}}%
\pgfpathlineto{\pgfqpoint{0.647939in}{2.661948in}}%
\pgfpathlineto{\pgfqpoint{0.715839in}{2.670415in}}%
\pgfpathlineto{\pgfqpoint{0.783556in}{2.680228in}}%
\pgfpathlineto{\pgfqpoint{0.851063in}{2.691399in}}%
\pgfpathlineto{\pgfqpoint{0.918336in}{2.703899in}}%
\pgfpathlineto{\pgfqpoint{0.985365in}{2.717656in}}%
\pgfpathlineto{\pgfqpoint{1.052150in}{2.732553in}}%
\pgfpathlineto{\pgfqpoint{1.118710in}{2.748426in}}%
\pgfpathlineto{\pgfqpoint{1.185084in}{2.765068in}}%
\pgfpathlineto{\pgfqpoint{1.251323in}{2.782240in}}%
\pgfpathlineto{\pgfqpoint{1.317494in}{2.799670in}}%
\pgfpathlineto{\pgfqpoint{1.383677in}{2.817059in}}%
\pgfpathlineto{\pgfqpoint{1.449953in}{2.834084in}}%
\pgfpathlineto{\pgfqpoint{1.516405in}{2.850409in}}%
\pgfpathlineto{\pgfqpoint{1.583102in}{2.865692in}}%
\pgfpathlineto{\pgfqpoint{1.650096in}{2.879603in}}%
\pgfpathlineto{\pgfqpoint{1.717415in}{2.891842in}}%
\pgfpathlineto{\pgfqpoint{1.785051in}{2.902163in}}%
\pgfusepath{stroke}%
\end{pgfscope}%
\begin{pgfscope}%
\pgfpathrectangle{\pgfqpoint{0.647939in}{0.492442in}}{\pgfqpoint{3.079299in}{3.079299in}}%
\pgfusepath{clip}%
\pgfsetbuttcap%
\pgfsetroundjoin%
\pgfsetlinewidth{0.803000pt}%
\definecolor{currentstroke}{rgb}{0.501961,0.501961,0.501961}%
\pgfsetstrokecolor{currentstroke}%
\pgfsetdash{}{0pt}%
\pgfpathmoveto{\pgfqpoint{0.647939in}{2.451996in}}%
\pgfpathlineto{\pgfqpoint{0.647939in}{2.451996in}}%
\pgfpathlineto{\pgfqpoint{0.715794in}{2.460806in}}%
\pgfpathlineto{\pgfqpoint{0.783447in}{2.471052in}}%
\pgfpathlineto{\pgfqpoint{0.850863in}{2.482756in}}%
\pgfpathlineto{\pgfqpoint{0.918012in}{2.495901in}}%
\pgfpathlineto{\pgfqpoint{0.984878in}{2.510424in}}%
\pgfpathlineto{\pgfqpoint{1.051457in}{2.526217in}}%
\pgfpathlineto{\pgfqpoint{1.117763in}{2.543119in}}%
\pgfpathlineto{\pgfqpoint{1.183833in}{2.560925in}}%
\pgfpathlineto{\pgfqpoint{1.249721in}{2.579395in}}%
\pgfpathlineto{\pgfqpoint{1.315500in}{2.598252in}}%
\pgfpathlineto{\pgfqpoint{1.381257in}{2.617186in}}%
\pgfpathlineto{\pgfqpoint{1.447088in}{2.635861in}}%
\pgfpathlineto{\pgfqpoint{1.513090in}{2.653919in}}%
\pgfpathlineto{\pgfqpoint{1.579353in}{2.670988in}}%
\pgfpathlineto{\pgfqpoint{1.645949in}{2.686695in}}%
\pgfpathlineto{\pgfqpoint{1.712924in}{2.700688in}}%
\pgfpathlineto{\pgfqpoint{1.780288in}{2.712656in}}%
\pgfpathlineto{\pgfqpoint{1.848012in}{2.722369in}}%
\pgfpathlineto{\pgfqpoint{1.916034in}{2.729701in}}%
\pgfpathlineto{\pgfqpoint{1.984270in}{2.734683in}}%
\pgfpathlineto{\pgfqpoint{2.052632in}{2.737545in}}%
\pgfpathlineto{\pgfqpoint{2.121046in}{2.738756in}}%
\pgfpathlineto{\pgfqpoint{2.189474in}{2.739036in}}%
\pgfpathlineto{\pgfqpoint{2.257901in}{2.739406in}}%
\pgfpathlineto{\pgfqpoint{2.326296in}{2.741218in}}%
\pgfpathlineto{\pgfqpoint{2.394512in}{2.746169in}}%
\pgfpathlineto{\pgfqpoint{2.462134in}{2.756114in}}%
\pgfpathlineto{\pgfqpoint{2.528412in}{2.772602in}}%
\pgfpathlineto{\pgfqpoint{2.592439in}{2.796303in}}%
\pgfpathlineto{\pgfqpoint{2.653551in}{2.826774in}}%
\pgfusepath{stroke}%
\end{pgfscope}%
\begin{pgfscope}%
\pgfpathrectangle{\pgfqpoint{0.647939in}{0.492442in}}{\pgfqpoint{3.079299in}{3.079299in}}%
\pgfusepath{clip}%
\pgfsetbuttcap%
\pgfsetroundjoin%
\pgfsetlinewidth{0.803000pt}%
\definecolor{currentstroke}{rgb}{0.501961,0.501961,0.501961}%
\pgfsetstrokecolor{currentstroke}%
\pgfsetdash{}{0pt}%
\pgfpathmoveto{\pgfqpoint{0.647939in}{2.382012in}}%
\pgfpathlineto{\pgfqpoint{0.647939in}{2.382012in}}%
\pgfpathlineto{\pgfqpoint{0.715778in}{2.390943in}}%
\pgfpathlineto{\pgfqpoint{0.783408in}{2.401341in}}%
\pgfpathlineto{\pgfqpoint{0.850790in}{2.413235in}}%
\pgfpathlineto{\pgfqpoint{0.917894in}{2.426609in}}%
\pgfpathlineto{\pgfqpoint{0.984699in}{2.441406in}}%
\pgfpathlineto{\pgfqpoint{1.051200in}{2.457520in}}%
\pgfpathlineto{\pgfqpoint{1.117411in}{2.474792in}}%
\pgfpathlineto{\pgfqpoint{1.183365in}{2.493021in}}%
\pgfpathlineto{\pgfqpoint{1.249119in}{2.511964in}}%
\pgfpathlineto{\pgfqpoint{1.314746in}{2.531344in}}%
\pgfpathlineto{\pgfqpoint{1.380336in}{2.550849in}}%
\pgfpathlineto{\pgfqpoint{1.445989in}{2.570139in}}%
\pgfpathlineto{\pgfqpoint{1.511810in}{2.588847in}}%
\pgfpathlineto{\pgfqpoint{1.577894in}{2.606594in}}%
\pgfpathlineto{\pgfqpoint{1.644324in}{2.622992in}}%
\pgfpathlineto{\pgfqpoint{1.711152in}{2.637670in}}%
\pgfpathlineto{\pgfqpoint{1.778396in}{2.650294in}}%
\pgfpathlineto{\pgfqpoint{1.846030in}{2.660604in}}%
\pgfpathlineto{\pgfqpoint{1.913994in}{2.668448in}}%
\pgfpathlineto{\pgfqpoint{1.982199in}{2.673830in}}%
\pgfpathlineto{\pgfqpoint{2.050548in}{2.676959in}}%
\pgfpathlineto{\pgfqpoint{2.118959in}{2.678305in}}%
\pgfpathlineto{\pgfqpoint{2.187386in}{2.678624in}}%
\pgfpathlineto{\pgfqpoint{2.255813in}{2.679011in}}%
\pgfpathlineto{\pgfqpoint{2.324202in}{2.680959in}}%
\pgfpathlineto{\pgfqpoint{2.392375in}{2.686376in}}%
\pgfusepath{stroke}%
\end{pgfscope}%
\begin{pgfscope}%
\pgfpathrectangle{\pgfqpoint{0.647939in}{0.492442in}}{\pgfqpoint{3.079299in}{3.079299in}}%
\pgfusepath{clip}%
\pgfsetbuttcap%
\pgfsetroundjoin%
\pgfsetlinewidth{0.803000pt}%
\definecolor{currentstroke}{rgb}{0.501961,0.501961,0.501961}%
\pgfsetstrokecolor{currentstroke}%
\pgfsetdash{}{0pt}%
\pgfpathmoveto{\pgfqpoint{0.647939in}{2.312028in}}%
\pgfpathlineto{\pgfqpoint{0.647939in}{2.312028in}}%
\pgfpathlineto{\pgfqpoint{0.715762in}{2.321083in}}%
\pgfpathlineto{\pgfqpoint{0.783366in}{2.331639in}}%
\pgfpathlineto{\pgfqpoint{0.850714in}{2.343727in}}%
\pgfpathlineto{\pgfqpoint{0.917770in}{2.357339in}}%
\pgfpathlineto{\pgfqpoint{0.984511in}{2.372420in}}%
\pgfpathlineto{\pgfqpoint{1.050930in}{2.388867in}}%
\pgfpathlineto{\pgfqpoint{1.117038in}{2.406526in}}%
\pgfpathlineto{\pgfqpoint{1.182869in}{2.425195in}}%
\pgfpathlineto{\pgfqpoint{1.248478in}{2.444634in}}%
\pgfpathlineto{\pgfqpoint{1.313940in}{2.464564in}}%
\pgfpathlineto{\pgfqpoint{1.379348in}{2.484670in}}%
\pgfpathlineto{\pgfqpoint{1.444807in}{2.504610in}}%
\pgfpathlineto{\pgfqpoint{1.510427in}{2.524011in}}%
\pgfpathlineto{\pgfqpoint{1.576312in}{2.542483in}}%
\pgfusepath{stroke}%
\end{pgfscope}%
\begin{pgfscope}%
\pgfpathrectangle{\pgfqpoint{0.647939in}{0.492442in}}{\pgfqpoint{3.079299in}{3.079299in}}%
\pgfusepath{clip}%
\pgfsetbuttcap%
\pgfsetroundjoin%
\pgfsetlinewidth{0.803000pt}%
\definecolor{currentstroke}{rgb}{0.501961,0.501961,0.501961}%
\pgfsetstrokecolor{currentstroke}%
\pgfsetdash{}{0pt}%
\pgfpathmoveto{\pgfqpoint{0.647939in}{2.242044in}}%
\pgfpathlineto{\pgfqpoint{0.647939in}{2.242044in}}%
\pgfpathlineto{\pgfqpoint{0.715744in}{2.251227in}}%
\pgfpathlineto{\pgfqpoint{0.783323in}{2.261944in}}%
\pgfpathlineto{\pgfqpoint{0.850634in}{2.274234in}}%
\pgfpathlineto{\pgfqpoint{0.917639in}{2.288091in}}%
\pgfpathlineto{\pgfqpoint{0.984313in}{2.303466in}}%
\pgfpathlineto{\pgfqpoint{1.050645in}{2.320260in}}%
\pgfpathlineto{\pgfqpoint{1.116644in}{2.338322in}}%
\pgfpathlineto{\pgfqpoint{1.182342in}{2.357452in}}%
\pgfpathlineto{\pgfqpoint{1.247795in}{2.377410in}}%
\pgfpathlineto{\pgfqpoint{1.313078in}{2.397917in}}%
\pgfusepath{stroke}%
\end{pgfscope}%
\begin{pgfscope}%
\pgfpathrectangle{\pgfqpoint{0.647939in}{0.492442in}}{\pgfqpoint{3.079299in}{3.079299in}}%
\pgfusepath{clip}%
\pgfsetbuttcap%
\pgfsetroundjoin%
\pgfsetlinewidth{0.803000pt}%
\definecolor{currentstroke}{rgb}{0.501961,0.501961,0.501961}%
\pgfsetstrokecolor{currentstroke}%
\pgfsetdash{}{0pt}%
\pgfpathmoveto{\pgfqpoint{0.647939in}{2.172060in}}%
\pgfpathlineto{\pgfqpoint{0.647939in}{2.172060in}}%
\pgfpathlineto{\pgfqpoint{0.715726in}{2.181374in}}%
\pgfpathlineto{\pgfqpoint{0.783278in}{2.192258in}}%
\pgfpathlineto{\pgfqpoint{0.850550in}{2.204755in}}%
\pgfpathlineto{\pgfqpoint{0.917502in}{2.218867in}}%
\pgfpathlineto{\pgfqpoint{0.984104in}{2.234548in}}%
\pgfpathlineto{\pgfqpoint{1.050343in}{2.251703in}}%
\pgfpathlineto{\pgfqpoint{1.116226in}{2.270184in}}%
\pgfpathlineto{\pgfqpoint{1.181782in}{2.289796in}}%
\pgfpathlineto{\pgfqpoint{1.247065in}{2.310298in}}%
\pgfpathlineto{\pgfqpoint{1.312154in}{2.331413in}}%
\pgfusepath{stroke}%
\end{pgfscope}%
\begin{pgfscope}%
\pgfpathrectangle{\pgfqpoint{0.647939in}{0.492442in}}{\pgfqpoint{3.079299in}{3.079299in}}%
\pgfusepath{clip}%
\pgfsetbuttcap%
\pgfsetroundjoin%
\pgfsetlinewidth{0.803000pt}%
\definecolor{currentstroke}{rgb}{0.501961,0.501961,0.501961}%
\pgfsetstrokecolor{currentstroke}%
\pgfsetdash{}{0pt}%
\pgfpathmoveto{\pgfqpoint{0.647939in}{2.102076in}}%
\pgfpathlineto{\pgfqpoint{0.647939in}{2.102076in}}%
\pgfpathlineto{\pgfqpoint{0.715707in}{2.111525in}}%
\pgfpathlineto{\pgfqpoint{0.783231in}{2.122581in}}%
\pgfpathlineto{\pgfqpoint{0.850462in}{2.135293in}}%
\pgfpathlineto{\pgfqpoint{0.917358in}{2.149668in}}%
\pgfpathlineto{\pgfqpoint{0.983884in}{2.165666in}}%
\pgfpathlineto{\pgfqpoint{1.050024in}{2.183196in}}%
\pgfpathlineto{\pgfqpoint{1.115782in}{2.202116in}}%
\pgfpathlineto{\pgfqpoint{1.181185in}{2.222232in}}%
\pgfpathlineto{\pgfqpoint{1.246286in}{2.243305in}}%
\pgfpathlineto{\pgfqpoint{1.311164in}{2.265059in}}%
\pgfpathlineto{\pgfqpoint{1.375919in}{2.287179in}}%
\pgfpathlineto{\pgfqpoint{1.440669in}{2.309313in}}%
\pgfpathlineto{\pgfqpoint{1.505543in}{2.331076in}}%
\pgfpathlineto{\pgfqpoint{1.570674in}{2.352056in}}%
\pgfpathlineto{\pgfqpoint{1.636184in}{2.371813in}}%
\pgfpathlineto{\pgfqpoint{1.702171in}{2.389899in}}%
\pgfpathlineto{\pgfqpoint{1.768698in}{2.405873in}}%
\pgfpathlineto{\pgfqpoint{1.835774in}{2.419337in}}%
\pgfpathlineto{\pgfqpoint{1.903350in}{2.429986in}}%
\pgfpathlineto{\pgfqpoint{1.971326in}{2.437663in}}%
\pgfpathlineto{\pgfqpoint{2.039571in}{2.442438in}}%
\pgfpathlineto{\pgfqpoint{2.107952in}{2.444696in}}%
\pgfpathlineto{\pgfqpoint{2.176376in}{2.445291in}}%
\pgfpathlineto{\pgfqpoint{2.244802in}{2.445748in}}%
\pgfpathlineto{\pgfqpoint{2.313139in}{2.448535in}}%
\pgfpathlineto{\pgfqpoint{2.380850in}{2.457321in}}%
\pgfpathlineto{\pgfqpoint{2.446288in}{2.476090in}}%
\pgfpathlineto{\pgfqpoint{2.507282in}{2.506159in}}%
\pgfpathlineto{\pgfqpoint{2.562463in}{2.544615in}}%
\pgfpathlineto{\pgfqpoint{2.614109in}{2.589207in}}%
\pgfusepath{stroke}%
\end{pgfscope}%
\begin{pgfscope}%
\pgfpathrectangle{\pgfqpoint{0.647939in}{0.492442in}}{\pgfqpoint{3.079299in}{3.079299in}}%
\pgfusepath{clip}%
\pgfsetbuttcap%
\pgfsetroundjoin%
\pgfsetlinewidth{0.803000pt}%
\definecolor{currentstroke}{rgb}{0.501961,0.501961,0.501961}%
\pgfsetstrokecolor{currentstroke}%
\pgfsetdash{}{0pt}%
\pgfpathmoveto{\pgfqpoint{0.647939in}{2.032092in}}%
\pgfpathlineto{\pgfqpoint{0.647939in}{2.032092in}}%
\pgfpathlineto{\pgfqpoint{0.715688in}{2.041680in}}%
\pgfpathlineto{\pgfqpoint{0.783182in}{2.052913in}}%
\pgfpathlineto{\pgfqpoint{0.850370in}{2.065848in}}%
\pgfpathlineto{\pgfqpoint{0.917206in}{2.080496in}}%
\pgfpathlineto{\pgfqpoint{0.983652in}{2.096823in}}%
\pgfpathlineto{\pgfqpoint{1.049687in}{2.114744in}}%
\pgfpathlineto{\pgfqpoint{1.115310in}{2.134121in}}%
\pgfpathlineto{\pgfqpoint{1.180548in}{2.154764in}}%
\pgfpathlineto{\pgfqpoint{1.245452in}{2.176437in}}%
\pgfpathlineto{\pgfqpoint{1.310100in}{2.198865in}}%
\pgfpathlineto{\pgfqpoint{1.374594in}{2.221733in}}%
\pgfpathlineto{\pgfqpoint{1.439057in}{2.244690in}}%
\pgfpathlineto{\pgfqpoint{1.503624in}{2.267349in}}%
\pgfpathlineto{\pgfqpoint{1.568438in}{2.289290in}}%
\pgfpathlineto{\pgfqpoint{1.633632in}{2.310066in}}%
\pgfpathlineto{\pgfqpoint{1.699319in}{2.329210in}}%
\pgfpathlineto{\pgfqpoint{1.765580in}{2.346254in}}%
\pgfpathlineto{\pgfqpoint{1.832437in}{2.360760in}}%
\pgfpathlineto{\pgfqpoint{1.899854in}{2.372370in}}%
\pgfpathlineto{\pgfqpoint{1.967731in}{2.380873in}}%
\pgfpathlineto{\pgfqpoint{2.035925in}{2.386279in}}%
\pgfpathlineto{\pgfqpoint{2.104289in}{2.388930in}}%
\pgfpathlineto{\pgfqpoint{2.172710in}{2.389663in}}%
\pgfpathlineto{\pgfqpoint{2.241136in}{2.390131in}}%
\pgfpathlineto{\pgfqpoint{2.309444in}{2.393264in}}%
\pgfpathlineto{\pgfqpoint{2.376825in}{2.403692in}}%
\pgfusepath{stroke}%
\end{pgfscope}%
\begin{pgfscope}%
\pgfpathrectangle{\pgfqpoint{0.647939in}{0.492442in}}{\pgfqpoint{3.079299in}{3.079299in}}%
\pgfusepath{clip}%
\pgfsetbuttcap%
\pgfsetroundjoin%
\pgfsetlinewidth{0.803000pt}%
\definecolor{currentstroke}{rgb}{0.501961,0.501961,0.501961}%
\pgfsetstrokecolor{currentstroke}%
\pgfsetdash{}{0pt}%
\pgfpathmoveto{\pgfqpoint{0.647939in}{1.962108in}}%
\pgfpathlineto{\pgfqpoint{0.647939in}{1.962108in}}%
\pgfpathlineto{\pgfqpoint{0.715667in}{1.971839in}}%
\pgfpathlineto{\pgfqpoint{0.783130in}{1.983256in}}%
\pgfpathlineto{\pgfqpoint{0.850273in}{1.996420in}}%
\pgfpathlineto{\pgfqpoint{0.917046in}{2.011351in}}%
\pgfpathlineto{\pgfqpoint{0.983406in}{2.028020in}}%
\pgfpathlineto{\pgfqpoint{1.049329in}{2.046349in}}%
\pgfpathlineto{\pgfqpoint{1.114809in}{2.066204in}}%
\pgfpathlineto{\pgfqpoint{1.179869in}{2.087399in}}%
\pgfpathlineto{\pgfqpoint{1.244559in}{2.109702in}}%
\pgfpathlineto{\pgfqpoint{1.308956in}{2.132839in}}%
\pgfpathlineto{\pgfqpoint{1.373163in}{2.156499in}}%
\pgfpathlineto{\pgfqpoint{1.437308in}{2.180330in}}%
\pgfpathlineto{\pgfqpoint{1.501533in}{2.203945in}}%
\pgfpathlineto{\pgfqpoint{1.565987in}{2.226920in}}%
\pgfpathlineto{\pgfqpoint{1.630819in}{2.248799in}}%
\pgfpathlineto{\pgfqpoint{1.696158in}{2.269102in}}%
\pgfusepath{stroke}%
\end{pgfscope}%
\begin{pgfscope}%
\pgfpathrectangle{\pgfqpoint{0.647939in}{0.492442in}}{\pgfqpoint{3.079299in}{3.079299in}}%
\pgfusepath{clip}%
\pgfsetbuttcap%
\pgfsetroundjoin%
\pgfsetlinewidth{0.803000pt}%
\definecolor{currentstroke}{rgb}{0.501961,0.501961,0.501961}%
\pgfsetstrokecolor{currentstroke}%
\pgfsetdash{}{0pt}%
\pgfpathmoveto{\pgfqpoint{0.647939in}{1.892124in}}%
\pgfpathlineto{\pgfqpoint{0.647939in}{1.892124in}}%
\pgfpathlineto{\pgfqpoint{0.715645in}{1.902002in}}%
\pgfpathlineto{\pgfqpoint{0.783076in}{1.913608in}}%
\pgfpathlineto{\pgfqpoint{0.850172in}{1.927010in}}%
\pgfpathlineto{\pgfqpoint{0.916877in}{1.942234in}}%
\pgfpathlineto{\pgfqpoint{0.983147in}{1.959260in}}%
\pgfpathlineto{\pgfqpoint{1.048949in}{1.978013in}}%
\pgfusepath{stroke}%
\end{pgfscope}%
\begin{pgfscope}%
\pgfpathrectangle{\pgfqpoint{0.647939in}{0.492442in}}{\pgfqpoint{3.079299in}{3.079299in}}%
\pgfusepath{clip}%
\pgfsetbuttcap%
\pgfsetroundjoin%
\pgfsetlinewidth{0.803000pt}%
\definecolor{currentstroke}{rgb}{0.501961,0.501961,0.501961}%
\pgfsetstrokecolor{currentstroke}%
\pgfsetdash{}{0pt}%
\pgfpathmoveto{\pgfqpoint{0.647939in}{1.752155in}}%
\pgfpathlineto{\pgfqpoint{0.647939in}{1.752155in}}%
\pgfpathlineto{\pgfqpoint{0.715599in}{1.762342in}}%
\pgfpathlineto{\pgfqpoint{0.782959in}{1.774345in}}%
\pgfpathlineto{\pgfqpoint{0.849952in}{1.788248in}}%
\pgfpathlineto{\pgfqpoint{0.916511in}{1.804094in}}%
\pgfpathlineto{\pgfqpoint{0.982580in}{1.821876in}}%
\pgfpathlineto{\pgfqpoint{1.048116in}{1.841535in}}%
\pgfpathlineto{\pgfqpoint{1.113099in}{1.862957in}}%
\pgfpathlineto{\pgfqpoint{1.177535in}{1.885974in}}%
\pgfpathlineto{\pgfqpoint{1.241464in}{1.910367in}}%
\pgfpathlineto{\pgfqpoint{1.304959in}{1.935874in}}%
\pgfpathlineto{\pgfqpoint{1.368121in}{1.962196in}}%
\pgfpathlineto{\pgfqpoint{1.431085in}{1.988990in}}%
\pgfpathlineto{\pgfqpoint{1.494009in}{2.015879in}}%
\pgfpathlineto{\pgfqpoint{1.557071in}{2.042442in}}%
\pgfpathlineto{\pgfqpoint{1.620457in}{2.068215in}}%
\pgfpathlineto{\pgfqpoint{1.684351in}{2.092691in}}%
\pgfpathlineto{\pgfqpoint{1.748917in}{2.115322in}}%
\pgfpathlineto{\pgfqpoint{1.814277in}{2.135526in}}%
\pgfpathlineto{\pgfqpoint{1.880483in}{2.152720in}}%
\pgfpathlineto{\pgfqpoint{1.947504in}{2.166369in}}%
\pgfpathlineto{\pgfqpoint{2.015206in}{2.176070in}}%
\pgfpathlineto{\pgfqpoint{2.083368in}{2.181745in}}%
\pgfpathlineto{\pgfqpoint{2.151743in}{2.183950in}}%
\pgfpathlineto{\pgfqpoint{2.220164in}{2.184782in}}%
\pgfpathlineto{\pgfqpoint{2.287536in}{2.192781in}}%
\pgfpathlineto{\pgfqpoint{2.287536in}{2.192781in}}%
\pgfusepath{stroke}%
\end{pgfscope}%
\begin{pgfscope}%
\pgfpathrectangle{\pgfqpoint{0.647939in}{0.492442in}}{\pgfqpoint{3.079299in}{3.079299in}}%
\pgfusepath{clip}%
\pgfsetbuttcap%
\pgfsetroundjoin%
\pgfsetlinewidth{0.803000pt}%
\definecolor{currentstroke}{rgb}{0.501961,0.501961,0.501961}%
\pgfsetstrokecolor{currentstroke}%
\pgfsetdash{}{0pt}%
\pgfpathmoveto{\pgfqpoint{0.647939in}{1.682171in}}%
\pgfpathlineto{\pgfqpoint{0.647939in}{1.682171in}}%
\pgfpathlineto{\pgfqpoint{0.715574in}{1.692520in}}%
\pgfpathlineto{\pgfqpoint{0.782896in}{1.704731in}}%
\pgfpathlineto{\pgfqpoint{0.849833in}{1.718899in}}%
\pgfpathlineto{\pgfqpoint{0.916313in}{1.735074in}}%
\pgfpathlineto{\pgfqpoint{0.982271in}{1.753258in}}%
\pgfpathlineto{\pgfqpoint{1.047659in}{1.773400in}}%
\pgfpathlineto{\pgfqpoint{1.112450in}{1.795393in}}%
\pgfpathlineto{\pgfqpoint{1.176643in}{1.819077in}}%
\pgfpathlineto{\pgfqpoint{1.240272in}{1.844241in}}%
\pgfpathlineto{\pgfqpoint{1.303406in}{1.870627in}}%
\pgfpathlineto{\pgfqpoint{1.366145in}{1.897941in}}%
\pgfpathlineto{\pgfqpoint{1.428623in}{1.925849in}}%
\pgfpathlineto{\pgfqpoint{1.491003in}{1.953979in}}%
\pgfpathlineto{\pgfqpoint{1.553468in}{1.981916in}}%
\pgfpathlineto{\pgfqpoint{1.616218in}{2.009204in}}%
\pgfpathlineto{\pgfqpoint{1.679454in}{2.035335in}}%
\pgfpathlineto{\pgfqpoint{1.743366in}{2.059754in}}%
\pgfusepath{stroke}%
\end{pgfscope}%
\begin{pgfscope}%
\pgfpathrectangle{\pgfqpoint{0.647939in}{0.492442in}}{\pgfqpoint{3.079299in}{3.079299in}}%
\pgfusepath{clip}%
\pgfsetbuttcap%
\pgfsetroundjoin%
\pgfsetlinewidth{0.803000pt}%
\definecolor{currentstroke}{rgb}{0.501961,0.501961,0.501961}%
\pgfsetstrokecolor{currentstroke}%
\pgfsetdash{}{0pt}%
\pgfpathmoveto{\pgfqpoint{0.647939in}{1.612187in}}%
\pgfpathlineto{\pgfqpoint{0.647939in}{1.612187in}}%
\pgfpathlineto{\pgfqpoint{0.715548in}{1.622702in}}%
\pgfpathlineto{\pgfqpoint{0.782830in}{1.635129in}}%
\pgfpathlineto{\pgfqpoint{0.849708in}{1.649571in}}%
\pgfpathlineto{\pgfqpoint{0.916102in}{1.666088in}}%
\pgfpathlineto{\pgfqpoint{0.981943in}{1.684692in}}%
\pgfpathlineto{\pgfqpoint{1.047172in}{1.705338in}}%
\pgfpathlineto{\pgfqpoint{1.111756in}{1.727930in}}%
\pgfpathlineto{\pgfqpoint{1.175685in}{1.752314in}}%
\pgfpathlineto{\pgfqpoint{1.238987in}{1.778287in}}%
\pgfpathlineto{\pgfqpoint{1.301725in}{1.805601in}}%
\pgfpathlineto{\pgfqpoint{1.363996in}{1.833965in}}%
\pgfusepath{stroke}%
\end{pgfscope}%
\begin{pgfscope}%
\pgfpathrectangle{\pgfqpoint{0.647939in}{0.492442in}}{\pgfqpoint{3.079299in}{3.079299in}}%
\pgfusepath{clip}%
\pgfsetbuttcap%
\pgfsetroundjoin%
\pgfsetlinewidth{0.803000pt}%
\definecolor{currentstroke}{rgb}{0.501961,0.501961,0.501961}%
\pgfsetstrokecolor{currentstroke}%
\pgfsetdash{}{0pt}%
\pgfpathmoveto{\pgfqpoint{0.647939in}{1.542203in}}%
\pgfpathlineto{\pgfqpoint{0.647939in}{1.542203in}}%
\pgfpathlineto{\pgfqpoint{0.715521in}{1.552890in}}%
\pgfpathlineto{\pgfqpoint{0.782761in}{1.565540in}}%
\pgfpathlineto{\pgfqpoint{0.849576in}{1.580267in}}%
\pgfpathlineto{\pgfqpoint{0.915880in}{1.597140in}}%
\pgfpathlineto{\pgfqpoint{0.981594in}{1.616181in}}%
\pgfpathlineto{\pgfqpoint{1.046653in}{1.637356in}}%
\pgfpathlineto{\pgfqpoint{1.111012in}{1.660575in}}%
\pgfpathlineto{\pgfqpoint{1.174655in}{1.685693in}}%
\pgfpathlineto{\pgfqpoint{1.237601in}{1.712516in}}%
\pgfpathlineto{\pgfqpoint{1.299904in}{1.740803in}}%
\pgfpathlineto{\pgfqpoint{1.361660in}{1.770272in}}%
\pgfpathlineto{\pgfqpoint{1.422996in}{1.800605in}}%
\pgfpathlineto{\pgfqpoint{1.484078in}{1.831450in}}%
\pgfpathlineto{\pgfqpoint{1.545098in}{1.862417in}}%
\pgfpathlineto{\pgfqpoint{1.606275in}{1.893072in}}%
\pgfpathlineto{\pgfqpoint{1.667841in}{1.922936in}}%
\pgfpathlineto{\pgfqpoint{1.730031in}{1.951465in}}%
\pgfpathlineto{\pgfqpoint{1.793068in}{1.978054in}}%
\pgfpathlineto{\pgfqpoint{1.857134in}{2.002028in}}%
\pgfpathlineto{\pgfqpoint{1.922343in}{2.022657in}}%
\pgfpathlineto{\pgfqpoint{1.988695in}{2.039205in}}%
\pgfpathlineto{\pgfqpoint{2.056035in}{2.051032in}}%
\pgfusepath{stroke}%
\end{pgfscope}%
\begin{pgfscope}%
\pgfpathrectangle{\pgfqpoint{0.647939in}{0.492442in}}{\pgfqpoint{3.079299in}{3.079299in}}%
\pgfusepath{clip}%
\pgfsetbuttcap%
\pgfsetroundjoin%
\pgfsetlinewidth{0.803000pt}%
\definecolor{currentstroke}{rgb}{0.501961,0.501961,0.501961}%
\pgfsetstrokecolor{currentstroke}%
\pgfsetdash{}{0pt}%
\pgfpathmoveto{\pgfqpoint{0.647939in}{1.472219in}}%
\pgfpathlineto{\pgfqpoint{0.647939in}{1.472219in}}%
\pgfpathlineto{\pgfqpoint{0.715492in}{1.483084in}}%
\pgfpathlineto{\pgfqpoint{0.782688in}{1.495965in}}%
\pgfpathlineto{\pgfqpoint{0.849436in}{1.510987in}}%
\pgfpathlineto{\pgfqpoint{0.915643in}{1.528231in}}%
\pgfpathlineto{\pgfqpoint{0.981223in}{1.547729in}}%
\pgfpathlineto{\pgfqpoint{1.046098in}{1.569456in}}%
\pgfpathlineto{\pgfqpoint{1.110214in}{1.593333in}}%
\pgfpathlineto{\pgfqpoint{1.173546in}{1.619223in}}%
\pgfpathlineto{\pgfqpoint{1.236102in}{1.646939in}}%
\pgfpathlineto{\pgfqpoint{1.297929in}{1.676249in}}%
\pgfpathlineto{\pgfqpoint{1.359114in}{1.706881in}}%
\pgfpathlineto{\pgfqpoint{1.419783in}{1.738527in}}%
\pgfpathlineto{\pgfqpoint{1.480096in}{1.770847in}}%
\pgfpathlineto{\pgfqpoint{1.540247in}{1.803468in}}%
\pgfpathlineto{\pgfqpoint{1.600459in}{1.835976in}}%
\pgfusepath{stroke}%
\end{pgfscope}%
\begin{pgfscope}%
\pgfpathrectangle{\pgfqpoint{0.647939in}{0.492442in}}{\pgfqpoint{3.079299in}{3.079299in}}%
\pgfusepath{clip}%
\pgfsetbuttcap%
\pgfsetroundjoin%
\pgfsetlinewidth{0.803000pt}%
\definecolor{currentstroke}{rgb}{0.501961,0.501961,0.501961}%
\pgfsetstrokecolor{currentstroke}%
\pgfsetdash{}{0pt}%
\pgfpathmoveto{\pgfqpoint{0.647939in}{1.402235in}}%
\pgfpathlineto{\pgfqpoint{0.647939in}{1.402235in}}%
\pgfpathlineto{\pgfqpoint{0.715462in}{1.413283in}}%
\pgfpathlineto{\pgfqpoint{0.782611in}{1.426404in}}%
\pgfpathlineto{\pgfqpoint{0.849288in}{1.441734in}}%
\pgfpathlineto{\pgfqpoint{0.915393in}{1.459364in}}%
\pgfpathlineto{\pgfqpoint{0.980827in}{1.479338in}}%
\pgfpathlineto{\pgfqpoint{1.045504in}{1.501644in}}%
\pgfpathlineto{\pgfqpoint{1.109357in}{1.526210in}}%
\pgfpathlineto{\pgfqpoint{1.172350in}{1.552911in}}%
\pgfusepath{stroke}%
\end{pgfscope}%
\begin{pgfscope}%
\pgfpathrectangle{\pgfqpoint{0.647939in}{0.492442in}}{\pgfqpoint{3.079299in}{3.079299in}}%
\pgfusepath{clip}%
\pgfsetbuttcap%
\pgfsetroundjoin%
\pgfsetlinewidth{0.803000pt}%
\definecolor{currentstroke}{rgb}{0.501961,0.501961,0.501961}%
\pgfsetstrokecolor{currentstroke}%
\pgfsetdash{}{0pt}%
\pgfpathmoveto{\pgfqpoint{0.647939in}{1.332251in}}%
\pgfpathlineto{\pgfqpoint{0.647939in}{1.332251in}}%
\pgfpathlineto{\pgfqpoint{0.715430in}{1.343489in}}%
\pgfpathlineto{\pgfqpoint{0.782529in}{1.356858in}}%
\pgfpathlineto{\pgfqpoint{0.849131in}{1.372507in}}%
\pgfpathlineto{\pgfqpoint{0.915126in}{1.390540in}}%
\pgfpathlineto{\pgfqpoint{0.980405in}{1.411013in}}%
\pgfpathlineto{\pgfqpoint{1.044868in}{1.433924in}}%
\pgfpathlineto{\pgfqpoint{1.108436in}{1.459215in}}%
\pgfpathlineto{\pgfqpoint{1.171059in}{1.486767in}}%
\pgfpathlineto{\pgfqpoint{1.232720in}{1.516412in}}%
\pgfpathlineto{\pgfqpoint{1.293445in}{1.547933in}}%
\pgfpathlineto{\pgfqpoint{1.353302in}{1.581078in}}%
\pgfpathlineto{\pgfqpoint{1.412398in}{1.615562in}}%
\pgfpathlineto{\pgfqpoint{1.470884in}{1.651074in}}%
\pgfpathlineto{\pgfqpoint{1.528946in}{1.687276in}}%
\pgfpathlineto{\pgfqpoint{1.586805in}{1.723801in}}%
\pgfpathlineto{\pgfqpoint{1.644716in}{1.760245in}}%
\pgfpathlineto{\pgfqpoint{1.702959in}{1.796154in}}%
\pgfpathlineto{\pgfqpoint{1.761832in}{1.831019in}}%
\pgfpathlineto{\pgfqpoint{1.821625in}{1.864267in}}%
\pgfpathlineto{\pgfqpoint{1.882608in}{1.895254in}}%
\pgfusepath{stroke}%
\end{pgfscope}%
\begin{pgfscope}%
\pgfpathrectangle{\pgfqpoint{0.647939in}{0.492442in}}{\pgfqpoint{3.079299in}{3.079299in}}%
\pgfusepath{clip}%
\pgfsetbuttcap%
\pgfsetroundjoin%
\pgfsetlinewidth{0.803000pt}%
\definecolor{currentstroke}{rgb}{0.501961,0.501961,0.501961}%
\pgfsetstrokecolor{currentstroke}%
\pgfsetdash{}{0pt}%
\pgfpathmoveto{\pgfqpoint{0.647939in}{1.262267in}}%
\pgfpathlineto{\pgfqpoint{0.647939in}{1.262267in}}%
\pgfpathlineto{\pgfqpoint{0.715397in}{1.273701in}}%
\pgfpathlineto{\pgfqpoint{0.782443in}{1.287328in}}%
\pgfpathlineto{\pgfqpoint{0.848965in}{1.303309in}}%
\pgfpathlineto{\pgfqpoint{0.914843in}{1.321762in}}%
\pgfpathlineto{\pgfqpoint{0.979954in}{1.342756in}}%
\pgfpathlineto{\pgfqpoint{1.044186in}{1.366303in}}%
\pgfpathlineto{\pgfqpoint{1.107444in}{1.392353in}}%
\pgfpathlineto{\pgfqpoint{1.169663in}{1.420801in}}%
\pgfpathlineto{\pgfqpoint{1.230811in}{1.451484in}}%
\pgfpathlineto{\pgfqpoint{1.290900in}{1.484195in}}%
\pgfpathlineto{\pgfqpoint{1.349984in}{1.518691in}}%
\pgfpathlineto{\pgfqpoint{1.408162in}{1.554698in}}%
\pgfpathlineto{\pgfqpoint{1.465573in}{1.591918in}}%
\pgfpathlineto{\pgfqpoint{1.522397in}{1.630032in}}%
\pgfusepath{stroke}%
\end{pgfscope}%
\begin{pgfscope}%
\pgfpathrectangle{\pgfqpoint{0.647939in}{0.492442in}}{\pgfqpoint{3.079299in}{3.079299in}}%
\pgfusepath{clip}%
\pgfsetbuttcap%
\pgfsetroundjoin%
\pgfsetlinewidth{0.803000pt}%
\definecolor{currentstroke}{rgb}{0.501961,0.501961,0.501961}%
\pgfsetstrokecolor{currentstroke}%
\pgfsetdash{}{0pt}%
\pgfpathmoveto{\pgfqpoint{0.647939in}{1.192283in}}%
\pgfpathlineto{\pgfqpoint{0.647939in}{1.192283in}}%
\pgfpathlineto{\pgfqpoint{0.715362in}{1.203920in}}%
\pgfpathlineto{\pgfqpoint{0.782352in}{1.217815in}}%
\pgfpathlineto{\pgfqpoint{0.848789in}{1.234142in}}%
\pgfpathlineto{\pgfqpoint{0.914541in}{1.253033in}}%
\pgfpathlineto{\pgfqpoint{0.979472in}{1.274573in}}%
\pgfusepath{stroke}%
\end{pgfscope}%
\begin{pgfscope}%
\pgfpathrectangle{\pgfqpoint{0.647939in}{0.492442in}}{\pgfqpoint{3.079299in}{3.079299in}}%
\pgfusepath{clip}%
\pgfsetbuttcap%
\pgfsetroundjoin%
\pgfsetlinewidth{0.803000pt}%
\definecolor{currentstroke}{rgb}{0.501961,0.501961,0.501961}%
\pgfsetstrokecolor{currentstroke}%
\pgfsetdash{}{0pt}%
\pgfpathmoveto{\pgfqpoint{0.647939in}{1.122299in}}%
\pgfpathlineto{\pgfqpoint{0.647939in}{1.122299in}}%
\pgfpathlineto{\pgfqpoint{0.715324in}{1.134146in}}%
\pgfpathlineto{\pgfqpoint{0.782256in}{1.148319in}}%
\pgfpathlineto{\pgfqpoint{0.848602in}{1.165006in}}%
\pgfpathlineto{\pgfqpoint{0.914219in}{1.184356in}}%
\pgfpathlineto{\pgfqpoint{0.978955in}{1.206467in}}%
\pgfpathlineto{\pgfqpoint{1.042666in}{1.231377in}}%
\pgfpathlineto{\pgfqpoint{1.105220in}{1.259063in}}%
\pgfpathlineto{\pgfqpoint{1.166514in}{1.289437in}}%
\pgfpathlineto{\pgfqpoint{1.226482in}{1.322353in}}%
\pgfpathlineto{\pgfqpoint{1.285100in}{1.357617in}}%
\pgfpathlineto{\pgfqpoint{1.342393in}{1.395000in}}%
\pgfpathlineto{\pgfqpoint{1.398434in}{1.434242in}}%
\pgfpathlineto{\pgfqpoint{1.453343in}{1.475055in}}%
\pgfpathlineto{\pgfqpoint{1.507299in}{1.517122in}}%
\pgfpathlineto{\pgfqpoint{1.560512in}{1.560122in}}%
\pgfpathlineto{\pgfqpoint{1.613227in}{1.603732in}}%
\pgfpathlineto{\pgfqpoint{1.665740in}{1.647586in}}%
\pgfpathlineto{\pgfqpoint{1.718368in}{1.691284in}}%
\pgfpathlineto{\pgfqpoint{1.771457in}{1.734421in}}%
\pgfpathlineto{\pgfqpoint{1.825358in}{1.776522in}}%
\pgfusepath{stroke}%
\end{pgfscope}%
\begin{pgfscope}%
\pgfpathrectangle{\pgfqpoint{0.647939in}{0.492442in}}{\pgfqpoint{3.079299in}{3.079299in}}%
\pgfusepath{clip}%
\pgfsetbuttcap%
\pgfsetroundjoin%
\pgfsetlinewidth{0.803000pt}%
\definecolor{currentstroke}{rgb}{0.501961,0.501961,0.501961}%
\pgfsetstrokecolor{currentstroke}%
\pgfsetdash{}{0pt}%
\pgfpathmoveto{\pgfqpoint{0.647939in}{1.052315in}}%
\pgfpathlineto{\pgfqpoint{0.647939in}{1.052315in}}%
\pgfpathlineto{\pgfqpoint{0.715285in}{1.064380in}}%
\pgfpathlineto{\pgfqpoint{0.782154in}{1.078842in}}%
\pgfpathlineto{\pgfqpoint{0.848403in}{1.095905in}}%
\pgfpathlineto{\pgfqpoint{0.913875in}{1.115733in}}%
\pgfpathlineto{\pgfqpoint{0.978402in}{1.138443in}}%
\pgfpathlineto{\pgfqpoint{1.041818in}{1.164086in}}%
\pgfpathlineto{\pgfqpoint{1.103972in}{1.192651in}}%
\pgfpathlineto{\pgfqpoint{1.164738in}{1.224059in}}%
\pgfpathlineto{\pgfqpoint{1.224028in}{1.258172in}}%
\pgfpathlineto{\pgfqpoint{1.281800in}{1.294798in}}%
\pgfpathlineto{\pgfqpoint{1.338063in}{1.333709in}}%
\pgfusepath{stroke}%
\end{pgfscope}%
\begin{pgfscope}%
\pgfpathrectangle{\pgfqpoint{0.647939in}{0.492442in}}{\pgfqpoint{3.079299in}{3.079299in}}%
\pgfusepath{clip}%
\pgfsetbuttcap%
\pgfsetroundjoin%
\pgfsetlinewidth{0.803000pt}%
\definecolor{currentstroke}{rgb}{0.501961,0.501961,0.501961}%
\pgfsetstrokecolor{currentstroke}%
\pgfsetdash{}{0pt}%
\pgfpathmoveto{\pgfqpoint{0.647939in}{0.982331in}}%
\pgfpathlineto{\pgfqpoint{0.647939in}{0.982331in}}%
\pgfpathlineto{\pgfqpoint{0.715244in}{0.994622in}}%
\pgfpathlineto{\pgfqpoint{0.782045in}{1.009384in}}%
\pgfpathlineto{\pgfqpoint{0.848191in}{1.026839in}}%
\pgfpathlineto{\pgfqpoint{0.913507in}{1.047169in}}%
\pgfpathlineto{\pgfqpoint{0.977807in}{1.070505in}}%
\pgfpathlineto{\pgfqpoint{1.040903in}{1.096918in}}%
\pgfpathlineto{\pgfqpoint{1.102621in}{1.126406in}}%
\pgfusepath{stroke}%
\end{pgfscope}%
\begin{pgfscope}%
\pgfpathrectangle{\pgfqpoint{0.647939in}{0.492442in}}{\pgfqpoint{3.079299in}{3.079299in}}%
\pgfusepath{clip}%
\pgfsetbuttcap%
\pgfsetroundjoin%
\pgfsetlinewidth{0.803000pt}%
\definecolor{currentstroke}{rgb}{0.501961,0.501961,0.501961}%
\pgfsetstrokecolor{currentstroke}%
\pgfsetdash{}{0pt}%
\pgfpathmoveto{\pgfqpoint{0.647939in}{0.912347in}}%
\pgfpathlineto{\pgfqpoint{0.647939in}{0.912347in}}%
\pgfpathlineto{\pgfqpoint{0.715200in}{0.924872in}}%
\pgfpathlineto{\pgfqpoint{0.781931in}{0.939947in}}%
\pgfpathlineto{\pgfqpoint{0.847965in}{0.957811in}}%
\pgfpathlineto{\pgfqpoint{0.913113in}{0.978665in}}%
\pgfpathlineto{\pgfqpoint{0.977168in}{1.002660in}}%
\pgfpathlineto{\pgfqpoint{1.039916in}{1.029879in}}%
\pgfpathlineto{\pgfqpoint{1.101157in}{1.060336in}}%
\pgfpathlineto{\pgfqpoint{1.160713in}{1.093964in}}%
\pgfpathlineto{\pgfqpoint{1.218447in}{1.130629in}}%
\pgfpathlineto{\pgfqpoint{1.274279in}{1.170137in}}%
\pgfpathlineto{\pgfqpoint{1.328190in}{1.212237in}}%
\pgfpathlineto{\pgfqpoint{1.380237in}{1.256615in}}%
\pgfpathlineto{\pgfqpoint{1.430540in}{1.302963in}}%
\pgfpathlineto{\pgfqpoint{1.479281in}{1.350953in}}%
\pgfpathlineto{\pgfqpoint{1.526711in}{1.400234in}}%
\pgfpathlineto{\pgfqpoint{1.573125in}{1.450473in}}%
\pgfusepath{stroke}%
\end{pgfscope}%
\begin{pgfscope}%
\pgfpathrectangle{\pgfqpoint{0.647939in}{0.492442in}}{\pgfqpoint{3.079299in}{3.079299in}}%
\pgfusepath{clip}%
\pgfsetbuttcap%
\pgfsetroundjoin%
\pgfsetlinewidth{0.803000pt}%
\definecolor{currentstroke}{rgb}{0.501961,0.501961,0.501961}%
\pgfsetstrokecolor{currentstroke}%
\pgfsetdash{}{0pt}%
\pgfpathmoveto{\pgfqpoint{0.647939in}{0.842362in}}%
\pgfpathlineto{\pgfqpoint{0.647939in}{0.842362in}}%
\pgfpathlineto{\pgfqpoint{0.715153in}{0.855131in}}%
\pgfpathlineto{\pgfqpoint{0.781808in}{0.870532in}}%
\pgfpathlineto{\pgfqpoint{0.847724in}{0.888823in}}%
\pgfpathlineto{\pgfqpoint{0.912691in}{0.910227in}}%
\pgfpathlineto{\pgfqpoint{0.976480in}{0.934912in}}%
\pgfpathlineto{\pgfqpoint{1.038850in}{0.962979in}}%
\pgfpathlineto{\pgfqpoint{1.099570in}{0.994450in}}%
\pgfpathlineto{\pgfqpoint{1.158433in}{1.029265in}}%
\pgfpathlineto{\pgfqpoint{1.215280in}{1.067285in}}%
\pgfpathlineto{\pgfqpoint{1.270008in}{1.108305in}}%
\pgfpathlineto{\pgfqpoint{1.322598in}{1.152037in}}%
\pgfusepath{stroke}%
\end{pgfscope}%
\begin{pgfscope}%
\pgfpathrectangle{\pgfqpoint{0.647939in}{0.492442in}}{\pgfqpoint{3.079299in}{3.079299in}}%
\pgfusepath{clip}%
\pgfsetbuttcap%
\pgfsetroundjoin%
\pgfsetlinewidth{0.803000pt}%
\definecolor{currentstroke}{rgb}{0.501961,0.501961,0.501961}%
\pgfsetstrokecolor{currentstroke}%
\pgfsetdash{}{0pt}%
\pgfpathmoveto{\pgfqpoint{0.647939in}{0.772378in}}%
\pgfpathlineto{\pgfqpoint{0.647939in}{0.772378in}}%
\pgfpathlineto{\pgfqpoint{0.715104in}{0.785400in}}%
\pgfpathlineto{\pgfqpoint{0.781678in}{0.801140in}}%
\pgfpathlineto{\pgfqpoint{0.847467in}{0.819878in}}%
\pgfpathlineto{\pgfqpoint{0.912238in}{0.841858in}}%
\pgfpathlineto{\pgfqpoint{0.975738in}{0.867267in}}%
\pgfpathlineto{\pgfqpoint{1.037695in}{0.896224in}}%
\pgfpathlineto{\pgfqpoint{1.097846in}{0.928759in}}%
\pgfpathlineto{\pgfqpoint{1.155954in}{0.964811in}}%
\pgfpathlineto{\pgfqpoint{1.211831in}{1.004234in}}%
\pgfpathlineto{\pgfqpoint{1.265367in}{1.046791in}}%
\pgfusepath{stroke}%
\end{pgfscope}%
\begin{pgfscope}%
\pgfpathrectangle{\pgfqpoint{0.647939in}{0.492442in}}{\pgfqpoint{3.079299in}{3.079299in}}%
\pgfusepath{clip}%
\pgfsetbuttcap%
\pgfsetroundjoin%
\pgfsetlinewidth{0.803000pt}%
\definecolor{currentstroke}{rgb}{0.501961,0.501961,0.501961}%
\pgfsetstrokecolor{currentstroke}%
\pgfsetdash{}{0pt}%
\pgfpathmoveto{\pgfqpoint{0.647939in}{0.702394in}}%
\pgfpathlineto{\pgfqpoint{0.647939in}{0.702394in}}%
\pgfpathlineto{\pgfqpoint{0.715051in}{0.715678in}}%
\pgfpathlineto{\pgfqpoint{0.781540in}{0.731773in}}%
\pgfpathlineto{\pgfqpoint{0.847191in}{0.750978in}}%
\pgfpathlineto{\pgfqpoint{0.911752in}{0.773562in}}%
\pgfpathlineto{\pgfqpoint{0.974937in}{0.799732in}}%
\pgfpathlineto{\pgfqpoint{1.036443in}{0.829622in}}%
\pgfpathlineto{\pgfqpoint{1.095972in}{0.863270in}}%
\pgfpathlineto{\pgfqpoint{1.153254in}{0.900610in}}%
\pgfpathlineto{\pgfqpoint{1.208079in}{0.941481in}}%
\pgfpathlineto{\pgfqpoint{1.260329in}{0.985597in}}%
\pgfusepath{stroke}%
\end{pgfscope}%
\begin{pgfscope}%
\pgfpathrectangle{\pgfqpoint{0.647939in}{0.492442in}}{\pgfqpoint{3.079299in}{3.079299in}}%
\pgfusepath{clip}%
\pgfsetbuttcap%
\pgfsetroundjoin%
\pgfsetlinewidth{0.803000pt}%
\definecolor{currentstroke}{rgb}{0.501961,0.501961,0.501961}%
\pgfsetstrokecolor{currentstroke}%
\pgfsetdash{}{0pt}%
\pgfpathmoveto{\pgfqpoint{0.647939in}{0.632410in}}%
\pgfpathlineto{\pgfqpoint{0.647939in}{0.632410in}}%
\pgfpathlineto{\pgfqpoint{0.714996in}{0.645967in}}%
\pgfpathlineto{\pgfqpoint{0.781392in}{0.662431in}}%
\pgfpathlineto{\pgfqpoint{0.846896in}{0.682127in}}%
\pgfpathlineto{\pgfqpoint{0.911227in}{0.705344in}}%
\pgfpathlineto{\pgfqpoint{0.974071in}{0.732313in}}%
\pgfpathlineto{\pgfqpoint{1.035085in}{0.763183in}}%
\pgfpathlineto{\pgfqpoint{1.093933in}{0.797993in}}%
\pgfpathlineto{\pgfqpoint{1.150315in}{0.836670in}}%
\pgfusepath{stroke}%
\end{pgfscope}%
\begin{pgfscope}%
\pgfpathrectangle{\pgfqpoint{0.647939in}{0.492442in}}{\pgfqpoint{3.079299in}{3.079299in}}%
\pgfusepath{clip}%
\pgfsetbuttcap%
\pgfsetroundjoin%
\pgfsetlinewidth{0.803000pt}%
\definecolor{currentstroke}{rgb}{0.501961,0.501961,0.501961}%
\pgfsetstrokecolor{currentstroke}%
\pgfsetdash{}{0pt}%
\pgfpathmoveto{\pgfqpoint{2.381290in}{0.599680in}}%
\pgfpathlineto{\pgfqpoint{2.312880in}{0.601157in}}%
\pgfpathlineto{\pgfqpoint{2.244455in}{0.601714in}}%
\pgfpathlineto{\pgfqpoint{2.176026in}{0.601824in}}%
\pgfpathlineto{\pgfqpoint{2.107598in}{0.602044in}}%
\pgfpathlineto{\pgfqpoint{2.039179in}{0.603027in}}%
\pgfpathlineto{\pgfqpoint{1.970804in}{0.605532in}}%
\pgfpathlineto{\pgfqpoint{1.902572in}{0.610463in}}%
\pgfpathlineto{\pgfqpoint{1.834707in}{0.618941in}}%
\pgfpathlineto{\pgfqpoint{1.767684in}{0.632410in}}%
\pgfusepath{stroke}%
\end{pgfscope}%
\begin{pgfscope}%
\pgfpathrectangle{\pgfqpoint{0.647939in}{0.492442in}}{\pgfqpoint{3.079299in}{3.079299in}}%
\pgfusepath{clip}%
\pgfsetbuttcap%
\pgfsetroundjoin%
\pgfsetlinewidth{0.803000pt}%
\definecolor{currentstroke}{rgb}{0.501961,0.501961,0.501961}%
\pgfsetstrokecolor{currentstroke}%
\pgfsetdash{}{0pt}%
\pgfpathmoveto{\pgfqpoint{3.587270in}{1.472219in}}%
\pgfpathlineto{\pgfqpoint{3.529178in}{1.508364in}}%
\pgfpathlineto{\pgfqpoint{3.472209in}{1.546257in}}%
\pgfpathlineto{\pgfqpoint{3.416304in}{1.585706in}}%
\pgfpathlineto{\pgfqpoint{3.361394in}{1.626530in}}%
\pgfpathlineto{\pgfqpoint{3.307408in}{1.668569in}}%
\pgfusepath{stroke}%
\end{pgfscope}%
\begin{pgfscope}%
\pgfpathrectangle{\pgfqpoint{0.647939in}{0.492442in}}{\pgfqpoint{3.079299in}{3.079299in}}%
\pgfusepath{clip}%
\pgfsetbuttcap%
\pgfsetroundjoin%
\pgfsetlinewidth{0.803000pt}%
\definecolor{currentstroke}{rgb}{0.501961,0.501961,0.501961}%
\pgfsetstrokecolor{currentstroke}%
\pgfsetdash{}{0pt}%
\pgfpathmoveto{\pgfqpoint{3.380499in}{2.519292in}}%
\pgfpathlineto{\pgfqpoint{3.373073in}{2.587239in}}%
\pgfpathlineto{\pgfqpoint{3.371743in}{2.655574in}}%
\pgfpathlineto{\pgfqpoint{3.376437in}{2.723766in}}%
\pgfpathlineto{\pgfqpoint{3.386902in}{2.791325in}}%
\pgfpathlineto{\pgfqpoint{3.402756in}{2.857837in}}%
\pgfpathlineto{\pgfqpoint{3.423549in}{2.922981in}}%
\pgfpathlineto{\pgfqpoint{3.448818in}{2.986527in}}%
\pgfpathlineto{\pgfqpoint{3.478145in}{3.048311in}}%
\pgfpathlineto{\pgfqpoint{3.511178in}{3.108201in}}%
\pgfpathlineto{\pgfqpoint{3.547627in}{3.166080in}}%
\pgfpathlineto{\pgfqpoint{3.587270in}{3.221821in}}%
\pgfusepath{stroke}%
\end{pgfscope}%
\begin{pgfscope}%
\pgfpathrectangle{\pgfqpoint{0.647939in}{0.492442in}}{\pgfqpoint{3.079299in}{3.079299in}}%
\pgfusepath{clip}%
\pgfsetbuttcap%
\pgfsetroundjoin%
\pgfsetlinewidth{0.803000pt}%
\definecolor{currentstroke}{rgb}{0.501961,0.501961,0.501961}%
\pgfsetstrokecolor{currentstroke}%
\pgfsetdash{}{0pt}%
\pgfpathmoveto{\pgfqpoint{3.517286in}{1.752155in}}%
\pgfpathlineto{\pgfqpoint{3.465283in}{1.796610in}}%
\pgfpathlineto{\pgfqpoint{3.415002in}{1.843005in}}%
\pgfpathlineto{\pgfqpoint{3.366488in}{1.891244in}}%
\pgfpathlineto{\pgfqpoint{3.319820in}{1.941270in}}%
\pgfpathlineto{\pgfqpoint{3.275133in}{1.993071in}}%
\pgfpathlineto{\pgfqpoint{3.232628in}{2.046674in}}%
\pgfpathlineto{\pgfqpoint{3.192598in}{2.102146in}}%
\pgfpathlineto{\pgfqpoint{3.155456in}{2.159580in}}%
\pgfpathlineto{\pgfqpoint{3.121764in}{2.219091in}}%
\pgfpathlineto{\pgfqpoint{3.092260in}{2.280768in}}%
\pgfpathlineto{\pgfqpoint{3.067851in}{2.344611in}}%
\pgfpathlineto{\pgfqpoint{3.049534in}{2.410437in}}%
\pgfpathlineto{\pgfqpoint{3.038221in}{2.477804in}}%
\pgfpathlineto{\pgfqpoint{3.034479in}{2.546011in}}%
\pgfpathlineto{\pgfqpoint{3.038323in}{2.614214in}}%
\pgfpathlineto{\pgfqpoint{3.049205in}{2.681664in}}%
\pgfpathlineto{\pgfqpoint{3.066237in}{2.747847in}}%
\pgfpathlineto{\pgfqpoint{3.088425in}{2.812506in}}%
\pgfpathlineto{\pgfqpoint{3.114858in}{2.875568in}}%
\pgfpathlineto{\pgfqpoint{3.144784in}{2.937062in}}%
\pgfpathlineto{\pgfqpoint{3.177623in}{2.997062in}}%
\pgfpathlineto{\pgfqpoint{3.212940in}{3.055643in}}%
\pgfpathlineto{\pgfqpoint{3.250423in}{3.112869in}}%
\pgfusepath{stroke}%
\end{pgfscope}%
\begin{pgfscope}%
\pgfpathrectangle{\pgfqpoint{0.647939in}{0.492442in}}{\pgfqpoint{3.079299in}{3.079299in}}%
\pgfusepath{clip}%
\pgfsetbuttcap%
\pgfsetroundjoin%
\pgfsetlinewidth{0.803000pt}%
\definecolor{currentstroke}{rgb}{0.501961,0.501961,0.501961}%
\pgfsetstrokecolor{currentstroke}%
\pgfsetdash{}{0pt}%
\pgfpathmoveto{\pgfqpoint{3.447302in}{1.402235in}}%
\pgfpathlineto{\pgfqpoint{3.389321in}{1.438569in}}%
\pgfpathlineto{\pgfqpoint{3.331995in}{1.475929in}}%
\pgfpathlineto{\pgfqpoint{3.275215in}{1.514115in}}%
\pgfpathlineto{\pgfqpoint{3.218867in}{1.552939in}}%
\pgfpathlineto{\pgfqpoint{3.162834in}{1.592214in}}%
\pgfpathlineto{\pgfqpoint{3.106990in}{1.631758in}}%
\pgfpathlineto{\pgfqpoint{3.051203in}{1.671381in}}%
\pgfpathlineto{\pgfqpoint{2.995338in}{1.710893in}}%
\pgfpathlineto{\pgfqpoint{2.939259in}{1.750097in}}%
\pgfpathlineto{\pgfqpoint{2.882823in}{1.788784in}}%
\pgfpathlineto{\pgfqpoint{2.825880in}{1.826721in}}%
\pgfpathlineto{\pgfqpoint{2.768279in}{1.863649in}}%
\pgfpathlineto{\pgfqpoint{2.709873in}{1.899279in}}%
\pgfpathlineto{\pgfqpoint{2.650515in}{1.933282in}}%
\pgfpathlineto{\pgfqpoint{2.590066in}{1.965291in}}%
\pgfpathlineto{\pgfqpoint{2.528413in}{1.994896in}}%
\pgfpathlineto{\pgfqpoint{2.465485in}{2.021665in}}%
\pgfpathlineto{\pgfqpoint{2.401282in}{2.045216in}}%
\pgfpathlineto{\pgfqpoint{2.335980in}{2.065461in}}%
\pgfusepath{stroke}%
\end{pgfscope}%
\begin{pgfscope}%
\pgfpathrectangle{\pgfqpoint{0.647939in}{0.492442in}}{\pgfqpoint{3.079299in}{3.079299in}}%
\pgfusepath{clip}%
\pgfsetbuttcap%
\pgfsetroundjoin%
\pgfsetlinewidth{0.803000pt}%
\definecolor{currentstroke}{rgb}{0.501961,0.501961,0.501961}%
\pgfsetstrokecolor{currentstroke}%
\pgfsetdash{}{0pt}%
\pgfpathmoveto{\pgfqpoint{1.711288in}{3.265820in}}%
\pgfpathlineto{\pgfqpoint{1.779193in}{3.274222in}}%
\pgfpathlineto{\pgfqpoint{1.847292in}{3.280855in}}%
\pgfpathlineto{\pgfqpoint{1.915541in}{3.285720in}}%
\pgfpathlineto{\pgfqpoint{1.983890in}{3.288925in}}%
\pgfpathlineto{\pgfqpoint{2.052292in}{3.290711in}}%
\pgfpathlineto{\pgfqpoint{2.120716in}{3.291443in}}%
\pgfpathlineto{\pgfqpoint{2.189144in}{3.291608in}}%
\pgfpathlineto{\pgfqpoint{2.257573in}{3.291805in}}%
\pgfusepath{stroke}%
\end{pgfscope}%
\begin{pgfscope}%
\pgfpathrectangle{\pgfqpoint{0.647939in}{0.492442in}}{\pgfqpoint{3.079299in}{3.079299in}}%
\pgfusepath{clip}%
\pgfsetbuttcap%
\pgfsetroundjoin%
\pgfsetlinewidth{0.803000pt}%
\definecolor{currentstroke}{rgb}{0.501961,0.501961,0.501961}%
\pgfsetstrokecolor{currentstroke}%
\pgfsetdash{}{0pt}%
\pgfpathmoveto{\pgfqpoint{2.317777in}{0.900415in}}%
\pgfpathlineto{\pgfqpoint{2.249354in}{0.901170in}}%
\pgfpathlineto{\pgfqpoint{2.180925in}{0.901331in}}%
\pgfpathlineto{\pgfqpoint{2.112497in}{0.901581in}}%
\pgfpathlineto{\pgfqpoint{2.044082in}{0.902757in}}%
\pgfpathlineto{\pgfqpoint{1.975740in}{0.905900in}}%
\pgfpathlineto{\pgfqpoint{1.907652in}{0.912347in}}%
\pgfusepath{stroke}%
\end{pgfscope}%
\begin{pgfscope}%
\pgfpathrectangle{\pgfqpoint{0.647939in}{0.492442in}}{\pgfqpoint{3.079299in}{3.079299in}}%
\pgfusepath{clip}%
\pgfsetbuttcap%
\pgfsetroundjoin%
\pgfsetlinewidth{0.803000pt}%
\definecolor{currentstroke}{rgb}{0.501961,0.501961,0.501961}%
\pgfsetstrokecolor{currentstroke}%
\pgfsetdash{}{0pt}%
\pgfpathmoveto{\pgfqpoint{3.343925in}{2.114272in}}%
\pgfpathlineto{\pgfqpoint{3.307334in}{2.172060in}}%
\pgfpathlineto{\pgfqpoint{3.274324in}{2.231960in}}%
\pgfpathlineto{\pgfqpoint{3.245431in}{2.293939in}}%
\pgfpathlineto{\pgfqpoint{3.221273in}{2.357903in}}%
\pgfpathlineto{\pgfqpoint{3.202512in}{2.423641in}}%
\pgfpathlineto{\pgfqpoint{3.189756in}{2.490786in}}%
\pgfpathlineto{\pgfqpoint{3.183435in}{2.558827in}}%
\pgfusepath{stroke}%
\end{pgfscope}%
\begin{pgfscope}%
\pgfpathrectangle{\pgfqpoint{0.647939in}{0.492442in}}{\pgfqpoint{3.079299in}{3.079299in}}%
\pgfusepath{clip}%
\pgfsetbuttcap%
\pgfsetroundjoin%
\pgfsetlinewidth{0.803000pt}%
\definecolor{currentstroke}{rgb}{0.501961,0.501961,0.501961}%
\pgfsetstrokecolor{currentstroke}%
\pgfsetdash{}{0pt}%
\pgfpathmoveto{\pgfqpoint{3.237350in}{1.752155in}}%
\pgfpathlineto{\pgfqpoint{3.186002in}{1.797380in}}%
\pgfpathlineto{\pgfqpoint{3.135523in}{1.843571in}}%
\pgfpathlineto{\pgfqpoint{3.085926in}{1.890708in}}%
\pgfpathlineto{\pgfqpoint{3.037268in}{1.938811in}}%
\pgfpathlineto{\pgfqpoint{2.989672in}{1.987963in}}%
\pgfpathlineto{\pgfqpoint{2.943359in}{2.038321in}}%
\pgfpathlineto{\pgfqpoint{2.898710in}{2.090147in}}%
\pgfpathlineto{\pgfqpoint{2.856354in}{2.143855in}}%
\pgfpathlineto{\pgfqpoint{2.817372in}{2.200033in}}%
\pgfpathlineto{\pgfqpoint{2.783574in}{2.259408in}}%
\pgfpathlineto{\pgfqpoint{2.757793in}{2.322555in}}%
\pgfpathlineto{\pgfqpoint{2.743543in}{2.389113in}}%
\pgfpathlineto{\pgfqpoint{2.742983in}{2.456670in}}%
\pgfpathlineto{\pgfqpoint{2.754211in}{2.520674in}}%
\pgfpathlineto{\pgfqpoint{2.775303in}{2.585539in}}%
\pgfpathlineto{\pgfqpoint{2.802842in}{2.648039in}}%
\pgfusepath{stroke}%
\end{pgfscope}%
\begin{pgfscope}%
\pgfpathrectangle{\pgfqpoint{0.647939in}{0.492442in}}{\pgfqpoint{3.079299in}{3.079299in}}%
\pgfusepath{clip}%
\pgfsetbuttcap%
\pgfsetroundjoin%
\pgfsetlinewidth{0.803000pt}%
\definecolor{currentstroke}{rgb}{0.501961,0.501961,0.501961}%
\pgfsetstrokecolor{currentstroke}%
\pgfsetdash{}{0pt}%
\pgfpathmoveto{\pgfqpoint{3.228637in}{2.664129in}}%
\pgfpathlineto{\pgfqpoint{3.237350in}{2.731932in}}%
\pgfpathlineto{\pgfqpoint{3.251706in}{2.798774in}}%
\pgfpathlineto{\pgfqpoint{3.271115in}{2.864334in}}%
\pgfpathlineto{\pgfqpoint{3.294975in}{2.928417in}}%
\pgfpathlineto{\pgfqpoint{3.322743in}{2.990915in}}%
\pgfusepath{stroke}%
\end{pgfscope}%
\begin{pgfscope}%
\pgfpathrectangle{\pgfqpoint{0.647939in}{0.492442in}}{\pgfqpoint{3.079299in}{3.079299in}}%
\pgfusepath{clip}%
\pgfsetbuttcap%
\pgfsetroundjoin%
\pgfsetlinewidth{0.803000pt}%
\definecolor{currentstroke}{rgb}{0.501961,0.501961,0.501961}%
\pgfsetstrokecolor{currentstroke}%
\pgfsetdash{}{0pt}%
\pgfpathmoveto{\pgfqpoint{1.137828in}{1.962108in}}%
\pgfpathlineto{\pgfqpoint{1.202395in}{1.984757in}}%
\pgfpathlineto{\pgfqpoint{1.266545in}{2.008567in}}%
\pgfpathlineto{\pgfqpoint{1.330364in}{2.033254in}}%
\pgfpathlineto{\pgfqpoint{1.393966in}{2.058494in}}%
\pgfpathlineto{\pgfqpoint{1.457494in}{2.083925in}}%
\pgfpathlineto{\pgfqpoint{1.521105in}{2.109145in}}%
\pgfpathlineto{\pgfqpoint{1.584969in}{2.133713in}}%
\pgfpathlineto{\pgfqpoint{1.649253in}{2.157152in}}%
\pgfpathlineto{\pgfqpoint{1.714108in}{2.178950in}}%
\pgfpathlineto{\pgfqpoint{1.779648in}{2.198573in}}%
\pgfpathlineto{\pgfqpoint{1.845932in}{2.215486in}}%
\pgfpathlineto{\pgfqpoint{1.912944in}{2.229206in}}%
\pgfpathlineto{\pgfqpoint{1.980584in}{2.239365in}}%
\pgfpathlineto{\pgfqpoint{2.048680in}{2.245832in}}%
\pgfpathlineto{\pgfqpoint{2.117020in}{2.248929in}}%
\pgfpathlineto{\pgfqpoint{2.185439in}{2.249762in}}%
\pgfpathlineto{\pgfqpoint{2.253836in}{2.251135in}}%
\pgfusepath{stroke}%
\end{pgfscope}%
\begin{pgfscope}%
\pgfpathrectangle{\pgfqpoint{0.647939in}{0.492442in}}{\pgfqpoint{3.079299in}{3.079299in}}%
\pgfusepath{clip}%
\pgfsetbuttcap%
\pgfsetroundjoin%
\pgfsetlinewidth{0.803000pt}%
\definecolor{currentstroke}{rgb}{0.501961,0.501961,0.501961}%
\pgfsetstrokecolor{currentstroke}%
\pgfsetdash{}{0pt}%
\pgfpathmoveto{\pgfqpoint{1.810494in}{2.878791in}}%
\pgfpathlineto{\pgfqpoint{1.878490in}{2.886379in}}%
\pgfpathlineto{\pgfqpoint{1.946694in}{2.891812in}}%
\pgfpathlineto{\pgfqpoint{2.015030in}{2.895229in}}%
\pgfpathlineto{\pgfqpoint{2.083433in}{2.896957in}}%
\pgfpathlineto{\pgfqpoint{2.151858in}{2.897526in}}%
\pgfpathlineto{\pgfqpoint{2.220286in}{2.897670in}}%
\pgfpathlineto{\pgfqpoint{2.288709in}{2.898335in}}%
\pgfpathlineto{\pgfqpoint{2.357088in}{2.900686in}}%
\pgfpathlineto{\pgfqpoint{2.425277in}{2.906066in}}%
\pgfpathlineto{\pgfqpoint{2.492949in}{2.915841in}}%
\pgfpathlineto{\pgfqpoint{2.559569in}{2.931109in}}%
\pgfpathlineto{\pgfqpoint{2.624501in}{2.952407in}}%
\pgfpathlineto{\pgfqpoint{2.687214in}{2.979571in}}%
\pgfpathlineto{\pgfqpoint{2.747461in}{3.011869in}}%
\pgfusepath{stroke}%
\end{pgfscope}%
\begin{pgfscope}%
\pgfpathrectangle{\pgfqpoint{0.647939in}{0.492442in}}{\pgfqpoint{3.079299in}{3.079299in}}%
\pgfusepath{clip}%
\pgfsetbuttcap%
\pgfsetroundjoin%
\pgfsetlinewidth{0.803000pt}%
\definecolor{currentstroke}{rgb}{0.501961,0.501961,0.501961}%
\pgfsetstrokecolor{currentstroke}%
\pgfsetdash{}{0pt}%
\pgfpathmoveto{\pgfqpoint{1.918865in}{2.928883in}}%
\pgfpathlineto{\pgfqpoint{1.987165in}{2.932952in}}%
\pgfpathlineto{\pgfqpoint{2.055551in}{2.935236in}}%
\pgfpathlineto{\pgfqpoint{2.123971in}{2.936171in}}%
\pgfpathlineto{\pgfqpoint{2.192399in}{2.936379in}}%
\pgfpathlineto{\pgfqpoint{2.260827in}{2.936682in}}%
\pgfpathlineto{\pgfqpoint{2.329235in}{2.938110in}}%
\pgfpathlineto{\pgfqpoint{2.397541in}{2.941885in}}%
\pgfusepath{stroke}%
\end{pgfscope}%
\begin{pgfscope}%
\pgfpathrectangle{\pgfqpoint{0.647939in}{0.492442in}}{\pgfqpoint{3.079299in}{3.079299in}}%
\pgfusepath{clip}%
\pgfsetbuttcap%
\pgfsetroundjoin%
\pgfsetlinewidth{0.803000pt}%
\definecolor{currentstroke}{rgb}{0.501961,0.501961,0.501961}%
\pgfsetstrokecolor{currentstroke}%
\pgfsetdash{}{0pt}%
\pgfpathmoveto{\pgfqpoint{1.602927in}{2.543001in}}%
\pgfpathlineto{\pgfqpoint{1.669313in}{2.559572in}}%
\pgfpathlineto{\pgfqpoint{1.736135in}{2.574271in}}%
\pgfpathlineto{\pgfqpoint{1.803404in}{2.586757in}}%
\pgfpathlineto{\pgfqpoint{1.871081in}{2.596777in}}%
\pgfpathlineto{\pgfqpoint{1.939091in}{2.604203in}}%
\pgfpathlineto{\pgfqpoint{2.007332in}{2.609084in}}%
\pgfpathlineto{\pgfqpoint{2.075702in}{2.611701in}}%
\pgfpathlineto{\pgfqpoint{2.144121in}{2.612635in}}%
\pgfpathlineto{\pgfqpoint{2.212550in}{2.612841in}}%
\pgfpathlineto{\pgfqpoint{2.280968in}{2.613709in}}%
\pgfpathlineto{\pgfqpoint{2.349282in}{2.617133in}}%
\pgfpathlineto{\pgfqpoint{2.417124in}{2.625466in}}%
\pgfpathlineto{\pgfqpoint{2.483624in}{2.641039in}}%
\pgfpathlineto{\pgfqpoint{2.547498in}{2.665144in}}%
\pgfpathlineto{\pgfqpoint{2.607767in}{2.697275in}}%
\pgfpathlineto{\pgfqpoint{2.664284in}{2.735688in}}%
\pgfpathlineto{\pgfqpoint{2.717578in}{2.778491in}}%
\pgfpathlineto{\pgfqpoint{2.768403in}{2.824229in}}%
\pgfpathlineto{\pgfqpoint{2.817445in}{2.871901in}}%
\pgfusepath{stroke}%
\end{pgfscope}%
\begin{pgfscope}%
\pgfpathrectangle{\pgfqpoint{0.647939in}{0.492442in}}{\pgfqpoint{3.079299in}{3.079299in}}%
\pgfusepath{clip}%
\pgfsetbuttcap%
\pgfsetroundjoin%
\pgfsetlinewidth{0.803000pt}%
\definecolor{currentstroke}{rgb}{0.501961,0.501961,0.501961}%
\pgfsetstrokecolor{currentstroke}%
\pgfsetdash{}{0pt}%
\pgfpathmoveto{\pgfqpoint{2.525889in}{1.243575in}}%
\pgfpathlineto{\pgfqpoint{2.457895in}{1.251121in}}%
\pgfpathlineto{\pgfqpoint{2.389657in}{1.256065in}}%
\pgfpathlineto{\pgfqpoint{2.321290in}{1.258809in}}%
\pgfpathlineto{\pgfqpoint{2.252874in}{1.259933in}}%
\pgfpathlineto{\pgfqpoint{2.184446in}{1.260186in}}%
\pgfpathlineto{\pgfqpoint{2.116019in}{1.260535in}}%
\pgfpathlineto{\pgfqpoint{2.047620in}{1.262267in}}%
\pgfusepath{stroke}%
\end{pgfscope}%
\begin{pgfscope}%
\pgfpathrectangle{\pgfqpoint{0.647939in}{0.492442in}}{\pgfqpoint{3.079299in}{3.079299in}}%
\pgfusepath{clip}%
\pgfsetbuttcap%
\pgfsetroundjoin%
\pgfsetlinewidth{0.803000pt}%
\definecolor{currentstroke}{rgb}{0.501961,0.501961,0.501961}%
\pgfsetstrokecolor{currentstroke}%
\pgfsetdash{}{0pt}%
\pgfpathmoveto{\pgfqpoint{1.417764in}{2.382012in}}%
\pgfpathlineto{\pgfqpoint{1.482907in}{2.402960in}}%
\pgfpathlineto{\pgfqpoint{1.548268in}{2.423212in}}%
\pgfpathlineto{\pgfqpoint{1.613962in}{2.442349in}}%
\pgfpathlineto{\pgfqpoint{1.680084in}{2.459943in}}%
\pgfpathlineto{\pgfqpoint{1.746692in}{2.475578in}}%
\pgfpathlineto{\pgfqpoint{1.813801in}{2.488878in}}%
\pgfpathlineto{\pgfqpoint{1.881376in}{2.499549in}}%
\pgfpathlineto{\pgfqpoint{1.949332in}{2.507424in}}%
\pgfpathlineto{\pgfqpoint{2.017556in}{2.512535in}}%
\pgfpathlineto{\pgfqpoint{2.085923in}{2.515202in}}%
\pgfpathlineto{\pgfqpoint{2.154342in}{2.516100in}}%
\pgfpathlineto{\pgfqpoint{2.222771in}{2.516360in}}%
\pgfpathlineto{\pgfqpoint{2.291175in}{2.517742in}}%
\pgfpathlineto{\pgfqpoint{2.359362in}{2.522823in}}%
\pgfpathlineto{\pgfqpoint{2.426552in}{2.534857in}}%
\pgfpathlineto{\pgfqpoint{2.491165in}{2.556441in}}%
\pgfpathlineto{\pgfqpoint{2.551688in}{2.587689in}}%
\pgfusepath{stroke}%
\end{pgfscope}%
\begin{pgfscope}%
\pgfpathrectangle{\pgfqpoint{0.647939in}{0.492442in}}{\pgfqpoint{3.079299in}{3.079299in}}%
\pgfusepath{clip}%
\pgfsetbuttcap%
\pgfsetroundjoin%
\pgfsetlinewidth{0.803000pt}%
\definecolor{currentstroke}{rgb}{0.501961,0.501961,0.501961}%
\pgfsetstrokecolor{currentstroke}%
\pgfsetdash{}{0pt}%
\pgfpathmoveto{\pgfqpoint{2.887429in}{2.032092in}}%
\pgfpathlineto{\pgfqpoint{2.840822in}{2.082172in}}%
\pgfpathlineto{\pgfqpoint{2.796192in}{2.134006in}}%
\pgfpathlineto{\pgfqpoint{2.754705in}{2.188344in}}%
\pgfpathlineto{\pgfqpoint{2.718521in}{2.246253in}}%
\pgfpathlineto{\pgfqpoint{2.691480in}{2.308747in}}%
\pgfpathlineto{\pgfqpoint{2.678706in}{2.375322in}}%
\pgfpathlineto{\pgfqpoint{2.681252in}{2.435180in}}%
\pgfpathlineto{\pgfqpoint{2.694627in}{2.492684in}}%
\pgfusepath{stroke}%
\end{pgfscope}%
\begin{pgfscope}%
\pgfpathrectangle{\pgfqpoint{0.647939in}{0.492442in}}{\pgfqpoint{3.079299in}{3.079299in}}%
\pgfusepath{clip}%
\pgfsetbuttcap%
\pgfsetroundjoin%
\pgfsetlinewidth{0.803000pt}%
\definecolor{currentstroke}{rgb}{0.501961,0.501961,0.501961}%
\pgfsetstrokecolor{currentstroke}%
\pgfsetdash{}{0pt}%
\pgfpathmoveto{\pgfqpoint{2.455547in}{1.449737in}}%
\pgfpathlineto{\pgfqpoint{2.387406in}{1.455822in}}%
\pgfpathlineto{\pgfqpoint{2.319074in}{1.459240in}}%
\pgfpathlineto{\pgfqpoint{2.250666in}{1.460646in}}%
\pgfpathlineto{\pgfqpoint{2.182238in}{1.460961in}}%
\pgfpathlineto{\pgfqpoint{2.113813in}{1.461478in}}%
\pgfpathlineto{\pgfqpoint{2.045458in}{1.464133in}}%
\pgfpathlineto{\pgfqpoint{1.977636in}{1.472219in}}%
\pgfusepath{stroke}%
\end{pgfscope}%
\begin{pgfscope}%
\pgfpathrectangle{\pgfqpoint{0.647939in}{0.492442in}}{\pgfqpoint{3.079299in}{3.079299in}}%
\pgfusepath{clip}%
\pgfsetbuttcap%
\pgfsetroundjoin%
\pgfsetlinewidth{0.803000pt}%
\definecolor{currentstroke}{rgb}{0.501961,0.501961,0.501961}%
\pgfsetstrokecolor{currentstroke}%
\pgfsetdash{}{0pt}%
\pgfpathmoveto{\pgfqpoint{2.802612in}{1.921616in}}%
\pgfpathlineto{\pgfqpoint{2.747461in}{1.962108in}}%
\pgfpathlineto{\pgfqpoint{2.691981in}{2.002136in}}%
\pgfpathlineto{\pgfqpoint{2.636216in}{2.041768in}}%
\pgfpathlineto{\pgfqpoint{2.580386in}{2.081312in}}%
\pgfpathlineto{\pgfqpoint{2.525197in}{2.121694in}}%
\pgfpathlineto{\pgfqpoint{2.473085in}{2.165731in}}%
\pgfpathlineto{\pgfqpoint{2.473085in}{2.165731in}}%
\pgfpathlineto{\pgfqpoint{2.445773in}{2.198866in}}%
\pgfpathlineto{\pgfqpoint{2.445773in}{2.198866in}}%
\pgfpathlineto{\pgfqpoint{2.434568in}{2.227312in}}%
\pgfpathlineto{\pgfqpoint{2.435870in}{2.259425in}}%
\pgfusepath{stroke}%
\end{pgfscope}%
\begin{pgfscope}%
\pgfpathrectangle{\pgfqpoint{0.647939in}{0.492442in}}{\pgfqpoint{3.079299in}{3.079299in}}%
\pgfusepath{clip}%
\pgfsetbuttcap%
\pgfsetroundjoin%
\pgfsetlinewidth{0.803000pt}%
\definecolor{currentstroke}{rgb}{0.501961,0.501961,0.501961}%
\pgfsetstrokecolor{currentstroke}%
\pgfsetdash{}{0pt}%
\pgfpathmoveto{\pgfqpoint{2.747461in}{2.102076in}}%
\pgfpathlineto{\pgfqpoint{2.701311in}{2.152537in}}%
\pgfpathlineto{\pgfqpoint{2.659177in}{2.206333in}}%
\pgfpathlineto{\pgfqpoint{2.625274in}{2.265417in}}%
\pgfpathlineto{\pgfqpoint{2.625274in}{2.265417in}}%
\pgfpathlineto{\pgfqpoint{2.609170in}{2.316635in}}%
\pgfpathlineto{\pgfqpoint{2.606758in}{2.371223in}}%
\pgfusepath{stroke}%
\end{pgfscope}%
\begin{pgfscope}%
\pgfpathrectangle{\pgfqpoint{0.647939in}{0.492442in}}{\pgfqpoint{3.079299in}{3.079299in}}%
\pgfusepath{clip}%
\pgfsetbuttcap%
\pgfsetroundjoin%
\pgfsetlinewidth{0.803000pt}%
\definecolor{currentstroke}{rgb}{0.501961,0.501961,0.501961}%
\pgfsetstrokecolor{currentstroke}%
\pgfsetdash{}{0pt}%
\pgfpathmoveto{\pgfqpoint{2.607493in}{1.892124in}}%
\pgfpathlineto{\pgfqpoint{2.544359in}{1.918457in}}%
\pgfpathlineto{\pgfqpoint{2.479849in}{1.941172in}}%
\pgfpathlineto{\pgfqpoint{2.414007in}{1.959624in}}%
\pgfpathlineto{\pgfqpoint{2.347003in}{1.973208in}}%
\pgfpathlineto{\pgfqpoint{2.279147in}{1.981518in}}%
\pgfpathlineto{\pgfqpoint{2.210844in}{1.984771in}}%
\pgfpathlineto{\pgfqpoint{2.210844in}{1.984771in}}%
\pgfpathlineto{\pgfqpoint{2.182919in}{1.985061in}}%
\pgfpathlineto{\pgfqpoint{2.182919in}{1.985061in}}%
\pgfpathlineto{\pgfqpoint{2.162216in}{1.985720in}}%
\pgfpathlineto{\pgfqpoint{2.162216in}{1.985720in}}%
\pgfpathlineto{\pgfqpoint{2.162216in}{1.985720in}}%
\pgfusepath{stroke}%
\end{pgfscope}%
\begin{pgfscope}%
\pgfpathrectangle{\pgfqpoint{0.647939in}{0.492442in}}{\pgfqpoint{3.079299in}{3.079299in}}%
\pgfusepath{clip}%
\pgfsetbuttcap%
\pgfsetroundjoin%
\pgfsetlinewidth{0.803000pt}%
\definecolor{currentstroke}{rgb}{0.501961,0.501961,0.501961}%
\pgfsetstrokecolor{currentstroke}%
\pgfsetdash{}{0pt}%
\pgfpathmoveto{\pgfqpoint{2.524398in}{1.647994in}}%
\pgfpathlineto{\pgfqpoint{2.457102in}{1.660198in}}%
\pgfpathlineto{\pgfqpoint{2.389219in}{1.668592in}}%
\pgfpathlineto{\pgfqpoint{2.320986in}{1.673496in}}%
\pgfpathlineto{\pgfqpoint{2.252602in}{1.675613in}}%
\pgfpathlineto{\pgfqpoint{2.184177in}{1.676116in}}%
\pgfpathlineto{\pgfqpoint{2.115759in}{1.676987in}}%
\pgfpathlineto{\pgfqpoint{2.047620in}{1.682171in}}%
\pgfpathlineto{\pgfqpoint{2.047620in}{1.682171in}}%
\pgfusepath{stroke}%
\end{pgfscope}%
\begin{pgfscope}%
\pgfpathrectangle{\pgfqpoint{0.647939in}{0.492442in}}{\pgfqpoint{3.079299in}{3.079299in}}%
\pgfusepath{clip}%
\pgfsetbuttcap%
\pgfsetroundjoin%
\pgfsetlinewidth{0.803000pt}%
\definecolor{currentstroke}{rgb}{0.501961,0.501961,0.501961}%
\pgfsetstrokecolor{currentstroke}%
\pgfsetdash{}{0pt}%
\pgfpathmoveto{\pgfqpoint{1.837668in}{2.312028in}}%
\pgfpathlineto{\pgfqpoint{1.904982in}{2.324207in}}%
\pgfpathlineto{\pgfqpoint{1.972801in}{2.333136in}}%
\pgfpathlineto{\pgfqpoint{2.040973in}{2.338792in}}%
\pgfpathlineto{\pgfqpoint{2.109333in}{2.341510in}}%
\pgfpathlineto{\pgfqpoint{2.177755in}{2.342238in}}%
\pgfpathlineto{\pgfqpoint{2.246176in}{2.342910in}}%
\pgfusepath{stroke}%
\end{pgfscope}%
\begin{pgfscope}%
\pgfpathrectangle{\pgfqpoint{0.647939in}{0.492442in}}{\pgfqpoint{3.079299in}{3.079299in}}%
\pgfusepath{clip}%
\pgfsetbuttcap%
\pgfsetroundjoin%
\pgfsetlinewidth{0.803000pt}%
\definecolor{currentstroke}{rgb}{0.501961,0.501961,0.501961}%
\pgfsetstrokecolor{currentstroke}%
\pgfsetdash{}{0pt}%
\pgfpathmoveto{\pgfqpoint{2.467525in}{1.822139in}}%
\pgfpathlineto{\pgfqpoint{2.400288in}{1.834599in}}%
\pgfpathlineto{\pgfqpoint{2.332368in}{1.842576in}}%
\pgfpathlineto{\pgfqpoint{2.264078in}{1.846464in}}%
\pgfpathlineto{\pgfqpoint{2.195664in}{1.847529in}}%
\pgfpathlineto{\pgfqpoint{2.127278in}{1.848999in}}%
\pgfpathlineto{\pgfqpoint{2.127278in}{1.848999in}}%
\pgfpathlineto{\pgfqpoint{2.082322in}{1.854724in}}%
\pgfpathlineto{\pgfqpoint{2.082322in}{1.854724in}}%
\pgfusepath{stroke}%
\end{pgfscope}%
\begin{pgfscope}%
\pgfpathrectangle{\pgfqpoint{0.647939in}{0.492442in}}{\pgfqpoint{3.079299in}{3.079299in}}%
\pgfusepath{clip}%
\pgfsetroundcap%
\pgfsetroundjoin%
\pgfsetlinewidth{0.803000pt}%
\definecolor{currentstroke}{rgb}{0.501961,0.501961,0.501961}%
\pgfsetstrokecolor{currentstroke}%
\pgfsetdash{}{0pt}%
\pgfpathmoveto{\pgfqpoint{2.110292in}{1.960811in}}%
\pgfpathquadraticcurveto{\pgfqpoint{2.113691in}{1.963812in}}{\pgfqpoint{2.107778in}{1.958591in}}%
\pgfusepath{stroke}%
\end{pgfscope}%
\begin{pgfscope}%
\pgfpathrectangle{\pgfqpoint{0.647939in}{0.492442in}}{\pgfqpoint{3.079299in}{3.079299in}}%
\pgfusepath{clip}%
\pgfsetroundcap%
\pgfsetroundjoin%
\definecolor{currentfill}{rgb}{0.501961,0.501961,0.501961}%
\pgfsetfillcolor{currentfill}%
\pgfsetlinewidth{0.803000pt}%
\definecolor{currentstroke}{rgb}{0.501961,0.501961,0.501961}%
\pgfsetstrokecolor{currentstroke}%
\pgfsetdash{}{0pt}%
\pgfpathmoveto{\pgfqpoint{2.035741in}{1.939453in}}%
\pgfpathlineto{\pgfqpoint{2.107778in}{1.958591in}}%
\pgfpathlineto{\pgfqpoint{2.079866in}{1.889479in}}%
\pgfpathlineto{\pgfqpoint{2.035741in}{1.939453in}}%
\pgfpathclose%
\pgfusepath{stroke,fill}%
\end{pgfscope}%
\begin{pgfscope}%
\pgfpathrectangle{\pgfqpoint{0.647939in}{0.492442in}}{\pgfqpoint{3.079299in}{3.079299in}}%
\pgfusepath{clip}%
\pgfsetroundcap%
\pgfsetroundjoin%
\pgfsetlinewidth{0.803000pt}%
\definecolor{currentstroke}{rgb}{0.501961,0.501961,0.501961}%
\pgfsetstrokecolor{currentstroke}%
\pgfsetdash{}{0pt}%
\pgfpathmoveto{\pgfqpoint{1.064119in}{0.599344in}}%
\pgfpathquadraticcurveto{\pgfqpoint{1.066731in}{0.601170in}}{\pgfqpoint{1.059162in}{0.595878in}}%
\pgfusepath{stroke}%
\end{pgfscope}%
\begin{pgfscope}%
\pgfpathrectangle{\pgfqpoint{0.647939in}{0.492442in}}{\pgfqpoint{3.079299in}{3.079299in}}%
\pgfusepath{clip}%
\pgfsetroundcap%
\pgfsetroundjoin%
\definecolor{currentfill}{rgb}{0.501961,0.501961,0.501961}%
\pgfsetfillcolor{currentfill}%
\pgfsetlinewidth{0.803000pt}%
\definecolor{currentstroke}{rgb}{0.501961,0.501961,0.501961}%
\pgfsetstrokecolor{currentstroke}%
\pgfsetdash{}{0pt}%
\pgfpathmoveto{\pgfqpoint{0.985426in}{0.584992in}}%
\pgfpathlineto{\pgfqpoint{1.059162in}{0.595878in}}%
\pgfpathlineto{\pgfqpoint{1.023629in}{0.530357in}}%
\pgfpathlineto{\pgfqpoint{0.985426in}{0.584992in}}%
\pgfpathclose%
\pgfusepath{stroke,fill}%
\end{pgfscope}%
\begin{pgfscope}%
\pgfpathrectangle{\pgfqpoint{0.647939in}{0.492442in}}{\pgfqpoint{3.079299in}{3.079299in}}%
\pgfusepath{clip}%
\pgfsetroundcap%
\pgfsetroundjoin%
\pgfsetlinewidth{0.803000pt}%
\definecolor{currentstroke}{rgb}{0.501961,0.501961,0.501961}%
\pgfsetstrokecolor{currentstroke}%
\pgfsetdash{}{0pt}%
\pgfpathmoveto{\pgfqpoint{1.309246in}{0.738971in}}%
\pgfpathquadraticcurveto{\pgfqpoint{1.310757in}{0.741779in}}{\pgfqpoint{1.306380in}{0.733648in}}%
\pgfusepath{stroke}%
\end{pgfscope}%
\begin{pgfscope}%
\pgfpathrectangle{\pgfqpoint{0.647939in}{0.492442in}}{\pgfqpoint{3.079299in}{3.079299in}}%
\pgfusepath{clip}%
\pgfsetroundcap%
\pgfsetroundjoin%
\definecolor{currentfill}{rgb}{0.501961,0.501961,0.501961}%
\pgfsetfillcolor{currentfill}%
\pgfsetlinewidth{0.803000pt}%
\definecolor{currentstroke}{rgb}{0.501961,0.501961,0.501961}%
\pgfsetstrokecolor{currentstroke}%
\pgfsetdash{}{0pt}%
\pgfpathmoveto{\pgfqpoint{1.245429in}{0.690747in}}%
\pgfpathlineto{\pgfqpoint{1.306380in}{0.733648in}}%
\pgfpathlineto{\pgfqpoint{1.304130in}{0.659146in}}%
\pgfpathlineto{\pgfqpoint{1.245429in}{0.690747in}}%
\pgfpathclose%
\pgfusepath{stroke,fill}%
\end{pgfscope}%
\begin{pgfscope}%
\pgfpathrectangle{\pgfqpoint{0.647939in}{0.492442in}}{\pgfqpoint{3.079299in}{3.079299in}}%
\pgfusepath{clip}%
\pgfsetroundcap%
\pgfsetroundjoin%
\pgfsetlinewidth{0.803000pt}%
\definecolor{currentstroke}{rgb}{0.501961,0.501961,0.501961}%
\pgfsetstrokecolor{currentstroke}%
\pgfsetdash{}{0pt}%
\pgfpathmoveto{\pgfqpoint{1.362904in}{0.655556in}}%
\pgfpathquadraticcurveto{\pgfqpoint{1.363472in}{0.658689in}}{\pgfqpoint{1.361824in}{0.649598in}}%
\pgfusepath{stroke}%
\end{pgfscope}%
\begin{pgfscope}%
\pgfpathrectangle{\pgfqpoint{0.647939in}{0.492442in}}{\pgfqpoint{3.079299in}{3.079299in}}%
\pgfusepath{clip}%
\pgfsetroundcap%
\pgfsetroundjoin%
\definecolor{currentfill}{rgb}{0.501961,0.501961,0.501961}%
\pgfsetfillcolor{currentfill}%
\pgfsetlinewidth{0.803000pt}%
\definecolor{currentstroke}{rgb}{0.501961,0.501961,0.501961}%
\pgfsetstrokecolor{currentstroke}%
\pgfsetdash{}{0pt}%
\pgfpathmoveto{\pgfqpoint{1.317138in}{0.589944in}}%
\pgfpathlineto{\pgfqpoint{1.361824in}{0.649598in}}%
\pgfpathlineto{\pgfqpoint{1.382736in}{0.578056in}}%
\pgfpathlineto{\pgfqpoint{1.317138in}{0.589944in}}%
\pgfpathclose%
\pgfusepath{stroke,fill}%
\end{pgfscope}%
\begin{pgfscope}%
\pgfpathrectangle{\pgfqpoint{0.647939in}{0.492442in}}{\pgfqpoint{3.079299in}{3.079299in}}%
\pgfusepath{clip}%
\pgfsetroundcap%
\pgfsetroundjoin%
\pgfsetlinewidth{0.803000pt}%
\definecolor{currentstroke}{rgb}{0.501961,0.501961,0.501961}%
\pgfsetstrokecolor{currentstroke}%
\pgfsetdash{}{0pt}%
\pgfpathmoveto{\pgfqpoint{1.503631in}{0.596103in}}%
\pgfpathquadraticcurveto{\pgfqpoint{1.502870in}{0.597224in}}{\pgfqpoint{1.509088in}{0.588068in}}%
\pgfusepath{stroke}%
\end{pgfscope}%
\begin{pgfscope}%
\pgfpathrectangle{\pgfqpoint{0.647939in}{0.492442in}}{\pgfqpoint{3.079299in}{3.079299in}}%
\pgfusepath{clip}%
\pgfsetroundcap%
\pgfsetroundjoin%
\definecolor{currentfill}{rgb}{0.501961,0.501961,0.501961}%
\pgfsetfillcolor{currentfill}%
\pgfsetlinewidth{0.803000pt}%
\definecolor{currentstroke}{rgb}{0.501961,0.501961,0.501961}%
\pgfsetstrokecolor{currentstroke}%
\pgfsetdash{}{0pt}%
\pgfpathmoveto{\pgfqpoint{1.518972in}{0.514191in}}%
\pgfpathlineto{\pgfqpoint{1.509088in}{0.588068in}}%
\pgfpathlineto{\pgfqpoint{1.574120in}{0.551649in}}%
\pgfpathlineto{\pgfqpoint{1.518972in}{0.514191in}}%
\pgfpathclose%
\pgfusepath{stroke,fill}%
\end{pgfscope}%
\begin{pgfscope}%
\pgfpathrectangle{\pgfqpoint{0.647939in}{0.492442in}}{\pgfqpoint{3.079299in}{3.079299in}}%
\pgfusepath{clip}%
\pgfsetroundcap%
\pgfsetroundjoin%
\pgfsetlinewidth{0.803000pt}%
\definecolor{currentstroke}{rgb}{0.501961,0.501961,0.501961}%
\pgfsetstrokecolor{currentstroke}%
\pgfsetdash{}{0pt}%
\pgfpathmoveto{\pgfqpoint{1.678380in}{0.531673in}}%
\pgfpathquadraticcurveto{\pgfqpoint{1.675420in}{0.532842in}}{\pgfqpoint{1.684014in}{0.529448in}}%
\pgfusepath{stroke}%
\end{pgfscope}%
\begin{pgfscope}%
\pgfpathrectangle{\pgfqpoint{0.647939in}{0.492442in}}{\pgfqpoint{3.079299in}{3.079299in}}%
\pgfusepath{clip}%
\pgfsetroundcap%
\pgfsetroundjoin%
\definecolor{currentfill}{rgb}{0.501961,0.501961,0.501961}%
\pgfsetfillcolor{currentfill}%
\pgfsetlinewidth{0.803000pt}%
\definecolor{currentstroke}{rgb}{0.501961,0.501961,0.501961}%
\pgfsetstrokecolor{currentstroke}%
\pgfsetdash{}{0pt}%
\pgfpathmoveto{\pgfqpoint{1.733775in}{0.473956in}}%
\pgfpathlineto{\pgfqpoint{1.684014in}{0.529448in}}%
\pgfpathlineto{\pgfqpoint{1.758264in}{0.535962in}}%
\pgfpathlineto{\pgfqpoint{1.733775in}{0.473956in}}%
\pgfpathclose%
\pgfusepath{stroke,fill}%
\end{pgfscope}%
\begin{pgfscope}%
\pgfpathrectangle{\pgfqpoint{0.647939in}{0.492442in}}{\pgfqpoint{3.079299in}{3.079299in}}%
\pgfusepath{clip}%
\pgfsetroundcap%
\pgfsetroundjoin%
\pgfsetlinewidth{0.803000pt}%
\definecolor{currentstroke}{rgb}{0.501961,0.501961,0.501961}%
\pgfsetstrokecolor{currentstroke}%
\pgfsetdash{}{0pt}%
\pgfpathmoveto{\pgfqpoint{2.092952in}{0.493031in}}%
\pgfpathquadraticcurveto{\pgfqpoint{2.089728in}{0.493065in}}{\pgfqpoint{2.098926in}{0.492968in}}%
\pgfusepath{stroke}%
\end{pgfscope}%
\begin{pgfscope}%
\pgfpathrectangle{\pgfqpoint{0.647939in}{0.492442in}}{\pgfqpoint{3.079299in}{3.079299in}}%
\pgfusepath{clip}%
\pgfsetroundcap%
\pgfsetroundjoin%
\definecolor{currentfill}{rgb}{0.501961,0.501961,0.501961}%
\pgfsetfillcolor{currentfill}%
\pgfsetlinewidth{0.803000pt}%
\definecolor{currentstroke}{rgb}{0.501961,0.501961,0.501961}%
\pgfsetstrokecolor{currentstroke}%
\pgfsetdash{}{0pt}%
\pgfpathmoveto{\pgfqpoint{2.165236in}{0.458931in}}%
\pgfpathlineto{\pgfqpoint{2.098926in}{0.492968in}}%
\pgfpathlineto{\pgfqpoint{2.165941in}{0.525594in}}%
\pgfpathlineto{\pgfqpoint{2.165236in}{0.458931in}}%
\pgfpathclose%
\pgfusepath{stroke,fill}%
\end{pgfscope}%
\begin{pgfscope}%
\pgfpathrectangle{\pgfqpoint{0.647939in}{0.492442in}}{\pgfqpoint{3.079299in}{3.079299in}}%
\pgfusepath{clip}%
\pgfsetroundcap%
\pgfsetroundjoin%
\pgfsetlinewidth{0.803000pt}%
\definecolor{currentstroke}{rgb}{0.501961,0.501961,0.501961}%
\pgfsetstrokecolor{currentstroke}%
\pgfsetdash{}{0pt}%
\pgfpathmoveto{\pgfqpoint{2.513822in}{0.509774in}}%
\pgfpathquadraticcurveto{\pgfqpoint{2.510606in}{0.510002in}}{\pgfqpoint{2.519782in}{0.509352in}}%
\pgfusepath{stroke}%
\end{pgfscope}%
\begin{pgfscope}%
\pgfpathrectangle{\pgfqpoint{0.647939in}{0.492442in}}{\pgfqpoint{3.079299in}{3.079299in}}%
\pgfusepath{clip}%
\pgfsetroundcap%
\pgfsetroundjoin%
\definecolor{currentfill}{rgb}{0.501961,0.501961,0.501961}%
\pgfsetfillcolor{currentfill}%
\pgfsetlinewidth{0.803000pt}%
\definecolor{currentstroke}{rgb}{0.501961,0.501961,0.501961}%
\pgfsetstrokecolor{currentstroke}%
\pgfsetdash{}{0pt}%
\pgfpathmoveto{\pgfqpoint{2.583928in}{0.471394in}}%
\pgfpathlineto{\pgfqpoint{2.519782in}{0.509352in}}%
\pgfpathlineto{\pgfqpoint{2.588636in}{0.537894in}}%
\pgfpathlineto{\pgfqpoint{2.583928in}{0.471394in}}%
\pgfpathclose%
\pgfusepath{stroke,fill}%
\end{pgfscope}%
\begin{pgfscope}%
\pgfpathrectangle{\pgfqpoint{0.647939in}{0.492442in}}{\pgfqpoint{3.079299in}{3.079299in}}%
\pgfusepath{clip}%
\pgfsetroundcap%
\pgfsetroundjoin%
\pgfsetlinewidth{0.803000pt}%
\definecolor{currentstroke}{rgb}{0.501961,0.501961,0.501961}%
\pgfsetstrokecolor{currentstroke}%
\pgfsetdash{}{0pt}%
\pgfpathmoveto{\pgfqpoint{1.975394in}{0.563539in}}%
\pgfpathquadraticcurveto{\pgfqpoint{1.972175in}{0.563706in}}{\pgfqpoint{1.981362in}{0.563231in}}%
\pgfusepath{stroke}%
\end{pgfscope}%
\begin{pgfscope}%
\pgfpathrectangle{\pgfqpoint{0.647939in}{0.492442in}}{\pgfqpoint{3.079299in}{3.079299in}}%
\pgfusepath{clip}%
\pgfsetroundcap%
\pgfsetroundjoin%
\definecolor{currentfill}{rgb}{0.501961,0.501961,0.501961}%
\pgfsetfillcolor{currentfill}%
\pgfsetlinewidth{0.803000pt}%
\definecolor{currentstroke}{rgb}{0.501961,0.501961,0.501961}%
\pgfsetstrokecolor{currentstroke}%
\pgfsetdash{}{0pt}%
\pgfpathmoveto{\pgfqpoint{2.046219in}{0.526502in}}%
\pgfpathlineto{\pgfqpoint{1.981362in}{0.563231in}}%
\pgfpathlineto{\pgfqpoint{2.049660in}{0.593080in}}%
\pgfpathlineto{\pgfqpoint{2.046219in}{0.526502in}}%
\pgfpathclose%
\pgfusepath{stroke,fill}%
\end{pgfscope}%
\begin{pgfscope}%
\pgfpathrectangle{\pgfqpoint{0.647939in}{0.492442in}}{\pgfqpoint{3.079299in}{3.079299in}}%
\pgfusepath{clip}%
\pgfsetroundcap%
\pgfsetroundjoin%
\pgfsetlinewidth{0.803000pt}%
\definecolor{currentstroke}{rgb}{0.501961,0.501961,0.501961}%
\pgfsetstrokecolor{currentstroke}%
\pgfsetdash{}{0pt}%
\pgfpathmoveto{\pgfqpoint{2.939431in}{0.538741in}}%
\pgfpathquadraticcurveto{\pgfqpoint{2.936315in}{0.539567in}}{\pgfqpoint{2.945206in}{0.537210in}}%
\pgfusepath{stroke}%
\end{pgfscope}%
\begin{pgfscope}%
\pgfpathrectangle{\pgfqpoint{0.647939in}{0.492442in}}{\pgfqpoint{3.079299in}{3.079299in}}%
\pgfusepath{clip}%
\pgfsetroundcap%
\pgfsetroundjoin%
\definecolor{currentfill}{rgb}{0.501961,0.501961,0.501961}%
\pgfsetfillcolor{currentfill}%
\pgfsetlinewidth{0.803000pt}%
\definecolor{currentstroke}{rgb}{0.501961,0.501961,0.501961}%
\pgfsetstrokecolor{currentstroke}%
\pgfsetdash{}{0pt}%
\pgfpathmoveto{\pgfqpoint{3.001104in}{0.487906in}}%
\pgfpathlineto{\pgfqpoint{2.945206in}{0.537210in}}%
\pgfpathlineto{\pgfqpoint{3.018188in}{0.552346in}}%
\pgfpathlineto{\pgfqpoint{3.001104in}{0.487906in}}%
\pgfpathclose%
\pgfusepath{stroke,fill}%
\end{pgfscope}%
\begin{pgfscope}%
\pgfpathrectangle{\pgfqpoint{0.647939in}{0.492442in}}{\pgfqpoint{3.079299in}{3.079299in}}%
\pgfusepath{clip}%
\pgfsetroundcap%
\pgfsetroundjoin%
\pgfsetlinewidth{0.803000pt}%
\definecolor{currentstroke}{rgb}{0.501961,0.501961,0.501961}%
\pgfsetstrokecolor{currentstroke}%
\pgfsetdash{}{0pt}%
\pgfpathmoveto{\pgfqpoint{2.348954in}{0.698716in}}%
\pgfpathquadraticcurveto{\pgfqpoint{2.345731in}{0.698786in}}{\pgfqpoint{2.354927in}{0.698585in}}%
\pgfusepath{stroke}%
\end{pgfscope}%
\begin{pgfscope}%
\pgfpathrectangle{\pgfqpoint{0.647939in}{0.492442in}}{\pgfqpoint{3.079299in}{3.079299in}}%
\pgfusepath{clip}%
\pgfsetroundcap%
\pgfsetroundjoin%
\definecolor{currentfill}{rgb}{0.501961,0.501961,0.501961}%
\pgfsetfillcolor{currentfill}%
\pgfsetlinewidth{0.803000pt}%
\definecolor{currentstroke}{rgb}{0.501961,0.501961,0.501961}%
\pgfsetstrokecolor{currentstroke}%
\pgfsetdash{}{0pt}%
\pgfpathmoveto{\pgfqpoint{2.420850in}{0.663803in}}%
\pgfpathlineto{\pgfqpoint{2.354927in}{0.698585in}}%
\pgfpathlineto{\pgfqpoint{2.422306in}{0.730454in}}%
\pgfpathlineto{\pgfqpoint{2.420850in}{0.663803in}}%
\pgfpathclose%
\pgfusepath{stroke,fill}%
\end{pgfscope}%
\begin{pgfscope}%
\pgfpathrectangle{\pgfqpoint{0.647939in}{0.492442in}}{\pgfqpoint{3.079299in}{3.079299in}}%
\pgfusepath{clip}%
\pgfsetroundcap%
\pgfsetroundjoin%
\pgfsetlinewidth{0.803000pt}%
\definecolor{currentstroke}{rgb}{0.501961,0.501961,0.501961}%
\pgfsetstrokecolor{currentstroke}%
\pgfsetdash{}{0pt}%
\pgfpathmoveto{\pgfqpoint{2.641756in}{0.759983in}}%
\pgfpathquadraticcurveto{\pgfqpoint{2.638566in}{0.760444in}}{\pgfqpoint{2.647670in}{0.759128in}}%
\pgfusepath{stroke}%
\end{pgfscope}%
\begin{pgfscope}%
\pgfpathrectangle{\pgfqpoint{0.647939in}{0.492442in}}{\pgfqpoint{3.079299in}{3.079299in}}%
\pgfusepath{clip}%
\pgfsetroundcap%
\pgfsetroundjoin%
\definecolor{currentfill}{rgb}{0.501961,0.501961,0.501961}%
\pgfsetfillcolor{currentfill}%
\pgfsetlinewidth{0.803000pt}%
\definecolor{currentstroke}{rgb}{0.501961,0.501961,0.501961}%
\pgfsetstrokecolor{currentstroke}%
\pgfsetdash{}{0pt}%
\pgfpathmoveto{\pgfqpoint{2.708884in}{0.716603in}}%
\pgfpathlineto{\pgfqpoint{2.647670in}{0.759128in}}%
\pgfpathlineto{\pgfqpoint{2.718418in}{0.782584in}}%
\pgfpathlineto{\pgfqpoint{2.708884in}{0.716603in}}%
\pgfpathclose%
\pgfusepath{stroke,fill}%
\end{pgfscope}%
\begin{pgfscope}%
\pgfpathrectangle{\pgfqpoint{0.647939in}{0.492442in}}{\pgfqpoint{3.079299in}{3.079299in}}%
\pgfusepath{clip}%
\pgfsetroundcap%
\pgfsetroundjoin%
\pgfsetlinewidth{0.803000pt}%
\definecolor{currentstroke}{rgb}{0.501961,0.501961,0.501961}%
\pgfsetstrokecolor{currentstroke}%
\pgfsetdash{}{0pt}%
\pgfpathmoveto{\pgfqpoint{2.524605in}{0.853526in}}%
\pgfpathquadraticcurveto{\pgfqpoint{2.521395in}{0.853824in}}{\pgfqpoint{2.530555in}{0.852974in}}%
\pgfusepath{stroke}%
\end{pgfscope}%
\begin{pgfscope}%
\pgfpathrectangle{\pgfqpoint{0.647939in}{0.492442in}}{\pgfqpoint{3.079299in}{3.079299in}}%
\pgfusepath{clip}%
\pgfsetroundcap%
\pgfsetroundjoin%
\definecolor{currentfill}{rgb}{0.501961,0.501961,0.501961}%
\pgfsetfillcolor{currentfill}%
\pgfsetlinewidth{0.803000pt}%
\definecolor{currentstroke}{rgb}{0.501961,0.501961,0.501961}%
\pgfsetstrokecolor{currentstroke}%
\pgfsetdash{}{0pt}%
\pgfpathmoveto{\pgfqpoint{2.593858in}{0.813626in}}%
\pgfpathlineto{\pgfqpoint{2.530555in}{0.852974in}}%
\pgfpathlineto{\pgfqpoint{2.600015in}{0.880007in}}%
\pgfpathlineto{\pgfqpoint{2.593858in}{0.813626in}}%
\pgfpathclose%
\pgfusepath{stroke,fill}%
\end{pgfscope}%
\begin{pgfscope}%
\pgfpathrectangle{\pgfqpoint{0.647939in}{0.492442in}}{\pgfqpoint{3.079299in}{3.079299in}}%
\pgfusepath{clip}%
\pgfsetroundcap%
\pgfsetroundjoin%
\pgfsetlinewidth{0.803000pt}%
\definecolor{currentstroke}{rgb}{0.501961,0.501961,0.501961}%
\pgfsetstrokecolor{currentstroke}%
\pgfsetdash{}{0pt}%
\pgfpathmoveto{\pgfqpoint{2.731889in}{0.906398in}}%
\pgfpathquadraticcurveto{\pgfqpoint{2.728735in}{0.907063in}}{\pgfqpoint{2.737736in}{0.905165in}}%
\pgfusepath{stroke}%
\end{pgfscope}%
\begin{pgfscope}%
\pgfpathrectangle{\pgfqpoint{0.647939in}{0.492442in}}{\pgfqpoint{3.079299in}{3.079299in}}%
\pgfusepath{clip}%
\pgfsetroundcap%
\pgfsetroundjoin%
\definecolor{currentfill}{rgb}{0.501961,0.501961,0.501961}%
\pgfsetfillcolor{currentfill}%
\pgfsetlinewidth{0.803000pt}%
\definecolor{currentstroke}{rgb}{0.501961,0.501961,0.501961}%
\pgfsetstrokecolor{currentstroke}%
\pgfsetdash{}{0pt}%
\pgfpathmoveto{\pgfqpoint{2.796091in}{0.858795in}}%
\pgfpathlineto{\pgfqpoint{2.737736in}{0.905165in}}%
\pgfpathlineto{\pgfqpoint{2.809845in}{0.924027in}}%
\pgfpathlineto{\pgfqpoint{2.796091in}{0.858795in}}%
\pgfpathclose%
\pgfusepath{stroke,fill}%
\end{pgfscope}%
\begin{pgfscope}%
\pgfpathrectangle{\pgfqpoint{0.647939in}{0.492442in}}{\pgfqpoint{3.079299in}{3.079299in}}%
\pgfusepath{clip}%
\pgfsetroundcap%
\pgfsetroundjoin%
\pgfsetlinewidth{0.803000pt}%
\definecolor{currentstroke}{rgb}{0.501961,0.501961,0.501961}%
\pgfsetstrokecolor{currentstroke}%
\pgfsetdash{}{0pt}%
\pgfpathmoveto{\pgfqpoint{2.602444in}{1.015956in}}%
\pgfpathquadraticcurveto{\pgfqpoint{2.599256in}{1.016429in}}{\pgfqpoint{2.608355in}{1.015078in}}%
\pgfusepath{stroke}%
\end{pgfscope}%
\begin{pgfscope}%
\pgfpathrectangle{\pgfqpoint{0.647939in}{0.492442in}}{\pgfqpoint{3.079299in}{3.079299in}}%
\pgfusepath{clip}%
\pgfsetroundcap%
\pgfsetroundjoin%
\definecolor{currentfill}{rgb}{0.501961,0.501961,0.501961}%
\pgfsetfillcolor{currentfill}%
\pgfsetlinewidth{0.803000pt}%
\definecolor{currentstroke}{rgb}{0.501961,0.501961,0.501961}%
\pgfsetstrokecolor{currentstroke}%
\pgfsetdash{}{0pt}%
\pgfpathmoveto{\pgfqpoint{2.669406in}{0.972320in}}%
\pgfpathlineto{\pgfqpoint{2.608355in}{1.015078in}}%
\pgfpathlineto{\pgfqpoint{2.679193in}{1.038264in}}%
\pgfpathlineto{\pgfqpoint{2.669406in}{0.972320in}}%
\pgfpathclose%
\pgfusepath{stroke,fill}%
\end{pgfscope}%
\begin{pgfscope}%
\pgfpathrectangle{\pgfqpoint{0.647939in}{0.492442in}}{\pgfqpoint{3.079299in}{3.079299in}}%
\pgfusepath{clip}%
\pgfsetroundcap%
\pgfsetroundjoin%
\pgfsetlinewidth{0.803000pt}%
\definecolor{currentstroke}{rgb}{0.501961,0.501961,0.501961}%
\pgfsetstrokecolor{currentstroke}%
\pgfsetdash{}{0pt}%
\pgfpathmoveto{\pgfqpoint{2.742398in}{1.075465in}}%
\pgfpathquadraticcurveto{\pgfqpoint{2.739267in}{1.076229in}}{\pgfqpoint{2.748203in}{1.074047in}}%
\pgfusepath{stroke}%
\end{pgfscope}%
\begin{pgfscope}%
\pgfpathrectangle{\pgfqpoint{0.647939in}{0.492442in}}{\pgfqpoint{3.079299in}{3.079299in}}%
\pgfusepath{clip}%
\pgfsetroundcap%
\pgfsetroundjoin%
\definecolor{currentfill}{rgb}{0.501961,0.501961,0.501961}%
\pgfsetfillcolor{currentfill}%
\pgfsetlinewidth{0.803000pt}%
\definecolor{currentstroke}{rgb}{0.501961,0.501961,0.501961}%
\pgfsetstrokecolor{currentstroke}%
\pgfsetdash{}{0pt}%
\pgfpathmoveto{\pgfqpoint{2.805060in}{1.025851in}}%
\pgfpathlineto{\pgfqpoint{2.748203in}{1.074047in}}%
\pgfpathlineto{\pgfqpoint{2.820874in}{1.090615in}}%
\pgfpathlineto{\pgfqpoint{2.805060in}{1.025851in}}%
\pgfpathclose%
\pgfusepath{stroke,fill}%
\end{pgfscope}%
\begin{pgfscope}%
\pgfpathrectangle{\pgfqpoint{0.647939in}{0.492442in}}{\pgfqpoint{3.079299in}{3.079299in}}%
\pgfusepath{clip}%
\pgfsetroundcap%
\pgfsetroundjoin%
\pgfsetlinewidth{0.803000pt}%
\definecolor{currentstroke}{rgb}{0.501961,0.501961,0.501961}%
\pgfsetstrokecolor{currentstroke}%
\pgfsetdash{}{0pt}%
\pgfpathmoveto{\pgfqpoint{2.814533in}{1.143057in}}%
\pgfpathquadraticcurveto{\pgfqpoint{2.811449in}{1.143998in}}{\pgfqpoint{2.820248in}{1.141313in}}%
\pgfusepath{stroke}%
\end{pgfscope}%
\begin{pgfscope}%
\pgfpathrectangle{\pgfqpoint{0.647939in}{0.492442in}}{\pgfqpoint{3.079299in}{3.079299in}}%
\pgfusepath{clip}%
\pgfsetroundcap%
\pgfsetroundjoin%
\definecolor{currentfill}{rgb}{0.501961,0.501961,0.501961}%
\pgfsetfillcolor{currentfill}%
\pgfsetlinewidth{0.803000pt}%
\definecolor{currentstroke}{rgb}{0.501961,0.501961,0.501961}%
\pgfsetstrokecolor{currentstroke}%
\pgfsetdash{}{0pt}%
\pgfpathmoveto{\pgfqpoint{2.874280in}{1.089970in}}%
\pgfpathlineto{\pgfqpoint{2.820248in}{1.141313in}}%
\pgfpathlineto{\pgfqpoint{2.893741in}{1.153733in}}%
\pgfpathlineto{\pgfqpoint{2.874280in}{1.089970in}}%
\pgfpathclose%
\pgfusepath{stroke,fill}%
\end{pgfscope}%
\begin{pgfscope}%
\pgfpathrectangle{\pgfqpoint{0.647939in}{0.492442in}}{\pgfqpoint{3.079299in}{3.079299in}}%
\pgfusepath{clip}%
\pgfsetroundcap%
\pgfsetroundjoin%
\pgfsetlinewidth{0.803000pt}%
\definecolor{currentstroke}{rgb}{0.501961,0.501961,0.501961}%
\pgfsetstrokecolor{currentstroke}%
\pgfsetdash{}{0pt}%
\pgfpathmoveto{\pgfqpoint{2.690251in}{1.268263in}}%
\pgfpathquadraticcurveto{\pgfqpoint{2.687122in}{1.269035in}}{\pgfqpoint{2.696053in}{1.266832in}}%
\pgfusepath{stroke}%
\end{pgfscope}%
\begin{pgfscope}%
\pgfpathrectangle{\pgfqpoint{0.647939in}{0.492442in}}{\pgfqpoint{3.079299in}{3.079299in}}%
\pgfusepath{clip}%
\pgfsetroundcap%
\pgfsetroundjoin%
\definecolor{currentfill}{rgb}{0.501961,0.501961,0.501961}%
\pgfsetfillcolor{currentfill}%
\pgfsetlinewidth{0.803000pt}%
\definecolor{currentstroke}{rgb}{0.501961,0.501961,0.501961}%
\pgfsetstrokecolor{currentstroke}%
\pgfsetdash{}{0pt}%
\pgfpathmoveto{\pgfqpoint{2.752796in}{1.218501in}}%
\pgfpathlineto{\pgfqpoint{2.696053in}{1.266832in}}%
\pgfpathlineto{\pgfqpoint{2.768763in}{1.283228in}}%
\pgfpathlineto{\pgfqpoint{2.752796in}{1.218501in}}%
\pgfpathclose%
\pgfusepath{stroke,fill}%
\end{pgfscope}%
\begin{pgfscope}%
\pgfpathrectangle{\pgfqpoint{0.647939in}{0.492442in}}{\pgfqpoint{3.079299in}{3.079299in}}%
\pgfusepath{clip}%
\pgfsetroundcap%
\pgfsetroundjoin%
\pgfsetlinewidth{0.803000pt}%
\definecolor{currentstroke}{rgb}{0.501961,0.501961,0.501961}%
\pgfsetstrokecolor{currentstroke}%
\pgfsetdash{}{0pt}%
\pgfpathmoveto{\pgfqpoint{2.764892in}{1.340376in}}%
\pgfpathquadraticcurveto{\pgfqpoint{2.761823in}{1.341363in}}{\pgfqpoint{2.770581in}{1.338547in}}%
\pgfusepath{stroke}%
\end{pgfscope}%
\begin{pgfscope}%
\pgfpathrectangle{\pgfqpoint{0.647939in}{0.492442in}}{\pgfqpoint{3.079299in}{3.079299in}}%
\pgfusepath{clip}%
\pgfsetroundcap%
\pgfsetroundjoin%
\definecolor{currentfill}{rgb}{0.501961,0.501961,0.501961}%
\pgfsetfillcolor{currentfill}%
\pgfsetlinewidth{0.803000pt}%
\definecolor{currentstroke}{rgb}{0.501961,0.501961,0.501961}%
\pgfsetstrokecolor{currentstroke}%
\pgfsetdash{}{0pt}%
\pgfpathmoveto{\pgfqpoint{2.823845in}{1.286408in}}%
\pgfpathlineto{\pgfqpoint{2.770581in}{1.338547in}}%
\pgfpathlineto{\pgfqpoint{2.844251in}{1.349875in}}%
\pgfpathlineto{\pgfqpoint{2.823845in}{1.286408in}}%
\pgfpathclose%
\pgfusepath{stroke,fill}%
\end{pgfscope}%
\begin{pgfscope}%
\pgfpathrectangle{\pgfqpoint{0.647939in}{0.492442in}}{\pgfqpoint{3.079299in}{3.079299in}}%
\pgfusepath{clip}%
\pgfsetroundcap%
\pgfsetroundjoin%
\pgfsetlinewidth{0.803000pt}%
\definecolor{currentstroke}{rgb}{0.501961,0.501961,0.501961}%
\pgfsetstrokecolor{currentstroke}%
\pgfsetdash{}{0pt}%
\pgfpathmoveto{\pgfqpoint{2.775082in}{1.432621in}}%
\pgfpathquadraticcurveto{\pgfqpoint{2.772048in}{1.433712in}}{\pgfqpoint{2.780705in}{1.430599in}}%
\pgfusepath{stroke}%
\end{pgfscope}%
\begin{pgfscope}%
\pgfpathrectangle{\pgfqpoint{0.647939in}{0.492442in}}{\pgfqpoint{3.079299in}{3.079299in}}%
\pgfusepath{clip}%
\pgfsetroundcap%
\pgfsetroundjoin%
\definecolor{currentfill}{rgb}{0.501961,0.501961,0.501961}%
\pgfsetfillcolor{currentfill}%
\pgfsetlinewidth{0.803000pt}%
\definecolor{currentstroke}{rgb}{0.501961,0.501961,0.501961}%
\pgfsetstrokecolor{currentstroke}%
\pgfsetdash{}{0pt}%
\pgfpathmoveto{\pgfqpoint{2.832155in}{1.376669in}}%
\pgfpathlineto{\pgfqpoint{2.780705in}{1.430599in}}%
\pgfpathlineto{\pgfqpoint{2.854719in}{1.439402in}}%
\pgfpathlineto{\pgfqpoint{2.832155in}{1.376669in}}%
\pgfpathclose%
\pgfusepath{stroke,fill}%
\end{pgfscope}%
\begin{pgfscope}%
\pgfpathrectangle{\pgfqpoint{0.647939in}{0.492442in}}{\pgfqpoint{3.079299in}{3.079299in}}%
\pgfusepath{clip}%
\pgfsetroundcap%
\pgfsetroundjoin%
\pgfsetlinewidth{0.803000pt}%
\definecolor{currentstroke}{rgb}{0.501961,0.501961,0.501961}%
\pgfsetstrokecolor{currentstroke}%
\pgfsetdash{}{0pt}%
\pgfpathmoveto{\pgfqpoint{2.849997in}{1.500250in}}%
\pgfpathquadraticcurveto{\pgfqpoint{2.847057in}{1.501572in}}{\pgfqpoint{2.855447in}{1.497801in}}%
\pgfusepath{stroke}%
\end{pgfscope}%
\begin{pgfscope}%
\pgfpathrectangle{\pgfqpoint{0.647939in}{0.492442in}}{\pgfqpoint{3.079299in}{3.079299in}}%
\pgfusepath{clip}%
\pgfsetroundcap%
\pgfsetroundjoin%
\definecolor{currentfill}{rgb}{0.501961,0.501961,0.501961}%
\pgfsetfillcolor{currentfill}%
\pgfsetlinewidth{0.803000pt}%
\definecolor{currentstroke}{rgb}{0.501961,0.501961,0.501961}%
\pgfsetstrokecolor{currentstroke}%
\pgfsetdash{}{0pt}%
\pgfpathmoveto{\pgfqpoint{2.902591in}{1.440069in}}%
\pgfpathlineto{\pgfqpoint{2.855447in}{1.497801in}}%
\pgfpathlineto{\pgfqpoint{2.929919in}{1.500877in}}%
\pgfpathlineto{\pgfqpoint{2.902591in}{1.440069in}}%
\pgfpathclose%
\pgfusepath{stroke,fill}%
\end{pgfscope}%
\begin{pgfscope}%
\pgfpathrectangle{\pgfqpoint{0.647939in}{0.492442in}}{\pgfqpoint{3.079299in}{3.079299in}}%
\pgfusepath{clip}%
\pgfsetroundcap%
\pgfsetroundjoin%
\pgfsetlinewidth{0.803000pt}%
\definecolor{currentstroke}{rgb}{0.501961,0.501961,0.501961}%
\pgfsetstrokecolor{currentstroke}%
\pgfsetdash{}{0pt}%
\pgfpathmoveto{\pgfqpoint{2.863189in}{1.594857in}}%
\pgfpathquadraticcurveto{\pgfqpoint{2.860317in}{1.596321in}}{\pgfqpoint{2.868512in}{1.592143in}}%
\pgfusepath{stroke}%
\end{pgfscope}%
\begin{pgfscope}%
\pgfpathrectangle{\pgfqpoint{0.647939in}{0.492442in}}{\pgfqpoint{3.079299in}{3.079299in}}%
\pgfusepath{clip}%
\pgfsetroundcap%
\pgfsetroundjoin%
\definecolor{currentfill}{rgb}{0.501961,0.501961,0.501961}%
\pgfsetfillcolor{currentfill}%
\pgfsetlinewidth{0.803000pt}%
\definecolor{currentstroke}{rgb}{0.501961,0.501961,0.501961}%
\pgfsetstrokecolor{currentstroke}%
\pgfsetdash{}{0pt}%
\pgfpathmoveto{\pgfqpoint{2.912763in}{1.532164in}}%
\pgfpathlineto{\pgfqpoint{2.868512in}{1.592143in}}%
\pgfpathlineto{\pgfqpoint{2.943045in}{1.591556in}}%
\pgfpathlineto{\pgfqpoint{2.912763in}{1.532164in}}%
\pgfpathclose%
\pgfusepath{stroke,fill}%
\end{pgfscope}%
\begin{pgfscope}%
\pgfpathrectangle{\pgfqpoint{0.647939in}{0.492442in}}{\pgfqpoint{3.079299in}{3.079299in}}%
\pgfusepath{clip}%
\pgfsetroundcap%
\pgfsetroundjoin%
\pgfsetlinewidth{0.803000pt}%
\definecolor{currentstroke}{rgb}{0.501961,0.501961,0.501961}%
\pgfsetstrokecolor{currentstroke}%
\pgfsetdash{}{0pt}%
\pgfpathmoveto{\pgfqpoint{2.937664in}{1.656821in}}%
\pgfpathquadraticcurveto{\pgfqpoint{2.934916in}{1.658508in}}{\pgfqpoint{2.942755in}{1.653695in}}%
\pgfusepath{stroke}%
\end{pgfscope}%
\begin{pgfscope}%
\pgfpathrectangle{\pgfqpoint{0.647939in}{0.492442in}}{\pgfqpoint{3.079299in}{3.079299in}}%
\pgfusepath{clip}%
\pgfsetroundcap%
\pgfsetroundjoin%
\definecolor{currentfill}{rgb}{0.501961,0.501961,0.501961}%
\pgfsetfillcolor{currentfill}%
\pgfsetlinewidth{0.803000pt}%
\definecolor{currentstroke}{rgb}{0.501961,0.501961,0.501961}%
\pgfsetstrokecolor{currentstroke}%
\pgfsetdash{}{0pt}%
\pgfpathmoveto{\pgfqpoint{2.982125in}{1.590405in}}%
\pgfpathlineto{\pgfqpoint{2.942755in}{1.653695in}}%
\pgfpathlineto{\pgfqpoint{3.017009in}{1.647217in}}%
\pgfpathlineto{\pgfqpoint{2.982125in}{1.590405in}}%
\pgfpathclose%
\pgfusepath{stroke,fill}%
\end{pgfscope}%
\begin{pgfscope}%
\pgfpathrectangle{\pgfqpoint{0.647939in}{0.492442in}}{\pgfqpoint{3.079299in}{3.079299in}}%
\pgfusepath{clip}%
\pgfsetroundcap%
\pgfsetroundjoin%
\pgfsetlinewidth{0.803000pt}%
\definecolor{currentstroke}{rgb}{0.501961,0.501961,0.501961}%
\pgfsetstrokecolor{currentstroke}%
\pgfsetdash{}{0pt}%
\pgfpathmoveto{\pgfqpoint{3.080307in}{1.765197in}}%
\pgfpathquadraticcurveto{\pgfqpoint{3.077807in}{1.767232in}}{\pgfqpoint{3.084941in}{1.761426in}}%
\pgfusepath{stroke}%
\end{pgfscope}%
\begin{pgfscope}%
\pgfpathrectangle{\pgfqpoint{0.647939in}{0.492442in}}{\pgfqpoint{3.079299in}{3.079299in}}%
\pgfusepath{clip}%
\pgfsetroundcap%
\pgfsetroundjoin%
\definecolor{currentfill}{rgb}{0.501961,0.501961,0.501961}%
\pgfsetfillcolor{currentfill}%
\pgfsetlinewidth{0.803000pt}%
\definecolor{currentstroke}{rgb}{0.501961,0.501961,0.501961}%
\pgfsetstrokecolor{currentstroke}%
\pgfsetdash{}{0pt}%
\pgfpathmoveto{\pgfqpoint{3.115609in}{1.693491in}}%
\pgfpathlineto{\pgfqpoint{3.084941in}{1.761426in}}%
\pgfpathlineto{\pgfqpoint{3.157689in}{1.745199in}}%
\pgfpathlineto{\pgfqpoint{3.115609in}{1.693491in}}%
\pgfpathclose%
\pgfusepath{stroke,fill}%
\end{pgfscope}%
\begin{pgfscope}%
\pgfpathrectangle{\pgfqpoint{0.647939in}{0.492442in}}{\pgfqpoint{3.079299in}{3.079299in}}%
\pgfusepath{clip}%
\pgfsetroundcap%
\pgfsetroundjoin%
\pgfsetlinewidth{0.803000pt}%
\definecolor{currentstroke}{rgb}{0.501961,0.501961,0.501961}%
\pgfsetstrokecolor{currentstroke}%
\pgfsetdash{}{0pt}%
\pgfpathmoveto{\pgfqpoint{2.987385in}{2.112076in}}%
\pgfpathquadraticcurveto{\pgfqpoint{2.985498in}{2.114688in}}{\pgfqpoint{2.990885in}{2.107231in}}%
\pgfusepath{stroke}%
\end{pgfscope}%
\begin{pgfscope}%
\pgfpathrectangle{\pgfqpoint{0.647939in}{0.492442in}}{\pgfqpoint{3.079299in}{3.079299in}}%
\pgfusepath{clip}%
\pgfsetroundcap%
\pgfsetroundjoin%
\definecolor{currentfill}{rgb}{0.501961,0.501961,0.501961}%
\pgfsetfillcolor{currentfill}%
\pgfsetlinewidth{0.803000pt}%
\definecolor{currentstroke}{rgb}{0.501961,0.501961,0.501961}%
\pgfsetstrokecolor{currentstroke}%
\pgfsetdash{}{0pt}%
\pgfpathmoveto{\pgfqpoint{3.002898in}{2.033670in}}%
\pgfpathlineto{\pgfqpoint{2.990885in}{2.107231in}}%
\pgfpathlineto{\pgfqpoint{3.056942in}{2.072705in}}%
\pgfpathlineto{\pgfqpoint{3.002898in}{2.033670in}}%
\pgfpathclose%
\pgfusepath{stroke,fill}%
\end{pgfscope}%
\begin{pgfscope}%
\pgfpathrectangle{\pgfqpoint{0.647939in}{0.492442in}}{\pgfqpoint{3.079299in}{3.079299in}}%
\pgfusepath{clip}%
\pgfsetroundcap%
\pgfsetroundjoin%
\pgfsetlinewidth{0.803000pt}%
\definecolor{currentstroke}{rgb}{0.501961,0.501961,0.501961}%
\pgfsetstrokecolor{currentstroke}%
\pgfsetdash{}{0pt}%
\pgfpathmoveto{\pgfqpoint{3.142482in}{2.051120in}}%
\pgfpathquadraticcurveto{\pgfqpoint{3.140531in}{2.053686in}}{\pgfqpoint{3.146099in}{2.046363in}}%
\pgfusepath{stroke}%
\end{pgfscope}%
\begin{pgfscope}%
\pgfpathrectangle{\pgfqpoint{0.647939in}{0.492442in}}{\pgfqpoint{3.079299in}{3.079299in}}%
\pgfusepath{clip}%
\pgfsetroundcap%
\pgfsetroundjoin%
\definecolor{currentfill}{rgb}{0.501961,0.501961,0.501961}%
\pgfsetfillcolor{currentfill}%
\pgfsetlinewidth{0.803000pt}%
\definecolor{currentstroke}{rgb}{0.501961,0.501961,0.501961}%
\pgfsetstrokecolor{currentstroke}%
\pgfsetdash{}{0pt}%
\pgfpathmoveto{\pgfqpoint{3.159920in}{1.973120in}}%
\pgfpathlineto{\pgfqpoint{3.146099in}{2.046363in}}%
\pgfpathlineto{\pgfqpoint{3.212986in}{2.013474in}}%
\pgfpathlineto{\pgfqpoint{3.159920in}{1.973120in}}%
\pgfpathclose%
\pgfusepath{stroke,fill}%
\end{pgfscope}%
\begin{pgfscope}%
\pgfpathrectangle{\pgfqpoint{0.647939in}{0.492442in}}{\pgfqpoint{3.079299in}{3.079299in}}%
\pgfusepath{clip}%
\pgfsetroundcap%
\pgfsetroundjoin%
\pgfsetlinewidth{0.803000pt}%
\definecolor{currentstroke}{rgb}{0.501961,0.501961,0.501961}%
\pgfsetstrokecolor{currentstroke}%
\pgfsetdash{}{0pt}%
\pgfpathmoveto{\pgfqpoint{3.126417in}{2.578340in}}%
\pgfpathquadraticcurveto{\pgfqpoint{3.126467in}{2.581526in}}{\pgfqpoint{3.126321in}{2.572291in}}%
\pgfusepath{stroke}%
\end{pgfscope}%
\begin{pgfscope}%
\pgfpathrectangle{\pgfqpoint{0.647939in}{0.492442in}}{\pgfqpoint{3.079299in}{3.079299in}}%
\pgfusepath{clip}%
\pgfsetroundcap%
\pgfsetroundjoin%
\definecolor{currentfill}{rgb}{0.501961,0.501961,0.501961}%
\pgfsetfillcolor{currentfill}%
\pgfsetlinewidth{0.803000pt}%
\definecolor{currentstroke}{rgb}{0.501961,0.501961,0.501961}%
\pgfsetstrokecolor{currentstroke}%
\pgfsetdash{}{0pt}%
\pgfpathmoveto{\pgfqpoint{3.091934in}{2.506161in}}%
\pgfpathlineto{\pgfqpoint{3.126321in}{2.572291in}}%
\pgfpathlineto{\pgfqpoint{3.158593in}{2.505104in}}%
\pgfpathlineto{\pgfqpoint{3.091934in}{2.506161in}}%
\pgfpathclose%
\pgfusepath{stroke,fill}%
\end{pgfscope}%
\begin{pgfscope}%
\pgfpathrectangle{\pgfqpoint{0.647939in}{0.492442in}}{\pgfqpoint{3.079299in}{3.079299in}}%
\pgfusepath{clip}%
\pgfsetroundcap%
\pgfsetroundjoin%
\pgfsetlinewidth{0.803000pt}%
\definecolor{currentstroke}{rgb}{0.501961,0.501961,0.501961}%
\pgfsetstrokecolor{currentstroke}%
\pgfsetdash{}{0pt}%
\pgfpathmoveto{\pgfqpoint{3.270705in}{2.651115in}}%
\pgfpathquadraticcurveto{\pgfqpoint{3.270893in}{2.654296in}}{\pgfqpoint{3.270347in}{2.645077in}}%
\pgfusepath{stroke}%
\end{pgfscope}%
\begin{pgfscope}%
\pgfpathrectangle{\pgfqpoint{0.647939in}{0.492442in}}{\pgfqpoint{3.079299in}{3.079299in}}%
\pgfusepath{clip}%
\pgfsetroundcap%
\pgfsetroundjoin%
\definecolor{currentfill}{rgb}{0.501961,0.501961,0.501961}%
\pgfsetfillcolor{currentfill}%
\pgfsetlinewidth{0.803000pt}%
\definecolor{currentstroke}{rgb}{0.501961,0.501961,0.501961}%
\pgfsetstrokecolor{currentstroke}%
\pgfsetdash{}{0pt}%
\pgfpathmoveto{\pgfqpoint{3.233126in}{2.580500in}}%
\pgfpathlineto{\pgfqpoint{3.270347in}{2.645077in}}%
\pgfpathlineto{\pgfqpoint{3.299676in}{2.576554in}}%
\pgfpathlineto{\pgfqpoint{3.233126in}{2.580500in}}%
\pgfpathclose%
\pgfusepath{stroke,fill}%
\end{pgfscope}%
\begin{pgfscope}%
\pgfpathrectangle{\pgfqpoint{0.647939in}{0.492442in}}{\pgfqpoint{3.079299in}{3.079299in}}%
\pgfusepath{clip}%
\pgfsetroundcap%
\pgfsetroundjoin%
\pgfsetlinewidth{0.803000pt}%
\definecolor{currentstroke}{rgb}{0.501961,0.501961,0.501961}%
\pgfsetstrokecolor{currentstroke}%
\pgfsetdash{}{0pt}%
\pgfpathmoveto{\pgfqpoint{3.518298in}{2.108016in}}%
\pgfpathquadraticcurveto{\pgfqpoint{3.516391in}{2.110615in}}{\pgfqpoint{3.521833in}{2.103197in}}%
\pgfusepath{stroke}%
\end{pgfscope}%
\begin{pgfscope}%
\pgfpathrectangle{\pgfqpoint{0.647939in}{0.492442in}}{\pgfqpoint{3.079299in}{3.079299in}}%
\pgfusepath{clip}%
\pgfsetroundcap%
\pgfsetroundjoin%
\definecolor{currentfill}{rgb}{0.501961,0.501961,0.501961}%
\pgfsetfillcolor{currentfill}%
\pgfsetlinewidth{0.803000pt}%
\definecolor{currentstroke}{rgb}{0.501961,0.501961,0.501961}%
\pgfsetstrokecolor{currentstroke}%
\pgfsetdash{}{0pt}%
\pgfpathmoveto{\pgfqpoint{3.534390in}{2.029726in}}%
\pgfpathlineto{\pgfqpoint{3.521833in}{2.103197in}}%
\pgfpathlineto{\pgfqpoint{3.588143in}{2.069159in}}%
\pgfpathlineto{\pgfqpoint{3.534390in}{2.029726in}}%
\pgfpathclose%
\pgfusepath{stroke,fill}%
\end{pgfscope}%
\begin{pgfscope}%
\pgfpathrectangle{\pgfqpoint{0.647939in}{0.492442in}}{\pgfqpoint{3.079299in}{3.079299in}}%
\pgfusepath{clip}%
\pgfsetroundcap%
\pgfsetroundjoin%
\pgfsetlinewidth{0.803000pt}%
\definecolor{currentstroke}{rgb}{0.501961,0.501961,0.501961}%
\pgfsetstrokecolor{currentstroke}%
\pgfsetdash{}{0pt}%
\pgfpathmoveto{\pgfqpoint{3.446530in}{2.663955in}}%
\pgfpathquadraticcurveto{\pgfqpoint{3.446608in}{2.667141in}}{\pgfqpoint{3.446382in}{2.657909in}}%
\pgfusepath{stroke}%
\end{pgfscope}%
\begin{pgfscope}%
\pgfpathrectangle{\pgfqpoint{0.647939in}{0.492442in}}{\pgfqpoint{3.079299in}{3.079299in}}%
\pgfusepath{clip}%
\pgfsetroundcap%
\pgfsetroundjoin%
\definecolor{currentfill}{rgb}{0.501961,0.501961,0.501961}%
\pgfsetfillcolor{currentfill}%
\pgfsetlinewidth{0.803000pt}%
\definecolor{currentstroke}{rgb}{0.501961,0.501961,0.501961}%
\pgfsetstrokecolor{currentstroke}%
\pgfsetdash{}{0pt}%
\pgfpathmoveto{\pgfqpoint{3.411433in}{2.592075in}}%
\pgfpathlineto{\pgfqpoint{3.446382in}{2.657909in}}%
\pgfpathlineto{\pgfqpoint{3.478080in}{2.590449in}}%
\pgfpathlineto{\pgfqpoint{3.411433in}{2.592075in}}%
\pgfpathclose%
\pgfusepath{stroke,fill}%
\end{pgfscope}%
\begin{pgfscope}%
\pgfpathrectangle{\pgfqpoint{0.647939in}{0.492442in}}{\pgfqpoint{3.079299in}{3.079299in}}%
\pgfusepath{clip}%
\pgfsetroundcap%
\pgfsetroundjoin%
\pgfsetlinewidth{0.803000pt}%
\definecolor{currentstroke}{rgb}{0.501961,0.501961,0.501961}%
\pgfsetstrokecolor{currentstroke}%
\pgfsetdash{}{0pt}%
\pgfpathmoveto{\pgfqpoint{3.544089in}{2.702902in}}%
\pgfpathquadraticcurveto{\pgfqpoint{3.544260in}{2.706085in}}{\pgfqpoint{3.543766in}{2.696862in}}%
\pgfusepath{stroke}%
\end{pgfscope}%
\begin{pgfscope}%
\pgfpathrectangle{\pgfqpoint{0.647939in}{0.492442in}}{\pgfqpoint{3.079299in}{3.079299in}}%
\pgfusepath{clip}%
\pgfsetroundcap%
\pgfsetroundjoin%
\definecolor{currentfill}{rgb}{0.501961,0.501961,0.501961}%
\pgfsetfillcolor{currentfill}%
\pgfsetlinewidth{0.803000pt}%
\definecolor{currentstroke}{rgb}{0.501961,0.501961,0.501961}%
\pgfsetstrokecolor{currentstroke}%
\pgfsetdash{}{0pt}%
\pgfpathmoveto{\pgfqpoint{3.506917in}{2.632073in}}%
\pgfpathlineto{\pgfqpoint{3.543766in}{2.696862in}}%
\pgfpathlineto{\pgfqpoint{3.573488in}{2.628509in}}%
\pgfpathlineto{\pgfqpoint{3.506917in}{2.632073in}}%
\pgfpathclose%
\pgfusepath{stroke,fill}%
\end{pgfscope}%
\begin{pgfscope}%
\pgfpathrectangle{\pgfqpoint{0.647939in}{0.492442in}}{\pgfqpoint{3.079299in}{3.079299in}}%
\pgfusepath{clip}%
\pgfsetroundcap%
\pgfsetroundjoin%
\pgfsetlinewidth{0.803000pt}%
\definecolor{currentstroke}{rgb}{0.501961,0.501961,0.501961}%
\pgfsetstrokecolor{currentstroke}%
\pgfsetdash{}{0pt}%
\pgfpathmoveto{\pgfqpoint{3.623561in}{2.662311in}}%
\pgfpathquadraticcurveto{\pgfqpoint{3.623499in}{2.665497in}}{\pgfqpoint{3.623679in}{2.656263in}}%
\pgfusepath{stroke}%
\end{pgfscope}%
\begin{pgfscope}%
\pgfpathrectangle{\pgfqpoint{0.647939in}{0.492442in}}{\pgfqpoint{3.079299in}{3.079299in}}%
\pgfusepath{clip}%
\pgfsetroundcap%
\pgfsetroundjoin%
\definecolor{currentfill}{rgb}{0.501961,0.501961,0.501961}%
\pgfsetfillcolor{currentfill}%
\pgfsetlinewidth{0.803000pt}%
\definecolor{currentstroke}{rgb}{0.501961,0.501961,0.501961}%
\pgfsetstrokecolor{currentstroke}%
\pgfsetdash{}{0pt}%
\pgfpathmoveto{\pgfqpoint{3.591655in}{2.588958in}}%
\pgfpathlineto{\pgfqpoint{3.623679in}{2.656263in}}%
\pgfpathlineto{\pgfqpoint{3.658309in}{2.590260in}}%
\pgfpathlineto{\pgfqpoint{3.591655in}{2.588958in}}%
\pgfpathclose%
\pgfusepath{stroke,fill}%
\end{pgfscope}%
\begin{pgfscope}%
\pgfpathrectangle{\pgfqpoint{0.647939in}{0.492442in}}{\pgfqpoint{3.079299in}{3.079299in}}%
\pgfusepath{clip}%
\pgfsetroundcap%
\pgfsetroundjoin%
\pgfsetlinewidth{0.803000pt}%
\definecolor{currentstroke}{rgb}{0.501961,0.501961,0.501961}%
\pgfsetstrokecolor{currentstroke}%
\pgfsetdash{}{0pt}%
\pgfpathmoveto{\pgfqpoint{3.704846in}{2.684384in}}%
\pgfpathquadraticcurveto{\pgfqpoint{3.704828in}{2.687569in}}{\pgfqpoint{3.704880in}{2.678333in}}%
\pgfusepath{stroke}%
\end{pgfscope}%
\begin{pgfscope}%
\pgfpathrectangle{\pgfqpoint{0.647939in}{0.492442in}}{\pgfqpoint{3.079299in}{3.079299in}}%
\pgfusepath{clip}%
\pgfsetroundcap%
\pgfsetroundjoin%
\definecolor{currentfill}{rgb}{0.501961,0.501961,0.501961}%
\pgfsetfillcolor{currentfill}%
\pgfsetlinewidth{0.803000pt}%
\definecolor{currentstroke}{rgb}{0.501961,0.501961,0.501961}%
\pgfsetstrokecolor{currentstroke}%
\pgfsetdash{}{0pt}%
\pgfpathmoveto{\pgfqpoint{3.671925in}{2.611478in}}%
\pgfpathlineto{\pgfqpoint{3.704880in}{2.678333in}}%
\pgfpathlineto{\pgfqpoint{3.738591in}{2.611856in}}%
\pgfpathlineto{\pgfqpoint{3.671925in}{2.611478in}}%
\pgfpathclose%
\pgfusepath{stroke,fill}%
\end{pgfscope}%
\begin{pgfscope}%
\pgfpathrectangle{\pgfqpoint{0.647939in}{0.492442in}}{\pgfqpoint{3.079299in}{3.079299in}}%
\pgfusepath{clip}%
\pgfsetroundcap%
\pgfsetroundjoin%
\pgfsetlinewidth{0.803000pt}%
\definecolor{currentstroke}{rgb}{0.501961,0.501961,0.501961}%
\pgfsetstrokecolor{currentstroke}%
\pgfsetdash{}{0pt}%
\pgfpathmoveto{\pgfqpoint{3.474068in}{3.407974in}}%
\pgfpathquadraticcurveto{\pgfqpoint{3.476340in}{3.410262in}}{\pgfqpoint{3.469859in}{3.403734in}}%
\pgfusepath{stroke}%
\end{pgfscope}%
\begin{pgfscope}%
\pgfpathrectangle{\pgfqpoint{0.647939in}{0.492442in}}{\pgfqpoint{3.079299in}{3.079299in}}%
\pgfusepath{clip}%
\pgfsetroundcap%
\pgfsetroundjoin%
\definecolor{currentfill}{rgb}{0.501961,0.501961,0.501961}%
\pgfsetfillcolor{currentfill}%
\pgfsetlinewidth{0.803000pt}%
\definecolor{currentstroke}{rgb}{0.501961,0.501961,0.501961}%
\pgfsetstrokecolor{currentstroke}%
\pgfsetdash{}{0pt}%
\pgfpathmoveto{\pgfqpoint{3.399233in}{3.379911in}}%
\pgfpathlineto{\pgfqpoint{3.469859in}{3.403734in}}%
\pgfpathlineto{\pgfqpoint{3.446542in}{3.332940in}}%
\pgfpathlineto{\pgfqpoint{3.399233in}{3.379911in}}%
\pgfpathclose%
\pgfusepath{stroke,fill}%
\end{pgfscope}%
\begin{pgfscope}%
\pgfpathrectangle{\pgfqpoint{0.647939in}{0.492442in}}{\pgfqpoint{3.079299in}{3.079299in}}%
\pgfusepath{clip}%
\pgfsetroundcap%
\pgfsetroundjoin%
\pgfsetlinewidth{0.803000pt}%
\definecolor{currentstroke}{rgb}{0.501961,0.501961,0.501961}%
\pgfsetstrokecolor{currentstroke}%
\pgfsetdash{}{0pt}%
\pgfpathmoveto{\pgfqpoint{2.164742in}{2.819618in}}%
\pgfpathquadraticcurveto{\pgfqpoint{2.167966in}{2.819626in}}{\pgfqpoint{2.158768in}{2.819604in}}%
\pgfusepath{stroke}%
\end{pgfscope}%
\begin{pgfscope}%
\pgfpathrectangle{\pgfqpoint{0.647939in}{0.492442in}}{\pgfqpoint{3.079299in}{3.079299in}}%
\pgfusepath{clip}%
\pgfsetroundcap%
\pgfsetroundjoin%
\definecolor{currentfill}{rgb}{0.501961,0.501961,0.501961}%
\pgfsetfillcolor{currentfill}%
\pgfsetlinewidth{0.803000pt}%
\definecolor{currentstroke}{rgb}{0.501961,0.501961,0.501961}%
\pgfsetstrokecolor{currentstroke}%
\pgfsetdash{}{0pt}%
\pgfpathmoveto{\pgfqpoint{2.092022in}{2.852779in}}%
\pgfpathlineto{\pgfqpoint{2.158768in}{2.819604in}}%
\pgfpathlineto{\pgfqpoint{2.092181in}{2.786112in}}%
\pgfpathlineto{\pgfqpoint{2.092022in}{2.852779in}}%
\pgfpathclose%
\pgfusepath{stroke,fill}%
\end{pgfscope}%
\begin{pgfscope}%
\pgfpathrectangle{\pgfqpoint{0.647939in}{0.492442in}}{\pgfqpoint{3.079299in}{3.079299in}}%
\pgfusepath{clip}%
\pgfsetroundcap%
\pgfsetroundjoin%
\pgfsetlinewidth{0.803000pt}%
\definecolor{currentstroke}{rgb}{0.501961,0.501961,0.501961}%
\pgfsetstrokecolor{currentstroke}%
\pgfsetdash{}{0pt}%
\pgfpathmoveto{\pgfqpoint{2.038597in}{3.085294in}}%
\pgfpathquadraticcurveto{\pgfqpoint{2.041820in}{3.085366in}}{\pgfqpoint{2.032624in}{3.085160in}}%
\pgfusepath{stroke}%
\end{pgfscope}%
\begin{pgfscope}%
\pgfpathrectangle{\pgfqpoint{0.647939in}{0.492442in}}{\pgfqpoint{3.079299in}{3.079299in}}%
\pgfusepath{clip}%
\pgfsetroundcap%
\pgfsetroundjoin%
\definecolor{currentfill}{rgb}{0.501961,0.501961,0.501961}%
\pgfsetfillcolor{currentfill}%
\pgfsetlinewidth{0.803000pt}%
\definecolor{currentstroke}{rgb}{0.501961,0.501961,0.501961}%
\pgfsetstrokecolor{currentstroke}%
\pgfsetdash{}{0pt}%
\pgfpathmoveto{\pgfqpoint{1.965225in}{3.116987in}}%
\pgfpathlineto{\pgfqpoint{2.032624in}{3.085160in}}%
\pgfpathlineto{\pgfqpoint{1.966723in}{3.050337in}}%
\pgfpathlineto{\pgfqpoint{1.965225in}{3.116987in}}%
\pgfpathclose%
\pgfusepath{stroke,fill}%
\end{pgfscope}%
\begin{pgfscope}%
\pgfpathrectangle{\pgfqpoint{0.647939in}{0.492442in}}{\pgfqpoint{3.079299in}{3.079299in}}%
\pgfusepath{clip}%
\pgfsetroundcap%
\pgfsetroundjoin%
\pgfsetlinewidth{0.803000pt}%
\definecolor{currentstroke}{rgb}{0.501961,0.501961,0.501961}%
\pgfsetstrokecolor{currentstroke}%
\pgfsetdash{}{0pt}%
\pgfpathmoveto{\pgfqpoint{1.902045in}{3.248257in}}%
\pgfpathquadraticcurveto{\pgfqpoint{1.905263in}{3.248459in}}{\pgfqpoint{1.896082in}{3.247883in}}%
\pgfusepath{stroke}%
\end{pgfscope}%
\begin{pgfscope}%
\pgfpathrectangle{\pgfqpoint{0.647939in}{0.492442in}}{\pgfqpoint{3.079299in}{3.079299in}}%
\pgfusepath{clip}%
\pgfsetroundcap%
\pgfsetroundjoin%
\definecolor{currentfill}{rgb}{0.501961,0.501961,0.501961}%
\pgfsetfillcolor{currentfill}%
\pgfsetlinewidth{0.803000pt}%
\definecolor{currentstroke}{rgb}{0.501961,0.501961,0.501961}%
\pgfsetstrokecolor{currentstroke}%
\pgfsetdash{}{0pt}%
\pgfpathmoveto{\pgfqpoint{1.827458in}{3.276974in}}%
\pgfpathlineto{\pgfqpoint{1.896082in}{3.247883in}}%
\pgfpathlineto{\pgfqpoint{1.831635in}{3.210438in}}%
\pgfpathlineto{\pgfqpoint{1.827458in}{3.276974in}}%
\pgfpathclose%
\pgfusepath{stroke,fill}%
\end{pgfscope}%
\begin{pgfscope}%
\pgfpathrectangle{\pgfqpoint{0.647939in}{0.492442in}}{\pgfqpoint{3.079299in}{3.079299in}}%
\pgfusepath{clip}%
\pgfsetroundcap%
\pgfsetroundjoin%
\pgfsetlinewidth{0.803000pt}%
\definecolor{currentstroke}{rgb}{0.501961,0.501961,0.501961}%
\pgfsetstrokecolor{currentstroke}%
\pgfsetdash{}{0pt}%
\pgfpathmoveto{\pgfqpoint{1.860711in}{3.365017in}}%
\pgfpathquadraticcurveto{\pgfqpoint{1.863927in}{3.365251in}}{\pgfqpoint{1.854753in}{3.364582in}}%
\pgfusepath{stroke}%
\end{pgfscope}%
\begin{pgfscope}%
\pgfpathrectangle{\pgfqpoint{0.647939in}{0.492442in}}{\pgfqpoint{3.079299in}{3.079299in}}%
\pgfusepath{clip}%
\pgfsetroundcap%
\pgfsetroundjoin%
\definecolor{currentfill}{rgb}{0.501961,0.501961,0.501961}%
\pgfsetfillcolor{currentfill}%
\pgfsetlinewidth{0.803000pt}%
\definecolor{currentstroke}{rgb}{0.501961,0.501961,0.501961}%
\pgfsetstrokecolor{currentstroke}%
\pgfsetdash{}{0pt}%
\pgfpathmoveto{\pgfqpoint{1.785837in}{3.392977in}}%
\pgfpathlineto{\pgfqpoint{1.854753in}{3.364582in}}%
\pgfpathlineto{\pgfqpoint{1.790688in}{3.326487in}}%
\pgfpathlineto{\pgfqpoint{1.785837in}{3.392977in}}%
\pgfpathclose%
\pgfusepath{stroke,fill}%
\end{pgfscope}%
\begin{pgfscope}%
\pgfpathrectangle{\pgfqpoint{0.647939in}{0.492442in}}{\pgfqpoint{3.079299in}{3.079299in}}%
\pgfusepath{clip}%
\pgfsetroundcap%
\pgfsetroundjoin%
\pgfsetlinewidth{0.803000pt}%
\definecolor{currentstroke}{rgb}{0.501961,0.501961,0.501961}%
\pgfsetstrokecolor{currentstroke}%
\pgfsetdash{}{0pt}%
\pgfpathmoveto{\pgfqpoint{1.768175in}{3.441023in}}%
\pgfpathquadraticcurveto{\pgfqpoint{1.771383in}{3.441353in}}{\pgfqpoint{1.762232in}{3.440412in}}%
\pgfusepath{stroke}%
\end{pgfscope}%
\begin{pgfscope}%
\pgfpathrectangle{\pgfqpoint{0.647939in}{0.492442in}}{\pgfqpoint{3.079299in}{3.079299in}}%
\pgfusepath{clip}%
\pgfsetroundcap%
\pgfsetroundjoin%
\definecolor{currentfill}{rgb}{0.501961,0.501961,0.501961}%
\pgfsetfillcolor{currentfill}%
\pgfsetlinewidth{0.803000pt}%
\definecolor{currentstroke}{rgb}{0.501961,0.501961,0.501961}%
\pgfsetstrokecolor{currentstroke}%
\pgfsetdash{}{0pt}%
\pgfpathmoveto{\pgfqpoint{1.692507in}{3.466755in}}%
\pgfpathlineto{\pgfqpoint{1.762232in}{3.440412in}}%
\pgfpathlineto{\pgfqpoint{1.699323in}{3.400437in}}%
\pgfpathlineto{\pgfqpoint{1.692507in}{3.466755in}}%
\pgfpathclose%
\pgfusepath{stroke,fill}%
\end{pgfscope}%
\begin{pgfscope}%
\pgfpathrectangle{\pgfqpoint{0.647939in}{0.492442in}}{\pgfqpoint{3.079299in}{3.079299in}}%
\pgfusepath{clip}%
\pgfsetroundcap%
\pgfsetroundjoin%
\pgfsetlinewidth{0.803000pt}%
\definecolor{currentstroke}{rgb}{0.501961,0.501961,0.501961}%
\pgfsetstrokecolor{currentstroke}%
\pgfsetdash{}{0pt}%
\pgfpathmoveto{\pgfqpoint{1.682690in}{3.506904in}}%
\pgfpathquadraticcurveto{\pgfqpoint{1.685888in}{3.507314in}}{\pgfqpoint{1.676765in}{3.506142in}}%
\pgfusepath{stroke}%
\end{pgfscope}%
\begin{pgfscope}%
\pgfpathrectangle{\pgfqpoint{0.647939in}{0.492442in}}{\pgfqpoint{3.079299in}{3.079299in}}%
\pgfusepath{clip}%
\pgfsetroundcap%
\pgfsetroundjoin%
\definecolor{currentfill}{rgb}{0.501961,0.501961,0.501961}%
\pgfsetfillcolor{currentfill}%
\pgfsetlinewidth{0.803000pt}%
\definecolor{currentstroke}{rgb}{0.501961,0.501961,0.501961}%
\pgfsetstrokecolor{currentstroke}%
\pgfsetdash{}{0pt}%
\pgfpathmoveto{\pgfqpoint{1.606394in}{3.530709in}}%
\pgfpathlineto{\pgfqpoint{1.676765in}{3.506142in}}%
\pgfpathlineto{\pgfqpoint{1.614889in}{3.464586in}}%
\pgfpathlineto{\pgfqpoint{1.606394in}{3.530709in}}%
\pgfpathclose%
\pgfusepath{stroke,fill}%
\end{pgfscope}%
\begin{pgfscope}%
\pgfpathrectangle{\pgfqpoint{0.647939in}{0.492442in}}{\pgfqpoint{3.079299in}{3.079299in}}%
\pgfusepath{clip}%
\pgfsetroundcap%
\pgfsetroundjoin%
\pgfsetlinewidth{0.803000pt}%
\definecolor{currentstroke}{rgb}{0.501961,0.501961,0.501961}%
\pgfsetstrokecolor{currentstroke}%
\pgfsetdash{}{0pt}%
\pgfpathmoveto{\pgfqpoint{2.014790in}{3.567167in}}%
\pgfpathquadraticcurveto{\pgfqpoint{2.018013in}{3.567235in}}{\pgfqpoint{2.008817in}{3.567040in}}%
\pgfusepath{stroke}%
\end{pgfscope}%
\begin{pgfscope}%
\pgfpathrectangle{\pgfqpoint{0.647939in}{0.492442in}}{\pgfqpoint{3.079299in}{3.079299in}}%
\pgfusepath{clip}%
\pgfsetroundcap%
\pgfsetroundjoin%
\definecolor{currentfill}{rgb}{0.501961,0.501961,0.501961}%
\pgfsetfillcolor{currentfill}%
\pgfsetlinewidth{0.803000pt}%
\definecolor{currentstroke}{rgb}{0.501961,0.501961,0.501961}%
\pgfsetstrokecolor{currentstroke}%
\pgfsetdash{}{0pt}%
\pgfpathmoveto{\pgfqpoint{1.941457in}{3.598950in}}%
\pgfpathlineto{\pgfqpoint{2.008817in}{3.567040in}}%
\pgfpathlineto{\pgfqpoint{1.942873in}{3.532298in}}%
\pgfpathlineto{\pgfqpoint{1.941457in}{3.598950in}}%
\pgfpathclose%
\pgfusepath{stroke,fill}%
\end{pgfscope}%
\begin{pgfscope}%
\pgfpathrectangle{\pgfqpoint{0.647939in}{0.492442in}}{\pgfqpoint{3.079299in}{3.079299in}}%
\pgfusepath{clip}%
\pgfsetroundcap%
\pgfsetroundjoin%
\pgfsetlinewidth{0.803000pt}%
\definecolor{currentstroke}{rgb}{0.501961,0.501961,0.501961}%
\pgfsetstrokecolor{currentstroke}%
\pgfsetdash{}{0pt}%
\pgfpathmoveto{\pgfqpoint{1.114754in}{3.485975in}}%
\pgfpathquadraticcurveto{\pgfqpoint{1.117917in}{3.486599in}}{\pgfqpoint{1.108893in}{3.484819in}}%
\pgfusepath{stroke}%
\end{pgfscope}%
\begin{pgfscope}%
\pgfpathrectangle{\pgfqpoint{0.647939in}{0.492442in}}{\pgfqpoint{3.079299in}{3.079299in}}%
\pgfusepath{clip}%
\pgfsetroundcap%
\pgfsetroundjoin%
\definecolor{currentfill}{rgb}{0.501961,0.501961,0.501961}%
\pgfsetfillcolor{currentfill}%
\pgfsetlinewidth{0.803000pt}%
\definecolor{currentstroke}{rgb}{0.501961,0.501961,0.501961}%
\pgfsetstrokecolor{currentstroke}%
\pgfsetdash{}{0pt}%
\pgfpathmoveto{\pgfqpoint{1.037036in}{3.504622in}}%
\pgfpathlineto{\pgfqpoint{1.108893in}{3.484819in}}%
\pgfpathlineto{\pgfqpoint{1.049936in}{3.439216in}}%
\pgfpathlineto{\pgfqpoint{1.037036in}{3.504622in}}%
\pgfpathclose%
\pgfusepath{stroke,fill}%
\end{pgfscope}%
\begin{pgfscope}%
\pgfpathrectangle{\pgfqpoint{0.647939in}{0.492442in}}{\pgfqpoint{3.079299in}{3.079299in}}%
\pgfusepath{clip}%
\pgfsetroundcap%
\pgfsetroundjoin%
\pgfsetlinewidth{0.803000pt}%
\definecolor{currentstroke}{rgb}{0.501961,0.501961,0.501961}%
\pgfsetstrokecolor{currentstroke}%
\pgfsetdash{}{0pt}%
\pgfpathmoveto{\pgfqpoint{0.895695in}{3.528692in}}%
\pgfpathquadraticcurveto{\pgfqpoint{0.898879in}{3.529195in}}{\pgfqpoint{0.889794in}{3.527759in}}%
\pgfusepath{stroke}%
\end{pgfscope}%
\begin{pgfscope}%
\pgfpathrectangle{\pgfqpoint{0.647939in}{0.492442in}}{\pgfqpoint{3.079299in}{3.079299in}}%
\pgfusepath{clip}%
\pgfsetroundcap%
\pgfsetroundjoin%
\definecolor{currentfill}{rgb}{0.501961,0.501961,0.501961}%
\pgfsetfillcolor{currentfill}%
\pgfsetlinewidth{0.803000pt}%
\definecolor{currentstroke}{rgb}{0.501961,0.501961,0.501961}%
\pgfsetstrokecolor{currentstroke}%
\pgfsetdash{}{0pt}%
\pgfpathmoveto{\pgfqpoint{0.818741in}{3.550277in}}%
\pgfpathlineto{\pgfqpoint{0.889794in}{3.527759in}}%
\pgfpathlineto{\pgfqpoint{0.829148in}{3.484428in}}%
\pgfpathlineto{\pgfqpoint{0.818741in}{3.550277in}}%
\pgfpathclose%
\pgfusepath{stroke,fill}%
\end{pgfscope}%
\begin{pgfscope}%
\pgfpathrectangle{\pgfqpoint{0.647939in}{0.492442in}}{\pgfqpoint{3.079299in}{3.079299in}}%
\pgfusepath{clip}%
\pgfsetroundcap%
\pgfsetroundjoin%
\pgfsetlinewidth{0.803000pt}%
\definecolor{currentstroke}{rgb}{0.501961,0.501961,0.501961}%
\pgfsetstrokecolor{currentstroke}%
\pgfsetdash{}{0pt}%
\pgfpathmoveto{\pgfqpoint{1.749491in}{3.154272in}}%
\pgfpathquadraticcurveto{\pgfqpoint{1.752689in}{3.154681in}}{\pgfqpoint{1.743565in}{3.153515in}}%
\pgfusepath{stroke}%
\end{pgfscope}%
\begin{pgfscope}%
\pgfpathrectangle{\pgfqpoint{0.647939in}{0.492442in}}{\pgfqpoint{3.079299in}{3.079299in}}%
\pgfusepath{clip}%
\pgfsetroundcap%
\pgfsetroundjoin%
\definecolor{currentfill}{rgb}{0.501961,0.501961,0.501961}%
\pgfsetfillcolor{currentfill}%
\pgfsetlinewidth{0.803000pt}%
\definecolor{currentstroke}{rgb}{0.501961,0.501961,0.501961}%
\pgfsetstrokecolor{currentstroke}%
\pgfsetdash{}{0pt}%
\pgfpathmoveto{\pgfqpoint{1.673210in}{3.178128in}}%
\pgfpathlineto{\pgfqpoint{1.743565in}{3.153515in}}%
\pgfpathlineto{\pgfqpoint{1.681662in}{3.111999in}}%
\pgfpathlineto{\pgfqpoint{1.673210in}{3.178128in}}%
\pgfpathclose%
\pgfusepath{stroke,fill}%
\end{pgfscope}%
\begin{pgfscope}%
\pgfpathrectangle{\pgfqpoint{0.647939in}{0.492442in}}{\pgfqpoint{3.079299in}{3.079299in}}%
\pgfusepath{clip}%
\pgfsetroundcap%
\pgfsetroundjoin%
\pgfsetlinewidth{0.803000pt}%
\definecolor{currentstroke}{rgb}{0.501961,0.501961,0.501961}%
\pgfsetstrokecolor{currentstroke}%
\pgfsetdash{}{0pt}%
\pgfpathmoveto{\pgfqpoint{1.882022in}{2.976208in}}%
\pgfpathquadraticcurveto{\pgfqpoint{1.885235in}{2.976480in}}{\pgfqpoint{1.876069in}{2.975705in}}%
\pgfusepath{stroke}%
\end{pgfscope}%
\begin{pgfscope}%
\pgfpathrectangle{\pgfqpoint{0.647939in}{0.492442in}}{\pgfqpoint{3.079299in}{3.079299in}}%
\pgfusepath{clip}%
\pgfsetroundcap%
\pgfsetroundjoin%
\definecolor{currentfill}{rgb}{0.501961,0.501961,0.501961}%
\pgfsetfillcolor{currentfill}%
\pgfsetlinewidth{0.803000pt}%
\definecolor{currentstroke}{rgb}{0.501961,0.501961,0.501961}%
\pgfsetstrokecolor{currentstroke}%
\pgfsetdash{}{0pt}%
\pgfpathmoveto{\pgfqpoint{1.806831in}{3.003302in}}%
\pgfpathlineto{\pgfqpoint{1.876069in}{2.975705in}}%
\pgfpathlineto{\pgfqpoint{1.812448in}{2.936872in}}%
\pgfpathlineto{\pgfqpoint{1.806831in}{3.003302in}}%
\pgfpathclose%
\pgfusepath{stroke,fill}%
\end{pgfscope}%
\begin{pgfscope}%
\pgfpathrectangle{\pgfqpoint{0.647939in}{0.492442in}}{\pgfqpoint{3.079299in}{3.079299in}}%
\pgfusepath{clip}%
\pgfsetroundcap%
\pgfsetroundjoin%
\pgfsetlinewidth{0.803000pt}%
\definecolor{currentstroke}{rgb}{0.501961,0.501961,0.501961}%
\pgfsetstrokecolor{currentstroke}%
\pgfsetdash{}{0pt}%
\pgfpathmoveto{\pgfqpoint{1.211961in}{2.772036in}}%
\pgfpathquadraticcurveto{\pgfqpoint{1.215082in}{2.772845in}}{\pgfqpoint{1.206178in}{2.770537in}}%
\pgfusepath{stroke}%
\end{pgfscope}%
\begin{pgfscope}%
\pgfpathrectangle{\pgfqpoint{0.647939in}{0.492442in}}{\pgfqpoint{3.079299in}{3.079299in}}%
\pgfusepath{clip}%
\pgfsetroundcap%
\pgfsetroundjoin%
\definecolor{currentfill}{rgb}{0.501961,0.501961,0.501961}%
\pgfsetfillcolor{currentfill}%
\pgfsetlinewidth{0.803000pt}%
\definecolor{currentstroke}{rgb}{0.501961,0.501961,0.501961}%
\pgfsetstrokecolor{currentstroke}%
\pgfsetdash{}{0pt}%
\pgfpathmoveto{\pgfqpoint{1.133280in}{2.786074in}}%
\pgfpathlineto{\pgfqpoint{1.206178in}{2.770537in}}%
\pgfpathlineto{\pgfqpoint{1.150010in}{2.721540in}}%
\pgfpathlineto{\pgfqpoint{1.133280in}{2.786074in}}%
\pgfpathclose%
\pgfusepath{stroke,fill}%
\end{pgfscope}%
\begin{pgfscope}%
\pgfpathrectangle{\pgfqpoint{0.647939in}{0.492442in}}{\pgfqpoint{3.079299in}{3.079299in}}%
\pgfusepath{clip}%
\pgfsetroundcap%
\pgfsetroundjoin%
\pgfsetlinewidth{0.803000pt}%
\definecolor{currentstroke}{rgb}{0.501961,0.501961,0.501961}%
\pgfsetstrokecolor{currentstroke}%
\pgfsetdash{}{0pt}%
\pgfpathmoveto{\pgfqpoint{1.606375in}{2.677361in}}%
\pgfpathquadraticcurveto{\pgfqpoint{1.609513in}{2.678101in}}{\pgfqpoint{1.600560in}{2.675990in}}%
\pgfusepath{stroke}%
\end{pgfscope}%
\begin{pgfscope}%
\pgfpathrectangle{\pgfqpoint{0.647939in}{0.492442in}}{\pgfqpoint{3.079299in}{3.079299in}}%
\pgfusepath{clip}%
\pgfsetroundcap%
\pgfsetroundjoin%
\definecolor{currentfill}{rgb}{0.501961,0.501961,0.501961}%
\pgfsetfillcolor{currentfill}%
\pgfsetlinewidth{0.803000pt}%
\definecolor{currentstroke}{rgb}{0.501961,0.501961,0.501961}%
\pgfsetstrokecolor{currentstroke}%
\pgfsetdash{}{0pt}%
\pgfpathmoveto{\pgfqpoint{1.528022in}{2.693129in}}%
\pgfpathlineto{\pgfqpoint{1.600560in}{2.675990in}}%
\pgfpathlineto{\pgfqpoint{1.543326in}{2.628242in}}%
\pgfpathlineto{\pgfqpoint{1.528022in}{2.693129in}}%
\pgfpathclose%
\pgfusepath{stroke,fill}%
\end{pgfscope}%
\begin{pgfscope}%
\pgfpathrectangle{\pgfqpoint{0.647939in}{0.492442in}}{\pgfqpoint{3.079299in}{3.079299in}}%
\pgfusepath{clip}%
\pgfsetroundcap%
\pgfsetroundjoin%
\pgfsetlinewidth{0.803000pt}%
\definecolor{currentstroke}{rgb}{0.501961,0.501961,0.501961}%
\pgfsetstrokecolor{currentstroke}%
\pgfsetdash{}{0pt}%
\pgfpathmoveto{\pgfqpoint{1.472697in}{2.577730in}}%
\pgfpathquadraticcurveto{\pgfqpoint{1.475798in}{2.578611in}}{\pgfqpoint{1.466950in}{2.576096in}}%
\pgfusepath{stroke}%
\end{pgfscope}%
\begin{pgfscope}%
\pgfpathrectangle{\pgfqpoint{0.647939in}{0.492442in}}{\pgfqpoint{3.079299in}{3.079299in}}%
\pgfusepath{clip}%
\pgfsetroundcap%
\pgfsetroundjoin%
\definecolor{currentfill}{rgb}{0.501961,0.501961,0.501961}%
\pgfsetfillcolor{currentfill}%
\pgfsetlinewidth{0.803000pt}%
\definecolor{currentstroke}{rgb}{0.501961,0.501961,0.501961}%
\pgfsetstrokecolor{currentstroke}%
\pgfsetdash{}{0pt}%
\pgfpathmoveto{\pgfqpoint{1.393710in}{2.589933in}}%
\pgfpathlineto{\pgfqpoint{1.466950in}{2.576096in}}%
\pgfpathlineto{\pgfqpoint{1.411937in}{2.525806in}}%
\pgfpathlineto{\pgfqpoint{1.393710in}{2.589933in}}%
\pgfpathclose%
\pgfusepath{stroke,fill}%
\end{pgfscope}%
\begin{pgfscope}%
\pgfpathrectangle{\pgfqpoint{0.647939in}{0.492442in}}{\pgfqpoint{3.079299in}{3.079299in}}%
\pgfusepath{clip}%
\pgfsetroundcap%
\pgfsetroundjoin%
\pgfsetlinewidth{0.803000pt}%
\definecolor{currentstroke}{rgb}{0.501961,0.501961,0.501961}%
\pgfsetstrokecolor{currentstroke}%
\pgfsetdash{}{0pt}%
\pgfpathmoveto{\pgfqpoint{1.143750in}{2.414101in}}%
\pgfpathquadraticcurveto{\pgfqpoint{1.146852in}{2.414981in}}{\pgfqpoint{1.138003in}{2.412471in}}%
\pgfusepath{stroke}%
\end{pgfscope}%
\begin{pgfscope}%
\pgfpathrectangle{\pgfqpoint{0.647939in}{0.492442in}}{\pgfqpoint{3.079299in}{3.079299in}}%
\pgfusepath{clip}%
\pgfsetroundcap%
\pgfsetroundjoin%
\definecolor{currentfill}{rgb}{0.501961,0.501961,0.501961}%
\pgfsetfillcolor{currentfill}%
\pgfsetlinewidth{0.803000pt}%
\definecolor{currentstroke}{rgb}{0.501961,0.501961,0.501961}%
\pgfsetstrokecolor{currentstroke}%
\pgfsetdash{}{0pt}%
\pgfpathmoveto{\pgfqpoint{1.064771in}{2.426351in}}%
\pgfpathlineto{\pgfqpoint{1.138003in}{2.412471in}}%
\pgfpathlineto{\pgfqpoint{1.082960in}{2.362213in}}%
\pgfpathlineto{\pgfqpoint{1.064771in}{2.426351in}}%
\pgfpathclose%
\pgfusepath{stroke,fill}%
\end{pgfscope}%
\begin{pgfscope}%
\pgfpathrectangle{\pgfqpoint{0.647939in}{0.492442in}}{\pgfqpoint{3.079299in}{3.079299in}}%
\pgfusepath{clip}%
\pgfsetroundcap%
\pgfsetroundjoin%
\pgfsetlinewidth{0.803000pt}%
\definecolor{currentstroke}{rgb}{0.501961,0.501961,0.501961}%
\pgfsetstrokecolor{currentstroke}%
\pgfsetdash{}{0pt}%
\pgfpathmoveto{\pgfqpoint{1.011228in}{2.310281in}}%
\pgfpathquadraticcurveto{\pgfqpoint{1.014353in}{2.311072in}}{\pgfqpoint{1.005436in}{2.308814in}}%
\pgfusepath{stroke}%
\end{pgfscope}%
\begin{pgfscope}%
\pgfpathrectangle{\pgfqpoint{0.647939in}{0.492442in}}{\pgfqpoint{3.079299in}{3.079299in}}%
\pgfusepath{clip}%
\pgfsetroundcap%
\pgfsetroundjoin%
\definecolor{currentfill}{rgb}{0.501961,0.501961,0.501961}%
\pgfsetfillcolor{currentfill}%
\pgfsetlinewidth{0.803000pt}%
\definecolor{currentstroke}{rgb}{0.501961,0.501961,0.501961}%
\pgfsetstrokecolor{currentstroke}%
\pgfsetdash{}{0pt}%
\pgfpathmoveto{\pgfqpoint{0.932628in}{2.324766in}}%
\pgfpathlineto{\pgfqpoint{1.005436in}{2.308814in}}%
\pgfpathlineto{\pgfqpoint{0.948990in}{2.260138in}}%
\pgfpathlineto{\pgfqpoint{0.932628in}{2.324766in}}%
\pgfpathclose%
\pgfusepath{stroke,fill}%
\end{pgfscope}%
\begin{pgfscope}%
\pgfpathrectangle{\pgfqpoint{0.647939in}{0.492442in}}{\pgfqpoint{3.079299in}{3.079299in}}%
\pgfusepath{clip}%
\pgfsetroundcap%
\pgfsetroundjoin%
\pgfsetlinewidth{0.803000pt}%
\definecolor{currentstroke}{rgb}{0.501961,0.501961,0.501961}%
\pgfsetstrokecolor{currentstroke}%
\pgfsetdash{}{0pt}%
\pgfpathmoveto{\pgfqpoint{1.010982in}{2.241509in}}%
\pgfpathquadraticcurveto{\pgfqpoint{1.014103in}{2.242317in}}{\pgfqpoint{1.005198in}{2.240011in}}%
\pgfusepath{stroke}%
\end{pgfscope}%
\begin{pgfscope}%
\pgfpathrectangle{\pgfqpoint{0.647939in}{0.492442in}}{\pgfqpoint{3.079299in}{3.079299in}}%
\pgfusepath{clip}%
\pgfsetroundcap%
\pgfsetroundjoin%
\definecolor{currentfill}{rgb}{0.501961,0.501961,0.501961}%
\pgfsetfillcolor{currentfill}%
\pgfsetlinewidth{0.803000pt}%
\definecolor{currentstroke}{rgb}{0.501961,0.501961,0.501961}%
\pgfsetstrokecolor{currentstroke}%
\pgfsetdash{}{0pt}%
\pgfpathmoveto{\pgfqpoint{0.932304in}{2.255565in}}%
\pgfpathlineto{\pgfqpoint{1.005198in}{2.240011in}}%
\pgfpathlineto{\pgfqpoint{0.949018in}{2.191028in}}%
\pgfpathlineto{\pgfqpoint{0.932304in}{2.255565in}}%
\pgfpathclose%
\pgfusepath{stroke,fill}%
\end{pgfscope}%
\begin{pgfscope}%
\pgfpathrectangle{\pgfqpoint{0.647939in}{0.492442in}}{\pgfqpoint{3.079299in}{3.079299in}}%
\pgfusepath{clip}%
\pgfsetroundcap%
\pgfsetroundjoin%
\pgfsetlinewidth{0.803000pt}%
\definecolor{currentstroke}{rgb}{0.501961,0.501961,0.501961}%
\pgfsetstrokecolor{currentstroke}%
\pgfsetdash{}{0pt}%
\pgfpathmoveto{\pgfqpoint{1.597255in}{2.360072in}}%
\pgfpathquadraticcurveto{\pgfqpoint{1.600342in}{2.361003in}}{\pgfqpoint{1.591536in}{2.358347in}}%
\pgfusepath{stroke}%
\end{pgfscope}%
\begin{pgfscope}%
\pgfpathrectangle{\pgfqpoint{0.647939in}{0.492442in}}{\pgfqpoint{3.079299in}{3.079299in}}%
\pgfusepath{clip}%
\pgfsetroundcap%
\pgfsetroundjoin%
\definecolor{currentfill}{rgb}{0.501961,0.501961,0.501961}%
\pgfsetfillcolor{currentfill}%
\pgfsetlinewidth{0.803000pt}%
\definecolor{currentstroke}{rgb}{0.501961,0.501961,0.501961}%
\pgfsetstrokecolor{currentstroke}%
\pgfsetdash{}{0pt}%
\pgfpathmoveto{\pgfqpoint{1.518084in}{2.371011in}}%
\pgfpathlineto{\pgfqpoint{1.591536in}{2.358347in}}%
\pgfpathlineto{\pgfqpoint{1.537334in}{2.307184in}}%
\pgfpathlineto{\pgfqpoint{1.518084in}{2.371011in}}%
\pgfpathclose%
\pgfusepath{stroke,fill}%
\end{pgfscope}%
\begin{pgfscope}%
\pgfpathrectangle{\pgfqpoint{0.647939in}{0.492442in}}{\pgfqpoint{3.079299in}{3.079299in}}%
\pgfusepath{clip}%
\pgfsetroundcap%
\pgfsetroundjoin%
\pgfsetlinewidth{0.803000pt}%
\definecolor{currentstroke}{rgb}{0.501961,0.501961,0.501961}%
\pgfsetstrokecolor{currentstroke}%
\pgfsetdash{}{0pt}%
\pgfpathmoveto{\pgfqpoint{1.465256in}{2.253884in}}%
\pgfpathquadraticcurveto{\pgfqpoint{1.468298in}{2.254952in}}{\pgfqpoint{1.459619in}{2.251906in}}%
\pgfusepath{stroke}%
\end{pgfscope}%
\begin{pgfscope}%
\pgfpathrectangle{\pgfqpoint{0.647939in}{0.492442in}}{\pgfqpoint{3.079299in}{3.079299in}}%
\pgfusepath{clip}%
\pgfsetroundcap%
\pgfsetroundjoin%
\definecolor{currentfill}{rgb}{0.501961,0.501961,0.501961}%
\pgfsetfillcolor{currentfill}%
\pgfsetlinewidth{0.803000pt}%
\definecolor{currentstroke}{rgb}{0.501961,0.501961,0.501961}%
\pgfsetstrokecolor{currentstroke}%
\pgfsetdash{}{0pt}%
\pgfpathmoveto{\pgfqpoint{1.385675in}{2.261283in}}%
\pgfpathlineto{\pgfqpoint{1.459619in}{2.251906in}}%
\pgfpathlineto{\pgfqpoint{1.407751in}{2.198378in}}%
\pgfpathlineto{\pgfqpoint{1.385675in}{2.261283in}}%
\pgfpathclose%
\pgfusepath{stroke,fill}%
\end{pgfscope}%
\begin{pgfscope}%
\pgfpathrectangle{\pgfqpoint{0.647939in}{0.492442in}}{\pgfqpoint{3.079299in}{3.079299in}}%
\pgfusepath{clip}%
\pgfsetroundcap%
\pgfsetroundjoin%
\pgfsetlinewidth{0.803000pt}%
\definecolor{currentstroke}{rgb}{0.501961,0.501961,0.501961}%
\pgfsetstrokecolor{currentstroke}%
\pgfsetdash{}{0pt}%
\pgfpathmoveto{\pgfqpoint{1.206118in}{2.096449in}}%
\pgfpathquadraticcurveto{\pgfqpoint{1.209166in}{2.097500in}}{\pgfqpoint{1.200470in}{2.094501in}}%
\pgfusepath{stroke}%
\end{pgfscope}%
\begin{pgfscope}%
\pgfpathrectangle{\pgfqpoint{0.647939in}{0.492442in}}{\pgfqpoint{3.079299in}{3.079299in}}%
\pgfusepath{clip}%
\pgfsetroundcap%
\pgfsetroundjoin%
\definecolor{currentfill}{rgb}{0.501961,0.501961,0.501961}%
\pgfsetfillcolor{currentfill}%
\pgfsetlinewidth{0.803000pt}%
\definecolor{currentstroke}{rgb}{0.501961,0.501961,0.501961}%
\pgfsetstrokecolor{currentstroke}%
\pgfsetdash{}{0pt}%
\pgfpathmoveto{\pgfqpoint{1.126579in}{2.104286in}}%
\pgfpathlineto{\pgfqpoint{1.200470in}{2.094501in}}%
\pgfpathlineto{\pgfqpoint{1.148308in}{2.041259in}}%
\pgfpathlineto{\pgfqpoint{1.126579in}{2.104286in}}%
\pgfpathclose%
\pgfusepath{stroke,fill}%
\end{pgfscope}%
\begin{pgfscope}%
\pgfpathrectangle{\pgfqpoint{0.647939in}{0.492442in}}{\pgfqpoint{3.079299in}{3.079299in}}%
\pgfusepath{clip}%
\pgfsetroundcap%
\pgfsetroundjoin%
\pgfsetlinewidth{0.803000pt}%
\definecolor{currentstroke}{rgb}{0.501961,0.501961,0.501961}%
\pgfsetstrokecolor{currentstroke}%
\pgfsetdash{}{0pt}%
\pgfpathmoveto{\pgfqpoint{0.810301in}{1.919046in}}%
\pgfpathquadraticcurveto{\pgfqpoint{0.813462in}{1.919677in}}{\pgfqpoint{0.804442in}{1.917875in}}%
\pgfusepath{stroke}%
\end{pgfscope}%
\begin{pgfscope}%
\pgfpathrectangle{\pgfqpoint{0.647939in}{0.492442in}}{\pgfqpoint{3.079299in}{3.079299in}}%
\pgfusepath{clip}%
\pgfsetroundcap%
\pgfsetroundjoin%
\definecolor{currentfill}{rgb}{0.501961,0.501961,0.501961}%
\pgfsetfillcolor{currentfill}%
\pgfsetlinewidth{0.803000pt}%
\definecolor{currentstroke}{rgb}{0.501961,0.501961,0.501961}%
\pgfsetstrokecolor{currentstroke}%
\pgfsetdash{}{0pt}%
\pgfpathmoveto{\pgfqpoint{0.732537in}{1.937505in}}%
\pgfpathlineto{\pgfqpoint{0.804442in}{1.917875in}}%
\pgfpathlineto{\pgfqpoint{0.745596in}{1.872130in}}%
\pgfpathlineto{\pgfqpoint{0.732537in}{1.937505in}}%
\pgfpathclose%
\pgfusepath{stroke,fill}%
\end{pgfscope}%
\begin{pgfscope}%
\pgfpathrectangle{\pgfqpoint{0.647939in}{0.492442in}}{\pgfqpoint{3.079299in}{3.079299in}}%
\pgfusepath{clip}%
\pgfsetroundcap%
\pgfsetroundjoin%
\pgfsetlinewidth{0.803000pt}%
\definecolor{currentstroke}{rgb}{0.501961,0.501961,0.501961}%
\pgfsetstrokecolor{currentstroke}%
\pgfsetdash{}{0pt}%
\pgfpathmoveto{\pgfqpoint{1.456617in}{1.999901in}}%
\pgfpathquadraticcurveto{\pgfqpoint{1.459582in}{2.001168in}}{\pgfqpoint{1.451124in}{1.997553in}}%
\pgfusepath{stroke}%
\end{pgfscope}%
\begin{pgfscope}%
\pgfpathrectangle{\pgfqpoint{0.647939in}{0.492442in}}{\pgfqpoint{3.079299in}{3.079299in}}%
\pgfusepath{clip}%
\pgfsetroundcap%
\pgfsetroundjoin%
\definecolor{currentfill}{rgb}{0.501961,0.501961,0.501961}%
\pgfsetfillcolor{currentfill}%
\pgfsetlinewidth{0.803000pt}%
\definecolor{currentstroke}{rgb}{0.501961,0.501961,0.501961}%
\pgfsetstrokecolor{currentstroke}%
\pgfsetdash{}{0pt}%
\pgfpathmoveto{\pgfqpoint{1.376721in}{2.002009in}}%
\pgfpathlineto{\pgfqpoint{1.451124in}{1.997553in}}%
\pgfpathlineto{\pgfqpoint{1.402918in}{1.940705in}}%
\pgfpathlineto{\pgfqpoint{1.376721in}{2.002009in}}%
\pgfpathclose%
\pgfusepath{stroke,fill}%
\end{pgfscope}%
\begin{pgfscope}%
\pgfpathrectangle{\pgfqpoint{0.647939in}{0.492442in}}{\pgfqpoint{3.079299in}{3.079299in}}%
\pgfusepath{clip}%
\pgfsetroundcap%
\pgfsetroundjoin%
\pgfsetlinewidth{0.803000pt}%
\definecolor{currentstroke}{rgb}{0.501961,0.501961,0.501961}%
\pgfsetstrokecolor{currentstroke}%
\pgfsetdash{}{0pt}%
\pgfpathmoveto{\pgfqpoint{1.202461in}{1.829288in}}%
\pgfpathquadraticcurveto{\pgfqpoint{1.205459in}{1.830473in}}{\pgfqpoint{1.196906in}{1.827091in}}%
\pgfusepath{stroke}%
\end{pgfscope}%
\begin{pgfscope}%
\pgfpathrectangle{\pgfqpoint{0.647939in}{0.492442in}}{\pgfqpoint{3.079299in}{3.079299in}}%
\pgfusepath{clip}%
\pgfsetroundcap%
\pgfsetroundjoin%
\definecolor{currentfill}{rgb}{0.501961,0.501961,0.501961}%
\pgfsetfillcolor{currentfill}%
\pgfsetlinewidth{0.803000pt}%
\definecolor{currentstroke}{rgb}{0.501961,0.501961,0.501961}%
\pgfsetstrokecolor{currentstroke}%
\pgfsetdash{}{0pt}%
\pgfpathmoveto{\pgfqpoint{1.122652in}{1.833571in}}%
\pgfpathlineto{\pgfqpoint{1.196906in}{1.827091in}}%
\pgfpathlineto{\pgfqpoint{1.147169in}{1.771576in}}%
\pgfpathlineto{\pgfqpoint{1.122652in}{1.833571in}}%
\pgfpathclose%
\pgfusepath{stroke,fill}%
\end{pgfscope}%
\begin{pgfscope}%
\pgfpathrectangle{\pgfqpoint{0.647939in}{0.492442in}}{\pgfqpoint{3.079299in}{3.079299in}}%
\pgfusepath{clip}%
\pgfsetroundcap%
\pgfsetroundjoin%
\pgfsetlinewidth{0.803000pt}%
\definecolor{currentstroke}{rgb}{0.501961,0.501961,0.501961}%
\pgfsetstrokecolor{currentstroke}%
\pgfsetdash{}{0pt}%
\pgfpathmoveto{\pgfqpoint{1.008411in}{1.693069in}}%
\pgfpathquadraticcurveto{\pgfqpoint{1.011484in}{1.694042in}}{\pgfqpoint{1.002714in}{1.691266in}}%
\pgfusepath{stroke}%
\end{pgfscope}%
\begin{pgfscope}%
\pgfpathrectangle{\pgfqpoint{0.647939in}{0.492442in}}{\pgfqpoint{3.079299in}{3.079299in}}%
\pgfusepath{clip}%
\pgfsetroundcap%
\pgfsetroundjoin%
\definecolor{currentfill}{rgb}{0.501961,0.501961,0.501961}%
\pgfsetfillcolor{currentfill}%
\pgfsetlinewidth{0.803000pt}%
\definecolor{currentstroke}{rgb}{0.501961,0.501961,0.501961}%
\pgfsetstrokecolor{currentstroke}%
\pgfsetdash{}{0pt}%
\pgfpathmoveto{\pgfqpoint{0.929097in}{1.702928in}}%
\pgfpathlineto{\pgfqpoint{1.002714in}{1.691266in}}%
\pgfpathlineto{\pgfqpoint{0.949215in}{1.639369in}}%
\pgfpathlineto{\pgfqpoint{0.929097in}{1.702928in}}%
\pgfpathclose%
\pgfusepath{stroke,fill}%
\end{pgfscope}%
\begin{pgfscope}%
\pgfpathrectangle{\pgfqpoint{0.647939in}{0.492442in}}{\pgfqpoint{3.079299in}{3.079299in}}%
\pgfusepath{clip}%
\pgfsetroundcap%
\pgfsetroundjoin%
\pgfsetlinewidth{0.803000pt}%
\definecolor{currentstroke}{rgb}{0.501961,0.501961,0.501961}%
\pgfsetstrokecolor{currentstroke}%
\pgfsetdash{}{0pt}%
\pgfpathmoveto{\pgfqpoint{1.324962in}{1.752760in}}%
\pgfpathquadraticcurveto{\pgfqpoint{1.327872in}{1.754149in}}{\pgfqpoint{1.319570in}{1.750187in}}%
\pgfusepath{stroke}%
\end{pgfscope}%
\begin{pgfscope}%
\pgfpathrectangle{\pgfqpoint{0.647939in}{0.492442in}}{\pgfqpoint{3.079299in}{3.079299in}}%
\pgfusepath{clip}%
\pgfsetroundcap%
\pgfsetroundjoin%
\definecolor{currentfill}{rgb}{0.501961,0.501961,0.501961}%
\pgfsetfillcolor{currentfill}%
\pgfsetlinewidth{0.803000pt}%
\definecolor{currentstroke}{rgb}{0.501961,0.501961,0.501961}%
\pgfsetstrokecolor{currentstroke}%
\pgfsetdash{}{0pt}%
\pgfpathmoveto{\pgfqpoint{1.245048in}{1.751560in}}%
\pgfpathlineto{\pgfqpoint{1.319570in}{1.750187in}}%
\pgfpathlineto{\pgfqpoint{1.273759in}{1.691392in}}%
\pgfpathlineto{\pgfqpoint{1.245048in}{1.751560in}}%
\pgfpathclose%
\pgfusepath{stroke,fill}%
\end{pgfscope}%
\begin{pgfscope}%
\pgfpathrectangle{\pgfqpoint{0.647939in}{0.492442in}}{\pgfqpoint{3.079299in}{3.079299in}}%
\pgfusepath{clip}%
\pgfsetroundcap%
\pgfsetroundjoin%
\pgfsetlinewidth{0.803000pt}%
\definecolor{currentstroke}{rgb}{0.501961,0.501961,0.501961}%
\pgfsetstrokecolor{currentstroke}%
\pgfsetdash{}{0pt}%
\pgfpathmoveto{\pgfqpoint{1.135912in}{1.603838in}}%
\pgfpathquadraticcurveto{\pgfqpoint{1.138896in}{1.605058in}}{\pgfqpoint{1.130381in}{1.601577in}}%
\pgfusepath{stroke}%
\end{pgfscope}%
\begin{pgfscope}%
\pgfpathrectangle{\pgfqpoint{0.647939in}{0.492442in}}{\pgfqpoint{3.079299in}{3.079299in}}%
\pgfusepath{clip}%
\pgfsetroundcap%
\pgfsetroundjoin%
\definecolor{currentfill}{rgb}{0.501961,0.501961,0.501961}%
\pgfsetfillcolor{currentfill}%
\pgfsetlinewidth{0.803000pt}%
\definecolor{currentstroke}{rgb}{0.501961,0.501961,0.501961}%
\pgfsetstrokecolor{currentstroke}%
\pgfsetdash{}{0pt}%
\pgfpathmoveto{\pgfqpoint{1.056058in}{1.607205in}}%
\pgfpathlineto{\pgfqpoint{1.130381in}{1.601577in}}%
\pgfpathlineto{\pgfqpoint{1.081285in}{1.545495in}}%
\pgfpathlineto{\pgfqpoint{1.056058in}{1.607205in}}%
\pgfpathclose%
\pgfusepath{stroke,fill}%
\end{pgfscope}%
\begin{pgfscope}%
\pgfpathrectangle{\pgfqpoint{0.647939in}{0.492442in}}{\pgfqpoint{3.079299in}{3.079299in}}%
\pgfusepath{clip}%
\pgfsetroundcap%
\pgfsetroundjoin%
\pgfsetlinewidth{0.803000pt}%
\definecolor{currentstroke}{rgb}{0.501961,0.501961,0.501961}%
\pgfsetstrokecolor{currentstroke}%
\pgfsetdash{}{0pt}%
\pgfpathmoveto{\pgfqpoint{0.876111in}{1.448887in}}%
\pgfpathquadraticcurveto{\pgfqpoint{0.879225in}{1.449718in}}{\pgfqpoint{0.870337in}{1.447348in}}%
\pgfusepath{stroke}%
\end{pgfscope}%
\begin{pgfscope}%
\pgfpathrectangle{\pgfqpoint{0.647939in}{0.492442in}}{\pgfqpoint{3.079299in}{3.079299in}}%
\pgfusepath{clip}%
\pgfsetroundcap%
\pgfsetroundjoin%
\definecolor{currentfill}{rgb}{0.501961,0.501961,0.501961}%
\pgfsetfillcolor{currentfill}%
\pgfsetlinewidth{0.803000pt}%
\definecolor{currentstroke}{rgb}{0.501961,0.501961,0.501961}%
\pgfsetstrokecolor{currentstroke}%
\pgfsetdash{}{0pt}%
\pgfpathmoveto{\pgfqpoint{0.797332in}{1.462376in}}%
\pgfpathlineto{\pgfqpoint{0.870337in}{1.447348in}}%
\pgfpathlineto{\pgfqpoint{0.814512in}{1.397960in}}%
\pgfpathlineto{\pgfqpoint{0.797332in}{1.462376in}}%
\pgfpathclose%
\pgfusepath{stroke,fill}%
\end{pgfscope}%
\begin{pgfscope}%
\pgfpathrectangle{\pgfqpoint{0.647939in}{0.492442in}}{\pgfqpoint{3.079299in}{3.079299in}}%
\pgfusepath{clip}%
\pgfsetroundcap%
\pgfsetroundjoin%
\pgfsetlinewidth{0.803000pt}%
\definecolor{currentstroke}{rgb}{0.501961,0.501961,0.501961}%
\pgfsetstrokecolor{currentstroke}%
\pgfsetdash{}{0pt}%
\pgfpathmoveto{\pgfqpoint{1.317733in}{1.561382in}}%
\pgfpathquadraticcurveto{\pgfqpoint{1.320553in}{1.562944in}}{\pgfqpoint{1.312506in}{1.558488in}}%
\pgfusepath{stroke}%
\end{pgfscope}%
\begin{pgfscope}%
\pgfpathrectangle{\pgfqpoint{0.647939in}{0.492442in}}{\pgfqpoint{3.079299in}{3.079299in}}%
\pgfusepath{clip}%
\pgfsetroundcap%
\pgfsetroundjoin%
\definecolor{currentfill}{rgb}{0.501961,0.501961,0.501961}%
\pgfsetfillcolor{currentfill}%
\pgfsetlinewidth{0.803000pt}%
\definecolor{currentstroke}{rgb}{0.501961,0.501961,0.501961}%
\pgfsetstrokecolor{currentstroke}%
\pgfsetdash{}{0pt}%
\pgfpathmoveto{\pgfqpoint{1.238036in}{1.555354in}}%
\pgfpathlineto{\pgfqpoint{1.312506in}{1.558488in}}%
\pgfpathlineto{\pgfqpoint{1.270331in}{1.497031in}}%
\pgfpathlineto{\pgfqpoint{1.238036in}{1.555354in}}%
\pgfpathclose%
\pgfusepath{stroke,fill}%
\end{pgfscope}%
\begin{pgfscope}%
\pgfpathrectangle{\pgfqpoint{0.647939in}{0.492442in}}{\pgfqpoint{3.079299in}{3.079299in}}%
\pgfusepath{clip}%
\pgfsetroundcap%
\pgfsetroundjoin%
\pgfsetlinewidth{0.803000pt}%
\definecolor{currentstroke}{rgb}{0.501961,0.501961,0.501961}%
\pgfsetstrokecolor{currentstroke}%
\pgfsetdash{}{0pt}%
\pgfpathmoveto{\pgfqpoint{1.132690in}{1.403896in}}%
\pgfpathquadraticcurveto{\pgfqpoint{1.135622in}{1.405237in}}{\pgfqpoint{1.127256in}{1.401412in}}%
\pgfusepath{stroke}%
\end{pgfscope}%
\begin{pgfscope}%
\pgfpathrectangle{\pgfqpoint{0.647939in}{0.492442in}}{\pgfqpoint{3.079299in}{3.079299in}}%
\pgfusepath{clip}%
\pgfsetroundcap%
\pgfsetroundjoin%
\definecolor{currentfill}{rgb}{0.501961,0.501961,0.501961}%
\pgfsetfillcolor{currentfill}%
\pgfsetlinewidth{0.803000pt}%
\definecolor{currentstroke}{rgb}{0.501961,0.501961,0.501961}%
\pgfsetstrokecolor{currentstroke}%
\pgfsetdash{}{0pt}%
\pgfpathmoveto{\pgfqpoint{1.052765in}{1.404005in}}%
\pgfpathlineto{\pgfqpoint{1.127256in}{1.401412in}}%
\pgfpathlineto{\pgfqpoint{1.080486in}{1.343376in}}%
\pgfpathlineto{\pgfqpoint{1.052765in}{1.404005in}}%
\pgfpathclose%
\pgfusepath{stroke,fill}%
\end{pgfscope}%
\begin{pgfscope}%
\pgfpathrectangle{\pgfqpoint{0.647939in}{0.492442in}}{\pgfqpoint{3.079299in}{3.079299in}}%
\pgfusepath{clip}%
\pgfsetroundcap%
\pgfsetroundjoin%
\pgfsetlinewidth{0.803000pt}%
\definecolor{currentstroke}{rgb}{0.501961,0.501961,0.501961}%
\pgfsetstrokecolor{currentstroke}%
\pgfsetdash{}{0pt}%
\pgfpathmoveto{\pgfqpoint{0.809310in}{1.224440in}}%
\pgfpathquadraticcurveto{\pgfqpoint{0.812440in}{1.225209in}}{\pgfqpoint{0.803507in}{1.223014in}}%
\pgfusepath{stroke}%
\end{pgfscope}%
\begin{pgfscope}%
\pgfpathrectangle{\pgfqpoint{0.647939in}{0.492442in}}{\pgfqpoint{3.079299in}{3.079299in}}%
\pgfusepath{clip}%
\pgfsetroundcap%
\pgfsetroundjoin%
\definecolor{currentfill}{rgb}{0.501961,0.501961,0.501961}%
\pgfsetfillcolor{currentfill}%
\pgfsetlinewidth{0.803000pt}%
\definecolor{currentstroke}{rgb}{0.501961,0.501961,0.501961}%
\pgfsetstrokecolor{currentstroke}%
\pgfsetdash{}{0pt}%
\pgfpathmoveto{\pgfqpoint{0.730812in}{1.239474in}}%
\pgfpathlineto{\pgfqpoint{0.803507in}{1.223014in}}%
\pgfpathlineto{\pgfqpoint{0.746721in}{1.174734in}}%
\pgfpathlineto{\pgfqpoint{0.730812in}{1.239474in}}%
\pgfpathclose%
\pgfusepath{stroke,fill}%
\end{pgfscope}%
\begin{pgfscope}%
\pgfpathrectangle{\pgfqpoint{0.647939in}{0.492442in}}{\pgfqpoint{3.079299in}{3.079299in}}%
\pgfusepath{clip}%
\pgfsetroundcap%
\pgfsetroundjoin%
\pgfsetlinewidth{0.803000pt}%
\definecolor{currentstroke}{rgb}{0.501961,0.501961,0.501961}%
\pgfsetstrokecolor{currentstroke}%
\pgfsetdash{}{0pt}%
\pgfpathmoveto{\pgfqpoint{1.308347in}{1.372785in}}%
\pgfpathquadraticcurveto{\pgfqpoint{1.311047in}{1.374547in}}{\pgfqpoint{1.303343in}{1.369520in}}%
\pgfusepath{stroke}%
\end{pgfscope}%
\begin{pgfscope}%
\pgfpathrectangle{\pgfqpoint{0.647939in}{0.492442in}}{\pgfqpoint{3.079299in}{3.079299in}}%
\pgfusepath{clip}%
\pgfsetroundcap%
\pgfsetroundjoin%
\definecolor{currentfill}{rgb}{0.501961,0.501961,0.501961}%
\pgfsetfillcolor{currentfill}%
\pgfsetlinewidth{0.803000pt}%
\definecolor{currentstroke}{rgb}{0.501961,0.501961,0.501961}%
\pgfsetstrokecolor{currentstroke}%
\pgfsetdash{}{0pt}%
\pgfpathmoveto{\pgfqpoint{1.229295in}{1.361007in}}%
\pgfpathlineto{\pgfqpoint{1.303343in}{1.369520in}}%
\pgfpathlineto{\pgfqpoint{1.265725in}{1.305174in}}%
\pgfpathlineto{\pgfqpoint{1.229295in}{1.361007in}}%
\pgfpathclose%
\pgfusepath{stroke,fill}%
\end{pgfscope}%
\begin{pgfscope}%
\pgfpathrectangle{\pgfqpoint{0.647939in}{0.492442in}}{\pgfqpoint{3.079299in}{3.079299in}}%
\pgfusepath{clip}%
\pgfsetroundcap%
\pgfsetroundjoin%
\pgfsetlinewidth{0.803000pt}%
\definecolor{currentstroke}{rgb}{0.501961,0.501961,0.501961}%
\pgfsetstrokecolor{currentstroke}%
\pgfsetdash{}{0pt}%
\pgfpathmoveto{\pgfqpoint{1.004133in}{1.148848in}}%
\pgfpathquadraticcurveto{\pgfqpoint{1.007122in}{1.150056in}}{\pgfqpoint{0.998593in}{1.146607in}}%
\pgfusepath{stroke}%
\end{pgfscope}%
\begin{pgfscope}%
\pgfpathrectangle{\pgfqpoint{0.647939in}{0.492442in}}{\pgfqpoint{3.079299in}{3.079299in}}%
\pgfusepath{clip}%
\pgfsetroundcap%
\pgfsetroundjoin%
\definecolor{currentfill}{rgb}{0.501961,0.501961,0.501961}%
\pgfsetfillcolor{currentfill}%
\pgfsetlinewidth{0.803000pt}%
\definecolor{currentstroke}{rgb}{0.501961,0.501961,0.501961}%
\pgfsetstrokecolor{currentstroke}%
\pgfsetdash{}{0pt}%
\pgfpathmoveto{\pgfqpoint{0.924292in}{1.152518in}}%
\pgfpathlineto{\pgfqpoint{0.998593in}{1.146607in}}%
\pgfpathlineto{\pgfqpoint{0.949284in}{1.090713in}}%
\pgfpathlineto{\pgfqpoint{0.924292in}{1.152518in}}%
\pgfpathclose%
\pgfusepath{stroke,fill}%
\end{pgfscope}%
\begin{pgfscope}%
\pgfpathrectangle{\pgfqpoint{0.647939in}{0.492442in}}{\pgfqpoint{3.079299in}{3.079299in}}%
\pgfusepath{clip}%
\pgfsetroundcap%
\pgfsetroundjoin%
\pgfsetlinewidth{0.803000pt}%
\definecolor{currentstroke}{rgb}{0.501961,0.501961,0.501961}%
\pgfsetstrokecolor{currentstroke}%
\pgfsetdash{}{0pt}%
\pgfpathmoveto{\pgfqpoint{0.874694in}{1.035088in}}%
\pgfpathquadraticcurveto{\pgfqpoint{0.877771in}{1.036046in}}{\pgfqpoint{0.868988in}{1.033312in}}%
\pgfusepath{stroke}%
\end{pgfscope}%
\begin{pgfscope}%
\pgfpathrectangle{\pgfqpoint{0.647939in}{0.492442in}}{\pgfqpoint{3.079299in}{3.079299in}}%
\pgfusepath{clip}%
\pgfsetroundcap%
\pgfsetroundjoin%
\definecolor{currentfill}{rgb}{0.501961,0.501961,0.501961}%
\pgfsetfillcolor{currentfill}%
\pgfsetlinewidth{0.803000pt}%
\definecolor{currentstroke}{rgb}{0.501961,0.501961,0.501961}%
\pgfsetstrokecolor{currentstroke}%
\pgfsetdash{}{0pt}%
\pgfpathmoveto{\pgfqpoint{0.795427in}{1.045326in}}%
\pgfpathlineto{\pgfqpoint{0.868988in}{1.033312in}}%
\pgfpathlineto{\pgfqpoint{0.815240in}{0.981672in}}%
\pgfpathlineto{\pgfqpoint{0.795427in}{1.045326in}}%
\pgfpathclose%
\pgfusepath{stroke,fill}%
\end{pgfscope}%
\begin{pgfscope}%
\pgfpathrectangle{\pgfqpoint{0.647939in}{0.492442in}}{\pgfqpoint{3.079299in}{3.079299in}}%
\pgfusepath{clip}%
\pgfsetroundcap%
\pgfsetroundjoin%
\pgfsetlinewidth{0.803000pt}%
\definecolor{currentstroke}{rgb}{0.501961,0.501961,0.501961}%
\pgfsetstrokecolor{currentstroke}%
\pgfsetdash{}{0pt}%
\pgfpathmoveto{\pgfqpoint{1.125322in}{1.073981in}}%
\pgfpathquadraticcurveto{\pgfqpoint{1.128129in}{1.075565in}}{\pgfqpoint{1.120118in}{1.071042in}}%
\pgfusepath{stroke}%
\end{pgfscope}%
\begin{pgfscope}%
\pgfpathrectangle{\pgfqpoint{0.647939in}{0.492442in}}{\pgfqpoint{3.079299in}{3.079299in}}%
\pgfusepath{clip}%
\pgfsetroundcap%
\pgfsetroundjoin%
\definecolor{currentfill}{rgb}{0.501961,0.501961,0.501961}%
\pgfsetfillcolor{currentfill}%
\pgfsetlinewidth{0.803000pt}%
\definecolor{currentstroke}{rgb}{0.501961,0.501961,0.501961}%
\pgfsetstrokecolor{currentstroke}%
\pgfsetdash{}{0pt}%
\pgfpathmoveto{\pgfqpoint{1.045677in}{1.067289in}}%
\pgfpathlineto{\pgfqpoint{1.120118in}{1.071042in}}%
\pgfpathlineto{\pgfqpoint{1.078456in}{1.009237in}}%
\pgfpathlineto{\pgfqpoint{1.045677in}{1.067289in}}%
\pgfpathclose%
\pgfusepath{stroke,fill}%
\end{pgfscope}%
\begin{pgfscope}%
\pgfpathrectangle{\pgfqpoint{0.647939in}{0.492442in}}{\pgfqpoint{3.079299in}{3.079299in}}%
\pgfusepath{clip}%
\pgfsetroundcap%
\pgfsetroundjoin%
\pgfsetlinewidth{0.803000pt}%
\definecolor{currentstroke}{rgb}{0.501961,0.501961,0.501961}%
\pgfsetstrokecolor{currentstroke}%
\pgfsetdash{}{0pt}%
\pgfpathmoveto{\pgfqpoint{1.001787in}{0.946300in}}%
\pgfpathquadraticcurveto{\pgfqpoint{1.004726in}{0.947623in}}{\pgfqpoint{0.996336in}{0.943848in}}%
\pgfusepath{stroke}%
\end{pgfscope}%
\begin{pgfscope}%
\pgfpathrectangle{\pgfqpoint{0.647939in}{0.492442in}}{\pgfqpoint{3.079299in}{3.079299in}}%
\pgfusepath{clip}%
\pgfsetroundcap%
\pgfsetroundjoin%
\definecolor{currentfill}{rgb}{0.501961,0.501961,0.501961}%
\pgfsetfillcolor{currentfill}%
\pgfsetlinewidth{0.803000pt}%
\definecolor{currentstroke}{rgb}{0.501961,0.501961,0.501961}%
\pgfsetstrokecolor{currentstroke}%
\pgfsetdash{}{0pt}%
\pgfpathmoveto{\pgfqpoint{0.921863in}{0.946887in}}%
\pgfpathlineto{\pgfqpoint{0.996336in}{0.943848in}}%
\pgfpathlineto{\pgfqpoint{0.949221in}{0.886092in}}%
\pgfpathlineto{\pgfqpoint{0.921863in}{0.946887in}}%
\pgfpathclose%
\pgfusepath{stroke,fill}%
\end{pgfscope}%
\begin{pgfscope}%
\pgfpathrectangle{\pgfqpoint{0.647939in}{0.492442in}}{\pgfqpoint{3.079299in}{3.079299in}}%
\pgfusepath{clip}%
\pgfsetroundcap%
\pgfsetroundjoin%
\pgfsetlinewidth{0.803000pt}%
\definecolor{currentstroke}{rgb}{0.501961,0.501961,0.501961}%
\pgfsetstrokecolor{currentstroke}%
\pgfsetdash{}{0pt}%
\pgfpathmoveto{\pgfqpoint{0.938004in}{0.852168in}}%
\pgfpathquadraticcurveto{\pgfqpoint{0.940996in}{0.853365in}}{\pgfqpoint{0.932455in}{0.849947in}}%
\pgfusepath{stroke}%
\end{pgfscope}%
\begin{pgfscope}%
\pgfpathrectangle{\pgfqpoint{0.647939in}{0.492442in}}{\pgfqpoint{3.079299in}{3.079299in}}%
\pgfusepath{clip}%
\pgfsetroundcap%
\pgfsetroundjoin%
\definecolor{currentfill}{rgb}{0.501961,0.501961,0.501961}%
\pgfsetfillcolor{currentfill}%
\pgfsetlinewidth{0.803000pt}%
\definecolor{currentstroke}{rgb}{0.501961,0.501961,0.501961}%
\pgfsetstrokecolor{currentstroke}%
\pgfsetdash{}{0pt}%
\pgfpathmoveto{\pgfqpoint{0.858176in}{0.856127in}}%
\pgfpathlineto{\pgfqpoint{0.932455in}{0.849947in}}%
\pgfpathlineto{\pgfqpoint{0.882943in}{0.794232in}}%
\pgfpathlineto{\pgfqpoint{0.858176in}{0.856127in}}%
\pgfpathclose%
\pgfusepath{stroke,fill}%
\end{pgfscope}%
\begin{pgfscope}%
\pgfpathrectangle{\pgfqpoint{0.647939in}{0.492442in}}{\pgfqpoint{3.079299in}{3.079299in}}%
\pgfusepath{clip}%
\pgfsetroundcap%
\pgfsetroundjoin%
\pgfsetlinewidth{0.803000pt}%
\definecolor{currentstroke}{rgb}{0.501961,0.501961,0.501961}%
\pgfsetstrokecolor{currentstroke}%
\pgfsetdash{}{0pt}%
\pgfpathmoveto{\pgfqpoint{0.937390in}{0.784181in}}%
\pgfpathquadraticcurveto{\pgfqpoint{0.940367in}{0.785414in}}{\pgfqpoint{0.931867in}{0.781893in}}%
\pgfusepath{stroke}%
\end{pgfscope}%
\begin{pgfscope}%
\pgfpathrectangle{\pgfqpoint{0.647939in}{0.492442in}}{\pgfqpoint{3.079299in}{3.079299in}}%
\pgfusepath{clip}%
\pgfsetroundcap%
\pgfsetroundjoin%
\definecolor{currentfill}{rgb}{0.501961,0.501961,0.501961}%
\pgfsetfillcolor{currentfill}%
\pgfsetlinewidth{0.803000pt}%
\definecolor{currentstroke}{rgb}{0.501961,0.501961,0.501961}%
\pgfsetstrokecolor{currentstroke}%
\pgfsetdash{}{0pt}%
\pgfpathmoveto{\pgfqpoint{0.857519in}{0.787179in}}%
\pgfpathlineto{\pgfqpoint{0.931867in}{0.781893in}}%
\pgfpathlineto{\pgfqpoint{0.883030in}{0.725586in}}%
\pgfpathlineto{\pgfqpoint{0.857519in}{0.787179in}}%
\pgfpathclose%
\pgfusepath{stroke,fill}%
\end{pgfscope}%
\begin{pgfscope}%
\pgfpathrectangle{\pgfqpoint{0.647939in}{0.492442in}}{\pgfqpoint{3.079299in}{3.079299in}}%
\pgfusepath{clip}%
\pgfsetroundcap%
\pgfsetroundjoin%
\pgfsetlinewidth{0.803000pt}%
\definecolor{currentstroke}{rgb}{0.501961,0.501961,0.501961}%
\pgfsetstrokecolor{currentstroke}%
\pgfsetdash{}{0pt}%
\pgfpathmoveto{\pgfqpoint{0.872999in}{0.691547in}}%
\pgfpathquadraticcurveto{\pgfqpoint{0.876030in}{0.692641in}}{\pgfqpoint{0.867377in}{0.689518in}}%
\pgfusepath{stroke}%
\end{pgfscope}%
\begin{pgfscope}%
\pgfpathrectangle{\pgfqpoint{0.647939in}{0.492442in}}{\pgfqpoint{3.079299in}{3.079299in}}%
\pgfusepath{clip}%
\pgfsetroundcap%
\pgfsetroundjoin%
\definecolor{currentfill}{rgb}{0.501961,0.501961,0.501961}%
\pgfsetfillcolor{currentfill}%
\pgfsetlinewidth{0.803000pt}%
\definecolor{currentstroke}{rgb}{0.501961,0.501961,0.501961}%
\pgfsetstrokecolor{currentstroke}%
\pgfsetdash{}{0pt}%
\pgfpathmoveto{\pgfqpoint{0.793353in}{0.698241in}}%
\pgfpathlineto{\pgfqpoint{0.867377in}{0.689518in}}%
\pgfpathlineto{\pgfqpoint{0.815984in}{0.635533in}}%
\pgfpathlineto{\pgfqpoint{0.793353in}{0.698241in}}%
\pgfpathclose%
\pgfusepath{stroke,fill}%
\end{pgfscope}%
\begin{pgfscope}%
\pgfpathrectangle{\pgfqpoint{0.647939in}{0.492442in}}{\pgfqpoint{3.079299in}{3.079299in}}%
\pgfusepath{clip}%
\pgfsetroundcap%
\pgfsetroundjoin%
\pgfsetlinewidth{0.803000pt}%
\definecolor{currentstroke}{rgb}{0.501961,0.501961,0.501961}%
\pgfsetstrokecolor{currentstroke}%
\pgfsetdash{}{0pt}%
\pgfpathmoveto{\pgfqpoint{2.079836in}{0.602443in}}%
\pgfpathquadraticcurveto{\pgfqpoint{2.076612in}{0.602489in}}{\pgfqpoint{2.085810in}{0.602357in}}%
\pgfusepath{stroke}%
\end{pgfscope}%
\begin{pgfscope}%
\pgfpathrectangle{\pgfqpoint{0.647939in}{0.492442in}}{\pgfqpoint{3.079299in}{3.079299in}}%
\pgfusepath{clip}%
\pgfsetroundcap%
\pgfsetroundjoin%
\definecolor{currentfill}{rgb}{0.501961,0.501961,0.501961}%
\pgfsetfillcolor{currentfill}%
\pgfsetlinewidth{0.803000pt}%
\definecolor{currentstroke}{rgb}{0.501961,0.501961,0.501961}%
\pgfsetstrokecolor{currentstroke}%
\pgfsetdash{}{0pt}%
\pgfpathmoveto{\pgfqpoint{2.151991in}{0.568070in}}%
\pgfpathlineto{\pgfqpoint{2.085810in}{0.602357in}}%
\pgfpathlineto{\pgfqpoint{2.152948in}{0.634730in}}%
\pgfpathlineto{\pgfqpoint{2.151991in}{0.568070in}}%
\pgfpathclose%
\pgfusepath{stroke,fill}%
\end{pgfscope}%
\begin{pgfscope}%
\pgfpathrectangle{\pgfqpoint{0.647939in}{0.492442in}}{\pgfqpoint{3.079299in}{3.079299in}}%
\pgfusepath{clip}%
\pgfsetroundcap%
\pgfsetroundjoin%
\pgfsetlinewidth{0.803000pt}%
\definecolor{currentstroke}{rgb}{0.501961,0.501961,0.501961}%
\pgfsetstrokecolor{currentstroke}%
\pgfsetdash{}{0pt}%
\pgfpathmoveto{\pgfqpoint{3.449525in}{1.562264in}}%
\pgfpathquadraticcurveto{\pgfqpoint{3.446891in}{1.564123in}}{\pgfqpoint{3.454407in}{1.558819in}}%
\pgfusepath{stroke}%
\end{pgfscope}%
\begin{pgfscope}%
\pgfpathrectangle{\pgfqpoint{0.647939in}{0.492442in}}{\pgfqpoint{3.079299in}{3.079299in}}%
\pgfusepath{clip}%
\pgfsetroundcap%
\pgfsetroundjoin%
\definecolor{currentfill}{rgb}{0.501961,0.501961,0.501961}%
\pgfsetfillcolor{currentfill}%
\pgfsetlinewidth{0.803000pt}%
\definecolor{currentstroke}{rgb}{0.501961,0.501961,0.501961}%
\pgfsetstrokecolor{currentstroke}%
\pgfsetdash{}{0pt}%
\pgfpathmoveto{\pgfqpoint{3.489659in}{1.493147in}}%
\pgfpathlineto{\pgfqpoint{3.454407in}{1.558819in}}%
\pgfpathlineto{\pgfqpoint{3.528096in}{1.547618in}}%
\pgfpathlineto{\pgfqpoint{3.489659in}{1.493147in}}%
\pgfpathclose%
\pgfusepath{stroke,fill}%
\end{pgfscope}%
\begin{pgfscope}%
\pgfpathrectangle{\pgfqpoint{0.647939in}{0.492442in}}{\pgfqpoint{3.079299in}{3.079299in}}%
\pgfusepath{clip}%
\pgfsetroundcap%
\pgfsetroundjoin%
\pgfsetlinewidth{0.803000pt}%
\definecolor{currentstroke}{rgb}{0.501961,0.501961,0.501961}%
\pgfsetstrokecolor{currentstroke}%
\pgfsetdash{}{0pt}%
\pgfpathmoveto{\pgfqpoint{3.411193in}{2.884270in}}%
\pgfpathquadraticcurveto{\pgfqpoint{3.412173in}{2.887339in}}{\pgfqpoint{3.409375in}{2.878575in}}%
\pgfusepath{stroke}%
\end{pgfscope}%
\begin{pgfscope}%
\pgfpathrectangle{\pgfqpoint{0.647939in}{0.492442in}}{\pgfqpoint{3.079299in}{3.079299in}}%
\pgfusepath{clip}%
\pgfsetroundcap%
\pgfsetroundjoin%
\definecolor{currentfill}{rgb}{0.501961,0.501961,0.501961}%
\pgfsetfillcolor{currentfill}%
\pgfsetlinewidth{0.803000pt}%
\definecolor{currentstroke}{rgb}{0.501961,0.501961,0.501961}%
\pgfsetstrokecolor{currentstroke}%
\pgfsetdash{}{0pt}%
\pgfpathmoveto{\pgfqpoint{3.357348in}{2.825201in}}%
\pgfpathlineto{\pgfqpoint{3.409375in}{2.878575in}}%
\pgfpathlineto{\pgfqpoint{3.420858in}{2.804929in}}%
\pgfpathlineto{\pgfqpoint{3.357348in}{2.825201in}}%
\pgfpathclose%
\pgfusepath{stroke,fill}%
\end{pgfscope}%
\begin{pgfscope}%
\pgfpathrectangle{\pgfqpoint{0.647939in}{0.492442in}}{\pgfqpoint{3.079299in}{3.079299in}}%
\pgfusepath{clip}%
\pgfsetroundcap%
\pgfsetroundjoin%
\pgfsetlinewidth{0.803000pt}%
\definecolor{currentstroke}{rgb}{0.501961,0.501961,0.501961}%
\pgfsetstrokecolor{currentstroke}%
\pgfsetdash{}{0pt}%
\pgfpathmoveto{\pgfqpoint{3.060400in}{2.371385in}}%
\pgfpathquadraticcurveto{\pgfqpoint{3.059546in}{2.374455in}}{\pgfqpoint{3.062022in}{2.365556in}}%
\pgfusepath{stroke}%
\end{pgfscope}%
\begin{pgfscope}%
\pgfpathrectangle{\pgfqpoint{0.647939in}{0.492442in}}{\pgfqpoint{3.079299in}{3.079299in}}%
\pgfusepath{clip}%
\pgfsetroundcap%
\pgfsetroundjoin%
\definecolor{currentfill}{rgb}{0.501961,0.501961,0.501961}%
\pgfsetfillcolor{currentfill}%
\pgfsetlinewidth{0.803000pt}%
\definecolor{currentstroke}{rgb}{0.501961,0.501961,0.501961}%
\pgfsetstrokecolor{currentstroke}%
\pgfsetdash{}{0pt}%
\pgfpathmoveto{\pgfqpoint{3.047781in}{2.292394in}}%
\pgfpathlineto{\pgfqpoint{3.062022in}{2.365556in}}%
\pgfpathlineto{\pgfqpoint{3.112007in}{2.310266in}}%
\pgfpathlineto{\pgfqpoint{3.047781in}{2.292394in}}%
\pgfpathclose%
\pgfusepath{stroke,fill}%
\end{pgfscope}%
\begin{pgfscope}%
\pgfpathrectangle{\pgfqpoint{0.647939in}{0.492442in}}{\pgfqpoint{3.079299in}{3.079299in}}%
\pgfusepath{clip}%
\pgfsetroundcap%
\pgfsetroundjoin%
\pgfsetlinewidth{0.803000pt}%
\definecolor{currentstroke}{rgb}{0.501961,0.501961,0.501961}%
\pgfsetstrokecolor{currentstroke}%
\pgfsetdash{}{0pt}%
\pgfpathmoveto{\pgfqpoint{2.916360in}{1.765795in}}%
\pgfpathquadraticcurveto{\pgfqpoint{2.913700in}{1.767618in}}{\pgfqpoint{2.921288in}{1.762417in}}%
\pgfusepath{stroke}%
\end{pgfscope}%
\begin{pgfscope}%
\pgfpathrectangle{\pgfqpoint{0.647939in}{0.492442in}}{\pgfqpoint{3.079299in}{3.079299in}}%
\pgfusepath{clip}%
\pgfsetroundcap%
\pgfsetroundjoin%
\definecolor{currentfill}{rgb}{0.501961,0.501961,0.501961}%
\pgfsetfillcolor{currentfill}%
\pgfsetlinewidth{0.803000pt}%
\definecolor{currentstroke}{rgb}{0.501961,0.501961,0.501961}%
\pgfsetstrokecolor{currentstroke}%
\pgfsetdash{}{0pt}%
\pgfpathmoveto{\pgfqpoint{2.957428in}{1.697229in}}%
\pgfpathlineto{\pgfqpoint{2.921288in}{1.762417in}}%
\pgfpathlineto{\pgfqpoint{2.995122in}{1.752217in}}%
\pgfpathlineto{\pgfqpoint{2.957428in}{1.697229in}}%
\pgfpathclose%
\pgfusepath{stroke,fill}%
\end{pgfscope}%
\begin{pgfscope}%
\pgfpathrectangle{\pgfqpoint{0.647939in}{0.492442in}}{\pgfqpoint{3.079299in}{3.079299in}}%
\pgfusepath{clip}%
\pgfsetroundcap%
\pgfsetroundjoin%
\pgfsetlinewidth{0.803000pt}%
\definecolor{currentstroke}{rgb}{0.501961,0.501961,0.501961}%
\pgfsetstrokecolor{currentstroke}%
\pgfsetdash{}{0pt}%
\pgfpathmoveto{\pgfqpoint{2.011645in}{3.289650in}}%
\pgfpathquadraticcurveto{\pgfqpoint{2.014868in}{3.289734in}}{\pgfqpoint{2.005673in}{3.289494in}}%
\pgfusepath{stroke}%
\end{pgfscope}%
\begin{pgfscope}%
\pgfpathrectangle{\pgfqpoint{0.647939in}{0.492442in}}{\pgfqpoint{3.079299in}{3.079299in}}%
\pgfusepath{clip}%
\pgfsetroundcap%
\pgfsetroundjoin%
\definecolor{currentfill}{rgb}{0.501961,0.501961,0.501961}%
\pgfsetfillcolor{currentfill}%
\pgfsetlinewidth{0.803000pt}%
\definecolor{currentstroke}{rgb}{0.501961,0.501961,0.501961}%
\pgfsetstrokecolor{currentstroke}%
\pgfsetdash{}{0pt}%
\pgfpathmoveto{\pgfqpoint{1.938159in}{3.321077in}}%
\pgfpathlineto{\pgfqpoint{2.005673in}{3.289494in}}%
\pgfpathlineto{\pgfqpoint{1.939898in}{3.254433in}}%
\pgfpathlineto{\pgfqpoint{1.938159in}{3.321077in}}%
\pgfpathclose%
\pgfusepath{stroke,fill}%
\end{pgfscope}%
\begin{pgfscope}%
\pgfpathrectangle{\pgfqpoint{0.647939in}{0.492442in}}{\pgfqpoint{3.079299in}{3.079299in}}%
\pgfusepath{clip}%
\pgfsetroundcap%
\pgfsetroundjoin%
\pgfsetlinewidth{0.803000pt}%
\definecolor{currentstroke}{rgb}{0.501961,0.501961,0.501961}%
\pgfsetstrokecolor{currentstroke}%
\pgfsetdash{}{0pt}%
\pgfpathmoveto{\pgfqpoint{2.153160in}{0.901432in}}%
\pgfpathquadraticcurveto{\pgfqpoint{2.149935in}{0.901444in}}{\pgfqpoint{2.159134in}{0.901410in}}%
\pgfusepath{stroke}%
\end{pgfscope}%
\begin{pgfscope}%
\pgfpathrectangle{\pgfqpoint{0.647939in}{0.492442in}}{\pgfqpoint{3.079299in}{3.079299in}}%
\pgfusepath{clip}%
\pgfsetroundcap%
\pgfsetroundjoin%
\definecolor{currentfill}{rgb}{0.501961,0.501961,0.501961}%
\pgfsetfillcolor{currentfill}%
\pgfsetlinewidth{0.803000pt}%
\definecolor{currentstroke}{rgb}{0.501961,0.501961,0.501961}%
\pgfsetstrokecolor{currentstroke}%
\pgfsetdash{}{0pt}%
\pgfpathmoveto{\pgfqpoint{2.225678in}{0.867833in}}%
\pgfpathlineto{\pgfqpoint{2.159134in}{0.901410in}}%
\pgfpathlineto{\pgfqpoint{2.225922in}{0.934500in}}%
\pgfpathlineto{\pgfqpoint{2.225678in}{0.867833in}}%
\pgfpathclose%
\pgfusepath{stroke,fill}%
\end{pgfscope}%
\begin{pgfscope}%
\pgfpathrectangle{\pgfqpoint{0.647939in}{0.492442in}}{\pgfqpoint{3.079299in}{3.079299in}}%
\pgfusepath{clip}%
\pgfsetroundcap%
\pgfsetroundjoin%
\pgfsetlinewidth{0.803000pt}%
\definecolor{currentstroke}{rgb}{0.501961,0.501961,0.501961}%
\pgfsetstrokecolor{currentstroke}%
\pgfsetdash{}{0pt}%
\pgfpathmoveto{\pgfqpoint{3.235605in}{2.319956in}}%
\pgfpathquadraticcurveto{\pgfqpoint{3.234479in}{2.322938in}}{\pgfqpoint{3.237741in}{2.314300in}}%
\pgfusepath{stroke}%
\end{pgfscope}%
\begin{pgfscope}%
\pgfpathrectangle{\pgfqpoint{0.647939in}{0.492442in}}{\pgfqpoint{3.079299in}{3.079299in}}%
\pgfusepath{clip}%
\pgfsetroundcap%
\pgfsetroundjoin%
\definecolor{currentfill}{rgb}{0.501961,0.501961,0.501961}%
\pgfsetfillcolor{currentfill}%
\pgfsetlinewidth{0.803000pt}%
\definecolor{currentstroke}{rgb}{0.501961,0.501961,0.501961}%
\pgfsetstrokecolor{currentstroke}%
\pgfsetdash{}{0pt}%
\pgfpathmoveto{\pgfqpoint{3.230113in}{2.240155in}}%
\pgfpathlineto{\pgfqpoint{3.237741in}{2.314300in}}%
\pgfpathlineto{\pgfqpoint{3.292479in}{2.263710in}}%
\pgfpathlineto{\pgfqpoint{3.230113in}{2.240155in}}%
\pgfpathclose%
\pgfusepath{stroke,fill}%
\end{pgfscope}%
\begin{pgfscope}%
\pgfpathrectangle{\pgfqpoint{0.647939in}{0.492442in}}{\pgfqpoint{3.079299in}{3.079299in}}%
\pgfusepath{clip}%
\pgfsetroundcap%
\pgfsetroundjoin%
\pgfsetlinewidth{0.803000pt}%
\definecolor{currentstroke}{rgb}{0.501961,0.501961,0.501961}%
\pgfsetstrokecolor{currentstroke}%
\pgfsetdash{}{0pt}%
\pgfpathmoveto{\pgfqpoint{2.881523in}{2.111940in}}%
\pgfpathquadraticcurveto{\pgfqpoint{2.879528in}{2.114470in}}{\pgfqpoint{2.885224in}{2.107247in}}%
\pgfusepath{stroke}%
\end{pgfscope}%
\begin{pgfscope}%
\pgfpathrectangle{\pgfqpoint{0.647939in}{0.492442in}}{\pgfqpoint{3.079299in}{3.079299in}}%
\pgfusepath{clip}%
\pgfsetroundcap%
\pgfsetroundjoin%
\definecolor{currentfill}{rgb}{0.501961,0.501961,0.501961}%
\pgfsetfillcolor{currentfill}%
\pgfsetlinewidth{0.803000pt}%
\definecolor{currentstroke}{rgb}{0.501961,0.501961,0.501961}%
\pgfsetstrokecolor{currentstroke}%
\pgfsetdash{}{0pt}%
\pgfpathmoveto{\pgfqpoint{2.900333in}{2.034258in}}%
\pgfpathlineto{\pgfqpoint{2.885224in}{2.107247in}}%
\pgfpathlineto{\pgfqpoint{2.952680in}{2.075540in}}%
\pgfpathlineto{\pgfqpoint{2.900333in}{2.034258in}}%
\pgfpathclose%
\pgfusepath{stroke,fill}%
\end{pgfscope}%
\begin{pgfscope}%
\pgfpathrectangle{\pgfqpoint{0.647939in}{0.492442in}}{\pgfqpoint{3.079299in}{3.079299in}}%
\pgfusepath{clip}%
\pgfsetroundcap%
\pgfsetroundjoin%
\pgfsetlinewidth{0.803000pt}%
\definecolor{currentstroke}{rgb}{0.501961,0.501961,0.501961}%
\pgfsetstrokecolor{currentstroke}%
\pgfsetdash{}{0pt}%
\pgfpathmoveto{\pgfqpoint{3.259600in}{2.825440in}}%
\pgfpathquadraticcurveto{\pgfqpoint{3.260505in}{2.828497in}}{\pgfqpoint{3.257884in}{2.819643in}}%
\pgfusepath{stroke}%
\end{pgfscope}%
\begin{pgfscope}%
\pgfpathrectangle{\pgfqpoint{0.647939in}{0.492442in}}{\pgfqpoint{3.079299in}{3.079299in}}%
\pgfusepath{clip}%
\pgfsetroundcap%
\pgfsetroundjoin%
\definecolor{currentfill}{rgb}{0.501961,0.501961,0.501961}%
\pgfsetfillcolor{currentfill}%
\pgfsetlinewidth{0.803000pt}%
\definecolor{currentstroke}{rgb}{0.501961,0.501961,0.501961}%
\pgfsetstrokecolor{currentstroke}%
\pgfsetdash{}{0pt}%
\pgfpathmoveto{\pgfqpoint{3.206998in}{2.765180in}}%
\pgfpathlineto{\pgfqpoint{3.257884in}{2.819643in}}%
\pgfpathlineto{\pgfqpoint{3.270922in}{2.746256in}}%
\pgfpathlineto{\pgfqpoint{3.206998in}{2.765180in}}%
\pgfpathclose%
\pgfusepath{stroke,fill}%
\end{pgfscope}%
\begin{pgfscope}%
\pgfpathrectangle{\pgfqpoint{0.647939in}{0.492442in}}{\pgfqpoint{3.079299in}{3.079299in}}%
\pgfusepath{clip}%
\pgfsetroundcap%
\pgfsetroundjoin%
\pgfsetlinewidth{0.803000pt}%
\definecolor{currentstroke}{rgb}{0.501961,0.501961,0.501961}%
\pgfsetstrokecolor{currentstroke}%
\pgfsetdash{}{0pt}%
\pgfpathmoveto{\pgfqpoint{1.675569in}{2.165997in}}%
\pgfpathquadraticcurveto{\pgfqpoint{1.678625in}{2.167024in}}{\pgfqpoint{1.669905in}{2.164093in}}%
\pgfusepath{stroke}%
\end{pgfscope}%
\begin{pgfscope}%
\pgfpathrectangle{\pgfqpoint{0.647939in}{0.492442in}}{\pgfqpoint{3.079299in}{3.079299in}}%
\pgfusepath{clip}%
\pgfsetroundcap%
\pgfsetroundjoin%
\definecolor{currentfill}{rgb}{0.501961,0.501961,0.501961}%
\pgfsetfillcolor{currentfill}%
\pgfsetlinewidth{0.803000pt}%
\definecolor{currentstroke}{rgb}{0.501961,0.501961,0.501961}%
\pgfsetstrokecolor{currentstroke}%
\pgfsetdash{}{0pt}%
\pgfpathmoveto{\pgfqpoint{1.596093in}{2.174450in}}%
\pgfpathlineto{\pgfqpoint{1.669905in}{2.164093in}}%
\pgfpathlineto{\pgfqpoint{1.617332in}{2.111257in}}%
\pgfpathlineto{\pgfqpoint{1.596093in}{2.174450in}}%
\pgfpathclose%
\pgfusepath{stroke,fill}%
\end{pgfscope}%
\begin{pgfscope}%
\pgfpathrectangle{\pgfqpoint{0.647939in}{0.492442in}}{\pgfqpoint{3.079299in}{3.079299in}}%
\pgfusepath{clip}%
\pgfsetroundcap%
\pgfsetroundjoin%
\pgfsetlinewidth{0.803000pt}%
\definecolor{currentstroke}{rgb}{0.501961,0.501961,0.501961}%
\pgfsetstrokecolor{currentstroke}%
\pgfsetdash{}{0pt}%
\pgfpathmoveto{\pgfqpoint{2.248050in}{2.897940in}}%
\pgfpathquadraticcurveto{\pgfqpoint{2.251274in}{2.897971in}}{\pgfqpoint{2.242076in}{2.897882in}}%
\pgfusepath{stroke}%
\end{pgfscope}%
\begin{pgfscope}%
\pgfpathrectangle{\pgfqpoint{0.647939in}{0.492442in}}{\pgfqpoint{3.079299in}{3.079299in}}%
\pgfusepath{clip}%
\pgfsetroundcap%
\pgfsetroundjoin%
\definecolor{currentfill}{rgb}{0.501961,0.501961,0.501961}%
\pgfsetfillcolor{currentfill}%
\pgfsetlinewidth{0.803000pt}%
\definecolor{currentstroke}{rgb}{0.501961,0.501961,0.501961}%
\pgfsetstrokecolor{currentstroke}%
\pgfsetdash{}{0pt}%
\pgfpathmoveto{\pgfqpoint{2.175088in}{2.930565in}}%
\pgfpathlineto{\pgfqpoint{2.242076in}{2.897882in}}%
\pgfpathlineto{\pgfqpoint{2.175736in}{2.863902in}}%
\pgfpathlineto{\pgfqpoint{2.175088in}{2.930565in}}%
\pgfpathclose%
\pgfusepath{stroke,fill}%
\end{pgfscope}%
\begin{pgfscope}%
\pgfpathrectangle{\pgfqpoint{0.647939in}{0.492442in}}{\pgfqpoint{3.079299in}{3.079299in}}%
\pgfusepath{clip}%
\pgfsetroundcap%
\pgfsetroundjoin%
\pgfsetlinewidth{0.803000pt}%
\definecolor{currentstroke}{rgb}{0.501961,0.501961,0.501961}%
\pgfsetstrokecolor{currentstroke}%
\pgfsetdash{}{0pt}%
\pgfpathmoveto{\pgfqpoint{2.151736in}{2.936255in}}%
\pgfpathquadraticcurveto{\pgfqpoint{2.154961in}{2.936265in}}{\pgfqpoint{2.145762in}{2.936237in}}%
\pgfusepath{stroke}%
\end{pgfscope}%
\begin{pgfscope}%
\pgfpathrectangle{\pgfqpoint{0.647939in}{0.492442in}}{\pgfqpoint{3.079299in}{3.079299in}}%
\pgfusepath{clip}%
\pgfsetroundcap%
\pgfsetroundjoin%
\definecolor{currentfill}{rgb}{0.501961,0.501961,0.501961}%
\pgfsetfillcolor{currentfill}%
\pgfsetlinewidth{0.803000pt}%
\definecolor{currentstroke}{rgb}{0.501961,0.501961,0.501961}%
\pgfsetstrokecolor{currentstroke}%
\pgfsetdash{}{0pt}%
\pgfpathmoveto{\pgfqpoint{2.078995in}{2.969368in}}%
\pgfpathlineto{\pgfqpoint{2.145762in}{2.936237in}}%
\pgfpathlineto{\pgfqpoint{2.079197in}{2.902702in}}%
\pgfpathlineto{\pgfqpoint{2.078995in}{2.969368in}}%
\pgfpathclose%
\pgfusepath{stroke,fill}%
\end{pgfscope}%
\begin{pgfscope}%
\pgfpathrectangle{\pgfqpoint{0.647939in}{0.492442in}}{\pgfqpoint{3.079299in}{3.079299in}}%
\pgfusepath{clip}%
\pgfsetroundcap%
\pgfsetroundjoin%
\pgfsetlinewidth{0.803000pt}%
\definecolor{currentstroke}{rgb}{0.501961,0.501961,0.501961}%
\pgfsetstrokecolor{currentstroke}%
\pgfsetdash{}{0pt}%
\pgfpathmoveto{\pgfqpoint{2.240311in}{2.613194in}}%
\pgfpathquadraticcurveto{\pgfqpoint{2.243535in}{2.613235in}}{\pgfqpoint{2.234337in}{2.613118in}}%
\pgfusepath{stroke}%
\end{pgfscope}%
\begin{pgfscope}%
\pgfpathrectangle{\pgfqpoint{0.647939in}{0.492442in}}{\pgfqpoint{3.079299in}{3.079299in}}%
\pgfusepath{clip}%
\pgfsetroundcap%
\pgfsetroundjoin%
\definecolor{currentfill}{rgb}{0.501961,0.501961,0.501961}%
\pgfsetfillcolor{currentfill}%
\pgfsetlinewidth{0.803000pt}%
\definecolor{currentstroke}{rgb}{0.501961,0.501961,0.501961}%
\pgfsetstrokecolor{currentstroke}%
\pgfsetdash{}{0pt}%
\pgfpathmoveto{\pgfqpoint{2.167253in}{2.645603in}}%
\pgfpathlineto{\pgfqpoint{2.234337in}{2.613118in}}%
\pgfpathlineto{\pgfqpoint{2.168099in}{2.578941in}}%
\pgfpathlineto{\pgfqpoint{2.167253in}{2.645603in}}%
\pgfpathclose%
\pgfusepath{stroke,fill}%
\end{pgfscope}%
\begin{pgfscope}%
\pgfpathrectangle{\pgfqpoint{0.647939in}{0.492442in}}{\pgfqpoint{3.079299in}{3.079299in}}%
\pgfusepath{clip}%
\pgfsetroundcap%
\pgfsetroundjoin%
\pgfsetlinewidth{0.803000pt}%
\definecolor{currentstroke}{rgb}{0.501961,0.501961,0.501961}%
\pgfsetstrokecolor{currentstroke}%
\pgfsetdash{}{0pt}%
\pgfpathmoveto{\pgfqpoint{2.293530in}{1.259265in}}%
\pgfpathquadraticcurveto{\pgfqpoint{2.290306in}{1.259318in}}{\pgfqpoint{2.299503in}{1.259167in}}%
\pgfusepath{stroke}%
\end{pgfscope}%
\begin{pgfscope}%
\pgfpathrectangle{\pgfqpoint{0.647939in}{0.492442in}}{\pgfqpoint{3.079299in}{3.079299in}}%
\pgfusepath{clip}%
\pgfsetroundcap%
\pgfsetroundjoin%
\definecolor{currentfill}{rgb}{0.501961,0.501961,0.501961}%
\pgfsetfillcolor{currentfill}%
\pgfsetlinewidth{0.803000pt}%
\definecolor{currentstroke}{rgb}{0.501961,0.501961,0.501961}%
\pgfsetstrokecolor{currentstroke}%
\pgfsetdash{}{0pt}%
\pgfpathmoveto{\pgfqpoint{2.365613in}{1.224743in}}%
\pgfpathlineto{\pgfqpoint{2.299503in}{1.259167in}}%
\pgfpathlineto{\pgfqpoint{2.366708in}{1.291400in}}%
\pgfpathlineto{\pgfqpoint{2.365613in}{1.224743in}}%
\pgfpathclose%
\pgfusepath{stroke,fill}%
\end{pgfscope}%
\begin{pgfscope}%
\pgfpathrectangle{\pgfqpoint{0.647939in}{0.492442in}}{\pgfqpoint{3.079299in}{3.079299in}}%
\pgfusepath{clip}%
\pgfsetroundcap%
\pgfsetroundjoin%
\pgfsetlinewidth{0.803000pt}%
\definecolor{currentstroke}{rgb}{0.501961,0.501961,0.501961}%
\pgfsetstrokecolor{currentstroke}%
\pgfsetdash{}{0pt}%
\pgfpathmoveto{\pgfqpoint{1.977015in}{2.509498in}}%
\pgfpathquadraticcurveto{\pgfqpoint{1.980229in}{2.509739in}}{\pgfqpoint{1.971056in}{2.509051in}}%
\pgfusepath{stroke}%
\end{pgfscope}%
\begin{pgfscope}%
\pgfpathrectangle{\pgfqpoint{0.647939in}{0.492442in}}{\pgfqpoint{3.079299in}{3.079299in}}%
\pgfusepath{clip}%
\pgfsetroundcap%
\pgfsetroundjoin%
\definecolor{currentfill}{rgb}{0.501961,0.501961,0.501961}%
\pgfsetfillcolor{currentfill}%
\pgfsetlinewidth{0.803000pt}%
\definecolor{currentstroke}{rgb}{0.501961,0.501961,0.501961}%
\pgfsetstrokecolor{currentstroke}%
\pgfsetdash{}{0pt}%
\pgfpathmoveto{\pgfqpoint{1.902086in}{2.537311in}}%
\pgfpathlineto{\pgfqpoint{1.971056in}{2.509051in}}%
\pgfpathlineto{\pgfqpoint{1.907066in}{2.470831in}}%
\pgfpathlineto{\pgfqpoint{1.902086in}{2.537311in}}%
\pgfpathclose%
\pgfusepath{stroke,fill}%
\end{pgfscope}%
\begin{pgfscope}%
\pgfpathrectangle{\pgfqpoint{0.647939in}{0.492442in}}{\pgfqpoint{3.079299in}{3.079299in}}%
\pgfusepath{clip}%
\pgfsetroundcap%
\pgfsetroundjoin%
\pgfsetlinewidth{0.803000pt}%
\definecolor{currentstroke}{rgb}{0.501961,0.501961,0.501961}%
\pgfsetstrokecolor{currentstroke}%
\pgfsetdash{}{0pt}%
\pgfpathmoveto{\pgfqpoint{2.739987in}{2.211898in}}%
\pgfpathquadraticcurveto{\pgfqpoint{2.738300in}{2.214598in}}{\pgfqpoint{2.743195in}{2.206764in}}%
\pgfusepath{stroke}%
\end{pgfscope}%
\begin{pgfscope}%
\pgfpathrectangle{\pgfqpoint{0.647939in}{0.492442in}}{\pgfqpoint{3.079299in}{3.079299in}}%
\pgfusepath{clip}%
\pgfsetroundcap%
\pgfsetroundjoin%
\definecolor{currentfill}{rgb}{0.501961,0.501961,0.501961}%
\pgfsetfillcolor{currentfill}%
\pgfsetlinewidth{0.803000pt}%
\definecolor{currentstroke}{rgb}{0.501961,0.501961,0.501961}%
\pgfsetstrokecolor{currentstroke}%
\pgfsetdash{}{0pt}%
\pgfpathmoveto{\pgfqpoint{2.750254in}{2.132563in}}%
\pgfpathlineto{\pgfqpoint{2.743195in}{2.206764in}}%
\pgfpathlineto{\pgfqpoint{2.806791in}{2.167890in}}%
\pgfpathlineto{\pgfqpoint{2.750254in}{2.132563in}}%
\pgfpathclose%
\pgfusepath{stroke,fill}%
\end{pgfscope}%
\begin{pgfscope}%
\pgfpathrectangle{\pgfqpoint{0.647939in}{0.492442in}}{\pgfqpoint{3.079299in}{3.079299in}}%
\pgfusepath{clip}%
\pgfsetroundcap%
\pgfsetroundjoin%
\pgfsetlinewidth{0.803000pt}%
\definecolor{currentstroke}{rgb}{0.501961,0.501961,0.501961}%
\pgfsetstrokecolor{currentstroke}%
\pgfsetdash{}{0pt}%
\pgfpathmoveto{\pgfqpoint{2.222900in}{1.460774in}}%
\pgfpathquadraticcurveto{\pgfqpoint{2.219676in}{1.460788in}}{\pgfqpoint{2.228874in}{1.460746in}}%
\pgfusepath{stroke}%
\end{pgfscope}%
\begin{pgfscope}%
\pgfpathrectangle{\pgfqpoint{0.647939in}{0.492442in}}{\pgfqpoint{3.079299in}{3.079299in}}%
\pgfusepath{clip}%
\pgfsetroundcap%
\pgfsetroundjoin%
\definecolor{currentfill}{rgb}{0.501961,0.501961,0.501961}%
\pgfsetfillcolor{currentfill}%
\pgfsetlinewidth{0.803000pt}%
\definecolor{currentstroke}{rgb}{0.501961,0.501961,0.501961}%
\pgfsetstrokecolor{currentstroke}%
\pgfsetdash{}{0pt}%
\pgfpathmoveto{\pgfqpoint{2.295387in}{1.427106in}}%
\pgfpathlineto{\pgfqpoint{2.228874in}{1.460746in}}%
\pgfpathlineto{\pgfqpoint{2.295694in}{1.493772in}}%
\pgfpathlineto{\pgfqpoint{2.295387in}{1.427106in}}%
\pgfpathclose%
\pgfusepath{stroke,fill}%
\end{pgfscope}%
\begin{pgfscope}%
\pgfpathrectangle{\pgfqpoint{0.647939in}{0.492442in}}{\pgfqpoint{3.079299in}{3.079299in}}%
\pgfusepath{clip}%
\pgfsetroundcap%
\pgfsetroundjoin%
\pgfsetlinewidth{0.803000pt}%
\definecolor{currentstroke}{rgb}{0.501961,0.501961,0.501961}%
\pgfsetstrokecolor{currentstroke}%
\pgfsetdash{}{0pt}%
\pgfpathmoveto{\pgfqpoint{2.613562in}{2.057813in}}%
\pgfpathquadraticcurveto{\pgfqpoint{2.610932in}{2.059676in}}{\pgfqpoint{2.618438in}{2.054360in}}%
\pgfusepath{stroke}%
\end{pgfscope}%
\begin{pgfscope}%
\pgfpathrectangle{\pgfqpoint{0.647939in}{0.492442in}}{\pgfqpoint{3.079299in}{3.079299in}}%
\pgfusepath{clip}%
\pgfsetroundcap%
\pgfsetroundjoin%
\definecolor{currentfill}{rgb}{0.501961,0.501961,0.501961}%
\pgfsetfillcolor{currentfill}%
\pgfsetlinewidth{0.803000pt}%
\definecolor{currentstroke}{rgb}{0.501961,0.501961,0.501961}%
\pgfsetstrokecolor{currentstroke}%
\pgfsetdash{}{0pt}%
\pgfpathmoveto{\pgfqpoint{2.653575in}{1.988625in}}%
\pgfpathlineto{\pgfqpoint{2.618438in}{2.054360in}}%
\pgfpathlineto{\pgfqpoint{2.692108in}{2.043028in}}%
\pgfpathlineto{\pgfqpoint{2.653575in}{1.988625in}}%
\pgfpathclose%
\pgfusepath{stroke,fill}%
\end{pgfscope}%
\begin{pgfscope}%
\pgfpathrectangle{\pgfqpoint{0.647939in}{0.492442in}}{\pgfqpoint{3.079299in}{3.079299in}}%
\pgfusepath{clip}%
\pgfsetroundcap%
\pgfsetroundjoin%
\pgfsetlinewidth{0.803000pt}%
\definecolor{currentstroke}{rgb}{0.501961,0.501961,0.501961}%
\pgfsetstrokecolor{currentstroke}%
\pgfsetdash{}{0pt}%
\pgfpathmoveto{\pgfqpoint{2.645354in}{2.230423in}}%
\pgfpathquadraticcurveto{\pgfqpoint{2.643790in}{2.233149in}}{\pgfqpoint{2.648408in}{2.225100in}}%
\pgfusepath{stroke}%
\end{pgfscope}%
\begin{pgfscope}%
\pgfpathrectangle{\pgfqpoint{0.647939in}{0.492442in}}{\pgfqpoint{3.079299in}{3.079299in}}%
\pgfusepath{clip}%
\pgfsetroundcap%
\pgfsetroundjoin%
\definecolor{currentfill}{rgb}{0.501961,0.501961,0.501961}%
\pgfsetfillcolor{currentfill}%
\pgfsetlinewidth{0.803000pt}%
\definecolor{currentstroke}{rgb}{0.501961,0.501961,0.501961}%
\pgfsetstrokecolor{currentstroke}%
\pgfsetdash{}{0pt}%
\pgfpathmoveto{\pgfqpoint{2.652677in}{2.150687in}}%
\pgfpathlineto{\pgfqpoint{2.648408in}{2.225100in}}%
\pgfpathlineto{\pgfqpoint{2.710500in}{2.183868in}}%
\pgfpathlineto{\pgfqpoint{2.652677in}{2.150687in}}%
\pgfpathclose%
\pgfusepath{stroke,fill}%
\end{pgfscope}%
\begin{pgfscope}%
\pgfpathrectangle{\pgfqpoint{0.647939in}{0.492442in}}{\pgfqpoint{3.079299in}{3.079299in}}%
\pgfusepath{clip}%
\pgfsetroundcap%
\pgfsetroundjoin%
\pgfsetlinewidth{0.803000pt}%
\definecolor{currentstroke}{rgb}{0.501961,0.501961,0.501961}%
\pgfsetstrokecolor{currentstroke}%
\pgfsetdash{}{0pt}%
\pgfpathmoveto{\pgfqpoint{2.386754in}{1.965149in}}%
\pgfpathquadraticcurveto{\pgfqpoint{2.383630in}{1.965783in}}{\pgfqpoint{2.392680in}{1.963948in}}%
\pgfusepath{stroke}%
\end{pgfscope}%
\begin{pgfscope}%
\pgfpathrectangle{\pgfqpoint{0.647939in}{0.492442in}}{\pgfqpoint{3.079299in}{3.079299in}}%
\pgfusepath{clip}%
\pgfsetroundcap%
\pgfsetroundjoin%
\definecolor{currentfill}{rgb}{0.501961,0.501961,0.501961}%
\pgfsetfillcolor{currentfill}%
\pgfsetlinewidth{0.803000pt}%
\definecolor{currentstroke}{rgb}{0.501961,0.501961,0.501961}%
\pgfsetstrokecolor{currentstroke}%
\pgfsetdash{}{0pt}%
\pgfpathmoveto{\pgfqpoint{2.451394in}{1.918032in}}%
\pgfpathlineto{\pgfqpoint{2.392680in}{1.963948in}}%
\pgfpathlineto{\pgfqpoint{2.464641in}{1.983369in}}%
\pgfpathlineto{\pgfqpoint{2.451394in}{1.918032in}}%
\pgfpathclose%
\pgfusepath{stroke,fill}%
\end{pgfscope}%
\begin{pgfscope}%
\pgfpathrectangle{\pgfqpoint{0.647939in}{0.492442in}}{\pgfqpoint{3.079299in}{3.079299in}}%
\pgfusepath{clip}%
\pgfsetroundcap%
\pgfsetroundjoin%
\pgfsetlinewidth{0.803000pt}%
\definecolor{currentstroke}{rgb}{0.501961,0.501961,0.501961}%
\pgfsetstrokecolor{currentstroke}%
\pgfsetdash{}{0pt}%
\pgfpathmoveto{\pgfqpoint{2.293239in}{1.674355in}}%
\pgfpathquadraticcurveto{\pgfqpoint{2.290016in}{1.674454in}}{\pgfqpoint{2.299211in}{1.674170in}}%
\pgfusepath{stroke}%
\end{pgfscope}%
\begin{pgfscope}%
\pgfpathrectangle{\pgfqpoint{0.647939in}{0.492442in}}{\pgfqpoint{3.079299in}{3.079299in}}%
\pgfusepath{clip}%
\pgfsetroundcap%
\pgfsetroundjoin%
\definecolor{currentfill}{rgb}{0.501961,0.501961,0.501961}%
\pgfsetfillcolor{currentfill}%
\pgfsetlinewidth{0.803000pt}%
\definecolor{currentstroke}{rgb}{0.501961,0.501961,0.501961}%
\pgfsetstrokecolor{currentstroke}%
\pgfsetdash{}{0pt}%
\pgfpathmoveto{\pgfqpoint{2.364814in}{1.638790in}}%
\pgfpathlineto{\pgfqpoint{2.299211in}{1.674170in}}%
\pgfpathlineto{\pgfqpoint{2.366877in}{1.705424in}}%
\pgfpathlineto{\pgfqpoint{2.364814in}{1.638790in}}%
\pgfpathclose%
\pgfusepath{stroke,fill}%
\end{pgfscope}%
\begin{pgfscope}%
\pgfpathrectangle{\pgfqpoint{0.647939in}{0.492442in}}{\pgfqpoint{3.079299in}{3.079299in}}%
\pgfusepath{clip}%
\pgfsetroundcap%
\pgfsetroundjoin%
\pgfsetlinewidth{0.803000pt}%
\definecolor{currentstroke}{rgb}{0.501961,0.501961,0.501961}%
\pgfsetstrokecolor{currentstroke}%
\pgfsetdash{}{0pt}%
\pgfpathmoveto{\pgfqpoint{2.068711in}{2.339895in}}%
\pgfpathquadraticcurveto{\pgfqpoint{2.071932in}{2.340023in}}{\pgfqpoint{2.062740in}{2.339658in}}%
\pgfusepath{stroke}%
\end{pgfscope}%
\begin{pgfscope}%
\pgfpathrectangle{\pgfqpoint{0.647939in}{0.492442in}}{\pgfqpoint{3.079299in}{3.079299in}}%
\pgfusepath{clip}%
\pgfsetroundcap%
\pgfsetroundjoin%
\definecolor{currentfill}{rgb}{0.501961,0.501961,0.501961}%
\pgfsetfillcolor{currentfill}%
\pgfsetlinewidth{0.803000pt}%
\definecolor{currentstroke}{rgb}{0.501961,0.501961,0.501961}%
\pgfsetstrokecolor{currentstroke}%
\pgfsetdash{}{0pt}%
\pgfpathmoveto{\pgfqpoint{1.994802in}{2.370316in}}%
\pgfpathlineto{\pgfqpoint{2.062740in}{2.339658in}}%
\pgfpathlineto{\pgfqpoint{1.997451in}{2.303701in}}%
\pgfpathlineto{\pgfqpoint{1.994802in}{2.370316in}}%
\pgfpathclose%
\pgfusepath{stroke,fill}%
\end{pgfscope}%
\begin{pgfscope}%
\pgfpathrectangle{\pgfqpoint{0.647939in}{0.492442in}}{\pgfqpoint{3.079299in}{3.079299in}}%
\pgfusepath{clip}%
\pgfsetroundcap%
\pgfsetroundjoin%
\pgfsetlinewidth{0.803000pt}%
\definecolor{currentstroke}{rgb}{0.501961,0.501961,0.501961}%
\pgfsetstrokecolor{currentstroke}%
\pgfsetdash{}{0pt}%
\pgfpathmoveto{\pgfqpoint{2.304659in}{1.844154in}}%
\pgfpathquadraticcurveto{\pgfqpoint{2.301441in}{1.844337in}}{\pgfqpoint{2.310626in}{1.843814in}}%
\pgfusepath{stroke}%
\end{pgfscope}%
\begin{pgfscope}%
\pgfpathrectangle{\pgfqpoint{0.647939in}{0.492442in}}{\pgfqpoint{3.079299in}{3.079299in}}%
\pgfusepath{clip}%
\pgfsetroundcap%
\pgfsetroundjoin%
\definecolor{currentfill}{rgb}{0.501961,0.501961,0.501961}%
\pgfsetfillcolor{currentfill}%
\pgfsetlinewidth{0.803000pt}%
\definecolor{currentstroke}{rgb}{0.501961,0.501961,0.501961}%
\pgfsetstrokecolor{currentstroke}%
\pgfsetdash{}{0pt}%
\pgfpathmoveto{\pgfqpoint{2.375290in}{1.806745in}}%
\pgfpathlineto{\pgfqpoint{2.310626in}{1.843814in}}%
\pgfpathlineto{\pgfqpoint{2.379079in}{1.873304in}}%
\pgfpathlineto{\pgfqpoint{2.375290in}{1.806745in}}%
\pgfpathclose%
\pgfusepath{stroke,fill}%
\end{pgfscope}%
\begin{pgfscope}%
\pgfpathrectangle{\pgfqpoint{0.647939in}{0.492442in}}{\pgfqpoint{3.079299in}{3.079299in}}%
\pgfusepath{clip}%
\pgfsetbuttcap%
\pgfsetroundjoin%
\pgfsetlinewidth{0.301125pt}%
\definecolor{currentstroke}{rgb}{0.500000,0.500000,0.500000}%
\pgfsetstrokecolor{currentstroke}%
\pgfsetstrokeopacity{0.300000}%
\pgfsetdash{}{0pt}%
\pgfpathmoveto{\pgfqpoint{0.647939in}{0.492442in}}%
\pgfpathlineto{\pgfqpoint{0.647939in}{0.492442in}}%
\pgfpathlineto{\pgfqpoint{0.715130in}{0.505310in}}%
\pgfpathlineto{\pgfqpoint{0.781653in}{0.521244in}}%
\pgfpathlineto{\pgfqpoint{0.847232in}{0.540664in}}%
\pgfpathlineto{\pgfqpoint{0.911520in}{0.563975in}}%
\pgfpathlineto{\pgfqpoint{0.974090in}{0.591537in}}%
\pgfpathlineto{\pgfqpoint{1.034454in}{0.623616in}}%
\pgfpathlineto{\pgfqpoint{1.092096in}{0.660338in}}%
\pgfpathlineto{\pgfqpoint{1.146539in}{0.701652in}}%
\pgfpathlineto{\pgfqpoint{1.197429in}{0.747270in}}%
\pgfpathlineto{\pgfqpoint{1.244603in}{0.796695in}}%
\pgfpathlineto{\pgfqpoint{1.288148in}{0.849347in}}%
\pgfpathlineto{\pgfqpoint{1.328381in}{0.904591in}}%
\pgfpathlineto{\pgfqpoint{1.365793in}{0.961807in}}%
\pgfpathlineto{\pgfqpoint{1.400962in}{1.020424in}}%
\pgfpathlineto{\pgfqpoint{1.434484in}{1.079969in}}%
\pgfpathlineto{\pgfqpoint{1.466927in}{1.140084in}}%
\pgfpathlineto{\pgfqpoint{1.498811in}{1.200480in}}%
\pgfpathlineto{\pgfqpoint{1.530595in}{1.260910in}}%
\pgfpathlineto{\pgfqpoint{1.562684in}{1.321158in}}%
\pgfpathlineto{\pgfqpoint{1.595440in}{1.381021in}}%
\pgfpathlineto{\pgfqpoint{1.629199in}{1.440299in}}%
\pgfpathlineto{\pgfqpoint{1.664291in}{1.498786in}}%
\pgfpathlineto{\pgfqpoint{1.701065in}{1.556260in}}%
\pgfpathlineto{\pgfqpoint{1.739917in}{1.612433in}}%
\pgfpathlineto{\pgfqpoint{1.781188in}{1.666679in}}%
\pgfpathlineto{\pgfqpoint{1.825321in}{1.718286in}}%
\pgfpathlineto{\pgfqpoint{1.873253in}{1.766176in}}%
\pgfpathlineto{\pgfqpoint{1.873253in}{1.766176in}}%
\pgfpathlineto{\pgfqpoint{1.915667in}{1.800390in}}%
\pgfpathlineto{\pgfqpoint{1.915667in}{1.800390in}}%
\pgfpathlineto{\pgfqpoint{1.928150in}{1.809940in}}%
\pgfpathlineto{\pgfqpoint{1.934713in}{1.814396in}}%
\pgfpathlineto{\pgfqpoint{1.938081in}{1.816330in}}%
\pgfpathlineto{\pgfqpoint{1.939342in}{1.816680in}}%
\pgfpathlineto{\pgfqpoint{1.939747in}{1.816663in}}%
\pgfpathlineto{\pgfqpoint{1.940580in}{1.817163in}}%
\pgfpathlineto{\pgfqpoint{1.941149in}{1.817795in}}%
\pgfpathlineto{\pgfqpoint{1.940780in}{1.817837in}}%
\pgfpathlineto{\pgfqpoint{1.939825in}{1.817190in}}%
\pgfpathlineto{\pgfqpoint{1.939352in}{1.816550in}}%
\pgfpathlineto{\pgfqpoint{1.939981in}{1.816687in}}%
\pgfpathlineto{\pgfqpoint{1.941041in}{1.817498in}}%
\pgfpathlineto{\pgfqpoint{1.941314in}{1.818068in}}%
\pgfpathlineto{\pgfqpoint{1.940434in}{1.817726in}}%
\pgfpathlineto{\pgfqpoint{1.939318in}{1.816783in}}%
\pgfpathlineto{\pgfqpoint{1.939271in}{1.816297in}}%
\pgfpathlineto{\pgfqpoint{1.940465in}{1.816909in}}%
\pgfpathlineto{\pgfqpoint{1.941544in}{1.817953in}}%
\pgfpathlineto{\pgfqpoint{1.941263in}{1.818249in}}%
\pgfpathlineto{\pgfqpoint{1.939828in}{1.817391in}}%
\pgfpathlineto{\pgfqpoint{1.938803in}{1.816262in}}%
\pgfpathlineto{\pgfqpoint{1.939485in}{1.816197in}}%
\pgfpathlineto{\pgfqpoint{1.941201in}{1.817382in}}%
\pgfpathlineto{\pgfqpoint{1.941974in}{1.818474in}}%
\pgfpathlineto{\pgfqpoint{1.940865in}{1.818225in}}%
\pgfpathlineto{\pgfqpoint{1.938960in}{1.816775in}}%
\pgfpathlineto{\pgfqpoint{1.938429in}{1.815697in}}%
\pgfpathlineto{\pgfqpoint{1.940138in}{1.816390in}}%
\pgfpathlineto{\pgfqpoint{1.942143in}{1.818133in}}%
\pgfpathlineto{\pgfqpoint{1.942163in}{1.818967in}}%
\pgfpathlineto{\pgfqpoint{1.939969in}{1.817844in}}%
\pgfpathlineto{\pgfqpoint{1.937857in}{1.815821in}}%
\pgfpathlineto{\pgfqpoint{1.938430in}{1.815217in}}%
\pgfpathlineto{\pgfqpoint{1.941357in}{1.817050in}}%
\pgfpathlineto{\pgfqpoint{1.943177in}{1.819159in}}%
\pgfusepath{stroke}%
\end{pgfscope}%
\begin{pgfscope}%
\pgfpathrectangle{\pgfqpoint{0.647939in}{0.492442in}}{\pgfqpoint{3.079299in}{3.079299in}}%
\pgfusepath{clip}%
\pgfsetbuttcap%
\pgfsetroundjoin%
\pgfsetlinewidth{0.301125pt}%
\definecolor{currentstroke}{rgb}{0.500000,0.500000,0.500000}%
\pgfsetstrokecolor{currentstroke}%
\pgfsetstrokeopacity{0.300000}%
\pgfsetdash{}{0pt}%
\pgfpathmoveto{\pgfqpoint{0.927875in}{0.492442in}}%
\pgfpathlineto{\pgfqpoint{0.927875in}{0.492442in}}%
\pgfpathlineto{\pgfqpoint{0.989318in}{0.522438in}}%
\pgfpathlineto{\pgfqpoint{1.048136in}{0.557273in}}%
\pgfpathlineto{\pgfqpoint{1.103751in}{0.596980in}}%
\pgfpathlineto{\pgfqpoint{1.155681in}{0.641365in}}%
\pgfpathlineto{\pgfqpoint{1.203648in}{0.690024in}}%
\pgfusepath{stroke}%
\end{pgfscope}%
\begin{pgfscope}%
\pgfpathrectangle{\pgfqpoint{0.647939in}{0.492442in}}{\pgfqpoint{3.079299in}{3.079299in}}%
\pgfusepath{clip}%
\pgfsetbuttcap%
\pgfsetroundjoin%
\pgfsetlinewidth{0.301125pt}%
\definecolor{currentstroke}{rgb}{0.500000,0.500000,0.500000}%
\pgfsetstrokecolor{currentstroke}%
\pgfsetstrokeopacity{0.300000}%
\pgfsetdash{}{0pt}%
\pgfpathmoveto{\pgfqpoint{1.137828in}{0.492442in}}%
\pgfpathlineto{\pgfqpoint{1.137828in}{0.492442in}}%
\pgfpathlineto{\pgfqpoint{1.182280in}{0.544278in}}%
\pgfpathlineto{\pgfqpoint{1.222381in}{0.599547in}}%
\pgfpathlineto{\pgfqpoint{1.258660in}{0.657392in}}%
\pgfpathlineto{\pgfqpoint{1.291824in}{0.717096in}}%
\pgfpathlineto{\pgfqpoint{1.322613in}{0.778067in}}%
\pgfpathlineto{\pgfqpoint{1.351715in}{0.839852in}}%
\pgfusepath{stroke}%
\end{pgfscope}%
\begin{pgfscope}%
\pgfpathrectangle{\pgfqpoint{0.647939in}{0.492442in}}{\pgfqpoint{3.079299in}{3.079299in}}%
\pgfusepath{clip}%
\pgfsetbuttcap%
\pgfsetroundjoin%
\pgfsetlinewidth{0.301125pt}%
\definecolor{currentstroke}{rgb}{0.500000,0.500000,0.500000}%
\pgfsetstrokecolor{currentstroke}%
\pgfsetstrokeopacity{0.300000}%
\pgfsetdash{}{0pt}%
\pgfpathmoveto{\pgfqpoint{1.417764in}{0.492442in}}%
\pgfpathlineto{\pgfqpoint{1.417764in}{0.492442in}}%
\pgfpathlineto{\pgfqpoint{1.417764in}{0.492442in}}%
\pgfpathlineto{\pgfqpoint{1.394247in}{0.542256in}}%
\pgfpathlineto{\pgfqpoint{1.380725in}{0.596109in}}%
\pgfpathlineto{\pgfqpoint{1.375715in}{0.655160in}}%
\pgfpathlineto{\pgfqpoint{1.378786in}{0.722688in}}%
\pgfpathlineto{\pgfqpoint{1.388314in}{0.790115in}}%
\pgfpathlineto{\pgfqpoint{1.402375in}{0.856875in}}%
\pgfpathlineto{\pgfqpoint{1.419754in}{0.922858in}}%
\pgfpathlineto{\pgfqpoint{1.439675in}{0.988140in}}%
\pgfpathlineto{\pgfqpoint{1.461650in}{1.052786in}}%
\pgfusepath{stroke}%
\end{pgfscope}%
\begin{pgfscope}%
\pgfpathrectangle{\pgfqpoint{0.647939in}{0.492442in}}{\pgfqpoint{3.079299in}{3.079299in}}%
\pgfusepath{clip}%
\pgfsetbuttcap%
\pgfsetroundjoin%
\pgfsetlinewidth{0.301125pt}%
\definecolor{currentstroke}{rgb}{0.500000,0.500000,0.500000}%
\pgfsetstrokecolor{currentstroke}%
\pgfsetstrokeopacity{0.300000}%
\pgfsetdash{}{0pt}%
\pgfpathmoveto{\pgfqpoint{1.627716in}{0.492442in}}%
\pgfpathlineto{\pgfqpoint{1.627716in}{0.492442in}}%
\pgfpathlineto{\pgfqpoint{1.564926in}{0.518982in}}%
\pgfpathlineto{\pgfqpoint{1.508450in}{0.556687in}}%
\pgfpathlineto{\pgfqpoint{1.465604in}{0.603279in}}%
\pgfpathlineto{\pgfqpoint{1.438935in}{0.651949in}}%
\pgfpathlineto{\pgfqpoint{1.423598in}{0.703233in}}%
\pgfusepath{stroke}%
\end{pgfscope}%
\begin{pgfscope}%
\pgfpathrectangle{\pgfqpoint{0.647939in}{0.492442in}}{\pgfqpoint{3.079299in}{3.079299in}}%
\pgfusepath{clip}%
\pgfsetbuttcap%
\pgfsetroundjoin%
\pgfsetlinewidth{0.301125pt}%
\definecolor{currentstroke}{rgb}{0.500000,0.500000,0.500000}%
\pgfsetstrokecolor{currentstroke}%
\pgfsetstrokeopacity{0.300000}%
\pgfsetdash{}{0pt}%
\pgfpathmoveto{\pgfqpoint{1.907652in}{0.492442in}}%
\pgfpathlineto{\pgfqpoint{1.907652in}{0.492442in}}%
\pgfpathlineto{\pgfqpoint{1.839292in}{0.495039in}}%
\pgfpathlineto{\pgfqpoint{1.771174in}{0.501178in}}%
\pgfpathlineto{\pgfqpoint{1.703691in}{0.512141in}}%
\pgfpathlineto{\pgfqpoint{1.637666in}{0.529639in}}%
\pgfpathlineto{\pgfqpoint{1.574771in}{0.555935in}}%
\pgfusepath{stroke}%
\end{pgfscope}%
\begin{pgfscope}%
\pgfpathrectangle{\pgfqpoint{0.647939in}{0.492442in}}{\pgfqpoint{3.079299in}{3.079299in}}%
\pgfusepath{clip}%
\pgfsetbuttcap%
\pgfsetroundjoin%
\pgfsetlinewidth{0.301125pt}%
\definecolor{currentstroke}{rgb}{0.500000,0.500000,0.500000}%
\pgfsetstrokecolor{currentstroke}%
\pgfsetstrokeopacity{0.300000}%
\pgfsetdash{}{0pt}%
\pgfpathmoveto{\pgfqpoint{2.719690in}{0.492442in}}%
\pgfpathlineto{\pgfqpoint{2.661312in}{0.497779in}}%
\pgfpathlineto{\pgfqpoint{2.593091in}{0.502989in}}%
\pgfpathlineto{\pgfqpoint{2.524746in}{0.506207in}}%
\pgfpathlineto{\pgfqpoint{2.456337in}{0.507579in}}%
\pgfpathlineto{\pgfqpoint{2.387913in}{0.507327in}}%
\pgfpathlineto{\pgfqpoint{2.319505in}{0.505749in}}%
\pgfpathlineto{\pgfqpoint{2.251125in}{0.503205in}}%
\pgfpathlineto{\pgfqpoint{2.182767in}{0.500106in}}%
\pgfpathlineto{\pgfqpoint{2.114412in}{0.496917in}}%
\pgfpathlineto{\pgfqpoint{2.046040in}{0.494162in}}%
\pgfpathlineto{\pgfqpoint{1.977636in}{0.492442in}}%
\pgfpathlineto{\pgfqpoint{1.977636in}{0.492442in}}%
\pgfusepath{stroke}%
\end{pgfscope}%
\begin{pgfscope}%
\pgfpathrectangle{\pgfqpoint{0.647939in}{0.492442in}}{\pgfqpoint{3.079299in}{3.079299in}}%
\pgfusepath{clip}%
\pgfsetbuttcap%
\pgfsetroundjoin%
\pgfsetlinewidth{0.301125pt}%
\definecolor{currentstroke}{rgb}{0.500000,0.500000,0.500000}%
\pgfsetstrokecolor{currentstroke}%
\pgfsetstrokeopacity{0.300000}%
\pgfsetdash{}{0pt}%
\pgfpathmoveto{\pgfqpoint{2.957413in}{0.492442in}}%
\pgfpathlineto{\pgfqpoint{2.957413in}{0.492442in}}%
\pgfpathlineto{\pgfqpoint{2.890444in}{0.506469in}}%
\pgfpathlineto{\pgfqpoint{2.823127in}{0.518717in}}%
\pgfpathlineto{\pgfqpoint{2.755486in}{0.529013in}}%
\pgfpathlineto{\pgfqpoint{2.687564in}{0.537248in}}%
\pgfpathlineto{\pgfqpoint{2.619420in}{0.543379in}}%
\pgfpathlineto{\pgfqpoint{2.551121in}{0.547439in}}%
\pgfpathlineto{\pgfqpoint{2.482731in}{0.549544in}}%
\pgfpathlineto{\pgfqpoint{2.414308in}{0.549894in}}%
\pgfpathlineto{\pgfqpoint{2.345893in}{0.548765in}}%
\pgfpathlineto{\pgfqpoint{2.277503in}{0.546495in}}%
\pgfpathlineto{\pgfqpoint{2.209142in}{0.543479in}}%
\pgfpathlineto{\pgfqpoint{2.140793in}{0.540172in}}%
\pgfpathlineto{\pgfqpoint{2.072434in}{0.537096in}}%
\pgfpathlineto{\pgfqpoint{2.004045in}{0.534842in}}%
\pgfpathlineto{\pgfqpoint{1.935628in}{0.534072in}}%
\pgfpathlineto{\pgfqpoint{1.867230in}{0.535569in}}%
\pgfpathlineto{\pgfqpoint{1.798992in}{0.540292in}}%
\pgfpathlineto{\pgfqpoint{1.731236in}{0.549471in}}%
\pgfpathlineto{\pgfqpoint{1.664648in}{0.564742in}}%
\pgfusepath{stroke}%
\end{pgfscope}%
\begin{pgfscope}%
\pgfpathrectangle{\pgfqpoint{0.647939in}{0.492442in}}{\pgfqpoint{3.079299in}{3.079299in}}%
\pgfusepath{clip}%
\pgfsetbuttcap%
\pgfsetroundjoin%
\pgfsetlinewidth{0.301125pt}%
\definecolor{currentstroke}{rgb}{0.500000,0.500000,0.500000}%
\pgfsetstrokecolor{currentstroke}%
\pgfsetstrokeopacity{0.300000}%
\pgfsetdash{}{0pt}%
\pgfpathmoveto{\pgfqpoint{3.167366in}{0.492442in}}%
\pgfpathlineto{\pgfqpoint{3.167366in}{0.492442in}}%
\pgfpathlineto{\pgfqpoint{3.101497in}{0.510979in}}%
\pgfpathlineto{\pgfqpoint{3.035353in}{0.528500in}}%
\pgfpathlineto{\pgfqpoint{2.968881in}{0.544730in}}%
\pgfpathlineto{\pgfqpoint{2.902053in}{0.559412in}}%
\pgfpathlineto{\pgfqpoint{2.834860in}{0.572317in}}%
\pgfusepath{stroke}%
\end{pgfscope}%
\begin{pgfscope}%
\pgfpathrectangle{\pgfqpoint{0.647939in}{0.492442in}}{\pgfqpoint{3.079299in}{3.079299in}}%
\pgfusepath{clip}%
\pgfsetbuttcap%
\pgfsetroundjoin%
\pgfsetlinewidth{0.301125pt}%
\definecolor{currentstroke}{rgb}{0.500000,0.500000,0.500000}%
\pgfsetstrokecolor{currentstroke}%
\pgfsetstrokeopacity{0.300000}%
\pgfsetdash{}{0pt}%
\pgfpathmoveto{\pgfqpoint{3.447302in}{0.492442in}}%
\pgfpathlineto{\pgfqpoint{3.447302in}{0.492442in}}%
\pgfpathlineto{\pgfqpoint{3.381966in}{0.512781in}}%
\pgfpathlineto{\pgfqpoint{3.316698in}{0.533339in}}%
\pgfpathlineto{\pgfqpoint{3.251411in}{0.553835in}}%
\pgfpathlineto{\pgfqpoint{3.186015in}{0.573977in}}%
\pgfpathlineto{\pgfqpoint{3.120424in}{0.593473in}}%
\pgfpathlineto{\pgfqpoint{3.054562in}{0.612029in}}%
\pgfpathlineto{\pgfqpoint{2.988368in}{0.629355in}}%
\pgfpathlineto{\pgfqpoint{2.921799in}{0.645175in}}%
\pgfpathlineto{\pgfqpoint{2.854837in}{0.659237in}}%
\pgfpathlineto{\pgfqpoint{2.787493in}{0.671321in}}%
\pgfpathlineto{\pgfqpoint{2.719802in}{0.681267in}}%
\pgfpathlineto{\pgfqpoint{2.651820in}{0.688978in}}%
\pgfpathlineto{\pgfqpoint{2.583620in}{0.694437in}}%
\pgfpathlineto{\pgfqpoint{2.515279in}{0.697714in}}%
\pgfpathlineto{\pgfqpoint{2.446869in}{0.698973in}}%
\pgfpathlineto{\pgfqpoint{2.378447in}{0.698466in}}%
\pgfpathlineto{\pgfqpoint{2.310049in}{0.696522in}}%
\pgfpathlineto{\pgfqpoint{2.241687in}{0.693537in}}%
\pgfpathlineto{\pgfqpoint{2.173351in}{0.689974in}}%
\pgfpathlineto{\pgfqpoint{2.105018in}{0.686369in}}%
\pgfpathlineto{\pgfqpoint{2.036658in}{0.683339in}}%
\pgfpathlineto{\pgfqpoint{1.968257in}{0.681585in}}%
\pgfpathlineto{\pgfqpoint{1.899844in}{0.681940in}}%
\pgfpathlineto{\pgfqpoint{1.831532in}{0.685439in}}%
\pgfpathlineto{\pgfqpoint{1.763629in}{0.693442in}}%
\pgfpathlineto{\pgfqpoint{1.696852in}{0.707809in}}%
\pgfpathlineto{\pgfqpoint{1.632808in}{0.731119in}}%
\pgfpathlineto{\pgfqpoint{1.574983in}{0.766563in}}%
\pgfpathlineto{\pgfqpoint{1.533034in}{0.810476in}}%
\pgfpathlineto{\pgfqpoint{1.507852in}{0.856236in}}%
\pgfpathlineto{\pgfqpoint{1.493885in}{0.904714in}}%
\pgfpathlineto{\pgfqpoint{1.488682in}{0.958032in}}%
\pgfusepath{stroke}%
\end{pgfscope}%
\begin{pgfscope}%
\pgfpathrectangle{\pgfqpoint{0.647939in}{0.492442in}}{\pgfqpoint{3.079299in}{3.079299in}}%
\pgfusepath{clip}%
\pgfsetbuttcap%
\pgfsetroundjoin%
\pgfsetlinewidth{0.301125pt}%
\definecolor{currentstroke}{rgb}{0.500000,0.500000,0.500000}%
\pgfsetstrokecolor{currentstroke}%
\pgfsetstrokeopacity{0.300000}%
\pgfsetdash{}{0pt}%
\pgfpathmoveto{\pgfqpoint{3.727238in}{0.492442in}}%
\pgfpathlineto{\pgfqpoint{3.727238in}{0.492442in}}%
\pgfpathlineto{\pgfqpoint{3.660963in}{0.509459in}}%
\pgfpathlineto{\pgfqpoint{3.595046in}{0.527814in}}%
\pgfpathlineto{\pgfqpoint{3.529446in}{0.547279in}}%
\pgfpathlineto{\pgfqpoint{3.464106in}{0.567603in}}%
\pgfpathlineto{\pgfqpoint{3.398954in}{0.588523in}}%
\pgfpathlineto{\pgfqpoint{3.333906in}{0.609763in}}%
\pgfpathlineto{\pgfqpoint{3.268868in}{0.631038in}}%
\pgfpathlineto{\pgfqpoint{3.203747in}{0.652055in}}%
\pgfpathlineto{\pgfqpoint{3.138450in}{0.672514in}}%
\pgfpathlineto{\pgfqpoint{3.072890in}{0.692112in}}%
\pgfpathlineto{\pgfqpoint{3.006994in}{0.710544in}}%
\pgfpathlineto{\pgfqpoint{2.940710in}{0.727518in}}%
\pgfpathlineto{\pgfqpoint{2.874008in}{0.742758in}}%
\pgfpathlineto{\pgfqpoint{2.806887in}{0.756021in}}%
\pgfpathlineto{\pgfqpoint{2.739375in}{0.767113in}}%
\pgfpathlineto{\pgfqpoint{2.671523in}{0.775898in}}%
\pgfpathlineto{\pgfqpoint{2.603408in}{0.782326in}}%
\pgfpathlineto{\pgfqpoint{2.535114in}{0.786437in}}%
\pgfpathlineto{\pgfqpoint{2.466721in}{0.788368in}}%
\pgfpathlineto{\pgfqpoint{2.398300in}{0.788343in}}%
\pgfpathlineto{\pgfqpoint{2.329896in}{0.786678in}}%
\pgfpathlineto{\pgfqpoint{2.261531in}{0.783771in}}%
\pgfpathlineto{\pgfqpoint{2.193201in}{0.780099in}}%
\pgfpathlineto{\pgfqpoint{2.124883in}{0.776208in}}%
\pgfpathlineto{\pgfqpoint{2.056544in}{0.772714in}}%
\pgfpathlineto{\pgfqpoint{1.988161in}{0.770337in}}%
\pgfpathlineto{\pgfqpoint{1.919746in}{0.769947in}}%
\pgfpathlineto{\pgfqpoint{1.851399in}{0.772641in}}%
\pgfpathlineto{\pgfqpoint{1.783416in}{0.779860in}}%
\pgfpathlineto{\pgfqpoint{1.716522in}{0.793610in}}%
\pgfpathlineto{\pgfqpoint{1.652445in}{0.816760in}}%
\pgfpathlineto{\pgfqpoint{1.595161in}{0.852894in}}%
\pgfpathlineto{\pgfqpoint{1.556062in}{0.896097in}}%
\pgfpathlineto{\pgfqpoint{1.533252in}{0.940939in}}%
\pgfusepath{stroke}%
\end{pgfscope}%
\begin{pgfscope}%
\pgfpathrectangle{\pgfqpoint{0.647939in}{0.492442in}}{\pgfqpoint{3.079299in}{3.079299in}}%
\pgfusepath{clip}%
\pgfsetbuttcap%
\pgfsetroundjoin%
\pgfsetlinewidth{0.301125pt}%
\definecolor{currentstroke}{rgb}{0.500000,0.500000,0.500000}%
\pgfsetstrokecolor{currentstroke}%
\pgfsetstrokeopacity{0.300000}%
\pgfsetdash{}{0pt}%
\pgfpathmoveto{\pgfqpoint{3.727238in}{0.562426in}}%
\pgfpathlineto{\pgfqpoint{3.727238in}{0.562426in}}%
\pgfpathlineto{\pgfqpoint{3.661096in}{0.579950in}}%
\pgfpathlineto{\pgfqpoint{3.595334in}{0.598855in}}%
\pgfpathlineto{\pgfqpoint{3.529912in}{0.618909in}}%
\pgfpathlineto{\pgfqpoint{3.464770in}{0.639858in}}%
\pgfpathlineto{\pgfqpoint{3.399833in}{0.661435in}}%
\pgfpathlineto{\pgfqpoint{3.335012in}{0.683361in}}%
\pgfpathlineto{\pgfqpoint{3.270212in}{0.705347in}}%
\pgfpathlineto{\pgfqpoint{3.205332in}{0.727096in}}%
\pgfpathlineto{\pgfqpoint{3.140274in}{0.748304in}}%
\pgfpathlineto{\pgfqpoint{3.074945in}{0.768660in}}%
\pgfpathlineto{\pgfqpoint{3.009267in}{0.787852in}}%
\pgfpathlineto{\pgfqpoint{2.943178in}{0.805573in}}%
\pgfpathlineto{\pgfqpoint{2.876646in}{0.821534in}}%
\pgfpathlineto{\pgfqpoint{2.809662in}{0.835473in}}%
\pgfpathlineto{\pgfqpoint{2.742254in}{0.847178in}}%
\pgfpathlineto{\pgfqpoint{2.674476in}{0.856495in}}%
\pgfpathlineto{\pgfqpoint{2.606403in}{0.863354in}}%
\pgfpathlineto{\pgfqpoint{2.538129in}{0.867783in}}%
\pgfpathlineto{\pgfqpoint{2.469744in}{0.869914in}}%
\pgfpathlineto{\pgfqpoint{2.401323in}{0.869973in}}%
\pgfpathlineto{\pgfqpoint{2.332921in}{0.868280in}}%
\pgfpathlineto{\pgfqpoint{2.264562in}{0.865245in}}%
\pgfpathlineto{\pgfqpoint{2.196244in}{0.861362in}}%
\pgfpathlineto{\pgfqpoint{2.127941in}{0.857209in}}%
\pgfpathlineto{\pgfqpoint{2.059617in}{0.853438in}}%
\pgfpathlineto{\pgfqpoint{1.991244in}{0.850816in}}%
\pgfpathlineto{\pgfqpoint{1.922830in}{0.850279in}}%
\pgfpathlineto{\pgfqpoint{1.854489in}{0.853039in}}%
\pgfpathlineto{\pgfqpoint{1.786573in}{0.860743in}}%
\pgfpathlineto{\pgfqpoint{1.719992in}{0.875752in}}%
\pgfpathlineto{\pgfqpoint{1.657044in}{0.901490in}}%
\pgfpathlineto{\pgfqpoint{1.657044in}{0.901490in}}%
\pgfpathlineto{\pgfqpoint{1.612709in}{0.932098in}}%
\pgfpathlineto{\pgfqpoint{1.577361in}{0.973822in}}%
\pgfpathlineto{\pgfqpoint{1.556953in}{1.017582in}}%
\pgfpathlineto{\pgfqpoint{1.546872in}{1.064217in}}%
\pgfpathlineto{\pgfqpoint{1.545163in}{1.116248in}}%
\pgfpathlineto{\pgfqpoint{1.551810in}{1.175672in}}%
\pgfusepath{stroke}%
\end{pgfscope}%
\begin{pgfscope}%
\pgfpathrectangle{\pgfqpoint{0.647939in}{0.492442in}}{\pgfqpoint{3.079299in}{3.079299in}}%
\pgfusepath{clip}%
\pgfsetbuttcap%
\pgfsetroundjoin%
\pgfsetlinewidth{0.301125pt}%
\definecolor{currentstroke}{rgb}{0.500000,0.500000,0.500000}%
\pgfsetstrokecolor{currentstroke}%
\pgfsetstrokeopacity{0.300000}%
\pgfsetdash{}{0pt}%
\pgfpathmoveto{\pgfqpoint{3.727238in}{0.632410in}}%
\pgfpathlineto{\pgfqpoint{3.727238in}{0.632410in}}%
\pgfpathlineto{\pgfqpoint{3.661241in}{0.650471in}}%
\pgfpathlineto{\pgfqpoint{3.595650in}{0.669959in}}%
\pgfpathlineto{\pgfqpoint{3.530423in}{0.690637in}}%
\pgfpathlineto{\pgfqpoint{3.465498in}{0.712250in}}%
\pgfpathlineto{\pgfqpoint{3.400797in}{0.734525in}}%
\pgfpathlineto{\pgfqpoint{3.336228in}{0.757182in}}%
\pgfpathlineto{\pgfqpoint{3.271691in}{0.779928in}}%
\pgfpathlineto{\pgfqpoint{3.207079in}{0.802463in}}%
\pgfpathlineto{\pgfqpoint{3.142289in}{0.824476in}}%
\pgfpathlineto{\pgfqpoint{3.077222in}{0.845651in}}%
\pgfpathlineto{\pgfqpoint{3.011790in}{0.865670in}}%
\pgfpathlineto{\pgfqpoint{2.945927in}{0.884210in}}%
\pgfpathlineto{\pgfqpoint{2.879590in}{0.900966in}}%
\pgfpathlineto{\pgfqpoint{2.812768in}{0.915658in}}%
\pgfpathlineto{\pgfqpoint{2.745484in}{0.928050in}}%
\pgfpathlineto{\pgfqpoint{2.677793in}{0.937969in}}%
\pgfpathlineto{\pgfqpoint{2.609774in}{0.945324in}}%
\pgfpathlineto{\pgfqpoint{2.541526in}{0.950124in}}%
\pgfpathlineto{\pgfqpoint{2.473150in}{0.952492in}}%
\pgfpathlineto{\pgfqpoint{2.404730in}{0.952655in}}%
\pgfpathlineto{\pgfqpoint{2.336329in}{0.950942in}}%
\pgfpathlineto{\pgfqpoint{2.267977in}{0.947770in}}%
\pgfpathlineto{\pgfqpoint{2.199673in}{0.943655in}}%
\pgfpathlineto{\pgfqpoint{2.131389in}{0.939201in}}%
\pgfpathlineto{\pgfqpoint{2.063084in}{0.935104in}}%
\pgfpathlineto{\pgfqpoint{1.994723in}{0.932188in}}%
\pgfpathlineto{\pgfqpoint{1.926314in}{0.931469in}}%
\pgfpathlineto{\pgfqpoint{1.857980in}{0.934292in}}%
\pgfpathlineto{\pgfqpoint{1.790146in}{0.942561in}}%
\pgfpathlineto{\pgfqpoint{1.723990in}{0.959105in}}%
\pgfpathlineto{\pgfqpoint{1.662668in}{0.988059in}}%
\pgfpathlineto{\pgfqpoint{1.662668in}{0.988059in}}%
\pgfpathlineto{\pgfqpoint{1.624300in}{1.019420in}}%
\pgfpathlineto{\pgfqpoint{1.595467in}{1.060476in}}%
\pgfusepath{stroke}%
\end{pgfscope}%
\begin{pgfscope}%
\pgfpathrectangle{\pgfqpoint{0.647939in}{0.492442in}}{\pgfqpoint{3.079299in}{3.079299in}}%
\pgfusepath{clip}%
\pgfsetbuttcap%
\pgfsetroundjoin%
\pgfsetlinewidth{0.301125pt}%
\definecolor{currentstroke}{rgb}{0.500000,0.500000,0.500000}%
\pgfsetstrokecolor{currentstroke}%
\pgfsetstrokeopacity{0.300000}%
\pgfsetdash{}{0pt}%
\pgfpathmoveto{\pgfqpoint{3.727238in}{0.702394in}}%
\pgfpathlineto{\pgfqpoint{3.727238in}{0.702394in}}%
\pgfpathlineto{\pgfqpoint{3.661400in}{0.721025in}}%
\pgfpathlineto{\pgfqpoint{3.595996in}{0.741132in}}%
\pgfpathlineto{\pgfqpoint{3.530983in}{0.762473in}}%
\pgfpathlineto{\pgfqpoint{3.466298in}{0.784790in}}%
\pgfpathlineto{\pgfqpoint{3.401858in}{0.807809in}}%
\pgfpathlineto{\pgfqpoint{3.337568in}{0.831246in}}%
\pgfpathlineto{\pgfqpoint{3.273323in}{0.854806in}}%
\pgfpathlineto{\pgfqpoint{3.209012in}{0.878185in}}%
\pgfpathlineto{\pgfqpoint{3.144524in}{0.901068in}}%
\pgfpathlineto{\pgfqpoint{3.079753in}{0.923133in}}%
\pgfpathlineto{\pgfqpoint{3.014604in}{0.944051in}}%
\pgfpathlineto{\pgfqpoint{2.949001in}{0.963492in}}%
\pgfpathlineto{\pgfqpoint{2.882893in}{0.981130in}}%
\pgfpathlineto{\pgfqpoint{2.816262in}{0.996664in}}%
\pgfpathlineto{\pgfqpoint{2.749127in}{1.009834in}}%
\pgfpathlineto{\pgfqpoint{2.681541in}{1.020442in}}%
\pgfpathlineto{\pgfqpoint{2.613589in}{1.028370in}}%
\pgfpathlineto{\pgfqpoint{2.545375in}{1.033606in}}%
\pgfpathlineto{\pgfqpoint{2.477009in}{1.036258in}}%
\pgfpathlineto{\pgfqpoint{2.408592in}{1.036553in}}%
\pgfpathlineto{\pgfqpoint{2.340192in}{1.034825in}}%
\pgfpathlineto{\pgfqpoint{2.271847in}{1.031508in}}%
\pgfpathlineto{\pgfqpoint{2.203560in}{1.027129in}}%
\pgfpathlineto{\pgfqpoint{2.135300in}{1.022325in}}%
\pgfpathlineto{\pgfqpoint{2.067019in}{1.017843in}}%
\pgfpathlineto{\pgfqpoint{1.998675in}{1.014567in}}%
\pgfpathlineto{\pgfqpoint{1.930272in}{1.013613in}}%
\pgfpathlineto{\pgfqpoint{1.861946in}{1.016490in}}%
\pgfpathlineto{\pgfqpoint{1.794216in}{1.025425in}}%
\pgfpathlineto{\pgfqpoint{1.728657in}{1.043890in}}%
\pgfpathlineto{\pgfqpoint{1.728657in}{1.043890in}}%
\pgfpathlineto{\pgfqpoint{1.679677in}{1.069180in}}%
\pgfpathlineto{\pgfqpoint{1.679677in}{1.069180in}}%
\pgfpathlineto{\pgfqpoint{1.644541in}{1.100143in}}%
\pgfpathlineto{\pgfqpoint{1.619161in}{1.140192in}}%
\pgfpathlineto{\pgfqpoint{1.606200in}{1.181953in}}%
\pgfpathlineto{\pgfqpoint{1.602088in}{1.227747in}}%
\pgfpathlineto{\pgfqpoint{1.606187in}{1.279662in}}%
\pgfpathlineto{\pgfqpoint{1.619252in}{1.339897in}}%
\pgfusepath{stroke}%
\end{pgfscope}%
\begin{pgfscope}%
\pgfpathrectangle{\pgfqpoint{0.647939in}{0.492442in}}{\pgfqpoint{3.079299in}{3.079299in}}%
\pgfusepath{clip}%
\pgfsetbuttcap%
\pgfsetroundjoin%
\pgfsetlinewidth{0.301125pt}%
\definecolor{currentstroke}{rgb}{0.500000,0.500000,0.500000}%
\pgfsetstrokecolor{currentstroke}%
\pgfsetstrokeopacity{0.300000}%
\pgfsetdash{}{0pt}%
\pgfpathmoveto{\pgfqpoint{3.727238in}{0.772378in}}%
\pgfpathlineto{\pgfqpoint{3.727238in}{0.772378in}}%
\pgfpathlineto{\pgfqpoint{3.661575in}{0.791615in}}%
\pgfpathlineto{\pgfqpoint{3.596377in}{0.812379in}}%
\pgfpathlineto{\pgfqpoint{3.531600in}{0.834426in}}%
\pgfpathlineto{\pgfqpoint{3.467178in}{0.857493in}}%
\pgfpathlineto{\pgfqpoint{3.403028in}{0.881306in}}%
\pgfpathlineto{\pgfqpoint{3.339048in}{0.905577in}}%
\pgfpathlineto{\pgfqpoint{3.275130in}{0.930010in}}%
\pgfpathlineto{\pgfqpoint{3.211157in}{0.954298in}}%
\pgfpathlineto{\pgfqpoint{3.147012in}{0.978123in}}%
\pgfpathlineto{\pgfqpoint{3.082579in}{1.001157in}}%
\pgfpathlineto{\pgfqpoint{3.017755in}{1.023062in}}%
\pgfpathlineto{\pgfqpoint{2.952455in}{1.043496in}}%
\pgfpathlineto{\pgfqpoint{2.886617in}{1.062117in}}%
\pgfpathlineto{\pgfqpoint{2.820214in}{1.078600in}}%
\pgfpathlineto{\pgfqpoint{2.753260in}{1.092656in}}%
\pgfpathlineto{\pgfqpoint{2.685804in}{1.104056in}}%
\pgfpathlineto{\pgfqpoint{2.617936in}{1.112656in}}%
\pgfpathlineto{\pgfqpoint{2.549767in}{1.118414in}}%
\pgfpathlineto{\pgfqpoint{2.481418in}{1.121415in}}%
\pgfpathlineto{\pgfqpoint{2.413002in}{1.121878in}}%
\pgfpathlineto{\pgfqpoint{2.344604in}{1.120147in}}%
\pgfpathlineto{\pgfqpoint{2.276268in}{1.116670in}}%
\pgfpathlineto{\pgfqpoint{2.208001in}{1.111998in}}%
\pgfpathlineto{\pgfqpoint{2.139770in}{1.106791in}}%
\pgfpathlineto{\pgfqpoint{2.071522in}{1.101845in}}%
\pgfpathlineto{\pgfqpoint{2.003201in}{1.098124in}}%
\pgfpathlineto{\pgfqpoint{1.934806in}{1.096859in}}%
\pgfpathlineto{\pgfqpoint{1.866496in}{1.099775in}}%
\pgfpathlineto{\pgfqpoint{1.798906in}{1.109522in}}%
\pgfpathlineto{\pgfqpoint{1.734218in}{1.130468in}}%
\pgfpathlineto{\pgfqpoint{1.734218in}{1.130468in}}%
\pgfpathlineto{\pgfqpoint{1.692090in}{1.155984in}}%
\pgfusepath{stroke}%
\end{pgfscope}%
\begin{pgfscope}%
\pgfpathrectangle{\pgfqpoint{0.647939in}{0.492442in}}{\pgfqpoint{3.079299in}{3.079299in}}%
\pgfusepath{clip}%
\pgfsetbuttcap%
\pgfsetroundjoin%
\pgfsetlinewidth{0.301125pt}%
\definecolor{currentstroke}{rgb}{0.500000,0.500000,0.500000}%
\pgfsetstrokecolor{currentstroke}%
\pgfsetstrokeopacity{0.300000}%
\pgfsetdash{}{0pt}%
\pgfpathmoveto{\pgfqpoint{3.727238in}{0.842362in}}%
\pgfpathlineto{\pgfqpoint{3.727238in}{0.842362in}}%
\pgfpathlineto{\pgfqpoint{3.661767in}{0.862244in}}%
\pgfpathlineto{\pgfqpoint{3.596797in}{0.883707in}}%
\pgfpathlineto{\pgfqpoint{3.532281in}{0.906506in}}%
\pgfpathlineto{\pgfqpoint{3.468152in}{0.930374in}}%
\pgfpathlineto{\pgfqpoint{3.404323in}{0.955035in}}%
\pgfpathlineto{\pgfqpoint{3.340690in}{0.980200in}}%
\pgfpathlineto{\pgfqpoint{3.277139in}{1.005572in}}%
\pgfpathlineto{\pgfqpoint{3.213548in}{1.030842in}}%
\pgfpathlineto{\pgfqpoint{3.149791in}{1.055690in}}%
\pgfpathlineto{\pgfqpoint{3.085747in}{1.079783in}}%
\pgfpathlineto{\pgfqpoint{3.021301in}{1.102775in}}%
\pgfpathlineto{\pgfqpoint{2.956357in}{1.124311in}}%
\pgfpathlineto{\pgfqpoint{2.890841in}{1.144033in}}%
\pgfpathlineto{\pgfqpoint{2.824715in}{1.161592in}}%
\pgfpathlineto{\pgfqpoint{2.757981in}{1.176666in}}%
\pgfpathlineto{\pgfqpoint{2.690689in}{1.188990in}}%
\pgfpathlineto{\pgfqpoint{2.622928in}{1.198381in}}%
\pgfpathlineto{\pgfqpoint{2.554818in}{1.204767in}}%
\pgfpathlineto{\pgfqpoint{2.486493in}{1.208203in}}%
\pgfpathlineto{\pgfqpoint{2.418081in}{1.208890in}}%
\pgfpathlineto{\pgfqpoint{2.349683in}{1.207173in}}%
\pgfpathlineto{\pgfqpoint{2.281358in}{1.203520in}}%
\pgfpathlineto{\pgfqpoint{2.213115in}{1.198509in}}%
\pgfpathlineto{\pgfqpoint{2.144923in}{1.192833in}}%
\pgfpathlineto{\pgfqpoint{2.076716in}{1.187327in}}%
\pgfpathlineto{\pgfqpoint{2.008429in}{1.183042in}}%
\pgfpathlineto{\pgfqpoint{1.940047in}{1.181359in}}%
\pgfpathlineto{\pgfqpoint{1.871754in}{1.184274in}}%
\pgfpathlineto{\pgfqpoint{1.804388in}{1.195039in}}%
\pgfpathlineto{\pgfqpoint{1.741111in}{1.219321in}}%
\pgfpathlineto{\pgfqpoint{1.741111in}{1.219321in}}%
\pgfpathlineto{\pgfqpoint{1.705896in}{1.245505in}}%
\pgfpathlineto{\pgfqpoint{1.680005in}{1.282105in}}%
\pgfpathlineto{\pgfqpoint{1.667747in}{1.319696in}}%
\pgfpathlineto{\pgfqpoint{1.664478in}{1.360613in}}%
\pgfpathlineto{\pgfqpoint{1.669372in}{1.407116in}}%
\pgfusepath{stroke}%
\end{pgfscope}%
\begin{pgfscope}%
\pgfpathrectangle{\pgfqpoint{0.647939in}{0.492442in}}{\pgfqpoint{3.079299in}{3.079299in}}%
\pgfusepath{clip}%
\pgfsetbuttcap%
\pgfsetroundjoin%
\pgfsetlinewidth{0.301125pt}%
\definecolor{currentstroke}{rgb}{0.500000,0.500000,0.500000}%
\pgfsetstrokecolor{currentstroke}%
\pgfsetstrokeopacity{0.300000}%
\pgfsetdash{}{0pt}%
\pgfpathmoveto{\pgfqpoint{3.727238in}{0.912347in}}%
\pgfpathlineto{\pgfqpoint{3.727238in}{0.912347in}}%
\pgfpathlineto{\pgfqpoint{3.661981in}{0.932915in}}%
\pgfpathlineto{\pgfqpoint{3.597262in}{0.955125in}}%
\pgfpathlineto{\pgfqpoint{3.533035in}{0.978725in}}%
\pgfpathlineto{\pgfqpoint{3.469232in}{1.003449in}}%
\pgfpathlineto{\pgfqpoint{3.405761in}{1.029019in}}%
\pgfpathlineto{\pgfqpoint{3.342517in}{1.055144in}}%
\pgfpathlineto{\pgfqpoint{3.279379in}{1.081528in}}%
\pgfpathlineto{\pgfqpoint{3.216221in}{1.107862in}}%
\pgfpathlineto{\pgfqpoint{3.152909in}{1.133824in}}%
\pgfpathlineto{\pgfqpoint{3.089314in}{1.159078in}}%
\pgfpathlineto{\pgfqpoint{3.025310in}{1.183272in}}%
\pgfpathlineto{\pgfqpoint{2.960788in}{1.206040in}}%
\pgfpathlineto{\pgfqpoint{2.895660in}{1.227005in}}%
\pgfpathlineto{\pgfqpoint{2.829874in}{1.245793in}}%
\pgfpathlineto{\pgfqpoint{2.763420in}{1.262047in}}%
\pgfpathlineto{\pgfqpoint{2.696336in}{1.275458in}}%
\pgfpathlineto{\pgfqpoint{2.628714in}{1.285798in}}%
\pgfpathlineto{\pgfqpoint{2.560683in}{1.292950in}}%
\pgfpathlineto{\pgfqpoint{2.492391in}{1.296936in}}%
\pgfpathlineto{\pgfqpoint{2.423987in}{1.297927in}}%
\pgfpathlineto{\pgfqpoint{2.355590in}{1.296257in}}%
\pgfpathlineto{\pgfqpoint{2.287277in}{1.292411in}}%
\pgfpathlineto{\pgfqpoint{2.219066in}{1.286999in}}%
\pgfpathlineto{\pgfqpoint{2.150923in}{1.280756in}}%
\pgfpathlineto{\pgfqpoint{2.082775in}{1.274569in}}%
\pgfpathlineto{\pgfqpoint{2.014536in}{1.269567in}}%
\pgfpathlineto{\pgfqpoint{1.946173in}{1.267303in}}%
\pgfpathlineto{\pgfqpoint{1.877900in}{1.270145in}}%
\pgfpathlineto{\pgfqpoint{1.810875in}{1.282220in}}%
\pgfpathlineto{\pgfqpoint{1.810875in}{1.282220in}}%
\pgfpathlineto{\pgfqpoint{1.764112in}{1.301254in}}%
\pgfpathlineto{\pgfqpoint{1.764112in}{1.301254in}}%
\pgfusepath{stroke}%
\end{pgfscope}%
\begin{pgfscope}%
\pgfpathrectangle{\pgfqpoint{0.647939in}{0.492442in}}{\pgfqpoint{3.079299in}{3.079299in}}%
\pgfusepath{clip}%
\pgfsetbuttcap%
\pgfsetroundjoin%
\pgfsetlinewidth{0.301125pt}%
\definecolor{currentstroke}{rgb}{0.500000,0.500000,0.500000}%
\pgfsetstrokecolor{currentstroke}%
\pgfsetstrokeopacity{0.300000}%
\pgfsetdash{}{0pt}%
\pgfpathmoveto{\pgfqpoint{3.727238in}{0.982331in}}%
\pgfpathlineto{\pgfqpoint{3.727238in}{0.982331in}}%
\pgfpathlineto{\pgfqpoint{3.662217in}{1.003633in}}%
\pgfpathlineto{\pgfqpoint{3.597778in}{1.026640in}}%
\pgfpathlineto{\pgfqpoint{3.533874in}{1.051098in}}%
\pgfpathlineto{\pgfqpoint{3.470433in}{1.076739in}}%
\pgfpathlineto{\pgfqpoint{3.407364in}{1.103282in}}%
\pgfpathlineto{\pgfqpoint{3.344556in}{1.130441in}}%
\pgfpathlineto{\pgfqpoint{3.281887in}{1.157919in}}%
\pgfpathlineto{\pgfqpoint{3.219223in}{1.185408in}}%
\pgfpathlineto{\pgfqpoint{3.156425in}{1.212587in}}%
\pgfpathlineto{\pgfqpoint{3.093352in}{1.239120in}}%
\pgfpathlineto{\pgfqpoint{3.029869in}{1.264650in}}%
\pgfpathlineto{\pgfqpoint{2.965851in}{1.288801in}}%
\pgfpathlineto{\pgfqpoint{2.901196in}{1.311180in}}%
\pgfpathlineto{\pgfqpoint{2.835834in}{1.331385in}}%
\pgfpathlineto{\pgfqpoint{2.769735in}{1.349023in}}%
\pgfpathlineto{\pgfqpoint{2.702927in}{1.363734in}}%
\pgfpathlineto{\pgfqpoint{2.635493in}{1.375229in}}%
\pgfpathlineto{\pgfqpoint{2.567570in}{1.383334in}}%
\pgfpathlineto{\pgfqpoint{2.499327in}{1.388022in}}%
\pgfpathlineto{\pgfqpoint{2.430935in}{1.389429in}}%
\pgfpathlineto{\pgfqpoint{2.362540in}{1.387867in}}%
\pgfpathlineto{\pgfqpoint{2.294241in}{1.383818in}}%
\pgfpathlineto{\pgfqpoint{2.226071in}{1.377926in}}%
\pgfpathlineto{\pgfqpoint{2.157996in}{1.370978in}}%
\pgfpathlineto{\pgfqpoint{2.089932in}{1.363928in}}%
\pgfpathlineto{\pgfqpoint{2.021768in}{1.358000in}}%
\pgfpathlineto{\pgfqpoint{1.953437in}{1.354924in}}%
\pgfpathlineto{\pgfqpoint{1.885179in}{1.357558in}}%
\pgfpathlineto{\pgfqpoint{1.818701in}{1.371403in}}%
\pgfpathlineto{\pgfqpoint{1.818701in}{1.371403in}}%
\pgfpathlineto{\pgfqpoint{1.780039in}{1.390238in}}%
\pgfpathlineto{\pgfqpoint{1.780039in}{1.390238in}}%
\pgfpathlineto{\pgfqpoint{1.753973in}{1.414933in}}%
\pgfpathlineto{\pgfqpoint{1.737950in}{1.447546in}}%
\pgfpathlineto{\pgfqpoint{1.732798in}{1.481574in}}%
\pgfpathlineto{\pgfqpoint{1.735774in}{1.519157in}}%
\pgfpathlineto{\pgfqpoint{1.747193in}{1.562745in}}%
\pgfusepath{stroke}%
\end{pgfscope}%
\begin{pgfscope}%
\pgfpathrectangle{\pgfqpoint{0.647939in}{0.492442in}}{\pgfqpoint{3.079299in}{3.079299in}}%
\pgfusepath{clip}%
\pgfsetbuttcap%
\pgfsetroundjoin%
\pgfsetlinewidth{0.301125pt}%
\definecolor{currentstroke}{rgb}{0.500000,0.500000,0.500000}%
\pgfsetstrokecolor{currentstroke}%
\pgfsetstrokeopacity{0.300000}%
\pgfsetdash{}{0pt}%
\pgfpathmoveto{\pgfqpoint{3.727238in}{1.052315in}}%
\pgfpathlineto{\pgfqpoint{3.727238in}{1.052315in}}%
\pgfpathlineto{\pgfqpoint{3.662480in}{1.074403in}}%
\pgfpathlineto{\pgfqpoint{3.598353in}{1.098263in}}%
\pgfpathlineto{\pgfqpoint{3.534808in}{1.123639in}}%
\pgfpathlineto{\pgfqpoint{3.471774in}{1.150263in}}%
\pgfpathlineto{\pgfqpoint{3.409156in}{1.177855in}}%
\pgfpathlineto{\pgfqpoint{3.346843in}{1.206129in}}%
\pgfpathlineto{\pgfqpoint{3.284706in}{1.234791in}}%
\pgfpathlineto{\pgfqpoint{3.222609in}{1.263538in}}%
\pgfpathlineto{\pgfqpoint{3.160405in}{1.292052in}}%
\pgfpathlineto{\pgfqpoint{3.097946in}{1.319998in}}%
\pgfpathlineto{\pgfqpoint{3.035083in}{1.347019in}}%
\pgfpathlineto{\pgfqpoint{2.971676in}{1.372732in}}%
\pgfpathlineto{\pgfqpoint{2.907604in}{1.396731in}}%
\pgfpathlineto{\pgfqpoint{2.842775in}{1.418587in}}%
\pgfpathlineto{\pgfqpoint{2.777139in}{1.437866in}}%
\pgfpathlineto{\pgfqpoint{2.710702in}{1.454154in}}%
\pgfpathlineto{\pgfqpoint{2.643534in}{1.467088in}}%
\pgfpathlineto{\pgfqpoint{2.575770in}{1.476408in}}%
\pgfpathlineto{\pgfqpoint{2.507598in}{1.482015in}}%
\pgfpathlineto{\pgfqpoint{2.439224in}{1.483995in}}%
\pgfpathlineto{\pgfqpoint{2.370831in}{1.482630in}}%
\pgfpathlineto{\pgfqpoint{2.302548in}{1.478391in}}%
\pgfpathlineto{\pgfqpoint{2.234432in}{1.471928in}}%
\pgfpathlineto{\pgfqpoint{2.166457in}{1.464079in}}%
\pgfpathlineto{\pgfqpoint{2.098521in}{1.455887in}}%
\pgfpathlineto{\pgfqpoint{2.030479in}{1.448690in}}%
\pgfpathlineto{\pgfqpoint{1.962219in}{1.444454in}}%
\pgfpathlineto{\pgfqpoint{1.893975in}{1.446664in}}%
\pgfpathlineto{\pgfqpoint{1.893975in}{1.446664in}}%
\pgfpathlineto{\pgfqpoint{1.840062in}{1.457987in}}%
\pgfpathlineto{\pgfqpoint{1.840062in}{1.457987in}}%
\pgfpathlineto{\pgfqpoint{1.805729in}{1.475358in}}%
\pgfpathlineto{\pgfqpoint{1.805729in}{1.475358in}}%
\pgfusepath{stroke}%
\end{pgfscope}%
\begin{pgfscope}%
\pgfpathrectangle{\pgfqpoint{0.647939in}{0.492442in}}{\pgfqpoint{3.079299in}{3.079299in}}%
\pgfusepath{clip}%
\pgfsetbuttcap%
\pgfsetroundjoin%
\pgfsetlinewidth{0.301125pt}%
\definecolor{currentstroke}{rgb}{0.500000,0.500000,0.500000}%
\pgfsetstrokecolor{currentstroke}%
\pgfsetstrokeopacity{0.300000}%
\pgfsetdash{}{0pt}%
\pgfpathmoveto{\pgfqpoint{3.727238in}{1.122299in}}%
\pgfpathlineto{\pgfqpoint{3.727238in}{1.122299in}}%
\pgfpathlineto{\pgfqpoint{3.662774in}{1.145229in}}%
\pgfpathlineto{\pgfqpoint{3.598995in}{1.170004in}}%
\pgfpathlineto{\pgfqpoint{3.535854in}{1.196366in}}%
\pgfpathlineto{\pgfqpoint{3.473277in}{1.224046in}}%
\pgfpathlineto{\pgfqpoint{3.411170in}{1.252767in}}%
\pgfpathlineto{\pgfqpoint{3.349418in}{1.282246in}}%
\pgfpathlineto{\pgfqpoint{3.287892in}{1.312195in}}%
\pgfpathlineto{\pgfqpoint{3.226449in}{1.342315in}}%
\pgfpathlineto{\pgfqpoint{3.164940in}{1.372298in}}%
\pgfpathlineto{\pgfqpoint{3.103205in}{1.401813in}}%
\pgfpathlineto{\pgfqpoint{3.041086in}{1.430507in}}%
\pgfpathlineto{\pgfqpoint{2.978427in}{1.457996in}}%
\pgfpathlineto{\pgfqpoint{2.915085in}{1.483864in}}%
\pgfpathlineto{\pgfqpoint{2.850941in}{1.507661in}}%
\pgfpathlineto{\pgfqpoint{2.785916in}{1.528912in}}%
\pgfpathlineto{\pgfqpoint{2.719986in}{1.547139in}}%
\pgfpathlineto{\pgfqpoint{2.653199in}{1.561897in}}%
\pgfpathlineto{\pgfqpoint{2.585684in}{1.572822in}}%
\pgfpathlineto{\pgfqpoint{2.517637in}{1.579694in}}%
\pgfpathlineto{\pgfqpoint{2.449297in}{1.582496in}}%
\pgfpathlineto{\pgfqpoint{2.380903in}{1.581459in}}%
\pgfpathlineto{\pgfqpoint{2.312638in}{1.577042in}}%
\pgfpathlineto{\pgfqpoint{2.244594in}{1.569910in}}%
\pgfpathlineto{\pgfqpoint{2.176763in}{1.560923in}}%
\pgfpathlineto{\pgfqpoint{2.109033in}{1.551174in}}%
\pgfpathlineto{\pgfqpoint{2.041209in}{1.542138in}}%
\pgfpathlineto{\pgfqpoint{1.973095in}{1.536074in}}%
\pgfpathlineto{\pgfqpoint{1.904931in}{1.537396in}}%
\pgfpathlineto{\pgfqpoint{1.904931in}{1.537396in}}%
\pgfpathlineto{\pgfqpoint{1.860581in}{1.547092in}}%
\pgfpathlineto{\pgfqpoint{1.860581in}{1.547092in}}%
\pgfpathlineto{\pgfqpoint{1.832283in}{1.562949in}}%
\pgfpathlineto{\pgfqpoint{1.832283in}{1.562949in}}%
\pgfpathlineto{\pgfqpoint{1.815833in}{1.584103in}}%
\pgfpathlineto{\pgfqpoint{1.809202in}{1.610188in}}%
\pgfpathlineto{\pgfqpoint{1.810977in}{1.638015in}}%
\pgfpathlineto{\pgfqpoint{1.820528in}{1.670083in}}%
\pgfpathlineto{\pgfqpoint{1.839321in}{1.707569in}}%
\pgfusepath{stroke}%
\end{pgfscope}%
\begin{pgfscope}%
\pgfpathrectangle{\pgfqpoint{0.647939in}{0.492442in}}{\pgfqpoint{3.079299in}{3.079299in}}%
\pgfusepath{clip}%
\pgfsetbuttcap%
\pgfsetroundjoin%
\pgfsetlinewidth{0.301125pt}%
\definecolor{currentstroke}{rgb}{0.500000,0.500000,0.500000}%
\pgfsetstrokecolor{currentstroke}%
\pgfsetstrokeopacity{0.300000}%
\pgfsetdash{}{0pt}%
\pgfpathmoveto{\pgfqpoint{3.727238in}{1.192283in}}%
\pgfpathlineto{\pgfqpoint{3.727238in}{1.192283in}}%
\pgfpathlineto{\pgfqpoint{3.663104in}{1.216117in}}%
\pgfpathlineto{\pgfqpoint{3.599716in}{1.241875in}}%
\pgfpathlineto{\pgfqpoint{3.537028in}{1.269298in}}%
\pgfpathlineto{\pgfqpoint{3.474968in}{1.298116in}}%
\pgfpathlineto{\pgfqpoint{3.413439in}{1.328056in}}%
\pgfpathlineto{\pgfqpoint{3.352328in}{1.358841in}}%
\pgfpathlineto{\pgfqpoint{3.291503in}{1.390190in}}%
\pgfpathlineto{\pgfqpoint{3.230821in}{1.421814in}}%
\pgfpathlineto{\pgfqpoint{3.170125in}{1.453412in}}%
\pgfpathlineto{\pgfqpoint{3.109252in}{1.484666in}}%
\pgfpathlineto{\pgfqpoint{3.048033in}{1.515235in}}%
\pgfpathlineto{\pgfqpoint{2.986299in}{1.544743in}}%
\pgfpathlineto{\pgfqpoint{2.923885in}{1.572775in}}%
\pgfpathlineto{\pgfqpoint{2.860640in}{1.598869in}}%
\pgfpathlineto{\pgfqpoint{2.796446in}{1.622517in}}%
\pgfpathlineto{\pgfqpoint{2.731233in}{1.643175in}}%
\pgfpathlineto{\pgfqpoint{2.665011in}{1.660294in}}%
\pgfpathlineto{\pgfqpoint{2.597885in}{1.673377in}}%
\pgfpathlineto{\pgfqpoint{2.530053in}{1.682044in}}%
\pgfpathlineto{\pgfqpoint{2.461792in}{1.686118in}}%
\pgfpathlineto{\pgfqpoint{2.393405in}{1.685683in}}%
\pgfpathlineto{\pgfqpoint{2.325158in}{1.681125in}}%
\pgfpathlineto{\pgfqpoint{2.257218in}{1.673123in}}%
\pgfpathlineto{\pgfqpoint{2.189615in}{1.662589in}}%
\pgfpathlineto{\pgfqpoint{2.122234in}{1.650676in}}%
\pgfpathlineto{\pgfqpoint{2.054825in}{1.638928in}}%
\pgfpathlineto{\pgfqpoint{1.987042in}{1.629811in}}%
\pgfpathlineto{\pgfqpoint{1.918959in}{1.629003in}}%
\pgfpathlineto{\pgfqpoint{1.918959in}{1.629003in}}%
\pgfpathlineto{\pgfqpoint{1.885271in}{1.636134in}}%
\pgfpathlineto{\pgfqpoint{1.885271in}{1.636134in}}%
\pgfusepath{stroke}%
\end{pgfscope}%
\begin{pgfscope}%
\pgfpathrectangle{\pgfqpoint{0.647939in}{0.492442in}}{\pgfqpoint{3.079299in}{3.079299in}}%
\pgfusepath{clip}%
\pgfsetbuttcap%
\pgfsetroundjoin%
\pgfsetlinewidth{0.301125pt}%
\definecolor{currentstroke}{rgb}{0.500000,0.500000,0.500000}%
\pgfsetstrokecolor{currentstroke}%
\pgfsetstrokeopacity{0.300000}%
\pgfsetdash{}{0pt}%
\pgfpathmoveto{\pgfqpoint{3.727238in}{1.262267in}}%
\pgfpathlineto{\pgfqpoint{3.727238in}{1.262267in}}%
\pgfpathlineto{\pgfqpoint{3.663475in}{1.287075in}}%
\pgfpathlineto{\pgfqpoint{3.600529in}{1.313891in}}%
\pgfpathlineto{\pgfqpoint{3.538354in}{1.342455in}}%
\pgfpathlineto{\pgfqpoint{3.476880in}{1.372501in}}%
\pgfpathlineto{\pgfqpoint{3.416011in}{1.403760in}}%
\pgfpathlineto{\pgfqpoint{3.355635in}{1.435961in}}%
\pgfpathlineto{\pgfqpoint{3.295620in}{1.468831in}}%
\pgfpathlineto{\pgfqpoint{3.235821in}{1.502095in}}%
\pgfpathlineto{\pgfqpoint{3.176086in}{1.535471in}}%
\pgfpathlineto{\pgfqpoint{3.116247in}{1.568661in}}%
\pgfpathlineto{\pgfqpoint{3.056132in}{1.601344in}}%
\pgfpathlineto{\pgfqpoint{2.995558in}{1.633165in}}%
\pgfpathlineto{\pgfqpoint{2.934342in}{1.663724in}}%
\pgfpathlineto{\pgfqpoint{2.872303in}{1.692565in}}%
\pgfpathlineto{\pgfqpoint{2.809278in}{1.719165in}}%
\pgfpathlineto{\pgfqpoint{2.745139in}{1.742932in}}%
\pgfpathlineto{\pgfqpoint{2.679823in}{1.763213in}}%
\pgfpathlineto{\pgfqpoint{2.613367in}{1.779333in}}%
\pgfpathlineto{\pgfqpoint{2.545935in}{1.790666in}}%
\pgfpathlineto{\pgfqpoint{2.477835in}{1.796762in}}%
\pgfpathlineto{\pgfqpoint{2.409470in}{1.797467in}}%
\pgfpathlineto{\pgfqpoint{2.341244in}{1.792985in}}%
\pgfpathlineto{\pgfqpoint{2.273464in}{1.783893in}}%
\pgfpathlineto{\pgfqpoint{2.206267in}{1.771098in}}%
\pgfpathlineto{\pgfqpoint{2.139580in}{1.755806in}}%
\pgfpathlineto{\pgfqpoint{2.073098in}{1.739615in}}%
\pgfpathlineto{\pgfqpoint{2.006270in}{1.725044in}}%
\pgfusepath{stroke}%
\end{pgfscope}%
\begin{pgfscope}%
\pgfpathrectangle{\pgfqpoint{0.647939in}{0.492442in}}{\pgfqpoint{3.079299in}{3.079299in}}%
\pgfusepath{clip}%
\pgfsetbuttcap%
\pgfsetroundjoin%
\pgfsetlinewidth{0.301125pt}%
\definecolor{currentstroke}{rgb}{0.500000,0.500000,0.500000}%
\pgfsetstrokecolor{currentstroke}%
\pgfsetstrokeopacity{0.300000}%
\pgfsetdash{}{0pt}%
\pgfpathmoveto{\pgfqpoint{3.727238in}{1.332251in}}%
\pgfpathlineto{\pgfqpoint{3.727238in}{1.332251in}}%
\pgfpathlineto{\pgfqpoint{3.663895in}{1.358109in}}%
\pgfpathlineto{\pgfqpoint{3.601447in}{1.386066in}}%
\pgfpathlineto{\pgfqpoint{3.539855in}{1.415863in}}%
\pgfpathlineto{\pgfqpoint{3.479048in}{1.447237in}}%
\pgfpathlineto{\pgfqpoint{3.418934in}{1.479921in}}%
\pgfpathlineto{\pgfqpoint{3.359402in}{1.513656in}}%
\pgfpathlineto{\pgfqpoint{3.300326in}{1.548186in}}%
\pgfpathlineto{\pgfqpoint{3.241567in}{1.583254in}}%
\pgfpathlineto{\pgfqpoint{3.182976in}{1.618602in}}%
\pgfpathlineto{\pgfqpoint{3.124393in}{1.653961in}}%
\pgfpathlineto{\pgfqpoint{3.065644in}{1.689042in}}%
\pgfpathlineto{\pgfqpoint{3.006547in}{1.723529in}}%
\pgfpathlineto{\pgfqpoint{2.946904in}{1.757061in}}%
\pgfpathlineto{\pgfqpoint{2.886512in}{1.789219in}}%
\pgfpathlineto{\pgfqpoint{2.825163in}{1.819504in}}%
\pgfpathlineto{\pgfqpoint{2.762661in}{1.847315in}}%
\pgfpathlineto{\pgfqpoint{2.698845in}{1.871935in}}%
\pgfpathlineto{\pgfqpoint{2.633632in}{1.892520in}}%
\pgfpathlineto{\pgfqpoint{2.567076in}{1.908137in}}%
\pgfpathlineto{\pgfqpoint{2.499430in}{1.917855in}}%
\pgfpathlineto{\pgfqpoint{2.431175in}{1.920926in}}%
\pgfpathlineto{\pgfqpoint{2.362961in}{1.917022in}}%
\pgfpathlineto{\pgfqpoint{2.295449in}{1.906417in}}%
\pgfpathlineto{\pgfqpoint{2.229086in}{1.889999in}}%
\pgfpathlineto{\pgfqpoint{2.163992in}{1.869067in}}%
\pgfpathlineto{\pgfqpoint{2.099918in}{1.845175in}}%
\pgfpathlineto{\pgfqpoint{2.036221in}{1.820306in}}%
\pgfusepath{stroke}%
\end{pgfscope}%
\begin{pgfscope}%
\pgfpathrectangle{\pgfqpoint{0.647939in}{0.492442in}}{\pgfqpoint{3.079299in}{3.079299in}}%
\pgfusepath{clip}%
\pgfsetbuttcap%
\pgfsetroundjoin%
\pgfsetlinewidth{0.301125pt}%
\definecolor{currentstroke}{rgb}{0.500000,0.500000,0.500000}%
\pgfsetstrokecolor{currentstroke}%
\pgfsetstrokeopacity{0.300000}%
\pgfsetdash{}{0pt}%
\pgfpathmoveto{\pgfqpoint{3.727238in}{1.402235in}}%
\pgfpathlineto{\pgfqpoint{3.727238in}{1.402235in}}%
\pgfpathlineto{\pgfqpoint{3.664371in}{1.429228in}}%
\pgfpathlineto{\pgfqpoint{3.602492in}{1.458418in}}%
\pgfpathlineto{\pgfqpoint{3.541562in}{1.489548in}}%
\pgfpathlineto{\pgfqpoint{3.481518in}{1.522356in}}%
\pgfpathlineto{\pgfqpoint{3.422271in}{1.556585in}}%
\pgfpathlineto{\pgfqpoint{3.363718in}{1.591991in}}%
\pgfpathlineto{\pgfqpoint{3.305743in}{1.628337in}}%
\pgfpathlineto{\pgfqpoint{3.248218in}{1.665392in}}%
\pgfpathlineto{\pgfqpoint{3.191006in}{1.702930in}}%
\pgfpathlineto{\pgfqpoint{3.133961in}{1.740722in}}%
\pgfpathlineto{\pgfqpoint{3.076925in}{1.778525in}}%
\pgfpathlineto{\pgfqpoint{3.019726in}{1.816079in}}%
\pgfpathlineto{\pgfqpoint{2.962182in}{1.853096in}}%
\pgfpathlineto{\pgfqpoint{2.904092in}{1.889248in}}%
\pgfpathlineto{\pgfqpoint{2.845240in}{1.924139in}}%
\pgfpathlineto{\pgfqpoint{2.785388in}{1.957274in}}%
\pgfpathlineto{\pgfqpoint{2.724280in}{1.988011in}}%
\pgfpathlineto{\pgfqpoint{2.661658in}{2.015499in}}%
\pgfpathlineto{\pgfqpoint{2.597309in}{2.038577in}}%
\pgfpathlineto{\pgfqpoint{2.531166in}{2.055668in}}%
\pgfpathlineto{\pgfqpoint{2.463526in}{2.064732in}}%
\pgfpathlineto{\pgfqpoint{2.395375in}{2.063558in}}%
\pgfpathlineto{\pgfqpoint{2.328473in}{2.050773in}}%
\pgfpathlineto{\pgfqpoint{2.264579in}{2.027140in}}%
\pgfpathlineto{\pgfqpoint{2.204301in}{1.995265in}}%
\pgfpathlineto{\pgfqpoint{2.147022in}{1.958206in}}%
\pgfpathlineto{\pgfqpoint{2.091586in}{1.918458in}}%
\pgfpathlineto{\pgfqpoint{2.036950in}{1.878132in}}%
\pgfusepath{stroke}%
\end{pgfscope}%
\begin{pgfscope}%
\pgfpathrectangle{\pgfqpoint{0.647939in}{0.492442in}}{\pgfqpoint{3.079299in}{3.079299in}}%
\pgfusepath{clip}%
\pgfsetbuttcap%
\pgfsetroundjoin%
\pgfsetlinewidth{0.301125pt}%
\definecolor{currentstroke}{rgb}{0.500000,0.500000,0.500000}%
\pgfsetstrokecolor{currentstroke}%
\pgfsetstrokeopacity{0.300000}%
\pgfsetdash{}{0pt}%
\pgfpathmoveto{\pgfqpoint{3.727238in}{1.472219in}}%
\pgfpathlineto{\pgfqpoint{3.727238in}{1.472219in}}%
\pgfpathlineto{\pgfqpoint{3.664915in}{1.500441in}}%
\pgfpathlineto{\pgfqpoint{3.603685in}{1.530966in}}%
\pgfpathlineto{\pgfqpoint{3.543516in}{1.563537in}}%
\pgfpathlineto{\pgfqpoint{3.484349in}{1.597897in}}%
\pgfpathlineto{\pgfqpoint{3.426106in}{1.633806in}}%
\pgfpathlineto{\pgfqpoint{3.368697in}{1.671035in}}%
\pgfpathlineto{\pgfqpoint{3.312019in}{1.709372in}}%
\pgfpathlineto{\pgfqpoint{3.255967in}{1.748618in}}%
\pgfpathlineto{\pgfqpoint{3.200427in}{1.788586in}}%
\pgfpathlineto{\pgfqpoint{3.145278in}{1.829092in}}%
\pgfpathlineto{\pgfqpoint{3.090396in}{1.869959in}}%
\pgfpathlineto{\pgfqpoint{3.035663in}{1.911022in}}%
\pgfpathlineto{\pgfqpoint{2.980950in}{1.952115in}}%
\pgfpathlineto{\pgfqpoint{2.926126in}{1.993056in}}%
\pgfpathlineto{\pgfqpoint{2.871055in}{2.033656in}}%
\pgfpathlineto{\pgfqpoint{2.815594in}{2.073716in}}%
\pgfpathlineto{\pgfqpoint{2.759585in}{2.113008in}}%
\pgfpathlineto{\pgfqpoint{2.702864in}{2.151256in}}%
\pgfpathlineto{\pgfqpoint{2.645250in}{2.188114in}}%
\pgfpathlineto{\pgfqpoint{2.586509in}{2.223096in}}%
\pgfpathlineto{\pgfqpoint{2.526253in}{2.255280in}}%
\pgfpathlineto{\pgfqpoint{2.463429in}{2.281155in}}%
\pgfpathlineto{\pgfqpoint{2.463429in}{2.281155in}}%
\pgfpathlineto{\pgfqpoint{2.440812in}{2.285659in}}%
\pgfpathlineto{\pgfqpoint{2.440812in}{2.285659in}}%
\pgfpathlineto{\pgfqpoint{2.420300in}{2.282416in}}%
\pgfpathlineto{\pgfqpoint{2.401941in}{2.270496in}}%
\pgfpathlineto{\pgfqpoint{2.385136in}{2.255239in}}%
\pgfpathlineto{\pgfqpoint{2.358429in}{2.226239in}}%
\pgfpathlineto{\pgfqpoint{2.314008in}{2.175942in}}%
\pgfpathlineto{\pgfqpoint{2.269271in}{2.125694in}}%
\pgfpathlineto{\pgfqpoint{2.223699in}{2.075587in}}%
\pgfpathlineto{\pgfqpoint{2.177280in}{2.026091in}}%
\pgfusepath{stroke}%
\end{pgfscope}%
\begin{pgfscope}%
\pgfpathrectangle{\pgfqpoint{0.647939in}{0.492442in}}{\pgfqpoint{3.079299in}{3.079299in}}%
\pgfusepath{clip}%
\pgfsetbuttcap%
\pgfsetroundjoin%
\pgfsetlinewidth{0.301125pt}%
\definecolor{currentstroke}{rgb}{0.500000,0.500000,0.500000}%
\pgfsetstrokecolor{currentstroke}%
\pgfsetstrokeopacity{0.300000}%
\pgfsetdash{}{0pt}%
\pgfpathmoveto{\pgfqpoint{3.727238in}{1.612187in}}%
\pgfpathlineto{\pgfqpoint{3.727238in}{1.612187in}}%
\pgfpathlineto{\pgfqpoint{3.666255in}{1.643193in}}%
\pgfpathlineto{\pgfqpoint{3.606629in}{1.676737in}}%
\pgfpathlineto{\pgfqpoint{3.548346in}{1.712567in}}%
\pgfpathlineto{\pgfqpoint{3.491374in}{1.750450in}}%
\pgfpathlineto{\pgfqpoint{3.435671in}{1.790177in}}%
\pgfpathlineto{\pgfqpoint{3.381191in}{1.831569in}}%
\pgfpathlineto{\pgfqpoint{3.327891in}{1.874471in}}%
\pgfpathlineto{\pgfqpoint{3.275734in}{1.918756in}}%
\pgfpathlineto{\pgfqpoint{3.224711in}{1.964342in}}%
\pgfpathlineto{\pgfqpoint{3.174841in}{2.011186in}}%
\pgfpathlineto{\pgfqpoint{3.126185in}{2.059287in}}%
\pgfpathlineto{\pgfqpoint{3.078874in}{2.108711in}}%
\pgfpathlineto{\pgfqpoint{3.033143in}{2.159594in}}%
\pgfpathlineto{\pgfqpoint{2.989387in}{2.212172in}}%
\pgfpathlineto{\pgfqpoint{2.948257in}{2.266822in}}%
\pgfpathlineto{\pgfqpoint{2.910841in}{2.324049in}}%
\pgfpathlineto{\pgfqpoint{2.878897in}{2.384438in}}%
\pgfpathlineto{\pgfqpoint{2.855043in}{2.448352in}}%
\pgfpathlineto{\pgfqpoint{2.842332in}{2.515271in}}%
\pgfpathlineto{\pgfqpoint{2.842537in}{2.583321in}}%
\pgfpathlineto{\pgfqpoint{2.854460in}{2.649708in}}%
\pgfpathlineto{\pgfqpoint{2.875565in}{2.714589in}}%
\pgfpathlineto{\pgfqpoint{2.903005in}{2.777134in}}%
\pgfpathlineto{\pgfqpoint{2.934771in}{2.837649in}}%
\pgfpathlineto{\pgfqpoint{2.969570in}{2.896504in}}%
\pgfpathlineto{\pgfqpoint{3.006601in}{2.954006in}}%
\pgfpathlineto{\pgfqpoint{3.045371in}{3.010373in}}%
\pgfpathlineto{\pgfqpoint{3.085575in}{3.065726in}}%
\pgfpathlineto{\pgfqpoint{3.127031in}{3.120145in}}%
\pgfpathlineto{\pgfqpoint{3.169632in}{3.173678in}}%
\pgfpathlineto{\pgfqpoint{3.213339in}{3.226317in}}%
\pgfpathlineto{\pgfqpoint{3.258139in}{3.278029in}}%
\pgfpathlineto{\pgfqpoint{3.304052in}{3.328755in}}%
\pgfpathlineto{\pgfqpoint{3.351121in}{3.378412in}}%
\pgfpathlineto{\pgfqpoint{3.399405in}{3.426888in}}%
\pgfpathlineto{\pgfqpoint{3.448975in}{3.474047in}}%
\pgfpathlineto{\pgfqpoint{3.499914in}{3.519722in}}%
\pgfpathlineto{\pgfqpoint{3.552312in}{3.563713in}}%
\pgfpathlineto{\pgfqpoint{3.562209in}{3.571741in}}%
\pgfusepath{stroke}%
\end{pgfscope}%
\begin{pgfscope}%
\pgfpathrectangle{\pgfqpoint{0.647939in}{0.492442in}}{\pgfqpoint{3.079299in}{3.079299in}}%
\pgfusepath{clip}%
\pgfsetbuttcap%
\pgfsetroundjoin%
\pgfsetlinewidth{0.301125pt}%
\definecolor{currentstroke}{rgb}{0.500000,0.500000,0.500000}%
\pgfsetstrokecolor{currentstroke}%
\pgfsetstrokeopacity{0.300000}%
\pgfsetdash{}{0pt}%
\pgfpathmoveto{\pgfqpoint{3.727238in}{1.682171in}}%
\pgfpathlineto{\pgfqpoint{3.727238in}{1.682171in}}%
\pgfpathlineto{\pgfqpoint{3.667088in}{1.714756in}}%
\pgfpathlineto{\pgfqpoint{3.608459in}{1.750008in}}%
\pgfpathlineto{\pgfqpoint{3.551355in}{1.787683in}}%
\pgfpathlineto{\pgfqpoint{3.495765in}{1.827562in}}%
\pgfpathlineto{\pgfqpoint{3.441675in}{1.869455in}}%
\pgfpathlineto{\pgfqpoint{3.389078in}{1.913209in}}%
\pgfpathlineto{\pgfqpoint{3.337972in}{1.958698in}}%
\pgfpathlineto{\pgfqpoint{3.288395in}{2.005846in}}%
\pgfpathlineto{\pgfqpoint{3.240424in}{2.054626in}}%
\pgfpathlineto{\pgfqpoint{3.194194in}{2.105057in}}%
\pgfpathlineto{\pgfqpoint{3.149936in}{2.157222in}}%
\pgfpathlineto{\pgfqpoint{3.108006in}{2.211267in}}%
\pgfpathlineto{\pgfqpoint{3.068947in}{2.267415in}}%
\pgfpathlineto{\pgfqpoint{3.033571in}{2.325937in}}%
\pgfpathlineto{\pgfqpoint{3.003044in}{2.387097in}}%
\pgfpathlineto{\pgfqpoint{2.978924in}{2.451013in}}%
\pgfpathlineto{\pgfqpoint{2.962976in}{2.517390in}}%
\pgfpathlineto{\pgfqpoint{2.956611in}{2.585308in}}%
\pgfpathlineto{\pgfqpoint{2.960161in}{2.653422in}}%
\pgfpathlineto{\pgfqpoint{2.972694in}{2.720505in}}%
\pgfpathlineto{\pgfqpoint{2.992564in}{2.785853in}}%
\pgfpathlineto{\pgfqpoint{3.018090in}{2.849254in}}%
\pgfusepath{stroke}%
\end{pgfscope}%
\begin{pgfscope}%
\pgfpathrectangle{\pgfqpoint{0.647939in}{0.492442in}}{\pgfqpoint{3.079299in}{3.079299in}}%
\pgfusepath{clip}%
\pgfsetbuttcap%
\pgfsetroundjoin%
\pgfsetlinewidth{0.301125pt}%
\definecolor{currentstroke}{rgb}{0.500000,0.500000,0.500000}%
\pgfsetstrokecolor{currentstroke}%
\pgfsetstrokeopacity{0.300000}%
\pgfsetdash{}{0pt}%
\pgfpathmoveto{\pgfqpoint{3.727238in}{1.752155in}}%
\pgfpathlineto{\pgfqpoint{3.727238in}{1.752155in}}%
\pgfpathlineto{\pgfqpoint{3.668057in}{1.786462in}}%
\pgfpathlineto{\pgfqpoint{3.610591in}{1.823575in}}%
\pgfpathlineto{\pgfqpoint{3.554866in}{1.863257in}}%
\pgfpathlineto{\pgfqpoint{3.500901in}{1.905304in}}%
\pgfpathlineto{\pgfqpoint{3.448723in}{1.949549in}}%
\pgfpathlineto{\pgfqpoint{3.398364in}{1.995856in}}%
\pgfpathlineto{\pgfqpoint{3.349898in}{2.044139in}}%
\pgfpathlineto{\pgfqpoint{3.303448in}{2.094363in}}%
\pgfpathlineto{\pgfqpoint{3.259207in}{2.146540in}}%
\pgfpathlineto{\pgfqpoint{3.217468in}{2.200735in}}%
\pgfpathlineto{\pgfqpoint{3.178655in}{2.257054in}}%
\pgfpathlineto{\pgfqpoint{3.143381in}{2.315636in}}%
\pgfpathlineto{\pgfqpoint{3.112471in}{2.376612in}}%
\pgfpathlineto{\pgfqpoint{3.086981in}{2.440018in}}%
\pgfpathlineto{\pgfqpoint{3.068123in}{2.505671in}}%
\pgfpathlineto{\pgfqpoint{3.057023in}{2.573048in}}%
\pgfpathlineto{\pgfqpoint{3.054347in}{2.641275in}}%
\pgfpathlineto{\pgfqpoint{3.060015in}{2.709322in}}%
\pgfpathlineto{\pgfqpoint{3.073235in}{2.776332in}}%
\pgfpathlineto{\pgfqpoint{3.092850in}{2.841787in}}%
\pgfpathlineto{\pgfqpoint{3.117680in}{2.905474in}}%
\pgfpathlineto{\pgfqpoint{3.146705in}{2.967382in}}%
\pgfpathlineto{\pgfqpoint{3.179136in}{3.027592in}}%
\pgfpathlineto{\pgfqpoint{3.214394in}{3.086201in}}%
\pgfpathlineto{\pgfqpoint{3.252064in}{3.143298in}}%
\pgfpathlineto{\pgfqpoint{3.291859in}{3.198945in}}%
\pgfpathlineto{\pgfqpoint{3.333598in}{3.253153in}}%
\pgfpathlineto{\pgfqpoint{3.377167in}{3.305898in}}%
\pgfpathlineto{\pgfqpoint{3.422502in}{3.357135in}}%
\pgfpathlineto{\pgfqpoint{3.469588in}{3.406769in}}%
\pgfpathlineto{\pgfqpoint{3.518435in}{3.454669in}}%
\pgfpathlineto{\pgfqpoint{3.569075in}{3.500667in}}%
\pgfusepath{stroke}%
\end{pgfscope}%
\begin{pgfscope}%
\pgfpathrectangle{\pgfqpoint{0.647939in}{0.492442in}}{\pgfqpoint{3.079299in}{3.079299in}}%
\pgfusepath{clip}%
\pgfsetbuttcap%
\pgfsetroundjoin%
\pgfsetlinewidth{0.301125pt}%
\definecolor{currentstroke}{rgb}{0.500000,0.500000,0.500000}%
\pgfsetstrokecolor{currentstroke}%
\pgfsetstrokeopacity{0.300000}%
\pgfsetdash{}{0pt}%
\pgfpathmoveto{\pgfqpoint{3.727238in}{1.892124in}}%
\pgfpathlineto{\pgfqpoint{3.727238in}{1.892124in}}%
\pgfpathlineto{\pgfqpoint{3.670528in}{1.930366in}}%
\pgfpathlineto{\pgfqpoint{3.616038in}{1.971709in}}%
\pgfpathlineto{\pgfqpoint{3.563857in}{2.015934in}}%
\pgfpathlineto{\pgfqpoint{3.514092in}{2.062864in}}%
\pgfpathlineto{\pgfqpoint{3.466875in}{2.112355in}}%
\pgfpathlineto{\pgfqpoint{3.422391in}{2.164314in}}%
\pgfpathlineto{\pgfqpoint{3.380901in}{2.218687in}}%
\pgfpathlineto{\pgfqpoint{3.342758in}{2.275451in}}%
\pgfpathlineto{\pgfqpoint{3.308429in}{2.334592in}}%
\pgfpathlineto{\pgfqpoint{3.278513in}{2.396072in}}%
\pgfpathlineto{\pgfqpoint{3.253724in}{2.459776in}}%
\pgfpathlineto{\pgfqpoint{3.234827in}{2.525450in}}%
\pgfpathlineto{\pgfqpoint{3.222520in}{2.592654in}}%
\pgfpathlineto{\pgfqpoint{3.217276in}{2.660766in}}%
\pgfpathlineto{\pgfqpoint{3.219202in}{2.729060in}}%
\pgfpathlineto{\pgfqpoint{3.228002in}{2.796826in}}%
\pgfpathlineto{\pgfqpoint{3.243073in}{2.863492in}}%
\pgfpathlineto{\pgfqpoint{3.263661in}{2.928677in}}%
\pgfpathlineto{\pgfqpoint{3.289013in}{2.992175in}}%
\pgfpathlineto{\pgfqpoint{3.318461in}{3.053892in}}%
\pgfpathlineto{\pgfqpoint{3.351459in}{3.113796in}}%
\pgfpathlineto{\pgfqpoint{3.387582in}{3.171879in}}%
\pgfpathlineto{\pgfqpoint{3.426528in}{3.228116in}}%
\pgfpathlineto{\pgfqpoint{3.468074in}{3.282461in}}%
\pgfpathlineto{\pgfqpoint{3.512065in}{3.334845in}}%
\pgfpathlineto{\pgfqpoint{3.558422in}{3.385149in}}%
\pgfpathlineto{\pgfqpoint{3.607097in}{3.433212in}}%
\pgfpathlineto{\pgfqpoint{3.658073in}{3.478825in}}%
\pgfpathlineto{\pgfqpoint{3.711346in}{3.521727in}}%
\pgfpathlineto{\pgfqpoint{3.727238in}{3.533841in}}%
\pgfusepath{stroke}%
\end{pgfscope}%
\begin{pgfscope}%
\pgfpathrectangle{\pgfqpoint{0.647939in}{0.492442in}}{\pgfqpoint{3.079299in}{3.079299in}}%
\pgfusepath{clip}%
\pgfsetbuttcap%
\pgfsetroundjoin%
\pgfsetlinewidth{0.301125pt}%
\definecolor{currentstroke}{rgb}{0.500000,0.500000,0.500000}%
\pgfsetstrokecolor{currentstroke}%
\pgfsetstrokeopacity{0.300000}%
\pgfsetdash{}{0pt}%
\pgfpathmoveto{\pgfqpoint{3.727238in}{2.032092in}}%
\pgfpathlineto{\pgfqpoint{3.727238in}{2.032092in}}%
\pgfpathlineto{\pgfqpoint{3.674000in}{2.075017in}}%
\pgfpathlineto{\pgfqpoint{3.623690in}{2.121338in}}%
\pgfpathlineto{\pgfqpoint{3.576504in}{2.170841in}}%
\pgfpathlineto{\pgfqpoint{3.532677in}{2.223337in}}%
\pgfpathlineto{\pgfqpoint{3.492502in}{2.278672in}}%
\pgfpathlineto{\pgfqpoint{3.456355in}{2.336712in}}%
\pgfpathlineto{\pgfqpoint{3.424697in}{2.397310in}}%
\pgfpathlineto{\pgfqpoint{3.398067in}{2.460269in}}%
\pgfpathlineto{\pgfqpoint{3.377046in}{2.525303in}}%
\pgfpathlineto{\pgfqpoint{3.362188in}{2.592003in}}%
\pgfpathlineto{\pgfqpoint{3.353912in}{2.659828in}}%
\pgfpathlineto{\pgfqpoint{3.352413in}{2.728143in}}%
\pgfpathlineto{\pgfqpoint{3.357610in}{2.796286in}}%
\pgfpathlineto{\pgfqpoint{3.369167in}{2.863649in}}%
\pgfpathlineto{\pgfqpoint{3.386565in}{2.929751in}}%
\pgfpathlineto{\pgfqpoint{3.409218in}{2.994251in}}%
\pgfpathlineto{\pgfqpoint{3.436559in}{3.056918in}}%
\pgfpathlineto{\pgfqpoint{3.468082in}{3.117599in}}%
\pgfpathlineto{\pgfqpoint{3.503369in}{3.176179in}}%
\pgfpathlineto{\pgfqpoint{3.542096in}{3.232551in}}%
\pgfpathlineto{\pgfqpoint{3.584035in}{3.286581in}}%
\pgfpathlineto{\pgfqpoint{3.629015in}{3.338104in}}%
\pgfpathlineto{\pgfqpoint{3.676918in}{3.386920in}}%
\pgfusepath{stroke}%
\end{pgfscope}%
\begin{pgfscope}%
\pgfpathrectangle{\pgfqpoint{0.647939in}{0.492442in}}{\pgfqpoint{3.079299in}{3.079299in}}%
\pgfusepath{clip}%
\pgfsetbuttcap%
\pgfsetroundjoin%
\pgfsetlinewidth{0.301125pt}%
\definecolor{currentstroke}{rgb}{0.500000,0.500000,0.500000}%
\pgfsetstrokecolor{currentstroke}%
\pgfsetstrokeopacity{0.300000}%
\pgfsetdash{}{0pt}%
\pgfpathmoveto{\pgfqpoint{3.727238in}{2.172060in}}%
\pgfpathlineto{\pgfqpoint{3.727238in}{2.172060in}}%
\pgfpathlineto{\pgfqpoint{3.678965in}{2.220480in}}%
\pgfpathlineto{\pgfqpoint{3.634614in}{2.272512in}}%
\pgfpathlineto{\pgfqpoint{3.594516in}{2.327889in}}%
\pgfpathlineto{\pgfqpoint{3.559059in}{2.386342in}}%
\pgfpathlineto{\pgfqpoint{3.528697in}{2.447586in}}%
\pgfpathlineto{\pgfqpoint{3.503921in}{2.511288in}}%
\pgfpathlineto{\pgfqpoint{3.485222in}{2.577028in}}%
\pgfpathlineto{\pgfqpoint{3.473018in}{2.644268in}}%
\pgfpathlineto{\pgfqpoint{3.467577in}{2.712380in}}%
\pgfpathlineto{\pgfqpoint{3.468953in}{2.780692in}}%
\pgfpathlineto{\pgfqpoint{3.476976in}{2.848553in}}%
\pgfpathlineto{\pgfqpoint{3.491291in}{2.915384in}}%
\pgfpathlineto{\pgfqpoint{3.511427in}{2.980712in}}%
\pgfpathlineto{\pgfqpoint{3.536882in}{3.044168in}}%
\pgfpathlineto{\pgfqpoint{3.567179in}{3.105466in}}%
\pgfpathlineto{\pgfqpoint{3.601900in}{3.164377in}}%
\pgfpathlineto{\pgfqpoint{3.640709in}{3.220678in}}%
\pgfpathlineto{\pgfqpoint{3.683349in}{3.274132in}}%
\pgfpathlineto{\pgfqpoint{3.727238in}{3.325423in}}%
\pgfusepath{stroke}%
\end{pgfscope}%
\begin{pgfscope}%
\pgfpathrectangle{\pgfqpoint{0.647939in}{0.492442in}}{\pgfqpoint{3.079299in}{3.079299in}}%
\pgfusepath{clip}%
\pgfsetbuttcap%
\pgfsetroundjoin%
\pgfsetlinewidth{0.301125pt}%
\definecolor{currentstroke}{rgb}{0.500000,0.500000,0.500000}%
\pgfsetstrokecolor{currentstroke}%
\pgfsetstrokeopacity{0.300000}%
\pgfsetdash{}{0pt}%
\pgfpathmoveto{\pgfqpoint{3.727238in}{2.312028in}}%
\pgfpathlineto{\pgfqpoint{3.727238in}{2.312028in}}%
\pgfpathlineto{\pgfqpoint{3.686128in}{2.366630in}}%
\pgfpathlineto{\pgfqpoint{3.650246in}{2.424798in}}%
\pgfpathlineto{\pgfqpoint{3.620053in}{2.486110in}}%
\pgfpathlineto{\pgfqpoint{3.596020in}{2.550081in}}%
\pgfpathlineto{\pgfqpoint{3.578588in}{2.616146in}}%
\pgfpathlineto{\pgfqpoint{3.568094in}{2.683654in}}%
\pgfpathlineto{\pgfqpoint{3.564718in}{2.751886in}}%
\pgfpathlineto{\pgfqpoint{3.568436in}{2.820107in}}%
\pgfpathlineto{\pgfqpoint{3.579026in}{2.887618in}}%
\pgfpathlineto{\pgfqpoint{3.596122in}{2.953790in}}%
\pgfpathlineto{\pgfqpoint{3.619269in}{3.018099in}}%
\pgfpathlineto{\pgfqpoint{3.648004in}{3.080120in}}%
\pgfpathlineto{\pgfqpoint{3.681898in}{3.139485in}}%
\pgfpathlineto{\pgfqpoint{3.720584in}{3.195853in}}%
\pgfpathlineto{\pgfqpoint{3.727238in}{3.204751in}}%
\pgfusepath{stroke}%
\end{pgfscope}%
\begin{pgfscope}%
\pgfpathrectangle{\pgfqpoint{0.647939in}{0.492442in}}{\pgfqpoint{3.079299in}{3.079299in}}%
\pgfusepath{clip}%
\pgfsetbuttcap%
\pgfsetroundjoin%
\pgfsetlinewidth{0.301125pt}%
\definecolor{currentstroke}{rgb}{0.500000,0.500000,0.500000}%
\pgfsetstrokecolor{currentstroke}%
\pgfsetstrokeopacity{0.300000}%
\pgfsetdash{}{0pt}%
\pgfpathmoveto{\pgfqpoint{3.727238in}{2.451996in}}%
\pgfpathlineto{\pgfqpoint{3.727238in}{2.451996in}}%
\pgfpathlineto{\pgfqpoint{3.696293in}{2.512916in}}%
\pgfpathlineto{\pgfqpoint{3.672086in}{2.576809in}}%
\pgfpathlineto{\pgfqpoint{3.655037in}{2.642965in}}%
\pgfpathlineto{\pgfqpoint{3.645456in}{2.710596in}}%
\pgfpathlineto{\pgfqpoint{3.643469in}{2.778868in}}%
\pgfpathlineto{\pgfqpoint{3.649007in}{2.846949in}}%
\pgfpathlineto{\pgfqpoint{3.661813in}{2.914057in}}%
\pgfpathlineto{\pgfqpoint{3.681499in}{2.979493in}}%
\pgfpathlineto{\pgfqpoint{3.707616in}{3.042648in}}%
\pgfpathlineto{\pgfqpoint{3.727238in}{3.084253in}}%
\pgfusepath{stroke}%
\end{pgfscope}%
\begin{pgfscope}%
\pgfpathrectangle{\pgfqpoint{0.647939in}{0.492442in}}{\pgfqpoint{3.079299in}{3.079299in}}%
\pgfusepath{clip}%
\pgfsetbuttcap%
\pgfsetroundjoin%
\pgfsetlinewidth{0.301125pt}%
\definecolor{currentstroke}{rgb}{0.500000,0.500000,0.500000}%
\pgfsetstrokecolor{currentstroke}%
\pgfsetstrokeopacity{0.300000}%
\pgfsetdash{}{0pt}%
\pgfpathmoveto{\pgfqpoint{3.727238in}{2.591964in}}%
\pgfpathlineto{\pgfqpoint{3.727238in}{2.591964in}}%
\pgfpathlineto{\pgfqpoint{3.709923in}{2.658018in}}%
\pgfpathlineto{\pgfqpoint{3.700564in}{2.725661in}}%
\pgfpathlineto{\pgfqpoint{3.699274in}{2.793949in}}%
\pgfpathlineto{\pgfqpoint{3.705962in}{2.861926in}}%
\pgfpathlineto{\pgfqpoint{3.720343in}{2.928696in}}%
\pgfpathlineto{\pgfqpoint{3.727238in}{2.953820in}}%
\pgfusepath{stroke}%
\end{pgfscope}%
\begin{pgfscope}%
\pgfpathrectangle{\pgfqpoint{0.647939in}{0.492442in}}{\pgfqpoint{3.079299in}{3.079299in}}%
\pgfusepath{clip}%
\pgfsetbuttcap%
\pgfsetroundjoin%
\pgfsetlinewidth{0.301125pt}%
\definecolor{currentstroke}{rgb}{0.500000,0.500000,0.500000}%
\pgfsetstrokecolor{currentstroke}%
\pgfsetstrokeopacity{0.300000}%
\pgfsetdash{}{0pt}%
\pgfpathmoveto{\pgfqpoint{0.647939in}{2.698133in}}%
\pgfpathlineto{\pgfqpoint{0.700886in}{2.704286in}}%
\pgfpathlineto{\pgfqpoint{0.768775in}{2.712836in}}%
\pgfpathlineto{\pgfqpoint{0.836488in}{2.722685in}}%
\pgfpathlineto{\pgfqpoint{0.904004in}{2.733807in}}%
\pgfpathlineto{\pgfqpoint{0.971310in}{2.746135in}}%
\pgfpathlineto{\pgfqpoint{1.038406in}{2.759559in}}%
\pgfpathlineto{\pgfqpoint{1.105309in}{2.773922in}}%
\pgfpathlineto{\pgfqpoint{1.172050in}{2.789023in}}%
\pgfpathlineto{\pgfqpoint{1.238678in}{2.804619in}}%
\pgfpathlineto{\pgfqpoint{1.305256in}{2.820423in}}%
\pgfpathlineto{\pgfqpoint{1.371862in}{2.836114in}}%
\pgfpathlineto{\pgfqpoint{1.438572in}{2.851347in}}%
\pgfpathlineto{\pgfqpoint{1.505465in}{2.865756in}}%
\pgfpathlineto{\pgfqpoint{1.572603in}{2.878962in}}%
\pgfpathlineto{\pgfqpoint{1.640031in}{2.890586in}}%
\pgfpathlineto{\pgfqpoint{1.707761in}{2.900273in}}%
\pgfpathlineto{\pgfqpoint{1.775773in}{2.907713in}}%
\pgfpathlineto{\pgfqpoint{1.844007in}{2.912677in}}%
\pgfpathlineto{\pgfqpoint{1.912380in}{2.915043in}}%
\pgfpathlineto{\pgfqpoint{1.980795in}{2.914827in}}%
\pgfpathlineto{\pgfqpoint{2.049164in}{2.912214in}}%
\pgfpathlineto{\pgfqpoint{2.117430in}{2.907591in}}%
\pgfpathlineto{\pgfqpoint{2.185590in}{2.901555in}}%
\pgfpathlineto{\pgfqpoint{2.253696in}{2.894922in}}%
\pgfpathlineto{\pgfqpoint{2.321847in}{2.888788in}}%
\pgfpathlineto{\pgfqpoint{2.390131in}{2.884571in}}%
\pgfpathlineto{\pgfqpoint{2.458513in}{2.884056in}}%
\pgfpathlineto{\pgfqpoint{2.526651in}{2.889224in}}%
\pgfpathlineto{\pgfqpoint{2.593773in}{2.901765in}}%
\pgfpathlineto{\pgfqpoint{2.658844in}{2.922419in}}%
\pgfpathlineto{\pgfqpoint{2.721040in}{2.950638in}}%
\pgfpathlineto{\pgfqpoint{2.780115in}{2.984983in}}%
\pgfpathlineto{\pgfqpoint{2.836356in}{3.023845in}}%
\pgfpathlineto{\pgfqpoint{2.890302in}{3.065879in}}%
\pgfpathlineto{\pgfqpoint{2.942517in}{3.110071in}}%
\pgfpathlineto{\pgfqpoint{2.993497in}{3.155687in}}%
\pgfpathlineto{\pgfqpoint{3.043660in}{3.202212in}}%
\pgfpathlineto{\pgfqpoint{3.093349in}{3.249248in}}%
\pgfpathlineto{\pgfqpoint{3.142847in}{3.296486in}}%
\pgfpathlineto{\pgfqpoint{3.192389in}{3.343681in}}%
\pgfpathlineto{\pgfqpoint{3.242182in}{3.390610in}}%
\pgfpathlineto{\pgfqpoint{3.292407in}{3.437079in}}%
\pgfpathlineto{\pgfqpoint{3.343232in}{3.482890in}}%
\pgfpathlineto{\pgfqpoint{3.394817in}{3.527842in}}%
\pgfpathlineto{\pgfqpoint{3.447302in}{3.571741in}}%
\pgfpathlineto{\pgfqpoint{3.447302in}{3.571741in}}%
\pgfusepath{stroke}%
\end{pgfscope}%
\begin{pgfscope}%
\pgfpathrectangle{\pgfqpoint{0.647939in}{0.492442in}}{\pgfqpoint{3.079299in}{3.079299in}}%
\pgfusepath{clip}%
\pgfsetbuttcap%
\pgfsetroundjoin%
\pgfsetlinewidth{0.301125pt}%
\definecolor{currentstroke}{rgb}{0.500000,0.500000,0.500000}%
\pgfsetstrokecolor{currentstroke}%
\pgfsetstrokeopacity{0.300000}%
\pgfsetdash{}{0pt}%
\pgfpathmoveto{\pgfqpoint{0.647939in}{2.970375in}}%
\pgfpathlineto{\pgfqpoint{0.711003in}{2.977505in}}%
\pgfpathlineto{\pgfqpoint{0.778925in}{2.985790in}}%
\pgfpathlineto{\pgfqpoint{0.846693in}{2.995260in}}%
\pgfpathlineto{\pgfqpoint{0.914290in}{3.005876in}}%
\pgfpathlineto{\pgfqpoint{0.981712in}{3.017558in}}%
\pgfpathlineto{\pgfqpoint{1.048965in}{3.030182in}}%
\pgfpathlineto{\pgfqpoint{1.116068in}{3.043581in}}%
\pgfpathlineto{\pgfqpoint{1.183057in}{3.057545in}}%
\pgfpathlineto{\pgfqpoint{1.249979in}{3.071826in}}%
\pgfpathlineto{\pgfqpoint{1.316891in}{3.086150in}}%
\pgfpathlineto{\pgfqpoint{1.383859in}{3.100216in}}%
\pgfpathlineto{\pgfqpoint{1.450944in}{3.113703in}}%
\pgfpathlineto{\pgfqpoint{1.518205in}{3.126278in}}%
\pgfpathlineto{\pgfqpoint{1.585685in}{3.137609in}}%
\pgfpathlineto{\pgfqpoint{1.653406in}{3.147382in}}%
\pgfpathlineto{\pgfqpoint{1.721364in}{3.155323in}}%
\pgfpathlineto{\pgfqpoint{1.789529in}{3.161218in}}%
\pgfpathlineto{\pgfqpoint{1.857847in}{3.164932in}}%
\pgfpathlineto{\pgfqpoint{1.926249in}{3.166446in}}%
\pgfpathlineto{\pgfqpoint{1.994668in}{3.165869in}}%
\pgfpathlineto{\pgfqpoint{2.063048in}{3.163452in}}%
\pgfpathlineto{\pgfqpoint{2.131364in}{3.159588in}}%
\pgfpathlineto{\pgfqpoint{2.199626in}{3.154822in}}%
\pgfpathlineto{\pgfqpoint{2.267875in}{3.149866in}}%
\pgfpathlineto{\pgfqpoint{2.336168in}{3.145604in}}%
\pgfpathlineto{\pgfqpoint{2.404538in}{3.143071in}}%
\pgfpathlineto{\pgfqpoint{2.472940in}{3.143415in}}%
\pgfpathlineto{\pgfqpoint{2.541183in}{3.147818in}}%
\pgfpathlineto{\pgfqpoint{2.608890in}{3.157315in}}%
\pgfpathlineto{\pgfqpoint{2.675525in}{3.172554in}}%
\pgfpathlineto{\pgfqpoint{2.740552in}{3.193617in}}%
\pgfpathlineto{\pgfqpoint{2.803603in}{3.220034in}}%
\pgfpathlineto{\pgfqpoint{2.864569in}{3.250995in}}%
\pgfpathlineto{\pgfqpoint{2.923573in}{3.285583in}}%
\pgfpathlineto{\pgfqpoint{2.980884in}{3.322937in}}%
\pgfpathlineto{\pgfqpoint{3.036834in}{3.362314in}}%
\pgfpathlineto{\pgfqpoint{3.091765in}{3.403102in}}%
\pgfpathlineto{\pgfqpoint{3.146001in}{3.444814in}}%
\pgfpathlineto{\pgfqpoint{3.199834in}{3.487052in}}%
\pgfpathlineto{\pgfqpoint{3.253530in}{3.529464in}}%
\pgfpathlineto{\pgfqpoint{3.307334in}{3.571741in}}%
\pgfpathlineto{\pgfqpoint{3.307334in}{3.571741in}}%
\pgfusepath{stroke}%
\end{pgfscope}%
\begin{pgfscope}%
\pgfpathrectangle{\pgfqpoint{0.647939in}{0.492442in}}{\pgfqpoint{3.079299in}{3.079299in}}%
\pgfusepath{clip}%
\pgfsetbuttcap%
\pgfsetroundjoin%
\pgfsetlinewidth{0.301125pt}%
\definecolor{currentstroke}{rgb}{0.500000,0.500000,0.500000}%
\pgfsetstrokecolor{currentstroke}%
\pgfsetstrokeopacity{0.300000}%
\pgfsetdash{}{0pt}%
\pgfpathmoveto{\pgfqpoint{0.647939in}{3.137982in}}%
\pgfpathlineto{\pgfqpoint{0.713796in}{3.145245in}}%
\pgfpathlineto{\pgfqpoint{0.781745in}{3.153317in}}%
\pgfpathlineto{\pgfqpoint{0.849550in}{3.162513in}}%
\pgfpathlineto{\pgfqpoint{0.917201in}{3.172788in}}%
\pgfpathlineto{\pgfqpoint{0.984693in}{3.184056in}}%
\pgfpathlineto{\pgfqpoint{1.052037in}{3.196189in}}%
\pgfpathlineto{\pgfqpoint{1.119252in}{3.209016in}}%
\pgfpathlineto{\pgfqpoint{1.186374in}{3.222326in}}%
\pgfpathlineto{\pgfqpoint{1.253447in}{3.235880in}}%
\pgfpathlineto{\pgfqpoint{1.320524in}{3.249411in}}%
\pgfpathlineto{\pgfqpoint{1.387664in}{3.262628in}}%
\pgfpathlineto{\pgfqpoint{1.454922in}{3.275228in}}%
\pgfpathlineto{\pgfqpoint{1.522346in}{3.286897in}}%
\pgfpathlineto{\pgfqpoint{1.589970in}{3.297331in}}%
\pgfpathlineto{\pgfqpoint{1.657809in}{3.306252in}}%
\pgfpathlineto{\pgfqpoint{1.725854in}{3.313425in}}%
\pgfpathlineto{\pgfqpoint{1.794072in}{3.318674in}}%
\pgfpathlineto{\pgfqpoint{1.862417in}{3.321909in}}%
\pgfpathlineto{\pgfqpoint{1.930827in}{3.323145in}}%
\pgfpathlineto{\pgfqpoint{1.999246in}{3.322518in}}%
\pgfpathlineto{\pgfqpoint{2.067634in}{3.320279in}}%
\pgfpathlineto{\pgfqpoint{2.135972in}{3.316810in}}%
\pgfpathlineto{\pgfqpoint{2.204272in}{3.312621in}}%
\pgfpathlineto{\pgfqpoint{2.272568in}{3.308361in}}%
\pgfpathlineto{\pgfqpoint{2.340901in}{3.304795in}}%
\pgfpathlineto{\pgfqpoint{2.409292in}{3.302787in}}%
\pgfpathlineto{\pgfqpoint{2.477701in}{3.303271in}}%
\pgfpathlineto{\pgfqpoint{2.545988in}{3.307183in}}%
\pgfpathlineto{\pgfqpoint{2.613885in}{3.315354in}}%
\pgfpathlineto{\pgfqpoint{2.681012in}{3.328348in}}%
\pgfpathlineto{\pgfqpoint{2.746971in}{3.346344in}}%
\pgfpathlineto{\pgfqpoint{2.811437in}{3.369128in}}%
\pgfpathlineto{\pgfqpoint{2.874243in}{3.396179in}}%
\pgfpathlineto{\pgfqpoint{2.935394in}{3.426814in}}%
\pgfpathlineto{\pgfqpoint{2.995033in}{3.460316in}}%
\pgfpathlineto{\pgfqpoint{3.053392in}{3.496015in}}%
\pgfpathlineto{\pgfqpoint{3.110743in}{3.533324in}}%
\pgfpathlineto{\pgfqpoint{3.167366in}{3.571741in}}%
\pgfpathlineto{\pgfqpoint{3.167366in}{3.571741in}}%
\pgfusepath{stroke}%
\end{pgfscope}%
\begin{pgfscope}%
\pgfpathrectangle{\pgfqpoint{0.647939in}{0.492442in}}{\pgfqpoint{3.079299in}{3.079299in}}%
\pgfusepath{clip}%
\pgfsetbuttcap%
\pgfsetroundjoin%
\pgfsetlinewidth{0.301125pt}%
\definecolor{currentstroke}{rgb}{0.500000,0.500000,0.500000}%
\pgfsetstrokecolor{currentstroke}%
\pgfsetstrokeopacity{0.300000}%
\pgfsetdash{}{0pt}%
\pgfpathmoveto{\pgfqpoint{0.647939in}{3.254394in}}%
\pgfpathlineto{\pgfqpoint{0.675401in}{3.257106in}}%
\pgfpathlineto{\pgfqpoint{0.743440in}{3.264386in}}%
\pgfpathlineto{\pgfqpoint{0.811351in}{3.272764in}}%
\pgfpathlineto{\pgfqpoint{0.879122in}{3.282211in}}%
\pgfpathlineto{\pgfqpoint{0.946747in}{3.292660in}}%
\pgfpathlineto{\pgfqpoint{1.014227in}{3.304006in}}%
\pgfpathlineto{\pgfqpoint{1.081577in}{3.316102in}}%
\pgfpathlineto{\pgfqpoint{1.148824in}{3.328763in}}%
\pgfpathlineto{\pgfqpoint{1.216006in}{3.341767in}}%
\pgfpathlineto{\pgfqpoint{1.283169in}{3.354864in}}%
\pgfpathlineto{\pgfqpoint{1.350367in}{3.367784in}}%
\pgfpathlineto{\pgfqpoint{1.417652in}{3.380239in}}%
\pgfpathlineto{\pgfqpoint{1.485073in}{3.391934in}}%
\pgfpathlineto{\pgfqpoint{1.552666in}{3.402573in}}%
\pgfpathlineto{\pgfqpoint{1.620455in}{3.411881in}}%
\pgfpathlineto{\pgfqpoint{1.688439in}{3.419616in}}%
\pgfpathlineto{\pgfqpoint{1.756599in}{3.425588in}}%
\pgfpathlineto{\pgfqpoint{1.824898in}{3.429679in}}%
\pgfpathlineto{\pgfqpoint{1.893285in}{3.431858in}}%
\pgfpathlineto{\pgfqpoint{1.961707in}{3.432205in}}%
\pgfpathlineto{\pgfqpoint{2.030118in}{3.430911in}}%
\pgfpathlineto{\pgfqpoint{2.098494in}{3.428278in}}%
\pgfpathlineto{\pgfqpoint{2.166829in}{3.424724in}}%
\pgfpathlineto{\pgfqpoint{2.235144in}{3.420779in}}%
\pgfpathlineto{\pgfqpoint{2.303473in}{3.417088in}}%
\pgfpathlineto{\pgfqpoint{2.371844in}{3.414392in}}%
\pgfpathlineto{\pgfqpoint{2.440256in}{3.413491in}}%
\pgfpathlineto{\pgfqpoint{2.508644in}{3.415213in}}%
\pgfpathlineto{\pgfqpoint{2.576853in}{3.420347in}}%
\pgfpathlineto{\pgfqpoint{2.644623in}{3.429541in}}%
\pgfpathlineto{\pgfqpoint{2.711630in}{3.443191in}}%
\pgfpathlineto{\pgfqpoint{2.777550in}{3.461367in}}%
\pgfpathlineto{\pgfqpoint{2.842140in}{3.483834in}}%
\pgfpathlineto{\pgfqpoint{2.905281in}{3.510120in}}%
\pgfpathlineto{\pgfqpoint{2.966991in}{3.539631in}}%
\pgfpathlineto{\pgfqpoint{3.027398in}{3.571741in}}%
\pgfpathlineto{\pgfqpoint{3.027398in}{3.571741in}}%
\pgfusepath{stroke}%
\end{pgfscope}%
\begin{pgfscope}%
\pgfpathrectangle{\pgfqpoint{0.647939in}{0.492442in}}{\pgfqpoint{3.079299in}{3.079299in}}%
\pgfusepath{clip}%
\pgfsetbuttcap%
\pgfsetroundjoin%
\pgfsetlinewidth{0.301125pt}%
\definecolor{currentstroke}{rgb}{0.500000,0.500000,0.500000}%
\pgfsetstrokecolor{currentstroke}%
\pgfsetstrokeopacity{0.300000}%
\pgfsetdash{}{0pt}%
\pgfpathmoveto{\pgfqpoint{0.647939in}{3.333566in}}%
\pgfpathlineto{\pgfqpoint{0.653634in}{3.334092in}}%
\pgfpathlineto{\pgfqpoint{0.721719in}{3.340926in}}%
\pgfpathlineto{\pgfqpoint{0.789686in}{3.348836in}}%
\pgfpathlineto{\pgfqpoint{0.857522in}{3.357806in}}%
\pgfpathlineto{\pgfqpoint{0.925219in}{3.367779in}}%
\pgfpathlineto{\pgfqpoint{0.992775in}{3.378659in}}%
\pgfpathlineto{\pgfqpoint{1.060204in}{3.390309in}}%
\pgfpathlineto{\pgfqpoint{1.127528in}{3.402554in}}%
\pgfpathlineto{\pgfqpoint{1.194780in}{3.415190in}}%
\pgfpathlineto{\pgfqpoint{1.262003in}{3.427982in}}%
\pgfpathlineto{\pgfqpoint{1.329245in}{3.440669in}}%
\pgfpathlineto{\pgfqpoint{1.396558in}{3.452972in}}%
\pgfpathlineto{\pgfqpoint{1.463990in}{3.464602in}}%
\pgfpathlineto{\pgfqpoint{1.531579in}{3.475271in}}%
\pgfpathlineto{\pgfqpoint{1.599350in}{3.484710in}}%
\pgfpathlineto{\pgfqpoint{1.667308in}{3.492677in}}%
\pgfpathlineto{\pgfqpoint{1.735440in}{3.498976in}}%
\pgfpathlineto{\pgfqpoint{1.803714in}{3.503478in}}%
\pgfpathlineto{\pgfqpoint{1.872084in}{3.506138in}}%
\pgfpathlineto{\pgfqpoint{1.940501in}{3.507005in}}%
\pgfpathlineto{\pgfqpoint{2.008920in}{3.506232in}}%
\pgfpathlineto{\pgfqpoint{2.077312in}{3.504077in}}%
\pgfpathlineto{\pgfqpoint{2.145667in}{3.500919in}}%
\pgfpathlineto{\pgfqpoint{2.213997in}{3.497244in}}%
\pgfpathlineto{\pgfqpoint{2.282330in}{3.493627in}}%
\pgfpathlineto{\pgfqpoint{2.350694in}{3.490726in}}%
\pgfpathlineto{\pgfqpoint{2.419101in}{3.489270in}}%
\pgfpathlineto{\pgfqpoint{2.487512in}{3.490031in}}%
\pgfpathlineto{\pgfqpoint{2.555816in}{3.493761in}}%
\pgfpathlineto{\pgfqpoint{2.623816in}{3.501106in}}%
\pgfpathlineto{\pgfqpoint{2.691246in}{3.512520in}}%
\pgfpathlineto{\pgfqpoint{2.757813in}{3.528196in}}%
\pgfpathlineto{\pgfqpoint{2.823262in}{3.548047in}}%
\pgfpathlineto{\pgfqpoint{2.887429in}{3.571741in}}%
\pgfpathlineto{\pgfqpoint{2.887429in}{3.571741in}}%
\pgfusepath{stroke}%
\end{pgfscope}%
\begin{pgfscope}%
\pgfpathrectangle{\pgfqpoint{0.647939in}{0.492442in}}{\pgfqpoint{3.079299in}{3.079299in}}%
\pgfusepath{clip}%
\pgfsetbuttcap%
\pgfsetroundjoin%
\pgfsetlinewidth{0.301125pt}%
\definecolor{currentstroke}{rgb}{0.500000,0.500000,0.500000}%
\pgfsetstrokecolor{currentstroke}%
\pgfsetstrokeopacity{0.300000}%
\pgfsetdash{}{0pt}%
\pgfpathmoveto{\pgfqpoint{1.656347in}{3.539113in}}%
\pgfpathlineto{\pgfqpoint{1.724467in}{3.545548in}}%
\pgfpathlineto{\pgfqpoint{1.792728in}{3.550234in}}%
\pgfpathlineto{\pgfqpoint{1.861090in}{3.553114in}}%
\pgfpathlineto{\pgfqpoint{1.929504in}{3.554227in}}%
\pgfpathlineto{\pgfqpoint{1.997926in}{3.553717in}}%
\pgfpathlineto{\pgfqpoint{2.066325in}{3.551826in}}%
\pgfpathlineto{\pgfqpoint{2.134690in}{3.548899in}}%
\pgfpathlineto{\pgfqpoint{2.203028in}{3.545383in}}%
\pgfpathlineto{\pgfqpoint{2.271365in}{3.541830in}}%
\pgfpathlineto{\pgfqpoint{2.339728in}{3.538875in}}%
\pgfpathlineto{\pgfqpoint{2.408130in}{3.537208in}}%
\pgfpathlineto{\pgfqpoint{2.476546in}{3.537552in}}%
\pgfpathlineto{\pgfqpoint{2.544888in}{3.540625in}}%
\pgfpathlineto{\pgfqpoint{2.612986in}{3.547067in}}%
\pgfpathlineto{\pgfqpoint{2.680601in}{3.557360in}}%
\pgfpathlineto{\pgfqpoint{2.747461in}{3.571741in}}%
\pgfpathlineto{\pgfqpoint{2.747461in}{3.571741in}}%
\pgfusepath{stroke}%
\end{pgfscope}%
\begin{pgfscope}%
\pgfpathrectangle{\pgfqpoint{0.647939in}{0.492442in}}{\pgfqpoint{3.079299in}{3.079299in}}%
\pgfusepath{clip}%
\pgfsetbuttcap%
\pgfsetroundjoin%
\pgfsetlinewidth{0.301125pt}%
\definecolor{currentstroke}{rgb}{0.500000,0.500000,0.500000}%
\pgfsetstrokecolor{currentstroke}%
\pgfsetstrokeopacity{0.300000}%
\pgfsetdash{}{0pt}%
\pgfpathmoveto{\pgfqpoint{0.647939in}{3.429982in}}%
\pgfpathlineto{\pgfqpoint{0.679318in}{3.433020in}}%
\pgfpathlineto{\pgfqpoint{0.747373in}{3.440142in}}%
\pgfpathlineto{\pgfqpoint{0.815310in}{3.448309in}}%
\pgfpathlineto{\pgfqpoint{0.883118in}{3.457486in}}%
\pgfpathlineto{\pgfqpoint{0.950794in}{3.467601in}}%
\pgfpathlineto{\pgfqpoint{1.018341in}{3.478545in}}%
\pgfpathlineto{\pgfqpoint{1.085774in}{3.490168in}}%
\pgfpathlineto{\pgfqpoint{1.153122in}{3.502281in}}%
\pgfpathlineto{\pgfqpoint{1.220420in}{3.514669in}}%
\pgfpathlineto{\pgfqpoint{1.287713in}{3.527090in}}%
\pgfpathlineto{\pgfqpoint{1.355046in}{3.539283in}}%
\pgfpathlineto{\pgfqpoint{1.422468in}{3.550974in}}%
\pgfpathlineto{\pgfqpoint{1.490020in}{3.561884in}}%
\pgfpathlineto{\pgfqpoint{1.557732in}{3.571741in}}%
\pgfpathlineto{\pgfqpoint{1.557732in}{3.571741in}}%
\pgfusepath{stroke}%
\end{pgfscope}%
\begin{pgfscope}%
\pgfpathrectangle{\pgfqpoint{0.647939in}{0.492442in}}{\pgfqpoint{3.079299in}{3.079299in}}%
\pgfusepath{clip}%
\pgfsetbuttcap%
\pgfsetroundjoin%
\pgfsetlinewidth{0.301125pt}%
\definecolor{currentstroke}{rgb}{0.500000,0.500000,0.500000}%
\pgfsetstrokecolor{currentstroke}%
\pgfsetstrokeopacity{0.300000}%
\pgfsetdash{}{0pt}%
\pgfpathmoveto{\pgfqpoint{0.647939in}{3.503413in}}%
\pgfpathlineto{\pgfqpoint{0.663680in}{3.504863in}}%
\pgfpathlineto{\pgfqpoint{0.731769in}{3.511657in}}%
\pgfpathlineto{\pgfqpoint{0.799747in}{3.519482in}}%
\pgfpathlineto{\pgfqpoint{0.867602in}{3.528310in}}%
\pgfpathlineto{\pgfqpoint{0.935329in}{3.538073in}}%
\pgfpathlineto{\pgfqpoint{1.002932in}{3.548664in}}%
\pgfpathlineto{\pgfqpoint{1.070424in}{3.559945in}}%
\pgfpathlineto{\pgfqpoint{1.137828in}{3.571741in}}%
\pgfpathlineto{\pgfqpoint{1.137828in}{3.571741in}}%
\pgfusepath{stroke}%
\end{pgfscope}%
\begin{pgfscope}%
\pgfpathrectangle{\pgfqpoint{0.647939in}{0.492442in}}{\pgfqpoint{3.079299in}{3.079299in}}%
\pgfusepath{clip}%
\pgfsetbuttcap%
\pgfsetroundjoin%
\pgfsetlinewidth{0.301125pt}%
\definecolor{currentstroke}{rgb}{0.500000,0.500000,0.500000}%
\pgfsetstrokecolor{currentstroke}%
\pgfsetstrokeopacity{0.300000}%
\pgfsetdash{}{0pt}%
\pgfpathmoveto{\pgfqpoint{0.647939in}{2.871901in}}%
\pgfpathlineto{\pgfqpoint{0.647939in}{2.871901in}}%
\pgfpathlineto{\pgfqpoint{0.715973in}{2.879214in}}%
\pgfpathlineto{\pgfqpoint{0.783864in}{2.887748in}}%
\pgfpathlineto{\pgfqpoint{0.851591in}{2.897505in}}%
\pgfpathlineto{\pgfqpoint{0.919136in}{2.908446in}}%
\pgfpathlineto{\pgfqpoint{0.986494in}{2.920492in}}%
\pgfpathlineto{\pgfqpoint{1.053670in}{2.933516in}}%
\pgfpathlineto{\pgfqpoint{1.120686in}{2.947345in}}%
\pgfpathlineto{\pgfqpoint{1.187578in}{2.961765in}}%
\pgfpathlineto{\pgfqpoint{1.254395in}{2.976529in}}%
\pgfpathlineto{\pgfqpoint{1.321199in}{2.991353in}}%
\pgfpathlineto{\pgfqpoint{1.388057in}{3.005929in}}%
\pgfpathlineto{\pgfqpoint{1.455038in}{3.019922in}}%
\pgfpathlineto{\pgfqpoint{1.522206in}{3.032985in}}%
\pgfpathlineto{\pgfqpoint{1.589608in}{3.044768in}}%
\pgfpathlineto{\pgfqpoint{1.657270in}{3.054939in}}%
\pgfpathlineto{\pgfqpoint{1.725189in}{3.063202in}}%
\pgfpathlineto{\pgfqpoint{1.793333in}{3.069320in}}%
\pgfpathlineto{\pgfqpoint{1.861644in}{3.073140in}}%
\pgfpathlineto{\pgfqpoint{1.930046in}{3.074629in}}%
\pgfpathlineto{\pgfqpoint{1.998462in}{3.073892in}}%
\pgfpathlineto{\pgfqpoint{2.066829in}{3.071179in}}%
\pgfpathlineto{\pgfqpoint{2.135121in}{3.066901in}}%
\pgfpathlineto{\pgfqpoint{2.203346in}{3.061644in}}%
\pgfpathlineto{\pgfqpoint{2.271558in}{3.056194in}}%
\pgfpathlineto{\pgfqpoint{2.339824in}{3.051536in}}%
\pgfpathlineto{\pgfqpoint{2.408185in}{3.048842in}}%
\pgfpathlineto{\pgfqpoint{2.476577in}{3.049453in}}%
\pgfpathlineto{\pgfqpoint{2.544743in}{3.054755in}}%
\pgfpathlineto{\pgfqpoint{2.612168in}{3.065927in}}%
\pgfpathlineto{\pgfqpoint{2.678171in}{3.083581in}}%
\pgfpathlineto{\pgfqpoint{2.742145in}{3.107571in}}%
\pgfpathlineto{\pgfqpoint{2.803767in}{3.137130in}}%
\pgfpathlineto{\pgfqpoint{2.863048in}{3.171194in}}%
\pgfusepath{stroke}%
\end{pgfscope}%
\begin{pgfscope}%
\pgfpathrectangle{\pgfqpoint{0.647939in}{0.492442in}}{\pgfqpoint{3.079299in}{3.079299in}}%
\pgfusepath{clip}%
\pgfsetbuttcap%
\pgfsetroundjoin%
\pgfsetlinewidth{0.301125pt}%
\definecolor{currentstroke}{rgb}{0.500000,0.500000,0.500000}%
\pgfsetstrokecolor{currentstroke}%
\pgfsetstrokeopacity{0.300000}%
\pgfsetdash{}{0pt}%
\pgfpathmoveto{\pgfqpoint{0.647939in}{2.801916in}}%
\pgfpathlineto{\pgfqpoint{0.647939in}{2.801916in}}%
\pgfpathlineto{\pgfqpoint{0.715963in}{2.809324in}}%
\pgfpathlineto{\pgfqpoint{0.783839in}{2.817978in}}%
\pgfpathlineto{\pgfqpoint{0.851544in}{2.827882in}}%
\pgfpathlineto{\pgfqpoint{0.919060in}{2.839002in}}%
\pgfpathlineto{\pgfqpoint{0.986379in}{2.851260in}}%
\pgfpathlineto{\pgfqpoint{1.053507in}{2.864531in}}%
\pgfpathlineto{\pgfqpoint{1.120464in}{2.878643in}}%
\pgfpathlineto{\pgfqpoint{1.187286in}{2.893381in}}%
\pgfusepath{stroke}%
\end{pgfscope}%
\begin{pgfscope}%
\pgfpathrectangle{\pgfqpoint{0.647939in}{0.492442in}}{\pgfqpoint{3.079299in}{3.079299in}}%
\pgfusepath{clip}%
\pgfsetbuttcap%
\pgfsetroundjoin%
\pgfsetlinewidth{0.301125pt}%
\definecolor{currentstroke}{rgb}{0.500000,0.500000,0.500000}%
\pgfsetstrokecolor{currentstroke}%
\pgfsetstrokeopacity{0.300000}%
\pgfsetdash{}{0pt}%
\pgfpathmoveto{\pgfqpoint{0.647939in}{2.591964in}}%
\pgfpathlineto{\pgfqpoint{0.647939in}{2.591964in}}%
\pgfpathlineto{\pgfqpoint{0.715929in}{2.599671in}}%
\pgfpathlineto{\pgfqpoint{0.783755in}{2.608703in}}%
\pgfpathlineto{\pgfqpoint{0.851389in}{2.619078in}}%
\pgfpathlineto{\pgfqpoint{0.918808in}{2.630770in}}%
\pgfpathlineto{\pgfqpoint{0.985999in}{2.643709in}}%
\pgfpathlineto{\pgfqpoint{1.052963in}{2.657778in}}%
\pgfpathlineto{\pgfqpoint{1.119720in}{2.672808in}}%
\pgfpathlineto{\pgfqpoint{1.186305in}{2.688582in}}%
\pgfpathlineto{\pgfqpoint{1.252772in}{2.704848in}}%
\pgfpathlineto{\pgfqpoint{1.319191in}{2.721312in}}%
\pgfpathlineto{\pgfqpoint{1.385642in}{2.737642in}}%
\pgfpathlineto{\pgfqpoint{1.452212in}{2.753479in}}%
\pgfpathlineto{\pgfqpoint{1.518984in}{2.768435in}}%
\pgfpathlineto{\pgfqpoint{1.586029in}{2.782104in}}%
\pgfpathlineto{\pgfqpoint{1.653393in}{2.794082in}}%
\pgfpathlineto{\pgfqpoint{1.721090in}{2.803987in}}%
\pgfpathlineto{\pgfqpoint{1.789093in}{2.811482in}}%
\pgfpathlineto{\pgfqpoint{1.857335in}{2.816306in}}%
\pgfpathlineto{\pgfqpoint{1.925719in}{2.818319in}}%
\pgfpathlineto{\pgfqpoint{1.994128in}{2.817541in}}%
\pgfpathlineto{\pgfqpoint{2.062461in}{2.814176in}}%
\pgfpathlineto{\pgfqpoint{2.130657in}{2.808631in}}%
\pgfpathlineto{\pgfqpoint{2.198716in}{2.801554in}}%
\pgfpathlineto{\pgfqpoint{2.266714in}{2.793891in}}%
\pgfpathlineto{\pgfqpoint{2.334788in}{2.786979in}}%
\pgfpathlineto{\pgfqpoint{2.403049in}{2.782638in}}%
\pgfpathlineto{\pgfqpoint{2.471391in}{2.783202in}}%
\pgfpathlineto{\pgfqpoint{2.539177in}{2.791251in}}%
\pgfpathlineto{\pgfqpoint{2.605131in}{2.808630in}}%
\pgfpathlineto{\pgfqpoint{2.667910in}{2.835339in}}%
\pgfpathlineto{\pgfqpoint{2.726932in}{2.869669in}}%
\pgfpathlineto{\pgfqpoint{2.782494in}{2.909427in}}%
\pgfpathlineto{\pgfqpoint{2.835321in}{2.952811in}}%
\pgfpathlineto{\pgfqpoint{2.886156in}{2.998565in}}%
\pgfpathlineto{\pgfqpoint{2.935609in}{3.045829in}}%
\pgfusepath{stroke}%
\end{pgfscope}%
\begin{pgfscope}%
\pgfpathrectangle{\pgfqpoint{0.647939in}{0.492442in}}{\pgfqpoint{3.079299in}{3.079299in}}%
\pgfusepath{clip}%
\pgfsetbuttcap%
\pgfsetroundjoin%
\pgfsetlinewidth{0.301125pt}%
\definecolor{currentstroke}{rgb}{0.500000,0.500000,0.500000}%
\pgfsetstrokecolor{currentstroke}%
\pgfsetstrokeopacity{0.300000}%
\pgfsetdash{}{0pt}%
\pgfpathmoveto{\pgfqpoint{0.647939in}{2.521980in}}%
\pgfpathlineto{\pgfqpoint{0.647939in}{2.521980in}}%
\pgfpathlineto{\pgfqpoint{0.715917in}{2.529791in}}%
\pgfpathlineto{\pgfqpoint{0.783725in}{2.538958in}}%
\pgfpathlineto{\pgfqpoint{0.851333in}{2.549499in}}%
\pgfpathlineto{\pgfqpoint{0.918716in}{2.561394in}}%
\pgfpathlineto{\pgfqpoint{0.985859in}{2.574577in}}%
\pgfpathlineto{\pgfqpoint{1.052763in}{2.588934in}}%
\pgfpathlineto{\pgfqpoint{1.119443in}{2.604295in}}%
\pgfpathlineto{\pgfqpoint{1.185938in}{2.620446in}}%
\pgfpathlineto{\pgfqpoint{1.252300in}{2.637132in}}%
\pgfpathlineto{\pgfqpoint{1.318603in}{2.654059in}}%
\pgfpathlineto{\pgfqpoint{1.384929in}{2.670890in}}%
\pgfpathlineto{\pgfqpoint{1.451370in}{2.687258in}}%
\pgfpathlineto{\pgfqpoint{1.518016in}{2.702766in}}%
\pgfpathlineto{\pgfqpoint{1.584945in}{2.716995in}}%
\pgfpathlineto{\pgfqpoint{1.652210in}{2.729519in}}%
\pgfpathlineto{\pgfqpoint{1.719830in}{2.739930in}}%
\pgfpathlineto{\pgfqpoint{1.787782in}{2.747860in}}%
\pgfpathlineto{\pgfqpoint{1.855999in}{2.753012in}}%
\pgfpathlineto{\pgfqpoint{1.924375in}{2.755205in}}%
\pgfpathlineto{\pgfqpoint{1.992782in}{2.754424in}}%
\pgfpathlineto{\pgfqpoint{2.061102in}{2.750846in}}%
\pgfpathlineto{\pgfqpoint{2.129259in}{2.744863in}}%
\pgfpathlineto{\pgfqpoint{2.197245in}{2.737127in}}%
\pgfpathlineto{\pgfqpoint{2.265142in}{2.728616in}}%
\pgfpathlineto{\pgfqpoint{2.333115in}{2.720781in}}%
\pgfpathlineto{\pgfqpoint{2.401320in}{2.715705in}}%
\pgfpathlineto{\pgfqpoint{2.469632in}{2.716217in}}%
\pgfpathlineto{\pgfqpoint{2.537180in}{2.725525in}}%
\pgfpathlineto{\pgfqpoint{2.602245in}{2.745653in}}%
\pgfusepath{stroke}%
\end{pgfscope}%
\begin{pgfscope}%
\pgfpathrectangle{\pgfqpoint{0.647939in}{0.492442in}}{\pgfqpoint{3.079299in}{3.079299in}}%
\pgfusepath{clip}%
\pgfsetbuttcap%
\pgfsetroundjoin%
\pgfsetlinewidth{0.301125pt}%
\definecolor{currentstroke}{rgb}{0.500000,0.500000,0.500000}%
\pgfsetstrokecolor{currentstroke}%
\pgfsetstrokeopacity{0.300000}%
\pgfsetdash{}{0pt}%
\pgfpathmoveto{\pgfqpoint{0.647939in}{2.451996in}}%
\pgfpathlineto{\pgfqpoint{0.647939in}{2.451996in}}%
\pgfpathlineto{\pgfqpoint{0.715904in}{2.459915in}}%
\pgfpathlineto{\pgfqpoint{0.783694in}{2.469219in}}%
\pgfpathlineto{\pgfqpoint{0.851274in}{2.479932in}}%
\pgfpathlineto{\pgfqpoint{0.918619in}{2.492038in}}%
\pgfpathlineto{\pgfqpoint{0.985712in}{2.505474in}}%
\pgfpathlineto{\pgfqpoint{1.052551in}{2.520128in}}%
\pgfpathlineto{\pgfqpoint{1.119151in}{2.535835in}}%
\pgfpathlineto{\pgfqpoint{1.185548in}{2.552380in}}%
\pgfpathlineto{\pgfqpoint{1.251798in}{2.569508in}}%
\pgfpathlineto{\pgfqpoint{1.317974in}{2.586921in}}%
\pgfpathlineto{\pgfqpoint{1.384164in}{2.604282in}}%
\pgfpathlineto{\pgfqpoint{1.450464in}{2.621215in}}%
\pgfpathlineto{\pgfqpoint{1.516969in}{2.637313in}}%
\pgfpathlineto{\pgfqpoint{1.583767in}{2.652143in}}%
\pgfpathlineto{\pgfqpoint{1.650919in}{2.665259in}}%
\pgfpathlineto{\pgfqpoint{1.718451in}{2.676225in}}%
\pgfpathlineto{\pgfqpoint{1.786344in}{2.684637in}}%
\pgfpathlineto{\pgfqpoint{1.854530in}{2.690158in}}%
\pgfpathlineto{\pgfqpoint{1.922897in}{2.692560in}}%
\pgfpathlineto{\pgfqpoint{1.991302in}{2.691781in}}%
\pgfpathlineto{\pgfqpoint{2.059606in}{2.687963in}}%
\pgfpathlineto{\pgfqpoint{2.127713in}{2.681470in}}%
\pgfpathlineto{\pgfqpoint{2.195603in}{2.672945in}}%
\pgfpathlineto{\pgfqpoint{2.263361in}{2.663394in}}%
\pgfpathlineto{\pgfqpoint{2.331187in}{2.654375in}}%
\pgfpathlineto{\pgfqpoint{2.399302in}{2.648298in}}%
\pgfpathlineto{\pgfqpoint{2.467566in}{2.648718in}}%
\pgfpathlineto{\pgfqpoint{2.534705in}{2.659806in}}%
\pgfpathlineto{\pgfqpoint{2.598346in}{2.683608in}}%
\pgfpathlineto{\pgfqpoint{2.654429in}{2.716306in}}%
\pgfpathlineto{\pgfqpoint{2.708712in}{2.757601in}}%
\pgfpathlineto{\pgfqpoint{2.759606in}{2.803200in}}%
\pgfpathlineto{\pgfqpoint{2.808207in}{2.851280in}}%
\pgfpathlineto{\pgfqpoint{2.855347in}{2.900825in}}%
\pgfpathlineto{\pgfqpoint{2.901601in}{2.951220in}}%
\pgfusepath{stroke}%
\end{pgfscope}%
\begin{pgfscope}%
\pgfpathrectangle{\pgfqpoint{0.647939in}{0.492442in}}{\pgfqpoint{3.079299in}{3.079299in}}%
\pgfusepath{clip}%
\pgfsetbuttcap%
\pgfsetroundjoin%
\pgfsetlinewidth{0.301125pt}%
\definecolor{currentstroke}{rgb}{0.500000,0.500000,0.500000}%
\pgfsetstrokecolor{currentstroke}%
\pgfsetstrokeopacity{0.300000}%
\pgfsetdash{}{0pt}%
\pgfpathmoveto{\pgfqpoint{0.647939in}{2.382012in}}%
\pgfpathlineto{\pgfqpoint{0.647939in}{2.382012in}}%
\pgfpathlineto{\pgfqpoint{0.715891in}{2.390042in}}%
\pgfpathlineto{\pgfqpoint{0.783661in}{2.399487in}}%
\pgfpathlineto{\pgfqpoint{0.851212in}{2.410378in}}%
\pgfpathlineto{\pgfqpoint{0.918518in}{2.422702in}}%
\pgfpathlineto{\pgfqpoint{0.985557in}{2.436400in}}%
\pgfpathlineto{\pgfqpoint{1.052327in}{2.451364in}}%
\pgfpathlineto{\pgfqpoint{1.118841in}{2.467431in}}%
\pgfpathlineto{\pgfqpoint{1.185134in}{2.484388in}}%
\pgfpathlineto{\pgfqpoint{1.251262in}{2.501979in}}%
\pgfpathlineto{\pgfqpoint{1.317301in}{2.519906in}}%
\pgfpathlineto{\pgfqpoint{1.383341in}{2.537827in}}%
\pgfpathlineto{\pgfqpoint{1.449485in}{2.555360in}}%
\pgfpathlineto{\pgfqpoint{1.515835in}{2.572090in}}%
\pgfpathlineto{\pgfqpoint{1.582485in}{2.587569in}}%
\pgfpathlineto{\pgfqpoint{1.649508in}{2.601329in}}%
\pgfpathlineto{\pgfqpoint{1.716938in}{2.612903in}}%
\pgfpathlineto{\pgfqpoint{1.784761in}{2.621852in}}%
\pgfusepath{stroke}%
\end{pgfscope}%
\begin{pgfscope}%
\pgfpathrectangle{\pgfqpoint{0.647939in}{0.492442in}}{\pgfqpoint{3.079299in}{3.079299in}}%
\pgfusepath{clip}%
\pgfsetbuttcap%
\pgfsetroundjoin%
\pgfsetlinewidth{0.301125pt}%
\definecolor{currentstroke}{rgb}{0.500000,0.500000,0.500000}%
\pgfsetstrokecolor{currentstroke}%
\pgfsetstrokeopacity{0.300000}%
\pgfsetdash{}{0pt}%
\pgfpathmoveto{\pgfqpoint{0.647939in}{2.312028in}}%
\pgfpathlineto{\pgfqpoint{0.647939in}{2.312028in}}%
\pgfpathlineto{\pgfqpoint{0.715878in}{2.320172in}}%
\pgfpathlineto{\pgfqpoint{0.783626in}{2.329763in}}%
\pgfpathlineto{\pgfqpoint{0.851148in}{2.340838in}}%
\pgfpathlineto{\pgfqpoint{0.918411in}{2.353387in}}%
\pgfpathlineto{\pgfqpoint{0.985394in}{2.367357in}}%
\pgfpathlineto{\pgfqpoint{1.052090in}{2.382644in}}%
\pgfpathlineto{\pgfqpoint{1.118512in}{2.399087in}}%
\pgfpathlineto{\pgfqpoint{1.184693in}{2.416475in}}%
\pgfpathlineto{\pgfqpoint{1.250690in}{2.434553in}}%
\pgfpathlineto{\pgfqpoint{1.316579in}{2.453021in}}%
\pgfpathlineto{\pgfqpoint{1.382456in}{2.471535in}}%
\pgfusepath{stroke}%
\end{pgfscope}%
\begin{pgfscope}%
\pgfpathrectangle{\pgfqpoint{0.647939in}{0.492442in}}{\pgfqpoint{3.079299in}{3.079299in}}%
\pgfusepath{clip}%
\pgfsetbuttcap%
\pgfsetroundjoin%
\pgfsetlinewidth{0.301125pt}%
\definecolor{currentstroke}{rgb}{0.500000,0.500000,0.500000}%
\pgfsetstrokecolor{currentstroke}%
\pgfsetstrokeopacity{0.300000}%
\pgfsetdash{}{0pt}%
\pgfpathmoveto{\pgfqpoint{0.647939in}{2.242044in}}%
\pgfpathlineto{\pgfqpoint{0.647939in}{2.242044in}}%
\pgfpathlineto{\pgfqpoint{0.715863in}{2.250305in}}%
\pgfpathlineto{\pgfqpoint{0.783590in}{2.260047in}}%
\pgfpathlineto{\pgfqpoint{0.851080in}{2.271311in}}%
\pgfpathlineto{\pgfqpoint{0.918299in}{2.284094in}}%
\pgfpathlineto{\pgfqpoint{0.985222in}{2.298347in}}%
\pgfpathlineto{\pgfqpoint{1.051840in}{2.313970in}}%
\pgfpathlineto{\pgfqpoint{1.118163in}{2.330806in}}%
\pgfusepath{stroke}%
\end{pgfscope}%
\begin{pgfscope}%
\pgfpathrectangle{\pgfqpoint{0.647939in}{0.492442in}}{\pgfqpoint{3.079299in}{3.079299in}}%
\pgfusepath{clip}%
\pgfsetbuttcap%
\pgfsetroundjoin%
\pgfsetlinewidth{0.301125pt}%
\definecolor{currentstroke}{rgb}{0.500000,0.500000,0.500000}%
\pgfsetstrokecolor{currentstroke}%
\pgfsetstrokeopacity{0.300000}%
\pgfsetdash{}{0pt}%
\pgfpathmoveto{\pgfqpoint{0.647939in}{2.172060in}}%
\pgfpathlineto{\pgfqpoint{0.647939in}{2.172060in}}%
\pgfpathlineto{\pgfqpoint{0.715848in}{2.180441in}}%
\pgfpathlineto{\pgfqpoint{0.783552in}{2.190339in}}%
\pgfpathlineto{\pgfqpoint{0.851009in}{2.201799in}}%
\pgfpathlineto{\pgfqpoint{0.918181in}{2.214824in}}%
\pgfpathlineto{\pgfqpoint{0.985041in}{2.229371in}}%
\pgfpathlineto{\pgfqpoint{1.051575in}{2.245344in}}%
\pgfpathlineto{\pgfqpoint{1.117792in}{2.262590in}}%
\pgfpathlineto{\pgfqpoint{1.183722in}{2.280903in}}%
\pgfpathlineto{\pgfqpoint{1.249422in}{2.300032in}}%
\pgfpathlineto{\pgfqpoint{1.314971in}{2.319675in}}%
\pgfpathlineto{\pgfqpoint{1.380469in}{2.339485in}}%
\pgfpathlineto{\pgfqpoint{1.446036in}{2.359066in}}%
\pgfpathlineto{\pgfqpoint{1.511797in}{2.377978in}}%
\pgfpathlineto{\pgfqpoint{1.577878in}{2.395732in}}%
\pgfpathlineto{\pgfqpoint{1.644387in}{2.411797in}}%
\pgfpathlineto{\pgfqpoint{1.711393in}{2.425607in}}%
\pgfpathlineto{\pgfqpoint{1.778914in}{2.436585in}}%
\pgfpathlineto{\pgfqpoint{1.846892in}{2.444168in}}%
\pgfpathlineto{\pgfqpoint{1.915185in}{2.447846in}}%
\pgfpathlineto{\pgfqpoint{1.983570in}{2.447213in}}%
\pgfpathlineto{\pgfqpoint{2.051761in}{2.442032in}}%
\pgfpathlineto{\pgfqpoint{2.119446in}{2.432271in}}%
\pgfpathlineto{\pgfqpoint{2.186345in}{2.418075in}}%
\pgfpathlineto{\pgfqpoint{2.252239in}{2.399768in}}%
\pgfpathlineto{\pgfqpoint{2.317029in}{2.377886in}}%
\pgfpathlineto{\pgfqpoint{2.380903in}{2.353594in}}%
\pgfpathlineto{\pgfqpoint{2.380903in}{2.353594in}}%
\pgfusepath{stroke}%
\end{pgfscope}%
\begin{pgfscope}%
\pgfpathrectangle{\pgfqpoint{0.647939in}{0.492442in}}{\pgfqpoint{3.079299in}{3.079299in}}%
\pgfusepath{clip}%
\pgfsetbuttcap%
\pgfsetroundjoin%
\pgfsetlinewidth{0.301125pt}%
\definecolor{currentstroke}{rgb}{0.500000,0.500000,0.500000}%
\pgfsetstrokecolor{currentstroke}%
\pgfsetstrokeopacity{0.300000}%
\pgfsetdash{}{0pt}%
\pgfpathmoveto{\pgfqpoint{0.647939in}{2.102076in}}%
\pgfpathlineto{\pgfqpoint{0.647939in}{2.102076in}}%
\pgfpathlineto{\pgfqpoint{0.715833in}{2.110581in}}%
\pgfpathlineto{\pgfqpoint{0.783513in}{2.120639in}}%
\pgfpathlineto{\pgfqpoint{0.850934in}{2.132302in}}%
\pgfpathlineto{\pgfqpoint{0.918057in}{2.145578in}}%
\pgfpathlineto{\pgfqpoint{0.984849in}{2.160430in}}%
\pgfpathlineto{\pgfqpoint{1.051294in}{2.176769in}}%
\pgfpathlineto{\pgfqpoint{1.117397in}{2.194444in}}%
\pgfpathlineto{\pgfqpoint{1.183187in}{2.213255in}}%
\pgfpathlineto{\pgfqpoint{1.248719in}{2.232951in}}%
\pgfpathlineto{\pgfqpoint{1.314073in}{2.253232in}}%
\pgfpathlineto{\pgfqpoint{1.379352in}{2.273750in}}%
\pgfpathlineto{\pgfqpoint{1.444682in}{2.294109in}}%
\pgfpathlineto{\pgfqpoint{1.510197in}{2.313859in}}%
\pgfpathlineto{\pgfqpoint{1.576033in}{2.332502in}}%
\pgfusepath{stroke}%
\end{pgfscope}%
\begin{pgfscope}%
\pgfpathrectangle{\pgfqpoint{0.647939in}{0.492442in}}{\pgfqpoint{3.079299in}{3.079299in}}%
\pgfusepath{clip}%
\pgfsetbuttcap%
\pgfsetroundjoin%
\pgfsetlinewidth{0.301125pt}%
\definecolor{currentstroke}{rgb}{0.500000,0.500000,0.500000}%
\pgfsetstrokecolor{currentstroke}%
\pgfsetstrokeopacity{0.300000}%
\pgfsetdash{}{0pt}%
\pgfpathmoveto{\pgfqpoint{0.647939in}{2.032092in}}%
\pgfpathlineto{\pgfqpoint{0.647939in}{2.032092in}}%
\pgfpathlineto{\pgfqpoint{0.715816in}{2.040725in}}%
\pgfpathlineto{\pgfqpoint{0.783471in}{2.050948in}}%
\pgfpathlineto{\pgfqpoint{0.850856in}{2.062821in}}%
\pgfpathlineto{\pgfqpoint{0.917926in}{2.076358in}}%
\pgfpathlineto{\pgfqpoint{0.984646in}{2.091528in}}%
\pgfpathlineto{\pgfqpoint{1.050996in}{2.108248in}}%
\pgfpathlineto{\pgfqpoint{1.116977in}{2.126373in}}%
\pgfpathlineto{\pgfqpoint{1.182615in}{2.145706in}}%
\pgfpathlineto{\pgfqpoint{1.247963in}{2.165999in}}%
\pgfpathlineto{\pgfqpoint{1.313103in}{2.186955in}}%
\pgfusepath{stroke}%
\end{pgfscope}%
\begin{pgfscope}%
\pgfpathrectangle{\pgfqpoint{0.647939in}{0.492442in}}{\pgfqpoint{3.079299in}{3.079299in}}%
\pgfusepath{clip}%
\pgfsetbuttcap%
\pgfsetroundjoin%
\pgfsetlinewidth{0.301125pt}%
\definecolor{currentstroke}{rgb}{0.500000,0.500000,0.500000}%
\pgfsetstrokecolor{currentstroke}%
\pgfsetstrokeopacity{0.300000}%
\pgfsetdash{}{0pt}%
\pgfpathmoveto{\pgfqpoint{0.647939in}{1.962108in}}%
\pgfpathlineto{\pgfqpoint{0.647939in}{1.962108in}}%
\pgfpathlineto{\pgfqpoint{0.715799in}{1.970873in}}%
\pgfpathlineto{\pgfqpoint{0.783428in}{1.981267in}}%
\pgfpathlineto{\pgfqpoint{0.850773in}{1.993357in}}%
\pgfpathlineto{\pgfqpoint{0.917787in}{2.007165in}}%
\pgfpathlineto{\pgfqpoint{0.984431in}{2.022666in}}%
\pgfpathlineto{\pgfqpoint{1.050679in}{2.039783in}}%
\pgfpathlineto{\pgfqpoint{1.116528in}{2.058380in}}%
\pgfpathlineto{\pgfqpoint{1.182001in}{2.078261in}}%
\pgfpathlineto{\pgfqpoint{1.247151in}{2.099185in}}%
\pgfpathlineto{\pgfqpoint{1.312056in}{2.120857in}}%
\pgfusepath{stroke}%
\end{pgfscope}%
\begin{pgfscope}%
\pgfpathrectangle{\pgfqpoint{0.647939in}{0.492442in}}{\pgfqpoint{3.079299in}{3.079299in}}%
\pgfusepath{clip}%
\pgfsetbuttcap%
\pgfsetroundjoin%
\pgfsetlinewidth{0.301125pt}%
\definecolor{currentstroke}{rgb}{0.500000,0.500000,0.500000}%
\pgfsetstrokecolor{currentstroke}%
\pgfsetstrokeopacity{0.300000}%
\pgfsetdash{}{0pt}%
\pgfpathmoveto{\pgfqpoint{0.647939in}{1.892124in}}%
\pgfpathlineto{\pgfqpoint{0.647939in}{1.892124in}}%
\pgfpathlineto{\pgfqpoint{0.715781in}{1.901024in}}%
\pgfpathlineto{\pgfqpoint{0.783382in}{1.911595in}}%
\pgfpathlineto{\pgfqpoint{0.850686in}{1.923910in}}%
\pgfpathlineto{\pgfqpoint{0.917641in}{1.937999in}}%
\pgfpathlineto{\pgfqpoint{0.984203in}{1.953846in}}%
\pgfpathlineto{\pgfqpoint{1.050341in}{1.971379in}}%
\pgfpathlineto{\pgfqpoint{1.116049in}{1.990469in}}%
\pgfpathlineto{\pgfqpoint{1.181344in}{2.010929in}}%
\pgfpathlineto{\pgfqpoint{1.246275in}{2.032518in}}%
\pgfpathlineto{\pgfqpoint{1.310922in}{2.054947in}}%
\pgfpathlineto{\pgfqpoint{1.375394in}{2.077876in}}%
\pgfpathlineto{\pgfqpoint{1.439831in}{2.100906in}}%
\pgfpathlineto{\pgfqpoint{1.504392in}{2.123582in}}%
\pgfpathlineto{\pgfqpoint{1.569251in}{2.145383in}}%
\pgfpathlineto{\pgfqpoint{1.634584in}{2.165708in}}%
\pgfpathlineto{\pgfqpoint{1.700545in}{2.183865in}}%
\pgfpathlineto{\pgfqpoint{1.767242in}{2.199055in}}%
\pgfpathlineto{\pgfqpoint{1.834691in}{2.210360in}}%
\pgfpathlineto{\pgfqpoint{1.902758in}{2.216726in}}%
\pgfpathlineto{\pgfqpoint{1.971082in}{2.216883in}}%
\pgfpathlineto{\pgfqpoint{2.038894in}{2.209122in}}%
\pgfpathlineto{\pgfqpoint{2.104347in}{2.190456in}}%
\pgfpathlineto{\pgfqpoint{2.104347in}{2.190456in}}%
\pgfpathlineto{\pgfqpoint{2.145752in}{2.167705in}}%
\pgfpathlineto{\pgfqpoint{2.145752in}{2.167705in}}%
\pgfpathlineto{\pgfqpoint{2.169534in}{2.143038in}}%
\pgfpathlineto{\pgfqpoint{2.169534in}{2.143038in}}%
\pgfpathlineto{\pgfqpoint{2.179904in}{2.116874in}}%
\pgfpathlineto{\pgfqpoint{2.179385in}{2.088556in}}%
\pgfusepath{stroke}%
\end{pgfscope}%
\begin{pgfscope}%
\pgfpathrectangle{\pgfqpoint{0.647939in}{0.492442in}}{\pgfqpoint{3.079299in}{3.079299in}}%
\pgfusepath{clip}%
\pgfsetbuttcap%
\pgfsetroundjoin%
\pgfsetlinewidth{0.301125pt}%
\definecolor{currentstroke}{rgb}{0.500000,0.500000,0.500000}%
\pgfsetstrokecolor{currentstroke}%
\pgfsetstrokeopacity{0.300000}%
\pgfsetdash{}{0pt}%
\pgfpathmoveto{\pgfqpoint{0.647939in}{1.822139in}}%
\pgfpathlineto{\pgfqpoint{0.647939in}{1.822139in}}%
\pgfpathlineto{\pgfqpoint{0.715762in}{1.831180in}}%
\pgfpathlineto{\pgfqpoint{0.783334in}{1.841934in}}%
\pgfpathlineto{\pgfqpoint{0.850595in}{1.854482in}}%
\pgfpathlineto{\pgfqpoint{0.917487in}{1.868864in}}%
\pgfpathlineto{\pgfqpoint{0.983961in}{1.885071in}}%
\pgfpathlineto{\pgfqpoint{1.049982in}{1.903039in}}%
\pgfpathlineto{\pgfqpoint{1.115536in}{1.922647in}}%
\pgfpathlineto{\pgfqpoint{1.180637in}{1.943714in}}%
\pgfpathlineto{\pgfqpoint{1.245329in}{1.966006in}}%
\pgfpathlineto{\pgfqpoint{1.309691in}{1.989240in}}%
\pgfpathlineto{\pgfqpoint{1.373833in}{2.013078in}}%
\pgfpathlineto{\pgfqpoint{1.437896in}{2.037128in}}%
\pgfpathlineto{\pgfqpoint{1.502048in}{2.060936in}}%
\pgfpathlineto{\pgfqpoint{1.566476in}{2.083981in}}%
\pgfpathlineto{\pgfqpoint{1.631375in}{2.105654in}}%
\pgfpathlineto{\pgfqpoint{1.696926in}{2.125239in}}%
\pgfusepath{stroke}%
\end{pgfscope}%
\begin{pgfscope}%
\pgfpathrectangle{\pgfqpoint{0.647939in}{0.492442in}}{\pgfqpoint{3.079299in}{3.079299in}}%
\pgfusepath{clip}%
\pgfsetbuttcap%
\pgfsetroundjoin%
\pgfsetlinewidth{0.301125pt}%
\definecolor{currentstroke}{rgb}{0.500000,0.500000,0.500000}%
\pgfsetstrokecolor{currentstroke}%
\pgfsetstrokeopacity{0.300000}%
\pgfsetdash{}{0pt}%
\pgfpathmoveto{\pgfqpoint{0.647939in}{1.752155in}}%
\pgfpathlineto{\pgfqpoint{0.647939in}{1.752155in}}%
\pgfpathlineto{\pgfqpoint{0.715743in}{1.761340in}}%
\pgfpathlineto{\pgfqpoint{0.783284in}{1.772283in}}%
\pgfpathlineto{\pgfqpoint{0.850498in}{1.785073in}}%
\pgfpathlineto{\pgfqpoint{0.917324in}{1.799760in}}%
\pgfpathlineto{\pgfqpoint{0.983705in}{1.816342in}}%
\pgfpathlineto{\pgfqpoint{1.049599in}{1.834765in}}%
\pgfpathlineto{\pgfqpoint{1.114987in}{1.854918in}}%
\pgfpathlineto{\pgfqpoint{1.179876in}{1.876625in}}%
\pgfpathlineto{\pgfqpoint{1.244307in}{1.899661in}}%
\pgfpathlineto{\pgfqpoint{1.308354in}{1.923747in}}%
\pgfpathlineto{\pgfqpoint{1.372127in}{1.948555in}}%
\pgfpathlineto{\pgfqpoint{1.435769in}{1.973697in}}%
\pgfusepath{stroke}%
\end{pgfscope}%
\begin{pgfscope}%
\pgfpathrectangle{\pgfqpoint{0.647939in}{0.492442in}}{\pgfqpoint{3.079299in}{3.079299in}}%
\pgfusepath{clip}%
\pgfsetbuttcap%
\pgfsetroundjoin%
\pgfsetlinewidth{0.301125pt}%
\definecolor{currentstroke}{rgb}{0.500000,0.500000,0.500000}%
\pgfsetstrokecolor{currentstroke}%
\pgfsetstrokeopacity{0.300000}%
\pgfsetdash{}{0pt}%
\pgfpathmoveto{\pgfqpoint{0.647939in}{1.682171in}}%
\pgfpathlineto{\pgfqpoint{0.647939in}{1.682171in}}%
\pgfpathlineto{\pgfqpoint{0.715722in}{1.691505in}}%
\pgfpathlineto{\pgfqpoint{0.783231in}{1.702644in}}%
\pgfpathlineto{\pgfqpoint{0.850396in}{1.715685in}}%
\pgfpathlineto{\pgfqpoint{0.917151in}{1.730689in}}%
\pgfpathlineto{\pgfqpoint{0.983431in}{1.747663in}}%
\pgfpathlineto{\pgfqpoint{1.049190in}{1.766563in}}%
\pgfusepath{stroke}%
\end{pgfscope}%
\begin{pgfscope}%
\pgfpathrectangle{\pgfqpoint{0.647939in}{0.492442in}}{\pgfqpoint{3.079299in}{3.079299in}}%
\pgfusepath{clip}%
\pgfsetbuttcap%
\pgfsetroundjoin%
\pgfsetlinewidth{0.301125pt}%
\definecolor{currentstroke}{rgb}{0.500000,0.500000,0.500000}%
\pgfsetstrokecolor{currentstroke}%
\pgfsetstrokeopacity{0.300000}%
\pgfsetdash{}{0pt}%
\pgfpathmoveto{\pgfqpoint{0.647939in}{1.612187in}}%
\pgfpathlineto{\pgfqpoint{0.647939in}{1.612187in}}%
\pgfpathlineto{\pgfqpoint{0.715700in}{1.621675in}}%
\pgfpathlineto{\pgfqpoint{0.783175in}{1.633016in}}%
\pgfpathlineto{\pgfqpoint{0.850289in}{1.646319in}}%
\pgfpathlineto{\pgfqpoint{0.916967in}{1.661652in}}%
\pgfpathlineto{\pgfqpoint{0.983141in}{1.679037in}}%
\pgfpathlineto{\pgfqpoint{1.048752in}{1.698437in}}%
\pgfpathlineto{\pgfqpoint{1.113765in}{1.719761in}}%
\pgfpathlineto{\pgfqpoint{1.178173in}{1.742855in}}%
\pgfpathlineto{\pgfqpoint{1.242001in}{1.767508in}}%
\pgfpathlineto{\pgfqpoint{1.305313in}{1.793462in}}%
\pgfpathlineto{\pgfqpoint{1.368212in}{1.820404in}}%
\pgfpathlineto{\pgfqpoint{1.430841in}{1.847971in}}%
\pgfpathlineto{\pgfqpoint{1.493382in}{1.875738in}}%
\pgfpathlineto{\pgfqpoint{1.556052in}{1.903209in}}%
\pgfpathlineto{\pgfqpoint{1.619100in}{1.929793in}}%
\pgfpathlineto{\pgfqpoint{1.682798in}{1.954766in}}%
\pgfpathlineto{\pgfqpoint{1.747419in}{1.977207in}}%
\pgfpathlineto{\pgfqpoint{1.813202in}{1.995867in}}%
\pgfpathlineto{\pgfqpoint{1.880272in}{2.008884in}}%
\pgfpathlineto{\pgfqpoint{1.948310in}{2.012874in}}%
\pgfpathlineto{\pgfqpoint{1.948310in}{2.012874in}}%
\pgfpathlineto{\pgfqpoint{1.996854in}{2.006001in}}%
\pgfpathlineto{\pgfqpoint{1.996854in}{2.006001in}}%
\pgfpathlineto{\pgfqpoint{2.025486in}{1.993123in}}%
\pgfpathlineto{\pgfqpoint{2.025486in}{1.993123in}}%
\pgfpathlineto{\pgfqpoint{2.041282in}{1.975310in}}%
\pgfpathlineto{\pgfqpoint{2.045314in}{1.952043in}}%
\pgfpathlineto{\pgfqpoint{2.040424in}{1.930247in}}%
\pgfusepath{stroke}%
\end{pgfscope}%
\begin{pgfscope}%
\pgfpathrectangle{\pgfqpoint{0.647939in}{0.492442in}}{\pgfqpoint{3.079299in}{3.079299in}}%
\pgfusepath{clip}%
\pgfsetbuttcap%
\pgfsetroundjoin%
\pgfsetlinewidth{0.301125pt}%
\definecolor{currentstroke}{rgb}{0.500000,0.500000,0.500000}%
\pgfsetstrokecolor{currentstroke}%
\pgfsetstrokeopacity{0.300000}%
\pgfsetdash{}{0pt}%
\pgfpathmoveto{\pgfqpoint{0.647939in}{1.542203in}}%
\pgfpathlineto{\pgfqpoint{0.647939in}{1.542203in}}%
\pgfpathlineto{\pgfqpoint{0.715678in}{1.551850in}}%
\pgfpathlineto{\pgfqpoint{0.783116in}{1.563401in}}%
\pgfpathlineto{\pgfqpoint{0.850175in}{1.576975in}}%
\pgfpathlineto{\pgfqpoint{0.916772in}{1.592653in}}%
\pgfpathlineto{\pgfqpoint{0.982830in}{1.610466in}}%
\pgfpathlineto{\pgfqpoint{1.048284in}{1.630391in}}%
\pgfpathlineto{\pgfqpoint{1.113085in}{1.652346in}}%
\pgfpathlineto{\pgfqpoint{1.177218in}{1.676190in}}%
\pgfpathlineto{\pgfqpoint{1.240698in}{1.701724in}}%
\pgfpathlineto{\pgfqpoint{1.303581in}{1.728698in}}%
\pgfpathlineto{\pgfqpoint{1.365963in}{1.756814in}}%
\pgfpathlineto{\pgfqpoint{1.427984in}{1.785722in}}%
\pgfpathlineto{\pgfqpoint{1.489824in}{1.815015in}}%
\pgfpathlineto{\pgfqpoint{1.551706in}{1.844216in}}%
\pgfpathlineto{\pgfqpoint{1.613893in}{1.872760in}}%
\pgfpathlineto{\pgfqpoint{1.676679in}{1.899945in}}%
\pgfpathlineto{\pgfqpoint{1.740386in}{1.924870in}}%
\pgfusepath{stroke}%
\end{pgfscope}%
\begin{pgfscope}%
\pgfpathrectangle{\pgfqpoint{0.647939in}{0.492442in}}{\pgfqpoint{3.079299in}{3.079299in}}%
\pgfusepath{clip}%
\pgfsetbuttcap%
\pgfsetroundjoin%
\pgfsetlinewidth{0.301125pt}%
\definecolor{currentstroke}{rgb}{0.500000,0.500000,0.500000}%
\pgfsetstrokecolor{currentstroke}%
\pgfsetstrokeopacity{0.300000}%
\pgfsetdash{}{0pt}%
\pgfpathmoveto{\pgfqpoint{0.647939in}{1.402235in}}%
\pgfpathlineto{\pgfqpoint{0.647939in}{1.402235in}}%
\pgfpathlineto{\pgfqpoint{0.715628in}{1.412216in}}%
\pgfpathlineto{\pgfqpoint{0.782988in}{1.424210in}}%
\pgfpathlineto{\pgfqpoint{0.849926in}{1.438361in}}%
\pgfpathlineto{\pgfqpoint{0.916344in}{1.454773in}}%
\pgfpathlineto{\pgfqpoint{0.982146in}{1.473505in}}%
\pgfpathlineto{\pgfqpoint{1.047242in}{1.494558in}}%
\pgfpathlineto{\pgfqpoint{1.111562in}{1.517878in}}%
\pgfpathlineto{\pgfqpoint{1.175063in}{1.543348in}}%
\pgfpathlineto{\pgfqpoint{1.237736in}{1.570796in}}%
\pgfpathlineto{\pgfqpoint{1.299612in}{1.599998in}}%
\pgfpathlineto{\pgfqpoint{1.360769in}{1.630683in}}%
\pgfpathlineto{\pgfqpoint{1.421329in}{1.662532in}}%
\pgfpathlineto{\pgfqpoint{1.481461in}{1.695183in}}%
\pgfpathlineto{\pgfqpoint{1.541383in}{1.728218in}}%
\pgfpathlineto{\pgfqpoint{1.601364in}{1.761147in}}%
\pgfpathlineto{\pgfqpoint{1.661722in}{1.793372in}}%
\pgfpathlineto{\pgfqpoint{1.722838in}{1.824121in}}%
\pgfpathlineto{\pgfqpoint{1.785157in}{1.852304in}}%
\pgfpathlineto{\pgfqpoint{1.849212in}{1.876121in}}%
\pgfpathlineto{\pgfqpoint{1.915569in}{1.891430in}}%
\pgfpathlineto{\pgfqpoint{1.915569in}{1.891430in}}%
\pgfpathlineto{\pgfqpoint{1.948374in}{1.892574in}}%
\pgfpathlineto{\pgfqpoint{1.948374in}{1.892574in}}%
\pgfpathlineto{\pgfqpoint{1.968209in}{1.887179in}}%
\pgfpathlineto{\pgfqpoint{1.968209in}{1.887179in}}%
\pgfpathlineto{\pgfqpoint{1.977199in}{1.876462in}}%
\pgfpathlineto{\pgfqpoint{1.976886in}{1.863069in}}%
\pgfusepath{stroke}%
\end{pgfscope}%
\begin{pgfscope}%
\pgfpathrectangle{\pgfqpoint{0.647939in}{0.492442in}}{\pgfqpoint{3.079299in}{3.079299in}}%
\pgfusepath{clip}%
\pgfsetbuttcap%
\pgfsetroundjoin%
\pgfsetlinewidth{0.301125pt}%
\definecolor{currentstroke}{rgb}{0.500000,0.500000,0.500000}%
\pgfsetstrokecolor{currentstroke}%
\pgfsetstrokeopacity{0.300000}%
\pgfsetdash{}{0pt}%
\pgfpathmoveto{\pgfqpoint{0.647939in}{1.332251in}}%
\pgfpathlineto{\pgfqpoint{0.647939in}{1.332251in}}%
\pgfpathlineto{\pgfqpoint{0.715602in}{1.342408in}}%
\pgfpathlineto{\pgfqpoint{0.782919in}{1.354636in}}%
\pgfpathlineto{\pgfqpoint{0.849790in}{1.369093in}}%
\pgfpathlineto{\pgfqpoint{0.916109in}{1.385897in}}%
\pgfpathlineto{\pgfqpoint{0.981767in}{1.405122in}}%
\pgfpathlineto{\pgfqpoint{1.046662in}{1.426783in}}%
\pgfpathlineto{\pgfqpoint{1.110708in}{1.450840in}}%
\pgfpathlineto{\pgfqpoint{1.173846in}{1.477191in}}%
\pgfpathlineto{\pgfqpoint{1.236051in}{1.505677in}}%
\pgfpathlineto{\pgfqpoint{1.297340in}{1.536088in}}%
\pgfpathlineto{\pgfqpoint{1.357773in}{1.568168in}}%
\pgfpathlineto{\pgfqpoint{1.417459in}{1.601619in}}%
\pgfpathlineto{\pgfqpoint{1.476556in}{1.636103in}}%
\pgfpathlineto{\pgfqpoint{1.535271in}{1.671237in}}%
\pgfpathlineto{\pgfqpoint{1.593864in}{1.706577in}}%
\pgfusepath{stroke}%
\end{pgfscope}%
\begin{pgfscope}%
\pgfpathrectangle{\pgfqpoint{0.647939in}{0.492442in}}{\pgfqpoint{3.079299in}{3.079299in}}%
\pgfusepath{clip}%
\pgfsetbuttcap%
\pgfsetroundjoin%
\pgfsetlinewidth{0.301125pt}%
\definecolor{currentstroke}{rgb}{0.500000,0.500000,0.500000}%
\pgfsetstrokecolor{currentstroke}%
\pgfsetstrokeopacity{0.300000}%
\pgfsetdash{}{0pt}%
\pgfpathmoveto{\pgfqpoint{0.647939in}{1.262267in}}%
\pgfpathlineto{\pgfqpoint{0.647939in}{1.262267in}}%
\pgfpathlineto{\pgfqpoint{0.715574in}{1.272606in}}%
\pgfpathlineto{\pgfqpoint{0.782845in}{1.285078in}}%
\pgfpathlineto{\pgfqpoint{0.849646in}{1.299854in}}%
\pgfpathlineto{\pgfqpoint{0.915859in}{1.317068in}}%
\pgfpathlineto{\pgfqpoint{0.981361in}{1.336809in}}%
\pgfpathlineto{\pgfqpoint{1.046037in}{1.359110in}}%
\pgfpathlineto{\pgfqpoint{1.109783in}{1.383945in}}%
\pgfpathlineto{\pgfqpoint{1.172521in}{1.411226in}}%
\pgfusepath{stroke}%
\end{pgfscope}%
\begin{pgfscope}%
\pgfpathrectangle{\pgfqpoint{0.647939in}{0.492442in}}{\pgfqpoint{3.079299in}{3.079299in}}%
\pgfusepath{clip}%
\pgfsetbuttcap%
\pgfsetroundjoin%
\pgfsetlinewidth{0.301125pt}%
\definecolor{currentstroke}{rgb}{0.500000,0.500000,0.500000}%
\pgfsetstrokecolor{currentstroke}%
\pgfsetstrokeopacity{0.300000}%
\pgfsetdash{}{0pt}%
\pgfpathmoveto{\pgfqpoint{0.647939in}{1.192283in}}%
\pgfpathlineto{\pgfqpoint{0.647939in}{1.192283in}}%
\pgfpathlineto{\pgfqpoint{0.715544in}{1.202810in}}%
\pgfpathlineto{\pgfqpoint{0.782768in}{1.215535in}}%
\pgfpathlineto{\pgfqpoint{0.849493in}{1.230644in}}%
\pgfpathlineto{\pgfqpoint{0.915591in}{1.248287in}}%
\pgfpathlineto{\pgfqpoint{0.980926in}{1.268571in}}%
\pgfpathlineto{\pgfqpoint{1.045363in}{1.291545in}}%
\pgfpathlineto{\pgfqpoint{1.108780in}{1.317200in}}%
\pgfpathlineto{\pgfqpoint{1.171079in}{1.345464in}}%
\pgfpathlineto{\pgfqpoint{1.232193in}{1.376206in}}%
\pgfpathlineto{\pgfqpoint{1.292098in}{1.409244in}}%
\pgfpathlineto{\pgfqpoint{1.350814in}{1.444352in}}%
\pgfpathlineto{\pgfqpoint{1.408412in}{1.481267in}}%
\pgfpathlineto{\pgfqpoint{1.465014in}{1.519699in}}%
\pgfpathlineto{\pgfqpoint{1.520790in}{1.559324in}}%
\pgfpathlineto{\pgfqpoint{1.575973in}{1.599767in}}%
\pgfpathlineto{\pgfqpoint{1.630865in}{1.640602in}}%
\pgfpathlineto{\pgfqpoint{1.685820in}{1.681343in}}%
\pgfpathlineto{\pgfqpoint{1.741270in}{1.721376in}}%
\pgfpathlineto{\pgfqpoint{1.797794in}{1.759835in}}%
\pgfpathlineto{\pgfqpoint{1.856243in}{1.795241in}}%
\pgfusepath{stroke}%
\end{pgfscope}%
\begin{pgfscope}%
\pgfpathrectangle{\pgfqpoint{0.647939in}{0.492442in}}{\pgfqpoint{3.079299in}{3.079299in}}%
\pgfusepath{clip}%
\pgfsetbuttcap%
\pgfsetroundjoin%
\pgfsetlinewidth{0.301125pt}%
\definecolor{currentstroke}{rgb}{0.500000,0.500000,0.500000}%
\pgfsetstrokecolor{currentstroke}%
\pgfsetstrokeopacity{0.300000}%
\pgfsetdash{}{0pt}%
\pgfpathmoveto{\pgfqpoint{0.647939in}{1.122299in}}%
\pgfpathlineto{\pgfqpoint{0.647939in}{1.122299in}}%
\pgfpathlineto{\pgfqpoint{0.715513in}{1.133022in}}%
\pgfpathlineto{\pgfqpoint{0.782685in}{1.146010in}}%
\pgfpathlineto{\pgfqpoint{0.849330in}{1.161466in}}%
\pgfpathlineto{\pgfqpoint{0.915304in}{1.179559in}}%
\pgfpathlineto{\pgfqpoint{0.980457in}{1.200413in}}%
\pgfpathlineto{\pgfqpoint{1.044634in}{1.224096in}}%
\pgfpathlineto{\pgfqpoint{1.107691in}{1.250616in}}%
\pgfpathlineto{\pgfqpoint{1.169505in}{1.279917in}}%
\pgfpathlineto{\pgfqpoint{1.229984in}{1.311882in}}%
\pgfpathlineto{\pgfqpoint{1.289077in}{1.346340in}}%
\pgfpathlineto{\pgfqpoint{1.346782in}{1.383076in}}%
\pgfpathlineto{\pgfqpoint{1.403146in}{1.421841in}}%
\pgfpathlineto{\pgfqpoint{1.458270in}{1.462360in}}%
\pgfpathlineto{\pgfqpoint{1.512316in}{1.504304in}}%
\pgfpathlineto{\pgfqpoint{1.565511in}{1.547316in}}%
\pgfusepath{stroke}%
\end{pgfscope}%
\begin{pgfscope}%
\pgfpathrectangle{\pgfqpoint{0.647939in}{0.492442in}}{\pgfqpoint{3.079299in}{3.079299in}}%
\pgfusepath{clip}%
\pgfsetbuttcap%
\pgfsetroundjoin%
\pgfsetlinewidth{0.301125pt}%
\definecolor{currentstroke}{rgb}{0.500000,0.500000,0.500000}%
\pgfsetstrokecolor{currentstroke}%
\pgfsetstrokeopacity{0.300000}%
\pgfsetdash{}{0pt}%
\pgfpathmoveto{\pgfqpoint{0.647939in}{1.052315in}}%
\pgfpathlineto{\pgfqpoint{0.647939in}{1.052315in}}%
\pgfpathlineto{\pgfqpoint{0.715480in}{1.063241in}}%
\pgfpathlineto{\pgfqpoint{0.782597in}{1.076503in}}%
\pgfpathlineto{\pgfqpoint{0.849155in}{1.092322in}}%
\pgfpathlineto{\pgfqpoint{0.914997in}{1.110886in}}%
\pgfpathlineto{\pgfqpoint{0.979953in}{1.132339in}}%
\pgfpathlineto{\pgfqpoint{1.043846in}{1.156769in}}%
\pgfpathlineto{\pgfqpoint{1.106507in}{1.184203in}}%
\pgfpathlineto{\pgfqpoint{1.167786in}{1.214598in}}%
\pgfpathlineto{\pgfqpoint{1.227560in}{1.247851in}}%
\pgfpathlineto{\pgfqpoint{1.285752in}{1.283799in}}%
\pgfpathlineto{\pgfqpoint{1.342332in}{1.322233in}}%
\pgfpathlineto{\pgfqpoint{1.397328in}{1.362913in}}%
\pgfusepath{stroke}%
\end{pgfscope}%
\begin{pgfscope}%
\pgfpathrectangle{\pgfqpoint{0.647939in}{0.492442in}}{\pgfqpoint{3.079299in}{3.079299in}}%
\pgfusepath{clip}%
\pgfsetbuttcap%
\pgfsetroundjoin%
\pgfsetlinewidth{0.301125pt}%
\definecolor{currentstroke}{rgb}{0.500000,0.500000,0.500000}%
\pgfsetstrokecolor{currentstroke}%
\pgfsetstrokeopacity{0.300000}%
\pgfsetdash{}{0pt}%
\pgfpathmoveto{\pgfqpoint{0.647939in}{0.982331in}}%
\pgfpathlineto{\pgfqpoint{0.647939in}{0.982331in}}%
\pgfpathlineto{\pgfqpoint{0.715445in}{0.993467in}}%
\pgfpathlineto{\pgfqpoint{0.782504in}{1.007015in}}%
\pgfpathlineto{\pgfqpoint{0.848970in}{1.023213in}}%
\pgfpathlineto{\pgfqpoint{0.914668in}{1.042271in}}%
\pgfpathlineto{\pgfqpoint{0.979409in}{1.064355in}}%
\pgfusepath{stroke}%
\end{pgfscope}%
\begin{pgfscope}%
\pgfpathrectangle{\pgfqpoint{0.647939in}{0.492442in}}{\pgfqpoint{3.079299in}{3.079299in}}%
\pgfusepath{clip}%
\pgfsetbuttcap%
\pgfsetroundjoin%
\pgfsetlinewidth{0.301125pt}%
\definecolor{currentstroke}{rgb}{0.500000,0.500000,0.500000}%
\pgfsetstrokecolor{currentstroke}%
\pgfsetstrokeopacity{0.300000}%
\pgfsetdash{}{0pt}%
\pgfpathmoveto{\pgfqpoint{0.647939in}{0.912347in}}%
\pgfpathlineto{\pgfqpoint{0.647939in}{0.912347in}}%
\pgfpathlineto{\pgfqpoint{0.715408in}{0.923702in}}%
\pgfpathlineto{\pgfqpoint{0.782405in}{0.937548in}}%
\pgfpathlineto{\pgfqpoint{0.848771in}{0.954143in}}%
\pgfpathlineto{\pgfqpoint{0.914314in}{0.973720in}}%
\pgfpathlineto{\pgfqpoint{0.978822in}{0.996467in}}%
\pgfpathlineto{\pgfqpoint{1.042065in}{1.022516in}}%
\pgfpathlineto{\pgfqpoint{1.103812in}{1.051929in}}%
\pgfpathlineto{\pgfqpoint{1.163845in}{1.084693in}}%
\pgfpathlineto{\pgfqpoint{1.221976in}{1.120717in}}%
\pgfpathlineto{\pgfqpoint{1.278065in}{1.159842in}}%
\pgfpathlineto{\pgfqpoint{1.332036in}{1.201852in}}%
\pgfpathlineto{\pgfqpoint{1.383897in}{1.246441in}}%
\pgfpathlineto{\pgfqpoint{1.433738in}{1.293261in}}%
\pgfpathlineto{\pgfqpoint{1.481732in}{1.341977in}}%
\pgfpathlineto{\pgfqpoint{1.528150in}{1.392190in}}%
\pgfpathlineto{\pgfqpoint{1.573336in}{1.443505in}}%
\pgfusepath{stroke}%
\end{pgfscope}%
\begin{pgfscope}%
\pgfpathrectangle{\pgfqpoint{0.647939in}{0.492442in}}{\pgfqpoint{3.079299in}{3.079299in}}%
\pgfusepath{clip}%
\pgfsetbuttcap%
\pgfsetroundjoin%
\pgfsetlinewidth{0.301125pt}%
\definecolor{currentstroke}{rgb}{0.500000,0.500000,0.500000}%
\pgfsetstrokecolor{currentstroke}%
\pgfsetstrokeopacity{0.300000}%
\pgfsetdash{}{0pt}%
\pgfpathmoveto{\pgfqpoint{0.647939in}{0.842362in}}%
\pgfpathlineto{\pgfqpoint{0.647939in}{0.842362in}}%
\pgfpathlineto{\pgfqpoint{0.715368in}{0.853945in}}%
\pgfpathlineto{\pgfqpoint{0.782300in}{0.868101in}}%
\pgfpathlineto{\pgfqpoint{0.848558in}{0.885113in}}%
\pgfpathlineto{\pgfqpoint{0.913934in}{0.905235in}}%
\pgfpathlineto{\pgfqpoint{0.978187in}{0.928681in}}%
\pgfpathlineto{\pgfqpoint{1.041057in}{0.955607in}}%
\pgfpathlineto{\pgfqpoint{1.102275in}{0.986091in}}%
\pgfusepath{stroke}%
\end{pgfscope}%
\begin{pgfscope}%
\pgfpathrectangle{\pgfqpoint{0.647939in}{0.492442in}}{\pgfqpoint{3.079299in}{3.079299in}}%
\pgfusepath{clip}%
\pgfsetbuttcap%
\pgfsetroundjoin%
\pgfsetlinewidth{0.301125pt}%
\definecolor{currentstroke}{rgb}{0.500000,0.500000,0.500000}%
\pgfsetstrokecolor{currentstroke}%
\pgfsetstrokeopacity{0.300000}%
\pgfsetdash{}{0pt}%
\pgfpathmoveto{\pgfqpoint{0.647939in}{0.772378in}}%
\pgfpathlineto{\pgfqpoint{0.647939in}{0.772378in}}%
\pgfpathlineto{\pgfqpoint{0.715326in}{0.784198in}}%
\pgfpathlineto{\pgfqpoint{0.782187in}{0.798678in}}%
\pgfpathlineto{\pgfqpoint{0.848330in}{0.816125in}}%
\pgfpathlineto{\pgfqpoint{0.913524in}{0.836821in}}%
\pgfpathlineto{\pgfqpoint{0.977499in}{0.861005in}}%
\pgfpathlineto{\pgfqpoint{1.039959in}{0.888854in}}%
\pgfpathlineto{\pgfqpoint{1.100594in}{0.920467in}}%
\pgfpathlineto{\pgfqpoint{1.159106in}{0.955845in}}%
\pgfpathlineto{\pgfqpoint{1.215234in}{0.994890in}}%
\pgfpathlineto{\pgfqpoint{1.268785in}{1.037415in}}%
\pgfpathlineto{\pgfqpoint{1.319672in}{1.083086in}}%
\pgfpathlineto{\pgfqpoint{1.367921in}{1.131523in}}%
\pgfpathlineto{\pgfqpoint{1.413689in}{1.182312in}}%
\pgfpathlineto{\pgfqpoint{1.457256in}{1.235012in}}%
\pgfpathlineto{\pgfqpoint{1.499001in}{1.289158in}}%
\pgfusepath{stroke}%
\end{pgfscope}%
\begin{pgfscope}%
\pgfpathrectangle{\pgfqpoint{0.647939in}{0.492442in}}{\pgfqpoint{3.079299in}{3.079299in}}%
\pgfusepath{clip}%
\pgfsetbuttcap%
\pgfsetroundjoin%
\pgfsetlinewidth{0.301125pt}%
\definecolor{currentstroke}{rgb}{0.500000,0.500000,0.500000}%
\pgfsetstrokecolor{currentstroke}%
\pgfsetstrokeopacity{0.300000}%
\pgfsetdash{}{0pt}%
\pgfpathmoveto{\pgfqpoint{0.647939in}{0.702394in}}%
\pgfpathlineto{\pgfqpoint{0.647939in}{0.702394in}}%
\pgfpathlineto{\pgfqpoint{0.715282in}{0.714460in}}%
\pgfpathlineto{\pgfqpoint{0.782067in}{0.729279in}}%
\pgfpathlineto{\pgfqpoint{0.848085in}{0.747183in}}%
\pgfpathlineto{\pgfqpoint{0.913081in}{0.768482in}}%
\pgfpathlineto{\pgfqpoint{0.976752in}{0.793444in}}%
\pgfpathlineto{\pgfqpoint{1.038761in}{0.822269in}}%
\pgfpathlineto{\pgfqpoint{1.098754in}{0.855069in}}%
\pgfpathlineto{\pgfqpoint{1.156386in}{0.891845in}}%
\pgfpathlineto{\pgfqpoint{1.211363in}{0.932486in}}%
\pgfpathlineto{\pgfqpoint{1.263477in}{0.976741in}}%
\pgfusepath{stroke}%
\end{pgfscope}%
\begin{pgfscope}%
\pgfpathrectangle{\pgfqpoint{0.647939in}{0.492442in}}{\pgfqpoint{3.079299in}{3.079299in}}%
\pgfusepath{clip}%
\pgfsetbuttcap%
\pgfsetroundjoin%
\pgfsetlinewidth{0.301125pt}%
\definecolor{currentstroke}{rgb}{0.500000,0.500000,0.500000}%
\pgfsetstrokecolor{currentstroke}%
\pgfsetstrokeopacity{0.300000}%
\pgfsetdash{}{0pt}%
\pgfpathmoveto{\pgfqpoint{0.647939in}{0.632410in}}%
\pgfpathlineto{\pgfqpoint{0.647939in}{0.632410in}}%
\pgfpathlineto{\pgfqpoint{0.715235in}{0.644732in}}%
\pgfpathlineto{\pgfqpoint{0.781938in}{0.659906in}}%
\pgfpathlineto{\pgfqpoint{0.847822in}{0.678290in}}%
\pgfpathlineto{\pgfqpoint{0.912602in}{0.700225in}}%
\pgfpathlineto{\pgfqpoint{0.975940in}{0.726007in}}%
\pgfpathlineto{\pgfqpoint{1.037453in}{0.755860in}}%
\pgfpathlineto{\pgfqpoint{1.096736in}{0.789907in}}%
\pgfpathlineto{\pgfqpoint{1.153401in}{0.828142in}}%
\pgfpathlineto{\pgfqpoint{1.207121in}{0.870426in}}%
\pgfpathlineto{\pgfqpoint{1.257690in}{0.916424in}}%
\pgfusepath{stroke}%
\end{pgfscope}%
\begin{pgfscope}%
\pgfpathrectangle{\pgfqpoint{0.647939in}{0.492442in}}{\pgfqpoint{3.079299in}{3.079299in}}%
\pgfusepath{clip}%
\pgfsetbuttcap%
\pgfsetroundjoin%
\pgfsetlinewidth{0.301125pt}%
\definecolor{currentstroke}{rgb}{0.500000,0.500000,0.500000}%
\pgfsetstrokecolor{currentstroke}%
\pgfsetstrokeopacity{0.300000}%
\pgfsetdash{}{0pt}%
\pgfpathmoveto{\pgfqpoint{0.647939in}{0.562426in}}%
\pgfpathlineto{\pgfqpoint{0.647939in}{0.562426in}}%
\pgfpathlineto{\pgfqpoint{0.715184in}{0.575015in}}%
\pgfpathlineto{\pgfqpoint{0.781800in}{0.590561in}}%
\pgfpathlineto{\pgfqpoint{0.847538in}{0.609449in}}%
\pgfpathlineto{\pgfqpoint{0.912083in}{0.632054in}}%
\pgfpathlineto{\pgfqpoint{0.975055in}{0.658702in}}%
\pgfpathlineto{\pgfqpoint{1.036021in}{0.689639in}}%
\pgfpathlineto{\pgfqpoint{1.094524in}{0.724993in}}%
\pgfpathlineto{\pgfqpoint{1.150127in}{0.764742in}}%
\pgfusepath{stroke}%
\end{pgfscope}%
\begin{pgfscope}%
\pgfpathrectangle{\pgfqpoint{0.647939in}{0.492442in}}{\pgfqpoint{3.079299in}{3.079299in}}%
\pgfusepath{clip}%
\pgfsetbuttcap%
\pgfsetroundjoin%
\pgfsetlinewidth{0.301125pt}%
\definecolor{currentstroke}{rgb}{0.500000,0.500000,0.500000}%
\pgfsetstrokecolor{currentstroke}%
\pgfsetstrokeopacity{0.300000}%
\pgfsetdash{}{0pt}%
\pgfpathmoveto{\pgfqpoint{3.727238in}{1.978382in}}%
\pgfpathlineto{\pgfqpoint{3.711218in}{1.990078in}}%
\pgfpathlineto{\pgfqpoint{3.657254in}{2.032092in}}%
\pgfpathlineto{\pgfqpoint{3.605959in}{2.077330in}}%
\pgfpathlineto{\pgfqpoint{3.557488in}{2.125583in}}%
\pgfpathlineto{\pgfqpoint{3.512038in}{2.176689in}}%
\pgfpathlineto{\pgfqpoint{3.469852in}{2.230519in}}%
\pgfusepath{stroke}%
\end{pgfscope}%
\begin{pgfscope}%
\pgfpathrectangle{\pgfqpoint{0.647939in}{0.492442in}}{\pgfqpoint{3.079299in}{3.079299in}}%
\pgfusepath{clip}%
\pgfsetbuttcap%
\pgfsetroundjoin%
\pgfsetlinewidth{0.301125pt}%
\definecolor{currentstroke}{rgb}{0.500000,0.500000,0.500000}%
\pgfsetstrokecolor{currentstroke}%
\pgfsetstrokeopacity{0.300000}%
\pgfsetdash{}{0pt}%
\pgfpathmoveto{\pgfqpoint{1.759642in}{0.598095in}}%
\pgfpathlineto{\pgfqpoint{1.692605in}{0.611325in}}%
\pgfpathlineto{\pgfqpoint{1.627716in}{0.632410in}}%
\pgfpathlineto{\pgfqpoint{1.567570in}{0.664123in}}%
\pgfpathlineto{\pgfqpoint{1.567570in}{0.664123in}}%
\pgfpathlineto{\pgfqpoint{1.524389in}{0.700223in}}%
\pgfpathlineto{\pgfqpoint{1.490755in}{0.746089in}}%
\pgfpathlineto{\pgfqpoint{1.470652in}{0.794144in}}%
\pgfusepath{stroke}%
\end{pgfscope}%
\begin{pgfscope}%
\pgfpathrectangle{\pgfqpoint{0.647939in}{0.492442in}}{\pgfqpoint{3.079299in}{3.079299in}}%
\pgfusepath{clip}%
\pgfsetbuttcap%
\pgfsetroundjoin%
\pgfsetlinewidth{0.301125pt}%
\definecolor{currentstroke}{rgb}{0.500000,0.500000,0.500000}%
\pgfsetstrokecolor{currentstroke}%
\pgfsetstrokeopacity{0.300000}%
\pgfsetdash{}{0pt}%
\pgfpathmoveto{\pgfqpoint{2.793691in}{0.618093in}}%
\pgfpathlineto{\pgfqpoint{2.725988in}{0.627964in}}%
\pgfpathlineto{\pgfqpoint{2.658003in}{0.635659in}}%
\pgfpathlineto{\pgfqpoint{2.589806in}{0.641158in}}%
\pgfpathlineto{\pgfqpoint{2.521470in}{0.644527in}}%
\pgfpathlineto{\pgfqpoint{2.453062in}{0.645916in}}%
\pgfpathlineto{\pgfqpoint{2.384639in}{0.645565in}}%
\pgfpathlineto{\pgfqpoint{2.316237in}{0.643795in}}%
\pgfpathlineto{\pgfqpoint{2.247867in}{0.640986in}}%
\pgfpathlineto{\pgfqpoint{2.179523in}{0.637584in}}%
\pgfpathlineto{\pgfqpoint{2.111183in}{0.634094in}}%
\pgfpathlineto{\pgfqpoint{2.042821in}{0.631100in}}%
\pgfpathlineto{\pgfqpoint{1.974422in}{0.629273in}}%
\pgfpathlineto{\pgfqpoint{1.906005in}{0.629394in}}%
\pgfpathlineto{\pgfqpoint{1.837668in}{0.632410in}}%
\pgfusepath{stroke}%
\end{pgfscope}%
\begin{pgfscope}%
\pgfpathrectangle{\pgfqpoint{0.647939in}{0.492442in}}{\pgfqpoint{3.079299in}{3.079299in}}%
\pgfusepath{clip}%
\pgfsetbuttcap%
\pgfsetroundjoin%
\pgfsetlinewidth{0.301125pt}%
\definecolor{currentstroke}{rgb}{0.500000,0.500000,0.500000}%
\pgfsetstrokecolor{currentstroke}%
\pgfsetstrokeopacity{0.300000}%
\pgfsetdash{}{0pt}%
\pgfpathmoveto{\pgfqpoint{0.647939in}{3.206425in}}%
\pgfpathlineto{\pgfqpoint{0.651887in}{3.206796in}}%
\pgfpathlineto{\pgfqpoint{0.719959in}{3.213751in}}%
\pgfpathlineto{\pgfqpoint{0.787907in}{3.221821in}}%
\pgfpathlineto{\pgfqpoint{0.855716in}{3.230991in}}%
\pgfpathlineto{\pgfqpoint{0.923375in}{3.241211in}}%
\pgfpathlineto{\pgfqpoint{0.990883in}{3.252388in}}%
\pgfusepath{stroke}%
\end{pgfscope}%
\begin{pgfscope}%
\pgfpathrectangle{\pgfqpoint{0.647939in}{0.492442in}}{\pgfqpoint{3.079299in}{3.079299in}}%
\pgfusepath{clip}%
\pgfsetbuttcap%
\pgfsetroundjoin%
\pgfsetlinewidth{0.301125pt}%
\definecolor{currentstroke}{rgb}{0.500000,0.500000,0.500000}%
\pgfsetstrokecolor{currentstroke}%
\pgfsetstrokeopacity{0.300000}%
\pgfsetdash{}{0pt}%
\pgfpathmoveto{\pgfqpoint{0.647939in}{3.066053in}}%
\pgfpathlineto{\pgfqpoint{0.651931in}{3.066437in}}%
\pgfpathlineto{\pgfqpoint{0.719985in}{3.073566in}}%
\pgfpathlineto{\pgfqpoint{0.787907in}{3.081853in}}%
\pgfpathlineto{\pgfqpoint{0.855679in}{3.091289in}}%
\pgfpathlineto{\pgfqpoint{0.923290in}{3.101826in}}%
\pgfpathlineto{\pgfqpoint{0.990734in}{3.113378in}}%
\pgfpathlineto{\pgfqpoint{1.058024in}{3.125807in}}%
\pgfpathlineto{\pgfqpoint{1.125180in}{3.138939in}}%
\pgfpathlineto{\pgfqpoint{1.192238in}{3.152564in}}%
\pgfpathlineto{\pgfqpoint{1.259246in}{3.166437in}}%
\pgfpathlineto{\pgfqpoint{1.326259in}{3.180286in}}%
\pgfpathlineto{\pgfqpoint{1.393337in}{3.193812in}}%
\pgfpathlineto{\pgfqpoint{1.460540in}{3.206699in}}%
\pgfpathlineto{\pgfqpoint{1.527919in}{3.218625in}}%
\pgfpathlineto{\pgfqpoint{1.595509in}{3.229278in}}%
\pgfpathlineto{\pgfqpoint{1.663325in}{3.238369in}}%
\pgfpathlineto{\pgfqpoint{1.731357in}{3.245652in}}%
\pgfpathlineto{\pgfqpoint{1.799573in}{3.250941in}}%
\pgfpathlineto{\pgfqpoint{1.867918in}{3.254143in}}%
\pgfpathlineto{\pgfqpoint{1.936329in}{3.255275in}}%
\pgfpathlineto{\pgfqpoint{2.004745in}{3.254470in}}%
\pgfpathlineto{\pgfqpoint{2.073124in}{3.251991in}}%
\pgfpathlineto{\pgfqpoint{2.141448in}{3.248240in}}%
\pgfpathlineto{\pgfqpoint{2.209731in}{3.243770in}}%
\pgfpathlineto{\pgfqpoint{2.278011in}{3.239274in}}%
\pgfpathlineto{\pgfqpoint{2.346336in}{3.235568in}}%
\pgfpathlineto{\pgfqpoint{2.414726in}{3.233590in}}%
\pgfpathlineto{\pgfqpoint{2.483130in}{3.234366in}}%
\pgfpathlineto{\pgfqpoint{2.551369in}{3.238934in}}%
\pgfpathlineto{\pgfqpoint{2.619113in}{3.248178in}}%
\pgfpathlineto{\pgfqpoint{2.685923in}{3.262645in}}%
\pgfpathlineto{\pgfqpoint{2.751354in}{3.282430in}}%
\pgfpathlineto{\pgfqpoint{2.815079in}{3.307186in}}%
\pgfusepath{stroke}%
\end{pgfscope}%
\begin{pgfscope}%
\pgfpathrectangle{\pgfqpoint{0.647939in}{0.492442in}}{\pgfqpoint{3.079299in}{3.079299in}}%
\pgfusepath{clip}%
\pgfsetbuttcap%
\pgfsetroundjoin%
\pgfsetlinewidth{0.301125pt}%
\definecolor{currentstroke}{rgb}{0.500000,0.500000,0.500000}%
\pgfsetstrokecolor{currentstroke}%
\pgfsetstrokeopacity{0.300000}%
\pgfsetdash{}{0pt}%
\pgfpathmoveto{\pgfqpoint{3.517286in}{1.962108in}}%
\pgfpathlineto{\pgfqpoint{3.466545in}{2.007987in}}%
\pgfpathlineto{\pgfqpoint{3.417947in}{2.056133in}}%
\pgfpathlineto{\pgfqpoint{3.371621in}{2.106466in}}%
\pgfpathlineto{\pgfqpoint{3.327763in}{2.158960in}}%
\pgfpathlineto{\pgfqpoint{3.286663in}{2.213633in}}%
\pgfpathlineto{\pgfqpoint{3.248722in}{2.270539in}}%
\pgfpathlineto{\pgfqpoint{3.214495in}{2.329744in}}%
\pgfpathlineto{\pgfqpoint{3.184708in}{2.391291in}}%
\pgfpathlineto{\pgfqpoint{3.160250in}{2.455122in}}%
\pgfpathlineto{\pgfqpoint{3.142090in}{2.520995in}}%
\pgfpathlineto{\pgfqpoint{3.131099in}{2.588410in}}%
\pgfpathlineto{\pgfqpoint{3.127798in}{2.656621in}}%
\pgfpathlineto{\pgfqpoint{3.132178in}{2.724780in}}%
\pgfpathlineto{\pgfqpoint{3.143699in}{2.792128in}}%
\pgfpathlineto{\pgfqpoint{3.161492in}{2.858122in}}%
\pgfusepath{stroke}%
\end{pgfscope}%
\begin{pgfscope}%
\pgfpathrectangle{\pgfqpoint{0.647939in}{0.492442in}}{\pgfqpoint{3.079299in}{3.079299in}}%
\pgfusepath{clip}%
\pgfsetbuttcap%
\pgfsetroundjoin%
\pgfsetlinewidth{0.301125pt}%
\definecolor{currentstroke}{rgb}{0.500000,0.500000,0.500000}%
\pgfsetstrokecolor{currentstroke}%
\pgfsetstrokeopacity{0.300000}%
\pgfsetdash{}{0pt}%
\pgfpathmoveto{\pgfqpoint{1.583407in}{3.344246in}}%
\pgfpathlineto{\pgfqpoint{1.651253in}{3.353121in}}%
\pgfpathlineto{\pgfqpoint{1.719298in}{3.360295in}}%
\pgfpathlineto{\pgfqpoint{1.787512in}{3.365597in}}%
\pgfpathlineto{\pgfqpoint{1.855851in}{3.368936in}}%
\pgfpathlineto{\pgfqpoint{1.924259in}{3.370318in}}%
\pgfpathlineto{\pgfqpoint{1.992681in}{3.369871in}}%
\pgfpathlineto{\pgfqpoint{2.061075in}{3.367841in}}%
\pgfpathlineto{\pgfqpoint{2.129424in}{3.364591in}}%
\pgfpathlineto{\pgfqpoint{2.197737in}{3.360606in}}%
\pgfpathlineto{\pgfqpoint{2.266041in}{3.356488in}}%
\pgfpathlineto{\pgfqpoint{2.334377in}{3.352965in}}%
\pgfpathlineto{\pgfqpoint{2.402766in}{3.350863in}}%
\pgfpathlineto{\pgfqpoint{2.471178in}{3.351059in}}%
\pgfpathlineto{\pgfqpoint{2.539496in}{3.354437in}}%
\pgfpathlineto{\pgfqpoint{2.607493in}{3.361789in}}%
\pgfusepath{stroke}%
\end{pgfscope}%
\begin{pgfscope}%
\pgfpathrectangle{\pgfqpoint{0.647939in}{0.492442in}}{\pgfqpoint{3.079299in}{3.079299in}}%
\pgfusepath{clip}%
\pgfsetbuttcap%
\pgfsetroundjoin%
\pgfsetlinewidth{0.301125pt}%
\definecolor{currentstroke}{rgb}{0.500000,0.500000,0.500000}%
\pgfsetstrokecolor{currentstroke}%
\pgfsetstrokeopacity{0.300000}%
\pgfsetdash{}{0pt}%
\pgfpathmoveto{\pgfqpoint{3.447302in}{1.682171in}}%
\pgfpathlineto{\pgfqpoint{3.390625in}{1.720503in}}%
\pgfpathlineto{\pgfqpoint{3.334851in}{1.760139in}}%
\pgfpathlineto{\pgfqpoint{3.279896in}{1.800905in}}%
\pgfpathlineto{\pgfqpoint{3.225678in}{1.842647in}}%
\pgfpathlineto{\pgfqpoint{3.172115in}{1.885225in}}%
\pgfpathlineto{\pgfqpoint{3.119129in}{1.928521in}}%
\pgfpathlineto{\pgfqpoint{3.066662in}{1.972440in}}%
\pgfpathlineto{\pgfqpoint{3.014671in}{2.016925in}}%
\pgfpathlineto{\pgfqpoint{2.963144in}{2.061943in}}%
\pgfpathlineto{\pgfqpoint{2.912114in}{2.107520in}}%
\pgfpathlineto{\pgfqpoint{2.861701in}{2.153773in}}%
\pgfpathlineto{\pgfqpoint{2.812172in}{2.200965in}}%
\pgfpathlineto{\pgfqpoint{2.764108in}{2.249627in}}%
\pgfpathlineto{\pgfqpoint{2.718812in}{2.300829in}}%
\pgfpathlineto{\pgfqpoint{2.679513in}{2.356583in}}%
\pgfpathlineto{\pgfqpoint{2.679513in}{2.356583in}}%
\pgfpathlineto{\pgfqpoint{2.657352in}{2.406450in}}%
\pgfpathlineto{\pgfqpoint{2.657352in}{2.406450in}}%
\pgfpathlineto{\pgfqpoint{2.650184in}{2.451739in}}%
\pgfpathlineto{\pgfqpoint{2.655232in}{2.497763in}}%
\pgfpathlineto{\pgfqpoint{2.669724in}{2.542723in}}%
\pgfpathlineto{\pgfqpoint{2.694767in}{2.594784in}}%
\pgfpathlineto{\pgfqpoint{2.729579in}{2.653455in}}%
\pgfpathlineto{\pgfqpoint{2.767319in}{2.710423in}}%
\pgfpathlineto{\pgfqpoint{2.806686in}{2.766309in}}%
\pgfpathlineto{\pgfqpoint{2.847090in}{2.821472in}}%
\pgfusepath{stroke}%
\end{pgfscope}%
\begin{pgfscope}%
\pgfpathrectangle{\pgfqpoint{0.647939in}{0.492442in}}{\pgfqpoint{3.079299in}{3.079299in}}%
\pgfusepath{clip}%
\pgfsetbuttcap%
\pgfsetroundjoin%
\pgfsetlinewidth{0.301125pt}%
\definecolor{currentstroke}{rgb}{0.500000,0.500000,0.500000}%
\pgfsetstrokecolor{currentstroke}%
\pgfsetstrokeopacity{0.300000}%
\pgfsetdash{}{0pt}%
\pgfpathmoveto{\pgfqpoint{3.384247in}{2.332388in}}%
\pgfpathlineto{\pgfqpoint{3.353392in}{2.393403in}}%
\pgfpathlineto{\pgfqpoint{3.327535in}{2.456686in}}%
\pgfpathlineto{\pgfqpoint{3.307334in}{2.521980in}}%
\pgfpathlineto{\pgfqpoint{3.293412in}{2.588880in}}%
\pgfpathlineto{\pgfqpoint{3.286227in}{2.656824in}}%
\pgfpathlineto{\pgfqpoint{3.285960in}{2.725147in}}%
\pgfpathlineto{\pgfqpoint{3.292458in}{2.793176in}}%
\pgfpathlineto{\pgfqpoint{3.305289in}{2.860311in}}%
\pgfpathlineto{\pgfqpoint{3.323836in}{2.926106in}}%
\pgfusepath{stroke}%
\end{pgfscope}%
\begin{pgfscope}%
\pgfpathrectangle{\pgfqpoint{0.647939in}{0.492442in}}{\pgfqpoint{3.079299in}{3.079299in}}%
\pgfusepath{clip}%
\pgfsetbuttcap%
\pgfsetroundjoin%
\pgfsetlinewidth{0.301125pt}%
\definecolor{currentstroke}{rgb}{0.500000,0.500000,0.500000}%
\pgfsetstrokecolor{currentstroke}%
\pgfsetstrokeopacity{0.300000}%
\pgfsetdash{}{0pt}%
\pgfpathmoveto{\pgfqpoint{1.255430in}{2.899204in}}%
\pgfpathlineto{\pgfqpoint{1.322135in}{2.914464in}}%
\pgfpathlineto{\pgfqpoint{1.388893in}{2.929496in}}%
\pgfpathlineto{\pgfqpoint{1.455775in}{2.943958in}}%
\pgfpathlineto{\pgfqpoint{1.522849in}{2.957489in}}%
\pgfpathlineto{\pgfqpoint{1.590170in}{2.969727in}}%
\pgfpathlineto{\pgfqpoint{1.657767in}{2.980320in}}%
\pgfpathlineto{\pgfqpoint{1.725639in}{2.988952in}}%
\pgfpathlineto{\pgfqpoint{1.793756in}{2.995360in}}%
\pgfpathlineto{\pgfqpoint{1.862055in}{2.999371in}}%
\pgfpathlineto{\pgfqpoint{1.930455in}{3.000930in}}%
\pgfpathlineto{\pgfqpoint{1.998868in}{3.000128in}}%
\pgfpathlineto{\pgfqpoint{2.067226in}{2.997212in}}%
\pgfpathlineto{\pgfqpoint{2.135495in}{2.992600in}}%
\pgfpathlineto{\pgfqpoint{2.203685in}{2.986904in}}%
\pgfpathlineto{\pgfqpoint{2.271855in}{2.980965in}}%
\pgfpathlineto{\pgfqpoint{2.340089in}{2.975856in}}%
\pgfpathlineto{\pgfqpoint{2.408435in}{2.972886in}}%
\pgfpathlineto{\pgfqpoint{2.476818in}{2.973585in}}%
\pgfpathlineto{\pgfqpoint{2.544915in}{2.979558in}}%
\pgfpathlineto{\pgfqpoint{2.612078in}{2.992128in}}%
\pgfpathlineto{\pgfqpoint{2.677477in}{3.011869in}}%
\pgfusepath{stroke}%
\end{pgfscope}%
\begin{pgfscope}%
\pgfpathrectangle{\pgfqpoint{0.647939in}{0.492442in}}{\pgfqpoint{3.079299in}{3.079299in}}%
\pgfusepath{clip}%
\pgfsetbuttcap%
\pgfsetroundjoin%
\pgfsetlinewidth{0.301125pt}%
\definecolor{currentstroke}{rgb}{0.500000,0.500000,0.500000}%
\pgfsetstrokecolor{currentstroke}%
\pgfsetstrokeopacity{0.300000}%
\pgfsetdash{}{0pt}%
\pgfpathmoveto{\pgfqpoint{1.277796in}{2.382012in}}%
\pgfpathlineto{\pgfqpoint{1.343514in}{2.401079in}}%
\pgfpathlineto{\pgfqpoint{1.409251in}{2.420083in}}%
\pgfpathlineto{\pgfqpoint{1.475122in}{2.438614in}}%
\pgfpathlineto{\pgfqpoint{1.541243in}{2.456223in}}%
\pgfpathlineto{\pgfqpoint{1.607720in}{2.472422in}}%
\pgfusepath{stroke}%
\end{pgfscope}%
\begin{pgfscope}%
\pgfpathrectangle{\pgfqpoint{0.647939in}{0.492442in}}{\pgfqpoint{3.079299in}{3.079299in}}%
\pgfusepath{clip}%
\pgfsetbuttcap%
\pgfsetroundjoin%
\pgfsetlinewidth{0.301125pt}%
\definecolor{currentstroke}{rgb}{0.500000,0.500000,0.500000}%
\pgfsetstrokecolor{currentstroke}%
\pgfsetstrokeopacity{0.300000}%
\pgfsetdash{}{0pt}%
\pgfpathmoveto{\pgfqpoint{3.027398in}{2.102076in}}%
\pgfpathlineto{\pgfqpoint{2.980178in}{2.151582in}}%
\pgfpathlineto{\pgfqpoint{2.934583in}{2.202584in}}%
\pgfpathlineto{\pgfqpoint{2.891246in}{2.255500in}}%
\pgfpathlineto{\pgfqpoint{2.851302in}{2.310978in}}%
\pgfpathlineto{\pgfqpoint{2.816777in}{2.369898in}}%
\pgfpathlineto{\pgfqpoint{2.791107in}{2.433009in}}%
\pgfpathlineto{\pgfqpoint{2.778717in}{2.499757in}}%
\pgfpathlineto{\pgfqpoint{2.781028in}{2.562381in}}%
\pgfusepath{stroke}%
\end{pgfscope}%
\begin{pgfscope}%
\pgfpathrectangle{\pgfqpoint{0.647939in}{0.492442in}}{\pgfqpoint{3.079299in}{3.079299in}}%
\pgfusepath{clip}%
\pgfsetbuttcap%
\pgfsetroundjoin%
\pgfsetlinewidth{0.301125pt}%
\definecolor{currentstroke}{rgb}{0.500000,0.500000,0.500000}%
\pgfsetstrokecolor{currentstroke}%
\pgfsetstrokeopacity{0.300000}%
\pgfsetdash{}{0pt}%
\pgfpathmoveto{\pgfqpoint{1.775980in}{2.872143in}}%
\pgfpathlineto{\pgfqpoint{1.844204in}{2.877237in}}%
\pgfpathlineto{\pgfqpoint{1.912574in}{2.879667in}}%
\pgfpathlineto{\pgfqpoint{1.980988in}{2.879434in}}%
\pgfpathlineto{\pgfqpoint{2.049352in}{2.876716in}}%
\pgfpathlineto{\pgfqpoint{2.117605in}{2.871901in}}%
\pgfusepath{stroke}%
\end{pgfscope}%
\begin{pgfscope}%
\pgfpathrectangle{\pgfqpoint{0.647939in}{0.492442in}}{\pgfqpoint{3.079299in}{3.079299in}}%
\pgfusepath{clip}%
\pgfsetbuttcap%
\pgfsetroundjoin%
\pgfsetlinewidth{0.301125pt}%
\definecolor{currentstroke}{rgb}{0.500000,0.500000,0.500000}%
\pgfsetstrokecolor{currentstroke}%
\pgfsetstrokeopacity{0.300000}%
\pgfsetdash{}{0pt}%
\pgfpathmoveto{\pgfqpoint{0.899774in}{1.515299in}}%
\pgfpathlineto{\pgfqpoint{0.965856in}{1.533025in}}%
\pgfpathlineto{\pgfqpoint{1.031301in}{1.552973in}}%
\pgfpathlineto{\pgfqpoint{1.096045in}{1.575090in}}%
\pgfpathlineto{\pgfqpoint{1.160050in}{1.599262in}}%
\pgfpathlineto{\pgfqpoint{1.223312in}{1.625323in}}%
\pgfpathlineto{\pgfqpoint{1.285863in}{1.653050in}}%
\pgfpathlineto{\pgfqpoint{1.347780in}{1.682171in}}%
\pgfusepath{stroke}%
\end{pgfscope}%
\begin{pgfscope}%
\pgfpathrectangle{\pgfqpoint{0.647939in}{0.492442in}}{\pgfqpoint{3.079299in}{3.079299in}}%
\pgfusepath{clip}%
\pgfsetbuttcap%
\pgfsetroundjoin%
\pgfsetlinewidth{0.301125pt}%
\definecolor{currentstroke}{rgb}{0.500000,0.500000,0.500000}%
\pgfsetstrokecolor{currentstroke}%
\pgfsetstrokeopacity{0.300000}%
\pgfsetdash{}{0pt}%
\pgfpathmoveto{\pgfqpoint{1.417764in}{2.172060in}}%
\pgfpathlineto{\pgfqpoint{1.482684in}{2.193687in}}%
\pgfpathlineto{\pgfqpoint{1.547860in}{2.214527in}}%
\pgfpathlineto{\pgfqpoint{1.613447in}{2.234021in}}%
\pgfpathlineto{\pgfqpoint{1.679584in}{2.251538in}}%
\pgfpathlineto{\pgfqpoint{1.746363in}{2.266377in}}%
\pgfpathlineto{\pgfqpoint{1.813803in}{2.277769in}}%
\pgfpathlineto{\pgfqpoint{1.881812in}{2.284881in}}%
\pgfpathlineto{\pgfqpoint{1.950148in}{2.286821in}}%
\pgfpathlineto{\pgfqpoint{2.018361in}{2.282622in}}%
\pgfpathlineto{\pgfqpoint{2.085684in}{2.271128in}}%
\pgfpathlineto{\pgfqpoint{2.150604in}{2.250358in}}%
\pgfpathlineto{\pgfqpoint{2.150604in}{2.250358in}}%
\pgfpathlineto{\pgfqpoint{2.198007in}{2.223910in}}%
\pgfpathlineto{\pgfqpoint{2.198007in}{2.223910in}}%
\pgfpathlineto{\pgfqpoint{2.223525in}{2.198308in}}%
\pgfpathlineto{\pgfqpoint{2.223525in}{2.198308in}}%
\pgfpathlineto{\pgfqpoint{2.234936in}{2.171896in}}%
\pgfpathlineto{\pgfqpoint{2.234531in}{2.142500in}}%
\pgfusepath{stroke}%
\end{pgfscope}%
\begin{pgfscope}%
\pgfpathrectangle{\pgfqpoint{0.647939in}{0.492442in}}{\pgfqpoint{3.079299in}{3.079299in}}%
\pgfusepath{clip}%
\pgfsetbuttcap%
\pgfsetroundjoin%
\pgfsetlinewidth{0.301125pt}%
\definecolor{currentstroke}{rgb}{0.500000,0.500000,0.500000}%
\pgfsetstrokecolor{currentstroke}%
\pgfsetstrokeopacity{0.300000}%
\pgfsetdash{}{0pt}%
\pgfpathmoveto{\pgfqpoint{2.944179in}{1.923880in}}%
\pgfpathlineto{\pgfqpoint{2.887429in}{1.962108in}}%
\pgfpathlineto{\pgfqpoint{2.830035in}{1.999351in}}%
\pgfpathlineto{\pgfqpoint{2.771766in}{2.035193in}}%
\pgfpathlineto{\pgfqpoint{2.712354in}{2.069088in}}%
\pgfpathlineto{\pgfqpoint{2.651483in}{2.100251in}}%
\pgfpathlineto{\pgfqpoint{2.588785in}{2.127475in}}%
\pgfpathlineto{\pgfqpoint{2.523886in}{2.148711in}}%
\pgfpathlineto{\pgfqpoint{2.456743in}{2.160197in}}%
\pgfpathlineto{\pgfqpoint{2.456743in}{2.160197in}}%
\pgfpathlineto{\pgfqpoint{2.400612in}{2.158193in}}%
\pgfusepath{stroke}%
\end{pgfscope}%
\begin{pgfscope}%
\pgfpathrectangle{\pgfqpoint{0.647939in}{0.492442in}}{\pgfqpoint{3.079299in}{3.079299in}}%
\pgfusepath{clip}%
\pgfsetbuttcap%
\pgfsetroundjoin%
\pgfsetlinewidth{0.301125pt}%
\definecolor{currentstroke}{rgb}{0.500000,0.500000,0.500000}%
\pgfsetstrokecolor{currentstroke}%
\pgfsetstrokeopacity{0.300000}%
\pgfsetdash{}{0pt}%
\pgfpathmoveto{\pgfqpoint{1.487748in}{1.962108in}}%
\pgfpathlineto{\pgfqpoint{1.551328in}{1.987402in}}%
\pgfpathlineto{\pgfqpoint{1.615317in}{2.011638in}}%
\pgfpathlineto{\pgfqpoint{1.679940in}{2.034104in}}%
\pgfpathlineto{\pgfqpoint{1.745410in}{2.053929in}}%
\pgfpathlineto{\pgfqpoint{1.811877in}{2.070004in}}%
\pgfpathlineto{\pgfqpoint{1.879357in}{2.080838in}}%
\pgfpathlineto{\pgfqpoint{1.947534in}{2.084176in}}%
\pgfpathlineto{\pgfqpoint{2.015017in}{2.075727in}}%
\pgfpathlineto{\pgfqpoint{2.015017in}{2.075727in}}%
\pgfpathlineto{\pgfqpoint{2.053992in}{2.060711in}}%
\pgfpathlineto{\pgfqpoint{2.053992in}{2.060711in}}%
\pgfpathlineto{\pgfqpoint{2.077449in}{2.041116in}}%
\pgfpathlineto{\pgfqpoint{2.077449in}{2.041116in}}%
\pgfpathlineto{\pgfqpoint{2.087930in}{2.018687in}}%
\pgfusepath{stroke}%
\end{pgfscope}%
\begin{pgfscope}%
\pgfpathrectangle{\pgfqpoint{0.647939in}{0.492442in}}{\pgfqpoint{3.079299in}{3.079299in}}%
\pgfusepath{clip}%
\pgfsetbuttcap%
\pgfsetroundjoin%
\pgfsetlinewidth{0.301125pt}%
\definecolor{currentstroke}{rgb}{0.500000,0.500000,0.500000}%
\pgfsetstrokecolor{currentstroke}%
\pgfsetstrokeopacity{0.300000}%
\pgfsetdash{}{0pt}%
\pgfpathmoveto{\pgfqpoint{1.849786in}{2.586780in}}%
\pgfpathlineto{\pgfqpoint{1.918126in}{2.589711in}}%
\pgfpathlineto{\pgfqpoint{1.986526in}{2.589058in}}%
\pgfpathlineto{\pgfqpoint{2.054802in}{2.584855in}}%
\pgfpathlineto{\pgfqpoint{2.122801in}{2.577381in}}%
\pgfpathlineto{\pgfqpoint{2.190455in}{2.567201in}}%
\pgfpathlineto{\pgfqpoint{2.257834in}{2.555284in}}%
\pgfpathlineto{\pgfqpoint{2.325199in}{2.543283in}}%
\pgfpathlineto{\pgfqpoint{2.392970in}{2.534228in}}%
\pgfpathlineto{\pgfqpoint{2.451555in}{2.532854in}}%
\pgfpathlineto{\pgfqpoint{2.512916in}{2.543937in}}%
\pgfpathlineto{\pgfqpoint{2.574107in}{2.573310in}}%
\pgfpathlineto{\pgfqpoint{2.628204in}{2.614751in}}%
\pgfpathlineto{\pgfqpoint{2.677477in}{2.661948in}}%
\pgfusepath{stroke}%
\end{pgfscope}%
\begin{pgfscope}%
\pgfpathrectangle{\pgfqpoint{0.647939in}{0.492442in}}{\pgfqpoint{3.079299in}{3.079299in}}%
\pgfusepath{clip}%
\pgfsetbuttcap%
\pgfsetroundjoin%
\pgfsetlinewidth{0.301125pt}%
\definecolor{currentstroke}{rgb}{0.500000,0.500000,0.500000}%
\pgfsetstrokecolor{currentstroke}%
\pgfsetstrokeopacity{0.300000}%
\pgfsetdash{}{0pt}%
\pgfpathmoveto{\pgfqpoint{1.494645in}{2.490098in}}%
\pgfpathlineto{\pgfqpoint{1.560995in}{2.506823in}}%
\pgfpathlineto{\pgfqpoint{1.627716in}{2.521980in}}%
\pgfpathlineto{\pgfqpoint{1.694869in}{2.535069in}}%
\pgfpathlineto{\pgfqpoint{1.762467in}{2.545590in}}%
\pgfpathlineto{\pgfqpoint{1.830465in}{2.553074in}}%
\pgfusepath{stroke}%
\end{pgfscope}%
\begin{pgfscope}%
\pgfpathrectangle{\pgfqpoint{0.647939in}{0.492442in}}{\pgfqpoint{3.079299in}{3.079299in}}%
\pgfusepath{clip}%
\pgfsetbuttcap%
\pgfsetroundjoin%
\pgfsetlinewidth{0.301125pt}%
\definecolor{currentstroke}{rgb}{0.500000,0.500000,0.500000}%
\pgfsetstrokecolor{currentstroke}%
\pgfsetstrokeopacity{0.300000}%
\pgfsetdash{}{0pt}%
\pgfpathmoveto{\pgfqpoint{1.627716in}{2.312028in}}%
\pgfpathlineto{\pgfqpoint{1.694287in}{2.327805in}}%
\pgfpathlineto{\pgfqpoint{1.761449in}{2.340802in}}%
\pgfpathlineto{\pgfqpoint{1.829181in}{2.350331in}}%
\pgfpathlineto{\pgfqpoint{1.897359in}{2.355708in}}%
\pgfpathlineto{\pgfqpoint{1.965730in}{2.356295in}}%
\pgfpathlineto{\pgfqpoint{2.033921in}{2.351502in}}%
\pgfpathlineto{\pgfqpoint{2.101413in}{2.340757in}}%
\pgfpathlineto{\pgfqpoint{2.167469in}{2.323314in}}%
\pgfpathlineto{\pgfqpoint{2.230620in}{2.297518in}}%
\pgfpathlineto{\pgfqpoint{2.230620in}{2.297518in}}%
\pgfpathlineto{\pgfqpoint{2.273946in}{2.269157in}}%
\pgfpathlineto{\pgfqpoint{2.273946in}{2.269157in}}%
\pgfusepath{stroke}%
\end{pgfscope}%
\begin{pgfscope}%
\pgfpathrectangle{\pgfqpoint{0.647939in}{0.492442in}}{\pgfqpoint{3.079299in}{3.079299in}}%
\pgfusepath{clip}%
\pgfsetbuttcap%
\pgfsetroundjoin%
\pgfsetlinewidth{0.301125pt}%
\definecolor{currentstroke}{rgb}{0.500000,0.500000,0.500000}%
\pgfsetstrokecolor{currentstroke}%
\pgfsetstrokeopacity{0.300000}%
\pgfsetdash{}{0pt}%
\pgfpathmoveto{\pgfqpoint{2.730015in}{2.198247in}}%
\pgfpathlineto{\pgfqpoint{2.677477in}{2.242044in}}%
\pgfpathlineto{\pgfqpoint{2.626463in}{2.287529in}}%
\pgfpathlineto{\pgfqpoint{2.580765in}{2.337979in}}%
\pgfpathlineto{\pgfqpoint{2.580765in}{2.337979in}}%
\pgfpathlineto{\pgfqpoint{2.562633in}{2.370756in}}%
\pgfpathlineto{\pgfqpoint{2.562633in}{2.370756in}}%
\pgfpathlineto{\pgfqpoint{2.557313in}{2.402171in}}%
\pgfpathlineto{\pgfqpoint{2.563035in}{2.434179in}}%
\pgfpathlineto{\pgfqpoint{2.576625in}{2.466382in}}%
\pgfpathlineto{\pgfqpoint{2.600798in}{2.507783in}}%
\pgfusepath{stroke}%
\end{pgfscope}%
\begin{pgfscope}%
\pgfpathrectangle{\pgfqpoint{0.647939in}{0.492442in}}{\pgfqpoint{3.079299in}{3.079299in}}%
\pgfusepath{clip}%
\pgfsetbuttcap%
\pgfsetroundjoin%
\pgfsetlinewidth{0.301125pt}%
\definecolor{currentstroke}{rgb}{0.500000,0.500000,0.500000}%
\pgfsetstrokecolor{currentstroke}%
\pgfsetstrokeopacity{0.300000}%
\pgfsetdash{}{0pt}%
\pgfpathmoveto{\pgfqpoint{1.909242in}{2.522182in}}%
\pgfpathlineto{\pgfqpoint{1.977636in}{2.521980in}}%
\pgfpathlineto{\pgfqpoint{2.045906in}{2.517806in}}%
\pgfpathlineto{\pgfqpoint{2.113837in}{2.509790in}}%
\pgfpathlineto{\pgfqpoint{2.181283in}{2.498362in}}%
\pgfpathlineto{\pgfqpoint{2.248243in}{2.484310in}}%
\pgfpathlineto{\pgfqpoint{2.314951in}{2.469080in}}%
\pgfpathlineto{\pgfqpoint{2.382029in}{2.455789in}}%
\pgfpathlineto{\pgfqpoint{2.449856in}{2.452902in}}%
\pgfpathlineto{\pgfqpoint{2.449856in}{2.452902in}}%
\pgfpathlineto{\pgfqpoint{2.490120in}{2.461966in}}%
\pgfusepath{stroke}%
\end{pgfscope}%
\begin{pgfscope}%
\pgfpathrectangle{\pgfqpoint{0.647939in}{0.492442in}}{\pgfqpoint{3.079299in}{3.079299in}}%
\pgfusepath{clip}%
\pgfsetbuttcap%
\pgfsetroundjoin%
\pgfsetlinewidth{0.301125pt}%
\definecolor{currentstroke}{rgb}{0.500000,0.500000,0.500000}%
\pgfsetstrokecolor{currentstroke}%
\pgfsetstrokeopacity{0.300000}%
\pgfsetdash{}{0pt}%
\pgfpathmoveto{\pgfqpoint{2.736190in}{1.916084in}}%
\pgfpathlineto{\pgfqpoint{2.672618in}{1.941292in}}%
\pgfpathlineto{\pgfqpoint{2.607493in}{1.962108in}}%
\pgfpathlineto{\pgfqpoint{2.540857in}{1.977329in}}%
\pgfpathlineto{\pgfqpoint{2.473058in}{1.985690in}}%
\pgfpathlineto{\pgfqpoint{2.404783in}{1.986094in}}%
\pgfpathlineto{\pgfqpoint{2.337000in}{1.978027in}}%
\pgfpathlineto{\pgfqpoint{2.270661in}{1.961885in}}%
\pgfusepath{stroke}%
\end{pgfscope}%
\begin{pgfscope}%
\pgfpathrectangle{\pgfqpoint{0.647939in}{0.492442in}}{\pgfqpoint{3.079299in}{3.079299in}}%
\pgfusepath{clip}%
\pgfsetroundcap%
\pgfsetroundjoin%
\pgfsetlinewidth{0.301125pt}%
\definecolor{currentstroke}{rgb}{0.500000,0.500000,0.500000}%
\pgfsetstrokecolor{currentstroke}%
\pgfsetstrokeopacity{0.300000}%
\pgfsetdash{}{0pt}%
\pgfpathmoveto{\pgfqpoint{1.414596in}{1.044643in}}%
\pgfusepath{stroke}%
\end{pgfscope}%
\begin{pgfscope}%
\pgfpathrectangle{\pgfqpoint{0.647939in}{0.492442in}}{\pgfqpoint{3.079299in}{3.079299in}}%
\pgfusepath{clip}%
\pgfsetroundcap%
\pgfsetroundjoin%
\definecolor{currentfill}{rgb}{0.500000,0.500000,0.500000}%
\pgfsetfillcolor{currentfill}%
\pgfsetfillopacity{0.300000}%
\pgfsetlinewidth{0.301125pt}%
\definecolor{currentstroke}{rgb}{0.500000,0.500000,0.500000}%
\pgfsetstrokecolor{currentstroke}%
\pgfsetstrokeopacity{0.300000}%
\pgfsetdash{}{0pt}%
\pgfpathmoveto{\pgfqpoint{0.000000in}{0.000000in}}%
\pgfpathlineto{\pgfqpoint{0.000000in}{0.000000in}}%
\pgfpathclose%
\pgfusepath{stroke,fill}%
\end{pgfscope}%
\begin{pgfscope}%
\pgfpathrectangle{\pgfqpoint{0.647939in}{0.492442in}}{\pgfqpoint{3.079299in}{3.079299in}}%
\pgfusepath{clip}%
\pgfsetroundcap%
\pgfsetroundjoin%
\pgfsetlinewidth{0.301125pt}%
\definecolor{currentstroke}{rgb}{0.500000,0.500000,0.500000}%
\pgfsetstrokecolor{currentstroke}%
\pgfsetstrokeopacity{0.300000}%
\pgfsetdash{}{0pt}%
\pgfpathmoveto{\pgfqpoint{1.070757in}{0.573424in}}%
\pgfusepath{stroke}%
\end{pgfscope}%
\begin{pgfscope}%
\pgfpathrectangle{\pgfqpoint{0.647939in}{0.492442in}}{\pgfqpoint{3.079299in}{3.079299in}}%
\pgfusepath{clip}%
\pgfsetroundcap%
\pgfsetroundjoin%
\definecolor{currentfill}{rgb}{0.500000,0.500000,0.500000}%
\pgfsetfillcolor{currentfill}%
\pgfsetfillopacity{0.300000}%
\pgfsetlinewidth{0.301125pt}%
\definecolor{currentstroke}{rgb}{0.500000,0.500000,0.500000}%
\pgfsetstrokecolor{currentstroke}%
\pgfsetstrokeopacity{0.300000}%
\pgfsetdash{}{0pt}%
\pgfpathmoveto{\pgfqpoint{0.000000in}{0.000000in}}%
\pgfpathlineto{\pgfqpoint{0.000000in}{0.000000in}}%
\pgfpathclose%
\pgfusepath{stroke,fill}%
\end{pgfscope}%
\begin{pgfscope}%
\pgfpathrectangle{\pgfqpoint{0.647939in}{0.492442in}}{\pgfqpoint{3.079299in}{3.079299in}}%
\pgfusepath{clip}%
\pgfsetroundcap%
\pgfsetroundjoin%
\pgfsetlinewidth{0.301125pt}%
\definecolor{currentstroke}{rgb}{0.500000,0.500000,0.500000}%
\pgfsetstrokecolor{currentstroke}%
\pgfsetstrokeopacity{0.300000}%
\pgfsetdash{}{0pt}%
\pgfpathmoveto{\pgfqpoint{1.272149in}{0.681676in}}%
\pgfusepath{stroke}%
\end{pgfscope}%
\begin{pgfscope}%
\pgfpathrectangle{\pgfqpoint{0.647939in}{0.492442in}}{\pgfqpoint{3.079299in}{3.079299in}}%
\pgfusepath{clip}%
\pgfsetroundcap%
\pgfsetroundjoin%
\definecolor{currentfill}{rgb}{0.500000,0.500000,0.500000}%
\pgfsetfillcolor{currentfill}%
\pgfsetfillopacity{0.300000}%
\pgfsetlinewidth{0.301125pt}%
\definecolor{currentstroke}{rgb}{0.500000,0.500000,0.500000}%
\pgfsetstrokecolor{currentstroke}%
\pgfsetstrokeopacity{0.300000}%
\pgfsetdash{}{0pt}%
\pgfpathmoveto{\pgfqpoint{0.000000in}{0.000000in}}%
\pgfpathlineto{\pgfqpoint{0.000000in}{0.000000in}}%
\pgfpathclose%
\pgfusepath{stroke,fill}%
\end{pgfscope}%
\begin{pgfscope}%
\pgfpathrectangle{\pgfqpoint{0.647939in}{0.492442in}}{\pgfqpoint{3.079299in}{3.079299in}}%
\pgfusepath{clip}%
\pgfsetroundcap%
\pgfsetroundjoin%
\pgfsetlinewidth{0.301125pt}%
\definecolor{currentstroke}{rgb}{0.500000,0.500000,0.500000}%
\pgfsetstrokecolor{currentstroke}%
\pgfsetstrokeopacity{0.300000}%
\pgfsetdash{}{0pt}%
\pgfpathmoveto{\pgfqpoint{1.382671in}{0.750179in}}%
\pgfusepath{stroke}%
\end{pgfscope}%
\begin{pgfscope}%
\pgfpathrectangle{\pgfqpoint{0.647939in}{0.492442in}}{\pgfqpoint{3.079299in}{3.079299in}}%
\pgfusepath{clip}%
\pgfsetroundcap%
\pgfsetroundjoin%
\definecolor{currentfill}{rgb}{0.500000,0.500000,0.500000}%
\pgfsetfillcolor{currentfill}%
\pgfsetfillopacity{0.300000}%
\pgfsetlinewidth{0.301125pt}%
\definecolor{currentstroke}{rgb}{0.500000,0.500000,0.500000}%
\pgfsetstrokecolor{currentstroke}%
\pgfsetstrokeopacity{0.300000}%
\pgfsetdash{}{0pt}%
\pgfpathmoveto{\pgfqpoint{0.000000in}{0.000000in}}%
\pgfpathlineto{\pgfqpoint{0.000000in}{0.000000in}}%
\pgfpathclose%
\pgfusepath{stroke,fill}%
\end{pgfscope}%
\begin{pgfscope}%
\pgfpathrectangle{\pgfqpoint{0.647939in}{0.492442in}}{\pgfqpoint{3.079299in}{3.079299in}}%
\pgfusepath{clip}%
\pgfsetroundcap%
\pgfsetroundjoin%
\pgfsetlinewidth{0.301125pt}%
\definecolor{currentstroke}{rgb}{0.500000,0.500000,0.500000}%
\pgfsetstrokecolor{currentstroke}%
\pgfsetstrokeopacity{0.300000}%
\pgfsetdash{}{0pt}%
\pgfpathmoveto{\pgfqpoint{1.489642in}{0.577139in}}%
\pgfusepath{stroke}%
\end{pgfscope}%
\begin{pgfscope}%
\pgfpathrectangle{\pgfqpoint{0.647939in}{0.492442in}}{\pgfqpoint{3.079299in}{3.079299in}}%
\pgfusepath{clip}%
\pgfsetroundcap%
\pgfsetroundjoin%
\definecolor{currentfill}{rgb}{0.500000,0.500000,0.500000}%
\pgfsetfillcolor{currentfill}%
\pgfsetfillopacity{0.300000}%
\pgfsetlinewidth{0.301125pt}%
\definecolor{currentstroke}{rgb}{0.500000,0.500000,0.500000}%
\pgfsetstrokecolor{currentstroke}%
\pgfsetstrokeopacity{0.300000}%
\pgfsetdash{}{0pt}%
\pgfpathmoveto{\pgfqpoint{0.000000in}{0.000000in}}%
\pgfpathlineto{\pgfqpoint{0.000000in}{0.000000in}}%
\pgfpathclose%
\pgfusepath{stroke,fill}%
\end{pgfscope}%
\begin{pgfscope}%
\pgfpathrectangle{\pgfqpoint{0.647939in}{0.492442in}}{\pgfqpoint{3.079299in}{3.079299in}}%
\pgfusepath{clip}%
\pgfsetroundcap%
\pgfsetroundjoin%
\pgfsetlinewidth{0.301125pt}%
\definecolor{currentstroke}{rgb}{0.500000,0.500000,0.500000}%
\pgfsetstrokecolor{currentstroke}%
\pgfsetstrokeopacity{0.300000}%
\pgfsetdash{}{0pt}%
\pgfpathmoveto{\pgfqpoint{1.743726in}{0.505637in}}%
\pgfusepath{stroke}%
\end{pgfscope}%
\begin{pgfscope}%
\pgfpathrectangle{\pgfqpoint{0.647939in}{0.492442in}}{\pgfqpoint{3.079299in}{3.079299in}}%
\pgfusepath{clip}%
\pgfsetroundcap%
\pgfsetroundjoin%
\definecolor{currentfill}{rgb}{0.500000,0.500000,0.500000}%
\pgfsetfillcolor{currentfill}%
\pgfsetfillopacity{0.300000}%
\pgfsetlinewidth{0.301125pt}%
\definecolor{currentstroke}{rgb}{0.500000,0.500000,0.500000}%
\pgfsetstrokecolor{currentstroke}%
\pgfsetstrokeopacity{0.300000}%
\pgfsetdash{}{0pt}%
\pgfpathmoveto{\pgfqpoint{0.000000in}{0.000000in}}%
\pgfpathlineto{\pgfqpoint{0.000000in}{0.000000in}}%
\pgfpathclose%
\pgfusepath{stroke,fill}%
\end{pgfscope}%
\begin{pgfscope}%
\pgfpathrectangle{\pgfqpoint{0.647939in}{0.492442in}}{\pgfqpoint{3.079299in}{3.079299in}}%
\pgfusepath{clip}%
\pgfsetroundcap%
\pgfsetroundjoin%
\pgfsetlinewidth{0.301125pt}%
\definecolor{currentstroke}{rgb}{0.500000,0.500000,0.500000}%
\pgfsetstrokecolor{currentstroke}%
\pgfsetstrokeopacity{0.300000}%
\pgfsetdash{}{0pt}%
\pgfpathmoveto{\pgfqpoint{2.360156in}{0.506686in}}%
\pgfusepath{stroke}%
\end{pgfscope}%
\begin{pgfscope}%
\pgfpathrectangle{\pgfqpoint{0.647939in}{0.492442in}}{\pgfqpoint{3.079299in}{3.079299in}}%
\pgfusepath{clip}%
\pgfsetroundcap%
\pgfsetroundjoin%
\definecolor{currentfill}{rgb}{0.500000,0.500000,0.500000}%
\pgfsetfillcolor{currentfill}%
\pgfsetfillopacity{0.300000}%
\pgfsetlinewidth{0.301125pt}%
\definecolor{currentstroke}{rgb}{0.500000,0.500000,0.500000}%
\pgfsetstrokecolor{currentstroke}%
\pgfsetstrokeopacity{0.300000}%
\pgfsetdash{}{0pt}%
\pgfpathmoveto{\pgfqpoint{0.000000in}{0.000000in}}%
\pgfpathlineto{\pgfqpoint{0.000000in}{0.000000in}}%
\pgfpathclose%
\pgfusepath{stroke,fill}%
\end{pgfscope}%
\begin{pgfscope}%
\pgfpathrectangle{\pgfqpoint{0.647939in}{0.492442in}}{\pgfqpoint{3.079299in}{3.079299in}}%
\pgfusepath{clip}%
\pgfsetroundcap%
\pgfsetroundjoin%
\pgfsetlinewidth{0.301125pt}%
\definecolor{currentstroke}{rgb}{0.500000,0.500000,0.500000}%
\pgfsetstrokecolor{currentstroke}%
\pgfsetstrokeopacity{0.300000}%
\pgfsetdash{}{0pt}%
\pgfpathmoveto{\pgfqpoint{2.318143in}{0.547844in}}%
\pgfusepath{stroke}%
\end{pgfscope}%
\begin{pgfscope}%
\pgfpathrectangle{\pgfqpoint{0.647939in}{0.492442in}}{\pgfqpoint{3.079299in}{3.079299in}}%
\pgfusepath{clip}%
\pgfsetroundcap%
\pgfsetroundjoin%
\definecolor{currentfill}{rgb}{0.500000,0.500000,0.500000}%
\pgfsetfillcolor{currentfill}%
\pgfsetfillopacity{0.300000}%
\pgfsetlinewidth{0.301125pt}%
\definecolor{currentstroke}{rgb}{0.500000,0.500000,0.500000}%
\pgfsetstrokecolor{currentstroke}%
\pgfsetstrokeopacity{0.300000}%
\pgfsetdash{}{0pt}%
\pgfpathmoveto{\pgfqpoint{0.000000in}{0.000000in}}%
\pgfpathlineto{\pgfqpoint{0.000000in}{0.000000in}}%
\pgfpathclose%
\pgfusepath{stroke,fill}%
\end{pgfscope}%
\begin{pgfscope}%
\pgfpathrectangle{\pgfqpoint{0.647939in}{0.492442in}}{\pgfqpoint{3.079299in}{3.079299in}}%
\pgfusepath{clip}%
\pgfsetroundcap%
\pgfsetroundjoin%
\pgfsetlinewidth{0.301125pt}%
\definecolor{currentstroke}{rgb}{0.500000,0.500000,0.500000}%
\pgfsetstrokecolor{currentstroke}%
\pgfsetstrokeopacity{0.300000}%
\pgfsetdash{}{0pt}%
\pgfpathmoveto{\pgfqpoint{3.008381in}{0.535085in}}%
\pgfusepath{stroke}%
\end{pgfscope}%
\begin{pgfscope}%
\pgfpathrectangle{\pgfqpoint{0.647939in}{0.492442in}}{\pgfqpoint{3.079299in}{3.079299in}}%
\pgfusepath{clip}%
\pgfsetroundcap%
\pgfsetroundjoin%
\definecolor{currentfill}{rgb}{0.500000,0.500000,0.500000}%
\pgfsetfillcolor{currentfill}%
\pgfsetfillopacity{0.300000}%
\pgfsetlinewidth{0.301125pt}%
\definecolor{currentstroke}{rgb}{0.500000,0.500000,0.500000}%
\pgfsetstrokecolor{currentstroke}%
\pgfsetstrokeopacity{0.300000}%
\pgfsetdash{}{0pt}%
\pgfpathmoveto{\pgfqpoint{0.000000in}{0.000000in}}%
\pgfpathlineto{\pgfqpoint{0.000000in}{0.000000in}}%
\pgfpathclose%
\pgfusepath{stroke,fill}%
\end{pgfscope}%
\begin{pgfscope}%
\pgfpathrectangle{\pgfqpoint{0.647939in}{0.492442in}}{\pgfqpoint{3.079299in}{3.079299in}}%
\pgfusepath{clip}%
\pgfsetroundcap%
\pgfsetroundjoin%
\pgfsetlinewidth{0.301125pt}%
\definecolor{currentstroke}{rgb}{0.500000,0.500000,0.500000}%
\pgfsetstrokecolor{currentstroke}%
\pgfsetstrokeopacity{0.300000}%
\pgfsetdash{}{0pt}%
\pgfpathmoveto{\pgfqpoint{2.419106in}{0.698767in}}%
\pgfusepath{stroke}%
\end{pgfscope}%
\begin{pgfscope}%
\pgfpathrectangle{\pgfqpoint{0.647939in}{0.492442in}}{\pgfqpoint{3.079299in}{3.079299in}}%
\pgfusepath{clip}%
\pgfsetroundcap%
\pgfsetroundjoin%
\definecolor{currentfill}{rgb}{0.500000,0.500000,0.500000}%
\pgfsetfillcolor{currentfill}%
\pgfsetfillopacity{0.300000}%
\pgfsetlinewidth{0.301125pt}%
\definecolor{currentstroke}{rgb}{0.500000,0.500000,0.500000}%
\pgfsetstrokecolor{currentstroke}%
\pgfsetstrokeopacity{0.300000}%
\pgfsetdash{}{0pt}%
\pgfpathmoveto{\pgfqpoint{0.000000in}{0.000000in}}%
\pgfpathlineto{\pgfqpoint{0.000000in}{0.000000in}}%
\pgfpathclose%
\pgfusepath{stroke,fill}%
\end{pgfscope}%
\begin{pgfscope}%
\pgfpathrectangle{\pgfqpoint{0.647939in}{0.492442in}}{\pgfqpoint{3.079299in}{3.079299in}}%
\pgfusepath{clip}%
\pgfsetroundcap%
\pgfsetroundjoin%
\pgfsetlinewidth{0.301125pt}%
\definecolor{currentstroke}{rgb}{0.500000,0.500000,0.500000}%
\pgfsetstrokecolor{currentstroke}%
\pgfsetstrokeopacity{0.300000}%
\pgfsetdash{}{0pt}%
\pgfpathmoveto{\pgfqpoint{2.643885in}{0.778506in}}%
\pgfusepath{stroke}%
\end{pgfscope}%
\begin{pgfscope}%
\pgfpathrectangle{\pgfqpoint{0.647939in}{0.492442in}}{\pgfqpoint{3.079299in}{3.079299in}}%
\pgfusepath{clip}%
\pgfsetroundcap%
\pgfsetroundjoin%
\definecolor{currentfill}{rgb}{0.500000,0.500000,0.500000}%
\pgfsetfillcolor{currentfill}%
\pgfsetfillopacity{0.300000}%
\pgfsetlinewidth{0.301125pt}%
\definecolor{currentstroke}{rgb}{0.500000,0.500000,0.500000}%
\pgfsetstrokecolor{currentstroke}%
\pgfsetstrokeopacity{0.300000}%
\pgfsetdash{}{0pt}%
\pgfpathmoveto{\pgfqpoint{0.000000in}{0.000000in}}%
\pgfpathlineto{\pgfqpoint{0.000000in}{0.000000in}}%
\pgfpathclose%
\pgfusepath{stroke,fill}%
\end{pgfscope}%
\begin{pgfscope}%
\pgfpathrectangle{\pgfqpoint{0.647939in}{0.492442in}}{\pgfqpoint{3.079299in}{3.079299in}}%
\pgfusepath{clip}%
\pgfsetroundcap%
\pgfsetroundjoin%
\pgfsetlinewidth{0.301125pt}%
\definecolor{currentstroke}{rgb}{0.500000,0.500000,0.500000}%
\pgfsetstrokecolor{currentstroke}%
\pgfsetstrokeopacity{0.300000}%
\pgfsetdash{}{0pt}%
\pgfpathmoveto{\pgfqpoint{2.578700in}{0.865151in}}%
\pgfusepath{stroke}%
\end{pgfscope}%
\begin{pgfscope}%
\pgfpathrectangle{\pgfqpoint{0.647939in}{0.492442in}}{\pgfqpoint{3.079299in}{3.079299in}}%
\pgfusepath{clip}%
\pgfsetroundcap%
\pgfsetroundjoin%
\definecolor{currentfill}{rgb}{0.500000,0.500000,0.500000}%
\pgfsetfillcolor{currentfill}%
\pgfsetfillopacity{0.300000}%
\pgfsetlinewidth{0.301125pt}%
\definecolor{currentstroke}{rgb}{0.500000,0.500000,0.500000}%
\pgfsetstrokecolor{currentstroke}%
\pgfsetstrokeopacity{0.300000}%
\pgfsetdash{}{0pt}%
\pgfpathmoveto{\pgfqpoint{0.000000in}{0.000000in}}%
\pgfpathlineto{\pgfqpoint{0.000000in}{0.000000in}}%
\pgfpathclose%
\pgfusepath{stroke,fill}%
\end{pgfscope}%
\begin{pgfscope}%
\pgfpathrectangle{\pgfqpoint{0.647939in}{0.492442in}}{\pgfqpoint{3.079299in}{3.079299in}}%
\pgfusepath{clip}%
\pgfsetroundcap%
\pgfsetroundjoin%
\pgfsetlinewidth{0.301125pt}%
\definecolor{currentstroke}{rgb}{0.500000,0.500000,0.500000}%
\pgfsetstrokecolor{currentstroke}%
\pgfsetstrokeopacity{0.300000}%
\pgfsetdash{}{0pt}%
\pgfpathmoveto{\pgfqpoint{2.650193in}{0.940954in}}%
\pgfusepath{stroke}%
\end{pgfscope}%
\begin{pgfscope}%
\pgfpathrectangle{\pgfqpoint{0.647939in}{0.492442in}}{\pgfqpoint{3.079299in}{3.079299in}}%
\pgfusepath{clip}%
\pgfsetroundcap%
\pgfsetroundjoin%
\definecolor{currentfill}{rgb}{0.500000,0.500000,0.500000}%
\pgfsetfillcolor{currentfill}%
\pgfsetfillopacity{0.300000}%
\pgfsetlinewidth{0.301125pt}%
\definecolor{currentstroke}{rgb}{0.500000,0.500000,0.500000}%
\pgfsetstrokecolor{currentstroke}%
\pgfsetstrokeopacity{0.300000}%
\pgfsetdash{}{0pt}%
\pgfpathmoveto{\pgfqpoint{0.000000in}{0.000000in}}%
\pgfpathlineto{\pgfqpoint{0.000000in}{0.000000in}}%
\pgfpathclose%
\pgfusepath{stroke,fill}%
\end{pgfscope}%
\begin{pgfscope}%
\pgfpathrectangle{\pgfqpoint{0.647939in}{0.492442in}}{\pgfqpoint{3.079299in}{3.079299in}}%
\pgfusepath{clip}%
\pgfsetroundcap%
\pgfsetroundjoin%
\pgfsetlinewidth{0.301125pt}%
\definecolor{currentstroke}{rgb}{0.500000,0.500000,0.500000}%
\pgfsetstrokecolor{currentstroke}%
\pgfsetstrokeopacity{0.300000}%
\pgfsetdash{}{0pt}%
\pgfpathmoveto{\pgfqpoint{2.585910in}{1.030495in}}%
\pgfusepath{stroke}%
\end{pgfscope}%
\begin{pgfscope}%
\pgfpathrectangle{\pgfqpoint{0.647939in}{0.492442in}}{\pgfqpoint{3.079299in}{3.079299in}}%
\pgfusepath{clip}%
\pgfsetroundcap%
\pgfsetroundjoin%
\definecolor{currentfill}{rgb}{0.500000,0.500000,0.500000}%
\pgfsetfillcolor{currentfill}%
\pgfsetfillopacity{0.300000}%
\pgfsetlinewidth{0.301125pt}%
\definecolor{currentstroke}{rgb}{0.500000,0.500000,0.500000}%
\pgfsetstrokecolor{currentstroke}%
\pgfsetstrokeopacity{0.300000}%
\pgfsetdash{}{0pt}%
\pgfpathmoveto{\pgfqpoint{0.000000in}{0.000000in}}%
\pgfpathlineto{\pgfqpoint{0.000000in}{0.000000in}}%
\pgfpathclose%
\pgfusepath{stroke,fill}%
\end{pgfscope}%
\begin{pgfscope}%
\pgfpathrectangle{\pgfqpoint{0.647939in}{0.492442in}}{\pgfqpoint{3.079299in}{3.079299in}}%
\pgfusepath{clip}%
\pgfsetroundcap%
\pgfsetroundjoin%
\pgfsetlinewidth{0.301125pt}%
\definecolor{currentstroke}{rgb}{0.500000,0.500000,0.500000}%
\pgfsetstrokecolor{currentstroke}%
\pgfsetstrokeopacity{0.300000}%
\pgfsetdash{}{0pt}%
\pgfpathmoveto{\pgfqpoint{2.725889in}{1.097282in}}%
\pgfusepath{stroke}%
\end{pgfscope}%
\begin{pgfscope}%
\pgfpathrectangle{\pgfqpoint{0.647939in}{0.492442in}}{\pgfqpoint{3.079299in}{3.079299in}}%
\pgfusepath{clip}%
\pgfsetroundcap%
\pgfsetroundjoin%
\definecolor{currentfill}{rgb}{0.500000,0.500000,0.500000}%
\pgfsetfillcolor{currentfill}%
\pgfsetfillopacity{0.300000}%
\pgfsetlinewidth{0.301125pt}%
\definecolor{currentstroke}{rgb}{0.500000,0.500000,0.500000}%
\pgfsetstrokecolor{currentstroke}%
\pgfsetstrokeopacity{0.300000}%
\pgfsetdash{}{0pt}%
\pgfpathmoveto{\pgfqpoint{0.000000in}{0.000000in}}%
\pgfpathlineto{\pgfqpoint{0.000000in}{0.000000in}}%
\pgfpathclose%
\pgfusepath{stroke,fill}%
\end{pgfscope}%
\begin{pgfscope}%
\pgfpathrectangle{\pgfqpoint{0.647939in}{0.492442in}}{\pgfqpoint{3.079299in}{3.079299in}}%
\pgfusepath{clip}%
\pgfsetroundcap%
\pgfsetroundjoin%
\pgfsetlinewidth{0.301125pt}%
\definecolor{currentstroke}{rgb}{0.500000,0.500000,0.500000}%
\pgfsetstrokecolor{currentstroke}%
\pgfsetstrokeopacity{0.300000}%
\pgfsetdash{}{0pt}%
\pgfpathmoveto{\pgfqpoint{2.663195in}{1.192800in}}%
\pgfusepath{stroke}%
\end{pgfscope}%
\begin{pgfscope}%
\pgfpathrectangle{\pgfqpoint{0.647939in}{0.492442in}}{\pgfqpoint{3.079299in}{3.079299in}}%
\pgfusepath{clip}%
\pgfsetroundcap%
\pgfsetroundjoin%
\definecolor{currentfill}{rgb}{0.500000,0.500000,0.500000}%
\pgfsetfillcolor{currentfill}%
\pgfsetfillopacity{0.300000}%
\pgfsetlinewidth{0.301125pt}%
\definecolor{currentstroke}{rgb}{0.500000,0.500000,0.500000}%
\pgfsetstrokecolor{currentstroke}%
\pgfsetstrokeopacity{0.300000}%
\pgfsetdash{}{0pt}%
\pgfpathmoveto{\pgfqpoint{0.000000in}{0.000000in}}%
\pgfpathlineto{\pgfqpoint{0.000000in}{0.000000in}}%
\pgfpathclose%
\pgfusepath{stroke,fill}%
\end{pgfscope}%
\begin{pgfscope}%
\pgfpathrectangle{\pgfqpoint{0.647939in}{0.492442in}}{\pgfqpoint{3.079299in}{3.079299in}}%
\pgfusepath{clip}%
\pgfsetroundcap%
\pgfsetroundjoin%
\pgfsetlinewidth{0.301125pt}%
\definecolor{currentstroke}{rgb}{0.500000,0.500000,0.500000}%
\pgfsetstrokecolor{currentstroke}%
\pgfsetstrokeopacity{0.300000}%
\pgfsetdash{}{0pt}%
\pgfpathmoveto{\pgfqpoint{2.802910in}{1.252388in}}%
\pgfusepath{stroke}%
\end{pgfscope}%
\begin{pgfscope}%
\pgfpathrectangle{\pgfqpoint{0.647939in}{0.492442in}}{\pgfqpoint{3.079299in}{3.079299in}}%
\pgfusepath{clip}%
\pgfsetroundcap%
\pgfsetroundjoin%
\definecolor{currentfill}{rgb}{0.500000,0.500000,0.500000}%
\pgfsetfillcolor{currentfill}%
\pgfsetfillopacity{0.300000}%
\pgfsetlinewidth{0.301125pt}%
\definecolor{currentstroke}{rgb}{0.500000,0.500000,0.500000}%
\pgfsetstrokecolor{currentstroke}%
\pgfsetstrokeopacity{0.300000}%
\pgfsetdash{}{0pt}%
\pgfpathmoveto{\pgfqpoint{0.000000in}{0.000000in}}%
\pgfpathlineto{\pgfqpoint{0.000000in}{0.000000in}}%
\pgfpathclose%
\pgfusepath{stroke,fill}%
\end{pgfscope}%
\begin{pgfscope}%
\pgfpathrectangle{\pgfqpoint{0.647939in}{0.492442in}}{\pgfqpoint{3.079299in}{3.079299in}}%
\pgfusepath{clip}%
\pgfsetroundcap%
\pgfsetroundjoin%
\pgfsetlinewidth{0.301125pt}%
\definecolor{currentstroke}{rgb}{0.500000,0.500000,0.500000}%
\pgfsetstrokecolor{currentstroke}%
\pgfsetstrokeopacity{0.300000}%
\pgfsetdash{}{0pt}%
\pgfpathmoveto{\pgfqpoint{2.675565in}{1.368399in}}%
\pgfusepath{stroke}%
\end{pgfscope}%
\begin{pgfscope}%
\pgfpathrectangle{\pgfqpoint{0.647939in}{0.492442in}}{\pgfqpoint{3.079299in}{3.079299in}}%
\pgfusepath{clip}%
\pgfsetroundcap%
\pgfsetroundjoin%
\definecolor{currentfill}{rgb}{0.500000,0.500000,0.500000}%
\pgfsetfillcolor{currentfill}%
\pgfsetfillopacity{0.300000}%
\pgfsetlinewidth{0.301125pt}%
\definecolor{currentstroke}{rgb}{0.500000,0.500000,0.500000}%
\pgfsetstrokecolor{currentstroke}%
\pgfsetstrokeopacity{0.300000}%
\pgfsetdash{}{0pt}%
\pgfpathmoveto{\pgfqpoint{0.000000in}{0.000000in}}%
\pgfpathlineto{\pgfqpoint{0.000000in}{0.000000in}}%
\pgfpathclose%
\pgfusepath{stroke,fill}%
\end{pgfscope}%
\begin{pgfscope}%
\pgfpathrectangle{\pgfqpoint{0.647939in}{0.492442in}}{\pgfqpoint{3.079299in}{3.079299in}}%
\pgfusepath{clip}%
\pgfsetroundcap%
\pgfsetroundjoin%
\pgfsetlinewidth{0.301125pt}%
\definecolor{currentstroke}{rgb}{0.500000,0.500000,0.500000}%
\pgfsetstrokecolor{currentstroke}%
\pgfsetstrokeopacity{0.300000}%
\pgfsetdash{}{0pt}%
\pgfpathmoveto{\pgfqpoint{2.816143in}{1.426410in}}%
\pgfusepath{stroke}%
\end{pgfscope}%
\begin{pgfscope}%
\pgfpathrectangle{\pgfqpoint{0.647939in}{0.492442in}}{\pgfqpoint{3.079299in}{3.079299in}}%
\pgfusepath{clip}%
\pgfsetroundcap%
\pgfsetroundjoin%
\definecolor{currentfill}{rgb}{0.500000,0.500000,0.500000}%
\pgfsetfillcolor{currentfill}%
\pgfsetfillopacity{0.300000}%
\pgfsetlinewidth{0.301125pt}%
\definecolor{currentstroke}{rgb}{0.500000,0.500000,0.500000}%
\pgfsetstrokecolor{currentstroke}%
\pgfsetstrokeopacity{0.300000}%
\pgfsetdash{}{0pt}%
\pgfpathmoveto{\pgfqpoint{0.000000in}{0.000000in}}%
\pgfpathlineto{\pgfqpoint{0.000000in}{0.000000in}}%
\pgfpathclose%
\pgfusepath{stroke,fill}%
\end{pgfscope}%
\begin{pgfscope}%
\pgfpathrectangle{\pgfqpoint{0.647939in}{0.492442in}}{\pgfqpoint{3.079299in}{3.079299in}}%
\pgfusepath{clip}%
\pgfsetroundcap%
\pgfsetroundjoin%
\pgfsetlinewidth{0.301125pt}%
\definecolor{currentstroke}{rgb}{0.500000,0.500000,0.500000}%
\pgfsetstrokecolor{currentstroke}%
\pgfsetstrokeopacity{0.300000}%
\pgfsetdash{}{0pt}%
\pgfpathmoveto{\pgfqpoint{2.759164in}{1.536308in}}%
\pgfusepath{stroke}%
\end{pgfscope}%
\begin{pgfscope}%
\pgfpathrectangle{\pgfqpoint{0.647939in}{0.492442in}}{\pgfqpoint{3.079299in}{3.079299in}}%
\pgfusepath{clip}%
\pgfsetroundcap%
\pgfsetroundjoin%
\definecolor{currentfill}{rgb}{0.500000,0.500000,0.500000}%
\pgfsetfillcolor{currentfill}%
\pgfsetfillopacity{0.300000}%
\pgfsetlinewidth{0.301125pt}%
\definecolor{currentstroke}{rgb}{0.500000,0.500000,0.500000}%
\pgfsetstrokecolor{currentstroke}%
\pgfsetstrokeopacity{0.300000}%
\pgfsetdash{}{0pt}%
\pgfpathmoveto{\pgfqpoint{0.000000in}{0.000000in}}%
\pgfpathlineto{\pgfqpoint{0.000000in}{0.000000in}}%
\pgfpathclose%
\pgfusepath{stroke,fill}%
\end{pgfscope}%
\begin{pgfscope}%
\pgfpathrectangle{\pgfqpoint{0.647939in}{0.492442in}}{\pgfqpoint{3.079299in}{3.079299in}}%
\pgfusepath{clip}%
\pgfsetroundcap%
\pgfsetroundjoin%
\pgfsetlinewidth{0.301125pt}%
\definecolor{currentstroke}{rgb}{0.500000,0.500000,0.500000}%
\pgfsetstrokecolor{currentstroke}%
\pgfsetstrokeopacity{0.300000}%
\pgfsetdash{}{0pt}%
\pgfpathmoveto{\pgfqpoint{2.834593in}{1.608464in}}%
\pgfusepath{stroke}%
\end{pgfscope}%
\begin{pgfscope}%
\pgfpathrectangle{\pgfqpoint{0.647939in}{0.492442in}}{\pgfqpoint{3.079299in}{3.079299in}}%
\pgfusepath{clip}%
\pgfsetroundcap%
\pgfsetroundjoin%
\definecolor{currentfill}{rgb}{0.500000,0.500000,0.500000}%
\pgfsetfillcolor{currentfill}%
\pgfsetfillopacity{0.300000}%
\pgfsetlinewidth{0.301125pt}%
\definecolor{currentstroke}{rgb}{0.500000,0.500000,0.500000}%
\pgfsetstrokecolor{currentstroke}%
\pgfsetstrokeopacity{0.300000}%
\pgfsetdash{}{0pt}%
\pgfpathmoveto{\pgfqpoint{0.000000in}{0.000000in}}%
\pgfpathlineto{\pgfqpoint{0.000000in}{0.000000in}}%
\pgfpathclose%
\pgfusepath{stroke,fill}%
\end{pgfscope}%
\begin{pgfscope}%
\pgfpathrectangle{\pgfqpoint{0.647939in}{0.492442in}}{\pgfqpoint{3.079299in}{3.079299in}}%
\pgfusepath{clip}%
\pgfsetroundcap%
\pgfsetroundjoin%
\pgfsetlinewidth{0.301125pt}%
\definecolor{currentstroke}{rgb}{0.500000,0.500000,0.500000}%
\pgfsetstrokecolor{currentstroke}%
\pgfsetstrokeopacity{0.300000}%
\pgfsetdash{}{0pt}%
\pgfpathmoveto{\pgfqpoint{2.909169in}{1.675426in}}%
\pgfusepath{stroke}%
\end{pgfscope}%
\begin{pgfscope}%
\pgfpathrectangle{\pgfqpoint{0.647939in}{0.492442in}}{\pgfqpoint{3.079299in}{3.079299in}}%
\pgfusepath{clip}%
\pgfsetroundcap%
\pgfsetroundjoin%
\definecolor{currentfill}{rgb}{0.500000,0.500000,0.500000}%
\pgfsetfillcolor{currentfill}%
\pgfsetfillopacity{0.300000}%
\pgfsetlinewidth{0.301125pt}%
\definecolor{currentstroke}{rgb}{0.500000,0.500000,0.500000}%
\pgfsetstrokecolor{currentstroke}%
\pgfsetstrokeopacity{0.300000}%
\pgfsetdash{}{0pt}%
\pgfpathmoveto{\pgfqpoint{0.000000in}{0.000000in}}%
\pgfpathlineto{\pgfqpoint{0.000000in}{0.000000in}}%
\pgfpathclose%
\pgfusepath{stroke,fill}%
\end{pgfscope}%
\begin{pgfscope}%
\pgfpathrectangle{\pgfqpoint{0.647939in}{0.492442in}}{\pgfqpoint{3.079299in}{3.079299in}}%
\pgfusepath{clip}%
\pgfsetroundcap%
\pgfsetroundjoin%
\pgfsetlinewidth{0.301125pt}%
\definecolor{currentstroke}{rgb}{0.500000,0.500000,0.500000}%
\pgfsetstrokecolor{currentstroke}%
\pgfsetstrokeopacity{0.300000}%
\pgfsetdash{}{0pt}%
\pgfpathmoveto{\pgfqpoint{2.922399in}{1.770110in}}%
\pgfusepath{stroke}%
\end{pgfscope}%
\begin{pgfscope}%
\pgfpathrectangle{\pgfqpoint{0.647939in}{0.492442in}}{\pgfqpoint{3.079299in}{3.079299in}}%
\pgfusepath{clip}%
\pgfsetroundcap%
\pgfsetroundjoin%
\definecolor{currentfill}{rgb}{0.500000,0.500000,0.500000}%
\pgfsetfillcolor{currentfill}%
\pgfsetfillopacity{0.300000}%
\pgfsetlinewidth{0.301125pt}%
\definecolor{currentstroke}{rgb}{0.500000,0.500000,0.500000}%
\pgfsetstrokecolor{currentstroke}%
\pgfsetstrokeopacity{0.300000}%
\pgfsetdash{}{0pt}%
\pgfpathmoveto{\pgfqpoint{0.000000in}{0.000000in}}%
\pgfpathlineto{\pgfqpoint{0.000000in}{0.000000in}}%
\pgfpathclose%
\pgfusepath{stroke,fill}%
\end{pgfscope}%
\begin{pgfscope}%
\pgfpathrectangle{\pgfqpoint{0.647939in}{0.492442in}}{\pgfqpoint{3.079299in}{3.079299in}}%
\pgfusepath{clip}%
\pgfsetroundcap%
\pgfsetroundjoin%
\pgfsetlinewidth{0.301125pt}%
\definecolor{currentstroke}{rgb}{0.500000,0.500000,0.500000}%
\pgfsetstrokecolor{currentstroke}%
\pgfsetstrokeopacity{0.300000}%
\pgfsetdash{}{0pt}%
\pgfpathmoveto{\pgfqpoint{2.938611in}{1.867765in}}%
\pgfusepath{stroke}%
\end{pgfscope}%
\begin{pgfscope}%
\pgfpathrectangle{\pgfqpoint{0.647939in}{0.492442in}}{\pgfqpoint{3.079299in}{3.079299in}}%
\pgfusepath{clip}%
\pgfsetroundcap%
\pgfsetroundjoin%
\definecolor{currentfill}{rgb}{0.500000,0.500000,0.500000}%
\pgfsetfillcolor{currentfill}%
\pgfsetfillopacity{0.300000}%
\pgfsetlinewidth{0.301125pt}%
\definecolor{currentstroke}{rgb}{0.500000,0.500000,0.500000}%
\pgfsetstrokecolor{currentstroke}%
\pgfsetstrokeopacity{0.300000}%
\pgfsetdash{}{0pt}%
\pgfpathmoveto{\pgfqpoint{0.000000in}{0.000000in}}%
\pgfpathlineto{\pgfqpoint{0.000000in}{0.000000in}}%
\pgfpathclose%
\pgfusepath{stroke,fill}%
\end{pgfscope}%
\begin{pgfscope}%
\pgfpathrectangle{\pgfqpoint{0.647939in}{0.492442in}}{\pgfqpoint{3.079299in}{3.079299in}}%
\pgfusepath{clip}%
\pgfsetroundcap%
\pgfsetroundjoin%
\pgfsetlinewidth{0.301125pt}%
\definecolor{currentstroke}{rgb}{0.500000,0.500000,0.500000}%
\pgfsetstrokecolor{currentstroke}%
\pgfsetstrokeopacity{0.300000}%
\pgfsetdash{}{0pt}%
\pgfpathmoveto{\pgfqpoint{2.958704in}{1.968727in}}%
\pgfusepath{stroke}%
\end{pgfscope}%
\begin{pgfscope}%
\pgfpathrectangle{\pgfqpoint{0.647939in}{0.492442in}}{\pgfqpoint{3.079299in}{3.079299in}}%
\pgfusepath{clip}%
\pgfsetroundcap%
\pgfsetroundjoin%
\definecolor{currentfill}{rgb}{0.500000,0.500000,0.500000}%
\pgfsetfillcolor{currentfill}%
\pgfsetfillopacity{0.300000}%
\pgfsetlinewidth{0.301125pt}%
\definecolor{currentstroke}{rgb}{0.500000,0.500000,0.500000}%
\pgfsetstrokecolor{currentstroke}%
\pgfsetstrokeopacity{0.300000}%
\pgfsetdash{}{0pt}%
\pgfpathmoveto{\pgfqpoint{0.000000in}{0.000000in}}%
\pgfpathlineto{\pgfqpoint{0.000000in}{0.000000in}}%
\pgfpathclose%
\pgfusepath{stroke,fill}%
\end{pgfscope}%
\begin{pgfscope}%
\pgfpathrectangle{\pgfqpoint{0.647939in}{0.492442in}}{\pgfqpoint{3.079299in}{3.079299in}}%
\pgfusepath{clip}%
\pgfsetroundcap%
\pgfsetroundjoin%
\pgfsetlinewidth{0.301125pt}%
\definecolor{currentstroke}{rgb}{0.500000,0.500000,0.500000}%
\pgfsetstrokecolor{currentstroke}%
\pgfsetstrokeopacity{0.300000}%
\pgfsetdash{}{0pt}%
\pgfpathmoveto{\pgfqpoint{2.842415in}{2.543016in}}%
\pgfusepath{stroke}%
\end{pgfscope}%
\begin{pgfscope}%
\pgfpathrectangle{\pgfqpoint{0.647939in}{0.492442in}}{\pgfqpoint{3.079299in}{3.079299in}}%
\pgfusepath{clip}%
\pgfsetroundcap%
\pgfsetroundjoin%
\definecolor{currentfill}{rgb}{0.500000,0.500000,0.500000}%
\pgfsetfillcolor{currentfill}%
\pgfsetfillopacity{0.300000}%
\pgfsetlinewidth{0.301125pt}%
\definecolor{currentstroke}{rgb}{0.500000,0.500000,0.500000}%
\pgfsetstrokecolor{currentstroke}%
\pgfsetstrokeopacity{0.300000}%
\pgfsetdash{}{0pt}%
\pgfpathmoveto{\pgfqpoint{0.000000in}{0.000000in}}%
\pgfpathlineto{\pgfqpoint{0.000000in}{0.000000in}}%
\pgfpathclose%
\pgfusepath{stroke,fill}%
\end{pgfscope}%
\begin{pgfscope}%
\pgfpathrectangle{\pgfqpoint{0.647939in}{0.492442in}}{\pgfqpoint{3.079299in}{3.079299in}}%
\pgfusepath{clip}%
\pgfsetroundcap%
\pgfsetroundjoin%
\pgfsetlinewidth{0.301125pt}%
\definecolor{currentstroke}{rgb}{0.500000,0.500000,0.500000}%
\pgfsetstrokecolor{currentstroke}%
\pgfsetstrokeopacity{0.300000}%
\pgfsetdash{}{0pt}%
\pgfpathmoveto{\pgfqpoint{3.176236in}{2.126224in}}%
\pgfusepath{stroke}%
\end{pgfscope}%
\begin{pgfscope}%
\pgfpathrectangle{\pgfqpoint{0.647939in}{0.492442in}}{\pgfqpoint{3.079299in}{3.079299in}}%
\pgfusepath{clip}%
\pgfsetroundcap%
\pgfsetroundjoin%
\definecolor{currentfill}{rgb}{0.500000,0.500000,0.500000}%
\pgfsetfillcolor{currentfill}%
\pgfsetfillopacity{0.300000}%
\pgfsetlinewidth{0.301125pt}%
\definecolor{currentstroke}{rgb}{0.500000,0.500000,0.500000}%
\pgfsetstrokecolor{currentstroke}%
\pgfsetstrokeopacity{0.300000}%
\pgfsetdash{}{0pt}%
\pgfpathmoveto{\pgfqpoint{0.000000in}{0.000000in}}%
\pgfpathlineto{\pgfqpoint{0.000000in}{0.000000in}}%
\pgfpathclose%
\pgfusepath{stroke,fill}%
\end{pgfscope}%
\begin{pgfscope}%
\pgfpathrectangle{\pgfqpoint{0.647939in}{0.492442in}}{\pgfqpoint{3.079299in}{3.079299in}}%
\pgfusepath{clip}%
\pgfsetroundcap%
\pgfsetroundjoin%
\pgfsetlinewidth{0.301125pt}%
\definecolor{currentstroke}{rgb}{0.500000,0.500000,0.500000}%
\pgfsetstrokecolor{currentstroke}%
\pgfsetstrokeopacity{0.300000}%
\pgfsetdash{}{0pt}%
\pgfpathmoveto{\pgfqpoint{3.063609in}{2.533075in}}%
\pgfusepath{stroke}%
\end{pgfscope}%
\begin{pgfscope}%
\pgfpathrectangle{\pgfqpoint{0.647939in}{0.492442in}}{\pgfqpoint{3.079299in}{3.079299in}}%
\pgfusepath{clip}%
\pgfsetroundcap%
\pgfsetroundjoin%
\definecolor{currentfill}{rgb}{0.500000,0.500000,0.500000}%
\pgfsetfillcolor{currentfill}%
\pgfsetfillopacity{0.300000}%
\pgfsetlinewidth{0.301125pt}%
\definecolor{currentstroke}{rgb}{0.500000,0.500000,0.500000}%
\pgfsetstrokecolor{currentstroke}%
\pgfsetstrokeopacity{0.300000}%
\pgfsetdash{}{0pt}%
\pgfpathmoveto{\pgfqpoint{0.000000in}{0.000000in}}%
\pgfpathlineto{\pgfqpoint{0.000000in}{0.000000in}}%
\pgfpathclose%
\pgfusepath{stroke,fill}%
\end{pgfscope}%
\begin{pgfscope}%
\pgfpathrectangle{\pgfqpoint{0.647939in}{0.492442in}}{\pgfqpoint{3.079299in}{3.079299in}}%
\pgfusepath{clip}%
\pgfsetroundcap%
\pgfsetroundjoin%
\pgfsetlinewidth{0.301125pt}%
\definecolor{currentstroke}{rgb}{0.500000,0.500000,0.500000}%
\pgfsetstrokecolor{currentstroke}%
\pgfsetstrokeopacity{0.300000}%
\pgfsetdash{}{0pt}%
\pgfpathmoveto{\pgfqpoint{3.218059in}{2.688544in}}%
\pgfusepath{stroke}%
\end{pgfscope}%
\begin{pgfscope}%
\pgfpathrectangle{\pgfqpoint{0.647939in}{0.492442in}}{\pgfqpoint{3.079299in}{3.079299in}}%
\pgfusepath{clip}%
\pgfsetroundcap%
\pgfsetroundjoin%
\definecolor{currentfill}{rgb}{0.500000,0.500000,0.500000}%
\pgfsetfillcolor{currentfill}%
\pgfsetfillopacity{0.300000}%
\pgfsetlinewidth{0.301125pt}%
\definecolor{currentstroke}{rgb}{0.500000,0.500000,0.500000}%
\pgfsetstrokecolor{currentstroke}%
\pgfsetstrokeopacity{0.300000}%
\pgfsetdash{}{0pt}%
\pgfpathmoveto{\pgfqpoint{0.000000in}{0.000000in}}%
\pgfpathlineto{\pgfqpoint{0.000000in}{0.000000in}}%
\pgfpathclose%
\pgfusepath{stroke,fill}%
\end{pgfscope}%
\begin{pgfscope}%
\pgfpathrectangle{\pgfqpoint{0.647939in}{0.492442in}}{\pgfqpoint{3.079299in}{3.079299in}}%
\pgfusepath{clip}%
\pgfsetroundcap%
\pgfsetroundjoin%
\pgfsetlinewidth{0.301125pt}%
\definecolor{currentstroke}{rgb}{0.500000,0.500000,0.500000}%
\pgfsetstrokecolor{currentstroke}%
\pgfsetstrokeopacity{0.300000}%
\pgfsetdash{}{0pt}%
\pgfpathmoveto{\pgfqpoint{3.353302in}{2.687615in}}%
\pgfusepath{stroke}%
\end{pgfscope}%
\begin{pgfscope}%
\pgfpathrectangle{\pgfqpoint{0.647939in}{0.492442in}}{\pgfqpoint{3.079299in}{3.079299in}}%
\pgfusepath{clip}%
\pgfsetroundcap%
\pgfsetroundjoin%
\definecolor{currentfill}{rgb}{0.500000,0.500000,0.500000}%
\pgfsetfillcolor{currentfill}%
\pgfsetfillopacity{0.300000}%
\pgfsetlinewidth{0.301125pt}%
\definecolor{currentstroke}{rgb}{0.500000,0.500000,0.500000}%
\pgfsetstrokecolor{currentstroke}%
\pgfsetstrokeopacity{0.300000}%
\pgfsetdash{}{0pt}%
\pgfpathmoveto{\pgfqpoint{0.000000in}{0.000000in}}%
\pgfpathlineto{\pgfqpoint{0.000000in}{0.000000in}}%
\pgfpathclose%
\pgfusepath{stroke,fill}%
\end{pgfscope}%
\begin{pgfscope}%
\pgfpathrectangle{\pgfqpoint{0.647939in}{0.492442in}}{\pgfqpoint{3.079299in}{3.079299in}}%
\pgfusepath{clip}%
\pgfsetroundcap%
\pgfsetroundjoin%
\pgfsetlinewidth{0.301125pt}%
\definecolor{currentstroke}{rgb}{0.500000,0.500000,0.500000}%
\pgfsetstrokecolor{currentstroke}%
\pgfsetstrokeopacity{0.300000}%
\pgfsetdash{}{0pt}%
\pgfpathmoveto{\pgfqpoint{3.468137in}{2.740165in}}%
\pgfusepath{stroke}%
\end{pgfscope}%
\begin{pgfscope}%
\pgfpathrectangle{\pgfqpoint{0.647939in}{0.492442in}}{\pgfqpoint{3.079299in}{3.079299in}}%
\pgfusepath{clip}%
\pgfsetroundcap%
\pgfsetroundjoin%
\definecolor{currentfill}{rgb}{0.500000,0.500000,0.500000}%
\pgfsetfillcolor{currentfill}%
\pgfsetfillopacity{0.300000}%
\pgfsetlinewidth{0.301125pt}%
\definecolor{currentstroke}{rgb}{0.500000,0.500000,0.500000}%
\pgfsetstrokecolor{currentstroke}%
\pgfsetstrokeopacity{0.300000}%
\pgfsetdash{}{0pt}%
\pgfpathmoveto{\pgfqpoint{0.000000in}{0.000000in}}%
\pgfpathlineto{\pgfqpoint{0.000000in}{0.000000in}}%
\pgfpathclose%
\pgfusepath{stroke,fill}%
\end{pgfscope}%
\begin{pgfscope}%
\pgfpathrectangle{\pgfqpoint{0.647939in}{0.492442in}}{\pgfqpoint{3.079299in}{3.079299in}}%
\pgfusepath{clip}%
\pgfsetroundcap%
\pgfsetroundjoin%
\pgfsetlinewidth{0.301125pt}%
\definecolor{currentstroke}{rgb}{0.500000,0.500000,0.500000}%
\pgfsetstrokecolor{currentstroke}%
\pgfsetstrokeopacity{0.300000}%
\pgfsetdash{}{0pt}%
\pgfpathmoveto{\pgfqpoint{3.566230in}{2.779634in}}%
\pgfusepath{stroke}%
\end{pgfscope}%
\begin{pgfscope}%
\pgfpathrectangle{\pgfqpoint{0.647939in}{0.492442in}}{\pgfqpoint{3.079299in}{3.079299in}}%
\pgfusepath{clip}%
\pgfsetroundcap%
\pgfsetroundjoin%
\definecolor{currentfill}{rgb}{0.500000,0.500000,0.500000}%
\pgfsetfillcolor{currentfill}%
\pgfsetfillopacity{0.300000}%
\pgfsetlinewidth{0.301125pt}%
\definecolor{currentstroke}{rgb}{0.500000,0.500000,0.500000}%
\pgfsetstrokecolor{currentstroke}%
\pgfsetstrokeopacity{0.300000}%
\pgfsetdash{}{0pt}%
\pgfpathmoveto{\pgfqpoint{0.000000in}{0.000000in}}%
\pgfpathlineto{\pgfqpoint{0.000000in}{0.000000in}}%
\pgfpathclose%
\pgfusepath{stroke,fill}%
\end{pgfscope}%
\begin{pgfscope}%
\pgfpathrectangle{\pgfqpoint{0.647939in}{0.492442in}}{\pgfqpoint{3.079299in}{3.079299in}}%
\pgfusepath{clip}%
\pgfsetroundcap%
\pgfsetroundjoin%
\pgfsetlinewidth{0.301125pt}%
\definecolor{currentstroke}{rgb}{0.500000,0.500000,0.500000}%
\pgfsetstrokecolor{currentstroke}%
\pgfsetstrokeopacity{0.300000}%
\pgfsetdash{}{0pt}%
\pgfpathmoveto{\pgfqpoint{3.644648in}{2.738365in}}%
\pgfusepath{stroke}%
\end{pgfscope}%
\begin{pgfscope}%
\pgfpathrectangle{\pgfqpoint{0.647939in}{0.492442in}}{\pgfqpoint{3.079299in}{3.079299in}}%
\pgfusepath{clip}%
\pgfsetroundcap%
\pgfsetroundjoin%
\definecolor{currentfill}{rgb}{0.500000,0.500000,0.500000}%
\pgfsetfillcolor{currentfill}%
\pgfsetfillopacity{0.300000}%
\pgfsetlinewidth{0.301125pt}%
\definecolor{currentstroke}{rgb}{0.500000,0.500000,0.500000}%
\pgfsetstrokecolor{currentstroke}%
\pgfsetstrokeopacity{0.300000}%
\pgfsetdash{}{0pt}%
\pgfpathmoveto{\pgfqpoint{0.000000in}{0.000000in}}%
\pgfpathlineto{\pgfqpoint{0.000000in}{0.000000in}}%
\pgfpathclose%
\pgfusepath{stroke,fill}%
\end{pgfscope}%
\begin{pgfscope}%
\pgfpathrectangle{\pgfqpoint{0.647939in}{0.492442in}}{\pgfqpoint{3.079299in}{3.079299in}}%
\pgfusepath{clip}%
\pgfsetroundcap%
\pgfsetroundjoin%
\pgfsetlinewidth{0.301125pt}%
\definecolor{currentstroke}{rgb}{0.500000,0.500000,0.500000}%
\pgfsetstrokecolor{currentstroke}%
\pgfsetstrokeopacity{0.300000}%
\pgfsetdash{}{0pt}%
\pgfpathmoveto{\pgfqpoint{3.700039in}{2.753436in}}%
\pgfusepath{stroke}%
\end{pgfscope}%
\begin{pgfscope}%
\pgfpathrectangle{\pgfqpoint{0.647939in}{0.492442in}}{\pgfqpoint{3.079299in}{3.079299in}}%
\pgfusepath{clip}%
\pgfsetroundcap%
\pgfsetroundjoin%
\definecolor{currentfill}{rgb}{0.500000,0.500000,0.500000}%
\pgfsetfillcolor{currentfill}%
\pgfsetfillopacity{0.300000}%
\pgfsetlinewidth{0.301125pt}%
\definecolor{currentstroke}{rgb}{0.500000,0.500000,0.500000}%
\pgfsetstrokecolor{currentstroke}%
\pgfsetstrokeopacity{0.300000}%
\pgfsetdash{}{0pt}%
\pgfpathmoveto{\pgfqpoint{0.000000in}{0.000000in}}%
\pgfpathlineto{\pgfqpoint{0.000000in}{0.000000in}}%
\pgfpathclose%
\pgfusepath{stroke,fill}%
\end{pgfscope}%
\begin{pgfscope}%
\pgfpathrectangle{\pgfqpoint{0.647939in}{0.492442in}}{\pgfqpoint{3.079299in}{3.079299in}}%
\pgfusepath{clip}%
\pgfsetroundcap%
\pgfsetroundjoin%
\pgfsetlinewidth{0.301125pt}%
\definecolor{currentstroke}{rgb}{0.500000,0.500000,0.500000}%
\pgfsetstrokecolor{currentstroke}%
\pgfsetstrokeopacity{0.300000}%
\pgfsetdash{}{0pt}%
\pgfpathmoveto{\pgfqpoint{2.145087in}{2.905142in}}%
\pgfusepath{stroke}%
\end{pgfscope}%
\begin{pgfscope}%
\pgfpathrectangle{\pgfqpoint{0.647939in}{0.492442in}}{\pgfqpoint{3.079299in}{3.079299in}}%
\pgfusepath{clip}%
\pgfsetroundcap%
\pgfsetroundjoin%
\definecolor{currentfill}{rgb}{0.500000,0.500000,0.500000}%
\pgfsetfillcolor{currentfill}%
\pgfsetfillopacity{0.300000}%
\pgfsetlinewidth{0.301125pt}%
\definecolor{currentstroke}{rgb}{0.500000,0.500000,0.500000}%
\pgfsetstrokecolor{currentstroke}%
\pgfsetstrokeopacity{0.300000}%
\pgfsetdash{}{0pt}%
\pgfpathmoveto{\pgfqpoint{0.000000in}{0.000000in}}%
\pgfpathlineto{\pgfqpoint{0.000000in}{0.000000in}}%
\pgfpathclose%
\pgfusepath{stroke,fill}%
\end{pgfscope}%
\begin{pgfscope}%
\pgfpathrectangle{\pgfqpoint{0.647939in}{0.492442in}}{\pgfqpoint{3.079299in}{3.079299in}}%
\pgfusepath{clip}%
\pgfsetroundcap%
\pgfsetroundjoin%
\pgfsetlinewidth{0.301125pt}%
\definecolor{currentstroke}{rgb}{0.500000,0.500000,0.500000}%
\pgfsetstrokecolor{currentstroke}%
\pgfsetstrokeopacity{0.300000}%
\pgfsetdash{}{0pt}%
\pgfpathmoveto{\pgfqpoint{2.022414in}{3.164888in}}%
\pgfusepath{stroke}%
\end{pgfscope}%
\begin{pgfscope}%
\pgfpathrectangle{\pgfqpoint{0.647939in}{0.492442in}}{\pgfqpoint{3.079299in}{3.079299in}}%
\pgfusepath{clip}%
\pgfsetroundcap%
\pgfsetroundjoin%
\definecolor{currentfill}{rgb}{0.500000,0.500000,0.500000}%
\pgfsetfillcolor{currentfill}%
\pgfsetfillopacity{0.300000}%
\pgfsetlinewidth{0.301125pt}%
\definecolor{currentstroke}{rgb}{0.500000,0.500000,0.500000}%
\pgfsetstrokecolor{currentstroke}%
\pgfsetstrokeopacity{0.300000}%
\pgfsetdash{}{0pt}%
\pgfpathmoveto{\pgfqpoint{0.000000in}{0.000000in}}%
\pgfpathlineto{\pgfqpoint{0.000000in}{0.000000in}}%
\pgfpathclose%
\pgfusepath{stroke,fill}%
\end{pgfscope}%
\begin{pgfscope}%
\pgfpathrectangle{\pgfqpoint{0.647939in}{0.492442in}}{\pgfqpoint{3.079299in}{3.079299in}}%
\pgfusepath{clip}%
\pgfsetroundcap%
\pgfsetroundjoin%
\pgfsetlinewidth{0.301125pt}%
\definecolor{currentstroke}{rgb}{0.500000,0.500000,0.500000}%
\pgfsetstrokecolor{currentstroke}%
\pgfsetstrokeopacity{0.300000}%
\pgfsetdash{}{0pt}%
\pgfpathmoveto{\pgfqpoint{1.958589in}{3.322891in}}%
\pgfusepath{stroke}%
\end{pgfscope}%
\begin{pgfscope}%
\pgfpathrectangle{\pgfqpoint{0.647939in}{0.492442in}}{\pgfqpoint{3.079299in}{3.079299in}}%
\pgfusepath{clip}%
\pgfsetroundcap%
\pgfsetroundjoin%
\definecolor{currentfill}{rgb}{0.500000,0.500000,0.500000}%
\pgfsetfillcolor{currentfill}%
\pgfsetfillopacity{0.300000}%
\pgfsetlinewidth{0.301125pt}%
\definecolor{currentstroke}{rgb}{0.500000,0.500000,0.500000}%
\pgfsetstrokecolor{currentstroke}%
\pgfsetstrokeopacity{0.300000}%
\pgfsetdash{}{0pt}%
\pgfpathmoveto{\pgfqpoint{0.000000in}{0.000000in}}%
\pgfpathlineto{\pgfqpoint{0.000000in}{0.000000in}}%
\pgfpathclose%
\pgfusepath{stroke,fill}%
\end{pgfscope}%
\begin{pgfscope}%
\pgfpathrectangle{\pgfqpoint{0.647939in}{0.492442in}}{\pgfqpoint{3.079299in}{3.079299in}}%
\pgfusepath{clip}%
\pgfsetroundcap%
\pgfsetroundjoin%
\pgfsetlinewidth{0.301125pt}%
\definecolor{currentstroke}{rgb}{0.500000,0.500000,0.500000}%
\pgfsetstrokecolor{currentstroke}%
\pgfsetstrokeopacity{0.300000}%
\pgfsetdash{}{0pt}%
\pgfpathmoveto{\pgfqpoint{1.852647in}{3.430563in}}%
\pgfusepath{stroke}%
\end{pgfscope}%
\begin{pgfscope}%
\pgfpathrectangle{\pgfqpoint{0.647939in}{0.492442in}}{\pgfqpoint{3.079299in}{3.079299in}}%
\pgfusepath{clip}%
\pgfsetroundcap%
\pgfsetroundjoin%
\definecolor{currentfill}{rgb}{0.500000,0.500000,0.500000}%
\pgfsetfillcolor{currentfill}%
\pgfsetfillopacity{0.300000}%
\pgfsetlinewidth{0.301125pt}%
\definecolor{currentstroke}{rgb}{0.500000,0.500000,0.500000}%
\pgfsetstrokecolor{currentstroke}%
\pgfsetstrokeopacity{0.300000}%
\pgfsetdash{}{0pt}%
\pgfpathmoveto{\pgfqpoint{0.000000in}{0.000000in}}%
\pgfpathlineto{\pgfqpoint{0.000000in}{0.000000in}}%
\pgfpathclose%
\pgfusepath{stroke,fill}%
\end{pgfscope}%
\begin{pgfscope}%
\pgfpathrectangle{\pgfqpoint{0.647939in}{0.492442in}}{\pgfqpoint{3.079299in}{3.079299in}}%
\pgfusepath{clip}%
\pgfsetroundcap%
\pgfsetroundjoin%
\pgfsetlinewidth{0.301125pt}%
\definecolor{currentstroke}{rgb}{0.500000,0.500000,0.500000}%
\pgfsetstrokecolor{currentstroke}%
\pgfsetstrokeopacity{0.300000}%
\pgfsetdash{}{0pt}%
\pgfpathmoveto{\pgfqpoint{1.763143in}{3.500803in}}%
\pgfusepath{stroke}%
\end{pgfscope}%
\begin{pgfscope}%
\pgfpathrectangle{\pgfqpoint{0.647939in}{0.492442in}}{\pgfqpoint{3.079299in}{3.079299in}}%
\pgfusepath{clip}%
\pgfsetroundcap%
\pgfsetroundjoin%
\definecolor{currentfill}{rgb}{0.500000,0.500000,0.500000}%
\pgfsetfillcolor{currentfill}%
\pgfsetfillopacity{0.300000}%
\pgfsetlinewidth{0.301125pt}%
\definecolor{currentstroke}{rgb}{0.500000,0.500000,0.500000}%
\pgfsetstrokecolor{currentstroke}%
\pgfsetstrokeopacity{0.300000}%
\pgfsetdash{}{0pt}%
\pgfpathmoveto{\pgfqpoint{0.000000in}{0.000000in}}%
\pgfpathlineto{\pgfqpoint{0.000000in}{0.000000in}}%
\pgfpathclose%
\pgfusepath{stroke,fill}%
\end{pgfscope}%
\begin{pgfscope}%
\pgfpathrectangle{\pgfqpoint{0.647939in}{0.492442in}}{\pgfqpoint{3.079299in}{3.079299in}}%
\pgfusepath{clip}%
\pgfsetroundcap%
\pgfsetroundjoin%
\pgfsetlinewidth{0.301125pt}%
\definecolor{currentstroke}{rgb}{0.500000,0.500000,0.500000}%
\pgfsetstrokecolor{currentstroke}%
\pgfsetstrokeopacity{0.300000}%
\pgfsetdash{}{0pt}%
\pgfpathmoveto{\pgfqpoint{2.162419in}{3.547472in}}%
\pgfusepath{stroke}%
\end{pgfscope}%
\begin{pgfscope}%
\pgfpathrectangle{\pgfqpoint{0.647939in}{0.492442in}}{\pgfqpoint{3.079299in}{3.079299in}}%
\pgfusepath{clip}%
\pgfsetroundcap%
\pgfsetroundjoin%
\definecolor{currentfill}{rgb}{0.500000,0.500000,0.500000}%
\pgfsetfillcolor{currentfill}%
\pgfsetfillopacity{0.300000}%
\pgfsetlinewidth{0.301125pt}%
\definecolor{currentstroke}{rgb}{0.500000,0.500000,0.500000}%
\pgfsetstrokecolor{currentstroke}%
\pgfsetstrokeopacity{0.300000}%
\pgfsetdash{}{0pt}%
\pgfpathmoveto{\pgfqpoint{0.000000in}{0.000000in}}%
\pgfpathlineto{\pgfqpoint{0.000000in}{0.000000in}}%
\pgfpathclose%
\pgfusepath{stroke,fill}%
\end{pgfscope}%
\begin{pgfscope}%
\pgfpathrectangle{\pgfqpoint{0.647939in}{0.492442in}}{\pgfqpoint{3.079299in}{3.079299in}}%
\pgfusepath{clip}%
\pgfsetroundcap%
\pgfsetroundjoin%
\pgfsetlinewidth{0.301125pt}%
\definecolor{currentstroke}{rgb}{0.500000,0.500000,0.500000}%
\pgfsetstrokecolor{currentstroke}%
\pgfsetstrokeopacity{0.300000}%
\pgfsetdash{}{0pt}%
\pgfpathmoveto{\pgfqpoint{1.113102in}{3.495083in}}%
\pgfusepath{stroke}%
\end{pgfscope}%
\begin{pgfscope}%
\pgfpathrectangle{\pgfqpoint{0.647939in}{0.492442in}}{\pgfqpoint{3.079299in}{3.079299in}}%
\pgfusepath{clip}%
\pgfsetroundcap%
\pgfsetroundjoin%
\definecolor{currentfill}{rgb}{0.500000,0.500000,0.500000}%
\pgfsetfillcolor{currentfill}%
\pgfsetfillopacity{0.300000}%
\pgfsetlinewidth{0.301125pt}%
\definecolor{currentstroke}{rgb}{0.500000,0.500000,0.500000}%
\pgfsetstrokecolor{currentstroke}%
\pgfsetstrokeopacity{0.300000}%
\pgfsetdash{}{0pt}%
\pgfpathmoveto{\pgfqpoint{0.000000in}{0.000000in}}%
\pgfpathlineto{\pgfqpoint{0.000000in}{0.000000in}}%
\pgfpathclose%
\pgfusepath{stroke,fill}%
\end{pgfscope}%
\begin{pgfscope}%
\pgfpathrectangle{\pgfqpoint{0.647939in}{0.492442in}}{\pgfqpoint{3.079299in}{3.079299in}}%
\pgfusepath{clip}%
\pgfsetroundcap%
\pgfsetroundjoin%
\pgfsetlinewidth{0.301125pt}%
\definecolor{currentstroke}{rgb}{0.500000,0.500000,0.500000}%
\pgfsetstrokecolor{currentstroke}%
\pgfsetstrokeopacity{0.300000}%
\pgfsetdash{}{0pt}%
\pgfpathmoveto{\pgfqpoint{0.895083in}{3.532271in}}%
\pgfusepath{stroke}%
\end{pgfscope}%
\begin{pgfscope}%
\pgfpathrectangle{\pgfqpoint{0.647939in}{0.492442in}}{\pgfqpoint{3.079299in}{3.079299in}}%
\pgfusepath{clip}%
\pgfsetroundcap%
\pgfsetroundjoin%
\definecolor{currentfill}{rgb}{0.500000,0.500000,0.500000}%
\pgfsetfillcolor{currentfill}%
\pgfsetfillopacity{0.300000}%
\pgfsetlinewidth{0.301125pt}%
\definecolor{currentstroke}{rgb}{0.500000,0.500000,0.500000}%
\pgfsetstrokecolor{currentstroke}%
\pgfsetstrokeopacity{0.300000}%
\pgfsetdash{}{0pt}%
\pgfpathmoveto{\pgfqpoint{0.000000in}{0.000000in}}%
\pgfpathlineto{\pgfqpoint{0.000000in}{0.000000in}}%
\pgfpathclose%
\pgfusepath{stroke,fill}%
\end{pgfscope}%
\begin{pgfscope}%
\pgfpathrectangle{\pgfqpoint{0.647939in}{0.492442in}}{\pgfqpoint{3.079299in}{3.079299in}}%
\pgfusepath{clip}%
\pgfsetroundcap%
\pgfsetroundjoin%
\pgfsetlinewidth{0.301125pt}%
\definecolor{currentstroke}{rgb}{0.500000,0.500000,0.500000}%
\pgfsetstrokecolor{currentstroke}%
\pgfsetstrokeopacity{0.300000}%
\pgfsetdash{}{0pt}%
\pgfpathmoveto{\pgfqpoint{1.752839in}{3.065685in}}%
\pgfusepath{stroke}%
\end{pgfscope}%
\begin{pgfscope}%
\pgfpathrectangle{\pgfqpoint{0.647939in}{0.492442in}}{\pgfqpoint{3.079299in}{3.079299in}}%
\pgfusepath{clip}%
\pgfsetroundcap%
\pgfsetroundjoin%
\definecolor{currentfill}{rgb}{0.500000,0.500000,0.500000}%
\pgfsetfillcolor{currentfill}%
\pgfsetfillopacity{0.300000}%
\pgfsetlinewidth{0.301125pt}%
\definecolor{currentstroke}{rgb}{0.500000,0.500000,0.500000}%
\pgfsetstrokecolor{currentstroke}%
\pgfsetstrokeopacity{0.300000}%
\pgfsetdash{}{0pt}%
\pgfpathmoveto{\pgfqpoint{0.000000in}{0.000000in}}%
\pgfpathlineto{\pgfqpoint{0.000000in}{0.000000in}}%
\pgfpathclose%
\pgfusepath{stroke,fill}%
\end{pgfscope}%
\begin{pgfscope}%
\pgfpathrectangle{\pgfqpoint{0.647939in}{0.492442in}}{\pgfqpoint{3.079299in}{3.079299in}}%
\pgfusepath{clip}%
\pgfsetroundcap%
\pgfsetroundjoin%
\pgfsetlinewidth{0.301125pt}%
\definecolor{currentstroke}{rgb}{0.500000,0.500000,0.500000}%
\pgfsetstrokecolor{currentstroke}%
\pgfsetstrokeopacity{0.300000}%
\pgfsetdash{}{0pt}%
\pgfpathmoveto{\pgfqpoint{0.946375in}{2.843976in}}%
\pgfusepath{stroke}%
\end{pgfscope}%
\begin{pgfscope}%
\pgfpathrectangle{\pgfqpoint{0.647939in}{0.492442in}}{\pgfqpoint{3.079299in}{3.079299in}}%
\pgfusepath{clip}%
\pgfsetroundcap%
\pgfsetroundjoin%
\definecolor{currentfill}{rgb}{0.500000,0.500000,0.500000}%
\pgfsetfillcolor{currentfill}%
\pgfsetfillopacity{0.300000}%
\pgfsetlinewidth{0.301125pt}%
\definecolor{currentstroke}{rgb}{0.500000,0.500000,0.500000}%
\pgfsetstrokecolor{currentstroke}%
\pgfsetstrokeopacity{0.300000}%
\pgfsetdash{}{0pt}%
\pgfpathmoveto{\pgfqpoint{0.000000in}{0.000000in}}%
\pgfpathlineto{\pgfqpoint{0.000000in}{0.000000in}}%
\pgfpathclose%
\pgfusepath{stroke,fill}%
\end{pgfscope}%
\begin{pgfscope}%
\pgfpathrectangle{\pgfqpoint{0.647939in}{0.492442in}}{\pgfqpoint{3.079299in}{3.079299in}}%
\pgfusepath{clip}%
\pgfsetroundcap%
\pgfsetroundjoin%
\pgfsetlinewidth{0.301125pt}%
\definecolor{currentstroke}{rgb}{0.500000,0.500000,0.500000}%
\pgfsetstrokecolor{currentstroke}%
\pgfsetstrokeopacity{0.300000}%
\pgfsetdash{}{0pt}%
\pgfpathmoveto{\pgfqpoint{1.816783in}{2.813439in}}%
\pgfusepath{stroke}%
\end{pgfscope}%
\begin{pgfscope}%
\pgfpathrectangle{\pgfqpoint{0.647939in}{0.492442in}}{\pgfqpoint{3.079299in}{3.079299in}}%
\pgfusepath{clip}%
\pgfsetroundcap%
\pgfsetroundjoin%
\definecolor{currentfill}{rgb}{0.500000,0.500000,0.500000}%
\pgfsetfillcolor{currentfill}%
\pgfsetfillopacity{0.300000}%
\pgfsetlinewidth{0.301125pt}%
\definecolor{currentstroke}{rgb}{0.500000,0.500000,0.500000}%
\pgfsetstrokecolor{currentstroke}%
\pgfsetstrokeopacity{0.300000}%
\pgfsetdash{}{0pt}%
\pgfpathmoveto{\pgfqpoint{0.000000in}{0.000000in}}%
\pgfpathlineto{\pgfqpoint{0.000000in}{0.000000in}}%
\pgfpathclose%
\pgfusepath{stroke,fill}%
\end{pgfscope}%
\begin{pgfscope}%
\pgfpathrectangle{\pgfqpoint{0.647939in}{0.492442in}}{\pgfqpoint{3.079299in}{3.079299in}}%
\pgfusepath{clip}%
\pgfsetroundcap%
\pgfsetroundjoin%
\pgfsetlinewidth{0.301125pt}%
\definecolor{currentstroke}{rgb}{0.500000,0.500000,0.500000}%
\pgfsetstrokecolor{currentstroke}%
\pgfsetstrokeopacity{0.300000}%
\pgfsetdash{}{0pt}%
\pgfpathmoveto{\pgfqpoint{1.612238in}{2.722077in}}%
\pgfusepath{stroke}%
\end{pgfscope}%
\begin{pgfscope}%
\pgfpathrectangle{\pgfqpoint{0.647939in}{0.492442in}}{\pgfqpoint{3.079299in}{3.079299in}}%
\pgfusepath{clip}%
\pgfsetroundcap%
\pgfsetroundjoin%
\definecolor{currentfill}{rgb}{0.500000,0.500000,0.500000}%
\pgfsetfillcolor{currentfill}%
\pgfsetfillopacity{0.300000}%
\pgfsetlinewidth{0.301125pt}%
\definecolor{currentstroke}{rgb}{0.500000,0.500000,0.500000}%
\pgfsetstrokecolor{currentstroke}%
\pgfsetstrokeopacity{0.300000}%
\pgfsetdash{}{0pt}%
\pgfpathmoveto{\pgfqpoint{0.000000in}{0.000000in}}%
\pgfpathlineto{\pgfqpoint{0.000000in}{0.000000in}}%
\pgfpathclose%
\pgfusepath{stroke,fill}%
\end{pgfscope}%
\begin{pgfscope}%
\pgfpathrectangle{\pgfqpoint{0.647939in}{0.492442in}}{\pgfqpoint{3.079299in}{3.079299in}}%
\pgfusepath{clip}%
\pgfsetroundcap%
\pgfsetroundjoin%
\pgfsetlinewidth{0.301125pt}%
\definecolor{currentstroke}{rgb}{0.500000,0.500000,0.500000}%
\pgfsetstrokecolor{currentstroke}%
\pgfsetstrokeopacity{0.300000}%
\pgfsetdash{}{0pt}%
\pgfpathmoveto{\pgfqpoint{1.814011in}{2.686877in}}%
\pgfusepath{stroke}%
\end{pgfscope}%
\begin{pgfscope}%
\pgfpathrectangle{\pgfqpoint{0.647939in}{0.492442in}}{\pgfqpoint{3.079299in}{3.079299in}}%
\pgfusepath{clip}%
\pgfsetroundcap%
\pgfsetroundjoin%
\definecolor{currentfill}{rgb}{0.500000,0.500000,0.500000}%
\pgfsetfillcolor{currentfill}%
\pgfsetfillopacity{0.300000}%
\pgfsetlinewidth{0.301125pt}%
\definecolor{currentstroke}{rgb}{0.500000,0.500000,0.500000}%
\pgfsetstrokecolor{currentstroke}%
\pgfsetstrokeopacity{0.300000}%
\pgfsetdash{}{0pt}%
\pgfpathmoveto{\pgfqpoint{0.000000in}{0.000000in}}%
\pgfpathlineto{\pgfqpoint{0.000000in}{0.000000in}}%
\pgfpathclose%
\pgfusepath{stroke,fill}%
\end{pgfscope}%
\begin{pgfscope}%
\pgfpathrectangle{\pgfqpoint{0.647939in}{0.492442in}}{\pgfqpoint{3.079299in}{3.079299in}}%
\pgfusepath{clip}%
\pgfsetroundcap%
\pgfsetroundjoin%
\pgfsetlinewidth{0.301125pt}%
\definecolor{currentstroke}{rgb}{0.500000,0.500000,0.500000}%
\pgfsetstrokecolor{currentstroke}%
\pgfsetstrokeopacity{0.300000}%
\pgfsetdash{}{0pt}%
\pgfpathmoveto{\pgfqpoint{1.211966in}{2.491526in}}%
\pgfusepath{stroke}%
\end{pgfscope}%
\begin{pgfscope}%
\pgfpathrectangle{\pgfqpoint{0.647939in}{0.492442in}}{\pgfqpoint{3.079299in}{3.079299in}}%
\pgfusepath{clip}%
\pgfsetroundcap%
\pgfsetroundjoin%
\definecolor{currentfill}{rgb}{0.500000,0.500000,0.500000}%
\pgfsetfillcolor{currentfill}%
\pgfsetfillopacity{0.300000}%
\pgfsetlinewidth{0.301125pt}%
\definecolor{currentstroke}{rgb}{0.500000,0.500000,0.500000}%
\pgfsetstrokecolor{currentstroke}%
\pgfsetstrokeopacity{0.300000}%
\pgfsetdash{}{0pt}%
\pgfpathmoveto{\pgfqpoint{0.000000in}{0.000000in}}%
\pgfpathlineto{\pgfqpoint{0.000000in}{0.000000in}}%
\pgfpathclose%
\pgfusepath{stroke,fill}%
\end{pgfscope}%
\begin{pgfscope}%
\pgfpathrectangle{\pgfqpoint{0.647939in}{0.492442in}}{\pgfqpoint{3.079299in}{3.079299in}}%
\pgfusepath{clip}%
\pgfsetroundcap%
\pgfsetroundjoin%
\pgfsetlinewidth{0.301125pt}%
\definecolor{currentstroke}{rgb}{0.500000,0.500000,0.500000}%
\pgfsetstrokecolor{currentstroke}%
\pgfsetstrokeopacity{0.300000}%
\pgfsetdash{}{0pt}%
\pgfpathmoveto{\pgfqpoint{1.012457in}{2.373560in}}%
\pgfusepath{stroke}%
\end{pgfscope}%
\begin{pgfscope}%
\pgfpathrectangle{\pgfqpoint{0.647939in}{0.492442in}}{\pgfqpoint{3.079299in}{3.079299in}}%
\pgfusepath{clip}%
\pgfsetroundcap%
\pgfsetroundjoin%
\definecolor{currentfill}{rgb}{0.500000,0.500000,0.500000}%
\pgfsetfillcolor{currentfill}%
\pgfsetfillopacity{0.300000}%
\pgfsetlinewidth{0.301125pt}%
\definecolor{currentstroke}{rgb}{0.500000,0.500000,0.500000}%
\pgfsetstrokecolor{currentstroke}%
\pgfsetstrokeopacity{0.300000}%
\pgfsetdash{}{0pt}%
\pgfpathmoveto{\pgfqpoint{0.000000in}{0.000000in}}%
\pgfpathlineto{\pgfqpoint{0.000000in}{0.000000in}}%
\pgfpathclose%
\pgfusepath{stroke,fill}%
\end{pgfscope}%
\begin{pgfscope}%
\pgfpathrectangle{\pgfqpoint{0.647939in}{0.492442in}}{\pgfqpoint{3.079299in}{3.079299in}}%
\pgfusepath{clip}%
\pgfsetroundcap%
\pgfsetroundjoin%
\pgfsetlinewidth{0.301125pt}%
\definecolor{currentstroke}{rgb}{0.500000,0.500000,0.500000}%
\pgfsetstrokecolor{currentstroke}%
\pgfsetstrokeopacity{0.300000}%
\pgfsetdash{}{0pt}%
\pgfpathmoveto{\pgfqpoint{0.878355in}{2.276498in}}%
\pgfusepath{stroke}%
\end{pgfscope}%
\begin{pgfscope}%
\pgfpathrectangle{\pgfqpoint{0.647939in}{0.492442in}}{\pgfqpoint{3.079299in}{3.079299in}}%
\pgfusepath{clip}%
\pgfsetroundcap%
\pgfsetroundjoin%
\definecolor{currentfill}{rgb}{0.500000,0.500000,0.500000}%
\pgfsetfillcolor{currentfill}%
\pgfsetfillopacity{0.300000}%
\pgfsetlinewidth{0.301125pt}%
\definecolor{currentstroke}{rgb}{0.500000,0.500000,0.500000}%
\pgfsetstrokecolor{currentstroke}%
\pgfsetstrokeopacity{0.300000}%
\pgfsetdash{}{0pt}%
\pgfpathmoveto{\pgfqpoint{0.000000in}{0.000000in}}%
\pgfpathlineto{\pgfqpoint{0.000000in}{0.000000in}}%
\pgfpathclose%
\pgfusepath{stroke,fill}%
\end{pgfscope}%
\begin{pgfscope}%
\pgfpathrectangle{\pgfqpoint{0.647939in}{0.492442in}}{\pgfqpoint{3.079299in}{3.079299in}}%
\pgfusepath{clip}%
\pgfsetroundcap%
\pgfsetroundjoin%
\pgfsetlinewidth{0.301125pt}%
\definecolor{currentstroke}{rgb}{0.500000,0.500000,0.500000}%
\pgfsetstrokecolor{currentstroke}%
\pgfsetstrokeopacity{0.300000}%
\pgfsetdash{}{0pt}%
\pgfpathmoveto{\pgfqpoint{1.472719in}{2.366740in}}%
\pgfusepath{stroke}%
\end{pgfscope}%
\begin{pgfscope}%
\pgfpathrectangle{\pgfqpoint{0.647939in}{0.492442in}}{\pgfqpoint{3.079299in}{3.079299in}}%
\pgfusepath{clip}%
\pgfsetroundcap%
\pgfsetroundjoin%
\definecolor{currentfill}{rgb}{0.500000,0.500000,0.500000}%
\pgfsetfillcolor{currentfill}%
\pgfsetfillopacity{0.300000}%
\pgfsetlinewidth{0.301125pt}%
\definecolor{currentstroke}{rgb}{0.500000,0.500000,0.500000}%
\pgfsetstrokecolor{currentstroke}%
\pgfsetstrokeopacity{0.300000}%
\pgfsetdash{}{0pt}%
\pgfpathmoveto{\pgfqpoint{0.000000in}{0.000000in}}%
\pgfpathlineto{\pgfqpoint{0.000000in}{0.000000in}}%
\pgfpathclose%
\pgfusepath{stroke,fill}%
\end{pgfscope}%
\begin{pgfscope}%
\pgfpathrectangle{\pgfqpoint{0.647939in}{0.492442in}}{\pgfqpoint{3.079299in}{3.079299in}}%
\pgfusepath{clip}%
\pgfsetroundcap%
\pgfsetroundjoin%
\pgfsetlinewidth{0.301125pt}%
\definecolor{currentstroke}{rgb}{0.500000,0.500000,0.500000}%
\pgfsetstrokecolor{currentstroke}%
\pgfsetstrokeopacity{0.300000}%
\pgfsetdash{}{0pt}%
\pgfpathmoveto{\pgfqpoint{1.144092in}{2.202077in}}%
\pgfusepath{stroke}%
\end{pgfscope}%
\begin{pgfscope}%
\pgfpathrectangle{\pgfqpoint{0.647939in}{0.492442in}}{\pgfqpoint{3.079299in}{3.079299in}}%
\pgfusepath{clip}%
\pgfsetroundcap%
\pgfsetroundjoin%
\definecolor{currentfill}{rgb}{0.500000,0.500000,0.500000}%
\pgfsetfillcolor{currentfill}%
\pgfsetfillopacity{0.300000}%
\pgfsetlinewidth{0.301125pt}%
\definecolor{currentstroke}{rgb}{0.500000,0.500000,0.500000}%
\pgfsetstrokecolor{currentstroke}%
\pgfsetstrokeopacity{0.300000}%
\pgfsetdash{}{0pt}%
\pgfpathmoveto{\pgfqpoint{0.000000in}{0.000000in}}%
\pgfpathlineto{\pgfqpoint{0.000000in}{0.000000in}}%
\pgfpathclose%
\pgfusepath{stroke,fill}%
\end{pgfscope}%
\begin{pgfscope}%
\pgfpathrectangle{\pgfqpoint{0.647939in}{0.492442in}}{\pgfqpoint{3.079299in}{3.079299in}}%
\pgfusepath{clip}%
\pgfsetroundcap%
\pgfsetroundjoin%
\pgfsetlinewidth{0.301125pt}%
\definecolor{currentstroke}{rgb}{0.500000,0.500000,0.500000}%
\pgfsetstrokecolor{currentstroke}%
\pgfsetstrokeopacity{0.300000}%
\pgfsetdash{}{0pt}%
\pgfpathmoveto{\pgfqpoint{1.011568in}{2.098312in}}%
\pgfusepath{stroke}%
\end{pgfscope}%
\begin{pgfscope}%
\pgfpathrectangle{\pgfqpoint{0.647939in}{0.492442in}}{\pgfqpoint{3.079299in}{3.079299in}}%
\pgfusepath{clip}%
\pgfsetroundcap%
\pgfsetroundjoin%
\definecolor{currentfill}{rgb}{0.500000,0.500000,0.500000}%
\pgfsetfillcolor{currentfill}%
\pgfsetfillopacity{0.300000}%
\pgfsetlinewidth{0.301125pt}%
\definecolor{currentstroke}{rgb}{0.500000,0.500000,0.500000}%
\pgfsetstrokecolor{currentstroke}%
\pgfsetstrokeopacity{0.300000}%
\pgfsetdash{}{0pt}%
\pgfpathmoveto{\pgfqpoint{0.000000in}{0.000000in}}%
\pgfpathlineto{\pgfqpoint{0.000000in}{0.000000in}}%
\pgfpathclose%
\pgfusepath{stroke,fill}%
\end{pgfscope}%
\begin{pgfscope}%
\pgfpathrectangle{\pgfqpoint{0.647939in}{0.492442in}}{\pgfqpoint{3.079299in}{3.079299in}}%
\pgfusepath{clip}%
\pgfsetroundcap%
\pgfsetroundjoin%
\pgfsetlinewidth{0.301125pt}%
\definecolor{currentstroke}{rgb}{0.500000,0.500000,0.500000}%
\pgfsetstrokecolor{currentstroke}%
\pgfsetstrokeopacity{0.300000}%
\pgfsetdash{}{0pt}%
\pgfpathmoveto{\pgfqpoint{1.011312in}{2.029612in}}%
\pgfusepath{stroke}%
\end{pgfscope}%
\begin{pgfscope}%
\pgfpathrectangle{\pgfqpoint{0.647939in}{0.492442in}}{\pgfqpoint{3.079299in}{3.079299in}}%
\pgfusepath{clip}%
\pgfsetroundcap%
\pgfsetroundjoin%
\definecolor{currentfill}{rgb}{0.500000,0.500000,0.500000}%
\pgfsetfillcolor{currentfill}%
\pgfsetfillopacity{0.300000}%
\pgfsetlinewidth{0.301125pt}%
\definecolor{currentstroke}{rgb}{0.500000,0.500000,0.500000}%
\pgfsetstrokecolor{currentstroke}%
\pgfsetstrokeopacity{0.300000}%
\pgfsetdash{}{0pt}%
\pgfpathmoveto{\pgfqpoint{0.000000in}{0.000000in}}%
\pgfpathlineto{\pgfqpoint{0.000000in}{0.000000in}}%
\pgfpathclose%
\pgfusepath{stroke,fill}%
\end{pgfscope}%
\begin{pgfscope}%
\pgfpathrectangle{\pgfqpoint{0.647939in}{0.492442in}}{\pgfqpoint{3.079299in}{3.079299in}}%
\pgfusepath{clip}%
\pgfsetroundcap%
\pgfsetroundjoin%
\pgfsetlinewidth{0.301125pt}%
\definecolor{currentstroke}{rgb}{0.500000,0.500000,0.500000}%
\pgfsetstrokecolor{currentstroke}%
\pgfsetstrokeopacity{0.300000}%
\pgfsetdash{}{0pt}%
\pgfpathmoveto{\pgfqpoint{1.466027in}{2.110107in}}%
\pgfusepath{stroke}%
\end{pgfscope}%
\begin{pgfscope}%
\pgfpathrectangle{\pgfqpoint{0.647939in}{0.492442in}}{\pgfqpoint{3.079299in}{3.079299in}}%
\pgfusepath{clip}%
\pgfsetroundcap%
\pgfsetroundjoin%
\definecolor{currentfill}{rgb}{0.500000,0.500000,0.500000}%
\pgfsetfillcolor{currentfill}%
\pgfsetfillopacity{0.300000}%
\pgfsetlinewidth{0.301125pt}%
\definecolor{currentstroke}{rgb}{0.500000,0.500000,0.500000}%
\pgfsetstrokecolor{currentstroke}%
\pgfsetstrokeopacity{0.300000}%
\pgfsetdash{}{0pt}%
\pgfpathmoveto{\pgfqpoint{0.000000in}{0.000000in}}%
\pgfpathlineto{\pgfqpoint{0.000000in}{0.000000in}}%
\pgfpathclose%
\pgfusepath{stroke,fill}%
\end{pgfscope}%
\begin{pgfscope}%
\pgfpathrectangle{\pgfqpoint{0.647939in}{0.492442in}}{\pgfqpoint{3.079299in}{3.079299in}}%
\pgfusepath{clip}%
\pgfsetroundcap%
\pgfsetroundjoin%
\pgfsetlinewidth{0.301125pt}%
\definecolor{currentstroke}{rgb}{0.500000,0.500000,0.500000}%
\pgfsetstrokecolor{currentstroke}%
\pgfsetstrokeopacity{0.300000}%
\pgfsetdash{}{0pt}%
\pgfpathmoveto{\pgfqpoint{1.206887in}{1.952759in}}%
\pgfusepath{stroke}%
\end{pgfscope}%
\begin{pgfscope}%
\pgfpathrectangle{\pgfqpoint{0.647939in}{0.492442in}}{\pgfqpoint{3.079299in}{3.079299in}}%
\pgfusepath{clip}%
\pgfsetroundcap%
\pgfsetroundjoin%
\definecolor{currentfill}{rgb}{0.500000,0.500000,0.500000}%
\pgfsetfillcolor{currentfill}%
\pgfsetfillopacity{0.300000}%
\pgfsetlinewidth{0.301125pt}%
\definecolor{currentstroke}{rgb}{0.500000,0.500000,0.500000}%
\pgfsetstrokecolor{currentstroke}%
\pgfsetstrokeopacity{0.300000}%
\pgfsetdash{}{0pt}%
\pgfpathmoveto{\pgfqpoint{0.000000in}{0.000000in}}%
\pgfpathlineto{\pgfqpoint{0.000000in}{0.000000in}}%
\pgfpathclose%
\pgfusepath{stroke,fill}%
\end{pgfscope}%
\begin{pgfscope}%
\pgfpathrectangle{\pgfqpoint{0.647939in}{0.492442in}}{\pgfqpoint{3.079299in}{3.079299in}}%
\pgfusepath{clip}%
\pgfsetroundcap%
\pgfsetroundjoin%
\pgfsetlinewidth{0.301125pt}%
\definecolor{currentstroke}{rgb}{0.500000,0.500000,0.500000}%
\pgfsetstrokecolor{currentstroke}%
\pgfsetstrokeopacity{0.300000}%
\pgfsetdash{}{0pt}%
\pgfpathmoveto{\pgfqpoint{1.076131in}{1.842942in}}%
\pgfusepath{stroke}%
\end{pgfscope}%
\begin{pgfscope}%
\pgfpathrectangle{\pgfqpoint{0.647939in}{0.492442in}}{\pgfqpoint{3.079299in}{3.079299in}}%
\pgfusepath{clip}%
\pgfsetroundcap%
\pgfsetroundjoin%
\definecolor{currentfill}{rgb}{0.500000,0.500000,0.500000}%
\pgfsetfillcolor{currentfill}%
\pgfsetfillopacity{0.300000}%
\pgfsetlinewidth{0.301125pt}%
\definecolor{currentstroke}{rgb}{0.500000,0.500000,0.500000}%
\pgfsetstrokecolor{currentstroke}%
\pgfsetstrokeopacity{0.300000}%
\pgfsetdash{}{0pt}%
\pgfpathmoveto{\pgfqpoint{0.000000in}{0.000000in}}%
\pgfpathlineto{\pgfqpoint{0.000000in}{0.000000in}}%
\pgfpathclose%
\pgfusepath{stroke,fill}%
\end{pgfscope}%
\begin{pgfscope}%
\pgfpathrectangle{\pgfqpoint{0.647939in}{0.492442in}}{\pgfqpoint{3.079299in}{3.079299in}}%
\pgfusepath{clip}%
\pgfsetroundcap%
\pgfsetroundjoin%
\pgfsetlinewidth{0.301125pt}%
\definecolor{currentstroke}{rgb}{0.500000,0.500000,0.500000}%
\pgfsetstrokecolor{currentstroke}%
\pgfsetstrokeopacity{0.300000}%
\pgfsetdash{}{0pt}%
\pgfpathmoveto{\pgfqpoint{0.810484in}{1.707935in}}%
\pgfusepath{stroke}%
\end{pgfscope}%
\begin{pgfscope}%
\pgfpathrectangle{\pgfqpoint{0.647939in}{0.492442in}}{\pgfqpoint{3.079299in}{3.079299in}}%
\pgfusepath{clip}%
\pgfsetroundcap%
\pgfsetroundjoin%
\definecolor{currentfill}{rgb}{0.500000,0.500000,0.500000}%
\pgfsetfillcolor{currentfill}%
\pgfsetfillopacity{0.300000}%
\pgfsetlinewidth{0.301125pt}%
\definecolor{currentstroke}{rgb}{0.500000,0.500000,0.500000}%
\pgfsetstrokecolor{currentstroke}%
\pgfsetstrokeopacity{0.300000}%
\pgfsetdash{}{0pt}%
\pgfpathmoveto{\pgfqpoint{0.000000in}{0.000000in}}%
\pgfpathlineto{\pgfqpoint{0.000000in}{0.000000in}}%
\pgfpathclose%
\pgfusepath{stroke,fill}%
\end{pgfscope}%
\begin{pgfscope}%
\pgfpathrectangle{\pgfqpoint{0.647939in}{0.492442in}}{\pgfqpoint{3.079299in}{3.079299in}}%
\pgfusepath{clip}%
\pgfsetroundcap%
\pgfsetroundjoin%
\pgfsetlinewidth{0.301125pt}%
\definecolor{currentstroke}{rgb}{0.500000,0.500000,0.500000}%
\pgfsetstrokecolor{currentstroke}%
\pgfsetstrokeopacity{0.300000}%
\pgfsetdash{}{0pt}%
\pgfpathmoveto{\pgfqpoint{1.393625in}{1.831590in}}%
\pgfusepath{stroke}%
\end{pgfscope}%
\begin{pgfscope}%
\pgfpathrectangle{\pgfqpoint{0.647939in}{0.492442in}}{\pgfqpoint{3.079299in}{3.079299in}}%
\pgfusepath{clip}%
\pgfsetroundcap%
\pgfsetroundjoin%
\definecolor{currentfill}{rgb}{0.500000,0.500000,0.500000}%
\pgfsetfillcolor{currentfill}%
\pgfsetfillopacity{0.300000}%
\pgfsetlinewidth{0.301125pt}%
\definecolor{currentstroke}{rgb}{0.500000,0.500000,0.500000}%
\pgfsetstrokecolor{currentstroke}%
\pgfsetstrokeopacity{0.300000}%
\pgfsetdash{}{0pt}%
\pgfpathmoveto{\pgfqpoint{0.000000in}{0.000000in}}%
\pgfpathlineto{\pgfqpoint{0.000000in}{0.000000in}}%
\pgfpathclose%
\pgfusepath{stroke,fill}%
\end{pgfscope}%
\begin{pgfscope}%
\pgfpathrectangle{\pgfqpoint{0.647939in}{0.492442in}}{\pgfqpoint{3.079299in}{3.079299in}}%
\pgfusepath{clip}%
\pgfsetroundcap%
\pgfsetroundjoin%
\pgfsetlinewidth{0.301125pt}%
\definecolor{currentstroke}{rgb}{0.500000,0.500000,0.500000}%
\pgfsetstrokecolor{currentstroke}%
\pgfsetstrokeopacity{0.300000}%
\pgfsetdash{}{0pt}%
\pgfpathmoveto{\pgfqpoint{1.202975in}{1.686551in}}%
\pgfusepath{stroke}%
\end{pgfscope}%
\begin{pgfscope}%
\pgfpathrectangle{\pgfqpoint{0.647939in}{0.492442in}}{\pgfqpoint{3.079299in}{3.079299in}}%
\pgfusepath{clip}%
\pgfsetroundcap%
\pgfsetroundjoin%
\definecolor{currentfill}{rgb}{0.500000,0.500000,0.500000}%
\pgfsetfillcolor{currentfill}%
\pgfsetfillopacity{0.300000}%
\pgfsetlinewidth{0.301125pt}%
\definecolor{currentstroke}{rgb}{0.500000,0.500000,0.500000}%
\pgfsetstrokecolor{currentstroke}%
\pgfsetstrokeopacity{0.300000}%
\pgfsetdash{}{0pt}%
\pgfpathmoveto{\pgfqpoint{0.000000in}{0.000000in}}%
\pgfpathlineto{\pgfqpoint{0.000000in}{0.000000in}}%
\pgfpathclose%
\pgfusepath{stroke,fill}%
\end{pgfscope}%
\begin{pgfscope}%
\pgfpathrectangle{\pgfqpoint{0.647939in}{0.492442in}}{\pgfqpoint{3.079299in}{3.079299in}}%
\pgfusepath{clip}%
\pgfsetroundcap%
\pgfsetroundjoin%
\pgfsetlinewidth{0.301125pt}%
\definecolor{currentstroke}{rgb}{0.500000,0.500000,0.500000}%
\pgfsetstrokecolor{currentstroke}%
\pgfsetstrokeopacity{0.300000}%
\pgfsetdash{}{0pt}%
\pgfpathmoveto{\pgfqpoint{1.324427in}{1.612449in}}%
\pgfusepath{stroke}%
\end{pgfscope}%
\begin{pgfscope}%
\pgfpathrectangle{\pgfqpoint{0.647939in}{0.492442in}}{\pgfqpoint{3.079299in}{3.079299in}}%
\pgfusepath{clip}%
\pgfsetroundcap%
\pgfsetroundjoin%
\definecolor{currentfill}{rgb}{0.500000,0.500000,0.500000}%
\pgfsetfillcolor{currentfill}%
\pgfsetfillopacity{0.300000}%
\pgfsetlinewidth{0.301125pt}%
\definecolor{currentstroke}{rgb}{0.500000,0.500000,0.500000}%
\pgfsetstrokecolor{currentstroke}%
\pgfsetstrokeopacity{0.300000}%
\pgfsetdash{}{0pt}%
\pgfpathmoveto{\pgfqpoint{0.000000in}{0.000000in}}%
\pgfpathlineto{\pgfqpoint{0.000000in}{0.000000in}}%
\pgfpathclose%
\pgfusepath{stroke,fill}%
\end{pgfscope}%
\begin{pgfscope}%
\pgfpathrectangle{\pgfqpoint{0.647939in}{0.492442in}}{\pgfqpoint{3.079299in}{3.079299in}}%
\pgfusepath{clip}%
\pgfsetroundcap%
\pgfsetroundjoin%
\pgfsetlinewidth{0.301125pt}%
\definecolor{currentstroke}{rgb}{0.500000,0.500000,0.500000}%
\pgfsetstrokecolor{currentstroke}%
\pgfsetstrokeopacity{0.300000}%
\pgfsetdash{}{0pt}%
\pgfpathmoveto{\pgfqpoint{1.136327in}{1.461532in}}%
\pgfusepath{stroke}%
\end{pgfscope}%
\begin{pgfscope}%
\pgfpathrectangle{\pgfqpoint{0.647939in}{0.492442in}}{\pgfqpoint{3.079299in}{3.079299in}}%
\pgfusepath{clip}%
\pgfsetroundcap%
\pgfsetroundjoin%
\definecolor{currentfill}{rgb}{0.500000,0.500000,0.500000}%
\pgfsetfillcolor{currentfill}%
\pgfsetfillopacity{0.300000}%
\pgfsetlinewidth{0.301125pt}%
\definecolor{currentstroke}{rgb}{0.500000,0.500000,0.500000}%
\pgfsetstrokecolor{currentstroke}%
\pgfsetstrokeopacity{0.300000}%
\pgfsetdash{}{0pt}%
\pgfpathmoveto{\pgfqpoint{0.000000in}{0.000000in}}%
\pgfpathlineto{\pgfqpoint{0.000000in}{0.000000in}}%
\pgfpathclose%
\pgfusepath{stroke,fill}%
\end{pgfscope}%
\begin{pgfscope}%
\pgfpathrectangle{\pgfqpoint{0.647939in}{0.492442in}}{\pgfqpoint{3.079299in}{3.079299in}}%
\pgfusepath{clip}%
\pgfsetroundcap%
\pgfsetroundjoin%
\pgfsetlinewidth{0.301125pt}%
\definecolor{currentstroke}{rgb}{0.500000,0.500000,0.500000}%
\pgfsetstrokecolor{currentstroke}%
\pgfsetstrokeopacity{0.300000}%
\pgfsetdash{}{0pt}%
\pgfpathmoveto{\pgfqpoint{0.876513in}{1.306838in}}%
\pgfusepath{stroke}%
\end{pgfscope}%
\begin{pgfscope}%
\pgfpathrectangle{\pgfqpoint{0.647939in}{0.492442in}}{\pgfqpoint{3.079299in}{3.079299in}}%
\pgfusepath{clip}%
\pgfsetroundcap%
\pgfsetroundjoin%
\definecolor{currentfill}{rgb}{0.500000,0.500000,0.500000}%
\pgfsetfillcolor{currentfill}%
\pgfsetfillopacity{0.300000}%
\pgfsetlinewidth{0.301125pt}%
\definecolor{currentstroke}{rgb}{0.500000,0.500000,0.500000}%
\pgfsetstrokecolor{currentstroke}%
\pgfsetstrokeopacity{0.300000}%
\pgfsetdash{}{0pt}%
\pgfpathmoveto{\pgfqpoint{0.000000in}{0.000000in}}%
\pgfpathlineto{\pgfqpoint{0.000000in}{0.000000in}}%
\pgfpathclose%
\pgfusepath{stroke,fill}%
\end{pgfscope}%
\begin{pgfscope}%
\pgfpathrectangle{\pgfqpoint{0.647939in}{0.492442in}}{\pgfqpoint{3.079299in}{3.079299in}}%
\pgfusepath{clip}%
\pgfsetroundcap%
\pgfsetroundjoin%
\pgfsetlinewidth{0.301125pt}%
\definecolor{currentstroke}{rgb}{0.500000,0.500000,0.500000}%
\pgfsetstrokecolor{currentstroke}%
\pgfsetstrokeopacity{0.300000}%
\pgfsetdash{}{0pt}%
\pgfpathmoveto{\pgfqpoint{1.256500in}{1.389612in}}%
\pgfusepath{stroke}%
\end{pgfscope}%
\begin{pgfscope}%
\pgfpathrectangle{\pgfqpoint{0.647939in}{0.492442in}}{\pgfqpoint{3.079299in}{3.079299in}}%
\pgfusepath{clip}%
\pgfsetroundcap%
\pgfsetroundjoin%
\definecolor{currentfill}{rgb}{0.500000,0.500000,0.500000}%
\pgfsetfillcolor{currentfill}%
\pgfsetfillopacity{0.300000}%
\pgfsetlinewidth{0.301125pt}%
\definecolor{currentstroke}{rgb}{0.500000,0.500000,0.500000}%
\pgfsetstrokecolor{currentstroke}%
\pgfsetstrokeopacity{0.300000}%
\pgfsetdash{}{0pt}%
\pgfpathmoveto{\pgfqpoint{0.000000in}{0.000000in}}%
\pgfpathlineto{\pgfqpoint{0.000000in}{0.000000in}}%
\pgfpathclose%
\pgfusepath{stroke,fill}%
\end{pgfscope}%
\begin{pgfscope}%
\pgfpathrectangle{\pgfqpoint{0.647939in}{0.492442in}}{\pgfqpoint{3.079299in}{3.079299in}}%
\pgfusepath{clip}%
\pgfsetroundcap%
\pgfsetroundjoin%
\pgfsetlinewidth{0.301125pt}%
\definecolor{currentstroke}{rgb}{0.500000,0.500000,0.500000}%
\pgfsetstrokecolor{currentstroke}%
\pgfsetstrokeopacity{0.300000}%
\pgfsetdash{}{0pt}%
\pgfpathmoveto{\pgfqpoint{1.132773in}{1.262505in}}%
\pgfusepath{stroke}%
\end{pgfscope}%
\begin{pgfscope}%
\pgfpathrectangle{\pgfqpoint{0.647939in}{0.492442in}}{\pgfqpoint{3.079299in}{3.079299in}}%
\pgfusepath{clip}%
\pgfsetroundcap%
\pgfsetroundjoin%
\definecolor{currentfill}{rgb}{0.500000,0.500000,0.500000}%
\pgfsetfillcolor{currentfill}%
\pgfsetfillopacity{0.300000}%
\pgfsetlinewidth{0.301125pt}%
\definecolor{currentstroke}{rgb}{0.500000,0.500000,0.500000}%
\pgfsetstrokecolor{currentstroke}%
\pgfsetstrokeopacity{0.300000}%
\pgfsetdash{}{0pt}%
\pgfpathmoveto{\pgfqpoint{0.000000in}{0.000000in}}%
\pgfpathlineto{\pgfqpoint{0.000000in}{0.000000in}}%
\pgfpathclose%
\pgfusepath{stroke,fill}%
\end{pgfscope}%
\begin{pgfscope}%
\pgfpathrectangle{\pgfqpoint{0.647939in}{0.492442in}}{\pgfqpoint{3.079299in}{3.079299in}}%
\pgfusepath{clip}%
\pgfsetroundcap%
\pgfsetroundjoin%
\pgfsetlinewidth{0.301125pt}%
\definecolor{currentstroke}{rgb}{0.500000,0.500000,0.500000}%
\pgfsetstrokecolor{currentstroke}%
\pgfsetstrokeopacity{0.300000}%
\pgfsetdash{}{0pt}%
\pgfpathmoveto{\pgfqpoint{1.005878in}{1.142252in}}%
\pgfusepath{stroke}%
\end{pgfscope}%
\begin{pgfscope}%
\pgfpathrectangle{\pgfqpoint{0.647939in}{0.492442in}}{\pgfqpoint{3.079299in}{3.079299in}}%
\pgfusepath{clip}%
\pgfsetroundcap%
\pgfsetroundjoin%
\definecolor{currentfill}{rgb}{0.500000,0.500000,0.500000}%
\pgfsetfillcolor{currentfill}%
\pgfsetfillopacity{0.300000}%
\pgfsetlinewidth{0.301125pt}%
\definecolor{currentstroke}{rgb}{0.500000,0.500000,0.500000}%
\pgfsetstrokecolor{currentstroke}%
\pgfsetstrokeopacity{0.300000}%
\pgfsetdash{}{0pt}%
\pgfpathmoveto{\pgfqpoint{0.000000in}{0.000000in}}%
\pgfpathlineto{\pgfqpoint{0.000000in}{0.000000in}}%
\pgfpathclose%
\pgfusepath{stroke,fill}%
\end{pgfscope}%
\begin{pgfscope}%
\pgfpathrectangle{\pgfqpoint{0.647939in}{0.492442in}}{\pgfqpoint{3.079299in}{3.079299in}}%
\pgfusepath{clip}%
\pgfsetroundcap%
\pgfsetroundjoin%
\pgfsetlinewidth{0.301125pt}%
\definecolor{currentstroke}{rgb}{0.500000,0.500000,0.500000}%
\pgfsetstrokecolor{currentstroke}%
\pgfsetstrokeopacity{0.300000}%
\pgfsetdash{}{0pt}%
\pgfpathmoveto{\pgfqpoint{0.809473in}{1.013588in}}%
\pgfusepath{stroke}%
\end{pgfscope}%
\begin{pgfscope}%
\pgfpathrectangle{\pgfqpoint{0.647939in}{0.492442in}}{\pgfqpoint{3.079299in}{3.079299in}}%
\pgfusepath{clip}%
\pgfsetroundcap%
\pgfsetroundjoin%
\definecolor{currentfill}{rgb}{0.500000,0.500000,0.500000}%
\pgfsetfillcolor{currentfill}%
\pgfsetfillopacity{0.300000}%
\pgfsetlinewidth{0.301125pt}%
\definecolor{currentstroke}{rgb}{0.500000,0.500000,0.500000}%
\pgfsetstrokecolor{currentstroke}%
\pgfsetstrokeopacity{0.300000}%
\pgfsetdash{}{0pt}%
\pgfpathmoveto{\pgfqpoint{0.000000in}{0.000000in}}%
\pgfpathlineto{\pgfqpoint{0.000000in}{0.000000in}}%
\pgfpathclose%
\pgfusepath{stroke,fill}%
\end{pgfscope}%
\begin{pgfscope}%
\pgfpathrectangle{\pgfqpoint{0.647939in}{0.492442in}}{\pgfqpoint{3.079299in}{3.079299in}}%
\pgfusepath{clip}%
\pgfsetroundcap%
\pgfsetroundjoin%
\pgfsetlinewidth{0.301125pt}%
\definecolor{currentstroke}{rgb}{0.500000,0.500000,0.500000}%
\pgfsetstrokecolor{currentstroke}%
\pgfsetstrokeopacity{0.300000}%
\pgfsetdash{}{0pt}%
\pgfpathmoveto{\pgfqpoint{1.128171in}{1.065224in}}%
\pgfusepath{stroke}%
\end{pgfscope}%
\begin{pgfscope}%
\pgfpathrectangle{\pgfqpoint{0.647939in}{0.492442in}}{\pgfqpoint{3.079299in}{3.079299in}}%
\pgfusepath{clip}%
\pgfsetroundcap%
\pgfsetroundjoin%
\definecolor{currentfill}{rgb}{0.500000,0.500000,0.500000}%
\pgfsetfillcolor{currentfill}%
\pgfsetfillopacity{0.300000}%
\pgfsetlinewidth{0.301125pt}%
\definecolor{currentstroke}{rgb}{0.500000,0.500000,0.500000}%
\pgfsetstrokecolor{currentstroke}%
\pgfsetstrokeopacity{0.300000}%
\pgfsetdash{}{0pt}%
\pgfpathmoveto{\pgfqpoint{0.000000in}{0.000000in}}%
\pgfpathlineto{\pgfqpoint{0.000000in}{0.000000in}}%
\pgfpathclose%
\pgfusepath{stroke,fill}%
\end{pgfscope}%
\begin{pgfscope}%
\pgfpathrectangle{\pgfqpoint{0.647939in}{0.492442in}}{\pgfqpoint{3.079299in}{3.079299in}}%
\pgfusepath{clip}%
\pgfsetroundcap%
\pgfsetroundjoin%
\pgfsetlinewidth{0.301125pt}%
\definecolor{currentstroke}{rgb}{0.500000,0.500000,0.500000}%
\pgfsetstrokecolor{currentstroke}%
\pgfsetstrokeopacity{0.300000}%
\pgfsetdash{}{0pt}%
\pgfpathmoveto{\pgfqpoint{0.875085in}{0.893277in}}%
\pgfusepath{stroke}%
\end{pgfscope}%
\begin{pgfscope}%
\pgfpathrectangle{\pgfqpoint{0.647939in}{0.492442in}}{\pgfqpoint{3.079299in}{3.079299in}}%
\pgfusepath{clip}%
\pgfsetroundcap%
\pgfsetroundjoin%
\definecolor{currentfill}{rgb}{0.500000,0.500000,0.500000}%
\pgfsetfillcolor{currentfill}%
\pgfsetfillopacity{0.300000}%
\pgfsetlinewidth{0.301125pt}%
\definecolor{currentstroke}{rgb}{0.500000,0.500000,0.500000}%
\pgfsetstrokecolor{currentstroke}%
\pgfsetstrokeopacity{0.300000}%
\pgfsetdash{}{0pt}%
\pgfpathmoveto{\pgfqpoint{0.000000in}{0.000000in}}%
\pgfpathlineto{\pgfqpoint{0.000000in}{0.000000in}}%
\pgfpathclose%
\pgfusepath{stroke,fill}%
\end{pgfscope}%
\begin{pgfscope}%
\pgfpathrectangle{\pgfqpoint{0.647939in}{0.492442in}}{\pgfqpoint{3.079299in}{3.079299in}}%
\pgfusepath{clip}%
\pgfsetroundcap%
\pgfsetroundjoin%
\pgfsetlinewidth{0.301125pt}%
\definecolor{currentstroke}{rgb}{0.500000,0.500000,0.500000}%
\pgfsetstrokecolor{currentstroke}%
\pgfsetstrokeopacity{0.300000}%
\pgfsetdash{}{0pt}%
\pgfpathmoveto{\pgfqpoint{1.124393in}{0.934857in}}%
\pgfusepath{stroke}%
\end{pgfscope}%
\begin{pgfscope}%
\pgfpathrectangle{\pgfqpoint{0.647939in}{0.492442in}}{\pgfqpoint{3.079299in}{3.079299in}}%
\pgfusepath{clip}%
\pgfsetroundcap%
\pgfsetroundjoin%
\definecolor{currentfill}{rgb}{0.500000,0.500000,0.500000}%
\pgfsetfillcolor{currentfill}%
\pgfsetfillopacity{0.300000}%
\pgfsetlinewidth{0.301125pt}%
\definecolor{currentstroke}{rgb}{0.500000,0.500000,0.500000}%
\pgfsetstrokecolor{currentstroke}%
\pgfsetstrokeopacity{0.300000}%
\pgfsetdash{}{0pt}%
\pgfpathmoveto{\pgfqpoint{0.000000in}{0.000000in}}%
\pgfpathlineto{\pgfqpoint{0.000000in}{0.000000in}}%
\pgfpathclose%
\pgfusepath{stroke,fill}%
\end{pgfscope}%
\begin{pgfscope}%
\pgfpathrectangle{\pgfqpoint{0.647939in}{0.492442in}}{\pgfqpoint{3.079299in}{3.079299in}}%
\pgfusepath{clip}%
\pgfsetroundcap%
\pgfsetroundjoin%
\pgfsetlinewidth{0.301125pt}%
\definecolor{currentstroke}{rgb}{0.500000,0.500000,0.500000}%
\pgfsetstrokecolor{currentstroke}%
\pgfsetstrokeopacity{0.300000}%
\pgfsetdash{}{0pt}%
\pgfpathmoveto{\pgfqpoint{0.938916in}{0.778611in}}%
\pgfusepath{stroke}%
\end{pgfscope}%
\begin{pgfscope}%
\pgfpathrectangle{\pgfqpoint{0.647939in}{0.492442in}}{\pgfqpoint{3.079299in}{3.079299in}}%
\pgfusepath{clip}%
\pgfsetroundcap%
\pgfsetroundjoin%
\definecolor{currentfill}{rgb}{0.500000,0.500000,0.500000}%
\pgfsetfillcolor{currentfill}%
\pgfsetfillopacity{0.300000}%
\pgfsetlinewidth{0.301125pt}%
\definecolor{currentstroke}{rgb}{0.500000,0.500000,0.500000}%
\pgfsetstrokecolor{currentstroke}%
\pgfsetstrokeopacity{0.300000}%
\pgfsetdash{}{0pt}%
\pgfpathmoveto{\pgfqpoint{0.000000in}{0.000000in}}%
\pgfpathlineto{\pgfqpoint{0.000000in}{0.000000in}}%
\pgfpathclose%
\pgfusepath{stroke,fill}%
\end{pgfscope}%
\begin{pgfscope}%
\pgfpathrectangle{\pgfqpoint{0.647939in}{0.492442in}}{\pgfqpoint{3.079299in}{3.079299in}}%
\pgfusepath{clip}%
\pgfsetroundcap%
\pgfsetroundjoin%
\pgfsetlinewidth{0.301125pt}%
\definecolor{currentstroke}{rgb}{0.500000,0.500000,0.500000}%
\pgfsetstrokecolor{currentstroke}%
\pgfsetstrokeopacity{0.300000}%
\pgfsetdash{}{0pt}%
\pgfpathmoveto{\pgfqpoint{0.938302in}{0.710686in}}%
\pgfusepath{stroke}%
\end{pgfscope}%
\begin{pgfscope}%
\pgfpathrectangle{\pgfqpoint{0.647939in}{0.492442in}}{\pgfqpoint{3.079299in}{3.079299in}}%
\pgfusepath{clip}%
\pgfsetroundcap%
\pgfsetroundjoin%
\definecolor{currentfill}{rgb}{0.500000,0.500000,0.500000}%
\pgfsetfillcolor{currentfill}%
\pgfsetfillopacity{0.300000}%
\pgfsetlinewidth{0.301125pt}%
\definecolor{currentstroke}{rgb}{0.500000,0.500000,0.500000}%
\pgfsetstrokecolor{currentstroke}%
\pgfsetstrokeopacity{0.300000}%
\pgfsetdash{}{0pt}%
\pgfpathmoveto{\pgfqpoint{0.000000in}{0.000000in}}%
\pgfpathlineto{\pgfqpoint{0.000000in}{0.000000in}}%
\pgfpathclose%
\pgfusepath{stroke,fill}%
\end{pgfscope}%
\begin{pgfscope}%
\pgfpathrectangle{\pgfqpoint{0.647939in}{0.492442in}}{\pgfqpoint{3.079299in}{3.079299in}}%
\pgfusepath{clip}%
\pgfsetroundcap%
\pgfsetroundjoin%
\pgfsetlinewidth{0.301125pt}%
\definecolor{currentstroke}{rgb}{0.500000,0.500000,0.500000}%
\pgfsetstrokecolor{currentstroke}%
\pgfsetstrokeopacity{0.300000}%
\pgfsetdash{}{0pt}%
\pgfpathmoveto{\pgfqpoint{0.873728in}{0.618622in}}%
\pgfusepath{stroke}%
\end{pgfscope}%
\begin{pgfscope}%
\pgfpathrectangle{\pgfqpoint{0.647939in}{0.492442in}}{\pgfqpoint{3.079299in}{3.079299in}}%
\pgfusepath{clip}%
\pgfsetroundcap%
\pgfsetroundjoin%
\definecolor{currentfill}{rgb}{0.500000,0.500000,0.500000}%
\pgfsetfillcolor{currentfill}%
\pgfsetfillopacity{0.300000}%
\pgfsetlinewidth{0.301125pt}%
\definecolor{currentstroke}{rgb}{0.500000,0.500000,0.500000}%
\pgfsetstrokecolor{currentstroke}%
\pgfsetstrokeopacity{0.300000}%
\pgfsetdash{}{0pt}%
\pgfpathmoveto{\pgfqpoint{0.000000in}{0.000000in}}%
\pgfpathlineto{\pgfqpoint{0.000000in}{0.000000in}}%
\pgfpathclose%
\pgfusepath{stroke,fill}%
\end{pgfscope}%
\begin{pgfscope}%
\pgfpathrectangle{\pgfqpoint{0.647939in}{0.492442in}}{\pgfqpoint{3.079299in}{3.079299in}}%
\pgfusepath{clip}%
\pgfsetroundcap%
\pgfsetroundjoin%
\pgfsetlinewidth{0.301125pt}%
\definecolor{currentstroke}{rgb}{0.500000,0.500000,0.500000}%
\pgfsetstrokecolor{currentstroke}%
\pgfsetstrokeopacity{0.300000}%
\pgfsetdash{}{0pt}%
\pgfpathmoveto{\pgfqpoint{3.586291in}{2.096909in}}%
\pgfusepath{stroke}%
\end{pgfscope}%
\begin{pgfscope}%
\pgfpathrectangle{\pgfqpoint{0.647939in}{0.492442in}}{\pgfqpoint{3.079299in}{3.079299in}}%
\pgfusepath{clip}%
\pgfsetroundcap%
\pgfsetroundjoin%
\definecolor{currentfill}{rgb}{0.500000,0.500000,0.500000}%
\pgfsetfillcolor{currentfill}%
\pgfsetfillopacity{0.300000}%
\pgfsetlinewidth{0.301125pt}%
\definecolor{currentstroke}{rgb}{0.500000,0.500000,0.500000}%
\pgfsetstrokecolor{currentstroke}%
\pgfsetstrokeopacity{0.300000}%
\pgfsetdash{}{0pt}%
\pgfpathmoveto{\pgfqpoint{0.000000in}{0.000000in}}%
\pgfpathlineto{\pgfqpoint{0.000000in}{0.000000in}}%
\pgfpathclose%
\pgfusepath{stroke,fill}%
\end{pgfscope}%
\begin{pgfscope}%
\pgfpathrectangle{\pgfqpoint{0.647939in}{0.492442in}}{\pgfqpoint{3.079299in}{3.079299in}}%
\pgfusepath{clip}%
\pgfsetroundcap%
\pgfsetroundjoin%
\pgfsetlinewidth{0.301125pt}%
\definecolor{currentstroke}{rgb}{0.500000,0.500000,0.500000}%
\pgfsetstrokecolor{currentstroke}%
\pgfsetstrokeopacity{0.300000}%
\pgfsetdash{}{0pt}%
\pgfpathmoveto{\pgfqpoint{1.603135in}{0.645371in}}%
\pgfusepath{stroke}%
\end{pgfscope}%
\begin{pgfscope}%
\pgfpathrectangle{\pgfqpoint{0.647939in}{0.492442in}}{\pgfqpoint{3.079299in}{3.079299in}}%
\pgfusepath{clip}%
\pgfsetroundcap%
\pgfsetroundjoin%
\definecolor{currentfill}{rgb}{0.500000,0.500000,0.500000}%
\pgfsetfillcolor{currentfill}%
\pgfsetfillopacity{0.300000}%
\pgfsetlinewidth{0.301125pt}%
\definecolor{currentstroke}{rgb}{0.500000,0.500000,0.500000}%
\pgfsetstrokecolor{currentstroke}%
\pgfsetstrokeopacity{0.300000}%
\pgfsetdash{}{0pt}%
\pgfpathmoveto{\pgfqpoint{0.000000in}{0.000000in}}%
\pgfpathlineto{\pgfqpoint{0.000000in}{0.000000in}}%
\pgfpathclose%
\pgfusepath{stroke,fill}%
\end{pgfscope}%
\begin{pgfscope}%
\pgfpathrectangle{\pgfqpoint{0.647939in}{0.492442in}}{\pgfqpoint{3.079299in}{3.079299in}}%
\pgfusepath{clip}%
\pgfsetroundcap%
\pgfsetroundjoin%
\pgfsetlinewidth{0.301125pt}%
\definecolor{currentstroke}{rgb}{0.500000,0.500000,0.500000}%
\pgfsetstrokecolor{currentstroke}%
\pgfsetstrokeopacity{0.300000}%
\pgfsetdash{}{0pt}%
\pgfpathmoveto{\pgfqpoint{2.288495in}{0.642655in}}%
\pgfusepath{stroke}%
\end{pgfscope}%
\begin{pgfscope}%
\pgfpathrectangle{\pgfqpoint{0.647939in}{0.492442in}}{\pgfqpoint{3.079299in}{3.079299in}}%
\pgfusepath{clip}%
\pgfsetroundcap%
\pgfsetroundjoin%
\definecolor{currentfill}{rgb}{0.500000,0.500000,0.500000}%
\pgfsetfillcolor{currentfill}%
\pgfsetfillopacity{0.300000}%
\pgfsetlinewidth{0.301125pt}%
\definecolor{currentstroke}{rgb}{0.500000,0.500000,0.500000}%
\pgfsetstrokecolor{currentstroke}%
\pgfsetstrokeopacity{0.300000}%
\pgfsetdash{}{0pt}%
\pgfpathmoveto{\pgfqpoint{0.000000in}{0.000000in}}%
\pgfpathlineto{\pgfqpoint{0.000000in}{0.000000in}}%
\pgfpathclose%
\pgfusepath{stroke,fill}%
\end{pgfscope}%
\begin{pgfscope}%
\pgfpathrectangle{\pgfqpoint{0.647939in}{0.492442in}}{\pgfqpoint{3.079299in}{3.079299in}}%
\pgfusepath{clip}%
\pgfsetroundcap%
\pgfsetroundjoin%
\pgfsetlinewidth{0.301125pt}%
\definecolor{currentstroke}{rgb}{0.500000,0.500000,0.500000}%
\pgfsetstrokecolor{currentstroke}%
\pgfsetstrokeopacity{0.300000}%
\pgfsetdash{}{0pt}%
\pgfpathmoveto{\pgfqpoint{0.815421in}{3.225542in}}%
\pgfusepath{stroke}%
\end{pgfscope}%
\begin{pgfscope}%
\pgfpathrectangle{\pgfqpoint{0.647939in}{0.492442in}}{\pgfqpoint{3.079299in}{3.079299in}}%
\pgfusepath{clip}%
\pgfsetroundcap%
\pgfsetroundjoin%
\definecolor{currentfill}{rgb}{0.500000,0.500000,0.500000}%
\pgfsetfillcolor{currentfill}%
\pgfsetfillopacity{0.300000}%
\pgfsetlinewidth{0.301125pt}%
\definecolor{currentstroke}{rgb}{0.500000,0.500000,0.500000}%
\pgfsetstrokecolor{currentstroke}%
\pgfsetstrokeopacity{0.300000}%
\pgfsetdash{}{0pt}%
\pgfpathmoveto{\pgfqpoint{0.000000in}{0.000000in}}%
\pgfpathlineto{\pgfqpoint{0.000000in}{0.000000in}}%
\pgfpathclose%
\pgfusepath{stroke,fill}%
\end{pgfscope}%
\begin{pgfscope}%
\pgfpathrectangle{\pgfqpoint{0.647939in}{0.492442in}}{\pgfqpoint{3.079299in}{3.079299in}}%
\pgfusepath{clip}%
\pgfsetroundcap%
\pgfsetroundjoin%
\pgfsetlinewidth{0.301125pt}%
\definecolor{currentstroke}{rgb}{0.500000,0.500000,0.500000}%
\pgfsetstrokecolor{currentstroke}%
\pgfsetstrokeopacity{0.300000}%
\pgfsetdash{}{0pt}%
\pgfpathmoveto{\pgfqpoint{1.690929in}{3.241324in}}%
\pgfusepath{stroke}%
\end{pgfscope}%
\begin{pgfscope}%
\pgfpathrectangle{\pgfqpoint{0.647939in}{0.492442in}}{\pgfqpoint{3.079299in}{3.079299in}}%
\pgfusepath{clip}%
\pgfsetroundcap%
\pgfsetroundjoin%
\definecolor{currentfill}{rgb}{0.500000,0.500000,0.500000}%
\pgfsetfillcolor{currentfill}%
\pgfsetfillopacity{0.300000}%
\pgfsetlinewidth{0.301125pt}%
\definecolor{currentstroke}{rgb}{0.500000,0.500000,0.500000}%
\pgfsetstrokecolor{currentstroke}%
\pgfsetstrokeopacity{0.300000}%
\pgfsetdash{}{0pt}%
\pgfpathmoveto{\pgfqpoint{0.000000in}{0.000000in}}%
\pgfpathlineto{\pgfqpoint{0.000000in}{0.000000in}}%
\pgfpathclose%
\pgfusepath{stroke,fill}%
\end{pgfscope}%
\begin{pgfscope}%
\pgfpathrectangle{\pgfqpoint{0.647939in}{0.492442in}}{\pgfqpoint{3.079299in}{3.079299in}}%
\pgfusepath{clip}%
\pgfsetroundcap%
\pgfsetroundjoin%
\pgfsetlinewidth{0.301125pt}%
\definecolor{currentstroke}{rgb}{0.500000,0.500000,0.500000}%
\pgfsetstrokecolor{currentstroke}%
\pgfsetstrokeopacity{0.300000}%
\pgfsetdash{}{0pt}%
\pgfpathmoveto{\pgfqpoint{3.202379in}{2.354777in}}%
\pgfusepath{stroke}%
\end{pgfscope}%
\begin{pgfscope}%
\pgfpathrectangle{\pgfqpoint{0.647939in}{0.492442in}}{\pgfqpoint{3.079299in}{3.079299in}}%
\pgfusepath{clip}%
\pgfsetroundcap%
\pgfsetroundjoin%
\definecolor{currentfill}{rgb}{0.500000,0.500000,0.500000}%
\pgfsetfillcolor{currentfill}%
\pgfsetfillopacity{0.300000}%
\pgfsetlinewidth{0.301125pt}%
\definecolor{currentstroke}{rgb}{0.500000,0.500000,0.500000}%
\pgfsetstrokecolor{currentstroke}%
\pgfsetstrokeopacity{0.300000}%
\pgfsetdash{}{0pt}%
\pgfpathmoveto{\pgfqpoint{0.000000in}{0.000000in}}%
\pgfpathlineto{\pgfqpoint{0.000000in}{0.000000in}}%
\pgfpathclose%
\pgfusepath{stroke,fill}%
\end{pgfscope}%
\begin{pgfscope}%
\pgfpathrectangle{\pgfqpoint{0.647939in}{0.492442in}}{\pgfqpoint{3.079299in}{3.079299in}}%
\pgfusepath{clip}%
\pgfsetroundcap%
\pgfsetroundjoin%
\pgfsetlinewidth{0.301125pt}%
\definecolor{currentstroke}{rgb}{0.500000,0.500000,0.500000}%
\pgfsetstrokecolor{currentstroke}%
\pgfsetstrokeopacity{0.300000}%
\pgfsetdash{}{0pt}%
\pgfpathmoveto{\pgfqpoint{2.088809in}{3.366522in}}%
\pgfusepath{stroke}%
\end{pgfscope}%
\begin{pgfscope}%
\pgfpathrectangle{\pgfqpoint{0.647939in}{0.492442in}}{\pgfqpoint{3.079299in}{3.079299in}}%
\pgfusepath{clip}%
\pgfsetroundcap%
\pgfsetroundjoin%
\definecolor{currentfill}{rgb}{0.500000,0.500000,0.500000}%
\pgfsetfillcolor{currentfill}%
\pgfsetfillopacity{0.300000}%
\pgfsetlinewidth{0.301125pt}%
\definecolor{currentstroke}{rgb}{0.500000,0.500000,0.500000}%
\pgfsetstrokecolor{currentstroke}%
\pgfsetstrokeopacity{0.300000}%
\pgfsetdash{}{0pt}%
\pgfpathmoveto{\pgfqpoint{0.000000in}{0.000000in}}%
\pgfpathlineto{\pgfqpoint{0.000000in}{0.000000in}}%
\pgfpathclose%
\pgfusepath{stroke,fill}%
\end{pgfscope}%
\begin{pgfscope}%
\pgfpathrectangle{\pgfqpoint{0.647939in}{0.492442in}}{\pgfqpoint{3.079299in}{3.079299in}}%
\pgfusepath{clip}%
\pgfsetroundcap%
\pgfsetroundjoin%
\pgfsetlinewidth{0.301125pt}%
\definecolor{currentstroke}{rgb}{0.500000,0.500000,0.500000}%
\pgfsetstrokecolor{currentstroke}%
\pgfsetstrokeopacity{0.300000}%
\pgfsetdash{}{0pt}%
\pgfpathmoveto{\pgfqpoint{2.841604in}{2.172922in}}%
\pgfusepath{stroke}%
\end{pgfscope}%
\begin{pgfscope}%
\pgfpathrectangle{\pgfqpoint{0.647939in}{0.492442in}}{\pgfqpoint{3.079299in}{3.079299in}}%
\pgfusepath{clip}%
\pgfsetroundcap%
\pgfsetroundjoin%
\definecolor{currentfill}{rgb}{0.500000,0.500000,0.500000}%
\pgfsetfillcolor{currentfill}%
\pgfsetfillopacity{0.300000}%
\pgfsetlinewidth{0.301125pt}%
\definecolor{currentstroke}{rgb}{0.500000,0.500000,0.500000}%
\pgfsetstrokecolor{currentstroke}%
\pgfsetstrokeopacity{0.300000}%
\pgfsetdash{}{0pt}%
\pgfpathmoveto{\pgfqpoint{0.000000in}{0.000000in}}%
\pgfpathlineto{\pgfqpoint{0.000000in}{0.000000in}}%
\pgfpathclose%
\pgfusepath{stroke,fill}%
\end{pgfscope}%
\begin{pgfscope}%
\pgfpathrectangle{\pgfqpoint{0.647939in}{0.492442in}}{\pgfqpoint{3.079299in}{3.079299in}}%
\pgfusepath{clip}%
\pgfsetroundcap%
\pgfsetroundjoin%
\pgfsetlinewidth{0.301125pt}%
\definecolor{currentstroke}{rgb}{0.500000,0.500000,0.500000}%
\pgfsetstrokecolor{currentstroke}%
\pgfsetstrokeopacity{0.300000}%
\pgfsetdash{}{0pt}%
\pgfpathmoveto{\pgfqpoint{3.290489in}{2.616515in}}%
\pgfusepath{stroke}%
\end{pgfscope}%
\begin{pgfscope}%
\pgfpathrectangle{\pgfqpoint{0.647939in}{0.492442in}}{\pgfqpoint{3.079299in}{3.079299in}}%
\pgfusepath{clip}%
\pgfsetroundcap%
\pgfsetroundjoin%
\definecolor{currentfill}{rgb}{0.500000,0.500000,0.500000}%
\pgfsetfillcolor{currentfill}%
\pgfsetfillopacity{0.300000}%
\pgfsetlinewidth{0.301125pt}%
\definecolor{currentstroke}{rgb}{0.500000,0.500000,0.500000}%
\pgfsetstrokecolor{currentstroke}%
\pgfsetstrokeopacity{0.300000}%
\pgfsetdash{}{0pt}%
\pgfpathmoveto{\pgfqpoint{0.000000in}{0.000000in}}%
\pgfpathlineto{\pgfqpoint{0.000000in}{0.000000in}}%
\pgfpathclose%
\pgfusepath{stroke,fill}%
\end{pgfscope}%
\begin{pgfscope}%
\pgfpathrectangle{\pgfqpoint{0.647939in}{0.492442in}}{\pgfqpoint{3.079299in}{3.079299in}}%
\pgfusepath{clip}%
\pgfsetroundcap%
\pgfsetroundjoin%
\pgfsetlinewidth{0.301125pt}%
\definecolor{currentstroke}{rgb}{0.500000,0.500000,0.500000}%
\pgfsetstrokecolor{currentstroke}%
\pgfsetstrokeopacity{0.300000}%
\pgfsetdash{}{0pt}%
\pgfpathmoveto{\pgfqpoint{1.958214in}{3.000605in}}%
\pgfusepath{stroke}%
\end{pgfscope}%
\begin{pgfscope}%
\pgfpathrectangle{\pgfqpoint{0.647939in}{0.492442in}}{\pgfqpoint{3.079299in}{3.079299in}}%
\pgfusepath{clip}%
\pgfsetroundcap%
\pgfsetroundjoin%
\definecolor{currentfill}{rgb}{0.500000,0.500000,0.500000}%
\pgfsetfillcolor{currentfill}%
\pgfsetfillopacity{0.300000}%
\pgfsetlinewidth{0.301125pt}%
\definecolor{currentstroke}{rgb}{0.500000,0.500000,0.500000}%
\pgfsetstrokecolor{currentstroke}%
\pgfsetstrokeopacity{0.300000}%
\pgfsetdash{}{0pt}%
\pgfpathmoveto{\pgfqpoint{0.000000in}{0.000000in}}%
\pgfpathlineto{\pgfqpoint{0.000000in}{0.000000in}}%
\pgfpathclose%
\pgfusepath{stroke,fill}%
\end{pgfscope}%
\begin{pgfscope}%
\pgfpathrectangle{\pgfqpoint{0.647939in}{0.492442in}}{\pgfqpoint{3.079299in}{3.079299in}}%
\pgfusepath{clip}%
\pgfsetroundcap%
\pgfsetroundjoin%
\pgfsetlinewidth{0.301125pt}%
\definecolor{currentstroke}{rgb}{0.500000,0.500000,0.500000}%
\pgfsetstrokecolor{currentstroke}%
\pgfsetstrokeopacity{0.300000}%
\pgfsetdash{}{0pt}%
\pgfpathmoveto{\pgfqpoint{1.435979in}{2.427602in}}%
\pgfusepath{stroke}%
\end{pgfscope}%
\begin{pgfscope}%
\pgfpathrectangle{\pgfqpoint{0.647939in}{0.492442in}}{\pgfqpoint{3.079299in}{3.079299in}}%
\pgfusepath{clip}%
\pgfsetroundcap%
\pgfsetroundjoin%
\definecolor{currentfill}{rgb}{0.500000,0.500000,0.500000}%
\pgfsetfillcolor{currentfill}%
\pgfsetfillopacity{0.300000}%
\pgfsetlinewidth{0.301125pt}%
\definecolor{currentstroke}{rgb}{0.500000,0.500000,0.500000}%
\pgfsetstrokecolor{currentstroke}%
\pgfsetstrokeopacity{0.300000}%
\pgfsetdash{}{0pt}%
\pgfpathmoveto{\pgfqpoint{0.000000in}{0.000000in}}%
\pgfpathlineto{\pgfqpoint{0.000000in}{0.000000in}}%
\pgfpathclose%
\pgfusepath{stroke,fill}%
\end{pgfscope}%
\begin{pgfscope}%
\pgfpathrectangle{\pgfqpoint{0.647939in}{0.492442in}}{\pgfqpoint{3.079299in}{3.079299in}}%
\pgfusepath{clip}%
\pgfsetroundcap%
\pgfsetroundjoin%
\pgfsetlinewidth{0.301125pt}%
\definecolor{currentstroke}{rgb}{0.500000,0.500000,0.500000}%
\pgfsetstrokecolor{currentstroke}%
\pgfsetstrokeopacity{0.300000}%
\pgfsetdash{}{0pt}%
\pgfpathmoveto{\pgfqpoint{2.874999in}{2.278065in}}%
\pgfusepath{stroke}%
\end{pgfscope}%
\begin{pgfscope}%
\pgfpathrectangle{\pgfqpoint{0.647939in}{0.492442in}}{\pgfqpoint{3.079299in}{3.079299in}}%
\pgfusepath{clip}%
\pgfsetroundcap%
\pgfsetroundjoin%
\definecolor{currentfill}{rgb}{0.500000,0.500000,0.500000}%
\pgfsetfillcolor{currentfill}%
\pgfsetfillopacity{0.300000}%
\pgfsetlinewidth{0.301125pt}%
\definecolor{currentstroke}{rgb}{0.500000,0.500000,0.500000}%
\pgfsetstrokecolor{currentstroke}%
\pgfsetstrokeopacity{0.300000}%
\pgfsetdash{}{0pt}%
\pgfpathmoveto{\pgfqpoint{0.000000in}{0.000000in}}%
\pgfpathlineto{\pgfqpoint{0.000000in}{0.000000in}}%
\pgfpathclose%
\pgfusepath{stroke,fill}%
\end{pgfscope}%
\begin{pgfscope}%
\pgfpathrectangle{\pgfqpoint{0.647939in}{0.492442in}}{\pgfqpoint{3.079299in}{3.079299in}}%
\pgfusepath{clip}%
\pgfsetroundcap%
\pgfsetroundjoin%
\pgfsetlinewidth{0.301125pt}%
\definecolor{currentstroke}{rgb}{0.500000,0.500000,0.500000}%
\pgfsetstrokecolor{currentstroke}%
\pgfsetstrokeopacity{0.300000}%
\pgfsetdash{}{0pt}%
\pgfpathmoveto{\pgfqpoint{1.940334in}{2.879572in}}%
\pgfusepath{stroke}%
\end{pgfscope}%
\begin{pgfscope}%
\pgfpathrectangle{\pgfqpoint{0.647939in}{0.492442in}}{\pgfqpoint{3.079299in}{3.079299in}}%
\pgfusepath{clip}%
\pgfsetroundcap%
\pgfsetroundjoin%
\definecolor{currentfill}{rgb}{0.500000,0.500000,0.500000}%
\pgfsetfillcolor{currentfill}%
\pgfsetfillopacity{0.300000}%
\pgfsetlinewidth{0.301125pt}%
\definecolor{currentstroke}{rgb}{0.500000,0.500000,0.500000}%
\pgfsetstrokecolor{currentstroke}%
\pgfsetstrokeopacity{0.300000}%
\pgfsetdash{}{0pt}%
\pgfpathmoveto{\pgfqpoint{0.000000in}{0.000000in}}%
\pgfpathlineto{\pgfqpoint{0.000000in}{0.000000in}}%
\pgfpathclose%
\pgfusepath{stroke,fill}%
\end{pgfscope}%
\begin{pgfscope}%
\pgfpathrectangle{\pgfqpoint{0.647939in}{0.492442in}}{\pgfqpoint{3.079299in}{3.079299in}}%
\pgfusepath{clip}%
\pgfsetroundcap%
\pgfsetroundjoin%
\pgfsetlinewidth{0.301125pt}%
\definecolor{currentstroke}{rgb}{0.500000,0.500000,0.500000}%
\pgfsetstrokecolor{currentstroke}%
\pgfsetstrokeopacity{0.300000}%
\pgfsetdash{}{0pt}%
\pgfpathmoveto{\pgfqpoint{1.122016in}{1.584898in}}%
\pgfusepath{stroke}%
\end{pgfscope}%
\begin{pgfscope}%
\pgfpathrectangle{\pgfqpoint{0.647939in}{0.492442in}}{\pgfqpoint{3.079299in}{3.079299in}}%
\pgfusepath{clip}%
\pgfsetroundcap%
\pgfsetroundjoin%
\definecolor{currentfill}{rgb}{0.500000,0.500000,0.500000}%
\pgfsetfillcolor{currentfill}%
\pgfsetfillopacity{0.300000}%
\pgfsetlinewidth{0.301125pt}%
\definecolor{currentstroke}{rgb}{0.500000,0.500000,0.500000}%
\pgfsetstrokecolor{currentstroke}%
\pgfsetstrokeopacity{0.300000}%
\pgfsetdash{}{0pt}%
\pgfpathmoveto{\pgfqpoint{0.000000in}{0.000000in}}%
\pgfpathlineto{\pgfqpoint{0.000000in}{0.000000in}}%
\pgfpathclose%
\pgfusepath{stroke,fill}%
\end{pgfscope}%
\begin{pgfscope}%
\pgfpathrectangle{\pgfqpoint{0.647939in}{0.492442in}}{\pgfqpoint{3.079299in}{3.079299in}}%
\pgfusepath{clip}%
\pgfsetroundcap%
\pgfsetroundjoin%
\pgfsetlinewidth{0.301125pt}%
\definecolor{currentstroke}{rgb}{0.500000,0.500000,0.500000}%
\pgfsetstrokecolor{currentstroke}%
\pgfsetstrokeopacity{0.300000}%
\pgfsetdash{}{0pt}%
\pgfpathmoveto{\pgfqpoint{1.841399in}{2.280655in}}%
\pgfusepath{stroke}%
\end{pgfscope}%
\begin{pgfscope}%
\pgfpathrectangle{\pgfqpoint{0.647939in}{0.492442in}}{\pgfqpoint{3.079299in}{3.079299in}}%
\pgfusepath{clip}%
\pgfsetroundcap%
\pgfsetroundjoin%
\definecolor{currentfill}{rgb}{0.500000,0.500000,0.500000}%
\pgfsetfillcolor{currentfill}%
\pgfsetfillopacity{0.300000}%
\pgfsetlinewidth{0.301125pt}%
\definecolor{currentstroke}{rgb}{0.500000,0.500000,0.500000}%
\pgfsetstrokecolor{currentstroke}%
\pgfsetstrokeopacity{0.300000}%
\pgfsetdash{}{0pt}%
\pgfpathmoveto{\pgfqpoint{0.000000in}{0.000000in}}%
\pgfpathlineto{\pgfqpoint{0.000000in}{0.000000in}}%
\pgfpathclose%
\pgfusepath{stroke,fill}%
\end{pgfscope}%
\begin{pgfscope}%
\pgfpathrectangle{\pgfqpoint{0.647939in}{0.492442in}}{\pgfqpoint{3.079299in}{3.079299in}}%
\pgfusepath{clip}%
\pgfsetroundcap%
\pgfsetroundjoin%
\pgfsetlinewidth{0.301125pt}%
\definecolor{currentstroke}{rgb}{0.500000,0.500000,0.500000}%
\pgfsetstrokecolor{currentstroke}%
\pgfsetstrokeopacity{0.300000}%
\pgfsetdash{}{0pt}%
\pgfpathmoveto{\pgfqpoint{2.687655in}{2.081733in}}%
\pgfusepath{stroke}%
\end{pgfscope}%
\begin{pgfscope}%
\pgfpathrectangle{\pgfqpoint{0.647939in}{0.492442in}}{\pgfqpoint{3.079299in}{3.079299in}}%
\pgfusepath{clip}%
\pgfsetroundcap%
\pgfsetroundjoin%
\definecolor{currentfill}{rgb}{0.500000,0.500000,0.500000}%
\pgfsetfillcolor{currentfill}%
\pgfsetfillopacity{0.300000}%
\pgfsetlinewidth{0.301125pt}%
\definecolor{currentstroke}{rgb}{0.500000,0.500000,0.500000}%
\pgfsetstrokecolor{currentstroke}%
\pgfsetstrokeopacity{0.300000}%
\pgfsetdash{}{0pt}%
\pgfpathmoveto{\pgfqpoint{0.000000in}{0.000000in}}%
\pgfpathlineto{\pgfqpoint{0.000000in}{0.000000in}}%
\pgfpathclose%
\pgfusepath{stroke,fill}%
\end{pgfscope}%
\begin{pgfscope}%
\pgfpathrectangle{\pgfqpoint{0.647939in}{0.492442in}}{\pgfqpoint{3.079299in}{3.079299in}}%
\pgfusepath{clip}%
\pgfsetroundcap%
\pgfsetroundjoin%
\pgfsetlinewidth{0.301125pt}%
\definecolor{currentstroke}{rgb}{0.500000,0.500000,0.500000}%
\pgfsetstrokecolor{currentstroke}%
\pgfsetstrokeopacity{0.300000}%
\pgfsetdash{}{0pt}%
\pgfpathmoveto{\pgfqpoint{1.772380in}{2.060451in}}%
\pgfusepath{stroke}%
\end{pgfscope}%
\begin{pgfscope}%
\pgfpathrectangle{\pgfqpoint{0.647939in}{0.492442in}}{\pgfqpoint{3.079299in}{3.079299in}}%
\pgfusepath{clip}%
\pgfsetroundcap%
\pgfsetroundjoin%
\definecolor{currentfill}{rgb}{0.500000,0.500000,0.500000}%
\pgfsetfillcolor{currentfill}%
\pgfsetfillopacity{0.300000}%
\pgfsetlinewidth{0.301125pt}%
\definecolor{currentstroke}{rgb}{0.500000,0.500000,0.500000}%
\pgfsetstrokecolor{currentstroke}%
\pgfsetstrokeopacity{0.300000}%
\pgfsetdash{}{0pt}%
\pgfpathmoveto{\pgfqpoint{0.000000in}{0.000000in}}%
\pgfpathlineto{\pgfqpoint{0.000000in}{0.000000in}}%
\pgfpathclose%
\pgfusepath{stroke,fill}%
\end{pgfscope}%
\begin{pgfscope}%
\pgfpathrectangle{\pgfqpoint{0.647939in}{0.492442in}}{\pgfqpoint{3.079299in}{3.079299in}}%
\pgfusepath{clip}%
\pgfsetroundcap%
\pgfsetroundjoin%
\pgfsetlinewidth{0.301125pt}%
\definecolor{currentstroke}{rgb}{0.500000,0.500000,0.500000}%
\pgfsetstrokecolor{currentstroke}%
\pgfsetstrokeopacity{0.300000}%
\pgfsetdash{}{0pt}%
\pgfpathmoveto{\pgfqpoint{2.285169in}{2.550414in}}%
\pgfusepath{stroke}%
\end{pgfscope}%
\begin{pgfscope}%
\pgfpathrectangle{\pgfqpoint{0.647939in}{0.492442in}}{\pgfqpoint{3.079299in}{3.079299in}}%
\pgfusepath{clip}%
\pgfsetroundcap%
\pgfsetroundjoin%
\definecolor{currentfill}{rgb}{0.500000,0.500000,0.500000}%
\pgfsetfillcolor{currentfill}%
\pgfsetfillopacity{0.300000}%
\pgfsetlinewidth{0.301125pt}%
\definecolor{currentstroke}{rgb}{0.500000,0.500000,0.500000}%
\pgfsetstrokecolor{currentstroke}%
\pgfsetstrokeopacity{0.300000}%
\pgfsetdash{}{0pt}%
\pgfpathmoveto{\pgfqpoint{0.000000in}{0.000000in}}%
\pgfpathlineto{\pgfqpoint{0.000000in}{0.000000in}}%
\pgfpathclose%
\pgfusepath{stroke,fill}%
\end{pgfscope}%
\begin{pgfscope}%
\pgfpathrectangle{\pgfqpoint{0.647939in}{0.492442in}}{\pgfqpoint{3.079299in}{3.079299in}}%
\pgfusepath{clip}%
\pgfsetroundcap%
\pgfsetroundjoin%
\pgfsetlinewidth{0.301125pt}%
\definecolor{currentstroke}{rgb}{0.500000,0.500000,0.500000}%
\pgfsetstrokecolor{currentstroke}%
\pgfsetstrokeopacity{0.300000}%
\pgfsetdash{}{0pt}%
\pgfpathmoveto{\pgfqpoint{1.654964in}{2.527291in}}%
\pgfusepath{stroke}%
\end{pgfscope}%
\begin{pgfscope}%
\pgfpathrectangle{\pgfqpoint{0.647939in}{0.492442in}}{\pgfqpoint{3.079299in}{3.079299in}}%
\pgfusepath{clip}%
\pgfsetroundcap%
\pgfsetroundjoin%
\definecolor{currentfill}{rgb}{0.500000,0.500000,0.500000}%
\pgfsetfillcolor{currentfill}%
\pgfsetfillopacity{0.300000}%
\pgfsetlinewidth{0.301125pt}%
\definecolor{currentstroke}{rgb}{0.500000,0.500000,0.500000}%
\pgfsetstrokecolor{currentstroke}%
\pgfsetstrokeopacity{0.300000}%
\pgfsetdash{}{0pt}%
\pgfpathmoveto{\pgfqpoint{0.000000in}{0.000000in}}%
\pgfpathlineto{\pgfqpoint{0.000000in}{0.000000in}}%
\pgfpathclose%
\pgfusepath{stroke,fill}%
\end{pgfscope}%
\begin{pgfscope}%
\pgfpathrectangle{\pgfqpoint{0.647939in}{0.492442in}}{\pgfqpoint{3.079299in}{3.079299in}}%
\pgfusepath{clip}%
\pgfsetroundcap%
\pgfsetroundjoin%
\pgfsetlinewidth{0.301125pt}%
\definecolor{currentstroke}{rgb}{0.500000,0.500000,0.500000}%
\pgfsetstrokecolor{currentstroke}%
\pgfsetstrokeopacity{0.300000}%
\pgfsetdash{}{0pt}%
\pgfpathmoveto{\pgfqpoint{1.925168in}{2.355947in}}%
\pgfusepath{stroke}%
\end{pgfscope}%
\begin{pgfscope}%
\pgfpathrectangle{\pgfqpoint{0.647939in}{0.492442in}}{\pgfqpoint{3.079299in}{3.079299in}}%
\pgfusepath{clip}%
\pgfsetroundcap%
\pgfsetroundjoin%
\definecolor{currentfill}{rgb}{0.500000,0.500000,0.500000}%
\pgfsetfillcolor{currentfill}%
\pgfsetfillopacity{0.300000}%
\pgfsetlinewidth{0.301125pt}%
\definecolor{currentstroke}{rgb}{0.500000,0.500000,0.500000}%
\pgfsetstrokecolor{currentstroke}%
\pgfsetstrokeopacity{0.300000}%
\pgfsetdash{}{0pt}%
\pgfpathmoveto{\pgfqpoint{0.000000in}{0.000000in}}%
\pgfpathlineto{\pgfqpoint{0.000000in}{0.000000in}}%
\pgfpathclose%
\pgfusepath{stroke,fill}%
\end{pgfscope}%
\begin{pgfscope}%
\pgfpathrectangle{\pgfqpoint{0.647939in}{0.492442in}}{\pgfqpoint{3.079299in}{3.079299in}}%
\pgfusepath{clip}%
\pgfsetroundcap%
\pgfsetroundjoin%
\pgfsetlinewidth{0.301125pt}%
\definecolor{currentstroke}{rgb}{0.500000,0.500000,0.500000}%
\pgfsetstrokecolor{currentstroke}%
\pgfsetstrokeopacity{0.300000}%
\pgfsetdash{}{0pt}%
\pgfpathmoveto{\pgfqpoint{2.607831in}{2.308098in}}%
\pgfusepath{stroke}%
\end{pgfscope}%
\begin{pgfscope}%
\pgfpathrectangle{\pgfqpoint{0.647939in}{0.492442in}}{\pgfqpoint{3.079299in}{3.079299in}}%
\pgfusepath{clip}%
\pgfsetroundcap%
\pgfsetroundjoin%
\definecolor{currentfill}{rgb}{0.500000,0.500000,0.500000}%
\pgfsetfillcolor{currentfill}%
\pgfsetfillopacity{0.300000}%
\pgfsetlinewidth{0.301125pt}%
\definecolor{currentstroke}{rgb}{0.500000,0.500000,0.500000}%
\pgfsetstrokecolor{currentstroke}%
\pgfsetstrokeopacity{0.300000}%
\pgfsetdash{}{0pt}%
\pgfpathmoveto{\pgfqpoint{0.000000in}{0.000000in}}%
\pgfpathlineto{\pgfqpoint{0.000000in}{0.000000in}}%
\pgfpathclose%
\pgfusepath{stroke,fill}%
\end{pgfscope}%
\begin{pgfscope}%
\pgfpathrectangle{\pgfqpoint{0.647939in}{0.492442in}}{\pgfqpoint{3.079299in}{3.079299in}}%
\pgfusepath{clip}%
\pgfsetroundcap%
\pgfsetroundjoin%
\pgfsetlinewidth{0.301125pt}%
\definecolor{currentstroke}{rgb}{0.500000,0.500000,0.500000}%
\pgfsetstrokecolor{currentstroke}%
\pgfsetstrokeopacity{0.300000}%
\pgfsetdash{}{0pt}%
\pgfpathmoveto{\pgfqpoint{2.208453in}{2.492660in}}%
\pgfusepath{stroke}%
\end{pgfscope}%
\begin{pgfscope}%
\pgfpathrectangle{\pgfqpoint{0.647939in}{0.492442in}}{\pgfqpoint{3.079299in}{3.079299in}}%
\pgfusepath{clip}%
\pgfsetroundcap%
\pgfsetroundjoin%
\definecolor{currentfill}{rgb}{0.500000,0.500000,0.500000}%
\pgfsetfillcolor{currentfill}%
\pgfsetfillopacity{0.300000}%
\pgfsetlinewidth{0.301125pt}%
\definecolor{currentstroke}{rgb}{0.500000,0.500000,0.500000}%
\pgfsetstrokecolor{currentstroke}%
\pgfsetstrokeopacity{0.300000}%
\pgfsetdash{}{0pt}%
\pgfpathmoveto{\pgfqpoint{0.000000in}{0.000000in}}%
\pgfpathlineto{\pgfqpoint{0.000000in}{0.000000in}}%
\pgfpathclose%
\pgfusepath{stroke,fill}%
\end{pgfscope}%
\begin{pgfscope}%
\pgfpathrectangle{\pgfqpoint{0.647939in}{0.492442in}}{\pgfqpoint{3.079299in}{3.079299in}}%
\pgfusepath{clip}%
\pgfsetroundcap%
\pgfsetroundjoin%
\pgfsetlinewidth{0.301125pt}%
\definecolor{currentstroke}{rgb}{0.500000,0.500000,0.500000}%
\pgfsetstrokecolor{currentstroke}%
\pgfsetstrokeopacity{0.300000}%
\pgfsetdash{}{0pt}%
\pgfpathmoveto{\pgfqpoint{2.513280in}{1.980729in}}%
\pgfusepath{stroke}%
\end{pgfscope}%
\begin{pgfscope}%
\pgfpathrectangle{\pgfqpoint{0.647939in}{0.492442in}}{\pgfqpoint{3.079299in}{3.079299in}}%
\pgfusepath{clip}%
\pgfsetroundcap%
\pgfsetroundjoin%
\definecolor{currentfill}{rgb}{0.500000,0.500000,0.500000}%
\pgfsetfillcolor{currentfill}%
\pgfsetfillopacity{0.300000}%
\pgfsetlinewidth{0.301125pt}%
\definecolor{currentstroke}{rgb}{0.500000,0.500000,0.500000}%
\pgfsetstrokecolor{currentstroke}%
\pgfsetstrokeopacity{0.300000}%
\pgfsetdash{}{0pt}%
\pgfpathmoveto{\pgfqpoint{0.000000in}{0.000000in}}%
\pgfpathlineto{\pgfqpoint{0.000000in}{0.000000in}}%
\pgfpathclose%
\pgfusepath{stroke,fill}%
\end{pgfscope}%
\begin{pgfscope}%
\pgfpathrectangle{\pgfqpoint{0.647939in}{0.492442in}}{\pgfqpoint{3.079299in}{3.079299in}}%
\pgfusepath{clip}%
\pgfsetbuttcap%
\pgfsetroundjoin%
\pgfsetlinewidth{0.301125pt}%
\definecolor{currentstroke}{rgb}{0.500000,0.500000,0.500000}%
\pgfsetstrokecolor{currentstroke}%
\pgfsetstrokeopacity{0.300000}%
\pgfsetdash{}{0pt}%
\pgfpathmoveto{\pgfqpoint{0.647939in}{0.492442in}}%
\pgfpathlineto{\pgfqpoint{0.647939in}{0.492442in}}%
\pgfpathlineto{\pgfqpoint{0.715074in}{0.505599in}}%
\pgfpathlineto{\pgfqpoint{0.781523in}{0.521835in}}%
\pgfpathlineto{\pgfqpoint{0.847012in}{0.541561in}}%
\pgfpathlineto{\pgfqpoint{0.911192in}{0.565166in}}%
\pgfpathlineto{\pgfqpoint{0.973646in}{0.592992in}}%
\pgfpathlineto{\pgfqpoint{1.033900in}{0.625280in}}%
\pgfpathlineto{\pgfqpoint{1.091462in}{0.662133in}}%
\pgfpathlineto{\pgfqpoint{1.145884in}{0.703485in}}%
\pgfpathlineto{\pgfqpoint{1.196840in}{0.749037in}}%
\pgfpathlineto{\pgfqpoint{1.244187in}{0.798307in}}%
\pgfpathlineto{\pgfqpoint{1.288012in}{0.850736in}}%
\pgfpathlineto{\pgfqpoint{1.328614in}{0.905717in}}%
\pgfpathlineto{\pgfqpoint{1.366456in}{0.962651in}}%
\pgfpathlineto{\pgfqpoint{1.402086in}{1.020989in}}%
\pgfpathlineto{\pgfqpoint{1.436070in}{1.080274in}}%
\pgfpathlineto{\pgfqpoint{1.468955in}{1.140154in}}%
\pgfpathlineto{\pgfqpoint{1.501246in}{1.200341in}}%
\pgfpathlineto{\pgfqpoint{1.533392in}{1.260589in}}%
\pgfpathlineto{\pgfqpoint{1.565795in}{1.320683in}}%
\pgfpathlineto{\pgfqpoint{1.598816in}{1.380424in}}%
\pgfpathlineto{\pgfqpoint{1.632789in}{1.439617in}}%
\pgfpathlineto{\pgfqpoint{1.668042in}{1.498066in}}%
\pgfpathlineto{\pgfqpoint{1.704915in}{1.555560in}}%
\pgfpathlineto{\pgfqpoint{1.743762in}{1.611761in}}%
\pgfpathlineto{\pgfqpoint{1.784887in}{1.666138in}}%
\pgfpathlineto{\pgfqpoint{1.828722in}{1.718195in}}%
\pgfpathlineto{\pgfqpoint{1.875931in}{1.767130in}}%
\pgfpathlineto{\pgfqpoint{1.927020in}{1.810551in}}%
\pgfpathlineto{\pgfqpoint{1.927020in}{1.810551in}}%
\pgfpathlineto{\pgfqpoint{1.952357in}{1.829707in}}%
\pgfpathlineto{\pgfqpoint{1.952357in}{1.829707in}}%
\pgfpathlineto{\pgfqpoint{1.961918in}{1.836269in}}%
\pgfpathlineto{\pgfqpoint{1.965940in}{1.838824in}}%
\pgfpathlineto{\pgfqpoint{1.968723in}{1.840211in}}%
\pgfpathlineto{\pgfqpoint{1.969636in}{1.840458in}}%
\pgfpathlineto{\pgfqpoint{1.969887in}{1.840374in}}%
\pgfpathlineto{\pgfqpoint{1.970696in}{1.840842in}}%
\pgfpathlineto{\pgfqpoint{1.971302in}{1.841417in}}%
\pgfpathlineto{\pgfqpoint{1.971159in}{1.841569in}}%
\pgfpathlineto{\pgfqpoint{1.970420in}{1.841171in}}%
\pgfpathlineto{\pgfqpoint{1.969785in}{1.840608in}}%
\pgfpathlineto{\pgfqpoint{1.969828in}{1.840400in}}%
\pgfpathlineto{\pgfqpoint{1.970489in}{1.840725in}}%
\pgfpathlineto{\pgfqpoint{1.971141in}{1.841267in}}%
\pgfpathlineto{\pgfqpoint{1.971190in}{1.841525in}}%
\pgfpathlineto{\pgfqpoint{1.970608in}{1.841270in}}%
\pgfpathlineto{\pgfqpoint{1.969953in}{1.840754in}}%
\pgfpathlineto{\pgfqpoint{1.969824in}{1.840458in}}%
\pgfpathlineto{\pgfqpoint{1.970322in}{1.840645in}}%
\pgfpathlineto{\pgfqpoint{1.970969in}{1.841127in}}%
\pgfpathlineto{\pgfqpoint{1.971170in}{1.841454in}}%
\pgfpathlineto{\pgfqpoint{1.970754in}{1.841331in}}%
\pgfpathlineto{\pgfqpoint{1.970125in}{1.840887in}}%
\pgfpathlineto{\pgfqpoint{1.969865in}{1.840539in}}%
\pgfpathlineto{\pgfqpoint{1.970198in}{1.840602in}}%
\pgfpathlineto{\pgfqpoint{1.970800in}{1.841003in}}%
\pgfpathlineto{\pgfqpoint{1.971111in}{1.841365in}}%
\pgfpathlineto{\pgfqpoint{1.970856in}{1.841358in}}%
\pgfpathlineto{\pgfqpoint{1.970289in}{1.841001in}}%
\pgfpathlineto{\pgfqpoint{1.969939in}{1.840633in}}%
\pgfpathlineto{\pgfqpoint{1.970118in}{1.840591in}}%
\pgfpathlineto{\pgfqpoint{1.970644in}{1.840901in}}%
\pgfpathlineto{\pgfqpoint{1.971024in}{1.841268in}}%
\pgfpathlineto{\pgfqpoint{1.970916in}{1.841354in}}%
\pgfpathlineto{\pgfqpoint{1.970436in}{1.841091in}}%
\pgfpathlineto{\pgfqpoint{1.970036in}{1.840731in}}%
\pgfpathlineto{\pgfqpoint{1.970078in}{1.840607in}}%
\pgfpathlineto{\pgfqpoint{1.970509in}{1.840824in}}%
\pgfpathlineto{\pgfqpoint{1.970920in}{1.841171in}}%
\pgfpathlineto{\pgfqpoint{1.970937in}{1.841327in}}%
\pgfpathlineto{\pgfqpoint{1.970558in}{1.841155in}}%
\pgfpathlineto{\pgfqpoint{1.970144in}{1.840826in}}%
\pgfpathlineto{\pgfqpoint{1.970074in}{1.840644in}}%
\pgfpathlineto{\pgfqpoint{1.970400in}{1.840771in}}%
\pgfpathlineto{\pgfqpoint{1.970809in}{1.841080in}}%
\pgfpathlineto{\pgfqpoint{1.970926in}{1.841282in}}%
\pgfpathlineto{\pgfqpoint{1.970653in}{1.841196in}}%
\pgfpathlineto{\pgfqpoint{1.970254in}{1.840911in}}%
\pgfpathlineto{\pgfqpoint{1.970099in}{1.840695in}}%
\pgfpathlineto{\pgfqpoint{1.970319in}{1.840743in}}%
\pgfpathlineto{\pgfqpoint{1.970701in}{1.841000in}}%
\pgfpathlineto{\pgfqpoint{1.970889in}{1.841225in}}%
\pgfpathlineto{\pgfqpoint{1.970719in}{1.841214in}}%
\pgfpathlineto{\pgfqpoint{1.970359in}{1.840984in}}%
\pgfpathlineto{\pgfqpoint{1.970145in}{1.840755in}}%
\pgfpathlineto{\pgfqpoint{1.970267in}{1.840735in}}%
\pgfpathlineto{\pgfqpoint{1.970601in}{1.840935in}}%
\pgfpathlineto{\pgfqpoint{1.970835in}{1.841164in}}%
\pgfpathlineto{\pgfqpoint{1.970759in}{1.841213in}}%
\pgfpathlineto{\pgfqpoint{1.970453in}{1.841042in}}%
\pgfpathlineto{\pgfqpoint{1.970206in}{1.840817in}}%
\pgfpathlineto{\pgfqpoint{1.970240in}{1.840744in}}%
\pgfpathlineto{\pgfqpoint{1.970515in}{1.840885in}}%
\pgfpathlineto{\pgfqpoint{1.970770in}{1.841103in}}%
\pgfusepath{stroke}%
\end{pgfscope}%
\begin{pgfscope}%
\pgfpathrectangle{\pgfqpoint{0.647939in}{0.492442in}}{\pgfqpoint{3.079299in}{3.079299in}}%
\pgfusepath{clip}%
\pgfsetbuttcap%
\pgfsetroundjoin%
\pgfsetlinewidth{0.301125pt}%
\definecolor{currentstroke}{rgb}{0.500000,0.500000,0.500000}%
\pgfsetstrokecolor{currentstroke}%
\pgfsetstrokeopacity{0.300000}%
\pgfsetdash{}{0pt}%
\pgfpathmoveto{\pgfqpoint{0.927875in}{0.492442in}}%
\pgfpathlineto{\pgfqpoint{0.927875in}{0.492442in}}%
\pgfpathlineto{\pgfqpoint{0.989177in}{0.522728in}}%
\pgfpathlineto{\pgfqpoint{1.047865in}{0.557785in}}%
\pgfpathlineto{\pgfqpoint{1.103390in}{0.597623in}}%
\pgfpathlineto{\pgfqpoint{1.155305in}{0.642033in}}%
\pgfpathlineto{\pgfqpoint{1.203357in}{0.690616in}}%
\pgfusepath{stroke}%
\end{pgfscope}%
\begin{pgfscope}%
\pgfpathrectangle{\pgfqpoint{0.647939in}{0.492442in}}{\pgfqpoint{3.079299in}{3.079299in}}%
\pgfusepath{clip}%
\pgfsetbuttcap%
\pgfsetroundjoin%
\pgfsetlinewidth{0.301125pt}%
\definecolor{currentstroke}{rgb}{0.500000,0.500000,0.500000}%
\pgfsetstrokecolor{currentstroke}%
\pgfsetstrokeopacity{0.300000}%
\pgfsetdash{}{0pt}%
\pgfpathmoveto{\pgfqpoint{1.137828in}{0.492442in}}%
\pgfpathlineto{\pgfqpoint{1.137828in}{0.492442in}}%
\pgfpathlineto{\pgfqpoint{1.182394in}{0.544191in}}%
\pgfpathlineto{\pgfqpoint{1.222722in}{0.599304in}}%
\pgfpathlineto{\pgfqpoint{1.259315in}{0.656965in}}%
\pgfpathlineto{\pgfqpoint{1.292842in}{0.716483in}}%
\pgfpathlineto{\pgfqpoint{1.324010in}{0.777281in}}%
\pgfpathlineto{\pgfqpoint{1.353483in}{0.838913in}}%
\pgfusepath{stroke}%
\end{pgfscope}%
\begin{pgfscope}%
\pgfpathrectangle{\pgfqpoint{0.647939in}{0.492442in}}{\pgfqpoint{3.079299in}{3.079299in}}%
\pgfusepath{clip}%
\pgfsetbuttcap%
\pgfsetroundjoin%
\pgfsetlinewidth{0.301125pt}%
\definecolor{currentstroke}{rgb}{0.500000,0.500000,0.500000}%
\pgfsetstrokecolor{currentstroke}%
\pgfsetstrokeopacity{0.300000}%
\pgfsetdash{}{0pt}%
\pgfpathmoveto{\pgfqpoint{1.417764in}{0.492442in}}%
\pgfpathlineto{\pgfqpoint{1.417764in}{0.492442in}}%
\pgfpathlineto{\pgfqpoint{1.417764in}{0.492442in}}%
\pgfpathlineto{\pgfqpoint{1.395721in}{0.544396in}}%
\pgfpathlineto{\pgfqpoint{1.383672in}{0.599883in}}%
\pgfpathlineto{\pgfqpoint{1.379987in}{0.661165in}}%
\pgfpathlineto{\pgfqpoint{1.384180in}{0.729051in}}%
\pgfpathlineto{\pgfqpoint{1.394374in}{0.796479in}}%
\pgfpathlineto{\pgfqpoint{1.408870in}{0.863204in}}%
\pgfpathlineto{\pgfqpoint{1.426534in}{0.929175in}}%
\pgfpathlineto{\pgfqpoint{1.446679in}{0.994451in}}%
\pgfpathlineto{\pgfqpoint{1.468837in}{1.059004in}}%
\pgfusepath{stroke}%
\end{pgfscope}%
\begin{pgfscope}%
\pgfpathrectangle{\pgfqpoint{0.647939in}{0.492442in}}{\pgfqpoint{3.079299in}{3.079299in}}%
\pgfusepath{clip}%
\pgfsetbuttcap%
\pgfsetroundjoin%
\pgfsetlinewidth{0.301125pt}%
\definecolor{currentstroke}{rgb}{0.500000,0.500000,0.500000}%
\pgfsetstrokecolor{currentstroke}%
\pgfsetstrokeopacity{0.300000}%
\pgfsetdash{}{0pt}%
\pgfpathmoveto{\pgfqpoint{1.767684in}{0.492442in}}%
\pgfpathlineto{\pgfqpoint{1.767684in}{0.492442in}}%
\pgfpathlineto{\pgfqpoint{1.700475in}{0.504933in}}%
\pgfpathlineto{\pgfqpoint{1.634972in}{0.524230in}}%
\pgfpathlineto{\pgfqpoint{1.573017in}{0.552601in}}%
\pgfpathlineto{\pgfqpoint{1.518068in}{0.592469in}}%
\pgfpathlineto{\pgfqpoint{1.478191in}{0.639643in}}%
\pgfpathlineto{\pgfqpoint{1.453532in}{0.688739in}}%
\pgfusepath{stroke}%
\end{pgfscope}%
\begin{pgfscope}%
\pgfpathrectangle{\pgfqpoint{0.647939in}{0.492442in}}{\pgfqpoint{3.079299in}{3.079299in}}%
\pgfusepath{clip}%
\pgfsetbuttcap%
\pgfsetroundjoin%
\pgfsetlinewidth{0.301125pt}%
\definecolor{currentstroke}{rgb}{0.500000,0.500000,0.500000}%
\pgfsetstrokecolor{currentstroke}%
\pgfsetstrokeopacity{0.300000}%
\pgfsetdash{}{0pt}%
\pgfpathmoveto{\pgfqpoint{2.653739in}{0.492442in}}%
\pgfpathlineto{\pgfqpoint{2.593174in}{0.496719in}}%
\pgfpathlineto{\pgfqpoint{2.524863in}{0.500577in}}%
\pgfpathlineto{\pgfqpoint{2.456470in}{0.502596in}}%
\pgfpathlineto{\pgfqpoint{2.388047in}{0.502999in}}%
\pgfpathlineto{\pgfqpoint{2.319626in}{0.502084in}}%
\pgfpathlineto{\pgfqpoint{2.251224in}{0.500216in}}%
\pgfpathlineto{\pgfqpoint{2.182838in}{0.497810in}}%
\pgfpathlineto{\pgfqpoint{2.114454in}{0.495341in}}%
\pgfpathlineto{\pgfqpoint{2.046055in}{0.493346in}}%
\pgfpathlineto{\pgfqpoint{1.977636in}{0.492442in}}%
\pgfpathlineto{\pgfqpoint{1.977636in}{0.492442in}}%
\pgfusepath{stroke}%
\end{pgfscope}%
\begin{pgfscope}%
\pgfpathrectangle{\pgfqpoint{0.647939in}{0.492442in}}{\pgfqpoint{3.079299in}{3.079299in}}%
\pgfusepath{clip}%
\pgfsetbuttcap%
\pgfsetroundjoin%
\pgfsetlinewidth{0.301125pt}%
\definecolor{currentstroke}{rgb}{0.500000,0.500000,0.500000}%
\pgfsetstrokecolor{currentstroke}%
\pgfsetstrokeopacity{0.300000}%
\pgfsetdash{}{0pt}%
\pgfpathmoveto{\pgfqpoint{2.887429in}{0.492442in}}%
\pgfpathlineto{\pgfqpoint{2.887429in}{0.492442in}}%
\pgfpathlineto{\pgfqpoint{2.820201in}{0.505169in}}%
\pgfpathlineto{\pgfqpoint{2.752638in}{0.515962in}}%
\pgfpathlineto{\pgfqpoint{2.684780in}{0.524711in}}%
\pgfpathlineto{\pgfqpoint{2.616686in}{0.531375in}}%
\pgfpathlineto{\pgfqpoint{2.548422in}{0.535989in}}%
\pgfpathlineto{\pgfqpoint{2.480052in}{0.538670in}}%
\pgfpathlineto{\pgfqpoint{2.411635in}{0.539620in}}%
\pgfpathlineto{\pgfqpoint{2.343212in}{0.539116in}}%
\pgfpathlineto{\pgfqpoint{2.274804in}{0.537499in}}%
\pgfpathlineto{\pgfqpoint{2.206415in}{0.535172in}}%
\pgfpathlineto{\pgfqpoint{2.138035in}{0.532602in}}%
\pgfpathlineto{\pgfqpoint{2.069644in}{0.530325in}}%
\pgfpathlineto{\pgfqpoint{2.001233in}{0.528942in}}%
\pgfpathlineto{\pgfqpoint{1.932812in}{0.529139in}}%
\pgfpathlineto{\pgfqpoint{1.864449in}{0.531732in}}%
\pgfpathlineto{\pgfqpoint{1.796314in}{0.537723in}}%
\pgfpathlineto{\pgfqpoint{1.728787in}{0.548403in}}%
\pgfpathlineto{\pgfqpoint{1.662653in}{0.565473in}}%
\pgfpathlineto{\pgfqpoint{1.599513in}{0.591179in}}%
\pgfpathlineto{\pgfqpoint{1.542524in}{0.628111in}}%
\pgfusepath{stroke}%
\end{pgfscope}%
\begin{pgfscope}%
\pgfpathrectangle{\pgfqpoint{0.647939in}{0.492442in}}{\pgfqpoint{3.079299in}{3.079299in}}%
\pgfusepath{clip}%
\pgfsetbuttcap%
\pgfsetroundjoin%
\pgfsetlinewidth{0.301125pt}%
\definecolor{currentstroke}{rgb}{0.500000,0.500000,0.500000}%
\pgfsetstrokecolor{currentstroke}%
\pgfsetstrokeopacity{0.300000}%
\pgfsetdash{}{0pt}%
\pgfpathmoveto{\pgfqpoint{3.097382in}{0.492442in}}%
\pgfpathlineto{\pgfqpoint{3.097382in}{0.492442in}}%
\pgfpathlineto{\pgfqpoint{3.031347in}{0.510372in}}%
\pgfpathlineto{\pgfqpoint{2.964979in}{0.527020in}}%
\pgfpathlineto{\pgfqpoint{2.898246in}{0.542132in}}%
\pgfpathlineto{\pgfqpoint{2.831141in}{0.555483in}}%
\pgfpathlineto{\pgfqpoint{2.763677in}{0.566890in}}%
\pgfpathlineto{\pgfqpoint{2.695897in}{0.576222in}}%
\pgfusepath{stroke}%
\end{pgfscope}%
\begin{pgfscope}%
\pgfpathrectangle{\pgfqpoint{0.647939in}{0.492442in}}{\pgfqpoint{3.079299in}{3.079299in}}%
\pgfusepath{clip}%
\pgfsetbuttcap%
\pgfsetroundjoin%
\pgfsetlinewidth{0.301125pt}%
\definecolor{currentstroke}{rgb}{0.500000,0.500000,0.500000}%
\pgfsetstrokecolor{currentstroke}%
\pgfsetstrokeopacity{0.300000}%
\pgfsetdash{}{0pt}%
\pgfpathmoveto{\pgfqpoint{3.307334in}{0.492442in}}%
\pgfpathlineto{\pgfqpoint{3.307334in}{0.492442in}}%
\pgfpathlineto{\pgfqpoint{3.242111in}{0.513143in}}%
\pgfpathlineto{\pgfqpoint{3.176770in}{0.533464in}}%
\pgfpathlineto{\pgfqpoint{3.111226in}{0.553118in}}%
\pgfpathlineto{\pgfqpoint{3.045405in}{0.571817in}}%
\pgfpathlineto{\pgfqpoint{2.979245in}{0.589278in}}%
\pgfpathlineto{\pgfqpoint{2.912708in}{0.605232in}}%
\pgfpathlineto{\pgfqpoint{2.845778in}{0.619435in}}%
\pgfpathlineto{\pgfqpoint{2.778464in}{0.631684in}}%
\pgfpathlineto{\pgfqpoint{2.710801in}{0.641827in}}%
\pgfpathlineto{\pgfqpoint{2.642847in}{0.649780in}}%
\pgfpathlineto{\pgfqpoint{2.574670in}{0.655536in}}%
\pgfpathlineto{\pgfqpoint{2.506346in}{0.659180in}}%
\pgfpathlineto{\pgfqpoint{2.437946in}{0.660885in}}%
\pgfpathlineto{\pgfqpoint{2.369522in}{0.660905in}}%
\pgfpathlineto{\pgfqpoint{2.301109in}{0.659572in}}%
\pgfpathlineto{\pgfqpoint{2.232719in}{0.657291in}}%
\pgfpathlineto{\pgfqpoint{2.164346in}{0.654544in}}%
\pgfpathlineto{\pgfqpoint{2.095969in}{0.651878in}}%
\pgfpathlineto{\pgfqpoint{2.027571in}{0.649911in}}%
\pgfpathlineto{\pgfqpoint{1.959151in}{0.649362in}}%
\pgfpathlineto{\pgfqpoint{1.890759in}{0.651092in}}%
\pgfpathlineto{\pgfqpoint{1.822552in}{0.656184in}}%
\pgfpathlineto{\pgfqpoint{1.754908in}{0.666045in}}%
\pgfpathlineto{\pgfqpoint{1.688650in}{0.682571in}}%
\pgfpathlineto{\pgfqpoint{1.625579in}{0.708335in}}%
\pgfpathlineto{\pgfqpoint{1.569404in}{0.746289in}}%
\pgfpathlineto{\pgfqpoint{1.529698in}{0.791429in}}%
\pgfpathlineto{\pgfqpoint{1.505701in}{0.838469in}}%
\pgfpathlineto{\pgfqpoint{1.492481in}{0.888492in}}%
\pgfpathlineto{\pgfqpoint{1.487919in}{0.943727in}}%
\pgfpathlineto{\pgfqpoint{1.491649in}{1.006353in}}%
\pgfusepath{stroke}%
\end{pgfscope}%
\begin{pgfscope}%
\pgfpathrectangle{\pgfqpoint{0.647939in}{0.492442in}}{\pgfqpoint{3.079299in}{3.079299in}}%
\pgfusepath{clip}%
\pgfsetbuttcap%
\pgfsetroundjoin%
\pgfsetlinewidth{0.301125pt}%
\definecolor{currentstroke}{rgb}{0.500000,0.500000,0.500000}%
\pgfsetstrokecolor{currentstroke}%
\pgfsetstrokeopacity{0.300000}%
\pgfsetdash{}{0pt}%
\pgfpathmoveto{\pgfqpoint{3.517286in}{0.492442in}}%
\pgfpathlineto{\pgfqpoint{3.517286in}{0.492442in}}%
\pgfpathlineto{\pgfqpoint{3.452023in}{0.513011in}}%
\pgfpathlineto{\pgfqpoint{3.386932in}{0.534119in}}%
\pgfpathlineto{\pgfqpoint{3.321927in}{0.555494in}}%
\pgfpathlineto{\pgfqpoint{3.256916in}{0.576851in}}%
\pgfpathlineto{\pgfqpoint{3.191806in}{0.597901in}}%
\pgfpathlineto{\pgfqpoint{3.126507in}{0.618353in}}%
\pgfpathlineto{\pgfqpoint{3.060935in}{0.637910in}}%
\pgfpathlineto{\pgfqpoint{2.995024in}{0.656283in}}%
\pgfpathlineto{\pgfqpoint{2.928723in}{0.673190in}}%
\pgfpathlineto{\pgfqpoint{2.862007in}{0.688369in}}%
\pgfpathlineto{\pgfqpoint{2.794876in}{0.701594in}}%
\pgfpathlineto{\pgfqpoint{2.727362in}{0.712684in}}%
\pgfpathlineto{\pgfqpoint{2.659518in}{0.721527in}}%
\pgfpathlineto{\pgfqpoint{2.591416in}{0.728090in}}%
\pgfpathlineto{\pgfqpoint{2.523135in}{0.732427in}}%
\pgfpathlineto{\pgfqpoint{2.454752in}{0.734684in}}%
\pgfpathlineto{\pgfqpoint{2.386330in}{0.735102in}}%
\pgfpathlineto{\pgfqpoint{2.317913in}{0.734013in}}%
\pgfpathlineto{\pgfqpoint{2.249521in}{0.731823in}}%
\pgfpathlineto{\pgfqpoint{2.181150in}{0.729007in}}%
\pgfpathlineto{\pgfqpoint{2.112783in}{0.726112in}}%
\pgfpathlineto{\pgfqpoint{2.044396in}{0.723772in}}%
\pgfpathlineto{\pgfqpoint{1.975981in}{0.722738in}}%
\pgfpathlineto{\pgfqpoint{1.907578in}{0.723903in}}%
\pgfpathlineto{\pgfqpoint{1.839330in}{0.728379in}}%
\pgfpathlineto{\pgfqpoint{1.771597in}{0.737641in}}%
\pgfpathlineto{\pgfqpoint{1.705221in}{0.753728in}}%
\pgfpathlineto{\pgfqpoint{1.642131in}{0.779460in}}%
\pgfpathlineto{\pgfqpoint{1.586401in}{0.818102in}}%
\pgfpathlineto{\pgfqpoint{1.549061in}{0.862580in}}%
\pgfpathlineto{\pgfqpoint{1.526919in}{0.908903in}}%
\pgfusepath{stroke}%
\end{pgfscope}%
\begin{pgfscope}%
\pgfpathrectangle{\pgfqpoint{0.647939in}{0.492442in}}{\pgfqpoint{3.079299in}{3.079299in}}%
\pgfusepath{clip}%
\pgfsetbuttcap%
\pgfsetroundjoin%
\pgfsetlinewidth{0.301125pt}%
\definecolor{currentstroke}{rgb}{0.500000,0.500000,0.500000}%
\pgfsetstrokecolor{currentstroke}%
\pgfsetstrokeopacity{0.300000}%
\pgfsetdash{}{0pt}%
\pgfpathmoveto{\pgfqpoint{3.727238in}{0.562426in}}%
\pgfpathlineto{\pgfqpoint{3.727238in}{0.562426in}}%
\pgfpathlineto{\pgfqpoint{3.661261in}{0.580562in}}%
\pgfpathlineto{\pgfqpoint{3.595682in}{0.600089in}}%
\pgfpathlineto{\pgfqpoint{3.530458in}{0.620777in}}%
\pgfpathlineto{\pgfqpoint{3.465528in}{0.642376in}}%
\pgfpathlineto{\pgfqpoint{3.400817in}{0.664622in}}%
\pgfpathlineto{\pgfqpoint{3.336234in}{0.687238in}}%
\pgfpathlineto{\pgfqpoint{3.271681in}{0.709940in}}%
\pgfpathlineto{\pgfqpoint{3.207056in}{0.732435in}}%
\pgfpathlineto{\pgfqpoint{3.142256in}{0.754421in}}%
\pgfpathlineto{\pgfqpoint{3.077187in}{0.775592in}}%
\pgfpathlineto{\pgfqpoint{3.011765in}{0.795640in}}%
\pgfpathlineto{\pgfqpoint{2.945922in}{0.814256in}}%
\pgfpathlineto{\pgfqpoint{2.879619in}{0.831148in}}%
\pgfpathlineto{\pgfqpoint{2.812843in}{0.846050in}}%
\pgfpathlineto{\pgfqpoint{2.745614in}{0.858741in}}%
\pgfpathlineto{\pgfqpoint{2.677981in}{0.869057in}}%
\pgfpathlineto{\pgfqpoint{2.610017in}{0.876916in}}%
\pgfpathlineto{\pgfqpoint{2.541814in}{0.882331in}}%
\pgfpathlineto{\pgfqpoint{2.473466in}{0.885423in}}%
\pgfpathlineto{\pgfqpoint{2.405052in}{0.886415in}}%
\pgfpathlineto{\pgfqpoint{2.336634in}{0.885626in}}%
\pgfpathlineto{\pgfqpoint{2.268242in}{0.883465in}}%
\pgfpathlineto{\pgfqpoint{2.199881in}{0.880435in}}%
\pgfpathlineto{\pgfqpoint{2.131532in}{0.877131in}}%
\pgfpathlineto{\pgfqpoint{2.063165in}{0.874238in}}%
\pgfpathlineto{\pgfqpoint{1.994763in}{0.872563in}}%
\pgfpathlineto{\pgfqpoint{1.926352in}{0.873101in}}%
\pgfpathlineto{\pgfqpoint{1.858078in}{0.877148in}}%
\pgfpathlineto{\pgfqpoint{1.790379in}{0.886495in}}%
\pgfpathlineto{\pgfqpoint{1.724372in}{0.903728in}}%
\pgfpathlineto{\pgfqpoint{1.662847in}{0.932489in}}%
\pgfpathlineto{\pgfqpoint{1.662847in}{0.932489in}}%
\pgfpathlineto{\pgfqpoint{1.621804in}{0.964746in}}%
\pgfpathlineto{\pgfqpoint{1.590166in}{1.007108in}}%
\pgfpathlineto{\pgfqpoint{1.572235in}{1.051502in}}%
\pgfpathlineto{\pgfqpoint{1.563983in}{1.099372in}}%
\pgfpathlineto{\pgfqpoint{1.564073in}{1.152929in}}%
\pgfpathlineto{\pgfqpoint{1.572772in}{1.214404in}}%
\pgfusepath{stroke}%
\end{pgfscope}%
\begin{pgfscope}%
\pgfpathrectangle{\pgfqpoint{0.647939in}{0.492442in}}{\pgfqpoint{3.079299in}{3.079299in}}%
\pgfusepath{clip}%
\pgfsetbuttcap%
\pgfsetroundjoin%
\pgfsetlinewidth{0.301125pt}%
\definecolor{currentstroke}{rgb}{0.500000,0.500000,0.500000}%
\pgfsetstrokecolor{currentstroke}%
\pgfsetstrokeopacity{0.300000}%
\pgfsetdash{}{0pt}%
\pgfpathmoveto{\pgfqpoint{3.727238in}{0.632410in}}%
\pgfpathlineto{\pgfqpoint{3.727238in}{0.632410in}}%
\pgfpathlineto{\pgfqpoint{3.661417in}{0.651103in}}%
\pgfpathlineto{\pgfqpoint{3.596021in}{0.671235in}}%
\pgfpathlineto{\pgfqpoint{3.531006in}{0.692569in}}%
\pgfpathlineto{\pgfqpoint{3.466309in}{0.714854in}}%
\pgfpathlineto{\pgfqpoint{3.401851in}{0.737822in}}%
\pgfpathlineto{\pgfqpoint{3.337538in}{0.761196in}}%
\pgfpathlineto{\pgfqpoint{3.273267in}{0.784686in}}%
\pgfpathlineto{\pgfqpoint{3.208932in}{0.807998in}}%
\pgfpathlineto{\pgfqpoint{3.144425in}{0.830826in}}%
\pgfpathlineto{\pgfqpoint{3.079642in}{0.852857in}}%
\pgfpathlineto{\pgfqpoint{3.014492in}{0.873774in}}%
\pgfpathlineto{\pgfqpoint{2.948902in}{0.893260in}}%
\pgfpathlineto{\pgfqpoint{2.882822in}{0.911005in}}%
\pgfpathlineto{\pgfqpoint{2.816234in}{0.926724in}}%
\pgfpathlineto{\pgfqpoint{2.749153in}{0.940173in}}%
\pgfpathlineto{\pgfqpoint{2.681627in}{0.951167in}}%
\pgfpathlineto{\pgfqpoint{2.613735in}{0.959600in}}%
\pgfpathlineto{\pgfqpoint{2.545571in}{0.965464in}}%
\pgfpathlineto{\pgfqpoint{2.477237in}{0.968868in}}%
\pgfpathlineto{\pgfqpoint{2.408828in}{0.970034in}}%
\pgfpathlineto{\pgfqpoint{2.340409in}{0.969287in}}%
\pgfpathlineto{\pgfqpoint{2.272021in}{0.967051in}}%
\pgfpathlineto{\pgfqpoint{2.203668in}{0.963841in}}%
\pgfpathlineto{\pgfqpoint{2.135332in}{0.960283in}}%
\pgfpathlineto{\pgfqpoint{2.066978in}{0.957112in}}%
\pgfpathlineto{\pgfqpoint{1.998582in}{0.955197in}}%
\pgfpathlineto{\pgfqpoint{1.930174in}{0.955628in}}%
\pgfpathlineto{\pgfqpoint{1.861917in}{0.959855in}}%
\pgfpathlineto{\pgfqpoint{1.794345in}{0.969966in}}%
\pgfpathlineto{\pgfqpoint{1.728932in}{0.989090in}}%
\pgfpathlineto{\pgfqpoint{1.669551in}{1.021595in}}%
\pgfpathlineto{\pgfqpoint{1.669551in}{1.021595in}}%
\pgfpathlineto{\pgfqpoint{1.634680in}{1.054707in}}%
\pgfpathlineto{\pgfqpoint{1.609672in}{1.096292in}}%
\pgfpathlineto{\pgfqpoint{1.596466in}{1.140231in}}%
\pgfusepath{stroke}%
\end{pgfscope}%
\begin{pgfscope}%
\pgfpathrectangle{\pgfqpoint{0.647939in}{0.492442in}}{\pgfqpoint{3.079299in}{3.079299in}}%
\pgfusepath{clip}%
\pgfsetbuttcap%
\pgfsetroundjoin%
\pgfsetlinewidth{0.301125pt}%
\definecolor{currentstroke}{rgb}{0.500000,0.500000,0.500000}%
\pgfsetstrokecolor{currentstroke}%
\pgfsetstrokeopacity{0.300000}%
\pgfsetdash{}{0pt}%
\pgfpathmoveto{\pgfqpoint{3.727238in}{0.702394in}}%
\pgfpathlineto{\pgfqpoint{3.727238in}{0.702394in}}%
\pgfpathlineto{\pgfqpoint{3.661589in}{0.721679in}}%
\pgfpathlineto{\pgfqpoint{3.596393in}{0.742451in}}%
\pgfpathlineto{\pgfqpoint{3.531607in}{0.764472in}}%
\pgfpathlineto{\pgfqpoint{3.467166in}{0.787486in}}%
\pgfpathlineto{\pgfqpoint{3.402988in}{0.811224in}}%
\pgfpathlineto{\pgfqpoint{3.338974in}{0.835405in}}%
\pgfpathlineto{\pgfqpoint{3.275019in}{0.859740in}}%
\pgfpathlineto{\pgfqpoint{3.211009in}{0.883931in}}%
\pgfpathlineto{\pgfqpoint{3.146830in}{0.907667in}}%
\pgfpathlineto{\pgfqpoint{3.082373in}{0.930631in}}%
\pgfpathlineto{\pgfqpoint{3.017536in}{0.952499in}}%
\pgfpathlineto{\pgfqpoint{2.952238in}{0.972940in}}%
\pgfpathlineto{\pgfqpoint{2.886419in}{0.991632in}}%
\pgfpathlineto{\pgfqpoint{2.820054in}{1.008267in}}%
\pgfpathlineto{\pgfqpoint{2.753151in}{1.022575in}}%
\pgfpathlineto{\pgfqpoint{2.685758in}{1.034343in}}%
\pgfpathlineto{\pgfqpoint{2.617953in}{1.043441in}}%
\pgfpathlineto{\pgfqpoint{2.549839in}{1.049838in}}%
\pgfpathlineto{\pgfqpoint{2.481527in}{1.053619in}}%
\pgfpathlineto{\pgfqpoint{2.413122in}{1.054999in}}%
\pgfpathlineto{\pgfqpoint{2.344704in}{1.054313in}}%
\pgfpathlineto{\pgfqpoint{2.276319in}{1.051998in}}%
\pgfpathlineto{\pgfqpoint{2.207976in}{1.048593in}}%
\pgfpathlineto{\pgfqpoint{2.139656in}{1.044746in}}%
\pgfpathlineto{\pgfqpoint{2.071318in}{1.041240in}}%
\pgfpathlineto{\pgfqpoint{2.002933in}{1.039025in}}%
\pgfpathlineto{\pgfqpoint{1.934526in}{1.039305in}}%
\pgfpathlineto{\pgfqpoint{1.866296in}{1.043731in}}%
\pgfpathlineto{\pgfqpoint{1.798894in}{1.054773in}}%
\pgfpathlineto{\pgfqpoint{1.734331in}{1.076302in}}%
\pgfpathlineto{\pgfqpoint{1.734331in}{1.076302in}}%
\pgfpathlineto{\pgfqpoint{1.689617in}{1.103067in}}%
\pgfusepath{stroke}%
\end{pgfscope}%
\begin{pgfscope}%
\pgfpathrectangle{\pgfqpoint{0.647939in}{0.492442in}}{\pgfqpoint{3.079299in}{3.079299in}}%
\pgfusepath{clip}%
\pgfsetbuttcap%
\pgfsetroundjoin%
\pgfsetlinewidth{0.301125pt}%
\definecolor{currentstroke}{rgb}{0.500000,0.500000,0.500000}%
\pgfsetstrokecolor{currentstroke}%
\pgfsetstrokeopacity{0.300000}%
\pgfsetdash{}{0pt}%
\pgfpathmoveto{\pgfqpoint{3.727238in}{0.772378in}}%
\pgfpathlineto{\pgfqpoint{3.727238in}{0.772378in}}%
\pgfpathlineto{\pgfqpoint{3.661777in}{0.792292in}}%
\pgfpathlineto{\pgfqpoint{3.596803in}{0.813745in}}%
\pgfpathlineto{\pgfqpoint{3.532270in}{0.836495in}}%
\pgfpathlineto{\pgfqpoint{3.468112in}{0.860286in}}%
\pgfpathlineto{\pgfqpoint{3.404243in}{0.884845in}}%
\pgfpathlineto{\pgfqpoint{3.340563in}{0.909890in}}%
\pgfpathlineto{\pgfqpoint{3.276960in}{0.935132in}}%
\pgfpathlineto{\pgfqpoint{3.213316in}{0.960268in}}%
\pgfpathlineto{\pgfqpoint{3.149509in}{0.984989in}}%
\pgfpathlineto{\pgfqpoint{3.085423in}{1.008970in}}%
\pgfpathlineto{\pgfqpoint{3.020948in}{1.031880in}}%
\pgfpathlineto{\pgfqpoint{2.955991in}{1.053379in}}%
\pgfpathlineto{\pgfqpoint{2.890482in}{1.073126in}}%
\pgfpathlineto{\pgfqpoint{2.824383in}{1.090793in}}%
\pgfpathlineto{\pgfqpoint{2.757697in}{1.106080in}}%
\pgfpathlineto{\pgfqpoint{2.690467in}{1.118744in}}%
\pgfpathlineto{\pgfqpoint{2.622773in}{1.128620in}}%
\pgfpathlineto{\pgfqpoint{2.554724in}{1.135649in}}%
\pgfpathlineto{\pgfqpoint{2.486442in}{1.139891in}}%
\pgfpathlineto{\pgfqpoint{2.418045in}{1.141545in}}%
\pgfpathlineto{\pgfqpoint{2.349627in}{1.140945in}}%
\pgfpathlineto{\pgfqpoint{2.281246in}{1.138550in}}%
\pgfpathlineto{\pgfqpoint{2.212915in}{1.134921in}}%
\pgfpathlineto{\pgfqpoint{2.144614in}{1.130741in}}%
\pgfpathlineto{\pgfqpoint{2.076298in}{1.126835in}}%
\pgfpathlineto{\pgfqpoint{2.007926in}{1.124239in}}%
\pgfpathlineto{\pgfqpoint{1.939524in}{1.124302in}}%
\pgfpathlineto{\pgfqpoint{1.871324in}{1.128937in}}%
\pgfpathlineto{\pgfqpoint{1.804185in}{1.141144in}}%
\pgfpathlineto{\pgfqpoint{1.740954in}{1.165843in}}%
\pgfpathlineto{\pgfqpoint{1.740954in}{1.165843in}}%
\pgfpathlineto{\pgfqpoint{1.702988in}{1.193251in}}%
\pgfpathlineto{\pgfqpoint{1.674253in}{1.231297in}}%
\pgfpathlineto{\pgfqpoint{1.659561in}{1.270911in}}%
\pgfpathlineto{\pgfqpoint{1.654394in}{1.313250in}}%
\pgfpathlineto{\pgfqpoint{1.657473in}{1.361358in}}%
\pgfpathlineto{\pgfqpoint{1.669397in}{1.416644in}}%
\pgfusepath{stroke}%
\end{pgfscope}%
\begin{pgfscope}%
\pgfpathrectangle{\pgfqpoint{0.647939in}{0.492442in}}{\pgfqpoint{3.079299in}{3.079299in}}%
\pgfusepath{clip}%
\pgfsetbuttcap%
\pgfsetroundjoin%
\pgfsetlinewidth{0.301125pt}%
\definecolor{currentstroke}{rgb}{0.500000,0.500000,0.500000}%
\pgfsetstrokecolor{currentstroke}%
\pgfsetstrokeopacity{0.300000}%
\pgfsetdash{}{0pt}%
\pgfpathmoveto{\pgfqpoint{3.727238in}{0.842362in}}%
\pgfpathlineto{\pgfqpoint{3.727238in}{0.842362in}}%
\pgfpathlineto{\pgfqpoint{3.661985in}{0.862945in}}%
\pgfpathlineto{\pgfqpoint{3.597254in}{0.885122in}}%
\pgfpathlineto{\pgfqpoint{3.533002in}{0.908651in}}%
\pgfpathlineto{\pgfqpoint{3.469157in}{0.933270in}}%
\pgfpathlineto{\pgfqpoint{3.405633in}{0.958706in}}%
\pgfpathlineto{\pgfqpoint{3.342325in}{0.984678in}}%
\pgfpathlineto{\pgfqpoint{3.279117in}{1.010894in}}%
\pgfpathlineto{\pgfqpoint{3.215886in}{1.037053in}}%
\pgfpathlineto{\pgfqpoint{3.152504in}{1.062841in}}%
\pgfpathlineto{\pgfqpoint{3.088845in}{1.087935in}}%
\pgfpathlineto{\pgfqpoint{3.024790in}{1.111993in}}%
\pgfpathlineto{\pgfqpoint{2.960234in}{1.134667in}}%
\pgfpathlineto{\pgfqpoint{2.895095in}{1.155599in}}%
\pgfpathlineto{\pgfqpoint{2.829322in}{1.174437in}}%
\pgfpathlineto{\pgfqpoint{2.762905in}{1.190852in}}%
\pgfpathlineto{\pgfqpoint{2.695880in}{1.204560in}}%
\pgfpathlineto{\pgfqpoint{2.628328in}{1.215359in}}%
\pgfpathlineto{\pgfqpoint{2.560364in}{1.223149in}}%
\pgfpathlineto{\pgfqpoint{2.492123in}{1.227962in}}%
\pgfpathlineto{\pgfqpoint{2.423739in}{1.229971in}}%
\pgfpathlineto{\pgfqpoint{2.355322in}{1.229500in}}%
\pgfpathlineto{\pgfqpoint{2.286944in}{1.227024in}}%
\pgfpathlineto{\pgfqpoint{2.218628in}{1.223136in}}%
\pgfpathlineto{\pgfqpoint{2.150353in}{1.218559in}}%
\pgfpathlineto{\pgfqpoint{2.082066in}{1.214169in}}%
\pgfpathlineto{\pgfqpoint{2.013714in}{1.211087in}}%
\pgfpathlineto{\pgfqpoint{1.945315in}{1.210836in}}%
\pgfpathlineto{\pgfqpoint{1.877153in}{1.215684in}}%
\pgfpathlineto{\pgfqpoint{1.810409in}{1.229392in}}%
\pgfpathlineto{\pgfqpoint{1.810409in}{1.229392in}}%
\pgfpathlineto{\pgfqpoint{1.761133in}{1.250352in}}%
\pgfpathlineto{\pgfqpoint{1.761133in}{1.250352in}}%
\pgfusepath{stroke}%
\end{pgfscope}%
\begin{pgfscope}%
\pgfpathrectangle{\pgfqpoint{0.647939in}{0.492442in}}{\pgfqpoint{3.079299in}{3.079299in}}%
\pgfusepath{clip}%
\pgfsetbuttcap%
\pgfsetroundjoin%
\pgfsetlinewidth{0.301125pt}%
\definecolor{currentstroke}{rgb}{0.500000,0.500000,0.500000}%
\pgfsetstrokecolor{currentstroke}%
\pgfsetstrokeopacity{0.300000}%
\pgfsetdash{}{0pt}%
\pgfpathmoveto{\pgfqpoint{3.727238in}{0.912347in}}%
\pgfpathlineto{\pgfqpoint{3.727238in}{0.912347in}}%
\pgfpathlineto{\pgfqpoint{3.662214in}{0.933642in}}%
\pgfpathlineto{\pgfqpoint{3.597755in}{0.956593in}}%
\pgfpathlineto{\pgfqpoint{3.533812in}{0.980951in}}%
\pgfpathlineto{\pgfqpoint{3.470316in}{1.006455in}}%
\pgfpathlineto{\pgfqpoint{3.407177in}{1.032831in}}%
\pgfpathlineto{\pgfqpoint{3.344286in}{1.059797in}}%
\pgfpathlineto{\pgfqpoint{3.281525in}{1.087063in}}%
\pgfpathlineto{\pgfqpoint{3.218763in}{1.114329in}}%
\pgfpathlineto{\pgfqpoint{3.155867in}{1.141282in}}%
\pgfpathlineto{\pgfqpoint{3.092702in}{1.167595in}}%
\pgfpathlineto{\pgfqpoint{3.029138in}{1.192924in}}%
\pgfpathlineto{\pgfqpoint{2.965058in}{1.216910in}}%
\pgfpathlineto{\pgfqpoint{2.900365in}{1.239181in}}%
\pgfpathlineto{\pgfqpoint{2.834993in}{1.259361in}}%
\pgfpathlineto{\pgfqpoint{2.768917in}{1.277086in}}%
\pgfpathlineto{\pgfqpoint{2.702158in}{1.292030in}}%
\pgfpathlineto{\pgfqpoint{2.634793in}{1.303937in}}%
\pgfpathlineto{\pgfqpoint{2.566944in}{1.312659in}}%
\pgfpathlineto{\pgfqpoint{2.498761in}{1.318187in}}%
\pgfpathlineto{\pgfqpoint{2.430395in}{1.320664in}}%
\pgfpathlineto{\pgfqpoint{2.361981in}{1.320394in}}%
\pgfpathlineto{\pgfqpoint{2.293607in}{1.317848in}}%
\pgfpathlineto{\pgfqpoint{2.225310in}{1.313655in}}%
\pgfpathlineto{\pgfqpoint{2.157070in}{1.308589in}}%
\pgfpathlineto{\pgfqpoint{2.088824in}{1.303593in}}%
\pgfpathlineto{\pgfqpoint{2.020504in}{1.299886in}}%
\pgfpathlineto{\pgfqpoint{1.952111in}{1.299186in}}%
\pgfpathlineto{\pgfqpoint{1.883988in}{1.304235in}}%
\pgfpathlineto{\pgfqpoint{1.817871in}{1.319971in}}%
\pgfpathlineto{\pgfqpoint{1.817871in}{1.319971in}}%
\pgfpathlineto{\pgfqpoint{1.776661in}{1.340791in}}%
\pgfpathlineto{\pgfqpoint{1.776661in}{1.340791in}}%
\pgfpathlineto{\pgfqpoint{1.748346in}{1.367419in}}%
\pgfusepath{stroke}%
\end{pgfscope}%
\begin{pgfscope}%
\pgfpathrectangle{\pgfqpoint{0.647939in}{0.492442in}}{\pgfqpoint{3.079299in}{3.079299in}}%
\pgfusepath{clip}%
\pgfsetbuttcap%
\pgfsetroundjoin%
\pgfsetlinewidth{0.301125pt}%
\definecolor{currentstroke}{rgb}{0.500000,0.500000,0.500000}%
\pgfsetstrokecolor{currentstroke}%
\pgfsetstrokeopacity{0.300000}%
\pgfsetdash{}{0pt}%
\pgfpathmoveto{\pgfqpoint{3.727238in}{0.982331in}}%
\pgfpathlineto{\pgfqpoint{3.727238in}{0.982331in}}%
\pgfpathlineto{\pgfqpoint{3.662469in}{1.004388in}}%
\pgfpathlineto{\pgfqpoint{3.598310in}{1.028164in}}%
\pgfpathlineto{\pgfqpoint{3.534713in}{1.053409in}}%
\pgfpathlineto{\pgfqpoint{3.471606in}{1.079861in}}%
\pgfpathlineto{\pgfqpoint{3.408897in}{1.107245in}}%
\pgfpathlineto{\pgfqpoint{3.346477in}{1.135281in}}%
\pgfpathlineto{\pgfqpoint{3.284221in}{1.163682in}}%
\pgfpathlineto{\pgfqpoint{3.221995in}{1.192150in}}%
\pgfpathlineto{\pgfqpoint{3.159660in}{1.220375in}}%
\pgfpathlineto{\pgfqpoint{3.097071in}{1.248030in}}%
\pgfpathlineto{\pgfqpoint{3.034088in}{1.274771in}}%
\pgfpathlineto{\pgfqpoint{2.970579in}{1.300232in}}%
\pgfpathlineto{\pgfqpoint{2.906430in}{1.324027in}}%
\pgfpathlineto{\pgfqpoint{2.841557in}{1.345755in}}%
\pgfpathlineto{\pgfqpoint{2.775915in}{1.365020in}}%
\pgfpathlineto{\pgfqpoint{2.709508in}{1.381444in}}%
\pgfpathlineto{\pgfqpoint{2.642402in}{1.394711in}}%
\pgfpathlineto{\pgfqpoint{2.574715in}{1.404601in}}%
\pgfpathlineto{\pgfqpoint{2.506614in}{1.411043in}}%
\pgfpathlineto{\pgfqpoint{2.438278in}{1.414140in}}%
\pgfpathlineto{\pgfqpoint{2.369867in}{1.414171in}}%
\pgfpathlineto{\pgfqpoint{2.301498in}{1.411597in}}%
\pgfpathlineto{\pgfqpoint{2.233226in}{1.407053in}}%
\pgfpathlineto{\pgfqpoint{2.165034in}{1.401368in}}%
\pgfpathlineto{\pgfqpoint{2.096852in}{1.395575in}}%
\pgfpathlineto{\pgfqpoint{2.028584in}{1.391021in}}%
\pgfpathlineto{\pgfqpoint{1.960209in}{1.389674in}}%
\pgfpathlineto{\pgfqpoint{1.892127in}{1.394907in}}%
\pgfpathlineto{\pgfqpoint{1.892127in}{1.394907in}}%
\pgfpathlineto{\pgfqpoint{1.836715in}{1.409062in}}%
\pgfpathlineto{\pgfqpoint{1.836715in}{1.409062in}}%
\pgfusepath{stroke}%
\end{pgfscope}%
\begin{pgfscope}%
\pgfpathrectangle{\pgfqpoint{0.647939in}{0.492442in}}{\pgfqpoint{3.079299in}{3.079299in}}%
\pgfusepath{clip}%
\pgfsetbuttcap%
\pgfsetroundjoin%
\pgfsetlinewidth{0.301125pt}%
\definecolor{currentstroke}{rgb}{0.500000,0.500000,0.500000}%
\pgfsetstrokecolor{currentstroke}%
\pgfsetstrokeopacity{0.300000}%
\pgfsetdash{}{0pt}%
\pgfpathmoveto{\pgfqpoint{3.727238in}{1.052315in}}%
\pgfpathlineto{\pgfqpoint{3.727238in}{1.052315in}}%
\pgfpathlineto{\pgfqpoint{3.662753in}{1.075187in}}%
\pgfpathlineto{\pgfqpoint{3.598929in}{1.099846in}}%
\pgfpathlineto{\pgfqpoint{3.535718in}{1.126042in}}%
\pgfpathlineto{\pgfqpoint{3.473047in}{1.153509in}}%
\pgfpathlineto{\pgfqpoint{3.410823in}{1.181977in}}%
\pgfpathlineto{\pgfqpoint{3.348934in}{1.211167in}}%
\pgfpathlineto{\pgfqpoint{3.287253in}{1.240797in}}%
\pgfpathlineto{\pgfqpoint{3.225643in}{1.270574in}}%
\pgfpathlineto{\pgfqpoint{3.163957in}{1.300193in}}%
\pgfpathlineto{\pgfqpoint{3.102044in}{1.329333in}}%
\pgfpathlineto{\pgfqpoint{3.039752in}{1.357650in}}%
\pgfpathlineto{\pgfqpoint{2.976936in}{1.384776in}}%
\pgfpathlineto{\pgfqpoint{2.913461in}{1.410316in}}%
\pgfpathlineto{\pgfqpoint{2.849220in}{1.433850in}}%
\pgfpathlineto{\pgfqpoint{2.784141in}{1.454942in}}%
\pgfpathlineto{\pgfqpoint{2.718207in}{1.473163in}}%
\pgfpathlineto{\pgfqpoint{2.651462in}{1.488125in}}%
\pgfpathlineto{\pgfqpoint{2.584019in}{1.499522in}}%
\pgfpathlineto{\pgfqpoint{2.516050in}{1.507183in}}%
\pgfpathlineto{\pgfqpoint{2.447761in}{1.511130in}}%
\pgfpathlineto{\pgfqpoint{2.379356in}{1.511605in}}%
\pgfpathlineto{\pgfqpoint{2.310991in}{1.509055in}}%
\pgfpathlineto{\pgfqpoint{2.242749in}{1.504129in}}%
\pgfpathlineto{\pgfqpoint{2.174628in}{1.497677in}}%
\pgfpathlineto{\pgfqpoint{2.106545in}{1.490804in}}%
\pgfpathlineto{\pgfqpoint{2.038370in}{1.485025in}}%
\pgfpathlineto{\pgfqpoint{1.970036in}{1.482646in}}%
\pgfpathlineto{\pgfqpoint{1.902078in}{1.488033in}}%
\pgfpathlineto{\pgfqpoint{1.902078in}{1.488033in}}%
\pgfpathlineto{\pgfqpoint{1.856413in}{1.501060in}}%
\pgfpathlineto{\pgfqpoint{1.856413in}{1.501060in}}%
\pgfpathlineto{\pgfqpoint{1.826676in}{1.519881in}}%
\pgfpathlineto{\pgfqpoint{1.826676in}{1.519881in}}%
\pgfpathlineto{\pgfqpoint{1.808622in}{1.544041in}}%
\pgfpathlineto{\pgfqpoint{1.800614in}{1.573264in}}%
\pgfusepath{stroke}%
\end{pgfscope}%
\begin{pgfscope}%
\pgfpathrectangle{\pgfqpoint{0.647939in}{0.492442in}}{\pgfqpoint{3.079299in}{3.079299in}}%
\pgfusepath{clip}%
\pgfsetbuttcap%
\pgfsetroundjoin%
\pgfsetlinewidth{0.301125pt}%
\definecolor{currentstroke}{rgb}{0.500000,0.500000,0.500000}%
\pgfsetstrokecolor{currentstroke}%
\pgfsetstrokeopacity{0.300000}%
\pgfsetdash{}{0pt}%
\pgfpathmoveto{\pgfqpoint{3.727238in}{1.122299in}}%
\pgfpathlineto{\pgfqpoint{3.727238in}{1.122299in}}%
\pgfpathlineto{\pgfqpoint{3.663070in}{1.146044in}}%
\pgfpathlineto{\pgfqpoint{3.599621in}{1.171651in}}%
\pgfpathlineto{\pgfqpoint{3.536842in}{1.198865in}}%
\pgfpathlineto{\pgfqpoint{3.474662in}{1.227424in}}%
\pgfpathlineto{\pgfqpoint{3.412986in}{1.257059in}}%
\pgfpathlineto{\pgfqpoint{3.351701in}{1.287497in}}%
\pgfpathlineto{\pgfqpoint{3.290679in}{1.318461in}}%
\pgfpathlineto{\pgfqpoint{3.229781in}{1.349666in}}%
\pgfpathlineto{\pgfqpoint{3.168854in}{1.380815in}}%
\pgfpathlineto{\pgfqpoint{3.107740in}{1.411595in}}%
\pgfpathlineto{\pgfqpoint{3.046279in}{1.441672in}}%
\pgfpathlineto{\pgfqpoint{2.984310in}{1.470684in}}%
\pgfpathlineto{\pgfqpoint{2.921680in}{1.498232in}}%
\pgfpathlineto{\pgfqpoint{2.858253in}{1.523884in}}%
\pgfpathlineto{\pgfqpoint{2.793924in}{1.547172in}}%
\pgfpathlineto{\pgfqpoint{2.728640in}{1.567610in}}%
\pgfpathlineto{\pgfqpoint{2.662414in}{1.584722in}}%
\pgfpathlineto{\pgfqpoint{2.595340in}{1.598096in}}%
\pgfpathlineto{\pgfqpoint{2.527590in}{1.607437in}}%
\pgfpathlineto{\pgfqpoint{2.459396in}{1.612635in}}%
\pgfpathlineto{\pgfqpoint{2.391008in}{1.613814in}}%
\pgfpathlineto{\pgfqpoint{2.322644in}{1.611376in}}%
\pgfpathlineto{\pgfqpoint{2.254442in}{1.605986in}}%
\pgfpathlineto{\pgfqpoint{2.186426in}{1.598536in}}%
\pgfpathlineto{\pgfqpoint{2.118508in}{1.590198in}}%
\pgfpathlineto{\pgfqpoint{2.050508in}{1.582602in}}%
\pgfpathlineto{\pgfqpoint{1.982261in}{1.578449in}}%
\pgfpathlineto{\pgfqpoint{1.914480in}{1.583739in}}%
\pgfpathlineto{\pgfqpoint{1.914480in}{1.583739in}}%
\pgfpathlineto{\pgfqpoint{1.879741in}{1.595099in}}%
\pgfpathlineto{\pgfqpoint{1.879741in}{1.595099in}}%
\pgfpathlineto{\pgfqpoint{1.857082in}{1.612552in}}%
\pgfpathlineto{\pgfqpoint{1.844841in}{1.638982in}}%
\pgfpathlineto{\pgfqpoint{1.843661in}{1.664017in}}%
\pgfpathlineto{\pgfqpoint{1.850030in}{1.692430in}}%
\pgfusepath{stroke}%
\end{pgfscope}%
\begin{pgfscope}%
\pgfpathrectangle{\pgfqpoint{0.647939in}{0.492442in}}{\pgfqpoint{3.079299in}{3.079299in}}%
\pgfusepath{clip}%
\pgfsetbuttcap%
\pgfsetroundjoin%
\pgfsetlinewidth{0.301125pt}%
\definecolor{currentstroke}{rgb}{0.500000,0.500000,0.500000}%
\pgfsetstrokecolor{currentstroke}%
\pgfsetstrokeopacity{0.300000}%
\pgfsetdash{}{0pt}%
\pgfpathmoveto{\pgfqpoint{3.727238in}{1.192283in}}%
\pgfpathlineto{\pgfqpoint{3.727238in}{1.192283in}}%
\pgfpathlineto{\pgfqpoint{3.663426in}{1.216966in}}%
\pgfpathlineto{\pgfqpoint{3.600397in}{1.243590in}}%
\pgfpathlineto{\pgfqpoint{3.538105in}{1.271900in}}%
\pgfpathlineto{\pgfqpoint{3.476479in}{1.301635in}}%
\pgfpathlineto{\pgfqpoint{3.415424in}{1.332529in}}%
\pgfpathlineto{\pgfqpoint{3.354828in}{1.364317in}}%
\pgfpathlineto{\pgfqpoint{3.294562in}{1.396728in}}%
\pgfpathlineto{\pgfqpoint{3.234485in}{1.429486in}}%
\pgfpathlineto{\pgfqpoint{3.174444in}{1.462312in}}%
\pgfpathlineto{\pgfqpoint{3.114280in}{1.494908in}}%
\pgfpathlineto{\pgfqpoint{3.053824in}{1.526957in}}%
\pgfpathlineto{\pgfqpoint{2.992904in}{1.558111in}}%
\pgfpathlineto{\pgfqpoint{2.931349in}{1.587983in}}%
\pgfpathlineto{\pgfqpoint{2.868994in}{1.616138in}}%
\pgfpathlineto{\pgfqpoint{2.805697in}{1.642092in}}%
\pgfpathlineto{\pgfqpoint{2.741353in}{1.665310in}}%
\pgfpathlineto{\pgfqpoint{2.675920in}{1.685228in}}%
\pgfpathlineto{\pgfqpoint{2.609439in}{1.701287in}}%
\pgfpathlineto{\pgfqpoint{2.542063in}{1.713008in}}%
\pgfpathlineto{\pgfqpoint{2.474047in}{1.720089in}}%
\pgfpathlineto{\pgfqpoint{2.405707in}{1.722476in}}%
\pgfpathlineto{\pgfqpoint{2.337348in}{1.720424in}}%
\pgfpathlineto{\pgfqpoint{2.269202in}{1.714507in}}%
\pgfpathlineto{\pgfqpoint{2.201365in}{1.705629in}}%
\pgfpathlineto{\pgfqpoint{2.133762in}{1.695048in}}%
\pgfpathlineto{\pgfqpoint{2.066150in}{1.684539in}}%
\pgfpathlineto{\pgfqpoint{1.998183in}{1.677190in}}%
\pgfpathlineto{\pgfqpoint{1.998183in}{1.677190in}}%
\pgfpathlineto{\pgfqpoint{1.940633in}{1.679026in}}%
\pgfpathlineto{\pgfqpoint{1.940633in}{1.679026in}}%
\pgfpathlineto{\pgfqpoint{1.912763in}{1.687190in}}%
\pgfpathlineto{\pgfqpoint{1.912763in}{1.687190in}}%
\pgfpathlineto{\pgfqpoint{1.895393in}{1.701752in}}%
\pgfusepath{stroke}%
\end{pgfscope}%
\begin{pgfscope}%
\pgfpathrectangle{\pgfqpoint{0.647939in}{0.492442in}}{\pgfqpoint{3.079299in}{3.079299in}}%
\pgfusepath{clip}%
\pgfsetbuttcap%
\pgfsetroundjoin%
\pgfsetlinewidth{0.301125pt}%
\definecolor{currentstroke}{rgb}{0.500000,0.500000,0.500000}%
\pgfsetstrokecolor{currentstroke}%
\pgfsetstrokeopacity{0.300000}%
\pgfsetdash{}{0pt}%
\pgfpathmoveto{\pgfqpoint{3.727238in}{1.262267in}}%
\pgfpathlineto{\pgfqpoint{3.727238in}{1.262267in}}%
\pgfpathlineto{\pgfqpoint{3.663827in}{1.287960in}}%
\pgfpathlineto{\pgfqpoint{3.601273in}{1.315677in}}%
\pgfpathlineto{\pgfqpoint{3.539531in}{1.345167in}}%
\pgfpathlineto{\pgfqpoint{3.478534in}{1.376169in}}%
\pgfpathlineto{\pgfqpoint{3.418188in}{1.408423in}}%
\pgfpathlineto{\pgfqpoint{3.358381in}{1.441668in}}%
\pgfpathlineto{\pgfqpoint{3.298987in}{1.475649in}}%
\pgfpathlineto{\pgfqpoint{3.239870in}{1.510109in}}%
\pgfpathlineto{\pgfqpoint{3.180880in}{1.544786in}}%
\pgfpathlineto{\pgfqpoint{3.121857in}{1.579407in}}%
\pgfpathlineto{\pgfqpoint{3.062633in}{1.613682in}}%
\pgfpathlineto{\pgfqpoint{3.003030in}{1.647290in}}%
\pgfpathlineto{\pgfqpoint{2.942864in}{1.679872in}}%
\pgfpathlineto{\pgfqpoint{2.881945in}{1.711018in}}%
\pgfpathlineto{\pgfqpoint{2.820090in}{1.740250in}}%
\pgfpathlineto{\pgfqpoint{2.757135in}{1.767017in}}%
\pgfpathlineto{\pgfqpoint{2.692960in}{1.790684in}}%
\pgfpathlineto{\pgfqpoint{2.627524in}{1.810552in}}%
\pgfpathlineto{\pgfqpoint{2.560896in}{1.825902in}}%
\pgfpathlineto{\pgfqpoint{2.493296in}{1.836085in}}%
\pgfpathlineto{\pgfqpoint{2.425089in}{1.840654in}}%
\pgfpathlineto{\pgfqpoint{2.356732in}{1.839517in}}%
\pgfpathlineto{\pgfqpoint{2.288661in}{1.833057in}}%
\pgfpathlineto{\pgfqpoint{2.221155in}{1.822096in}}%
\pgfpathlineto{\pgfqpoint{2.154249in}{1.807848in}}%
\pgfpathlineto{\pgfqpoint{2.087681in}{1.792038in}}%
\pgfpathlineto{\pgfqpoint{2.020807in}{1.777710in}}%
\pgfpathlineto{\pgfqpoint{2.020807in}{1.777710in}}%
\pgfpathlineto{\pgfqpoint{1.970533in}{1.772932in}}%
\pgfpathlineto{\pgfqpoint{1.970533in}{1.772932in}}%
\pgfpathlineto{\pgfqpoint{1.949433in}{1.776177in}}%
\pgfpathlineto{\pgfqpoint{1.949433in}{1.776177in}}%
\pgfusepath{stroke}%
\end{pgfscope}%
\begin{pgfscope}%
\pgfpathrectangle{\pgfqpoint{0.647939in}{0.492442in}}{\pgfqpoint{3.079299in}{3.079299in}}%
\pgfusepath{clip}%
\pgfsetbuttcap%
\pgfsetroundjoin%
\pgfsetlinewidth{0.301125pt}%
\definecolor{currentstroke}{rgb}{0.500000,0.500000,0.500000}%
\pgfsetstrokecolor{currentstroke}%
\pgfsetstrokeopacity{0.300000}%
\pgfsetdash{}{0pt}%
\pgfpathmoveto{\pgfqpoint{3.727238in}{1.332251in}}%
\pgfpathlineto{\pgfqpoint{3.727238in}{1.332251in}}%
\pgfpathlineto{\pgfqpoint{3.664280in}{1.359032in}}%
\pgfpathlineto{\pgfqpoint{3.602262in}{1.387930in}}%
\pgfpathlineto{\pgfqpoint{3.541146in}{1.418691in}}%
\pgfpathlineto{\pgfqpoint{3.480863in}{1.451060in}}%
\pgfpathlineto{\pgfqpoint{3.421325in}{1.484781in}}%
\pgfpathlineto{\pgfqpoint{3.362426in}{1.519610in}}%
\pgfpathlineto{\pgfqpoint{3.304047in}{1.555306in}}%
\pgfpathlineto{\pgfqpoint{3.246059in}{1.591634in}}%
\pgfpathlineto{\pgfqpoint{3.188323in}{1.628359in}}%
\pgfpathlineto{\pgfqpoint{3.130687in}{1.665242in}}%
\pgfpathlineto{\pgfqpoint{3.072992in}{1.702032in}}%
\pgfpathlineto{\pgfqpoint{3.015064in}{1.738454in}}%
\pgfpathlineto{\pgfqpoint{2.956719in}{1.774201in}}%
\pgfpathlineto{\pgfqpoint{2.897763in}{1.808922in}}%
\pgfpathlineto{\pgfqpoint{2.837991in}{1.842207in}}%
\pgfpathlineto{\pgfqpoint{2.777189in}{1.873556in}}%
\pgfpathlineto{\pgfqpoint{2.715147in}{1.902352in}}%
\pgfpathlineto{\pgfqpoint{2.651682in}{1.927826in}}%
\pgfpathlineto{\pgfqpoint{2.586683in}{1.949032in}}%
\pgfpathlineto{\pgfqpoint{2.520190in}{1.964844in}}%
\pgfpathlineto{\pgfqpoint{2.452485in}{1.974035in}}%
\pgfpathlineto{\pgfqpoint{2.384200in}{1.975541in}}%
\pgfpathlineto{\pgfqpoint{2.316258in}{1.968818in}}%
\pgfpathlineto{\pgfqpoint{2.249591in}{1.954110in}}%
\pgfpathlineto{\pgfqpoint{2.184811in}{1.932524in}}%
\pgfpathlineto{\pgfqpoint{2.121997in}{1.905785in}}%
\pgfpathlineto{\pgfqpoint{2.060641in}{1.876054in}}%
\pgfusepath{stroke}%
\end{pgfscope}%
\begin{pgfscope}%
\pgfpathrectangle{\pgfqpoint{0.647939in}{0.492442in}}{\pgfqpoint{3.079299in}{3.079299in}}%
\pgfusepath{clip}%
\pgfsetbuttcap%
\pgfsetroundjoin%
\pgfsetlinewidth{0.301125pt}%
\definecolor{currentstroke}{rgb}{0.500000,0.500000,0.500000}%
\pgfsetstrokecolor{currentstroke}%
\pgfsetstrokeopacity{0.300000}%
\pgfsetdash{}{0pt}%
\pgfpathmoveto{\pgfqpoint{3.727238in}{1.402235in}}%
\pgfpathlineto{\pgfqpoint{3.727238in}{1.402235in}}%
\pgfpathlineto{\pgfqpoint{3.664794in}{1.430191in}}%
\pgfpathlineto{\pgfqpoint{3.603387in}{1.460363in}}%
\pgfpathlineto{\pgfqpoint{3.542982in}{1.492497in}}%
\pgfpathlineto{\pgfqpoint{3.483517in}{1.526341in}}%
\pgfpathlineto{\pgfqpoint{3.424910in}{1.561654in}}%
\pgfpathlineto{\pgfqpoint{3.367066in}{1.598204in}}%
\pgfpathlineto{\pgfqpoint{3.309877in}{1.635774in}}%
\pgfpathlineto{\pgfqpoint{3.253228in}{1.674155in}}%
\pgfpathlineto{\pgfqpoint{3.196998in}{1.713149in}}%
\pgfpathlineto{\pgfqpoint{3.141057in}{1.752554in}}%
\pgfpathlineto{\pgfqpoint{3.085265in}{1.792170in}}%
\pgfpathlineto{\pgfqpoint{3.029480in}{1.831795in}}%
\pgfpathlineto{\pgfqpoint{2.973554in}{1.871218in}}%
\pgfpathlineto{\pgfqpoint{2.917325in}{1.910205in}}%
\pgfpathlineto{\pgfqpoint{2.860617in}{1.948490in}}%
\pgfpathlineto{\pgfqpoint{2.803237in}{1.985756in}}%
\pgfpathlineto{\pgfqpoint{2.744974in}{2.021612in}}%
\pgfpathlineto{\pgfqpoint{2.685585in}{2.055547in}}%
\pgfpathlineto{\pgfqpoint{2.624784in}{2.086846in}}%
\pgfpathlineto{\pgfqpoint{2.562234in}{2.114421in}}%
\pgfpathlineto{\pgfqpoint{2.497565in}{2.136401in}}%
\pgfpathlineto{\pgfqpoint{2.430642in}{2.149086in}}%
\pgfpathlineto{\pgfqpoint{2.430642in}{2.149086in}}%
\pgfpathlineto{\pgfqpoint{2.377103in}{2.148138in}}%
\pgfpathlineto{\pgfqpoint{2.377103in}{2.148138in}}%
\pgfpathlineto{\pgfqpoint{2.331965in}{2.136077in}}%
\pgfpathlineto{\pgfqpoint{2.290217in}{2.113924in}}%
\pgfpathlineto{\pgfqpoint{2.250464in}{2.084369in}}%
\pgfpathlineto{\pgfqpoint{2.203095in}{2.042250in}}%
\pgfusepath{stroke}%
\end{pgfscope}%
\begin{pgfscope}%
\pgfpathrectangle{\pgfqpoint{0.647939in}{0.492442in}}{\pgfqpoint{3.079299in}{3.079299in}}%
\pgfusepath{clip}%
\pgfsetbuttcap%
\pgfsetroundjoin%
\pgfsetlinewidth{0.301125pt}%
\definecolor{currentstroke}{rgb}{0.500000,0.500000,0.500000}%
\pgfsetstrokecolor{currentstroke}%
\pgfsetstrokeopacity{0.300000}%
\pgfsetdash{}{0pt}%
\pgfpathmoveto{\pgfqpoint{3.727238in}{1.542203in}}%
\pgfpathlineto{\pgfqpoint{3.727238in}{1.542203in}}%
\pgfpathlineto{\pgfqpoint{3.666053in}{1.572810in}}%
\pgfpathlineto{\pgfqpoint{3.606145in}{1.605849in}}%
\pgfpathlineto{\pgfqpoint{3.547492in}{1.641074in}}%
\pgfpathlineto{\pgfqpoint{3.490055in}{1.678252in}}%
\pgfpathlineto{\pgfqpoint{3.433781in}{1.717169in}}%
\pgfpathlineto{\pgfqpoint{3.378610in}{1.757636in}}%
\pgfpathlineto{\pgfqpoint{3.324483in}{1.799492in}}%
\pgfpathlineto{\pgfqpoint{3.271339in}{1.842591in}}%
\pgfpathlineto{\pgfqpoint{3.219133in}{1.886820in}}%
\pgfpathlineto{\pgfqpoint{3.167842in}{1.932107in}}%
\pgfpathlineto{\pgfqpoint{3.117463in}{1.978406in}}%
\pgfpathlineto{\pgfqpoint{3.068035in}{2.025715in}}%
\pgfpathlineto{\pgfqpoint{3.019661in}{2.074100in}}%
\pgfpathlineto{\pgfqpoint{2.972535in}{2.123696in}}%
\pgfpathlineto{\pgfqpoint{2.927008in}{2.174758in}}%
\pgfpathlineto{\pgfqpoint{2.883697in}{2.227697in}}%
\pgfpathlineto{\pgfqpoint{2.843695in}{2.283139in}}%
\pgfpathlineto{\pgfqpoint{2.808930in}{2.341926in}}%
\pgfpathlineto{\pgfqpoint{2.782648in}{2.404804in}}%
\pgfpathlineto{\pgfqpoint{2.769089in}{2.471368in}}%
\pgfpathlineto{\pgfqpoint{2.770163in}{2.535286in}}%
\pgfpathlineto{\pgfqpoint{2.782632in}{2.595870in}}%
\pgfpathlineto{\pgfqpoint{2.805703in}{2.660029in}}%
\pgfpathlineto{\pgfqpoint{2.835085in}{2.721678in}}%
\pgfpathlineto{\pgfqpoint{2.868476in}{2.781308in}}%
\pgfpathlineto{\pgfqpoint{2.904528in}{2.839395in}}%
\pgfpathlineto{\pgfqpoint{2.942463in}{2.896294in}}%
\pgfpathlineto{\pgfqpoint{2.981827in}{2.952241in}}%
\pgfpathlineto{\pgfqpoint{3.022355in}{3.007360in}}%
\pgfpathlineto{\pgfqpoint{3.063888in}{3.061720in}}%
\pgfpathlineto{\pgfqpoint{3.106343in}{3.115367in}}%
\pgfpathlineto{\pgfqpoint{3.149696in}{3.168299in}}%
\pgfpathlineto{\pgfqpoint{3.193945in}{3.220484in}}%
\pgfpathlineto{\pgfqpoint{3.239113in}{3.271875in}}%
\pgfpathlineto{\pgfqpoint{3.285250in}{3.322401in}}%
\pgfpathlineto{\pgfqpoint{3.332411in}{3.371973in}}%
\pgfpathlineto{\pgfqpoint{3.380668in}{3.420477in}}%
\pgfpathlineto{\pgfqpoint{3.430103in}{3.467779in}}%
\pgfpathlineto{\pgfqpoint{3.480804in}{3.513720in}}%
\pgfpathlineto{\pgfqpoint{3.532864in}{3.558114in}}%
\pgfpathlineto{\pgfqpoint{3.549362in}{3.571741in}}%
\pgfusepath{stroke}%
\end{pgfscope}%
\begin{pgfscope}%
\pgfpathrectangle{\pgfqpoint{0.647939in}{0.492442in}}{\pgfqpoint{3.079299in}{3.079299in}}%
\pgfusepath{clip}%
\pgfsetbuttcap%
\pgfsetroundjoin%
\pgfsetlinewidth{0.301125pt}%
\definecolor{currentstroke}{rgb}{0.500000,0.500000,0.500000}%
\pgfsetstrokecolor{currentstroke}%
\pgfsetstrokeopacity{0.300000}%
\pgfsetdash{}{0pt}%
\pgfpathmoveto{\pgfqpoint{3.727238in}{1.682171in}}%
\pgfpathlineto{\pgfqpoint{3.727238in}{1.682171in}}%
\pgfpathlineto{\pgfqpoint{3.667725in}{1.715904in}}%
\pgfpathlineto{\pgfqpoint{3.609811in}{1.752317in}}%
\pgfpathlineto{\pgfqpoint{3.553507in}{1.791176in}}%
\pgfpathlineto{\pgfqpoint{3.498814in}{1.832275in}}%
\pgfpathlineto{\pgfqpoint{3.445736in}{1.875441in}}%
\pgfpathlineto{\pgfqpoint{3.394279in}{1.920530in}}%
\pgfpathlineto{\pgfqpoint{3.344473in}{1.967434in}}%
\pgfpathlineto{\pgfqpoint{3.296390in}{2.016103in}}%
\pgfpathlineto{\pgfqpoint{3.250148in}{2.066524in}}%
\pgfpathlineto{\pgfqpoint{3.205944in}{2.118734in}}%
\pgfpathlineto{\pgfqpoint{3.164077in}{2.172834in}}%
\pgfpathlineto{\pgfqpoint{3.124993in}{2.228969in}}%
\pgfpathlineto{\pgfqpoint{3.089334in}{2.287321in}}%
\pgfpathlineto{\pgfqpoint{3.057998in}{2.348080in}}%
\pgfpathlineto{\pgfqpoint{3.032159in}{2.411340in}}%
\pgfpathlineto{\pgfqpoint{3.013191in}{2.476952in}}%
\pgfpathlineto{\pgfqpoint{3.002376in}{2.544365in}}%
\pgfpathlineto{\pgfqpoint{3.000442in}{2.612599in}}%
\pgfpathlineto{\pgfqpoint{3.007199in}{2.680528in}}%
\pgfpathlineto{\pgfqpoint{3.021655in}{2.747276in}}%
\pgfpathlineto{\pgfqpoint{3.042465in}{2.812360in}}%
\pgfpathlineto{\pgfqpoint{3.068323in}{2.875642in}}%
\pgfpathlineto{\pgfqpoint{3.098160in}{2.937166in}}%
\pgfpathlineto{\pgfqpoint{3.131172in}{2.997062in}}%
\pgfpathlineto{\pgfqpoint{3.166778in}{3.055464in}}%
\pgfpathlineto{\pgfqpoint{3.204582in}{3.112478in}}%
\pgfpathlineto{\pgfqpoint{3.244318in}{3.168168in}}%
\pgfpathlineto{\pgfqpoint{3.285803in}{3.222572in}}%
\pgfusepath{stroke}%
\end{pgfscope}%
\begin{pgfscope}%
\pgfpathrectangle{\pgfqpoint{0.647939in}{0.492442in}}{\pgfqpoint{3.079299in}{3.079299in}}%
\pgfusepath{clip}%
\pgfsetbuttcap%
\pgfsetroundjoin%
\pgfsetlinewidth{0.301125pt}%
\definecolor{currentstroke}{rgb}{0.500000,0.500000,0.500000}%
\pgfsetstrokecolor{currentstroke}%
\pgfsetstrokeopacity{0.300000}%
\pgfsetdash{}{0pt}%
\pgfpathmoveto{\pgfqpoint{3.727238in}{1.822139in}}%
\pgfpathlineto{\pgfqpoint{3.727238in}{1.822139in}}%
\pgfpathlineto{\pgfqpoint{3.669990in}{1.859577in}}%
\pgfpathlineto{\pgfqpoint{3.614787in}{1.899971in}}%
\pgfpathlineto{\pgfqpoint{3.561695in}{1.943102in}}%
\pgfpathlineto{\pgfqpoint{3.510789in}{1.988797in}}%
\pgfpathlineto{\pgfqpoint{3.462157in}{2.036905in}}%
\pgfpathlineto{\pgfqpoint{3.415933in}{2.087327in}}%
\pgfpathlineto{\pgfqpoint{3.372305in}{2.140013in}}%
\pgfpathlineto{\pgfqpoint{3.331541in}{2.194939in}}%
\pgfpathlineto{\pgfqpoint{3.294022in}{2.252120in}}%
\pgfpathlineto{\pgfqpoint{3.260242in}{2.311575in}}%
\pgfpathlineto{\pgfqpoint{3.230832in}{2.373295in}}%
\pgfpathlineto{\pgfqpoint{3.206544in}{2.437187in}}%
\pgfpathlineto{\pgfqpoint{3.188189in}{2.503010in}}%
\pgfpathlineto{\pgfqpoint{3.176503in}{2.570326in}}%
\pgfpathlineto{\pgfqpoint{3.171963in}{2.638498in}}%
\pgfpathlineto{\pgfqpoint{3.174640in}{2.706771in}}%
\pgfpathlineto{\pgfqpoint{3.184178in}{2.774429in}}%
\pgfpathlineto{\pgfqpoint{3.199898in}{2.840939in}}%
\pgfpathlineto{\pgfqpoint{3.220999in}{2.905960in}}%
\pgfpathlineto{\pgfqpoint{3.246698in}{2.969319in}}%
\pgfpathlineto{\pgfqpoint{3.276310in}{3.030961in}}%
\pgfpathlineto{\pgfqpoint{3.309282in}{3.090881in}}%
\pgfpathlineto{\pgfqpoint{3.345192in}{3.149095in}}%
\pgfpathlineto{\pgfqpoint{3.383728in}{3.205611in}}%
\pgfpathlineto{\pgfqpoint{3.424677in}{3.260411in}}%
\pgfpathlineto{\pgfqpoint{3.467902in}{3.313437in}}%
\pgfpathlineto{\pgfqpoint{3.513305in}{3.364606in}}%
\pgfpathlineto{\pgfqpoint{3.560851in}{3.413792in}}%
\pgfpathlineto{\pgfqpoint{3.610528in}{3.460822in}}%
\pgfpathlineto{\pgfqpoint{3.662339in}{3.505487in}}%
\pgfpathlineto{\pgfqpoint{3.716301in}{3.547523in}}%
\pgfpathlineto{\pgfqpoint{3.727238in}{3.555602in}}%
\pgfusepath{stroke}%
\end{pgfscope}%
\begin{pgfscope}%
\pgfpathrectangle{\pgfqpoint{0.647939in}{0.492442in}}{\pgfqpoint{3.079299in}{3.079299in}}%
\pgfusepath{clip}%
\pgfsetbuttcap%
\pgfsetroundjoin%
\pgfsetlinewidth{0.301125pt}%
\definecolor{currentstroke}{rgb}{0.500000,0.500000,0.500000}%
\pgfsetstrokecolor{currentstroke}%
\pgfsetstrokeopacity{0.300000}%
\pgfsetdash{}{0pt}%
\pgfpathmoveto{\pgfqpoint{3.727238in}{1.962108in}}%
\pgfpathlineto{\pgfqpoint{3.727238in}{1.962108in}}%
\pgfpathlineto{\pgfqpoint{3.673129in}{2.003936in}}%
\pgfpathlineto{\pgfqpoint{3.621684in}{2.049004in}}%
\pgfpathlineto{\pgfqpoint{3.573064in}{2.097107in}}%
\pgfpathlineto{\pgfqpoint{3.527446in}{2.148063in}}%
\pgfpathlineto{\pgfqpoint{3.485061in}{2.201738in}}%
\pgfpathlineto{\pgfqpoint{3.446217in}{2.258023in}}%
\pgfpathlineto{\pgfqpoint{3.411299in}{2.316816in}}%
\pgfpathlineto{\pgfqpoint{3.380783in}{2.377998in}}%
\pgfpathlineto{\pgfqpoint{3.355226in}{2.441401in}}%
\pgfpathlineto{\pgfqpoint{3.335219in}{2.506757in}}%
\pgfpathlineto{\pgfqpoint{3.321312in}{2.573670in}}%
\pgfpathlineto{\pgfqpoint{3.313920in}{2.641610in}}%
\pgfpathlineto{\pgfqpoint{3.313212in}{2.709941in}}%
\pgfpathlineto{\pgfqpoint{3.319068in}{2.778024in}}%
\pgfpathlineto{\pgfqpoint{3.331116in}{2.845298in}}%
\pgfpathlineto{\pgfqpoint{3.348822in}{2.911323in}}%
\pgfpathlineto{\pgfqpoint{3.371586in}{2.975793in}}%
\pgfpathlineto{\pgfqpoint{3.398834in}{3.038512in}}%
\pgfpathlineto{\pgfqpoint{3.430060in}{3.099357in}}%
\pgfpathlineto{\pgfqpoint{3.464852in}{3.158239in}}%
\pgfpathlineto{\pgfqpoint{3.502888in}{3.215082in}}%
\pgfpathlineto{\pgfqpoint{3.543933in}{3.269795in}}%
\pgfpathlineto{\pgfqpoint{3.587831in}{3.322250in}}%
\pgfpathlineto{\pgfqpoint{3.634468in}{3.372287in}}%
\pgfpathlineto{\pgfqpoint{3.683773in}{3.419692in}}%
\pgfpathlineto{\pgfqpoint{3.727238in}{3.459220in}}%
\pgfusepath{stroke}%
\end{pgfscope}%
\begin{pgfscope}%
\pgfpathrectangle{\pgfqpoint{0.647939in}{0.492442in}}{\pgfqpoint{3.079299in}{3.079299in}}%
\pgfusepath{clip}%
\pgfsetbuttcap%
\pgfsetroundjoin%
\pgfsetlinewidth{0.301125pt}%
\definecolor{currentstroke}{rgb}{0.500000,0.500000,0.500000}%
\pgfsetstrokecolor{currentstroke}%
\pgfsetstrokeopacity{0.300000}%
\pgfsetdash{}{0pt}%
\pgfpathmoveto{\pgfqpoint{3.727238in}{2.032092in}}%
\pgfpathlineto{\pgfqpoint{3.727238in}{2.032092in}}%
\pgfpathlineto{\pgfqpoint{3.675148in}{2.076401in}}%
\pgfpathlineto{\pgfqpoint{3.626121in}{2.124077in}}%
\pgfpathlineto{\pgfqpoint{3.580371in}{2.174907in}}%
\pgfpathlineto{\pgfqpoint{3.538150in}{2.228702in}}%
\pgfpathlineto{\pgfqpoint{3.499772in}{2.285296in}}%
\pgfpathlineto{\pgfqpoint{3.465622in}{2.344528in}}%
\pgfpathlineto{\pgfqpoint{3.436164in}{2.406219in}}%
\pgfusepath{stroke}%
\end{pgfscope}%
\begin{pgfscope}%
\pgfpathrectangle{\pgfqpoint{0.647939in}{0.492442in}}{\pgfqpoint{3.079299in}{3.079299in}}%
\pgfusepath{clip}%
\pgfsetbuttcap%
\pgfsetroundjoin%
\pgfsetlinewidth{0.301125pt}%
\definecolor{currentstroke}{rgb}{0.500000,0.500000,0.500000}%
\pgfsetstrokecolor{currentstroke}%
\pgfsetstrokeopacity{0.300000}%
\pgfsetdash{}{0pt}%
\pgfpathmoveto{\pgfqpoint{3.727238in}{2.172060in}}%
\pgfpathlineto{\pgfqpoint{3.727238in}{2.172060in}}%
\pgfpathlineto{\pgfqpoint{3.680431in}{2.221897in}}%
\pgfpathlineto{\pgfqpoint{3.637702in}{2.275265in}}%
\pgfpathlineto{\pgfqpoint{3.599389in}{2.331888in}}%
\pgfpathlineto{\pgfqpoint{3.565885in}{2.391481in}}%
\pgfpathlineto{\pgfqpoint{3.537633in}{2.453730in}}%
\pgfpathlineto{\pgfqpoint{3.515100in}{2.518262in}}%
\pgfpathlineto{\pgfqpoint{3.498719in}{2.584611in}}%
\pgfpathlineto{\pgfqpoint{3.488836in}{2.652224in}}%
\pgfpathlineto{\pgfqpoint{3.485631in}{2.720478in}}%
\pgfpathlineto{\pgfqpoint{3.489086in}{2.788725in}}%
\pgfpathlineto{\pgfqpoint{3.498986in}{2.856352in}}%
\pgfpathlineto{\pgfqpoint{3.514955in}{2.922817in}}%
\pgfpathlineto{\pgfqpoint{3.536543in}{2.987678in}}%
\pgfpathlineto{\pgfqpoint{3.563282in}{3.050598in}}%
\pgfpathlineto{\pgfqpoint{3.594739in}{3.111303in}}%
\pgfpathlineto{\pgfqpoint{3.630541in}{3.169560in}}%
\pgfpathlineto{\pgfqpoint{3.670385in}{3.225135in}}%
\pgfpathlineto{\pgfqpoint{3.714041in}{3.277764in}}%
\pgfpathlineto{\pgfqpoint{3.727238in}{3.292587in}}%
\pgfusepath{stroke}%
\end{pgfscope}%
\begin{pgfscope}%
\pgfpathrectangle{\pgfqpoint{0.647939in}{0.492442in}}{\pgfqpoint{3.079299in}{3.079299in}}%
\pgfusepath{clip}%
\pgfsetbuttcap%
\pgfsetroundjoin%
\pgfsetlinewidth{0.301125pt}%
\definecolor{currentstroke}{rgb}{0.500000,0.500000,0.500000}%
\pgfsetstrokecolor{currentstroke}%
\pgfsetstrokeopacity{0.300000}%
\pgfsetdash{}{0pt}%
\pgfpathmoveto{\pgfqpoint{3.727238in}{2.312028in}}%
\pgfpathlineto{\pgfqpoint{3.727238in}{2.312028in}}%
\pgfpathlineto{\pgfqpoint{3.687968in}{2.367966in}}%
\pgfpathlineto{\pgfqpoint{3.654066in}{2.427307in}}%
\pgfpathlineto{\pgfqpoint{3.625977in}{2.489606in}}%
\pgfpathlineto{\pgfqpoint{3.604139in}{2.554355in}}%
\pgfpathlineto{\pgfqpoint{3.588938in}{2.620966in}}%
\pgfpathlineto{\pgfqpoint{3.580645in}{2.688780in}}%
\pgfpathlineto{\pgfqpoint{3.579370in}{2.757091in}}%
\pgfpathlineto{\pgfqpoint{3.585035in}{2.825190in}}%
\pgfpathlineto{\pgfqpoint{3.597387in}{2.892407in}}%
\pgfpathlineto{\pgfqpoint{3.616046in}{2.958152in}}%
\pgfpathlineto{\pgfqpoint{3.640581in}{3.021944in}}%
\pgfpathlineto{\pgfqpoint{3.670558in}{3.083374in}}%
\pgfpathlineto{\pgfqpoint{3.705589in}{3.142074in}}%
\pgfpathlineto{\pgfqpoint{3.727238in}{3.175176in}}%
\pgfusepath{stroke}%
\end{pgfscope}%
\begin{pgfscope}%
\pgfpathrectangle{\pgfqpoint{0.647939in}{0.492442in}}{\pgfqpoint{3.079299in}{3.079299in}}%
\pgfusepath{clip}%
\pgfsetbuttcap%
\pgfsetroundjoin%
\pgfsetlinewidth{0.301125pt}%
\definecolor{currentstroke}{rgb}{0.500000,0.500000,0.500000}%
\pgfsetstrokecolor{currentstroke}%
\pgfsetstrokeopacity{0.300000}%
\pgfsetdash{}{0pt}%
\pgfpathmoveto{\pgfqpoint{3.727238in}{2.451996in}}%
\pgfpathlineto{\pgfqpoint{3.727238in}{2.451996in}}%
\pgfpathlineto{\pgfqpoint{3.698476in}{2.513979in}}%
\pgfpathlineto{\pgfqpoint{3.676501in}{2.578675in}}%
\pgfpathlineto{\pgfqpoint{3.661682in}{2.645365in}}%
\pgfpathlineto{\pgfqpoint{3.654262in}{2.713268in}}%
\pgfpathlineto{\pgfqpoint{3.654306in}{2.781574in}}%
\pgfpathlineto{\pgfqpoint{3.661696in}{2.849486in}}%
\pgfpathlineto{\pgfqpoint{3.676153in}{2.916267in}}%
\pgfpathlineto{\pgfqpoint{3.697291in}{2.981254in}}%
\pgfpathlineto{\pgfqpoint{3.724677in}{3.043872in}}%
\pgfpathlineto{\pgfqpoint{3.727238in}{3.049042in}}%
\pgfusepath{stroke}%
\end{pgfscope}%
\begin{pgfscope}%
\pgfpathrectangle{\pgfqpoint{0.647939in}{0.492442in}}{\pgfqpoint{3.079299in}{3.079299in}}%
\pgfusepath{clip}%
\pgfsetbuttcap%
\pgfsetroundjoin%
\pgfsetlinewidth{0.301125pt}%
\definecolor{currentstroke}{rgb}{0.500000,0.500000,0.500000}%
\pgfsetstrokecolor{currentstroke}%
\pgfsetstrokeopacity{0.300000}%
\pgfsetdash{}{0pt}%
\pgfpathmoveto{\pgfqpoint{3.727238in}{2.562220in}}%
\pgfpathlineto{\pgfqpoint{3.716207in}{2.604871in}}%
\pgfpathlineto{\pgfqpoint{3.702972in}{2.671885in}}%
\pgfpathlineto{\pgfqpoint{3.697536in}{2.739974in}}%
\pgfpathlineto{\pgfqpoint{3.699911in}{2.808234in}}%
\pgfpathlineto{\pgfqpoint{3.709920in}{2.875801in}}%
\pgfpathlineto{\pgfqpoint{3.727238in}{2.941885in}}%
\pgfpathlineto{\pgfqpoint{3.727238in}{2.941885in}}%
\pgfusepath{stroke}%
\end{pgfscope}%
\begin{pgfscope}%
\pgfpathrectangle{\pgfqpoint{0.647939in}{0.492442in}}{\pgfqpoint{3.079299in}{3.079299in}}%
\pgfusepath{clip}%
\pgfsetbuttcap%
\pgfsetroundjoin%
\pgfsetlinewidth{0.301125pt}%
\definecolor{currentstroke}{rgb}{0.500000,0.500000,0.500000}%
\pgfsetstrokecolor{currentstroke}%
\pgfsetstrokeopacity{0.300000}%
\pgfsetdash{}{0pt}%
\pgfpathmoveto{\pgfqpoint{3.521413in}{3.090484in}}%
\pgfpathlineto{\pgfqpoint{3.555420in}{3.149811in}}%
\pgfpathlineto{\pgfqpoint{3.593229in}{3.206795in}}%
\pgfpathlineto{\pgfqpoint{3.634584in}{3.261268in}}%
\pgfpathlineto{\pgfqpoint{3.679301in}{3.313014in}}%
\pgfpathlineto{\pgfqpoint{3.727238in}{3.361789in}}%
\pgfpathlineto{\pgfqpoint{3.727238in}{3.361789in}}%
\pgfusepath{stroke}%
\end{pgfscope}%
\begin{pgfscope}%
\pgfpathrectangle{\pgfqpoint{0.647939in}{0.492442in}}{\pgfqpoint{3.079299in}{3.079299in}}%
\pgfusepath{clip}%
\pgfsetbuttcap%
\pgfsetroundjoin%
\pgfsetlinewidth{0.301125pt}%
\definecolor{currentstroke}{rgb}{0.500000,0.500000,0.500000}%
\pgfsetstrokecolor{currentstroke}%
\pgfsetstrokeopacity{0.300000}%
\pgfsetdash{}{0pt}%
\pgfpathmoveto{\pgfqpoint{3.362405in}{3.287339in}}%
\pgfpathlineto{\pgfqpoint{3.407208in}{3.339045in}}%
\pgfpathlineto{\pgfqpoint{3.453716in}{3.389221in}}%
\pgfpathlineto{\pgfqpoint{3.501938in}{3.437754in}}%
\pgfpathlineto{\pgfqpoint{3.551906in}{3.484483in}}%
\pgfpathlineto{\pgfqpoint{3.603660in}{3.529224in}}%
\pgfpathlineto{\pgfqpoint{3.657254in}{3.571741in}}%
\pgfpathlineto{\pgfqpoint{3.657254in}{3.571741in}}%
\pgfusepath{stroke}%
\end{pgfscope}%
\begin{pgfscope}%
\pgfpathrectangle{\pgfqpoint{0.647939in}{0.492442in}}{\pgfqpoint{3.079299in}{3.079299in}}%
\pgfusepath{clip}%
\pgfsetbuttcap%
\pgfsetroundjoin%
\pgfsetlinewidth{0.301125pt}%
\definecolor{currentstroke}{rgb}{0.500000,0.500000,0.500000}%
\pgfsetstrokecolor{currentstroke}%
\pgfsetstrokeopacity{0.300000}%
\pgfsetdash{}{0pt}%
\pgfpathmoveto{\pgfqpoint{0.647939in}{2.662712in}}%
\pgfpathlineto{\pgfqpoint{0.706674in}{2.669857in}}%
\pgfpathlineto{\pgfqpoint{0.774514in}{2.678785in}}%
\pgfpathlineto{\pgfqpoint{0.842167in}{2.689036in}}%
\pgfpathlineto{\pgfqpoint{0.909611in}{2.700582in}}%
\pgfpathlineto{\pgfqpoint{0.976833in}{2.713359in}}%
\pgfpathlineto{\pgfqpoint{1.043834in}{2.727252in}}%
\pgfpathlineto{\pgfqpoint{1.110630in}{2.742105in}}%
\pgfpathlineto{\pgfqpoint{1.177254in}{2.757714in}}%
\pgfpathlineto{\pgfqpoint{1.243758in}{2.773829in}}%
\pgfpathlineto{\pgfqpoint{1.310208in}{2.790164in}}%
\pgfpathlineto{\pgfqpoint{1.376682in}{2.806405in}}%
\pgfpathlineto{\pgfqpoint{1.443260in}{2.822208in}}%
\pgfpathlineto{\pgfqpoint{1.510022in}{2.837210in}}%
\pgfpathlineto{\pgfqpoint{1.577035in}{2.851036in}}%
\pgfpathlineto{\pgfqpoint{1.644348in}{2.863312in}}%
\pgfpathlineto{\pgfqpoint{1.711977in}{2.873687in}}%
\pgfpathlineto{\pgfqpoint{1.779904in}{2.881863in}}%
\pgfpathlineto{\pgfqpoint{1.848076in}{2.887617in}}%
\pgfpathlineto{\pgfqpoint{1.916415in}{2.890837in}}%
\pgfpathlineto{\pgfqpoint{1.984829in}{2.891553in}}%
\pgfpathlineto{\pgfqpoint{2.053231in}{2.889978in}}%
\pgfpathlineto{\pgfqpoint{2.121567in}{2.886526in}}%
\pgfpathlineto{\pgfqpoint{2.189832in}{2.881815in}}%
\pgfpathlineto{\pgfqpoint{2.258069in}{2.876705in}}%
\pgfpathlineto{\pgfqpoint{2.326355in}{2.872336in}}%
\pgfpathlineto{\pgfqpoint{2.394731in}{2.870191in}}%
\pgfpathlineto{\pgfqpoint{2.463081in}{2.872052in}}%
\pgfpathlineto{\pgfqpoint{2.530972in}{2.879770in}}%
\pgfpathlineto{\pgfqpoint{2.597592in}{2.894788in}}%
\pgfpathlineto{\pgfqpoint{2.661979in}{2.917550in}}%
\pgfpathlineto{\pgfqpoint{2.723462in}{2.947336in}}%
\pgfpathlineto{\pgfqpoint{2.781926in}{2.982737in}}%
\pgfpathlineto{\pgfqpoint{2.837705in}{3.022275in}}%
\pgfpathlineto{\pgfqpoint{2.891322in}{3.064731in}}%
\pgfpathlineto{\pgfqpoint{2.943311in}{3.109192in}}%
\pgfpathlineto{\pgfqpoint{2.994137in}{3.154982in}}%
\pgfpathlineto{\pgfqpoint{3.044192in}{3.201624in}}%
\pgfpathlineto{\pgfqpoint{3.093803in}{3.248744in}}%
\pgfpathlineto{\pgfqpoint{3.143238in}{3.296047in}}%
\pgfpathlineto{\pgfqpoint{3.192725in}{3.343300in}}%
\pgfpathlineto{\pgfqpoint{3.242465in}{3.390287in}}%
\pgfpathlineto{\pgfqpoint{3.292632in}{3.436818in}}%
\pgfpathlineto{\pgfqpoint{3.343393in}{3.482700in}}%
\pgfpathlineto{\pgfqpoint{3.394905in}{3.527738in}}%
\pgfpathlineto{\pgfqpoint{3.447302in}{3.571741in}}%
\pgfpathlineto{\pgfqpoint{3.447302in}{3.571741in}}%
\pgfusepath{stroke}%
\end{pgfscope}%
\begin{pgfscope}%
\pgfpathrectangle{\pgfqpoint{0.647939in}{0.492442in}}{\pgfqpoint{3.079299in}{3.079299in}}%
\pgfusepath{clip}%
\pgfsetbuttcap%
\pgfsetroundjoin%
\pgfsetlinewidth{0.301125pt}%
\definecolor{currentstroke}{rgb}{0.500000,0.500000,0.500000}%
\pgfsetstrokecolor{currentstroke}%
\pgfsetstrokeopacity{0.300000}%
\pgfsetdash{}{0pt}%
\pgfpathmoveto{\pgfqpoint{0.647939in}{2.941137in}}%
\pgfpathlineto{\pgfqpoint{0.715698in}{2.949115in}}%
\pgfpathlineto{\pgfqpoint{0.783580in}{2.957724in}}%
\pgfpathlineto{\pgfqpoint{0.851298in}{2.967540in}}%
\pgfpathlineto{\pgfqpoint{0.918837in}{2.978522in}}%
\pgfpathlineto{\pgfqpoint{0.986191in}{2.990587in}}%
\pgfpathlineto{\pgfqpoint{1.053368in}{3.003610in}}%
\pgfpathlineto{\pgfqpoint{1.120388in}{3.017419in}}%
\pgfpathlineto{\pgfqpoint{1.187288in}{3.031802in}}%
\pgfpathlineto{\pgfqpoint{1.254116in}{3.046517in}}%
\pgfpathlineto{\pgfqpoint{1.320931in}{3.061289in}}%
\pgfpathlineto{\pgfqpoint{1.387799in}{3.075818in}}%
\pgfpathlineto{\pgfqpoint{1.454787in}{3.089784in}}%
\pgfpathlineto{\pgfqpoint{1.521953in}{3.102857in}}%
\pgfpathlineto{\pgfqpoint{1.589344in}{3.114708in}}%
\pgfpathlineto{\pgfqpoint{1.656983in}{3.125030in}}%
\pgfpathlineto{\pgfqpoint{1.724871in}{3.133553in}}%
\pgfpathlineto{\pgfqpoint{1.792980in}{3.140069in}}%
\pgfpathlineto{\pgfqpoint{1.861259in}{3.144451in}}%
\pgfpathlineto{\pgfqpoint{1.929642in}{3.146690in}}%
\pgfpathlineto{\pgfqpoint{1.998063in}{3.146907in}}%
\pgfpathlineto{\pgfqpoint{2.066469in}{3.145359in}}%
\pgfpathlineto{\pgfqpoint{2.134833in}{3.142451in}}%
\pgfpathlineto{\pgfqpoint{2.203161in}{3.138745in}}%
\pgfpathlineto{\pgfqpoint{2.271486in}{3.134979in}}%
\pgfpathlineto{\pgfqpoint{2.339848in}{3.132052in}}%
\pgfpathlineto{\pgfqpoint{2.408256in}{3.130995in}}%
\pgfpathlineto{\pgfqpoint{2.476630in}{3.132944in}}%
\pgfpathlineto{\pgfqpoint{2.544745in}{3.139031in}}%
\pgfpathlineto{\pgfqpoint{2.612195in}{3.150206in}}%
\pgfpathlineto{\pgfqpoint{2.678455in}{3.167004in}}%
\pgfpathlineto{\pgfqpoint{2.743036in}{3.189403in}}%
\pgfpathlineto{\pgfqpoint{2.805631in}{3.216892in}}%
\pgfpathlineto{\pgfqpoint{2.866177in}{3.248672in}}%
\pgfpathlineto{\pgfqpoint{2.924822in}{3.283869in}}%
\pgfpathlineto{\pgfqpoint{2.981838in}{3.321673in}}%
\pgfpathlineto{\pgfqpoint{3.037550in}{3.361386in}}%
\pgfpathlineto{\pgfqpoint{3.092288in}{3.402433in}}%
\pgfpathlineto{\pgfqpoint{3.146367in}{3.444350in}}%
\pgfpathlineto{\pgfqpoint{3.200065in}{3.486759in}}%
\pgfpathlineto{\pgfqpoint{3.253642in}{3.529322in}}%
\pgfpathlineto{\pgfqpoint{3.307334in}{3.571741in}}%
\pgfpathlineto{\pgfqpoint{3.307334in}{3.571741in}}%
\pgfusepath{stroke}%
\end{pgfscope}%
\begin{pgfscope}%
\pgfpathrectangle{\pgfqpoint{0.647939in}{0.492442in}}{\pgfqpoint{3.079299in}{3.079299in}}%
\pgfusepath{clip}%
\pgfsetbuttcap%
\pgfsetroundjoin%
\pgfsetlinewidth{0.301125pt}%
\definecolor{currentstroke}{rgb}{0.500000,0.500000,0.500000}%
\pgfsetstrokecolor{currentstroke}%
\pgfsetstrokeopacity{0.300000}%
\pgfsetdash{}{0pt}%
\pgfpathmoveto{\pgfqpoint{0.647939in}{3.113223in}}%
\pgfpathlineto{\pgfqpoint{0.649276in}{3.113353in}}%
\pgfpathlineto{\pgfqpoint{0.717322in}{3.120561in}}%
\pgfpathlineto{\pgfqpoint{0.785235in}{3.128921in}}%
\pgfpathlineto{\pgfqpoint{0.852998in}{3.138424in}}%
\pgfpathlineto{\pgfqpoint{0.920599in}{3.149023in}}%
\pgfpathlineto{\pgfqpoint{0.988034in}{3.160630in}}%
\pgfpathlineto{\pgfqpoint{1.055313in}{3.173116in}}%
\pgfpathlineto{\pgfqpoint{1.122457in}{3.186306in}}%
\pgfpathlineto{\pgfqpoint{1.189503in}{3.199992in}}%
\pgfpathlineto{\pgfqpoint{1.256496in}{3.213938in}}%
\pgfpathlineto{\pgfqpoint{1.323490in}{3.227874in}}%
\pgfpathlineto{\pgfqpoint{1.390546in}{3.241514in}}%
\pgfpathlineto{\pgfqpoint{1.457720in}{3.254554in}}%
\pgfpathlineto{\pgfqpoint{1.525062in}{3.266683in}}%
\pgfpathlineto{\pgfqpoint{1.592611in}{3.277602in}}%
\pgfpathlineto{\pgfqpoint{1.660380in}{3.287037in}}%
\pgfpathlineto{\pgfqpoint{1.728365in}{3.294756in}}%
\pgfpathlineto{\pgfqpoint{1.796537in}{3.300590in}}%
\pgfpathlineto{\pgfqpoint{1.864849in}{3.304455in}}%
\pgfpathlineto{\pgfqpoint{1.933243in}{3.306372in}}%
\pgfpathlineto{\pgfqpoint{2.001666in}{3.306481in}}%
\pgfpathlineto{\pgfqpoint{2.070075in}{3.305040in}}%
\pgfpathlineto{\pgfqpoint{2.138453in}{3.302437in}}%
\pgfpathlineto{\pgfqpoint{2.206804in}{3.299198in}}%
\pgfpathlineto{\pgfqpoint{2.275157in}{3.295978in}}%
\pgfpathlineto{\pgfqpoint{2.343540in}{3.293546in}}%
\pgfpathlineto{\pgfqpoint{2.411954in}{3.292761in}}%
\pgfpathlineto{\pgfqpoint{2.480342in}{3.294543in}}%
\pgfpathlineto{\pgfqpoint{2.548538in}{3.299801in}}%
\pgfpathlineto{\pgfqpoint{2.616258in}{3.309310in}}%
\pgfpathlineto{\pgfqpoint{2.683131in}{3.323563in}}%
\pgfpathlineto{\pgfqpoint{2.748777in}{3.342678in}}%
\pgfpathlineto{\pgfqpoint{2.812906in}{3.366405in}}%
\pgfpathlineto{\pgfqpoint{2.875382in}{3.394219in}}%
\pgfpathlineto{\pgfqpoint{2.936231in}{3.425454in}}%
\pgfpathlineto{\pgfqpoint{2.995608in}{3.459422in}}%
\pgfpathlineto{\pgfqpoint{3.053744in}{3.495486in}}%
\pgfpathlineto{\pgfqpoint{3.110905in}{3.533086in}}%
\pgfpathlineto{\pgfqpoint{3.167366in}{3.571741in}}%
\pgfpathlineto{\pgfqpoint{3.167366in}{3.571741in}}%
\pgfusepath{stroke}%
\end{pgfscope}%
\begin{pgfscope}%
\pgfpathrectangle{\pgfqpoint{0.647939in}{0.492442in}}{\pgfqpoint{3.079299in}{3.079299in}}%
\pgfusepath{clip}%
\pgfsetbuttcap%
\pgfsetroundjoin%
\pgfsetlinewidth{0.301125pt}%
\definecolor{currentstroke}{rgb}{0.500000,0.500000,0.500000}%
\pgfsetstrokecolor{currentstroke}%
\pgfsetstrokeopacity{0.300000}%
\pgfsetdash{}{0pt}%
\pgfpathmoveto{\pgfqpoint{0.647939in}{3.232039in}}%
\pgfpathlineto{\pgfqpoint{0.677970in}{3.235115in}}%
\pgfpathlineto{\pgfqpoint{0.745981in}{3.242647in}}%
\pgfpathlineto{\pgfqpoint{0.813859in}{3.251294in}}%
\pgfpathlineto{\pgfqpoint{0.881590in}{3.261025in}}%
\pgfpathlineto{\pgfqpoint{0.949167in}{3.271774in}}%
\pgfpathlineto{\pgfqpoint{1.016593in}{3.283435in}}%
\pgfpathlineto{\pgfqpoint{1.083883in}{3.295860in}}%
\pgfpathlineto{\pgfqpoint{1.151065in}{3.308863in}}%
\pgfpathlineto{\pgfqpoint{1.218177in}{3.322221in}}%
\pgfpathlineto{\pgfqpoint{1.285268in}{3.335688in}}%
\pgfpathlineto{\pgfqpoint{1.352390in}{3.348995in}}%
\pgfpathlineto{\pgfqpoint{1.419599in}{3.361857in}}%
\pgfpathlineto{\pgfqpoint{1.486944in}{3.373979in}}%
\pgfpathlineto{\pgfqpoint{1.554465in}{3.385070in}}%
\pgfpathlineto{\pgfqpoint{1.622186in}{3.394857in}}%
\pgfpathlineto{\pgfqpoint{1.690110in}{3.403102in}}%
\pgfpathlineto{\pgfqpoint{1.758221in}{3.409620in}}%
\pgfpathlineto{\pgfqpoint{1.826482in}{3.414297in}}%
\pgfpathlineto{\pgfqpoint{1.894847in}{3.417105in}}%
\pgfpathlineto{\pgfqpoint{1.963262in}{3.418130in}}%
\pgfpathlineto{\pgfqpoint{2.031684in}{3.417564in}}%
\pgfpathlineto{\pgfqpoint{2.100085in}{3.415715in}}%
\pgfpathlineto{\pgfqpoint{2.168459in}{3.413004in}}%
\pgfpathlineto{\pgfqpoint{2.236821in}{3.409969in}}%
\pgfpathlineto{\pgfqpoint{2.305195in}{3.407259in}}%
\pgfpathlineto{\pgfqpoint{2.373599in}{3.405611in}}%
\pgfpathlineto{\pgfqpoint{2.442017in}{3.405816in}}%
\pgfpathlineto{\pgfqpoint{2.510366in}{3.408684in}}%
\pgfpathlineto{\pgfqpoint{2.578478in}{3.414973in}}%
\pgfpathlineto{\pgfqpoint{2.646087in}{3.425288in}}%
\pgfpathlineto{\pgfqpoint{2.712875in}{3.439974in}}%
\pgfpathlineto{\pgfqpoint{2.778539in}{3.459063in}}%
\pgfpathlineto{\pgfqpoint{2.842858in}{3.482300in}}%
\pgfpathlineto{\pgfqpoint{2.905736in}{3.509215in}}%
\pgfpathlineto{\pgfqpoint{2.967204in}{3.539230in}}%
\pgfpathlineto{\pgfqpoint{3.027398in}{3.571741in}}%
\pgfpathlineto{\pgfqpoint{3.027398in}{3.571741in}}%
\pgfusepath{stroke}%
\end{pgfscope}%
\begin{pgfscope}%
\pgfpathrectangle{\pgfqpoint{0.647939in}{0.492442in}}{\pgfqpoint{3.079299in}{3.079299in}}%
\pgfusepath{clip}%
\pgfsetbuttcap%
\pgfsetroundjoin%
\pgfsetlinewidth{0.301125pt}%
\definecolor{currentstroke}{rgb}{0.500000,0.500000,0.500000}%
\pgfsetstrokecolor{currentstroke}%
\pgfsetstrokeopacity{0.300000}%
\pgfsetdash{}{0pt}%
\pgfpathmoveto{\pgfqpoint{0.647939in}{3.313746in}}%
\pgfpathlineto{\pgfqpoint{0.655418in}{3.314462in}}%
\pgfpathlineto{\pgfqpoint{0.723479in}{3.321523in}}%
\pgfpathlineto{\pgfqpoint{0.791418in}{3.329677in}}%
\pgfpathlineto{\pgfqpoint{0.859219in}{3.338905in}}%
\pgfpathlineto{\pgfqpoint{0.926874in}{3.349152in}}%
\pgfpathlineto{\pgfqpoint{0.994384in}{3.360320in}}%
\pgfpathlineto{\pgfqpoint{1.061760in}{3.372270in}}%
\pgfpathlineto{\pgfqpoint{1.129026in}{3.384831in}}%
\pgfpathlineto{\pgfqpoint{1.196215in}{3.397798in}}%
\pgfpathlineto{\pgfqpoint{1.263370in}{3.410937in}}%
\pgfpathlineto{\pgfqpoint{1.330543in}{3.423990in}}%
\pgfpathlineto{\pgfqpoint{1.397784in}{3.436677in}}%
\pgfpathlineto{\pgfqpoint{1.465145in}{3.448713in}}%
\pgfpathlineto{\pgfqpoint{1.532665in}{3.459813in}}%
\pgfpathlineto{\pgfqpoint{1.600370in}{3.469712in}}%
\pgfpathlineto{\pgfqpoint{1.668269in}{3.478171in}}%
\pgfpathlineto{\pgfqpoint{1.736351in}{3.484997in}}%
\pgfpathlineto{\pgfqpoint{1.804585in}{3.490064in}}%
\pgfpathlineto{\pgfqpoint{1.872930in}{3.493329in}}%
\pgfpathlineto{\pgfqpoint{1.941335in}{3.494845in}}%
\pgfpathlineto{\pgfqpoint{2.009759in}{3.494766in}}%
\pgfpathlineto{\pgfqpoint{2.078171in}{3.493353in}}%
\pgfpathlineto{\pgfqpoint{2.146558in}{3.490990in}}%
\pgfpathlineto{\pgfqpoint{2.214928in}{3.488163in}}%
\pgfpathlineto{\pgfqpoint{2.283303in}{3.485447in}}%
\pgfpathlineto{\pgfqpoint{2.351701in}{3.483501in}}%
\pgfpathlineto{\pgfqpoint{2.420122in}{3.483045in}}%
\pgfpathlineto{\pgfqpoint{2.488514in}{3.484838in}}%
\pgfpathlineto{\pgfqpoint{2.556752in}{3.489609in}}%
\pgfpathlineto{\pgfqpoint{2.624635in}{3.497968in}}%
\pgfpathlineto{\pgfqpoint{2.691899in}{3.510331in}}%
\pgfpathlineto{\pgfqpoint{2.758263in}{3.526858in}}%
\pgfpathlineto{\pgfqpoint{2.823488in}{3.547440in}}%
\pgfpathlineto{\pgfqpoint{2.887429in}{3.571741in}}%
\pgfpathlineto{\pgfqpoint{2.887429in}{3.571741in}}%
\pgfusepath{stroke}%
\end{pgfscope}%
\begin{pgfscope}%
\pgfpathrectangle{\pgfqpoint{0.647939in}{0.492442in}}{\pgfqpoint{3.079299in}{3.079299in}}%
\pgfusepath{clip}%
\pgfsetbuttcap%
\pgfsetroundjoin%
\pgfsetlinewidth{0.301125pt}%
\definecolor{currentstroke}{rgb}{0.500000,0.500000,0.500000}%
\pgfsetstrokecolor{currentstroke}%
\pgfsetstrokeopacity{0.300000}%
\pgfsetdash{}{0pt}%
\pgfpathmoveto{\pgfqpoint{1.248384in}{3.475561in}}%
\pgfpathlineto{\pgfqpoint{1.315601in}{3.488380in}}%
\pgfpathlineto{\pgfqpoint{1.382877in}{3.500886in}}%
\pgfpathlineto{\pgfqpoint{1.450260in}{3.512800in}}%
\pgfpathlineto{\pgfqpoint{1.517789in}{3.523847in}}%
\pgfpathlineto{\pgfqpoint{1.585493in}{3.533762in}}%
\pgfpathlineto{\pgfqpoint{1.653381in}{3.542306in}}%
\pgfpathlineto{\pgfqpoint{1.721447in}{3.549288in}}%
\pgfpathlineto{\pgfqpoint{1.789665in}{3.554574in}}%
\pgfpathlineto{\pgfqpoint{1.857996in}{3.558109in}}%
\pgfpathlineto{\pgfqpoint{1.926395in}{3.559926in}}%
\pgfpathlineto{\pgfqpoint{1.994819in}{3.560165in}}%
\pgfpathlineto{\pgfqpoint{2.063236in}{3.559067in}}%
\pgfpathlineto{\pgfqpoint{2.131632in}{3.556974in}}%
\pgfpathlineto{\pgfqpoint{2.200009in}{3.554323in}}%
\pgfpathlineto{\pgfqpoint{2.268385in}{3.551650in}}%
\pgfpathlineto{\pgfqpoint{2.336781in}{3.549576in}}%
\pgfpathlineto{\pgfqpoint{2.405199in}{3.548779in}}%
\pgfpathlineto{\pgfqpoint{2.473606in}{3.549959in}}%
\pgfpathlineto{\pgfqpoint{2.541909in}{3.553801in}}%
\pgfpathlineto{\pgfqpoint{2.609943in}{3.560911in}}%
\pgfpathlineto{\pgfqpoint{2.677477in}{3.571741in}}%
\pgfpathlineto{\pgfqpoint{2.677477in}{3.571741in}}%
\pgfusepath{stroke}%
\end{pgfscope}%
\begin{pgfscope}%
\pgfpathrectangle{\pgfqpoint{0.647939in}{0.492442in}}{\pgfqpoint{3.079299in}{3.079299in}}%
\pgfusepath{clip}%
\pgfsetbuttcap%
\pgfsetroundjoin%
\pgfsetlinewidth{0.301125pt}%
\definecolor{currentstroke}{rgb}{0.500000,0.500000,0.500000}%
\pgfsetstrokecolor{currentstroke}%
\pgfsetstrokeopacity{0.300000}%
\pgfsetdash{}{0pt}%
\pgfpathmoveto{\pgfqpoint{0.647939in}{3.449462in}}%
\pgfpathlineto{\pgfqpoint{0.674896in}{3.452105in}}%
\pgfpathlineto{\pgfqpoint{0.742942in}{3.459311in}}%
\pgfpathlineto{\pgfqpoint{0.810869in}{3.467569in}}%
\pgfpathlineto{\pgfqpoint{0.878664in}{3.476844in}}%
\pgfpathlineto{\pgfqpoint{0.946324in}{3.487064in}}%
\pgfpathlineto{\pgfqpoint{1.013851in}{3.498124in}}%
\pgfpathlineto{\pgfqpoint{1.081262in}{3.509879in}}%
\pgfpathlineto{\pgfqpoint{1.148582in}{3.522147in}}%
\pgfpathlineto{\pgfqpoint{1.215847in}{3.534713in}}%
\pgfpathlineto{\pgfqpoint{1.283101in}{3.547340in}}%
\pgfpathlineto{\pgfqpoint{1.350391in}{3.559772in}}%
\pgfpathlineto{\pgfqpoint{1.417764in}{3.571741in}}%
\pgfpathlineto{\pgfqpoint{1.417764in}{3.571741in}}%
\pgfusepath{stroke}%
\end{pgfscope}%
\begin{pgfscope}%
\pgfpathrectangle{\pgfqpoint{0.647939in}{0.492442in}}{\pgfqpoint{3.079299in}{3.079299in}}%
\pgfusepath{clip}%
\pgfsetbuttcap%
\pgfsetroundjoin%
\pgfsetlinewidth{0.301125pt}%
\definecolor{currentstroke}{rgb}{0.500000,0.500000,0.500000}%
\pgfsetstrokecolor{currentstroke}%
\pgfsetstrokeopacity{0.300000}%
\pgfsetdash{}{0pt}%
\pgfpathmoveto{\pgfqpoint{0.647939in}{3.526527in}}%
\pgfpathlineto{\pgfqpoint{0.658663in}{3.527526in}}%
\pgfpathlineto{\pgfqpoint{0.726744in}{3.534390in}}%
\pgfpathlineto{\pgfqpoint{0.794714in}{3.542289in}}%
\pgfpathlineto{\pgfqpoint{0.862558in}{3.551196in}}%
\pgfpathlineto{\pgfqpoint{0.930273in}{3.561048in}}%
\pgfpathlineto{\pgfqpoint{0.997859in}{3.571741in}}%
\pgfpathlineto{\pgfqpoint{0.997859in}{3.571741in}}%
\pgfusepath{stroke}%
\end{pgfscope}%
\begin{pgfscope}%
\pgfpathrectangle{\pgfqpoint{0.647939in}{0.492442in}}{\pgfqpoint{3.079299in}{3.079299in}}%
\pgfusepath{clip}%
\pgfsetbuttcap%
\pgfsetroundjoin%
\pgfsetlinewidth{0.301125pt}%
\definecolor{currentstroke}{rgb}{0.500000,0.500000,0.500000}%
\pgfsetstrokecolor{currentstroke}%
\pgfsetstrokeopacity{0.300000}%
\pgfsetdash{}{0pt}%
\pgfpathmoveto{\pgfqpoint{0.647939in}{3.011869in}}%
\pgfpathlineto{\pgfqpoint{0.647939in}{3.011869in}}%
\pgfpathlineto{\pgfqpoint{0.715973in}{3.019184in}}%
\pgfpathlineto{\pgfqpoint{0.783869in}{3.027683in}}%
\pgfpathlineto{\pgfqpoint{0.851607in}{3.037362in}}%
\pgfpathlineto{\pgfqpoint{0.919173in}{3.048178in}}%
\pgfpathlineto{\pgfqpoint{0.986562in}{3.060046in}}%
\pgfpathlineto{\pgfqpoint{1.053783in}{3.072839in}}%
\pgfpathlineto{\pgfqpoint{1.120857in}{3.086384in}}%
\pgfpathlineto{\pgfqpoint{1.187819in}{3.100472in}}%
\pgfpathlineto{\pgfqpoint{1.254718in}{3.114862in}}%
\pgfpathlineto{\pgfqpoint{1.321610in}{3.129283in}}%
\pgfpathlineto{\pgfqpoint{1.388559in}{3.143438in}}%
\pgfpathlineto{\pgfqpoint{1.455626in}{3.157016in}}%
\pgfpathlineto{\pgfqpoint{1.522867in}{3.169694in}}%
\pgfpathlineto{\pgfqpoint{1.590325in}{3.181156in}}%
\pgfpathlineto{\pgfqpoint{1.658021in}{3.191108in}}%
\pgfpathlineto{\pgfqpoint{1.725950in}{3.199297in}}%
\pgfpathlineto{\pgfqpoint{1.794085in}{3.205531in}}%
\pgfpathlineto{\pgfqpoint{1.862379in}{3.209701in}}%
\pgfpathlineto{\pgfqpoint{1.930767in}{3.211811in}}%
\pgfpathlineto{\pgfqpoint{1.999188in}{3.211991in}}%
\pgfpathlineto{\pgfqpoint{2.067596in}{3.210497in}}%
\pgfpathlineto{\pgfqpoint{2.135966in}{3.207730in}}%
\pgfpathlineto{\pgfqpoint{2.204305in}{3.204232in}}%
\pgfpathlineto{\pgfqpoint{2.272643in}{3.200706in}}%
\pgfpathlineto{\pgfqpoint{2.341014in}{3.197993in}}%
\pgfpathlineto{\pgfqpoint{2.409425in}{3.197048in}}%
\pgfpathlineto{\pgfqpoint{2.477806in}{3.198909in}}%
\pgfpathlineto{\pgfqpoint{2.545960in}{3.204605in}}%
\pgfpathlineto{\pgfqpoint{2.613538in}{3.215011in}}%
\pgfpathlineto{\pgfqpoint{2.680086in}{3.230647in}}%
\pgfusepath{stroke}%
\end{pgfscope}%
\begin{pgfscope}%
\pgfpathrectangle{\pgfqpoint{0.647939in}{0.492442in}}{\pgfqpoint{3.079299in}{3.079299in}}%
\pgfusepath{clip}%
\pgfsetbuttcap%
\pgfsetroundjoin%
\pgfsetlinewidth{0.301125pt}%
\definecolor{currentstroke}{rgb}{0.500000,0.500000,0.500000}%
\pgfsetstrokecolor{currentstroke}%
\pgfsetstrokeopacity{0.300000}%
\pgfsetdash{}{0pt}%
\pgfpathmoveto{\pgfqpoint{0.647939in}{2.871901in}}%
\pgfpathlineto{\pgfqpoint{0.647939in}{2.871901in}}%
\pgfpathlineto{\pgfqpoint{0.715953in}{2.879402in}}%
\pgfpathlineto{\pgfqpoint{0.783819in}{2.888134in}}%
\pgfpathlineto{\pgfqpoint{0.851514in}{2.898101in}}%
\pgfpathlineto{\pgfqpoint{0.919024in}{2.909262in}}%
\pgfpathlineto{\pgfqpoint{0.986340in}{2.921539in}}%
\pgfpathlineto{\pgfqpoint{1.053468in}{2.934806in}}%
\pgfpathlineto{\pgfqpoint{1.120431in}{2.948891in}}%
\pgfpathlineto{\pgfqpoint{1.187263in}{2.963583in}}%
\pgfusepath{stroke}%
\end{pgfscope}%
\begin{pgfscope}%
\pgfpathrectangle{\pgfqpoint{0.647939in}{0.492442in}}{\pgfqpoint{3.079299in}{3.079299in}}%
\pgfusepath{clip}%
\pgfsetbuttcap%
\pgfsetroundjoin%
\pgfsetlinewidth{0.301125pt}%
\definecolor{currentstroke}{rgb}{0.500000,0.500000,0.500000}%
\pgfsetstrokecolor{currentstroke}%
\pgfsetstrokeopacity{0.300000}%
\pgfsetdash{}{0pt}%
\pgfpathmoveto{\pgfqpoint{0.647939in}{2.801916in}}%
\pgfpathlineto{\pgfqpoint{0.647939in}{2.801916in}}%
\pgfpathlineto{\pgfqpoint{0.715942in}{2.809514in}}%
\pgfpathlineto{\pgfqpoint{0.783792in}{2.818369in}}%
\pgfpathlineto{\pgfqpoint{0.851465in}{2.828485in}}%
\pgfpathlineto{\pgfqpoint{0.918944in}{2.839827in}}%
\pgfpathlineto{\pgfqpoint{0.986220in}{2.852319in}}%
\pgfpathlineto{\pgfqpoint{1.053299in}{2.865835in}}%
\pgfpathlineto{\pgfqpoint{1.120201in}{2.880206in}}%
\pgfpathlineto{\pgfqpoint{1.186962in}{2.895219in}}%
\pgfpathlineto{\pgfqpoint{1.253633in}{2.910627in}}%
\pgfpathlineto{\pgfqpoint{1.320278in}{2.926149in}}%
\pgfpathlineto{\pgfqpoint{1.386969in}{2.941475in}}%
\pgfpathlineto{\pgfqpoint{1.453777in}{2.956272in}}%
\pgfpathlineto{\pgfqpoint{1.520773in}{2.970190in}}%
\pgfpathlineto{\pgfqpoint{1.588011in}{2.982878in}}%
\pgfpathlineto{\pgfqpoint{1.655524in}{2.993997in}}%
\pgfpathlineto{\pgfqpoint{1.723316in}{3.003244in}}%
\pgfpathlineto{\pgfqpoint{1.791362in}{3.010371in}}%
\pgfpathlineto{\pgfqpoint{1.859608in}{3.015216in}}%
\pgfpathlineto{\pgfqpoint{1.927980in}{3.017733in}}%
\pgfpathlineto{\pgfqpoint{1.996399in}{3.018021in}}%
\pgfpathlineto{\pgfqpoint{2.064800in}{3.016331in}}%
\pgfpathlineto{\pgfqpoint{2.133148in}{3.013085in}}%
\pgfpathlineto{\pgfqpoint{2.201447in}{3.008885in}}%
\pgfpathlineto{\pgfqpoint{2.269738in}{3.004546in}}%
\pgfpathlineto{\pgfqpoint{2.338075in}{3.001105in}}%
\pgfpathlineto{\pgfqpoint{2.406474in}{2.999798in}}%
\pgfpathlineto{\pgfqpoint{2.474828in}{3.002023in}}%
\pgfpathlineto{\pgfqpoint{2.542815in}{3.009192in}}%
\pgfpathlineto{\pgfqpoint{2.609859in}{3.022434in}}%
\pgfpathlineto{\pgfqpoint{2.675251in}{3.042233in}}%
\pgfpathlineto{\pgfqpoint{2.738418in}{3.068285in}}%
\pgfpathlineto{\pgfqpoint{2.799119in}{3.099706in}}%
\pgfpathlineto{\pgfqpoint{2.857448in}{3.135393in}}%
\pgfpathlineto{\pgfqpoint{2.913714in}{3.174288in}}%
\pgfpathlineto{\pgfqpoint{2.968309in}{3.215500in}}%
\pgfpathlineto{\pgfqpoint{3.021644in}{3.258341in}}%
\pgfpathlineto{\pgfqpoint{3.074088in}{3.302283in}}%
\pgfpathlineto{\pgfqpoint{3.125969in}{3.346894in}}%
\pgfusepath{stroke}%
\end{pgfscope}%
\begin{pgfscope}%
\pgfpathrectangle{\pgfqpoint{0.647939in}{0.492442in}}{\pgfqpoint{3.079299in}{3.079299in}}%
\pgfusepath{clip}%
\pgfsetbuttcap%
\pgfsetroundjoin%
\pgfsetlinewidth{0.301125pt}%
\definecolor{currentstroke}{rgb}{0.500000,0.500000,0.500000}%
\pgfsetstrokecolor{currentstroke}%
\pgfsetstrokeopacity{0.300000}%
\pgfsetdash{}{0pt}%
\pgfpathmoveto{\pgfqpoint{0.647939in}{2.731932in}}%
\pgfpathlineto{\pgfqpoint{0.647939in}{2.731932in}}%
\pgfpathlineto{\pgfqpoint{0.715930in}{2.739629in}}%
\pgfpathlineto{\pgfqpoint{0.783764in}{2.748609in}}%
\pgfpathlineto{\pgfqpoint{0.851414in}{2.758879in}}%
\pgfpathlineto{\pgfqpoint{0.918861in}{2.770409in}}%
\pgfpathlineto{\pgfqpoint{0.986095in}{2.783122in}}%
\pgfpathlineto{\pgfqpoint{1.053121in}{2.796898in}}%
\pgfpathlineto{\pgfqpoint{1.119958in}{2.811566in}}%
\pgfpathlineto{\pgfqpoint{1.186643in}{2.826912in}}%
\pgfpathlineto{\pgfqpoint{1.253228in}{2.842690in}}%
\pgfpathlineto{\pgfqpoint{1.319778in}{2.858615in}}%
\pgfpathlineto{\pgfqpoint{1.386368in}{2.874371in}}%
\pgfusepath{stroke}%
\end{pgfscope}%
\begin{pgfscope}%
\pgfpathrectangle{\pgfqpoint{0.647939in}{0.492442in}}{\pgfqpoint{3.079299in}{3.079299in}}%
\pgfusepath{clip}%
\pgfsetbuttcap%
\pgfsetroundjoin%
\pgfsetlinewidth{0.301125pt}%
\definecolor{currentstroke}{rgb}{0.500000,0.500000,0.500000}%
\pgfsetstrokecolor{currentstroke}%
\pgfsetstrokeopacity{0.300000}%
\pgfsetdash{}{0pt}%
\pgfpathmoveto{\pgfqpoint{0.647939in}{2.591964in}}%
\pgfpathlineto{\pgfqpoint{0.647939in}{2.591964in}}%
\pgfpathlineto{\pgfqpoint{0.715907in}{2.599867in}}%
\pgfpathlineto{\pgfqpoint{0.783705in}{2.609107in}}%
\pgfpathlineto{\pgfqpoint{0.851304in}{2.619701in}}%
\pgfpathlineto{\pgfqpoint{0.918683in}{2.631623in}}%
\pgfpathlineto{\pgfqpoint{0.985827in}{2.644804in}}%
\pgfpathlineto{\pgfqpoint{1.052737in}{2.659128in}}%
\pgfpathlineto{\pgfqpoint{1.119433in}{2.674425in}}%
\pgfpathlineto{\pgfqpoint{1.185950in}{2.690483in}}%
\pgfpathlineto{\pgfqpoint{1.252342in}{2.707052in}}%
\pgfpathlineto{\pgfqpoint{1.318679in}{2.723843in}}%
\pgfpathlineto{\pgfqpoint{1.385041in}{2.740531in}}%
\pgfpathlineto{\pgfqpoint{1.451516in}{2.756765in}}%
\pgfpathlineto{\pgfqpoint{1.518186in}{2.772166in}}%
\pgfpathlineto{\pgfqpoint{1.585126in}{2.786344in}}%
\pgfpathlineto{\pgfqpoint{1.652384in}{2.798909in}}%
\pgfpathlineto{\pgfqpoint{1.719979in}{2.809494in}}%
\pgfpathlineto{\pgfqpoint{1.787891in}{2.817782in}}%
\pgfpathlineto{\pgfqpoint{1.856063in}{2.823531in}}%
\pgfpathlineto{\pgfqpoint{1.924408in}{2.826618in}}%
\pgfpathlineto{\pgfqpoint{1.992822in}{2.827080in}}%
\pgfpathlineto{\pgfqpoint{2.061213in}{2.825141in}}%
\pgfpathlineto{\pgfqpoint{2.129523in}{2.821225in}}%
\pgfpathlineto{\pgfqpoint{2.197750in}{2.815996in}}%
\pgfpathlineto{\pgfqpoint{2.265950in}{2.810403in}}%
\pgfpathlineto{\pgfqpoint{2.334215in}{2.805761in}}%
\pgfpathlineto{\pgfqpoint{2.402586in}{2.803797in}}%
\pgfpathlineto{\pgfqpoint{2.470880in}{2.806601in}}%
\pgfpathlineto{\pgfqpoint{2.538473in}{2.816324in}}%
\pgfpathlineto{\pgfqpoint{2.604281in}{2.834413in}}%
\pgfpathlineto{\pgfqpoint{2.667214in}{2.860859in}}%
\pgfpathlineto{\pgfqpoint{2.726758in}{2.894326in}}%
\pgfpathlineto{\pgfqpoint{2.783091in}{2.933008in}}%
\pgfpathlineto{\pgfqpoint{2.836785in}{2.975318in}}%
\pgfpathlineto{\pgfqpoint{2.888486in}{3.020091in}}%
\pgfusepath{stroke}%
\end{pgfscope}%
\begin{pgfscope}%
\pgfpathrectangle{\pgfqpoint{0.647939in}{0.492442in}}{\pgfqpoint{3.079299in}{3.079299in}}%
\pgfusepath{clip}%
\pgfsetbuttcap%
\pgfsetroundjoin%
\pgfsetlinewidth{0.301125pt}%
\definecolor{currentstroke}{rgb}{0.500000,0.500000,0.500000}%
\pgfsetstrokecolor{currentstroke}%
\pgfsetstrokeopacity{0.300000}%
\pgfsetdash{}{0pt}%
\pgfpathmoveto{\pgfqpoint{0.647939in}{2.521980in}}%
\pgfpathlineto{\pgfqpoint{0.647939in}{2.521980in}}%
\pgfpathlineto{\pgfqpoint{0.715894in}{2.529990in}}%
\pgfpathlineto{\pgfqpoint{0.783673in}{2.539366in}}%
\pgfpathlineto{\pgfqpoint{0.851246in}{2.550129in}}%
\pgfpathlineto{\pgfqpoint{0.918587in}{2.562258in}}%
\pgfpathlineto{\pgfqpoint{0.985682in}{2.575685in}}%
\pgfpathlineto{\pgfqpoint{1.052530in}{2.590298in}}%
\pgfpathlineto{\pgfqpoint{1.119147in}{2.605930in}}%
\pgfpathlineto{\pgfqpoint{1.185571in}{2.622368in}}%
\pgfpathlineto{\pgfqpoint{1.251856in}{2.639360in}}%
\pgfpathlineto{\pgfqpoint{1.318073in}{2.656617in}}%
\pgfpathlineto{\pgfqpoint{1.384307in}{2.673810in}}%
\pgfpathlineto{\pgfqpoint{1.450648in}{2.690580in}}%
\pgfpathlineto{\pgfqpoint{1.517188in}{2.706540in}}%
\pgfpathlineto{\pgfqpoint{1.584004in}{2.721287in}}%
\pgfpathlineto{\pgfqpoint{1.651155in}{2.734411in}}%
\pgfpathlineto{\pgfqpoint{1.718665in}{2.745523in}}%
\pgfpathlineto{\pgfqpoint{1.786518in}{2.754276in}}%
\pgfpathlineto{\pgfqpoint{1.854656in}{2.760397in}}%
\pgfpathlineto{\pgfqpoint{1.922988in}{2.763726in}}%
\pgfpathlineto{\pgfqpoint{1.991400in}{2.764269in}}%
\pgfpathlineto{\pgfqpoint{2.059786in}{2.762231in}}%
\pgfpathlineto{\pgfqpoint{2.128078in}{2.758029in}}%
\pgfpathlineto{\pgfqpoint{2.196268in}{2.752342in}}%
\pgfpathlineto{\pgfqpoint{2.264417in}{2.746167in}}%
\pgfpathlineto{\pgfqpoint{2.332639in}{2.740940in}}%
\pgfpathlineto{\pgfqpoint{2.400993in}{2.738629in}}%
\pgfpathlineto{\pgfqpoint{2.469250in}{2.741718in}}%
\pgfpathlineto{\pgfqpoint{2.536584in}{2.752773in}}%
\pgfusepath{stroke}%
\end{pgfscope}%
\begin{pgfscope}%
\pgfpathrectangle{\pgfqpoint{0.647939in}{0.492442in}}{\pgfqpoint{3.079299in}{3.079299in}}%
\pgfusepath{clip}%
\pgfsetbuttcap%
\pgfsetroundjoin%
\pgfsetlinewidth{0.301125pt}%
\definecolor{currentstroke}{rgb}{0.500000,0.500000,0.500000}%
\pgfsetstrokecolor{currentstroke}%
\pgfsetstrokeopacity{0.300000}%
\pgfsetdash{}{0pt}%
\pgfpathmoveto{\pgfqpoint{0.647939in}{2.451996in}}%
\pgfpathlineto{\pgfqpoint{0.647939in}{2.451996in}}%
\pgfpathlineto{\pgfqpoint{0.715881in}{2.460116in}}%
\pgfpathlineto{\pgfqpoint{0.783640in}{2.469632in}}%
\pgfpathlineto{\pgfqpoint{0.851185in}{2.480570in}}%
\pgfpathlineto{\pgfqpoint{0.918487in}{2.492911in}}%
\pgfpathlineto{\pgfqpoint{0.985530in}{2.506594in}}%
\pgfpathlineto{\pgfqpoint{1.052311in}{2.521508in}}%
\pgfpathlineto{\pgfqpoint{1.118846in}{2.537488in}}%
\pgfpathlineto{\pgfqpoint{1.185170in}{2.554323in}}%
\pgfpathlineto{\pgfqpoint{1.251340in}{2.571759in}}%
\pgfpathlineto{\pgfqpoint{1.317427in}{2.589506in}}%
\pgfpathlineto{\pgfqpoint{1.383520in}{2.607232in}}%
\pgfpathlineto{\pgfqpoint{1.449715in}{2.624571in}}%
\pgfpathlineto{\pgfqpoint{1.516108in}{2.641128in}}%
\pgfpathlineto{\pgfqpoint{1.582787in}{2.656485in}}%
\pgfpathlineto{\pgfqpoint{1.649817in}{2.670215in}}%
\pgfpathlineto{\pgfqpoint{1.717229in}{2.681903in}}%
\pgfpathlineto{\pgfqpoint{1.785013in}{2.691169in}}%
\pgfusepath{stroke}%
\end{pgfscope}%
\begin{pgfscope}%
\pgfpathrectangle{\pgfqpoint{0.647939in}{0.492442in}}{\pgfqpoint{3.079299in}{3.079299in}}%
\pgfusepath{clip}%
\pgfsetbuttcap%
\pgfsetroundjoin%
\pgfsetlinewidth{0.301125pt}%
\definecolor{currentstroke}{rgb}{0.500000,0.500000,0.500000}%
\pgfsetstrokecolor{currentstroke}%
\pgfsetstrokeopacity{0.300000}%
\pgfsetdash{}{0pt}%
\pgfpathmoveto{\pgfqpoint{0.647939in}{2.382012in}}%
\pgfpathlineto{\pgfqpoint{0.647939in}{2.382012in}}%
\pgfpathlineto{\pgfqpoint{0.715867in}{2.390245in}}%
\pgfpathlineto{\pgfqpoint{0.783606in}{2.399906in}}%
\pgfpathlineto{\pgfqpoint{0.851120in}{2.411023in}}%
\pgfpathlineto{\pgfqpoint{0.918382in}{2.423586in}}%
\pgfpathlineto{\pgfqpoint{0.985370in}{2.437534in}}%
\pgfpathlineto{\pgfqpoint{1.052080in}{2.452760in}}%
\pgfpathlineto{\pgfqpoint{1.118526in}{2.469103in}}%
\pgfpathlineto{\pgfqpoint{1.184744in}{2.486352in}}%
\pgfpathlineto{\pgfqpoint{1.250789in}{2.504255in}}%
\pgfpathlineto{\pgfqpoint{1.316735in}{2.522517in}}%
\pgfpathlineto{\pgfqpoint{1.382675in}{2.540806in}}%
\pgfusepath{stroke}%
\end{pgfscope}%
\begin{pgfscope}%
\pgfpathrectangle{\pgfqpoint{0.647939in}{0.492442in}}{\pgfqpoint{3.079299in}{3.079299in}}%
\pgfusepath{clip}%
\pgfsetbuttcap%
\pgfsetroundjoin%
\pgfsetlinewidth{0.301125pt}%
\definecolor{currentstroke}{rgb}{0.500000,0.500000,0.500000}%
\pgfsetstrokecolor{currentstroke}%
\pgfsetstrokeopacity{0.300000}%
\pgfsetdash{}{0pt}%
\pgfpathmoveto{\pgfqpoint{0.647939in}{2.312028in}}%
\pgfpathlineto{\pgfqpoint{0.647939in}{2.312028in}}%
\pgfpathlineto{\pgfqpoint{0.715853in}{2.320377in}}%
\pgfpathlineto{\pgfqpoint{0.783570in}{2.330187in}}%
\pgfpathlineto{\pgfqpoint{0.851053in}{2.341490in}}%
\pgfpathlineto{\pgfqpoint{0.918271in}{2.354281in}}%
\pgfpathlineto{\pgfqpoint{0.985201in}{2.368504in}}%
\pgfpathlineto{\pgfqpoint{1.051836in}{2.384056in}}%
\pgfpathlineto{\pgfqpoint{1.118188in}{2.400777in}}%
\pgfusepath{stroke}%
\end{pgfscope}%
\begin{pgfscope}%
\pgfpathrectangle{\pgfqpoint{0.647939in}{0.492442in}}{\pgfqpoint{3.079299in}{3.079299in}}%
\pgfusepath{clip}%
\pgfsetbuttcap%
\pgfsetroundjoin%
\pgfsetlinewidth{0.301125pt}%
\definecolor{currentstroke}{rgb}{0.500000,0.500000,0.500000}%
\pgfsetstrokecolor{currentstroke}%
\pgfsetstrokeopacity{0.300000}%
\pgfsetdash{}{0pt}%
\pgfpathmoveto{\pgfqpoint{0.647939in}{2.242044in}}%
\pgfpathlineto{\pgfqpoint{0.647939in}{2.242044in}}%
\pgfpathlineto{\pgfqpoint{0.715838in}{2.250513in}}%
\pgfpathlineto{\pgfqpoint{0.783532in}{2.260475in}}%
\pgfpathlineto{\pgfqpoint{0.850983in}{2.271971in}}%
\pgfpathlineto{\pgfqpoint{0.918155in}{2.284998in}}%
\pgfpathlineto{\pgfqpoint{0.985023in}{2.299507in}}%
\pgfpathlineto{\pgfqpoint{1.051578in}{2.315397in}}%
\pgfpathlineto{\pgfqpoint{1.117828in}{2.332514in}}%
\pgfpathlineto{\pgfqpoint{1.183808in}{2.350650in}}%
\pgfpathlineto{\pgfqpoint{1.249572in}{2.369556in}}%
\pgfpathlineto{\pgfqpoint{1.315199in}{2.388936in}}%
\pgfpathlineto{\pgfqpoint{1.380786in}{2.408451in}}%
\pgfpathlineto{\pgfqpoint{1.446445in}{2.427721in}}%
\pgfpathlineto{\pgfqpoint{1.512294in}{2.446326in}}%
\pgfpathlineto{\pgfqpoint{1.578447in}{2.463811in}}%
\pgfpathlineto{\pgfqpoint{1.645001in}{2.479690in}}%
\pgfpathlineto{\pgfqpoint{1.712018in}{2.493462in}}%
\pgfpathlineto{\pgfqpoint{1.779510in}{2.504639in}}%
\pgfpathlineto{\pgfqpoint{1.847430in}{2.512772in}}%
\pgfpathlineto{\pgfqpoint{1.915668in}{2.517499in}}%
\pgfpathlineto{\pgfqpoint{1.984060in}{2.518598in}}%
\pgfpathlineto{\pgfqpoint{2.052417in}{2.516064in}}%
\pgfpathlineto{\pgfqpoint{2.120571in}{2.510166in}}%
\pgfpathlineto{\pgfqpoint{2.188436in}{2.501501in}}%
\pgfpathlineto{\pgfqpoint{2.256076in}{2.491158in}}%
\pgfpathlineto{\pgfqpoint{2.323757in}{2.481118in}}%
\pgfpathlineto{\pgfqpoint{2.391846in}{2.475317in}}%
\pgfpathlineto{\pgfqpoint{2.459374in}{2.481401in}}%
\pgfpathlineto{\pgfqpoint{2.459374in}{2.481401in}}%
\pgfpathlineto{\pgfqpoint{2.507267in}{2.497683in}}%
\pgfpathlineto{\pgfqpoint{2.551174in}{2.523754in}}%
\pgfpathlineto{\pgfqpoint{2.594558in}{2.559131in}}%
\pgfpathlineto{\pgfqpoint{2.643053in}{2.607012in}}%
\pgfpathlineto{\pgfqpoint{2.689070in}{2.657488in}}%
\pgfpathlineto{\pgfqpoint{2.733704in}{2.709224in}}%
\pgfpathlineto{\pgfqpoint{2.777613in}{2.761640in}}%
\pgfpathlineto{\pgfqpoint{2.821133in}{2.814385in}}%
\pgfusepath{stroke}%
\end{pgfscope}%
\begin{pgfscope}%
\pgfpathrectangle{\pgfqpoint{0.647939in}{0.492442in}}{\pgfqpoint{3.079299in}{3.079299in}}%
\pgfusepath{clip}%
\pgfsetbuttcap%
\pgfsetroundjoin%
\pgfsetlinewidth{0.301125pt}%
\definecolor{currentstroke}{rgb}{0.500000,0.500000,0.500000}%
\pgfsetstrokecolor{currentstroke}%
\pgfsetstrokeopacity{0.300000}%
\pgfsetdash{}{0pt}%
\pgfpathmoveto{\pgfqpoint{0.647939in}{2.172060in}}%
\pgfpathlineto{\pgfqpoint{0.647939in}{2.172060in}}%
\pgfpathlineto{\pgfqpoint{0.715822in}{2.180652in}}%
\pgfpathlineto{\pgfqpoint{0.783493in}{2.190772in}}%
\pgfpathlineto{\pgfqpoint{0.850909in}{2.202467in}}%
\pgfpathlineto{\pgfqpoint{0.918033in}{2.215739in}}%
\pgfpathlineto{\pgfqpoint{0.984836in}{2.230544in}}%
\pgfpathlineto{\pgfqpoint{1.051304in}{2.246787in}}%
\pgfpathlineto{\pgfqpoint{1.117447in}{2.264317in}}%
\pgfpathlineto{\pgfqpoint{1.183293in}{2.282929in}}%
\pgfpathlineto{\pgfqpoint{1.248900in}{2.302375in}}%
\pgfpathlineto{\pgfqpoint{1.314345in}{2.322360in}}%
\pgfpathlineto{\pgfqpoint{1.379729in}{2.342544in}}%
\pgfpathlineto{\pgfqpoint{1.445170in}{2.362543in}}%
\pgfpathlineto{\pgfqpoint{1.510793in}{2.381931in}}%
\pgfpathlineto{\pgfqpoint{1.576723in}{2.400240in}}%
\pgfpathlineto{\pgfqpoint{1.643068in}{2.416967in}}%
\pgfpathlineto{\pgfqpoint{1.709906in}{2.431581in}}%
\pgfpathlineto{\pgfqpoint{1.777262in}{2.443550in}}%
\pgfpathlineto{\pgfqpoint{1.845094in}{2.452368in}}%
\pgfpathlineto{\pgfqpoint{1.913291in}{2.457600in}}%
\pgfpathlineto{\pgfqpoint{1.981674in}{2.458934in}}%
\pgfpathlineto{\pgfqpoint{2.050018in}{2.456251in}}%
\pgfpathlineto{\pgfqpoint{2.118106in}{2.449704in}}%
\pgfpathlineto{\pgfqpoint{2.185789in}{2.439772in}}%
\pgfpathlineto{\pgfqpoint{2.253082in}{2.427403in}}%
\pgfpathlineto{\pgfqpoint{2.320284in}{2.414539in}}%
\pgfpathlineto{\pgfqpoint{2.388055in}{2.406044in}}%
\pgfpathlineto{\pgfqpoint{2.388055in}{2.406044in}}%
\pgfpathlineto{\pgfqpoint{2.434723in}{2.408557in}}%
\pgfpathlineto{\pgfqpoint{2.434723in}{2.408557in}}%
\pgfusepath{stroke}%
\end{pgfscope}%
\begin{pgfscope}%
\pgfpathrectangle{\pgfqpoint{0.647939in}{0.492442in}}{\pgfqpoint{3.079299in}{3.079299in}}%
\pgfusepath{clip}%
\pgfsetbuttcap%
\pgfsetroundjoin%
\pgfsetlinewidth{0.301125pt}%
\definecolor{currentstroke}{rgb}{0.500000,0.500000,0.500000}%
\pgfsetstrokecolor{currentstroke}%
\pgfsetstrokeopacity{0.300000}%
\pgfsetdash{}{0pt}%
\pgfpathmoveto{\pgfqpoint{0.647939in}{2.102076in}}%
\pgfpathlineto{\pgfqpoint{0.647939in}{2.102076in}}%
\pgfpathlineto{\pgfqpoint{0.715806in}{2.110794in}}%
\pgfpathlineto{\pgfqpoint{0.783452in}{2.121078in}}%
\pgfpathlineto{\pgfqpoint{0.850831in}{2.132978in}}%
\pgfpathlineto{\pgfqpoint{0.917904in}{2.146504in}}%
\pgfpathlineto{\pgfqpoint{0.984637in}{2.161618in}}%
\pgfpathlineto{\pgfqpoint{1.051015in}{2.178229in}}%
\pgfpathlineto{\pgfqpoint{1.117041in}{2.196190in}}%
\pgfpathlineto{\pgfqpoint{1.182744in}{2.215301in}}%
\pgfpathlineto{\pgfqpoint{1.248179in}{2.235316in}}%
\pgfpathlineto{\pgfqpoint{1.313425in}{2.255940in}}%
\pgfpathlineto{\pgfqpoint{1.378586in}{2.276833in}}%
\pgfpathlineto{\pgfqpoint{1.443784in}{2.297609in}}%
\pgfpathlineto{\pgfqpoint{1.509154in}{2.317837in}}%
\pgfpathlineto{\pgfqpoint{1.574829in}{2.337038in}}%
\pgfusepath{stroke}%
\end{pgfscope}%
\begin{pgfscope}%
\pgfpathrectangle{\pgfqpoint{0.647939in}{0.492442in}}{\pgfqpoint{3.079299in}{3.079299in}}%
\pgfusepath{clip}%
\pgfsetbuttcap%
\pgfsetroundjoin%
\pgfsetlinewidth{0.301125pt}%
\definecolor{currentstroke}{rgb}{0.500000,0.500000,0.500000}%
\pgfsetstrokecolor{currentstroke}%
\pgfsetstrokeopacity{0.300000}%
\pgfsetdash{}{0pt}%
\pgfpathmoveto{\pgfqpoint{0.647939in}{2.032092in}}%
\pgfpathlineto{\pgfqpoint{0.647939in}{2.032092in}}%
\pgfpathlineto{\pgfqpoint{0.715788in}{2.040940in}}%
\pgfpathlineto{\pgfqpoint{0.783408in}{2.051392in}}%
\pgfpathlineto{\pgfqpoint{0.850750in}{2.063505in}}%
\pgfpathlineto{\pgfqpoint{0.917768in}{2.077295in}}%
\pgfpathlineto{\pgfqpoint{0.984428in}{2.092730in}}%
\pgfpathlineto{\pgfqpoint{1.050708in}{2.109724in}}%
\pgfpathlineto{\pgfqpoint{1.116609in}{2.128137in}}%
\pgfpathlineto{\pgfqpoint{1.182157in}{2.147772in}}%
\pgfpathlineto{\pgfqpoint{1.247405in}{2.168385in}}%
\pgfpathlineto{\pgfqpoint{1.312434in}{2.189684in}}%
\pgfusepath{stroke}%
\end{pgfscope}%
\begin{pgfscope}%
\pgfpathrectangle{\pgfqpoint{0.647939in}{0.492442in}}{\pgfqpoint{3.079299in}{3.079299in}}%
\pgfusepath{clip}%
\pgfsetbuttcap%
\pgfsetroundjoin%
\pgfsetlinewidth{0.301125pt}%
\definecolor{currentstroke}{rgb}{0.500000,0.500000,0.500000}%
\pgfsetstrokecolor{currentstroke}%
\pgfsetstrokeopacity{0.300000}%
\pgfsetdash{}{0pt}%
\pgfpathmoveto{\pgfqpoint{0.647939in}{1.962108in}}%
\pgfpathlineto{\pgfqpoint{0.647939in}{1.962108in}}%
\pgfpathlineto{\pgfqpoint{0.715771in}{1.971091in}}%
\pgfpathlineto{\pgfqpoint{0.783363in}{1.981716in}}%
\pgfpathlineto{\pgfqpoint{0.850664in}{1.994050in}}%
\pgfpathlineto{\pgfqpoint{0.917625in}{2.008113in}}%
\pgfpathlineto{\pgfqpoint{0.984206in}{2.023881in}}%
\pgfpathlineto{\pgfqpoint{1.050381in}{2.041276in}}%
\pgfpathlineto{\pgfqpoint{1.116148in}{2.060162in}}%
\pgfpathlineto{\pgfqpoint{1.181528in}{2.080347in}}%
\pgfpathlineto{\pgfqpoint{1.246574in}{2.101591in}}%
\pgfpathlineto{\pgfqpoint{1.311364in}{2.123605in}}%
\pgfpathlineto{\pgfqpoint{1.376006in}{2.146051in}}%
\pgfpathlineto{\pgfqpoint{1.440632in}{2.168544in}}%
\pgfpathlineto{\pgfqpoint{1.505393in}{2.190645in}}%
\pgfpathlineto{\pgfqpoint{1.570446in}{2.211861in}}%
\pgfpathlineto{\pgfqpoint{1.635946in}{2.231640in}}%
\pgfpathlineto{\pgfqpoint{1.702026in}{2.249369in}}%
\pgfpathlineto{\pgfqpoint{1.768770in}{2.264373in}}%
\pgfpathlineto{\pgfqpoint{1.836184in}{2.275938in}}%
\pgfpathlineto{\pgfqpoint{1.904167in}{2.283327in}}%
\pgfpathlineto{\pgfqpoint{1.972490in}{2.285807in}}%
\pgfpathlineto{\pgfqpoint{2.040769in}{2.282651in}}%
\pgfpathlineto{\pgfqpoint{2.108424in}{2.273047in}}%
\pgfpathlineto{\pgfqpoint{2.174472in}{2.255745in}}%
\pgfpathlineto{\pgfqpoint{2.236053in}{2.227018in}}%
\pgfpathlineto{\pgfqpoint{2.236053in}{2.227018in}}%
\pgfpathlineto{\pgfqpoint{2.259929in}{2.206471in}}%
\pgfpathlineto{\pgfqpoint{2.259929in}{2.206471in}}%
\pgfpathlineto{\pgfqpoint{2.270905in}{2.185204in}}%
\pgfpathlineto{\pgfqpoint{2.270333in}{2.160797in}}%
\pgfpathlineto{\pgfqpoint{2.261921in}{2.137986in}}%
\pgfusepath{stroke}%
\end{pgfscope}%
\begin{pgfscope}%
\pgfpathrectangle{\pgfqpoint{0.647939in}{0.492442in}}{\pgfqpoint{3.079299in}{3.079299in}}%
\pgfusepath{clip}%
\pgfsetbuttcap%
\pgfsetroundjoin%
\pgfsetlinewidth{0.301125pt}%
\definecolor{currentstroke}{rgb}{0.500000,0.500000,0.500000}%
\pgfsetstrokecolor{currentstroke}%
\pgfsetstrokeopacity{0.300000}%
\pgfsetdash{}{0pt}%
\pgfpathmoveto{\pgfqpoint{0.647939in}{1.892124in}}%
\pgfpathlineto{\pgfqpoint{0.647939in}{1.892124in}}%
\pgfpathlineto{\pgfqpoint{0.715752in}{1.901245in}}%
\pgfpathlineto{\pgfqpoint{0.783316in}{1.912050in}}%
\pgfpathlineto{\pgfqpoint{0.850574in}{1.924612in}}%
\pgfpathlineto{\pgfqpoint{0.917474in}{1.938960in}}%
\pgfpathlineto{\pgfqpoint{0.983971in}{1.955076in}}%
\pgfpathlineto{\pgfqpoint{1.050034in}{1.972888in}}%
\pgfpathlineto{\pgfqpoint{1.115656in}{1.992270in}}%
\pgfpathlineto{\pgfqpoint{1.180855in}{2.013033in}}%
\pgfpathlineto{\pgfqpoint{1.245679in}{2.034942in}}%
\pgfpathlineto{\pgfqpoint{1.310207in}{2.057712in}}%
\pgfpathlineto{\pgfqpoint{1.374547in}{2.081009in}}%
\pgfpathlineto{\pgfqpoint{1.438835in}{2.104449in}}%
\pgfpathlineto{\pgfqpoint{1.503230in}{2.127594in}}%
\pgfpathlineto{\pgfqpoint{1.567902in}{2.149948in}}%
\pgfpathlineto{\pgfqpoint{1.633022in}{2.170947in}}%
\pgfpathlineto{\pgfqpoint{1.698745in}{2.189955in}}%
\pgfpathlineto{\pgfqpoint{1.765183in}{2.206254in}}%
\pgfusepath{stroke}%
\end{pgfscope}%
\begin{pgfscope}%
\pgfpathrectangle{\pgfqpoint{0.647939in}{0.492442in}}{\pgfqpoint{3.079299in}{3.079299in}}%
\pgfusepath{clip}%
\pgfsetbuttcap%
\pgfsetroundjoin%
\pgfsetlinewidth{0.301125pt}%
\definecolor{currentstroke}{rgb}{0.500000,0.500000,0.500000}%
\pgfsetstrokecolor{currentstroke}%
\pgfsetstrokeopacity{0.300000}%
\pgfsetdash{}{0pt}%
\pgfpathmoveto{\pgfqpoint{0.647939in}{1.822139in}}%
\pgfpathlineto{\pgfqpoint{0.647939in}{1.822139in}}%
\pgfpathlineto{\pgfqpoint{0.715732in}{1.831404in}}%
\pgfpathlineto{\pgfqpoint{0.783266in}{1.842394in}}%
\pgfpathlineto{\pgfqpoint{0.850479in}{1.855192in}}%
\pgfpathlineto{\pgfqpoint{0.917315in}{1.869836in}}%
\pgfpathlineto{\pgfqpoint{0.983722in}{1.886314in}}%
\pgfpathlineto{\pgfqpoint{1.049665in}{1.904564in}}%
\pgfpathlineto{\pgfqpoint{1.115131in}{1.924465in}}%
\pgfpathlineto{\pgfqpoint{1.180132in}{1.945837in}}%
\pgfpathlineto{\pgfqpoint{1.244714in}{1.968447in}}%
\pgfpathlineto{\pgfqpoint{1.308953in}{1.992019in}}%
\pgfpathlineto{\pgfqpoint{1.372958in}{2.016221in}}%
\pgfpathlineto{\pgfqpoint{1.436868in}{2.040674in}}%
\pgfpathlineto{\pgfqpoint{1.500848in}{2.064942in}}%
\pgfpathlineto{\pgfqpoint{1.565080in}{2.088530in}}%
\pgfpathlineto{\pgfqpoint{1.629754in}{2.110867in}}%
\pgfpathlineto{\pgfqpoint{1.695048in}{2.131300in}}%
\pgfusepath{stroke}%
\end{pgfscope}%
\begin{pgfscope}%
\pgfpathrectangle{\pgfqpoint{0.647939in}{0.492442in}}{\pgfqpoint{3.079299in}{3.079299in}}%
\pgfusepath{clip}%
\pgfsetbuttcap%
\pgfsetroundjoin%
\pgfsetlinewidth{0.301125pt}%
\definecolor{currentstroke}{rgb}{0.500000,0.500000,0.500000}%
\pgfsetstrokecolor{currentstroke}%
\pgfsetstrokeopacity{0.300000}%
\pgfsetdash{}{0pt}%
\pgfpathmoveto{\pgfqpoint{0.647939in}{1.752155in}}%
\pgfpathlineto{\pgfqpoint{0.647939in}{1.752155in}}%
\pgfpathlineto{\pgfqpoint{0.715712in}{1.761567in}}%
\pgfpathlineto{\pgfqpoint{0.783213in}{1.772749in}}%
\pgfpathlineto{\pgfqpoint{0.850379in}{1.785793in}}%
\pgfpathlineto{\pgfqpoint{0.917146in}{1.800743in}}%
\pgfpathlineto{\pgfqpoint{0.983457in}{1.817600in}}%
\pgfpathlineto{\pgfqpoint{1.049271in}{1.836307in}}%
\pgfpathlineto{\pgfqpoint{1.114568in}{1.856753in}}%
\pgfpathlineto{\pgfqpoint{1.179355in}{1.878765in}}%
\pgfpathlineto{\pgfqpoint{1.243672in}{1.902117in}}%
\pgfpathlineto{\pgfqpoint{1.307593in}{1.926537in}}%
\pgfpathlineto{\pgfqpoint{1.371225in}{1.951702in}}%
\pgfpathlineto{\pgfqpoint{1.434710in}{1.977238in}}%
\pgfusepath{stroke}%
\end{pgfscope}%
\begin{pgfscope}%
\pgfpathrectangle{\pgfqpoint{0.647939in}{0.492442in}}{\pgfqpoint{3.079299in}{3.079299in}}%
\pgfusepath{clip}%
\pgfsetbuttcap%
\pgfsetroundjoin%
\pgfsetlinewidth{0.301125pt}%
\definecolor{currentstroke}{rgb}{0.500000,0.500000,0.500000}%
\pgfsetstrokecolor{currentstroke}%
\pgfsetstrokeopacity{0.300000}%
\pgfsetdash{}{0pt}%
\pgfpathmoveto{\pgfqpoint{0.647939in}{1.682171in}}%
\pgfpathlineto{\pgfqpoint{0.647939in}{1.682171in}}%
\pgfpathlineto{\pgfqpoint{0.715690in}{1.691735in}}%
\pgfpathlineto{\pgfqpoint{0.783158in}{1.703116in}}%
\pgfpathlineto{\pgfqpoint{0.850273in}{1.716414in}}%
\pgfpathlineto{\pgfqpoint{0.916967in}{1.731684in}}%
\pgfpathlineto{\pgfqpoint{0.983176in}{1.748936in}}%
\pgfpathlineto{\pgfqpoint{1.048851in}{1.768121in}}%
\pgfusepath{stroke}%
\end{pgfscope}%
\begin{pgfscope}%
\pgfpathrectangle{\pgfqpoint{0.647939in}{0.492442in}}{\pgfqpoint{3.079299in}{3.079299in}}%
\pgfusepath{clip}%
\pgfsetbuttcap%
\pgfsetroundjoin%
\pgfsetlinewidth{0.301125pt}%
\definecolor{currentstroke}{rgb}{0.500000,0.500000,0.500000}%
\pgfsetstrokecolor{currentstroke}%
\pgfsetstrokeopacity{0.300000}%
\pgfsetdash{}{0pt}%
\pgfpathmoveto{\pgfqpoint{0.647939in}{1.612187in}}%
\pgfpathlineto{\pgfqpoint{0.647939in}{1.612187in}}%
\pgfpathlineto{\pgfqpoint{0.715667in}{1.621907in}}%
\pgfpathlineto{\pgfqpoint{0.783100in}{1.633494in}}%
\pgfpathlineto{\pgfqpoint{0.850162in}{1.647056in}}%
\pgfpathlineto{\pgfqpoint{0.916777in}{1.662660in}}%
\pgfpathlineto{\pgfqpoint{0.982877in}{1.680324in}}%
\pgfpathlineto{\pgfqpoint{1.048402in}{1.700011in}}%
\pgfpathlineto{\pgfqpoint{1.113319in}{1.721628in}}%
\pgfpathlineto{\pgfqpoint{1.177618in}{1.745022in}}%
\pgfpathlineto{\pgfqpoint{1.241325in}{1.769985in}}%
\pgfpathlineto{\pgfqpoint{1.304505in}{1.796258in}}%
\pgfpathlineto{\pgfqpoint{1.367259in}{1.823538in}}%
\pgfpathlineto{\pgfqpoint{1.429726in}{1.851469in}}%
\pgfpathlineto{\pgfqpoint{1.492084in}{1.879644in}}%
\pgfpathlineto{\pgfqpoint{1.554544in}{1.907591in}}%
\pgfpathlineto{\pgfqpoint{1.617344in}{1.934758in}}%
\pgfpathlineto{\pgfqpoint{1.680741in}{1.960489in}}%
\pgfpathlineto{\pgfqpoint{1.744994in}{1.983975in}}%
\pgfpathlineto{\pgfqpoint{1.810334in}{2.004180in}}%
\pgfpathlineto{\pgfqpoint{1.876906in}{2.019693in}}%
\pgfpathlineto{\pgfqpoint{1.944635in}{2.028308in}}%
\pgfpathlineto{\pgfqpoint{2.012549in}{2.025408in}}%
\pgfpathlineto{\pgfqpoint{2.012549in}{2.025408in}}%
\pgfpathlineto{\pgfqpoint{2.047765in}{2.015001in}}%
\pgfpathlineto{\pgfqpoint{2.047765in}{2.015001in}}%
\pgfpathlineto{\pgfqpoint{2.068117in}{2.000021in}}%
\pgfpathlineto{\pgfqpoint{2.068117in}{2.000021in}}%
\pgfpathlineto{\pgfqpoint{2.076482in}{1.981206in}}%
\pgfpathlineto{\pgfqpoint{2.074713in}{1.960454in}}%
\pgfpathlineto{\pgfqpoint{2.066458in}{1.940230in}}%
\pgfusepath{stroke}%
\end{pgfscope}%
\begin{pgfscope}%
\pgfpathrectangle{\pgfqpoint{0.647939in}{0.492442in}}{\pgfqpoint{3.079299in}{3.079299in}}%
\pgfusepath{clip}%
\pgfsetbuttcap%
\pgfsetroundjoin%
\pgfsetlinewidth{0.301125pt}%
\definecolor{currentstroke}{rgb}{0.500000,0.500000,0.500000}%
\pgfsetstrokecolor{currentstroke}%
\pgfsetstrokeopacity{0.300000}%
\pgfsetdash{}{0pt}%
\pgfpathmoveto{\pgfqpoint{0.647939in}{1.472219in}}%
\pgfpathlineto{\pgfqpoint{0.647939in}{1.472219in}}%
\pgfpathlineto{\pgfqpoint{0.715619in}{1.482268in}}%
\pgfpathlineto{\pgfqpoint{0.782974in}{1.494289in}}%
\pgfpathlineto{\pgfqpoint{0.849919in}{1.508411in}}%
\pgfpathlineto{\pgfqpoint{0.916362in}{1.524725in}}%
\pgfpathlineto{\pgfqpoint{0.982217in}{1.543270in}}%
\pgfpathlineto{\pgfqpoint{1.047408in}{1.564034in}}%
\pgfpathlineto{\pgfqpoint{1.111876in}{1.586945in}}%
\pgfpathlineto{\pgfqpoint{1.175593in}{1.611873in}}%
\pgfpathlineto{\pgfqpoint{1.238565in}{1.638634in}}%
\pgfpathlineto{\pgfqpoint{1.300836in}{1.666990in}}%
\pgfpathlineto{\pgfqpoint{1.362494in}{1.696658in}}%
\pgfpathlineto{\pgfqpoint{1.423671in}{1.727311in}}%
\pgfpathlineto{\pgfqpoint{1.484538in}{1.758573in}}%
\pgfpathlineto{\pgfqpoint{1.545313in}{1.790015in}}%
\pgfpathlineto{\pgfqpoint{1.606250in}{1.821139in}}%
\pgfpathlineto{\pgfqpoint{1.667642in}{1.851347in}}%
\pgfpathlineto{\pgfqpoint{1.729815in}{1.879894in}}%
\pgfpathlineto{\pgfqpoint{1.793122in}{1.905788in}}%
\pgfpathlineto{\pgfqpoint{1.857923in}{1.927578in}}%
\pgfpathlineto{\pgfqpoint{1.924515in}{1.942667in}}%
\pgfpathlineto{\pgfqpoint{1.924515in}{1.942667in}}%
\pgfpathlineto{\pgfqpoint{1.980248in}{1.945710in}}%
\pgfpathlineto{\pgfqpoint{1.980248in}{1.945710in}}%
\pgfpathlineto{\pgfqpoint{2.007622in}{1.940285in}}%
\pgfpathlineto{\pgfqpoint{2.007622in}{1.940285in}}%
\pgfusepath{stroke}%
\end{pgfscope}%
\begin{pgfscope}%
\pgfpathrectangle{\pgfqpoint{0.647939in}{0.492442in}}{\pgfqpoint{3.079299in}{3.079299in}}%
\pgfusepath{clip}%
\pgfsetbuttcap%
\pgfsetroundjoin%
\pgfsetlinewidth{0.301125pt}%
\definecolor{currentstroke}{rgb}{0.500000,0.500000,0.500000}%
\pgfsetstrokecolor{currentstroke}%
\pgfsetstrokeopacity{0.300000}%
\pgfsetdash{}{0pt}%
\pgfpathmoveto{\pgfqpoint{0.647939in}{1.402235in}}%
\pgfpathlineto{\pgfqpoint{0.647939in}{1.402235in}}%
\pgfpathlineto{\pgfqpoint{0.715592in}{1.412457in}}%
\pgfpathlineto{\pgfqpoint{0.782906in}{1.424707in}}%
\pgfpathlineto{\pgfqpoint{0.849786in}{1.439127in}}%
\pgfpathlineto{\pgfqpoint{0.916135in}{1.455818in}}%
\pgfpathlineto{\pgfqpoint{0.981854in}{1.474835in}}%
\pgfpathlineto{\pgfqpoint{1.046856in}{1.496177in}}%
\pgfpathlineto{\pgfqpoint{1.111070in}{1.519786in}}%
\pgfpathlineto{\pgfqpoint{1.174455in}{1.545544in}}%
\pgfpathlineto{\pgfqpoint{1.237002in}{1.573276in}}%
\pgfpathlineto{\pgfqpoint{1.298746in}{1.602759in}}%
\pgfpathlineto{\pgfqpoint{1.359762in}{1.633722in}}%
\pgfpathlineto{\pgfqpoint{1.420171in}{1.665856in}}%
\pgfpathlineto{\pgfqpoint{1.480141in}{1.698804in}}%
\pgfpathlineto{\pgfqpoint{1.539882in}{1.732168in}}%
\pgfpathlineto{\pgfqpoint{1.599649in}{1.765485in}}%
\pgfusepath{stroke}%
\end{pgfscope}%
\begin{pgfscope}%
\pgfpathrectangle{\pgfqpoint{0.647939in}{0.492442in}}{\pgfqpoint{3.079299in}{3.079299in}}%
\pgfusepath{clip}%
\pgfsetbuttcap%
\pgfsetroundjoin%
\pgfsetlinewidth{0.301125pt}%
\definecolor{currentstroke}{rgb}{0.500000,0.500000,0.500000}%
\pgfsetstrokecolor{currentstroke}%
\pgfsetstrokeopacity{0.300000}%
\pgfsetdash{}{0pt}%
\pgfpathmoveto{\pgfqpoint{0.647939in}{1.332251in}}%
\pgfpathlineto{\pgfqpoint{0.647939in}{1.332251in}}%
\pgfpathlineto{\pgfqpoint{0.715565in}{1.342652in}}%
\pgfpathlineto{\pgfqpoint{0.782834in}{1.355140in}}%
\pgfpathlineto{\pgfqpoint{0.849646in}{1.369869in}}%
\pgfpathlineto{\pgfqpoint{0.915893in}{1.386954in}}%
\pgfpathlineto{\pgfqpoint{0.981466in}{1.406466in}}%
\pgfpathlineto{\pgfqpoint{1.046263in}{1.428416in}}%
\pgfpathlineto{\pgfqpoint{1.110200in}{1.452760in}}%
\pgfpathlineto{\pgfqpoint{1.173220in}{1.479392in}}%
\pgfusepath{stroke}%
\end{pgfscope}%
\begin{pgfscope}%
\pgfpathrectangle{\pgfqpoint{0.647939in}{0.492442in}}{\pgfqpoint{3.079299in}{3.079299in}}%
\pgfusepath{clip}%
\pgfsetbuttcap%
\pgfsetroundjoin%
\pgfsetlinewidth{0.301125pt}%
\definecolor{currentstroke}{rgb}{0.500000,0.500000,0.500000}%
\pgfsetstrokecolor{currentstroke}%
\pgfsetstrokeopacity{0.300000}%
\pgfsetdash{}{0pt}%
\pgfpathmoveto{\pgfqpoint{0.647939in}{1.262267in}}%
\pgfpathlineto{\pgfqpoint{0.647939in}{1.262267in}}%
\pgfpathlineto{\pgfqpoint{0.715535in}{1.272854in}}%
\pgfpathlineto{\pgfqpoint{0.782757in}{1.285588in}}%
\pgfpathlineto{\pgfqpoint{0.849497in}{1.300639in}}%
\pgfpathlineto{\pgfqpoint{0.915635in}{1.318137in}}%
\pgfpathlineto{\pgfqpoint{0.981049in}{1.338167in}}%
\pgfpathlineto{\pgfqpoint{1.045625in}{1.360756in}}%
\pgfpathlineto{\pgfqpoint{1.109259in}{1.385874in}}%
\pgfpathlineto{\pgfqpoint{1.171879in}{1.413428in}}%
\pgfusepath{stroke}%
\end{pgfscope}%
\begin{pgfscope}%
\pgfpathrectangle{\pgfqpoint{0.647939in}{0.492442in}}{\pgfqpoint{3.079299in}{3.079299in}}%
\pgfusepath{clip}%
\pgfsetbuttcap%
\pgfsetroundjoin%
\pgfsetlinewidth{0.301125pt}%
\definecolor{currentstroke}{rgb}{0.500000,0.500000,0.500000}%
\pgfsetstrokecolor{currentstroke}%
\pgfsetstrokeopacity{0.300000}%
\pgfsetdash{}{0pt}%
\pgfpathmoveto{\pgfqpoint{0.647939in}{1.192283in}}%
\pgfpathlineto{\pgfqpoint{0.647939in}{1.192283in}}%
\pgfpathlineto{\pgfqpoint{0.715504in}{1.203062in}}%
\pgfpathlineto{\pgfqpoint{0.782677in}{1.216053in}}%
\pgfpathlineto{\pgfqpoint{0.849338in}{1.231439in}}%
\pgfpathlineto{\pgfqpoint{0.915359in}{1.249369in}}%
\pgfpathlineto{\pgfqpoint{0.980603in}{1.269942in}}%
\pgfpathlineto{\pgfqpoint{1.044937in}{1.293203in}}%
\pgfpathlineto{\pgfqpoint{1.108241in}{1.319137in}}%
\pgfpathlineto{\pgfqpoint{1.170420in}{1.347664in}}%
\pgfpathlineto{\pgfqpoint{1.231411in}{1.378649in}}%
\pgfpathlineto{\pgfqpoint{1.291195in}{1.411905in}}%
\pgfpathlineto{\pgfqpoint{1.349797in}{1.447203in}}%
\pgfpathlineto{\pgfqpoint{1.407292in}{1.484281in}}%
\pgfpathlineto{\pgfqpoint{1.463804in}{1.522847in}}%
\pgfpathlineto{\pgfqpoint{1.519502in}{1.562581in}}%
\pgfpathlineto{\pgfqpoint{1.574621in}{1.603115in}}%
\pgfpathlineto{\pgfqpoint{1.629449in}{1.644040in}}%
\pgfpathlineto{\pgfqpoint{1.684316in}{1.684902in}}%
\pgfpathlineto{\pgfqpoint{1.739618in}{1.725150in}}%
\pgfpathlineto{\pgfqpoint{1.795861in}{1.764062in}}%
\pgfpathlineto{\pgfqpoint{1.853670in}{1.800565in}}%
\pgfusepath{stroke}%
\end{pgfscope}%
\begin{pgfscope}%
\pgfpathrectangle{\pgfqpoint{0.647939in}{0.492442in}}{\pgfqpoint{3.079299in}{3.079299in}}%
\pgfusepath{clip}%
\pgfsetbuttcap%
\pgfsetroundjoin%
\pgfsetlinewidth{0.301125pt}%
\definecolor{currentstroke}{rgb}{0.500000,0.500000,0.500000}%
\pgfsetstrokecolor{currentstroke}%
\pgfsetstrokeopacity{0.300000}%
\pgfsetdash{}{0pt}%
\pgfpathmoveto{\pgfqpoint{0.647939in}{1.122299in}}%
\pgfpathlineto{\pgfqpoint{0.647939in}{1.122299in}}%
\pgfpathlineto{\pgfqpoint{0.715472in}{1.133277in}}%
\pgfpathlineto{\pgfqpoint{0.782591in}{1.146534in}}%
\pgfpathlineto{\pgfqpoint{0.849170in}{1.162272in}}%
\pgfpathlineto{\pgfqpoint{0.915064in}{1.180653in}}%
\pgfpathlineto{\pgfqpoint{0.980124in}{1.201797in}}%
\pgfpathlineto{\pgfqpoint{1.044195in}{1.225765in}}%
\pgfpathlineto{\pgfqpoint{1.107137in}{1.252557in}}%
\pgfpathlineto{\pgfqpoint{1.168830in}{1.282110in}}%
\pgfpathlineto{\pgfqpoint{1.229190in}{1.314299in}}%
\pgfpathlineto{\pgfqpoint{1.288172in}{1.348948in}}%
\pgfpathlineto{\pgfqpoint{1.345780in}{1.385837in}}%
\pgfpathlineto{\pgfqpoint{1.402069in}{1.424715in}}%
\pgfpathlineto{\pgfqpoint{1.457144in}{1.465303in}}%
\pgfusepath{stroke}%
\end{pgfscope}%
\begin{pgfscope}%
\pgfpathrectangle{\pgfqpoint{0.647939in}{0.492442in}}{\pgfqpoint{3.079299in}{3.079299in}}%
\pgfusepath{clip}%
\pgfsetbuttcap%
\pgfsetroundjoin%
\pgfsetlinewidth{0.301125pt}%
\definecolor{currentstroke}{rgb}{0.500000,0.500000,0.500000}%
\pgfsetstrokecolor{currentstroke}%
\pgfsetstrokeopacity{0.300000}%
\pgfsetdash{}{0pt}%
\pgfpathmoveto{\pgfqpoint{0.647939in}{1.052315in}}%
\pgfpathlineto{\pgfqpoint{0.647939in}{1.052315in}}%
\pgfpathlineto{\pgfqpoint{0.715437in}{1.063499in}}%
\pgfpathlineto{\pgfqpoint{0.782500in}{1.077034in}}%
\pgfpathlineto{\pgfqpoint{0.848990in}{1.093138in}}%
\pgfpathlineto{\pgfqpoint{0.914749in}{1.111992in}}%
\pgfpathlineto{\pgfqpoint{0.979608in}{1.133735in}}%
\pgfusepath{stroke}%
\end{pgfscope}%
\begin{pgfscope}%
\pgfpathrectangle{\pgfqpoint{0.647939in}{0.492442in}}{\pgfqpoint{3.079299in}{3.079299in}}%
\pgfusepath{clip}%
\pgfsetbuttcap%
\pgfsetroundjoin%
\pgfsetlinewidth{0.301125pt}%
\definecolor{currentstroke}{rgb}{0.500000,0.500000,0.500000}%
\pgfsetstrokecolor{currentstroke}%
\pgfsetstrokeopacity{0.300000}%
\pgfsetdash{}{0pt}%
\pgfpathmoveto{\pgfqpoint{0.647939in}{0.982331in}}%
\pgfpathlineto{\pgfqpoint{0.647939in}{0.982331in}}%
\pgfpathlineto{\pgfqpoint{0.715401in}{0.993729in}}%
\pgfpathlineto{\pgfqpoint{0.782404in}{1.007554in}}%
\pgfpathlineto{\pgfqpoint{0.848798in}{1.024039in}}%
\pgfpathlineto{\pgfqpoint{0.914411in}{1.043390in}}%
\pgfpathlineto{\pgfqpoint{0.979053in}{1.065763in}}%
\pgfpathlineto{\pgfqpoint{1.042525in}{1.091259in}}%
\pgfpathlineto{\pgfqpoint{1.104633in}{1.119912in}}%
\pgfpathlineto{\pgfqpoint{1.165204in}{1.151683in}}%
\pgfpathlineto{\pgfqpoint{1.224094in}{1.186467in}}%
\pgfpathlineto{\pgfqpoint{1.281206in}{1.224098in}}%
\pgfpathlineto{\pgfqpoint{1.336497in}{1.264361in}}%
\pgfpathlineto{\pgfqpoint{1.389992in}{1.306992in}}%
\pgfpathlineto{\pgfqpoint{1.441784in}{1.351664in}}%
\pgfpathlineto{\pgfqpoint{1.492036in}{1.398065in}}%
\pgfpathlineto{\pgfqpoint{1.540975in}{1.445847in}}%
\pgfpathlineto{\pgfqpoint{1.588903in}{1.494631in}}%
\pgfpathlineto{\pgfqpoint{1.636183in}{1.544038in}}%
\pgfpathlineto{\pgfqpoint{1.683236in}{1.593634in}}%
\pgfpathlineto{\pgfqpoint{1.730569in}{1.642969in}}%
\pgfusepath{stroke}%
\end{pgfscope}%
\begin{pgfscope}%
\pgfpathrectangle{\pgfqpoint{0.647939in}{0.492442in}}{\pgfqpoint{3.079299in}{3.079299in}}%
\pgfusepath{clip}%
\pgfsetbuttcap%
\pgfsetroundjoin%
\pgfsetlinewidth{0.301125pt}%
\definecolor{currentstroke}{rgb}{0.500000,0.500000,0.500000}%
\pgfsetstrokecolor{currentstroke}%
\pgfsetstrokeopacity{0.300000}%
\pgfsetdash{}{0pt}%
\pgfpathmoveto{\pgfqpoint{0.647939in}{0.912347in}}%
\pgfpathlineto{\pgfqpoint{0.647939in}{0.912347in}}%
\pgfpathlineto{\pgfqpoint{0.715362in}{0.923967in}}%
\pgfpathlineto{\pgfqpoint{0.782301in}{0.938093in}}%
\pgfpathlineto{\pgfqpoint{0.848593in}{0.954979in}}%
\pgfpathlineto{\pgfqpoint{0.914048in}{0.974850in}}%
\pgfpathlineto{\pgfqpoint{0.978453in}{0.997886in}}%
\pgfpathlineto{\pgfqpoint{1.041583in}{1.024208in}}%
\pgfpathlineto{\pgfqpoint{1.103213in}{1.053866in}}%
\pgfpathlineto{\pgfqpoint{1.163135in}{1.086833in}}%
\pgfpathlineto{\pgfqpoint{1.221174in}{1.123008in}}%
\pgfusepath{stroke}%
\end{pgfscope}%
\begin{pgfscope}%
\pgfpathrectangle{\pgfqpoint{0.647939in}{0.492442in}}{\pgfqpoint{3.079299in}{3.079299in}}%
\pgfusepath{clip}%
\pgfsetbuttcap%
\pgfsetroundjoin%
\pgfsetlinewidth{0.301125pt}%
\definecolor{currentstroke}{rgb}{0.500000,0.500000,0.500000}%
\pgfsetstrokecolor{currentstroke}%
\pgfsetstrokeopacity{0.300000}%
\pgfsetdash{}{0pt}%
\pgfpathmoveto{\pgfqpoint{0.647939in}{0.842362in}}%
\pgfpathlineto{\pgfqpoint{0.647939in}{0.842362in}}%
\pgfpathlineto{\pgfqpoint{0.715321in}{0.854214in}}%
\pgfpathlineto{\pgfqpoint{0.782192in}{0.868654in}}%
\pgfpathlineto{\pgfqpoint{0.848374in}{0.885959in}}%
\pgfpathlineto{\pgfqpoint{0.913658in}{0.906376in}}%
\pgfpathlineto{\pgfqpoint{0.977806in}{0.930110in}}%
\pgfpathlineto{\pgfqpoint{1.040561in}{0.957302in}}%
\pgfpathlineto{\pgfqpoint{1.101665in}{0.988018in}}%
\pgfusepath{stroke}%
\end{pgfscope}%
\begin{pgfscope}%
\pgfpathrectangle{\pgfqpoint{0.647939in}{0.492442in}}{\pgfqpoint{3.079299in}{3.079299in}}%
\pgfusepath{clip}%
\pgfsetbuttcap%
\pgfsetroundjoin%
\pgfsetlinewidth{0.301125pt}%
\definecolor{currentstroke}{rgb}{0.500000,0.500000,0.500000}%
\pgfsetstrokecolor{currentstroke}%
\pgfsetstrokeopacity{0.300000}%
\pgfsetdash{}{0pt}%
\pgfpathmoveto{\pgfqpoint{0.647939in}{0.772378in}}%
\pgfpathlineto{\pgfqpoint{0.647939in}{0.772378in}}%
\pgfpathlineto{\pgfqpoint{0.715278in}{0.784471in}}%
\pgfpathlineto{\pgfqpoint{0.782075in}{0.799239in}}%
\pgfpathlineto{\pgfqpoint{0.848139in}{0.816982in}}%
\pgfpathlineto{\pgfqpoint{0.913238in}{0.837974in}}%
\pgfpathlineto{\pgfqpoint{0.977105in}{0.862442in}}%
\pgfpathlineto{\pgfqpoint{1.039450in}{0.890551in}}%
\pgfpathlineto{\pgfqpoint{1.099973in}{0.922380in}}%
\pgfpathlineto{\pgfqpoint{1.158391in}{0.957916in}}%
\pgfpathlineto{\pgfqpoint{1.214462in}{0.997049in}}%
\pgfpathlineto{\pgfqpoint{1.268012in}{1.039577in}}%
\pgfpathlineto{\pgfqpoint{1.318974in}{1.085168in}}%
\pgfpathlineto{\pgfqpoint{1.367383in}{1.133452in}}%
\pgfpathlineto{\pgfqpoint{1.413399in}{1.184024in}}%
\pgfpathlineto{\pgfqpoint{1.457292in}{1.236458in}}%
\pgfpathlineto{\pgfqpoint{1.499425in}{1.290312in}}%
\pgfusepath{stroke}%
\end{pgfscope}%
\begin{pgfscope}%
\pgfpathrectangle{\pgfqpoint{0.647939in}{0.492442in}}{\pgfqpoint{3.079299in}{3.079299in}}%
\pgfusepath{clip}%
\pgfsetbuttcap%
\pgfsetroundjoin%
\pgfsetlinewidth{0.301125pt}%
\definecolor{currentstroke}{rgb}{0.500000,0.500000,0.500000}%
\pgfsetstrokecolor{currentstroke}%
\pgfsetstrokeopacity{0.300000}%
\pgfsetdash{}{0pt}%
\pgfpathmoveto{\pgfqpoint{0.647939in}{0.702394in}}%
\pgfpathlineto{\pgfqpoint{0.647939in}{0.702394in}}%
\pgfpathlineto{\pgfqpoint{0.715232in}{0.714737in}}%
\pgfpathlineto{\pgfqpoint{0.781951in}{0.729847in}}%
\pgfpathlineto{\pgfqpoint{0.847888in}{0.748050in}}%
\pgfpathlineto{\pgfqpoint{0.912785in}{0.769646in}}%
\pgfpathlineto{\pgfqpoint{0.976345in}{0.794889in}}%
\pgfpathlineto{\pgfqpoint{1.038239in}{0.823963in}}%
\pgfpathlineto{\pgfqpoint{1.098124in}{0.856962in}}%
\pgfpathlineto{\pgfqpoint{1.155675in}{0.893870in}}%
\pgfpathlineto{\pgfqpoint{1.210618in}{0.934562in}}%
\pgfpathlineto{\pgfqpoint{1.262769in}{0.978779in}}%
\pgfusepath{stroke}%
\end{pgfscope}%
\begin{pgfscope}%
\pgfpathrectangle{\pgfqpoint{0.647939in}{0.492442in}}{\pgfqpoint{3.079299in}{3.079299in}}%
\pgfusepath{clip}%
\pgfsetbuttcap%
\pgfsetroundjoin%
\pgfsetlinewidth{0.301125pt}%
\definecolor{currentstroke}{rgb}{0.500000,0.500000,0.500000}%
\pgfsetstrokecolor{currentstroke}%
\pgfsetstrokeopacity{0.300000}%
\pgfsetdash{}{0pt}%
\pgfpathmoveto{\pgfqpoint{0.647939in}{0.632410in}}%
\pgfpathlineto{\pgfqpoint{0.647939in}{0.632410in}}%
\pgfpathlineto{\pgfqpoint{0.715182in}{0.645013in}}%
\pgfpathlineto{\pgfqpoint{0.781818in}{0.660482in}}%
\pgfpathlineto{\pgfqpoint{0.847617in}{0.679167in}}%
\pgfpathlineto{\pgfqpoint{0.912296in}{0.701399in}}%
\pgfpathlineto{\pgfqpoint{0.975520in}{0.727457in}}%
\pgfpathlineto{\pgfqpoint{1.036918in}{0.757548in}}%
\pgfpathlineto{\pgfqpoint{1.096101in}{0.791774in}}%
\pgfpathlineto{\pgfqpoint{1.152701in}{0.830112in}}%
\pgfpathlineto{\pgfqpoint{1.206416in}{0.872410in}}%
\pgfpathlineto{\pgfqpoint{1.257063in}{0.918325in}}%
\pgfusepath{stroke}%
\end{pgfscope}%
\begin{pgfscope}%
\pgfpathrectangle{\pgfqpoint{0.647939in}{0.492442in}}{\pgfqpoint{3.079299in}{3.079299in}}%
\pgfusepath{clip}%
\pgfsetbuttcap%
\pgfsetroundjoin%
\pgfsetlinewidth{0.301125pt}%
\definecolor{currentstroke}{rgb}{0.500000,0.500000,0.500000}%
\pgfsetstrokecolor{currentstroke}%
\pgfsetstrokeopacity{0.300000}%
\pgfsetdash{}{0pt}%
\pgfpathmoveto{\pgfqpoint{0.647939in}{0.562426in}}%
\pgfpathlineto{\pgfqpoint{0.647939in}{0.562426in}}%
\pgfpathlineto{\pgfqpoint{0.715130in}{0.575300in}}%
\pgfpathlineto{\pgfqpoint{0.781676in}{0.591144in}}%
\pgfpathlineto{\pgfqpoint{0.847326in}{0.610336in}}%
\pgfpathlineto{\pgfqpoint{0.911766in}{0.633237in}}%
\pgfpathlineto{\pgfqpoint{0.974623in}{0.660155in}}%
\pgfpathlineto{\pgfqpoint{1.035477in}{0.691317in}}%
\pgfpathlineto{\pgfqpoint{1.093887in}{0.726828in}}%
\pgfpathlineto{\pgfqpoint{1.149445in}{0.766648in}}%
\pgfusepath{stroke}%
\end{pgfscope}%
\begin{pgfscope}%
\pgfpathrectangle{\pgfqpoint{0.647939in}{0.492442in}}{\pgfqpoint{3.079299in}{3.079299in}}%
\pgfusepath{clip}%
\pgfsetbuttcap%
\pgfsetroundjoin%
\pgfsetlinewidth{0.301125pt}%
\definecolor{currentstroke}{rgb}{0.500000,0.500000,0.500000}%
\pgfsetstrokecolor{currentstroke}%
\pgfsetstrokeopacity{0.300000}%
\pgfsetdash{}{0pt}%
\pgfpathmoveto{\pgfqpoint{0.647939in}{3.385273in}}%
\pgfpathlineto{\pgfqpoint{0.658835in}{3.386312in}}%
\pgfpathlineto{\pgfqpoint{0.726899in}{3.393342in}}%
\pgfpathlineto{\pgfqpoint{0.794844in}{3.401445in}}%
\pgfpathlineto{\pgfqpoint{0.862656in}{3.410598in}}%
\pgfpathlineto{\pgfqpoint{0.930327in}{3.420742in}}%
\pgfpathlineto{\pgfqpoint{0.997859in}{3.431773in}}%
\pgfpathlineto{\pgfqpoint{1.065266in}{3.443554in}}%
\pgfpathlineto{\pgfqpoint{1.132569in}{3.455912in}}%
\pgfusepath{stroke}%
\end{pgfscope}%
\begin{pgfscope}%
\pgfpathrectangle{\pgfqpoint{0.647939in}{0.492442in}}{\pgfqpoint{3.079299in}{3.079299in}}%
\pgfusepath{clip}%
\pgfsetbuttcap%
\pgfsetroundjoin%
\pgfsetlinewidth{0.301125pt}%
\definecolor{currentstroke}{rgb}{0.500000,0.500000,0.500000}%
\pgfsetstrokecolor{currentstroke}%
\pgfsetstrokeopacity{0.300000}%
\pgfsetdash{}{0pt}%
\pgfpathmoveto{\pgfqpoint{3.292341in}{0.638131in}}%
\pgfpathlineto{\pgfqpoint{3.227528in}{0.660080in}}%
\pgfpathlineto{\pgfqpoint{3.162573in}{0.681601in}}%
\pgfpathlineto{\pgfqpoint{3.097382in}{0.702394in}}%
\pgfpathlineto{\pgfqpoint{3.031873in}{0.722159in}}%
\pgfpathlineto{\pgfqpoint{2.965980in}{0.740596in}}%
\pgfpathlineto{\pgfqpoint{2.899660in}{0.757422in}}%
\pgfpathlineto{\pgfqpoint{2.832894in}{0.772373in}}%
\pgfpathlineto{\pgfqpoint{2.765693in}{0.785226in}}%
\pgfpathlineto{\pgfqpoint{2.698099in}{0.795814in}}%
\pgfpathlineto{\pgfqpoint{2.630179in}{0.804043in}}%
\pgfpathlineto{\pgfqpoint{2.562014in}{0.809909in}}%
\pgfpathlineto{\pgfqpoint{2.493690in}{0.813502in}}%
\pgfpathlineto{\pgfqpoint{2.425286in}{0.815010in}}%
\pgfpathlineto{\pgfqpoint{2.356863in}{0.814716in}}%
\pgfpathlineto{\pgfqpoint{2.288459in}{0.813003in}}%
\pgfpathlineto{\pgfqpoint{2.220083in}{0.810330in}}%
\pgfpathlineto{\pgfqpoint{2.151725in}{0.807229in}}%
\pgfpathlineto{\pgfqpoint{2.083358in}{0.804314in}}%
\pgfpathlineto{\pgfqpoint{2.014962in}{0.802307in}}%
\pgfpathlineto{\pgfqpoint{1.946544in}{0.802081in}}%
\pgfpathlineto{\pgfqpoint{1.878193in}{0.804723in}}%
\pgfusepath{stroke}%
\end{pgfscope}%
\begin{pgfscope}%
\pgfpathrectangle{\pgfqpoint{0.647939in}{0.492442in}}{\pgfqpoint{3.079299in}{3.079299in}}%
\pgfusepath{clip}%
\pgfsetbuttcap%
\pgfsetroundjoin%
\pgfsetlinewidth{0.301125pt}%
\definecolor{currentstroke}{rgb}{0.500000,0.500000,0.500000}%
\pgfsetstrokecolor{currentstroke}%
\pgfsetstrokeopacity{0.300000}%
\pgfsetdash{}{0pt}%
\pgfpathmoveto{\pgfqpoint{3.517286in}{1.892124in}}%
\pgfpathlineto{\pgfqpoint{3.465562in}{1.936898in}}%
\pgfpathlineto{\pgfqpoint{3.415764in}{1.983806in}}%
\pgfpathlineto{\pgfqpoint{3.367975in}{2.032758in}}%
\pgfpathlineto{\pgfqpoint{3.322328in}{2.083712in}}%
\pgfpathlineto{\pgfqpoint{3.279033in}{2.136675in}}%
\pgfpathlineto{\pgfqpoint{3.238391in}{2.191693in}}%
\pgfpathlineto{\pgfqpoint{3.200841in}{2.248855in}}%
\pgfpathlineto{\pgfqpoint{3.166974in}{2.308260in}}%
\pgfpathlineto{\pgfqpoint{3.137571in}{2.369978in}}%
\pgfpathlineto{\pgfqpoint{3.113588in}{2.433974in}}%
\pgfpathlineto{\pgfqpoint{3.096071in}{2.500005in}}%
\pgfpathlineto{\pgfqpoint{3.085939in}{2.567544in}}%
\pgfpathlineto{\pgfqpoint{3.083700in}{2.635801in}}%
\pgfpathlineto{\pgfqpoint{3.089257in}{2.703881in}}%
\pgfpathlineto{\pgfqpoint{3.101945in}{2.771015in}}%
\pgfpathlineto{\pgfqpoint{3.120780in}{2.836710in}}%
\pgfusepath{stroke}%
\end{pgfscope}%
\begin{pgfscope}%
\pgfpathrectangle{\pgfqpoint{0.647939in}{0.492442in}}{\pgfqpoint{3.079299in}{3.079299in}}%
\pgfusepath{clip}%
\pgfsetbuttcap%
\pgfsetroundjoin%
\pgfsetlinewidth{0.301125pt}%
\definecolor{currentstroke}{rgb}{0.500000,0.500000,0.500000}%
\pgfsetstrokecolor{currentstroke}%
\pgfsetstrokeopacity{0.300000}%
\pgfsetdash{}{0pt}%
\pgfpathmoveto{\pgfqpoint{1.722327in}{3.334872in}}%
\pgfpathlineto{\pgfqpoint{1.790495in}{3.340747in}}%
\pgfpathlineto{\pgfqpoint{1.858802in}{3.344693in}}%
\pgfpathlineto{\pgfqpoint{1.927194in}{3.346725in}}%
\pgfpathlineto{\pgfqpoint{1.995617in}{3.346975in}}%
\pgfpathlineto{\pgfqpoint{2.064030in}{3.345696in}}%
\pgfpathlineto{\pgfqpoint{2.132413in}{3.343258in}}%
\pgfpathlineto{\pgfqpoint{2.200771in}{3.340156in}}%
\pgfpathlineto{\pgfqpoint{2.269128in}{3.337010in}}%
\pgfpathlineto{\pgfqpoint{2.337510in}{3.334561in}}%
\pgfpathlineto{\pgfqpoint{2.405924in}{3.333630in}}%
\pgfpathlineto{\pgfqpoint{2.474320in}{3.335091in}}%
\pgfpathlineto{\pgfqpoint{2.542559in}{3.339809in}}%
\pgfpathlineto{\pgfqpoint{2.610392in}{3.348536in}}%
\pgfpathlineto{\pgfqpoint{2.677477in}{3.361789in}}%
\pgfusepath{stroke}%
\end{pgfscope}%
\begin{pgfscope}%
\pgfpathrectangle{\pgfqpoint{0.647939in}{0.492442in}}{\pgfqpoint{3.079299in}{3.079299in}}%
\pgfusepath{clip}%
\pgfsetbuttcap%
\pgfsetroundjoin%
\pgfsetlinewidth{0.301125pt}%
\definecolor{currentstroke}{rgb}{0.500000,0.500000,0.500000}%
\pgfsetstrokecolor{currentstroke}%
\pgfsetstrokeopacity{0.300000}%
\pgfsetdash{}{0pt}%
\pgfpathmoveto{\pgfqpoint{2.529560in}{0.775304in}}%
\pgfpathlineto{\pgfqpoint{2.461187in}{0.777818in}}%
\pgfpathlineto{\pgfqpoint{2.392767in}{0.778415in}}%
\pgfpathlineto{\pgfqpoint{2.324349in}{0.777420in}}%
\pgfpathlineto{\pgfqpoint{2.255957in}{0.775246in}}%
\pgfpathlineto{\pgfqpoint{2.187589in}{0.772378in}}%
\pgfusepath{stroke}%
\end{pgfscope}%
\begin{pgfscope}%
\pgfpathrectangle{\pgfqpoint{0.647939in}{0.492442in}}{\pgfqpoint{3.079299in}{3.079299in}}%
\pgfusepath{clip}%
\pgfsetbuttcap%
\pgfsetroundjoin%
\pgfsetlinewidth{0.301125pt}%
\definecolor{currentstroke}{rgb}{0.500000,0.500000,0.500000}%
\pgfsetstrokecolor{currentstroke}%
\pgfsetstrokeopacity{0.300000}%
\pgfsetdash{}{0pt}%
\pgfpathmoveto{\pgfqpoint{3.447302in}{1.612187in}}%
\pgfpathlineto{\pgfqpoint{3.390163in}{1.649829in}}%
\pgfpathlineto{\pgfqpoint{3.333843in}{1.688688in}}%
\pgfpathlineto{\pgfqpoint{3.278249in}{1.728578in}}%
\pgfpathlineto{\pgfqpoint{3.223281in}{1.769329in}}%
\pgfpathlineto{\pgfqpoint{3.168844in}{1.810786in}}%
\pgfusepath{stroke}%
\end{pgfscope}%
\begin{pgfscope}%
\pgfpathrectangle{\pgfqpoint{0.647939in}{0.492442in}}{\pgfqpoint{3.079299in}{3.079299in}}%
\pgfusepath{clip}%
\pgfsetbuttcap%
\pgfsetroundjoin%
\pgfsetlinewidth{0.301125pt}%
\definecolor{currentstroke}{rgb}{0.500000,0.500000,0.500000}%
\pgfsetstrokecolor{currentstroke}%
\pgfsetstrokeopacity{0.300000}%
\pgfsetdash{}{0pt}%
\pgfpathmoveto{\pgfqpoint{3.447302in}{2.451996in}}%
\pgfpathlineto{\pgfqpoint{3.426589in}{2.517141in}}%
\pgfpathlineto{\pgfqpoint{3.411948in}{2.583900in}}%
\pgfpathlineto{\pgfqpoint{3.403740in}{2.651739in}}%
\pgfpathlineto{\pgfqpoint{3.402128in}{2.720052in}}%
\pgfpathlineto{\pgfqpoint{3.407039in}{2.788213in}}%
\pgfpathlineto{\pgfqpoint{3.418185in}{2.855651in}}%
\pgfpathlineto{\pgfqpoint{3.435126in}{2.921881in}}%
\pgfpathlineto{\pgfqpoint{3.457349in}{2.986537in}}%
\pgfusepath{stroke}%
\end{pgfscope}%
\begin{pgfscope}%
\pgfpathrectangle{\pgfqpoint{0.647939in}{0.492442in}}{\pgfqpoint{3.079299in}{3.079299in}}%
\pgfusepath{clip}%
\pgfsetbuttcap%
\pgfsetroundjoin%
\pgfsetlinewidth{0.301125pt}%
\definecolor{currentstroke}{rgb}{0.500000,0.500000,0.500000}%
\pgfsetstrokecolor{currentstroke}%
\pgfsetstrokeopacity{0.300000}%
\pgfsetdash{}{0pt}%
\pgfpathmoveto{\pgfqpoint{3.416054in}{2.185683in}}%
\pgfpathlineto{\pgfqpoint{3.377318in}{2.242044in}}%
\pgfpathlineto{\pgfqpoint{3.342278in}{2.300771in}}%
\pgfpathlineto{\pgfqpoint{3.311464in}{2.361814in}}%
\pgfpathlineto{\pgfqpoint{3.285502in}{2.425061in}}%
\pgfpathlineto{\pgfqpoint{3.265075in}{2.490290in}}%
\pgfpathlineto{\pgfqpoint{3.250835in}{2.557125in}}%
\pgfpathlineto{\pgfqpoint{3.243275in}{2.625029in}}%
\pgfpathlineto{\pgfqpoint{3.242607in}{2.693350in}}%
\pgfpathlineto{\pgfqpoint{3.248694in}{2.761414in}}%
\pgfpathlineto{\pgfqpoint{3.261089in}{2.828632in}}%
\pgfusepath{stroke}%
\end{pgfscope}%
\begin{pgfscope}%
\pgfpathrectangle{\pgfqpoint{0.647939in}{0.492442in}}{\pgfqpoint{3.079299in}{3.079299in}}%
\pgfusepath{clip}%
\pgfsetbuttcap%
\pgfsetroundjoin%
\pgfsetlinewidth{0.301125pt}%
\definecolor{currentstroke}{rgb}{0.500000,0.500000,0.500000}%
\pgfsetstrokecolor{currentstroke}%
\pgfsetstrokeopacity{0.300000}%
\pgfsetdash{}{0pt}%
\pgfpathmoveto{\pgfqpoint{3.412325in}{1.804384in}}%
\pgfpathlineto{\pgfqpoint{3.359201in}{1.847501in}}%
\pgfpathlineto{\pgfqpoint{3.307334in}{1.892124in}}%
\pgfpathlineto{\pgfqpoint{3.256722in}{1.938166in}}%
\pgfpathlineto{\pgfqpoint{3.207399in}{1.985584in}}%
\pgfpathlineto{\pgfqpoint{3.159436in}{2.034377in}}%
\pgfpathlineto{\pgfqpoint{3.112981in}{2.084603in}}%
\pgfpathlineto{\pgfqpoint{3.068271in}{2.136385in}}%
\pgfpathlineto{\pgfqpoint{3.025702in}{2.189929in}}%
\pgfpathlineto{\pgfqpoint{2.985892in}{2.245539in}}%
\pgfpathlineto{\pgfqpoint{2.949810in}{2.303614in}}%
\pgfpathlineto{\pgfqpoint{2.918922in}{2.364569in}}%
\pgfpathlineto{\pgfqpoint{2.895278in}{2.428610in}}%
\pgfpathlineto{\pgfqpoint{2.881207in}{2.495317in}}%
\pgfpathlineto{\pgfqpoint{2.878288in}{2.563374in}}%
\pgfpathlineto{\pgfqpoint{2.886298in}{2.631062in}}%
\pgfpathlineto{\pgfqpoint{2.903427in}{2.697131in}}%
\pgfpathlineto{\pgfqpoint{2.927413in}{2.761107in}}%
\pgfpathlineto{\pgfqpoint{2.956344in}{2.823053in}}%
\pgfusepath{stroke}%
\end{pgfscope}%
\begin{pgfscope}%
\pgfpathrectangle{\pgfqpoint{0.647939in}{0.492442in}}{\pgfqpoint{3.079299in}{3.079299in}}%
\pgfusepath{clip}%
\pgfsetbuttcap%
\pgfsetroundjoin%
\pgfsetlinewidth{0.301125pt}%
\definecolor{currentstroke}{rgb}{0.500000,0.500000,0.500000}%
\pgfsetstrokecolor{currentstroke}%
\pgfsetstrokeopacity{0.300000}%
\pgfsetdash{}{0pt}%
\pgfpathmoveto{\pgfqpoint{1.741575in}{3.054259in}}%
\pgfpathlineto{\pgfqpoint{1.809701in}{3.060577in}}%
\pgfpathlineto{\pgfqpoint{1.877997in}{3.064655in}}%
\pgfpathlineto{\pgfqpoint{1.946391in}{3.066491in}}%
\pgfpathlineto{\pgfqpoint{2.014811in}{3.066221in}}%
\pgfpathlineto{\pgfqpoint{2.083203in}{3.064147in}}%
\pgfpathlineto{\pgfqpoint{2.151545in}{3.060741in}}%
\pgfpathlineto{\pgfqpoint{2.219851in}{3.056645in}}%
\pgfpathlineto{\pgfqpoint{2.288164in}{3.052678in}}%
\pgfpathlineto{\pgfqpoint{2.356529in}{3.049850in}}%
\pgfpathlineto{\pgfqpoint{2.424938in}{3.049364in}}%
\pgfpathlineto{\pgfqpoint{2.493253in}{3.052543in}}%
\pgfpathlineto{\pgfqpoint{2.561133in}{3.060645in}}%
\pgfpathlineto{\pgfqpoint{2.628036in}{3.074601in}}%
\pgfpathlineto{\pgfqpoint{2.693337in}{3.094753in}}%
\pgfpathlineto{\pgfqpoint{2.756542in}{3.120771in}}%
\pgfpathlineto{\pgfqpoint{2.817445in}{3.151837in}}%
\pgfusepath{stroke}%
\end{pgfscope}%
\begin{pgfscope}%
\pgfpathrectangle{\pgfqpoint{0.647939in}{0.492442in}}{\pgfqpoint{3.079299in}{3.079299in}}%
\pgfusepath{clip}%
\pgfsetbuttcap%
\pgfsetroundjoin%
\pgfsetlinewidth{0.301125pt}%
\definecolor{currentstroke}{rgb}{0.500000,0.500000,0.500000}%
\pgfsetstrokecolor{currentstroke}%
\pgfsetstrokeopacity{0.300000}%
\pgfsetdash{}{0pt}%
\pgfpathmoveto{\pgfqpoint{2.846745in}{2.876337in}}%
\pgfpathlineto{\pgfqpoint{2.892230in}{2.927416in}}%
\pgfpathlineto{\pgfqpoint{2.937353in}{2.978830in}}%
\pgfpathlineto{\pgfqpoint{2.982347in}{3.030365in}}%
\pgfpathlineto{\pgfqpoint{3.027398in}{3.081853in}}%
\pgfpathlineto{\pgfqpoint{3.072655in}{3.133164in}}%
\pgfpathlineto{\pgfqpoint{3.118248in}{3.184180in}}%
\pgfusepath{stroke}%
\end{pgfscope}%
\begin{pgfscope}%
\pgfpathrectangle{\pgfqpoint{0.647939in}{0.492442in}}{\pgfqpoint{3.079299in}{3.079299in}}%
\pgfusepath{clip}%
\pgfsetbuttcap%
\pgfsetroundjoin%
\pgfsetlinewidth{0.301125pt}%
\definecolor{currentstroke}{rgb}{0.500000,0.500000,0.500000}%
\pgfsetstrokecolor{currentstroke}%
\pgfsetstrokeopacity{0.300000}%
\pgfsetdash{}{0pt}%
\pgfpathmoveto{\pgfqpoint{1.222709in}{1.081771in}}%
\pgfpathlineto{\pgfqpoint{1.277796in}{1.122299in}}%
\pgfpathlineto{\pgfqpoint{1.330615in}{1.165744in}}%
\pgfpathlineto{\pgfqpoint{1.381215in}{1.211752in}}%
\pgfpathlineto{\pgfqpoint{1.429708in}{1.259967in}}%
\pgfpathlineto{\pgfqpoint{1.476304in}{1.310017in}}%
\pgfpathlineto{\pgfqpoint{1.521302in}{1.361513in}}%
\pgfpathlineto{\pgfqpoint{1.565071in}{1.414047in}}%
\pgfusepath{stroke}%
\end{pgfscope}%
\begin{pgfscope}%
\pgfpathrectangle{\pgfqpoint{0.647939in}{0.492442in}}{\pgfqpoint{3.079299in}{3.079299in}}%
\pgfusepath{clip}%
\pgfsetbuttcap%
\pgfsetroundjoin%
\pgfsetlinewidth{0.301125pt}%
\definecolor{currentstroke}{rgb}{0.500000,0.500000,0.500000}%
\pgfsetstrokecolor{currentstroke}%
\pgfsetstrokeopacity{0.300000}%
\pgfsetdash{}{0pt}%
\pgfpathmoveto{\pgfqpoint{3.097382in}{1.892124in}}%
\pgfpathlineto{\pgfqpoint{3.044568in}{1.935629in}}%
\pgfpathlineto{\pgfqpoint{2.992106in}{1.979555in}}%
\pgfpathlineto{\pgfqpoint{2.939959in}{2.023853in}}%
\pgfpathlineto{\pgfqpoint{2.888134in}{2.068525in}}%
\pgfpathlineto{\pgfqpoint{2.836696in}{2.113639in}}%
\pgfpathlineto{\pgfqpoint{2.785836in}{2.159391in}}%
\pgfpathlineto{\pgfqpoint{2.735973in}{2.206222in}}%
\pgfpathlineto{\pgfqpoint{2.688119in}{2.255057in}}%
\pgfpathlineto{\pgfqpoint{2.644864in}{2.307839in}}%
\pgfpathlineto{\pgfqpoint{2.644864in}{2.307839in}}%
\pgfpathlineto{\pgfqpoint{2.617287in}{2.357318in}}%
\pgfpathlineto{\pgfqpoint{2.617287in}{2.357318in}}%
\pgfpathlineto{\pgfqpoint{2.606893in}{2.399243in}}%
\pgfpathlineto{\pgfqpoint{2.609262in}{2.443311in}}%
\pgfpathlineto{\pgfqpoint{2.621305in}{2.484340in}}%
\pgfpathlineto{\pgfqpoint{2.643097in}{2.530997in}}%
\pgfusepath{stroke}%
\end{pgfscope}%
\begin{pgfscope}%
\pgfpathrectangle{\pgfqpoint{0.647939in}{0.492442in}}{\pgfqpoint{3.079299in}{3.079299in}}%
\pgfusepath{clip}%
\pgfsetbuttcap%
\pgfsetroundjoin%
\pgfsetlinewidth{0.301125pt}%
\definecolor{currentstroke}{rgb}{0.500000,0.500000,0.500000}%
\pgfsetstrokecolor{currentstroke}%
\pgfsetstrokeopacity{0.300000}%
\pgfsetdash{}{0pt}%
\pgfpathmoveto{\pgfqpoint{1.653919in}{2.910996in}}%
\pgfpathlineto{\pgfqpoint{1.721635in}{2.920781in}}%
\pgfpathlineto{\pgfqpoint{1.789630in}{2.928373in}}%
\pgfpathlineto{\pgfqpoint{1.857848in}{2.933580in}}%
\pgfpathlineto{\pgfqpoint{1.926209in}{2.936327in}}%
\pgfpathlineto{\pgfqpoint{1.994627in}{2.936689in}}%
\pgfpathlineto{\pgfqpoint{2.063024in}{2.934910in}}%
\pgfpathlineto{\pgfqpoint{2.131359in}{2.931416in}}%
\pgfpathlineto{\pgfqpoint{2.199633in}{2.926836in}}%
\pgfpathlineto{\pgfqpoint{2.267893in}{2.922037in}}%
\pgfpathlineto{\pgfqpoint{2.336206in}{2.918157in}}%
\pgfpathlineto{\pgfqpoint{2.404596in}{2.916602in}}%
\pgfpathlineto{\pgfqpoint{2.472933in}{2.918999in}}%
\pgfpathlineto{\pgfqpoint{2.540807in}{2.927004in}}%
\pgfpathlineto{\pgfqpoint{2.607493in}{2.941885in}}%
\pgfusepath{stroke}%
\end{pgfscope}%
\begin{pgfscope}%
\pgfpathrectangle{\pgfqpoint{0.647939in}{0.492442in}}{\pgfqpoint{3.079299in}{3.079299in}}%
\pgfusepath{clip}%
\pgfsetbuttcap%
\pgfsetroundjoin%
\pgfsetlinewidth{0.301125pt}%
\definecolor{currentstroke}{rgb}{0.500000,0.500000,0.500000}%
\pgfsetstrokecolor{currentstroke}%
\pgfsetstrokeopacity{0.300000}%
\pgfsetdash{}{0pt}%
\pgfpathmoveto{\pgfqpoint{1.277796in}{2.451996in}}%
\pgfpathlineto{\pgfqpoint{1.343585in}{2.470817in}}%
\pgfpathlineto{\pgfqpoint{1.409398in}{2.489556in}}%
\pgfpathlineto{\pgfqpoint{1.475343in}{2.507821in}}%
\pgfpathlineto{\pgfqpoint{1.541528in}{2.525190in}}%
\pgfpathlineto{\pgfqpoint{1.608050in}{2.541209in}}%
\pgfusepath{stroke}%
\end{pgfscope}%
\begin{pgfscope}%
\pgfpathrectangle{\pgfqpoint{0.647939in}{0.492442in}}{\pgfqpoint{3.079299in}{3.079299in}}%
\pgfusepath{clip}%
\pgfsetbuttcap%
\pgfsetroundjoin%
\pgfsetlinewidth{0.301125pt}%
\definecolor{currentstroke}{rgb}{0.500000,0.500000,0.500000}%
\pgfsetstrokecolor{currentstroke}%
\pgfsetstrokeopacity{0.300000}%
\pgfsetdash{}{0pt}%
\pgfpathmoveto{\pgfqpoint{1.287670in}{1.509526in}}%
\pgfpathlineto{\pgfqpoint{1.347780in}{1.542203in}}%
\pgfpathlineto{\pgfqpoint{1.407057in}{1.576371in}}%
\pgfpathlineto{\pgfqpoint{1.465646in}{1.611712in}}%
\pgfpathlineto{\pgfqpoint{1.523737in}{1.647869in}}%
\pgfpathlineto{\pgfqpoint{1.581567in}{1.684442in}}%
\pgfusepath{stroke}%
\end{pgfscope}%
\begin{pgfscope}%
\pgfpathrectangle{\pgfqpoint{0.647939in}{0.492442in}}{\pgfqpoint{3.079299in}{3.079299in}}%
\pgfusepath{clip}%
\pgfsetbuttcap%
\pgfsetroundjoin%
\pgfsetlinewidth{0.301125pt}%
\definecolor{currentstroke}{rgb}{0.500000,0.500000,0.500000}%
\pgfsetstrokecolor{currentstroke}%
\pgfsetstrokeopacity{0.300000}%
\pgfsetdash{}{0pt}%
\pgfpathmoveto{\pgfqpoint{2.249177in}{1.275293in}}%
\pgfpathlineto{\pgfqpoint{2.180906in}{1.270666in}}%
\pgfpathlineto{\pgfqpoint{2.112653in}{1.265773in}}%
\pgfpathlineto{\pgfqpoint{2.044351in}{1.261680in}}%
\pgfpathlineto{\pgfqpoint{1.975967in}{1.259816in}}%
\pgfpathlineto{\pgfqpoint{1.907652in}{1.262267in}}%
\pgfusepath{stroke}%
\end{pgfscope}%
\begin{pgfscope}%
\pgfpathrectangle{\pgfqpoint{0.647939in}{0.492442in}}{\pgfqpoint{3.079299in}{3.079299in}}%
\pgfusepath{clip}%
\pgfsetbuttcap%
\pgfsetroundjoin%
\pgfsetlinewidth{0.301125pt}%
\definecolor{currentstroke}{rgb}{0.500000,0.500000,0.500000}%
\pgfsetstrokecolor{currentstroke}%
\pgfsetstrokeopacity{0.300000}%
\pgfsetdash{}{0pt}%
\pgfpathmoveto{\pgfqpoint{3.040562in}{2.265346in}}%
\pgfpathlineto{\pgfqpoint{3.006954in}{2.324886in}}%
\pgfpathlineto{\pgfqpoint{2.978688in}{2.387101in}}%
\pgfpathlineto{\pgfqpoint{2.957413in}{2.451996in}}%
\pgfpathlineto{\pgfqpoint{2.944874in}{2.519084in}}%
\pgfpathlineto{\pgfqpoint{2.942122in}{2.587235in}}%
\pgfpathlineto{\pgfqpoint{2.949001in}{2.655107in}}%
\pgfusepath{stroke}%
\end{pgfscope}%
\begin{pgfscope}%
\pgfpathrectangle{\pgfqpoint{0.647939in}{0.492442in}}{\pgfqpoint{3.079299in}{3.079299in}}%
\pgfusepath{clip}%
\pgfsetbuttcap%
\pgfsetroundjoin%
\pgfsetlinewidth{0.301125pt}%
\definecolor{currentstroke}{rgb}{0.500000,0.500000,0.500000}%
\pgfsetstrokecolor{currentstroke}%
\pgfsetstrokeopacity{0.300000}%
\pgfsetdash{}{0pt}%
\pgfpathmoveto{\pgfqpoint{1.832527in}{2.677198in}}%
\pgfpathlineto{\pgfqpoint{1.900783in}{2.681784in}}%
\pgfpathlineto{\pgfqpoint{1.969174in}{2.683354in}}%
\pgfpathlineto{\pgfqpoint{2.037572in}{2.681991in}}%
\pgfpathlineto{\pgfqpoint{2.105873in}{2.678008in}}%
\pgfpathlineto{\pgfqpoint{2.174032in}{2.672002in}}%
\pgfpathlineto{\pgfqpoint{2.242095in}{2.664942in}}%
\pgfpathlineto{\pgfqpoint{2.310193in}{2.658266in}}%
\pgfpathlineto{\pgfqpoint{2.378461in}{2.654073in}}%
\pgfpathlineto{\pgfqpoint{2.446772in}{2.655325in}}%
\pgfpathlineto{\pgfqpoint{2.514232in}{2.665486in}}%
\pgfpathlineto{\pgfqpoint{2.578961in}{2.686837in}}%
\pgfpathlineto{\pgfqpoint{2.639333in}{2.718603in}}%
\pgfpathlineto{\pgfqpoint{2.695199in}{2.757889in}}%
\pgfpathlineto{\pgfqpoint{2.747461in}{2.801916in}}%
\pgfusepath{stroke}%
\end{pgfscope}%
\begin{pgfscope}%
\pgfpathrectangle{\pgfqpoint{0.647939in}{0.492442in}}{\pgfqpoint{3.079299in}{3.079299in}}%
\pgfusepath{clip}%
\pgfsetbuttcap%
\pgfsetroundjoin%
\pgfsetlinewidth{0.301125pt}%
\definecolor{currentstroke}{rgb}{0.500000,0.500000,0.500000}%
\pgfsetstrokecolor{currentstroke}%
\pgfsetstrokeopacity{0.300000}%
\pgfsetdash{}{0pt}%
\pgfpathmoveto{\pgfqpoint{1.487748in}{1.962108in}}%
\pgfpathlineto{\pgfqpoint{1.551090in}{1.987993in}}%
\pgfpathlineto{\pgfqpoint{1.614810in}{2.012927in}}%
\pgfpathlineto{\pgfqpoint{1.679129in}{2.036259in}}%
\pgfpathlineto{\pgfqpoint{1.744251in}{2.057209in}}%
\pgfpathlineto{\pgfqpoint{1.810337in}{2.074827in}}%
\pgfpathlineto{\pgfqpoint{1.877439in}{2.087912in}}%
\pgfpathlineto{\pgfqpoint{1.945405in}{2.094827in}}%
\pgfpathlineto{\pgfqpoint{2.013583in}{2.092916in}}%
\pgfpathlineto{\pgfqpoint{2.013583in}{2.092916in}}%
\pgfpathlineto{\pgfqpoint{2.067921in}{2.080962in}}%
\pgfpathlineto{\pgfqpoint{2.067921in}{2.080962in}}%
\pgfpathlineto{\pgfqpoint{2.099015in}{2.064686in}}%
\pgfpathlineto{\pgfqpoint{2.099015in}{2.064686in}}%
\pgfpathlineto{\pgfqpoint{2.115585in}{2.045180in}}%
\pgfpathlineto{\pgfqpoint{2.119837in}{2.019832in}}%
\pgfpathlineto{\pgfqpoint{2.114842in}{1.997560in}}%
\pgfpathlineto{\pgfqpoint{2.102300in}{1.972441in}}%
\pgfusepath{stroke}%
\end{pgfscope}%
\begin{pgfscope}%
\pgfpathrectangle{\pgfqpoint{0.647939in}{0.492442in}}{\pgfqpoint{3.079299in}{3.079299in}}%
\pgfusepath{clip}%
\pgfsetbuttcap%
\pgfsetroundjoin%
\pgfsetlinewidth{0.301125pt}%
\definecolor{currentstroke}{rgb}{0.500000,0.500000,0.500000}%
\pgfsetstrokecolor{currentstroke}%
\pgfsetstrokeopacity{0.300000}%
\pgfsetdash{}{0pt}%
\pgfpathmoveto{\pgfqpoint{2.817445in}{2.032092in}}%
\pgfpathlineto{\pgfqpoint{2.761507in}{2.071484in}}%
\pgfpathlineto{\pgfqpoint{2.704917in}{2.109937in}}%
\pgfpathlineto{\pgfqpoint{2.647545in}{2.147194in}}%
\pgfpathlineto{\pgfqpoint{2.589264in}{2.182962in}}%
\pgfpathlineto{\pgfqpoint{2.529945in}{2.216902in}}%
\pgfpathlineto{\pgfqpoint{2.469525in}{2.248568in}}%
\pgfpathlineto{\pgfqpoint{2.469525in}{2.248568in}}%
\pgfpathlineto{\pgfqpoint{2.437266in}{2.263873in}}%
\pgfpathlineto{\pgfqpoint{2.437266in}{2.263873in}}%
\pgfpathlineto{\pgfqpoint{2.419118in}{2.271495in}}%
\pgfpathlineto{\pgfqpoint{2.419118in}{2.271495in}}%
\pgfpathlineto{\pgfqpoint{2.409999in}{2.274216in}}%
\pgfpathlineto{\pgfqpoint{2.404674in}{2.274310in}}%
\pgfpathlineto{\pgfqpoint{2.400261in}{2.271768in}}%
\pgfpathlineto{\pgfqpoint{2.392443in}{2.260955in}}%
\pgfpathlineto{\pgfqpoint{2.379026in}{2.247120in}}%
\pgfpathlineto{\pgfqpoint{2.363630in}{2.230503in}}%
\pgfpathlineto{\pgfqpoint{2.338509in}{2.201974in}}%
\pgfpathlineto{\pgfqpoint{2.294925in}{2.152474in}}%
\pgfusepath{stroke}%
\end{pgfscope}%
\begin{pgfscope}%
\pgfpathrectangle{\pgfqpoint{0.647939in}{0.492442in}}{\pgfqpoint{3.079299in}{3.079299in}}%
\pgfusepath{clip}%
\pgfsetbuttcap%
\pgfsetroundjoin%
\pgfsetlinewidth{0.301125pt}%
\definecolor{currentstroke}{rgb}{0.500000,0.500000,0.500000}%
\pgfsetstrokecolor{currentstroke}%
\pgfsetstrokeopacity{0.300000}%
\pgfsetdash{}{0pt}%
\pgfpathmoveto{\pgfqpoint{1.657736in}{2.569231in}}%
\pgfpathlineto{\pgfqpoint{1.725024in}{2.581606in}}%
\pgfpathlineto{\pgfqpoint{1.792732in}{2.591396in}}%
\pgfpathlineto{\pgfqpoint{1.860799in}{2.598234in}}%
\pgfpathlineto{\pgfqpoint{1.929110in}{2.601871in}}%
\pgfpathlineto{\pgfqpoint{1.997516in}{2.602225in}}%
\pgfpathlineto{\pgfqpoint{2.065867in}{2.599421in}}%
\pgfpathlineto{\pgfqpoint{2.134054in}{2.593844in}}%
\pgfpathlineto{\pgfqpoint{2.202051in}{2.586217in}}%
\pgfpathlineto{\pgfqpoint{2.269955in}{2.577762in}}%
\pgfpathlineto{\pgfqpoint{2.337978in}{2.570457in}}%
\pgfpathlineto{\pgfqpoint{2.406250in}{2.567476in}}%
\pgfpathlineto{\pgfqpoint{2.472214in}{2.573106in}}%
\pgfpathlineto{\pgfqpoint{2.537509in}{2.591964in}}%
\pgfusepath{stroke}%
\end{pgfscope}%
\begin{pgfscope}%
\pgfpathrectangle{\pgfqpoint{0.647939in}{0.492442in}}{\pgfqpoint{3.079299in}{3.079299in}}%
\pgfusepath{clip}%
\pgfsetbuttcap%
\pgfsetroundjoin%
\pgfsetlinewidth{0.301125pt}%
\definecolor{currentstroke}{rgb}{0.500000,0.500000,0.500000}%
\pgfsetstrokecolor{currentstroke}%
\pgfsetstrokeopacity{0.300000}%
\pgfsetdash{}{0pt}%
\pgfpathmoveto{\pgfqpoint{1.627716in}{2.312028in}}%
\pgfpathlineto{\pgfqpoint{1.694086in}{2.328638in}}%
\pgfpathlineto{\pgfqpoint{1.761045in}{2.342652in}}%
\pgfpathlineto{\pgfqpoint{1.828591in}{2.353450in}}%
\pgfpathlineto{\pgfqpoint{1.896629in}{2.360439in}}%
\pgfpathlineto{\pgfqpoint{1.964962in}{2.363105in}}%
\pgfpathlineto{\pgfqpoint{2.033308in}{2.361051in}}%
\pgfpathlineto{\pgfqpoint{2.101320in}{2.354027in}}%
\pgfpathlineto{\pgfqpoint{2.168614in}{2.341927in}}%
\pgfpathlineto{\pgfqpoint{2.234785in}{2.324708in}}%
\pgfpathlineto{\pgfqpoint{2.299234in}{2.302003in}}%
\pgfpathlineto{\pgfqpoint{2.357094in}{2.269312in}}%
\pgfpathlineto{\pgfqpoint{2.357094in}{2.269312in}}%
\pgfusepath{stroke}%
\end{pgfscope}%
\begin{pgfscope}%
\pgfpathrectangle{\pgfqpoint{0.647939in}{0.492442in}}{\pgfqpoint{3.079299in}{3.079299in}}%
\pgfusepath{clip}%
\pgfsetbuttcap%
\pgfsetroundjoin%
\pgfsetlinewidth{0.301125pt}%
\definecolor{currentstroke}{rgb}{0.500000,0.500000,0.500000}%
\pgfsetstrokecolor{currentstroke}%
\pgfsetstrokeopacity{0.300000}%
\pgfsetdash{}{0pt}%
\pgfpathmoveto{\pgfqpoint{2.247921in}{1.560940in}}%
\pgfpathlineto{\pgfqpoint{2.179855in}{1.553943in}}%
\pgfpathlineto{\pgfqpoint{2.111854in}{1.546302in}}%
\pgfpathlineto{\pgfqpoint{2.043761in}{1.539617in}}%
\pgfpathlineto{\pgfqpoint{1.975464in}{1.536470in}}%
\pgfpathlineto{\pgfqpoint{1.907652in}{1.542203in}}%
\pgfpathlineto{\pgfqpoint{1.907652in}{1.542203in}}%
\pgfusepath{stroke}%
\end{pgfscope}%
\begin{pgfscope}%
\pgfpathrectangle{\pgfqpoint{0.647939in}{0.492442in}}{\pgfqpoint{3.079299in}{3.079299in}}%
\pgfusepath{clip}%
\pgfsetbuttcap%
\pgfsetroundjoin%
\pgfsetlinewidth{0.301125pt}%
\definecolor{currentstroke}{rgb}{0.500000,0.500000,0.500000}%
\pgfsetstrokecolor{currentstroke}%
\pgfsetstrokeopacity{0.300000}%
\pgfsetdash{}{0pt}%
\pgfpathmoveto{\pgfqpoint{2.622283in}{2.234289in}}%
\pgfpathlineto{\pgfqpoint{2.577426in}{2.275031in}}%
\pgfpathlineto{\pgfqpoint{2.550202in}{2.305599in}}%
\pgfpathlineto{\pgfqpoint{2.533033in}{2.334815in}}%
\pgfpathlineto{\pgfqpoint{2.526939in}{2.367908in}}%
\pgfpathlineto{\pgfqpoint{2.526939in}{2.367908in}}%
\pgfpathlineto{\pgfqpoint{2.536270in}{2.407200in}}%
\pgfpathlineto{\pgfqpoint{2.536270in}{2.407200in}}%
\pgfpathlineto{\pgfqpoint{2.568792in}{2.466032in}}%
\pgfpathlineto{\pgfqpoint{2.607493in}{2.521980in}}%
\pgfusepath{stroke}%
\end{pgfscope}%
\begin{pgfscope}%
\pgfpathrectangle{\pgfqpoint{0.647939in}{0.492442in}}{\pgfqpoint{3.079299in}{3.079299in}}%
\pgfusepath{clip}%
\pgfsetbuttcap%
\pgfsetroundjoin%
\pgfsetlinewidth{0.301125pt}%
\definecolor{currentstroke}{rgb}{0.500000,0.500000,0.500000}%
\pgfsetstrokecolor{currentstroke}%
\pgfsetstrokeopacity{0.300000}%
\pgfsetdash{}{0pt}%
\pgfpathmoveto{\pgfqpoint{2.736329in}{1.846401in}}%
\pgfpathlineto{\pgfqpoint{2.672620in}{1.871276in}}%
\pgfpathlineto{\pgfqpoint{2.607493in}{1.892124in}}%
\pgfpathlineto{\pgfqpoint{2.540996in}{1.908022in}}%
\pgfpathlineto{\pgfqpoint{2.473393in}{1.918071in}}%
\pgfpathlineto{\pgfqpoint{2.405154in}{1.921540in}}%
\pgfpathlineto{\pgfqpoint{2.336917in}{1.918105in}}%
\pgfpathlineto{\pgfqpoint{2.269329in}{1.908027in}}%
\pgfpathlineto{\pgfqpoint{2.202825in}{1.892192in}}%
\pgfusepath{stroke}%
\end{pgfscope}%
\begin{pgfscope}%
\pgfpathrectangle{\pgfqpoint{0.647939in}{0.492442in}}{\pgfqpoint{3.079299in}{3.079299in}}%
\pgfusepath{clip}%
\pgfsetbuttcap%
\pgfsetroundjoin%
\pgfsetlinewidth{0.301125pt}%
\definecolor{currentstroke}{rgb}{0.500000,0.500000,0.500000}%
\pgfsetstrokecolor{currentstroke}%
\pgfsetstrokeopacity{0.300000}%
\pgfsetdash{}{0pt}%
\pgfpathmoveto{\pgfqpoint{2.602784in}{2.011854in}}%
\pgfpathlineto{\pgfqpoint{2.537509in}{2.032092in}}%
\pgfpathlineto{\pgfqpoint{2.470556in}{2.045569in}}%
\pgfpathlineto{\pgfqpoint{2.402499in}{2.050175in}}%
\pgfpathlineto{\pgfqpoint{2.334676in}{2.043844in}}%
\pgfpathlineto{\pgfqpoint{2.269053in}{2.025791in}}%
\pgfusepath{stroke}%
\end{pgfscope}%
\begin{pgfscope}%
\pgfpathrectangle{\pgfqpoint{0.647939in}{0.492442in}}{\pgfqpoint{3.079299in}{3.079299in}}%
\pgfusepath{clip}%
\pgfsetbuttcap%
\pgfsetroundjoin%
\pgfsetlinewidth{0.301125pt}%
\definecolor{currentstroke}{rgb}{0.500000,0.500000,0.500000}%
\pgfsetstrokecolor{currentstroke}%
\pgfsetstrokeopacity{0.300000}%
\pgfsetdash{}{0pt}%
\pgfpathmoveto{\pgfqpoint{1.837668in}{2.172060in}}%
\pgfpathlineto{\pgfqpoint{1.905466in}{2.180877in}}%
\pgfpathlineto{\pgfqpoint{1.973737in}{2.183818in}}%
\pgfpathlineto{\pgfqpoint{2.041862in}{2.179256in}}%
\pgfpathlineto{\pgfqpoint{2.108317in}{2.164263in}}%
\pgfpathlineto{\pgfqpoint{2.108317in}{2.164263in}}%
\pgfpathlineto{\pgfqpoint{2.150340in}{2.144585in}}%
\pgfpathlineto{\pgfqpoint{2.150340in}{2.144585in}}%
\pgfpathlineto{\pgfqpoint{2.173433in}{2.123542in}}%
\pgfpathlineto{\pgfqpoint{2.173433in}{2.123542in}}%
\pgfpathlineto{\pgfqpoint{2.183343in}{2.100698in}}%
\pgfpathlineto{\pgfqpoint{2.182424in}{2.075603in}}%
\pgfusepath{stroke}%
\end{pgfscope}%
\begin{pgfscope}%
\pgfpathrectangle{\pgfqpoint{0.647939in}{0.492442in}}{\pgfqpoint{3.079299in}{3.079299in}}%
\pgfusepath{clip}%
\pgfsetroundcap%
\pgfsetroundjoin%
\pgfsetlinewidth{0.301125pt}%
\definecolor{currentstroke}{rgb}{0.500000,0.500000,0.500000}%
\pgfsetstrokecolor{currentstroke}%
\pgfsetstrokeopacity{0.300000}%
\pgfsetdash{}{0pt}%
\pgfpathmoveto{\pgfqpoint{1.415909in}{1.045102in}}%
\pgfusepath{stroke}%
\end{pgfscope}%
\begin{pgfscope}%
\pgfpathrectangle{\pgfqpoint{0.647939in}{0.492442in}}{\pgfqpoint{3.079299in}{3.079299in}}%
\pgfusepath{clip}%
\pgfsetroundcap%
\pgfsetroundjoin%
\definecolor{currentfill}{rgb}{0.500000,0.500000,0.500000}%
\pgfsetfillcolor{currentfill}%
\pgfsetfillopacity{0.300000}%
\pgfsetlinewidth{0.301125pt}%
\definecolor{currentstroke}{rgb}{0.500000,0.500000,0.500000}%
\pgfsetstrokecolor{currentstroke}%
\pgfsetstrokeopacity{0.300000}%
\pgfsetdash{}{0pt}%
\pgfpathmoveto{\pgfqpoint{0.000000in}{0.000000in}}%
\pgfpathlineto{\pgfqpoint{0.000000in}{0.000000in}}%
\pgfpathclose%
\pgfusepath{stroke,fill}%
\end{pgfscope}%
\begin{pgfscope}%
\pgfpathrectangle{\pgfqpoint{0.647939in}{0.492442in}}{\pgfqpoint{3.079299in}{3.079299in}}%
\pgfusepath{clip}%
\pgfsetroundcap%
\pgfsetroundjoin%
\pgfsetlinewidth{0.301125pt}%
\definecolor{currentstroke}{rgb}{0.500000,0.500000,0.500000}%
\pgfsetstrokecolor{currentstroke}%
\pgfsetstrokeopacity{0.300000}%
\pgfsetdash{}{0pt}%
\pgfpathmoveto{\pgfqpoint{1.070449in}{0.573989in}}%
\pgfusepath{stroke}%
\end{pgfscope}%
\begin{pgfscope}%
\pgfpathrectangle{\pgfqpoint{0.647939in}{0.492442in}}{\pgfqpoint{3.079299in}{3.079299in}}%
\pgfusepath{clip}%
\pgfsetroundcap%
\pgfsetroundjoin%
\definecolor{currentfill}{rgb}{0.500000,0.500000,0.500000}%
\pgfsetfillcolor{currentfill}%
\pgfsetfillopacity{0.300000}%
\pgfsetlinewidth{0.301125pt}%
\definecolor{currentstroke}{rgb}{0.500000,0.500000,0.500000}%
\pgfsetstrokecolor{currentstroke}%
\pgfsetstrokeopacity{0.300000}%
\pgfsetdash{}{0pt}%
\pgfpathmoveto{\pgfqpoint{0.000000in}{0.000000in}}%
\pgfpathlineto{\pgfqpoint{0.000000in}{0.000000in}}%
\pgfpathclose%
\pgfusepath{stroke,fill}%
\end{pgfscope}%
\begin{pgfscope}%
\pgfpathrectangle{\pgfqpoint{0.647939in}{0.492442in}}{\pgfqpoint{3.079299in}{3.079299in}}%
\pgfusepath{clip}%
\pgfsetroundcap%
\pgfsetroundjoin%
\pgfsetlinewidth{0.301125pt}%
\definecolor{currentstroke}{rgb}{0.500000,0.500000,0.500000}%
\pgfsetstrokecolor{currentstroke}%
\pgfsetstrokeopacity{0.300000}%
\pgfsetdash{}{0pt}%
\pgfpathmoveto{\pgfqpoint{1.272952in}{0.681173in}}%
\pgfusepath{stroke}%
\end{pgfscope}%
\begin{pgfscope}%
\pgfpathrectangle{\pgfqpoint{0.647939in}{0.492442in}}{\pgfqpoint{3.079299in}{3.079299in}}%
\pgfusepath{clip}%
\pgfsetroundcap%
\pgfsetroundjoin%
\definecolor{currentfill}{rgb}{0.500000,0.500000,0.500000}%
\pgfsetfillcolor{currentfill}%
\pgfsetfillopacity{0.300000}%
\pgfsetlinewidth{0.301125pt}%
\definecolor{currentstroke}{rgb}{0.500000,0.500000,0.500000}%
\pgfsetstrokecolor{currentstroke}%
\pgfsetstrokeopacity{0.300000}%
\pgfsetdash{}{0pt}%
\pgfpathmoveto{\pgfqpoint{0.000000in}{0.000000in}}%
\pgfpathlineto{\pgfqpoint{0.000000in}{0.000000in}}%
\pgfpathclose%
\pgfusepath{stroke,fill}%
\end{pgfscope}%
\begin{pgfscope}%
\pgfpathrectangle{\pgfqpoint{0.647939in}{0.492442in}}{\pgfqpoint{3.079299in}{3.079299in}}%
\pgfusepath{clip}%
\pgfsetroundcap%
\pgfsetroundjoin%
\pgfsetlinewidth{0.301125pt}%
\definecolor{currentstroke}{rgb}{0.500000,0.500000,0.500000}%
\pgfsetstrokecolor{currentstroke}%
\pgfsetstrokeopacity{0.300000}%
\pgfsetdash{}{0pt}%
\pgfpathmoveto{\pgfqpoint{1.388337in}{0.756542in}}%
\pgfusepath{stroke}%
\end{pgfscope}%
\begin{pgfscope}%
\pgfpathrectangle{\pgfqpoint{0.647939in}{0.492442in}}{\pgfqpoint{3.079299in}{3.079299in}}%
\pgfusepath{clip}%
\pgfsetroundcap%
\pgfsetroundjoin%
\definecolor{currentfill}{rgb}{0.500000,0.500000,0.500000}%
\pgfsetfillcolor{currentfill}%
\pgfsetfillopacity{0.300000}%
\pgfsetlinewidth{0.301125pt}%
\definecolor{currentstroke}{rgb}{0.500000,0.500000,0.500000}%
\pgfsetstrokecolor{currentstroke}%
\pgfsetstrokeopacity{0.300000}%
\pgfsetdash{}{0pt}%
\pgfpathmoveto{\pgfqpoint{0.000000in}{0.000000in}}%
\pgfpathlineto{\pgfqpoint{0.000000in}{0.000000in}}%
\pgfpathclose%
\pgfusepath{stroke,fill}%
\end{pgfscope}%
\begin{pgfscope}%
\pgfpathrectangle{\pgfqpoint{0.647939in}{0.492442in}}{\pgfqpoint{3.079299in}{3.079299in}}%
\pgfusepath{clip}%
\pgfsetroundcap%
\pgfsetroundjoin%
\pgfsetlinewidth{0.301125pt}%
\definecolor{currentstroke}{rgb}{0.500000,0.500000,0.500000}%
\pgfsetstrokecolor{currentstroke}%
\pgfsetstrokeopacity{0.300000}%
\pgfsetdash{}{0pt}%
\pgfpathmoveto{\pgfqpoint{1.609712in}{0.535797in}}%
\pgfusepath{stroke}%
\end{pgfscope}%
\begin{pgfscope}%
\pgfpathrectangle{\pgfqpoint{0.647939in}{0.492442in}}{\pgfqpoint{3.079299in}{3.079299in}}%
\pgfusepath{clip}%
\pgfsetroundcap%
\pgfsetroundjoin%
\definecolor{currentfill}{rgb}{0.500000,0.500000,0.500000}%
\pgfsetfillcolor{currentfill}%
\pgfsetfillopacity{0.300000}%
\pgfsetlinewidth{0.301125pt}%
\definecolor{currentstroke}{rgb}{0.500000,0.500000,0.500000}%
\pgfsetstrokecolor{currentstroke}%
\pgfsetstrokeopacity{0.300000}%
\pgfsetdash{}{0pt}%
\pgfpathmoveto{\pgfqpoint{0.000000in}{0.000000in}}%
\pgfpathlineto{\pgfqpoint{0.000000in}{0.000000in}}%
\pgfpathclose%
\pgfusepath{stroke,fill}%
\end{pgfscope}%
\begin{pgfscope}%
\pgfpathrectangle{\pgfqpoint{0.647939in}{0.492442in}}{\pgfqpoint{3.079299in}{3.079299in}}%
\pgfusepath{clip}%
\pgfsetroundcap%
\pgfsetroundjoin%
\pgfsetlinewidth{0.301125pt}%
\definecolor{currentstroke}{rgb}{0.500000,0.500000,0.500000}%
\pgfsetstrokecolor{currentstroke}%
\pgfsetstrokeopacity{0.300000}%
\pgfsetdash{}{0pt}%
\pgfpathmoveto{\pgfqpoint{2.291871in}{0.501326in}}%
\pgfusepath{stroke}%
\end{pgfscope}%
\begin{pgfscope}%
\pgfpathrectangle{\pgfqpoint{0.647939in}{0.492442in}}{\pgfqpoint{3.079299in}{3.079299in}}%
\pgfusepath{clip}%
\pgfsetroundcap%
\pgfsetroundjoin%
\definecolor{currentfill}{rgb}{0.500000,0.500000,0.500000}%
\pgfsetfillcolor{currentfill}%
\pgfsetfillopacity{0.300000}%
\pgfsetlinewidth{0.301125pt}%
\definecolor{currentstroke}{rgb}{0.500000,0.500000,0.500000}%
\pgfsetstrokecolor{currentstroke}%
\pgfsetstrokeopacity{0.300000}%
\pgfsetdash{}{0pt}%
\pgfpathmoveto{\pgfqpoint{0.000000in}{0.000000in}}%
\pgfpathlineto{\pgfqpoint{0.000000in}{0.000000in}}%
\pgfpathclose%
\pgfusepath{stroke,fill}%
\end{pgfscope}%
\begin{pgfscope}%
\pgfpathrectangle{\pgfqpoint{0.647939in}{0.492442in}}{\pgfqpoint{3.079299in}{3.079299in}}%
\pgfusepath{clip}%
\pgfsetroundcap%
\pgfsetroundjoin%
\pgfsetlinewidth{0.301125pt}%
\definecolor{currentstroke}{rgb}{0.500000,0.500000,0.500000}%
\pgfsetstrokecolor{currentstroke}%
\pgfsetstrokeopacity{0.300000}%
\pgfsetdash{}{0pt}%
\pgfpathmoveto{\pgfqpoint{2.247054in}{0.536555in}}%
\pgfusepath{stroke}%
\end{pgfscope}%
\begin{pgfscope}%
\pgfpathrectangle{\pgfqpoint{0.647939in}{0.492442in}}{\pgfqpoint{3.079299in}{3.079299in}}%
\pgfusepath{clip}%
\pgfsetroundcap%
\pgfsetroundjoin%
\definecolor{currentfill}{rgb}{0.500000,0.500000,0.500000}%
\pgfsetfillcolor{currentfill}%
\pgfsetfillopacity{0.300000}%
\pgfsetlinewidth{0.301125pt}%
\definecolor{currentstroke}{rgb}{0.500000,0.500000,0.500000}%
\pgfsetstrokecolor{currentstroke}%
\pgfsetstrokeopacity{0.300000}%
\pgfsetdash{}{0pt}%
\pgfpathmoveto{\pgfqpoint{0.000000in}{0.000000in}}%
\pgfpathlineto{\pgfqpoint{0.000000in}{0.000000in}}%
\pgfpathclose%
\pgfusepath{stroke,fill}%
\end{pgfscope}%
\begin{pgfscope}%
\pgfpathrectangle{\pgfqpoint{0.647939in}{0.492442in}}{\pgfqpoint{3.079299in}{3.079299in}}%
\pgfusepath{clip}%
\pgfsetroundcap%
\pgfsetroundjoin%
\pgfsetlinewidth{0.301125pt}%
\definecolor{currentstroke}{rgb}{0.500000,0.500000,0.500000}%
\pgfsetstrokecolor{currentstroke}%
\pgfsetstrokeopacity{0.300000}%
\pgfsetdash{}{0pt}%
\pgfpathmoveto{\pgfqpoint{2.937901in}{0.533152in}}%
\pgfusepath{stroke}%
\end{pgfscope}%
\begin{pgfscope}%
\pgfpathrectangle{\pgfqpoint{0.647939in}{0.492442in}}{\pgfqpoint{3.079299in}{3.079299in}}%
\pgfusepath{clip}%
\pgfsetroundcap%
\pgfsetroundjoin%
\definecolor{currentfill}{rgb}{0.500000,0.500000,0.500000}%
\pgfsetfillcolor{currentfill}%
\pgfsetfillopacity{0.300000}%
\pgfsetlinewidth{0.301125pt}%
\definecolor{currentstroke}{rgb}{0.500000,0.500000,0.500000}%
\pgfsetstrokecolor{currentstroke}%
\pgfsetstrokeopacity{0.300000}%
\pgfsetdash{}{0pt}%
\pgfpathmoveto{\pgfqpoint{0.000000in}{0.000000in}}%
\pgfpathlineto{\pgfqpoint{0.000000in}{0.000000in}}%
\pgfpathclose%
\pgfusepath{stroke,fill}%
\end{pgfscope}%
\begin{pgfscope}%
\pgfpathrectangle{\pgfqpoint{0.647939in}{0.492442in}}{\pgfqpoint{3.079299in}{3.079299in}}%
\pgfusepath{clip}%
\pgfsetroundcap%
\pgfsetroundjoin%
\pgfsetlinewidth{0.301125pt}%
\definecolor{currentstroke}{rgb}{0.500000,0.500000,0.500000}%
\pgfsetstrokecolor{currentstroke}%
\pgfsetstrokeopacity{0.300000}%
\pgfsetdash{}{0pt}%
\pgfpathmoveto{\pgfqpoint{2.273359in}{0.658647in}}%
\pgfusepath{stroke}%
\end{pgfscope}%
\begin{pgfscope}%
\pgfpathrectangle{\pgfqpoint{0.647939in}{0.492442in}}{\pgfqpoint{3.079299in}{3.079299in}}%
\pgfusepath{clip}%
\pgfsetroundcap%
\pgfsetroundjoin%
\definecolor{currentfill}{rgb}{0.500000,0.500000,0.500000}%
\pgfsetfillcolor{currentfill}%
\pgfsetfillopacity{0.300000}%
\pgfsetlinewidth{0.301125pt}%
\definecolor{currentstroke}{rgb}{0.500000,0.500000,0.500000}%
\pgfsetstrokecolor{currentstroke}%
\pgfsetstrokeopacity{0.300000}%
\pgfsetdash{}{0pt}%
\pgfpathmoveto{\pgfqpoint{0.000000in}{0.000000in}}%
\pgfpathlineto{\pgfqpoint{0.000000in}{0.000000in}}%
\pgfpathclose%
\pgfusepath{stroke,fill}%
\end{pgfscope}%
\begin{pgfscope}%
\pgfpathrectangle{\pgfqpoint{0.647939in}{0.492442in}}{\pgfqpoint{3.079299in}{3.079299in}}%
\pgfusepath{clip}%
\pgfsetroundcap%
\pgfsetroundjoin%
\pgfsetlinewidth{0.301125pt}%
\definecolor{currentstroke}{rgb}{0.500000,0.500000,0.500000}%
\pgfsetstrokecolor{currentstroke}%
\pgfsetstrokeopacity{0.300000}%
\pgfsetdash{}{0pt}%
\pgfpathmoveto{\pgfqpoint{2.495388in}{0.733343in}}%
\pgfusepath{stroke}%
\end{pgfscope}%
\begin{pgfscope}%
\pgfpathrectangle{\pgfqpoint{0.647939in}{0.492442in}}{\pgfqpoint{3.079299in}{3.079299in}}%
\pgfusepath{clip}%
\pgfsetroundcap%
\pgfsetroundjoin%
\definecolor{currentfill}{rgb}{0.500000,0.500000,0.500000}%
\pgfsetfillcolor{currentfill}%
\pgfsetfillopacity{0.300000}%
\pgfsetlinewidth{0.301125pt}%
\definecolor{currentstroke}{rgb}{0.500000,0.500000,0.500000}%
\pgfsetstrokecolor{currentstroke}%
\pgfsetstrokeopacity{0.300000}%
\pgfsetdash{}{0pt}%
\pgfpathmoveto{\pgfqpoint{0.000000in}{0.000000in}}%
\pgfpathlineto{\pgfqpoint{0.000000in}{0.000000in}}%
\pgfpathclose%
\pgfusepath{stroke,fill}%
\end{pgfscope}%
\begin{pgfscope}%
\pgfpathrectangle{\pgfqpoint{0.647939in}{0.492442in}}{\pgfqpoint{3.079299in}{3.079299in}}%
\pgfusepath{clip}%
\pgfsetroundcap%
\pgfsetroundjoin%
\pgfsetlinewidth{0.301125pt}%
\definecolor{currentstroke}{rgb}{0.500000,0.500000,0.500000}%
\pgfsetstrokecolor{currentstroke}%
\pgfsetstrokeopacity{0.300000}%
\pgfsetdash{}{0pt}%
\pgfpathmoveto{\pgfqpoint{2.582343in}{0.879113in}}%
\pgfusepath{stroke}%
\end{pgfscope}%
\begin{pgfscope}%
\pgfpathrectangle{\pgfqpoint{0.647939in}{0.492442in}}{\pgfqpoint{3.079299in}{3.079299in}}%
\pgfusepath{clip}%
\pgfsetroundcap%
\pgfsetroundjoin%
\definecolor{currentfill}{rgb}{0.500000,0.500000,0.500000}%
\pgfsetfillcolor{currentfill}%
\pgfsetfillopacity{0.300000}%
\pgfsetlinewidth{0.301125pt}%
\definecolor{currentstroke}{rgb}{0.500000,0.500000,0.500000}%
\pgfsetstrokecolor{currentstroke}%
\pgfsetstrokeopacity{0.300000}%
\pgfsetdash{}{0pt}%
\pgfpathmoveto{\pgfqpoint{0.000000in}{0.000000in}}%
\pgfpathlineto{\pgfqpoint{0.000000in}{0.000000in}}%
\pgfpathclose%
\pgfusepath{stroke,fill}%
\end{pgfscope}%
\begin{pgfscope}%
\pgfpathrectangle{\pgfqpoint{0.647939in}{0.492442in}}{\pgfqpoint{3.079299in}{3.079299in}}%
\pgfusepath{clip}%
\pgfsetroundcap%
\pgfsetroundjoin%
\pgfsetlinewidth{0.301125pt}%
\definecolor{currentstroke}{rgb}{0.500000,0.500000,0.500000}%
\pgfsetstrokecolor{currentstroke}%
\pgfsetstrokeopacity{0.300000}%
\pgfsetdash{}{0pt}%
\pgfpathmoveto{\pgfqpoint{2.654079in}{0.954588in}}%
\pgfusepath{stroke}%
\end{pgfscope}%
\begin{pgfscope}%
\pgfpathrectangle{\pgfqpoint{0.647939in}{0.492442in}}{\pgfqpoint{3.079299in}{3.079299in}}%
\pgfusepath{clip}%
\pgfsetroundcap%
\pgfsetroundjoin%
\definecolor{currentfill}{rgb}{0.500000,0.500000,0.500000}%
\pgfsetfillcolor{currentfill}%
\pgfsetfillopacity{0.300000}%
\pgfsetlinewidth{0.301125pt}%
\definecolor{currentstroke}{rgb}{0.500000,0.500000,0.500000}%
\pgfsetstrokecolor{currentstroke}%
\pgfsetstrokeopacity{0.300000}%
\pgfsetdash{}{0pt}%
\pgfpathmoveto{\pgfqpoint{0.000000in}{0.000000in}}%
\pgfpathlineto{\pgfqpoint{0.000000in}{0.000000in}}%
\pgfpathclose%
\pgfusepath{stroke,fill}%
\end{pgfscope}%
\begin{pgfscope}%
\pgfpathrectangle{\pgfqpoint{0.647939in}{0.492442in}}{\pgfqpoint{3.079299in}{3.079299in}}%
\pgfusepath{clip}%
\pgfsetroundcap%
\pgfsetroundjoin%
\pgfsetlinewidth{0.301125pt}%
\definecolor{currentstroke}{rgb}{0.500000,0.500000,0.500000}%
\pgfsetstrokecolor{currentstroke}%
\pgfsetstrokeopacity{0.300000}%
\pgfsetdash{}{0pt}%
\pgfpathmoveto{\pgfqpoint{2.725805in}{1.027350in}}%
\pgfusepath{stroke}%
\end{pgfscope}%
\begin{pgfscope}%
\pgfpathrectangle{\pgfqpoint{0.647939in}{0.492442in}}{\pgfqpoint{3.079299in}{3.079299in}}%
\pgfusepath{clip}%
\pgfsetroundcap%
\pgfsetroundjoin%
\definecolor{currentfill}{rgb}{0.500000,0.500000,0.500000}%
\pgfsetfillcolor{currentfill}%
\pgfsetfillopacity{0.300000}%
\pgfsetlinewidth{0.301125pt}%
\definecolor{currentstroke}{rgb}{0.500000,0.500000,0.500000}%
\pgfsetstrokecolor{currentstroke}%
\pgfsetstrokeopacity{0.300000}%
\pgfsetdash{}{0pt}%
\pgfpathmoveto{\pgfqpoint{0.000000in}{0.000000in}}%
\pgfpathlineto{\pgfqpoint{0.000000in}{0.000000in}}%
\pgfpathclose%
\pgfusepath{stroke,fill}%
\end{pgfscope}%
\begin{pgfscope}%
\pgfpathrectangle{\pgfqpoint{0.647939in}{0.492442in}}{\pgfqpoint{3.079299in}{3.079299in}}%
\pgfusepath{clip}%
\pgfsetroundcap%
\pgfsetroundjoin%
\pgfsetlinewidth{0.301125pt}%
\definecolor{currentstroke}{rgb}{0.500000,0.500000,0.500000}%
\pgfsetstrokecolor{currentstroke}%
\pgfsetstrokeopacity{0.300000}%
\pgfsetdash{}{0pt}%
\pgfpathmoveto{\pgfqpoint{2.595161in}{1.131472in}}%
\pgfusepath{stroke}%
\end{pgfscope}%
\begin{pgfscope}%
\pgfpathrectangle{\pgfqpoint{0.647939in}{0.492442in}}{\pgfqpoint{3.079299in}{3.079299in}}%
\pgfusepath{clip}%
\pgfsetroundcap%
\pgfsetroundjoin%
\definecolor{currentfill}{rgb}{0.500000,0.500000,0.500000}%
\pgfsetfillcolor{currentfill}%
\pgfsetfillopacity{0.300000}%
\pgfsetlinewidth{0.301125pt}%
\definecolor{currentstroke}{rgb}{0.500000,0.500000,0.500000}%
\pgfsetstrokecolor{currentstroke}%
\pgfsetstrokeopacity{0.300000}%
\pgfsetdash{}{0pt}%
\pgfpathmoveto{\pgfqpoint{0.000000in}{0.000000in}}%
\pgfpathlineto{\pgfqpoint{0.000000in}{0.000000in}}%
\pgfpathclose%
\pgfusepath{stroke,fill}%
\end{pgfscope}%
\begin{pgfscope}%
\pgfpathrectangle{\pgfqpoint{0.647939in}{0.492442in}}{\pgfqpoint{3.079299in}{3.079299in}}%
\pgfusepath{clip}%
\pgfsetroundcap%
\pgfsetroundjoin%
\pgfsetlinewidth{0.301125pt}%
\definecolor{currentstroke}{rgb}{0.500000,0.500000,0.500000}%
\pgfsetstrokecolor{currentstroke}%
\pgfsetstrokeopacity{0.300000}%
\pgfsetdash{}{0pt}%
\pgfpathmoveto{\pgfqpoint{2.802373in}{1.181097in}}%
\pgfusepath{stroke}%
\end{pgfscope}%
\begin{pgfscope}%
\pgfpathrectangle{\pgfqpoint{0.647939in}{0.492442in}}{\pgfqpoint{3.079299in}{3.079299in}}%
\pgfusepath{clip}%
\pgfsetroundcap%
\pgfsetroundjoin%
\definecolor{currentfill}{rgb}{0.500000,0.500000,0.500000}%
\pgfsetfillcolor{currentfill}%
\pgfsetfillopacity{0.300000}%
\pgfsetlinewidth{0.301125pt}%
\definecolor{currentstroke}{rgb}{0.500000,0.500000,0.500000}%
\pgfsetstrokecolor{currentstroke}%
\pgfsetstrokeopacity{0.300000}%
\pgfsetdash{}{0pt}%
\pgfpathmoveto{\pgfqpoint{0.000000in}{0.000000in}}%
\pgfpathlineto{\pgfqpoint{0.000000in}{0.000000in}}%
\pgfpathclose%
\pgfusepath{stroke,fill}%
\end{pgfscope}%
\begin{pgfscope}%
\pgfpathrectangle{\pgfqpoint{0.647939in}{0.492442in}}{\pgfqpoint{3.079299in}{3.079299in}}%
\pgfusepath{clip}%
\pgfsetroundcap%
\pgfsetroundjoin%
\pgfsetlinewidth{0.301125pt}%
\definecolor{currentstroke}{rgb}{0.500000,0.500000,0.500000}%
\pgfsetstrokecolor{currentstroke}%
\pgfsetstrokeopacity{0.300000}%
\pgfsetdash{}{0pt}%
\pgfpathmoveto{\pgfqpoint{2.741829in}{1.283150in}}%
\pgfusepath{stroke}%
\end{pgfscope}%
\begin{pgfscope}%
\pgfpathrectangle{\pgfqpoint{0.647939in}{0.492442in}}{\pgfqpoint{3.079299in}{3.079299in}}%
\pgfusepath{clip}%
\pgfsetroundcap%
\pgfsetroundjoin%
\definecolor{currentfill}{rgb}{0.500000,0.500000,0.500000}%
\pgfsetfillcolor{currentfill}%
\pgfsetfillopacity{0.300000}%
\pgfsetlinewidth{0.301125pt}%
\definecolor{currentstroke}{rgb}{0.500000,0.500000,0.500000}%
\pgfsetstrokecolor{currentstroke}%
\pgfsetstrokeopacity{0.300000}%
\pgfsetdash{}{0pt}%
\pgfpathmoveto{\pgfqpoint{0.000000in}{0.000000in}}%
\pgfpathlineto{\pgfqpoint{0.000000in}{0.000000in}}%
\pgfpathclose%
\pgfusepath{stroke,fill}%
\end{pgfscope}%
\begin{pgfscope}%
\pgfpathrectangle{\pgfqpoint{0.647939in}{0.492442in}}{\pgfqpoint{3.079299in}{3.079299in}}%
\pgfusepath{clip}%
\pgfsetroundcap%
\pgfsetroundjoin%
\pgfsetlinewidth{0.301125pt}%
\definecolor{currentstroke}{rgb}{0.500000,0.500000,0.500000}%
\pgfsetstrokecolor{currentstroke}%
\pgfsetstrokeopacity{0.300000}%
\pgfsetdash{}{0pt}%
\pgfpathmoveto{\pgfqpoint{2.814922in}{1.353572in}}%
\pgfusepath{stroke}%
\end{pgfscope}%
\begin{pgfscope}%
\pgfpathrectangle{\pgfqpoint{0.647939in}{0.492442in}}{\pgfqpoint{3.079299in}{3.079299in}}%
\pgfusepath{clip}%
\pgfsetroundcap%
\pgfsetroundjoin%
\definecolor{currentfill}{rgb}{0.500000,0.500000,0.500000}%
\pgfsetfillcolor{currentfill}%
\pgfsetfillopacity{0.300000}%
\pgfsetlinewidth{0.301125pt}%
\definecolor{currentstroke}{rgb}{0.500000,0.500000,0.500000}%
\pgfsetstrokecolor{currentstroke}%
\pgfsetstrokeopacity{0.300000}%
\pgfsetdash{}{0pt}%
\pgfpathmoveto{\pgfqpoint{0.000000in}{0.000000in}}%
\pgfpathlineto{\pgfqpoint{0.000000in}{0.000000in}}%
\pgfpathclose%
\pgfusepath{stroke,fill}%
\end{pgfscope}%
\begin{pgfscope}%
\pgfpathrectangle{\pgfqpoint{0.647939in}{0.492442in}}{\pgfqpoint{3.079299in}{3.079299in}}%
\pgfusepath{clip}%
\pgfsetroundcap%
\pgfsetroundjoin%
\pgfsetlinewidth{0.301125pt}%
\definecolor{currentstroke}{rgb}{0.500000,0.500000,0.500000}%
\pgfsetstrokecolor{currentstroke}%
\pgfsetstrokeopacity{0.300000}%
\pgfsetdash{}{0pt}%
\pgfpathmoveto{\pgfqpoint{2.757387in}{1.462336in}}%
\pgfusepath{stroke}%
\end{pgfscope}%
\begin{pgfscope}%
\pgfpathrectangle{\pgfqpoint{0.647939in}{0.492442in}}{\pgfqpoint{3.079299in}{3.079299in}}%
\pgfusepath{clip}%
\pgfsetroundcap%
\pgfsetroundjoin%
\definecolor{currentfill}{rgb}{0.500000,0.500000,0.500000}%
\pgfsetfillcolor{currentfill}%
\pgfsetfillopacity{0.300000}%
\pgfsetlinewidth{0.301125pt}%
\definecolor{currentstroke}{rgb}{0.500000,0.500000,0.500000}%
\pgfsetstrokecolor{currentstroke}%
\pgfsetstrokeopacity{0.300000}%
\pgfsetdash{}{0pt}%
\pgfpathmoveto{\pgfqpoint{0.000000in}{0.000000in}}%
\pgfpathlineto{\pgfqpoint{0.000000in}{0.000000in}}%
\pgfpathclose%
\pgfusepath{stroke,fill}%
\end{pgfscope}%
\begin{pgfscope}%
\pgfpathrectangle{\pgfqpoint{0.647939in}{0.492442in}}{\pgfqpoint{3.079299in}{3.079299in}}%
\pgfusepath{clip}%
\pgfsetroundcap%
\pgfsetroundjoin%
\pgfsetlinewidth{0.301125pt}%
\definecolor{currentstroke}{rgb}{0.500000,0.500000,0.500000}%
\pgfsetstrokecolor{currentstroke}%
\pgfsetstrokeopacity{0.300000}%
\pgfsetdash{}{0pt}%
\pgfpathmoveto{\pgfqpoint{2.767434in}{1.555465in}}%
\pgfusepath{stroke}%
\end{pgfscope}%
\begin{pgfscope}%
\pgfpathrectangle{\pgfqpoint{0.647939in}{0.492442in}}{\pgfqpoint{3.079299in}{3.079299in}}%
\pgfusepath{clip}%
\pgfsetroundcap%
\pgfsetroundjoin%
\definecolor{currentfill}{rgb}{0.500000,0.500000,0.500000}%
\pgfsetfillcolor{currentfill}%
\pgfsetfillopacity{0.300000}%
\pgfsetlinewidth{0.301125pt}%
\definecolor{currentstroke}{rgb}{0.500000,0.500000,0.500000}%
\pgfsetstrokecolor{currentstroke}%
\pgfsetstrokeopacity{0.300000}%
\pgfsetdash{}{0pt}%
\pgfpathmoveto{\pgfqpoint{0.000000in}{0.000000in}}%
\pgfpathlineto{\pgfqpoint{0.000000in}{0.000000in}}%
\pgfpathclose%
\pgfusepath{stroke,fill}%
\end{pgfscope}%
\begin{pgfscope}%
\pgfpathrectangle{\pgfqpoint{0.647939in}{0.492442in}}{\pgfqpoint{3.079299in}{3.079299in}}%
\pgfusepath{clip}%
\pgfsetroundcap%
\pgfsetroundjoin%
\pgfsetlinewidth{0.301125pt}%
\definecolor{currentstroke}{rgb}{0.500000,0.500000,0.500000}%
\pgfsetstrokecolor{currentstroke}%
\pgfsetstrokeopacity{0.300000}%
\pgfsetdash{}{0pt}%
\pgfpathmoveto{\pgfqpoint{2.843311in}{1.626669in}}%
\pgfusepath{stroke}%
\end{pgfscope}%
\begin{pgfscope}%
\pgfpathrectangle{\pgfqpoint{0.647939in}{0.492442in}}{\pgfqpoint{3.079299in}{3.079299in}}%
\pgfusepath{clip}%
\pgfsetroundcap%
\pgfsetroundjoin%
\definecolor{currentfill}{rgb}{0.500000,0.500000,0.500000}%
\pgfsetfillcolor{currentfill}%
\pgfsetfillopacity{0.300000}%
\pgfsetlinewidth{0.301125pt}%
\definecolor{currentstroke}{rgb}{0.500000,0.500000,0.500000}%
\pgfsetstrokecolor{currentstroke}%
\pgfsetstrokeopacity{0.300000}%
\pgfsetdash{}{0pt}%
\pgfpathmoveto{\pgfqpoint{0.000000in}{0.000000in}}%
\pgfpathlineto{\pgfqpoint{0.000000in}{0.000000in}}%
\pgfpathclose%
\pgfusepath{stroke,fill}%
\end{pgfscope}%
\begin{pgfscope}%
\pgfpathrectangle{\pgfqpoint{0.647939in}{0.492442in}}{\pgfqpoint{3.079299in}{3.079299in}}%
\pgfusepath{clip}%
\pgfsetroundcap%
\pgfsetroundjoin%
\pgfsetlinewidth{0.301125pt}%
\definecolor{currentstroke}{rgb}{0.500000,0.500000,0.500000}%
\pgfsetstrokecolor{currentstroke}%
\pgfsetstrokeopacity{0.300000}%
\pgfsetdash{}{0pt}%
\pgfpathmoveto{\pgfqpoint{2.856846in}{1.722879in}}%
\pgfusepath{stroke}%
\end{pgfscope}%
\begin{pgfscope}%
\pgfpathrectangle{\pgfqpoint{0.647939in}{0.492442in}}{\pgfqpoint{3.079299in}{3.079299in}}%
\pgfusepath{clip}%
\pgfsetroundcap%
\pgfsetroundjoin%
\definecolor{currentfill}{rgb}{0.500000,0.500000,0.500000}%
\pgfsetfillcolor{currentfill}%
\pgfsetfillopacity{0.300000}%
\pgfsetlinewidth{0.301125pt}%
\definecolor{currentstroke}{rgb}{0.500000,0.500000,0.500000}%
\pgfsetstrokecolor{currentstroke}%
\pgfsetstrokeopacity{0.300000}%
\pgfsetdash{}{0pt}%
\pgfpathmoveto{\pgfqpoint{0.000000in}{0.000000in}}%
\pgfpathlineto{\pgfqpoint{0.000000in}{0.000000in}}%
\pgfpathclose%
\pgfusepath{stroke,fill}%
\end{pgfscope}%
\begin{pgfscope}%
\pgfpathrectangle{\pgfqpoint{0.647939in}{0.492442in}}{\pgfqpoint{3.079299in}{3.079299in}}%
\pgfusepath{clip}%
\pgfsetroundcap%
\pgfsetroundjoin%
\pgfsetlinewidth{0.301125pt}%
\definecolor{currentstroke}{rgb}{0.500000,0.500000,0.500000}%
\pgfsetstrokecolor{currentstroke}%
\pgfsetstrokeopacity{0.300000}%
\pgfsetdash{}{0pt}%
\pgfpathmoveto{\pgfqpoint{2.932797in}{1.788289in}}%
\pgfusepath{stroke}%
\end{pgfscope}%
\begin{pgfscope}%
\pgfpathrectangle{\pgfqpoint{0.647939in}{0.492442in}}{\pgfqpoint{3.079299in}{3.079299in}}%
\pgfusepath{clip}%
\pgfsetroundcap%
\pgfsetroundjoin%
\definecolor{currentfill}{rgb}{0.500000,0.500000,0.500000}%
\pgfsetfillcolor{currentfill}%
\pgfsetfillopacity{0.300000}%
\pgfsetlinewidth{0.301125pt}%
\definecolor{currentstroke}{rgb}{0.500000,0.500000,0.500000}%
\pgfsetstrokecolor{currentstroke}%
\pgfsetstrokeopacity{0.300000}%
\pgfsetdash{}{0pt}%
\pgfpathmoveto{\pgfqpoint{0.000000in}{0.000000in}}%
\pgfpathlineto{\pgfqpoint{0.000000in}{0.000000in}}%
\pgfpathclose%
\pgfusepath{stroke,fill}%
\end{pgfscope}%
\begin{pgfscope}%
\pgfpathrectangle{\pgfqpoint{0.647939in}{0.492442in}}{\pgfqpoint{3.079299in}{3.079299in}}%
\pgfusepath{clip}%
\pgfsetroundcap%
\pgfsetroundjoin%
\pgfsetlinewidth{0.301125pt}%
\definecolor{currentstroke}{rgb}{0.500000,0.500000,0.500000}%
\pgfsetstrokecolor{currentstroke}%
\pgfsetstrokeopacity{0.300000}%
\pgfsetdash{}{0pt}%
\pgfpathmoveto{\pgfqpoint{3.006788in}{1.847791in}}%
\pgfusepath{stroke}%
\end{pgfscope}%
\begin{pgfscope}%
\pgfpathrectangle{\pgfqpoint{0.647939in}{0.492442in}}{\pgfqpoint{3.079299in}{3.079299in}}%
\pgfusepath{clip}%
\pgfsetroundcap%
\pgfsetroundjoin%
\definecolor{currentfill}{rgb}{0.500000,0.500000,0.500000}%
\pgfsetfillcolor{currentfill}%
\pgfsetfillopacity{0.300000}%
\pgfsetlinewidth{0.301125pt}%
\definecolor{currentstroke}{rgb}{0.500000,0.500000,0.500000}%
\pgfsetstrokecolor{currentstroke}%
\pgfsetstrokeopacity{0.300000}%
\pgfsetdash{}{0pt}%
\pgfpathmoveto{\pgfqpoint{0.000000in}{0.000000in}}%
\pgfpathlineto{\pgfqpoint{0.000000in}{0.000000in}}%
\pgfpathclose%
\pgfusepath{stroke,fill}%
\end{pgfscope}%
\begin{pgfscope}%
\pgfpathrectangle{\pgfqpoint{0.647939in}{0.492442in}}{\pgfqpoint{3.079299in}{3.079299in}}%
\pgfusepath{clip}%
\pgfsetroundcap%
\pgfsetroundjoin%
\pgfsetlinewidth{0.301125pt}%
\definecolor{currentstroke}{rgb}{0.500000,0.500000,0.500000}%
\pgfsetstrokecolor{currentstroke}%
\pgfsetstrokeopacity{0.300000}%
\pgfsetdash{}{0pt}%
\pgfpathmoveto{\pgfqpoint{2.769555in}{2.499114in}}%
\pgfusepath{stroke}%
\end{pgfscope}%
\begin{pgfscope}%
\pgfpathrectangle{\pgfqpoint{0.647939in}{0.492442in}}{\pgfqpoint{3.079299in}{3.079299in}}%
\pgfusepath{clip}%
\pgfsetroundcap%
\pgfsetroundjoin%
\definecolor{currentfill}{rgb}{0.500000,0.500000,0.500000}%
\pgfsetfillcolor{currentfill}%
\pgfsetfillopacity{0.300000}%
\pgfsetlinewidth{0.301125pt}%
\definecolor{currentstroke}{rgb}{0.500000,0.500000,0.500000}%
\pgfsetstrokecolor{currentstroke}%
\pgfsetstrokeopacity{0.300000}%
\pgfsetdash{}{0pt}%
\pgfpathmoveto{\pgfqpoint{0.000000in}{0.000000in}}%
\pgfpathlineto{\pgfqpoint{0.000000in}{0.000000in}}%
\pgfpathclose%
\pgfusepath{stroke,fill}%
\end{pgfscope}%
\begin{pgfscope}%
\pgfpathrectangle{\pgfqpoint{0.647939in}{0.492442in}}{\pgfqpoint{3.079299in}{3.079299in}}%
\pgfusepath{clip}%
\pgfsetroundcap%
\pgfsetroundjoin%
\pgfsetlinewidth{0.301125pt}%
\definecolor{currentstroke}{rgb}{0.500000,0.500000,0.500000}%
\pgfsetstrokecolor{currentstroke}%
\pgfsetstrokeopacity{0.300000}%
\pgfsetdash{}{0pt}%
\pgfpathmoveto{\pgfqpoint{3.076589in}{2.312034in}}%
\pgfusepath{stroke}%
\end{pgfscope}%
\begin{pgfscope}%
\pgfpathrectangle{\pgfqpoint{0.647939in}{0.492442in}}{\pgfqpoint{3.079299in}{3.079299in}}%
\pgfusepath{clip}%
\pgfsetroundcap%
\pgfsetroundjoin%
\definecolor{currentfill}{rgb}{0.500000,0.500000,0.500000}%
\pgfsetfillcolor{currentfill}%
\pgfsetfillopacity{0.300000}%
\pgfsetlinewidth{0.301125pt}%
\definecolor{currentstroke}{rgb}{0.500000,0.500000,0.500000}%
\pgfsetstrokecolor{currentstroke}%
\pgfsetstrokeopacity{0.300000}%
\pgfsetdash{}{0pt}%
\pgfpathmoveto{\pgfqpoint{0.000000in}{0.000000in}}%
\pgfpathlineto{\pgfqpoint{0.000000in}{0.000000in}}%
\pgfpathclose%
\pgfusepath{stroke,fill}%
\end{pgfscope}%
\begin{pgfscope}%
\pgfpathrectangle{\pgfqpoint{0.647939in}{0.492442in}}{\pgfqpoint{3.079299in}{3.079299in}}%
\pgfusepath{clip}%
\pgfsetroundcap%
\pgfsetroundjoin%
\pgfsetlinewidth{0.301125pt}%
\definecolor{currentstroke}{rgb}{0.500000,0.500000,0.500000}%
\pgfsetstrokecolor{currentstroke}%
\pgfsetstrokeopacity{0.300000}%
\pgfsetdash{}{0pt}%
\pgfpathmoveto{\pgfqpoint{3.173052in}{2.666267in}}%
\pgfusepath{stroke}%
\end{pgfscope}%
\begin{pgfscope}%
\pgfpathrectangle{\pgfqpoint{0.647939in}{0.492442in}}{\pgfqpoint{3.079299in}{3.079299in}}%
\pgfusepath{clip}%
\pgfsetroundcap%
\pgfsetroundjoin%
\definecolor{currentfill}{rgb}{0.500000,0.500000,0.500000}%
\pgfsetfillcolor{currentfill}%
\pgfsetfillopacity{0.300000}%
\pgfsetlinewidth{0.301125pt}%
\definecolor{currentstroke}{rgb}{0.500000,0.500000,0.500000}%
\pgfsetstrokecolor{currentstroke}%
\pgfsetstrokeopacity{0.300000}%
\pgfsetdash{}{0pt}%
\pgfpathmoveto{\pgfqpoint{0.000000in}{0.000000in}}%
\pgfpathlineto{\pgfqpoint{0.000000in}{0.000000in}}%
\pgfpathclose%
\pgfusepath{stroke,fill}%
\end{pgfscope}%
\begin{pgfscope}%
\pgfpathrectangle{\pgfqpoint{0.647939in}{0.492442in}}{\pgfqpoint{3.079299in}{3.079299in}}%
\pgfusepath{clip}%
\pgfsetroundcap%
\pgfsetroundjoin%
\pgfsetlinewidth{0.301125pt}%
\definecolor{currentstroke}{rgb}{0.500000,0.500000,0.500000}%
\pgfsetstrokecolor{currentstroke}%
\pgfsetstrokeopacity{0.300000}%
\pgfsetdash{}{0pt}%
\pgfpathmoveto{\pgfqpoint{3.313632in}{2.669403in}}%
\pgfusepath{stroke}%
\end{pgfscope}%
\begin{pgfscope}%
\pgfpathrectangle{\pgfqpoint{0.647939in}{0.492442in}}{\pgfqpoint{3.079299in}{3.079299in}}%
\pgfusepath{clip}%
\pgfsetroundcap%
\pgfsetroundjoin%
\definecolor{currentfill}{rgb}{0.500000,0.500000,0.500000}%
\pgfsetfillcolor{currentfill}%
\pgfsetfillopacity{0.300000}%
\pgfsetlinewidth{0.301125pt}%
\definecolor{currentstroke}{rgb}{0.500000,0.500000,0.500000}%
\pgfsetstrokecolor{currentstroke}%
\pgfsetstrokeopacity{0.300000}%
\pgfsetdash{}{0pt}%
\pgfpathmoveto{\pgfqpoint{0.000000in}{0.000000in}}%
\pgfpathlineto{\pgfqpoint{0.000000in}{0.000000in}}%
\pgfpathclose%
\pgfusepath{stroke,fill}%
\end{pgfscope}%
\begin{pgfscope}%
\pgfpathrectangle{\pgfqpoint{0.647939in}{0.492442in}}{\pgfqpoint{3.079299in}{3.079299in}}%
\pgfusepath{clip}%
\pgfsetroundcap%
\pgfsetroundjoin%
\pgfsetlinewidth{0.301125pt}%
\definecolor{currentstroke}{rgb}{0.500000,0.500000,0.500000}%
\pgfsetstrokecolor{currentstroke}%
\pgfsetstrokeopacity{0.300000}%
\pgfsetdash{}{0pt}%
\pgfpathmoveto{\pgfqpoint{3.563240in}{2.196735in}}%
\pgfusepath{stroke}%
\end{pgfscope}%
\begin{pgfscope}%
\pgfpathrectangle{\pgfqpoint{0.647939in}{0.492442in}}{\pgfqpoint{3.079299in}{3.079299in}}%
\pgfusepath{clip}%
\pgfsetroundcap%
\pgfsetroundjoin%
\definecolor{currentfill}{rgb}{0.500000,0.500000,0.500000}%
\pgfsetfillcolor{currentfill}%
\pgfsetfillopacity{0.300000}%
\pgfsetlinewidth{0.301125pt}%
\definecolor{currentstroke}{rgb}{0.500000,0.500000,0.500000}%
\pgfsetstrokecolor{currentstroke}%
\pgfsetstrokeopacity{0.300000}%
\pgfsetdash{}{0pt}%
\pgfpathmoveto{\pgfqpoint{0.000000in}{0.000000in}}%
\pgfpathlineto{\pgfqpoint{0.000000in}{0.000000in}}%
\pgfpathclose%
\pgfusepath{stroke,fill}%
\end{pgfscope}%
\begin{pgfscope}%
\pgfpathrectangle{\pgfqpoint{0.647939in}{0.492442in}}{\pgfqpoint{3.079299in}{3.079299in}}%
\pgfusepath{clip}%
\pgfsetroundcap%
\pgfsetroundjoin%
\pgfsetlinewidth{0.301125pt}%
\definecolor{currentstroke}{rgb}{0.500000,0.500000,0.500000}%
\pgfsetstrokecolor{currentstroke}%
\pgfsetstrokeopacity{0.300000}%
\pgfsetdash{}{0pt}%
\pgfpathmoveto{\pgfqpoint{3.487036in}{2.748236in}}%
\pgfusepath{stroke}%
\end{pgfscope}%
\begin{pgfscope}%
\pgfpathrectangle{\pgfqpoint{0.647939in}{0.492442in}}{\pgfqpoint{3.079299in}{3.079299in}}%
\pgfusepath{clip}%
\pgfsetroundcap%
\pgfsetroundjoin%
\definecolor{currentfill}{rgb}{0.500000,0.500000,0.500000}%
\pgfsetfillcolor{currentfill}%
\pgfsetfillopacity{0.300000}%
\pgfsetlinewidth{0.301125pt}%
\definecolor{currentstroke}{rgb}{0.500000,0.500000,0.500000}%
\pgfsetstrokecolor{currentstroke}%
\pgfsetstrokeopacity{0.300000}%
\pgfsetdash{}{0pt}%
\pgfpathmoveto{\pgfqpoint{0.000000in}{0.000000in}}%
\pgfpathlineto{\pgfqpoint{0.000000in}{0.000000in}}%
\pgfpathclose%
\pgfusepath{stroke,fill}%
\end{pgfscope}%
\begin{pgfscope}%
\pgfpathrectangle{\pgfqpoint{0.647939in}{0.492442in}}{\pgfqpoint{3.079299in}{3.079299in}}%
\pgfusepath{clip}%
\pgfsetroundcap%
\pgfsetroundjoin%
\pgfsetlinewidth{0.301125pt}%
\definecolor{currentstroke}{rgb}{0.500000,0.500000,0.500000}%
\pgfsetstrokecolor{currentstroke}%
\pgfsetstrokeopacity{0.300000}%
\pgfsetdash{}{0pt}%
\pgfpathmoveto{\pgfqpoint{3.580127in}{2.716565in}}%
\pgfusepath{stroke}%
\end{pgfscope}%
\begin{pgfscope}%
\pgfpathrectangle{\pgfqpoint{0.647939in}{0.492442in}}{\pgfqpoint{3.079299in}{3.079299in}}%
\pgfusepath{clip}%
\pgfsetroundcap%
\pgfsetroundjoin%
\definecolor{currentfill}{rgb}{0.500000,0.500000,0.500000}%
\pgfsetfillcolor{currentfill}%
\pgfsetfillopacity{0.300000}%
\pgfsetlinewidth{0.301125pt}%
\definecolor{currentstroke}{rgb}{0.500000,0.500000,0.500000}%
\pgfsetstrokecolor{currentstroke}%
\pgfsetstrokeopacity{0.300000}%
\pgfsetdash{}{0pt}%
\pgfpathmoveto{\pgfqpoint{0.000000in}{0.000000in}}%
\pgfpathlineto{\pgfqpoint{0.000000in}{0.000000in}}%
\pgfpathclose%
\pgfusepath{stroke,fill}%
\end{pgfscope}%
\begin{pgfscope}%
\pgfpathrectangle{\pgfqpoint{0.647939in}{0.492442in}}{\pgfqpoint{3.079299in}{3.079299in}}%
\pgfusepath{clip}%
\pgfsetroundcap%
\pgfsetroundjoin%
\pgfsetlinewidth{0.301125pt}%
\definecolor{currentstroke}{rgb}{0.500000,0.500000,0.500000}%
\pgfsetstrokecolor{currentstroke}%
\pgfsetstrokeopacity{0.300000}%
\pgfsetdash{}{0pt}%
\pgfpathmoveto{\pgfqpoint{3.654280in}{2.741050in}}%
\pgfusepath{stroke}%
\end{pgfscope}%
\begin{pgfscope}%
\pgfpathrectangle{\pgfqpoint{0.647939in}{0.492442in}}{\pgfqpoint{3.079299in}{3.079299in}}%
\pgfusepath{clip}%
\pgfsetroundcap%
\pgfsetroundjoin%
\definecolor{currentfill}{rgb}{0.500000,0.500000,0.500000}%
\pgfsetfillcolor{currentfill}%
\pgfsetfillopacity{0.300000}%
\pgfsetlinewidth{0.301125pt}%
\definecolor{currentstroke}{rgb}{0.500000,0.500000,0.500000}%
\pgfsetstrokecolor{currentstroke}%
\pgfsetstrokeopacity{0.300000}%
\pgfsetdash{}{0pt}%
\pgfpathmoveto{\pgfqpoint{0.000000in}{0.000000in}}%
\pgfpathlineto{\pgfqpoint{0.000000in}{0.000000in}}%
\pgfpathclose%
\pgfusepath{stroke,fill}%
\end{pgfscope}%
\begin{pgfscope}%
\pgfpathrectangle{\pgfqpoint{0.647939in}{0.492442in}}{\pgfqpoint{3.079299in}{3.079299in}}%
\pgfusepath{clip}%
\pgfsetroundcap%
\pgfsetroundjoin%
\pgfsetlinewidth{0.301125pt}%
\definecolor{currentstroke}{rgb}{0.500000,0.500000,0.500000}%
\pgfsetstrokecolor{currentstroke}%
\pgfsetstrokeopacity{0.300000}%
\pgfsetdash{}{0pt}%
\pgfpathmoveto{\pgfqpoint{3.698502in}{2.767738in}}%
\pgfusepath{stroke}%
\end{pgfscope}%
\begin{pgfscope}%
\pgfpathrectangle{\pgfqpoint{0.647939in}{0.492442in}}{\pgfqpoint{3.079299in}{3.079299in}}%
\pgfusepath{clip}%
\pgfsetroundcap%
\pgfsetroundjoin%
\definecolor{currentfill}{rgb}{0.500000,0.500000,0.500000}%
\pgfsetfillcolor{currentfill}%
\pgfsetfillopacity{0.300000}%
\pgfsetlinewidth{0.301125pt}%
\definecolor{currentstroke}{rgb}{0.500000,0.500000,0.500000}%
\pgfsetstrokecolor{currentstroke}%
\pgfsetstrokeopacity{0.300000}%
\pgfsetdash{}{0pt}%
\pgfpathmoveto{\pgfqpoint{0.000000in}{0.000000in}}%
\pgfpathlineto{\pgfqpoint{0.000000in}{0.000000in}}%
\pgfpathclose%
\pgfusepath{stroke,fill}%
\end{pgfscope}%
\begin{pgfscope}%
\pgfpathrectangle{\pgfqpoint{0.647939in}{0.492442in}}{\pgfqpoint{3.079299in}{3.079299in}}%
\pgfusepath{clip}%
\pgfsetroundcap%
\pgfsetroundjoin%
\pgfsetlinewidth{0.301125pt}%
\definecolor{currentstroke}{rgb}{0.500000,0.500000,0.500000}%
\pgfsetstrokecolor{currentstroke}%
\pgfsetstrokeopacity{0.300000}%
\pgfsetdash{}{0pt}%
\pgfpathmoveto{\pgfqpoint{3.610009in}{3.228898in}}%
\pgfusepath{stroke}%
\end{pgfscope}%
\begin{pgfscope}%
\pgfpathrectangle{\pgfqpoint{0.647939in}{0.492442in}}{\pgfqpoint{3.079299in}{3.079299in}}%
\pgfusepath{clip}%
\pgfsetroundcap%
\pgfsetroundjoin%
\definecolor{currentfill}{rgb}{0.500000,0.500000,0.500000}%
\pgfsetfillcolor{currentfill}%
\pgfsetfillopacity{0.300000}%
\pgfsetlinewidth{0.301125pt}%
\definecolor{currentstroke}{rgb}{0.500000,0.500000,0.500000}%
\pgfsetstrokecolor{currentstroke}%
\pgfsetstrokeopacity{0.300000}%
\pgfsetdash{}{0pt}%
\pgfpathmoveto{\pgfqpoint{0.000000in}{0.000000in}}%
\pgfpathlineto{\pgfqpoint{0.000000in}{0.000000in}}%
\pgfpathclose%
\pgfusepath{stroke,fill}%
\end{pgfscope}%
\begin{pgfscope}%
\pgfpathrectangle{\pgfqpoint{0.647939in}{0.492442in}}{\pgfqpoint{3.079299in}{3.079299in}}%
\pgfusepath{clip}%
\pgfsetroundcap%
\pgfsetroundjoin%
\pgfsetlinewidth{0.301125pt}%
\definecolor{currentstroke}{rgb}{0.500000,0.500000,0.500000}%
\pgfsetstrokecolor{currentstroke}%
\pgfsetstrokeopacity{0.300000}%
\pgfsetdash{}{0pt}%
\pgfpathmoveto{\pgfqpoint{3.473283in}{3.408914in}}%
\pgfusepath{stroke}%
\end{pgfscope}%
\begin{pgfscope}%
\pgfpathrectangle{\pgfqpoint{0.647939in}{0.492442in}}{\pgfqpoint{3.079299in}{3.079299in}}%
\pgfusepath{clip}%
\pgfsetroundcap%
\pgfsetroundjoin%
\definecolor{currentfill}{rgb}{0.500000,0.500000,0.500000}%
\pgfsetfillcolor{currentfill}%
\pgfsetfillopacity{0.300000}%
\pgfsetlinewidth{0.301125pt}%
\definecolor{currentstroke}{rgb}{0.500000,0.500000,0.500000}%
\pgfsetstrokecolor{currentstroke}%
\pgfsetstrokeopacity{0.300000}%
\pgfsetdash{}{0pt}%
\pgfpathmoveto{\pgfqpoint{0.000000in}{0.000000in}}%
\pgfpathlineto{\pgfqpoint{0.000000in}{0.000000in}}%
\pgfpathclose%
\pgfusepath{stroke,fill}%
\end{pgfscope}%
\begin{pgfscope}%
\pgfpathrectangle{\pgfqpoint{0.647939in}{0.492442in}}{\pgfqpoint{3.079299in}{3.079299in}}%
\pgfusepath{clip}%
\pgfsetroundcap%
\pgfsetroundjoin%
\pgfsetlinewidth{0.301125pt}%
\definecolor{currentstroke}{rgb}{0.500000,0.500000,0.500000}%
\pgfsetstrokecolor{currentstroke}%
\pgfsetstrokeopacity{0.300000}%
\pgfsetdash{}{0pt}%
\pgfpathmoveto{\pgfqpoint{2.149266in}{2.884615in}}%
\pgfusepath{stroke}%
\end{pgfscope}%
\begin{pgfscope}%
\pgfpathrectangle{\pgfqpoint{0.647939in}{0.492442in}}{\pgfqpoint{3.079299in}{3.079299in}}%
\pgfusepath{clip}%
\pgfsetroundcap%
\pgfsetroundjoin%
\definecolor{currentfill}{rgb}{0.500000,0.500000,0.500000}%
\pgfsetfillcolor{currentfill}%
\pgfsetfillopacity{0.300000}%
\pgfsetlinewidth{0.301125pt}%
\definecolor{currentstroke}{rgb}{0.500000,0.500000,0.500000}%
\pgfsetstrokecolor{currentstroke}%
\pgfsetstrokeopacity{0.300000}%
\pgfsetdash{}{0pt}%
\pgfpathmoveto{\pgfqpoint{0.000000in}{0.000000in}}%
\pgfpathlineto{\pgfqpoint{0.000000in}{0.000000in}}%
\pgfpathclose%
\pgfusepath{stroke,fill}%
\end{pgfscope}%
\begin{pgfscope}%
\pgfpathrectangle{\pgfqpoint{0.647939in}{0.492442in}}{\pgfqpoint{3.079299in}{3.079299in}}%
\pgfusepath{clip}%
\pgfsetroundcap%
\pgfsetroundjoin%
\pgfsetlinewidth{0.301125pt}%
\definecolor{currentstroke}{rgb}{0.500000,0.500000,0.500000}%
\pgfsetstrokecolor{currentstroke}%
\pgfsetstrokeopacity{0.300000}%
\pgfsetdash{}{0pt}%
\pgfpathmoveto{\pgfqpoint{2.025819in}{3.146279in}}%
\pgfusepath{stroke}%
\end{pgfscope}%
\begin{pgfscope}%
\pgfpathrectangle{\pgfqpoint{0.647939in}{0.492442in}}{\pgfqpoint{3.079299in}{3.079299in}}%
\pgfusepath{clip}%
\pgfsetroundcap%
\pgfsetroundjoin%
\definecolor{currentfill}{rgb}{0.500000,0.500000,0.500000}%
\pgfsetfillcolor{currentfill}%
\pgfsetfillopacity{0.300000}%
\pgfsetlinewidth{0.301125pt}%
\definecolor{currentstroke}{rgb}{0.500000,0.500000,0.500000}%
\pgfsetstrokecolor{currentstroke}%
\pgfsetstrokeopacity{0.300000}%
\pgfsetdash{}{0pt}%
\pgfpathmoveto{\pgfqpoint{0.000000in}{0.000000in}}%
\pgfpathlineto{\pgfqpoint{0.000000in}{0.000000in}}%
\pgfpathclose%
\pgfusepath{stroke,fill}%
\end{pgfscope}%
\begin{pgfscope}%
\pgfpathrectangle{\pgfqpoint{0.647939in}{0.492442in}}{\pgfqpoint{3.079299in}{3.079299in}}%
\pgfusepath{clip}%
\pgfsetroundcap%
\pgfsetroundjoin%
\pgfsetlinewidth{0.301125pt}%
\definecolor{currentstroke}{rgb}{0.500000,0.500000,0.500000}%
\pgfsetstrokecolor{currentstroke}%
\pgfsetstrokeopacity{0.300000}%
\pgfsetdash{}{0pt}%
\pgfpathmoveto{\pgfqpoint{1.892601in}{3.305233in}}%
\pgfusepath{stroke}%
\end{pgfscope}%
\begin{pgfscope}%
\pgfpathrectangle{\pgfqpoint{0.647939in}{0.492442in}}{\pgfqpoint{3.079299in}{3.079299in}}%
\pgfusepath{clip}%
\pgfsetroundcap%
\pgfsetroundjoin%
\definecolor{currentfill}{rgb}{0.500000,0.500000,0.500000}%
\pgfsetfillcolor{currentfill}%
\pgfsetfillopacity{0.300000}%
\pgfsetlinewidth{0.301125pt}%
\definecolor{currentstroke}{rgb}{0.500000,0.500000,0.500000}%
\pgfsetstrokecolor{currentstroke}%
\pgfsetstrokeopacity{0.300000}%
\pgfsetdash{}{0pt}%
\pgfpathmoveto{\pgfqpoint{0.000000in}{0.000000in}}%
\pgfpathlineto{\pgfqpoint{0.000000in}{0.000000in}}%
\pgfpathclose%
\pgfusepath{stroke,fill}%
\end{pgfscope}%
\begin{pgfscope}%
\pgfpathrectangle{\pgfqpoint{0.647939in}{0.492442in}}{\pgfqpoint{3.079299in}{3.079299in}}%
\pgfusepath{clip}%
\pgfsetroundcap%
\pgfsetroundjoin%
\pgfsetlinewidth{0.301125pt}%
\definecolor{currentstroke}{rgb}{0.500000,0.500000,0.500000}%
\pgfsetstrokecolor{currentstroke}%
\pgfsetstrokeopacity{0.300000}%
\pgfsetdash{}{0pt}%
\pgfpathmoveto{\pgfqpoint{1.854222in}{3.415436in}}%
\pgfusepath{stroke}%
\end{pgfscope}%
\begin{pgfscope}%
\pgfpathrectangle{\pgfqpoint{0.647939in}{0.492442in}}{\pgfqpoint{3.079299in}{3.079299in}}%
\pgfusepath{clip}%
\pgfsetroundcap%
\pgfsetroundjoin%
\definecolor{currentfill}{rgb}{0.500000,0.500000,0.500000}%
\pgfsetfillcolor{currentfill}%
\pgfsetfillopacity{0.300000}%
\pgfsetlinewidth{0.301125pt}%
\definecolor{currentstroke}{rgb}{0.500000,0.500000,0.500000}%
\pgfsetstrokecolor{currentstroke}%
\pgfsetstrokeopacity{0.300000}%
\pgfsetdash{}{0pt}%
\pgfpathmoveto{\pgfqpoint{0.000000in}{0.000000in}}%
\pgfpathlineto{\pgfqpoint{0.000000in}{0.000000in}}%
\pgfpathclose%
\pgfusepath{stroke,fill}%
\end{pgfscope}%
\begin{pgfscope}%
\pgfpathrectangle{\pgfqpoint{0.647939in}{0.492442in}}{\pgfqpoint{3.079299in}{3.079299in}}%
\pgfusepath{clip}%
\pgfsetroundcap%
\pgfsetroundjoin%
\pgfsetlinewidth{0.301125pt}%
\definecolor{currentstroke}{rgb}{0.500000,0.500000,0.500000}%
\pgfsetstrokecolor{currentstroke}%
\pgfsetstrokeopacity{0.300000}%
\pgfsetdash{}{0pt}%
\pgfpathmoveto{\pgfqpoint{1.764038in}{3.487053in}}%
\pgfusepath{stroke}%
\end{pgfscope}%
\begin{pgfscope}%
\pgfpathrectangle{\pgfqpoint{0.647939in}{0.492442in}}{\pgfqpoint{3.079299in}{3.079299in}}%
\pgfusepath{clip}%
\pgfsetroundcap%
\pgfsetroundjoin%
\definecolor{currentfill}{rgb}{0.500000,0.500000,0.500000}%
\pgfsetfillcolor{currentfill}%
\pgfsetfillopacity{0.300000}%
\pgfsetlinewidth{0.301125pt}%
\definecolor{currentstroke}{rgb}{0.500000,0.500000,0.500000}%
\pgfsetstrokecolor{currentstroke}%
\pgfsetstrokeopacity{0.300000}%
\pgfsetdash{}{0pt}%
\pgfpathmoveto{\pgfqpoint{0.000000in}{0.000000in}}%
\pgfpathlineto{\pgfqpoint{0.000000in}{0.000000in}}%
\pgfpathclose%
\pgfusepath{stroke,fill}%
\end{pgfscope}%
\begin{pgfscope}%
\pgfpathrectangle{\pgfqpoint{0.647939in}{0.492442in}}{\pgfqpoint{3.079299in}{3.079299in}}%
\pgfusepath{clip}%
\pgfsetroundcap%
\pgfsetroundjoin%
\pgfsetlinewidth{0.301125pt}%
\definecolor{currentstroke}{rgb}{0.500000,0.500000,0.500000}%
\pgfsetstrokecolor{currentstroke}%
\pgfsetstrokeopacity{0.300000}%
\pgfsetdash{}{0pt}%
\pgfpathmoveto{\pgfqpoint{1.954159in}{3.560023in}}%
\pgfusepath{stroke}%
\end{pgfscope}%
\begin{pgfscope}%
\pgfpathrectangle{\pgfqpoint{0.647939in}{0.492442in}}{\pgfqpoint{3.079299in}{3.079299in}}%
\pgfusepath{clip}%
\pgfsetroundcap%
\pgfsetroundjoin%
\definecolor{currentfill}{rgb}{0.500000,0.500000,0.500000}%
\pgfsetfillcolor{currentfill}%
\pgfsetfillopacity{0.300000}%
\pgfsetlinewidth{0.301125pt}%
\definecolor{currentstroke}{rgb}{0.500000,0.500000,0.500000}%
\pgfsetstrokecolor{currentstroke}%
\pgfsetstrokeopacity{0.300000}%
\pgfsetdash{}{0pt}%
\pgfpathmoveto{\pgfqpoint{0.000000in}{0.000000in}}%
\pgfpathlineto{\pgfqpoint{0.000000in}{0.000000in}}%
\pgfpathclose%
\pgfusepath{stroke,fill}%
\end{pgfscope}%
\begin{pgfscope}%
\pgfpathrectangle{\pgfqpoint{0.647939in}{0.492442in}}{\pgfqpoint{3.079299in}{3.079299in}}%
\pgfusepath{clip}%
\pgfsetroundcap%
\pgfsetroundjoin%
\pgfsetlinewidth{0.301125pt}%
\definecolor{currentstroke}{rgb}{0.500000,0.500000,0.500000}%
\pgfsetstrokecolor{currentstroke}%
\pgfsetstrokeopacity{0.300000}%
\pgfsetdash{}{0pt}%
\pgfpathmoveto{\pgfqpoint{1.041204in}{3.502894in}}%
\pgfusepath{stroke}%
\end{pgfscope}%
\begin{pgfscope}%
\pgfpathrectangle{\pgfqpoint{0.647939in}{0.492442in}}{\pgfqpoint{3.079299in}{3.079299in}}%
\pgfusepath{clip}%
\pgfsetroundcap%
\pgfsetroundjoin%
\definecolor{currentfill}{rgb}{0.500000,0.500000,0.500000}%
\pgfsetfillcolor{currentfill}%
\pgfsetfillopacity{0.300000}%
\pgfsetlinewidth{0.301125pt}%
\definecolor{currentstroke}{rgb}{0.500000,0.500000,0.500000}%
\pgfsetstrokecolor{currentstroke}%
\pgfsetstrokeopacity{0.300000}%
\pgfsetdash{}{0pt}%
\pgfpathmoveto{\pgfqpoint{0.000000in}{0.000000in}}%
\pgfpathlineto{\pgfqpoint{0.000000in}{0.000000in}}%
\pgfpathclose%
\pgfusepath{stroke,fill}%
\end{pgfscope}%
\begin{pgfscope}%
\pgfpathrectangle{\pgfqpoint{0.647939in}{0.492442in}}{\pgfqpoint{3.079299in}{3.079299in}}%
\pgfusepath{clip}%
\pgfsetroundcap%
\pgfsetroundjoin%
\pgfsetlinewidth{0.301125pt}%
\definecolor{currentstroke}{rgb}{0.500000,0.500000,0.500000}%
\pgfsetstrokecolor{currentstroke}%
\pgfsetstrokeopacity{0.300000}%
\pgfsetdash{}{0pt}%
\pgfpathmoveto{\pgfqpoint{0.822242in}{3.545903in}}%
\pgfusepath{stroke}%
\end{pgfscope}%
\begin{pgfscope}%
\pgfpathrectangle{\pgfqpoint{0.647939in}{0.492442in}}{\pgfqpoint{3.079299in}{3.079299in}}%
\pgfusepath{clip}%
\pgfsetroundcap%
\pgfsetroundjoin%
\definecolor{currentfill}{rgb}{0.500000,0.500000,0.500000}%
\pgfsetfillcolor{currentfill}%
\pgfsetfillopacity{0.300000}%
\pgfsetlinewidth{0.301125pt}%
\definecolor{currentstroke}{rgb}{0.500000,0.500000,0.500000}%
\pgfsetstrokecolor{currentstroke}%
\pgfsetstrokeopacity{0.300000}%
\pgfsetdash{}{0pt}%
\pgfpathmoveto{\pgfqpoint{0.000000in}{0.000000in}}%
\pgfpathlineto{\pgfqpoint{0.000000in}{0.000000in}}%
\pgfpathclose%
\pgfusepath{stroke,fill}%
\end{pgfscope}%
\begin{pgfscope}%
\pgfpathrectangle{\pgfqpoint{0.647939in}{0.492442in}}{\pgfqpoint{3.079299in}{3.079299in}}%
\pgfusepath{clip}%
\pgfsetroundcap%
\pgfsetroundjoin%
\pgfsetlinewidth{0.301125pt}%
\definecolor{currentstroke}{rgb}{0.500000,0.500000,0.500000}%
\pgfsetstrokecolor{currentstroke}%
\pgfsetstrokeopacity{0.300000}%
\pgfsetdash{}{0pt}%
\pgfpathmoveto{\pgfqpoint{1.617794in}{3.185194in}}%
\pgfusepath{stroke}%
\end{pgfscope}%
\begin{pgfscope}%
\pgfpathrectangle{\pgfqpoint{0.647939in}{0.492442in}}{\pgfqpoint{3.079299in}{3.079299in}}%
\pgfusepath{clip}%
\pgfsetroundcap%
\pgfsetroundjoin%
\definecolor{currentfill}{rgb}{0.500000,0.500000,0.500000}%
\pgfsetfillcolor{currentfill}%
\pgfsetfillopacity{0.300000}%
\pgfsetlinewidth{0.301125pt}%
\definecolor{currentstroke}{rgb}{0.500000,0.500000,0.500000}%
\pgfsetstrokecolor{currentstroke}%
\pgfsetstrokeopacity{0.300000}%
\pgfsetdash{}{0pt}%
\pgfpathmoveto{\pgfqpoint{0.000000in}{0.000000in}}%
\pgfpathlineto{\pgfqpoint{0.000000in}{0.000000in}}%
\pgfpathclose%
\pgfusepath{stroke,fill}%
\end{pgfscope}%
\begin{pgfscope}%
\pgfpathrectangle{\pgfqpoint{0.647939in}{0.492442in}}{\pgfqpoint{3.079299in}{3.079299in}}%
\pgfusepath{clip}%
\pgfsetroundcap%
\pgfsetroundjoin%
\pgfsetlinewidth{0.301125pt}%
\definecolor{currentstroke}{rgb}{0.500000,0.500000,0.500000}%
\pgfsetstrokecolor{currentstroke}%
\pgfsetstrokeopacity{0.300000}%
\pgfsetdash{}{0pt}%
\pgfpathmoveto{\pgfqpoint{0.946338in}{2.914243in}}%
\pgfusepath{stroke}%
\end{pgfscope}%
\begin{pgfscope}%
\pgfpathrectangle{\pgfqpoint{0.647939in}{0.492442in}}{\pgfqpoint{3.079299in}{3.079299in}}%
\pgfusepath{clip}%
\pgfsetroundcap%
\pgfsetroundjoin%
\definecolor{currentfill}{rgb}{0.500000,0.500000,0.500000}%
\pgfsetfillcolor{currentfill}%
\pgfsetfillopacity{0.300000}%
\pgfsetlinewidth{0.301125pt}%
\definecolor{currentstroke}{rgb}{0.500000,0.500000,0.500000}%
\pgfsetstrokecolor{currentstroke}%
\pgfsetstrokeopacity{0.300000}%
\pgfsetdash{}{0pt}%
\pgfpathmoveto{\pgfqpoint{0.000000in}{0.000000in}}%
\pgfpathlineto{\pgfqpoint{0.000000in}{0.000000in}}%
\pgfpathclose%
\pgfusepath{stroke,fill}%
\end{pgfscope}%
\begin{pgfscope}%
\pgfpathrectangle{\pgfqpoint{0.647939in}{0.492442in}}{\pgfqpoint{3.079299in}{3.079299in}}%
\pgfusepath{clip}%
\pgfsetroundcap%
\pgfsetroundjoin%
\pgfsetlinewidth{0.301125pt}%
\definecolor{currentstroke}{rgb}{0.500000,0.500000,0.500000}%
\pgfsetstrokecolor{currentstroke}%
\pgfsetstrokeopacity{0.300000}%
\pgfsetdash{}{0pt}%
\pgfpathmoveto{\pgfqpoint{1.887351in}{3.016237in}}%
\pgfusepath{stroke}%
\end{pgfscope}%
\begin{pgfscope}%
\pgfpathrectangle{\pgfqpoint{0.647939in}{0.492442in}}{\pgfqpoint{3.079299in}{3.079299in}}%
\pgfusepath{clip}%
\pgfsetroundcap%
\pgfsetroundjoin%
\definecolor{currentfill}{rgb}{0.500000,0.500000,0.500000}%
\pgfsetfillcolor{currentfill}%
\pgfsetfillopacity{0.300000}%
\pgfsetlinewidth{0.301125pt}%
\definecolor{currentstroke}{rgb}{0.500000,0.500000,0.500000}%
\pgfsetstrokecolor{currentstroke}%
\pgfsetstrokeopacity{0.300000}%
\pgfsetdash{}{0pt}%
\pgfpathmoveto{\pgfqpoint{0.000000in}{0.000000in}}%
\pgfpathlineto{\pgfqpoint{0.000000in}{0.000000in}}%
\pgfpathclose%
\pgfusepath{stroke,fill}%
\end{pgfscope}%
\begin{pgfscope}%
\pgfpathrectangle{\pgfqpoint{0.647939in}{0.492442in}}{\pgfqpoint{3.079299in}{3.079299in}}%
\pgfusepath{clip}%
\pgfsetroundcap%
\pgfsetroundjoin%
\pgfsetlinewidth{0.301125pt}%
\definecolor{currentstroke}{rgb}{0.500000,0.500000,0.500000}%
\pgfsetstrokecolor{currentstroke}%
\pgfsetstrokeopacity{0.300000}%
\pgfsetdash{}{0pt}%
\pgfpathmoveto{\pgfqpoint{1.013292in}{2.788712in}}%
\pgfusepath{stroke}%
\end{pgfscope}%
\begin{pgfscope}%
\pgfpathrectangle{\pgfqpoint{0.647939in}{0.492442in}}{\pgfqpoint{3.079299in}{3.079299in}}%
\pgfusepath{clip}%
\pgfsetroundcap%
\pgfsetroundjoin%
\definecolor{currentfill}{rgb}{0.500000,0.500000,0.500000}%
\pgfsetfillcolor{currentfill}%
\pgfsetfillopacity{0.300000}%
\pgfsetlinewidth{0.301125pt}%
\definecolor{currentstroke}{rgb}{0.500000,0.500000,0.500000}%
\pgfsetstrokecolor{currentstroke}%
\pgfsetstrokeopacity{0.300000}%
\pgfsetdash{}{0pt}%
\pgfpathmoveto{\pgfqpoint{0.000000in}{0.000000in}}%
\pgfpathlineto{\pgfqpoint{0.000000in}{0.000000in}}%
\pgfpathclose%
\pgfusepath{stroke,fill}%
\end{pgfscope}%
\begin{pgfscope}%
\pgfpathrectangle{\pgfqpoint{0.647939in}{0.492442in}}{\pgfqpoint{3.079299in}{3.079299in}}%
\pgfusepath{clip}%
\pgfsetroundcap%
\pgfsetroundjoin%
\pgfsetlinewidth{0.301125pt}%
\definecolor{currentstroke}{rgb}{0.500000,0.500000,0.500000}%
\pgfsetstrokecolor{currentstroke}%
\pgfsetstrokeopacity{0.300000}%
\pgfsetdash{}{0pt}%
\pgfpathmoveto{\pgfqpoint{1.747535in}{2.812857in}}%
\pgfusepath{stroke}%
\end{pgfscope}%
\begin{pgfscope}%
\pgfpathrectangle{\pgfqpoint{0.647939in}{0.492442in}}{\pgfqpoint{3.079299in}{3.079299in}}%
\pgfusepath{clip}%
\pgfsetroundcap%
\pgfsetroundjoin%
\definecolor{currentfill}{rgb}{0.500000,0.500000,0.500000}%
\pgfsetfillcolor{currentfill}%
\pgfsetfillopacity{0.300000}%
\pgfsetlinewidth{0.301125pt}%
\definecolor{currentstroke}{rgb}{0.500000,0.500000,0.500000}%
\pgfsetstrokecolor{currentstroke}%
\pgfsetstrokeopacity{0.300000}%
\pgfsetdash{}{0pt}%
\pgfpathmoveto{\pgfqpoint{0.000000in}{0.000000in}}%
\pgfpathlineto{\pgfqpoint{0.000000in}{0.000000in}}%
\pgfpathclose%
\pgfusepath{stroke,fill}%
\end{pgfscope}%
\begin{pgfscope}%
\pgfpathrectangle{\pgfqpoint{0.647939in}{0.492442in}}{\pgfqpoint{3.079299in}{3.079299in}}%
\pgfusepath{clip}%
\pgfsetroundcap%
\pgfsetroundjoin%
\pgfsetlinewidth{0.301125pt}%
\definecolor{currentstroke}{rgb}{0.500000,0.500000,0.500000}%
\pgfsetstrokecolor{currentstroke}%
\pgfsetstrokeopacity{0.300000}%
\pgfsetdash{}{0pt}%
\pgfpathmoveto{\pgfqpoint{1.544299in}{2.712524in}}%
\pgfusepath{stroke}%
\end{pgfscope}%
\begin{pgfscope}%
\pgfpathrectangle{\pgfqpoint{0.647939in}{0.492442in}}{\pgfqpoint{3.079299in}{3.079299in}}%
\pgfusepath{clip}%
\pgfsetroundcap%
\pgfsetroundjoin%
\definecolor{currentfill}{rgb}{0.500000,0.500000,0.500000}%
\pgfsetfillcolor{currentfill}%
\pgfsetfillopacity{0.300000}%
\pgfsetlinewidth{0.301125pt}%
\definecolor{currentstroke}{rgb}{0.500000,0.500000,0.500000}%
\pgfsetstrokecolor{currentstroke}%
\pgfsetstrokeopacity{0.300000}%
\pgfsetdash{}{0pt}%
\pgfpathmoveto{\pgfqpoint{0.000000in}{0.000000in}}%
\pgfpathlineto{\pgfqpoint{0.000000in}{0.000000in}}%
\pgfpathclose%
\pgfusepath{stroke,fill}%
\end{pgfscope}%
\begin{pgfscope}%
\pgfpathrectangle{\pgfqpoint{0.647939in}{0.492442in}}{\pgfqpoint{3.079299in}{3.079299in}}%
\pgfusepath{clip}%
\pgfsetroundcap%
\pgfsetroundjoin%
\pgfsetlinewidth{0.301125pt}%
\definecolor{currentstroke}{rgb}{0.500000,0.500000,0.500000}%
\pgfsetstrokecolor{currentstroke}%
\pgfsetstrokeopacity{0.300000}%
\pgfsetdash{}{0pt}%
\pgfpathmoveto{\pgfqpoint{1.212019in}{2.561398in}}%
\pgfusepath{stroke}%
\end{pgfscope}%
\begin{pgfscope}%
\pgfpathrectangle{\pgfqpoint{0.647939in}{0.492442in}}{\pgfqpoint{3.079299in}{3.079299in}}%
\pgfusepath{clip}%
\pgfsetroundcap%
\pgfsetroundjoin%
\definecolor{currentfill}{rgb}{0.500000,0.500000,0.500000}%
\pgfsetfillcolor{currentfill}%
\pgfsetfillopacity{0.300000}%
\pgfsetlinewidth{0.301125pt}%
\definecolor{currentstroke}{rgb}{0.500000,0.500000,0.500000}%
\pgfsetstrokecolor{currentstroke}%
\pgfsetstrokeopacity{0.300000}%
\pgfsetdash{}{0pt}%
\pgfpathmoveto{\pgfqpoint{0.000000in}{0.000000in}}%
\pgfpathlineto{\pgfqpoint{0.000000in}{0.000000in}}%
\pgfpathclose%
\pgfusepath{stroke,fill}%
\end{pgfscope}%
\begin{pgfscope}%
\pgfpathrectangle{\pgfqpoint{0.647939in}{0.492442in}}{\pgfqpoint{3.079299in}{3.079299in}}%
\pgfusepath{clip}%
\pgfsetroundcap%
\pgfsetroundjoin%
\pgfsetlinewidth{0.301125pt}%
\definecolor{currentstroke}{rgb}{0.500000,0.500000,0.500000}%
\pgfsetstrokecolor{currentstroke}%
\pgfsetstrokeopacity{0.300000}%
\pgfsetdash{}{0pt}%
\pgfpathmoveto{\pgfqpoint{1.012438in}{2.443712in}}%
\pgfusepath{stroke}%
\end{pgfscope}%
\begin{pgfscope}%
\pgfpathrectangle{\pgfqpoint{0.647939in}{0.492442in}}{\pgfqpoint{3.079299in}{3.079299in}}%
\pgfusepath{clip}%
\pgfsetroundcap%
\pgfsetroundjoin%
\definecolor{currentfill}{rgb}{0.500000,0.500000,0.500000}%
\pgfsetfillcolor{currentfill}%
\pgfsetfillopacity{0.300000}%
\pgfsetlinewidth{0.301125pt}%
\definecolor{currentstroke}{rgb}{0.500000,0.500000,0.500000}%
\pgfsetstrokecolor{currentstroke}%
\pgfsetstrokeopacity{0.300000}%
\pgfsetdash{}{0pt}%
\pgfpathmoveto{\pgfqpoint{0.000000in}{0.000000in}}%
\pgfpathlineto{\pgfqpoint{0.000000in}{0.000000in}}%
\pgfpathclose%
\pgfusepath{stroke,fill}%
\end{pgfscope}%
\begin{pgfscope}%
\pgfpathrectangle{\pgfqpoint{0.647939in}{0.492442in}}{\pgfqpoint{3.079299in}{3.079299in}}%
\pgfusepath{clip}%
\pgfsetroundcap%
\pgfsetroundjoin%
\pgfsetlinewidth{0.301125pt}%
\definecolor{currentstroke}{rgb}{0.500000,0.500000,0.500000}%
\pgfsetstrokecolor{currentstroke}%
\pgfsetstrokeopacity{0.300000}%
\pgfsetdash{}{0pt}%
\pgfpathmoveto{\pgfqpoint{0.878328in}{2.346680in}}%
\pgfusepath{stroke}%
\end{pgfscope}%
\begin{pgfscope}%
\pgfpathrectangle{\pgfqpoint{0.647939in}{0.492442in}}{\pgfqpoint{3.079299in}{3.079299in}}%
\pgfusepath{clip}%
\pgfsetroundcap%
\pgfsetroundjoin%
\definecolor{currentfill}{rgb}{0.500000,0.500000,0.500000}%
\pgfsetfillcolor{currentfill}%
\pgfsetfillopacity{0.300000}%
\pgfsetlinewidth{0.301125pt}%
\definecolor{currentstroke}{rgb}{0.500000,0.500000,0.500000}%
\pgfsetstrokecolor{currentstroke}%
\pgfsetstrokeopacity{0.300000}%
\pgfsetdash{}{0pt}%
\pgfpathmoveto{\pgfqpoint{0.000000in}{0.000000in}}%
\pgfpathlineto{\pgfqpoint{0.000000in}{0.000000in}}%
\pgfpathclose%
\pgfusepath{stroke,fill}%
\end{pgfscope}%
\begin{pgfscope}%
\pgfpathrectangle{\pgfqpoint{0.647939in}{0.492442in}}{\pgfqpoint{3.079299in}{3.079299in}}%
\pgfusepath{clip}%
\pgfsetroundcap%
\pgfsetroundjoin%
\pgfsetlinewidth{0.301125pt}%
\definecolor{currentstroke}{rgb}{0.500000,0.500000,0.500000}%
\pgfsetstrokecolor{currentstroke}%
\pgfsetstrokeopacity{0.300000}%
\pgfsetdash{}{0pt}%
\pgfpathmoveto{\pgfqpoint{1.807069in}{2.507939in}}%
\pgfusepath{stroke}%
\end{pgfscope}%
\begin{pgfscope}%
\pgfpathrectangle{\pgfqpoint{0.647939in}{0.492442in}}{\pgfqpoint{3.079299in}{3.079299in}}%
\pgfusepath{clip}%
\pgfsetroundcap%
\pgfsetroundjoin%
\definecolor{currentfill}{rgb}{0.500000,0.500000,0.500000}%
\pgfsetfillcolor{currentfill}%
\pgfsetfillopacity{0.300000}%
\pgfsetlinewidth{0.301125pt}%
\definecolor{currentstroke}{rgb}{0.500000,0.500000,0.500000}%
\pgfsetstrokecolor{currentstroke}%
\pgfsetstrokeopacity{0.300000}%
\pgfsetdash{}{0pt}%
\pgfpathmoveto{\pgfqpoint{0.000000in}{0.000000in}}%
\pgfpathlineto{\pgfqpoint{0.000000in}{0.000000in}}%
\pgfpathclose%
\pgfusepath{stroke,fill}%
\end{pgfscope}%
\begin{pgfscope}%
\pgfpathrectangle{\pgfqpoint{0.647939in}{0.492442in}}{\pgfqpoint{3.079299in}{3.079299in}}%
\pgfusepath{clip}%
\pgfsetroundcap%
\pgfsetroundjoin%
\pgfsetlinewidth{0.301125pt}%
\definecolor{currentstroke}{rgb}{0.500000,0.500000,0.500000}%
\pgfsetstrokecolor{currentstroke}%
\pgfsetstrokeopacity{0.300000}%
\pgfsetdash{}{0pt}%
\pgfpathmoveto{\pgfqpoint{1.537545in}{2.389360in}}%
\pgfusepath{stroke}%
\end{pgfscope}%
\begin{pgfscope}%
\pgfpathrectangle{\pgfqpoint{0.647939in}{0.492442in}}{\pgfqpoint{3.079299in}{3.079299in}}%
\pgfusepath{clip}%
\pgfsetroundcap%
\pgfsetroundjoin%
\definecolor{currentfill}{rgb}{0.500000,0.500000,0.500000}%
\pgfsetfillcolor{currentfill}%
\pgfsetfillopacity{0.300000}%
\pgfsetlinewidth{0.301125pt}%
\definecolor{currentstroke}{rgb}{0.500000,0.500000,0.500000}%
\pgfsetstrokecolor{currentstroke}%
\pgfsetstrokeopacity{0.300000}%
\pgfsetdash{}{0pt}%
\pgfpathmoveto{\pgfqpoint{0.000000in}{0.000000in}}%
\pgfpathlineto{\pgfqpoint{0.000000in}{0.000000in}}%
\pgfpathclose%
\pgfusepath{stroke,fill}%
\end{pgfscope}%
\begin{pgfscope}%
\pgfpathrectangle{\pgfqpoint{0.647939in}{0.492442in}}{\pgfqpoint{3.079299in}{3.079299in}}%
\pgfusepath{clip}%
\pgfsetroundcap%
\pgfsetroundjoin%
\pgfsetlinewidth{0.301125pt}%
\definecolor{currentstroke}{rgb}{0.500000,0.500000,0.500000}%
\pgfsetstrokecolor{currentstroke}%
\pgfsetstrokeopacity{0.300000}%
\pgfsetdash{}{0pt}%
\pgfpathmoveto{\pgfqpoint{1.143701in}{2.203945in}}%
\pgfusepath{stroke}%
\end{pgfscope}%
\begin{pgfscope}%
\pgfpathrectangle{\pgfqpoint{0.647939in}{0.492442in}}{\pgfqpoint{3.079299in}{3.079299in}}%
\pgfusepath{clip}%
\pgfsetroundcap%
\pgfsetroundjoin%
\definecolor{currentfill}{rgb}{0.500000,0.500000,0.500000}%
\pgfsetfillcolor{currentfill}%
\pgfsetfillopacity{0.300000}%
\pgfsetlinewidth{0.301125pt}%
\definecolor{currentstroke}{rgb}{0.500000,0.500000,0.500000}%
\pgfsetstrokecolor{currentstroke}%
\pgfsetstrokeopacity{0.300000}%
\pgfsetdash{}{0pt}%
\pgfpathmoveto{\pgfqpoint{0.000000in}{0.000000in}}%
\pgfpathlineto{\pgfqpoint{0.000000in}{0.000000in}}%
\pgfpathclose%
\pgfusepath{stroke,fill}%
\end{pgfscope}%
\begin{pgfscope}%
\pgfpathrectangle{\pgfqpoint{0.647939in}{0.492442in}}{\pgfqpoint{3.079299in}{3.079299in}}%
\pgfusepath{clip}%
\pgfsetroundcap%
\pgfsetroundjoin%
\pgfsetlinewidth{0.301125pt}%
\definecolor{currentstroke}{rgb}{0.500000,0.500000,0.500000}%
\pgfsetstrokecolor{currentstroke}%
\pgfsetstrokeopacity{0.300000}%
\pgfsetdash{}{0pt}%
\pgfpathmoveto{\pgfqpoint{1.011322in}{2.099625in}}%
\pgfusepath{stroke}%
\end{pgfscope}%
\begin{pgfscope}%
\pgfpathrectangle{\pgfqpoint{0.647939in}{0.492442in}}{\pgfqpoint{3.079299in}{3.079299in}}%
\pgfusepath{clip}%
\pgfsetroundcap%
\pgfsetroundjoin%
\definecolor{currentfill}{rgb}{0.500000,0.500000,0.500000}%
\pgfsetfillcolor{currentfill}%
\pgfsetfillopacity{0.300000}%
\pgfsetlinewidth{0.301125pt}%
\definecolor{currentstroke}{rgb}{0.500000,0.500000,0.500000}%
\pgfsetstrokecolor{currentstroke}%
\pgfsetstrokeopacity{0.300000}%
\pgfsetdash{}{0pt}%
\pgfpathmoveto{\pgfqpoint{0.000000in}{0.000000in}}%
\pgfpathlineto{\pgfqpoint{0.000000in}{0.000000in}}%
\pgfpathclose%
\pgfusepath{stroke,fill}%
\end{pgfscope}%
\begin{pgfscope}%
\pgfpathrectangle{\pgfqpoint{0.647939in}{0.492442in}}{\pgfqpoint{3.079299in}{3.079299in}}%
\pgfusepath{clip}%
\pgfsetroundcap%
\pgfsetroundjoin%
\pgfsetlinewidth{0.301125pt}%
\definecolor{currentstroke}{rgb}{0.500000,0.500000,0.500000}%
\pgfsetstrokecolor{currentstroke}%
\pgfsetstrokeopacity{0.300000}%
\pgfsetdash{}{0pt}%
\pgfpathmoveto{\pgfqpoint{1.466909in}{2.177512in}}%
\pgfusepath{stroke}%
\end{pgfscope}%
\begin{pgfscope}%
\pgfpathrectangle{\pgfqpoint{0.647939in}{0.492442in}}{\pgfqpoint{3.079299in}{3.079299in}}%
\pgfusepath{clip}%
\pgfsetroundcap%
\pgfsetroundjoin%
\definecolor{currentfill}{rgb}{0.500000,0.500000,0.500000}%
\pgfsetfillcolor{currentfill}%
\pgfsetfillopacity{0.300000}%
\pgfsetlinewidth{0.301125pt}%
\definecolor{currentstroke}{rgb}{0.500000,0.500000,0.500000}%
\pgfsetstrokecolor{currentstroke}%
\pgfsetstrokeopacity{0.300000}%
\pgfsetdash{}{0pt}%
\pgfpathmoveto{\pgfqpoint{0.000000in}{0.000000in}}%
\pgfpathlineto{\pgfqpoint{0.000000in}{0.000000in}}%
\pgfpathclose%
\pgfusepath{stroke,fill}%
\end{pgfscope}%
\begin{pgfscope}%
\pgfpathrectangle{\pgfqpoint{0.647939in}{0.492442in}}{\pgfqpoint{3.079299in}{3.079299in}}%
\pgfusepath{clip}%
\pgfsetroundcap%
\pgfsetroundjoin%
\pgfsetlinewidth{0.301125pt}%
\definecolor{currentstroke}{rgb}{0.500000,0.500000,0.500000}%
\pgfsetstrokecolor{currentstroke}%
\pgfsetstrokeopacity{0.300000}%
\pgfsetdash{}{0pt}%
\pgfpathmoveto{\pgfqpoint{1.207158in}{2.021923in}}%
\pgfusepath{stroke}%
\end{pgfscope}%
\begin{pgfscope}%
\pgfpathrectangle{\pgfqpoint{0.647939in}{0.492442in}}{\pgfqpoint{3.079299in}{3.079299in}}%
\pgfusepath{clip}%
\pgfsetroundcap%
\pgfsetroundjoin%
\definecolor{currentfill}{rgb}{0.500000,0.500000,0.500000}%
\pgfsetfillcolor{currentfill}%
\pgfsetfillopacity{0.300000}%
\pgfsetlinewidth{0.301125pt}%
\definecolor{currentstroke}{rgb}{0.500000,0.500000,0.500000}%
\pgfsetstrokecolor{currentstroke}%
\pgfsetstrokeopacity{0.300000}%
\pgfsetdash{}{0pt}%
\pgfpathmoveto{\pgfqpoint{0.000000in}{0.000000in}}%
\pgfpathlineto{\pgfqpoint{0.000000in}{0.000000in}}%
\pgfpathclose%
\pgfusepath{stroke,fill}%
\end{pgfscope}%
\begin{pgfscope}%
\pgfpathrectangle{\pgfqpoint{0.647939in}{0.492442in}}{\pgfqpoint{3.079299in}{3.079299in}}%
\pgfusepath{clip}%
\pgfsetroundcap%
\pgfsetroundjoin%
\pgfsetlinewidth{0.301125pt}%
\definecolor{currentstroke}{rgb}{0.500000,0.500000,0.500000}%
\pgfsetstrokecolor{currentstroke}%
\pgfsetstrokeopacity{0.300000}%
\pgfsetdash{}{0pt}%
\pgfpathmoveto{\pgfqpoint{1.206337in}{1.955011in}}%
\pgfusepath{stroke}%
\end{pgfscope}%
\begin{pgfscope}%
\pgfpathrectangle{\pgfqpoint{0.647939in}{0.492442in}}{\pgfqpoint{3.079299in}{3.079299in}}%
\pgfusepath{clip}%
\pgfsetroundcap%
\pgfsetroundjoin%
\definecolor{currentfill}{rgb}{0.500000,0.500000,0.500000}%
\pgfsetfillcolor{currentfill}%
\pgfsetfillopacity{0.300000}%
\pgfsetlinewidth{0.301125pt}%
\definecolor{currentstroke}{rgb}{0.500000,0.500000,0.500000}%
\pgfsetstrokecolor{currentstroke}%
\pgfsetstrokeopacity{0.300000}%
\pgfsetdash{}{0pt}%
\pgfpathmoveto{\pgfqpoint{0.000000in}{0.000000in}}%
\pgfpathlineto{\pgfqpoint{0.000000in}{0.000000in}}%
\pgfpathclose%
\pgfusepath{stroke,fill}%
\end{pgfscope}%
\begin{pgfscope}%
\pgfpathrectangle{\pgfqpoint{0.647939in}{0.492442in}}{\pgfqpoint{3.079299in}{3.079299in}}%
\pgfusepath{clip}%
\pgfsetroundcap%
\pgfsetroundjoin%
\pgfsetlinewidth{0.301125pt}%
\definecolor{currentstroke}{rgb}{0.500000,0.500000,0.500000}%
\pgfsetstrokecolor{currentstroke}%
\pgfsetstrokeopacity{0.300000}%
\pgfsetdash{}{0pt}%
\pgfpathmoveto{\pgfqpoint{1.075766in}{1.844603in}}%
\pgfusepath{stroke}%
\end{pgfscope}%
\begin{pgfscope}%
\pgfpathrectangle{\pgfqpoint{0.647939in}{0.492442in}}{\pgfqpoint{3.079299in}{3.079299in}}%
\pgfusepath{clip}%
\pgfsetroundcap%
\pgfsetroundjoin%
\definecolor{currentfill}{rgb}{0.500000,0.500000,0.500000}%
\pgfsetfillcolor{currentfill}%
\pgfsetfillopacity{0.300000}%
\pgfsetlinewidth{0.301125pt}%
\definecolor{currentstroke}{rgb}{0.500000,0.500000,0.500000}%
\pgfsetstrokecolor{currentstroke}%
\pgfsetstrokeopacity{0.300000}%
\pgfsetdash{}{0pt}%
\pgfpathmoveto{\pgfqpoint{0.000000in}{0.000000in}}%
\pgfpathlineto{\pgfqpoint{0.000000in}{0.000000in}}%
\pgfpathclose%
\pgfusepath{stroke,fill}%
\end{pgfscope}%
\begin{pgfscope}%
\pgfpathrectangle{\pgfqpoint{0.647939in}{0.492442in}}{\pgfqpoint{3.079299in}{3.079299in}}%
\pgfusepath{clip}%
\pgfsetroundcap%
\pgfsetroundjoin%
\pgfsetlinewidth{0.301125pt}%
\definecolor{currentstroke}{rgb}{0.500000,0.500000,0.500000}%
\pgfsetstrokecolor{currentstroke}%
\pgfsetstrokeopacity{0.300000}%
\pgfsetdash{}{0pt}%
\pgfpathmoveto{\pgfqpoint{0.810391in}{1.708512in}}%
\pgfusepath{stroke}%
\end{pgfscope}%
\begin{pgfscope}%
\pgfpathrectangle{\pgfqpoint{0.647939in}{0.492442in}}{\pgfqpoint{3.079299in}{3.079299in}}%
\pgfusepath{clip}%
\pgfsetroundcap%
\pgfsetroundjoin%
\definecolor{currentfill}{rgb}{0.500000,0.500000,0.500000}%
\pgfsetfillcolor{currentfill}%
\pgfsetfillopacity{0.300000}%
\pgfsetlinewidth{0.301125pt}%
\definecolor{currentstroke}{rgb}{0.500000,0.500000,0.500000}%
\pgfsetstrokecolor{currentstroke}%
\pgfsetstrokeopacity{0.300000}%
\pgfsetdash{}{0pt}%
\pgfpathmoveto{\pgfqpoint{0.000000in}{0.000000in}}%
\pgfpathlineto{\pgfqpoint{0.000000in}{0.000000in}}%
\pgfpathclose%
\pgfusepath{stroke,fill}%
\end{pgfscope}%
\begin{pgfscope}%
\pgfpathrectangle{\pgfqpoint{0.647939in}{0.492442in}}{\pgfqpoint{3.079299in}{3.079299in}}%
\pgfusepath{clip}%
\pgfsetroundcap%
\pgfsetroundjoin%
\pgfsetlinewidth{0.301125pt}%
\definecolor{currentstroke}{rgb}{0.500000,0.500000,0.500000}%
\pgfsetstrokecolor{currentstroke}%
\pgfsetstrokeopacity{0.300000}%
\pgfsetdash{}{0pt}%
\pgfpathmoveto{\pgfqpoint{1.392606in}{1.834871in}}%
\pgfusepath{stroke}%
\end{pgfscope}%
\begin{pgfscope}%
\pgfpathrectangle{\pgfqpoint{0.647939in}{0.492442in}}{\pgfqpoint{3.079299in}{3.079299in}}%
\pgfusepath{clip}%
\pgfsetroundcap%
\pgfsetroundjoin%
\definecolor{currentfill}{rgb}{0.500000,0.500000,0.500000}%
\pgfsetfillcolor{currentfill}%
\pgfsetfillopacity{0.300000}%
\pgfsetlinewidth{0.301125pt}%
\definecolor{currentstroke}{rgb}{0.500000,0.500000,0.500000}%
\pgfsetstrokecolor{currentstroke}%
\pgfsetstrokeopacity{0.300000}%
\pgfsetdash{}{0pt}%
\pgfpathmoveto{\pgfqpoint{0.000000in}{0.000000in}}%
\pgfpathlineto{\pgfqpoint{0.000000in}{0.000000in}}%
\pgfpathclose%
\pgfusepath{stroke,fill}%
\end{pgfscope}%
\begin{pgfscope}%
\pgfpathrectangle{\pgfqpoint{0.647939in}{0.492442in}}{\pgfqpoint{3.079299in}{3.079299in}}%
\pgfusepath{clip}%
\pgfsetroundcap%
\pgfsetroundjoin%
\pgfsetlinewidth{0.301125pt}%
\definecolor{currentstroke}{rgb}{0.500000,0.500000,0.500000}%
\pgfsetstrokecolor{currentstroke}%
\pgfsetstrokeopacity{0.300000}%
\pgfsetdash{}{0pt}%
\pgfpathmoveto{\pgfqpoint{1.325855in}{1.679028in}}%
\pgfusepath{stroke}%
\end{pgfscope}%
\begin{pgfscope}%
\pgfpathrectangle{\pgfqpoint{0.647939in}{0.492442in}}{\pgfqpoint{3.079299in}{3.079299in}}%
\pgfusepath{clip}%
\pgfsetroundcap%
\pgfsetroundjoin%
\definecolor{currentfill}{rgb}{0.500000,0.500000,0.500000}%
\pgfsetfillcolor{currentfill}%
\pgfsetfillopacity{0.300000}%
\pgfsetlinewidth{0.301125pt}%
\definecolor{currentstroke}{rgb}{0.500000,0.500000,0.500000}%
\pgfsetstrokecolor{currentstroke}%
\pgfsetstrokeopacity{0.300000}%
\pgfsetdash{}{0pt}%
\pgfpathmoveto{\pgfqpoint{0.000000in}{0.000000in}}%
\pgfpathlineto{\pgfqpoint{0.000000in}{0.000000in}}%
\pgfpathclose%
\pgfusepath{stroke,fill}%
\end{pgfscope}%
\begin{pgfscope}%
\pgfpathrectangle{\pgfqpoint{0.647939in}{0.492442in}}{\pgfqpoint{3.079299in}{3.079299in}}%
\pgfusepath{clip}%
\pgfsetroundcap%
\pgfsetroundjoin%
\pgfsetlinewidth{0.301125pt}%
\definecolor{currentstroke}{rgb}{0.500000,0.500000,0.500000}%
\pgfsetstrokecolor{currentstroke}%
\pgfsetstrokeopacity{0.300000}%
\pgfsetdash{}{0pt}%
\pgfpathmoveto{\pgfqpoint{1.136789in}{1.530238in}}%
\pgfusepath{stroke}%
\end{pgfscope}%
\begin{pgfscope}%
\pgfpathrectangle{\pgfqpoint{0.647939in}{0.492442in}}{\pgfqpoint{3.079299in}{3.079299in}}%
\pgfusepath{clip}%
\pgfsetroundcap%
\pgfsetroundjoin%
\definecolor{currentfill}{rgb}{0.500000,0.500000,0.500000}%
\pgfsetfillcolor{currentfill}%
\pgfsetfillopacity{0.300000}%
\pgfsetlinewidth{0.301125pt}%
\definecolor{currentstroke}{rgb}{0.500000,0.500000,0.500000}%
\pgfsetstrokecolor{currentstroke}%
\pgfsetstrokeopacity{0.300000}%
\pgfsetdash{}{0pt}%
\pgfpathmoveto{\pgfqpoint{0.000000in}{0.000000in}}%
\pgfpathlineto{\pgfqpoint{0.000000in}{0.000000in}}%
\pgfpathclose%
\pgfusepath{stroke,fill}%
\end{pgfscope}%
\begin{pgfscope}%
\pgfpathrectangle{\pgfqpoint{0.647939in}{0.492442in}}{\pgfqpoint{3.079299in}{3.079299in}}%
\pgfusepath{clip}%
\pgfsetroundcap%
\pgfsetroundjoin%
\pgfsetlinewidth{0.301125pt}%
\definecolor{currentstroke}{rgb}{0.500000,0.500000,0.500000}%
\pgfsetstrokecolor{currentstroke}%
\pgfsetstrokeopacity{0.300000}%
\pgfsetdash{}{0pt}%
\pgfpathmoveto{\pgfqpoint{0.876526in}{1.376801in}}%
\pgfusepath{stroke}%
\end{pgfscope}%
\begin{pgfscope}%
\pgfpathrectangle{\pgfqpoint{0.647939in}{0.492442in}}{\pgfqpoint{3.079299in}{3.079299in}}%
\pgfusepath{clip}%
\pgfsetroundcap%
\pgfsetroundjoin%
\definecolor{currentfill}{rgb}{0.500000,0.500000,0.500000}%
\pgfsetfillcolor{currentfill}%
\pgfsetfillopacity{0.300000}%
\pgfsetlinewidth{0.301125pt}%
\definecolor{currentstroke}{rgb}{0.500000,0.500000,0.500000}%
\pgfsetstrokecolor{currentstroke}%
\pgfsetstrokeopacity{0.300000}%
\pgfsetdash{}{0pt}%
\pgfpathmoveto{\pgfqpoint{0.000000in}{0.000000in}}%
\pgfpathlineto{\pgfqpoint{0.000000in}{0.000000in}}%
\pgfpathclose%
\pgfusepath{stroke,fill}%
\end{pgfscope}%
\begin{pgfscope}%
\pgfpathrectangle{\pgfqpoint{0.647939in}{0.492442in}}{\pgfqpoint{3.079299in}{3.079299in}}%
\pgfusepath{clip}%
\pgfsetroundcap%
\pgfsetroundjoin%
\pgfsetlinewidth{0.301125pt}%
\definecolor{currentstroke}{rgb}{0.500000,0.500000,0.500000}%
\pgfsetstrokecolor{currentstroke}%
\pgfsetstrokeopacity{0.300000}%
\pgfsetdash{}{0pt}%
\pgfpathmoveto{\pgfqpoint{0.876333in}{1.307739in}}%
\pgfusepath{stroke}%
\end{pgfscope}%
\begin{pgfscope}%
\pgfpathrectangle{\pgfqpoint{0.647939in}{0.492442in}}{\pgfqpoint{3.079299in}{3.079299in}}%
\pgfusepath{clip}%
\pgfsetroundcap%
\pgfsetroundjoin%
\definecolor{currentfill}{rgb}{0.500000,0.500000,0.500000}%
\pgfsetfillcolor{currentfill}%
\pgfsetfillopacity{0.300000}%
\pgfsetlinewidth{0.301125pt}%
\definecolor{currentstroke}{rgb}{0.500000,0.500000,0.500000}%
\pgfsetstrokecolor{currentstroke}%
\pgfsetstrokeopacity{0.300000}%
\pgfsetdash{}{0pt}%
\pgfpathmoveto{\pgfqpoint{0.000000in}{0.000000in}}%
\pgfpathlineto{\pgfqpoint{0.000000in}{0.000000in}}%
\pgfpathclose%
\pgfusepath{stroke,fill}%
\end{pgfscope}%
\begin{pgfscope}%
\pgfpathrectangle{\pgfqpoint{0.647939in}{0.492442in}}{\pgfqpoint{3.079299in}{3.079299in}}%
\pgfusepath{clip}%
\pgfsetroundcap%
\pgfsetroundjoin%
\pgfsetlinewidth{0.301125pt}%
\definecolor{currentstroke}{rgb}{0.500000,0.500000,0.500000}%
\pgfsetstrokecolor{currentstroke}%
\pgfsetstrokeopacity{0.300000}%
\pgfsetdash{}{0pt}%
\pgfpathmoveto{\pgfqpoint{1.255669in}{1.392143in}}%
\pgfusepath{stroke}%
\end{pgfscope}%
\begin{pgfscope}%
\pgfpathrectangle{\pgfqpoint{0.647939in}{0.492442in}}{\pgfqpoint{3.079299in}{3.079299in}}%
\pgfusepath{clip}%
\pgfsetroundcap%
\pgfsetroundjoin%
\definecolor{currentfill}{rgb}{0.500000,0.500000,0.500000}%
\pgfsetfillcolor{currentfill}%
\pgfsetfillopacity{0.300000}%
\pgfsetlinewidth{0.301125pt}%
\definecolor{currentstroke}{rgb}{0.500000,0.500000,0.500000}%
\pgfsetstrokecolor{currentstroke}%
\pgfsetstrokeopacity{0.300000}%
\pgfsetdash{}{0pt}%
\pgfpathmoveto{\pgfqpoint{0.000000in}{0.000000in}}%
\pgfpathlineto{\pgfqpoint{0.000000in}{0.000000in}}%
\pgfpathclose%
\pgfusepath{stroke,fill}%
\end{pgfscope}%
\begin{pgfscope}%
\pgfpathrectangle{\pgfqpoint{0.647939in}{0.492442in}}{\pgfqpoint{3.079299in}{3.079299in}}%
\pgfusepath{clip}%
\pgfsetroundcap%
\pgfsetroundjoin%
\pgfsetlinewidth{0.301125pt}%
\definecolor{currentstroke}{rgb}{0.500000,0.500000,0.500000}%
\pgfsetstrokecolor{currentstroke}%
\pgfsetstrokeopacity{0.300000}%
\pgfsetdash{}{0pt}%
\pgfpathmoveto{\pgfqpoint{1.069734in}{1.236636in}}%
\pgfusepath{stroke}%
\end{pgfscope}%
\begin{pgfscope}%
\pgfpathrectangle{\pgfqpoint{0.647939in}{0.492442in}}{\pgfqpoint{3.079299in}{3.079299in}}%
\pgfusepath{clip}%
\pgfsetroundcap%
\pgfsetroundjoin%
\definecolor{currentfill}{rgb}{0.500000,0.500000,0.500000}%
\pgfsetfillcolor{currentfill}%
\pgfsetfillopacity{0.300000}%
\pgfsetlinewidth{0.301125pt}%
\definecolor{currentstroke}{rgb}{0.500000,0.500000,0.500000}%
\pgfsetstrokecolor{currentstroke}%
\pgfsetstrokeopacity{0.300000}%
\pgfsetdash{}{0pt}%
\pgfpathmoveto{\pgfqpoint{0.000000in}{0.000000in}}%
\pgfpathlineto{\pgfqpoint{0.000000in}{0.000000in}}%
\pgfpathclose%
\pgfusepath{stroke,fill}%
\end{pgfscope}%
\begin{pgfscope}%
\pgfpathrectangle{\pgfqpoint{0.647939in}{0.492442in}}{\pgfqpoint{3.079299in}{3.079299in}}%
\pgfusepath{clip}%
\pgfsetroundcap%
\pgfsetroundjoin%
\pgfsetlinewidth{0.301125pt}%
\definecolor{currentstroke}{rgb}{0.500000,0.500000,0.500000}%
\pgfsetstrokecolor{currentstroke}%
\pgfsetstrokeopacity{0.300000}%
\pgfsetdash{}{0pt}%
\pgfpathmoveto{\pgfqpoint{0.809479in}{1.083568in}}%
\pgfusepath{stroke}%
\end{pgfscope}%
\begin{pgfscope}%
\pgfpathrectangle{\pgfqpoint{0.647939in}{0.492442in}}{\pgfqpoint{3.079299in}{3.079299in}}%
\pgfusepath{clip}%
\pgfsetroundcap%
\pgfsetroundjoin%
\definecolor{currentfill}{rgb}{0.500000,0.500000,0.500000}%
\pgfsetfillcolor{currentfill}%
\pgfsetfillopacity{0.300000}%
\pgfsetlinewidth{0.301125pt}%
\definecolor{currentstroke}{rgb}{0.500000,0.500000,0.500000}%
\pgfsetstrokecolor{currentstroke}%
\pgfsetstrokeopacity{0.300000}%
\pgfsetdash{}{0pt}%
\pgfpathmoveto{\pgfqpoint{0.000000in}{0.000000in}}%
\pgfpathlineto{\pgfqpoint{0.000000in}{0.000000in}}%
\pgfpathclose%
\pgfusepath{stroke,fill}%
\end{pgfscope}%
\begin{pgfscope}%
\pgfpathrectangle{\pgfqpoint{0.647939in}{0.492442in}}{\pgfqpoint{3.079299in}{3.079299in}}%
\pgfusepath{clip}%
\pgfsetroundcap%
\pgfsetroundjoin%
\pgfsetlinewidth{0.301125pt}%
\definecolor{currentstroke}{rgb}{0.500000,0.500000,0.500000}%
\pgfsetstrokecolor{currentstroke}%
\pgfsetstrokeopacity{0.300000}%
\pgfsetdash{}{0pt}%
\pgfpathmoveto{\pgfqpoint{1.247268in}{1.201736in}}%
\pgfusepath{stroke}%
\end{pgfscope}%
\begin{pgfscope}%
\pgfpathrectangle{\pgfqpoint{0.647939in}{0.492442in}}{\pgfqpoint{3.079299in}{3.079299in}}%
\pgfusepath{clip}%
\pgfsetroundcap%
\pgfsetroundjoin%
\definecolor{currentfill}{rgb}{0.500000,0.500000,0.500000}%
\pgfsetfillcolor{currentfill}%
\pgfsetfillopacity{0.300000}%
\pgfsetlinewidth{0.301125pt}%
\definecolor{currentstroke}{rgb}{0.500000,0.500000,0.500000}%
\pgfsetstrokecolor{currentstroke}%
\pgfsetstrokeopacity{0.300000}%
\pgfsetdash{}{0pt}%
\pgfpathmoveto{\pgfqpoint{0.000000in}{0.000000in}}%
\pgfpathlineto{\pgfqpoint{0.000000in}{0.000000in}}%
\pgfpathclose%
\pgfusepath{stroke,fill}%
\end{pgfscope}%
\begin{pgfscope}%
\pgfpathrectangle{\pgfqpoint{0.647939in}{0.492442in}}{\pgfqpoint{3.079299in}{3.079299in}}%
\pgfusepath{clip}%
\pgfsetroundcap%
\pgfsetroundjoin%
\pgfsetlinewidth{0.301125pt}%
\definecolor{currentstroke}{rgb}{0.500000,0.500000,0.500000}%
\pgfsetstrokecolor{currentstroke}%
\pgfsetstrokeopacity{0.300000}%
\pgfsetdash{}{0pt}%
\pgfpathmoveto{\pgfqpoint{0.940181in}{0.984197in}}%
\pgfusepath{stroke}%
\end{pgfscope}%
\begin{pgfscope}%
\pgfpathrectangle{\pgfqpoint{0.647939in}{0.492442in}}{\pgfqpoint{3.079299in}{3.079299in}}%
\pgfusepath{clip}%
\pgfsetroundcap%
\pgfsetroundjoin%
\definecolor{currentfill}{rgb}{0.500000,0.500000,0.500000}%
\pgfsetfillcolor{currentfill}%
\pgfsetfillopacity{0.300000}%
\pgfsetlinewidth{0.301125pt}%
\definecolor{currentstroke}{rgb}{0.500000,0.500000,0.500000}%
\pgfsetstrokecolor{currentstroke}%
\pgfsetstrokeopacity{0.300000}%
\pgfsetdash{}{0pt}%
\pgfpathmoveto{\pgfqpoint{0.000000in}{0.000000in}}%
\pgfpathlineto{\pgfqpoint{0.000000in}{0.000000in}}%
\pgfpathclose%
\pgfusepath{stroke,fill}%
\end{pgfscope}%
\begin{pgfscope}%
\pgfpathrectangle{\pgfqpoint{0.647939in}{0.492442in}}{\pgfqpoint{3.079299in}{3.079299in}}%
\pgfusepath{clip}%
\pgfsetroundcap%
\pgfsetroundjoin%
\pgfsetlinewidth{0.301125pt}%
\definecolor{currentstroke}{rgb}{0.500000,0.500000,0.500000}%
\pgfsetstrokecolor{currentstroke}%
\pgfsetstrokeopacity{0.300000}%
\pgfsetdash{}{0pt}%
\pgfpathmoveto{\pgfqpoint{0.874864in}{0.894244in}}%
\pgfusepath{stroke}%
\end{pgfscope}%
\begin{pgfscope}%
\pgfpathrectangle{\pgfqpoint{0.647939in}{0.492442in}}{\pgfqpoint{3.079299in}{3.079299in}}%
\pgfusepath{clip}%
\pgfsetroundcap%
\pgfsetroundjoin%
\definecolor{currentfill}{rgb}{0.500000,0.500000,0.500000}%
\pgfsetfillcolor{currentfill}%
\pgfsetfillopacity{0.300000}%
\pgfsetlinewidth{0.301125pt}%
\definecolor{currentstroke}{rgb}{0.500000,0.500000,0.500000}%
\pgfsetstrokecolor{currentstroke}%
\pgfsetstrokeopacity{0.300000}%
\pgfsetdash{}{0pt}%
\pgfpathmoveto{\pgfqpoint{0.000000in}{0.000000in}}%
\pgfpathlineto{\pgfqpoint{0.000000in}{0.000000in}}%
\pgfpathclose%
\pgfusepath{stroke,fill}%
\end{pgfscope}%
\begin{pgfscope}%
\pgfpathrectangle{\pgfqpoint{0.647939in}{0.492442in}}{\pgfqpoint{3.079299in}{3.079299in}}%
\pgfusepath{clip}%
\pgfsetroundcap%
\pgfsetroundjoin%
\pgfsetlinewidth{0.301125pt}%
\definecolor{currentstroke}{rgb}{0.500000,0.500000,0.500000}%
\pgfsetstrokecolor{currentstroke}%
\pgfsetstrokeopacity{0.300000}%
\pgfsetdash{}{0pt}%
\pgfpathmoveto{\pgfqpoint{1.123677in}{0.936799in}}%
\pgfusepath{stroke}%
\end{pgfscope}%
\begin{pgfscope}%
\pgfpathrectangle{\pgfqpoint{0.647939in}{0.492442in}}{\pgfqpoint{3.079299in}{3.079299in}}%
\pgfusepath{clip}%
\pgfsetroundcap%
\pgfsetroundjoin%
\definecolor{currentfill}{rgb}{0.500000,0.500000,0.500000}%
\pgfsetfillcolor{currentfill}%
\pgfsetfillopacity{0.300000}%
\pgfsetlinewidth{0.301125pt}%
\definecolor{currentstroke}{rgb}{0.500000,0.500000,0.500000}%
\pgfsetstrokecolor{currentstroke}%
\pgfsetstrokeopacity{0.300000}%
\pgfsetdash{}{0pt}%
\pgfpathmoveto{\pgfqpoint{0.000000in}{0.000000in}}%
\pgfpathlineto{\pgfqpoint{0.000000in}{0.000000in}}%
\pgfpathclose%
\pgfusepath{stroke,fill}%
\end{pgfscope}%
\begin{pgfscope}%
\pgfpathrectangle{\pgfqpoint{0.647939in}{0.492442in}}{\pgfqpoint{3.079299in}{3.079299in}}%
\pgfusepath{clip}%
\pgfsetroundcap%
\pgfsetroundjoin%
\pgfsetlinewidth{0.301125pt}%
\definecolor{currentstroke}{rgb}{0.500000,0.500000,0.500000}%
\pgfsetstrokecolor{currentstroke}%
\pgfsetstrokeopacity{0.300000}%
\pgfsetdash{}{0pt}%
\pgfpathmoveto{\pgfqpoint{0.938575in}{0.779888in}}%
\pgfusepath{stroke}%
\end{pgfscope}%
\begin{pgfscope}%
\pgfpathrectangle{\pgfqpoint{0.647939in}{0.492442in}}{\pgfqpoint{3.079299in}{3.079299in}}%
\pgfusepath{clip}%
\pgfsetroundcap%
\pgfsetroundjoin%
\definecolor{currentfill}{rgb}{0.500000,0.500000,0.500000}%
\pgfsetfillcolor{currentfill}%
\pgfsetfillopacity{0.300000}%
\pgfsetlinewidth{0.301125pt}%
\definecolor{currentstroke}{rgb}{0.500000,0.500000,0.500000}%
\pgfsetstrokecolor{currentstroke}%
\pgfsetstrokeopacity{0.300000}%
\pgfsetdash{}{0pt}%
\pgfpathmoveto{\pgfqpoint{0.000000in}{0.000000in}}%
\pgfpathlineto{\pgfqpoint{0.000000in}{0.000000in}}%
\pgfpathclose%
\pgfusepath{stroke,fill}%
\end{pgfscope}%
\begin{pgfscope}%
\pgfpathrectangle{\pgfqpoint{0.647939in}{0.492442in}}{\pgfqpoint{3.079299in}{3.079299in}}%
\pgfusepath{clip}%
\pgfsetroundcap%
\pgfsetroundjoin%
\pgfsetlinewidth{0.301125pt}%
\definecolor{currentstroke}{rgb}{0.500000,0.500000,0.500000}%
\pgfsetstrokecolor{currentstroke}%
\pgfsetstrokeopacity{0.300000}%
\pgfsetdash{}{0pt}%
\pgfpathmoveto{\pgfqpoint{0.937950in}{0.711972in}}%
\pgfusepath{stroke}%
\end{pgfscope}%
\begin{pgfscope}%
\pgfpathrectangle{\pgfqpoint{0.647939in}{0.492442in}}{\pgfqpoint{3.079299in}{3.079299in}}%
\pgfusepath{clip}%
\pgfsetroundcap%
\pgfsetroundjoin%
\definecolor{currentfill}{rgb}{0.500000,0.500000,0.500000}%
\pgfsetfillcolor{currentfill}%
\pgfsetfillopacity{0.300000}%
\pgfsetlinewidth{0.301125pt}%
\definecolor{currentstroke}{rgb}{0.500000,0.500000,0.500000}%
\pgfsetstrokecolor{currentstroke}%
\pgfsetstrokeopacity{0.300000}%
\pgfsetdash{}{0pt}%
\pgfpathmoveto{\pgfqpoint{0.000000in}{0.000000in}}%
\pgfpathlineto{\pgfqpoint{0.000000in}{0.000000in}}%
\pgfpathclose%
\pgfusepath{stroke,fill}%
\end{pgfscope}%
\begin{pgfscope}%
\pgfpathrectangle{\pgfqpoint{0.647939in}{0.492442in}}{\pgfqpoint{3.079299in}{3.079299in}}%
\pgfusepath{clip}%
\pgfsetroundcap%
\pgfsetroundjoin%
\pgfsetlinewidth{0.301125pt}%
\definecolor{currentstroke}{rgb}{0.500000,0.500000,0.500000}%
\pgfsetstrokecolor{currentstroke}%
\pgfsetstrokeopacity{0.300000}%
\pgfsetdash{}{0pt}%
\pgfpathmoveto{\pgfqpoint{0.873473in}{0.619628in}}%
\pgfusepath{stroke}%
\end{pgfscope}%
\begin{pgfscope}%
\pgfpathrectangle{\pgfqpoint{0.647939in}{0.492442in}}{\pgfqpoint{3.079299in}{3.079299in}}%
\pgfusepath{clip}%
\pgfsetroundcap%
\pgfsetroundjoin%
\definecolor{currentfill}{rgb}{0.500000,0.500000,0.500000}%
\pgfsetfillcolor{currentfill}%
\pgfsetfillopacity{0.300000}%
\pgfsetlinewidth{0.301125pt}%
\definecolor{currentstroke}{rgb}{0.500000,0.500000,0.500000}%
\pgfsetstrokecolor{currentstroke}%
\pgfsetstrokeopacity{0.300000}%
\pgfsetdash{}{0pt}%
\pgfpathmoveto{\pgfqpoint{0.000000in}{0.000000in}}%
\pgfpathlineto{\pgfqpoint{0.000000in}{0.000000in}}%
\pgfpathclose%
\pgfusepath{stroke,fill}%
\end{pgfscope}%
\begin{pgfscope}%
\pgfpathrectangle{\pgfqpoint{0.647939in}{0.492442in}}{\pgfqpoint{3.079299in}{3.079299in}}%
\pgfusepath{clip}%
\pgfsetroundcap%
\pgfsetroundjoin%
\pgfsetlinewidth{0.301125pt}%
\definecolor{currentstroke}{rgb}{0.500000,0.500000,0.500000}%
\pgfsetstrokecolor{currentstroke}%
\pgfsetstrokeopacity{0.300000}%
\pgfsetdash{}{0pt}%
\pgfpathmoveto{\pgfqpoint{0.890114in}{3.414714in}}%
\pgfusepath{stroke}%
\end{pgfscope}%
\begin{pgfscope}%
\pgfpathrectangle{\pgfqpoint{0.647939in}{0.492442in}}{\pgfqpoint{3.079299in}{3.079299in}}%
\pgfusepath{clip}%
\pgfsetroundcap%
\pgfsetroundjoin%
\definecolor{currentfill}{rgb}{0.500000,0.500000,0.500000}%
\pgfsetfillcolor{currentfill}%
\pgfsetfillopacity{0.300000}%
\pgfsetlinewidth{0.301125pt}%
\definecolor{currentstroke}{rgb}{0.500000,0.500000,0.500000}%
\pgfsetstrokecolor{currentstroke}%
\pgfsetstrokeopacity{0.300000}%
\pgfsetdash{}{0pt}%
\pgfpathmoveto{\pgfqpoint{0.000000in}{0.000000in}}%
\pgfpathlineto{\pgfqpoint{0.000000in}{0.000000in}}%
\pgfpathclose%
\pgfusepath{stroke,fill}%
\end{pgfscope}%
\begin{pgfscope}%
\pgfpathrectangle{\pgfqpoint{0.647939in}{0.492442in}}{\pgfqpoint{3.079299in}{3.079299in}}%
\pgfusepath{clip}%
\pgfsetroundcap%
\pgfsetroundjoin%
\pgfsetlinewidth{0.301125pt}%
\definecolor{currentstroke}{rgb}{0.500000,0.500000,0.500000}%
\pgfsetstrokecolor{currentstroke}%
\pgfsetstrokeopacity{0.300000}%
\pgfsetdash{}{0pt}%
\pgfpathmoveto{\pgfqpoint{2.602520in}{0.806423in}}%
\pgfusepath{stroke}%
\end{pgfscope}%
\begin{pgfscope}%
\pgfpathrectangle{\pgfqpoint{0.647939in}{0.492442in}}{\pgfqpoint{3.079299in}{3.079299in}}%
\pgfusepath{clip}%
\pgfsetroundcap%
\pgfsetroundjoin%
\definecolor{currentfill}{rgb}{0.500000,0.500000,0.500000}%
\pgfsetfillcolor{currentfill}%
\pgfsetfillopacity{0.300000}%
\pgfsetlinewidth{0.301125pt}%
\definecolor{currentstroke}{rgb}{0.500000,0.500000,0.500000}%
\pgfsetstrokecolor{currentstroke}%
\pgfsetstrokeopacity{0.300000}%
\pgfsetdash{}{0pt}%
\pgfpathmoveto{\pgfqpoint{0.000000in}{0.000000in}}%
\pgfpathlineto{\pgfqpoint{0.000000in}{0.000000in}}%
\pgfpathclose%
\pgfusepath{stroke,fill}%
\end{pgfscope}%
\begin{pgfscope}%
\pgfpathrectangle{\pgfqpoint{0.647939in}{0.492442in}}{\pgfqpoint{3.079299in}{3.079299in}}%
\pgfusepath{clip}%
\pgfsetroundcap%
\pgfsetroundjoin%
\pgfsetlinewidth{0.301125pt}%
\definecolor{currentstroke}{rgb}{0.500000,0.500000,0.500000}%
\pgfsetstrokecolor{currentstroke}%
\pgfsetstrokeopacity{0.300000}%
\pgfsetdash{}{0pt}%
\pgfpathmoveto{\pgfqpoint{3.187099in}{2.272959in}}%
\pgfusepath{stroke}%
\end{pgfscope}%
\begin{pgfscope}%
\pgfpathrectangle{\pgfqpoint{0.647939in}{0.492442in}}{\pgfqpoint{3.079299in}{3.079299in}}%
\pgfusepath{clip}%
\pgfsetroundcap%
\pgfsetroundjoin%
\definecolor{currentfill}{rgb}{0.500000,0.500000,0.500000}%
\pgfsetfillcolor{currentfill}%
\pgfsetfillopacity{0.300000}%
\pgfsetlinewidth{0.301125pt}%
\definecolor{currentstroke}{rgb}{0.500000,0.500000,0.500000}%
\pgfsetstrokecolor{currentstroke}%
\pgfsetstrokeopacity{0.300000}%
\pgfsetdash{}{0pt}%
\pgfpathmoveto{\pgfqpoint{0.000000in}{0.000000in}}%
\pgfpathlineto{\pgfqpoint{0.000000in}{0.000000in}}%
\pgfpathclose%
\pgfusepath{stroke,fill}%
\end{pgfscope}%
\begin{pgfscope}%
\pgfpathrectangle{\pgfqpoint{0.647939in}{0.492442in}}{\pgfqpoint{3.079299in}{3.079299in}}%
\pgfusepath{clip}%
\pgfsetroundcap%
\pgfsetroundjoin%
\pgfsetlinewidth{0.301125pt}%
\definecolor{currentstroke}{rgb}{0.500000,0.500000,0.500000}%
\pgfsetstrokecolor{currentstroke}%
\pgfsetstrokeopacity{0.300000}%
\pgfsetdash{}{0pt}%
\pgfpathmoveto{\pgfqpoint{2.160150in}{3.341999in}}%
\pgfusepath{stroke}%
\end{pgfscope}%
\begin{pgfscope}%
\pgfpathrectangle{\pgfqpoint{0.647939in}{0.492442in}}{\pgfqpoint{3.079299in}{3.079299in}}%
\pgfusepath{clip}%
\pgfsetroundcap%
\pgfsetroundjoin%
\definecolor{currentfill}{rgb}{0.500000,0.500000,0.500000}%
\pgfsetfillcolor{currentfill}%
\pgfsetfillopacity{0.300000}%
\pgfsetlinewidth{0.301125pt}%
\definecolor{currentstroke}{rgb}{0.500000,0.500000,0.500000}%
\pgfsetstrokecolor{currentstroke}%
\pgfsetstrokeopacity{0.300000}%
\pgfsetdash{}{0pt}%
\pgfpathmoveto{\pgfqpoint{0.000000in}{0.000000in}}%
\pgfpathlineto{\pgfqpoint{0.000000in}{0.000000in}}%
\pgfpathclose%
\pgfusepath{stroke,fill}%
\end{pgfscope}%
\begin{pgfscope}%
\pgfpathrectangle{\pgfqpoint{0.647939in}{0.492442in}}{\pgfqpoint{3.079299in}{3.079299in}}%
\pgfusepath{clip}%
\pgfsetroundcap%
\pgfsetroundjoin%
\pgfsetlinewidth{0.301125pt}%
\definecolor{currentstroke}{rgb}{0.500000,0.500000,0.500000}%
\pgfsetstrokecolor{currentstroke}%
\pgfsetstrokeopacity{0.300000}%
\pgfsetdash{}{0pt}%
\pgfpathmoveto{\pgfqpoint{2.365006in}{0.778012in}}%
\pgfusepath{stroke}%
\end{pgfscope}%
\begin{pgfscope}%
\pgfpathrectangle{\pgfqpoint{0.647939in}{0.492442in}}{\pgfqpoint{3.079299in}{3.079299in}}%
\pgfusepath{clip}%
\pgfsetroundcap%
\pgfsetroundjoin%
\definecolor{currentfill}{rgb}{0.500000,0.500000,0.500000}%
\pgfsetfillcolor{currentfill}%
\pgfsetfillopacity{0.300000}%
\pgfsetlinewidth{0.301125pt}%
\definecolor{currentstroke}{rgb}{0.500000,0.500000,0.500000}%
\pgfsetstrokecolor{currentstroke}%
\pgfsetstrokeopacity{0.300000}%
\pgfsetdash{}{0pt}%
\pgfpathmoveto{\pgfqpoint{0.000000in}{0.000000in}}%
\pgfpathlineto{\pgfqpoint{0.000000in}{0.000000in}}%
\pgfpathclose%
\pgfusepath{stroke,fill}%
\end{pgfscope}%
\begin{pgfscope}%
\pgfpathrectangle{\pgfqpoint{0.647939in}{0.492442in}}{\pgfqpoint{3.079299in}{3.079299in}}%
\pgfusepath{clip}%
\pgfsetroundcap%
\pgfsetroundjoin%
\pgfsetlinewidth{0.301125pt}%
\definecolor{currentstroke}{rgb}{0.500000,0.500000,0.500000}%
\pgfsetstrokecolor{currentstroke}%
\pgfsetstrokeopacity{0.300000}%
\pgfsetdash{}{0pt}%
\pgfpathmoveto{\pgfqpoint{3.311285in}{1.704874in}}%
\pgfusepath{stroke}%
\end{pgfscope}%
\begin{pgfscope}%
\pgfpathrectangle{\pgfqpoint{0.647939in}{0.492442in}}{\pgfqpoint{3.079299in}{3.079299in}}%
\pgfusepath{clip}%
\pgfsetroundcap%
\pgfsetroundjoin%
\definecolor{currentfill}{rgb}{0.500000,0.500000,0.500000}%
\pgfsetfillcolor{currentfill}%
\pgfsetfillopacity{0.300000}%
\pgfsetlinewidth{0.301125pt}%
\definecolor{currentstroke}{rgb}{0.500000,0.500000,0.500000}%
\pgfsetstrokecolor{currentstroke}%
\pgfsetstrokeopacity{0.300000}%
\pgfsetdash{}{0pt}%
\pgfpathmoveto{\pgfqpoint{0.000000in}{0.000000in}}%
\pgfpathlineto{\pgfqpoint{0.000000in}{0.000000in}}%
\pgfpathclose%
\pgfusepath{stroke,fill}%
\end{pgfscope}%
\begin{pgfscope}%
\pgfpathrectangle{\pgfqpoint{0.647939in}{0.492442in}}{\pgfqpoint{3.079299in}{3.079299in}}%
\pgfusepath{clip}%
\pgfsetroundcap%
\pgfsetroundjoin%
\pgfsetlinewidth{0.301125pt}%
\definecolor{currentstroke}{rgb}{0.500000,0.500000,0.500000}%
\pgfsetstrokecolor{currentstroke}%
\pgfsetstrokeopacity{0.300000}%
\pgfsetdash{}{0pt}%
\pgfpathmoveto{\pgfqpoint{3.404125in}{2.747775in}}%
\pgfusepath{stroke}%
\end{pgfscope}%
\begin{pgfscope}%
\pgfpathrectangle{\pgfqpoint{0.647939in}{0.492442in}}{\pgfqpoint{3.079299in}{3.079299in}}%
\pgfusepath{clip}%
\pgfsetroundcap%
\pgfsetroundjoin%
\definecolor{currentfill}{rgb}{0.500000,0.500000,0.500000}%
\pgfsetfillcolor{currentfill}%
\pgfsetfillopacity{0.300000}%
\pgfsetlinewidth{0.301125pt}%
\definecolor{currentstroke}{rgb}{0.500000,0.500000,0.500000}%
\pgfsetstrokecolor{currentstroke}%
\pgfsetstrokeopacity{0.300000}%
\pgfsetdash{}{0pt}%
\pgfpathmoveto{\pgfqpoint{0.000000in}{0.000000in}}%
\pgfpathlineto{\pgfqpoint{0.000000in}{0.000000in}}%
\pgfpathclose%
\pgfusepath{stroke,fill}%
\end{pgfscope}%
\begin{pgfscope}%
\pgfpathrectangle{\pgfqpoint{0.647939in}{0.492442in}}{\pgfqpoint{3.079299in}{3.079299in}}%
\pgfusepath{clip}%
\pgfsetroundcap%
\pgfsetroundjoin%
\pgfsetlinewidth{0.301125pt}%
\definecolor{currentstroke}{rgb}{0.500000,0.500000,0.500000}%
\pgfsetstrokecolor{currentstroke}%
\pgfsetstrokeopacity{0.300000}%
\pgfsetdash{}{0pt}%
\pgfpathmoveto{\pgfqpoint{3.277194in}{2.451592in}}%
\pgfusepath{stroke}%
\end{pgfscope}%
\begin{pgfscope}%
\pgfpathrectangle{\pgfqpoint{0.647939in}{0.492442in}}{\pgfqpoint{3.079299in}{3.079299in}}%
\pgfusepath{clip}%
\pgfsetroundcap%
\pgfsetroundjoin%
\definecolor{currentfill}{rgb}{0.500000,0.500000,0.500000}%
\pgfsetfillcolor{currentfill}%
\pgfsetfillopacity{0.300000}%
\pgfsetlinewidth{0.301125pt}%
\definecolor{currentstroke}{rgb}{0.500000,0.500000,0.500000}%
\pgfsetstrokecolor{currentstroke}%
\pgfsetstrokeopacity{0.300000}%
\pgfsetdash{}{0pt}%
\pgfpathmoveto{\pgfqpoint{0.000000in}{0.000000in}}%
\pgfpathlineto{\pgfqpoint{0.000000in}{0.000000in}}%
\pgfpathclose%
\pgfusepath{stroke,fill}%
\end{pgfscope}%
\begin{pgfscope}%
\pgfpathrectangle{\pgfqpoint{0.647939in}{0.492442in}}{\pgfqpoint{3.079299in}{3.079299in}}%
\pgfusepath{clip}%
\pgfsetroundcap%
\pgfsetroundjoin%
\pgfsetlinewidth{0.301125pt}%
\definecolor{currentstroke}{rgb}{0.500000,0.500000,0.500000}%
\pgfsetstrokecolor{currentstroke}%
\pgfsetstrokeopacity{0.300000}%
\pgfsetdash{}{0pt}%
\pgfpathmoveto{\pgfqpoint{3.009549in}{2.212494in}}%
\pgfusepath{stroke}%
\end{pgfscope}%
\begin{pgfscope}%
\pgfpathrectangle{\pgfqpoint{0.647939in}{0.492442in}}{\pgfqpoint{3.079299in}{3.079299in}}%
\pgfusepath{clip}%
\pgfsetroundcap%
\pgfsetroundjoin%
\definecolor{currentfill}{rgb}{0.500000,0.500000,0.500000}%
\pgfsetfillcolor{currentfill}%
\pgfsetfillopacity{0.300000}%
\pgfsetlinewidth{0.301125pt}%
\definecolor{currentstroke}{rgb}{0.500000,0.500000,0.500000}%
\pgfsetstrokecolor{currentstroke}%
\pgfsetstrokeopacity{0.300000}%
\pgfsetdash{}{0pt}%
\pgfpathmoveto{\pgfqpoint{0.000000in}{0.000000in}}%
\pgfpathlineto{\pgfqpoint{0.000000in}{0.000000in}}%
\pgfpathclose%
\pgfusepath{stroke,fill}%
\end{pgfscope}%
\begin{pgfscope}%
\pgfpathrectangle{\pgfqpoint{0.647939in}{0.492442in}}{\pgfqpoint{3.079299in}{3.079299in}}%
\pgfusepath{clip}%
\pgfsetroundcap%
\pgfsetroundjoin%
\pgfsetlinewidth{0.301125pt}%
\definecolor{currentstroke}{rgb}{0.500000,0.500000,0.500000}%
\pgfsetstrokecolor{currentstroke}%
\pgfsetstrokeopacity{0.300000}%
\pgfsetdash{}{0pt}%
\pgfpathmoveto{\pgfqpoint{2.247570in}{3.055035in}}%
\pgfusepath{stroke}%
\end{pgfscope}%
\begin{pgfscope}%
\pgfpathrectangle{\pgfqpoint{0.647939in}{0.492442in}}{\pgfqpoint{3.079299in}{3.079299in}}%
\pgfusepath{clip}%
\pgfsetroundcap%
\pgfsetroundjoin%
\definecolor{currentfill}{rgb}{0.500000,0.500000,0.500000}%
\pgfsetfillcolor{currentfill}%
\pgfsetfillopacity{0.300000}%
\pgfsetlinewidth{0.301125pt}%
\definecolor{currentstroke}{rgb}{0.500000,0.500000,0.500000}%
\pgfsetstrokecolor{currentstroke}%
\pgfsetstrokeopacity{0.300000}%
\pgfsetdash{}{0pt}%
\pgfpathmoveto{\pgfqpoint{0.000000in}{0.000000in}}%
\pgfpathlineto{\pgfqpoint{0.000000in}{0.000000in}}%
\pgfpathclose%
\pgfusepath{stroke,fill}%
\end{pgfscope}%
\begin{pgfscope}%
\pgfpathrectangle{\pgfqpoint{0.647939in}{0.492442in}}{\pgfqpoint{3.079299in}{3.079299in}}%
\pgfusepath{clip}%
\pgfsetroundcap%
\pgfsetroundjoin%
\pgfsetlinewidth{0.301125pt}%
\definecolor{currentstroke}{rgb}{0.500000,0.500000,0.500000}%
\pgfsetstrokecolor{currentstroke}%
\pgfsetstrokeopacity{0.300000}%
\pgfsetdash{}{0pt}%
\pgfpathmoveto{\pgfqpoint{3.000627in}{3.051257in}}%
\pgfusepath{stroke}%
\end{pgfscope}%
\begin{pgfscope}%
\pgfpathrectangle{\pgfqpoint{0.647939in}{0.492442in}}{\pgfqpoint{3.079299in}{3.079299in}}%
\pgfusepath{clip}%
\pgfsetroundcap%
\pgfsetroundjoin%
\definecolor{currentfill}{rgb}{0.500000,0.500000,0.500000}%
\pgfsetfillcolor{currentfill}%
\pgfsetfillopacity{0.300000}%
\pgfsetlinewidth{0.301125pt}%
\definecolor{currentstroke}{rgb}{0.500000,0.500000,0.500000}%
\pgfsetstrokecolor{currentstroke}%
\pgfsetstrokeopacity{0.300000}%
\pgfsetdash{}{0pt}%
\pgfpathmoveto{\pgfqpoint{0.000000in}{0.000000in}}%
\pgfpathlineto{\pgfqpoint{0.000000in}{0.000000in}}%
\pgfpathclose%
\pgfusepath{stroke,fill}%
\end{pgfscope}%
\begin{pgfscope}%
\pgfpathrectangle{\pgfqpoint{0.647939in}{0.492442in}}{\pgfqpoint{3.079299in}{3.079299in}}%
\pgfusepath{clip}%
\pgfsetroundcap%
\pgfsetroundjoin%
\pgfsetlinewidth{0.301125pt}%
\definecolor{currentstroke}{rgb}{0.500000,0.500000,0.500000}%
\pgfsetstrokecolor{currentstroke}%
\pgfsetstrokeopacity{0.300000}%
\pgfsetdash{}{0pt}%
\pgfpathmoveto{\pgfqpoint{1.400892in}{1.231316in}}%
\pgfusepath{stroke}%
\end{pgfscope}%
\begin{pgfscope}%
\pgfpathrectangle{\pgfqpoint{0.647939in}{0.492442in}}{\pgfqpoint{3.079299in}{3.079299in}}%
\pgfusepath{clip}%
\pgfsetroundcap%
\pgfsetroundjoin%
\definecolor{currentfill}{rgb}{0.500000,0.500000,0.500000}%
\pgfsetfillcolor{currentfill}%
\pgfsetfillopacity{0.300000}%
\pgfsetlinewidth{0.301125pt}%
\definecolor{currentstroke}{rgb}{0.500000,0.500000,0.500000}%
\pgfsetstrokecolor{currentstroke}%
\pgfsetstrokeopacity{0.300000}%
\pgfsetdash{}{0pt}%
\pgfpathmoveto{\pgfqpoint{0.000000in}{0.000000in}}%
\pgfpathlineto{\pgfqpoint{0.000000in}{0.000000in}}%
\pgfpathclose%
\pgfusepath{stroke,fill}%
\end{pgfscope}%
\begin{pgfscope}%
\pgfpathrectangle{\pgfqpoint{0.647939in}{0.492442in}}{\pgfqpoint{3.079299in}{3.079299in}}%
\pgfusepath{clip}%
\pgfsetroundcap%
\pgfsetroundjoin%
\pgfsetlinewidth{0.301125pt}%
\definecolor{currentstroke}{rgb}{0.500000,0.500000,0.500000}%
\pgfsetstrokecolor{currentstroke}%
\pgfsetstrokeopacity{0.300000}%
\pgfsetdash{}{0pt}%
\pgfpathmoveto{\pgfqpoint{2.765603in}{2.178393in}}%
\pgfusepath{stroke}%
\end{pgfscope}%
\begin{pgfscope}%
\pgfpathrectangle{\pgfqpoint{0.647939in}{0.492442in}}{\pgfqpoint{3.079299in}{3.079299in}}%
\pgfusepath{clip}%
\pgfsetroundcap%
\pgfsetroundjoin%
\definecolor{currentfill}{rgb}{0.500000,0.500000,0.500000}%
\pgfsetfillcolor{currentfill}%
\pgfsetfillopacity{0.300000}%
\pgfsetlinewidth{0.301125pt}%
\definecolor{currentstroke}{rgb}{0.500000,0.500000,0.500000}%
\pgfsetstrokecolor{currentstroke}%
\pgfsetstrokeopacity{0.300000}%
\pgfsetdash{}{0pt}%
\pgfpathmoveto{\pgfqpoint{0.000000in}{0.000000in}}%
\pgfpathlineto{\pgfqpoint{0.000000in}{0.000000in}}%
\pgfpathclose%
\pgfusepath{stroke,fill}%
\end{pgfscope}%
\begin{pgfscope}%
\pgfpathrectangle{\pgfqpoint{0.647939in}{0.492442in}}{\pgfqpoint{3.079299in}{3.079299in}}%
\pgfusepath{clip}%
\pgfsetroundcap%
\pgfsetroundjoin%
\pgfsetlinewidth{0.301125pt}%
\definecolor{currentstroke}{rgb}{0.500000,0.500000,0.500000}%
\pgfsetstrokecolor{currentstroke}%
\pgfsetstrokeopacity{0.300000}%
\pgfsetdash{}{0pt}%
\pgfpathmoveto{\pgfqpoint{2.090752in}{2.933492in}}%
\pgfusepath{stroke}%
\end{pgfscope}%
\begin{pgfscope}%
\pgfpathrectangle{\pgfqpoint{0.647939in}{0.492442in}}{\pgfqpoint{3.079299in}{3.079299in}}%
\pgfusepath{clip}%
\pgfsetroundcap%
\pgfsetroundjoin%
\definecolor{currentfill}{rgb}{0.500000,0.500000,0.500000}%
\pgfsetfillcolor{currentfill}%
\pgfsetfillopacity{0.300000}%
\pgfsetlinewidth{0.301125pt}%
\definecolor{currentstroke}{rgb}{0.500000,0.500000,0.500000}%
\pgfsetstrokecolor{currentstroke}%
\pgfsetstrokeopacity{0.300000}%
\pgfsetdash{}{0pt}%
\pgfpathmoveto{\pgfqpoint{0.000000in}{0.000000in}}%
\pgfpathlineto{\pgfqpoint{0.000000in}{0.000000in}}%
\pgfpathclose%
\pgfusepath{stroke,fill}%
\end{pgfscope}%
\begin{pgfscope}%
\pgfpathrectangle{\pgfqpoint{0.647939in}{0.492442in}}{\pgfqpoint{3.079299in}{3.079299in}}%
\pgfusepath{clip}%
\pgfsetroundcap%
\pgfsetroundjoin%
\pgfsetlinewidth{0.301125pt}%
\definecolor{currentstroke}{rgb}{0.500000,0.500000,0.500000}%
\pgfsetstrokecolor{currentstroke}%
\pgfsetstrokeopacity{0.300000}%
\pgfsetdash{}{0pt}%
\pgfpathmoveto{\pgfqpoint{1.436156in}{2.496967in}}%
\pgfusepath{stroke}%
\end{pgfscope}%
\begin{pgfscope}%
\pgfpathrectangle{\pgfqpoint{0.647939in}{0.492442in}}{\pgfqpoint{3.079299in}{3.079299in}}%
\pgfusepath{clip}%
\pgfsetroundcap%
\pgfsetroundjoin%
\definecolor{currentfill}{rgb}{0.500000,0.500000,0.500000}%
\pgfsetfillcolor{currentfill}%
\pgfsetfillopacity{0.300000}%
\pgfsetlinewidth{0.301125pt}%
\definecolor{currentstroke}{rgb}{0.500000,0.500000,0.500000}%
\pgfsetstrokecolor{currentstroke}%
\pgfsetstrokeopacity{0.300000}%
\pgfsetdash{}{0pt}%
\pgfpathmoveto{\pgfqpoint{0.000000in}{0.000000in}}%
\pgfpathlineto{\pgfqpoint{0.000000in}{0.000000in}}%
\pgfpathclose%
\pgfusepath{stroke,fill}%
\end{pgfscope}%
\begin{pgfscope}%
\pgfpathrectangle{\pgfqpoint{0.647939in}{0.492442in}}{\pgfqpoint{3.079299in}{3.079299in}}%
\pgfusepath{clip}%
\pgfsetroundcap%
\pgfsetroundjoin%
\pgfsetlinewidth{0.301125pt}%
\definecolor{currentstroke}{rgb}{0.500000,0.500000,0.500000}%
\pgfsetstrokecolor{currentstroke}%
\pgfsetstrokeopacity{0.300000}%
\pgfsetdash{}{0pt}%
\pgfpathmoveto{\pgfqpoint{1.430830in}{1.590711in}}%
\pgfusepath{stroke}%
\end{pgfscope}%
\begin{pgfscope}%
\pgfpathrectangle{\pgfqpoint{0.647939in}{0.492442in}}{\pgfqpoint{3.079299in}{3.079299in}}%
\pgfusepath{clip}%
\pgfsetroundcap%
\pgfsetroundjoin%
\definecolor{currentfill}{rgb}{0.500000,0.500000,0.500000}%
\pgfsetfillcolor{currentfill}%
\pgfsetfillopacity{0.300000}%
\pgfsetlinewidth{0.301125pt}%
\definecolor{currentstroke}{rgb}{0.500000,0.500000,0.500000}%
\pgfsetstrokecolor{currentstroke}%
\pgfsetstrokeopacity{0.300000}%
\pgfsetdash{}{0pt}%
\pgfpathmoveto{\pgfqpoint{0.000000in}{0.000000in}}%
\pgfpathlineto{\pgfqpoint{0.000000in}{0.000000in}}%
\pgfpathclose%
\pgfusepath{stroke,fill}%
\end{pgfscope}%
\begin{pgfscope}%
\pgfpathrectangle{\pgfqpoint{0.647939in}{0.492442in}}{\pgfqpoint{3.079299in}{3.079299in}}%
\pgfusepath{clip}%
\pgfsetroundcap%
\pgfsetroundjoin%
\pgfsetlinewidth{0.301125pt}%
\definecolor{currentstroke}{rgb}{0.500000,0.500000,0.500000}%
\pgfsetstrokecolor{currentstroke}%
\pgfsetstrokeopacity{0.300000}%
\pgfsetdash{}{0pt}%
\pgfpathmoveto{\pgfqpoint{2.084938in}{1.264112in}}%
\pgfusepath{stroke}%
\end{pgfscope}%
\begin{pgfscope}%
\pgfpathrectangle{\pgfqpoint{0.647939in}{0.492442in}}{\pgfqpoint{3.079299in}{3.079299in}}%
\pgfusepath{clip}%
\pgfsetroundcap%
\pgfsetroundjoin%
\definecolor{currentfill}{rgb}{0.500000,0.500000,0.500000}%
\pgfsetfillcolor{currentfill}%
\pgfsetfillopacity{0.300000}%
\pgfsetlinewidth{0.301125pt}%
\definecolor{currentstroke}{rgb}{0.500000,0.500000,0.500000}%
\pgfsetstrokecolor{currentstroke}%
\pgfsetstrokeopacity{0.300000}%
\pgfsetdash{}{0pt}%
\pgfpathmoveto{\pgfqpoint{0.000000in}{0.000000in}}%
\pgfpathlineto{\pgfqpoint{0.000000in}{0.000000in}}%
\pgfpathclose%
\pgfusepath{stroke,fill}%
\end{pgfscope}%
\begin{pgfscope}%
\pgfpathrectangle{\pgfqpoint{0.647939in}{0.492442in}}{\pgfqpoint{3.079299in}{3.079299in}}%
\pgfusepath{clip}%
\pgfsetroundcap%
\pgfsetroundjoin%
\pgfsetlinewidth{0.301125pt}%
\definecolor{currentstroke}{rgb}{0.500000,0.500000,0.500000}%
\pgfsetstrokecolor{currentstroke}%
\pgfsetstrokeopacity{0.300000}%
\pgfsetdash{}{0pt}%
\pgfpathmoveto{\pgfqpoint{2.970035in}{2.413496in}}%
\pgfusepath{stroke}%
\end{pgfscope}%
\begin{pgfscope}%
\pgfpathrectangle{\pgfqpoint{0.647939in}{0.492442in}}{\pgfqpoint{3.079299in}{3.079299in}}%
\pgfusepath{clip}%
\pgfsetroundcap%
\pgfsetroundjoin%
\definecolor{currentfill}{rgb}{0.500000,0.500000,0.500000}%
\pgfsetfillcolor{currentfill}%
\pgfsetfillopacity{0.300000}%
\pgfsetlinewidth{0.301125pt}%
\definecolor{currentstroke}{rgb}{0.500000,0.500000,0.500000}%
\pgfsetstrokecolor{currentstroke}%
\pgfsetstrokeopacity{0.300000}%
\pgfsetdash{}{0pt}%
\pgfpathmoveto{\pgfqpoint{0.000000in}{0.000000in}}%
\pgfpathlineto{\pgfqpoint{0.000000in}{0.000000in}}%
\pgfpathclose%
\pgfusepath{stroke,fill}%
\end{pgfscope}%
\begin{pgfscope}%
\pgfpathrectangle{\pgfqpoint{0.647939in}{0.492442in}}{\pgfqpoint{3.079299in}{3.079299in}}%
\pgfusepath{clip}%
\pgfsetroundcap%
\pgfsetroundjoin%
\pgfsetlinewidth{0.301125pt}%
\definecolor{currentstroke}{rgb}{0.500000,0.500000,0.500000}%
\pgfsetstrokecolor{currentstroke}%
\pgfsetstrokeopacity{0.300000}%
\pgfsetdash{}{0pt}%
\pgfpathmoveto{\pgfqpoint{2.269727in}{2.662233in}}%
\pgfusepath{stroke}%
\end{pgfscope}%
\begin{pgfscope}%
\pgfpathrectangle{\pgfqpoint{0.647939in}{0.492442in}}{\pgfqpoint{3.079299in}{3.079299in}}%
\pgfusepath{clip}%
\pgfsetroundcap%
\pgfsetroundjoin%
\definecolor{currentfill}{rgb}{0.500000,0.500000,0.500000}%
\pgfsetfillcolor{currentfill}%
\pgfsetfillopacity{0.300000}%
\pgfsetlinewidth{0.301125pt}%
\definecolor{currentstroke}{rgb}{0.500000,0.500000,0.500000}%
\pgfsetstrokecolor{currentstroke}%
\pgfsetstrokeopacity{0.300000}%
\pgfsetdash{}{0pt}%
\pgfpathmoveto{\pgfqpoint{0.000000in}{0.000000in}}%
\pgfpathlineto{\pgfqpoint{0.000000in}{0.000000in}}%
\pgfpathclose%
\pgfusepath{stroke,fill}%
\end{pgfscope}%
\begin{pgfscope}%
\pgfpathrectangle{\pgfqpoint{0.647939in}{0.492442in}}{\pgfqpoint{3.079299in}{3.079299in}}%
\pgfusepath{clip}%
\pgfsetroundcap%
\pgfsetroundjoin%
\pgfsetlinewidth{0.301125pt}%
\definecolor{currentstroke}{rgb}{0.500000,0.500000,0.500000}%
\pgfsetstrokecolor{currentstroke}%
\pgfsetstrokeopacity{0.300000}%
\pgfsetdash{}{0pt}%
\pgfpathmoveto{\pgfqpoint{1.837630in}{2.080149in}}%
\pgfusepath{stroke}%
\end{pgfscope}%
\begin{pgfscope}%
\pgfpathrectangle{\pgfqpoint{0.647939in}{0.492442in}}{\pgfqpoint{3.079299in}{3.079299in}}%
\pgfusepath{clip}%
\pgfsetroundcap%
\pgfsetroundjoin%
\definecolor{currentfill}{rgb}{0.500000,0.500000,0.500000}%
\pgfsetfillcolor{currentfill}%
\pgfsetfillopacity{0.300000}%
\pgfsetlinewidth{0.301125pt}%
\definecolor{currentstroke}{rgb}{0.500000,0.500000,0.500000}%
\pgfsetstrokecolor{currentstroke}%
\pgfsetstrokeopacity{0.300000}%
\pgfsetdash{}{0pt}%
\pgfpathmoveto{\pgfqpoint{0.000000in}{0.000000in}}%
\pgfpathlineto{\pgfqpoint{0.000000in}{0.000000in}}%
\pgfpathclose%
\pgfusepath{stroke,fill}%
\end{pgfscope}%
\begin{pgfscope}%
\pgfpathrectangle{\pgfqpoint{0.647939in}{0.492442in}}{\pgfqpoint{3.079299in}{3.079299in}}%
\pgfusepath{clip}%
\pgfsetroundcap%
\pgfsetroundjoin%
\pgfsetlinewidth{0.301125pt}%
\definecolor{currentstroke}{rgb}{0.500000,0.500000,0.500000}%
\pgfsetstrokecolor{currentstroke}%
\pgfsetstrokeopacity{0.300000}%
\pgfsetdash{}{0pt}%
\pgfpathmoveto{\pgfqpoint{2.565137in}{2.196767in}}%
\pgfusepath{stroke}%
\end{pgfscope}%
\begin{pgfscope}%
\pgfpathrectangle{\pgfqpoint{0.647939in}{0.492442in}}{\pgfqpoint{3.079299in}{3.079299in}}%
\pgfusepath{clip}%
\pgfsetroundcap%
\pgfsetroundjoin%
\definecolor{currentfill}{rgb}{0.500000,0.500000,0.500000}%
\pgfsetfillcolor{currentfill}%
\pgfsetfillopacity{0.300000}%
\pgfsetlinewidth{0.301125pt}%
\definecolor{currentstroke}{rgb}{0.500000,0.500000,0.500000}%
\pgfsetstrokecolor{currentstroke}%
\pgfsetstrokeopacity{0.300000}%
\pgfsetdash{}{0pt}%
\pgfpathmoveto{\pgfqpoint{0.000000in}{0.000000in}}%
\pgfpathlineto{\pgfqpoint{0.000000in}{0.000000in}}%
\pgfpathclose%
\pgfusepath{stroke,fill}%
\end{pgfscope}%
\begin{pgfscope}%
\pgfpathrectangle{\pgfqpoint{0.647939in}{0.492442in}}{\pgfqpoint{3.079299in}{3.079299in}}%
\pgfusepath{clip}%
\pgfsetroundcap%
\pgfsetroundjoin%
\pgfsetlinewidth{0.301125pt}%
\definecolor{currentstroke}{rgb}{0.500000,0.500000,0.500000}%
\pgfsetstrokecolor{currentstroke}%
\pgfsetstrokeopacity{0.300000}%
\pgfsetdash{}{0pt}%
\pgfpathmoveto{\pgfqpoint{2.093535in}{2.597158in}}%
\pgfusepath{stroke}%
\end{pgfscope}%
\begin{pgfscope}%
\pgfpathrectangle{\pgfqpoint{0.647939in}{0.492442in}}{\pgfqpoint{3.079299in}{3.079299in}}%
\pgfusepath{clip}%
\pgfsetroundcap%
\pgfsetroundjoin%
\definecolor{currentfill}{rgb}{0.500000,0.500000,0.500000}%
\pgfsetfillcolor{currentfill}%
\pgfsetfillopacity{0.300000}%
\pgfsetlinewidth{0.301125pt}%
\definecolor{currentstroke}{rgb}{0.500000,0.500000,0.500000}%
\pgfsetstrokecolor{currentstroke}%
\pgfsetstrokeopacity{0.300000}%
\pgfsetdash{}{0pt}%
\pgfpathmoveto{\pgfqpoint{0.000000in}{0.000000in}}%
\pgfpathlineto{\pgfqpoint{0.000000in}{0.000000in}}%
\pgfpathclose%
\pgfusepath{stroke,fill}%
\end{pgfscope}%
\begin{pgfscope}%
\pgfpathrectangle{\pgfqpoint{0.647939in}{0.492442in}}{\pgfqpoint{3.079299in}{3.079299in}}%
\pgfusepath{clip}%
\pgfsetroundcap%
\pgfsetroundjoin%
\pgfsetlinewidth{0.301125pt}%
\definecolor{currentstroke}{rgb}{0.500000,0.500000,0.500000}%
\pgfsetstrokecolor{currentstroke}%
\pgfsetstrokeopacity{0.300000}%
\pgfsetdash{}{0pt}%
\pgfpathmoveto{\pgfqpoint{1.992694in}{2.362271in}}%
\pgfusepath{stroke}%
\end{pgfscope}%
\begin{pgfscope}%
\pgfpathrectangle{\pgfqpoint{0.647939in}{0.492442in}}{\pgfqpoint{3.079299in}{3.079299in}}%
\pgfusepath{clip}%
\pgfsetroundcap%
\pgfsetroundjoin%
\definecolor{currentfill}{rgb}{0.500000,0.500000,0.500000}%
\pgfsetfillcolor{currentfill}%
\pgfsetfillopacity{0.300000}%
\pgfsetlinewidth{0.301125pt}%
\definecolor{currentstroke}{rgb}{0.500000,0.500000,0.500000}%
\pgfsetstrokecolor{currentstroke}%
\pgfsetstrokeopacity{0.300000}%
\pgfsetdash{}{0pt}%
\pgfpathmoveto{\pgfqpoint{0.000000in}{0.000000in}}%
\pgfpathlineto{\pgfqpoint{0.000000in}{0.000000in}}%
\pgfpathclose%
\pgfusepath{stroke,fill}%
\end{pgfscope}%
\begin{pgfscope}%
\pgfpathrectangle{\pgfqpoint{0.647939in}{0.492442in}}{\pgfqpoint{3.079299in}{3.079299in}}%
\pgfusepath{clip}%
\pgfsetroundcap%
\pgfsetroundjoin%
\pgfsetlinewidth{0.301125pt}%
\definecolor{currentstroke}{rgb}{0.500000,0.500000,0.500000}%
\pgfsetstrokecolor{currentstroke}%
\pgfsetstrokeopacity{0.300000}%
\pgfsetdash{}{0pt}%
\pgfpathmoveto{\pgfqpoint{2.084225in}{1.543590in}}%
\pgfusepath{stroke}%
\end{pgfscope}%
\begin{pgfscope}%
\pgfpathrectangle{\pgfqpoint{0.647939in}{0.492442in}}{\pgfqpoint{3.079299in}{3.079299in}}%
\pgfusepath{clip}%
\pgfsetroundcap%
\pgfsetroundjoin%
\definecolor{currentfill}{rgb}{0.500000,0.500000,0.500000}%
\pgfsetfillcolor{currentfill}%
\pgfsetfillopacity{0.300000}%
\pgfsetlinewidth{0.301125pt}%
\definecolor{currentstroke}{rgb}{0.500000,0.500000,0.500000}%
\pgfsetstrokecolor{currentstroke}%
\pgfsetstrokeopacity{0.300000}%
\pgfsetdash{}{0pt}%
\pgfpathmoveto{\pgfqpoint{0.000000in}{0.000000in}}%
\pgfpathlineto{\pgfqpoint{0.000000in}{0.000000in}}%
\pgfpathclose%
\pgfusepath{stroke,fill}%
\end{pgfscope}%
\begin{pgfscope}%
\pgfpathrectangle{\pgfqpoint{0.647939in}{0.492442in}}{\pgfqpoint{3.079299in}{3.079299in}}%
\pgfusepath{clip}%
\pgfsetroundcap%
\pgfsetroundjoin%
\pgfsetlinewidth{0.301125pt}%
\definecolor{currentstroke}{rgb}{0.500000,0.500000,0.500000}%
\pgfsetstrokecolor{currentstroke}%
\pgfsetstrokeopacity{0.300000}%
\pgfsetdash{}{0pt}%
\pgfpathmoveto{\pgfqpoint{2.526939in}{2.367908in}}%
\pgfusepath{stroke}%
\end{pgfscope}%
\begin{pgfscope}%
\pgfpathrectangle{\pgfqpoint{0.647939in}{0.492442in}}{\pgfqpoint{3.079299in}{3.079299in}}%
\pgfusepath{clip}%
\pgfsetroundcap%
\pgfsetroundjoin%
\definecolor{currentfill}{rgb}{0.500000,0.500000,0.500000}%
\pgfsetfillcolor{currentfill}%
\pgfsetfillopacity{0.300000}%
\pgfsetlinewidth{0.301125pt}%
\definecolor{currentstroke}{rgb}{0.500000,0.500000,0.500000}%
\pgfsetstrokecolor{currentstroke}%
\pgfsetstrokeopacity{0.300000}%
\pgfsetdash{}{0pt}%
\pgfpathmoveto{\pgfqpoint{0.000000in}{0.000000in}}%
\pgfpathlineto{\pgfqpoint{0.000000in}{0.000000in}}%
\pgfpathclose%
\pgfusepath{stroke,fill}%
\end{pgfscope}%
\begin{pgfscope}%
\pgfpathrectangle{\pgfqpoint{0.647939in}{0.492442in}}{\pgfqpoint{3.079299in}{3.079299in}}%
\pgfusepath{clip}%
\pgfsetroundcap%
\pgfsetroundjoin%
\pgfsetlinewidth{0.301125pt}%
\definecolor{currentstroke}{rgb}{0.500000,0.500000,0.500000}%
\pgfsetstrokecolor{currentstroke}%
\pgfsetstrokeopacity{0.300000}%
\pgfsetdash{}{0pt}%
\pgfpathmoveto{\pgfqpoint{2.513499in}{1.912109in}}%
\pgfusepath{stroke}%
\end{pgfscope}%
\begin{pgfscope}%
\pgfpathrectangle{\pgfqpoint{0.647939in}{0.492442in}}{\pgfqpoint{3.079299in}{3.079299in}}%
\pgfusepath{clip}%
\pgfsetroundcap%
\pgfsetroundjoin%
\definecolor{currentfill}{rgb}{0.500000,0.500000,0.500000}%
\pgfsetfillcolor{currentfill}%
\pgfsetfillopacity{0.300000}%
\pgfsetlinewidth{0.301125pt}%
\definecolor{currentstroke}{rgb}{0.500000,0.500000,0.500000}%
\pgfsetstrokecolor{currentstroke}%
\pgfsetstrokeopacity{0.300000}%
\pgfsetdash{}{0pt}%
\pgfpathmoveto{\pgfqpoint{0.000000in}{0.000000in}}%
\pgfpathlineto{\pgfqpoint{0.000000in}{0.000000in}}%
\pgfpathclose%
\pgfusepath{stroke,fill}%
\end{pgfscope}%
\begin{pgfscope}%
\pgfpathrectangle{\pgfqpoint{0.647939in}{0.492442in}}{\pgfqpoint{3.079299in}{3.079299in}}%
\pgfusepath{clip}%
\pgfsetroundcap%
\pgfsetroundjoin%
\pgfsetlinewidth{0.301125pt}%
\definecolor{currentstroke}{rgb}{0.500000,0.500000,0.500000}%
\pgfsetstrokecolor{currentstroke}%
\pgfsetstrokeopacity{0.300000}%
\pgfsetdash{}{0pt}%
\pgfpathmoveto{\pgfqpoint{2.442875in}{2.047442in}}%
\pgfusepath{stroke}%
\end{pgfscope}%
\begin{pgfscope}%
\pgfpathrectangle{\pgfqpoint{0.647939in}{0.492442in}}{\pgfqpoint{3.079299in}{3.079299in}}%
\pgfusepath{clip}%
\pgfsetroundcap%
\pgfsetroundjoin%
\definecolor{currentfill}{rgb}{0.500000,0.500000,0.500000}%
\pgfsetfillcolor{currentfill}%
\pgfsetfillopacity{0.300000}%
\pgfsetlinewidth{0.301125pt}%
\definecolor{currentstroke}{rgb}{0.500000,0.500000,0.500000}%
\pgfsetstrokecolor{currentstroke}%
\pgfsetstrokeopacity{0.300000}%
\pgfsetdash{}{0pt}%
\pgfpathmoveto{\pgfqpoint{0.000000in}{0.000000in}}%
\pgfpathlineto{\pgfqpoint{0.000000in}{0.000000in}}%
\pgfpathclose%
\pgfusepath{stroke,fill}%
\end{pgfscope}%
\begin{pgfscope}%
\pgfpathrectangle{\pgfqpoint{0.647939in}{0.492442in}}{\pgfqpoint{3.079299in}{3.079299in}}%
\pgfusepath{clip}%
\pgfsetroundcap%
\pgfsetroundjoin%
\pgfsetlinewidth{0.301125pt}%
\definecolor{currentstroke}{rgb}{0.500000,0.500000,0.500000}%
\pgfsetstrokecolor{currentstroke}%
\pgfsetstrokeopacity{0.300000}%
\pgfsetdash{}{0pt}%
\pgfpathmoveto{\pgfqpoint{2.001446in}{2.181962in}}%
\pgfusepath{stroke}%
\end{pgfscope}%
\begin{pgfscope}%
\pgfpathrectangle{\pgfqpoint{0.647939in}{0.492442in}}{\pgfqpoint{3.079299in}{3.079299in}}%
\pgfusepath{clip}%
\pgfsetroundcap%
\pgfsetroundjoin%
\definecolor{currentfill}{rgb}{0.500000,0.500000,0.500000}%
\pgfsetfillcolor{currentfill}%
\pgfsetfillopacity{0.300000}%
\pgfsetlinewidth{0.301125pt}%
\definecolor{currentstroke}{rgb}{0.500000,0.500000,0.500000}%
\pgfsetstrokecolor{currentstroke}%
\pgfsetstrokeopacity{0.300000}%
\pgfsetdash{}{0pt}%
\pgfpathmoveto{\pgfqpoint{0.000000in}{0.000000in}}%
\pgfpathlineto{\pgfqpoint{0.000000in}{0.000000in}}%
\pgfpathclose%
\pgfusepath{stroke,fill}%
\end{pgfscope}%
\begin{pgfscope}%
\pgfpathrectangle{\pgfqpoint{0.647939in}{0.492442in}}{\pgfqpoint{3.079299in}{3.079299in}}%
\pgfusepath{clip}%
\pgfsetbuttcap%
\pgfsetroundjoin%
\pgfsetlinewidth{0.301125pt}%
\definecolor{currentstroke}{rgb}{0.500000,0.500000,0.500000}%
\pgfsetstrokecolor{currentstroke}%
\pgfsetstrokeopacity{0.300000}%
\pgfsetdash{}{0pt}%
\pgfpathmoveto{\pgfqpoint{0.647939in}{0.492442in}}%
\pgfpathlineto{\pgfqpoint{0.647939in}{0.492442in}}%
\pgfpathlineto{\pgfqpoint{0.715017in}{0.505884in}}%
\pgfpathlineto{\pgfqpoint{0.781393in}{0.522417in}}%
\pgfpathlineto{\pgfqpoint{0.846791in}{0.542442in}}%
\pgfpathlineto{\pgfqpoint{0.910865in}{0.566335in}}%
\pgfpathlineto{\pgfqpoint{0.973206in}{0.594416in}}%
\pgfpathlineto{\pgfqpoint{1.033352in}{0.626907in}}%
\pgfpathlineto{\pgfqpoint{1.090836in}{0.663888in}}%
\pgfpathlineto{\pgfqpoint{1.145236in}{0.705278in}}%
\pgfpathlineto{\pgfqpoint{1.196252in}{0.750767in}}%
\pgfpathlineto{\pgfqpoint{1.243760in}{0.799891in}}%
\pgfpathlineto{\pgfqpoint{1.287847in}{0.852109in}}%
\pgfpathlineto{\pgfqpoint{1.328798in}{0.906838in}}%
\pgfpathlineto{\pgfqpoint{1.367050in}{0.963500in}}%
\pgfpathlineto{\pgfqpoint{1.403122in}{1.021567in}}%
\pgfpathlineto{\pgfqpoint{1.437554in}{1.080598in}}%
\pgfpathlineto{\pgfqpoint{1.470872in}{1.140247in}}%
\pgfpathlineto{\pgfqpoint{1.503566in}{1.200227in}}%
\pgfpathlineto{\pgfqpoint{1.536076in}{1.260296in}}%
\pgfpathlineto{\pgfqpoint{1.568798in}{1.320238in}}%
\pgfpathlineto{\pgfqpoint{1.602092in}{1.379858in}}%
\pgfpathlineto{\pgfqpoint{1.636290in}{1.438965in}}%
\pgfpathlineto{\pgfqpoint{1.671714in}{1.497370in}}%
\pgfpathlineto{\pgfqpoint{1.708688in}{1.554833in}}%
\pgfpathlineto{\pgfqpoint{1.747531in}{1.611012in}}%
\pgfpathlineto{\pgfqpoint{1.788547in}{1.665508in}}%
\pgfpathlineto{\pgfqpoint{1.832122in}{1.717912in}}%
\pgfpathlineto{\pgfqpoint{1.878727in}{1.767552in}}%
\pgfpathlineto{\pgfqpoint{1.928666in}{1.813130in}}%
\pgfpathlineto{\pgfqpoint{1.981112in}{1.850658in}}%
\pgfpathlineto{\pgfqpoint{1.981112in}{1.850658in}}%
\pgfpathlineto{\pgfqpoint{1.993226in}{1.860176in}}%
\pgfpathlineto{\pgfqpoint{1.999705in}{1.864905in}}%
\pgfpathlineto{\pgfqpoint{2.002762in}{1.867028in}}%
\pgfpathlineto{\pgfqpoint{2.004102in}{1.867736in}}%
\pgfpathlineto{\pgfqpoint{2.005239in}{1.868409in}}%
\pgfpathlineto{\pgfqpoint{2.006137in}{1.869142in}}%
\pgfpathlineto{\pgfqpoint{2.006287in}{1.869498in}}%
\pgfpathlineto{\pgfqpoint{2.005638in}{1.869287in}}%
\pgfpathlineto{\pgfqpoint{2.004615in}{1.868657in}}%
\pgfpathlineto{\pgfqpoint{2.003864in}{1.868016in}}%
\pgfpathlineto{\pgfqpoint{2.003836in}{1.867758in}}%
\pgfpathlineto{\pgfqpoint{2.004515in}{1.868028in}}%
\pgfpathlineto{\pgfqpoint{2.005430in}{1.868621in}}%
\pgfpathlineto{\pgfqpoint{2.006011in}{1.869153in}}%
\pgfpathlineto{\pgfqpoint{2.005939in}{1.869316in}}%
\pgfpathlineto{\pgfqpoint{2.005289in}{1.869030in}}%
\pgfpathlineto{\pgfqpoint{2.004483in}{1.868489in}}%
\pgfpathlineto{\pgfqpoint{2.004023in}{1.868037in}}%
\pgfpathlineto{\pgfqpoint{2.004176in}{1.867950in}}%
\pgfpathlineto{\pgfqpoint{2.004813in}{1.868260in}}%
\pgfpathlineto{\pgfqpoint{2.005508in}{1.868749in}}%
\pgfpathlineto{\pgfqpoint{2.005840in}{1.869110in}}%
\pgfpathlineto{\pgfqpoint{2.005635in}{1.869137in}}%
\pgfpathlineto{\pgfqpoint{2.005045in}{1.868831in}}%
\pgfpathlineto{\pgfqpoint{2.004447in}{1.868394in}}%
\pgfpathlineto{\pgfqpoint{2.004209in}{1.868102in}}%
\pgfpathlineto{\pgfqpoint{2.004457in}{1.868125in}}%
\pgfpathlineto{\pgfqpoint{2.005010in}{1.868431in}}%
\pgfpathlineto{\pgfqpoint{2.005508in}{1.868813in}}%
\pgfpathlineto{\pgfqpoint{2.005658in}{1.869036in}}%
\pgfpathlineto{\pgfqpoint{2.005391in}{1.868978in}}%
\pgfpathlineto{\pgfqpoint{2.004891in}{1.868688in}}%
\pgfpathlineto{\pgfqpoint{2.004474in}{1.868355in}}%
\pgfpathlineto{\pgfqpoint{2.004392in}{1.868185in}}%
\pgfpathlineto{\pgfqpoint{2.004675in}{1.868274in}}%
\pgfpathlineto{\pgfqpoint{2.005127in}{1.868550in}}%
\pgfpathlineto{\pgfqpoint{2.005462in}{1.868832in}}%
\pgfpathlineto{\pgfqpoint{2.005487in}{1.868952in}}%
\pgfpathlineto{\pgfqpoint{2.005205in}{1.868846in}}%
\pgfpathlineto{\pgfqpoint{2.004805in}{1.868592in}}%
\pgfpathlineto{\pgfqpoint{2.004536in}{1.868352in}}%
\pgfpathlineto{\pgfqpoint{2.004555in}{1.868272in}}%
\pgfpathlineto{\pgfqpoint{2.004834in}{1.868392in}}%
\pgfpathlineto{\pgfqpoint{2.005185in}{1.868625in}}%
\pgfpathlineto{\pgfqpoint{2.005391in}{1.868823in}}%
\pgfpathlineto{\pgfqpoint{2.005338in}{1.868869in}}%
\pgfpathlineto{\pgfqpoint{2.005073in}{1.868744in}}%
\pgfpathlineto{\pgfqpoint{2.004769in}{1.868534in}}%
\pgfpathlineto{\pgfqpoint{2.004613in}{1.868371in}}%
\pgfpathlineto{\pgfqpoint{2.004692in}{1.868352in}}%
\pgfpathlineto{\pgfqpoint{2.004943in}{1.868480in}}%
\pgfpathlineto{\pgfqpoint{2.005202in}{1.868667in}}%
\pgfpathlineto{\pgfqpoint{2.005313in}{1.868797in}}%
\pgfpathlineto{\pgfqpoint{2.005217in}{1.868794in}}%
\pgfpathlineto{\pgfqpoint{2.004986in}{1.868669in}}%
\pgfpathlineto{\pgfqpoint{2.004766in}{1.868504in}}%
\pgfpathlineto{\pgfqpoint{2.004692in}{1.868403in}}%
\pgfpathlineto{\pgfqpoint{2.004799in}{1.868422in}}%
\pgfpathlineto{\pgfqpoint{2.005012in}{1.868543in}}%
\pgfpathlineto{\pgfqpoint{2.005194in}{1.868686in}}%
\pgfpathlineto{\pgfqpoint{2.005237in}{1.868763in}}%
\pgfpathlineto{\pgfqpoint{2.005124in}{1.868732in}}%
\pgfusepath{stroke}%
\end{pgfscope}%
\begin{pgfscope}%
\pgfpathrectangle{\pgfqpoint{0.647939in}{0.492442in}}{\pgfqpoint{3.079299in}{3.079299in}}%
\pgfusepath{clip}%
\pgfsetbuttcap%
\pgfsetroundjoin%
\pgfsetlinewidth{0.301125pt}%
\definecolor{currentstroke}{rgb}{0.500000,0.500000,0.500000}%
\pgfsetstrokecolor{currentstroke}%
\pgfsetstrokeopacity{0.300000}%
\pgfsetdash{}{0pt}%
\pgfpathmoveto{\pgfqpoint{0.927875in}{0.492442in}}%
\pgfpathlineto{\pgfqpoint{0.927875in}{0.492442in}}%
\pgfpathlineto{\pgfqpoint{0.989039in}{0.523009in}}%
\pgfpathlineto{\pgfqpoint{1.047599in}{0.558281in}}%
\pgfpathlineto{\pgfqpoint{1.103037in}{0.598245in}}%
\pgfpathlineto{\pgfqpoint{1.154936in}{0.642682in}}%
\pgfpathlineto{\pgfqpoint{1.203067in}{0.691193in}}%
\pgfusepath{stroke}%
\end{pgfscope}%
\begin{pgfscope}%
\pgfpathrectangle{\pgfqpoint{0.647939in}{0.492442in}}{\pgfqpoint{3.079299in}{3.079299in}}%
\pgfusepath{clip}%
\pgfsetbuttcap%
\pgfsetroundjoin%
\pgfsetlinewidth{0.301125pt}%
\definecolor{currentstroke}{rgb}{0.500000,0.500000,0.500000}%
\pgfsetstrokecolor{currentstroke}%
\pgfsetstrokeopacity{0.300000}%
\pgfsetdash{}{0pt}%
\pgfpathmoveto{\pgfqpoint{1.137828in}{0.492442in}}%
\pgfpathlineto{\pgfqpoint{1.137828in}{0.492442in}}%
\pgfpathlineto{\pgfqpoint{1.182502in}{0.544108in}}%
\pgfpathlineto{\pgfqpoint{1.223046in}{0.599073in}}%
\pgfpathlineto{\pgfqpoint{1.259938in}{0.656554in}}%
\pgfpathlineto{\pgfqpoint{1.293814in}{0.715888in}}%
\pgfpathlineto{\pgfqpoint{1.325351in}{0.776515in}}%
\pgfpathlineto{\pgfqpoint{1.355187in}{0.837995in}}%
\pgfpathlineto{\pgfqpoint{1.383887in}{0.899999in}}%
\pgfpathlineto{\pgfqpoint{1.411928in}{0.962286in}}%
\pgfusepath{stroke}%
\end{pgfscope}%
\begin{pgfscope}%
\pgfpathrectangle{\pgfqpoint{0.647939in}{0.492442in}}{\pgfqpoint{3.079299in}{3.079299in}}%
\pgfusepath{clip}%
\pgfsetbuttcap%
\pgfsetroundjoin%
\pgfsetlinewidth{0.301125pt}%
\definecolor{currentstroke}{rgb}{0.500000,0.500000,0.500000}%
\pgfsetstrokecolor{currentstroke}%
\pgfsetstrokeopacity{0.300000}%
\pgfsetdash{}{0pt}%
\pgfpathmoveto{\pgfqpoint{1.417764in}{0.492442in}}%
\pgfpathlineto{\pgfqpoint{1.417764in}{0.492442in}}%
\pgfpathlineto{\pgfqpoint{1.395066in}{0.556157in}}%
\pgfpathlineto{\pgfqpoint{1.385923in}{0.613517in}}%
\pgfpathlineto{\pgfqpoint{1.384842in}{0.678418in}}%
\pgfpathlineto{\pgfqpoint{1.391086in}{0.746268in}}%
\pgfpathlineto{\pgfqpoint{1.402651in}{0.813536in}}%
\pgfusepath{stroke}%
\end{pgfscope}%
\begin{pgfscope}%
\pgfpathrectangle{\pgfqpoint{0.647939in}{0.492442in}}{\pgfqpoint{3.079299in}{3.079299in}}%
\pgfusepath{clip}%
\pgfsetbuttcap%
\pgfsetroundjoin%
\pgfsetlinewidth{0.301125pt}%
\definecolor{currentstroke}{rgb}{0.500000,0.500000,0.500000}%
\pgfsetstrokecolor{currentstroke}%
\pgfsetstrokeopacity{0.300000}%
\pgfsetdash{}{0pt}%
\pgfpathmoveto{\pgfqpoint{1.697700in}{0.492442in}}%
\pgfpathlineto{\pgfqpoint{1.697700in}{0.492442in}}%
\pgfpathlineto{\pgfqpoint{1.632720in}{0.513388in}}%
\pgfpathlineto{\pgfqpoint{1.571653in}{0.543573in}}%
\pgfpathlineto{\pgfqpoint{1.518057in}{0.585207in}}%
\pgfpathlineto{\pgfqpoint{1.479805in}{0.633318in}}%
\pgfpathlineto{\pgfqpoint{1.456081in}{0.683431in}}%
\pgfusepath{stroke}%
\end{pgfscope}%
\begin{pgfscope}%
\pgfpathrectangle{\pgfqpoint{0.647939in}{0.492442in}}{\pgfqpoint{3.079299in}{3.079299in}}%
\pgfusepath{clip}%
\pgfsetbuttcap%
\pgfsetroundjoin%
\pgfsetlinewidth{0.301125pt}%
\definecolor{currentstroke}{rgb}{0.500000,0.500000,0.500000}%
\pgfsetstrokecolor{currentstroke}%
\pgfsetstrokeopacity{0.300000}%
\pgfsetdash{}{0pt}%
\pgfpathmoveto{\pgfqpoint{2.569158in}{0.492442in}}%
\pgfpathlineto{\pgfqpoint{2.526479in}{0.494680in}}%
\pgfpathlineto{\pgfqpoint{2.458111in}{0.497393in}}%
\pgfpathlineto{\pgfqpoint{2.389695in}{0.498491in}}%
\pgfpathlineto{\pgfqpoint{2.321269in}{0.498274in}}%
\pgfpathlineto{\pgfqpoint{2.252851in}{0.497109in}}%
\pgfpathlineto{\pgfqpoint{2.184444in}{0.495417in}}%
\pgfpathlineto{\pgfqpoint{2.116037in}{0.493679in}}%
\pgfpathlineto{\pgfqpoint{2.047620in}{0.492442in}}%
\pgfpathlineto{\pgfqpoint{2.047620in}{0.492442in}}%
\pgfusepath{stroke}%
\end{pgfscope}%
\begin{pgfscope}%
\pgfpathrectangle{\pgfqpoint{0.647939in}{0.492442in}}{\pgfqpoint{3.079299in}{3.079299in}}%
\pgfusepath{clip}%
\pgfsetbuttcap%
\pgfsetroundjoin%
\pgfsetlinewidth{0.301125pt}%
\definecolor{currentstroke}{rgb}{0.500000,0.500000,0.500000}%
\pgfsetstrokecolor{currentstroke}%
\pgfsetstrokeopacity{0.300000}%
\pgfsetdash{}{0pt}%
\pgfpathmoveto{\pgfqpoint{2.887429in}{0.492442in}}%
\pgfpathlineto{\pgfqpoint{2.887429in}{0.492442in}}%
\pgfpathlineto{\pgfqpoint{2.820329in}{0.505824in}}%
\pgfpathlineto{\pgfqpoint{2.752876in}{0.517287in}}%
\pgfpathlineto{\pgfqpoint{2.685109in}{0.526717in}}%
\pgfpathlineto{\pgfqpoint{2.617086in}{0.534069in}}%
\pgfpathlineto{\pgfqpoint{2.548872in}{0.539374in}}%
\pgfpathlineto{\pgfqpoint{2.480533in}{0.542747in}}%
\pgfpathlineto{\pgfqpoint{2.412129in}{0.544389in}}%
\pgfpathlineto{\pgfqpoint{2.343704in}{0.544577in}}%
\pgfpathlineto{\pgfqpoint{2.275283in}{0.543656in}}%
\pgfpathlineto{\pgfqpoint{2.206874in}{0.542037in}}%
\pgfpathlineto{\pgfqpoint{2.138470in}{0.540196in}}%
\pgfpathlineto{\pgfqpoint{2.070058in}{0.538685in}}%
\pgfpathlineto{\pgfqpoint{2.001635in}{0.538125in}}%
\pgfpathlineto{\pgfqpoint{1.933225in}{0.539231in}}%
\pgfpathlineto{\pgfqpoint{1.864910in}{0.542855in}}%
\pgfpathlineto{\pgfqpoint{1.796896in}{0.550055in}}%
\pgfpathlineto{\pgfqpoint{1.729622in}{0.562193in}}%
\pgfpathlineto{\pgfqpoint{1.663995in}{0.581063in}}%
\pgfpathlineto{\pgfqpoint{1.601837in}{0.608980in}}%
\pgfpathlineto{\pgfqpoint{1.546605in}{0.648423in}}%
\pgfusepath{stroke}%
\end{pgfscope}%
\begin{pgfscope}%
\pgfpathrectangle{\pgfqpoint{0.647939in}{0.492442in}}{\pgfqpoint{3.079299in}{3.079299in}}%
\pgfusepath{clip}%
\pgfsetbuttcap%
\pgfsetroundjoin%
\pgfsetlinewidth{0.301125pt}%
\definecolor{currentstroke}{rgb}{0.500000,0.500000,0.500000}%
\pgfsetstrokecolor{currentstroke}%
\pgfsetstrokeopacity{0.300000}%
\pgfsetdash{}{0pt}%
\pgfpathmoveto{\pgfqpoint{3.097382in}{0.492442in}}%
\pgfpathlineto{\pgfqpoint{3.097382in}{0.492442in}}%
\pgfpathlineto{\pgfqpoint{3.031524in}{0.511012in}}%
\pgfpathlineto{\pgfqpoint{2.965325in}{0.528321in}}%
\pgfpathlineto{\pgfqpoint{2.898751in}{0.544114in}}%
\pgfpathlineto{\pgfqpoint{2.831788in}{0.558166in}}%
\pgfpathlineto{\pgfqpoint{2.764450in}{0.570289in}}%
\pgfpathlineto{\pgfqpoint{2.696774in}{0.580351in}}%
\pgfusepath{stroke}%
\end{pgfscope}%
\begin{pgfscope}%
\pgfpathrectangle{\pgfqpoint{0.647939in}{0.492442in}}{\pgfqpoint{3.079299in}{3.079299in}}%
\pgfusepath{clip}%
\pgfsetbuttcap%
\pgfsetroundjoin%
\pgfsetlinewidth{0.301125pt}%
\definecolor{currentstroke}{rgb}{0.500000,0.500000,0.500000}%
\pgfsetstrokecolor{currentstroke}%
\pgfsetstrokeopacity{0.300000}%
\pgfsetdash{}{0pt}%
\pgfpathmoveto{\pgfqpoint{3.377318in}{0.492442in}}%
\pgfpathlineto{\pgfqpoint{3.377318in}{0.492442in}}%
\pgfpathlineto{\pgfqpoint{3.312388in}{0.514043in}}%
\pgfpathlineto{\pgfqpoint{3.247441in}{0.535592in}}%
\pgfpathlineto{\pgfqpoint{3.182384in}{0.556806in}}%
\pgfpathlineto{\pgfqpoint{3.117129in}{0.577398in}}%
\pgfpathlineto{\pgfqpoint{3.051595in}{0.597081in}}%
\pgfpathlineto{\pgfqpoint{2.985716in}{0.615571in}}%
\pgfpathlineto{\pgfqpoint{2.919444in}{0.632594in}}%
\pgfpathlineto{\pgfqpoint{2.852756in}{0.647899in}}%
\pgfpathlineto{\pgfqpoint{2.785655in}{0.661269in}}%
\pgfpathlineto{\pgfqpoint{2.718171in}{0.672538in}}%
\pgfpathlineto{\pgfqpoint{2.650357in}{0.681606in}}%
\pgfpathlineto{\pgfqpoint{2.582282in}{0.688451in}}%
\pgfpathlineto{\pgfqpoint{2.514022in}{0.693136in}}%
\pgfpathlineto{\pgfqpoint{2.445653in}{0.695824in}}%
\pgfpathlineto{\pgfqpoint{2.377236in}{0.696768in}}%
\pgfpathlineto{\pgfqpoint{2.308812in}{0.696300in}}%
\pgfpathlineto{\pgfqpoint{2.240400in}{0.694828in}}%
\pgfpathlineto{\pgfqpoint{2.172001in}{0.692834in}}%
\pgfpathlineto{\pgfqpoint{2.103600in}{0.690889in}}%
\pgfpathlineto{\pgfqpoint{2.035185in}{0.689648in}}%
\pgfpathlineto{\pgfqpoint{1.966764in}{0.689869in}}%
\pgfpathlineto{\pgfqpoint{1.898402in}{0.692467in}}%
\pgfpathlineto{\pgfqpoint{1.830282in}{0.698595in}}%
\pgfpathlineto{\pgfqpoint{1.762847in}{0.709784in}}%
\pgfpathlineto{\pgfqpoint{1.697087in}{0.728118in}}%
\pgfpathlineto{\pgfqpoint{1.635145in}{0.756353in}}%
\pgfpathlineto{\pgfqpoint{1.581265in}{0.797379in}}%
\pgfpathlineto{\pgfqpoint{1.545833in}{0.843106in}}%
\pgfpathlineto{\pgfqpoint{1.524761in}{0.890650in}}%
\pgfpathlineto{\pgfqpoint{1.513788in}{0.941610in}}%
\pgfpathlineto{\pgfqpoint{1.511280in}{0.998220in}}%
\pgfpathlineto{\pgfqpoint{1.517206in}{1.062654in}}%
\pgfpathlineto{\pgfqpoint{1.530443in}{1.129530in}}%
\pgfpathlineto{\pgfqpoint{1.548777in}{1.195205in}}%
\pgfusepath{stroke}%
\end{pgfscope}%
\begin{pgfscope}%
\pgfpathrectangle{\pgfqpoint{0.647939in}{0.492442in}}{\pgfqpoint{3.079299in}{3.079299in}}%
\pgfusepath{clip}%
\pgfsetbuttcap%
\pgfsetroundjoin%
\pgfsetlinewidth{0.301125pt}%
\definecolor{currentstroke}{rgb}{0.500000,0.500000,0.500000}%
\pgfsetstrokecolor{currentstroke}%
\pgfsetstrokeopacity{0.300000}%
\pgfsetdash{}{0pt}%
\pgfpathmoveto{\pgfqpoint{3.587270in}{0.492442in}}%
\pgfpathlineto{\pgfqpoint{3.587270in}{0.492442in}}%
\pgfpathlineto{\pgfqpoint{3.521980in}{0.512921in}}%
\pgfpathlineto{\pgfqpoint{3.456957in}{0.534238in}}%
\pgfpathlineto{\pgfqpoint{3.392127in}{0.556135in}}%
\pgfpathlineto{\pgfqpoint{3.327402in}{0.578343in}}%
\pgfpathlineto{\pgfqpoint{3.262689in}{0.600583in}}%
\pgfpathlineto{\pgfqpoint{3.197889in}{0.622572in}}%
\pgfpathlineto{\pgfqpoint{3.132908in}{0.644014in}}%
\pgfpathlineto{\pgfqpoint{3.067656in}{0.664615in}}%
\pgfpathlineto{\pgfqpoint{3.002059in}{0.684081in}}%
\pgfpathlineto{\pgfqpoint{2.936059in}{0.702127in}}%
\pgfpathlineto{\pgfqpoint{2.869622in}{0.718484in}}%
\pgfpathlineto{\pgfqpoint{2.802742in}{0.732915in}}%
\pgfpathlineto{\pgfqpoint{2.735439in}{0.745223in}}%
\pgfpathlineto{\pgfqpoint{2.667764in}{0.755277in}}%
\pgfpathlineto{\pgfqpoint{2.599786in}{0.763023in}}%
\pgfpathlineto{\pgfqpoint{2.531587in}{0.768502in}}%
\pgfpathlineto{\pgfqpoint{2.463249in}{0.771844in}}%
\pgfpathlineto{\pgfqpoint{2.394842in}{0.773276in}}%
\pgfpathlineto{\pgfqpoint{2.326417in}{0.773119in}}%
\pgfpathlineto{\pgfqpoint{2.258003in}{0.771787in}}%
\pgfpathlineto{\pgfqpoint{2.189604in}{0.769774in}}%
\pgfpathlineto{\pgfqpoint{2.121208in}{0.767647in}}%
\pgfpathlineto{\pgfqpoint{2.052800in}{0.766065in}}%
\pgfpathlineto{\pgfqpoint{1.984377in}{0.765805in}}%
\pgfpathlineto{\pgfqpoint{1.915994in}{0.767828in}}%
\pgfpathlineto{\pgfqpoint{1.847827in}{0.773350in}}%
\pgfpathlineto{\pgfqpoint{1.780313in}{0.783980in}}%
\pgfpathlineto{\pgfqpoint{1.714467in}{0.801960in}}%
\pgfpathlineto{\pgfqpoint{1.652616in}{0.830331in}}%
\pgfpathlineto{\pgfqpoint{1.652616in}{0.830331in}}%
\pgfpathlineto{\pgfqpoint{1.607644in}{0.863749in}}%
\pgfusepath{stroke}%
\end{pgfscope}%
\begin{pgfscope}%
\pgfpathrectangle{\pgfqpoint{0.647939in}{0.492442in}}{\pgfqpoint{3.079299in}{3.079299in}}%
\pgfusepath{clip}%
\pgfsetbuttcap%
\pgfsetroundjoin%
\pgfsetlinewidth{0.301125pt}%
\definecolor{currentstroke}{rgb}{0.500000,0.500000,0.500000}%
\pgfsetstrokecolor{currentstroke}%
\pgfsetstrokeopacity{0.300000}%
\pgfsetdash{}{0pt}%
\pgfpathmoveto{\pgfqpoint{3.727238in}{0.562426in}}%
\pgfpathlineto{\pgfqpoint{3.727238in}{0.562426in}}%
\pgfpathlineto{\pgfqpoint{3.661434in}{0.581178in}}%
\pgfpathlineto{\pgfqpoint{3.596044in}{0.601333in}}%
\pgfpathlineto{\pgfqpoint{3.531027in}{0.622661in}}%
\pgfpathlineto{\pgfqpoint{3.466320in}{0.644916in}}%
\pgfpathlineto{\pgfqpoint{3.401844in}{0.667835in}}%
\pgfpathlineto{\pgfqpoint{3.337509in}{0.691148in}}%
\pgfpathlineto{\pgfqpoint{3.273215in}{0.714574in}}%
\pgfpathlineto{\pgfqpoint{3.208857in}{0.737822in}}%
\pgfpathlineto{\pgfqpoint{3.144331in}{0.760598in}}%
\pgfpathlineto{\pgfqpoint{3.079537in}{0.782596in}}%
\pgfpathlineto{\pgfqpoint{3.014387in}{0.803513in}}%
\pgfpathlineto{\pgfqpoint{2.948809in}{0.823042in}}%
\pgfpathlineto{\pgfqpoint{2.882756in}{0.840888in}}%
\pgfpathlineto{\pgfqpoint{2.816208in}{0.856780in}}%
\pgfpathlineto{\pgfqpoint{2.749179in}{0.870489in}}%
\pgfpathlineto{\pgfqpoint{2.681712in}{0.881842in}}%
\pgfpathlineto{\pgfqpoint{2.613878in}{0.890743in}}%
\pgfpathlineto{\pgfqpoint{2.545765in}{0.897191in}}%
\pgfpathlineto{\pgfqpoint{2.477469in}{0.901293in}}%
\pgfpathlineto{\pgfqpoint{2.409077in}{0.903266in}}%
\pgfpathlineto{\pgfqpoint{2.340654in}{0.903429in}}%
\pgfpathlineto{\pgfqpoint{2.272239in}{0.902195in}}%
\pgfpathlineto{\pgfqpoint{2.203844in}{0.900068in}}%
\pgfpathlineto{\pgfqpoint{2.135458in}{0.897662in}}%
\pgfpathlineto{\pgfqpoint{2.067059in}{0.895697in}}%
\pgfpathlineto{\pgfqpoint{1.998640in}{0.895026in}}%
\pgfpathlineto{\pgfqpoint{1.930250in}{0.896707in}}%
\pgfpathlineto{\pgfqpoint{1.862077in}{0.902129in}}%
\pgfpathlineto{\pgfqpoint{1.794654in}{0.913242in}}%
\pgfpathlineto{\pgfqpoint{1.729352in}{0.932870in}}%
\pgfpathlineto{\pgfqpoint{1.669524in}{0.964837in}}%
\pgfpathlineto{\pgfqpoint{1.669524in}{0.964837in}}%
\pgfpathlineto{\pgfqpoint{1.631798in}{0.998737in}}%
\pgfpathlineto{\pgfqpoint{1.603853in}{1.041583in}}%
\pgfpathlineto{\pgfqpoint{1.588272in}{1.086889in}}%
\pgfusepath{stroke}%
\end{pgfscope}%
\begin{pgfscope}%
\pgfpathrectangle{\pgfqpoint{0.647939in}{0.492442in}}{\pgfqpoint{3.079299in}{3.079299in}}%
\pgfusepath{clip}%
\pgfsetbuttcap%
\pgfsetroundjoin%
\pgfsetlinewidth{0.301125pt}%
\definecolor{currentstroke}{rgb}{0.500000,0.500000,0.500000}%
\pgfsetstrokecolor{currentstroke}%
\pgfsetstrokeopacity{0.300000}%
\pgfsetdash{}{0pt}%
\pgfpathmoveto{\pgfqpoint{3.727238in}{0.632410in}}%
\pgfpathlineto{\pgfqpoint{3.727238in}{0.632410in}}%
\pgfpathlineto{\pgfqpoint{3.661602in}{0.651741in}}%
\pgfpathlineto{\pgfqpoint{3.596408in}{0.672520in}}%
\pgfpathlineto{\pgfqpoint{3.531614in}{0.694517in}}%
\pgfpathlineto{\pgfqpoint{3.467155in}{0.717480in}}%
\pgfpathlineto{\pgfqpoint{3.402950in}{0.741146in}}%
\pgfpathlineto{\pgfqpoint{3.338905in}{0.765243in}}%
\pgfpathlineto{\pgfqpoint{3.274914in}{0.789486in}}%
\pgfpathlineto{\pgfqpoint{3.210870in}{0.813584in}}%
\pgfpathlineto{\pgfqpoint{3.146660in}{0.837237in}}%
\pgfpathlineto{\pgfqpoint{3.082179in}{0.860137in}}%
\pgfpathlineto{\pgfqpoint{3.017331in}{0.881969in}}%
\pgfpathlineto{\pgfqpoint{2.952035in}{0.902420in}}%
\pgfpathlineto{\pgfqpoint{2.886235in}{0.921179in}}%
\pgfpathlineto{\pgfqpoint{2.819904in}{0.937955in}}%
\pgfpathlineto{\pgfqpoint{2.753051in}{0.952497in}}%
\pgfpathlineto{\pgfqpoint{2.685717in}{0.964607in}}%
\pgfpathlineto{\pgfqpoint{2.617974in}{0.974167in}}%
\pgfpathlineto{\pgfqpoint{2.549917in}{0.981153in}}%
\pgfpathlineto{\pgfqpoint{2.481647in}{0.985653in}}%
\pgfpathlineto{\pgfqpoint{2.413264in}{0.987880in}}%
\pgfpathlineto{\pgfqpoint{2.344842in}{0.988158in}}%
\pgfpathlineto{\pgfqpoint{2.276428in}{0.986914in}}%
\pgfpathlineto{\pgfqpoint{2.208037in}{0.984676in}}%
\pgfpathlineto{\pgfqpoint{2.139658in}{0.982079in}}%
\pgfpathlineto{\pgfqpoint{2.071265in}{0.979891in}}%
\pgfpathlineto{\pgfqpoint{2.002849in}{0.979043in}}%
\pgfpathlineto{\pgfqpoint{1.934463in}{0.980703in}}%
\pgfpathlineto{\pgfqpoint{1.866328in}{0.986449in}}%
\pgfpathlineto{\pgfqpoint{1.799100in}{0.998567in}}%
\pgfpathlineto{\pgfqpoint{1.734614in}{1.020499in}}%
\pgfpathlineto{\pgfqpoint{1.734614in}{1.020499in}}%
\pgfpathlineto{\pgfqpoint{1.687099in}{1.048534in}}%
\pgfpathlineto{\pgfqpoint{1.648114in}{1.088817in}}%
\pgfpathlineto{\pgfqpoint{1.625778in}{1.131280in}}%
\pgfpathlineto{\pgfqpoint{1.614647in}{1.175963in}}%
\pgfpathlineto{\pgfqpoint{1.612176in}{1.225552in}}%
\pgfpathlineto{\pgfqpoint{1.618161in}{1.281630in}}%
\pgfpathlineto{\pgfqpoint{1.633452in}{1.345895in}}%
\pgfusepath{stroke}%
\end{pgfscope}%
\begin{pgfscope}%
\pgfpathrectangle{\pgfqpoint{0.647939in}{0.492442in}}{\pgfqpoint{3.079299in}{3.079299in}}%
\pgfusepath{clip}%
\pgfsetbuttcap%
\pgfsetroundjoin%
\pgfsetlinewidth{0.301125pt}%
\definecolor{currentstroke}{rgb}{0.500000,0.500000,0.500000}%
\pgfsetstrokecolor{currentstroke}%
\pgfsetstrokeopacity{0.300000}%
\pgfsetdash{}{0pt}%
\pgfpathmoveto{\pgfqpoint{3.727238in}{0.702394in}}%
\pgfpathlineto{\pgfqpoint{3.727238in}{0.702394in}}%
\pgfpathlineto{\pgfqpoint{3.661786in}{0.722338in}}%
\pgfpathlineto{\pgfqpoint{3.596808in}{0.743781in}}%
\pgfpathlineto{\pgfqpoint{3.532259in}{0.766487in}}%
\pgfpathlineto{\pgfqpoint{3.468074in}{0.790203in}}%
\pgfpathlineto{\pgfqpoint{3.404168in}{0.814666in}}%
\pgfpathlineto{\pgfqpoint{3.340443in}{0.839598in}}%
\pgfpathlineto{\pgfqpoint{3.276792in}{0.864717in}}%
\pgfpathlineto{\pgfqpoint{3.213098in}{0.889729in}}%
\pgfpathlineto{\pgfqpoint{3.149246in}{0.914329in}}%
\pgfpathlineto{\pgfqpoint{3.085121in}{0.938207in}}%
\pgfpathlineto{\pgfqpoint{3.020618in}{0.961042in}}%
\pgfpathlineto{\pgfqpoint{2.955649in}{0.982507in}}%
\pgfpathlineto{\pgfqpoint{2.890147in}{1.002279in}}%
\pgfpathlineto{\pgfqpoint{2.824076in}{1.020047in}}%
\pgfpathlineto{\pgfqpoint{2.757434in}{1.035533in}}%
\pgfpathlineto{\pgfqpoint{2.690263in}{1.048512in}}%
\pgfpathlineto{\pgfqpoint{2.622634in}{1.058835in}}%
\pgfpathlineto{\pgfqpoint{2.554647in}{1.066455in}}%
\pgfpathlineto{\pgfqpoint{2.486413in}{1.071436in}}%
\pgfpathlineto{\pgfqpoint{2.418042in}{1.073976in}}%
\pgfpathlineto{\pgfqpoint{2.349622in}{1.074401in}}%
\pgfpathlineto{\pgfqpoint{2.281209in}{1.073157in}}%
\pgfpathlineto{\pgfqpoint{2.212822in}{1.070794in}}%
\pgfpathlineto{\pgfqpoint{2.144451in}{1.067980in}}%
\pgfpathlineto{\pgfqpoint{2.076068in}{1.065528in}}%
\pgfpathlineto{\pgfqpoint{2.007655in}{1.064451in}}%
\pgfpathlineto{\pgfqpoint{1.939272in}{1.066059in}}%
\pgfpathlineto{\pgfqpoint{1.871185in}{1.072172in}}%
\pgfpathlineto{\pgfqpoint{1.804249in}{1.085541in}}%
\pgfpathlineto{\pgfqpoint{1.741003in}{1.110451in}}%
\pgfpathlineto{\pgfqpoint{1.741003in}{1.110451in}}%
\pgfpathlineto{\pgfqpoint{1.700193in}{1.138989in}}%
\pgfpathlineto{\pgfqpoint{1.668482in}{1.178463in}}%
\pgfusepath{stroke}%
\end{pgfscope}%
\begin{pgfscope}%
\pgfpathrectangle{\pgfqpoint{0.647939in}{0.492442in}}{\pgfqpoint{3.079299in}{3.079299in}}%
\pgfusepath{clip}%
\pgfsetbuttcap%
\pgfsetroundjoin%
\pgfsetlinewidth{0.301125pt}%
\definecolor{currentstroke}{rgb}{0.500000,0.500000,0.500000}%
\pgfsetstrokecolor{currentstroke}%
\pgfsetstrokeopacity{0.300000}%
\pgfsetdash{}{0pt}%
\pgfpathmoveto{\pgfqpoint{3.727238in}{0.772378in}}%
\pgfpathlineto{\pgfqpoint{3.727238in}{0.772378in}}%
\pgfpathlineto{\pgfqpoint{3.661988in}{0.792974in}}%
\pgfpathlineto{\pgfqpoint{3.597248in}{0.815121in}}%
\pgfpathlineto{\pgfqpoint{3.532970in}{0.838582in}}%
\pgfpathlineto{\pgfqpoint{3.469087in}{0.863101in}}%
\pgfpathlineto{\pgfqpoint{3.405512in}{0.888412in}}%
\pgfpathlineto{\pgfqpoint{3.342145in}{0.914239in}}%
\pgfpathlineto{\pgfqpoint{3.278872in}{0.940297in}}%
\pgfpathlineto{\pgfqpoint{3.215574in}{0.966292in}}%
\pgfpathlineto{\pgfqpoint{3.152127in}{0.991920in}}%
\pgfpathlineto{\pgfqpoint{3.088409in}{1.016864in}}%
\pgfpathlineto{\pgfqpoint{3.024306in}{1.040798in}}%
\pgfpathlineto{\pgfqpoint{2.959720in}{1.063386in}}%
\pgfpathlineto{\pgfqpoint{2.894571in}{1.084289in}}%
\pgfpathlineto{\pgfqpoint{2.828811in}{1.103177in}}%
\pgfpathlineto{\pgfqpoint{2.762431in}{1.119742in}}%
\pgfpathlineto{\pgfqpoint{2.695461in}{1.133726in}}%
\pgfpathlineto{\pgfqpoint{2.627976in}{1.144945in}}%
\pgfpathlineto{\pgfqpoint{2.560079in}{1.153319in}}%
\pgfpathlineto{\pgfqpoint{2.491894in}{1.158886in}}%
\pgfpathlineto{\pgfqpoint{2.423541in}{1.161821in}}%
\pgfpathlineto{\pgfqpoint{2.355124in}{1.162443in}}%
\pgfpathlineto{\pgfqpoint{2.286711in}{1.161211in}}%
\pgfpathlineto{\pgfqpoint{2.218330in}{1.158708in}}%
\pgfpathlineto{\pgfqpoint{2.149970in}{1.155639in}}%
\pgfpathlineto{\pgfqpoint{2.081598in}{1.152870in}}%
\pgfpathlineto{\pgfqpoint{2.013191in}{1.151498in}}%
\pgfpathlineto{\pgfqpoint{1.944809in}{1.153007in}}%
\pgfpathlineto{\pgfqpoint{1.876783in}{1.159541in}}%
\pgfpathlineto{\pgfqpoint{1.810274in}{1.174504in}}%
\pgfpathlineto{\pgfqpoint{1.810274in}{1.174504in}}%
\pgfpathlineto{\pgfqpoint{1.758282in}{1.197205in}}%
\pgfpathlineto{\pgfqpoint{1.758282in}{1.197205in}}%
\pgfusepath{stroke}%
\end{pgfscope}%
\begin{pgfscope}%
\pgfpathrectangle{\pgfqpoint{0.647939in}{0.492442in}}{\pgfqpoint{3.079299in}{3.079299in}}%
\pgfusepath{clip}%
\pgfsetbuttcap%
\pgfsetroundjoin%
\pgfsetlinewidth{0.301125pt}%
\definecolor{currentstroke}{rgb}{0.500000,0.500000,0.500000}%
\pgfsetstrokecolor{currentstroke}%
\pgfsetstrokeopacity{0.300000}%
\pgfsetdash{}{0pt}%
\pgfpathmoveto{\pgfqpoint{3.727238in}{0.842362in}}%
\pgfpathlineto{\pgfqpoint{3.727238in}{0.842362in}}%
\pgfpathlineto{\pgfqpoint{3.662212in}{0.863652in}}%
\pgfpathlineto{\pgfqpoint{3.597733in}{0.886549in}}%
\pgfpathlineto{\pgfqpoint{3.533754in}{0.910813in}}%
\pgfpathlineto{\pgfqpoint{3.470207in}{0.936188in}}%
\pgfpathlineto{\pgfqpoint{3.407001in}{0.962406in}}%
\pgfpathlineto{\pgfqpoint{3.344033in}{0.989192in}}%
\pgfpathlineto{\pgfqpoint{3.281186in}{1.016260in}}%
\pgfpathlineto{\pgfqpoint{3.218334in}{1.043318in}}%
\pgfpathlineto{\pgfqpoint{3.155349in}{1.070061in}}%
\pgfpathlineto{\pgfqpoint{3.092100in}{1.096170in}}%
\pgfpathlineto{\pgfqpoint{3.028463in}{1.121315in}}%
\pgfpathlineto{\pgfqpoint{2.964327in}{1.145152in}}%
\pgfpathlineto{\pgfqpoint{2.899600in}{1.167327in}}%
\pgfpathlineto{\pgfqpoint{2.834221in}{1.187488in}}%
\pgfpathlineto{\pgfqpoint{2.768165in}{1.205298in}}%
\pgfpathlineto{\pgfqpoint{2.701454in}{1.220458in}}%
\pgfpathlineto{\pgfqpoint{2.634154in}{1.232743in}}%
\pgfpathlineto{\pgfqpoint{2.566378in}{1.242027in}}%
\pgfpathlineto{\pgfqpoint{2.498257in}{1.248315in}}%
\pgfpathlineto{\pgfqpoint{2.429932in}{1.251757in}}%
\pgfpathlineto{\pgfqpoint{2.361520in}{1.252652in}}%
\pgfpathlineto{\pgfqpoint{2.293108in}{1.251460in}}%
\pgfpathlineto{\pgfqpoint{2.224734in}{1.248796in}}%
\pgfpathlineto{\pgfqpoint{2.156388in}{1.245417in}}%
\pgfpathlineto{\pgfqpoint{2.088034in}{1.242254in}}%
\pgfpathlineto{\pgfqpoint{2.019636in}{1.240506in}}%
\pgfpathlineto{\pgfqpoint{1.951254in}{1.241850in}}%
\pgfpathlineto{\pgfqpoint{1.883302in}{1.248872in}}%
\pgfpathlineto{\pgfqpoint{1.817453in}{1.265962in}}%
\pgfpathlineto{\pgfqpoint{1.817453in}{1.265962in}}%
\pgfpathlineto{\pgfqpoint{1.773468in}{1.288507in}}%
\pgfpathlineto{\pgfqpoint{1.773468in}{1.288507in}}%
\pgfpathlineto{\pgfqpoint{1.742685in}{1.316821in}}%
\pgfpathlineto{\pgfqpoint{1.722043in}{1.353540in}}%
\pgfpathlineto{\pgfqpoint{1.713141in}{1.391833in}}%
\pgfpathlineto{\pgfqpoint{1.712912in}{1.434098in}}%
\pgfusepath{stroke}%
\end{pgfscope}%
\begin{pgfscope}%
\pgfpathrectangle{\pgfqpoint{0.647939in}{0.492442in}}{\pgfqpoint{3.079299in}{3.079299in}}%
\pgfusepath{clip}%
\pgfsetbuttcap%
\pgfsetroundjoin%
\pgfsetlinewidth{0.301125pt}%
\definecolor{currentstroke}{rgb}{0.500000,0.500000,0.500000}%
\pgfsetstrokecolor{currentstroke}%
\pgfsetstrokeopacity{0.300000}%
\pgfsetdash{}{0pt}%
\pgfpathmoveto{\pgfqpoint{3.727238in}{0.912347in}}%
\pgfpathlineto{\pgfqpoint{3.727238in}{0.912347in}}%
\pgfpathlineto{\pgfqpoint{3.662459in}{0.934375in}}%
\pgfpathlineto{\pgfqpoint{3.598270in}{0.958072in}}%
\pgfpathlineto{\pgfqpoint{3.534624in}{0.983194in}}%
\pgfpathlineto{\pgfqpoint{3.471449in}{1.009483in}}%
\pgfpathlineto{\pgfqpoint{3.408656in}{1.036672in}}%
\pgfpathlineto{\pgfqpoint{3.346136in}{1.064487in}}%
\pgfpathlineto{\pgfqpoint{3.283769in}{1.092644in}}%
\pgfpathlineto{\pgfqpoint{3.221426in}{1.120852in}}%
\pgfpathlineto{\pgfqpoint{3.158969in}{1.148809in}}%
\pgfpathlineto{\pgfqpoint{3.096262in}{1.176197in}}%
\pgfpathlineto{\pgfqpoint{3.033171in}{1.202683in}}%
\pgfpathlineto{\pgfqpoint{2.969571in}{1.227916in}}%
\pgfpathlineto{\pgfqpoint{2.905355in}{1.251529in}}%
\pgfpathlineto{\pgfqpoint{2.840445in}{1.273149in}}%
\pgfpathlineto{\pgfqpoint{2.774799in}{1.292407in}}%
\pgfpathlineto{\pgfqpoint{2.708423in}{1.308964in}}%
\pgfpathlineto{\pgfqpoint{2.641374in}{1.322537in}}%
\pgfpathlineto{\pgfqpoint{2.573762in}{1.332943in}}%
\pgfpathlineto{\pgfqpoint{2.505733in}{1.340136in}}%
\pgfpathlineto{\pgfqpoint{2.437447in}{1.344230in}}%
\pgfpathlineto{\pgfqpoint{2.369045in}{1.345506in}}%
\pgfpathlineto{\pgfqpoint{2.300634in}{1.344413in}}%
\pgfpathlineto{\pgfqpoint{2.232268in}{1.341577in}}%
\pgfpathlineto{\pgfqpoint{2.163943in}{1.337809in}}%
\pgfpathlineto{\pgfqpoint{2.095614in}{1.334131in}}%
\pgfpathlineto{\pgfqpoint{2.027233in}{1.331875in}}%
\pgfpathlineto{\pgfqpoint{1.958855in}{1.332966in}}%
\pgfpathlineto{\pgfqpoint{1.890997in}{1.340588in}}%
\pgfpathlineto{\pgfqpoint{1.826249in}{1.360704in}}%
\pgfpathlineto{\pgfqpoint{1.826249in}{1.360704in}}%
\pgfpathlineto{\pgfqpoint{1.790537in}{1.383535in}}%
\pgfpathlineto{\pgfqpoint{1.790537in}{1.383535in}}%
\pgfpathlineto{\pgfqpoint{1.766665in}{1.412071in}}%
\pgfpathlineto{\pgfqpoint{1.753067in}{1.447008in}}%
\pgfpathlineto{\pgfqpoint{1.749428in}{1.483502in}}%
\pgfpathlineto{\pgfqpoint{1.754019in}{1.524923in}}%
\pgfpathlineto{\pgfqpoint{1.767509in}{1.572483in}}%
\pgfusepath{stroke}%
\end{pgfscope}%
\begin{pgfscope}%
\pgfpathrectangle{\pgfqpoint{0.647939in}{0.492442in}}{\pgfqpoint{3.079299in}{3.079299in}}%
\pgfusepath{clip}%
\pgfsetbuttcap%
\pgfsetroundjoin%
\pgfsetlinewidth{0.301125pt}%
\definecolor{currentstroke}{rgb}{0.500000,0.500000,0.500000}%
\pgfsetstrokecolor{currentstroke}%
\pgfsetstrokeopacity{0.300000}%
\pgfsetdash{}{0pt}%
\pgfpathmoveto{\pgfqpoint{3.727238in}{0.982331in}}%
\pgfpathlineto{\pgfqpoint{3.727238in}{0.982331in}}%
\pgfpathlineto{\pgfqpoint{3.662734in}{1.005149in}}%
\pgfpathlineto{\pgfqpoint{3.598867in}{1.029699in}}%
\pgfpathlineto{\pgfqpoint{3.535591in}{1.055738in}}%
\pgfpathlineto{\pgfqpoint{3.472832in}{1.083006in}}%
\pgfpathlineto{\pgfqpoint{3.410500in}{1.111237in}}%
\pgfpathlineto{\pgfqpoint{3.348485in}{1.140158in}}%
\pgfpathlineto{\pgfqpoint{3.286662in}{1.169491in}}%
\pgfpathlineto{\pgfqpoint{3.224899in}{1.198949in}}%
\pgfpathlineto{\pgfqpoint{3.163054in}{1.228232in}}%
\pgfpathlineto{\pgfqpoint{3.100980in}{1.257027in}}%
\pgfpathlineto{\pgfqpoint{3.038535in}{1.285002in}}%
\pgfpathlineto{\pgfqpoint{2.975578in}{1.311803in}}%
\pgfpathlineto{\pgfqpoint{2.911987in}{1.337053in}}%
\pgfpathlineto{\pgfqpoint{2.847662in}{1.360359in}}%
\pgfpathlineto{\pgfqpoint{2.782540in}{1.381320in}}%
\pgfpathlineto{\pgfqpoint{2.716606in}{1.399550in}}%
\pgfpathlineto{\pgfqpoint{2.649902in}{1.414707in}}%
\pgfpathlineto{\pgfqpoint{2.582529in}{1.426533in}}%
\pgfpathlineto{\pgfqpoint{2.514638in}{1.434900in}}%
\pgfpathlineto{\pgfqpoint{2.446411in}{1.439859in}}%
\pgfpathlineto{\pgfqpoint{2.378024in}{1.441661in}}%
\pgfpathlineto{\pgfqpoint{2.309614in}{1.440748in}}%
\pgfpathlineto{\pgfqpoint{2.241258in}{1.437754in}}%
\pgfpathlineto{\pgfqpoint{2.172962in}{1.433515in}}%
\pgfpathlineto{\pgfqpoint{2.104674in}{1.429140in}}%
\pgfpathlineto{\pgfqpoint{2.036322in}{1.426144in}}%
\pgfpathlineto{\pgfqpoint{1.967954in}{1.426804in}}%
\pgfpathlineto{\pgfqpoint{1.900289in}{1.435231in}}%
\pgfpathlineto{\pgfqpoint{1.900289in}{1.435231in}}%
\pgfpathlineto{\pgfqpoint{1.852276in}{1.451088in}}%
\pgfpathlineto{\pgfqpoint{1.852276in}{1.451088in}}%
\pgfpathlineto{\pgfqpoint{1.820527in}{1.472455in}}%
\pgfpathlineto{\pgfqpoint{1.820527in}{1.472455in}}%
\pgfusepath{stroke}%
\end{pgfscope}%
\begin{pgfscope}%
\pgfpathrectangle{\pgfqpoint{0.647939in}{0.492442in}}{\pgfqpoint{3.079299in}{3.079299in}}%
\pgfusepath{clip}%
\pgfsetbuttcap%
\pgfsetroundjoin%
\pgfsetlinewidth{0.301125pt}%
\definecolor{currentstroke}{rgb}{0.500000,0.500000,0.500000}%
\pgfsetstrokecolor{currentstroke}%
\pgfsetstrokeopacity{0.300000}%
\pgfsetdash{}{0pt}%
\pgfpathmoveto{\pgfqpoint{3.727238in}{1.052315in}}%
\pgfpathlineto{\pgfqpoint{3.727238in}{1.052315in}}%
\pgfpathlineto{\pgfqpoint{3.663039in}{1.075977in}}%
\pgfpathlineto{\pgfqpoint{3.599532in}{1.101442in}}%
\pgfpathlineto{\pgfqpoint{3.536669in}{1.128461in}}%
\pgfpathlineto{\pgfqpoint{3.474378in}{1.156778in}}%
\pgfpathlineto{\pgfqpoint{3.412565in}{1.186128in}}%
\pgfpathlineto{\pgfqpoint{3.351121in}{1.216243in}}%
\pgfpathlineto{\pgfqpoint{3.289919in}{1.246849in}}%
\pgfpathlineto{\pgfqpoint{3.228823in}{1.277666in}}%
\pgfpathlineto{\pgfqpoint{3.167686in}{1.308402in}}%
\pgfpathlineto{\pgfqpoint{3.106356in}{1.338750in}}%
\pgfpathlineto{\pgfqpoint{3.044679in}{1.368383in}}%
\pgfpathlineto{\pgfqpoint{2.982503in}{1.396951in}}%
\pgfpathlineto{\pgfqpoint{2.919686in}{1.424073in}}%
\pgfpathlineto{\pgfqpoint{2.856104in}{1.449343in}}%
\pgfpathlineto{\pgfqpoint{2.791666in}{1.472328in}}%
\pgfpathlineto{\pgfqpoint{2.726327in}{1.492593in}}%
\pgfpathlineto{\pgfqpoint{2.660103in}{1.509724in}}%
\pgfpathlineto{\pgfqpoint{2.593080in}{1.523376in}}%
\pgfpathlineto{\pgfqpoint{2.525410in}{1.533319in}}%
\pgfpathlineto{\pgfqpoint{2.457291in}{1.539499in}}%
\pgfpathlineto{\pgfqpoint{2.388934in}{1.542073in}}%
\pgfpathlineto{\pgfqpoint{2.320525in}{1.541456in}}%
\pgfpathlineto{\pgfqpoint{2.252178in}{1.538302in}}%
\pgfpathlineto{\pgfqpoint{2.183922in}{1.533490in}}%
\pgfpathlineto{\pgfqpoint{2.115699in}{1.528199in}}%
\pgfpathlineto{\pgfqpoint{2.047402in}{1.524095in}}%
\pgfpathlineto{\pgfqpoint{1.979044in}{1.523951in}}%
\pgfpathlineto{\pgfqpoint{1.911729in}{1.533482in}}%
\pgfpathlineto{\pgfqpoint{1.911729in}{1.533482in}}%
\pgfpathlineto{\pgfqpoint{1.874509in}{1.548206in}}%
\pgfpathlineto{\pgfqpoint{1.874509in}{1.548206in}}%
\pgfpathlineto{\pgfqpoint{1.850084in}{1.568672in}}%
\pgfpathlineto{\pgfqpoint{1.835839in}{1.597978in}}%
\pgfpathlineto{\pgfqpoint{1.833130in}{1.627161in}}%
\pgfpathlineto{\pgfqpoint{1.838435in}{1.659214in}}%
\pgfpathlineto{\pgfqpoint{1.852161in}{1.695931in}}%
\pgfusepath{stroke}%
\end{pgfscope}%
\begin{pgfscope}%
\pgfpathrectangle{\pgfqpoint{0.647939in}{0.492442in}}{\pgfqpoint{3.079299in}{3.079299in}}%
\pgfusepath{clip}%
\pgfsetbuttcap%
\pgfsetroundjoin%
\pgfsetlinewidth{0.301125pt}%
\definecolor{currentstroke}{rgb}{0.500000,0.500000,0.500000}%
\pgfsetstrokecolor{currentstroke}%
\pgfsetstrokeopacity{0.300000}%
\pgfsetdash{}{0pt}%
\pgfpathmoveto{\pgfqpoint{3.727238in}{1.122299in}}%
\pgfpathlineto{\pgfqpoint{3.727238in}{1.122299in}}%
\pgfpathlineto{\pgfqpoint{3.663381in}{1.146866in}}%
\pgfpathlineto{\pgfqpoint{3.600275in}{1.173310in}}%
\pgfpathlineto{\pgfqpoint{3.537875in}{1.201382in}}%
\pgfpathlineto{\pgfqpoint{3.476109in}{1.230825in}}%
\pgfpathlineto{\pgfqpoint{3.414884in}{1.261381in}}%
\pgfpathlineto{\pgfqpoint{3.354089in}{1.292784in}}%
\pgfpathlineto{\pgfqpoint{3.293596in}{1.324769in}}%
\pgfpathlineto{\pgfqpoint{3.233267in}{1.357061in}}%
\pgfpathlineto{\pgfqpoint{3.172953in}{1.389383in}}%
\pgfpathlineto{\pgfqpoint{3.112500in}{1.421440in}}%
\pgfpathlineto{\pgfqpoint{3.051744in}{1.452919in}}%
\pgfpathlineto{\pgfqpoint{2.990524in}{1.483480in}}%
\pgfpathlineto{\pgfqpoint{2.928679in}{1.512749in}}%
\pgfpathlineto{\pgfqpoint{2.866061in}{1.540315in}}%
\pgfpathlineto{\pgfqpoint{2.802543in}{1.565727in}}%
\pgfpathlineto{\pgfqpoint{2.738039in}{1.588504in}}%
\pgfpathlineto{\pgfqpoint{2.672518in}{1.608153in}}%
\pgfpathlineto{\pgfqpoint{2.606030in}{1.624211in}}%
\pgfpathlineto{\pgfqpoint{2.538714in}{1.636311in}}%
\pgfpathlineto{\pgfqpoint{2.470786in}{1.644253in}}%
\pgfpathlineto{\pgfqpoint{2.402499in}{1.648059in}}%
\pgfpathlineto{\pgfqpoint{2.334099in}{1.648026in}}%
\pgfpathlineto{\pgfqpoint{2.265765in}{1.644743in}}%
\pgfpathlineto{\pgfqpoint{2.197573in}{1.639133in}}%
\pgfpathlineto{\pgfqpoint{2.129468in}{1.632489in}}%
\pgfpathlineto{\pgfqpoint{2.061296in}{1.626688in}}%
\pgfpathlineto{\pgfqpoint{1.992977in}{1.625101in}}%
\pgfpathlineto{\pgfqpoint{1.992977in}{1.625101in}}%
\pgfpathlineto{\pgfqpoint{1.937586in}{1.632333in}}%
\pgfpathlineto{\pgfqpoint{1.937586in}{1.632333in}}%
\pgfpathlineto{\pgfqpoint{1.907244in}{1.644860in}}%
\pgfpathlineto{\pgfqpoint{1.907244in}{1.644860in}}%
\pgfpathlineto{\pgfqpoint{1.888792in}{1.662945in}}%
\pgfpathlineto{\pgfqpoint{1.880851in}{1.687681in}}%
\pgfpathlineto{\pgfqpoint{1.882196in}{1.711742in}}%
\pgfusepath{stroke}%
\end{pgfscope}%
\begin{pgfscope}%
\pgfpathrectangle{\pgfqpoint{0.647939in}{0.492442in}}{\pgfqpoint{3.079299in}{3.079299in}}%
\pgfusepath{clip}%
\pgfsetbuttcap%
\pgfsetroundjoin%
\pgfsetlinewidth{0.301125pt}%
\definecolor{currentstroke}{rgb}{0.500000,0.500000,0.500000}%
\pgfsetstrokecolor{currentstroke}%
\pgfsetstrokeopacity{0.300000}%
\pgfsetdash{}{0pt}%
\pgfpathmoveto{\pgfqpoint{3.727238in}{1.192283in}}%
\pgfpathlineto{\pgfqpoint{3.727238in}{1.192283in}}%
\pgfpathlineto{\pgfqpoint{3.663764in}{1.217822in}}%
\pgfpathlineto{\pgfqpoint{3.601110in}{1.245316in}}%
\pgfpathlineto{\pgfqpoint{3.539232in}{1.274519in}}%
\pgfpathlineto{\pgfqpoint{3.478060in}{1.305175in}}%
\pgfpathlineto{\pgfqpoint{3.417500in}{1.337029in}}%
\pgfpathlineto{\pgfqpoint{3.357443in}{1.369821in}}%
\pgfpathlineto{\pgfqpoint{3.297762in}{1.403296in}}%
\pgfpathlineto{\pgfqpoint{3.238323in}{1.437198in}}%
\pgfpathlineto{\pgfqpoint{3.178977in}{1.471262in}}%
\pgfpathlineto{\pgfqpoint{3.119569in}{1.505216in}}%
\pgfpathlineto{\pgfqpoint{3.059933in}{1.538767in}}%
\pgfpathlineto{\pgfqpoint{2.999899in}{1.571597in}}%
\pgfpathlineto{\pgfqpoint{2.939293in}{1.603352in}}%
\pgfpathlineto{\pgfqpoint{2.877941in}{1.633634in}}%
\pgfpathlineto{\pgfqpoint{2.815681in}{1.661991in}}%
\pgfpathlineto{\pgfqpoint{2.752377in}{1.687918in}}%
\pgfpathlineto{\pgfqpoint{2.687938in}{1.710856in}}%
\pgfpathlineto{\pgfqpoint{2.622347in}{1.730224in}}%
\pgfpathlineto{\pgfqpoint{2.555683in}{1.745462in}}%
\pgfpathlineto{\pgfqpoint{2.488138in}{1.756115in}}%
\pgfpathlineto{\pgfqpoint{2.420003in}{1.761935in}}%
\pgfpathlineto{\pgfqpoint{2.351624in}{1.763012in}}%
\pgfpathlineto{\pgfqpoint{2.283304in}{1.759811in}}%
\pgfpathlineto{\pgfqpoint{2.215219in}{1.753169in}}%
\pgfpathlineto{\pgfqpoint{2.147364in}{1.744373in}}%
\pgfpathlineto{\pgfqpoint{2.079525in}{1.735434in}}%
\pgfpathlineto{\pgfqpoint{2.011386in}{1.730524in}}%
\pgfpathlineto{\pgfqpoint{2.011386in}{1.730524in}}%
\pgfpathlineto{\pgfqpoint{1.969963in}{1.733993in}}%
\pgfpathlineto{\pgfqpoint{1.969963in}{1.733993in}}%
\pgfpathlineto{\pgfqpoint{1.947205in}{1.742790in}}%
\pgfpathlineto{\pgfqpoint{1.947205in}{1.742790in}}%
\pgfusepath{stroke}%
\end{pgfscope}%
\begin{pgfscope}%
\pgfpathrectangle{\pgfqpoint{0.647939in}{0.492442in}}{\pgfqpoint{3.079299in}{3.079299in}}%
\pgfusepath{clip}%
\pgfsetbuttcap%
\pgfsetroundjoin%
\pgfsetlinewidth{0.301125pt}%
\definecolor{currentstroke}{rgb}{0.500000,0.500000,0.500000}%
\pgfsetstrokecolor{currentstroke}%
\pgfsetstrokeopacity{0.300000}%
\pgfsetdash{}{0pt}%
\pgfpathmoveto{\pgfqpoint{3.727238in}{1.262267in}}%
\pgfpathlineto{\pgfqpoint{3.727238in}{1.262267in}}%
\pgfpathlineto{\pgfqpoint{3.664196in}{1.288850in}}%
\pgfpathlineto{\pgfqpoint{3.602051in}{1.317475in}}%
\pgfpathlineto{\pgfqpoint{3.540763in}{1.347895in}}%
\pgfpathlineto{\pgfqpoint{3.480263in}{1.379856in}}%
\pgfpathlineto{\pgfqpoint{3.420461in}{1.413106in}}%
\pgfpathlineto{\pgfqpoint{3.361249in}{1.447401in}}%
\pgfpathlineto{\pgfqpoint{3.302510in}{1.482499in}}%
\pgfpathlineto{\pgfqpoint{3.244110in}{1.518162in}}%
\pgfpathlineto{\pgfqpoint{3.185909in}{1.554149in}}%
\pgfpathlineto{\pgfqpoint{3.127757in}{1.590214in}}%
\pgfpathlineto{\pgfqpoint{3.069494in}{1.626097in}}%
\pgfpathlineto{\pgfqpoint{3.010949in}{1.661517in}}%
\pgfpathlineto{\pgfqpoint{2.951943in}{1.696160in}}%
\pgfpathlineto{\pgfqpoint{2.892287in}{1.729666in}}%
\pgfpathlineto{\pgfqpoint{2.831792in}{1.761625in}}%
\pgfpathlineto{\pgfqpoint{2.770273in}{1.791550in}}%
\pgfpathlineto{\pgfqpoint{2.707561in}{1.818869in}}%
\pgfpathlineto{\pgfqpoint{2.643532in}{1.842911in}}%
\pgfpathlineto{\pgfqpoint{2.578138in}{1.862916in}}%
\pgfpathlineto{\pgfqpoint{2.511468in}{1.878088in}}%
\pgfpathlineto{\pgfqpoint{2.443790in}{1.887710in}}%
\pgfpathlineto{\pgfqpoint{2.375537in}{1.891291in}}%
\pgfpathlineto{\pgfqpoint{2.307237in}{1.888737in}}%
\pgfpathlineto{\pgfqpoint{2.239372in}{1.880494in}}%
\pgfpathlineto{\pgfqpoint{2.172212in}{1.867601in}}%
\pgfpathlineto{\pgfqpoint{2.105631in}{1.851893in}}%
\pgfpathlineto{\pgfqpoint{2.038854in}{1.837291in}}%
\pgfpathlineto{\pgfqpoint{2.038854in}{1.837291in}}%
\pgfusepath{stroke}%
\end{pgfscope}%
\begin{pgfscope}%
\pgfpathrectangle{\pgfqpoint{0.647939in}{0.492442in}}{\pgfqpoint{3.079299in}{3.079299in}}%
\pgfusepath{clip}%
\pgfsetbuttcap%
\pgfsetroundjoin%
\pgfsetlinewidth{0.301125pt}%
\definecolor{currentstroke}{rgb}{0.500000,0.500000,0.500000}%
\pgfsetstrokecolor{currentstroke}%
\pgfsetstrokeopacity{0.300000}%
\pgfsetdash{}{0pt}%
\pgfpathmoveto{\pgfqpoint{3.727238in}{1.332251in}}%
\pgfpathlineto{\pgfqpoint{3.727238in}{1.332251in}}%
\pgfpathlineto{\pgfqpoint{3.664683in}{1.359960in}}%
\pgfpathlineto{\pgfqpoint{3.603115in}{1.389803in}}%
\pgfpathlineto{\pgfqpoint{3.542495in}{1.421532in}}%
\pgfpathlineto{\pgfqpoint{3.482759in}{1.454898in}}%
\pgfpathlineto{\pgfqpoint{3.423823in}{1.489661in}}%
\pgfpathlineto{\pgfqpoint{3.365587in}{1.525586in}}%
\pgfpathlineto{\pgfqpoint{3.307942in}{1.562453in}}%
\pgfpathlineto{\pgfqpoint{3.250766in}{1.600045in}}%
\pgfpathlineto{\pgfqpoint{3.193933in}{1.638153in}}%
\pgfpathlineto{\pgfqpoint{3.137306in}{1.676568in}}%
\pgfpathlineto{\pgfqpoint{3.080738in}{1.715069in}}%
\pgfpathlineto{\pgfqpoint{3.024076in}{1.753428in}}%
\pgfpathlineto{\pgfqpoint{2.967159in}{1.791405in}}%
\pgfpathlineto{\pgfqpoint{2.909813in}{1.828730in}}%
\pgfpathlineto{\pgfqpoint{2.851853in}{1.865091in}}%
\pgfpathlineto{\pgfqpoint{2.793078in}{1.900115in}}%
\pgfpathlineto{\pgfqpoint{2.733275in}{1.933343in}}%
\pgfpathlineto{\pgfqpoint{2.672223in}{1.964196in}}%
\pgfpathlineto{\pgfqpoint{2.609700in}{1.991919in}}%
\pgfpathlineto{\pgfqpoint{2.545514in}{2.015500in}}%
\pgfpathlineto{\pgfqpoint{2.479594in}{2.033559in}}%
\pgfpathlineto{\pgfqpoint{2.412153in}{2.044242in}}%
\pgfpathlineto{\pgfqpoint{2.343996in}{2.045145in}}%
\pgfpathlineto{\pgfqpoint{2.276967in}{2.033826in}}%
\pgfpathlineto{\pgfqpoint{2.214919in}{2.010379in}}%
\pgfpathlineto{\pgfqpoint{2.159971in}{1.979178in}}%
\pgfpathlineto{\pgfqpoint{2.105353in}{1.940519in}}%
\pgfpathlineto{\pgfqpoint{2.053406in}{1.901523in}}%
\pgfpathlineto{\pgfqpoint{2.053406in}{1.901523in}}%
\pgfpathlineto{\pgfqpoint{2.032315in}{1.886456in}}%
\pgfpathlineto{\pgfqpoint{2.032315in}{1.886456in}}%
\pgfpathlineto{\pgfqpoint{2.017973in}{1.876647in}}%
\pgfpathlineto{\pgfqpoint{2.017973in}{1.876647in}}%
\pgfusepath{stroke}%
\end{pgfscope}%
\begin{pgfscope}%
\pgfpathrectangle{\pgfqpoint{0.647939in}{0.492442in}}{\pgfqpoint{3.079299in}{3.079299in}}%
\pgfusepath{clip}%
\pgfsetbuttcap%
\pgfsetroundjoin%
\pgfsetlinewidth{0.301125pt}%
\definecolor{currentstroke}{rgb}{0.500000,0.500000,0.500000}%
\pgfsetstrokecolor{currentstroke}%
\pgfsetstrokeopacity{0.300000}%
\pgfsetdash{}{0pt}%
\pgfpathmoveto{\pgfqpoint{3.727238in}{1.472219in}}%
\pgfpathlineto{\pgfqpoint{3.727238in}{1.472219in}}%
\pgfpathlineto{\pgfqpoint{3.665869in}{1.502458in}}%
\pgfpathlineto{\pgfqpoint{3.605705in}{1.535031in}}%
\pgfpathlineto{\pgfqpoint{3.546720in}{1.569698in}}%
\pgfpathlineto{\pgfqpoint{3.488866in}{1.606224in}}%
\pgfpathlineto{\pgfqpoint{3.432082in}{1.644396in}}%
\pgfpathlineto{\pgfqpoint{3.376298in}{1.684017in}}%
\pgfpathlineto{\pgfqpoint{3.321443in}{1.724915in}}%
\pgfpathlineto{\pgfqpoint{3.267439in}{1.766933in}}%
\pgfpathlineto{\pgfqpoint{3.214213in}{1.809933in}}%
\pgfpathlineto{\pgfqpoint{3.161710in}{1.853811in}}%
\pgfpathlineto{\pgfqpoint{3.109885in}{1.898487in}}%
\pgfpathlineto{\pgfqpoint{3.058711in}{1.943907in}}%
\pgfpathlineto{\pgfqpoint{3.008200in}{1.990061in}}%
\pgfpathlineto{\pgfqpoint{2.958410in}{2.036991in}}%
\pgfpathlineto{\pgfqpoint{2.909482in}{2.084813in}}%
\pgfpathlineto{\pgfqpoint{2.861693in}{2.133766in}}%
\pgfpathlineto{\pgfqpoint{2.815570in}{2.184276in}}%
\pgfpathlineto{\pgfqpoint{2.772115in}{2.237076in}}%
\pgfpathlineto{\pgfqpoint{2.733337in}{2.293321in}}%
\pgfpathlineto{\pgfqpoint{2.703241in}{2.354426in}}%
\pgfpathlineto{\pgfqpoint{2.703241in}{2.354426in}}%
\pgfpathlineto{\pgfqpoint{2.689191in}{2.409732in}}%
\pgfpathlineto{\pgfqpoint{2.688655in}{2.467350in}}%
\pgfpathlineto{\pgfqpoint{2.699505in}{2.520754in}}%
\pgfpathlineto{\pgfqpoint{2.720504in}{2.577531in}}%
\pgfpathlineto{\pgfqpoint{2.750913in}{2.638655in}}%
\pgfpathlineto{\pgfqpoint{2.785446in}{2.697634in}}%
\pgfpathlineto{\pgfqpoint{2.822461in}{2.755120in}}%
\pgfpathlineto{\pgfqpoint{2.861103in}{2.811547in}}%
\pgfpathlineto{\pgfqpoint{2.900906in}{2.867160in}}%
\pgfpathlineto{\pgfqpoint{2.941619in}{2.922123in}}%
\pgfpathlineto{\pgfqpoint{2.983104in}{2.976523in}}%
\pgfpathlineto{\pgfqpoint{3.025294in}{3.030379in}}%
\pgfpathlineto{\pgfqpoint{3.068163in}{3.083693in}}%
\pgfpathlineto{\pgfqpoint{3.111722in}{3.136455in}}%
\pgfpathlineto{\pgfqpoint{3.156000in}{3.188616in}}%
\pgfpathlineto{\pgfqpoint{3.201036in}{3.240124in}}%
\pgfpathlineto{\pgfqpoint{3.246893in}{3.290905in}}%
\pgfpathlineto{\pgfqpoint{3.293634in}{3.340874in}}%
\pgfpathlineto{\pgfqpoint{3.341339in}{3.389922in}}%
\pgfpathlineto{\pgfqpoint{3.390092in}{3.437929in}}%
\pgfpathlineto{\pgfqpoint{3.439982in}{3.484751in}}%
\pgfpathlineto{\pgfqpoint{3.491107in}{3.530221in}}%
\pgfpathlineto{\pgfqpoint{3.539121in}{3.571741in}}%
\pgfusepath{stroke}%
\end{pgfscope}%
\begin{pgfscope}%
\pgfpathrectangle{\pgfqpoint{0.647939in}{0.492442in}}{\pgfqpoint{3.079299in}{3.079299in}}%
\pgfusepath{clip}%
\pgfsetbuttcap%
\pgfsetroundjoin%
\pgfsetlinewidth{0.301125pt}%
\definecolor{currentstroke}{rgb}{0.500000,0.500000,0.500000}%
\pgfsetstrokecolor{currentstroke}%
\pgfsetstrokeopacity{0.300000}%
\pgfsetdash{}{0pt}%
\pgfpathmoveto{\pgfqpoint{3.727238in}{1.612187in}}%
\pgfpathlineto{\pgfqpoint{3.727238in}{1.612187in}}%
\pgfpathlineto{\pgfqpoint{3.667426in}{1.645392in}}%
\pgfpathlineto{\pgfqpoint{3.609110in}{1.681163in}}%
\pgfpathlineto{\pgfqpoint{3.552290in}{1.719267in}}%
\pgfpathlineto{\pgfqpoint{3.496953in}{1.759497in}}%
\pgfpathlineto{\pgfqpoint{3.443082in}{1.801673in}}%
\pgfpathlineto{\pgfqpoint{3.390664in}{1.845643in}}%
\pgfpathlineto{\pgfqpoint{3.339696in}{1.891287in}}%
\pgfpathlineto{\pgfqpoint{3.290207in}{1.938530in}}%
\pgfpathlineto{\pgfqpoint{3.242259in}{1.987336in}}%
\pgfpathlineto{\pgfqpoint{3.195969in}{2.037714in}}%
\pgfpathlineto{\pgfqpoint{3.151534in}{2.089730in}}%
\pgfpathlineto{\pgfqpoint{3.109254in}{2.143509in}}%
\pgfpathlineto{\pgfqpoint{3.069590in}{2.199238in}}%
\pgfpathlineto{\pgfqpoint{3.033224in}{2.257154in}}%
\pgfpathlineto{\pgfqpoint{3.001131in}{2.317515in}}%
\pgfpathlineto{\pgfqpoint{2.974634in}{2.380496in}}%
\pgfpathlineto{\pgfqpoint{2.955327in}{2.445993in}}%
\pgfpathlineto{\pgfqpoint{2.944720in}{2.513414in}}%
\pgfpathlineto{\pgfqpoint{2.943620in}{2.581642in}}%
\pgfpathlineto{\pgfqpoint{2.951680in}{2.649413in}}%
\pgfpathlineto{\pgfqpoint{2.967633in}{2.715809in}}%
\pgfpathlineto{\pgfqpoint{2.989887in}{2.780410in}}%
\pgfpathlineto{\pgfqpoint{3.016993in}{2.843166in}}%
\pgfpathlineto{\pgfqpoint{3.047827in}{2.904201in}}%
\pgfpathlineto{\pgfqpoint{3.081582in}{2.963684in}}%
\pgfpathlineto{\pgfqpoint{3.117693in}{3.021776in}}%
\pgfpathlineto{\pgfqpoint{3.155776in}{3.078603in}}%
\pgfpathlineto{\pgfqpoint{3.195572in}{3.134253in}}%
\pgfpathlineto{\pgfqpoint{3.236916in}{3.188765in}}%
\pgfpathlineto{\pgfqpoint{3.279715in}{3.242140in}}%
\pgfpathlineto{\pgfqpoint{3.323915in}{3.294364in}}%
\pgfpathlineto{\pgfqpoint{3.369511in}{3.345376in}}%
\pgfpathlineto{\pgfqpoint{3.416516in}{3.395091in}}%
\pgfpathlineto{\pgfqpoint{3.464971in}{3.443393in}}%
\pgfusepath{stroke}%
\end{pgfscope}%
\begin{pgfscope}%
\pgfpathrectangle{\pgfqpoint{0.647939in}{0.492442in}}{\pgfqpoint{3.079299in}{3.079299in}}%
\pgfusepath{clip}%
\pgfsetbuttcap%
\pgfsetroundjoin%
\pgfsetlinewidth{0.301125pt}%
\definecolor{currentstroke}{rgb}{0.500000,0.500000,0.500000}%
\pgfsetstrokecolor{currentstroke}%
\pgfsetstrokeopacity{0.300000}%
\pgfsetdash{}{0pt}%
\pgfpathmoveto{\pgfqpoint{3.727238in}{1.752155in}}%
\pgfpathlineto{\pgfqpoint{3.727238in}{1.752155in}}%
\pgfpathlineto{\pgfqpoint{3.669513in}{1.788857in}}%
\pgfpathlineto{\pgfqpoint{3.613681in}{1.828381in}}%
\pgfpathlineto{\pgfqpoint{3.559786in}{1.870512in}}%
\pgfpathlineto{\pgfqpoint{3.507879in}{1.915070in}}%
\pgfpathlineto{\pgfqpoint{3.458016in}{1.961904in}}%
\pgfpathlineto{\pgfqpoint{3.410280in}{2.010906in}}%
\pgfpathlineto{\pgfqpoint{3.364809in}{2.062013in}}%
\pgfpathlineto{\pgfqpoint{3.321800in}{2.115206in}}%
\pgfpathlineto{\pgfqpoint{3.281534in}{2.170502in}}%
\pgfpathlineto{\pgfqpoint{3.244401in}{2.227941in}}%
\pgfpathlineto{\pgfqpoint{3.210934in}{2.287576in}}%
\pgfpathlineto{\pgfqpoint{3.181808in}{2.349433in}}%
\pgfpathlineto{\pgfqpoint{3.157834in}{2.413444in}}%
\pgfpathlineto{\pgfqpoint{3.139885in}{2.479383in}}%
\pgfpathlineto{\pgfqpoint{3.128744in}{2.546796in}}%
\pgfpathlineto{\pgfqpoint{3.124898in}{2.615003in}}%
\pgfpathlineto{\pgfqpoint{3.128374in}{2.683225in}}%
\pgfpathlineto{\pgfqpoint{3.138727in}{2.750761in}}%
\pgfpathlineto{\pgfqpoint{3.155208in}{2.817088in}}%
\pgfpathlineto{\pgfqpoint{3.176947in}{2.881905in}}%
\pgfpathlineto{\pgfqpoint{3.203115in}{2.945082in}}%
\pgfpathlineto{\pgfqpoint{3.233009in}{3.006597in}}%
\pgfpathlineto{\pgfqpoint{3.266077in}{3.066473in}}%
\pgfpathlineto{\pgfqpoint{3.301900in}{3.124745in}}%
\pgfpathlineto{\pgfqpoint{3.340172in}{3.181442in}}%
\pgfpathlineto{\pgfqpoint{3.380676in}{3.236571in}}%
\pgfpathlineto{\pgfqpoint{3.423271in}{3.290107in}}%
\pgfusepath{stroke}%
\end{pgfscope}%
\begin{pgfscope}%
\pgfpathrectangle{\pgfqpoint{0.647939in}{0.492442in}}{\pgfqpoint{3.079299in}{3.079299in}}%
\pgfusepath{clip}%
\pgfsetbuttcap%
\pgfsetroundjoin%
\pgfsetlinewidth{0.301125pt}%
\definecolor{currentstroke}{rgb}{0.500000,0.500000,0.500000}%
\pgfsetstrokecolor{currentstroke}%
\pgfsetstrokeopacity{0.300000}%
\pgfsetdash{}{0pt}%
\pgfpathmoveto{\pgfqpoint{3.727238in}{1.892124in}}%
\pgfpathlineto{\pgfqpoint{3.727238in}{1.892124in}}%
\pgfpathlineto{\pgfqpoint{3.672364in}{1.932954in}}%
\pgfpathlineto{\pgfqpoint{3.619933in}{1.976875in}}%
\pgfpathlineto{\pgfqpoint{3.570063in}{2.023686in}}%
\pgfpathlineto{\pgfqpoint{3.522894in}{2.073218in}}%
\pgfpathlineto{\pgfqpoint{3.478603in}{2.125338in}}%
\pgfpathlineto{\pgfqpoint{3.437430in}{2.179950in}}%
\pgfpathlineto{\pgfqpoint{3.399681in}{2.236980in}}%
\pgfpathlineto{\pgfqpoint{3.365766in}{2.296365in}}%
\pgfpathlineto{\pgfqpoint{3.336184in}{2.358016in}}%
\pgfpathlineto{\pgfqpoint{3.311512in}{2.421781in}}%
\pgfpathlineto{\pgfqpoint{3.292365in}{2.487404in}}%
\pgfpathlineto{\pgfqpoint{3.279302in}{2.554487in}}%
\pgfpathlineto{\pgfqpoint{3.272722in}{2.622501in}}%
\pgfpathlineto{\pgfqpoint{3.272767in}{2.690832in}}%
\pgfpathlineto{\pgfqpoint{3.279285in}{2.758861in}}%
\pgfpathlineto{\pgfqpoint{3.291865in}{2.826050in}}%
\pgfpathlineto{\pgfqpoint{3.309939in}{2.891986in}}%
\pgfpathlineto{\pgfqpoint{3.332893in}{2.956393in}}%
\pgfpathlineto{\pgfqpoint{3.360138in}{3.019114in}}%
\pgfpathlineto{\pgfqpoint{3.391171in}{3.080059in}}%
\pgfpathlineto{\pgfqpoint{3.425584in}{3.139170in}}%
\pgfpathlineto{\pgfqpoint{3.463062in}{3.196392in}}%
\pgfpathlineto{\pgfqpoint{3.503368in}{3.251660in}}%
\pgfpathlineto{\pgfqpoint{3.546334in}{3.304884in}}%
\pgfpathlineto{\pgfqpoint{3.591863in}{3.355934in}}%
\pgfpathlineto{\pgfqpoint{3.639888in}{3.404644in}}%
\pgfpathlineto{\pgfqpoint{3.690370in}{3.450798in}}%
\pgfpathlineto{\pgfqpoint{3.727238in}{3.482764in}}%
\pgfusepath{stroke}%
\end{pgfscope}%
\begin{pgfscope}%
\pgfpathrectangle{\pgfqpoint{0.647939in}{0.492442in}}{\pgfqpoint{3.079299in}{3.079299in}}%
\pgfusepath{clip}%
\pgfsetbuttcap%
\pgfsetroundjoin%
\pgfsetlinewidth{0.301125pt}%
\definecolor{currentstroke}{rgb}{0.500000,0.500000,0.500000}%
\pgfsetstrokecolor{currentstroke}%
\pgfsetstrokeopacity{0.300000}%
\pgfsetdash{}{0pt}%
\pgfpathmoveto{\pgfqpoint{3.727238in}{1.962108in}}%
\pgfpathlineto{\pgfqpoint{3.727238in}{1.962108in}}%
\pgfpathlineto{\pgfqpoint{3.674184in}{2.005265in}}%
\pgfpathlineto{\pgfqpoint{3.623918in}{2.051642in}}%
\pgfpathlineto{\pgfqpoint{3.576614in}{2.101036in}}%
\pgfpathlineto{\pgfqpoint{3.532468in}{2.153269in}}%
\pgfpathlineto{\pgfqpoint{3.491731in}{2.208201in}}%
\pgfpathlineto{\pgfqpoint{3.454726in}{2.265710in}}%
\pgfpathlineto{\pgfqpoint{3.421854in}{2.325674in}}%
\pgfpathlineto{\pgfqpoint{3.393591in}{2.387933in}}%
\pgfpathlineto{\pgfqpoint{3.370464in}{2.452270in}}%
\pgfpathlineto{\pgfqpoint{3.353010in}{2.518361in}}%
\pgfusepath{stroke}%
\end{pgfscope}%
\begin{pgfscope}%
\pgfpathrectangle{\pgfqpoint{0.647939in}{0.492442in}}{\pgfqpoint{3.079299in}{3.079299in}}%
\pgfusepath{clip}%
\pgfsetbuttcap%
\pgfsetroundjoin%
\pgfsetlinewidth{0.301125pt}%
\definecolor{currentstroke}{rgb}{0.500000,0.500000,0.500000}%
\pgfsetstrokecolor{currentstroke}%
\pgfsetstrokeopacity{0.300000}%
\pgfsetdash{}{0pt}%
\pgfpathmoveto{\pgfqpoint{3.727238in}{2.102076in}}%
\pgfpathlineto{\pgfqpoint{3.727238in}{2.102076in}}%
\pgfpathlineto{\pgfqpoint{3.678895in}{2.150434in}}%
\pgfpathlineto{\pgfqpoint{3.634229in}{2.202206in}}%
\pgfpathlineto{\pgfqpoint{3.593525in}{2.257147in}}%
\pgfpathlineto{\pgfqpoint{3.557118in}{2.315017in}}%
\pgfpathlineto{\pgfqpoint{3.525401in}{2.375575in}}%
\pgfpathlineto{\pgfqpoint{3.498810in}{2.438547in}}%
\pgfpathlineto{\pgfqpoint{3.477804in}{2.503587in}}%
\pgfpathlineto{\pgfqpoint{3.462800in}{2.570263in}}%
\pgfpathlineto{\pgfqpoint{3.454115in}{2.638054in}}%
\pgfpathlineto{\pgfqpoint{3.451904in}{2.706364in}}%
\pgfpathlineto{\pgfqpoint{3.456121in}{2.774575in}}%
\pgfpathlineto{\pgfqpoint{3.466529in}{2.842122in}}%
\pgfpathlineto{\pgfqpoint{3.482743in}{2.908524in}}%
\pgfpathlineto{\pgfqpoint{3.504318in}{2.973394in}}%
\pgfpathlineto{\pgfqpoint{3.530793in}{3.036432in}}%
\pgfpathlineto{\pgfqpoint{3.561745in}{3.097404in}}%
\pgfpathlineto{\pgfqpoint{3.596809in}{3.156111in}}%
\pgfpathlineto{\pgfqpoint{3.635689in}{3.212366in}}%
\pgfpathlineto{\pgfqpoint{3.678170in}{3.265955in}}%
\pgfpathlineto{\pgfqpoint{3.724080in}{3.316639in}}%
\pgfpathlineto{\pgfqpoint{3.727238in}{3.319904in}}%
\pgfusepath{stroke}%
\end{pgfscope}%
\begin{pgfscope}%
\pgfpathrectangle{\pgfqpoint{0.647939in}{0.492442in}}{\pgfqpoint{3.079299in}{3.079299in}}%
\pgfusepath{clip}%
\pgfsetbuttcap%
\pgfsetroundjoin%
\pgfsetlinewidth{0.301125pt}%
\definecolor{currentstroke}{rgb}{0.500000,0.500000,0.500000}%
\pgfsetstrokecolor{currentstroke}%
\pgfsetstrokeopacity{0.300000}%
\pgfsetdash{}{0pt}%
\pgfpathmoveto{\pgfqpoint{3.727238in}{2.242044in}}%
\pgfpathlineto{\pgfqpoint{3.727238in}{2.242044in}}%
\pgfpathlineto{\pgfqpoint{3.685552in}{2.296224in}}%
\pgfpathlineto{\pgfqpoint{3.648696in}{2.353795in}}%
\pgfpathlineto{\pgfqpoint{3.617071in}{2.414389in}}%
\pgfpathlineto{\pgfqpoint{3.591097in}{2.477606in}}%
\pgfpathlineto{\pgfqpoint{3.571183in}{2.542984in}}%
\pgfpathlineto{\pgfqpoint{3.557685in}{2.609980in}}%
\pgfpathlineto{\pgfqpoint{3.550838in}{2.677972in}}%
\pgfpathlineto{\pgfqpoint{3.550716in}{2.746302in}}%
\pgfpathlineto{\pgfqpoint{3.557212in}{2.814325in}}%
\pgfpathlineto{\pgfqpoint{3.570060in}{2.881449in}}%
\pgfpathlineto{\pgfqpoint{3.588891in}{2.947159in}}%
\pgfpathlineto{\pgfqpoint{3.613287in}{3.011022in}}%
\pgfpathlineto{\pgfqpoint{3.642832in}{3.072681in}}%
\pgfpathlineto{\pgfqpoint{3.677152in}{3.131816in}}%
\pgfpathlineto{\pgfqpoint{3.715929in}{3.188126in}}%
\pgfpathlineto{\pgfqpoint{3.727238in}{3.203282in}}%
\pgfusepath{stroke}%
\end{pgfscope}%
\begin{pgfscope}%
\pgfpathrectangle{\pgfqpoint{0.647939in}{0.492442in}}{\pgfqpoint{3.079299in}{3.079299in}}%
\pgfusepath{clip}%
\pgfsetbuttcap%
\pgfsetroundjoin%
\pgfsetlinewidth{0.301125pt}%
\definecolor{currentstroke}{rgb}{0.500000,0.500000,0.500000}%
\pgfsetstrokecolor{currentstroke}%
\pgfsetstrokeopacity{0.300000}%
\pgfsetdash{}{0pt}%
\pgfpathmoveto{\pgfqpoint{3.727238in}{2.382012in}}%
\pgfpathlineto{\pgfqpoint{3.727238in}{2.382012in}}%
\pgfpathlineto{\pgfqpoint{3.694845in}{2.442180in}}%
\pgfpathlineto{\pgfqpoint{3.668623in}{2.505277in}}%
\pgfpathlineto{\pgfqpoint{3.648965in}{2.570712in}}%
\pgfpathlineto{\pgfqpoint{3.636192in}{2.637835in}}%
\pgfpathlineto{\pgfqpoint{3.630502in}{2.705926in}}%
\pgfpathlineto{\pgfqpoint{3.631920in}{2.774233in}}%
\pgfpathlineto{\pgfqpoint{3.640300in}{2.842040in}}%
\pgfpathlineto{\pgfqpoint{3.655363in}{2.908689in}}%
\pgfpathlineto{\pgfqpoint{3.676730in}{2.973602in}}%
\pgfpathlineto{\pgfqpoint{3.703993in}{3.036276in}}%
\pgfpathlineto{\pgfqpoint{3.727238in}{3.083755in}}%
\pgfusepath{stroke}%
\end{pgfscope}%
\begin{pgfscope}%
\pgfpathrectangle{\pgfqpoint{0.647939in}{0.492442in}}{\pgfqpoint{3.079299in}{3.079299in}}%
\pgfusepath{clip}%
\pgfsetbuttcap%
\pgfsetroundjoin%
\pgfsetlinewidth{0.301125pt}%
\definecolor{currentstroke}{rgb}{0.500000,0.500000,0.500000}%
\pgfsetstrokecolor{currentstroke}%
\pgfsetstrokeopacity{0.300000}%
\pgfsetdash{}{0pt}%
\pgfpathmoveto{\pgfqpoint{3.727238in}{2.532084in}}%
\pgfpathlineto{\pgfqpoint{3.724863in}{2.539118in}}%
\pgfpathlineto{\pgfqpoint{3.706653in}{2.604960in}}%
\pgfpathlineto{\pgfqpoint{3.695876in}{2.672420in}}%
\pgfpathlineto{\pgfqpoint{3.692666in}{2.740658in}}%
\pgfpathlineto{\pgfqpoint{3.696980in}{2.808831in}}%
\pgfpathlineto{\pgfqpoint{3.708616in}{2.876148in}}%
\pgfpathlineto{\pgfqpoint{3.727238in}{2.941885in}}%
\pgfpathlineto{\pgfqpoint{3.727238in}{2.941885in}}%
\pgfusepath{stroke}%
\end{pgfscope}%
\begin{pgfscope}%
\pgfpathrectangle{\pgfqpoint{0.647939in}{0.492442in}}{\pgfqpoint{3.079299in}{3.079299in}}%
\pgfusepath{clip}%
\pgfsetbuttcap%
\pgfsetroundjoin%
\pgfsetlinewidth{0.301125pt}%
\definecolor{currentstroke}{rgb}{0.500000,0.500000,0.500000}%
\pgfsetstrokecolor{currentstroke}%
\pgfsetstrokeopacity{0.300000}%
\pgfsetdash{}{0pt}%
\pgfpathmoveto{\pgfqpoint{0.647939in}{2.627863in}}%
\pgfpathlineto{\pgfqpoint{0.712750in}{2.636111in}}%
\pgfpathlineto{\pgfqpoint{0.780539in}{2.645425in}}%
\pgfpathlineto{\pgfqpoint{0.848128in}{2.656087in}}%
\pgfpathlineto{\pgfqpoint{0.915495in}{2.668070in}}%
\pgfpathlineto{\pgfqpoint{0.982628in}{2.681306in}}%
\pgfpathlineto{\pgfqpoint{1.049528in}{2.695680in}}%
\pgfpathlineto{\pgfqpoint{1.116211in}{2.711031in}}%
\pgfpathlineto{\pgfqpoint{1.182714in}{2.727148in}}%
\pgfpathlineto{\pgfqpoint{1.249090in}{2.743783in}}%
\pgfpathlineto{\pgfqpoint{1.315406in}{2.760655in}}%
\pgfpathlineto{\pgfqpoint{1.381741in}{2.777450in}}%
\pgfpathlineto{\pgfqpoint{1.448181in}{2.793827in}}%
\pgfpathlineto{\pgfqpoint{1.514806in}{2.809425in}}%
\pgfpathlineto{\pgfqpoint{1.581689in}{2.823872in}}%
\pgfpathlineto{\pgfqpoint{1.648880in}{2.836800in}}%
\pgfpathlineto{\pgfqpoint{1.716399in}{2.847867in}}%
\pgfpathlineto{\pgfqpoint{1.784233in}{2.856781in}}%
\pgfpathlineto{\pgfqpoint{1.852334in}{2.863330in}}%
\pgfpathlineto{\pgfqpoint{1.920629in}{2.867414in}}%
\pgfpathlineto{\pgfqpoint{1.989026in}{2.869084in}}%
\pgfpathlineto{\pgfqpoint{2.057445in}{2.868575in}}%
\pgfpathlineto{\pgfqpoint{2.125832in}{2.866320in}}%
\pgfpathlineto{\pgfqpoint{2.194177in}{2.862967in}}%
\pgfpathlineto{\pgfqpoint{2.262514in}{2.859415in}}%
\pgfpathlineto{\pgfqpoint{2.330889in}{2.856862in}}%
\pgfpathlineto{\pgfqpoint{2.399294in}{2.856819in}}%
\pgfpathlineto{\pgfqpoint{2.467535in}{2.861028in}}%
\pgfpathlineto{\pgfqpoint{2.535102in}{2.871199in}}%
\pgfpathlineto{\pgfqpoint{2.601173in}{2.888524in}}%
\pgfpathlineto{\pgfqpoint{2.664876in}{2.913190in}}%
\pgfpathlineto{\pgfqpoint{2.725685in}{2.944369in}}%
\pgfpathlineto{\pgfqpoint{2.783588in}{2.980705in}}%
\pgfpathlineto{\pgfqpoint{2.838946in}{3.020842in}}%
\pgfpathlineto{\pgfqpoint{2.892265in}{3.063676in}}%
\pgfpathlineto{\pgfqpoint{2.944049in}{3.108377in}}%
\pgfpathlineto{\pgfqpoint{2.994735in}{3.154326in}}%
\pgfpathlineto{\pgfqpoint{3.044692in}{3.201073in}}%
\pgfpathlineto{\pgfqpoint{3.094231in}{3.248270in}}%
\pgfpathlineto{\pgfqpoint{3.143608in}{3.295634in}}%
\pgfpathlineto{\pgfqpoint{3.193044in}{3.342940in}}%
\pgfpathlineto{\pgfqpoint{3.242734in}{3.389980in}}%
\pgfpathlineto{\pgfqpoint{3.292848in}{3.436569in}}%
\pgfpathlineto{\pgfqpoint{3.343548in}{3.482519in}}%
\pgfpathlineto{\pgfqpoint{3.394988in}{3.527638in}}%
\pgfpathlineto{\pgfqpoint{3.447302in}{3.571741in}}%
\pgfpathlineto{\pgfqpoint{3.447302in}{3.571741in}}%
\pgfusepath{stroke}%
\end{pgfscope}%
\begin{pgfscope}%
\pgfpathrectangle{\pgfqpoint{0.647939in}{0.492442in}}{\pgfqpoint{3.079299in}{3.079299in}}%
\pgfusepath{clip}%
\pgfsetbuttcap%
\pgfsetroundjoin%
\pgfsetlinewidth{0.301125pt}%
\definecolor{currentstroke}{rgb}{0.500000,0.500000,0.500000}%
\pgfsetstrokecolor{currentstroke}%
\pgfsetstrokeopacity{0.300000}%
\pgfsetdash{}{0pt}%
\pgfpathmoveto{\pgfqpoint{0.647939in}{2.912986in}}%
\pgfpathlineto{\pgfqpoint{0.652614in}{2.913474in}}%
\pgfpathlineto{\pgfqpoint{0.720603in}{2.921187in}}%
\pgfpathlineto{\pgfqpoint{0.788442in}{2.930129in}}%
\pgfpathlineto{\pgfqpoint{0.856107in}{2.940299in}}%
\pgfpathlineto{\pgfqpoint{0.923585in}{2.951653in}}%
\pgfpathlineto{\pgfqpoint{0.990868in}{2.964109in}}%
\pgfpathlineto{\pgfqpoint{1.057966in}{2.977532in}}%
\pgfpathlineto{\pgfqpoint{1.124900in}{2.991752in}}%
\pgfpathlineto{\pgfqpoint{1.191707in}{3.006559in}}%
\pgfpathlineto{\pgfqpoint{1.258437in}{3.021711in}}%
\pgfpathlineto{\pgfqpoint{1.325151in}{3.036933in}}%
\pgfpathlineto{\pgfqpoint{1.391917in}{3.051926in}}%
\pgfpathlineto{\pgfqpoint{1.458803in}{3.066371in}}%
\pgfpathlineto{\pgfqpoint{1.525871in}{3.079940in}}%
\pgfpathlineto{\pgfqpoint{1.593168in}{3.092310in}}%
\pgfpathlineto{\pgfqpoint{1.660722in}{3.103179in}}%
\pgfpathlineto{\pgfqpoint{1.728535in}{3.112283in}}%
\pgfpathlineto{\pgfqpoint{1.796582in}{3.119420in}}%
\pgfpathlineto{\pgfqpoint{1.864815in}{3.124475in}}%
\pgfpathlineto{\pgfqpoint{1.933170in}{3.127446in}}%
\pgfpathlineto{\pgfqpoint{2.001584in}{3.128461in}}%
\pgfpathlineto{\pgfqpoint{2.070005in}{3.127787in}}%
\pgfpathlineto{\pgfqpoint{2.138404in}{3.125844in}}%
\pgfpathlineto{\pgfqpoint{2.206782in}{3.123218in}}%
\pgfpathlineto{\pgfqpoint{2.275162in}{3.120660in}}%
\pgfpathlineto{\pgfqpoint{2.343568in}{3.119074in}}%
\pgfpathlineto{\pgfqpoint{2.411982in}{3.119491in}}%
\pgfpathlineto{\pgfqpoint{2.480292in}{3.123026in}}%
\pgfpathlineto{\pgfqpoint{2.548239in}{3.130760in}}%
\pgfpathlineto{\pgfqpoint{2.615399in}{3.143553in}}%
\pgfpathlineto{\pgfqpoint{2.681268in}{3.161825in}}%
\pgfpathlineto{\pgfqpoint{2.745406in}{3.185473in}}%
\pgfpathlineto{\pgfqpoint{2.807558in}{3.213958in}}%
\pgfpathlineto{\pgfqpoint{2.867704in}{3.246497in}}%
\pgfpathlineto{\pgfqpoint{2.926008in}{3.282258in}}%
\pgfpathlineto{\pgfqpoint{2.982745in}{3.320480in}}%
\pgfpathlineto{\pgfqpoint{3.038232in}{3.360508in}}%
\pgfpathlineto{\pgfqpoint{3.092789in}{3.401798in}}%
\pgfpathlineto{\pgfqpoint{3.146717in}{3.443909in}}%
\pgfpathlineto{\pgfqpoint{3.200287in}{3.486479in}}%
\pgfpathlineto{\pgfqpoint{3.253750in}{3.529186in}}%
\pgfpathlineto{\pgfqpoint{3.307334in}{3.571741in}}%
\pgfpathlineto{\pgfqpoint{3.307334in}{3.571741in}}%
\pgfusepath{stroke}%
\end{pgfscope}%
\begin{pgfscope}%
\pgfpathrectangle{\pgfqpoint{0.647939in}{0.492442in}}{\pgfqpoint{3.079299in}{3.079299in}}%
\pgfusepath{clip}%
\pgfsetbuttcap%
\pgfsetroundjoin%
\pgfsetlinewidth{0.301125pt}%
\definecolor{currentstroke}{rgb}{0.500000,0.500000,0.500000}%
\pgfsetstrokecolor{currentstroke}%
\pgfsetstrokeopacity{0.300000}%
\pgfsetdash{}{0pt}%
\pgfpathmoveto{\pgfqpoint{0.647939in}{3.088280in}}%
\pgfpathlineto{\pgfqpoint{0.653013in}{3.088794in}}%
\pgfpathlineto{\pgfqpoint{0.721029in}{3.096277in}}%
\pgfpathlineto{\pgfqpoint{0.788905in}{3.104931in}}%
\pgfpathlineto{\pgfqpoint{0.856623in}{3.114745in}}%
\pgfpathlineto{\pgfqpoint{0.924171in}{3.125673in}}%
\pgfpathlineto{\pgfqpoint{0.991547in}{3.137624in}}%
\pgfpathlineto{\pgfqpoint{1.058759in}{3.150464in}}%
\pgfpathlineto{\pgfqpoint{1.125831in}{3.164019in}}%
\pgfpathlineto{\pgfqpoint{1.192798in}{3.178085in}}%
\pgfpathlineto{\pgfqpoint{1.259708in}{3.192422in}}%
\pgfpathlineto{\pgfqpoint{1.326617in}{3.206765in}}%
\pgfpathlineto{\pgfqpoint{1.393585in}{3.220827in}}%
\pgfpathlineto{\pgfqpoint{1.460672in}{3.234305in}}%
\pgfpathlineto{\pgfqpoint{1.527931in}{3.246891in}}%
\pgfpathlineto{\pgfqpoint{1.595400in}{3.258291in}}%
\pgfpathlineto{\pgfqpoint{1.663096in}{3.268237in}}%
\pgfpathlineto{\pgfqpoint{1.731017in}{3.276500in}}%
\pgfpathlineto{\pgfqpoint{1.799138in}{3.282916in}}%
\pgfpathlineto{\pgfqpoint{1.867412in}{3.287408in}}%
\pgfpathlineto{\pgfqpoint{1.935784in}{3.290005in}}%
\pgfpathlineto{\pgfqpoint{2.004202in}{3.290849in}}%
\pgfpathlineto{\pgfqpoint{2.072624in}{3.290205in}}%
\pgfpathlineto{\pgfqpoint{2.141029in}{3.288470in}}%
\pgfpathlineto{\pgfqpoint{2.209419in}{3.286182in}}%
\pgfpathlineto{\pgfqpoint{2.277813in}{3.284004in}}%
\pgfpathlineto{\pgfqpoint{2.346226in}{3.282702in}}%
\pgfpathlineto{\pgfqpoint{2.414643in}{3.283131in}}%
\pgfpathlineto{\pgfqpoint{2.482986in}{3.286194in}}%
\pgfpathlineto{\pgfqpoint{2.551067in}{3.292768in}}%
\pgfpathlineto{\pgfqpoint{2.618593in}{3.303573in}}%
\pgfpathlineto{\pgfqpoint{2.685198in}{3.319032in}}%
\pgfpathlineto{\pgfqpoint{2.750528in}{3.339211in}}%
\pgfpathlineto{\pgfqpoint{2.814324in}{3.363830in}}%
\pgfpathlineto{\pgfqpoint{2.876478in}{3.392363in}}%
\pgfpathlineto{\pgfqpoint{2.937037in}{3.424164in}}%
\pgfpathlineto{\pgfqpoint{2.996161in}{3.458573in}}%
\pgfpathlineto{\pgfqpoint{3.054082in}{3.494982in}}%
\pgfpathlineto{\pgfqpoint{3.111062in}{3.532858in}}%
\pgfpathlineto{\pgfqpoint{3.167366in}{3.571741in}}%
\pgfpathlineto{\pgfqpoint{3.167366in}{3.571741in}}%
\pgfusepath{stroke}%
\end{pgfscope}%
\begin{pgfscope}%
\pgfpathrectangle{\pgfqpoint{0.647939in}{0.492442in}}{\pgfqpoint{3.079299in}{3.079299in}}%
\pgfusepath{clip}%
\pgfsetbuttcap%
\pgfsetroundjoin%
\pgfsetlinewidth{0.301125pt}%
\definecolor{currentstroke}{rgb}{0.500000,0.500000,0.500000}%
\pgfsetstrokecolor{currentstroke}%
\pgfsetstrokeopacity{0.300000}%
\pgfsetdash{}{0pt}%
\pgfpathmoveto{\pgfqpoint{0.647939in}{3.209984in}}%
\pgfpathlineto{\pgfqpoint{0.680712in}{3.213461in}}%
\pgfpathlineto{\pgfqpoint{0.748694in}{3.221250in}}%
\pgfpathlineto{\pgfqpoint{0.816536in}{3.230169in}}%
\pgfpathlineto{\pgfqpoint{0.884225in}{3.240188in}}%
\pgfpathlineto{\pgfqpoint{0.951753in}{3.251240in}}%
\pgfpathlineto{\pgfqpoint{1.019124in}{3.263219in}}%
\pgfpathlineto{\pgfqpoint{1.086352in}{3.275974in}}%
\pgfpathlineto{\pgfqpoint{1.153467in}{3.289317in}}%
\pgfpathlineto{\pgfqpoint{1.220508in}{3.303030in}}%
\pgfpathlineto{\pgfqpoint{1.287523in}{3.316867in}}%
\pgfpathlineto{\pgfqpoint{1.354568in}{3.330560in}}%
\pgfpathlineto{\pgfqpoint{1.421698in}{3.343825in}}%
\pgfpathlineto{\pgfqpoint{1.488965in}{3.356372in}}%
\pgfpathlineto{\pgfqpoint{1.556411in}{3.367910in}}%
\pgfpathlineto{\pgfqpoint{1.624062in}{3.378172in}}%
\pgfpathlineto{\pgfqpoint{1.691923in}{3.386924in}}%
\pgfpathlineto{\pgfqpoint{1.759980in}{3.393983in}}%
\pgfpathlineto{\pgfqpoint{1.828200in}{3.399240in}}%
\pgfpathlineto{\pgfqpoint{1.896536in}{3.402674in}}%
\pgfpathlineto{\pgfqpoint{1.964938in}{3.404372in}}%
\pgfpathlineto{\pgfqpoint{2.033362in}{3.404531in}}%
\pgfpathlineto{\pgfqpoint{2.101781in}{3.403462in}}%
\pgfpathlineto{\pgfqpoint{2.170183in}{3.401591in}}%
\pgfpathlineto{\pgfqpoint{2.238579in}{3.399462in}}%
\pgfpathlineto{\pgfqpoint{2.306985in}{3.397729in}}%
\pgfpathlineto{\pgfqpoint{2.375406in}{3.397119in}}%
\pgfpathlineto{\pgfqpoint{2.443811in}{3.398416in}}%
\pgfpathlineto{\pgfqpoint{2.512104in}{3.402409in}}%
\pgfpathlineto{\pgfqpoint{2.580102in}{3.409825in}}%
\pgfpathlineto{\pgfqpoint{2.647537in}{3.421224in}}%
\pgfpathlineto{\pgfqpoint{2.714099in}{3.436906in}}%
\pgfpathlineto{\pgfqpoint{2.779506in}{3.456868in}}%
\pgfpathlineto{\pgfqpoint{2.843558in}{3.480839in}}%
\pgfpathlineto{\pgfqpoint{2.906178in}{3.508353in}}%
\pgfpathlineto{\pgfqpoint{2.967412in}{3.538847in}}%
\pgfpathlineto{\pgfqpoint{3.027398in}{3.571741in}}%
\pgfpathlineto{\pgfqpoint{3.027398in}{3.571741in}}%
\pgfusepath{stroke}%
\end{pgfscope}%
\begin{pgfscope}%
\pgfpathrectangle{\pgfqpoint{0.647939in}{0.492442in}}{\pgfqpoint{3.079299in}{3.079299in}}%
\pgfusepath{clip}%
\pgfsetbuttcap%
\pgfsetroundjoin%
\pgfsetlinewidth{0.301125pt}%
\definecolor{currentstroke}{rgb}{0.500000,0.500000,0.500000}%
\pgfsetstrokecolor{currentstroke}%
\pgfsetstrokeopacity{0.300000}%
\pgfsetdash{}{0pt}%
\pgfpathmoveto{\pgfqpoint{0.647939in}{3.294182in}}%
\pgfpathlineto{\pgfqpoint{0.657368in}{3.295115in}}%
\pgfpathlineto{\pgfqpoint{0.725404in}{3.302408in}}%
\pgfpathlineto{\pgfqpoint{0.793313in}{3.310809in}}%
\pgfpathlineto{\pgfqpoint{0.861078in}{3.320299in}}%
\pgfpathlineto{\pgfqpoint{0.928691in}{3.330822in}}%
\pgfpathlineto{\pgfqpoint{0.996152in}{3.342278in}}%
\pgfpathlineto{\pgfqpoint{1.063474in}{3.354531in}}%
\pgfpathlineto{\pgfqpoint{1.130680in}{3.367408in}}%
\pgfpathlineto{\pgfqpoint{1.197804in}{3.380706in}}%
\pgfpathlineto{\pgfqpoint{1.264891in}{3.394191in}}%
\pgfpathlineto{\pgfqpoint{1.331991in}{3.407607in}}%
\pgfpathlineto{\pgfqpoint{1.399160in}{3.420676in}}%
\pgfpathlineto{\pgfqpoint{1.466448in}{3.433114in}}%
\pgfpathlineto{\pgfqpoint{1.533896in}{3.444642in}}%
\pgfpathlineto{\pgfqpoint{1.601533in}{3.454996in}}%
\pgfpathlineto{\pgfqpoint{1.669370in}{3.463940in}}%
\pgfpathlineto{\pgfqpoint{1.737398in}{3.471287in}}%
\pgfpathlineto{\pgfqpoint{1.805589in}{3.476912in}}%
\pgfpathlineto{\pgfqpoint{1.873902in}{3.480778in}}%
\pgfpathlineto{\pgfqpoint{1.942291in}{3.482936in}}%
\pgfpathlineto{\pgfqpoint{2.010713in}{3.483545in}}%
\pgfpathlineto{\pgfqpoint{2.079136in}{3.482869in}}%
\pgfpathlineto{\pgfqpoint{2.147545in}{3.481294in}}%
\pgfpathlineto{\pgfqpoint{2.215945in}{3.479309in}}%
\pgfpathlineto{\pgfqpoint{2.284350in}{3.477488in}}%
\pgfpathlineto{\pgfqpoint{2.352768in}{3.476485in}}%
\pgfpathlineto{\pgfqpoint{2.421188in}{3.477015in}}%
\pgfpathlineto{\pgfqpoint{2.489545in}{3.479822in}}%
\pgfpathlineto{\pgfqpoint{2.557705in}{3.485610in}}%
\pgfpathlineto{\pgfqpoint{2.625461in}{3.494953in}}%
\pgfpathlineto{\pgfqpoint{2.692551in}{3.508232in}}%
\pgfpathlineto{\pgfqpoint{2.758708in}{3.525577in}}%
\pgfpathlineto{\pgfqpoint{2.823710in}{3.546860in}}%
\pgfpathlineto{\pgfqpoint{2.887429in}{3.571741in}}%
\pgfpathlineto{\pgfqpoint{2.887429in}{3.571741in}}%
\pgfusepath{stroke}%
\end{pgfscope}%
\begin{pgfscope}%
\pgfpathrectangle{\pgfqpoint{0.647939in}{0.492442in}}{\pgfqpoint{3.079299in}{3.079299in}}%
\pgfusepath{clip}%
\pgfsetbuttcap%
\pgfsetroundjoin%
\pgfsetlinewidth{0.301125pt}%
\definecolor{currentstroke}{rgb}{0.500000,0.500000,0.500000}%
\pgfsetstrokecolor{currentstroke}%
\pgfsetstrokeopacity{0.300000}%
\pgfsetdash{}{0pt}%
\pgfpathmoveto{\pgfqpoint{0.647939in}{3.366142in}}%
\pgfpathlineto{\pgfqpoint{0.709177in}{3.372885in}}%
\pgfpathlineto{\pgfqpoint{0.777130in}{3.380920in}}%
\pgfpathlineto{\pgfqpoint{0.844948in}{3.390027in}}%
\pgfpathlineto{\pgfqpoint{0.912621in}{3.400156in}}%
\pgfpathlineto{\pgfqpoint{0.980148in}{3.411219in}}%
\pgfpathlineto{\pgfqpoint{1.047537in}{3.423092in}}%
\pgfpathlineto{\pgfqpoint{1.114810in}{3.435613in}}%
\pgfpathlineto{\pgfqpoint{1.181999in}{3.448582in}}%
\pgfpathlineto{\pgfqpoint{1.249145in}{3.461771in}}%
\pgfpathlineto{\pgfqpoint{1.316296in}{3.474930in}}%
\pgfpathlineto{\pgfqpoint{1.383504in}{3.487796in}}%
\pgfpathlineto{\pgfqpoint{1.450818in}{3.500094in}}%
\pgfpathlineto{\pgfqpoint{1.518279in}{3.511550in}}%
\pgfpathlineto{\pgfqpoint{1.585917in}{3.521903in}}%
\pgfpathlineto{\pgfqpoint{1.653745in}{3.530917in}}%
\pgfpathlineto{\pgfqpoint{1.721757in}{3.538404in}}%
\pgfpathlineto{\pgfqpoint{1.789931in}{3.544233in}}%
\pgfpathlineto{\pgfqpoint{1.858230in}{3.548350in}}%
\pgfpathlineto{\pgfqpoint{1.926610in}{3.550792in}}%
\pgfpathlineto{\pgfqpoint{1.995029in}{3.551698in}}%
\pgfpathlineto{\pgfqpoint{2.063454in}{3.551311in}}%
\pgfpathlineto{\pgfqpoint{2.131868in}{3.549972in}}%
\pgfpathlineto{\pgfqpoint{2.200272in}{3.548120in}}%
\pgfpathlineto{\pgfqpoint{2.268676in}{3.546288in}}%
\pgfpathlineto{\pgfqpoint{2.337092in}{3.545095in}}%
\pgfpathlineto{\pgfqpoint{2.405515in}{3.545209in}}%
\pgfpathlineto{\pgfqpoint{2.473900in}{3.547316in}}%
\pgfpathlineto{\pgfqpoint{2.542144in}{3.552081in}}%
\pgfpathlineto{\pgfqpoint{2.610080in}{3.560082in}}%
\pgfpathlineto{\pgfqpoint{2.677477in}{3.571741in}}%
\pgfpathlineto{\pgfqpoint{2.677477in}{3.571741in}}%
\pgfusepath{stroke}%
\end{pgfscope}%
\begin{pgfscope}%
\pgfpathrectangle{\pgfqpoint{0.647939in}{0.492442in}}{\pgfqpoint{3.079299in}{3.079299in}}%
\pgfusepath{clip}%
\pgfsetbuttcap%
\pgfsetroundjoin%
\pgfsetlinewidth{0.301125pt}%
\definecolor{currentstroke}{rgb}{0.500000,0.500000,0.500000}%
\pgfsetstrokecolor{currentstroke}%
\pgfsetstrokeopacity{0.300000}%
\pgfsetdash{}{0pt}%
\pgfpathmoveto{\pgfqpoint{1.019186in}{3.473741in}}%
\pgfpathlineto{\pgfqpoint{1.086534in}{3.485847in}}%
\pgfpathlineto{\pgfqpoint{1.153789in}{3.498467in}}%
\pgfpathlineto{\pgfqpoint{1.220987in}{3.511389in}}%
\pgfpathlineto{\pgfqpoint{1.288172in}{3.524377in}}%
\pgfpathlineto{\pgfqpoint{1.355393in}{3.537174in}}%
\pgfpathlineto{\pgfqpoint{1.422699in}{3.549513in}}%
\pgfpathlineto{\pgfqpoint{1.490135in}{3.561123in}}%
\pgfpathlineto{\pgfqpoint{1.557732in}{3.571741in}}%
\pgfpathlineto{\pgfqpoint{1.557732in}{3.571741in}}%
\pgfusepath{stroke}%
\end{pgfscope}%
\begin{pgfscope}%
\pgfpathrectangle{\pgfqpoint{0.647939in}{0.492442in}}{\pgfqpoint{3.079299in}{3.079299in}}%
\pgfusepath{clip}%
\pgfsetbuttcap%
\pgfsetroundjoin%
\pgfsetlinewidth{0.301125pt}%
\definecolor{currentstroke}{rgb}{0.500000,0.500000,0.500000}%
\pgfsetstrokecolor{currentstroke}%
\pgfsetstrokeopacity{0.300000}%
\pgfsetdash{}{0pt}%
\pgfpathmoveto{\pgfqpoint{0.647939in}{3.500389in}}%
\pgfpathlineto{\pgfqpoint{0.664107in}{3.501961in}}%
\pgfpathlineto{\pgfqpoint{0.732159in}{3.509111in}}%
\pgfpathlineto{\pgfqpoint{0.800092in}{3.517311in}}%
\pgfpathlineto{\pgfqpoint{0.867895in}{3.526533in}}%
\pgfpathlineto{\pgfqpoint{0.935561in}{3.536708in}}%
\pgfpathlineto{\pgfqpoint{1.003095in}{3.547732in}}%
\pgfpathlineto{\pgfqpoint{1.070509in}{3.559467in}}%
\pgfpathlineto{\pgfqpoint{1.137828in}{3.571741in}}%
\pgfpathlineto{\pgfqpoint{1.137828in}{3.571741in}}%
\pgfusepath{stroke}%
\end{pgfscope}%
\begin{pgfscope}%
\pgfpathrectangle{\pgfqpoint{0.647939in}{0.492442in}}{\pgfqpoint{3.079299in}{3.079299in}}%
\pgfusepath{clip}%
\pgfsetbuttcap%
\pgfsetroundjoin%
\pgfsetlinewidth{0.301125pt}%
\definecolor{currentstroke}{rgb}{0.500000,0.500000,0.500000}%
\pgfsetstrokecolor{currentstroke}%
\pgfsetstrokeopacity{0.300000}%
\pgfsetdash{}{0pt}%
\pgfpathmoveto{\pgfqpoint{0.647939in}{3.011869in}}%
\pgfpathlineto{\pgfqpoint{0.647939in}{3.011869in}}%
\pgfpathlineto{\pgfqpoint{0.715953in}{3.019366in}}%
\pgfpathlineto{\pgfqpoint{0.783825in}{3.028058in}}%
\pgfpathlineto{\pgfqpoint{0.851533in}{3.037940in}}%
\pgfpathlineto{\pgfqpoint{0.919064in}{3.048969in}}%
\pgfpathlineto{\pgfqpoint{0.986414in}{3.061060in}}%
\pgfpathlineto{\pgfqpoint{1.053590in}{3.074088in}}%
\pgfpathlineto{\pgfqpoint{1.120613in}{3.087881in}}%
\pgfpathlineto{\pgfqpoint{1.187520in}{3.102231in}}%
\pgfpathlineto{\pgfqpoint{1.254358in}{3.116900in}}%
\pgfpathlineto{\pgfqpoint{1.321185in}{3.131620in}}%
\pgfpathlineto{\pgfqpoint{1.388064in}{3.146102in}}%
\pgfpathlineto{\pgfqpoint{1.455057in}{3.160036in}}%
\pgfpathlineto{\pgfqpoint{1.522224in}{3.173110in}}%
\pgfpathlineto{\pgfqpoint{1.589606in}{3.185012in}}%
\pgfpathlineto{\pgfqpoint{1.657227in}{3.195457in}}%
\pgfpathlineto{\pgfqpoint{1.725088in}{3.204200in}}%
\pgfpathlineto{\pgfqpoint{1.793164in}{3.211057in}}%
\pgfpathlineto{\pgfqpoint{1.861412in}{3.215923in}}%
\pgfpathlineto{\pgfqpoint{1.929772in}{3.218808in}}%
\pgfpathlineto{\pgfqpoint{1.998187in}{3.219845in}}%
\pgfpathlineto{\pgfqpoint{2.066609in}{3.219291in}}%
\pgfpathlineto{\pgfqpoint{2.135013in}{3.217544in}}%
\pgfpathlineto{\pgfqpoint{2.203400in}{3.215141in}}%
\pgfpathlineto{\pgfqpoint{2.271787in}{3.212771in}}%
\pgfpathlineto{\pgfqpoint{2.340196in}{3.211259in}}%
\pgfpathlineto{\pgfqpoint{2.408613in}{3.211522in}}%
\pgfpathlineto{\pgfqpoint{2.476952in}{3.214546in}}%
\pgfpathlineto{\pgfqpoint{2.545014in}{3.221286in}}%
\pgfpathlineto{\pgfqpoint{2.612462in}{3.232536in}}%
\pgfpathlineto{\pgfqpoint{2.678877in}{3.248759in}}%
\pgfpathlineto{\pgfqpoint{2.743861in}{3.269991in}}%
\pgfpathlineto{\pgfqpoint{2.807142in}{3.295878in}}%
\pgfpathlineto{\pgfqpoint{2.868628in}{3.325806in}}%
\pgfpathlineto{\pgfqpoint{2.928399in}{3.359055in}}%
\pgfpathlineto{\pgfqpoint{2.986652in}{3.394916in}}%
\pgfusepath{stroke}%
\end{pgfscope}%
\begin{pgfscope}%
\pgfpathrectangle{\pgfqpoint{0.647939in}{0.492442in}}{\pgfqpoint{3.079299in}{3.079299in}}%
\pgfusepath{clip}%
\pgfsetbuttcap%
\pgfsetroundjoin%
\pgfsetlinewidth{0.301125pt}%
\definecolor{currentstroke}{rgb}{0.500000,0.500000,0.500000}%
\pgfsetstrokecolor{currentstroke}%
\pgfsetstrokeopacity{0.300000}%
\pgfsetdash{}{0pt}%
\pgfpathmoveto{\pgfqpoint{0.647939in}{2.801916in}}%
\pgfpathlineto{\pgfqpoint{0.647939in}{2.801916in}}%
\pgfpathlineto{\pgfqpoint{0.715920in}{2.809703in}}%
\pgfpathlineto{\pgfqpoint{0.783744in}{2.818756in}}%
\pgfpathlineto{\pgfqpoint{0.851386in}{2.829082in}}%
\pgfpathlineto{\pgfqpoint{0.918827in}{2.840645in}}%
\pgfpathlineto{\pgfqpoint{0.986060in}{2.853367in}}%
\pgfpathlineto{\pgfqpoint{1.053090in}{2.867126in}}%
\pgfpathlineto{\pgfqpoint{1.119936in}{2.881753in}}%
\pgfpathlineto{\pgfqpoint{1.186635in}{2.897036in}}%
\pgfpathlineto{\pgfqpoint{1.253239in}{2.912732in}}%
\pgfpathlineto{\pgfqpoint{1.319811in}{2.928565in}}%
\pgfpathlineto{\pgfqpoint{1.386423in}{2.944229in}}%
\pgfpathlineto{\pgfqpoint{1.453148in}{2.959398in}}%
\pgfpathlineto{\pgfqpoint{1.520057in}{2.973732in}}%
\pgfpathlineto{\pgfqpoint{1.587206in}{2.986887in}}%
\pgfpathlineto{\pgfqpoint{1.654630in}{2.998535in}}%
\pgfpathlineto{\pgfqpoint{1.722337in}{3.008384in}}%
\pgfpathlineto{\pgfqpoint{1.790309in}{3.016199in}}%
\pgfpathlineto{\pgfqpoint{1.858495in}{3.021825in}}%
\pgfpathlineto{\pgfqpoint{1.926830in}{3.025227in}}%
\pgfpathlineto{\pgfqpoint{1.995239in}{3.026508in}}%
\pgfpathlineto{\pgfqpoint{2.063659in}{3.025926in}}%
\pgfpathlineto{\pgfqpoint{2.132055in}{3.023904in}}%
\pgfpathlineto{\pgfqpoint{2.200423in}{3.021042in}}%
\pgfpathlineto{\pgfqpoint{2.268790in}{3.018141in}}%
\pgfpathlineto{\pgfqpoint{2.337186in}{3.016207in}}%
\pgfpathlineto{\pgfqpoint{2.405598in}{3.016422in}}%
\pgfpathlineto{\pgfqpoint{2.473891in}{3.020088in}}%
\pgfpathlineto{\pgfqpoint{2.541744in}{3.028474in}}%
\pgfpathlineto{\pgfqpoint{2.608626in}{3.042565in}}%
\pgfpathlineto{\pgfqpoint{2.673911in}{3.062771in}}%
\pgfpathlineto{\pgfqpoint{2.737093in}{3.088823in}}%
\pgfpathlineto{\pgfqpoint{2.797957in}{3.119949in}}%
\pgfpathlineto{\pgfqpoint{2.856574in}{3.155167in}}%
\pgfusepath{stroke}%
\end{pgfscope}%
\begin{pgfscope}%
\pgfpathrectangle{\pgfqpoint{0.647939in}{0.492442in}}{\pgfqpoint{3.079299in}{3.079299in}}%
\pgfusepath{clip}%
\pgfsetbuttcap%
\pgfsetroundjoin%
\pgfsetlinewidth{0.301125pt}%
\definecolor{currentstroke}{rgb}{0.500000,0.500000,0.500000}%
\pgfsetstrokecolor{currentstroke}%
\pgfsetstrokeopacity{0.300000}%
\pgfsetdash{}{0pt}%
\pgfpathmoveto{\pgfqpoint{0.647939in}{2.731932in}}%
\pgfpathlineto{\pgfqpoint{0.647939in}{2.731932in}}%
\pgfpathlineto{\pgfqpoint{0.715909in}{2.739820in}}%
\pgfpathlineto{\pgfqpoint{0.783715in}{2.749000in}}%
\pgfpathlineto{\pgfqpoint{0.851332in}{2.759483in}}%
\pgfpathlineto{\pgfqpoint{0.918741in}{2.771235in}}%
\pgfpathlineto{\pgfqpoint{0.985931in}{2.784182in}}%
\pgfpathlineto{\pgfqpoint{1.052906in}{2.798203in}}%
\pgfpathlineto{\pgfqpoint{1.119685in}{2.813129in}}%
\pgfpathlineto{\pgfqpoint{1.186307in}{2.828749in}}%
\pgfpathlineto{\pgfqpoint{1.252822in}{2.844819in}}%
\pgfpathlineto{\pgfqpoint{1.319296in}{2.861057in}}%
\pgfpathlineto{\pgfqpoint{1.385804in}{2.877156in}}%
\pgfpathlineto{\pgfqpoint{1.452424in}{2.892782in}}%
\pgfusepath{stroke}%
\end{pgfscope}%
\begin{pgfscope}%
\pgfpathrectangle{\pgfqpoint{0.647939in}{0.492442in}}{\pgfqpoint{3.079299in}{3.079299in}}%
\pgfusepath{clip}%
\pgfsetbuttcap%
\pgfsetroundjoin%
\pgfsetlinewidth{0.301125pt}%
\definecolor{currentstroke}{rgb}{0.500000,0.500000,0.500000}%
\pgfsetstrokecolor{currentstroke}%
\pgfsetstrokeopacity{0.300000}%
\pgfsetdash{}{0pt}%
\pgfpathmoveto{\pgfqpoint{0.647939in}{2.521980in}}%
\pgfpathlineto{\pgfqpoint{0.647939in}{2.521980in}}%
\pgfpathlineto{\pgfqpoint{0.715870in}{2.530187in}}%
\pgfpathlineto{\pgfqpoint{0.783620in}{2.539772in}}%
\pgfpathlineto{\pgfqpoint{0.851158in}{2.550754in}}%
\pgfpathlineto{\pgfqpoint{0.918457in}{2.563113in}}%
\pgfpathlineto{\pgfqpoint{0.985503in}{2.576781in}}%
\pgfpathlineto{\pgfqpoint{1.052295in}{2.591648in}}%
\pgfpathlineto{\pgfqpoint{1.118849in}{2.607546in}}%
\pgfpathlineto{\pgfqpoint{1.185203in}{2.624265in}}%
\pgfpathlineto{\pgfqpoint{1.251410in}{2.641558in}}%
\pgfpathlineto{\pgfqpoint{1.317542in}{2.659138in}}%
\pgfpathlineto{\pgfqpoint{1.383683in}{2.676684in}}%
\pgfpathlineto{\pgfqpoint{1.449925in}{2.693844in}}%
\pgfpathlineto{\pgfqpoint{1.516357in}{2.710244in}}%
\pgfpathlineto{\pgfqpoint{1.583062in}{2.725493in}}%
\pgfpathlineto{\pgfqpoint{1.650098in}{2.739196in}}%
\pgfpathlineto{\pgfqpoint{1.717494in}{2.750982in}}%
\pgfpathlineto{\pgfqpoint{1.785242in}{2.760523in}}%
\pgfpathlineto{\pgfqpoint{1.853292in}{2.767567in}}%
\pgfpathlineto{\pgfqpoint{1.921564in}{2.771975in}}%
\pgfpathlineto{\pgfqpoint{1.989957in}{2.773770in}}%
\pgfpathlineto{\pgfqpoint{2.058373in}{2.773174in}}%
\pgfpathlineto{\pgfqpoint{2.126748in}{2.770624in}}%
\pgfpathlineto{\pgfqpoint{2.195069in}{2.766812in}}%
\pgfpathlineto{\pgfqpoint{2.263376in}{2.762738in}}%
\pgfpathlineto{\pgfqpoint{2.331734in}{2.759794in}}%
\pgfpathlineto{\pgfqpoint{2.400128in}{2.759817in}}%
\pgfpathlineto{\pgfqpoint{2.468271in}{2.764994in}}%
\pgfpathlineto{\pgfqpoint{2.535398in}{2.777434in}}%
\pgfpathlineto{\pgfqpoint{2.600356in}{2.798318in}}%
\pgfpathlineto{\pgfqpoint{2.662159in}{2.827278in}}%
\pgfpathlineto{\pgfqpoint{2.720501in}{2.862784in}}%
\pgfpathlineto{\pgfqpoint{2.775715in}{2.903042in}}%
\pgfpathlineto{\pgfqpoint{2.828435in}{2.946578in}}%
\pgfpathlineto{\pgfqpoint{2.879294in}{2.992313in}}%
\pgfpathlineto{\pgfqpoint{2.928826in}{3.039488in}}%
\pgfusepath{stroke}%
\end{pgfscope}%
\begin{pgfscope}%
\pgfpathrectangle{\pgfqpoint{0.647939in}{0.492442in}}{\pgfqpoint{3.079299in}{3.079299in}}%
\pgfusepath{clip}%
\pgfsetbuttcap%
\pgfsetroundjoin%
\pgfsetlinewidth{0.301125pt}%
\definecolor{currentstroke}{rgb}{0.500000,0.500000,0.500000}%
\pgfsetstrokecolor{currentstroke}%
\pgfsetstrokeopacity{0.300000}%
\pgfsetdash{}{0pt}%
\pgfpathmoveto{\pgfqpoint{0.647939in}{2.451996in}}%
\pgfpathlineto{\pgfqpoint{0.647939in}{2.451996in}}%
\pgfpathlineto{\pgfqpoint{0.715856in}{2.460315in}}%
\pgfpathlineto{\pgfqpoint{0.783586in}{2.470042in}}%
\pgfpathlineto{\pgfqpoint{0.851094in}{2.481202in}}%
\pgfpathlineto{\pgfqpoint{0.918353in}{2.493776in}}%
\pgfpathlineto{\pgfqpoint{0.985346in}{2.507703in}}%
\pgfpathlineto{\pgfqpoint{1.052069in}{2.522873in}}%
\pgfpathlineto{\pgfqpoint{1.118539in}{2.539122in}}%
\pgfpathlineto{\pgfqpoint{1.184790in}{2.556240in}}%
\pgfpathlineto{\pgfqpoint{1.250879in}{2.573979in}}%
\pgfpathlineto{\pgfqpoint{1.316878in}{2.592052in}}%
\pgfpathlineto{\pgfqpoint{1.382875in}{2.610134in}}%
\pgfusepath{stroke}%
\end{pgfscope}%
\begin{pgfscope}%
\pgfpathrectangle{\pgfqpoint{0.647939in}{0.492442in}}{\pgfqpoint{3.079299in}{3.079299in}}%
\pgfusepath{clip}%
\pgfsetbuttcap%
\pgfsetroundjoin%
\pgfsetlinewidth{0.301125pt}%
\definecolor{currentstroke}{rgb}{0.500000,0.500000,0.500000}%
\pgfsetstrokecolor{currentstroke}%
\pgfsetstrokeopacity{0.300000}%
\pgfsetdash{}{0pt}%
\pgfpathmoveto{\pgfqpoint{0.647939in}{2.382012in}}%
\pgfpathlineto{\pgfqpoint{0.647939in}{2.382012in}}%
\pgfpathlineto{\pgfqpoint{0.715842in}{2.390446in}}%
\pgfpathlineto{\pgfqpoint{0.783550in}{2.400320in}}%
\pgfpathlineto{\pgfqpoint{0.851027in}{2.411663in}}%
\pgfpathlineto{\pgfqpoint{0.918244in}{2.424460in}}%
\pgfpathlineto{\pgfqpoint{0.985180in}{2.438655in}}%
\pgfpathlineto{\pgfqpoint{1.051831in}{2.454140in}}%
\pgfpathlineto{\pgfqpoint{1.118210in}{2.470754in}}%
\pgfusepath{stroke}%
\end{pgfscope}%
\begin{pgfscope}%
\pgfpathrectangle{\pgfqpoint{0.647939in}{0.492442in}}{\pgfqpoint{3.079299in}{3.079299in}}%
\pgfusepath{clip}%
\pgfsetbuttcap%
\pgfsetroundjoin%
\pgfsetlinewidth{0.301125pt}%
\definecolor{currentstroke}{rgb}{0.500000,0.500000,0.500000}%
\pgfsetstrokecolor{currentstroke}%
\pgfsetstrokeopacity{0.300000}%
\pgfsetdash{}{0pt}%
\pgfpathmoveto{\pgfqpoint{0.647939in}{2.312028in}}%
\pgfpathlineto{\pgfqpoint{0.647939in}{2.312028in}}%
\pgfpathlineto{\pgfqpoint{0.715827in}{2.320581in}}%
\pgfpathlineto{\pgfqpoint{0.783513in}{2.330606in}}%
\pgfpathlineto{\pgfqpoint{0.850958in}{2.342137in}}%
\pgfpathlineto{\pgfqpoint{0.918130in}{2.355166in}}%
\pgfpathlineto{\pgfqpoint{0.985006in}{2.369638in}}%
\pgfpathlineto{\pgfqpoint{1.051579in}{2.385451in}}%
\pgfpathlineto{\pgfqpoint{1.117861in}{2.402446in}}%
\pgfpathlineto{\pgfqpoint{1.183887in}{2.420417in}}%
\pgfpathlineto{\pgfqpoint{1.249710in}{2.439116in}}%
\pgfpathlineto{\pgfqpoint{1.315408in}{2.458253in}}%
\pgfpathlineto{\pgfqpoint{1.381075in}{2.477498in}}%
\pgfpathlineto{\pgfqpoint{1.446816in}{2.496485in}}%
\pgfpathlineto{\pgfqpoint{1.512743in}{2.514814in}}%
\pgfpathlineto{\pgfqpoint{1.578959in}{2.532059in}}%
\pgfpathlineto{\pgfqpoint{1.645553in}{2.547774in}}%
\pgfpathlineto{\pgfqpoint{1.712578in}{2.561513in}}%
\pgfpathlineto{\pgfqpoint{1.780045in}{2.572856in}}%
\pgfpathlineto{\pgfqpoint{1.847913in}{2.581443in}}%
\pgfpathlineto{\pgfqpoint{1.916094in}{2.587014in}}%
\pgfpathlineto{\pgfqpoint{1.984460in}{2.589463in}}%
\pgfpathlineto{\pgfqpoint{2.052870in}{2.588910in}}%
\pgfpathlineto{\pgfqpoint{2.121215in}{2.585754in}}%
\pgfpathlineto{\pgfqpoint{2.189454in}{2.580723in}}%
\pgfpathlineto{\pgfqpoint{2.257644in}{2.575005in}}%
\pgfpathlineto{\pgfqpoint{2.325912in}{2.570473in}}%
\pgfpathlineto{\pgfqpoint{2.394268in}{2.570061in}}%
\pgfpathlineto{\pgfqpoint{2.462011in}{2.577814in}}%
\pgfpathlineto{\pgfqpoint{2.527156in}{2.597200in}}%
\pgfpathlineto{\pgfqpoint{2.586111in}{2.627170in}}%
\pgfpathlineto{\pgfqpoint{2.641823in}{2.666432in}}%
\pgfpathlineto{\pgfqpoint{2.693585in}{2.711004in}}%
\pgfpathlineto{\pgfqpoint{2.742568in}{2.758650in}}%
\pgfpathlineto{\pgfqpoint{2.789762in}{2.808121in}}%
\pgfpathlineto{\pgfqpoint{2.835824in}{2.858674in}}%
\pgfpathlineto{\pgfqpoint{2.881199in}{2.909855in}}%
\pgfusepath{stroke}%
\end{pgfscope}%
\begin{pgfscope}%
\pgfpathrectangle{\pgfqpoint{0.647939in}{0.492442in}}{\pgfqpoint{3.079299in}{3.079299in}}%
\pgfusepath{clip}%
\pgfsetbuttcap%
\pgfsetroundjoin%
\pgfsetlinewidth{0.301125pt}%
\definecolor{currentstroke}{rgb}{0.500000,0.500000,0.500000}%
\pgfsetstrokecolor{currentstroke}%
\pgfsetstrokeopacity{0.300000}%
\pgfsetdash{}{0pt}%
\pgfpathmoveto{\pgfqpoint{0.647939in}{2.242044in}}%
\pgfpathlineto{\pgfqpoint{0.647939in}{2.242044in}}%
\pgfpathlineto{\pgfqpoint{0.715811in}{2.250719in}}%
\pgfpathlineto{\pgfqpoint{0.783473in}{2.260900in}}%
\pgfpathlineto{\pgfqpoint{0.850885in}{2.272625in}}%
\pgfpathlineto{\pgfqpoint{0.918009in}{2.285893in}}%
\pgfpathlineto{\pgfqpoint{0.984822in}{2.300654in}}%
\pgfpathlineto{\pgfqpoint{1.051313in}{2.316808in}}%
\pgfpathlineto{\pgfqpoint{1.117492in}{2.334200in}}%
\pgfpathlineto{\pgfqpoint{1.183391in}{2.352628in}}%
\pgfpathlineto{\pgfqpoint{1.249066in}{2.371842in}}%
\pgfpathlineto{\pgfqpoint{1.314593in}{2.391554in}}%
\pgfpathlineto{\pgfqpoint{1.380071in}{2.411432in}}%
\pgfpathlineto{\pgfqpoint{1.445611in}{2.431105in}}%
\pgfpathlineto{\pgfqpoint{1.511329in}{2.450167in}}%
\pgfpathlineto{\pgfqpoint{1.577341in}{2.468180in}}%
\pgfpathlineto{\pgfqpoint{1.643744in}{2.484682in}}%
\pgfpathlineto{\pgfqpoint{1.710604in}{2.499201in}}%
\pgfpathlineto{\pgfqpoint{1.777943in}{2.511281in}}%
\pgfpathlineto{\pgfqpoint{1.845725in}{2.520516in}}%
\pgfpathlineto{\pgfqpoint{1.913859in}{2.526595in}}%
\pgfpathlineto{\pgfqpoint{1.982209in}{2.529351in}}%
\pgfpathlineto{\pgfqpoint{2.050617in}{2.528840in}}%
\pgfpathlineto{\pgfqpoint{2.118946in}{2.525417in}}%
\pgfpathlineto{\pgfqpoint{2.187137in}{2.519799in}}%
\pgfpathlineto{\pgfqpoint{2.255249in}{2.513223in}}%
\pgfpathlineto{\pgfqpoint{2.323448in}{2.507787in}}%
\pgfpathlineto{\pgfqpoint{2.391778in}{2.507089in}}%
\pgfpathlineto{\pgfqpoint{2.459156in}{2.516534in}}%
\pgfpathlineto{\pgfqpoint{2.522250in}{2.539929in}}%
\pgfpathlineto{\pgfqpoint{2.573114in}{2.570635in}}%
\pgfpathlineto{\pgfqpoint{2.624452in}{2.611839in}}%
\pgfpathlineto{\pgfqpoint{2.673848in}{2.658947in}}%
\pgfusepath{stroke}%
\end{pgfscope}%
\begin{pgfscope}%
\pgfpathrectangle{\pgfqpoint{0.647939in}{0.492442in}}{\pgfqpoint{3.079299in}{3.079299in}}%
\pgfusepath{clip}%
\pgfsetbuttcap%
\pgfsetroundjoin%
\pgfsetlinewidth{0.301125pt}%
\definecolor{currentstroke}{rgb}{0.500000,0.500000,0.500000}%
\pgfsetstrokecolor{currentstroke}%
\pgfsetstrokeopacity{0.300000}%
\pgfsetdash{}{0pt}%
\pgfpathmoveto{\pgfqpoint{0.647939in}{2.172060in}}%
\pgfpathlineto{\pgfqpoint{0.647939in}{2.172060in}}%
\pgfpathlineto{\pgfqpoint{0.715795in}{2.180860in}}%
\pgfpathlineto{\pgfqpoint{0.783433in}{2.191202in}}%
\pgfpathlineto{\pgfqpoint{0.850808in}{2.203129in}}%
\pgfpathlineto{\pgfqpoint{0.917883in}{2.216645in}}%
\pgfpathlineto{\pgfqpoint{0.984629in}{2.231704in}}%
\pgfpathlineto{\pgfqpoint{1.051032in}{2.248213in}}%
\pgfpathlineto{\pgfqpoint{1.117100in}{2.266021in}}%
\pgfpathlineto{\pgfqpoint{1.182863in}{2.284926in}}%
\pgfpathlineto{\pgfqpoint{1.248376in}{2.304683in}}%
\pgfpathlineto{\pgfqpoint{1.313719in}{2.325001in}}%
\pgfpathlineto{\pgfqpoint{1.378990in}{2.345548in}}%
\pgfpathlineto{\pgfqpoint{1.444306in}{2.365951in}}%
\pgfpathlineto{\pgfqpoint{1.509792in}{2.385797in}}%
\pgfpathlineto{\pgfqpoint{1.575572in}{2.404639in}}%
\pgfusepath{stroke}%
\end{pgfscope}%
\begin{pgfscope}%
\pgfpathrectangle{\pgfqpoint{0.647939in}{0.492442in}}{\pgfqpoint{3.079299in}{3.079299in}}%
\pgfusepath{clip}%
\pgfsetbuttcap%
\pgfsetroundjoin%
\pgfsetlinewidth{0.301125pt}%
\definecolor{currentstroke}{rgb}{0.500000,0.500000,0.500000}%
\pgfsetstrokecolor{currentstroke}%
\pgfsetstrokeopacity{0.300000}%
\pgfsetdash{}{0pt}%
\pgfpathmoveto{\pgfqpoint{0.647939in}{2.102076in}}%
\pgfpathlineto{\pgfqpoint{0.647939in}{2.102076in}}%
\pgfpathlineto{\pgfqpoint{0.715778in}{2.111005in}}%
\pgfpathlineto{\pgfqpoint{0.783390in}{2.121512in}}%
\pgfpathlineto{\pgfqpoint{0.850727in}{2.133648in}}%
\pgfpathlineto{\pgfqpoint{0.917750in}{2.147421in}}%
\pgfpathlineto{\pgfqpoint{0.984424in}{2.162791in}}%
\pgfpathlineto{\pgfqpoint{1.050734in}{2.179670in}}%
\pgfpathlineto{\pgfqpoint{1.116683in}{2.197912in}}%
\pgfpathlineto{\pgfqpoint{1.182300in}{2.217318in}}%
\pgfpathlineto{\pgfqpoint{1.247638in}{2.237644in}}%
\pgfpathlineto{\pgfqpoint{1.312778in}{2.258601in}}%
\pgfusepath{stroke}%
\end{pgfscope}%
\begin{pgfscope}%
\pgfpathrectangle{\pgfqpoint{0.647939in}{0.492442in}}{\pgfqpoint{3.079299in}{3.079299in}}%
\pgfusepath{clip}%
\pgfsetbuttcap%
\pgfsetroundjoin%
\pgfsetlinewidth{0.301125pt}%
\definecolor{currentstroke}{rgb}{0.500000,0.500000,0.500000}%
\pgfsetstrokecolor{currentstroke}%
\pgfsetstrokeopacity{0.300000}%
\pgfsetdash{}{0pt}%
\pgfpathmoveto{\pgfqpoint{0.647939in}{2.032092in}}%
\pgfpathlineto{\pgfqpoint{0.647939in}{2.032092in}}%
\pgfpathlineto{\pgfqpoint{0.715760in}{2.041154in}}%
\pgfpathlineto{\pgfqpoint{0.783345in}{2.051832in}}%
\pgfpathlineto{\pgfqpoint{0.850643in}{2.064183in}}%
\pgfpathlineto{\pgfqpoint{0.917609in}{2.078222in}}%
\pgfpathlineto{\pgfqpoint{0.984208in}{2.093917in}}%
\pgfpathlineto{\pgfqpoint{1.050418in}{2.111181in}}%
\pgfpathlineto{\pgfqpoint{1.116239in}{2.129877in}}%
\pgfpathlineto{\pgfqpoint{1.181698in}{2.149807in}}%
\pgfpathlineto{\pgfqpoint{1.246847in}{2.170732in}}%
\pgfpathlineto{\pgfqpoint{1.311766in}{2.192365in}}%
\pgfpathlineto{\pgfqpoint{1.376558in}{2.214374in}}%
\pgfpathlineto{\pgfqpoint{1.441351in}{2.236383in}}%
\pgfpathlineto{\pgfqpoint{1.506284in}{2.257971in}}%
\pgfpathlineto{\pgfqpoint{1.571502in}{2.278674in}}%
\pgfpathlineto{\pgfqpoint{1.637144in}{2.297982in}}%
\pgfpathlineto{\pgfqpoint{1.703321in}{2.315349in}}%
\pgfpathlineto{\pgfqpoint{1.770103in}{2.330199in}}%
\pgfpathlineto{\pgfqpoint{1.837488in}{2.341965in}}%
\pgfpathlineto{\pgfqpoint{1.905396in}{2.350121in}}%
\pgfpathlineto{\pgfqpoint{1.973662in}{2.354238in}}%
\pgfpathlineto{\pgfqpoint{2.042050in}{2.354047in}}%
\pgfpathlineto{\pgfqpoint{2.110292in}{2.349515in}}%
\pgfpathlineto{\pgfqpoint{2.178160in}{2.340988in}}%
\pgfpathlineto{\pgfqpoint{2.245582in}{2.329375in}}%
\pgfpathlineto{\pgfqpoint{2.312863in}{2.316984in}}%
\pgfpathlineto{\pgfqpoint{2.380402in}{2.313703in}}%
\pgfpathlineto{\pgfqpoint{2.380402in}{2.313703in}}%
\pgfpathlineto{\pgfqpoint{2.410003in}{2.321032in}}%
\pgfpathlineto{\pgfqpoint{2.438398in}{2.338507in}}%
\pgfpathlineto{\pgfqpoint{2.464050in}{2.361279in}}%
\pgfpathlineto{\pgfqpoint{2.497952in}{2.397795in}}%
\pgfpathlineto{\pgfqpoint{2.541784in}{2.449472in}}%
\pgfpathlineto{\pgfqpoint{2.584883in}{2.502246in}}%
\pgfpathlineto{\pgfqpoint{2.627513in}{2.555425in}}%
\pgfpathlineto{\pgfqpoint{2.669982in}{2.608931in}}%
\pgfusepath{stroke}%
\end{pgfscope}%
\begin{pgfscope}%
\pgfpathrectangle{\pgfqpoint{0.647939in}{0.492442in}}{\pgfqpoint{3.079299in}{3.079299in}}%
\pgfusepath{clip}%
\pgfsetbuttcap%
\pgfsetroundjoin%
\pgfsetlinewidth{0.301125pt}%
\definecolor{currentstroke}{rgb}{0.500000,0.500000,0.500000}%
\pgfsetstrokecolor{currentstroke}%
\pgfsetstrokeopacity{0.300000}%
\pgfsetdash{}{0pt}%
\pgfpathmoveto{\pgfqpoint{0.647939in}{1.962108in}}%
\pgfpathlineto{\pgfqpoint{0.647939in}{1.962108in}}%
\pgfpathlineto{\pgfqpoint{0.715741in}{1.971307in}}%
\pgfpathlineto{\pgfqpoint{0.783298in}{1.982161in}}%
\pgfpathlineto{\pgfqpoint{0.850554in}{1.994736in}}%
\pgfpathlineto{\pgfqpoint{0.917461in}{2.009051in}}%
\pgfpathlineto{\pgfqpoint{0.983979in}{2.025082in}}%
\pgfpathlineto{\pgfqpoint{1.050082in}{2.042749in}}%
\pgfpathlineto{\pgfqpoint{1.115767in}{2.061919in}}%
\pgfpathlineto{\pgfqpoint{1.181055in}{2.082401in}}%
\pgfpathlineto{\pgfqpoint{1.245997in}{2.103957in}}%
\pgfpathlineto{\pgfqpoint{1.310673in}{2.126303in}}%
\pgfpathlineto{\pgfqpoint{1.375189in}{2.149110in}}%
\pgfpathlineto{\pgfqpoint{1.439675in}{2.172002in}}%
\pgfpathlineto{\pgfqpoint{1.504279in}{2.194557in}}%
\pgfpathlineto{\pgfqpoint{1.569157in}{2.216305in}}%
\pgfpathlineto{\pgfqpoint{1.634461in}{2.236725in}}%
\pgfpathlineto{\pgfqpoint{1.700324in}{2.255247in}}%
\pgfpathlineto{\pgfqpoint{1.766836in}{2.271258in}}%
\pgfusepath{stroke}%
\end{pgfscope}%
\begin{pgfscope}%
\pgfpathrectangle{\pgfqpoint{0.647939in}{0.492442in}}{\pgfqpoint{3.079299in}{3.079299in}}%
\pgfusepath{clip}%
\pgfsetbuttcap%
\pgfsetroundjoin%
\pgfsetlinewidth{0.301125pt}%
\definecolor{currentstroke}{rgb}{0.500000,0.500000,0.500000}%
\pgfsetstrokecolor{currentstroke}%
\pgfsetstrokeopacity{0.300000}%
\pgfsetdash{}{0pt}%
\pgfpathmoveto{\pgfqpoint{0.647939in}{1.892124in}}%
\pgfpathlineto{\pgfqpoint{0.647939in}{1.892124in}}%
\pgfpathlineto{\pgfqpoint{0.715722in}{1.901464in}}%
\pgfpathlineto{\pgfqpoint{0.783248in}{1.912501in}}%
\pgfpathlineto{\pgfqpoint{0.850461in}{1.925306in}}%
\pgfpathlineto{\pgfqpoint{0.917305in}{1.939909in}}%
\pgfpathlineto{\pgfqpoint{0.983737in}{1.956290in}}%
\pgfpathlineto{\pgfqpoint{1.049726in}{1.974377in}}%
\pgfpathlineto{\pgfqpoint{1.115263in}{1.994044in}}%
\pgfpathlineto{\pgfqpoint{1.180366in}{2.015105in}}%
\pgfpathlineto{\pgfqpoint{1.245084in}{2.037325in}}%
\pgfpathlineto{\pgfqpoint{1.309494in}{2.060425in}}%
\pgfpathlineto{\pgfqpoint{1.373703in}{2.084079in}}%
\pgfpathlineto{\pgfqpoint{1.437847in}{2.107914in}}%
\pgfpathlineto{\pgfqpoint{1.502079in}{2.131508in}}%
\pgfpathlineto{\pgfqpoint{1.566567in}{2.154386in}}%
\pgfpathlineto{\pgfqpoint{1.631480in}{2.176021in}}%
\pgfpathlineto{\pgfqpoint{1.696969in}{2.195822in}}%
\pgfusepath{stroke}%
\end{pgfscope}%
\begin{pgfscope}%
\pgfpathrectangle{\pgfqpoint{0.647939in}{0.492442in}}{\pgfqpoint{3.079299in}{3.079299in}}%
\pgfusepath{clip}%
\pgfsetbuttcap%
\pgfsetroundjoin%
\pgfsetlinewidth{0.301125pt}%
\definecolor{currentstroke}{rgb}{0.500000,0.500000,0.500000}%
\pgfsetstrokecolor{currentstroke}%
\pgfsetstrokeopacity{0.300000}%
\pgfsetdash{}{0pt}%
\pgfpathmoveto{\pgfqpoint{0.647939in}{1.822139in}}%
\pgfpathlineto{\pgfqpoint{0.647939in}{1.822139in}}%
\pgfpathlineto{\pgfqpoint{0.715701in}{1.831625in}}%
\pgfpathlineto{\pgfqpoint{0.783196in}{1.842850in}}%
\pgfpathlineto{\pgfqpoint{0.850362in}{1.855895in}}%
\pgfpathlineto{\pgfqpoint{0.917140in}{1.870796in}}%
\pgfpathlineto{\pgfqpoint{0.983481in}{1.887542in}}%
\pgfpathlineto{\pgfqpoint{1.049347in}{1.906069in}}%
\pgfpathlineto{\pgfqpoint{1.114724in}{1.926256in}}%
\pgfpathlineto{\pgfqpoint{1.179627in}{1.947925in}}%
\pgfpathlineto{\pgfqpoint{1.244100in}{1.970846in}}%
\pgfpathlineto{\pgfqpoint{1.308217in}{1.994744in}}%
\pgfpathlineto{\pgfqpoint{1.372088in}{2.019298in}}%
\pgfpathlineto{\pgfqpoint{1.435849in}{2.044138in}}%
\pgfusepath{stroke}%
\end{pgfscope}%
\begin{pgfscope}%
\pgfpathrectangle{\pgfqpoint{0.647939in}{0.492442in}}{\pgfqpoint{3.079299in}{3.079299in}}%
\pgfusepath{clip}%
\pgfsetbuttcap%
\pgfsetroundjoin%
\pgfsetlinewidth{0.301125pt}%
\definecolor{currentstroke}{rgb}{0.500000,0.500000,0.500000}%
\pgfsetstrokecolor{currentstroke}%
\pgfsetstrokeopacity{0.300000}%
\pgfsetdash{}{0pt}%
\pgfpathmoveto{\pgfqpoint{0.647939in}{1.752155in}}%
\pgfpathlineto{\pgfqpoint{0.647939in}{1.752155in}}%
\pgfpathlineto{\pgfqpoint{0.715680in}{1.761791in}}%
\pgfpathlineto{\pgfqpoint{0.783142in}{1.773211in}}%
\pgfpathlineto{\pgfqpoint{0.850258in}{1.786504in}}%
\pgfpathlineto{\pgfqpoint{0.916966in}{1.801715in}}%
\pgfpathlineto{\pgfqpoint{0.983208in}{1.818842in}}%
\pgfpathlineto{\pgfqpoint{1.048943in}{1.837827in}}%
\pgfusepath{stroke}%
\end{pgfscope}%
\begin{pgfscope}%
\pgfpathrectangle{\pgfqpoint{0.647939in}{0.492442in}}{\pgfqpoint{3.079299in}{3.079299in}}%
\pgfusepath{clip}%
\pgfsetbuttcap%
\pgfsetroundjoin%
\pgfsetlinewidth{0.301125pt}%
\definecolor{currentstroke}{rgb}{0.500000,0.500000,0.500000}%
\pgfsetstrokecolor{currentstroke}%
\pgfsetstrokeopacity{0.300000}%
\pgfsetdash{}{0pt}%
\pgfpathmoveto{\pgfqpoint{0.647939in}{1.682171in}}%
\pgfpathlineto{\pgfqpoint{0.647939in}{1.682171in}}%
\pgfpathlineto{\pgfqpoint{0.715658in}{1.691962in}}%
\pgfpathlineto{\pgfqpoint{0.783084in}{1.703583in}}%
\pgfpathlineto{\pgfqpoint{0.850149in}{1.717134in}}%
\pgfpathlineto{\pgfqpoint{0.916782in}{1.732667in}}%
\pgfpathlineto{\pgfqpoint{0.982919in}{1.750191in}}%
\pgfpathlineto{\pgfqpoint{1.048512in}{1.769657in}}%
\pgfpathlineto{\pgfqpoint{1.113533in}{1.790961in}}%
\pgfpathlineto{\pgfqpoint{1.177981in}{1.813940in}}%
\pgfpathlineto{\pgfqpoint{1.241893in}{1.838380in}}%
\pgfpathlineto{\pgfqpoint{1.305334in}{1.864016in}}%
\pgfpathlineto{\pgfqpoint{1.368411in}{1.890542in}}%
\pgfpathlineto{\pgfqpoint{1.431262in}{1.917601in}}%
\pgfpathlineto{\pgfqpoint{1.494057in}{1.944788in}}%
\pgfpathlineto{\pgfqpoint{1.556994in}{1.971642in}}%
\pgfpathlineto{\pgfqpoint{1.620290in}{1.997635in}}%
\pgfpathlineto{\pgfqpoint{1.684166in}{2.022153in}}%
\pgfpathlineto{\pgfqpoint{1.748832in}{2.044480in}}%
\pgfpathlineto{\pgfqpoint{1.814456in}{2.063765in}}%
\pgfpathlineto{\pgfqpoint{1.881122in}{2.078974in}}%
\pgfpathlineto{\pgfqpoint{1.948760in}{2.088786in}}%
\pgfpathlineto{\pgfqpoint{2.017004in}{2.091270in}}%
\pgfpathlineto{\pgfqpoint{2.084455in}{2.082553in}}%
\pgfpathlineto{\pgfqpoint{2.084455in}{2.082553in}}%
\pgfpathlineto{\pgfqpoint{2.118677in}{2.069565in}}%
\pgfpathlineto{\pgfqpoint{2.118677in}{2.069565in}}%
\pgfpathlineto{\pgfqpoint{2.137566in}{2.053393in}}%
\pgfpathlineto{\pgfqpoint{2.137566in}{2.053393in}}%
\pgfpathlineto{\pgfqpoint{2.143995in}{2.035219in}}%
\pgfpathlineto{\pgfqpoint{2.141114in}{2.015522in}}%
\pgfusepath{stroke}%
\end{pgfscope}%
\begin{pgfscope}%
\pgfpathrectangle{\pgfqpoint{0.647939in}{0.492442in}}{\pgfqpoint{3.079299in}{3.079299in}}%
\pgfusepath{clip}%
\pgfsetbuttcap%
\pgfsetroundjoin%
\pgfsetlinewidth{0.301125pt}%
\definecolor{currentstroke}{rgb}{0.500000,0.500000,0.500000}%
\pgfsetstrokecolor{currentstroke}%
\pgfsetstrokeopacity{0.300000}%
\pgfsetdash{}{0pt}%
\pgfpathmoveto{\pgfqpoint{0.647939in}{1.542203in}}%
\pgfpathlineto{\pgfqpoint{0.647939in}{1.542203in}}%
\pgfpathlineto{\pgfqpoint{0.715609in}{1.552318in}}%
\pgfpathlineto{\pgfqpoint{0.782960in}{1.564365in}}%
\pgfpathlineto{\pgfqpoint{0.849911in}{1.578460in}}%
\pgfpathlineto{\pgfqpoint{0.916379in}{1.594679in}}%
\pgfpathlineto{\pgfqpoint{0.982284in}{1.613050in}}%
\pgfpathlineto{\pgfqpoint{1.047560in}{1.633545in}}%
\pgfpathlineto{\pgfqpoint{1.112163in}{1.656079in}}%
\pgfpathlineto{\pgfqpoint{1.176074in}{1.680510in}}%
\pgfpathlineto{\pgfqpoint{1.239311in}{1.706640in}}%
\pgfpathlineto{\pgfqpoint{1.301929in}{1.734225in}}%
\pgfpathlineto{\pgfqpoint{1.364022in}{1.762974in}}%
\pgfpathlineto{\pgfqpoint{1.425725in}{1.792554in}}%
\pgfpathlineto{\pgfqpoint{1.487211in}{1.822584in}}%
\pgfpathlineto{\pgfqpoint{1.548689in}{1.852632in}}%
\pgfpathlineto{\pgfqpoint{1.610396in}{1.882199in}}%
\pgfpathlineto{\pgfqpoint{1.672597in}{1.910705in}}%
\pgfpathlineto{\pgfqpoint{1.735568in}{1.937449in}}%
\pgfpathlineto{\pgfqpoint{1.799581in}{1.961559in}}%
\pgfpathlineto{\pgfqpoint{1.864871in}{1.981900in}}%
\pgfpathlineto{\pgfqpoint{1.931554in}{1.996833in}}%
\pgfpathlineto{\pgfqpoint{1.999417in}{2.003374in}}%
\pgfpathlineto{\pgfqpoint{1.999417in}{2.003374in}}%
\pgfpathlineto{\pgfqpoint{2.046961in}{1.998849in}}%
\pgfpathlineto{\pgfqpoint{2.046961in}{1.998849in}}%
\pgfpathlineto{\pgfqpoint{2.070757in}{1.989405in}}%
\pgfpathlineto{\pgfqpoint{2.070757in}{1.989405in}}%
\pgfpathlineto{\pgfqpoint{2.082244in}{1.976116in}}%
\pgfusepath{stroke}%
\end{pgfscope}%
\begin{pgfscope}%
\pgfpathrectangle{\pgfqpoint{0.647939in}{0.492442in}}{\pgfqpoint{3.079299in}{3.079299in}}%
\pgfusepath{clip}%
\pgfsetbuttcap%
\pgfsetroundjoin%
\pgfsetlinewidth{0.301125pt}%
\definecolor{currentstroke}{rgb}{0.500000,0.500000,0.500000}%
\pgfsetstrokecolor{currentstroke}%
\pgfsetstrokeopacity{0.300000}%
\pgfsetdash{}{0pt}%
\pgfpathmoveto{\pgfqpoint{0.647939in}{1.472219in}}%
\pgfpathlineto{\pgfqpoint{0.647939in}{1.472219in}}%
\pgfpathlineto{\pgfqpoint{0.715583in}{1.482504in}}%
\pgfpathlineto{\pgfqpoint{0.782893in}{1.494775in}}%
\pgfpathlineto{\pgfqpoint{0.849782in}{1.509159in}}%
\pgfpathlineto{\pgfqpoint{0.916158in}{1.525743in}}%
\pgfpathlineto{\pgfqpoint{0.981934in}{1.544567in}}%
\pgfpathlineto{\pgfqpoint{1.047034in}{1.565613in}}%
\pgfpathlineto{\pgfqpoint{1.111400in}{1.588809in}}%
\pgfpathlineto{\pgfqpoint{1.175005in}{1.614022in}}%
\pgfpathlineto{\pgfqpoint{1.237855in}{1.641066in}}%
\pgfpathlineto{\pgfqpoint{1.299996in}{1.669707in}}%
\pgfpathlineto{\pgfqpoint{1.361514in}{1.699664in}}%
\pgfpathlineto{\pgfqpoint{1.422539in}{1.730616in}}%
\pgfpathlineto{\pgfqpoint{1.483241in}{1.762200in}}%
\pgfusepath{stroke}%
\end{pgfscope}%
\begin{pgfscope}%
\pgfpathrectangle{\pgfqpoint{0.647939in}{0.492442in}}{\pgfqpoint{3.079299in}{3.079299in}}%
\pgfusepath{clip}%
\pgfsetbuttcap%
\pgfsetroundjoin%
\pgfsetlinewidth{0.301125pt}%
\definecolor{currentstroke}{rgb}{0.500000,0.500000,0.500000}%
\pgfsetstrokecolor{currentstroke}%
\pgfsetstrokeopacity{0.300000}%
\pgfsetdash{}{0pt}%
\pgfpathmoveto{\pgfqpoint{0.647939in}{1.402235in}}%
\pgfpathlineto{\pgfqpoint{0.647939in}{1.402235in}}%
\pgfpathlineto{\pgfqpoint{0.715556in}{1.412696in}}%
\pgfpathlineto{\pgfqpoint{0.782822in}{1.425199in}}%
\pgfpathlineto{\pgfqpoint{0.849645in}{1.439883in}}%
\pgfpathlineto{\pgfqpoint{0.915924in}{1.456848in}}%
\pgfpathlineto{\pgfqpoint{0.981562in}{1.476145in}}%
\pgfpathlineto{\pgfqpoint{1.046470in}{1.497771in}}%
\pgfpathlineto{\pgfqpoint{1.110579in}{1.521662in}}%
\pgfpathlineto{\pgfqpoint{1.173850in}{1.547698in}}%
\pgfpathlineto{\pgfqpoint{1.236275in}{1.575705in}}%
\pgfpathlineto{\pgfqpoint{1.297888in}{1.605458in}}%
\pgfpathlineto{\pgfqpoint{1.358767in}{1.636690in}}%
\pgfpathlineto{\pgfqpoint{1.419032in}{1.669094in}}%
\pgfpathlineto{\pgfqpoint{1.478846in}{1.702325in}}%
\pgfpathlineto{\pgfqpoint{1.538414in}{1.735998in}}%
\pgfpathlineto{\pgfqpoint{1.597982in}{1.769673in}}%
\pgfpathlineto{\pgfqpoint{1.657830in}{1.802842in}}%
\pgfpathlineto{\pgfqpoint{1.718278in}{1.834893in}}%
\pgfpathlineto{\pgfqpoint{1.779678in}{1.865061in}}%
\pgfpathlineto{\pgfqpoint{1.842407in}{1.892318in}}%
\pgfpathlineto{\pgfqpoint{1.906850in}{1.915103in}}%
\pgfpathlineto{\pgfqpoint{1.973328in}{1.930215in}}%
\pgfpathlineto{\pgfqpoint{1.973328in}{1.930215in}}%
\pgfpathlineto{\pgfqpoint{2.011565in}{1.932138in}}%
\pgfpathlineto{\pgfqpoint{2.011565in}{1.932138in}}%
\pgfusepath{stroke}%
\end{pgfscope}%
\begin{pgfscope}%
\pgfpathrectangle{\pgfqpoint{0.647939in}{0.492442in}}{\pgfqpoint{3.079299in}{3.079299in}}%
\pgfusepath{clip}%
\pgfsetbuttcap%
\pgfsetroundjoin%
\pgfsetlinewidth{0.301125pt}%
\definecolor{currentstroke}{rgb}{0.500000,0.500000,0.500000}%
\pgfsetstrokecolor{currentstroke}%
\pgfsetstrokeopacity{0.300000}%
\pgfsetdash{}{0pt}%
\pgfpathmoveto{\pgfqpoint{0.647939in}{1.332251in}}%
\pgfpathlineto{\pgfqpoint{0.647939in}{1.332251in}}%
\pgfpathlineto{\pgfqpoint{0.715527in}{1.342895in}}%
\pgfpathlineto{\pgfqpoint{0.782748in}{1.355638in}}%
\pgfpathlineto{\pgfqpoint{0.849500in}{1.370634in}}%
\pgfpathlineto{\pgfqpoint{0.915675in}{1.387996in}}%
\pgfpathlineto{\pgfqpoint{0.981163in}{1.407789in}}%
\pgfpathlineto{\pgfqpoint{1.045864in}{1.430022in}}%
\pgfpathlineto{\pgfqpoint{1.109694in}{1.454645in}}%
\pgfpathlineto{\pgfqpoint{1.172598in}{1.481550in}}%
\pgfusepath{stroke}%
\end{pgfscope}%
\begin{pgfscope}%
\pgfpathrectangle{\pgfqpoint{0.647939in}{0.492442in}}{\pgfqpoint{3.079299in}{3.079299in}}%
\pgfusepath{clip}%
\pgfsetbuttcap%
\pgfsetroundjoin%
\pgfsetlinewidth{0.301125pt}%
\definecolor{currentstroke}{rgb}{0.500000,0.500000,0.500000}%
\pgfsetstrokecolor{currentstroke}%
\pgfsetstrokeopacity{0.300000}%
\pgfsetdash{}{0pt}%
\pgfpathmoveto{\pgfqpoint{0.647939in}{1.262267in}}%
\pgfpathlineto{\pgfqpoint{0.647939in}{1.262267in}}%
\pgfpathlineto{\pgfqpoint{0.715496in}{1.273099in}}%
\pgfpathlineto{\pgfqpoint{0.782669in}{1.286093in}}%
\pgfpathlineto{\pgfqpoint{0.849346in}{1.301414in}}%
\pgfpathlineto{\pgfqpoint{0.915410in}{1.319191in}}%
\pgfpathlineto{\pgfqpoint{0.980737in}{1.339503in}}%
\pgfpathlineto{\pgfqpoint{1.045213in}{1.362374in}}%
\pgfpathlineto{\pgfqpoint{1.108738in}{1.387768in}}%
\pgfpathlineto{\pgfqpoint{1.171240in}{1.415587in}}%
\pgfpathlineto{\pgfqpoint{1.232681in}{1.445679in}}%
\pgfpathlineto{\pgfqpoint{1.293064in}{1.477844in}}%
\pgfpathlineto{\pgfqpoint{1.352437in}{1.511838in}}%
\pgfpathlineto{\pgfqpoint{1.410897in}{1.547385in}}%
\pgfpathlineto{\pgfqpoint{1.468581in}{1.584180in}}%
\pgfpathlineto{\pgfqpoint{1.525676in}{1.621885in}}%
\pgfpathlineto{\pgfqpoint{1.582411in}{1.660128in}}%
\pgfpathlineto{\pgfqpoint{1.639071in}{1.698483in}}%
\pgfpathlineto{\pgfqpoint{1.695989in}{1.736456in}}%
\pgfpathlineto{\pgfqpoint{1.753531in}{1.773463in}}%
\pgfpathlineto{\pgfqpoint{1.812118in}{1.808752in}}%
\pgfpathlineto{\pgfqpoint{1.872249in}{1.841265in}}%
\pgfusepath{stroke}%
\end{pgfscope}%
\begin{pgfscope}%
\pgfpathrectangle{\pgfqpoint{0.647939in}{0.492442in}}{\pgfqpoint{3.079299in}{3.079299in}}%
\pgfusepath{clip}%
\pgfsetbuttcap%
\pgfsetroundjoin%
\pgfsetlinewidth{0.301125pt}%
\definecolor{currentstroke}{rgb}{0.500000,0.500000,0.500000}%
\pgfsetstrokecolor{currentstroke}%
\pgfsetstrokeopacity{0.300000}%
\pgfsetdash{}{0pt}%
\pgfpathmoveto{\pgfqpoint{0.647939in}{1.192283in}}%
\pgfpathlineto{\pgfqpoint{0.647939in}{1.192283in}}%
\pgfpathlineto{\pgfqpoint{0.715464in}{1.203310in}}%
\pgfpathlineto{\pgfqpoint{0.782585in}{1.216564in}}%
\pgfpathlineto{\pgfqpoint{0.849183in}{1.232224in}}%
\pgfpathlineto{\pgfqpoint{0.915126in}{1.250435in}}%
\pgfpathlineto{\pgfqpoint{0.980280in}{1.271292in}}%
\pgfpathlineto{\pgfqpoint{1.044512in}{1.294833in}}%
\pgfpathlineto{\pgfqpoint{1.107705in}{1.321036in}}%
\pgfpathlineto{\pgfqpoint{1.169765in}{1.349819in}}%
\pgfpathlineto{\pgfqpoint{1.230637in}{1.381039in}}%
\pgfpathlineto{\pgfqpoint{1.290304in}{1.414505in}}%
\pgfpathlineto{\pgfqpoint{1.348796in}{1.449986in}}%
\pgfpathlineto{\pgfqpoint{1.406192in}{1.487219in}}%
\pgfpathlineto{\pgfqpoint{1.462616in}{1.525914in}}%
\pgfpathlineto{\pgfqpoint{1.518241in}{1.565753in}}%
\pgfusepath{stroke}%
\end{pgfscope}%
\begin{pgfscope}%
\pgfpathrectangle{\pgfqpoint{0.647939in}{0.492442in}}{\pgfqpoint{3.079299in}{3.079299in}}%
\pgfusepath{clip}%
\pgfsetbuttcap%
\pgfsetroundjoin%
\pgfsetlinewidth{0.301125pt}%
\definecolor{currentstroke}{rgb}{0.500000,0.500000,0.500000}%
\pgfsetstrokecolor{currentstroke}%
\pgfsetstrokeopacity{0.300000}%
\pgfsetdash{}{0pt}%
\pgfpathmoveto{\pgfqpoint{0.647939in}{1.122299in}}%
\pgfpathlineto{\pgfqpoint{0.647939in}{1.122299in}}%
\pgfpathlineto{\pgfqpoint{0.715430in}{1.133529in}}%
\pgfpathlineto{\pgfqpoint{0.782496in}{1.147052in}}%
\pgfpathlineto{\pgfqpoint{0.849009in}{1.163066in}}%
\pgfpathlineto{\pgfqpoint{0.914824in}{1.181731in}}%
\pgfpathlineto{\pgfqpoint{0.979790in}{1.203158in}}%
\pgfusepath{stroke}%
\end{pgfscope}%
\begin{pgfscope}%
\pgfpathrectangle{\pgfqpoint{0.647939in}{0.492442in}}{\pgfqpoint{3.079299in}{3.079299in}}%
\pgfusepath{clip}%
\pgfsetbuttcap%
\pgfsetroundjoin%
\pgfsetlinewidth{0.301125pt}%
\definecolor{currentstroke}{rgb}{0.500000,0.500000,0.500000}%
\pgfsetstrokecolor{currentstroke}%
\pgfsetstrokeopacity{0.300000}%
\pgfsetdash{}{0pt}%
\pgfpathmoveto{\pgfqpoint{0.647939in}{1.052315in}}%
\pgfpathlineto{\pgfqpoint{0.647939in}{1.052315in}}%
\pgfpathlineto{\pgfqpoint{0.715394in}{1.063754in}}%
\pgfpathlineto{\pgfqpoint{0.782402in}{1.077559in}}%
\pgfpathlineto{\pgfqpoint{0.848823in}{1.093942in}}%
\pgfpathlineto{\pgfqpoint{0.914500in}{1.113082in}}%
\pgfpathlineto{\pgfqpoint{0.979263in}{1.135108in}}%
\pgfpathlineto{\pgfqpoint{1.042941in}{1.160095in}}%
\pgfpathlineto{\pgfqpoint{1.105371in}{1.188050in}}%
\pgfpathlineto{\pgfqpoint{1.166415in}{1.218914in}}%
\pgfpathlineto{\pgfqpoint{1.225967in}{1.252565in}}%
\pgfpathlineto{\pgfqpoint{1.283965in}{1.288827in}}%
\pgfpathlineto{\pgfqpoint{1.340399in}{1.327482in}}%
\pgfpathlineto{\pgfqpoint{1.395313in}{1.368279in}}%
\pgfpathlineto{\pgfqpoint{1.448812in}{1.410915in}}%
\pgfpathlineto{\pgfqpoint{1.501063in}{1.455064in}}%
\pgfpathlineto{\pgfqpoint{1.552277in}{1.500416in}}%
\pgfpathlineto{\pgfqpoint{1.602729in}{1.546613in}}%
\pgfpathlineto{\pgfqpoint{1.652748in}{1.593262in}}%
\pgfpathlineto{\pgfqpoint{1.702714in}{1.639967in}}%
\pgfusepath{stroke}%
\end{pgfscope}%
\begin{pgfscope}%
\pgfpathrectangle{\pgfqpoint{0.647939in}{0.492442in}}{\pgfqpoint{3.079299in}{3.079299in}}%
\pgfusepath{clip}%
\pgfsetbuttcap%
\pgfsetroundjoin%
\pgfsetlinewidth{0.301125pt}%
\definecolor{currentstroke}{rgb}{0.500000,0.500000,0.500000}%
\pgfsetstrokecolor{currentstroke}%
\pgfsetstrokeopacity{0.300000}%
\pgfsetdash{}{0pt}%
\pgfpathmoveto{\pgfqpoint{0.647939in}{0.982331in}}%
\pgfpathlineto{\pgfqpoint{0.647939in}{0.982331in}}%
\pgfpathlineto{\pgfqpoint{0.715357in}{0.993988in}}%
\pgfpathlineto{\pgfqpoint{0.782302in}{1.008085in}}%
\pgfpathlineto{\pgfqpoint{0.848626in}{1.024854in}}%
\pgfpathlineto{\pgfqpoint{0.914153in}{1.044491in}}%
\pgfpathlineto{\pgfqpoint{0.978696in}{1.067147in}}%
\pgfpathlineto{\pgfqpoint{1.042059in}{1.092914in}}%
\pgfpathlineto{\pgfqpoint{1.104053in}{1.121814in}}%
\pgfpathlineto{\pgfqpoint{1.164511in}{1.153800in}}%
\pgfpathlineto{\pgfqpoint{1.223300in}{1.188756in}}%
\pgfpathlineto{\pgfqpoint{1.280333in}{1.226509in}}%
\pgfpathlineto{\pgfqpoint{1.335579in}{1.266838in}}%
\pgfpathlineto{\pgfqpoint{1.389072in}{1.309472in}}%
\pgfpathlineto{\pgfqpoint{1.440917in}{1.354087in}}%
\pgfpathlineto{\pgfqpoint{1.491275in}{1.400376in}}%
\pgfusepath{stroke}%
\end{pgfscope}%
\begin{pgfscope}%
\pgfpathrectangle{\pgfqpoint{0.647939in}{0.492442in}}{\pgfqpoint{3.079299in}{3.079299in}}%
\pgfusepath{clip}%
\pgfsetbuttcap%
\pgfsetroundjoin%
\pgfsetlinewidth{0.301125pt}%
\definecolor{currentstroke}{rgb}{0.500000,0.500000,0.500000}%
\pgfsetstrokecolor{currentstroke}%
\pgfsetstrokeopacity{0.300000}%
\pgfsetdash{}{0pt}%
\pgfpathmoveto{\pgfqpoint{0.647939in}{0.912347in}}%
\pgfpathlineto{\pgfqpoint{0.647939in}{0.912347in}}%
\pgfpathlineto{\pgfqpoint{0.715316in}{0.924230in}}%
\pgfpathlineto{\pgfqpoint{0.782196in}{0.938632in}}%
\pgfpathlineto{\pgfqpoint{0.848415in}{0.955803in}}%
\pgfpathlineto{\pgfqpoint{0.913781in}{0.975962in}}%
\pgfpathlineto{\pgfqpoint{0.978085in}{0.999280in}}%
\pgfpathlineto{\pgfqpoint{1.041104in}{1.025867in}}%
\pgfpathlineto{\pgfqpoint{1.102620in}{1.055762in}}%
\pgfusepath{stroke}%
\end{pgfscope}%
\begin{pgfscope}%
\pgfpathrectangle{\pgfqpoint{0.647939in}{0.492442in}}{\pgfqpoint{3.079299in}{3.079299in}}%
\pgfusepath{clip}%
\pgfsetbuttcap%
\pgfsetroundjoin%
\pgfsetlinewidth{0.301125pt}%
\definecolor{currentstroke}{rgb}{0.500000,0.500000,0.500000}%
\pgfsetstrokecolor{currentstroke}%
\pgfsetstrokeopacity{0.300000}%
\pgfsetdash{}{0pt}%
\pgfpathmoveto{\pgfqpoint{0.647939in}{0.842362in}}%
\pgfpathlineto{\pgfqpoint{0.647939in}{0.842362in}}%
\pgfpathlineto{\pgfqpoint{0.715274in}{0.854480in}}%
\pgfpathlineto{\pgfqpoint{0.782083in}{0.869200in}}%
\pgfpathlineto{\pgfqpoint{0.848189in}{0.886793in}}%
\pgfpathlineto{\pgfqpoint{0.913382in}{0.907499in}}%
\pgfpathlineto{\pgfqpoint{0.977426in}{0.931513in}}%
\pgfpathlineto{\pgfqpoint{1.040069in}{0.958964in}}%
\pgfpathlineto{\pgfqpoint{1.101059in}{0.989904in}}%
\pgfpathlineto{\pgfqpoint{1.160165in}{1.024303in}}%
\pgfpathlineto{\pgfqpoint{1.217191in}{1.062045in}}%
\pgfpathlineto{\pgfqpoint{1.272001in}{1.102943in}}%
\pgfpathlineto{\pgfqpoint{1.324544in}{1.146724in}}%
\pgfpathlineto{\pgfqpoint{1.374852in}{1.193042in}}%
\pgfpathlineto{\pgfqpoint{1.423046in}{1.241568in}}%
\pgfpathlineto{\pgfqpoint{1.469346in}{1.291899in}}%
\pgfpathlineto{\pgfqpoint{1.514044in}{1.343649in}}%
\pgfpathlineto{\pgfqpoint{1.557510in}{1.396443in}}%
\pgfusepath{stroke}%
\end{pgfscope}%
\begin{pgfscope}%
\pgfpathrectangle{\pgfqpoint{0.647939in}{0.492442in}}{\pgfqpoint{3.079299in}{3.079299in}}%
\pgfusepath{clip}%
\pgfsetbuttcap%
\pgfsetroundjoin%
\pgfsetlinewidth{0.301125pt}%
\definecolor{currentstroke}{rgb}{0.500000,0.500000,0.500000}%
\pgfsetstrokecolor{currentstroke}%
\pgfsetstrokeopacity{0.300000}%
\pgfsetdash{}{0pt}%
\pgfpathmoveto{\pgfqpoint{0.647939in}{0.772378in}}%
\pgfpathlineto{\pgfqpoint{0.647939in}{0.772378in}}%
\pgfpathlineto{\pgfqpoint{0.715229in}{0.784740in}}%
\pgfpathlineto{\pgfqpoint{0.781963in}{0.799791in}}%
\pgfpathlineto{\pgfqpoint{0.847948in}{0.817825in}}%
\pgfpathlineto{\pgfqpoint{0.912952in}{0.839107in}}%
\pgfpathlineto{\pgfqpoint{0.976713in}{0.863853in}}%
\pgfpathlineto{\pgfqpoint{1.038944in}{0.892212in}}%
\pgfpathlineto{\pgfqpoint{1.099358in}{0.924251in}}%
\pgfpathlineto{\pgfqpoint{1.157684in}{0.959941in}}%
\pgfpathlineto{\pgfqpoint{1.213699in}{0.999159in}}%
\pgfpathlineto{\pgfqpoint{1.267246in}{1.041693in}}%
\pgfpathlineto{\pgfqpoint{1.318276in}{1.087210in}}%
\pgfusepath{stroke}%
\end{pgfscope}%
\begin{pgfscope}%
\pgfpathrectangle{\pgfqpoint{0.647939in}{0.492442in}}{\pgfqpoint{3.079299in}{3.079299in}}%
\pgfusepath{clip}%
\pgfsetbuttcap%
\pgfsetroundjoin%
\pgfsetlinewidth{0.301125pt}%
\definecolor{currentstroke}{rgb}{0.500000,0.500000,0.500000}%
\pgfsetstrokecolor{currentstroke}%
\pgfsetstrokeopacity{0.300000}%
\pgfsetdash{}{0pt}%
\pgfpathmoveto{\pgfqpoint{0.647939in}{0.702394in}}%
\pgfpathlineto{\pgfqpoint{0.647939in}{0.702394in}}%
\pgfpathlineto{\pgfqpoint{0.715181in}{0.715010in}}%
\pgfpathlineto{\pgfqpoint{0.781834in}{0.730407in}}%
\pgfpathlineto{\pgfqpoint{0.847689in}{0.748903in}}%
\pgfpathlineto{\pgfqpoint{0.912489in}{0.770789in}}%
\pgfpathlineto{\pgfqpoint{0.975941in}{0.796306in}}%
\pgfpathlineto{\pgfqpoint{1.037721in}{0.825622in}}%
\pgfpathlineto{\pgfqpoint{1.097501in}{0.858813in}}%
\pgfpathlineto{\pgfqpoint{1.154973in}{0.895850in}}%
\pgfpathlineto{\pgfqpoint{1.209882in}{0.936593in}}%
\pgfpathlineto{\pgfqpoint{1.262066in}{0.980776in}}%
\pgfusepath{stroke}%
\end{pgfscope}%
\begin{pgfscope}%
\pgfpathrectangle{\pgfqpoint{0.647939in}{0.492442in}}{\pgfqpoint{3.079299in}{3.079299in}}%
\pgfusepath{clip}%
\pgfsetbuttcap%
\pgfsetroundjoin%
\pgfsetlinewidth{0.301125pt}%
\definecolor{currentstroke}{rgb}{0.500000,0.500000,0.500000}%
\pgfsetstrokecolor{currentstroke}%
\pgfsetstrokeopacity{0.300000}%
\pgfsetdash{}{0pt}%
\pgfpathmoveto{\pgfqpoint{0.647939in}{0.632410in}}%
\pgfpathlineto{\pgfqpoint{0.647939in}{0.632410in}}%
\pgfpathlineto{\pgfqpoint{0.715130in}{0.645290in}}%
\pgfpathlineto{\pgfqpoint{0.781697in}{0.661049in}}%
\pgfpathlineto{\pgfqpoint{0.847411in}{0.680030in}}%
\pgfpathlineto{\pgfqpoint{0.911990in}{0.702551in}}%
\pgfpathlineto{\pgfqpoint{0.975103in}{0.728879in}}%
\pgfpathlineto{\pgfqpoint{1.036389in}{0.759201in}}%
\pgfpathlineto{\pgfqpoint{1.095473in}{0.793600in}}%
\pgfpathlineto{\pgfqpoint{1.152009in}{0.832037in}}%
\pgfpathlineto{\pgfqpoint{1.205718in}{0.874349in}}%
\pgfpathlineto{\pgfqpoint{1.256437in}{0.920187in}}%
\pgfusepath{stroke}%
\end{pgfscope}%
\begin{pgfscope}%
\pgfpathrectangle{\pgfqpoint{0.647939in}{0.492442in}}{\pgfqpoint{3.079299in}{3.079299in}}%
\pgfusepath{clip}%
\pgfsetbuttcap%
\pgfsetroundjoin%
\pgfsetlinewidth{0.301125pt}%
\definecolor{currentstroke}{rgb}{0.500000,0.500000,0.500000}%
\pgfsetstrokecolor{currentstroke}%
\pgfsetstrokeopacity{0.300000}%
\pgfsetdash{}{0pt}%
\pgfpathmoveto{\pgfqpoint{0.647939in}{0.562426in}}%
\pgfpathlineto{\pgfqpoint{0.647939in}{0.562426in}}%
\pgfpathlineto{\pgfqpoint{0.715075in}{0.575581in}}%
\pgfpathlineto{\pgfqpoint{0.781550in}{0.591718in}}%
\pgfpathlineto{\pgfqpoint{0.847112in}{0.611208in}}%
\pgfpathlineto{\pgfqpoint{0.911450in}{0.634398in}}%
\pgfpathlineto{\pgfqpoint{0.974194in}{0.661580in}}%
\pgfpathlineto{\pgfqpoint{1.034937in}{0.692959in}}%
\pgfpathlineto{\pgfqpoint{1.093257in}{0.728622in}}%
\pgfpathlineto{\pgfqpoint{1.148771in}{0.768512in}}%
\pgfusepath{stroke}%
\end{pgfscope}%
\begin{pgfscope}%
\pgfpathrectangle{\pgfqpoint{0.647939in}{0.492442in}}{\pgfqpoint{3.079299in}{3.079299in}}%
\pgfusepath{clip}%
\pgfsetbuttcap%
\pgfsetroundjoin%
\pgfsetlinewidth{0.301125pt}%
\definecolor{currentstroke}{rgb}{0.500000,0.500000,0.500000}%
\pgfsetstrokecolor{currentstroke}%
\pgfsetstrokeopacity{0.300000}%
\pgfsetdash{}{0pt}%
\pgfpathmoveto{\pgfqpoint{2.515525in}{0.601298in}}%
\pgfpathlineto{\pgfqpoint{2.447151in}{0.603882in}}%
\pgfpathlineto{\pgfqpoint{2.378733in}{0.604807in}}%
\pgfpathlineto{\pgfqpoint{2.310308in}{0.604392in}}%
\pgfpathlineto{\pgfqpoint{2.241894in}{0.603025in}}%
\pgfpathlineto{\pgfqpoint{2.173491in}{0.601161in}}%
\pgfpathlineto{\pgfqpoint{2.105087in}{0.599333in}}%
\pgfpathlineto{\pgfqpoint{2.036670in}{0.598154in}}%
\pgfpathlineto{\pgfqpoint{1.968248in}{0.598319in}}%
\pgfpathlineto{\pgfqpoint{1.899874in}{0.600654in}}%
\pgfpathlineto{\pgfqpoint{1.831696in}{0.606173in}}%
\pgfpathlineto{\pgfqpoint{1.764058in}{0.616181in}}%
\pgfpathlineto{\pgfqpoint{1.697700in}{0.632410in}}%
\pgfusepath{stroke}%
\end{pgfscope}%
\begin{pgfscope}%
\pgfpathrectangle{\pgfqpoint{0.647939in}{0.492442in}}{\pgfqpoint{3.079299in}{3.079299in}}%
\pgfusepath{clip}%
\pgfsetbuttcap%
\pgfsetroundjoin%
\pgfsetlinewidth{0.301125pt}%
\definecolor{currentstroke}{rgb}{0.500000,0.500000,0.500000}%
\pgfsetstrokecolor{currentstroke}%
\pgfsetstrokeopacity{0.300000}%
\pgfsetdash{}{0pt}%
\pgfpathmoveto{\pgfqpoint{3.587270in}{1.612187in}}%
\pgfpathlineto{\pgfqpoint{3.529609in}{1.649013in}}%
\pgfpathlineto{\pgfqpoint{3.473209in}{1.687745in}}%
\pgfpathlineto{\pgfqpoint{3.418025in}{1.728191in}}%
\pgfpathlineto{\pgfqpoint{3.364000in}{1.770178in}}%
\pgfpathlineto{\pgfqpoint{3.311091in}{1.813562in}}%
\pgfpathlineto{\pgfqpoint{3.259268in}{1.858238in}}%
\pgfpathlineto{\pgfqpoint{3.208517in}{1.904129in}}%
\pgfpathlineto{\pgfqpoint{3.158851in}{1.951189in}}%
\pgfpathlineto{\pgfqpoint{3.110331in}{1.999429in}}%
\pgfpathlineto{\pgfqpoint{3.063075in}{2.048908in}}%
\pgfpathlineto{\pgfqpoint{3.017300in}{2.099754in}}%
\pgfpathlineto{\pgfqpoint{2.973370in}{2.152190in}}%
\pgfpathlineto{\pgfqpoint{2.931873in}{2.206563in}}%
\pgfpathlineto{\pgfqpoint{2.893782in}{2.263353in}}%
\pgfpathlineto{\pgfqpoint{2.860664in}{2.323129in}}%
\pgfpathlineto{\pgfqpoint{2.834887in}{2.386328in}}%
\pgfpathlineto{\pgfqpoint{2.819386in}{2.452694in}}%
\pgfpathlineto{\pgfqpoint{2.816312in}{2.520695in}}%
\pgfpathlineto{\pgfqpoint{2.825339in}{2.588188in}}%
\pgfpathlineto{\pgfqpoint{2.844047in}{2.653781in}}%
\pgfpathlineto{\pgfqpoint{2.869651in}{2.717087in}}%
\pgfusepath{stroke}%
\end{pgfscope}%
\begin{pgfscope}%
\pgfpathrectangle{\pgfqpoint{0.647939in}{0.492442in}}{\pgfqpoint{3.079299in}{3.079299in}}%
\pgfusepath{clip}%
\pgfsetbuttcap%
\pgfsetroundjoin%
\pgfsetlinewidth{0.301125pt}%
\definecolor{currentstroke}{rgb}{0.500000,0.500000,0.500000}%
\pgfsetstrokecolor{currentstroke}%
\pgfsetstrokeopacity{0.300000}%
\pgfsetdash{}{0pt}%
\pgfpathmoveto{\pgfqpoint{3.430673in}{2.919504in}}%
\pgfpathlineto{\pgfqpoint{3.453646in}{2.983909in}}%
\pgfpathlineto{\pgfqpoint{3.481183in}{3.046505in}}%
\pgfpathlineto{\pgfqpoint{3.512856in}{3.107119in}}%
\pgfpathlineto{\pgfqpoint{3.548312in}{3.165605in}}%
\pgfpathlineto{\pgfqpoint{3.587270in}{3.221821in}}%
\pgfusepath{stroke}%
\end{pgfscope}%
\begin{pgfscope}%
\pgfpathrectangle{\pgfqpoint{0.647939in}{0.492442in}}{\pgfqpoint{3.079299in}{3.079299in}}%
\pgfusepath{clip}%
\pgfsetbuttcap%
\pgfsetroundjoin%
\pgfsetlinewidth{0.301125pt}%
\definecolor{currentstroke}{rgb}{0.500000,0.500000,0.500000}%
\pgfsetstrokecolor{currentstroke}%
\pgfsetstrokeopacity{0.300000}%
\pgfsetdash{}{0pt}%
\pgfpathmoveto{\pgfqpoint{3.492545in}{3.333088in}}%
\pgfpathlineto{\pgfqpoint{3.538849in}{3.383447in}}%
\pgfpathlineto{\pgfqpoint{3.587270in}{3.431773in}}%
\pgfpathlineto{\pgfqpoint{3.637802in}{3.477884in}}%
\pgfpathlineto{\pgfqpoint{3.690459in}{3.521547in}}%
\pgfpathlineto{\pgfqpoint{3.727238in}{3.550564in}}%
\pgfusepath{stroke}%
\end{pgfscope}%
\begin{pgfscope}%
\pgfpathrectangle{\pgfqpoint{0.647939in}{0.492442in}}{\pgfqpoint{3.079299in}{3.079299in}}%
\pgfusepath{clip}%
\pgfsetbuttcap%
\pgfsetroundjoin%
\pgfsetlinewidth{0.301125pt}%
\definecolor{currentstroke}{rgb}{0.500000,0.500000,0.500000}%
\pgfsetstrokecolor{currentstroke}%
\pgfsetstrokeopacity{0.300000}%
\pgfsetdash{}{0pt}%
\pgfpathmoveto{\pgfqpoint{3.517286in}{1.822139in}}%
\pgfpathlineto{\pgfqpoint{3.464693in}{1.865893in}}%
\pgfpathlineto{\pgfqpoint{3.413837in}{1.911656in}}%
\pgfpathlineto{\pgfqpoint{3.364767in}{1.959327in}}%
\pgfpathlineto{\pgfqpoint{3.317564in}{2.008848in}}%
\pgfpathlineto{\pgfqpoint{3.272373in}{2.060207in}}%
\pgfpathlineto{\pgfqpoint{3.229411in}{2.113443in}}%
\pgfpathlineto{\pgfqpoint{3.188999in}{2.168635in}}%
\pgfpathlineto{\pgfqpoint{3.151594in}{2.225897in}}%
\pgfpathlineto{\pgfqpoint{3.117841in}{2.285367in}}%
\pgfpathlineto{\pgfqpoint{3.088589in}{2.347154in}}%
\pgfpathlineto{\pgfqpoint{3.064889in}{2.411250in}}%
\pgfpathlineto{\pgfqpoint{3.047885in}{2.477412in}}%
\pgfpathlineto{\pgfqpoint{3.038565in}{2.545071in}}%
\pgfpathlineto{\pgfqpoint{3.037420in}{2.613351in}}%
\pgfpathlineto{\pgfqpoint{3.044234in}{2.681303in}}%
\pgfpathlineto{\pgfqpoint{3.058191in}{2.748176in}}%
\pgfpathlineto{\pgfqpoint{3.078183in}{2.813528in}}%
\pgfpathlineto{\pgfqpoint{3.103107in}{2.877189in}}%
\pgfusepath{stroke}%
\end{pgfscope}%
\begin{pgfscope}%
\pgfpathrectangle{\pgfqpoint{0.647939in}{0.492442in}}{\pgfqpoint{3.079299in}{3.079299in}}%
\pgfusepath{clip}%
\pgfsetbuttcap%
\pgfsetroundjoin%
\pgfsetlinewidth{0.301125pt}%
\definecolor{currentstroke}{rgb}{0.500000,0.500000,0.500000}%
\pgfsetstrokecolor{currentstroke}%
\pgfsetstrokeopacity{0.300000}%
\pgfsetdash{}{0pt}%
\pgfpathmoveto{\pgfqpoint{1.859143in}{3.333216in}}%
\pgfpathlineto{\pgfqpoint{1.927511in}{3.335954in}}%
\pgfpathlineto{\pgfqpoint{1.995927in}{3.336964in}}%
\pgfpathlineto{\pgfqpoint{2.064350in}{3.336499in}}%
\pgfpathlineto{\pgfqpoint{2.132760in}{3.334935in}}%
\pgfpathlineto{\pgfqpoint{2.201154in}{3.332765in}}%
\pgfpathlineto{\pgfqpoint{2.269548in}{3.330614in}}%
\pgfpathlineto{\pgfqpoint{2.337960in}{3.329215in}}%
\pgfpathlineto{\pgfqpoint{2.406380in}{3.329380in}}%
\pgfpathlineto{\pgfqpoint{2.474743in}{3.331963in}}%
\pgfpathlineto{\pgfqpoint{2.542896in}{3.337794in}}%
\pgfpathlineto{\pgfqpoint{2.610585in}{3.347580in}}%
\pgfpathlineto{\pgfqpoint{2.677477in}{3.361789in}}%
\pgfusepath{stroke}%
\end{pgfscope}%
\begin{pgfscope}%
\pgfpathrectangle{\pgfqpoint{0.647939in}{0.492442in}}{\pgfqpoint{3.079299in}{3.079299in}}%
\pgfusepath{clip}%
\pgfsetbuttcap%
\pgfsetroundjoin%
\pgfsetlinewidth{0.301125pt}%
\definecolor{currentstroke}{rgb}{0.500000,0.500000,0.500000}%
\pgfsetstrokecolor{currentstroke}%
\pgfsetstrokeopacity{0.300000}%
\pgfsetdash{}{0pt}%
\pgfpathmoveto{\pgfqpoint{3.447302in}{1.542203in}}%
\pgfpathlineto{\pgfqpoint{3.389746in}{1.579206in}}%
\pgfpathlineto{\pgfqpoint{3.332937in}{1.617347in}}%
\pgfpathlineto{\pgfqpoint{3.276773in}{1.656432in}}%
\pgfpathlineto{\pgfqpoint{3.221146in}{1.696279in}}%
\pgfpathlineto{\pgfqpoint{3.165945in}{1.736715in}}%
\pgfpathlineto{\pgfqpoint{3.111054in}{1.777572in}}%
\pgfpathlineto{\pgfqpoint{3.056354in}{1.818681in}}%
\pgfpathlineto{\pgfqpoint{3.001727in}{1.859887in}}%
\pgfpathlineto{\pgfqpoint{2.947056in}{1.901034in}}%
\pgfpathlineto{\pgfqpoint{2.892222in}{1.941959in}}%
\pgfpathlineto{\pgfqpoint{2.837103in}{1.982497in}}%
\pgfpathlineto{\pgfqpoint{2.781585in}{2.022480in}}%
\pgfpathlineto{\pgfqpoint{2.725558in}{2.061739in}}%
\pgfpathlineto{\pgfqpoint{2.668922in}{2.100111in}}%
\pgfpathlineto{\pgfqpoint{2.611624in}{2.137487in}}%
\pgfpathlineto{\pgfqpoint{2.553723in}{2.173909in}}%
\pgfpathlineto{\pgfqpoint{2.495663in}{2.209987in}}%
\pgfpathlineto{\pgfqpoint{2.441210in}{2.249892in}}%
\pgfpathlineto{\pgfqpoint{2.441210in}{2.249892in}}%
\pgfpathlineto{\pgfqpoint{2.427959in}{2.267840in}}%
\pgfpathlineto{\pgfqpoint{2.427959in}{2.267840in}}%
\pgfusepath{stroke}%
\end{pgfscope}%
\begin{pgfscope}%
\pgfpathrectangle{\pgfqpoint{0.647939in}{0.492442in}}{\pgfqpoint{3.079299in}{3.079299in}}%
\pgfusepath{clip}%
\pgfsetbuttcap%
\pgfsetroundjoin%
\pgfsetlinewidth{0.301125pt}%
\definecolor{currentstroke}{rgb}{0.500000,0.500000,0.500000}%
\pgfsetstrokecolor{currentstroke}%
\pgfsetstrokeopacity{0.300000}%
\pgfsetdash{}{0pt}%
\pgfpathmoveto{\pgfqpoint{2.654652in}{0.810841in}}%
\pgfpathlineto{\pgfqpoint{2.586653in}{0.818391in}}%
\pgfpathlineto{\pgfqpoint{2.518436in}{0.823619in}}%
\pgfpathlineto{\pgfqpoint{2.450083in}{0.826674in}}%
\pgfpathlineto{\pgfqpoint{2.381670in}{0.827812in}}%
\pgfpathlineto{\pgfqpoint{2.313247in}{0.827386in}}%
\pgfpathlineto{\pgfqpoint{2.244837in}{0.825827in}}%
\pgfpathlineto{\pgfqpoint{2.176443in}{0.823653in}}%
\pgfpathlineto{\pgfqpoint{2.108049in}{0.821473in}}%
\pgfpathlineto{\pgfqpoint{2.039639in}{0.820011in}}%
\pgfpathlineto{\pgfqpoint{1.971219in}{0.820128in}}%
\pgfpathlineto{\pgfqpoint{1.902869in}{0.822881in}}%
\pgfpathlineto{\pgfqpoint{1.834825in}{0.829656in}}%
\pgfpathlineto{\pgfqpoint{1.767684in}{0.842362in}}%
\pgfpathlineto{\pgfqpoint{1.702913in}{0.863707in}}%
\pgfpathlineto{\pgfqpoint{1.643875in}{0.897224in}}%
\pgfpathlineto{\pgfqpoint{1.643875in}{0.897224in}}%
\pgfpathlineto{\pgfqpoint{1.606084in}{0.932659in}}%
\pgfpathlineto{\pgfqpoint{1.578134in}{0.976751in}}%
\pgfusepath{stroke}%
\end{pgfscope}%
\begin{pgfscope}%
\pgfpathrectangle{\pgfqpoint{0.647939in}{0.492442in}}{\pgfqpoint{3.079299in}{3.079299in}}%
\pgfusepath{clip}%
\pgfsetbuttcap%
\pgfsetroundjoin%
\pgfsetlinewidth{0.301125pt}%
\definecolor{currentstroke}{rgb}{0.500000,0.500000,0.500000}%
\pgfsetstrokecolor{currentstroke}%
\pgfsetstrokeopacity{0.300000}%
\pgfsetdash{}{0pt}%
\pgfpathmoveto{\pgfqpoint{3.371380in}{2.597547in}}%
\pgfpathlineto{\pgfqpoint{3.366984in}{2.665748in}}%
\pgfpathlineto{\pgfqpoint{3.369030in}{2.734067in}}%
\pgfpathlineto{\pgfqpoint{3.377318in}{2.801916in}}%
\pgfpathlineto{\pgfqpoint{3.391458in}{2.868792in}}%
\pgfpathlineto{\pgfqpoint{3.410960in}{2.934312in}}%
\pgfusepath{stroke}%
\end{pgfscope}%
\begin{pgfscope}%
\pgfpathrectangle{\pgfqpoint{0.647939in}{0.492442in}}{\pgfqpoint{3.079299in}{3.079299in}}%
\pgfusepath{clip}%
\pgfsetbuttcap%
\pgfsetroundjoin%
\pgfsetlinewidth{0.301125pt}%
\definecolor{currentstroke}{rgb}{0.500000,0.500000,0.500000}%
\pgfsetstrokecolor{currentstroke}%
\pgfsetstrokeopacity{0.300000}%
\pgfsetdash{}{0pt}%
\pgfpathmoveto{\pgfqpoint{3.383816in}{2.128712in}}%
\pgfpathlineto{\pgfqpoint{3.343911in}{2.184259in}}%
\pgfpathlineto{\pgfqpoint{3.307334in}{2.242044in}}%
\pgfpathlineto{\pgfqpoint{3.274569in}{2.302068in}}%
\pgfpathlineto{\pgfqpoint{3.246227in}{2.364291in}}%
\pgfpathlineto{\pgfqpoint{3.223011in}{2.428587in}}%
\pgfpathlineto{\pgfqpoint{3.205654in}{2.494688in}}%
\pgfpathlineto{\pgfqpoint{3.194791in}{2.562145in}}%
\pgfpathlineto{\pgfqpoint{3.190817in}{2.630350in}}%
\pgfusepath{stroke}%
\end{pgfscope}%
\begin{pgfscope}%
\pgfpathrectangle{\pgfqpoint{0.647939in}{0.492442in}}{\pgfqpoint{3.079299in}{3.079299in}}%
\pgfusepath{clip}%
\pgfsetbuttcap%
\pgfsetroundjoin%
\pgfsetlinewidth{0.301125pt}%
\definecolor{currentstroke}{rgb}{0.500000,0.500000,0.500000}%
\pgfsetstrokecolor{currentstroke}%
\pgfsetstrokeopacity{0.300000}%
\pgfsetdash{}{0pt}%
\pgfpathmoveto{\pgfqpoint{1.788849in}{3.057034in}}%
\pgfpathlineto{\pgfqpoint{1.857045in}{3.062550in}}%
\pgfpathlineto{\pgfqpoint{1.925383in}{3.065895in}}%
\pgfpathlineto{\pgfqpoint{1.993792in}{3.067175in}}%
\pgfpathlineto{\pgfqpoint{2.062213in}{3.066648in}}%
\pgfpathlineto{\pgfqpoint{2.130612in}{3.064726in}}%
\pgfpathlineto{\pgfqpoint{2.198986in}{3.061991in}}%
\pgfpathlineto{\pgfqpoint{2.267357in}{3.059208in}}%
\pgfpathlineto{\pgfqpoint{2.335756in}{3.057330in}}%
\pgfpathlineto{\pgfqpoint{2.404170in}{3.057474in}}%
\pgfpathlineto{\pgfqpoint{2.472482in}{3.060855in}}%
\pgfpathlineto{\pgfqpoint{2.540412in}{3.068665in}}%
\pgfpathlineto{\pgfqpoint{2.607493in}{3.081853in}}%
\pgfusepath{stroke}%
\end{pgfscope}%
\begin{pgfscope}%
\pgfpathrectangle{\pgfqpoint{0.647939in}{0.492442in}}{\pgfqpoint{3.079299in}{3.079299in}}%
\pgfusepath{clip}%
\pgfsetbuttcap%
\pgfsetroundjoin%
\pgfsetlinewidth{0.301125pt}%
\definecolor{currentstroke}{rgb}{0.500000,0.500000,0.500000}%
\pgfsetstrokecolor{currentstroke}%
\pgfsetstrokeopacity{0.300000}%
\pgfsetdash{}{0pt}%
\pgfpathmoveto{\pgfqpoint{3.167366in}{1.892124in}}%
\pgfpathlineto{\pgfqpoint{3.116837in}{1.938261in}}%
\pgfpathlineto{\pgfqpoint{3.067202in}{1.985356in}}%
\pgfpathlineto{\pgfqpoint{3.018544in}{2.033458in}}%
\pgfpathlineto{\pgfqpoint{2.971027in}{2.082682in}}%
\pgfpathlineto{\pgfqpoint{2.924946in}{2.133247in}}%
\pgfpathlineto{\pgfqpoint{2.880825in}{2.185514in}}%
\pgfpathlineto{\pgfqpoint{2.839574in}{2.240052in}}%
\pgfpathlineto{\pgfqpoint{2.802810in}{2.297668in}}%
\pgfpathlineto{\pgfqpoint{2.773338in}{2.359225in}}%
\pgfpathlineto{\pgfqpoint{2.755279in}{2.424838in}}%
\pgfpathlineto{\pgfqpoint{2.751951in}{2.491134in}}%
\pgfpathlineto{\pgfqpoint{2.761117in}{2.551922in}}%
\pgfusepath{stroke}%
\end{pgfscope}%
\begin{pgfscope}%
\pgfpathrectangle{\pgfqpoint{0.647939in}{0.492442in}}{\pgfqpoint{3.079299in}{3.079299in}}%
\pgfusepath{clip}%
\pgfsetbuttcap%
\pgfsetroundjoin%
\pgfsetlinewidth{0.301125pt}%
\definecolor{currentstroke}{rgb}{0.500000,0.500000,0.500000}%
\pgfsetstrokecolor{currentstroke}%
\pgfsetstrokeopacity{0.300000}%
\pgfsetdash{}{0pt}%
\pgfpathmoveto{\pgfqpoint{1.737657in}{2.916293in}}%
\pgfpathlineto{\pgfqpoint{1.805628in}{2.924105in}}%
\pgfpathlineto{\pgfqpoint{1.873824in}{2.929598in}}%
\pgfpathlineto{\pgfqpoint{1.942169in}{2.932738in}}%
\pgfpathlineto{\pgfqpoint{2.010582in}{2.933640in}}%
\pgfpathlineto{\pgfqpoint{2.078996in}{2.932589in}}%
\pgfpathlineto{\pgfqpoint{2.147377in}{2.930080in}}%
\pgfpathlineto{\pgfqpoint{2.215728in}{2.926820in}}%
\pgfpathlineto{\pgfqpoint{2.284085in}{2.923730in}}%
\pgfpathlineto{\pgfqpoint{2.352483in}{2.921973in}}%
\pgfpathlineto{\pgfqpoint{2.420882in}{2.922954in}}%
\pgfpathlineto{\pgfqpoint{2.489054in}{2.928240in}}%
\pgfpathlineto{\pgfqpoint{2.556496in}{2.939284in}}%
\pgfpathlineto{\pgfqpoint{2.622477in}{2.956992in}}%
\pgfpathlineto{\pgfqpoint{2.686274in}{2.981431in}}%
\pgfpathlineto{\pgfqpoint{2.747461in}{3.011869in}}%
\pgfpathlineto{\pgfqpoint{2.806002in}{3.047156in}}%
\pgfusepath{stroke}%
\end{pgfscope}%
\begin{pgfscope}%
\pgfpathrectangle{\pgfqpoint{0.647939in}{0.492442in}}{\pgfqpoint{3.079299in}{3.079299in}}%
\pgfusepath{clip}%
\pgfsetbuttcap%
\pgfsetroundjoin%
\pgfsetlinewidth{0.301125pt}%
\definecolor{currentstroke}{rgb}{0.500000,0.500000,0.500000}%
\pgfsetstrokecolor{currentstroke}%
\pgfsetstrokeopacity{0.300000}%
\pgfsetdash{}{0pt}%
\pgfpathmoveto{\pgfqpoint{2.957413in}{1.962108in}}%
\pgfpathlineto{\pgfqpoint{2.905081in}{2.006183in}}%
\pgfpathlineto{\pgfqpoint{2.852993in}{2.050549in}}%
\pgfpathlineto{\pgfqpoint{2.801235in}{2.095293in}}%
\pgfpathlineto{\pgfqpoint{2.750020in}{2.140654in}}%
\pgfpathlineto{\pgfqpoint{2.699852in}{2.187152in}}%
\pgfpathlineto{\pgfqpoint{2.651935in}{2.235931in}}%
\pgfpathlineto{\pgfqpoint{2.609615in}{2.289393in}}%
\pgfpathlineto{\pgfqpoint{2.609615in}{2.289393in}}%
\pgfpathlineto{\pgfqpoint{2.586968in}{2.334304in}}%
\pgfpathlineto{\pgfqpoint{2.586968in}{2.334304in}}%
\pgfpathlineto{\pgfqpoint{2.579307in}{2.374894in}}%
\pgfpathlineto{\pgfqpoint{2.583730in}{2.416709in}}%
\pgfpathlineto{\pgfqpoint{2.597249in}{2.457202in}}%
\pgfusepath{stroke}%
\end{pgfscope}%
\begin{pgfscope}%
\pgfpathrectangle{\pgfqpoint{0.647939in}{0.492442in}}{\pgfqpoint{3.079299in}{3.079299in}}%
\pgfusepath{clip}%
\pgfsetbuttcap%
\pgfsetroundjoin%
\pgfsetlinewidth{0.301125pt}%
\definecolor{currentstroke}{rgb}{0.500000,0.500000,0.500000}%
\pgfsetstrokecolor{currentstroke}%
\pgfsetstrokeopacity{0.300000}%
\pgfsetdash{}{0pt}%
\pgfpathmoveto{\pgfqpoint{2.995937in}{2.185547in}}%
\pgfpathlineto{\pgfqpoint{2.957413in}{2.242044in}}%
\pgfpathlineto{\pgfqpoint{2.923051in}{2.301144in}}%
\pgfpathlineto{\pgfqpoint{2.894530in}{2.363225in}}%
\pgfpathlineto{\pgfqpoint{2.874061in}{2.428324in}}%
\pgfpathlineto{\pgfqpoint{2.863833in}{2.495706in}}%
\pgfpathlineto{\pgfqpoint{2.864848in}{2.563829in}}%
\pgfpathlineto{\pgfqpoint{2.876231in}{2.631059in}}%
\pgfpathlineto{\pgfqpoint{2.895888in}{2.696446in}}%
\pgfusepath{stroke}%
\end{pgfscope}%
\begin{pgfscope}%
\pgfpathrectangle{\pgfqpoint{0.647939in}{0.492442in}}{\pgfqpoint{3.079299in}{3.079299in}}%
\pgfusepath{clip}%
\pgfsetbuttcap%
\pgfsetroundjoin%
\pgfsetlinewidth{0.301125pt}%
\definecolor{currentstroke}{rgb}{0.500000,0.500000,0.500000}%
\pgfsetstrokecolor{currentstroke}%
\pgfsetstrokeopacity{0.300000}%
\pgfsetdash{}{0pt}%
\pgfpathmoveto{\pgfqpoint{1.460512in}{2.616120in}}%
\pgfpathlineto{\pgfqpoint{1.526801in}{2.633092in}}%
\pgfpathlineto{\pgfqpoint{1.593384in}{2.648858in}}%
\pgfpathlineto{\pgfqpoint{1.660326in}{2.663006in}}%
\pgfpathlineto{\pgfqpoint{1.727660in}{2.675139in}}%
\pgfpathlineto{\pgfqpoint{1.795375in}{2.684904in}}%
\pgfpathlineto{\pgfqpoint{1.863418in}{2.692022in}}%
\pgfpathlineto{\pgfqpoint{1.931693in}{2.696347in}}%
\pgfpathlineto{\pgfqpoint{2.000089in}{2.697895in}}%
\pgfpathlineto{\pgfqpoint{2.068498in}{2.696884in}}%
\pgfpathlineto{\pgfqpoint{2.136849in}{2.693783in}}%
\pgfpathlineto{\pgfqpoint{2.205134in}{2.689369in}}%
\pgfpathlineto{\pgfqpoint{2.273410in}{2.684820in}}%
\pgfpathlineto{\pgfqpoint{2.341758in}{2.681803in}}%
\pgfpathlineto{\pgfqpoint{2.410126in}{2.682571in}}%
\pgfpathlineto{\pgfqpoint{2.478036in}{2.689850in}}%
\pgfpathlineto{\pgfqpoint{2.544327in}{2.706057in}}%
\pgfpathlineto{\pgfqpoint{2.607493in}{2.731932in}}%
\pgfusepath{stroke}%
\end{pgfscope}%
\begin{pgfscope}%
\pgfpathrectangle{\pgfqpoint{0.647939in}{0.492442in}}{\pgfqpoint{3.079299in}{3.079299in}}%
\pgfusepath{clip}%
\pgfsetbuttcap%
\pgfsetroundjoin%
\pgfsetlinewidth{0.301125pt}%
\definecolor{currentstroke}{rgb}{0.500000,0.500000,0.500000}%
\pgfsetstrokecolor{currentstroke}%
\pgfsetstrokeopacity{0.300000}%
\pgfsetdash{}{0pt}%
\pgfpathmoveto{\pgfqpoint{1.487748in}{2.032092in}}%
\pgfpathlineto{\pgfqpoint{1.551478in}{2.057006in}}%
\pgfpathlineto{\pgfqpoint{1.615583in}{2.080933in}}%
\pgfpathlineto{\pgfqpoint{1.680251in}{2.103279in}}%
\pgfpathlineto{\pgfqpoint{1.745649in}{2.123365in}}%
\pgfpathlineto{\pgfqpoint{1.811889in}{2.140422in}}%
\pgfpathlineto{\pgfqpoint{1.878995in}{2.153585in}}%
\pgfpathlineto{\pgfqpoint{1.946856in}{2.161868in}}%
\pgfpathlineto{\pgfqpoint{2.015160in}{2.164099in}}%
\pgfpathlineto{\pgfqpoint{2.083223in}{2.158666in}}%
\pgfpathlineto{\pgfqpoint{2.149133in}{2.142022in}}%
\pgfpathlineto{\pgfqpoint{2.149133in}{2.142022in}}%
\pgfpathlineto{\pgfqpoint{2.179714in}{2.125829in}}%
\pgfpathlineto{\pgfqpoint{2.179714in}{2.125829in}}%
\pgfpathlineto{\pgfqpoint{2.196198in}{2.107275in}}%
\pgfpathlineto{\pgfqpoint{2.196198in}{2.107275in}}%
\pgfpathlineto{\pgfqpoint{2.200208in}{2.087569in}}%
\pgfpathlineto{\pgfqpoint{2.195600in}{2.067579in}}%
\pgfusepath{stroke}%
\end{pgfscope}%
\begin{pgfscope}%
\pgfpathrectangle{\pgfqpoint{0.647939in}{0.492442in}}{\pgfqpoint{3.079299in}{3.079299in}}%
\pgfusepath{clip}%
\pgfsetbuttcap%
\pgfsetroundjoin%
\pgfsetlinewidth{0.301125pt}%
\definecolor{currentstroke}{rgb}{0.500000,0.500000,0.500000}%
\pgfsetstrokecolor{currentstroke}%
\pgfsetstrokeopacity{0.300000}%
\pgfsetdash{}{0pt}%
\pgfpathmoveto{\pgfqpoint{2.747461in}{1.752155in}}%
\pgfpathlineto{\pgfqpoint{2.683611in}{1.776679in}}%
\pgfpathlineto{\pgfqpoint{2.618499in}{1.797594in}}%
\pgfpathlineto{\pgfqpoint{2.552172in}{1.814223in}}%
\pgfpathlineto{\pgfqpoint{2.484811in}{1.825947in}}%
\pgfpathlineto{\pgfqpoint{2.416740in}{1.832343in}}%
\pgfpathlineto{\pgfqpoint{2.348377in}{1.833333in}}%
\pgfpathlineto{\pgfqpoint{2.280120in}{1.829257in}}%
\pgfpathlineto{\pgfqpoint{2.212234in}{1.820926in}}%
\pgfusepath{stroke}%
\end{pgfscope}%
\begin{pgfscope}%
\pgfpathrectangle{\pgfqpoint{0.647939in}{0.492442in}}{\pgfqpoint{3.079299in}{3.079299in}}%
\pgfusepath{clip}%
\pgfsetbuttcap%
\pgfsetroundjoin%
\pgfsetlinewidth{0.301125pt}%
\definecolor{currentstroke}{rgb}{0.500000,0.500000,0.500000}%
\pgfsetstrokecolor{currentstroke}%
\pgfsetstrokeopacity{0.300000}%
\pgfsetdash{}{0pt}%
\pgfpathmoveto{\pgfqpoint{1.627716in}{2.382012in}}%
\pgfpathlineto{\pgfqpoint{1.694154in}{2.398354in}}%
\pgfpathlineto{\pgfqpoint{1.761130in}{2.412304in}}%
\pgfpathlineto{\pgfqpoint{1.828639in}{2.423359in}}%
\pgfpathlineto{\pgfqpoint{1.896606in}{2.431088in}}%
\pgfpathlineto{\pgfqpoint{1.964882in}{2.435185in}}%
\pgfpathlineto{\pgfqpoint{2.033280in}{2.435530in}}%
\pgfpathlineto{\pgfqpoint{2.101604in}{2.432261in}}%
\pgfpathlineto{\pgfqpoint{2.169718in}{2.425872in}}%
\pgfpathlineto{\pgfqpoint{2.237622in}{2.417449in}}%
\pgfpathlineto{\pgfqpoint{2.305540in}{2.409175in}}%
\pgfpathlineto{\pgfqpoint{2.373724in}{2.405791in}}%
\pgfpathlineto{\pgfqpoint{2.373724in}{2.405791in}}%
\pgfpathlineto{\pgfqpoint{2.427022in}{2.412344in}}%
\pgfpathlineto{\pgfqpoint{2.427022in}{2.412344in}}%
\pgfpathlineto{\pgfqpoint{2.471335in}{2.428905in}}%
\pgfusepath{stroke}%
\end{pgfscope}%
\begin{pgfscope}%
\pgfpathrectangle{\pgfqpoint{0.647939in}{0.492442in}}{\pgfqpoint{3.079299in}{3.079299in}}%
\pgfusepath{clip}%
\pgfsetbuttcap%
\pgfsetroundjoin%
\pgfsetlinewidth{0.301125pt}%
\definecolor{currentstroke}{rgb}{0.500000,0.500000,0.500000}%
\pgfsetstrokecolor{currentstroke}%
\pgfsetstrokeopacity{0.300000}%
\pgfsetdash{}{0pt}%
\pgfpathmoveto{\pgfqpoint{2.677477in}{2.032092in}}%
\pgfpathlineto{\pgfqpoint{2.617280in}{2.064554in}}%
\pgfpathlineto{\pgfqpoint{2.555568in}{2.093991in}}%
\pgfpathlineto{\pgfqpoint{2.492022in}{2.119129in}}%
\pgfpathlineto{\pgfqpoint{2.426272in}{2.137440in}}%
\pgfpathlineto{\pgfqpoint{2.358641in}{2.142025in}}%
\pgfpathlineto{\pgfqpoint{2.358641in}{2.142025in}}%
\pgfpathlineto{\pgfqpoint{2.320462in}{2.134134in}}%
\pgfpathlineto{\pgfqpoint{2.283755in}{2.114918in}}%
\pgfpathlineto{\pgfqpoint{2.252487in}{2.091288in}}%
\pgfusepath{stroke}%
\end{pgfscope}%
\begin{pgfscope}%
\pgfpathrectangle{\pgfqpoint{0.647939in}{0.492442in}}{\pgfqpoint{3.079299in}{3.079299in}}%
\pgfusepath{clip}%
\pgfsetbuttcap%
\pgfsetroundjoin%
\pgfsetlinewidth{0.301125pt}%
\definecolor{currentstroke}{rgb}{0.500000,0.500000,0.500000}%
\pgfsetstrokecolor{currentstroke}%
\pgfsetstrokeopacity{0.300000}%
\pgfsetdash{}{0pt}%
\pgfpathmoveto{\pgfqpoint{2.251800in}{1.703152in}}%
\pgfpathlineto{\pgfqpoint{2.183716in}{1.696368in}}%
\pgfpathlineto{\pgfqpoint{2.115732in}{1.688589in}}%
\pgfpathlineto{\pgfqpoint{2.047620in}{1.682171in}}%
\pgfpathlineto{\pgfqpoint{1.979438in}{1.682316in}}%
\pgfpathlineto{\pgfqpoint{1.979438in}{1.682316in}}%
\pgfpathlineto{\pgfqpoint{1.943356in}{1.689562in}}%
\pgfpathlineto{\pgfqpoint{1.943356in}{1.689562in}}%
\pgfusepath{stroke}%
\end{pgfscope}%
\begin{pgfscope}%
\pgfpathrectangle{\pgfqpoint{0.647939in}{0.492442in}}{\pgfqpoint{3.079299in}{3.079299in}}%
\pgfusepath{clip}%
\pgfsetbuttcap%
\pgfsetroundjoin%
\pgfsetlinewidth{0.301125pt}%
\definecolor{currentstroke}{rgb}{0.500000,0.500000,0.500000}%
\pgfsetstrokecolor{currentstroke}%
\pgfsetstrokeopacity{0.300000}%
\pgfsetdash{}{0pt}%
\pgfpathmoveto{\pgfqpoint{1.837668in}{2.242044in}}%
\pgfpathlineto{\pgfqpoint{1.905388in}{2.251582in}}%
\pgfpathlineto{\pgfqpoint{1.973585in}{2.256517in}}%
\pgfpathlineto{\pgfqpoint{2.041949in}{2.256225in}}%
\pgfpathlineto{\pgfqpoint{2.110030in}{2.250115in}}%
\pgfpathlineto{\pgfqpoint{2.177182in}{2.237475in}}%
\pgfpathlineto{\pgfqpoint{2.241917in}{2.216223in}}%
\pgfpathlineto{\pgfqpoint{2.241917in}{2.216223in}}%
\pgfpathlineto{\pgfqpoint{2.270678in}{2.200220in}}%
\pgfpathlineto{\pgfqpoint{2.270678in}{2.200220in}}%
\pgfpathlineto{\pgfqpoint{2.284866in}{2.185182in}}%
\pgfpathlineto{\pgfqpoint{2.284866in}{2.185182in}}%
\pgfpathlineto{\pgfqpoint{2.288038in}{2.168010in}}%
\pgfpathlineto{\pgfqpoint{2.283103in}{2.151531in}}%
\pgfusepath{stroke}%
\end{pgfscope}%
\begin{pgfscope}%
\pgfpathrectangle{\pgfqpoint{0.647939in}{0.492442in}}{\pgfqpoint{3.079299in}{3.079299in}}%
\pgfusepath{clip}%
\pgfsetroundcap%
\pgfsetroundjoin%
\pgfsetlinewidth{0.301125pt}%
\definecolor{currentstroke}{rgb}{0.500000,0.500000,0.500000}%
\pgfsetstrokecolor{currentstroke}%
\pgfsetstrokeopacity{0.300000}%
\pgfsetdash{}{0pt}%
\pgfpathmoveto{\pgfqpoint{1.417127in}{1.045577in}}%
\pgfusepath{stroke}%
\end{pgfscope}%
\begin{pgfscope}%
\pgfpathrectangle{\pgfqpoint{0.647939in}{0.492442in}}{\pgfqpoint{3.079299in}{3.079299in}}%
\pgfusepath{clip}%
\pgfsetroundcap%
\pgfsetroundjoin%
\definecolor{currentfill}{rgb}{0.500000,0.500000,0.500000}%
\pgfsetfillcolor{currentfill}%
\pgfsetfillopacity{0.300000}%
\pgfsetlinewidth{0.301125pt}%
\definecolor{currentstroke}{rgb}{0.500000,0.500000,0.500000}%
\pgfsetstrokecolor{currentstroke}%
\pgfsetstrokeopacity{0.300000}%
\pgfsetdash{}{0pt}%
\pgfpathmoveto{\pgfqpoint{0.000000in}{0.000000in}}%
\pgfpathlineto{\pgfqpoint{0.000000in}{0.000000in}}%
\pgfpathclose%
\pgfusepath{stroke,fill}%
\end{pgfscope}%
\begin{pgfscope}%
\pgfpathrectangle{\pgfqpoint{0.647939in}{0.492442in}}{\pgfqpoint{3.079299in}{3.079299in}}%
\pgfusepath{clip}%
\pgfsetroundcap%
\pgfsetroundjoin%
\pgfsetlinewidth{0.301125pt}%
\definecolor{currentstroke}{rgb}{0.500000,0.500000,0.500000}%
\pgfsetstrokecolor{currentstroke}%
\pgfsetstrokeopacity{0.300000}%
\pgfsetdash{}{0pt}%
\pgfpathmoveto{\pgfqpoint{1.070148in}{0.574536in}}%
\pgfusepath{stroke}%
\end{pgfscope}%
\begin{pgfscope}%
\pgfpathrectangle{\pgfqpoint{0.647939in}{0.492442in}}{\pgfqpoint{3.079299in}{3.079299in}}%
\pgfusepath{clip}%
\pgfsetroundcap%
\pgfsetroundjoin%
\definecolor{currentfill}{rgb}{0.500000,0.500000,0.500000}%
\pgfsetfillcolor{currentfill}%
\pgfsetfillopacity{0.300000}%
\pgfsetlinewidth{0.301125pt}%
\definecolor{currentstroke}{rgb}{0.500000,0.500000,0.500000}%
\pgfsetstrokecolor{currentstroke}%
\pgfsetstrokeopacity{0.300000}%
\pgfsetdash{}{0pt}%
\pgfpathmoveto{\pgfqpoint{0.000000in}{0.000000in}}%
\pgfpathlineto{\pgfqpoint{0.000000in}{0.000000in}}%
\pgfpathclose%
\pgfusepath{stroke,fill}%
\end{pgfscope}%
\begin{pgfscope}%
\pgfpathrectangle{\pgfqpoint{0.647939in}{0.492442in}}{\pgfqpoint{3.079299in}{3.079299in}}%
\pgfusepath{clip}%
\pgfsetroundcap%
\pgfsetroundjoin%
\pgfsetlinewidth{0.301125pt}%
\definecolor{currentstroke}{rgb}{0.500000,0.500000,0.500000}%
\pgfsetstrokecolor{currentstroke}%
\pgfsetstrokeopacity{0.300000}%
\pgfsetdash{}{0pt}%
\pgfpathmoveto{\pgfqpoint{1.306641in}{0.740548in}}%
\pgfusepath{stroke}%
\end{pgfscope}%
\begin{pgfscope}%
\pgfpathrectangle{\pgfqpoint{0.647939in}{0.492442in}}{\pgfqpoint{3.079299in}{3.079299in}}%
\pgfusepath{clip}%
\pgfsetroundcap%
\pgfsetroundjoin%
\definecolor{currentfill}{rgb}{0.500000,0.500000,0.500000}%
\pgfsetfillcolor{currentfill}%
\pgfsetfillopacity{0.300000}%
\pgfsetlinewidth{0.301125pt}%
\definecolor{currentstroke}{rgb}{0.500000,0.500000,0.500000}%
\pgfsetstrokecolor{currentstroke}%
\pgfsetstrokeopacity{0.300000}%
\pgfsetdash{}{0pt}%
\pgfpathmoveto{\pgfqpoint{0.000000in}{0.000000in}}%
\pgfpathlineto{\pgfqpoint{0.000000in}{0.000000in}}%
\pgfpathclose%
\pgfusepath{stroke,fill}%
\end{pgfscope}%
\begin{pgfscope}%
\pgfpathrectangle{\pgfqpoint{0.647939in}{0.492442in}}{\pgfqpoint{3.079299in}{3.079299in}}%
\pgfusepath{clip}%
\pgfsetroundcap%
\pgfsetroundjoin%
\pgfsetlinewidth{0.301125pt}%
\definecolor{currentstroke}{rgb}{0.500000,0.500000,0.500000}%
\pgfsetstrokecolor{currentstroke}%
\pgfsetstrokeopacity{0.300000}%
\pgfsetdash{}{0pt}%
\pgfpathmoveto{\pgfqpoint{1.385460in}{0.641309in}}%
\pgfusepath{stroke}%
\end{pgfscope}%
\begin{pgfscope}%
\pgfpathrectangle{\pgfqpoint{0.647939in}{0.492442in}}{\pgfqpoint{3.079299in}{3.079299in}}%
\pgfusepath{clip}%
\pgfsetroundcap%
\pgfsetroundjoin%
\definecolor{currentfill}{rgb}{0.500000,0.500000,0.500000}%
\pgfsetfillcolor{currentfill}%
\pgfsetfillopacity{0.300000}%
\pgfsetlinewidth{0.301125pt}%
\definecolor{currentstroke}{rgb}{0.500000,0.500000,0.500000}%
\pgfsetstrokecolor{currentstroke}%
\pgfsetstrokeopacity{0.300000}%
\pgfsetdash{}{0pt}%
\pgfpathmoveto{\pgfqpoint{0.000000in}{0.000000in}}%
\pgfpathlineto{\pgfqpoint{0.000000in}{0.000000in}}%
\pgfpathclose%
\pgfusepath{stroke,fill}%
\end{pgfscope}%
\begin{pgfscope}%
\pgfpathrectangle{\pgfqpoint{0.647939in}{0.492442in}}{\pgfqpoint{3.079299in}{3.079299in}}%
\pgfusepath{clip}%
\pgfsetroundcap%
\pgfsetroundjoin%
\pgfsetlinewidth{0.301125pt}%
\definecolor{currentstroke}{rgb}{0.500000,0.500000,0.500000}%
\pgfsetstrokecolor{currentstroke}%
\pgfsetstrokeopacity{0.300000}%
\pgfsetdash{}{0pt}%
\pgfpathmoveto{\pgfqpoint{1.549696in}{0.560629in}}%
\pgfusepath{stroke}%
\end{pgfscope}%
\begin{pgfscope}%
\pgfpathrectangle{\pgfqpoint{0.647939in}{0.492442in}}{\pgfqpoint{3.079299in}{3.079299in}}%
\pgfusepath{clip}%
\pgfsetroundcap%
\pgfsetroundjoin%
\definecolor{currentfill}{rgb}{0.500000,0.500000,0.500000}%
\pgfsetfillcolor{currentfill}%
\pgfsetfillopacity{0.300000}%
\pgfsetlinewidth{0.301125pt}%
\definecolor{currentstroke}{rgb}{0.500000,0.500000,0.500000}%
\pgfsetstrokecolor{currentstroke}%
\pgfsetstrokeopacity{0.300000}%
\pgfsetdash{}{0pt}%
\pgfpathmoveto{\pgfqpoint{0.000000in}{0.000000in}}%
\pgfpathlineto{\pgfqpoint{0.000000in}{0.000000in}}%
\pgfpathclose%
\pgfusepath{stroke,fill}%
\end{pgfscope}%
\begin{pgfscope}%
\pgfpathrectangle{\pgfqpoint{0.647939in}{0.492442in}}{\pgfqpoint{3.079299in}{3.079299in}}%
\pgfusepath{clip}%
\pgfsetroundcap%
\pgfsetroundjoin%
\pgfsetlinewidth{0.301125pt}%
\definecolor{currentstroke}{rgb}{0.500000,0.500000,0.500000}%
\pgfsetstrokecolor{currentstroke}%
\pgfsetstrokeopacity{0.300000}%
\pgfsetdash{}{0pt}%
\pgfpathmoveto{\pgfqpoint{2.293508in}{0.497801in}}%
\pgfusepath{stroke}%
\end{pgfscope}%
\begin{pgfscope}%
\pgfpathrectangle{\pgfqpoint{0.647939in}{0.492442in}}{\pgfqpoint{3.079299in}{3.079299in}}%
\pgfusepath{clip}%
\pgfsetroundcap%
\pgfsetroundjoin%
\definecolor{currentfill}{rgb}{0.500000,0.500000,0.500000}%
\pgfsetfillcolor{currentfill}%
\pgfsetfillopacity{0.300000}%
\pgfsetlinewidth{0.301125pt}%
\definecolor{currentstroke}{rgb}{0.500000,0.500000,0.500000}%
\pgfsetstrokecolor{currentstroke}%
\pgfsetstrokeopacity{0.300000}%
\pgfsetdash{}{0pt}%
\pgfpathmoveto{\pgfqpoint{0.000000in}{0.000000in}}%
\pgfpathlineto{\pgfqpoint{0.000000in}{0.000000in}}%
\pgfpathclose%
\pgfusepath{stroke,fill}%
\end{pgfscope}%
\begin{pgfscope}%
\pgfpathrectangle{\pgfqpoint{0.647939in}{0.492442in}}{\pgfqpoint{3.079299in}{3.079299in}}%
\pgfusepath{clip}%
\pgfsetroundcap%
\pgfsetroundjoin%
\pgfsetlinewidth{0.301125pt}%
\definecolor{currentstroke}{rgb}{0.500000,0.500000,0.500000}%
\pgfsetstrokecolor{currentstroke}%
\pgfsetstrokeopacity{0.300000}%
\pgfsetdash{}{0pt}%
\pgfpathmoveto{\pgfqpoint{2.247525in}{0.542999in}}%
\pgfusepath{stroke}%
\end{pgfscope}%
\begin{pgfscope}%
\pgfpathrectangle{\pgfqpoint{0.647939in}{0.492442in}}{\pgfqpoint{3.079299in}{3.079299in}}%
\pgfusepath{clip}%
\pgfsetroundcap%
\pgfsetroundjoin%
\definecolor{currentfill}{rgb}{0.500000,0.500000,0.500000}%
\pgfsetfillcolor{currentfill}%
\pgfsetfillopacity{0.300000}%
\pgfsetlinewidth{0.301125pt}%
\definecolor{currentstroke}{rgb}{0.500000,0.500000,0.500000}%
\pgfsetstrokecolor{currentstroke}%
\pgfsetstrokeopacity{0.300000}%
\pgfsetdash{}{0pt}%
\pgfpathmoveto{\pgfqpoint{0.000000in}{0.000000in}}%
\pgfpathlineto{\pgfqpoint{0.000000in}{0.000000in}}%
\pgfpathclose%
\pgfusepath{stroke,fill}%
\end{pgfscope}%
\begin{pgfscope}%
\pgfpathrectangle{\pgfqpoint{0.647939in}{0.492442in}}{\pgfqpoint{3.079299in}{3.079299in}}%
\pgfusepath{clip}%
\pgfsetroundcap%
\pgfsetroundjoin%
\pgfsetlinewidth{0.301125pt}%
\definecolor{currentstroke}{rgb}{0.500000,0.500000,0.500000}%
\pgfsetstrokecolor{currentstroke}%
\pgfsetstrokeopacity{0.300000}%
\pgfsetdash{}{0pt}%
\pgfpathmoveto{\pgfqpoint{2.938312in}{0.534729in}}%
\pgfusepath{stroke}%
\end{pgfscope}%
\begin{pgfscope}%
\pgfpathrectangle{\pgfqpoint{0.647939in}{0.492442in}}{\pgfqpoint{3.079299in}{3.079299in}}%
\pgfusepath{clip}%
\pgfsetroundcap%
\pgfsetroundjoin%
\definecolor{currentfill}{rgb}{0.500000,0.500000,0.500000}%
\pgfsetfillcolor{currentfill}%
\pgfsetfillopacity{0.300000}%
\pgfsetlinewidth{0.301125pt}%
\definecolor{currentstroke}{rgb}{0.500000,0.500000,0.500000}%
\pgfsetstrokecolor{currentstroke}%
\pgfsetstrokeopacity{0.300000}%
\pgfsetdash{}{0pt}%
\pgfpathmoveto{\pgfqpoint{0.000000in}{0.000000in}}%
\pgfpathlineto{\pgfqpoint{0.000000in}{0.000000in}}%
\pgfpathclose%
\pgfusepath{stroke,fill}%
\end{pgfscope}%
\begin{pgfscope}%
\pgfpathrectangle{\pgfqpoint{0.647939in}{0.492442in}}{\pgfqpoint{3.079299in}{3.079299in}}%
\pgfusepath{clip}%
\pgfsetroundcap%
\pgfsetroundjoin%
\pgfsetlinewidth{0.301125pt}%
\definecolor{currentstroke}{rgb}{0.500000,0.500000,0.500000}%
\pgfsetstrokecolor{currentstroke}%
\pgfsetstrokeopacity{0.300000}%
\pgfsetdash{}{0pt}%
\pgfpathmoveto{\pgfqpoint{2.281053in}{0.695703in}}%
\pgfusepath{stroke}%
\end{pgfscope}%
\begin{pgfscope}%
\pgfpathrectangle{\pgfqpoint{0.647939in}{0.492442in}}{\pgfqpoint{3.079299in}{3.079299in}}%
\pgfusepath{clip}%
\pgfsetroundcap%
\pgfsetroundjoin%
\definecolor{currentfill}{rgb}{0.500000,0.500000,0.500000}%
\pgfsetfillcolor{currentfill}%
\pgfsetfillopacity{0.300000}%
\pgfsetlinewidth{0.301125pt}%
\definecolor{currentstroke}{rgb}{0.500000,0.500000,0.500000}%
\pgfsetstrokecolor{currentstroke}%
\pgfsetstrokeopacity{0.300000}%
\pgfsetdash{}{0pt}%
\pgfpathmoveto{\pgfqpoint{0.000000in}{0.000000in}}%
\pgfpathlineto{\pgfqpoint{0.000000in}{0.000000in}}%
\pgfpathclose%
\pgfusepath{stroke,fill}%
\end{pgfscope}%
\begin{pgfscope}%
\pgfpathrectangle{\pgfqpoint{0.647939in}{0.492442in}}{\pgfqpoint{3.079299in}{3.079299in}}%
\pgfusepath{clip}%
\pgfsetroundcap%
\pgfsetroundjoin%
\pgfsetlinewidth{0.301125pt}%
\definecolor{currentstroke}{rgb}{0.500000,0.500000,0.500000}%
\pgfsetstrokecolor{currentstroke}%
\pgfsetstrokeopacity{0.300000}%
\pgfsetdash{}{0pt}%
\pgfpathmoveto{\pgfqpoint{2.640181in}{0.758420in}}%
\pgfusepath{stroke}%
\end{pgfscope}%
\begin{pgfscope}%
\pgfpathrectangle{\pgfqpoint{0.647939in}{0.492442in}}{\pgfqpoint{3.079299in}{3.079299in}}%
\pgfusepath{clip}%
\pgfsetroundcap%
\pgfsetroundjoin%
\definecolor{currentfill}{rgb}{0.500000,0.500000,0.500000}%
\pgfsetfillcolor{currentfill}%
\pgfsetfillopacity{0.300000}%
\pgfsetlinewidth{0.301125pt}%
\definecolor{currentstroke}{rgb}{0.500000,0.500000,0.500000}%
\pgfsetstrokecolor{currentstroke}%
\pgfsetstrokeopacity{0.300000}%
\pgfsetdash{}{0pt}%
\pgfpathmoveto{\pgfqpoint{0.000000in}{0.000000in}}%
\pgfpathlineto{\pgfqpoint{0.000000in}{0.000000in}}%
\pgfpathclose%
\pgfusepath{stroke,fill}%
\end{pgfscope}%
\begin{pgfscope}%
\pgfpathrectangle{\pgfqpoint{0.647939in}{0.492442in}}{\pgfqpoint{3.079299in}{3.079299in}}%
\pgfusepath{clip}%
\pgfsetroundcap%
\pgfsetroundjoin%
\pgfsetlinewidth{0.301125pt}%
\definecolor{currentstroke}{rgb}{0.500000,0.500000,0.500000}%
\pgfsetstrokecolor{currentstroke}%
\pgfsetstrokeopacity{0.300000}%
\pgfsetdash{}{0pt}%
\pgfpathmoveto{\pgfqpoint{2.654187in}{0.885454in}}%
\pgfusepath{stroke}%
\end{pgfscope}%
\begin{pgfscope}%
\pgfpathrectangle{\pgfqpoint{0.647939in}{0.492442in}}{\pgfqpoint{3.079299in}{3.079299in}}%
\pgfusepath{clip}%
\pgfsetroundcap%
\pgfsetroundjoin%
\definecolor{currentfill}{rgb}{0.500000,0.500000,0.500000}%
\pgfsetfillcolor{currentfill}%
\pgfsetfillopacity{0.300000}%
\pgfsetlinewidth{0.301125pt}%
\definecolor{currentstroke}{rgb}{0.500000,0.500000,0.500000}%
\pgfsetstrokecolor{currentstroke}%
\pgfsetstrokeopacity{0.300000}%
\pgfsetdash{}{0pt}%
\pgfpathmoveto{\pgfqpoint{0.000000in}{0.000000in}}%
\pgfpathlineto{\pgfqpoint{0.000000in}{0.000000in}}%
\pgfpathclose%
\pgfusepath{stroke,fill}%
\end{pgfscope}%
\begin{pgfscope}%
\pgfpathrectangle{\pgfqpoint{0.647939in}{0.492442in}}{\pgfqpoint{3.079299in}{3.079299in}}%
\pgfusepath{clip}%
\pgfsetroundcap%
\pgfsetroundjoin%
\pgfsetlinewidth{0.301125pt}%
\definecolor{currentstroke}{rgb}{0.500000,0.500000,0.500000}%
\pgfsetstrokecolor{currentstroke}%
\pgfsetstrokeopacity{0.300000}%
\pgfsetdash{}{0pt}%
\pgfpathmoveto{\pgfqpoint{2.590359in}{0.977002in}}%
\pgfusepath{stroke}%
\end{pgfscope}%
\begin{pgfscope}%
\pgfpathrectangle{\pgfqpoint{0.647939in}{0.492442in}}{\pgfqpoint{3.079299in}{3.079299in}}%
\pgfusepath{clip}%
\pgfsetroundcap%
\pgfsetroundjoin%
\definecolor{currentfill}{rgb}{0.500000,0.500000,0.500000}%
\pgfsetfillcolor{currentfill}%
\pgfsetfillopacity{0.300000}%
\pgfsetlinewidth{0.301125pt}%
\definecolor{currentstroke}{rgb}{0.500000,0.500000,0.500000}%
\pgfsetstrokecolor{currentstroke}%
\pgfsetstrokeopacity{0.300000}%
\pgfsetdash{}{0pt}%
\pgfpathmoveto{\pgfqpoint{0.000000in}{0.000000in}}%
\pgfpathlineto{\pgfqpoint{0.000000in}{0.000000in}}%
\pgfpathclose%
\pgfusepath{stroke,fill}%
\end{pgfscope}%
\begin{pgfscope}%
\pgfpathrectangle{\pgfqpoint{0.647939in}{0.492442in}}{\pgfqpoint{3.079299in}{3.079299in}}%
\pgfusepath{clip}%
\pgfsetroundcap%
\pgfsetroundjoin%
\pgfsetlinewidth{0.301125pt}%
\definecolor{currentstroke}{rgb}{0.500000,0.500000,0.500000}%
\pgfsetstrokecolor{currentstroke}%
\pgfsetstrokeopacity{0.300000}%
\pgfsetdash{}{0pt}%
\pgfpathmoveto{\pgfqpoint{2.730179in}{1.040799in}}%
\pgfusepath{stroke}%
\end{pgfscope}%
\begin{pgfscope}%
\pgfpathrectangle{\pgfqpoint{0.647939in}{0.492442in}}{\pgfqpoint{3.079299in}{3.079299in}}%
\pgfusepath{clip}%
\pgfsetroundcap%
\pgfsetroundjoin%
\definecolor{currentfill}{rgb}{0.500000,0.500000,0.500000}%
\pgfsetfillcolor{currentfill}%
\pgfsetfillopacity{0.300000}%
\pgfsetlinewidth{0.301125pt}%
\definecolor{currentstroke}{rgb}{0.500000,0.500000,0.500000}%
\pgfsetstrokecolor{currentstroke}%
\pgfsetstrokeopacity{0.300000}%
\pgfsetdash{}{0pt}%
\pgfpathmoveto{\pgfqpoint{0.000000in}{0.000000in}}%
\pgfpathlineto{\pgfqpoint{0.000000in}{0.000000in}}%
\pgfpathclose%
\pgfusepath{stroke,fill}%
\end{pgfscope}%
\begin{pgfscope}%
\pgfpathrectangle{\pgfqpoint{0.647939in}{0.492442in}}{\pgfqpoint{3.079299in}{3.079299in}}%
\pgfusepath{clip}%
\pgfsetroundcap%
\pgfsetroundjoin%
\pgfsetlinewidth{0.301125pt}%
\definecolor{currentstroke}{rgb}{0.500000,0.500000,0.500000}%
\pgfsetstrokecolor{currentstroke}%
\pgfsetstrokeopacity{0.300000}%
\pgfsetdash{}{0pt}%
\pgfpathmoveto{\pgfqpoint{2.801877in}{1.109899in}}%
\pgfusepath{stroke}%
\end{pgfscope}%
\begin{pgfscope}%
\pgfpathrectangle{\pgfqpoint{0.647939in}{0.492442in}}{\pgfqpoint{3.079299in}{3.079299in}}%
\pgfusepath{clip}%
\pgfsetroundcap%
\pgfsetroundjoin%
\definecolor{currentfill}{rgb}{0.500000,0.500000,0.500000}%
\pgfsetfillcolor{currentfill}%
\pgfsetfillopacity{0.300000}%
\pgfsetlinewidth{0.301125pt}%
\definecolor{currentstroke}{rgb}{0.500000,0.500000,0.500000}%
\pgfsetstrokecolor{currentstroke}%
\pgfsetstrokeopacity{0.300000}%
\pgfsetdash{}{0pt}%
\pgfpathmoveto{\pgfqpoint{0.000000in}{0.000000in}}%
\pgfpathlineto{\pgfqpoint{0.000000in}{0.000000in}}%
\pgfpathclose%
\pgfusepath{stroke,fill}%
\end{pgfscope}%
\begin{pgfscope}%
\pgfpathrectangle{\pgfqpoint{0.647939in}{0.492442in}}{\pgfqpoint{3.079299in}{3.079299in}}%
\pgfusepath{clip}%
\pgfsetroundcap%
\pgfsetroundjoin%
\pgfsetlinewidth{0.301125pt}%
\definecolor{currentstroke}{rgb}{0.500000,0.500000,0.500000}%
\pgfsetstrokecolor{currentstroke}%
\pgfsetstrokeopacity{0.300000}%
\pgfsetdash{}{0pt}%
\pgfpathmoveto{\pgfqpoint{2.674146in}{1.225443in}}%
\pgfusepath{stroke}%
\end{pgfscope}%
\begin{pgfscope}%
\pgfpathrectangle{\pgfqpoint{0.647939in}{0.492442in}}{\pgfqpoint{3.079299in}{3.079299in}}%
\pgfusepath{clip}%
\pgfsetroundcap%
\pgfsetroundjoin%
\definecolor{currentfill}{rgb}{0.500000,0.500000,0.500000}%
\pgfsetfillcolor{currentfill}%
\pgfsetfillopacity{0.300000}%
\pgfsetlinewidth{0.301125pt}%
\definecolor{currentstroke}{rgb}{0.500000,0.500000,0.500000}%
\pgfsetstrokecolor{currentstroke}%
\pgfsetstrokeopacity{0.300000}%
\pgfsetdash{}{0pt}%
\pgfpathmoveto{\pgfqpoint{0.000000in}{0.000000in}}%
\pgfpathlineto{\pgfqpoint{0.000000in}{0.000000in}}%
\pgfpathclose%
\pgfusepath{stroke,fill}%
\end{pgfscope}%
\begin{pgfscope}%
\pgfpathrectangle{\pgfqpoint{0.647939in}{0.492442in}}{\pgfqpoint{3.079299in}{3.079299in}}%
\pgfusepath{clip}%
\pgfsetroundcap%
\pgfsetroundjoin%
\pgfsetlinewidth{0.301125pt}%
\definecolor{currentstroke}{rgb}{0.500000,0.500000,0.500000}%
\pgfsetstrokecolor{currentstroke}%
\pgfsetstrokeopacity{0.300000}%
\pgfsetdash{}{0pt}%
\pgfpathmoveto{\pgfqpoint{2.681217in}{1.314471in}}%
\pgfusepath{stroke}%
\end{pgfscope}%
\begin{pgfscope}%
\pgfpathrectangle{\pgfqpoint{0.647939in}{0.492442in}}{\pgfqpoint{3.079299in}{3.079299in}}%
\pgfusepath{clip}%
\pgfsetroundcap%
\pgfsetroundjoin%
\definecolor{currentfill}{rgb}{0.500000,0.500000,0.500000}%
\pgfsetfillcolor{currentfill}%
\pgfsetfillopacity{0.300000}%
\pgfsetlinewidth{0.301125pt}%
\definecolor{currentstroke}{rgb}{0.500000,0.500000,0.500000}%
\pgfsetstrokecolor{currentstroke}%
\pgfsetstrokeopacity{0.300000}%
\pgfsetdash{}{0pt}%
\pgfpathmoveto{\pgfqpoint{0.000000in}{0.000000in}}%
\pgfpathlineto{\pgfqpoint{0.000000in}{0.000000in}}%
\pgfpathclose%
\pgfusepath{stroke,fill}%
\end{pgfscope}%
\begin{pgfscope}%
\pgfpathrectangle{\pgfqpoint{0.647939in}{0.492442in}}{\pgfqpoint{3.079299in}{3.079299in}}%
\pgfusepath{clip}%
\pgfsetroundcap%
\pgfsetroundjoin%
\pgfsetlinewidth{0.301125pt}%
\definecolor{currentstroke}{rgb}{0.500000,0.500000,0.500000}%
\pgfsetstrokecolor{currentstroke}%
\pgfsetstrokeopacity{0.300000}%
\pgfsetdash{}{0pt}%
\pgfpathmoveto{\pgfqpoint{2.821238in}{1.368864in}}%
\pgfusepath{stroke}%
\end{pgfscope}%
\begin{pgfscope}%
\pgfpathrectangle{\pgfqpoint{0.647939in}{0.492442in}}{\pgfqpoint{3.079299in}{3.079299in}}%
\pgfusepath{clip}%
\pgfsetroundcap%
\pgfsetroundjoin%
\definecolor{currentfill}{rgb}{0.500000,0.500000,0.500000}%
\pgfsetfillcolor{currentfill}%
\pgfsetfillopacity{0.300000}%
\pgfsetlinewidth{0.301125pt}%
\definecolor{currentstroke}{rgb}{0.500000,0.500000,0.500000}%
\pgfsetstrokecolor{currentstroke}%
\pgfsetstrokeopacity{0.300000}%
\pgfsetdash{}{0pt}%
\pgfpathmoveto{\pgfqpoint{0.000000in}{0.000000in}}%
\pgfpathlineto{\pgfqpoint{0.000000in}{0.000000in}}%
\pgfpathclose%
\pgfusepath{stroke,fill}%
\end{pgfscope}%
\begin{pgfscope}%
\pgfpathrectangle{\pgfqpoint{0.647939in}{0.492442in}}{\pgfqpoint{3.079299in}{3.079299in}}%
\pgfusepath{clip}%
\pgfsetroundcap%
\pgfsetroundjoin%
\pgfsetlinewidth{0.301125pt}%
\definecolor{currentstroke}{rgb}{0.500000,0.500000,0.500000}%
\pgfsetstrokecolor{currentstroke}%
\pgfsetstrokeopacity{0.300000}%
\pgfsetdash{}{0pt}%
\pgfpathmoveto{\pgfqpoint{2.765154in}{1.480551in}}%
\pgfusepath{stroke}%
\end{pgfscope}%
\begin{pgfscope}%
\pgfpathrectangle{\pgfqpoint{0.647939in}{0.492442in}}{\pgfqpoint{3.079299in}{3.079299in}}%
\pgfusepath{clip}%
\pgfsetroundcap%
\pgfsetroundjoin%
\definecolor{currentfill}{rgb}{0.500000,0.500000,0.500000}%
\pgfsetfillcolor{currentfill}%
\pgfsetfillopacity{0.300000}%
\pgfsetlinewidth{0.301125pt}%
\definecolor{currentstroke}{rgb}{0.500000,0.500000,0.500000}%
\pgfsetstrokecolor{currentstroke}%
\pgfsetstrokeopacity{0.300000}%
\pgfsetdash{}{0pt}%
\pgfpathmoveto{\pgfqpoint{0.000000in}{0.000000in}}%
\pgfpathlineto{\pgfqpoint{0.000000in}{0.000000in}}%
\pgfpathclose%
\pgfusepath{stroke,fill}%
\end{pgfscope}%
\begin{pgfscope}%
\pgfpathrectangle{\pgfqpoint{0.647939in}{0.492442in}}{\pgfqpoint{3.079299in}{3.079299in}}%
\pgfusepath{clip}%
\pgfsetroundcap%
\pgfsetroundjoin%
\pgfsetlinewidth{0.301125pt}%
\definecolor{currentstroke}{rgb}{0.500000,0.500000,0.500000}%
\pgfsetstrokecolor{currentstroke}%
\pgfsetstrokeopacity{0.300000}%
\pgfsetdash{}{0pt}%
\pgfpathmoveto{\pgfqpoint{2.840288in}{1.550626in}}%
\pgfusepath{stroke}%
\end{pgfscope}%
\begin{pgfscope}%
\pgfpathrectangle{\pgfqpoint{0.647939in}{0.492442in}}{\pgfqpoint{3.079299in}{3.079299in}}%
\pgfusepath{clip}%
\pgfsetroundcap%
\pgfsetroundjoin%
\definecolor{currentfill}{rgb}{0.500000,0.500000,0.500000}%
\pgfsetfillcolor{currentfill}%
\pgfsetfillopacity{0.300000}%
\pgfsetlinewidth{0.301125pt}%
\definecolor{currentstroke}{rgb}{0.500000,0.500000,0.500000}%
\pgfsetstrokecolor{currentstroke}%
\pgfsetstrokeopacity{0.300000}%
\pgfsetdash{}{0pt}%
\pgfpathmoveto{\pgfqpoint{0.000000in}{0.000000in}}%
\pgfpathlineto{\pgfqpoint{0.000000in}{0.000000in}}%
\pgfpathclose%
\pgfusepath{stroke,fill}%
\end{pgfscope}%
\begin{pgfscope}%
\pgfpathrectangle{\pgfqpoint{0.647939in}{0.492442in}}{\pgfqpoint{3.079299in}{3.079299in}}%
\pgfusepath{clip}%
\pgfsetroundcap%
\pgfsetroundjoin%
\pgfsetlinewidth{0.301125pt}%
\definecolor{currentstroke}{rgb}{0.500000,0.500000,0.500000}%
\pgfsetstrokecolor{currentstroke}%
\pgfsetstrokeopacity{0.300000}%
\pgfsetdash{}{0pt}%
\pgfpathmoveto{\pgfqpoint{2.914398in}{1.615639in}}%
\pgfusepath{stroke}%
\end{pgfscope}%
\begin{pgfscope}%
\pgfpathrectangle{\pgfqpoint{0.647939in}{0.492442in}}{\pgfqpoint{3.079299in}{3.079299in}}%
\pgfusepath{clip}%
\pgfsetroundcap%
\pgfsetroundjoin%
\definecolor{currentfill}{rgb}{0.500000,0.500000,0.500000}%
\pgfsetfillcolor{currentfill}%
\pgfsetfillopacity{0.300000}%
\pgfsetlinewidth{0.301125pt}%
\definecolor{currentstroke}{rgb}{0.500000,0.500000,0.500000}%
\pgfsetstrokecolor{currentstroke}%
\pgfsetstrokeopacity{0.300000}%
\pgfsetdash{}{0pt}%
\pgfpathmoveto{\pgfqpoint{0.000000in}{0.000000in}}%
\pgfpathlineto{\pgfqpoint{0.000000in}{0.000000in}}%
\pgfpathclose%
\pgfusepath{stroke,fill}%
\end{pgfscope}%
\begin{pgfscope}%
\pgfpathrectangle{\pgfqpoint{0.647939in}{0.492442in}}{\pgfqpoint{3.079299in}{3.079299in}}%
\pgfusepath{clip}%
\pgfsetroundcap%
\pgfsetroundjoin%
\pgfsetlinewidth{0.301125pt}%
\definecolor{currentstroke}{rgb}{0.500000,0.500000,0.500000}%
\pgfsetstrokecolor{currentstroke}%
\pgfsetstrokeopacity{0.300000}%
\pgfsetdash{}{0pt}%
\pgfpathmoveto{\pgfqpoint{2.927737in}{1.709755in}}%
\pgfusepath{stroke}%
\end{pgfscope}%
\begin{pgfscope}%
\pgfpathrectangle{\pgfqpoint{0.647939in}{0.492442in}}{\pgfqpoint{3.079299in}{3.079299in}}%
\pgfusepath{clip}%
\pgfsetroundcap%
\pgfsetroundjoin%
\definecolor{currentfill}{rgb}{0.500000,0.500000,0.500000}%
\pgfsetfillcolor{currentfill}%
\pgfsetfillopacity{0.300000}%
\pgfsetlinewidth{0.301125pt}%
\definecolor{currentstroke}{rgb}{0.500000,0.500000,0.500000}%
\pgfsetstrokecolor{currentstroke}%
\pgfsetstrokeopacity{0.300000}%
\pgfsetdash{}{0pt}%
\pgfpathmoveto{\pgfqpoint{0.000000in}{0.000000in}}%
\pgfpathlineto{\pgfqpoint{0.000000in}{0.000000in}}%
\pgfpathclose%
\pgfusepath{stroke,fill}%
\end{pgfscope}%
\begin{pgfscope}%
\pgfpathrectangle{\pgfqpoint{0.647939in}{0.492442in}}{\pgfqpoint{3.079299in}{3.079299in}}%
\pgfusepath{clip}%
\pgfsetroundcap%
\pgfsetroundjoin%
\pgfsetlinewidth{0.301125pt}%
\definecolor{currentstroke}{rgb}{0.500000,0.500000,0.500000}%
\pgfsetstrokecolor{currentstroke}%
\pgfsetstrokeopacity{0.300000}%
\pgfsetdash{}{0pt}%
\pgfpathmoveto{\pgfqpoint{2.886295in}{1.843484in}}%
\pgfusepath{stroke}%
\end{pgfscope}%
\begin{pgfscope}%
\pgfpathrectangle{\pgfqpoint{0.647939in}{0.492442in}}{\pgfqpoint{3.079299in}{3.079299in}}%
\pgfusepath{clip}%
\pgfsetroundcap%
\pgfsetroundjoin%
\definecolor{currentfill}{rgb}{0.500000,0.500000,0.500000}%
\pgfsetfillcolor{currentfill}%
\pgfsetfillopacity{0.300000}%
\pgfsetlinewidth{0.301125pt}%
\definecolor{currentstroke}{rgb}{0.500000,0.500000,0.500000}%
\pgfsetstrokecolor{currentstroke}%
\pgfsetstrokeopacity{0.300000}%
\pgfsetdash{}{0pt}%
\pgfpathmoveto{\pgfqpoint{0.000000in}{0.000000in}}%
\pgfpathlineto{\pgfqpoint{0.000000in}{0.000000in}}%
\pgfpathclose%
\pgfusepath{stroke,fill}%
\end{pgfscope}%
\begin{pgfscope}%
\pgfpathrectangle{\pgfqpoint{0.647939in}{0.492442in}}{\pgfqpoint{3.079299in}{3.079299in}}%
\pgfusepath{clip}%
\pgfsetroundcap%
\pgfsetroundjoin%
\pgfsetlinewidth{0.301125pt}%
\definecolor{currentstroke}{rgb}{0.500000,0.500000,0.500000}%
\pgfsetstrokecolor{currentstroke}%
\pgfsetstrokeopacity{0.300000}%
\pgfsetdash{}{0pt}%
\pgfpathmoveto{\pgfqpoint{2.688932in}{2.437500in}}%
\pgfusepath{stroke}%
\end{pgfscope}%
\begin{pgfscope}%
\pgfpathrectangle{\pgfqpoint{0.647939in}{0.492442in}}{\pgfqpoint{3.079299in}{3.079299in}}%
\pgfusepath{clip}%
\pgfsetroundcap%
\pgfsetroundjoin%
\definecolor{currentfill}{rgb}{0.500000,0.500000,0.500000}%
\pgfsetfillcolor{currentfill}%
\pgfsetfillopacity{0.300000}%
\pgfsetlinewidth{0.301125pt}%
\definecolor{currentstroke}{rgb}{0.500000,0.500000,0.500000}%
\pgfsetstrokecolor{currentstroke}%
\pgfsetstrokeopacity{0.300000}%
\pgfsetdash{}{0pt}%
\pgfpathmoveto{\pgfqpoint{0.000000in}{0.000000in}}%
\pgfpathlineto{\pgfqpoint{0.000000in}{0.000000in}}%
\pgfpathclose%
\pgfusepath{stroke,fill}%
\end{pgfscope}%
\begin{pgfscope}%
\pgfpathrectangle{\pgfqpoint{0.647939in}{0.492442in}}{\pgfqpoint{3.079299in}{3.079299in}}%
\pgfusepath{clip}%
\pgfsetroundcap%
\pgfsetroundjoin%
\pgfsetlinewidth{0.301125pt}%
\definecolor{currentstroke}{rgb}{0.500000,0.500000,0.500000}%
\pgfsetstrokecolor{currentstroke}%
\pgfsetstrokeopacity{0.300000}%
\pgfsetdash{}{0pt}%
\pgfpathmoveto{\pgfqpoint{2.966781in}{2.407136in}}%
\pgfusepath{stroke}%
\end{pgfscope}%
\begin{pgfscope}%
\pgfpathrectangle{\pgfqpoint{0.647939in}{0.492442in}}{\pgfqpoint{3.079299in}{3.079299in}}%
\pgfusepath{clip}%
\pgfsetroundcap%
\pgfsetroundjoin%
\definecolor{currentfill}{rgb}{0.500000,0.500000,0.500000}%
\pgfsetfillcolor{currentfill}%
\pgfsetfillopacity{0.300000}%
\pgfsetlinewidth{0.301125pt}%
\definecolor{currentstroke}{rgb}{0.500000,0.500000,0.500000}%
\pgfsetstrokecolor{currentstroke}%
\pgfsetstrokeopacity{0.300000}%
\pgfsetdash{}{0pt}%
\pgfpathmoveto{\pgfqpoint{0.000000in}{0.000000in}}%
\pgfpathlineto{\pgfqpoint{0.000000in}{0.000000in}}%
\pgfpathclose%
\pgfusepath{stroke,fill}%
\end{pgfscope}%
\begin{pgfscope}%
\pgfpathrectangle{\pgfqpoint{0.647939in}{0.492442in}}{\pgfqpoint{3.079299in}{3.079299in}}%
\pgfusepath{clip}%
\pgfsetroundcap%
\pgfsetroundjoin%
\pgfsetlinewidth{0.301125pt}%
\definecolor{currentstroke}{rgb}{0.500000,0.500000,0.500000}%
\pgfsetstrokecolor{currentstroke}%
\pgfsetstrokeopacity{0.300000}%
\pgfsetdash{}{0pt}%
\pgfpathmoveto{\pgfqpoint{3.150533in}{2.440264in}}%
\pgfusepath{stroke}%
\end{pgfscope}%
\begin{pgfscope}%
\pgfpathrectangle{\pgfqpoint{0.647939in}{0.492442in}}{\pgfqpoint{3.079299in}{3.079299in}}%
\pgfusepath{clip}%
\pgfsetroundcap%
\pgfsetroundjoin%
\definecolor{currentfill}{rgb}{0.500000,0.500000,0.500000}%
\pgfsetfillcolor{currentfill}%
\pgfsetfillopacity{0.300000}%
\pgfsetlinewidth{0.301125pt}%
\definecolor{currentstroke}{rgb}{0.500000,0.500000,0.500000}%
\pgfsetstrokecolor{currentstroke}%
\pgfsetstrokeopacity{0.300000}%
\pgfsetdash{}{0pt}%
\pgfpathmoveto{\pgfqpoint{0.000000in}{0.000000in}}%
\pgfpathlineto{\pgfqpoint{0.000000in}{0.000000in}}%
\pgfpathclose%
\pgfusepath{stroke,fill}%
\end{pgfscope}%
\begin{pgfscope}%
\pgfpathrectangle{\pgfqpoint{0.647939in}{0.492442in}}{\pgfqpoint{3.079299in}{3.079299in}}%
\pgfusepath{clip}%
\pgfsetroundcap%
\pgfsetroundjoin%
\pgfsetlinewidth{0.301125pt}%
\definecolor{currentstroke}{rgb}{0.500000,0.500000,0.500000}%
\pgfsetstrokecolor{currentstroke}%
\pgfsetstrokeopacity{0.300000}%
\pgfsetdash{}{0pt}%
\pgfpathmoveto{\pgfqpoint{3.272740in}{2.650294in}}%
\pgfusepath{stroke}%
\end{pgfscope}%
\begin{pgfscope}%
\pgfpathrectangle{\pgfqpoint{0.647939in}{0.492442in}}{\pgfqpoint{3.079299in}{3.079299in}}%
\pgfusepath{clip}%
\pgfsetroundcap%
\pgfsetroundjoin%
\definecolor{currentfill}{rgb}{0.500000,0.500000,0.500000}%
\pgfsetfillcolor{currentfill}%
\pgfsetfillopacity{0.300000}%
\pgfsetlinewidth{0.301125pt}%
\definecolor{currentstroke}{rgb}{0.500000,0.500000,0.500000}%
\pgfsetstrokecolor{currentstroke}%
\pgfsetstrokeopacity{0.300000}%
\pgfsetdash{}{0pt}%
\pgfpathmoveto{\pgfqpoint{0.000000in}{0.000000in}}%
\pgfpathlineto{\pgfqpoint{0.000000in}{0.000000in}}%
\pgfpathclose%
\pgfusepath{stroke,fill}%
\end{pgfscope}%
\begin{pgfscope}%
\pgfpathrectangle{\pgfqpoint{0.647939in}{0.492442in}}{\pgfqpoint{3.079299in}{3.079299in}}%
\pgfusepath{clip}%
\pgfsetroundcap%
\pgfsetroundjoin%
\pgfsetlinewidth{0.301125pt}%
\definecolor{currentstroke}{rgb}{0.500000,0.500000,0.500000}%
\pgfsetstrokecolor{currentstroke}%
\pgfsetstrokeopacity{0.300000}%
\pgfsetdash{}{0pt}%
\pgfpathmoveto{\pgfqpoint{3.515938in}{2.175558in}}%
\pgfusepath{stroke}%
\end{pgfscope}%
\begin{pgfscope}%
\pgfpathrectangle{\pgfqpoint{0.647939in}{0.492442in}}{\pgfqpoint{3.079299in}{3.079299in}}%
\pgfusepath{clip}%
\pgfsetroundcap%
\pgfsetroundjoin%
\definecolor{currentfill}{rgb}{0.500000,0.500000,0.500000}%
\pgfsetfillcolor{currentfill}%
\pgfsetfillopacity{0.300000}%
\pgfsetlinewidth{0.301125pt}%
\definecolor{currentstroke}{rgb}{0.500000,0.500000,0.500000}%
\pgfsetstrokecolor{currentstroke}%
\pgfsetstrokeopacity{0.300000}%
\pgfsetdash{}{0pt}%
\pgfpathmoveto{\pgfqpoint{0.000000in}{0.000000in}}%
\pgfpathlineto{\pgfqpoint{0.000000in}{0.000000in}}%
\pgfpathclose%
\pgfusepath{stroke,fill}%
\end{pgfscope}%
\begin{pgfscope}%
\pgfpathrectangle{\pgfqpoint{0.647939in}{0.492442in}}{\pgfqpoint{3.079299in}{3.079299in}}%
\pgfusepath{clip}%
\pgfsetroundcap%
\pgfsetroundjoin%
\pgfsetlinewidth{0.301125pt}%
\definecolor{currentstroke}{rgb}{0.500000,0.500000,0.500000}%
\pgfsetstrokecolor{currentstroke}%
\pgfsetstrokeopacity{0.300000}%
\pgfsetdash{}{0pt}%
\pgfpathmoveto{\pgfqpoint{3.453620in}{2.734108in}}%
\pgfusepath{stroke}%
\end{pgfscope}%
\begin{pgfscope}%
\pgfpathrectangle{\pgfqpoint{0.647939in}{0.492442in}}{\pgfqpoint{3.079299in}{3.079299in}}%
\pgfusepath{clip}%
\pgfsetroundcap%
\pgfsetroundjoin%
\definecolor{currentfill}{rgb}{0.500000,0.500000,0.500000}%
\pgfsetfillcolor{currentfill}%
\pgfsetfillopacity{0.300000}%
\pgfsetlinewidth{0.301125pt}%
\definecolor{currentstroke}{rgb}{0.500000,0.500000,0.500000}%
\pgfsetstrokecolor{currentstroke}%
\pgfsetstrokeopacity{0.300000}%
\pgfsetdash{}{0pt}%
\pgfpathmoveto{\pgfqpoint{0.000000in}{0.000000in}}%
\pgfpathlineto{\pgfqpoint{0.000000in}{0.000000in}}%
\pgfpathclose%
\pgfusepath{stroke,fill}%
\end{pgfscope}%
\begin{pgfscope}%
\pgfpathrectangle{\pgfqpoint{0.647939in}{0.492442in}}{\pgfqpoint{3.079299in}{3.079299in}}%
\pgfusepath{clip}%
\pgfsetroundcap%
\pgfsetroundjoin%
\pgfsetlinewidth{0.301125pt}%
\definecolor{currentstroke}{rgb}{0.500000,0.500000,0.500000}%
\pgfsetstrokecolor{currentstroke}%
\pgfsetstrokeopacity{0.300000}%
\pgfsetdash{}{0pt}%
\pgfpathmoveto{\pgfqpoint{3.550788in}{2.705764in}}%
\pgfusepath{stroke}%
\end{pgfscope}%
\begin{pgfscope}%
\pgfpathrectangle{\pgfqpoint{0.647939in}{0.492442in}}{\pgfqpoint{3.079299in}{3.079299in}}%
\pgfusepath{clip}%
\pgfsetroundcap%
\pgfsetroundjoin%
\definecolor{currentfill}{rgb}{0.500000,0.500000,0.500000}%
\pgfsetfillcolor{currentfill}%
\pgfsetfillopacity{0.300000}%
\pgfsetlinewidth{0.301125pt}%
\definecolor{currentstroke}{rgb}{0.500000,0.500000,0.500000}%
\pgfsetstrokecolor{currentstroke}%
\pgfsetstrokeopacity{0.300000}%
\pgfsetdash{}{0pt}%
\pgfpathmoveto{\pgfqpoint{0.000000in}{0.000000in}}%
\pgfpathlineto{\pgfqpoint{0.000000in}{0.000000in}}%
\pgfpathclose%
\pgfusepath{stroke,fill}%
\end{pgfscope}%
\begin{pgfscope}%
\pgfpathrectangle{\pgfqpoint{0.647939in}{0.492442in}}{\pgfqpoint{3.079299in}{3.079299in}}%
\pgfusepath{clip}%
\pgfsetroundcap%
\pgfsetroundjoin%
\pgfsetlinewidth{0.301125pt}%
\definecolor{currentstroke}{rgb}{0.500000,0.500000,0.500000}%
\pgfsetstrokecolor{currentstroke}%
\pgfsetstrokeopacity{0.300000}%
\pgfsetdash{}{0pt}%
\pgfpathmoveto{\pgfqpoint{3.631078in}{2.733709in}}%
\pgfusepath{stroke}%
\end{pgfscope}%
\begin{pgfscope}%
\pgfpathrectangle{\pgfqpoint{0.647939in}{0.492442in}}{\pgfqpoint{3.079299in}{3.079299in}}%
\pgfusepath{clip}%
\pgfsetroundcap%
\pgfsetroundjoin%
\definecolor{currentfill}{rgb}{0.500000,0.500000,0.500000}%
\pgfsetfillcolor{currentfill}%
\pgfsetfillopacity{0.300000}%
\pgfsetlinewidth{0.301125pt}%
\definecolor{currentstroke}{rgb}{0.500000,0.500000,0.500000}%
\pgfsetstrokecolor{currentstroke}%
\pgfsetstrokeopacity{0.300000}%
\pgfsetdash{}{0pt}%
\pgfpathmoveto{\pgfqpoint{0.000000in}{0.000000in}}%
\pgfpathlineto{\pgfqpoint{0.000000in}{0.000000in}}%
\pgfpathclose%
\pgfusepath{stroke,fill}%
\end{pgfscope}%
\begin{pgfscope}%
\pgfpathrectangle{\pgfqpoint{0.647939in}{0.492442in}}{\pgfqpoint{3.079299in}{3.079299in}}%
\pgfusepath{clip}%
\pgfsetroundcap%
\pgfsetroundjoin%
\pgfsetlinewidth{0.301125pt}%
\definecolor{currentstroke}{rgb}{0.500000,0.500000,0.500000}%
\pgfsetstrokecolor{currentstroke}%
\pgfsetstrokeopacity{0.300000}%
\pgfsetdash{}{0pt}%
\pgfpathmoveto{\pgfqpoint{3.694570in}{2.700175in}}%
\pgfusepath{stroke}%
\end{pgfscope}%
\begin{pgfscope}%
\pgfpathrectangle{\pgfqpoint{0.647939in}{0.492442in}}{\pgfqpoint{3.079299in}{3.079299in}}%
\pgfusepath{clip}%
\pgfsetroundcap%
\pgfsetroundjoin%
\definecolor{currentfill}{rgb}{0.500000,0.500000,0.500000}%
\pgfsetfillcolor{currentfill}%
\pgfsetfillopacity{0.300000}%
\pgfsetlinewidth{0.301125pt}%
\definecolor{currentstroke}{rgb}{0.500000,0.500000,0.500000}%
\pgfsetstrokecolor{currentstroke}%
\pgfsetstrokeopacity{0.300000}%
\pgfsetdash{}{0pt}%
\pgfpathmoveto{\pgfqpoint{0.000000in}{0.000000in}}%
\pgfpathlineto{\pgfqpoint{0.000000in}{0.000000in}}%
\pgfpathclose%
\pgfusepath{stroke,fill}%
\end{pgfscope}%
\begin{pgfscope}%
\pgfpathrectangle{\pgfqpoint{0.647939in}{0.492442in}}{\pgfqpoint{3.079299in}{3.079299in}}%
\pgfusepath{clip}%
\pgfsetroundcap%
\pgfsetroundjoin%
\pgfsetlinewidth{0.301125pt}%
\definecolor{currentstroke}{rgb}{0.500000,0.500000,0.500000}%
\pgfsetstrokecolor{currentstroke}%
\pgfsetstrokeopacity{0.300000}%
\pgfsetdash{}{0pt}%
\pgfpathmoveto{\pgfqpoint{2.153564in}{2.864959in}}%
\pgfusepath{stroke}%
\end{pgfscope}%
\begin{pgfscope}%
\pgfpathrectangle{\pgfqpoint{0.647939in}{0.492442in}}{\pgfqpoint{3.079299in}{3.079299in}}%
\pgfusepath{clip}%
\pgfsetroundcap%
\pgfsetroundjoin%
\definecolor{currentfill}{rgb}{0.500000,0.500000,0.500000}%
\pgfsetfillcolor{currentfill}%
\pgfsetfillopacity{0.300000}%
\pgfsetlinewidth{0.301125pt}%
\definecolor{currentstroke}{rgb}{0.500000,0.500000,0.500000}%
\pgfsetstrokecolor{currentstroke}%
\pgfsetstrokeopacity{0.300000}%
\pgfsetdash{}{0pt}%
\pgfpathmoveto{\pgfqpoint{0.000000in}{0.000000in}}%
\pgfpathlineto{\pgfqpoint{0.000000in}{0.000000in}}%
\pgfpathclose%
\pgfusepath{stroke,fill}%
\end{pgfscope}%
\begin{pgfscope}%
\pgfpathrectangle{\pgfqpoint{0.647939in}{0.492442in}}{\pgfqpoint{3.079299in}{3.079299in}}%
\pgfusepath{clip}%
\pgfsetroundcap%
\pgfsetroundjoin%
\pgfsetlinewidth{0.301125pt}%
\definecolor{currentstroke}{rgb}{0.500000,0.500000,0.500000}%
\pgfsetstrokecolor{currentstroke}%
\pgfsetstrokeopacity{0.300000}%
\pgfsetdash{}{0pt}%
\pgfpathmoveto{\pgfqpoint{2.029347in}{3.128187in}}%
\pgfusepath{stroke}%
\end{pgfscope}%
\begin{pgfscope}%
\pgfpathrectangle{\pgfqpoint{0.647939in}{0.492442in}}{\pgfqpoint{3.079299in}{3.079299in}}%
\pgfusepath{clip}%
\pgfsetroundcap%
\pgfsetroundjoin%
\definecolor{currentfill}{rgb}{0.500000,0.500000,0.500000}%
\pgfsetfillcolor{currentfill}%
\pgfsetfillopacity{0.300000}%
\pgfsetlinewidth{0.301125pt}%
\definecolor{currentstroke}{rgb}{0.500000,0.500000,0.500000}%
\pgfsetstrokecolor{currentstroke}%
\pgfsetstrokeopacity{0.300000}%
\pgfsetdash{}{0pt}%
\pgfpathmoveto{\pgfqpoint{0.000000in}{0.000000in}}%
\pgfpathlineto{\pgfqpoint{0.000000in}{0.000000in}}%
\pgfpathclose%
\pgfusepath{stroke,fill}%
\end{pgfscope}%
\begin{pgfscope}%
\pgfpathrectangle{\pgfqpoint{0.647939in}{0.492442in}}{\pgfqpoint{3.079299in}{3.079299in}}%
\pgfusepath{clip}%
\pgfsetroundcap%
\pgfsetroundjoin%
\pgfsetlinewidth{0.301125pt}%
\definecolor{currentstroke}{rgb}{0.500000,0.500000,0.500000}%
\pgfsetstrokecolor{currentstroke}%
\pgfsetstrokeopacity{0.300000}%
\pgfsetdash{}{0pt}%
\pgfpathmoveto{\pgfqpoint{1.895154in}{3.288462in}}%
\pgfusepath{stroke}%
\end{pgfscope}%
\begin{pgfscope}%
\pgfpathrectangle{\pgfqpoint{0.647939in}{0.492442in}}{\pgfqpoint{3.079299in}{3.079299in}}%
\pgfusepath{clip}%
\pgfsetroundcap%
\pgfsetroundjoin%
\definecolor{currentfill}{rgb}{0.500000,0.500000,0.500000}%
\pgfsetfillcolor{currentfill}%
\pgfsetfillopacity{0.300000}%
\pgfsetlinewidth{0.301125pt}%
\definecolor{currentstroke}{rgb}{0.500000,0.500000,0.500000}%
\pgfsetstrokecolor{currentstroke}%
\pgfsetstrokeopacity{0.300000}%
\pgfsetdash{}{0pt}%
\pgfpathmoveto{\pgfqpoint{0.000000in}{0.000000in}}%
\pgfpathlineto{\pgfqpoint{0.000000in}{0.000000in}}%
\pgfpathclose%
\pgfusepath{stroke,fill}%
\end{pgfscope}%
\begin{pgfscope}%
\pgfpathrectangle{\pgfqpoint{0.647939in}{0.492442in}}{\pgfqpoint{3.079299in}{3.079299in}}%
\pgfusepath{clip}%
\pgfsetroundcap%
\pgfsetroundjoin%
\pgfsetlinewidth{0.301125pt}%
\definecolor{currentstroke}{rgb}{0.500000,0.500000,0.500000}%
\pgfsetstrokecolor{currentstroke}%
\pgfsetstrokeopacity{0.300000}%
\pgfsetdash{}{0pt}%
\pgfpathmoveto{\pgfqpoint{1.855928in}{3.400633in}}%
\pgfusepath{stroke}%
\end{pgfscope}%
\begin{pgfscope}%
\pgfpathrectangle{\pgfqpoint{0.647939in}{0.492442in}}{\pgfqpoint{3.079299in}{3.079299in}}%
\pgfusepath{clip}%
\pgfsetroundcap%
\pgfsetroundjoin%
\definecolor{currentfill}{rgb}{0.500000,0.500000,0.500000}%
\pgfsetfillcolor{currentfill}%
\pgfsetfillopacity{0.300000}%
\pgfsetlinewidth{0.301125pt}%
\definecolor{currentstroke}{rgb}{0.500000,0.500000,0.500000}%
\pgfsetstrokecolor{currentstroke}%
\pgfsetstrokeopacity{0.300000}%
\pgfsetdash{}{0pt}%
\pgfpathmoveto{\pgfqpoint{0.000000in}{0.000000in}}%
\pgfpathlineto{\pgfqpoint{0.000000in}{0.000000in}}%
\pgfpathclose%
\pgfusepath{stroke,fill}%
\end{pgfscope}%
\begin{pgfscope}%
\pgfpathrectangle{\pgfqpoint{0.647939in}{0.492442in}}{\pgfqpoint{3.079299in}{3.079299in}}%
\pgfusepath{clip}%
\pgfsetroundcap%
\pgfsetroundjoin%
\pgfsetlinewidth{0.301125pt}%
\definecolor{currentstroke}{rgb}{0.500000,0.500000,0.500000}%
\pgfsetstrokecolor{currentstroke}%
\pgfsetstrokeopacity{0.300000}%
\pgfsetdash{}{0pt}%
\pgfpathmoveto{\pgfqpoint{1.765067in}{3.473569in}}%
\pgfusepath{stroke}%
\end{pgfscope}%
\begin{pgfscope}%
\pgfpathrectangle{\pgfqpoint{0.647939in}{0.492442in}}{\pgfqpoint{3.079299in}{3.079299in}}%
\pgfusepath{clip}%
\pgfsetroundcap%
\pgfsetroundjoin%
\definecolor{currentfill}{rgb}{0.500000,0.500000,0.500000}%
\pgfsetfillcolor{currentfill}%
\pgfsetfillopacity{0.300000}%
\pgfsetlinewidth{0.301125pt}%
\definecolor{currentstroke}{rgb}{0.500000,0.500000,0.500000}%
\pgfsetstrokecolor{currentstroke}%
\pgfsetstrokeopacity{0.300000}%
\pgfsetdash{}{0pt}%
\pgfpathmoveto{\pgfqpoint{0.000000in}{0.000000in}}%
\pgfpathlineto{\pgfqpoint{0.000000in}{0.000000in}}%
\pgfpathclose%
\pgfusepath{stroke,fill}%
\end{pgfscope}%
\begin{pgfscope}%
\pgfpathrectangle{\pgfqpoint{0.647939in}{0.492442in}}{\pgfqpoint{3.079299in}{3.079299in}}%
\pgfusepath{clip}%
\pgfsetroundcap%
\pgfsetroundjoin%
\pgfsetlinewidth{0.301125pt}%
\definecolor{currentstroke}{rgb}{0.500000,0.500000,0.500000}%
\pgfsetstrokecolor{currentstroke}%
\pgfsetstrokeopacity{0.300000}%
\pgfsetdash{}{0pt}%
\pgfpathmoveto{\pgfqpoint{1.681342in}{3.533955in}}%
\pgfusepath{stroke}%
\end{pgfscope}%
\begin{pgfscope}%
\pgfpathrectangle{\pgfqpoint{0.647939in}{0.492442in}}{\pgfqpoint{3.079299in}{3.079299in}}%
\pgfusepath{clip}%
\pgfsetroundcap%
\pgfsetroundjoin%
\definecolor{currentfill}{rgb}{0.500000,0.500000,0.500000}%
\pgfsetfillcolor{currentfill}%
\pgfsetfillopacity{0.300000}%
\pgfsetlinewidth{0.301125pt}%
\definecolor{currentstroke}{rgb}{0.500000,0.500000,0.500000}%
\pgfsetstrokecolor{currentstroke}%
\pgfsetstrokeopacity{0.300000}%
\pgfsetdash{}{0pt}%
\pgfpathmoveto{\pgfqpoint{0.000000in}{0.000000in}}%
\pgfpathlineto{\pgfqpoint{0.000000in}{0.000000in}}%
\pgfpathclose%
\pgfusepath{stroke,fill}%
\end{pgfscope}%
\begin{pgfscope}%
\pgfpathrectangle{\pgfqpoint{0.647939in}{0.492442in}}{\pgfqpoint{3.079299in}{3.079299in}}%
\pgfusepath{clip}%
\pgfsetroundcap%
\pgfsetroundjoin%
\pgfsetlinewidth{0.301125pt}%
\definecolor{currentstroke}{rgb}{0.500000,0.500000,0.500000}%
\pgfsetstrokecolor{currentstroke}%
\pgfsetstrokeopacity{0.300000}%
\pgfsetdash{}{0pt}%
\pgfpathmoveto{\pgfqpoint{1.248248in}{3.516659in}}%
\pgfusepath{stroke}%
\end{pgfscope}%
\begin{pgfscope}%
\pgfpathrectangle{\pgfqpoint{0.647939in}{0.492442in}}{\pgfqpoint{3.079299in}{3.079299in}}%
\pgfusepath{clip}%
\pgfsetroundcap%
\pgfsetroundjoin%
\definecolor{currentfill}{rgb}{0.500000,0.500000,0.500000}%
\pgfsetfillcolor{currentfill}%
\pgfsetfillopacity{0.300000}%
\pgfsetlinewidth{0.301125pt}%
\definecolor{currentstroke}{rgb}{0.500000,0.500000,0.500000}%
\pgfsetstrokecolor{currentstroke}%
\pgfsetstrokeopacity{0.300000}%
\pgfsetdash{}{0pt}%
\pgfpathmoveto{\pgfqpoint{0.000000in}{0.000000in}}%
\pgfpathlineto{\pgfqpoint{0.000000in}{0.000000in}}%
\pgfpathclose%
\pgfusepath{stroke,fill}%
\end{pgfscope}%
\begin{pgfscope}%
\pgfpathrectangle{\pgfqpoint{0.647939in}{0.492442in}}{\pgfqpoint{3.079299in}{3.079299in}}%
\pgfusepath{clip}%
\pgfsetroundcap%
\pgfsetroundjoin%
\pgfsetlinewidth{0.301125pt}%
\definecolor{currentstroke}{rgb}{0.500000,0.500000,0.500000}%
\pgfsetstrokecolor{currentstroke}%
\pgfsetstrokeopacity{0.300000}%
\pgfsetdash{}{0pt}%
\pgfpathmoveto{\pgfqpoint{0.895351in}{3.530662in}}%
\pgfusepath{stroke}%
\end{pgfscope}%
\begin{pgfscope}%
\pgfpathrectangle{\pgfqpoint{0.647939in}{0.492442in}}{\pgfqpoint{3.079299in}{3.079299in}}%
\pgfusepath{clip}%
\pgfsetroundcap%
\pgfsetroundjoin%
\definecolor{currentfill}{rgb}{0.500000,0.500000,0.500000}%
\pgfsetfillcolor{currentfill}%
\pgfsetfillopacity{0.300000}%
\pgfsetlinewidth{0.301125pt}%
\definecolor{currentstroke}{rgb}{0.500000,0.500000,0.500000}%
\pgfsetstrokecolor{currentstroke}%
\pgfsetstrokeopacity{0.300000}%
\pgfsetdash{}{0pt}%
\pgfpathmoveto{\pgfqpoint{0.000000in}{0.000000in}}%
\pgfpathlineto{\pgfqpoint{0.000000in}{0.000000in}}%
\pgfpathclose%
\pgfusepath{stroke,fill}%
\end{pgfscope}%
\begin{pgfscope}%
\pgfpathrectangle{\pgfqpoint{0.647939in}{0.492442in}}{\pgfqpoint{3.079299in}{3.079299in}}%
\pgfusepath{clip}%
\pgfsetroundcap%
\pgfsetroundjoin%
\pgfsetlinewidth{0.301125pt}%
\definecolor{currentstroke}{rgb}{0.500000,0.500000,0.500000}%
\pgfsetstrokecolor{currentstroke}%
\pgfsetstrokeopacity{0.300000}%
\pgfsetdash{}{0pt}%
\pgfpathmoveto{\pgfqpoint{1.820856in}{3.213031in}}%
\pgfusepath{stroke}%
\end{pgfscope}%
\begin{pgfscope}%
\pgfpathrectangle{\pgfqpoint{0.647939in}{0.492442in}}{\pgfqpoint{3.079299in}{3.079299in}}%
\pgfusepath{clip}%
\pgfsetroundcap%
\pgfsetroundjoin%
\definecolor{currentfill}{rgb}{0.500000,0.500000,0.500000}%
\pgfsetfillcolor{currentfill}%
\pgfsetfillopacity{0.300000}%
\pgfsetlinewidth{0.301125pt}%
\definecolor{currentstroke}{rgb}{0.500000,0.500000,0.500000}%
\pgfsetstrokecolor{currentstroke}%
\pgfsetstrokeopacity{0.300000}%
\pgfsetdash{}{0pt}%
\pgfpathmoveto{\pgfqpoint{0.000000in}{0.000000in}}%
\pgfpathlineto{\pgfqpoint{0.000000in}{0.000000in}}%
\pgfpathclose%
\pgfusepath{stroke,fill}%
\end{pgfscope}%
\begin{pgfscope}%
\pgfpathrectangle{\pgfqpoint{0.647939in}{0.492442in}}{\pgfqpoint{3.079299in}{3.079299in}}%
\pgfusepath{clip}%
\pgfsetroundcap%
\pgfsetroundjoin%
\pgfsetlinewidth{0.301125pt}%
\definecolor{currentstroke}{rgb}{0.500000,0.500000,0.500000}%
\pgfsetstrokecolor{currentstroke}%
\pgfsetstrokeopacity{0.300000}%
\pgfsetdash{}{0pt}%
\pgfpathmoveto{\pgfqpoint{1.749918in}{3.011555in}}%
\pgfusepath{stroke}%
\end{pgfscope}%
\begin{pgfscope}%
\pgfpathrectangle{\pgfqpoint{0.647939in}{0.492442in}}{\pgfqpoint{3.079299in}{3.079299in}}%
\pgfusepath{clip}%
\pgfsetroundcap%
\pgfsetroundjoin%
\definecolor{currentfill}{rgb}{0.500000,0.500000,0.500000}%
\pgfsetfillcolor{currentfill}%
\pgfsetfillopacity{0.300000}%
\pgfsetlinewidth{0.301125pt}%
\definecolor{currentstroke}{rgb}{0.500000,0.500000,0.500000}%
\pgfsetstrokecolor{currentstroke}%
\pgfsetstrokeopacity{0.300000}%
\pgfsetdash{}{0pt}%
\pgfpathmoveto{\pgfqpoint{0.000000in}{0.000000in}}%
\pgfpathlineto{\pgfqpoint{0.000000in}{0.000000in}}%
\pgfpathclose%
\pgfusepath{stroke,fill}%
\end{pgfscope}%
\begin{pgfscope}%
\pgfpathrectangle{\pgfqpoint{0.647939in}{0.492442in}}{\pgfqpoint{3.079299in}{3.079299in}}%
\pgfusepath{clip}%
\pgfsetroundcap%
\pgfsetroundjoin%
\pgfsetlinewidth{0.301125pt}%
\definecolor{currentstroke}{rgb}{0.500000,0.500000,0.500000}%
\pgfsetstrokecolor{currentstroke}%
\pgfsetstrokeopacity{0.300000}%
\pgfsetdash{}{0pt}%
\pgfpathmoveto{\pgfqpoint{1.080002in}{2.804260in}}%
\pgfusepath{stroke}%
\end{pgfscope}%
\begin{pgfscope}%
\pgfpathrectangle{\pgfqpoint{0.647939in}{0.492442in}}{\pgfqpoint{3.079299in}{3.079299in}}%
\pgfusepath{clip}%
\pgfsetroundcap%
\pgfsetroundjoin%
\definecolor{currentfill}{rgb}{0.500000,0.500000,0.500000}%
\pgfsetfillcolor{currentfill}%
\pgfsetfillopacity{0.300000}%
\pgfsetlinewidth{0.301125pt}%
\definecolor{currentstroke}{rgb}{0.500000,0.500000,0.500000}%
\pgfsetstrokecolor{currentstroke}%
\pgfsetstrokeopacity{0.300000}%
\pgfsetdash{}{0pt}%
\pgfpathmoveto{\pgfqpoint{0.000000in}{0.000000in}}%
\pgfpathlineto{\pgfqpoint{0.000000in}{0.000000in}}%
\pgfpathclose%
\pgfusepath{stroke,fill}%
\end{pgfscope}%
\begin{pgfscope}%
\pgfpathrectangle{\pgfqpoint{0.647939in}{0.492442in}}{\pgfqpoint{3.079299in}{3.079299in}}%
\pgfusepath{clip}%
\pgfsetroundcap%
\pgfsetroundjoin%
\pgfsetlinewidth{0.301125pt}%
\definecolor{currentstroke}{rgb}{0.500000,0.500000,0.500000}%
\pgfsetstrokecolor{currentstroke}%
\pgfsetstrokeopacity{0.300000}%
\pgfsetdash{}{0pt}%
\pgfpathmoveto{\pgfqpoint{1.812854in}{2.763381in}}%
\pgfusepath{stroke}%
\end{pgfscope}%
\begin{pgfscope}%
\pgfpathrectangle{\pgfqpoint{0.647939in}{0.492442in}}{\pgfqpoint{3.079299in}{3.079299in}}%
\pgfusepath{clip}%
\pgfsetroundcap%
\pgfsetroundjoin%
\definecolor{currentfill}{rgb}{0.500000,0.500000,0.500000}%
\pgfsetfillcolor{currentfill}%
\pgfsetfillopacity{0.300000}%
\pgfsetlinewidth{0.301125pt}%
\definecolor{currentstroke}{rgb}{0.500000,0.500000,0.500000}%
\pgfsetstrokecolor{currentstroke}%
\pgfsetstrokeopacity{0.300000}%
\pgfsetdash{}{0pt}%
\pgfpathmoveto{\pgfqpoint{0.000000in}{0.000000in}}%
\pgfpathlineto{\pgfqpoint{0.000000in}{0.000000in}}%
\pgfpathclose%
\pgfusepath{stroke,fill}%
\end{pgfscope}%
\begin{pgfscope}%
\pgfpathrectangle{\pgfqpoint{0.647939in}{0.492442in}}{\pgfqpoint{3.079299in}{3.079299in}}%
\pgfusepath{clip}%
\pgfsetroundcap%
\pgfsetroundjoin%
\pgfsetlinewidth{0.301125pt}%
\definecolor{currentstroke}{rgb}{0.500000,0.500000,0.500000}%
\pgfsetstrokecolor{currentstroke}%
\pgfsetstrokeopacity{0.300000}%
\pgfsetdash{}{0pt}%
\pgfpathmoveto{\pgfqpoint{1.012419in}{2.513858in}}%
\pgfusepath{stroke}%
\end{pgfscope}%
\begin{pgfscope}%
\pgfpathrectangle{\pgfqpoint{0.647939in}{0.492442in}}{\pgfqpoint{3.079299in}{3.079299in}}%
\pgfusepath{clip}%
\pgfsetroundcap%
\pgfsetroundjoin%
\definecolor{currentfill}{rgb}{0.500000,0.500000,0.500000}%
\pgfsetfillcolor{currentfill}%
\pgfsetfillopacity{0.300000}%
\pgfsetlinewidth{0.301125pt}%
\definecolor{currentstroke}{rgb}{0.500000,0.500000,0.500000}%
\pgfsetstrokecolor{currentstroke}%
\pgfsetstrokeopacity{0.300000}%
\pgfsetdash{}{0pt}%
\pgfpathmoveto{\pgfqpoint{0.000000in}{0.000000in}}%
\pgfpathlineto{\pgfqpoint{0.000000in}{0.000000in}}%
\pgfpathclose%
\pgfusepath{stroke,fill}%
\end{pgfscope}%
\begin{pgfscope}%
\pgfpathrectangle{\pgfqpoint{0.647939in}{0.492442in}}{\pgfqpoint{3.079299in}{3.079299in}}%
\pgfusepath{clip}%
\pgfsetroundcap%
\pgfsetroundjoin%
\pgfsetlinewidth{0.301125pt}%
\definecolor{currentstroke}{rgb}{0.500000,0.500000,0.500000}%
\pgfsetstrokecolor{currentstroke}%
\pgfsetstrokeopacity{0.300000}%
\pgfsetdash{}{0pt}%
\pgfpathmoveto{\pgfqpoint{0.878301in}{2.416855in}}%
\pgfusepath{stroke}%
\end{pgfscope}%
\begin{pgfscope}%
\pgfpathrectangle{\pgfqpoint{0.647939in}{0.492442in}}{\pgfqpoint{3.079299in}{3.079299in}}%
\pgfusepath{clip}%
\pgfsetroundcap%
\pgfsetroundjoin%
\definecolor{currentfill}{rgb}{0.500000,0.500000,0.500000}%
\pgfsetfillcolor{currentfill}%
\pgfsetfillopacity{0.300000}%
\pgfsetlinewidth{0.301125pt}%
\definecolor{currentstroke}{rgb}{0.500000,0.500000,0.500000}%
\pgfsetstrokecolor{currentstroke}%
\pgfsetstrokeopacity{0.300000}%
\pgfsetdash{}{0pt}%
\pgfpathmoveto{\pgfqpoint{0.000000in}{0.000000in}}%
\pgfpathlineto{\pgfqpoint{0.000000in}{0.000000in}}%
\pgfpathclose%
\pgfusepath{stroke,fill}%
\end{pgfscope}%
\begin{pgfscope}%
\pgfpathrectangle{\pgfqpoint{0.647939in}{0.492442in}}{\pgfqpoint{3.079299in}{3.079299in}}%
\pgfusepath{clip}%
\pgfsetroundcap%
\pgfsetroundjoin%
\pgfsetlinewidth{0.301125pt}%
\definecolor{currentstroke}{rgb}{0.500000,0.500000,0.500000}%
\pgfsetstrokecolor{currentstroke}%
\pgfsetstrokeopacity{0.300000}%
\pgfsetdash{}{0pt}%
\pgfpathmoveto{\pgfqpoint{1.807583in}{2.576340in}}%
\pgfusepath{stroke}%
\end{pgfscope}%
\begin{pgfscope}%
\pgfpathrectangle{\pgfqpoint{0.647939in}{0.492442in}}{\pgfqpoint{3.079299in}{3.079299in}}%
\pgfusepath{clip}%
\pgfsetroundcap%
\pgfsetroundjoin%
\definecolor{currentfill}{rgb}{0.500000,0.500000,0.500000}%
\pgfsetfillcolor{currentfill}%
\pgfsetfillopacity{0.300000}%
\pgfsetlinewidth{0.301125pt}%
\definecolor{currentstroke}{rgb}{0.500000,0.500000,0.500000}%
\pgfsetstrokecolor{currentstroke}%
\pgfsetstrokeopacity{0.300000}%
\pgfsetdash{}{0pt}%
\pgfpathmoveto{\pgfqpoint{0.000000in}{0.000000in}}%
\pgfpathlineto{\pgfqpoint{0.000000in}{0.000000in}}%
\pgfpathclose%
\pgfusepath{stroke,fill}%
\end{pgfscope}%
\begin{pgfscope}%
\pgfpathrectangle{\pgfqpoint{0.647939in}{0.492442in}}{\pgfqpoint{3.079299in}{3.079299in}}%
\pgfusepath{clip}%
\pgfsetroundcap%
\pgfsetroundjoin%
\pgfsetlinewidth{0.301125pt}%
\definecolor{currentstroke}{rgb}{0.500000,0.500000,0.500000}%
\pgfsetstrokecolor{currentstroke}%
\pgfsetstrokeopacity{0.300000}%
\pgfsetdash{}{0pt}%
\pgfpathmoveto{\pgfqpoint{1.670873in}{2.490573in}}%
\pgfusepath{stroke}%
\end{pgfscope}%
\begin{pgfscope}%
\pgfpathrectangle{\pgfqpoint{0.647939in}{0.492442in}}{\pgfqpoint{3.079299in}{3.079299in}}%
\pgfusepath{clip}%
\pgfsetroundcap%
\pgfsetroundjoin%
\definecolor{currentfill}{rgb}{0.500000,0.500000,0.500000}%
\pgfsetfillcolor{currentfill}%
\pgfsetfillopacity{0.300000}%
\pgfsetlinewidth{0.301125pt}%
\definecolor{currentstroke}{rgb}{0.500000,0.500000,0.500000}%
\pgfsetstrokecolor{currentstroke}%
\pgfsetstrokeopacity{0.300000}%
\pgfsetdash{}{0pt}%
\pgfpathmoveto{\pgfqpoint{0.000000in}{0.000000in}}%
\pgfpathlineto{\pgfqpoint{0.000000in}{0.000000in}}%
\pgfpathclose%
\pgfusepath{stroke,fill}%
\end{pgfscope}%
\begin{pgfscope}%
\pgfpathrectangle{\pgfqpoint{0.647939in}{0.492442in}}{\pgfqpoint{3.079299in}{3.079299in}}%
\pgfusepath{clip}%
\pgfsetroundcap%
\pgfsetroundjoin%
\pgfsetlinewidth{0.301125pt}%
\definecolor{currentstroke}{rgb}{0.500000,0.500000,0.500000}%
\pgfsetstrokecolor{currentstroke}%
\pgfsetstrokeopacity{0.300000}%
\pgfsetdash{}{0pt}%
\pgfpathmoveto{\pgfqpoint{1.143784in}{2.273692in}}%
\pgfusepath{stroke}%
\end{pgfscope}%
\begin{pgfscope}%
\pgfpathrectangle{\pgfqpoint{0.647939in}{0.492442in}}{\pgfqpoint{3.079299in}{3.079299in}}%
\pgfusepath{clip}%
\pgfsetroundcap%
\pgfsetroundjoin%
\definecolor{currentfill}{rgb}{0.500000,0.500000,0.500000}%
\pgfsetfillcolor{currentfill}%
\pgfsetfillopacity{0.300000}%
\pgfsetlinewidth{0.301125pt}%
\definecolor{currentstroke}{rgb}{0.500000,0.500000,0.500000}%
\pgfsetstrokecolor{currentstroke}%
\pgfsetstrokeopacity{0.300000}%
\pgfsetdash{}{0pt}%
\pgfpathmoveto{\pgfqpoint{0.000000in}{0.000000in}}%
\pgfpathlineto{\pgfqpoint{0.000000in}{0.000000in}}%
\pgfpathclose%
\pgfusepath{stroke,fill}%
\end{pgfscope}%
\begin{pgfscope}%
\pgfpathrectangle{\pgfqpoint{0.647939in}{0.492442in}}{\pgfqpoint{3.079299in}{3.079299in}}%
\pgfusepath{clip}%
\pgfsetroundcap%
\pgfsetroundjoin%
\pgfsetlinewidth{0.301125pt}%
\definecolor{currentstroke}{rgb}{0.500000,0.500000,0.500000}%
\pgfsetstrokecolor{currentstroke}%
\pgfsetstrokeopacity{0.300000}%
\pgfsetdash{}{0pt}%
\pgfpathmoveto{\pgfqpoint{1.011330in}{2.169640in}}%
\pgfusepath{stroke}%
\end{pgfscope}%
\begin{pgfscope}%
\pgfpathrectangle{\pgfqpoint{0.647939in}{0.492442in}}{\pgfqpoint{3.079299in}{3.079299in}}%
\pgfusepath{clip}%
\pgfsetroundcap%
\pgfsetroundjoin%
\definecolor{currentfill}{rgb}{0.500000,0.500000,0.500000}%
\pgfsetfillcolor{currentfill}%
\pgfsetfillopacity{0.300000}%
\pgfsetlinewidth{0.301125pt}%
\definecolor{currentstroke}{rgb}{0.500000,0.500000,0.500000}%
\pgfsetstrokecolor{currentstroke}%
\pgfsetstrokeopacity{0.300000}%
\pgfsetdash{}{0pt}%
\pgfpathmoveto{\pgfqpoint{0.000000in}{0.000000in}}%
\pgfpathlineto{\pgfqpoint{0.000000in}{0.000000in}}%
\pgfpathclose%
\pgfusepath{stroke,fill}%
\end{pgfscope}%
\begin{pgfscope}%
\pgfpathrectangle{\pgfqpoint{0.647939in}{0.492442in}}{\pgfqpoint{3.079299in}{3.079299in}}%
\pgfusepath{clip}%
\pgfsetroundcap%
\pgfsetroundjoin%
\pgfsetlinewidth{0.301125pt}%
\definecolor{currentstroke}{rgb}{0.500000,0.500000,0.500000}%
\pgfsetstrokecolor{currentstroke}%
\pgfsetstrokeopacity{0.300000}%
\pgfsetdash{}{0pt}%
\pgfpathmoveto{\pgfqpoint{1.730419in}{2.321374in}}%
\pgfusepath{stroke}%
\end{pgfscope}%
\begin{pgfscope}%
\pgfpathrectangle{\pgfqpoint{0.647939in}{0.492442in}}{\pgfqpoint{3.079299in}{3.079299in}}%
\pgfusepath{clip}%
\pgfsetroundcap%
\pgfsetroundjoin%
\definecolor{currentfill}{rgb}{0.500000,0.500000,0.500000}%
\pgfsetfillcolor{currentfill}%
\pgfsetfillopacity{0.300000}%
\pgfsetlinewidth{0.301125pt}%
\definecolor{currentstroke}{rgb}{0.500000,0.500000,0.500000}%
\pgfsetstrokecolor{currentstroke}%
\pgfsetstrokeopacity{0.300000}%
\pgfsetdash{}{0pt}%
\pgfpathmoveto{\pgfqpoint{0.000000in}{0.000000in}}%
\pgfpathlineto{\pgfqpoint{0.000000in}{0.000000in}}%
\pgfpathclose%
\pgfusepath{stroke,fill}%
\end{pgfscope}%
\begin{pgfscope}%
\pgfpathrectangle{\pgfqpoint{0.647939in}{0.492442in}}{\pgfqpoint{3.079299in}{3.079299in}}%
\pgfusepath{clip}%
\pgfsetroundcap%
\pgfsetroundjoin%
\pgfsetlinewidth{0.301125pt}%
\definecolor{currentstroke}{rgb}{0.500000,0.500000,0.500000}%
\pgfsetstrokecolor{currentstroke}%
\pgfsetstrokeopacity{0.300000}%
\pgfsetdash{}{0pt}%
\pgfpathmoveto{\pgfqpoint{1.207406in}{2.091147in}}%
\pgfusepath{stroke}%
\end{pgfscope}%
\begin{pgfscope}%
\pgfpathrectangle{\pgfqpoint{0.647939in}{0.492442in}}{\pgfqpoint{3.079299in}{3.079299in}}%
\pgfusepath{clip}%
\pgfsetroundcap%
\pgfsetroundjoin%
\definecolor{currentfill}{rgb}{0.500000,0.500000,0.500000}%
\pgfsetfillcolor{currentfill}%
\pgfsetfillopacity{0.300000}%
\pgfsetlinewidth{0.301125pt}%
\definecolor{currentstroke}{rgb}{0.500000,0.500000,0.500000}%
\pgfsetstrokecolor{currentstroke}%
\pgfsetstrokeopacity{0.300000}%
\pgfsetdash{}{0pt}%
\pgfpathmoveto{\pgfqpoint{0.000000in}{0.000000in}}%
\pgfpathlineto{\pgfqpoint{0.000000in}{0.000000in}}%
\pgfpathclose%
\pgfusepath{stroke,fill}%
\end{pgfscope}%
\begin{pgfscope}%
\pgfpathrectangle{\pgfqpoint{0.647939in}{0.492442in}}{\pgfqpoint{3.079299in}{3.079299in}}%
\pgfusepath{clip}%
\pgfsetroundcap%
\pgfsetroundjoin%
\pgfsetlinewidth{0.301125pt}%
\definecolor{currentstroke}{rgb}{0.500000,0.500000,0.500000}%
\pgfsetstrokecolor{currentstroke}%
\pgfsetstrokeopacity{0.300000}%
\pgfsetdash{}{0pt}%
\pgfpathmoveto{\pgfqpoint{1.206626in}{2.024121in}}%
\pgfusepath{stroke}%
\end{pgfscope}%
\begin{pgfscope}%
\pgfpathrectangle{\pgfqpoint{0.647939in}{0.492442in}}{\pgfqpoint{3.079299in}{3.079299in}}%
\pgfusepath{clip}%
\pgfsetroundcap%
\pgfsetroundjoin%
\definecolor{currentfill}{rgb}{0.500000,0.500000,0.500000}%
\pgfsetfillcolor{currentfill}%
\pgfsetfillopacity{0.300000}%
\pgfsetlinewidth{0.301125pt}%
\definecolor{currentstroke}{rgb}{0.500000,0.500000,0.500000}%
\pgfsetstrokecolor{currentstroke}%
\pgfsetstrokeopacity{0.300000}%
\pgfsetdash{}{0pt}%
\pgfpathmoveto{\pgfqpoint{0.000000in}{0.000000in}}%
\pgfpathlineto{\pgfqpoint{0.000000in}{0.000000in}}%
\pgfpathclose%
\pgfusepath{stroke,fill}%
\end{pgfscope}%
\begin{pgfscope}%
\pgfpathrectangle{\pgfqpoint{0.647939in}{0.492442in}}{\pgfqpoint{3.079299in}{3.079299in}}%
\pgfusepath{clip}%
\pgfsetroundcap%
\pgfsetroundjoin%
\pgfsetlinewidth{0.301125pt}%
\definecolor{currentstroke}{rgb}{0.500000,0.500000,0.500000}%
\pgfsetstrokecolor{currentstroke}%
\pgfsetstrokeopacity{0.300000}%
\pgfsetdash{}{0pt}%
\pgfpathmoveto{\pgfqpoint{1.075874in}{1.914260in}}%
\pgfusepath{stroke}%
\end{pgfscope}%
\begin{pgfscope}%
\pgfpathrectangle{\pgfqpoint{0.647939in}{0.492442in}}{\pgfqpoint{3.079299in}{3.079299in}}%
\pgfusepath{clip}%
\pgfsetroundcap%
\pgfsetroundjoin%
\definecolor{currentfill}{rgb}{0.500000,0.500000,0.500000}%
\pgfsetfillcolor{currentfill}%
\pgfsetfillopacity{0.300000}%
\pgfsetlinewidth{0.301125pt}%
\definecolor{currentstroke}{rgb}{0.500000,0.500000,0.500000}%
\pgfsetstrokecolor{currentstroke}%
\pgfsetstrokeopacity{0.300000}%
\pgfsetdash{}{0pt}%
\pgfpathmoveto{\pgfqpoint{0.000000in}{0.000000in}}%
\pgfpathlineto{\pgfqpoint{0.000000in}{0.000000in}}%
\pgfpathclose%
\pgfusepath{stroke,fill}%
\end{pgfscope}%
\begin{pgfscope}%
\pgfpathrectangle{\pgfqpoint{0.647939in}{0.492442in}}{\pgfqpoint{3.079299in}{3.079299in}}%
\pgfusepath{clip}%
\pgfsetroundcap%
\pgfsetroundjoin%
\pgfsetlinewidth{0.301125pt}%
\definecolor{currentstroke}{rgb}{0.500000,0.500000,0.500000}%
\pgfsetstrokecolor{currentstroke}%
\pgfsetstrokeopacity{0.300000}%
\pgfsetdash{}{0pt}%
\pgfpathmoveto{\pgfqpoint{0.810375in}{1.778605in}}%
\pgfusepath{stroke}%
\end{pgfscope}%
\begin{pgfscope}%
\pgfpathrectangle{\pgfqpoint{0.647939in}{0.492442in}}{\pgfqpoint{3.079299in}{3.079299in}}%
\pgfusepath{clip}%
\pgfsetroundcap%
\pgfsetroundjoin%
\definecolor{currentfill}{rgb}{0.500000,0.500000,0.500000}%
\pgfsetfillcolor{currentfill}%
\pgfsetfillopacity{0.300000}%
\pgfsetlinewidth{0.301125pt}%
\definecolor{currentstroke}{rgb}{0.500000,0.500000,0.500000}%
\pgfsetstrokecolor{currentstroke}%
\pgfsetstrokeopacity{0.300000}%
\pgfsetdash{}{0pt}%
\pgfpathmoveto{\pgfqpoint{0.000000in}{0.000000in}}%
\pgfpathlineto{\pgfqpoint{0.000000in}{0.000000in}}%
\pgfpathclose%
\pgfusepath{stroke,fill}%
\end{pgfscope}%
\begin{pgfscope}%
\pgfpathrectangle{\pgfqpoint{0.647939in}{0.492442in}}{\pgfqpoint{3.079299in}{3.079299in}}%
\pgfusepath{clip}%
\pgfsetroundcap%
\pgfsetroundjoin%
\pgfsetlinewidth{0.301125pt}%
\definecolor{currentstroke}{rgb}{0.500000,0.500000,0.500000}%
\pgfsetstrokecolor{currentstroke}%
\pgfsetstrokeopacity{0.300000}%
\pgfsetdash{}{0pt}%
\pgfpathmoveto{\pgfqpoint{1.393914in}{1.901521in}}%
\pgfusepath{stroke}%
\end{pgfscope}%
\begin{pgfscope}%
\pgfpathrectangle{\pgfqpoint{0.647939in}{0.492442in}}{\pgfqpoint{3.079299in}{3.079299in}}%
\pgfusepath{clip}%
\pgfsetroundcap%
\pgfsetroundjoin%
\definecolor{currentfill}{rgb}{0.500000,0.500000,0.500000}%
\pgfsetfillcolor{currentfill}%
\pgfsetfillopacity{0.300000}%
\pgfsetlinewidth{0.301125pt}%
\definecolor{currentstroke}{rgb}{0.500000,0.500000,0.500000}%
\pgfsetstrokecolor{currentstroke}%
\pgfsetstrokeopacity{0.300000}%
\pgfsetdash{}{0pt}%
\pgfpathmoveto{\pgfqpoint{0.000000in}{0.000000in}}%
\pgfpathlineto{\pgfqpoint{0.000000in}{0.000000in}}%
\pgfpathclose%
\pgfusepath{stroke,fill}%
\end{pgfscope}%
\begin{pgfscope}%
\pgfpathrectangle{\pgfqpoint{0.647939in}{0.492442in}}{\pgfqpoint{3.079299in}{3.079299in}}%
\pgfusepath{clip}%
\pgfsetroundcap%
\pgfsetroundjoin%
\pgfsetlinewidth{0.301125pt}%
\definecolor{currentstroke}{rgb}{0.500000,0.500000,0.500000}%
\pgfsetstrokecolor{currentstroke}%
\pgfsetstrokeopacity{0.300000}%
\pgfsetdash{}{0pt}%
\pgfpathmoveto{\pgfqpoint{1.389059in}{1.774976in}}%
\pgfusepath{stroke}%
\end{pgfscope}%
\begin{pgfscope}%
\pgfpathrectangle{\pgfqpoint{0.647939in}{0.492442in}}{\pgfqpoint{3.079299in}{3.079299in}}%
\pgfusepath{clip}%
\pgfsetroundcap%
\pgfsetroundjoin%
\definecolor{currentfill}{rgb}{0.500000,0.500000,0.500000}%
\pgfsetfillcolor{currentfill}%
\pgfsetfillopacity{0.300000}%
\pgfsetlinewidth{0.301125pt}%
\definecolor{currentstroke}{rgb}{0.500000,0.500000,0.500000}%
\pgfsetstrokecolor{currentstroke}%
\pgfsetstrokeopacity{0.300000}%
\pgfsetdash{}{0pt}%
\pgfpathmoveto{\pgfqpoint{0.000000in}{0.000000in}}%
\pgfpathlineto{\pgfqpoint{0.000000in}{0.000000in}}%
\pgfpathclose%
\pgfusepath{stroke,fill}%
\end{pgfscope}%
\begin{pgfscope}%
\pgfpathrectangle{\pgfqpoint{0.647939in}{0.492442in}}{\pgfqpoint{3.079299in}{3.079299in}}%
\pgfusepath{clip}%
\pgfsetroundcap%
\pgfsetroundjoin%
\pgfsetlinewidth{0.301125pt}%
\definecolor{currentstroke}{rgb}{0.500000,0.500000,0.500000}%
\pgfsetstrokecolor{currentstroke}%
\pgfsetstrokeopacity{0.300000}%
\pgfsetdash{}{0pt}%
\pgfpathmoveto{\pgfqpoint{1.073151in}{1.575025in}}%
\pgfusepath{stroke}%
\end{pgfscope}%
\begin{pgfscope}%
\pgfpathrectangle{\pgfqpoint{0.647939in}{0.492442in}}{\pgfqpoint{3.079299in}{3.079299in}}%
\pgfusepath{clip}%
\pgfsetroundcap%
\pgfsetroundjoin%
\definecolor{currentfill}{rgb}{0.500000,0.500000,0.500000}%
\pgfsetfillcolor{currentfill}%
\pgfsetfillopacity{0.300000}%
\pgfsetlinewidth{0.301125pt}%
\definecolor{currentstroke}{rgb}{0.500000,0.500000,0.500000}%
\pgfsetstrokecolor{currentstroke}%
\pgfsetstrokeopacity{0.300000}%
\pgfsetdash{}{0pt}%
\pgfpathmoveto{\pgfqpoint{0.000000in}{0.000000in}}%
\pgfpathlineto{\pgfqpoint{0.000000in}{0.000000in}}%
\pgfpathclose%
\pgfusepath{stroke,fill}%
\end{pgfscope}%
\begin{pgfscope}%
\pgfpathrectangle{\pgfqpoint{0.647939in}{0.492442in}}{\pgfqpoint{3.079299in}{3.079299in}}%
\pgfusepath{clip}%
\pgfsetroundcap%
\pgfsetroundjoin%
\pgfsetlinewidth{0.301125pt}%
\definecolor{currentstroke}{rgb}{0.500000,0.500000,0.500000}%
\pgfsetstrokecolor{currentstroke}%
\pgfsetstrokeopacity{0.300000}%
\pgfsetdash{}{0pt}%
\pgfpathmoveto{\pgfqpoint{1.322590in}{1.618131in}}%
\pgfusepath{stroke}%
\end{pgfscope}%
\begin{pgfscope}%
\pgfpathrectangle{\pgfqpoint{0.647939in}{0.492442in}}{\pgfqpoint{3.079299in}{3.079299in}}%
\pgfusepath{clip}%
\pgfsetroundcap%
\pgfsetroundjoin%
\definecolor{currentfill}{rgb}{0.500000,0.500000,0.500000}%
\pgfsetfillcolor{currentfill}%
\pgfsetfillopacity{0.300000}%
\pgfsetlinewidth{0.301125pt}%
\definecolor{currentstroke}{rgb}{0.500000,0.500000,0.500000}%
\pgfsetstrokecolor{currentstroke}%
\pgfsetstrokeopacity{0.300000}%
\pgfsetdash{}{0pt}%
\pgfpathmoveto{\pgfqpoint{0.000000in}{0.000000in}}%
\pgfpathlineto{\pgfqpoint{0.000000in}{0.000000in}}%
\pgfpathclose%
\pgfusepath{stroke,fill}%
\end{pgfscope}%
\begin{pgfscope}%
\pgfpathrectangle{\pgfqpoint{0.647939in}{0.492442in}}{\pgfqpoint{3.079299in}{3.079299in}}%
\pgfusepath{clip}%
\pgfsetroundcap%
\pgfsetroundjoin%
\pgfsetlinewidth{0.301125pt}%
\definecolor{currentstroke}{rgb}{0.500000,0.500000,0.500000}%
\pgfsetstrokecolor{currentstroke}%
\pgfsetstrokeopacity{0.300000}%
\pgfsetdash{}{0pt}%
\pgfpathmoveto{\pgfqpoint{0.876351in}{1.377679in}}%
\pgfusepath{stroke}%
\end{pgfscope}%
\begin{pgfscope}%
\pgfpathrectangle{\pgfqpoint{0.647939in}{0.492442in}}{\pgfqpoint{3.079299in}{3.079299in}}%
\pgfusepath{clip}%
\pgfsetroundcap%
\pgfsetroundjoin%
\definecolor{currentfill}{rgb}{0.500000,0.500000,0.500000}%
\pgfsetfillcolor{currentfill}%
\pgfsetfillopacity{0.300000}%
\pgfsetlinewidth{0.301125pt}%
\definecolor{currentstroke}{rgb}{0.500000,0.500000,0.500000}%
\pgfsetstrokecolor{currentstroke}%
\pgfsetstrokeopacity{0.300000}%
\pgfsetdash{}{0pt}%
\pgfpathmoveto{\pgfqpoint{0.000000in}{0.000000in}}%
\pgfpathlineto{\pgfqpoint{0.000000in}{0.000000in}}%
\pgfpathclose%
\pgfusepath{stroke,fill}%
\end{pgfscope}%
\begin{pgfscope}%
\pgfpathrectangle{\pgfqpoint{0.647939in}{0.492442in}}{\pgfqpoint{3.079299in}{3.079299in}}%
\pgfusepath{clip}%
\pgfsetroundcap%
\pgfsetroundjoin%
\pgfsetlinewidth{0.301125pt}%
\definecolor{currentstroke}{rgb}{0.500000,0.500000,0.500000}%
\pgfsetstrokecolor{currentstroke}%
\pgfsetstrokeopacity{0.300000}%
\pgfsetdash{}{0pt}%
\pgfpathmoveto{\pgfqpoint{1.257182in}{1.458730in}}%
\pgfusepath{stroke}%
\end{pgfscope}%
\begin{pgfscope}%
\pgfpathrectangle{\pgfqpoint{0.647939in}{0.492442in}}{\pgfqpoint{3.079299in}{3.079299in}}%
\pgfusepath{clip}%
\pgfsetroundcap%
\pgfsetroundjoin%
\definecolor{currentfill}{rgb}{0.500000,0.500000,0.500000}%
\pgfsetfillcolor{currentfill}%
\pgfsetfillopacity{0.300000}%
\pgfsetlinewidth{0.301125pt}%
\definecolor{currentstroke}{rgb}{0.500000,0.500000,0.500000}%
\pgfsetstrokecolor{currentstroke}%
\pgfsetstrokeopacity{0.300000}%
\pgfsetdash{}{0pt}%
\pgfpathmoveto{\pgfqpoint{0.000000in}{0.000000in}}%
\pgfpathlineto{\pgfqpoint{0.000000in}{0.000000in}}%
\pgfpathclose%
\pgfusepath{stroke,fill}%
\end{pgfscope}%
\begin{pgfscope}%
\pgfpathrectangle{\pgfqpoint{0.647939in}{0.492442in}}{\pgfqpoint{3.079299in}{3.079299in}}%
\pgfusepath{clip}%
\pgfsetroundcap%
\pgfsetroundjoin%
\pgfsetlinewidth{0.301125pt}%
\definecolor{currentstroke}{rgb}{0.500000,0.500000,0.500000}%
\pgfsetstrokecolor{currentstroke}%
\pgfsetstrokeopacity{0.300000}%
\pgfsetdash{}{0pt}%
\pgfpathmoveto{\pgfqpoint{1.132886in}{1.332715in}}%
\pgfusepath{stroke}%
\end{pgfscope}%
\begin{pgfscope}%
\pgfpathrectangle{\pgfqpoint{0.647939in}{0.492442in}}{\pgfqpoint{3.079299in}{3.079299in}}%
\pgfusepath{clip}%
\pgfsetroundcap%
\pgfsetroundjoin%
\definecolor{currentfill}{rgb}{0.500000,0.500000,0.500000}%
\pgfsetfillcolor{currentfill}%
\pgfsetfillopacity{0.300000}%
\pgfsetlinewidth{0.301125pt}%
\definecolor{currentstroke}{rgb}{0.500000,0.500000,0.500000}%
\pgfsetstrokecolor{currentstroke}%
\pgfsetstrokeopacity{0.300000}%
\pgfsetdash{}{0pt}%
\pgfpathmoveto{\pgfqpoint{0.000000in}{0.000000in}}%
\pgfpathlineto{\pgfqpoint{0.000000in}{0.000000in}}%
\pgfpathclose%
\pgfusepath{stroke,fill}%
\end{pgfscope}%
\begin{pgfscope}%
\pgfpathrectangle{\pgfqpoint{0.647939in}{0.492442in}}{\pgfqpoint{3.079299in}{3.079299in}}%
\pgfusepath{clip}%
\pgfsetroundcap%
\pgfsetroundjoin%
\pgfsetlinewidth{0.301125pt}%
\definecolor{currentstroke}{rgb}{0.500000,0.500000,0.500000}%
\pgfsetstrokecolor{currentstroke}%
\pgfsetstrokeopacity{0.300000}%
\pgfsetdash{}{0pt}%
\pgfpathmoveto{\pgfqpoint{0.809484in}{1.153550in}}%
\pgfusepath{stroke}%
\end{pgfscope}%
\begin{pgfscope}%
\pgfpathrectangle{\pgfqpoint{0.647939in}{0.492442in}}{\pgfqpoint{3.079299in}{3.079299in}}%
\pgfusepath{clip}%
\pgfsetroundcap%
\pgfsetroundjoin%
\definecolor{currentfill}{rgb}{0.500000,0.500000,0.500000}%
\pgfsetfillcolor{currentfill}%
\pgfsetfillopacity{0.300000}%
\pgfsetlinewidth{0.301125pt}%
\definecolor{currentstroke}{rgb}{0.500000,0.500000,0.500000}%
\pgfsetstrokecolor{currentstroke}%
\pgfsetstrokeopacity{0.300000}%
\pgfsetdash{}{0pt}%
\pgfpathmoveto{\pgfqpoint{0.000000in}{0.000000in}}%
\pgfpathlineto{\pgfqpoint{0.000000in}{0.000000in}}%
\pgfpathclose%
\pgfusepath{stroke,fill}%
\end{pgfscope}%
\begin{pgfscope}%
\pgfpathrectangle{\pgfqpoint{0.647939in}{0.492442in}}{\pgfqpoint{3.079299in}{3.079299in}}%
\pgfusepath{clip}%
\pgfsetroundcap%
\pgfsetroundjoin%
\pgfsetlinewidth{0.301125pt}%
\definecolor{currentstroke}{rgb}{0.500000,0.500000,0.500000}%
\pgfsetstrokecolor{currentstroke}%
\pgfsetstrokeopacity{0.300000}%
\pgfsetdash{}{0pt}%
\pgfpathmoveto{\pgfqpoint{1.190579in}{1.232568in}}%
\pgfusepath{stroke}%
\end{pgfscope}%
\begin{pgfscope}%
\pgfpathrectangle{\pgfqpoint{0.647939in}{0.492442in}}{\pgfqpoint{3.079299in}{3.079299in}}%
\pgfusepath{clip}%
\pgfsetroundcap%
\pgfsetroundjoin%
\definecolor{currentfill}{rgb}{0.500000,0.500000,0.500000}%
\pgfsetfillcolor{currentfill}%
\pgfsetfillopacity{0.300000}%
\pgfsetlinewidth{0.301125pt}%
\definecolor{currentstroke}{rgb}{0.500000,0.500000,0.500000}%
\pgfsetstrokecolor{currentstroke}%
\pgfsetstrokeopacity{0.300000}%
\pgfsetdash{}{0pt}%
\pgfpathmoveto{\pgfqpoint{0.000000in}{0.000000in}}%
\pgfpathlineto{\pgfqpoint{0.000000in}{0.000000in}}%
\pgfpathclose%
\pgfusepath{stroke,fill}%
\end{pgfscope}%
\begin{pgfscope}%
\pgfpathrectangle{\pgfqpoint{0.647939in}{0.492442in}}{\pgfqpoint{3.079299in}{3.079299in}}%
\pgfusepath{clip}%
\pgfsetroundcap%
\pgfsetroundjoin%
\pgfsetlinewidth{0.301125pt}%
\definecolor{currentstroke}{rgb}{0.500000,0.500000,0.500000}%
\pgfsetstrokecolor{currentstroke}%
\pgfsetstrokeopacity{0.300000}%
\pgfsetdash{}{0pt}%
\pgfpathmoveto{\pgfqpoint{1.067214in}{1.104640in}}%
\pgfusepath{stroke}%
\end{pgfscope}%
\begin{pgfscope}%
\pgfpathrectangle{\pgfqpoint{0.647939in}{0.492442in}}{\pgfqpoint{3.079299in}{3.079299in}}%
\pgfusepath{clip}%
\pgfsetroundcap%
\pgfsetroundjoin%
\definecolor{currentfill}{rgb}{0.500000,0.500000,0.500000}%
\pgfsetfillcolor{currentfill}%
\pgfsetfillopacity{0.300000}%
\pgfsetlinewidth{0.301125pt}%
\definecolor{currentstroke}{rgb}{0.500000,0.500000,0.500000}%
\pgfsetstrokecolor{currentstroke}%
\pgfsetstrokeopacity{0.300000}%
\pgfsetdash{}{0pt}%
\pgfpathmoveto{\pgfqpoint{0.000000in}{0.000000in}}%
\pgfpathlineto{\pgfqpoint{0.000000in}{0.000000in}}%
\pgfpathclose%
\pgfusepath{stroke,fill}%
\end{pgfscope}%
\begin{pgfscope}%
\pgfpathrectangle{\pgfqpoint{0.647939in}{0.492442in}}{\pgfqpoint{3.079299in}{3.079299in}}%
\pgfusepath{clip}%
\pgfsetroundcap%
\pgfsetroundjoin%
\pgfsetlinewidth{0.301125pt}%
\definecolor{currentstroke}{rgb}{0.500000,0.500000,0.500000}%
\pgfsetstrokecolor{currentstroke}%
\pgfsetstrokeopacity{0.300000}%
\pgfsetdash{}{0pt}%
\pgfpathmoveto{\pgfqpoint{0.874938in}{0.963983in}}%
\pgfusepath{stroke}%
\end{pgfscope}%
\begin{pgfscope}%
\pgfpathrectangle{\pgfqpoint{0.647939in}{0.492442in}}{\pgfqpoint{3.079299in}{3.079299in}}%
\pgfusepath{clip}%
\pgfsetroundcap%
\pgfsetroundjoin%
\definecolor{currentfill}{rgb}{0.500000,0.500000,0.500000}%
\pgfsetfillcolor{currentfill}%
\pgfsetfillopacity{0.300000}%
\pgfsetlinewidth{0.301125pt}%
\definecolor{currentstroke}{rgb}{0.500000,0.500000,0.500000}%
\pgfsetstrokecolor{currentstroke}%
\pgfsetstrokeopacity{0.300000}%
\pgfsetdash{}{0pt}%
\pgfpathmoveto{\pgfqpoint{0.000000in}{0.000000in}}%
\pgfpathlineto{\pgfqpoint{0.000000in}{0.000000in}}%
\pgfpathclose%
\pgfusepath{stroke,fill}%
\end{pgfscope}%
\begin{pgfscope}%
\pgfpathrectangle{\pgfqpoint{0.647939in}{0.492442in}}{\pgfqpoint{3.079299in}{3.079299in}}%
\pgfusepath{clip}%
\pgfsetroundcap%
\pgfsetroundjoin%
\pgfsetlinewidth{0.301125pt}%
\definecolor{currentstroke}{rgb}{0.500000,0.500000,0.500000}%
\pgfsetstrokecolor{currentstroke}%
\pgfsetstrokeopacity{0.300000}%
\pgfsetdash{}{0pt}%
\pgfpathmoveto{\pgfqpoint{1.125042in}{1.003862in}}%
\pgfusepath{stroke}%
\end{pgfscope}%
\begin{pgfscope}%
\pgfpathrectangle{\pgfqpoint{0.647939in}{0.492442in}}{\pgfqpoint{3.079299in}{3.079299in}}%
\pgfusepath{clip}%
\pgfsetroundcap%
\pgfsetroundjoin%
\definecolor{currentfill}{rgb}{0.500000,0.500000,0.500000}%
\pgfsetfillcolor{currentfill}%
\pgfsetfillopacity{0.300000}%
\pgfsetlinewidth{0.301125pt}%
\definecolor{currentstroke}{rgb}{0.500000,0.500000,0.500000}%
\pgfsetstrokecolor{currentstroke}%
\pgfsetstrokeopacity{0.300000}%
\pgfsetdash{}{0pt}%
\pgfpathmoveto{\pgfqpoint{0.000000in}{0.000000in}}%
\pgfpathlineto{\pgfqpoint{0.000000in}{0.000000in}}%
\pgfpathclose%
\pgfusepath{stroke,fill}%
\end{pgfscope}%
\begin{pgfscope}%
\pgfpathrectangle{\pgfqpoint{0.647939in}{0.492442in}}{\pgfqpoint{3.079299in}{3.079299in}}%
\pgfusepath{clip}%
\pgfsetroundcap%
\pgfsetroundjoin%
\pgfsetlinewidth{0.301125pt}%
\definecolor{currentstroke}{rgb}{0.500000,0.500000,0.500000}%
\pgfsetstrokecolor{currentstroke}%
\pgfsetstrokeopacity{0.300000}%
\pgfsetdash{}{0pt}%
\pgfpathmoveto{\pgfqpoint{1.001964in}{0.875360in}}%
\pgfusepath{stroke}%
\end{pgfscope}%
\begin{pgfscope}%
\pgfpathrectangle{\pgfqpoint{0.647939in}{0.492442in}}{\pgfqpoint{3.079299in}{3.079299in}}%
\pgfusepath{clip}%
\pgfsetroundcap%
\pgfsetroundjoin%
\definecolor{currentfill}{rgb}{0.500000,0.500000,0.500000}%
\pgfsetfillcolor{currentfill}%
\pgfsetfillopacity{0.300000}%
\pgfsetlinewidth{0.301125pt}%
\definecolor{currentstroke}{rgb}{0.500000,0.500000,0.500000}%
\pgfsetstrokecolor{currentstroke}%
\pgfsetstrokeopacity{0.300000}%
\pgfsetdash{}{0pt}%
\pgfpathmoveto{\pgfqpoint{0.000000in}{0.000000in}}%
\pgfpathlineto{\pgfqpoint{0.000000in}{0.000000in}}%
\pgfpathclose%
\pgfusepath{stroke,fill}%
\end{pgfscope}%
\begin{pgfscope}%
\pgfpathrectangle{\pgfqpoint{0.647939in}{0.492442in}}{\pgfqpoint{3.079299in}{3.079299in}}%
\pgfusepath{clip}%
\pgfsetroundcap%
\pgfsetroundjoin%
\pgfsetlinewidth{0.301125pt}%
\definecolor{currentstroke}{rgb}{0.500000,0.500000,0.500000}%
\pgfsetstrokecolor{currentstroke}%
\pgfsetstrokeopacity{0.300000}%
\pgfsetdash{}{0pt}%
\pgfpathmoveto{\pgfqpoint{0.938235in}{0.781143in}}%
\pgfusepath{stroke}%
\end{pgfscope}%
\begin{pgfscope}%
\pgfpathrectangle{\pgfqpoint{0.647939in}{0.492442in}}{\pgfqpoint{3.079299in}{3.079299in}}%
\pgfusepath{clip}%
\pgfsetroundcap%
\pgfsetroundjoin%
\definecolor{currentfill}{rgb}{0.500000,0.500000,0.500000}%
\pgfsetfillcolor{currentfill}%
\pgfsetfillopacity{0.300000}%
\pgfsetlinewidth{0.301125pt}%
\definecolor{currentstroke}{rgb}{0.500000,0.500000,0.500000}%
\pgfsetstrokecolor{currentstroke}%
\pgfsetstrokeopacity{0.300000}%
\pgfsetdash{}{0pt}%
\pgfpathmoveto{\pgfqpoint{0.000000in}{0.000000in}}%
\pgfpathlineto{\pgfqpoint{0.000000in}{0.000000in}}%
\pgfpathclose%
\pgfusepath{stroke,fill}%
\end{pgfscope}%
\begin{pgfscope}%
\pgfpathrectangle{\pgfqpoint{0.647939in}{0.492442in}}{\pgfqpoint{3.079299in}{3.079299in}}%
\pgfusepath{clip}%
\pgfsetroundcap%
\pgfsetroundjoin%
\pgfsetlinewidth{0.301125pt}%
\definecolor{currentstroke}{rgb}{0.500000,0.500000,0.500000}%
\pgfsetstrokecolor{currentstroke}%
\pgfsetstrokeopacity{0.300000}%
\pgfsetdash{}{0pt}%
\pgfpathmoveto{\pgfqpoint{0.937599in}{0.713234in}}%
\pgfusepath{stroke}%
\end{pgfscope}%
\begin{pgfscope}%
\pgfpathrectangle{\pgfqpoint{0.647939in}{0.492442in}}{\pgfqpoint{3.079299in}{3.079299in}}%
\pgfusepath{clip}%
\pgfsetroundcap%
\pgfsetroundjoin%
\definecolor{currentfill}{rgb}{0.500000,0.500000,0.500000}%
\pgfsetfillcolor{currentfill}%
\pgfsetfillopacity{0.300000}%
\pgfsetlinewidth{0.301125pt}%
\definecolor{currentstroke}{rgb}{0.500000,0.500000,0.500000}%
\pgfsetstrokecolor{currentstroke}%
\pgfsetstrokeopacity{0.300000}%
\pgfsetdash{}{0pt}%
\pgfpathmoveto{\pgfqpoint{0.000000in}{0.000000in}}%
\pgfpathlineto{\pgfqpoint{0.000000in}{0.000000in}}%
\pgfpathclose%
\pgfusepath{stroke,fill}%
\end{pgfscope}%
\begin{pgfscope}%
\pgfpathrectangle{\pgfqpoint{0.647939in}{0.492442in}}{\pgfqpoint{3.079299in}{3.079299in}}%
\pgfusepath{clip}%
\pgfsetroundcap%
\pgfsetroundjoin%
\pgfsetlinewidth{0.301125pt}%
\definecolor{currentstroke}{rgb}{0.500000,0.500000,0.500000}%
\pgfsetstrokecolor{currentstroke}%
\pgfsetstrokeopacity{0.300000}%
\pgfsetdash{}{0pt}%
\pgfpathmoveto{\pgfqpoint{0.873218in}{0.620618in}}%
\pgfusepath{stroke}%
\end{pgfscope}%
\begin{pgfscope}%
\pgfpathrectangle{\pgfqpoint{0.647939in}{0.492442in}}{\pgfqpoint{3.079299in}{3.079299in}}%
\pgfusepath{clip}%
\pgfsetroundcap%
\pgfsetroundjoin%
\definecolor{currentfill}{rgb}{0.500000,0.500000,0.500000}%
\pgfsetfillcolor{currentfill}%
\pgfsetfillopacity{0.300000}%
\pgfsetlinewidth{0.301125pt}%
\definecolor{currentstroke}{rgb}{0.500000,0.500000,0.500000}%
\pgfsetstrokecolor{currentstroke}%
\pgfsetstrokeopacity{0.300000}%
\pgfsetdash{}{0pt}%
\pgfpathmoveto{\pgfqpoint{0.000000in}{0.000000in}}%
\pgfpathlineto{\pgfqpoint{0.000000in}{0.000000in}}%
\pgfpathclose%
\pgfusepath{stroke,fill}%
\end{pgfscope}%
\begin{pgfscope}%
\pgfpathrectangle{\pgfqpoint{0.647939in}{0.492442in}}{\pgfqpoint{3.079299in}{3.079299in}}%
\pgfusepath{clip}%
\pgfsetroundcap%
\pgfsetroundjoin%
\pgfsetlinewidth{0.301125pt}%
\definecolor{currentstroke}{rgb}{0.500000,0.500000,0.500000}%
\pgfsetstrokecolor{currentstroke}%
\pgfsetstrokeopacity{0.300000}%
\pgfsetdash{}{0pt}%
\pgfpathmoveto{\pgfqpoint{2.145735in}{0.600419in}}%
\pgfusepath{stroke}%
\end{pgfscope}%
\begin{pgfscope}%
\pgfpathrectangle{\pgfqpoint{0.647939in}{0.492442in}}{\pgfqpoint{3.079299in}{3.079299in}}%
\pgfusepath{clip}%
\pgfsetroundcap%
\pgfsetroundjoin%
\definecolor{currentfill}{rgb}{0.500000,0.500000,0.500000}%
\pgfsetfillcolor{currentfill}%
\pgfsetfillopacity{0.300000}%
\pgfsetlinewidth{0.301125pt}%
\definecolor{currentstroke}{rgb}{0.500000,0.500000,0.500000}%
\pgfsetstrokecolor{currentstroke}%
\pgfsetstrokeopacity{0.300000}%
\pgfsetdash{}{0pt}%
\pgfpathmoveto{\pgfqpoint{0.000000in}{0.000000in}}%
\pgfpathlineto{\pgfqpoint{0.000000in}{0.000000in}}%
\pgfpathclose%
\pgfusepath{stroke,fill}%
\end{pgfscope}%
\begin{pgfscope}%
\pgfpathrectangle{\pgfqpoint{0.647939in}{0.492442in}}{\pgfqpoint{3.079299in}{3.079299in}}%
\pgfusepath{clip}%
\pgfsetroundcap%
\pgfsetroundjoin%
\pgfsetlinewidth{0.301125pt}%
\definecolor{currentstroke}{rgb}{0.500000,0.500000,0.500000}%
\pgfsetstrokecolor{currentstroke}%
\pgfsetstrokeopacity{0.300000}%
\pgfsetdash{}{0pt}%
\pgfpathmoveto{\pgfqpoint{3.044501in}{2.069539in}}%
\pgfusepath{stroke}%
\end{pgfscope}%
\begin{pgfscope}%
\pgfpathrectangle{\pgfqpoint{0.647939in}{0.492442in}}{\pgfqpoint{3.079299in}{3.079299in}}%
\pgfusepath{clip}%
\pgfsetroundcap%
\pgfsetroundjoin%
\definecolor{currentfill}{rgb}{0.500000,0.500000,0.500000}%
\pgfsetfillcolor{currentfill}%
\pgfsetfillopacity{0.300000}%
\pgfsetlinewidth{0.301125pt}%
\definecolor{currentstroke}{rgb}{0.500000,0.500000,0.500000}%
\pgfsetstrokecolor{currentstroke}%
\pgfsetstrokeopacity{0.300000}%
\pgfsetdash{}{0pt}%
\pgfpathmoveto{\pgfqpoint{0.000000in}{0.000000in}}%
\pgfpathlineto{\pgfqpoint{0.000000in}{0.000000in}}%
\pgfpathclose%
\pgfusepath{stroke,fill}%
\end{pgfscope}%
\begin{pgfscope}%
\pgfpathrectangle{\pgfqpoint{0.647939in}{0.492442in}}{\pgfqpoint{3.079299in}{3.079299in}}%
\pgfusepath{clip}%
\pgfsetroundcap%
\pgfsetroundjoin%
\pgfsetlinewidth{0.301125pt}%
\definecolor{currentstroke}{rgb}{0.500000,0.500000,0.500000}%
\pgfsetstrokecolor{currentstroke}%
\pgfsetstrokeopacity{0.300000}%
\pgfsetdash{}{0pt}%
\pgfpathmoveto{\pgfqpoint{3.494034in}{3.071099in}}%
\pgfusepath{stroke}%
\end{pgfscope}%
\begin{pgfscope}%
\pgfpathrectangle{\pgfqpoint{0.647939in}{0.492442in}}{\pgfqpoint{3.079299in}{3.079299in}}%
\pgfusepath{clip}%
\pgfsetroundcap%
\pgfsetroundjoin%
\definecolor{currentfill}{rgb}{0.500000,0.500000,0.500000}%
\pgfsetfillcolor{currentfill}%
\pgfsetfillopacity{0.300000}%
\pgfsetlinewidth{0.301125pt}%
\definecolor{currentstroke}{rgb}{0.500000,0.500000,0.500000}%
\pgfsetstrokecolor{currentstroke}%
\pgfsetstrokeopacity{0.300000}%
\pgfsetdash{}{0pt}%
\pgfpathmoveto{\pgfqpoint{0.000000in}{0.000000in}}%
\pgfpathlineto{\pgfqpoint{0.000000in}{0.000000in}}%
\pgfpathclose%
\pgfusepath{stroke,fill}%
\end{pgfscope}%
\begin{pgfscope}%
\pgfpathrectangle{\pgfqpoint{0.647939in}{0.492442in}}{\pgfqpoint{3.079299in}{3.079299in}}%
\pgfusepath{clip}%
\pgfsetroundcap%
\pgfsetroundjoin%
\pgfsetlinewidth{0.301125pt}%
\definecolor{currentstroke}{rgb}{0.500000,0.500000,0.500000}%
\pgfsetstrokecolor{currentstroke}%
\pgfsetstrokeopacity{0.300000}%
\pgfsetdash{}{0pt}%
\pgfpathmoveto{\pgfqpoint{3.607774in}{3.450483in}}%
\pgfusepath{stroke}%
\end{pgfscope}%
\begin{pgfscope}%
\pgfpathrectangle{\pgfqpoint{0.647939in}{0.492442in}}{\pgfqpoint{3.079299in}{3.079299in}}%
\pgfusepath{clip}%
\pgfsetroundcap%
\pgfsetroundjoin%
\definecolor{currentfill}{rgb}{0.500000,0.500000,0.500000}%
\pgfsetfillcolor{currentfill}%
\pgfsetfillopacity{0.300000}%
\pgfsetlinewidth{0.301125pt}%
\definecolor{currentstroke}{rgb}{0.500000,0.500000,0.500000}%
\pgfsetstrokecolor{currentstroke}%
\pgfsetstrokeopacity{0.300000}%
\pgfsetdash{}{0pt}%
\pgfpathmoveto{\pgfqpoint{0.000000in}{0.000000in}}%
\pgfpathlineto{\pgfqpoint{0.000000in}{0.000000in}}%
\pgfpathclose%
\pgfusepath{stroke,fill}%
\end{pgfscope}%
\begin{pgfscope}%
\pgfpathrectangle{\pgfqpoint{0.647939in}{0.492442in}}{\pgfqpoint{3.079299in}{3.079299in}}%
\pgfusepath{clip}%
\pgfsetroundcap%
\pgfsetroundjoin%
\pgfsetlinewidth{0.301125pt}%
\definecolor{currentstroke}{rgb}{0.500000,0.500000,0.500000}%
\pgfsetstrokecolor{currentstroke}%
\pgfsetstrokeopacity{0.300000}%
\pgfsetdash{}{0pt}%
\pgfpathmoveto{\pgfqpoint{3.137899in}{2.250028in}}%
\pgfusepath{stroke}%
\end{pgfscope}%
\begin{pgfscope}%
\pgfpathrectangle{\pgfqpoint{0.647939in}{0.492442in}}{\pgfqpoint{3.079299in}{3.079299in}}%
\pgfusepath{clip}%
\pgfsetroundcap%
\pgfsetroundjoin%
\definecolor{currentfill}{rgb}{0.500000,0.500000,0.500000}%
\pgfsetfillcolor{currentfill}%
\pgfsetfillopacity{0.300000}%
\pgfsetlinewidth{0.301125pt}%
\definecolor{currentstroke}{rgb}{0.500000,0.500000,0.500000}%
\pgfsetstrokecolor{currentstroke}%
\pgfsetstrokeopacity{0.300000}%
\pgfsetdash{}{0pt}%
\pgfpathmoveto{\pgfqpoint{0.000000in}{0.000000in}}%
\pgfpathlineto{\pgfqpoint{0.000000in}{0.000000in}}%
\pgfpathclose%
\pgfusepath{stroke,fill}%
\end{pgfscope}%
\begin{pgfscope}%
\pgfpathrectangle{\pgfqpoint{0.647939in}{0.492442in}}{\pgfqpoint{3.079299in}{3.079299in}}%
\pgfusepath{clip}%
\pgfsetroundcap%
\pgfsetroundjoin%
\pgfsetlinewidth{0.301125pt}%
\definecolor{currentstroke}{rgb}{0.500000,0.500000,0.500000}%
\pgfsetstrokecolor{currentstroke}%
\pgfsetstrokeopacity{0.300000}%
\pgfsetdash{}{0pt}%
\pgfpathmoveto{\pgfqpoint{2.228906in}{3.331892in}}%
\pgfusepath{stroke}%
\end{pgfscope}%
\begin{pgfscope}%
\pgfpathrectangle{\pgfqpoint{0.647939in}{0.492442in}}{\pgfqpoint{3.079299in}{3.079299in}}%
\pgfusepath{clip}%
\pgfsetroundcap%
\pgfsetroundjoin%
\definecolor{currentfill}{rgb}{0.500000,0.500000,0.500000}%
\pgfsetfillcolor{currentfill}%
\pgfsetfillopacity{0.300000}%
\pgfsetlinewidth{0.301125pt}%
\definecolor{currentstroke}{rgb}{0.500000,0.500000,0.500000}%
\pgfsetstrokecolor{currentstroke}%
\pgfsetstrokeopacity{0.300000}%
\pgfsetdash{}{0pt}%
\pgfpathmoveto{\pgfqpoint{0.000000in}{0.000000in}}%
\pgfpathlineto{\pgfqpoint{0.000000in}{0.000000in}}%
\pgfpathclose%
\pgfusepath{stroke,fill}%
\end{pgfscope}%
\begin{pgfscope}%
\pgfpathrectangle{\pgfqpoint{0.647939in}{0.492442in}}{\pgfqpoint{3.079299in}{3.079299in}}%
\pgfusepath{clip}%
\pgfsetroundcap%
\pgfsetroundjoin%
\pgfsetlinewidth{0.301125pt}%
\definecolor{currentstroke}{rgb}{0.500000,0.500000,0.500000}%
\pgfsetstrokecolor{currentstroke}%
\pgfsetstrokeopacity{0.300000}%
\pgfsetdash{}{0pt}%
\pgfpathmoveto{\pgfqpoint{2.924807in}{1.917640in}}%
\pgfusepath{stroke}%
\end{pgfscope}%
\begin{pgfscope}%
\pgfpathrectangle{\pgfqpoint{0.647939in}{0.492442in}}{\pgfqpoint{3.079299in}{3.079299in}}%
\pgfusepath{clip}%
\pgfsetroundcap%
\pgfsetroundjoin%
\definecolor{currentfill}{rgb}{0.500000,0.500000,0.500000}%
\pgfsetfillcolor{currentfill}%
\pgfsetfillopacity{0.300000}%
\pgfsetlinewidth{0.301125pt}%
\definecolor{currentstroke}{rgb}{0.500000,0.500000,0.500000}%
\pgfsetstrokecolor{currentstroke}%
\pgfsetstrokeopacity{0.300000}%
\pgfsetdash{}{0pt}%
\pgfpathmoveto{\pgfqpoint{0.000000in}{0.000000in}}%
\pgfpathlineto{\pgfqpoint{0.000000in}{0.000000in}}%
\pgfpathclose%
\pgfusepath{stroke,fill}%
\end{pgfscope}%
\begin{pgfscope}%
\pgfpathrectangle{\pgfqpoint{0.647939in}{0.492442in}}{\pgfqpoint{3.079299in}{3.079299in}}%
\pgfusepath{clip}%
\pgfsetroundcap%
\pgfsetroundjoin%
\pgfsetlinewidth{0.301125pt}%
\definecolor{currentstroke}{rgb}{0.500000,0.500000,0.500000}%
\pgfsetstrokecolor{currentstroke}%
\pgfsetstrokeopacity{0.300000}%
\pgfsetdash{}{0pt}%
\pgfpathmoveto{\pgfqpoint{2.080291in}{0.820880in}}%
\pgfusepath{stroke}%
\end{pgfscope}%
\begin{pgfscope}%
\pgfpathrectangle{\pgfqpoint{0.647939in}{0.492442in}}{\pgfqpoint{3.079299in}{3.079299in}}%
\pgfusepath{clip}%
\pgfsetroundcap%
\pgfsetroundjoin%
\definecolor{currentfill}{rgb}{0.500000,0.500000,0.500000}%
\pgfsetfillcolor{currentfill}%
\pgfsetfillopacity{0.300000}%
\pgfsetlinewidth{0.301125pt}%
\definecolor{currentstroke}{rgb}{0.500000,0.500000,0.500000}%
\pgfsetstrokecolor{currentstroke}%
\pgfsetstrokeopacity{0.300000}%
\pgfsetdash{}{0pt}%
\pgfpathmoveto{\pgfqpoint{0.000000in}{0.000000in}}%
\pgfpathlineto{\pgfqpoint{0.000000in}{0.000000in}}%
\pgfpathclose%
\pgfusepath{stroke,fill}%
\end{pgfscope}%
\begin{pgfscope}%
\pgfpathrectangle{\pgfqpoint{0.647939in}{0.492442in}}{\pgfqpoint{3.079299in}{3.079299in}}%
\pgfusepath{clip}%
\pgfsetroundcap%
\pgfsetroundjoin%
\pgfsetlinewidth{0.301125pt}%
\definecolor{currentstroke}{rgb}{0.500000,0.500000,0.500000}%
\pgfsetstrokecolor{currentstroke}%
\pgfsetstrokeopacity{0.300000}%
\pgfsetdash{}{0pt}%
\pgfpathmoveto{\pgfqpoint{3.372401in}{2.761664in}}%
\pgfusepath{stroke}%
\end{pgfscope}%
\begin{pgfscope}%
\pgfpathrectangle{\pgfqpoint{0.647939in}{0.492442in}}{\pgfqpoint{3.079299in}{3.079299in}}%
\pgfusepath{clip}%
\pgfsetroundcap%
\pgfsetroundjoin%
\definecolor{currentfill}{rgb}{0.500000,0.500000,0.500000}%
\pgfsetfillcolor{currentfill}%
\pgfsetfillopacity{0.300000}%
\pgfsetlinewidth{0.301125pt}%
\definecolor{currentstroke}{rgb}{0.500000,0.500000,0.500000}%
\pgfsetstrokecolor{currentstroke}%
\pgfsetstrokeopacity{0.300000}%
\pgfsetdash{}{0pt}%
\pgfpathmoveto{\pgfqpoint{0.000000in}{0.000000in}}%
\pgfpathlineto{\pgfqpoint{0.000000in}{0.000000in}}%
\pgfpathclose%
\pgfusepath{stroke,fill}%
\end{pgfscope}%
\begin{pgfscope}%
\pgfpathrectangle{\pgfqpoint{0.647939in}{0.492442in}}{\pgfqpoint{3.079299in}{3.079299in}}%
\pgfusepath{clip}%
\pgfsetroundcap%
\pgfsetroundjoin%
\pgfsetlinewidth{0.301125pt}%
\definecolor{currentstroke}{rgb}{0.500000,0.500000,0.500000}%
\pgfsetstrokecolor{currentstroke}%
\pgfsetstrokeopacity{0.300000}%
\pgfsetdash{}{0pt}%
\pgfpathmoveto{\pgfqpoint{3.263041in}{2.327377in}}%
\pgfusepath{stroke}%
\end{pgfscope}%
\begin{pgfscope}%
\pgfpathrectangle{\pgfqpoint{0.647939in}{0.492442in}}{\pgfqpoint{3.079299in}{3.079299in}}%
\pgfusepath{clip}%
\pgfsetroundcap%
\pgfsetroundjoin%
\definecolor{currentfill}{rgb}{0.500000,0.500000,0.500000}%
\pgfsetfillcolor{currentfill}%
\pgfsetfillopacity{0.300000}%
\pgfsetlinewidth{0.301125pt}%
\definecolor{currentstroke}{rgb}{0.500000,0.500000,0.500000}%
\pgfsetstrokecolor{currentstroke}%
\pgfsetstrokeopacity{0.300000}%
\pgfsetdash{}{0pt}%
\pgfpathmoveto{\pgfqpoint{0.000000in}{0.000000in}}%
\pgfpathlineto{\pgfqpoint{0.000000in}{0.000000in}}%
\pgfpathclose%
\pgfusepath{stroke,fill}%
\end{pgfscope}%
\begin{pgfscope}%
\pgfpathrectangle{\pgfqpoint{0.647939in}{0.492442in}}{\pgfqpoint{3.079299in}{3.079299in}}%
\pgfusepath{clip}%
\pgfsetroundcap%
\pgfsetroundjoin%
\pgfsetlinewidth{0.301125pt}%
\definecolor{currentstroke}{rgb}{0.500000,0.500000,0.500000}%
\pgfsetstrokecolor{currentstroke}%
\pgfsetstrokeopacity{0.300000}%
\pgfsetdash{}{0pt}%
\pgfpathmoveto{\pgfqpoint{2.158355in}{3.063616in}}%
\pgfusepath{stroke}%
\end{pgfscope}%
\begin{pgfscope}%
\pgfpathrectangle{\pgfqpoint{0.647939in}{0.492442in}}{\pgfqpoint{3.079299in}{3.079299in}}%
\pgfusepath{clip}%
\pgfsetroundcap%
\pgfsetroundjoin%
\definecolor{currentfill}{rgb}{0.500000,0.500000,0.500000}%
\pgfsetfillcolor{currentfill}%
\pgfsetfillopacity{0.300000}%
\pgfsetlinewidth{0.301125pt}%
\definecolor{currentstroke}{rgb}{0.500000,0.500000,0.500000}%
\pgfsetstrokecolor{currentstroke}%
\pgfsetstrokeopacity{0.300000}%
\pgfsetdash{}{0pt}%
\pgfpathmoveto{\pgfqpoint{0.000000in}{0.000000in}}%
\pgfpathlineto{\pgfqpoint{0.000000in}{0.000000in}}%
\pgfpathclose%
\pgfusepath{stroke,fill}%
\end{pgfscope}%
\begin{pgfscope}%
\pgfpathrectangle{\pgfqpoint{0.647939in}{0.492442in}}{\pgfqpoint{3.079299in}{3.079299in}}%
\pgfusepath{clip}%
\pgfsetroundcap%
\pgfsetroundjoin%
\pgfsetlinewidth{0.301125pt}%
\definecolor{currentstroke}{rgb}{0.500000,0.500000,0.500000}%
\pgfsetstrokecolor{currentstroke}%
\pgfsetstrokeopacity{0.300000}%
\pgfsetdash{}{0pt}%
\pgfpathmoveto{\pgfqpoint{2.907043in}{2.154455in}}%
\pgfusepath{stroke}%
\end{pgfscope}%
\begin{pgfscope}%
\pgfpathrectangle{\pgfqpoint{0.647939in}{0.492442in}}{\pgfqpoint{3.079299in}{3.079299in}}%
\pgfusepath{clip}%
\pgfsetroundcap%
\pgfsetroundjoin%
\definecolor{currentfill}{rgb}{0.500000,0.500000,0.500000}%
\pgfsetfillcolor{currentfill}%
\pgfsetfillopacity{0.300000}%
\pgfsetlinewidth{0.301125pt}%
\definecolor{currentstroke}{rgb}{0.500000,0.500000,0.500000}%
\pgfsetstrokecolor{currentstroke}%
\pgfsetstrokeopacity{0.300000}%
\pgfsetdash{}{0pt}%
\pgfpathmoveto{\pgfqpoint{0.000000in}{0.000000in}}%
\pgfpathlineto{\pgfqpoint{0.000000in}{0.000000in}}%
\pgfpathclose%
\pgfusepath{stroke,fill}%
\end{pgfscope}%
\begin{pgfscope}%
\pgfpathrectangle{\pgfqpoint{0.647939in}{0.492442in}}{\pgfqpoint{3.079299in}{3.079299in}}%
\pgfusepath{clip}%
\pgfsetroundcap%
\pgfsetroundjoin%
\pgfsetlinewidth{0.301125pt}%
\definecolor{currentstroke}{rgb}{0.500000,0.500000,0.500000}%
\pgfsetstrokecolor{currentstroke}%
\pgfsetstrokeopacity{0.300000}%
\pgfsetdash{}{0pt}%
\pgfpathmoveto{\pgfqpoint{2.243465in}{2.925566in}}%
\pgfusepath{stroke}%
\end{pgfscope}%
\begin{pgfscope}%
\pgfpathrectangle{\pgfqpoint{0.647939in}{0.492442in}}{\pgfqpoint{3.079299in}{3.079299in}}%
\pgfusepath{clip}%
\pgfsetroundcap%
\pgfsetroundjoin%
\definecolor{currentfill}{rgb}{0.500000,0.500000,0.500000}%
\pgfsetfillcolor{currentfill}%
\pgfsetfillopacity{0.300000}%
\pgfsetlinewidth{0.301125pt}%
\definecolor{currentstroke}{rgb}{0.500000,0.500000,0.500000}%
\pgfsetstrokecolor{currentstroke}%
\pgfsetstrokeopacity{0.300000}%
\pgfsetdash{}{0pt}%
\pgfpathmoveto{\pgfqpoint{0.000000in}{0.000000in}}%
\pgfpathlineto{\pgfqpoint{0.000000in}{0.000000in}}%
\pgfpathclose%
\pgfusepath{stroke,fill}%
\end{pgfscope}%
\begin{pgfscope}%
\pgfpathrectangle{\pgfqpoint{0.647939in}{0.492442in}}{\pgfqpoint{3.079299in}{3.079299in}}%
\pgfusepath{clip}%
\pgfsetroundcap%
\pgfsetroundjoin%
\pgfsetlinewidth{0.301125pt}%
\definecolor{currentstroke}{rgb}{0.500000,0.500000,0.500000}%
\pgfsetstrokecolor{currentstroke}%
\pgfsetstrokeopacity{0.300000}%
\pgfsetdash{}{0pt}%
\pgfpathmoveto{\pgfqpoint{2.729664in}{2.159521in}}%
\pgfusepath{stroke}%
\end{pgfscope}%
\begin{pgfscope}%
\pgfpathrectangle{\pgfqpoint{0.647939in}{0.492442in}}{\pgfqpoint{3.079299in}{3.079299in}}%
\pgfusepath{clip}%
\pgfsetroundcap%
\pgfsetroundjoin%
\definecolor{currentfill}{rgb}{0.500000,0.500000,0.500000}%
\pgfsetfillcolor{currentfill}%
\pgfsetfillopacity{0.300000}%
\pgfsetlinewidth{0.301125pt}%
\definecolor{currentstroke}{rgb}{0.500000,0.500000,0.500000}%
\pgfsetstrokecolor{currentstroke}%
\pgfsetstrokeopacity{0.300000}%
\pgfsetdash{}{0pt}%
\pgfpathmoveto{\pgfqpoint{0.000000in}{0.000000in}}%
\pgfpathlineto{\pgfqpoint{0.000000in}{0.000000in}}%
\pgfpathclose%
\pgfusepath{stroke,fill}%
\end{pgfscope}%
\begin{pgfscope}%
\pgfpathrectangle{\pgfqpoint{0.647939in}{0.492442in}}{\pgfqpoint{3.079299in}{3.079299in}}%
\pgfusepath{clip}%
\pgfsetroundcap%
\pgfsetroundjoin%
\pgfsetlinewidth{0.301125pt}%
\definecolor{currentstroke}{rgb}{0.500000,0.500000,0.500000}%
\pgfsetstrokecolor{currentstroke}%
\pgfsetstrokeopacity{0.300000}%
\pgfsetdash{}{0pt}%
\pgfpathmoveto{\pgfqpoint{2.886205in}{2.389703in}}%
\pgfusepath{stroke}%
\end{pgfscope}%
\begin{pgfscope}%
\pgfpathrectangle{\pgfqpoint{0.647939in}{0.492442in}}{\pgfqpoint{3.079299in}{3.079299in}}%
\pgfusepath{clip}%
\pgfsetroundcap%
\pgfsetroundjoin%
\definecolor{currentfill}{rgb}{0.500000,0.500000,0.500000}%
\pgfsetfillcolor{currentfill}%
\pgfsetfillopacity{0.300000}%
\pgfsetlinewidth{0.301125pt}%
\definecolor{currentstroke}{rgb}{0.500000,0.500000,0.500000}%
\pgfsetstrokecolor{currentstroke}%
\pgfsetstrokeopacity{0.300000}%
\pgfsetdash{}{0pt}%
\pgfpathmoveto{\pgfqpoint{0.000000in}{0.000000in}}%
\pgfpathlineto{\pgfqpoint{0.000000in}{0.000000in}}%
\pgfpathclose%
\pgfusepath{stroke,fill}%
\end{pgfscope}%
\begin{pgfscope}%
\pgfpathrectangle{\pgfqpoint{0.647939in}{0.492442in}}{\pgfqpoint{3.079299in}{3.079299in}}%
\pgfusepath{clip}%
\pgfsetroundcap%
\pgfsetroundjoin%
\pgfsetlinewidth{0.301125pt}%
\definecolor{currentstroke}{rgb}{0.500000,0.500000,0.500000}%
\pgfsetstrokecolor{currentstroke}%
\pgfsetstrokeopacity{0.300000}%
\pgfsetdash{}{0pt}%
\pgfpathmoveto{\pgfqpoint{2.027847in}{2.697485in}}%
\pgfusepath{stroke}%
\end{pgfscope}%
\begin{pgfscope}%
\pgfpathrectangle{\pgfqpoint{0.647939in}{0.492442in}}{\pgfqpoint{3.079299in}{3.079299in}}%
\pgfusepath{clip}%
\pgfsetroundcap%
\pgfsetroundjoin%
\definecolor{currentfill}{rgb}{0.500000,0.500000,0.500000}%
\pgfsetfillcolor{currentfill}%
\pgfsetfillopacity{0.300000}%
\pgfsetlinewidth{0.301125pt}%
\definecolor{currentstroke}{rgb}{0.500000,0.500000,0.500000}%
\pgfsetstrokecolor{currentstroke}%
\pgfsetstrokeopacity{0.300000}%
\pgfsetdash{}{0pt}%
\pgfpathmoveto{\pgfqpoint{0.000000in}{0.000000in}}%
\pgfpathlineto{\pgfqpoint{0.000000in}{0.000000in}}%
\pgfpathclose%
\pgfusepath{stroke,fill}%
\end{pgfscope}%
\begin{pgfscope}%
\pgfpathrectangle{\pgfqpoint{0.647939in}{0.492442in}}{\pgfqpoint{3.079299in}{3.079299in}}%
\pgfusepath{clip}%
\pgfsetroundcap%
\pgfsetroundjoin%
\pgfsetlinewidth{0.301125pt}%
\definecolor{currentstroke}{rgb}{0.500000,0.500000,0.500000}%
\pgfsetstrokecolor{currentstroke}%
\pgfsetstrokeopacity{0.300000}%
\pgfsetdash{}{0pt}%
\pgfpathmoveto{\pgfqpoint{1.839118in}{2.145763in}}%
\pgfusepath{stroke}%
\end{pgfscope}%
\begin{pgfscope}%
\pgfpathrectangle{\pgfqpoint{0.647939in}{0.492442in}}{\pgfqpoint{3.079299in}{3.079299in}}%
\pgfusepath{clip}%
\pgfsetroundcap%
\pgfsetroundjoin%
\definecolor{currentfill}{rgb}{0.500000,0.500000,0.500000}%
\pgfsetfillcolor{currentfill}%
\pgfsetfillopacity{0.300000}%
\pgfsetlinewidth{0.301125pt}%
\definecolor{currentstroke}{rgb}{0.500000,0.500000,0.500000}%
\pgfsetstrokecolor{currentstroke}%
\pgfsetstrokeopacity{0.300000}%
\pgfsetdash{}{0pt}%
\pgfpathmoveto{\pgfqpoint{0.000000in}{0.000000in}}%
\pgfpathlineto{\pgfqpoint{0.000000in}{0.000000in}}%
\pgfpathclose%
\pgfusepath{stroke,fill}%
\end{pgfscope}%
\begin{pgfscope}%
\pgfpathrectangle{\pgfqpoint{0.647939in}{0.492442in}}{\pgfqpoint{3.079299in}{3.079299in}}%
\pgfusepath{clip}%
\pgfsetroundcap%
\pgfsetroundjoin%
\pgfsetlinewidth{0.301125pt}%
\definecolor{currentstroke}{rgb}{0.500000,0.500000,0.500000}%
\pgfsetstrokecolor{currentstroke}%
\pgfsetstrokeopacity{0.300000}%
\pgfsetdash{}{0pt}%
\pgfpathmoveto{\pgfqpoint{2.524774in}{1.818992in}}%
\pgfusepath{stroke}%
\end{pgfscope}%
\begin{pgfscope}%
\pgfpathrectangle{\pgfqpoint{0.647939in}{0.492442in}}{\pgfqpoint{3.079299in}{3.079299in}}%
\pgfusepath{clip}%
\pgfsetroundcap%
\pgfsetroundjoin%
\definecolor{currentfill}{rgb}{0.500000,0.500000,0.500000}%
\pgfsetfillcolor{currentfill}%
\pgfsetfillopacity{0.300000}%
\pgfsetlinewidth{0.301125pt}%
\definecolor{currentstroke}{rgb}{0.500000,0.500000,0.500000}%
\pgfsetstrokecolor{currentstroke}%
\pgfsetstrokeopacity{0.300000}%
\pgfsetdash{}{0pt}%
\pgfpathmoveto{\pgfqpoint{0.000000in}{0.000000in}}%
\pgfpathlineto{\pgfqpoint{0.000000in}{0.000000in}}%
\pgfpathclose%
\pgfusepath{stroke,fill}%
\end{pgfscope}%
\begin{pgfscope}%
\pgfpathrectangle{\pgfqpoint{0.647939in}{0.492442in}}{\pgfqpoint{3.079299in}{3.079299in}}%
\pgfusepath{clip}%
\pgfsetroundcap%
\pgfsetroundjoin%
\pgfsetlinewidth{0.301125pt}%
\definecolor{currentstroke}{rgb}{0.500000,0.500000,0.500000}%
\pgfsetstrokecolor{currentstroke}%
\pgfsetstrokeopacity{0.300000}%
\pgfsetdash{}{0pt}%
\pgfpathmoveto{\pgfqpoint{2.061003in}{2.434204in}}%
\pgfusepath{stroke}%
\end{pgfscope}%
\begin{pgfscope}%
\pgfpathrectangle{\pgfqpoint{0.647939in}{0.492442in}}{\pgfqpoint{3.079299in}{3.079299in}}%
\pgfusepath{clip}%
\pgfsetroundcap%
\pgfsetroundjoin%
\definecolor{currentfill}{rgb}{0.500000,0.500000,0.500000}%
\pgfsetfillcolor{currentfill}%
\pgfsetfillopacity{0.300000}%
\pgfsetlinewidth{0.301125pt}%
\definecolor{currentstroke}{rgb}{0.500000,0.500000,0.500000}%
\pgfsetstrokecolor{currentstroke}%
\pgfsetstrokeopacity{0.300000}%
\pgfsetdash{}{0pt}%
\pgfpathmoveto{\pgfqpoint{0.000000in}{0.000000in}}%
\pgfpathlineto{\pgfqpoint{0.000000in}{0.000000in}}%
\pgfpathclose%
\pgfusepath{stroke,fill}%
\end{pgfscope}%
\begin{pgfscope}%
\pgfpathrectangle{\pgfqpoint{0.647939in}{0.492442in}}{\pgfqpoint{3.079299in}{3.079299in}}%
\pgfusepath{clip}%
\pgfsetroundcap%
\pgfsetroundjoin%
\pgfsetlinewidth{0.301125pt}%
\definecolor{currentstroke}{rgb}{0.500000,0.500000,0.500000}%
\pgfsetstrokecolor{currentstroke}%
\pgfsetstrokeopacity{0.300000}%
\pgfsetdash{}{0pt}%
\pgfpathmoveto{\pgfqpoint{2.465279in}{2.126577in}}%
\pgfusepath{stroke}%
\end{pgfscope}%
\begin{pgfscope}%
\pgfpathrectangle{\pgfqpoint{0.647939in}{0.492442in}}{\pgfqpoint{3.079299in}{3.079299in}}%
\pgfusepath{clip}%
\pgfsetroundcap%
\pgfsetroundjoin%
\definecolor{currentfill}{rgb}{0.500000,0.500000,0.500000}%
\pgfsetfillcolor{currentfill}%
\pgfsetfillopacity{0.300000}%
\pgfsetlinewidth{0.301125pt}%
\definecolor{currentstroke}{rgb}{0.500000,0.500000,0.500000}%
\pgfsetstrokecolor{currentstroke}%
\pgfsetstrokeopacity{0.300000}%
\pgfsetdash{}{0pt}%
\pgfpathmoveto{\pgfqpoint{0.000000in}{0.000000in}}%
\pgfpathlineto{\pgfqpoint{0.000000in}{0.000000in}}%
\pgfpathclose%
\pgfusepath{stroke,fill}%
\end{pgfscope}%
\begin{pgfscope}%
\pgfpathrectangle{\pgfqpoint{0.647939in}{0.492442in}}{\pgfqpoint{3.079299in}{3.079299in}}%
\pgfusepath{clip}%
\pgfsetroundcap%
\pgfsetroundjoin%
\pgfsetlinewidth{0.301125pt}%
\definecolor{currentstroke}{rgb}{0.500000,0.500000,0.500000}%
\pgfsetstrokecolor{currentstroke}%
\pgfsetstrokeopacity{0.300000}%
\pgfsetdash{}{0pt}%
\pgfpathmoveto{\pgfqpoint{2.088095in}{1.685985in}}%
\pgfusepath{stroke}%
\end{pgfscope}%
\begin{pgfscope}%
\pgfpathrectangle{\pgfqpoint{0.647939in}{0.492442in}}{\pgfqpoint{3.079299in}{3.079299in}}%
\pgfusepath{clip}%
\pgfsetroundcap%
\pgfsetroundjoin%
\definecolor{currentfill}{rgb}{0.500000,0.500000,0.500000}%
\pgfsetfillcolor{currentfill}%
\pgfsetfillopacity{0.300000}%
\pgfsetlinewidth{0.301125pt}%
\definecolor{currentstroke}{rgb}{0.500000,0.500000,0.500000}%
\pgfsetstrokecolor{currentstroke}%
\pgfsetstrokeopacity{0.300000}%
\pgfsetdash{}{0pt}%
\pgfpathmoveto{\pgfqpoint{0.000000in}{0.000000in}}%
\pgfpathlineto{\pgfqpoint{0.000000in}{0.000000in}}%
\pgfpathclose%
\pgfusepath{stroke,fill}%
\end{pgfscope}%
\begin{pgfscope}%
\pgfpathrectangle{\pgfqpoint{0.647939in}{0.492442in}}{\pgfqpoint{3.079299in}{3.079299in}}%
\pgfusepath{clip}%
\pgfsetroundcap%
\pgfsetroundjoin%
\pgfsetlinewidth{0.301125pt}%
\definecolor{currentstroke}{rgb}{0.500000,0.500000,0.500000}%
\pgfsetstrokecolor{currentstroke}%
\pgfsetstrokeopacity{0.300000}%
\pgfsetdash{}{0pt}%
\pgfpathmoveto{\pgfqpoint{2.069640in}{2.253739in}}%
\pgfusepath{stroke}%
\end{pgfscope}%
\begin{pgfscope}%
\pgfpathrectangle{\pgfqpoint{0.647939in}{0.492442in}}{\pgfqpoint{3.079299in}{3.079299in}}%
\pgfusepath{clip}%
\pgfsetroundcap%
\pgfsetroundjoin%
\definecolor{currentfill}{rgb}{0.500000,0.500000,0.500000}%
\pgfsetfillcolor{currentfill}%
\pgfsetfillopacity{0.300000}%
\pgfsetlinewidth{0.301125pt}%
\definecolor{currentstroke}{rgb}{0.500000,0.500000,0.500000}%
\pgfsetstrokecolor{currentstroke}%
\pgfsetstrokeopacity{0.300000}%
\pgfsetdash{}{0pt}%
\pgfpathmoveto{\pgfqpoint{0.000000in}{0.000000in}}%
\pgfpathlineto{\pgfqpoint{0.000000in}{0.000000in}}%
\pgfpathclose%
\pgfusepath{stroke,fill}%
\end{pgfscope}%
\begin{pgfscope}%
\pgfpathrectangle{\pgfqpoint{0.647939in}{0.492442in}}{\pgfqpoint{3.079299in}{3.079299in}}%
\pgfusepath{clip}%
\pgfsetbuttcap%
\pgfsetroundjoin%
\pgfsetlinewidth{0.301125pt}%
\definecolor{currentstroke}{rgb}{0.500000,0.500000,0.500000}%
\pgfsetstrokecolor{currentstroke}%
\pgfsetstrokeopacity{0.300000}%
\pgfsetdash{}{0pt}%
\pgfpathmoveto{\pgfqpoint{0.647939in}{0.492442in}}%
\pgfpathlineto{\pgfqpoint{0.647939in}{0.492442in}}%
\pgfpathlineto{\pgfqpoint{0.714960in}{0.506165in}}%
\pgfpathlineto{\pgfqpoint{0.781262in}{0.522990in}}%
\pgfpathlineto{\pgfqpoint{0.846570in}{0.543308in}}%
\pgfpathlineto{\pgfqpoint{0.910539in}{0.567481in}}%
\pgfpathlineto{\pgfqpoint{0.972768in}{0.595812in}}%
\pgfpathlineto{\pgfqpoint{1.032810in}{0.628499in}}%
\pgfpathlineto{\pgfqpoint{1.090217in}{0.665604in}}%
\pgfpathlineto{\pgfqpoint{1.144595in}{0.707032in}}%
\pgfpathlineto{\pgfqpoint{1.195666in}{0.752462in}}%
\pgfpathlineto{\pgfqpoint{1.243323in}{0.801449in}}%
\pgfpathlineto{\pgfqpoint{1.287657in}{0.853468in}}%
\pgfpathlineto{\pgfqpoint{1.328937in}{0.907954in}}%
\pgfpathlineto{\pgfqpoint{1.367578in}{0.964355in}}%
\pgfpathlineto{\pgfqpoint{1.404074in}{1.022159in}}%
\pgfpathlineto{\pgfqpoint{1.438940in}{1.080940in}}%
\pgfpathlineto{\pgfqpoint{1.472682in}{1.140360in}}%
\pgfpathlineto{\pgfqpoint{1.505773in}{1.200137in}}%
\pgfpathlineto{\pgfqpoint{1.538646in}{1.260026in}}%
\pgfpathlineto{\pgfqpoint{1.571691in}{1.319817in}}%
\pgfpathlineto{\pgfqpoint{1.605263in}{1.379314in}}%
\pgfpathlineto{\pgfqpoint{1.639691in}{1.438332in}}%
\pgfpathlineto{\pgfqpoint{1.675294in}{1.496682in}}%
\pgfpathlineto{\pgfqpoint{1.712375in}{1.554078in}}%
\pgfpathlineto{\pgfqpoint{1.751236in}{1.610234in}}%
\pgfpathlineto{\pgfqpoint{1.792173in}{1.664836in}}%
\pgfpathlineto{\pgfqpoint{1.835508in}{1.717475in}}%
\pgfpathlineto{\pgfqpoint{1.881626in}{1.767649in}}%
\pgfpathlineto{\pgfqpoint{1.930751in}{1.814548in}}%
\pgfpathlineto{\pgfqpoint{1.983241in}{1.857224in}}%
\pgfpathlineto{\pgfqpoint{1.983241in}{1.857224in}}%
\pgfpathlineto{\pgfqpoint{2.014724in}{1.880442in}}%
\pgfpathlineto{\pgfqpoint{2.014724in}{1.880442in}}%
\pgfpathlineto{\pgfqpoint{2.029402in}{1.891032in}}%
\pgfpathlineto{\pgfqpoint{2.038196in}{1.897128in}}%
\pgfpathlineto{\pgfqpoint{2.042696in}{1.900118in}}%
\pgfpathlineto{\pgfqpoint{2.044665in}{1.901347in}}%
\pgfpathlineto{\pgfqpoint{2.045923in}{1.901898in}}%
\pgfpathlineto{\pgfqpoint{2.049374in}{1.903622in}}%
\pgfpathlineto{\pgfqpoint{2.049374in}{1.903622in}}%
\pgfpathlineto{\pgfqpoint{2.049301in}{1.904012in}}%
\pgfpathlineto{\pgfqpoint{2.048645in}{1.903855in}}%
\pgfpathlineto{\pgfqpoint{2.048427in}{1.903846in}}%
\pgfpathlineto{\pgfqpoint{2.048650in}{1.904061in}}%
\pgfpathlineto{\pgfqpoint{2.049098in}{1.904405in}}%
\pgfpathlineto{\pgfqpoint{2.049373in}{1.904634in}}%
\pgfpathlineto{\pgfqpoint{2.049286in}{1.904627in}}%
\pgfpathlineto{\pgfqpoint{2.049021in}{1.904511in}}%
\pgfpathlineto{\pgfqpoint{2.048750in}{1.904420in}}%
\pgfpathlineto{\pgfqpoint{2.048387in}{1.904351in}}%
\pgfpathlineto{\pgfqpoint{2.047155in}{1.904007in}}%
\pgfpathlineto{\pgfqpoint{2.041578in}{1.901497in}}%
\pgfpathlineto{\pgfqpoint{2.041578in}{1.901497in}}%
\pgfpathlineto{\pgfqpoint{2.043383in}{1.901826in}}%
\pgfpathlineto{\pgfqpoint{2.044531in}{1.902075in}}%
\pgfpathlineto{\pgfqpoint{2.044695in}{1.901977in}}%
\pgfpathlineto{\pgfqpoint{2.044183in}{1.901613in}}%
\pgfpathlineto{\pgfqpoint{2.043334in}{1.901103in}}%
\pgfpathlineto{\pgfqpoint{2.043000in}{1.900883in}}%
\pgfpathlineto{\pgfqpoint{2.043408in}{1.901066in}}%
\pgfpathlineto{\pgfqpoint{2.043911in}{1.901299in}}%
\pgfpathlineto{\pgfqpoint{2.044129in}{1.901370in}}%
\pgfpathlineto{\pgfqpoint{2.044057in}{1.901263in}}%
\pgfpathlineto{\pgfqpoint{2.043905in}{1.901080in}}%
\pgfpathlineto{\pgfqpoint{2.044017in}{1.901000in}}%
\pgfpathlineto{\pgfqpoint{2.044760in}{1.901159in}}%
\pgfpathlineto{\pgfqpoint{2.047753in}{1.902403in}}%
\pgfpathlineto{\pgfqpoint{2.047753in}{1.902403in}}%
\pgfpathlineto{\pgfqpoint{2.048932in}{1.903630in}}%
\pgfpathlineto{\pgfqpoint{2.048682in}{1.903790in}}%
\pgfpathlineto{\pgfqpoint{2.048488in}{1.903832in}}%
\pgfpathlineto{\pgfqpoint{2.048631in}{1.904010in}}%
\pgfpathlineto{\pgfqpoint{2.049004in}{1.904308in}}%
\pgfpathlineto{\pgfqpoint{2.049301in}{1.904550in}}%
\pgfpathlineto{\pgfqpoint{2.049304in}{1.904601in}}%
\pgfpathlineto{\pgfqpoint{2.049109in}{1.904525in}}%
\pgfpathlineto{\pgfqpoint{2.048889in}{1.904448in}}%
\pgfpathlineto{\pgfqpoint{2.048661in}{1.904408in}}%
\pgfpathlineto{\pgfqpoint{2.048164in}{1.904319in}}%
\pgfpathlineto{\pgfqpoint{2.045935in}{1.903497in}}%
\pgfpathlineto{\pgfqpoint{2.045935in}{1.903497in}}%
\pgfpathlineto{\pgfqpoint{2.043315in}{1.901844in}}%
\pgfpathlineto{\pgfqpoint{2.043315in}{1.901844in}}%
\pgfpathlineto{\pgfqpoint{2.043712in}{1.901716in}}%
\pgfpathlineto{\pgfqpoint{2.044213in}{1.901772in}}%
\pgfpathlineto{\pgfqpoint{2.044283in}{1.901689in}}%
\pgfpathlineto{\pgfqpoint{2.043961in}{1.901443in}}%
\pgfpathlineto{\pgfqpoint{2.043523in}{1.901152in}}%
\pgfpathlineto{\pgfqpoint{2.043381in}{1.901033in}}%
\pgfpathlineto{\pgfqpoint{2.043596in}{1.901113in}}%
\pgfpathlineto{\pgfqpoint{2.043878in}{1.901227in}}%
\pgfpathlineto{\pgfqpoint{2.044026in}{1.901256in}}%
\pgfpathlineto{\pgfqpoint{2.044046in}{1.901189in}}%
\pgfpathlineto{\pgfqpoint{2.044122in}{1.901103in}}%
\pgfpathlineto{\pgfqpoint{2.044648in}{1.901146in}}%
\pgfpathlineto{\pgfqpoint{2.047195in}{1.902122in}}%
\pgfpathlineto{\pgfqpoint{2.047195in}{1.902122in}}%
\pgfpathlineto{\pgfqpoint{2.049060in}{1.903663in}}%
\pgfpathlineto{\pgfqpoint{2.049254in}{1.904084in}}%
\pgfpathlineto{\pgfqpoint{2.048834in}{1.904012in}}%
\pgfpathlineto{\pgfqpoint{2.048639in}{1.903997in}}%
\pgfpathlineto{\pgfqpoint{2.048767in}{1.904146in}}%
\pgfpathlineto{\pgfqpoint{2.049058in}{1.904390in}}%
\pgfpathlineto{\pgfqpoint{2.049243in}{1.904564in}}%
\pgfpathlineto{\pgfqpoint{2.049185in}{1.904582in}}%
\pgfpathlineto{\pgfqpoint{2.048970in}{1.904510in}}%
\pgfpathlineto{\pgfqpoint{2.048681in}{1.904427in}}%
\pgfpathlineto{\pgfqpoint{2.048133in}{1.904300in}}%
\pgfpathlineto{\pgfqpoint{2.045865in}{1.903460in}}%
\pgfpathlineto{\pgfqpoint{2.045865in}{1.903460in}}%
\pgfpathlineto{\pgfqpoint{2.043345in}{1.901849in}}%
\pgfpathlineto{\pgfqpoint{2.043345in}{1.901849in}}%
\pgfpathlineto{\pgfqpoint{2.043717in}{1.901714in}}%
\pgfpathlineto{\pgfqpoint{2.044202in}{1.901764in}}%
\pgfpathlineto{\pgfqpoint{2.044270in}{1.901681in}}%
\pgfpathlineto{\pgfqpoint{2.043956in}{1.901439in}}%
\pgfpathlineto{\pgfqpoint{2.043530in}{1.901155in}}%
\pgfpathlineto{\pgfqpoint{2.043391in}{1.901038in}}%
\pgfpathlineto{\pgfqpoint{2.043600in}{1.901114in}}%
\pgfusepath{stroke}%
\end{pgfscope}%
\begin{pgfscope}%
\pgfpathrectangle{\pgfqpoint{0.647939in}{0.492442in}}{\pgfqpoint{3.079299in}{3.079299in}}%
\pgfusepath{clip}%
\pgfsetbuttcap%
\pgfsetroundjoin%
\pgfsetlinewidth{0.301125pt}%
\definecolor{currentstroke}{rgb}{0.500000,0.500000,0.500000}%
\pgfsetstrokecolor{currentstroke}%
\pgfsetstrokeopacity{0.300000}%
\pgfsetdash{}{0pt}%
\pgfpathmoveto{\pgfqpoint{0.997859in}{0.492442in}}%
\pgfpathlineto{\pgfqpoint{0.997859in}{0.492442in}}%
\pgfpathlineto{\pgfqpoint{1.055367in}{0.529381in}}%
\pgfpathlineto{\pgfqpoint{1.109565in}{0.571022in}}%
\pgfpathlineto{\pgfqpoint{1.160087in}{0.617024in}}%
\pgfpathlineto{\pgfqpoint{1.206787in}{0.666917in}}%
\pgfpathlineto{\pgfqpoint{1.249768in}{0.720048in}}%
\pgfusepath{stroke}%
\end{pgfscope}%
\begin{pgfscope}%
\pgfpathrectangle{\pgfqpoint{0.647939in}{0.492442in}}{\pgfqpoint{3.079299in}{3.079299in}}%
\pgfusepath{clip}%
\pgfsetbuttcap%
\pgfsetroundjoin%
\pgfsetlinewidth{0.301125pt}%
\definecolor{currentstroke}{rgb}{0.500000,0.500000,0.500000}%
\pgfsetstrokecolor{currentstroke}%
\pgfsetstrokeopacity{0.300000}%
\pgfsetdash{}{0pt}%
\pgfpathmoveto{\pgfqpoint{1.277796in}{0.492442in}}%
\pgfpathlineto{\pgfqpoint{1.277796in}{0.492442in}}%
\pgfpathlineto{\pgfqpoint{1.296526in}{0.558141in}}%
\pgfpathlineto{\pgfqpoint{1.316105in}{0.623574in}}%
\pgfpathlineto{\pgfqpoint{1.336469in}{0.688770in}}%
\pgfpathlineto{\pgfqpoint{1.357608in}{0.753745in}}%
\pgfpathlineto{\pgfqpoint{1.379519in}{0.818428in}}%
\pgfpathlineto{\pgfqpoint{1.402208in}{0.882833in}}%
\pgfpathlineto{\pgfqpoint{1.425712in}{0.947013in}}%
\pgfpathlineto{\pgfqpoint{1.450088in}{1.010877in}}%
\pgfpathlineto{\pgfqpoint{1.475377in}{1.074322in}}%
\pgfusepath{stroke}%
\end{pgfscope}%
\begin{pgfscope}%
\pgfpathrectangle{\pgfqpoint{0.647939in}{0.492442in}}{\pgfqpoint{3.079299in}{3.079299in}}%
\pgfusepath{clip}%
\pgfsetbuttcap%
\pgfsetroundjoin%
\pgfsetlinewidth{0.301125pt}%
\definecolor{currentstroke}{rgb}{0.500000,0.500000,0.500000}%
\pgfsetstrokecolor{currentstroke}%
\pgfsetstrokeopacity{0.300000}%
\pgfsetdash{}{0pt}%
\pgfpathmoveto{\pgfqpoint{1.417764in}{0.492442in}}%
\pgfpathlineto{\pgfqpoint{1.417764in}{0.492442in}}%
\pgfpathlineto{\pgfqpoint{1.397146in}{0.556959in}}%
\pgfpathlineto{\pgfqpoint{1.389186in}{0.616167in}}%
\pgfpathlineto{\pgfqpoint{1.389317in}{0.683506in}}%
\pgfpathlineto{\pgfqpoint{1.396394in}{0.751374in}}%
\pgfpathlineto{\pgfqpoint{1.408492in}{0.818538in}}%
\pgfusepath{stroke}%
\end{pgfscope}%
\begin{pgfscope}%
\pgfpathrectangle{\pgfqpoint{0.647939in}{0.492442in}}{\pgfqpoint{3.079299in}{3.079299in}}%
\pgfusepath{clip}%
\pgfsetbuttcap%
\pgfsetroundjoin%
\pgfsetlinewidth{0.301125pt}%
\definecolor{currentstroke}{rgb}{0.500000,0.500000,0.500000}%
\pgfsetstrokecolor{currentstroke}%
\pgfsetstrokeopacity{0.300000}%
\pgfsetdash{}{0pt}%
\pgfpathmoveto{\pgfqpoint{1.697700in}{0.492442in}}%
\pgfpathlineto{\pgfqpoint{1.697700in}{0.492442in}}%
\pgfpathlineto{\pgfqpoint{1.633281in}{0.515018in}}%
\pgfpathlineto{\pgfqpoint{1.573178in}{0.547038in}}%
\pgfpathlineto{\pgfqpoint{1.521056in}{0.590510in}}%
\pgfpathlineto{\pgfqpoint{1.484696in}{0.639454in}}%
\pgfpathlineto{\pgfqpoint{1.462113in}{0.690609in}}%
\pgfusepath{stroke}%
\end{pgfscope}%
\begin{pgfscope}%
\pgfpathrectangle{\pgfqpoint{0.647939in}{0.492442in}}{\pgfqpoint{3.079299in}{3.079299in}}%
\pgfusepath{clip}%
\pgfsetbuttcap%
\pgfsetroundjoin%
\pgfsetlinewidth{0.301125pt}%
\definecolor{currentstroke}{rgb}{0.500000,0.500000,0.500000}%
\pgfsetstrokecolor{currentstroke}%
\pgfsetstrokeopacity{0.300000}%
\pgfsetdash{}{0pt}%
\pgfpathmoveto{\pgfqpoint{2.475563in}{0.492442in}}%
\pgfpathlineto{\pgfqpoint{2.458143in}{0.493081in}}%
\pgfpathlineto{\pgfqpoint{2.389741in}{0.494850in}}%
\pgfpathlineto{\pgfqpoint{2.321316in}{0.495315in}}%
\pgfpathlineto{\pgfqpoint{2.252890in}{0.494848in}}%
\pgfpathlineto{\pgfqpoint{2.184468in}{0.493874in}}%
\pgfpathlineto{\pgfqpoint{2.116047in}{0.492886in}}%
\pgfpathlineto{\pgfqpoint{2.047620in}{0.492442in}}%
\pgfpathlineto{\pgfqpoint{2.047620in}{0.492442in}}%
\pgfusepath{stroke}%
\end{pgfscope}%
\begin{pgfscope}%
\pgfpathrectangle{\pgfqpoint{0.647939in}{0.492442in}}{\pgfqpoint{3.079299in}{3.079299in}}%
\pgfusepath{clip}%
\pgfsetbuttcap%
\pgfsetroundjoin%
\pgfsetlinewidth{0.301125pt}%
\definecolor{currentstroke}{rgb}{0.500000,0.500000,0.500000}%
\pgfsetstrokecolor{currentstroke}%
\pgfsetstrokeopacity{0.300000}%
\pgfsetdash{}{0pt}%
\pgfpathmoveto{\pgfqpoint{2.817445in}{0.492442in}}%
\pgfpathlineto{\pgfqpoint{2.817445in}{0.492442in}}%
\pgfpathlineto{\pgfqpoint{2.750080in}{0.504408in}}%
\pgfpathlineto{\pgfqpoint{2.682388in}{0.514362in}}%
\pgfpathlineto{\pgfqpoint{2.614426in}{0.522259in}}%
\pgfpathlineto{\pgfqpoint{2.546257in}{0.528130in}}%
\pgfpathlineto{\pgfqpoint{2.477950in}{0.532095in}}%
\pgfpathlineto{\pgfqpoint{2.409563in}{0.534355in}}%
\pgfpathlineto{\pgfqpoint{2.341143in}{0.535189in}}%
\pgfpathlineto{\pgfqpoint{2.272717in}{0.534946in}}%
\pgfpathlineto{\pgfqpoint{2.204294in}{0.534042in}}%
\pgfpathlineto{\pgfqpoint{2.135874in}{0.532969in}}%
\pgfpathlineto{\pgfqpoint{2.067449in}{0.532291in}}%
\pgfpathlineto{\pgfqpoint{1.999025in}{0.532643in}}%
\pgfpathlineto{\pgfqpoint{1.930639in}{0.534765in}}%
\pgfpathlineto{\pgfqpoint{1.862397in}{0.539541in}}%
\pgfpathlineto{\pgfqpoint{1.794542in}{0.548080in}}%
\pgfpathlineto{\pgfqpoint{1.727584in}{0.561800in}}%
\pgfpathlineto{\pgfqpoint{1.662545in}{0.582554in}}%
\pgfpathlineto{\pgfqpoint{1.601451in}{0.612661in}}%
\pgfpathlineto{\pgfqpoint{1.547965in}{0.654399in}}%
\pgfusepath{stroke}%
\end{pgfscope}%
\begin{pgfscope}%
\pgfpathrectangle{\pgfqpoint{0.647939in}{0.492442in}}{\pgfqpoint{3.079299in}{3.079299in}}%
\pgfusepath{clip}%
\pgfsetbuttcap%
\pgfsetroundjoin%
\pgfsetlinewidth{0.301125pt}%
\definecolor{currentstroke}{rgb}{0.500000,0.500000,0.500000}%
\pgfsetstrokecolor{currentstroke}%
\pgfsetstrokeopacity{0.300000}%
\pgfsetdash{}{0pt}%
\pgfpathmoveto{\pgfqpoint{3.027398in}{0.492442in}}%
\pgfpathlineto{\pgfqpoint{3.027398in}{0.492442in}}%
\pgfpathlineto{\pgfqpoint{2.961307in}{0.510161in}}%
\pgfpathlineto{\pgfqpoint{2.894835in}{0.526379in}}%
\pgfpathlineto{\pgfqpoint{2.827965in}{0.540872in}}%
\pgfpathlineto{\pgfqpoint{2.760712in}{0.553456in}}%
\pgfpathlineto{\pgfqpoint{2.693109in}{0.564002in}}%
\pgfpathlineto{\pgfqpoint{2.625214in}{0.572450in}}%
\pgfusepath{stroke}%
\end{pgfscope}%
\begin{pgfscope}%
\pgfpathrectangle{\pgfqpoint{0.647939in}{0.492442in}}{\pgfqpoint{3.079299in}{3.079299in}}%
\pgfusepath{clip}%
\pgfsetbuttcap%
\pgfsetroundjoin%
\pgfsetlinewidth{0.301125pt}%
\definecolor{currentstroke}{rgb}{0.500000,0.500000,0.500000}%
\pgfsetstrokecolor{currentstroke}%
\pgfsetstrokeopacity{0.300000}%
\pgfsetdash{}{0pt}%
\pgfpathmoveto{\pgfqpoint{3.237350in}{0.492442in}}%
\pgfpathlineto{\pgfqpoint{3.237350in}{0.492442in}}%
\pgfpathlineto{\pgfqpoint{3.172338in}{0.513792in}}%
\pgfpathlineto{\pgfqpoint{3.107118in}{0.534496in}}%
\pgfpathlineto{\pgfqpoint{3.041611in}{0.554272in}}%
\pgfpathlineto{\pgfqpoint{2.975754in}{0.572840in}}%
\pgfpathlineto{\pgfqpoint{2.909501in}{0.589938in}}%
\pgfpathlineto{\pgfqpoint{2.842833in}{0.605326in}}%
\pgfpathlineto{\pgfqpoint{2.775754in}{0.618801in}}%
\pgfpathlineto{\pgfqpoint{2.708293in}{0.630210in}}%
\pgfpathlineto{\pgfqpoint{2.640503in}{0.639465in}}%
\pgfpathlineto{\pgfqpoint{2.572451in}{0.646557in}}%
\pgfpathlineto{\pgfqpoint{2.504215in}{0.651569in}}%
\pgfpathlineto{\pgfqpoint{2.435863in}{0.654670in}}%
\pgfpathlineto{\pgfqpoint{2.367455in}{0.656117in}}%
\pgfpathlineto{\pgfqpoint{2.299029in}{0.656246in}}%
\pgfpathlineto{\pgfqpoint{2.230605in}{0.655481in}}%
\pgfpathlineto{\pgfqpoint{2.162186in}{0.654324in}}%
\pgfpathlineto{\pgfqpoint{2.093764in}{0.653347in}}%
\pgfpathlineto{\pgfqpoint{2.025339in}{0.653212in}}%
\pgfpathlineto{\pgfqpoint{1.956934in}{0.654693in}}%
\pgfpathlineto{\pgfqpoint{1.888644in}{0.658739in}}%
\pgfpathlineto{\pgfqpoint{1.820705in}{0.666550in}}%
\pgfpathlineto{\pgfqpoint{1.753639in}{0.679687in}}%
\pgfpathlineto{\pgfqpoint{1.688561in}{0.700242in}}%
\pgfpathlineto{\pgfqpoint{1.627802in}{0.730885in}}%
\pgfpathlineto{\pgfqpoint{1.575715in}{0.774101in}}%
\pgfpathlineto{\pgfqpoint{1.541978in}{0.820997in}}%
\pgfpathlineto{\pgfqpoint{1.521786in}{0.869952in}}%
\pgfpathlineto{\pgfqpoint{1.511401in}{0.922726in}}%
\pgfpathlineto{\pgfqpoint{1.509494in}{0.981236in}}%
\pgfusepath{stroke}%
\end{pgfscope}%
\begin{pgfscope}%
\pgfpathrectangle{\pgfqpoint{0.647939in}{0.492442in}}{\pgfqpoint{3.079299in}{3.079299in}}%
\pgfusepath{clip}%
\pgfsetbuttcap%
\pgfsetroundjoin%
\pgfsetlinewidth{0.301125pt}%
\definecolor{currentstroke}{rgb}{0.500000,0.500000,0.500000}%
\pgfsetstrokecolor{currentstroke}%
\pgfsetstrokeopacity{0.300000}%
\pgfsetdash{}{0pt}%
\pgfpathmoveto{\pgfqpoint{3.447302in}{0.492442in}}%
\pgfpathlineto{\pgfqpoint{3.447302in}{0.492442in}}%
\pgfpathlineto{\pgfqpoint{3.382565in}{0.514613in}}%
\pgfpathlineto{\pgfqpoint{3.317920in}{0.537052in}}%
\pgfpathlineto{\pgfqpoint{3.253273in}{0.559485in}}%
\pgfpathlineto{\pgfqpoint{3.188528in}{0.581632in}}%
\pgfpathlineto{\pgfqpoint{3.123591in}{0.603208in}}%
\pgfpathlineto{\pgfqpoint{3.058377in}{0.623926in}}%
\pgfpathlineto{\pgfqpoint{2.992812in}{0.643501in}}%
\pgfpathlineto{\pgfqpoint{2.926841in}{0.661655in}}%
\pgfpathlineto{\pgfqpoint{2.860433in}{0.678130in}}%
\pgfpathlineto{\pgfqpoint{2.793581in}{0.692698in}}%
\pgfpathlineto{\pgfqpoint{2.726309in}{0.705177in}}%
\pgfpathlineto{\pgfqpoint{2.658667in}{0.715449in}}%
\pgfpathlineto{\pgfqpoint{2.590721in}{0.723474in}}%
\pgfpathlineto{\pgfqpoint{2.522551in}{0.729300in}}%
\pgfpathlineto{\pgfqpoint{2.454234in}{0.733068in}}%
\pgfpathlineto{\pgfqpoint{2.385838in}{0.735018in}}%
\pgfpathlineto{\pgfqpoint{2.317414in}{0.735487in}}%
\pgfpathlineto{\pgfqpoint{2.248990in}{0.734892in}}%
\pgfpathlineto{\pgfqpoint{2.180571in}{0.733727in}}%
\pgfpathlineto{\pgfqpoint{2.112152in}{0.732569in}}%
\pgfpathlineto{\pgfqpoint{2.043727in}{0.732099in}}%
\pgfpathlineto{\pgfqpoint{1.975314in}{0.733131in}}%
\pgfpathlineto{\pgfqpoint{1.906996in}{0.736645in}}%
\pgfpathlineto{\pgfqpoint{1.838994in}{0.743888in}}%
\pgfpathlineto{\pgfqpoint{1.771827in}{0.756518in}}%
\pgfpathlineto{\pgfqpoint{1.706654in}{0.776815in}}%
\pgfpathlineto{\pgfqpoint{1.646019in}{0.807744in}}%
\pgfpathlineto{\pgfqpoint{1.646019in}{0.807744in}}%
\pgfpathlineto{\pgfqpoint{1.602616in}{0.843117in}}%
\pgfpathlineto{\pgfqpoint{1.568707in}{0.888319in}}%
\pgfpathlineto{\pgfqpoint{1.548348in}{0.935967in}}%
\pgfpathlineto{\pgfqpoint{1.537931in}{0.987269in}}%
\pgfpathlineto{\pgfqpoint{1.536042in}{1.043986in}}%
\pgfpathlineto{\pgfqpoint{1.542712in}{1.108793in}}%
\pgfpathlineto{\pgfqpoint{1.556682in}{1.175538in}}%
\pgfusepath{stroke}%
\end{pgfscope}%
\begin{pgfscope}%
\pgfpathrectangle{\pgfqpoint{0.647939in}{0.492442in}}{\pgfqpoint{3.079299in}{3.079299in}}%
\pgfusepath{clip}%
\pgfsetbuttcap%
\pgfsetroundjoin%
\pgfsetlinewidth{0.301125pt}%
\definecolor{currentstroke}{rgb}{0.500000,0.500000,0.500000}%
\pgfsetstrokecolor{currentstroke}%
\pgfsetstrokeopacity{0.300000}%
\pgfsetdash{}{0pt}%
\pgfpathmoveto{\pgfqpoint{3.657254in}{0.492442in}}%
\pgfpathlineto{\pgfqpoint{3.657254in}{0.492442in}}%
\pgfpathlineto{\pgfqpoint{3.591844in}{0.512529in}}%
\pgfpathlineto{\pgfqpoint{3.526786in}{0.533736in}}%
\pgfpathlineto{\pgfqpoint{3.462020in}{0.555820in}}%
\pgfpathlineto{\pgfqpoint{3.397469in}{0.578526in}}%
\pgfpathlineto{\pgfqpoint{3.333045in}{0.601589in}}%
\pgfpathlineto{\pgfqpoint{3.268649in}{0.624735in}}%
\pgfpathlineto{\pgfqpoint{3.204183in}{0.647681in}}%
\pgfpathlineto{\pgfqpoint{3.139545in}{0.670140in}}%
\pgfpathlineto{\pgfqpoint{3.074643in}{0.691818in}}%
\pgfpathlineto{\pgfqpoint{3.009393in}{0.712421in}}%
\pgfpathlineto{\pgfqpoint{2.943729in}{0.731656in}}%
\pgfpathlineto{\pgfqpoint{2.877608in}{0.749249in}}%
\pgfpathlineto{\pgfqpoint{2.811014in}{0.764950in}}%
\pgfpathlineto{\pgfqpoint{2.743962in}{0.778550in}}%
\pgfpathlineto{\pgfqpoint{2.676493in}{0.789898in}}%
\pgfpathlineto{\pgfqpoint{2.608672in}{0.798917in}}%
\pgfpathlineto{\pgfqpoint{2.540583in}{0.805624in}}%
\pgfpathlineto{\pgfqpoint{2.472312in}{0.810137in}}%
\pgfpathlineto{\pgfqpoint{2.403938in}{0.812674in}}%
\pgfpathlineto{\pgfqpoint{2.335519in}{0.813546in}}%
\pgfpathlineto{\pgfqpoint{2.267094in}{0.813160in}}%
\pgfpathlineto{\pgfqpoint{2.198675in}{0.812019in}}%
\pgfpathlineto{\pgfqpoint{2.130258in}{0.810720in}}%
\pgfpathlineto{\pgfqpoint{2.061836in}{0.809958in}}%
\pgfpathlineto{\pgfqpoint{1.993416in}{0.810553in}}%
\pgfpathlineto{\pgfqpoint{1.925069in}{0.813519in}}%
\pgfpathlineto{\pgfqpoint{1.857004in}{0.820172in}}%
\pgfpathlineto{\pgfqpoint{1.789752in}{0.832295in}}%
\pgfpathlineto{\pgfqpoint{1.724556in}{0.852361in}}%
\pgfpathlineto{\pgfqpoint{1.664239in}{0.883624in}}%
\pgfpathlineto{\pgfqpoint{1.664239in}{0.883624in}}%
\pgfpathlineto{\pgfqpoint{1.622904in}{0.918468in}}%
\pgfusepath{stroke}%
\end{pgfscope}%
\begin{pgfscope}%
\pgfpathrectangle{\pgfqpoint{0.647939in}{0.492442in}}{\pgfqpoint{3.079299in}{3.079299in}}%
\pgfusepath{clip}%
\pgfsetbuttcap%
\pgfsetroundjoin%
\pgfsetlinewidth{0.301125pt}%
\definecolor{currentstroke}{rgb}{0.500000,0.500000,0.500000}%
\pgfsetstrokecolor{currentstroke}%
\pgfsetstrokeopacity{0.300000}%
\pgfsetdash{}{0pt}%
\pgfpathmoveto{\pgfqpoint{3.727238in}{0.562426in}}%
\pgfpathlineto{\pgfqpoint{3.727238in}{0.562426in}}%
\pgfpathlineto{\pgfqpoint{3.661614in}{0.581800in}}%
\pgfpathlineto{\pgfqpoint{3.596423in}{0.602587in}}%
\pgfpathlineto{\pgfqpoint{3.531621in}{0.624560in}}%
\pgfpathlineto{\pgfqpoint{3.467144in}{0.647475in}}%
\pgfpathlineto{\pgfqpoint{3.402915in}{0.671074in}}%
\pgfpathlineto{\pgfqpoint{3.338839in}{0.695090in}}%
\pgfpathlineto{\pgfqpoint{3.274816in}{0.719246in}}%
\pgfpathlineto{\pgfqpoint{3.210738in}{0.743258in}}%
\pgfpathlineto{\pgfqpoint{3.146500in}{0.766832in}}%
\pgfpathlineto{\pgfqpoint{3.081997in}{0.789671in}}%
\pgfpathlineto{\pgfqpoint{3.017138in}{0.811472in}}%
\pgfpathlineto{\pgfqpoint{2.951844in}{0.831931in}}%
\pgfpathlineto{\pgfqpoint{2.886062in}{0.850753in}}%
\pgfpathlineto{\pgfqpoint{2.819764in}{0.867663in}}%
\pgfpathlineto{\pgfqpoint{2.752958in}{0.882423in}}%
\pgfpathlineto{\pgfqpoint{2.685681in}{0.894850in}}%
\pgfpathlineto{\pgfqpoint{2.617998in}{0.904839in}}%
\pgfpathlineto{\pgfqpoint{2.549998in}{0.912371in}}%
\pgfpathlineto{\pgfqpoint{2.481775in}{0.917537in}}%
\pgfpathlineto{\pgfqpoint{2.413420in}{0.920545in}}%
\pgfpathlineto{\pgfqpoint{2.345007in}{0.921714in}}%
\pgfpathlineto{\pgfqpoint{2.276582in}{0.921459in}}%
\pgfpathlineto{\pgfqpoint{2.208164in}{0.920298in}}%
\pgfpathlineto{\pgfqpoint{2.139750in}{0.918850in}}%
\pgfpathlineto{\pgfqpoint{2.071330in}{0.917868in}}%
\pgfpathlineto{\pgfqpoint{2.002910in}{0.918262in}}%
\pgfpathlineto{\pgfqpoint{1.934564in}{0.921159in}}%
\pgfpathlineto{\pgfqpoint{1.866534in}{0.928060in}}%
\pgfpathlineto{\pgfqpoint{1.799464in}{0.941078in}}%
\pgfpathlineto{\pgfqpoint{1.735022in}{0.963286in}}%
\pgfpathlineto{\pgfqpoint{1.677181in}{0.998643in}}%
\pgfpathlineto{\pgfqpoint{1.638085in}{1.041206in}}%
\pgfpathlineto{\pgfqpoint{1.615603in}{1.085320in}}%
\pgfpathlineto{\pgfqpoint{1.604101in}{1.132007in}}%
\pgfpathlineto{\pgfqpoint{1.601295in}{1.183410in}}%
\pgfpathlineto{\pgfqpoint{1.606998in}{1.241858in}}%
\pgfpathlineto{\pgfqpoint{1.621875in}{1.308307in}}%
\pgfusepath{stroke}%
\end{pgfscope}%
\begin{pgfscope}%
\pgfpathrectangle{\pgfqpoint{0.647939in}{0.492442in}}{\pgfqpoint{3.079299in}{3.079299in}}%
\pgfusepath{clip}%
\pgfsetbuttcap%
\pgfsetroundjoin%
\pgfsetlinewidth{0.301125pt}%
\definecolor{currentstroke}{rgb}{0.500000,0.500000,0.500000}%
\pgfsetstrokecolor{currentstroke}%
\pgfsetstrokeopacity{0.300000}%
\pgfsetdash{}{0pt}%
\pgfpathmoveto{\pgfqpoint{3.727238in}{0.632410in}}%
\pgfpathlineto{\pgfqpoint{3.727238in}{0.632410in}}%
\pgfpathlineto{\pgfqpoint{3.661794in}{0.652383in}}%
\pgfpathlineto{\pgfqpoint{3.596813in}{0.673816in}}%
\pgfpathlineto{\pgfqpoint{3.532249in}{0.696479in}}%
\pgfpathlineto{\pgfqpoint{3.468038in}{0.720126in}}%
\pgfpathlineto{\pgfqpoint{3.404097in}{0.744497in}}%
\pgfpathlineto{\pgfqpoint{3.340331in}{0.769323in}}%
\pgfpathlineto{\pgfqpoint{3.276634in}{0.794326in}}%
\pgfpathlineto{\pgfqpoint{3.212894in}{0.819220in}}%
\pgfpathlineto{\pgfqpoint{3.148999in}{0.843709in}}%
\pgfpathlineto{\pgfqpoint{3.084837in}{0.867490in}}%
\pgfpathlineto{\pgfqpoint{3.020310in}{0.890255in}}%
\pgfpathlineto{\pgfqpoint{2.955331in}{0.911689in}}%
\pgfpathlineto{\pgfqpoint{2.889836in}{0.931485in}}%
\pgfpathlineto{\pgfqpoint{2.823789in}{0.949350in}}%
\pgfpathlineto{\pgfqpoint{2.757190in}{0.965021in}}%
\pgfpathlineto{\pgfqpoint{2.690075in}{0.978292in}}%
\pgfpathlineto{\pgfqpoint{2.622508in}{0.989029in}}%
\pgfpathlineto{\pgfqpoint{2.554583in}{0.997193in}}%
\pgfpathlineto{\pgfqpoint{2.486401in}{1.002855in}}%
\pgfpathlineto{\pgfqpoint{2.418064in}{1.006208in}}%
\pgfpathlineto{\pgfqpoint{2.349655in}{1.007575in}}%
\pgfpathlineto{\pgfqpoint{2.281230in}{1.007387in}}%
\pgfpathlineto{\pgfqpoint{2.212813in}{1.006185in}}%
\pgfpathlineto{\pgfqpoint{2.144402in}{1.004620in}}%
\pgfpathlineto{\pgfqpoint{2.075984in}{1.003488in}}%
\pgfpathlineto{\pgfqpoint{2.007564in}{1.003779in}}%
\pgfpathlineto{\pgfqpoint{1.939226in}{1.006757in}}%
\pgfpathlineto{\pgfqpoint{1.871262in}{1.014143in}}%
\pgfpathlineto{\pgfqpoint{1.804503in}{1.028453in}}%
\pgfpathlineto{\pgfqpoint{1.741211in}{1.053443in}}%
\pgfpathlineto{\pgfqpoint{1.741211in}{1.053443in}}%
\pgfpathlineto{\pgfqpoint{1.697514in}{1.083032in}}%
\pgfpathlineto{\pgfqpoint{1.662762in}{1.123869in}}%
\pgfusepath{stroke}%
\end{pgfscope}%
\begin{pgfscope}%
\pgfpathrectangle{\pgfqpoint{0.647939in}{0.492442in}}{\pgfqpoint{3.079299in}{3.079299in}}%
\pgfusepath{clip}%
\pgfsetbuttcap%
\pgfsetroundjoin%
\pgfsetlinewidth{0.301125pt}%
\definecolor{currentstroke}{rgb}{0.500000,0.500000,0.500000}%
\pgfsetstrokecolor{currentstroke}%
\pgfsetstrokeopacity{0.300000}%
\pgfsetdash{}{0pt}%
\pgfpathmoveto{\pgfqpoint{3.727238in}{0.702394in}}%
\pgfpathlineto{\pgfqpoint{3.727238in}{0.702394in}}%
\pgfpathlineto{\pgfqpoint{3.661992in}{0.723002in}}%
\pgfpathlineto{\pgfqpoint{3.597241in}{0.745121in}}%
\pgfpathlineto{\pgfqpoint{3.532939in}{0.768517in}}%
\pgfpathlineto{\pgfqpoint{3.469020in}{0.792942in}}%
\pgfpathlineto{\pgfqpoint{3.405399in}{0.818135in}}%
\pgfpathlineto{\pgfqpoint{3.341976in}{0.843825in}}%
\pgfpathlineto{\pgfqpoint{3.278643in}{0.869735in}}%
\pgfpathlineto{\pgfqpoint{3.215281in}{0.895577in}}%
\pgfpathlineto{\pgfqpoint{3.151773in}{0.921054in}}%
\pgfpathlineto{\pgfqpoint{3.088001in}{0.945859in}}%
\pgfpathlineto{\pgfqpoint{3.023856in}{0.969678in}}%
\pgfpathlineto{\pgfqpoint{2.959241in}{0.992188in}}%
\pgfpathlineto{\pgfqpoint{2.894084in}{1.013067in}}%
\pgfpathlineto{\pgfqpoint{2.828338in}{1.032003in}}%
\pgfpathlineto{\pgfqpoint{2.761991in}{1.048709in}}%
\pgfpathlineto{\pgfqpoint{2.695074in}{1.062948in}}%
\pgfpathlineto{\pgfqpoint{2.627653in}{1.074556in}}%
\pgfpathlineto{\pgfqpoint{2.559823in}{1.083464in}}%
\pgfpathlineto{\pgfqpoint{2.491695in}{1.089722in}}%
\pgfpathlineto{\pgfqpoint{2.423383in}{1.093502in}}%
\pgfpathlineto{\pgfqpoint{2.354980in}{1.095120in}}%
\pgfpathlineto{\pgfqpoint{2.286556in}{1.095023in}}%
\pgfpathlineto{\pgfqpoint{2.218140in}{1.093778in}}%
\pgfpathlineto{\pgfqpoint{2.149732in}{1.092079in}}%
\pgfpathlineto{\pgfqpoint{2.081317in}{1.090769in}}%
\pgfpathlineto{\pgfqpoint{2.012897in}{1.090923in}}%
\pgfpathlineto{\pgfqpoint{1.944566in}{1.093970in}}%
\pgfpathlineto{\pgfqpoint{1.876686in}{1.101925in}}%
\pgfpathlineto{\pgfqpoint{1.810370in}{1.117863in}}%
\pgfpathlineto{\pgfqpoint{1.748904in}{1.146458in}}%
\pgfpathlineto{\pgfqpoint{1.748904in}{1.146458in}}%
\pgfpathlineto{\pgfqpoint{1.711926in}{1.176896in}}%
\pgfpathlineto{\pgfqpoint{1.684596in}{1.217059in}}%
\pgfpathlineto{\pgfqpoint{1.670495in}{1.258817in}}%
\pgfpathlineto{\pgfqpoint{1.665697in}{1.304202in}}%
\pgfpathlineto{\pgfqpoint{1.669333in}{1.355044in}}%
\pgfpathlineto{\pgfqpoint{1.682037in}{1.413436in}}%
\pgfusepath{stroke}%
\end{pgfscope}%
\begin{pgfscope}%
\pgfpathrectangle{\pgfqpoint{0.647939in}{0.492442in}}{\pgfqpoint{3.079299in}{3.079299in}}%
\pgfusepath{clip}%
\pgfsetbuttcap%
\pgfsetroundjoin%
\pgfsetlinewidth{0.301125pt}%
\definecolor{currentstroke}{rgb}{0.500000,0.500000,0.500000}%
\pgfsetstrokecolor{currentstroke}%
\pgfsetstrokeopacity{0.300000}%
\pgfsetdash{}{0pt}%
\pgfpathmoveto{\pgfqpoint{3.727238in}{0.772378in}}%
\pgfpathlineto{\pgfqpoint{3.727238in}{0.772378in}}%
\pgfpathlineto{\pgfqpoint{3.662210in}{0.793661in}}%
\pgfpathlineto{\pgfqpoint{3.597712in}{0.816508in}}%
\pgfpathlineto{\pgfqpoint{3.533700in}{0.840684in}}%
\pgfpathlineto{\pgfqpoint{3.470104in}{0.865937in}}%
\pgfpathlineto{\pgfqpoint{3.406837in}{0.892007in}}%
\pgfpathlineto{\pgfqpoint{3.343797in}{0.918622in}}%
\pgfpathlineto{\pgfqpoint{3.280870in}{0.945505in}}%
\pgfpathlineto{\pgfqpoint{3.217935in}{0.972368in}}%
\pgfpathlineto{\pgfqpoint{3.154867in}{0.998915in}}%
\pgfpathlineto{\pgfqpoint{3.091540in}{1.024836in}}%
\pgfpathlineto{\pgfqpoint{3.027836in}{1.049812in}}%
\pgfpathlineto{\pgfqpoint{2.963649in}{1.073513in}}%
\pgfpathlineto{\pgfqpoint{2.898893in}{1.095603in}}%
\pgfpathlineto{\pgfqpoint{2.833508in}{1.115750in}}%
\pgfpathlineto{\pgfqpoint{2.767474in}{1.133641in}}%
\pgfpathlineto{\pgfqpoint{2.700806in}{1.149004in}}%
\pgfpathlineto{\pgfqpoint{2.633569in}{1.161636in}}%
\pgfpathlineto{\pgfqpoint{2.565863in}{1.171433in}}%
\pgfpathlineto{\pgfqpoint{2.497808in}{1.178413in}}%
\pgfpathlineto{\pgfqpoint{2.429530in}{1.182728in}}%
\pgfpathlineto{\pgfqpoint{2.361138in}{1.184677in}}%
\pgfpathlineto{\pgfqpoint{2.292715in}{1.184709in}}%
\pgfpathlineto{\pgfqpoint{2.224300in}{1.183423in}}%
\pgfpathlineto{\pgfqpoint{2.155897in}{1.181560in}}%
\pgfpathlineto{\pgfqpoint{2.087486in}{1.180032in}}%
\pgfpathlineto{\pgfqpoint{2.019067in}{1.180010in}}%
\pgfpathlineto{\pgfqpoint{1.950742in}{1.183110in}}%
\pgfpathlineto{\pgfqpoint{1.882969in}{1.191751in}}%
\pgfpathlineto{\pgfqpoint{1.817319in}{1.209818in}}%
\pgfpathlineto{\pgfqpoint{1.817319in}{1.209818in}}%
\pgfpathlineto{\pgfqpoint{1.770197in}{1.234065in}}%
\pgfpathlineto{\pgfqpoint{1.770197in}{1.234065in}}%
\pgfpathlineto{\pgfqpoint{1.737009in}{1.263808in}}%
\pgfusepath{stroke}%
\end{pgfscope}%
\begin{pgfscope}%
\pgfpathrectangle{\pgfqpoint{0.647939in}{0.492442in}}{\pgfqpoint{3.079299in}{3.079299in}}%
\pgfusepath{clip}%
\pgfsetbuttcap%
\pgfsetroundjoin%
\pgfsetlinewidth{0.301125pt}%
\definecolor{currentstroke}{rgb}{0.500000,0.500000,0.500000}%
\pgfsetstrokecolor{currentstroke}%
\pgfsetstrokeopacity{0.300000}%
\pgfsetdash{}{0pt}%
\pgfpathmoveto{\pgfqpoint{3.727238in}{0.842362in}}%
\pgfpathlineto{\pgfqpoint{3.727238in}{0.842362in}}%
\pgfpathlineto{\pgfqpoint{3.662450in}{0.864364in}}%
\pgfpathlineto{\pgfqpoint{3.598233in}{0.887986in}}%
\pgfpathlineto{\pgfqpoint{3.534540in}{0.912991in}}%
\pgfpathlineto{\pgfqpoint{3.471302in}{0.939128in}}%
\pgfpathlineto{\pgfqpoint{3.408430in}{0.966134in}}%
\pgfpathlineto{\pgfqpoint{3.345817in}{0.993740in}}%
\pgfpathlineto{\pgfqpoint{3.283348in}{1.021669in}}%
\pgfpathlineto{\pgfqpoint{3.220895in}{1.049636in}}%
\pgfpathlineto{\pgfqpoint{3.158328in}{1.077344in}}%
\pgfpathlineto{\pgfqpoint{3.095514in}{1.104486in}}%
\pgfpathlineto{\pgfqpoint{3.032325in}{1.130737in}}%
\pgfpathlineto{\pgfqpoint{2.968643in}{1.155763in}}%
\pgfpathlineto{\pgfqpoint{2.904367in}{1.179214in}}%
\pgfpathlineto{\pgfqpoint{2.839425in}{1.200740in}}%
\pgfpathlineto{\pgfqpoint{2.773777in}{1.219999in}}%
\pgfpathlineto{\pgfqpoint{2.707431in}{1.236683in}}%
\pgfpathlineto{\pgfqpoint{2.640438in}{1.250540in}}%
\pgfpathlineto{\pgfqpoint{2.572898in}{1.261416in}}%
\pgfpathlineto{\pgfqpoint{2.504942in}{1.269287in}}%
\pgfpathlineto{\pgfqpoint{2.436712in}{1.274277in}}%
\pgfpathlineto{\pgfqpoint{2.368337in}{1.276667in}}%
\pgfpathlineto{\pgfqpoint{2.299916in}{1.276898in}}%
\pgfpathlineto{\pgfqpoint{2.231503in}{1.275583in}}%
\pgfpathlineto{\pgfqpoint{2.163106in}{1.273518in}}%
\pgfpathlineto{\pgfqpoint{2.094702in}{1.271702in}}%
\pgfpathlineto{\pgfqpoint{2.026286in}{1.271439in}}%
\pgfpathlineto{\pgfqpoint{1.957971in}{1.274584in}}%
\pgfpathlineto{\pgfqpoint{1.890342in}{1.284092in}}%
\pgfpathlineto{\pgfqpoint{1.825756in}{1.305107in}}%
\pgfpathlineto{\pgfqpoint{1.825756in}{1.305107in}}%
\pgfpathlineto{\pgfqpoint{1.786826in}{1.329540in}}%
\pgfpathlineto{\pgfqpoint{1.786826in}{1.329540in}}%
\pgfpathlineto{\pgfqpoint{1.760162in}{1.359592in}}%
\pgfpathlineto{\pgfqpoint{1.743696in}{1.396785in}}%
\pgfpathlineto{\pgfqpoint{1.737858in}{1.435744in}}%
\pgfpathlineto{\pgfqpoint{1.740477in}{1.479233in}}%
\pgfpathlineto{\pgfqpoint{1.751840in}{1.528480in}}%
\pgfusepath{stroke}%
\end{pgfscope}%
\begin{pgfscope}%
\pgfpathrectangle{\pgfqpoint{0.647939in}{0.492442in}}{\pgfqpoint{3.079299in}{3.079299in}}%
\pgfusepath{clip}%
\pgfsetbuttcap%
\pgfsetroundjoin%
\pgfsetlinewidth{0.301125pt}%
\definecolor{currentstroke}{rgb}{0.500000,0.500000,0.500000}%
\pgfsetstrokecolor{currentstroke}%
\pgfsetstrokeopacity{0.300000}%
\pgfsetdash{}{0pt}%
\pgfpathmoveto{\pgfqpoint{3.727238in}{0.912347in}}%
\pgfpathlineto{\pgfqpoint{3.727238in}{0.912347in}}%
\pgfpathlineto{\pgfqpoint{3.662715in}{0.935113in}}%
\pgfpathlineto{\pgfqpoint{3.598809in}{0.959562in}}%
\pgfpathlineto{\pgfqpoint{3.535471in}{0.985453in}}%
\pgfpathlineto{\pgfqpoint{3.472632in}{1.012533in}}%
\pgfpathlineto{\pgfqpoint{3.410200in}{1.040542in}}%
\pgfpathlineto{\pgfqpoint{3.348067in}{1.069211in}}%
\pgfpathlineto{\pgfqpoint{3.286115in}{1.098267in}}%
\pgfpathlineto{\pgfqpoint{3.224211in}{1.127428in}}%
\pgfpathlineto{\pgfqpoint{3.162220in}{1.156401in}}%
\pgfpathlineto{\pgfqpoint{3.100001in}{1.184881in}}%
\pgfpathlineto{\pgfqpoint{3.037416in}{1.212543in}}%
\pgfpathlineto{\pgfqpoint{2.974335in}{1.239050in}}%
\pgfpathlineto{\pgfqpoint{2.910641in}{1.264041in}}%
\pgfpathlineto{\pgfqpoint{2.846244in}{1.287149in}}%
\pgfpathlineto{\pgfqpoint{2.781086in}{1.308002in}}%
\pgfpathlineto{\pgfqpoint{2.715155in}{1.326251in}}%
\pgfpathlineto{\pgfqpoint{2.648491in}{1.341594in}}%
\pgfpathlineto{\pgfqpoint{2.581184in}{1.353813in}}%
\pgfpathlineto{\pgfqpoint{2.513371in}{1.362815in}}%
\pgfpathlineto{\pgfqpoint{2.445212in}{1.368675in}}%
\pgfpathlineto{\pgfqpoint{2.376863in}{1.371652in}}%
\pgfpathlineto{\pgfqpoint{2.308447in}{1.372180in}}%
\pgfpathlineto{\pgfqpoint{2.240035in}{1.370881in}}%
\pgfpathlineto{\pgfqpoint{2.171646in}{1.368579in}}%
\pgfpathlineto{\pgfqpoint{2.103254in}{1.366370in}}%
\pgfpathlineto{\pgfqpoint{2.034841in}{1.365743in}}%
\pgfpathlineto{\pgfqpoint{1.966544in}{1.368892in}}%
\pgfpathlineto{\pgfqpoint{1.899175in}{1.379588in}}%
\pgfpathlineto{\pgfqpoint{1.899175in}{1.379588in}}%
\pgfpathlineto{\pgfqpoint{1.848247in}{1.397909in}}%
\pgfpathlineto{\pgfqpoint{1.848247in}{1.397909in}}%
\pgfpathlineto{\pgfqpoint{1.814119in}{1.421500in}}%
\pgfusepath{stroke}%
\end{pgfscope}%
\begin{pgfscope}%
\pgfpathrectangle{\pgfqpoint{0.647939in}{0.492442in}}{\pgfqpoint{3.079299in}{3.079299in}}%
\pgfusepath{clip}%
\pgfsetbuttcap%
\pgfsetroundjoin%
\pgfsetlinewidth{0.301125pt}%
\definecolor{currentstroke}{rgb}{0.500000,0.500000,0.500000}%
\pgfsetstrokecolor{currentstroke}%
\pgfsetstrokeopacity{0.300000}%
\pgfsetdash{}{0pt}%
\pgfpathmoveto{\pgfqpoint{3.727238in}{0.982331in}}%
\pgfpathlineto{\pgfqpoint{3.727238in}{0.982331in}}%
\pgfpathlineto{\pgfqpoint{3.663010in}{1.005915in}}%
\pgfpathlineto{\pgfqpoint{3.599449in}{1.031246in}}%
\pgfpathlineto{\pgfqpoint{3.536507in}{1.058083in}}%
\pgfpathlineto{\pgfqpoint{3.474113in}{1.086172in}}%
\pgfpathlineto{\pgfqpoint{3.412174in}{1.115256in}}%
\pgfpathlineto{\pgfqpoint{3.350582in}{1.145069in}}%
\pgfpathlineto{\pgfqpoint{3.289215in}{1.175343in}}%
\pgfpathlineto{\pgfqpoint{3.227938in}{1.205800in}}%
\pgfpathlineto{\pgfqpoint{3.166610in}{1.236154in}}%
\pgfpathlineto{\pgfqpoint{3.105086in}{1.266106in}}%
\pgfpathlineto{\pgfqpoint{3.043216in}{1.295335in}}%
\pgfpathlineto{\pgfqpoint{2.980858in}{1.323504in}}%
\pgfpathlineto{\pgfqpoint{2.917878in}{1.350248in}}%
\pgfpathlineto{\pgfqpoint{2.854164in}{1.375183in}}%
\pgfpathlineto{\pgfqpoint{2.789635in}{1.397912in}}%
\pgfpathlineto{\pgfqpoint{2.724253in}{1.418041in}}%
\pgfpathlineto{\pgfqpoint{2.658036in}{1.435209in}}%
\pgfpathlineto{\pgfqpoint{2.591064in}{1.449124in}}%
\pgfpathlineto{\pgfqpoint{2.523471in}{1.459610in}}%
\pgfpathlineto{\pgfqpoint{2.455430in}{1.466655in}}%
\pgfpathlineto{\pgfqpoint{2.387125in}{1.470447in}}%
\pgfpathlineto{\pgfqpoint{2.318716in}{1.471407in}}%
\pgfpathlineto{\pgfqpoint{2.250305in}{1.470176in}}%
\pgfpathlineto{\pgfqpoint{2.181926in}{1.467623in}}%
\pgfpathlineto{\pgfqpoint{2.113551in}{1.464917in}}%
\pgfpathlineto{\pgfqpoint{2.045145in}{1.463730in}}%
\pgfpathlineto{\pgfqpoint{1.976861in}{1.466764in}}%
\pgfpathlineto{\pgfqpoint{1.909970in}{1.479181in}}%
\pgfpathlineto{\pgfqpoint{1.909970in}{1.479181in}}%
\pgfpathlineto{\pgfqpoint{1.869504in}{1.496692in}}%
\pgfpathlineto{\pgfqpoint{1.869504in}{1.496692in}}%
\pgfpathlineto{\pgfqpoint{1.842535in}{1.519753in}}%
\pgfpathlineto{\pgfqpoint{1.825834in}{1.551628in}}%
\pgfpathlineto{\pgfqpoint{1.821081in}{1.583874in}}%
\pgfpathlineto{\pgfqpoint{1.824660in}{1.618769in}}%
\pgfpathlineto{\pgfqpoint{1.836832in}{1.659074in}}%
\pgfpathlineto{\pgfqpoint{1.858837in}{1.704846in}}%
\pgfusepath{stroke}%
\end{pgfscope}%
\begin{pgfscope}%
\pgfpathrectangle{\pgfqpoint{0.647939in}{0.492442in}}{\pgfqpoint{3.079299in}{3.079299in}}%
\pgfusepath{clip}%
\pgfsetbuttcap%
\pgfsetroundjoin%
\pgfsetlinewidth{0.301125pt}%
\definecolor{currentstroke}{rgb}{0.500000,0.500000,0.500000}%
\pgfsetstrokecolor{currentstroke}%
\pgfsetstrokeopacity{0.300000}%
\pgfsetdash{}{0pt}%
\pgfpathmoveto{\pgfqpoint{3.727238in}{1.052315in}}%
\pgfpathlineto{\pgfqpoint{3.727238in}{1.052315in}}%
\pgfpathlineto{\pgfqpoint{3.663338in}{1.076773in}}%
\pgfpathlineto{\pgfqpoint{3.600162in}{1.103048in}}%
\pgfpathlineto{\pgfqpoint{3.537662in}{1.130897in}}%
\pgfpathlineto{\pgfqpoint{3.475767in}{1.160069in}}%
\pgfpathlineto{\pgfqpoint{3.414385in}{1.190307in}}%
\pgfpathlineto{\pgfqpoint{3.353405in}{1.221352in}}%
\pgfpathlineto{\pgfqpoint{3.292705in}{1.252941in}}%
\pgfpathlineto{\pgfqpoint{3.232148in}{1.284803in}}%
\pgfpathlineto{\pgfqpoint{3.171589in}{1.316661in}}%
\pgfpathlineto{\pgfqpoint{3.110876in}{1.348226in}}%
\pgfpathlineto{\pgfqpoint{3.049857in}{1.379189in}}%
\pgfpathlineto{\pgfqpoint{2.988375in}{1.409219in}}%
\pgfpathlineto{\pgfqpoint{2.926281in}{1.437957in}}%
\pgfpathlineto{\pgfqpoint{2.863439in}{1.465012in}}%
\pgfpathlineto{\pgfqpoint{2.799738in}{1.489966in}}%
\pgfpathlineto{\pgfqpoint{2.735104in}{1.512384in}}%
\pgfpathlineto{\pgfqpoint{2.669519in}{1.531833in}}%
\pgfpathlineto{\pgfqpoint{2.603037in}{1.547930in}}%
\pgfpathlineto{\pgfqpoint{2.535783in}{1.560392in}}%
\pgfpathlineto{\pgfqpoint{2.467941in}{1.569091in}}%
\pgfpathlineto{\pgfqpoint{2.399724in}{1.574110in}}%
\pgfpathlineto{\pgfqpoint{2.331336in}{1.575779in}}%
\pgfpathlineto{\pgfqpoint{2.262927in}{1.574708in}}%
\pgfpathlineto{\pgfqpoint{2.194562in}{1.571825in}}%
\pgfpathlineto{\pgfqpoint{2.126218in}{1.568423in}}%
\pgfpathlineto{\pgfqpoint{2.057834in}{1.566413in}}%
\pgfpathlineto{\pgfqpoint{1.989561in}{1.569168in}}%
\pgfpathlineto{\pgfqpoint{1.989561in}{1.569168in}}%
\pgfpathlineto{\pgfqpoint{1.933122in}{1.580495in}}%
\pgfpathlineto{\pgfqpoint{1.933122in}{1.580495in}}%
\pgfpathlineto{\pgfqpoint{1.900169in}{1.596631in}}%
\pgfpathlineto{\pgfqpoint{1.900169in}{1.596631in}}%
\pgfpathlineto{\pgfqpoint{1.879730in}{1.618036in}}%
\pgfusepath{stroke}%
\end{pgfscope}%
\begin{pgfscope}%
\pgfpathrectangle{\pgfqpoint{0.647939in}{0.492442in}}{\pgfqpoint{3.079299in}{3.079299in}}%
\pgfusepath{clip}%
\pgfsetbuttcap%
\pgfsetroundjoin%
\pgfsetlinewidth{0.301125pt}%
\definecolor{currentstroke}{rgb}{0.500000,0.500000,0.500000}%
\pgfsetstrokecolor{currentstroke}%
\pgfsetstrokeopacity{0.300000}%
\pgfsetdash{}{0pt}%
\pgfpathmoveto{\pgfqpoint{3.727238in}{1.122299in}}%
\pgfpathlineto{\pgfqpoint{3.727238in}{1.122299in}}%
\pgfpathlineto{\pgfqpoint{3.663705in}{1.147693in}}%
\pgfpathlineto{\pgfqpoint{3.600960in}{1.174979in}}%
\pgfpathlineto{\pgfqpoint{3.538955in}{1.203914in}}%
\pgfpathlineto{\pgfqpoint{3.477622in}{1.234248in}}%
\pgfpathlineto{\pgfqpoint{3.416867in}{1.265728in}}%
\pgfpathlineto{\pgfqpoint{3.356581in}{1.298099in}}%
\pgfpathlineto{\pgfqpoint{3.296641in}{1.331107in}}%
\pgfpathlineto{\pgfqpoint{3.236910in}{1.364495in}}%
\pgfpathlineto{\pgfqpoint{3.177245in}{1.397999in}}%
\pgfpathlineto{\pgfqpoint{3.117494in}{1.431347in}}%
\pgfpathlineto{\pgfqpoint{3.057496in}{1.464247in}}%
\pgfpathlineto{\pgfqpoint{2.997089in}{1.496385in}}%
\pgfpathlineto{\pgfqpoint{2.936108in}{1.527415in}}%
\pgfpathlineto{\pgfqpoint{2.874394in}{1.556953in}}%
\pgfpathlineto{\pgfqpoint{2.811804in}{1.584575in}}%
\pgfpathlineto{\pgfqpoint{2.748222in}{1.609818in}}%
\pgfpathlineto{\pgfqpoint{2.683582in}{1.632190in}}%
\pgfpathlineto{\pgfqpoint{2.617881in}{1.651205in}}%
\pgfpathlineto{\pgfqpoint{2.551203in}{1.666423in}}%
\pgfpathlineto{\pgfqpoint{2.483718in}{1.677528in}}%
\pgfpathlineto{\pgfqpoint{2.415671in}{1.684426in}}%
\pgfpathlineto{\pgfqpoint{2.347334in}{1.687312in}}%
\pgfpathlineto{\pgfqpoint{2.278930in}{1.686702in}}%
\pgfpathlineto{\pgfqpoint{2.210587in}{1.683476in}}%
\pgfpathlineto{\pgfqpoint{2.142306in}{1.678986in}}%
\pgfpathlineto{\pgfqpoint{2.073981in}{1.675465in}}%
\pgfpathlineto{\pgfqpoint{2.005768in}{1.677493in}}%
\pgfpathlineto{\pgfqpoint{2.005768in}{1.677493in}}%
\pgfpathlineto{\pgfqpoint{1.963817in}{1.686382in}}%
\pgfpathlineto{\pgfqpoint{1.963817in}{1.686382in}}%
\pgfpathlineto{\pgfqpoint{1.939080in}{1.700182in}}%
\pgfpathlineto{\pgfqpoint{1.939080in}{1.700182in}}%
\pgfusepath{stroke}%
\end{pgfscope}%
\begin{pgfscope}%
\pgfpathrectangle{\pgfqpoint{0.647939in}{0.492442in}}{\pgfqpoint{3.079299in}{3.079299in}}%
\pgfusepath{clip}%
\pgfsetbuttcap%
\pgfsetroundjoin%
\pgfsetlinewidth{0.301125pt}%
\definecolor{currentstroke}{rgb}{0.500000,0.500000,0.500000}%
\pgfsetstrokecolor{currentstroke}%
\pgfsetstrokeopacity{0.300000}%
\pgfsetdash{}{0pt}%
\pgfpathmoveto{\pgfqpoint{3.727238in}{1.192283in}}%
\pgfpathlineto{\pgfqpoint{3.727238in}{1.192283in}}%
\pgfpathlineto{\pgfqpoint{3.664117in}{1.218682in}}%
\pgfpathlineto{\pgfqpoint{3.601856in}{1.247053in}}%
\pgfpathlineto{\pgfqpoint{3.540410in}{1.277153in}}%
\pgfpathlineto{\pgfqpoint{3.479710in}{1.308734in}}%
\pgfpathlineto{\pgfqpoint{3.419666in}{1.341548in}}%
\pgfpathlineto{\pgfqpoint{3.360171in}{1.375349in}}%
\pgfpathlineto{\pgfqpoint{3.301105in}{1.409896in}}%
\pgfpathlineto{\pgfqpoint{3.242337in}{1.444947in}}%
\pgfpathlineto{\pgfqpoint{3.183725in}{1.480260in}}%
\pgfpathlineto{\pgfqpoint{3.125120in}{1.515584in}}%
\pgfpathlineto{\pgfqpoint{3.066364in}{1.550655in}}%
\pgfpathlineto{\pgfqpoint{3.007290in}{1.585186in}}%
\pgfpathlineto{\pgfqpoint{2.947726in}{1.618862in}}%
\pgfpathlineto{\pgfqpoint{2.887495in}{1.651327in}}%
\pgfpathlineto{\pgfqpoint{2.826425in}{1.682176in}}%
\pgfpathlineto{\pgfqpoint{2.764355in}{1.710948in}}%
\pgfpathlineto{\pgfqpoint{2.701149in}{1.737121in}}%
\pgfpathlineto{\pgfqpoint{2.636727in}{1.760118in}}%
\pgfpathlineto{\pgfqpoint{2.571092in}{1.779343in}}%
\pgfpathlineto{\pgfqpoint{2.504352in}{1.794238in}}%
\pgfpathlineto{\pgfqpoint{2.436732in}{1.804375in}}%
\pgfpathlineto{\pgfqpoint{2.368555in}{1.809580in}}%
\pgfpathlineto{\pgfqpoint{2.300171in}{1.810048in}}%
\pgfpathlineto{\pgfqpoint{2.231861in}{1.806469in}}%
\pgfpathlineto{\pgfqpoint{2.163729in}{1.800185in}}%
\pgfpathlineto{\pgfqpoint{2.095621in}{1.793637in}}%
\pgfpathlineto{\pgfqpoint{2.027540in}{1.793432in}}%
\pgfpathlineto{\pgfqpoint{2.027540in}{1.793432in}}%
\pgfpathlineto{\pgfqpoint{2.003355in}{1.798560in}}%
\pgfpathlineto{\pgfqpoint{2.003355in}{1.798560in}}%
\pgfpathlineto{\pgfqpoint{1.988117in}{1.809622in}}%
\pgfpathlineto{\pgfqpoint{1.988117in}{1.809622in}}%
\pgfpathlineto{\pgfqpoint{1.984445in}{1.823679in}}%
\pgfpathlineto{\pgfqpoint{1.988228in}{1.839101in}}%
\pgfpathlineto{\pgfqpoint{1.997020in}{1.854560in}}%
\pgfusepath{stroke}%
\end{pgfscope}%
\begin{pgfscope}%
\pgfpathrectangle{\pgfqpoint{0.647939in}{0.492442in}}{\pgfqpoint{3.079299in}{3.079299in}}%
\pgfusepath{clip}%
\pgfsetbuttcap%
\pgfsetroundjoin%
\pgfsetlinewidth{0.301125pt}%
\definecolor{currentstroke}{rgb}{0.500000,0.500000,0.500000}%
\pgfsetstrokecolor{currentstroke}%
\pgfsetstrokeopacity{0.300000}%
\pgfsetdash{}{0pt}%
\pgfpathmoveto{\pgfqpoint{3.727238in}{1.262267in}}%
\pgfpathlineto{\pgfqpoint{3.727238in}{1.262267in}}%
\pgfpathlineto{\pgfqpoint{3.664581in}{1.289746in}}%
\pgfpathlineto{\pgfqpoint{3.602865in}{1.319283in}}%
\pgfpathlineto{\pgfqpoint{3.542049in}{1.350635in}}%
\pgfpathlineto{\pgfqpoint{3.482067in}{1.383557in}}%
\pgfpathlineto{\pgfqpoint{3.422833in}{1.417808in}}%
\pgfpathlineto{\pgfqpoint{3.364244in}{1.453157in}}%
\pgfpathlineto{\pgfqpoint{3.306189in}{1.489377in}}%
\pgfpathlineto{\pgfqpoint{3.248545in}{1.526247in}}%
\pgfpathlineto{\pgfqpoint{3.191180in}{1.563550in}}%
\pgfpathlineto{\pgfqpoint{3.133955in}{1.601068in}}%
\pgfpathlineto{\pgfqpoint{3.076719in}{1.638570in}}%
\pgfpathlineto{\pgfqpoint{3.019315in}{1.675811in}}%
\pgfpathlineto{\pgfqpoint{2.961577in}{1.712530in}}%
\pgfpathlineto{\pgfqpoint{2.903332in}{1.748434in}}%
\pgfpathlineto{\pgfqpoint{2.844399in}{1.783188in}}%
\pgfpathlineto{\pgfqpoint{2.784588in}{1.816400in}}%
\pgfpathlineto{\pgfqpoint{2.723709in}{1.847600in}}%
\pgfpathlineto{\pgfqpoint{2.661586in}{1.876226in}}%
\pgfpathlineto{\pgfqpoint{2.598077in}{1.901599in}}%
\pgfpathlineto{\pgfqpoint{2.533111in}{1.922925in}}%
\pgfpathlineto{\pgfqpoint{2.466743in}{1.939303in}}%
\pgfpathlineto{\pgfqpoint{2.399214in}{1.949794in}}%
\pgfpathlineto{\pgfqpoint{2.330992in}{1.953566in}}%
\pgfpathlineto{\pgfqpoint{2.262755in}{1.950144in}}%
\pgfpathlineto{\pgfqpoint{2.195247in}{1.939693in}}%
\pgfpathlineto{\pgfqpoint{2.129004in}{1.923166in}}%
\pgfpathlineto{\pgfqpoint{2.129004in}{1.923166in}}%
\pgfusepath{stroke}%
\end{pgfscope}%
\begin{pgfscope}%
\pgfpathrectangle{\pgfqpoint{0.647939in}{0.492442in}}{\pgfqpoint{3.079299in}{3.079299in}}%
\pgfusepath{clip}%
\pgfsetbuttcap%
\pgfsetroundjoin%
\pgfsetlinewidth{0.301125pt}%
\definecolor{currentstroke}{rgb}{0.500000,0.500000,0.500000}%
\pgfsetstrokecolor{currentstroke}%
\pgfsetstrokeopacity{0.300000}%
\pgfsetdash{}{0pt}%
\pgfpathmoveto{\pgfqpoint{3.727238in}{1.332251in}}%
\pgfpathlineto{\pgfqpoint{3.727238in}{1.332251in}}%
\pgfpathlineto{\pgfqpoint{3.665105in}{1.360894in}}%
\pgfpathlineto{\pgfqpoint{3.604006in}{1.391685in}}%
\pgfpathlineto{\pgfqpoint{3.543904in}{1.424384in}}%
\pgfpathlineto{\pgfqpoint{3.484739in}{1.458751in}}%
\pgfpathlineto{\pgfqpoint{3.426431in}{1.494556in}}%
\pgfpathlineto{\pgfqpoint{3.368889in}{1.531581in}}%
\pgfpathlineto{\pgfqpoint{3.312011in}{1.569620in}}%
\pgfpathlineto{\pgfqpoint{3.255690in}{1.608479in}}%
\pgfpathlineto{\pgfqpoint{3.199810in}{1.647972in}}%
\pgfpathlineto{\pgfqpoint{3.144249in}{1.687912in}}%
\pgfpathlineto{\pgfqpoint{3.088882in}{1.728119in}}%
\pgfpathlineto{\pgfqpoint{3.033583in}{1.768420in}}%
\pgfpathlineto{\pgfqpoint{2.978223in}{1.808637in}}%
\pgfpathlineto{\pgfqpoint{2.922666in}{1.848577in}}%
\pgfpathlineto{\pgfqpoint{2.866769in}{1.888039in}}%
\pgfpathlineto{\pgfqpoint{2.810389in}{1.926802in}}%
\pgfpathlineto{\pgfqpoint{2.753376in}{1.964619in}}%
\pgfpathlineto{\pgfqpoint{2.695573in}{2.001209in}}%
\pgfpathlineto{\pgfqpoint{2.636815in}{2.036240in}}%
\pgfpathlineto{\pgfqpoint{2.576939in}{2.069318in}}%
\pgfpathlineto{\pgfqpoint{2.515791in}{2.099956in}}%
\pgfpathlineto{\pgfqpoint{2.453208in}{2.127485in}}%
\pgfpathlineto{\pgfqpoint{2.388941in}{2.150620in}}%
\pgfpathlineto{\pgfqpoint{2.388941in}{2.150620in}}%
\pgfpathlineto{\pgfqpoint{2.345346in}{2.161232in}}%
\pgfpathlineto{\pgfqpoint{2.345346in}{2.161232in}}%
\pgfpathlineto{\pgfqpoint{2.325295in}{2.162181in}}%
\pgfpathlineto{\pgfqpoint{2.325295in}{2.162181in}}%
\pgfpathlineto{\pgfqpoint{2.312860in}{2.158433in}}%
\pgfpathlineto{\pgfqpoint{2.302394in}{2.151721in}}%
\pgfpathlineto{\pgfqpoint{2.287825in}{2.137404in}}%
\pgfpathlineto{\pgfqpoint{2.266156in}{2.114285in}}%
\pgfpathlineto{\pgfqpoint{2.231244in}{2.077012in}}%
\pgfpathlineto{\pgfqpoint{2.186861in}{2.031622in}}%
\pgfpathlineto{\pgfqpoint{2.141286in}{1.987139in}}%
\pgfpathlineto{\pgfqpoint{2.096564in}{1.947236in}}%
\pgfusepath{stroke}%
\end{pgfscope}%
\begin{pgfscope}%
\pgfpathrectangle{\pgfqpoint{0.647939in}{0.492442in}}{\pgfqpoint{3.079299in}{3.079299in}}%
\pgfusepath{clip}%
\pgfsetbuttcap%
\pgfsetroundjoin%
\pgfsetlinewidth{0.301125pt}%
\definecolor{currentstroke}{rgb}{0.500000,0.500000,0.500000}%
\pgfsetstrokecolor{currentstroke}%
\pgfsetstrokeopacity{0.300000}%
\pgfsetdash{}{0pt}%
\pgfpathmoveto{\pgfqpoint{3.727238in}{1.472219in}}%
\pgfpathlineto{\pgfqpoint{3.727238in}{1.472219in}}%
\pgfpathlineto{\pgfqpoint{3.666380in}{1.503471in}}%
\pgfpathlineto{\pgfqpoint{3.606784in}{1.537073in}}%
\pgfpathlineto{\pgfqpoint{3.548430in}{1.572789in}}%
\pgfpathlineto{\pgfqpoint{3.491275in}{1.610399in}}%
\pgfpathlineto{\pgfqpoint{3.435267in}{1.649701in}}%
\pgfpathlineto{\pgfqpoint{3.380351in}{1.690514in}}%
\pgfpathlineto{\pgfqpoint{3.326467in}{1.732682in}}%
\pgfpathlineto{\pgfqpoint{3.273556in}{1.776067in}}%
\pgfpathlineto{\pgfqpoint{3.221576in}{1.820561in}}%
\pgfpathlineto{\pgfqpoint{3.170504in}{1.866095in}}%
\pgfpathlineto{\pgfqpoint{3.120335in}{1.912622in}}%
\pgfpathlineto{\pgfqpoint{3.071107in}{1.960141in}}%
\pgfpathlineto{\pgfqpoint{3.022912in}{2.008705in}}%
\pgfpathlineto{\pgfqpoint{2.975924in}{2.058433in}}%
\pgfpathlineto{\pgfqpoint{2.930449in}{2.109546in}}%
\pgfpathlineto{\pgfqpoint{2.887015in}{2.162392in}}%
\pgfpathlineto{\pgfqpoint{2.846526in}{2.217494in}}%
\pgfpathlineto{\pgfqpoint{2.810514in}{2.275567in}}%
\pgfpathlineto{\pgfqpoint{2.781518in}{2.337340in}}%
\pgfpathlineto{\pgfqpoint{2.763113in}{2.402873in}}%
\pgfpathlineto{\pgfqpoint{2.758377in}{2.470634in}}%
\pgfpathlineto{\pgfqpoint{2.766364in}{2.534148in}}%
\pgfpathlineto{\pgfqpoint{2.784975in}{2.598661in}}%
\pgfpathlineto{\pgfqpoint{2.811206in}{2.661691in}}%
\pgfpathlineto{\pgfqpoint{2.842262in}{2.722556in}}%
\pgfpathlineto{\pgfqpoint{2.876527in}{2.781709in}}%
\pgfpathlineto{\pgfqpoint{2.913030in}{2.839533in}}%
\pgfpathlineto{\pgfqpoint{2.951189in}{2.896302in}}%
\pgfpathlineto{\pgfqpoint{2.990653in}{2.952182in}}%
\pgfpathlineto{\pgfqpoint{3.031209in}{3.007277in}}%
\pgfpathlineto{\pgfqpoint{3.072729in}{3.061652in}}%
\pgfpathlineto{\pgfqpoint{3.115152in}{3.115328in}}%
\pgfpathlineto{\pgfqpoint{3.158461in}{3.168295in}}%
\pgfpathlineto{\pgfqpoint{3.202656in}{3.220528in}}%
\pgfpathlineto{\pgfqpoint{3.247767in}{3.271970in}}%
\pgfpathlineto{\pgfqpoint{3.293839in}{3.322556in}}%
\pgfpathlineto{\pgfqpoint{3.340932in}{3.372193in}}%
\pgfpathlineto{\pgfqpoint{3.389116in}{3.420770in}}%
\pgfpathlineto{\pgfqpoint{3.438471in}{3.468157in}}%
\pgfpathlineto{\pgfqpoint{3.489083in}{3.514196in}}%
\pgfpathlineto{\pgfqpoint{3.541038in}{3.558713in}}%
\pgfpathlineto{\pgfqpoint{3.556724in}{3.571741in}}%
\pgfusepath{stroke}%
\end{pgfscope}%
\begin{pgfscope}%
\pgfpathrectangle{\pgfqpoint{0.647939in}{0.492442in}}{\pgfqpoint{3.079299in}{3.079299in}}%
\pgfusepath{clip}%
\pgfsetbuttcap%
\pgfsetroundjoin%
\pgfsetlinewidth{0.301125pt}%
\definecolor{currentstroke}{rgb}{0.500000,0.500000,0.500000}%
\pgfsetstrokecolor{currentstroke}%
\pgfsetstrokeopacity{0.300000}%
\pgfsetdash{}{0pt}%
\pgfpathmoveto{\pgfqpoint{3.727238in}{1.542203in}}%
\pgfpathlineto{\pgfqpoint{3.727238in}{1.542203in}}%
\pgfpathlineto{\pgfqpoint{3.667157in}{1.574921in}}%
\pgfpathlineto{\pgfqpoint{3.608480in}{1.610099in}}%
\pgfpathlineto{\pgfqpoint{3.551197in}{1.647509in}}%
\pgfpathlineto{\pgfqpoint{3.495285in}{1.686938in}}%
\pgfpathlineto{\pgfqpoint{3.440712in}{1.728204in}}%
\pgfpathlineto{\pgfqpoint{3.387449in}{1.771149in}}%
\pgfpathlineto{\pgfqpoint{3.335467in}{1.815637in}}%
\pgfpathlineto{\pgfqpoint{3.284757in}{1.861572in}}%
\pgfpathlineto{\pgfqpoint{3.235344in}{1.908897in}}%
\pgfpathlineto{\pgfqpoint{3.187280in}{1.957590in}}%
\pgfpathlineto{\pgfqpoint{3.140677in}{2.007682in}}%
\pgfpathlineto{\pgfqpoint{3.095725in}{2.059254in}}%
\pgfpathlineto{\pgfqpoint{3.052722in}{2.112459in}}%
\pgfpathlineto{\pgfqpoint{3.012140in}{2.167525in}}%
\pgfpathlineto{\pgfqpoint{2.974695in}{2.224751in}}%
\pgfpathlineto{\pgfqpoint{2.941452in}{2.284482in}}%
\pgfpathlineto{\pgfqpoint{2.913918in}{2.347003in}}%
\pgfpathlineto{\pgfqpoint{2.893993in}{2.412287in}}%
\pgfpathlineto{\pgfqpoint{2.883521in}{2.479684in}}%
\pgfpathlineto{\pgfqpoint{2.883415in}{2.547882in}}%
\pgfpathlineto{\pgfqpoint{2.893088in}{2.615426in}}%
\pgfpathlineto{\pgfqpoint{2.910872in}{2.681345in}}%
\pgfpathlineto{\pgfqpoint{2.934859in}{2.745318in}}%
\pgfpathlineto{\pgfqpoint{2.963451in}{2.807406in}}%
\pgfpathlineto{\pgfqpoint{2.995479in}{2.867818in}}%
\pgfpathlineto{\pgfqpoint{3.030138in}{2.926777in}}%
\pgfpathlineto{\pgfqpoint{3.066885in}{2.984473in}}%
\pgfpathlineto{\pgfqpoint{3.105358in}{3.041046in}}%
\pgfpathlineto{\pgfqpoint{3.145327in}{3.096574in}}%
\pgfusepath{stroke}%
\end{pgfscope}%
\begin{pgfscope}%
\pgfpathrectangle{\pgfqpoint{0.647939in}{0.492442in}}{\pgfqpoint{3.079299in}{3.079299in}}%
\pgfusepath{clip}%
\pgfsetbuttcap%
\pgfsetroundjoin%
\pgfsetlinewidth{0.301125pt}%
\definecolor{currentstroke}{rgb}{0.500000,0.500000,0.500000}%
\pgfsetstrokecolor{currentstroke}%
\pgfsetstrokeopacity{0.300000}%
\pgfsetdash{}{0pt}%
\pgfpathmoveto{\pgfqpoint{3.727238in}{1.612187in}}%
\pgfpathlineto{\pgfqpoint{3.727238in}{1.612187in}}%
\pgfpathlineto{\pgfqpoint{3.668051in}{1.646494in}}%
\pgfpathlineto{\pgfqpoint{3.610433in}{1.683376in}}%
\pgfpathlineto{\pgfqpoint{3.554391in}{1.722613in}}%
\pgfpathlineto{\pgfqpoint{3.499921in}{1.764008in}}%
\pgfpathlineto{\pgfqpoint{3.447022in}{1.807395in}}%
\pgfpathlineto{\pgfqpoint{3.395696in}{1.852633in}}%
\pgfpathlineto{\pgfqpoint{3.345963in}{1.899616in}}%
\pgfpathlineto{\pgfqpoint{3.297883in}{1.948290in}}%
\pgfpathlineto{\pgfqpoint{3.251558in}{1.998634in}}%
\pgfpathlineto{\pgfqpoint{3.207151in}{2.050676in}}%
\pgfpathlineto{\pgfqpoint{3.164922in}{2.104496in}}%
\pgfpathlineto{\pgfqpoint{3.125249in}{2.160218in}}%
\pgfpathlineto{\pgfqpoint{3.088672in}{2.218005in}}%
\pgfpathlineto{\pgfqpoint{3.055954in}{2.278043in}}%
\pgfpathlineto{\pgfqpoint{3.028104in}{2.340462in}}%
\pgfpathlineto{\pgfqpoint{3.006352in}{2.405226in}}%
\pgfpathlineto{\pgfqpoint{2.991970in}{2.471978in}}%
\pgfpathlineto{\pgfqpoint{2.985905in}{2.539976in}}%
\pgfpathlineto{\pgfqpoint{2.988392in}{2.608212in}}%
\pgfpathlineto{\pgfqpoint{2.998847in}{2.675707in}}%
\pgfpathlineto{\pgfqpoint{3.016124in}{2.741813in}}%
\pgfpathlineto{\pgfqpoint{3.038957in}{2.806237in}}%
\pgfpathlineto{\pgfqpoint{3.066211in}{2.868942in}}%
\pgfpathlineto{\pgfqpoint{3.096985in}{2.930012in}}%
\pgfusepath{stroke}%
\end{pgfscope}%
\begin{pgfscope}%
\pgfpathrectangle{\pgfqpoint{0.647939in}{0.492442in}}{\pgfqpoint{3.079299in}{3.079299in}}%
\pgfusepath{clip}%
\pgfsetbuttcap%
\pgfsetroundjoin%
\pgfsetlinewidth{0.301125pt}%
\definecolor{currentstroke}{rgb}{0.500000,0.500000,0.500000}%
\pgfsetstrokecolor{currentstroke}%
\pgfsetstrokeopacity{0.300000}%
\pgfsetdash{}{0pt}%
\pgfpathmoveto{\pgfqpoint{3.727238in}{1.752155in}}%
\pgfpathlineto{\pgfqpoint{3.727238in}{1.752155in}}%
\pgfpathlineto{\pgfqpoint{3.670290in}{1.790049in}}%
\pgfpathlineto{\pgfqpoint{3.615324in}{1.830769in}}%
\pgfpathlineto{\pgfqpoint{3.562398in}{1.874108in}}%
\pgfpathlineto{\pgfqpoint{3.511575in}{1.919897in}}%
\pgfpathlineto{\pgfqpoint{3.462927in}{1.967991in}}%
\pgfpathlineto{\pgfqpoint{3.416567in}{2.018295in}}%
\pgfpathlineto{\pgfqpoint{3.372657in}{2.070748in}}%
\pgfpathlineto{\pgfqpoint{3.331425in}{2.125325in}}%
\pgfpathlineto{\pgfqpoint{3.293196in}{2.182040in}}%
\pgfpathlineto{\pgfqpoint{3.258393in}{2.240913in}}%
\pgfpathlineto{\pgfqpoint{3.227565in}{2.301952in}}%
\pgfpathlineto{\pgfqpoint{3.201383in}{2.365110in}}%
\pgfpathlineto{\pgfqpoint{3.180596in}{2.430224in}}%
\pgfpathlineto{\pgfqpoint{3.165942in}{2.496969in}}%
\pgfpathlineto{\pgfqpoint{3.157997in}{2.564827in}}%
\pgfpathlineto{\pgfqpoint{3.157020in}{2.633142in}}%
\pgfpathlineto{\pgfqpoint{3.162871in}{2.701227in}}%
\pgfpathlineto{\pgfqpoint{3.175057in}{2.768482in}}%
\pgfpathlineto{\pgfqpoint{3.192876in}{2.834477in}}%
\pgfpathlineto{\pgfqpoint{3.215566in}{2.898973in}}%
\pgfpathlineto{\pgfqpoint{3.242422in}{2.961860in}}%
\pgfpathlineto{\pgfqpoint{3.272845in}{3.023111in}}%
\pgfpathlineto{\pgfqpoint{3.306359in}{3.082736in}}%
\pgfpathlineto{\pgfqpoint{3.342603in}{3.140750in}}%
\pgfpathlineto{\pgfqpoint{3.381310in}{3.197158in}}%
\pgfpathlineto{\pgfqpoint{3.422291in}{3.251936in}}%
\pgfpathlineto{\pgfqpoint{3.465433in}{3.305028in}}%
\pgfpathlineto{\pgfqpoint{3.510658in}{3.356358in}}%
\pgfpathlineto{\pgfqpoint{3.557938in}{3.405802in}}%
\pgfpathlineto{\pgfqpoint{3.607268in}{3.453197in}}%
\pgfpathlineto{\pgfqpoint{3.658659in}{3.498348in}}%
\pgfpathlineto{\pgfqpoint{3.712131in}{3.541009in}}%
\pgfpathlineto{\pgfqpoint{3.727238in}{3.552469in}}%
\pgfusepath{stroke}%
\end{pgfscope}%
\begin{pgfscope}%
\pgfpathrectangle{\pgfqpoint{0.647939in}{0.492442in}}{\pgfqpoint{3.079299in}{3.079299in}}%
\pgfusepath{clip}%
\pgfsetbuttcap%
\pgfsetroundjoin%
\pgfsetlinewidth{0.301125pt}%
\definecolor{currentstroke}{rgb}{0.500000,0.500000,0.500000}%
\pgfsetstrokecolor{currentstroke}%
\pgfsetstrokeopacity{0.300000}%
\pgfsetdash{}{0pt}%
\pgfpathmoveto{\pgfqpoint{3.727238in}{1.892124in}}%
\pgfpathlineto{\pgfqpoint{3.727238in}{1.892124in}}%
\pgfpathlineto{\pgfqpoint{3.673339in}{1.934230in}}%
\pgfpathlineto{\pgfqpoint{3.621993in}{1.979414in}}%
\pgfpathlineto{\pgfqpoint{3.573336in}{2.027483in}}%
\pgfpathlineto{\pgfqpoint{3.527519in}{2.078267in}}%
\pgfpathlineto{\pgfqpoint{3.484741in}{2.131634in}}%
\pgfpathlineto{\pgfqpoint{3.445265in}{2.187481in}}%
\pgfpathlineto{\pgfqpoint{3.409423in}{2.245721in}}%
\pgfpathlineto{\pgfqpoint{3.377627in}{2.306258in}}%
\pgfpathlineto{\pgfqpoint{3.350375in}{2.368962in}}%
\pgfpathlineto{\pgfqpoint{3.328212in}{2.433629in}}%
\pgfpathlineto{\pgfqpoint{3.311678in}{2.499948in}}%
\pgfpathlineto{\pgfqpoint{3.301228in}{2.567485in}}%
\pgfpathlineto{\pgfqpoint{3.297136in}{2.635706in}}%
\pgfpathlineto{\pgfqpoint{3.299432in}{2.704015in}}%
\pgfpathlineto{\pgfqpoint{3.307891in}{2.771840in}}%
\pgfpathlineto{\pgfqpoint{3.322082in}{2.838706in}}%
\pgfpathlineto{\pgfqpoint{3.341472in}{2.904264in}}%
\pgfpathlineto{\pgfqpoint{3.365505in}{2.968276in}}%
\pgfpathlineto{\pgfqpoint{3.393660in}{3.030596in}}%
\pgfpathlineto{\pgfqpoint{3.425492in}{3.091126in}}%
\pgfpathlineto{\pgfqpoint{3.460646in}{3.149795in}}%
\pgfpathlineto{\pgfqpoint{3.498845in}{3.206533in}}%
\pgfpathlineto{\pgfqpoint{3.539883in}{3.261259in}}%
\pgfpathlineto{\pgfqpoint{3.583621in}{3.313852in}}%
\pgfpathlineto{\pgfqpoint{3.629971in}{3.364156in}}%
\pgfpathlineto{\pgfqpoint{3.678872in}{3.411981in}}%
\pgfpathlineto{\pgfqpoint{3.727238in}{3.456854in}}%
\pgfusepath{stroke}%
\end{pgfscope}%
\begin{pgfscope}%
\pgfpathrectangle{\pgfqpoint{0.647939in}{0.492442in}}{\pgfqpoint{3.079299in}{3.079299in}}%
\pgfusepath{clip}%
\pgfsetbuttcap%
\pgfsetroundjoin%
\pgfsetlinewidth{0.301125pt}%
\definecolor{currentstroke}{rgb}{0.500000,0.500000,0.500000}%
\pgfsetstrokecolor{currentstroke}%
\pgfsetstrokeopacity{0.300000}%
\pgfsetdash{}{0pt}%
\pgfpathmoveto{\pgfqpoint{3.727238in}{1.962108in}}%
\pgfpathlineto{\pgfqpoint{3.727238in}{1.962108in}}%
\pgfpathlineto{\pgfqpoint{3.675278in}{2.006574in}}%
\pgfpathlineto{\pgfqpoint{3.626229in}{2.054234in}}%
\pgfpathlineto{\pgfqpoint{3.580277in}{2.104886in}}%
\pgfpathlineto{\pgfqpoint{3.537636in}{2.158353in}}%
\pgfpathlineto{\pgfqpoint{3.498576in}{2.214485in}}%
\pgfpathlineto{\pgfqpoint{3.463427in}{2.273141in}}%
\pgfpathlineto{\pgfqpoint{3.432598in}{2.334172in}}%
\pgfusepath{stroke}%
\end{pgfscope}%
\begin{pgfscope}%
\pgfpathrectangle{\pgfqpoint{0.647939in}{0.492442in}}{\pgfqpoint{3.079299in}{3.079299in}}%
\pgfusepath{clip}%
\pgfsetbuttcap%
\pgfsetroundjoin%
\pgfsetlinewidth{0.301125pt}%
\definecolor{currentstroke}{rgb}{0.500000,0.500000,0.500000}%
\pgfsetstrokecolor{currentstroke}%
\pgfsetstrokeopacity{0.300000}%
\pgfsetdash{}{0pt}%
\pgfpathmoveto{\pgfqpoint{3.727238in}{2.102076in}}%
\pgfpathlineto{\pgfqpoint{3.727238in}{2.102076in}}%
\pgfpathlineto{\pgfqpoint{3.680275in}{2.151772in}}%
\pgfpathlineto{\pgfqpoint{3.637124in}{2.204814in}}%
\pgfpathlineto{\pgfqpoint{3.598084in}{2.260948in}}%
\pgfpathlineto{\pgfqpoint{3.563492in}{2.319922in}}%
\pgfpathlineto{\pgfqpoint{3.533739in}{2.381470in}}%
\pgfpathlineto{\pgfqpoint{3.509243in}{2.445287in}}%
\pgfpathlineto{\pgfqpoint{3.490422in}{2.510999in}}%
\pgfpathlineto{\pgfqpoint{3.477633in}{2.578144in}}%
\pgfpathlineto{\pgfqpoint{3.471127in}{2.646180in}}%
\pgfpathlineto{\pgfqpoint{3.470979in}{2.714520in}}%
\pgfpathlineto{\pgfqpoint{3.477084in}{2.782590in}}%
\pgfpathlineto{\pgfqpoint{3.489176in}{2.849864in}}%
\pgfpathlineto{\pgfqpoint{3.506879in}{2.915895in}}%
\pgfpathlineto{\pgfqpoint{3.529765in}{2.980326in}}%
\pgfpathlineto{\pgfqpoint{3.557412in}{3.042869in}}%
\pgfpathlineto{\pgfqpoint{3.589433in}{3.103294in}}%
\pgfpathlineto{\pgfqpoint{3.625502in}{3.161395in}}%
\pgfpathlineto{\pgfqpoint{3.665353in}{3.216967in}}%
\pgfpathlineto{\pgfqpoint{3.708790in}{3.269782in}}%
\pgfpathlineto{\pgfqpoint{3.727238in}{3.290763in}}%
\pgfusepath{stroke}%
\end{pgfscope}%
\begin{pgfscope}%
\pgfpathrectangle{\pgfqpoint{0.647939in}{0.492442in}}{\pgfqpoint{3.079299in}{3.079299in}}%
\pgfusepath{clip}%
\pgfsetbuttcap%
\pgfsetroundjoin%
\pgfsetlinewidth{0.301125pt}%
\definecolor{currentstroke}{rgb}{0.500000,0.500000,0.500000}%
\pgfsetstrokecolor{currentstroke}%
\pgfsetstrokeopacity{0.300000}%
\pgfsetdash{}{0pt}%
\pgfpathmoveto{\pgfqpoint{3.727238in}{2.242044in}}%
\pgfpathlineto{\pgfqpoint{3.727238in}{2.242044in}}%
\pgfpathlineto{\pgfqpoint{3.687263in}{2.297498in}}%
\pgfpathlineto{\pgfqpoint{3.652245in}{2.356206in}}%
\pgfpathlineto{\pgfqpoint{3.622575in}{2.417784in}}%
\pgfpathlineto{\pgfqpoint{3.598651in}{2.481810in}}%
\pgfpathlineto{\pgfqpoint{3.580844in}{2.547798in}}%
\pgfpathlineto{\pgfqpoint{3.569455in}{2.615182in}}%
\pgfpathlineto{\pgfqpoint{3.564657in}{2.683346in}}%
\pgfpathlineto{\pgfqpoint{3.566464in}{2.751655in}}%
\pgfpathlineto{\pgfqpoint{3.574727in}{2.819494in}}%
\pgfpathlineto{\pgfqpoint{3.589170in}{2.886304in}}%
\pgfpathlineto{\pgfqpoint{3.609425in}{2.951595in}}%
\pgfpathlineto{\pgfqpoint{3.635090in}{3.014960in}}%
\pgfpathlineto{\pgfqpoint{3.665782in}{3.076054in}}%
\pgfpathlineto{\pgfqpoint{3.701163in}{3.134562in}}%
\pgfpathlineto{\pgfqpoint{3.727238in}{3.174160in}}%
\pgfusepath{stroke}%
\end{pgfscope}%
\begin{pgfscope}%
\pgfpathrectangle{\pgfqpoint{0.647939in}{0.492442in}}{\pgfqpoint{3.079299in}{3.079299in}}%
\pgfusepath{clip}%
\pgfsetbuttcap%
\pgfsetroundjoin%
\pgfsetlinewidth{0.301125pt}%
\definecolor{currentstroke}{rgb}{0.500000,0.500000,0.500000}%
\pgfsetstrokecolor{currentstroke}%
\pgfsetstrokeopacity{0.300000}%
\pgfsetdash{}{0pt}%
\pgfpathmoveto{\pgfqpoint{3.727238in}{2.382012in}}%
\pgfpathlineto{\pgfqpoint{3.727238in}{2.382012in}}%
\pgfpathlineto{\pgfqpoint{3.696872in}{2.443229in}}%
\pgfpathlineto{\pgfqpoint{3.672735in}{2.507154in}}%
\pgfpathlineto{\pgfqpoint{3.655182in}{2.573190in}}%
\pgfpathlineto{\pgfqpoint{3.644485in}{2.640680in}}%
\pgfpathlineto{\pgfqpoint{3.640781in}{2.708907in}}%
\pgfpathlineto{\pgfqpoint{3.644049in}{2.777153in}}%
\pgfpathlineto{\pgfqpoint{3.654114in}{2.844735in}}%
\pgfpathlineto{\pgfqpoint{3.670675in}{2.911033in}}%
\pgfpathlineto{\pgfqpoint{3.693371in}{2.975500in}}%
\pgfpathlineto{\pgfqpoint{3.721815in}{3.037652in}}%
\pgfpathlineto{\pgfqpoint{3.727238in}{3.048237in}}%
\pgfusepath{stroke}%
\end{pgfscope}%
\begin{pgfscope}%
\pgfpathrectangle{\pgfqpoint{0.647939in}{0.492442in}}{\pgfqpoint{3.079299in}{3.079299in}}%
\pgfusepath{clip}%
\pgfsetbuttcap%
\pgfsetroundjoin%
\pgfsetlinewidth{0.301125pt}%
\definecolor{currentstroke}{rgb}{0.500000,0.500000,0.500000}%
\pgfsetstrokecolor{currentstroke}%
\pgfsetstrokeopacity{0.300000}%
\pgfsetdash{}{0pt}%
\pgfpathmoveto{\pgfqpoint{3.727238in}{2.521980in}}%
\pgfpathlineto{\pgfqpoint{3.727238in}{2.521980in}}%
\pgfpathlineto{\pgfqpoint{3.709398in}{2.587922in}}%
\pgfpathlineto{\pgfqpoint{3.698825in}{2.655405in}}%
\pgfpathlineto{\pgfqpoint{3.695641in}{2.723640in}}%
\pgfpathlineto{\pgfqpoint{3.699804in}{2.791833in}}%
\pgfpathlineto{\pgfqpoint{3.711127in}{2.859216in}}%
\pgfusepath{stroke}%
\end{pgfscope}%
\begin{pgfscope}%
\pgfpathrectangle{\pgfqpoint{0.647939in}{0.492442in}}{\pgfqpoint{3.079299in}{3.079299in}}%
\pgfusepath{clip}%
\pgfsetbuttcap%
\pgfsetroundjoin%
\pgfsetlinewidth{0.301125pt}%
\definecolor{currentstroke}{rgb}{0.500000,0.500000,0.500000}%
\pgfsetstrokecolor{currentstroke}%
\pgfsetstrokeopacity{0.300000}%
\pgfsetdash{}{0pt}%
\pgfpathmoveto{\pgfqpoint{3.519102in}{3.092011in}}%
\pgfpathlineto{\pgfqpoint{3.554079in}{3.150779in}}%
\pgfpathlineto{\pgfqpoint{3.592564in}{3.207315in}}%
\pgfpathlineto{\pgfqpoint{3.634346in}{3.261466in}}%
\pgfpathlineto{\pgfqpoint{3.679275in}{3.313033in}}%
\pgfpathlineto{\pgfqpoint{3.727238in}{3.361789in}}%
\pgfpathlineto{\pgfqpoint{3.727238in}{3.361789in}}%
\pgfusepath{stroke}%
\end{pgfscope}%
\begin{pgfscope}%
\pgfpathrectangle{\pgfqpoint{0.647939in}{0.492442in}}{\pgfqpoint{3.079299in}{3.079299in}}%
\pgfusepath{clip}%
\pgfsetbuttcap%
\pgfsetroundjoin%
\pgfsetlinewidth{0.301125pt}%
\definecolor{currentstroke}{rgb}{0.500000,0.500000,0.500000}%
\pgfsetstrokecolor{currentstroke}%
\pgfsetstrokeopacity{0.300000}%
\pgfsetdash{}{0pt}%
\pgfpathmoveto{\pgfqpoint{3.237697in}{3.124232in}}%
\pgfpathlineto{\pgfqpoint{3.277625in}{3.179784in}}%
\pgfpathlineto{\pgfqpoint{3.319301in}{3.234040in}}%
\pgfpathlineto{\pgfqpoint{3.362642in}{3.286979in}}%
\pgfpathlineto{\pgfqpoint{3.407606in}{3.338546in}}%
\pgfpathlineto{\pgfqpoint{3.454185in}{3.388660in}}%
\pgfpathlineto{\pgfqpoint{3.502396in}{3.437204in}}%
\pgfpathlineto{\pgfqpoint{3.552280in}{3.484025in}}%
\pgfpathlineto{\pgfqpoint{3.603881in}{3.528945in}}%
\pgfpathlineto{\pgfqpoint{3.657254in}{3.571741in}}%
\pgfpathlineto{\pgfqpoint{3.657254in}{3.571741in}}%
\pgfusepath{stroke}%
\end{pgfscope}%
\begin{pgfscope}%
\pgfpathrectangle{\pgfqpoint{0.647939in}{0.492442in}}{\pgfqpoint{3.079299in}{3.079299in}}%
\pgfusepath{clip}%
\pgfsetbuttcap%
\pgfsetroundjoin%
\pgfsetlinewidth{0.301125pt}%
\definecolor{currentstroke}{rgb}{0.500000,0.500000,0.500000}%
\pgfsetstrokecolor{currentstroke}%
\pgfsetstrokeopacity{0.300000}%
\pgfsetdash{}{0pt}%
\pgfpathmoveto{\pgfqpoint{0.647939in}{2.594352in}}%
\pgfpathlineto{\pgfqpoint{0.651191in}{2.594719in}}%
\pgfpathlineto{\pgfqpoint{0.719104in}{2.603069in}}%
\pgfpathlineto{\pgfqpoint{0.786836in}{2.612779in}}%
\pgfpathlineto{\pgfqpoint{0.854357in}{2.623864in}}%
\pgfpathlineto{\pgfqpoint{0.921643in}{2.636296in}}%
\pgfpathlineto{\pgfqpoint{0.988682in}{2.650003in}}%
\pgfpathlineto{\pgfqpoint{1.055475in}{2.664866in}}%
\pgfpathlineto{\pgfqpoint{1.122041in}{2.680716in}}%
\pgfpathlineto{\pgfqpoint{1.188418in}{2.697344in}}%
\pgfpathlineto{\pgfqpoint{1.254659in}{2.714507in}}%
\pgfpathlineto{\pgfqpoint{1.320835in}{2.731921in}}%
\pgfpathlineto{\pgfqpoint{1.387026in}{2.749274in}}%
\pgfpathlineto{\pgfqpoint{1.453321in}{2.766228in}}%
\pgfpathlineto{\pgfqpoint{1.519804in}{2.782422in}}%
\pgfpathlineto{\pgfqpoint{1.586550in}{2.797490in}}%
\pgfpathlineto{\pgfqpoint{1.653611in}{2.811070in}}%
\pgfpathlineto{\pgfqpoint{1.721013in}{2.822829in}}%
\pgfpathlineto{\pgfqpoint{1.788747in}{2.832485in}}%
\pgfpathlineto{\pgfqpoint{1.856768in}{2.839835in}}%
\pgfpathlineto{\pgfqpoint{1.925005in}{2.844794in}}%
\pgfpathlineto{\pgfqpoint{1.993373in}{2.847435in}}%
\pgfpathlineto{\pgfqpoint{2.061792in}{2.848012in}}%
\pgfpathlineto{\pgfqpoint{2.130209in}{2.846977in}}%
\pgfpathlineto{\pgfqpoint{2.198609in}{2.845010in}}%
\pgfpathlineto{\pgfqpoint{2.267009in}{2.843053in}}%
\pgfpathlineto{\pgfqpoint{2.335426in}{2.842349in}}%
\pgfpathlineto{\pgfqpoint{2.403795in}{2.844414in}}%
\pgfpathlineto{\pgfqpoint{2.471852in}{2.850924in}}%
\pgfpathlineto{\pgfqpoint{2.539031in}{2.863432in}}%
\pgfpathlineto{\pgfqpoint{2.604522in}{2.882889in}}%
\pgfpathlineto{\pgfqpoint{2.667554in}{2.909274in}}%
\pgfpathlineto{\pgfqpoint{2.727730in}{2.941693in}}%
\pgfpathlineto{\pgfqpoint{2.785117in}{2.978859in}}%
\pgfpathlineto{\pgfqpoint{2.840091in}{3.019529in}}%
\pgfpathlineto{\pgfqpoint{2.893140in}{3.062702in}}%
\pgfpathlineto{\pgfqpoint{2.944737in}{3.107621in}}%
\pgfpathlineto{\pgfqpoint{2.995295in}{3.153712in}}%
\pgfpathlineto{\pgfqpoint{3.045162in}{3.200556in}}%
\pgfpathlineto{\pgfqpoint{3.094635in}{3.247822in}}%
\pgfpathlineto{\pgfqpoint{3.143959in}{3.295242in}}%
\pgfpathlineto{\pgfqpoint{3.193348in}{3.342599in}}%
\pgfpathlineto{\pgfqpoint{3.242991in}{3.389689in}}%
\pgfpathlineto{\pgfqpoint{3.293054in}{3.436332in}}%
\pgfpathlineto{\pgfqpoint{3.343696in}{3.482346in}}%
\pgfpathlineto{\pgfqpoint{3.395068in}{3.527543in}}%
\pgfpathlineto{\pgfqpoint{3.447302in}{3.571741in}}%
\pgfpathlineto{\pgfqpoint{3.447302in}{3.571741in}}%
\pgfusepath{stroke}%
\end{pgfscope}%
\begin{pgfscope}%
\pgfpathrectangle{\pgfqpoint{0.647939in}{0.492442in}}{\pgfqpoint{3.079299in}{3.079299in}}%
\pgfusepath{clip}%
\pgfsetbuttcap%
\pgfsetroundjoin%
\pgfsetlinewidth{0.301125pt}%
\definecolor{currentstroke}{rgb}{0.500000,0.500000,0.500000}%
\pgfsetstrokecolor{currentstroke}%
\pgfsetstrokeopacity{0.300000}%
\pgfsetdash{}{0pt}%
\pgfpathmoveto{\pgfqpoint{0.647939in}{2.884655in}}%
\pgfpathlineto{\pgfqpoint{0.657745in}{2.885723in}}%
\pgfpathlineto{\pgfqpoint{0.725698in}{2.893754in}}%
\pgfpathlineto{\pgfqpoint{0.793491in}{2.903036in}}%
\pgfpathlineto{\pgfqpoint{0.861101in}{2.913567in}}%
\pgfpathlineto{\pgfqpoint{0.928514in}{2.925301in}}%
\pgfpathlineto{\pgfqpoint{0.995723in}{2.938148in}}%
\pgfpathlineto{\pgfqpoint{1.062739in}{2.951975in}}%
\pgfpathlineto{\pgfqpoint{1.129583in}{2.966610in}}%
\pgfpathlineto{\pgfqpoint{1.196294in}{2.981845in}}%
\pgfpathlineto{\pgfqpoint{1.262922in}{2.997437in}}%
\pgfpathlineto{\pgfqpoint{1.329532in}{3.013112in}}%
\pgfpathlineto{\pgfqpoint{1.396192in}{3.028569in}}%
\pgfpathlineto{\pgfqpoint{1.462973in}{3.043491in}}%
\pgfpathlineto{\pgfqpoint{1.529939in}{3.057554in}}%
\pgfpathlineto{\pgfqpoint{1.597139in}{3.070442in}}%
\pgfpathlineto{\pgfqpoint{1.664603in}{3.081857in}}%
\pgfpathlineto{\pgfqpoint{1.732336in}{3.091542in}}%
\pgfpathlineto{\pgfqpoint{1.800315in}{3.099301in}}%
\pgfpathlineto{\pgfqpoint{1.868495in}{3.105029in}}%
\pgfpathlineto{\pgfqpoint{1.936815in}{3.108734in}}%
\pgfpathlineto{\pgfqpoint{2.005213in}{3.110551in}}%
\pgfpathlineto{\pgfqpoint{2.073637in}{3.110758in}}%
\pgfpathlineto{\pgfqpoint{2.142058in}{3.109790in}}%
\pgfpathlineto{\pgfqpoint{2.210469in}{3.108252in}}%
\pgfpathlineto{\pgfqpoint{2.278884in}{3.106911in}}%
\pgfpathlineto{\pgfqpoint{2.347307in}{3.106667in}}%
\pgfpathlineto{\pgfqpoint{2.415695in}{3.108550in}}%
\pgfpathlineto{\pgfqpoint{2.483907in}{3.113646in}}%
\pgfpathlineto{\pgfqpoint{2.551650in}{3.122985in}}%
\pgfpathlineto{\pgfqpoint{2.618496in}{3.137326in}}%
\pgfpathlineto{\pgfqpoint{2.683966in}{3.156989in}}%
\pgfpathlineto{\pgfqpoint{2.747665in}{3.181805in}}%
\pgfpathlineto{\pgfqpoint{2.809391in}{3.211214in}}%
\pgfpathlineto{\pgfqpoint{2.869154in}{3.244457in}}%
\pgfpathlineto{\pgfqpoint{2.927135in}{3.280743in}}%
\pgfpathlineto{\pgfqpoint{2.983608in}{3.319355in}}%
\pgfpathlineto{\pgfqpoint{3.038882in}{3.359677in}}%
\pgfpathlineto{\pgfqpoint{3.093266in}{3.401195in}}%
\pgfpathlineto{\pgfqpoint{3.147052in}{3.443489in}}%
\pgfpathlineto{\pgfqpoint{3.200500in}{3.486212in}}%
\pgfpathlineto{\pgfqpoint{3.253853in}{3.529056in}}%
\pgfpathlineto{\pgfqpoint{3.307334in}{3.571741in}}%
\pgfpathlineto{\pgfqpoint{3.307334in}{3.571741in}}%
\pgfusepath{stroke}%
\end{pgfscope}%
\begin{pgfscope}%
\pgfpathrectangle{\pgfqpoint{0.647939in}{0.492442in}}{\pgfqpoint{3.079299in}{3.079299in}}%
\pgfusepath{clip}%
\pgfsetbuttcap%
\pgfsetroundjoin%
\pgfsetlinewidth{0.301125pt}%
\definecolor{currentstroke}{rgb}{0.500000,0.500000,0.500000}%
\pgfsetstrokecolor{currentstroke}%
\pgfsetstrokeopacity{0.300000}%
\pgfsetdash{}{0pt}%
\pgfpathmoveto{\pgfqpoint{0.647939in}{3.063660in}}%
\pgfpathlineto{\pgfqpoint{0.656927in}{3.064607in}}%
\pgfpathlineto{\pgfqpoint{0.724911in}{3.072370in}}%
\pgfpathlineto{\pgfqpoint{0.792748in}{3.081323in}}%
\pgfpathlineto{\pgfqpoint{0.860420in}{3.091455in}}%
\pgfpathlineto{\pgfqpoint{0.927914in}{3.102717in}}%
\pgfpathlineto{\pgfqpoint{0.995227in}{3.115013in}}%
\pgfpathlineto{\pgfqpoint{1.062370in}{3.128209in}}%
\pgfpathlineto{\pgfqpoint{1.129366in}{3.142133in}}%
\pgfpathlineto{\pgfqpoint{1.196252in}{3.156580in}}%
\pgfpathlineto{\pgfqpoint{1.263076in}{3.171311in}}%
\pgfpathlineto{\pgfqpoint{1.329896in}{3.186061in}}%
\pgfpathlineto{\pgfqpoint{1.396775in}{3.200543in}}%
\pgfpathlineto{\pgfqpoint{1.463773in}{3.214455in}}%
\pgfpathlineto{\pgfqpoint{1.530946in}{3.227496in}}%
\pgfpathlineto{\pgfqpoint{1.598332in}{3.239375in}}%
\pgfpathlineto{\pgfqpoint{1.665952in}{3.249828in}}%
\pgfpathlineto{\pgfqpoint{1.733806in}{3.258633in}}%
\pgfpathlineto{\pgfqpoint{1.801869in}{3.265630in}}%
\pgfpathlineto{\pgfqpoint{1.870099in}{3.270749in}}%
\pgfpathlineto{\pgfqpoint{1.938442in}{3.274023in}}%
\pgfpathlineto{\pgfqpoint{2.006848in}{3.275601in}}%
\pgfpathlineto{\pgfqpoint{2.075273in}{3.275755in}}%
\pgfpathlineto{\pgfqpoint{2.143695in}{3.274893in}}%
\pgfpathlineto{\pgfqpoint{2.212111in}{3.273561in}}%
\pgfpathlineto{\pgfqpoint{2.280530in}{3.272424in}}%
\pgfpathlineto{\pgfqpoint{2.348954in}{3.272246in}}%
\pgfpathlineto{\pgfqpoint{2.417353in}{3.273878in}}%
\pgfpathlineto{\pgfqpoint{2.485626in}{3.278206in}}%
\pgfpathlineto{\pgfqpoint{2.553569in}{3.286070in}}%
\pgfpathlineto{\pgfqpoint{2.620883in}{3.298126in}}%
\pgfpathlineto{\pgfqpoint{2.687211in}{3.314740in}}%
\pgfpathlineto{\pgfqpoint{2.752224in}{3.335929in}}%
\pgfpathlineto{\pgfqpoint{2.815692in}{3.361392in}}%
\pgfpathlineto{\pgfqpoint{2.877534in}{3.390604in}}%
\pgfpathlineto{\pgfqpoint{2.937811in}{3.422939in}}%
\pgfpathlineto{\pgfqpoint{2.996693in}{3.457765in}}%
\pgfpathlineto{\pgfqpoint{3.054408in}{3.494502in}}%
\pgfpathlineto{\pgfqpoint{3.111212in}{3.532641in}}%
\pgfpathlineto{\pgfqpoint{3.167366in}{3.571741in}}%
\pgfpathlineto{\pgfqpoint{3.167366in}{3.571741in}}%
\pgfusepath{stroke}%
\end{pgfscope}%
\begin{pgfscope}%
\pgfpathrectangle{\pgfqpoint{0.647939in}{0.492442in}}{\pgfqpoint{3.079299in}{3.079299in}}%
\pgfusepath{clip}%
\pgfsetbuttcap%
\pgfsetroundjoin%
\pgfsetlinewidth{0.301125pt}%
\definecolor{currentstroke}{rgb}{0.500000,0.500000,0.500000}%
\pgfsetstrokecolor{currentstroke}%
\pgfsetstrokeopacity{0.300000}%
\pgfsetdash{}{0pt}%
\pgfpathmoveto{\pgfqpoint{0.647939in}{3.188223in}}%
\pgfpathlineto{\pgfqpoint{0.683618in}{3.192142in}}%
\pgfpathlineto{\pgfqpoint{0.751569in}{3.200190in}}%
\pgfpathlineto{\pgfqpoint{0.819374in}{3.209385in}}%
\pgfpathlineto{\pgfqpoint{0.887019in}{3.219696in}}%
\pgfpathlineto{\pgfqpoint{0.954496in}{3.231055in}}%
\pgfpathlineto{\pgfqpoint{1.021809in}{3.243354in}}%
\pgfpathlineto{\pgfqpoint{1.088974in}{3.256439in}}%
\pgfpathlineto{\pgfqpoint{1.156020in}{3.270125in}}%
\pgfpathlineto{\pgfqpoint{1.222987in}{3.284192in}}%
\pgfpathlineto{\pgfqpoint{1.289925in}{3.298399in}}%
\pgfpathlineto{\pgfqpoint{1.356890in}{3.312477in}}%
\pgfpathlineto{\pgfqpoint{1.423939in}{3.326145in}}%
\pgfpathlineto{\pgfqpoint{1.491127in}{3.339112in}}%
\pgfpathlineto{\pgfqpoint{1.558496in}{3.351094in}}%
\pgfpathlineto{\pgfqpoint{1.626074in}{3.361826in}}%
\pgfpathlineto{\pgfqpoint{1.693868in}{3.371079in}}%
\pgfpathlineto{\pgfqpoint{1.761868in}{3.378676in}}%
\pgfpathlineto{\pgfqpoint{1.830041in}{3.384508in}}%
\pgfpathlineto{\pgfqpoint{1.898344in}{3.388563in}}%
\pgfpathlineto{\pgfqpoint{1.966726in}{3.390930in}}%
\pgfpathlineto{\pgfqpoint{2.035145in}{3.391810in}}%
\pgfpathlineto{\pgfqpoint{2.103572in}{3.391518in}}%
\pgfpathlineto{\pgfqpoint{2.171992in}{3.390482in}}%
\pgfpathlineto{\pgfqpoint{2.240410in}{3.389256in}}%
\pgfpathlineto{\pgfqpoint{2.308833in}{3.388491in}}%
\pgfpathlineto{\pgfqpoint{2.377255in}{3.388910in}}%
\pgfpathlineto{\pgfqpoint{2.445631in}{3.391284in}}%
\pgfpathlineto{\pgfqpoint{2.513852in}{3.396380in}}%
\pgfpathlineto{\pgfqpoint{2.581721in}{3.404892in}}%
\pgfpathlineto{\pgfqpoint{2.648970in}{3.417341in}}%
\pgfpathlineto{\pgfqpoint{2.715302in}{3.433979in}}%
\pgfpathlineto{\pgfqpoint{2.780450in}{3.454775in}}%
\pgfpathlineto{\pgfqpoint{2.844239in}{3.479446in}}%
\pgfpathlineto{\pgfqpoint{2.906608in}{3.507530in}}%
\pgfpathlineto{\pgfqpoint{2.967612in}{3.538482in}}%
\pgfpathlineto{\pgfqpoint{3.027398in}{3.571741in}}%
\pgfpathlineto{\pgfqpoint{3.027398in}{3.571741in}}%
\pgfusepath{stroke}%
\end{pgfscope}%
\begin{pgfscope}%
\pgfpathrectangle{\pgfqpoint{0.647939in}{0.492442in}}{\pgfqpoint{3.079299in}{3.079299in}}%
\pgfusepath{clip}%
\pgfsetbuttcap%
\pgfsetroundjoin%
\pgfsetlinewidth{0.301125pt}%
\definecolor{currentstroke}{rgb}{0.500000,0.500000,0.500000}%
\pgfsetstrokecolor{currentstroke}%
\pgfsetstrokeopacity{0.300000}%
\pgfsetdash{}{0pt}%
\pgfpathmoveto{\pgfqpoint{0.647939in}{3.274873in}}%
\pgfpathlineto{\pgfqpoint{0.659474in}{3.276053in}}%
\pgfpathlineto{\pgfqpoint{0.727485in}{3.283580in}}%
\pgfpathlineto{\pgfqpoint{0.795362in}{3.292230in}}%
\pgfpathlineto{\pgfqpoint{0.863090in}{3.301985in}}%
\pgfpathlineto{\pgfqpoint{0.930659in}{3.312786in}}%
\pgfpathlineto{\pgfqpoint{0.998070in}{3.324532in}}%
\pgfpathlineto{\pgfqpoint{1.065336in}{3.337088in}}%
\pgfpathlineto{\pgfqpoint{1.132480in}{3.350282in}}%
\pgfpathlineto{\pgfqpoint{1.199538in}{3.363910in}}%
\pgfpathlineto{\pgfqpoint{1.266555in}{3.377742in}}%
\pgfpathlineto{\pgfqpoint{1.333582in}{3.391518in}}%
\pgfpathlineto{\pgfqpoint{1.400676in}{3.404966in}}%
\pgfpathlineto{\pgfqpoint{1.467889in}{3.417803in}}%
\pgfpathlineto{\pgfqpoint{1.535264in}{3.429753in}}%
\pgfpathlineto{\pgfqpoint{1.602831in}{3.440557in}}%
\pgfpathlineto{\pgfqpoint{1.670603in}{3.449982in}}%
\pgfpathlineto{\pgfqpoint{1.738573in}{3.457843in}}%
\pgfpathlineto{\pgfqpoint{1.806716in}{3.464022in}}%
\pgfpathlineto{\pgfqpoint{1.874994in}{3.468480in}}%
\pgfpathlineto{\pgfqpoint{1.943360in}{3.471275in}}%
\pgfpathlineto{\pgfqpoint{2.011772in}{3.472565in}}%
\pgfpathlineto{\pgfqpoint{2.080199in}{3.472621in}}%
\pgfpathlineto{\pgfqpoint{2.148622in}{3.471830in}}%
\pgfpathlineto{\pgfqpoint{2.217042in}{3.470679in}}%
\pgfpathlineto{\pgfqpoint{2.285463in}{3.469743in}}%
\pgfpathlineto{\pgfqpoint{2.353889in}{3.469673in}}%
\pgfpathlineto{\pgfqpoint{2.422294in}{3.471176in}}%
\pgfpathlineto{\pgfqpoint{2.490603in}{3.474979in}}%
\pgfpathlineto{\pgfqpoint{2.558671in}{3.481758in}}%
\pgfpathlineto{\pgfqpoint{2.626289in}{3.492056in}}%
\pgfpathlineto{\pgfqpoint{2.693200in}{3.506219in}}%
\pgfpathlineto{\pgfqpoint{2.759148in}{3.524349in}}%
\pgfpathlineto{\pgfqpoint{2.823928in}{3.546304in}}%
\pgfpathlineto{\pgfqpoint{2.887429in}{3.571741in}}%
\pgfpathlineto{\pgfqpoint{2.887429in}{3.571741in}}%
\pgfusepath{stroke}%
\end{pgfscope}%
\begin{pgfscope}%
\pgfpathrectangle{\pgfqpoint{0.647939in}{0.492442in}}{\pgfqpoint{3.079299in}{3.079299in}}%
\pgfusepath{clip}%
\pgfsetbuttcap%
\pgfsetroundjoin%
\pgfsetlinewidth{0.301125pt}%
\definecolor{currentstroke}{rgb}{0.500000,0.500000,0.500000}%
\pgfsetstrokecolor{currentstroke}%
\pgfsetstrokeopacity{0.300000}%
\pgfsetdash{}{0pt}%
\pgfpathmoveto{\pgfqpoint{0.647939in}{3.350077in}}%
\pgfpathlineto{\pgfqpoint{0.710434in}{3.357162in}}%
\pgfpathlineto{\pgfqpoint{0.778360in}{3.365421in}}%
\pgfpathlineto{\pgfqpoint{0.846146in}{3.374765in}}%
\pgfpathlineto{\pgfqpoint{0.913781in}{3.385145in}}%
\pgfpathlineto{\pgfqpoint{0.981263in}{3.396473in}}%
\pgfpathlineto{\pgfqpoint{1.048604in}{3.408624in}}%
\pgfpathlineto{\pgfqpoint{1.115822in}{3.421436in}}%
\pgfpathlineto{\pgfqpoint{1.182951in}{3.434710in}}%
\pgfpathlineto{\pgfqpoint{1.250033in}{3.448218in}}%
\pgfpathlineto{\pgfqpoint{1.317117in}{3.461716in}}%
\pgfpathlineto{\pgfqpoint{1.384256in}{3.474939in}}%
\pgfpathlineto{\pgfqpoint{1.451499in}{3.487616in}}%
\pgfpathlineto{\pgfqpoint{1.518890in}{3.499477in}}%
\pgfpathlineto{\pgfqpoint{1.586461in}{3.510262in}}%
\pgfpathlineto{\pgfqpoint{1.654226in}{3.519740in}}%
\pgfpathlineto{\pgfqpoint{1.722182in}{3.527724in}}%
\pgfpathlineto{\pgfqpoint{1.790308in}{3.534089in}}%
\pgfpathlineto{\pgfqpoint{1.858570in}{3.538780in}}%
\pgfpathlineto{\pgfqpoint{1.926926in}{3.541838in}}%
\pgfpathlineto{\pgfqpoint{1.995333in}{3.543403in}}%
\pgfpathlineto{\pgfqpoint{2.063758in}{3.543718in}}%
\pgfpathlineto{\pgfqpoint{2.132184in}{3.543124in}}%
\pgfpathlineto{\pgfqpoint{2.200604in}{3.542060in}}%
\pgfpathlineto{\pgfqpoint{2.269025in}{3.541058in}}%
\pgfpathlineto{\pgfqpoint{2.337451in}{3.540732in}}%
\pgfpathlineto{\pgfqpoint{2.405866in}{3.541739in}}%
\pgfpathlineto{\pgfqpoint{2.474217in}{3.544753in}}%
\pgfpathlineto{\pgfqpoint{2.542394in}{3.550416in}}%
\pgfpathlineto{\pgfqpoint{2.610222in}{3.559281in}}%
\pgfpathlineto{\pgfqpoint{2.677477in}{3.571741in}}%
\pgfpathlineto{\pgfqpoint{2.677477in}{3.571741in}}%
\pgfusepath{stroke}%
\end{pgfscope}%
\begin{pgfscope}%
\pgfpathrectangle{\pgfqpoint{0.647939in}{0.492442in}}{\pgfqpoint{3.079299in}{3.079299in}}%
\pgfusepath{clip}%
\pgfsetbuttcap%
\pgfsetroundjoin%
\pgfsetlinewidth{0.301125pt}%
\definecolor{currentstroke}{rgb}{0.500000,0.500000,0.500000}%
\pgfsetstrokecolor{currentstroke}%
\pgfsetstrokeopacity{0.300000}%
\pgfsetdash{}{0pt}%
\pgfpathmoveto{\pgfqpoint{0.647939in}{3.408639in}}%
\pgfpathlineto{\pgfqpoint{0.683389in}{3.412388in}}%
\pgfpathlineto{\pgfqpoint{0.751376in}{3.420130in}}%
\pgfpathlineto{\pgfqpoint{0.819232in}{3.428950in}}%
\pgfpathlineto{\pgfqpoint{0.886944in}{3.438812in}}%
\pgfpathlineto{\pgfqpoint{0.954508in}{3.449644in}}%
\pgfpathlineto{\pgfqpoint{1.021930in}{3.461334in}}%
\pgfpathlineto{\pgfqpoint{1.089225in}{3.473731in}}%
\pgfpathlineto{\pgfqpoint{1.156423in}{3.486650in}}%
\pgfpathlineto{\pgfqpoint{1.223561in}{3.499879in}}%
\pgfpathlineto{\pgfqpoint{1.290684in}{3.513184in}}%
\pgfpathlineto{\pgfqpoint{1.357842in}{3.526310in}}%
\pgfpathlineto{\pgfqpoint{1.425085in}{3.538992in}}%
\pgfpathlineto{\pgfqpoint{1.492457in}{3.550960in}}%
\pgfpathlineto{\pgfqpoint{1.559995in}{3.561955in}}%
\pgfpathlineto{\pgfqpoint{1.627716in}{3.571741in}}%
\pgfpathlineto{\pgfqpoint{1.627716in}{3.571741in}}%
\pgfusepath{stroke}%
\end{pgfscope}%
\begin{pgfscope}%
\pgfpathrectangle{\pgfqpoint{0.647939in}{0.492442in}}{\pgfqpoint{3.079299in}{3.079299in}}%
\pgfusepath{clip}%
\pgfsetbuttcap%
\pgfsetroundjoin%
\pgfsetlinewidth{0.301125pt}%
\definecolor{currentstroke}{rgb}{0.500000,0.500000,0.500000}%
\pgfsetstrokecolor{currentstroke}%
\pgfsetstrokeopacity{0.300000}%
\pgfsetdash{}{0pt}%
\pgfpathmoveto{\pgfqpoint{0.647939in}{3.485222in}}%
\pgfpathlineto{\pgfqpoint{0.667178in}{3.487159in}}%
\pgfpathlineto{\pgfqpoint{0.735204in}{3.494547in}}%
\pgfpathlineto{\pgfqpoint{0.803107in}{3.502998in}}%
\pgfpathlineto{\pgfqpoint{0.870873in}{3.512481in}}%
\pgfpathlineto{\pgfqpoint{0.938499in}{3.522926in}}%
\pgfpathlineto{\pgfqpoint{1.005986in}{3.534230in}}%
\pgfpathlineto{\pgfqpoint{1.073349in}{3.546253in}}%
\pgfpathlineto{\pgfqpoint{1.140613in}{3.558825in}}%
\pgfpathlineto{\pgfqpoint{1.207812in}{3.571741in}}%
\pgfpathlineto{\pgfqpoint{1.207812in}{3.571741in}}%
\pgfusepath{stroke}%
\end{pgfscope}%
\begin{pgfscope}%
\pgfpathrectangle{\pgfqpoint{0.647939in}{0.492442in}}{\pgfqpoint{3.079299in}{3.079299in}}%
\pgfusepath{clip}%
\pgfsetbuttcap%
\pgfsetroundjoin%
\pgfsetlinewidth{0.301125pt}%
\definecolor{currentstroke}{rgb}{0.500000,0.500000,0.500000}%
\pgfsetstrokecolor{currentstroke}%
\pgfsetstrokeopacity{0.300000}%
\pgfsetdash{}{0pt}%
\pgfpathmoveto{\pgfqpoint{0.647939in}{2.801916in}}%
\pgfpathlineto{\pgfqpoint{0.647939in}{2.801916in}}%
\pgfpathlineto{\pgfqpoint{0.715899in}{2.809890in}}%
\pgfpathlineto{\pgfqpoint{0.783696in}{2.819141in}}%
\pgfpathlineto{\pgfqpoint{0.851305in}{2.829674in}}%
\pgfpathlineto{\pgfqpoint{0.918709in}{2.841455in}}%
\pgfpathlineto{\pgfqpoint{0.985898in}{2.854405in}}%
\pgfpathlineto{\pgfqpoint{1.052878in}{2.868404in}}%
\pgfpathlineto{\pgfqpoint{1.119668in}{2.883283in}}%
\pgfpathlineto{\pgfqpoint{1.186306in}{2.898832in}}%
\pgfpathlineto{\pgfqpoint{1.252843in}{2.914812in}}%
\pgfpathlineto{\pgfqpoint{1.319342in}{2.930950in}}%
\pgfpathlineto{\pgfqpoint{1.385875in}{2.946945in}}%
\pgfpathlineto{\pgfqpoint{1.452516in}{2.962478in}}%
\pgfusepath{stroke}%
\end{pgfscope}%
\begin{pgfscope}%
\pgfpathrectangle{\pgfqpoint{0.647939in}{0.492442in}}{\pgfqpoint{3.079299in}{3.079299in}}%
\pgfusepath{clip}%
\pgfsetbuttcap%
\pgfsetroundjoin%
\pgfsetlinewidth{0.301125pt}%
\definecolor{currentstroke}{rgb}{0.500000,0.500000,0.500000}%
\pgfsetstrokecolor{currentstroke}%
\pgfsetstrokeopacity{0.300000}%
\pgfsetdash{}{0pt}%
\pgfpathmoveto{\pgfqpoint{0.647939in}{2.731932in}}%
\pgfpathlineto{\pgfqpoint{0.647939in}{2.731932in}}%
\pgfpathlineto{\pgfqpoint{0.715886in}{2.740009in}}%
\pgfpathlineto{\pgfqpoint{0.783665in}{2.749389in}}%
\pgfpathlineto{\pgfqpoint{0.851250in}{2.760082in}}%
\pgfpathlineto{\pgfqpoint{0.918619in}{2.772054in}}%
\pgfpathlineto{\pgfqpoint{0.985764in}{2.785232in}}%
\pgfpathlineto{\pgfqpoint{1.052688in}{2.799495in}}%
\pgfpathlineto{\pgfqpoint{1.119410in}{2.814676in}}%
\pgfpathlineto{\pgfqpoint{1.185968in}{2.830565in}}%
\pgfpathlineto{\pgfqpoint{1.252413in}{2.846920in}}%
\pgfpathlineto{\pgfqpoint{1.318811in}{2.863467in}}%
\pgfpathlineto{\pgfqpoint{1.385237in}{2.879901in}}%
\pgfpathlineto{\pgfqpoint{1.451770in}{2.895895in}}%
\pgfpathlineto{\pgfqpoint{1.518484in}{2.911111in}}%
\pgfpathlineto{\pgfqpoint{1.585443in}{2.925201in}}%
\pgfpathlineto{\pgfqpoint{1.652690in}{2.937834in}}%
\pgfpathlineto{\pgfqpoint{1.720240in}{2.948710in}}%
\pgfpathlineto{\pgfqpoint{1.788081in}{2.957585in}}%
\pgfpathlineto{\pgfqpoint{1.856170in}{2.964298in}}%
\pgfpathlineto{\pgfqpoint{1.924441in}{2.968798in}}%
\pgfpathlineto{\pgfqpoint{1.992820in}{2.971182in}}%
\pgfpathlineto{\pgfqpoint{2.061241in}{2.971712in}}%
\pgfpathlineto{\pgfqpoint{2.129661in}{2.970817in}}%
\pgfpathlineto{\pgfqpoint{2.198068in}{2.969116in}}%
\pgfpathlineto{\pgfqpoint{2.266476in}{2.967434in}}%
\pgfpathlineto{\pgfqpoint{2.334896in}{2.966820in}}%
\pgfpathlineto{\pgfqpoint{2.403285in}{2.968513in}}%
\pgfpathlineto{\pgfqpoint{2.471463in}{2.973850in}}%
\pgfpathlineto{\pgfqpoint{2.539052in}{2.984102in}}%
\pgfpathlineto{\pgfqpoint{2.605478in}{3.000184in}}%
\pgfpathlineto{\pgfqpoint{2.670113in}{3.022386in}}%
\pgfpathlineto{\pgfqpoint{2.732501in}{3.050315in}}%
\pgfpathlineto{\pgfqpoint{2.792492in}{3.083120in}}%
\pgfpathlineto{\pgfqpoint{2.850216in}{3.119789in}}%
\pgfpathlineto{\pgfqpoint{2.905985in}{3.159381in}}%
\pgfpathlineto{\pgfqpoint{2.960183in}{3.201116in}}%
\pgfusepath{stroke}%
\end{pgfscope}%
\begin{pgfscope}%
\pgfpathrectangle{\pgfqpoint{0.647939in}{0.492442in}}{\pgfqpoint{3.079299in}{3.079299in}}%
\pgfusepath{clip}%
\pgfsetbuttcap%
\pgfsetroundjoin%
\pgfsetlinewidth{0.301125pt}%
\definecolor{currentstroke}{rgb}{0.500000,0.500000,0.500000}%
\pgfsetstrokecolor{currentstroke}%
\pgfsetstrokeopacity{0.300000}%
\pgfsetdash{}{0pt}%
\pgfpathmoveto{\pgfqpoint{0.647939in}{2.661948in}}%
\pgfpathlineto{\pgfqpoint{0.647939in}{2.661948in}}%
\pgfpathlineto{\pgfqpoint{0.715873in}{2.670130in}}%
\pgfpathlineto{\pgfqpoint{0.783634in}{2.679644in}}%
\pgfpathlineto{\pgfqpoint{0.851192in}{2.690500in}}%
\pgfpathlineto{\pgfqpoint{0.918526in}{2.702671in}}%
\pgfpathlineto{\pgfqpoint{0.985624in}{2.716083in}}%
\pgfpathlineto{\pgfqpoint{1.052489in}{2.730620in}}%
\pgfpathlineto{\pgfqpoint{1.119139in}{2.746115in}}%
\pgfusepath{stroke}%
\end{pgfscope}%
\begin{pgfscope}%
\pgfpathrectangle{\pgfqpoint{0.647939in}{0.492442in}}{\pgfqpoint{3.079299in}{3.079299in}}%
\pgfusepath{clip}%
\pgfsetbuttcap%
\pgfsetroundjoin%
\pgfsetlinewidth{0.301125pt}%
\definecolor{currentstroke}{rgb}{0.500000,0.500000,0.500000}%
\pgfsetstrokecolor{currentstroke}%
\pgfsetstrokeopacity{0.300000}%
\pgfsetdash{}{0pt}%
\pgfpathmoveto{\pgfqpoint{0.647939in}{2.521980in}}%
\pgfpathlineto{\pgfqpoint{0.647939in}{2.521980in}}%
\pgfpathlineto{\pgfqpoint{0.715846in}{2.530382in}}%
\pgfpathlineto{\pgfqpoint{0.783567in}{2.540173in}}%
\pgfpathlineto{\pgfqpoint{0.851068in}{2.551373in}}%
\pgfpathlineto{\pgfqpoint{0.918325in}{2.563959in}}%
\pgfpathlineto{\pgfqpoint{0.985322in}{2.577866in}}%
\pgfpathlineto{\pgfqpoint{1.052058in}{2.592982in}}%
\pgfpathlineto{\pgfqpoint{1.118549in}{2.609142in}}%
\pgfpathlineto{\pgfqpoint{1.184833in}{2.626138in}}%
\pgfpathlineto{\pgfqpoint{1.250962in}{2.643725in}}%
\pgfpathlineto{\pgfqpoint{1.317009in}{2.661622in}}%
\pgfpathlineto{\pgfqpoint{1.383058in}{2.679512in}}%
\pgfusepath{stroke}%
\end{pgfscope}%
\begin{pgfscope}%
\pgfpathrectangle{\pgfqpoint{0.647939in}{0.492442in}}{\pgfqpoint{3.079299in}{3.079299in}}%
\pgfusepath{clip}%
\pgfsetbuttcap%
\pgfsetroundjoin%
\pgfsetlinewidth{0.301125pt}%
\definecolor{currentstroke}{rgb}{0.500000,0.500000,0.500000}%
\pgfsetstrokecolor{currentstroke}%
\pgfsetstrokeopacity{0.300000}%
\pgfsetdash{}{0pt}%
\pgfpathmoveto{\pgfqpoint{0.647939in}{2.451996in}}%
\pgfpathlineto{\pgfqpoint{0.647939in}{2.451996in}}%
\pgfpathlineto{\pgfqpoint{0.715832in}{2.460513in}}%
\pgfpathlineto{\pgfqpoint{0.783531in}{2.470449in}}%
\pgfpathlineto{\pgfqpoint{0.851002in}{2.481828in}}%
\pgfpathlineto{\pgfqpoint{0.918218in}{2.494632in}}%
\pgfpathlineto{\pgfqpoint{0.985160in}{2.508800in}}%
\pgfpathlineto{\pgfqpoint{1.051825in}{2.524222in}}%
\pgfpathlineto{\pgfqpoint{1.118230in}{2.540735in}}%
\pgfusepath{stroke}%
\end{pgfscope}%
\begin{pgfscope}%
\pgfpathrectangle{\pgfqpoint{0.647939in}{0.492442in}}{\pgfqpoint{3.079299in}{3.079299in}}%
\pgfusepath{clip}%
\pgfsetbuttcap%
\pgfsetroundjoin%
\pgfsetlinewidth{0.301125pt}%
\definecolor{currentstroke}{rgb}{0.500000,0.500000,0.500000}%
\pgfsetstrokecolor{currentstroke}%
\pgfsetstrokeopacity{0.300000}%
\pgfsetdash{}{0pt}%
\pgfpathmoveto{\pgfqpoint{0.647939in}{2.382012in}}%
\pgfpathlineto{\pgfqpoint{0.647939in}{2.382012in}}%
\pgfpathlineto{\pgfqpoint{0.715817in}{2.390646in}}%
\pgfpathlineto{\pgfqpoint{0.783494in}{2.400731in}}%
\pgfpathlineto{\pgfqpoint{0.850933in}{2.412296in}}%
\pgfpathlineto{\pgfqpoint{0.918105in}{2.425326in}}%
\pgfpathlineto{\pgfqpoint{0.984989in}{2.439764in}}%
\pgfpathlineto{\pgfqpoint{1.051580in}{2.455503in}}%
\pgfpathlineto{\pgfqpoint{1.117892in}{2.472384in}}%
\pgfpathlineto{\pgfqpoint{1.183959in}{2.490201in}}%
\pgfpathlineto{\pgfqpoint{1.249837in}{2.508707in}}%
\pgfpathlineto{\pgfqpoint{1.315600in}{2.527619in}}%
\pgfpathlineto{\pgfqpoint{1.381339in}{2.546616in}}%
\pgfpathlineto{\pgfqpoint{1.447154in}{2.565345in}}%
\pgfpathlineto{\pgfqpoint{1.513150in}{2.583424in}}%
\pgfpathlineto{\pgfqpoint{1.579422in}{2.600454in}}%
\pgfpathlineto{\pgfqpoint{1.646050in}{2.616024in}}%
\pgfpathlineto{\pgfqpoint{1.713083in}{2.629733in}}%
\pgfpathlineto{\pgfqpoint{1.780528in}{2.641219in}}%
\pgfpathlineto{\pgfqpoint{1.848351in}{2.650186in}}%
\pgfpathlineto{\pgfqpoint{1.916476in}{2.656450in}}%
\pgfpathlineto{\pgfqpoint{1.984799in}{2.659980in}}%
\pgfpathlineto{\pgfqpoint{2.053210in}{2.660966in}}%
\pgfpathlineto{\pgfqpoint{2.121623in}{2.659858in}}%
\pgfpathlineto{\pgfqpoint{2.190006in}{2.657403in}}%
\pgfpathlineto{\pgfqpoint{2.258382in}{2.654728in}}%
\pgfpathlineto{\pgfqpoint{2.326789in}{2.653450in}}%
\pgfpathlineto{\pgfqpoint{2.395127in}{2.655801in}}%
\pgfpathlineto{\pgfqpoint{2.462871in}{2.664455in}}%
\pgfpathlineto{\pgfqpoint{2.528853in}{2.681687in}}%
\pgfpathlineto{\pgfqpoint{2.591684in}{2.708122in}}%
\pgfpathlineto{\pgfqpoint{2.650629in}{2.742458in}}%
\pgfpathlineto{\pgfqpoint{2.705896in}{2.782576in}}%
\pgfpathlineto{\pgfqpoint{2.758203in}{2.826593in}}%
\pgfpathlineto{\pgfqpoint{2.808302in}{2.873131in}}%
\pgfpathlineto{\pgfqpoint{2.856856in}{2.921305in}}%
\pgfpathlineto{\pgfqpoint{2.904372in}{2.970520in}}%
\pgfpathlineto{\pgfqpoint{2.951227in}{3.020364in}}%
\pgfpathlineto{\pgfqpoint{2.997721in}{3.070556in}}%
\pgfusepath{stroke}%
\end{pgfscope}%
\begin{pgfscope}%
\pgfpathrectangle{\pgfqpoint{0.647939in}{0.492442in}}{\pgfqpoint{3.079299in}{3.079299in}}%
\pgfusepath{clip}%
\pgfsetbuttcap%
\pgfsetroundjoin%
\pgfsetlinewidth{0.301125pt}%
\definecolor{currentstroke}{rgb}{0.500000,0.500000,0.500000}%
\pgfsetstrokecolor{currentstroke}%
\pgfsetstrokeopacity{0.300000}%
\pgfsetdash{}{0pt}%
\pgfpathmoveto{\pgfqpoint{0.647939in}{2.312028in}}%
\pgfpathlineto{\pgfqpoint{0.647939in}{2.312028in}}%
\pgfpathlineto{\pgfqpoint{0.715801in}{2.320783in}}%
\pgfpathlineto{\pgfqpoint{0.783455in}{2.331022in}}%
\pgfpathlineto{\pgfqpoint{0.850861in}{2.342777in}}%
\pgfpathlineto{\pgfqpoint{0.917987in}{2.356041in}}%
\pgfpathlineto{\pgfqpoint{0.984809in}{2.370760in}}%
\pgfpathlineto{\pgfqpoint{1.051321in}{2.386829in}}%
\pgfpathlineto{\pgfqpoint{1.117533in}{2.404093in}}%
\pgfpathlineto{\pgfqpoint{1.183481in}{2.422348in}}%
\pgfpathlineto{\pgfqpoint{1.249218in}{2.441347in}}%
\pgfpathlineto{\pgfqpoint{1.314821in}{2.460807in}}%
\pgfpathlineto{\pgfqpoint{1.380384in}{2.480403in}}%
\pgfpathlineto{\pgfqpoint{1.446012in}{2.499779in}}%
\pgfpathlineto{\pgfqpoint{1.511815in}{2.518548in}}%
\pgfpathlineto{\pgfqpoint{1.577898in}{2.536297in}}%
\pgfpathlineto{\pgfqpoint{1.644351in}{2.552601in}}%
\pgfpathlineto{\pgfqpoint{1.711230in}{2.567036in}}%
\pgfpathlineto{\pgfqpoint{1.778554in}{2.579209in}}%
\pgfpathlineto{\pgfqpoint{1.846292in}{2.588791in}}%
\pgfpathlineto{\pgfqpoint{1.914367in}{2.595556in}}%
\pgfpathlineto{\pgfqpoint{1.982669in}{2.599434in}}%
\pgfpathlineto{\pgfqpoint{2.051075in}{2.600575in}}%
\pgfpathlineto{\pgfqpoint{2.119486in}{2.599417in}}%
\pgfpathlineto{\pgfqpoint{2.187860in}{2.596729in}}%
\pgfpathlineto{\pgfqpoint{2.256222in}{2.593718in}}%
\pgfpathlineto{\pgfqpoint{2.324621in}{2.592196in}}%
\pgfpathlineto{\pgfqpoint{2.392937in}{2.594773in}}%
\pgfpathlineto{\pgfqpoint{2.460457in}{2.604690in}}%
\pgfpathlineto{\pgfqpoint{2.525618in}{2.624510in}}%
\pgfpathlineto{\pgfqpoint{2.586838in}{2.654333in}}%
\pgfpathlineto{\pgfqpoint{2.643683in}{2.692017in}}%
\pgfusepath{stroke}%
\end{pgfscope}%
\begin{pgfscope}%
\pgfpathrectangle{\pgfqpoint{0.647939in}{0.492442in}}{\pgfqpoint{3.079299in}{3.079299in}}%
\pgfusepath{clip}%
\pgfsetbuttcap%
\pgfsetroundjoin%
\pgfsetlinewidth{0.301125pt}%
\definecolor{currentstroke}{rgb}{0.500000,0.500000,0.500000}%
\pgfsetstrokecolor{currentstroke}%
\pgfsetstrokeopacity{0.300000}%
\pgfsetdash{}{0pt}%
\pgfpathmoveto{\pgfqpoint{0.647939in}{2.242044in}}%
\pgfpathlineto{\pgfqpoint{0.647939in}{2.242044in}}%
\pgfpathlineto{\pgfqpoint{0.715785in}{2.250923in}}%
\pgfpathlineto{\pgfqpoint{0.783414in}{2.261320in}}%
\pgfpathlineto{\pgfqpoint{0.850785in}{2.273273in}}%
\pgfpathlineto{\pgfqpoint{0.917862in}{2.286779in}}%
\pgfpathlineto{\pgfqpoint{0.984620in}{2.301788in}}%
\pgfpathlineto{\pgfqpoint{1.051047in}{2.318201in}}%
\pgfpathlineto{\pgfqpoint{1.117154in}{2.335865in}}%
\pgfpathlineto{\pgfqpoint{1.182972in}{2.354578in}}%
\pgfpathlineto{\pgfqpoint{1.248558in}{2.374094in}}%
\pgfpathlineto{\pgfqpoint{1.313987in}{2.394130in}}%
\pgfpathlineto{\pgfqpoint{1.379357in}{2.414360in}}%
\pgfpathlineto{\pgfqpoint{1.444778in}{2.434424in}}%
\pgfpathlineto{\pgfqpoint{1.510367in}{2.453928in}}%
\pgfpathlineto{\pgfqpoint{1.576238in}{2.472449in}}%
\pgfusepath{stroke}%
\end{pgfscope}%
\begin{pgfscope}%
\pgfpathrectangle{\pgfqpoint{0.647939in}{0.492442in}}{\pgfqpoint{3.079299in}{3.079299in}}%
\pgfusepath{clip}%
\pgfsetbuttcap%
\pgfsetroundjoin%
\pgfsetlinewidth{0.301125pt}%
\definecolor{currentstroke}{rgb}{0.500000,0.500000,0.500000}%
\pgfsetstrokecolor{currentstroke}%
\pgfsetstrokeopacity{0.300000}%
\pgfsetdash{}{0pt}%
\pgfpathmoveto{\pgfqpoint{0.647939in}{2.172060in}}%
\pgfpathlineto{\pgfqpoint{0.647939in}{2.172060in}}%
\pgfpathlineto{\pgfqpoint{0.715768in}{2.181067in}}%
\pgfpathlineto{\pgfqpoint{0.783371in}{2.191627in}}%
\pgfpathlineto{\pgfqpoint{0.850706in}{2.203784in}}%
\pgfpathlineto{\pgfqpoint{0.917731in}{2.217541in}}%
\pgfpathlineto{\pgfqpoint{0.984420in}{2.232852in}}%
\pgfpathlineto{\pgfqpoint{1.050758in}{2.249622in}}%
\pgfpathlineto{\pgfqpoint{1.116751in}{2.267703in}}%
\pgfpathlineto{\pgfqpoint{1.182431in}{2.286895in}}%
\pgfpathlineto{\pgfqpoint{1.247852in}{2.306955in}}%
\pgfpathlineto{\pgfqpoint{1.313093in}{2.327597in}}%
\pgfusepath{stroke}%
\end{pgfscope}%
\begin{pgfscope}%
\pgfpathrectangle{\pgfqpoint{0.647939in}{0.492442in}}{\pgfqpoint{3.079299in}{3.079299in}}%
\pgfusepath{clip}%
\pgfsetbuttcap%
\pgfsetroundjoin%
\pgfsetlinewidth{0.301125pt}%
\definecolor{currentstroke}{rgb}{0.500000,0.500000,0.500000}%
\pgfsetstrokecolor{currentstroke}%
\pgfsetstrokeopacity{0.300000}%
\pgfsetdash{}{0pt}%
\pgfpathmoveto{\pgfqpoint{0.647939in}{2.102076in}}%
\pgfpathlineto{\pgfqpoint{0.647939in}{2.102076in}}%
\pgfpathlineto{\pgfqpoint{0.715750in}{2.111214in}}%
\pgfpathlineto{\pgfqpoint{0.783327in}{2.121943in}}%
\pgfpathlineto{\pgfqpoint{0.850622in}{2.134311in}}%
\pgfpathlineto{\pgfqpoint{0.917594in}{2.148327in}}%
\pgfpathlineto{\pgfqpoint{0.984209in}{2.163951in}}%
\pgfpathlineto{\pgfqpoint{1.050451in}{2.181094in}}%
\pgfpathlineto{\pgfqpoint{1.116323in}{2.199611in}}%
\pgfpathlineto{\pgfqpoint{1.181854in}{2.219305in}}%
\pgfpathlineto{\pgfqpoint{1.247098in}{2.239935in}}%
\pgfpathlineto{\pgfqpoint{1.312132in}{2.261217in}}%
\pgfusepath{stroke}%
\end{pgfscope}%
\begin{pgfscope}%
\pgfpathrectangle{\pgfqpoint{0.647939in}{0.492442in}}{\pgfqpoint{3.079299in}{3.079299in}}%
\pgfusepath{clip}%
\pgfsetbuttcap%
\pgfsetroundjoin%
\pgfsetlinewidth{0.301125pt}%
\definecolor{currentstroke}{rgb}{0.500000,0.500000,0.500000}%
\pgfsetstrokecolor{currentstroke}%
\pgfsetstrokeopacity{0.300000}%
\pgfsetdash{}{0pt}%
\pgfpathmoveto{\pgfqpoint{0.647939in}{2.032092in}}%
\pgfpathlineto{\pgfqpoint{0.647939in}{2.032092in}}%
\pgfpathlineto{\pgfqpoint{0.715731in}{2.041366in}}%
\pgfpathlineto{\pgfqpoint{0.783280in}{2.052268in}}%
\pgfpathlineto{\pgfqpoint{0.850535in}{2.064854in}}%
\pgfpathlineto{\pgfqpoint{0.917449in}{2.079139in}}%
\pgfpathlineto{\pgfqpoint{0.983987in}{2.095089in}}%
\pgfpathlineto{\pgfqpoint{1.050126in}{2.112620in}}%
\pgfpathlineto{\pgfqpoint{1.115869in}{2.131592in}}%
\pgfpathlineto{\pgfqpoint{1.181239in}{2.151812in}}%
\pgfpathlineto{\pgfqpoint{1.246289in}{2.173042in}}%
\pgfpathlineto{\pgfqpoint{1.311098in}{2.194999in}}%
\pgfpathlineto{\pgfqpoint{1.375770in}{2.217359in}}%
\pgfpathlineto{\pgfqpoint{1.440430in}{2.239757in}}%
\pgfpathlineto{\pgfqpoint{1.505215in}{2.261786in}}%
\pgfpathlineto{\pgfqpoint{1.570269in}{2.283000in}}%
\pgfpathlineto{\pgfqpoint{1.635729in}{2.302917in}}%
\pgfpathlineto{\pgfqpoint{1.701709in}{2.321025in}}%
\pgfpathlineto{\pgfqpoint{1.768281in}{2.336796in}}%
\pgfpathlineto{\pgfqpoint{1.835458in}{2.349720in}}%
\pgfpathlineto{\pgfqpoint{1.903179in}{2.359350in}}%
\pgfpathlineto{\pgfqpoint{1.971313in}{2.365359in}}%
\pgfpathlineto{\pgfqpoint{2.039674in}{2.367620in}}%
\pgfpathlineto{\pgfqpoint{2.108066in}{2.366299in}}%
\pgfpathlineto{\pgfqpoint{2.176352in}{2.362054in}}%
\pgfpathlineto{\pgfqpoint{2.244542in}{2.356358in}}%
\pgfpathlineto{\pgfqpoint{2.312816in}{2.352394in}}%
\pgfpathlineto{\pgfqpoint{2.380581in}{2.357546in}}%
\pgfpathlineto{\pgfqpoint{2.380581in}{2.357546in}}%
\pgfpathlineto{\pgfqpoint{2.425466in}{2.371404in}}%
\pgfpathlineto{\pgfqpoint{2.467470in}{2.395861in}}%
\pgfpathlineto{\pgfqpoint{2.506009in}{2.427515in}}%
\pgfpathlineto{\pgfqpoint{2.551419in}{2.473202in}}%
\pgfpathlineto{\pgfqpoint{2.596796in}{2.524070in}}%
\pgfpathlineto{\pgfqpoint{2.640855in}{2.576246in}}%
\pgfusepath{stroke}%
\end{pgfscope}%
\begin{pgfscope}%
\pgfpathrectangle{\pgfqpoint{0.647939in}{0.492442in}}{\pgfqpoint{3.079299in}{3.079299in}}%
\pgfusepath{clip}%
\pgfsetbuttcap%
\pgfsetroundjoin%
\pgfsetlinewidth{0.301125pt}%
\definecolor{currentstroke}{rgb}{0.500000,0.500000,0.500000}%
\pgfsetstrokecolor{currentstroke}%
\pgfsetstrokeopacity{0.300000}%
\pgfsetdash{}{0pt}%
\pgfpathmoveto{\pgfqpoint{0.647939in}{1.962108in}}%
\pgfpathlineto{\pgfqpoint{0.647939in}{1.962108in}}%
\pgfpathlineto{\pgfqpoint{0.715712in}{1.971521in}}%
\pgfpathlineto{\pgfqpoint{0.783231in}{1.982602in}}%
\pgfpathlineto{\pgfqpoint{0.850443in}{1.995414in}}%
\pgfpathlineto{\pgfqpoint{0.917296in}{2.009979in}}%
\pgfpathlineto{\pgfqpoint{0.983751in}{2.026268in}}%
\pgfpathlineto{\pgfqpoint{1.049782in}{2.044203in}}%
\pgfpathlineto{\pgfqpoint{1.115384in}{2.063651in}}%
\pgfpathlineto{\pgfqpoint{1.180581in}{2.084423in}}%
\pgfpathlineto{\pgfqpoint{1.245421in}{2.106283in}}%
\pgfpathlineto{\pgfqpoint{1.309985in}{2.128953in}}%
\pgfpathlineto{\pgfqpoint{1.374376in}{2.152109in}}%
\pgfpathlineto{\pgfqpoint{1.438724in}{2.175386in}}%
\pgfpathlineto{\pgfqpoint{1.503174in}{2.198378in}}%
\pgfpathlineto{\pgfqpoint{1.567879in}{2.220633in}}%
\pgfpathlineto{\pgfqpoint{1.632992in}{2.241660in}}%
\pgfpathlineto{\pgfqpoint{1.698642in}{2.260927in}}%
\pgfusepath{stroke}%
\end{pgfscope}%
\begin{pgfscope}%
\pgfpathrectangle{\pgfqpoint{0.647939in}{0.492442in}}{\pgfqpoint{3.079299in}{3.079299in}}%
\pgfusepath{clip}%
\pgfsetbuttcap%
\pgfsetroundjoin%
\pgfsetlinewidth{0.301125pt}%
\definecolor{currentstroke}{rgb}{0.500000,0.500000,0.500000}%
\pgfsetstrokecolor{currentstroke}%
\pgfsetstrokeopacity{0.300000}%
\pgfsetdash{}{0pt}%
\pgfpathmoveto{\pgfqpoint{0.647939in}{1.892124in}}%
\pgfpathlineto{\pgfqpoint{0.647939in}{1.892124in}}%
\pgfpathlineto{\pgfqpoint{0.715692in}{1.901681in}}%
\pgfpathlineto{\pgfqpoint{0.783180in}{1.912947in}}%
\pgfpathlineto{\pgfqpoint{0.850346in}{1.925993in}}%
\pgfpathlineto{\pgfqpoint{0.917135in}{1.940847in}}%
\pgfpathlineto{\pgfqpoint{0.983502in}{1.957489in}}%
\pgfpathlineto{\pgfqpoint{1.049416in}{1.975846in}}%
\pgfpathlineto{\pgfqpoint{1.114868in}{1.995792in}}%
\pgfpathlineto{\pgfqpoint{1.179877in}{2.017143in}}%
\pgfpathlineto{\pgfqpoint{1.244490in}{2.039667in}}%
\pgfpathlineto{\pgfqpoint{1.308784in}{2.063088in}}%
\pgfpathlineto{\pgfqpoint{1.372865in}{2.087088in}}%
\pgfpathlineto{\pgfqpoint{1.436866in}{2.111303in}}%
\pgfpathlineto{\pgfqpoint{1.500938in}{2.135326in}}%
\pgfpathlineto{\pgfqpoint{1.565247in}{2.158705in}}%
\pgfpathlineto{\pgfqpoint{1.629957in}{2.180940in}}%
\pgfpathlineto{\pgfqpoint{1.695220in}{2.201483in}}%
\pgfpathlineto{\pgfqpoint{1.761151in}{2.219745in}}%
\pgfpathlineto{\pgfqpoint{1.827810in}{2.235110in}}%
\pgfpathlineto{\pgfqpoint{1.895174in}{2.246970in}}%
\pgfpathlineto{\pgfqpoint{1.963115in}{2.254791in}}%
\pgfpathlineto{\pgfqpoint{2.031414in}{2.258172in}}%
\pgfpathlineto{\pgfqpoint{2.099784in}{2.256921in}}%
\pgfpathlineto{\pgfqpoint{2.167931in}{2.251169in}}%
\pgfpathlineto{\pgfqpoint{2.235676in}{2.241689in}}%
\pgfpathlineto{\pgfqpoint{2.303334in}{2.231878in}}%
\pgfpathlineto{\pgfqpoint{2.303334in}{2.231878in}}%
\pgfpathlineto{\pgfqpoint{2.330597in}{2.230452in}}%
\pgfpathlineto{\pgfqpoint{2.330597in}{2.230452in}}%
\pgfpathlineto{\pgfqpoint{2.352423in}{2.234923in}}%
\pgfpathlineto{\pgfqpoint{2.373407in}{2.249177in}}%
\pgfpathlineto{\pgfqpoint{2.391322in}{2.265167in}}%
\pgfpathlineto{\pgfqpoint{2.418045in}{2.294215in}}%
\pgfpathlineto{\pgfqpoint{2.458710in}{2.342048in}}%
\pgfusepath{stroke}%
\end{pgfscope}%
\begin{pgfscope}%
\pgfpathrectangle{\pgfqpoint{0.647939in}{0.492442in}}{\pgfqpoint{3.079299in}{3.079299in}}%
\pgfusepath{clip}%
\pgfsetbuttcap%
\pgfsetroundjoin%
\pgfsetlinewidth{0.301125pt}%
\definecolor{currentstroke}{rgb}{0.500000,0.500000,0.500000}%
\pgfsetstrokecolor{currentstroke}%
\pgfsetstrokeopacity{0.300000}%
\pgfsetdash{}{0pt}%
\pgfpathmoveto{\pgfqpoint{0.647939in}{1.822139in}}%
\pgfpathlineto{\pgfqpoint{0.647939in}{1.822139in}}%
\pgfpathlineto{\pgfqpoint{0.715670in}{1.831845in}}%
\pgfpathlineto{\pgfqpoint{0.783126in}{1.843302in}}%
\pgfpathlineto{\pgfqpoint{0.850244in}{1.856590in}}%
\pgfpathlineto{\pgfqpoint{0.916965in}{1.871745in}}%
\pgfpathlineto{\pgfqpoint{0.983238in}{1.888755in}}%
\pgfpathlineto{\pgfqpoint{1.049027in}{1.907553in}}%
\pgfusepath{stroke}%
\end{pgfscope}%
\begin{pgfscope}%
\pgfpathrectangle{\pgfqpoint{0.647939in}{0.492442in}}{\pgfqpoint{3.079299in}{3.079299in}}%
\pgfusepath{clip}%
\pgfsetbuttcap%
\pgfsetroundjoin%
\pgfsetlinewidth{0.301125pt}%
\definecolor{currentstroke}{rgb}{0.500000,0.500000,0.500000}%
\pgfsetstrokecolor{currentstroke}%
\pgfsetstrokeopacity{0.300000}%
\pgfsetdash{}{0pt}%
\pgfpathmoveto{\pgfqpoint{0.647939in}{1.682171in}}%
\pgfpathlineto{\pgfqpoint{0.647939in}{1.682171in}}%
\pgfpathlineto{\pgfqpoint{0.715625in}{1.692186in}}%
\pgfpathlineto{\pgfqpoint{0.783010in}{1.704046in}}%
\pgfpathlineto{\pgfqpoint{0.850023in}{1.717846in}}%
\pgfpathlineto{\pgfqpoint{0.916595in}{1.733639in}}%
\pgfpathlineto{\pgfqpoint{0.982661in}{1.751430in}}%
\pgfpathlineto{\pgfqpoint{1.048171in}{1.771170in}}%
\pgfpathlineto{\pgfqpoint{1.113099in}{1.792754in}}%
\pgfpathlineto{\pgfqpoint{1.177445in}{1.816021in}}%
\pgfpathlineto{\pgfqpoint{1.241242in}{1.840756in}}%
\pgfpathlineto{\pgfqpoint{1.304559in}{1.866700in}}%
\pgfpathlineto{\pgfqpoint{1.367499in}{1.893549in}}%
\pgfpathlineto{\pgfqpoint{1.430198in}{1.920958in}}%
\pgfpathlineto{\pgfqpoint{1.492823in}{1.948536in}}%
\pgfpathlineto{\pgfqpoint{1.555566in}{1.975841in}}%
\pgfpathlineto{\pgfqpoint{1.618637in}{2.002376in}}%
\pgfpathlineto{\pgfqpoint{1.682248in}{2.027576in}}%
\pgfpathlineto{\pgfqpoint{1.746602in}{2.050797in}}%
\pgfpathlineto{\pgfqpoint{1.811860in}{2.071306in}}%
\pgfpathlineto{\pgfqpoint{1.878114in}{2.088267in}}%
\pgfpathlineto{\pgfqpoint{1.945341in}{2.100726in}}%
\pgfpathlineto{\pgfqpoint{2.013346in}{2.107544in}}%
\pgfpathlineto{\pgfqpoint{2.081641in}{2.107121in}}%
\pgfpathlineto{\pgfqpoint{2.148585in}{2.095544in}}%
\pgfpathlineto{\pgfqpoint{2.148585in}{2.095544in}}%
\pgfpathlineto{\pgfqpoint{2.172565in}{2.085216in}}%
\pgfpathlineto{\pgfqpoint{2.172565in}{2.085216in}}%
\pgfpathlineto{\pgfqpoint{2.185879in}{2.072124in}}%
\pgfpathlineto{\pgfqpoint{2.185879in}{2.072124in}}%
\pgfusepath{stroke}%
\end{pgfscope}%
\begin{pgfscope}%
\pgfpathrectangle{\pgfqpoint{0.647939in}{0.492442in}}{\pgfqpoint{3.079299in}{3.079299in}}%
\pgfusepath{clip}%
\pgfsetbuttcap%
\pgfsetroundjoin%
\pgfsetlinewidth{0.301125pt}%
\definecolor{currentstroke}{rgb}{0.500000,0.500000,0.500000}%
\pgfsetstrokecolor{currentstroke}%
\pgfsetstrokeopacity{0.300000}%
\pgfsetdash{}{0pt}%
\pgfpathmoveto{\pgfqpoint{0.647939in}{1.612187in}}%
\pgfpathlineto{\pgfqpoint{0.647939in}{1.612187in}}%
\pgfpathlineto{\pgfqpoint{0.715600in}{1.622365in}}%
\pgfpathlineto{\pgfqpoint{0.782947in}{1.634436in}}%
\pgfpathlineto{\pgfqpoint{0.849904in}{1.648506in}}%
\pgfpathlineto{\pgfqpoint{0.916393in}{1.664637in}}%
\pgfpathlineto{\pgfqpoint{0.982345in}{1.682845in}}%
\pgfpathlineto{\pgfqpoint{1.047701in}{1.703089in}}%
\pgfpathlineto{\pgfqpoint{1.112425in}{1.725273in}}%
\pgfpathlineto{\pgfqpoint{1.176511in}{1.749244in}}%
\pgfpathlineto{\pgfqpoint{1.239984in}{1.774798in}}%
\pgfpathlineto{\pgfqpoint{1.302907in}{1.801680in}}%
\pgfpathlineto{\pgfqpoint{1.365380in}{1.829596in}}%
\pgfusepath{stroke}%
\end{pgfscope}%
\begin{pgfscope}%
\pgfpathrectangle{\pgfqpoint{0.647939in}{0.492442in}}{\pgfqpoint{3.079299in}{3.079299in}}%
\pgfusepath{clip}%
\pgfsetbuttcap%
\pgfsetroundjoin%
\pgfsetlinewidth{0.301125pt}%
\definecolor{currentstroke}{rgb}{0.500000,0.500000,0.500000}%
\pgfsetstrokecolor{currentstroke}%
\pgfsetstrokeopacity{0.300000}%
\pgfsetdash{}{0pt}%
\pgfpathmoveto{\pgfqpoint{0.647939in}{1.542203in}}%
\pgfpathlineto{\pgfqpoint{0.647939in}{1.542203in}}%
\pgfpathlineto{\pgfqpoint{0.715574in}{1.552549in}}%
\pgfpathlineto{\pgfqpoint{0.782881in}{1.564839in}}%
\pgfpathlineto{\pgfqpoint{0.849778in}{1.579189in}}%
\pgfpathlineto{\pgfqpoint{0.916180in}{1.595673in}}%
\pgfpathlineto{\pgfqpoint{0.982009in}{1.614316in}}%
\pgfpathlineto{\pgfqpoint{1.047198in}{1.635087in}}%
\pgfpathlineto{\pgfqpoint{1.111702in}{1.657899in}}%
\pgfpathlineto{\pgfqpoint{1.175505in}{1.682611in}}%
\pgfpathlineto{\pgfqpoint{1.238623in}{1.709025in}}%
\pgfpathlineto{\pgfqpoint{1.301113in}{1.736899in}}%
\pgfpathlineto{\pgfqpoint{1.363068in}{1.765946in}}%
\pgfpathlineto{\pgfqpoint{1.424621in}{1.795839in}}%
\pgfpathlineto{\pgfqpoint{1.485941in}{1.826207in}}%
\pgfpathlineto{\pgfqpoint{1.547230in}{1.856638in}}%
\pgfpathlineto{\pgfqpoint{1.608718in}{1.886663in}}%
\pgfpathlineto{\pgfqpoint{1.670653in}{1.915744in}}%
\pgfpathlineto{\pgfqpoint{1.733296in}{1.943256in}}%
\pgfpathlineto{\pgfqpoint{1.796894in}{1.968453in}}%
\pgfpathlineto{\pgfqpoint{1.861664in}{1.990424in}}%
\pgfpathlineto{\pgfqpoint{1.927732in}{2.008025in}}%
\pgfpathlineto{\pgfqpoint{1.995056in}{2.019636in}}%
\pgfpathlineto{\pgfqpoint{2.063128in}{2.021983in}}%
\pgfpathlineto{\pgfqpoint{2.063128in}{2.021983in}}%
\pgfpathlineto{\pgfqpoint{2.097577in}{2.016583in}}%
\pgfpathlineto{\pgfqpoint{2.097577in}{2.016583in}}%
\pgfpathlineto{\pgfqpoint{2.115775in}{2.007906in}}%
\pgfpathlineto{\pgfqpoint{2.115775in}{2.007906in}}%
\pgfusepath{stroke}%
\end{pgfscope}%
\begin{pgfscope}%
\pgfpathrectangle{\pgfqpoint{0.647939in}{0.492442in}}{\pgfqpoint{3.079299in}{3.079299in}}%
\pgfusepath{clip}%
\pgfsetbuttcap%
\pgfsetroundjoin%
\pgfsetlinewidth{0.301125pt}%
\definecolor{currentstroke}{rgb}{0.500000,0.500000,0.500000}%
\pgfsetstrokecolor{currentstroke}%
\pgfsetstrokeopacity{0.300000}%
\pgfsetdash{}{0pt}%
\pgfpathmoveto{\pgfqpoint{0.647939in}{1.472219in}}%
\pgfpathlineto{\pgfqpoint{0.647939in}{1.472219in}}%
\pgfpathlineto{\pgfqpoint{0.715547in}{1.482738in}}%
\pgfpathlineto{\pgfqpoint{0.782812in}{1.495255in}}%
\pgfpathlineto{\pgfqpoint{0.849644in}{1.509897in}}%
\pgfpathlineto{\pgfqpoint{0.915953in}{1.526748in}}%
\pgfpathlineto{\pgfqpoint{0.981650in}{1.545845in}}%
\pgfpathlineto{\pgfqpoint{1.046660in}{1.567168in}}%
\pgfpathlineto{\pgfqpoint{1.110925in}{1.590641in}}%
\pgfpathlineto{\pgfqpoint{1.174419in}{1.616130in}}%
\pgfpathlineto{\pgfqpoint{1.237150in}{1.643450in}}%
\pgfpathlineto{\pgfqpoint{1.299164in}{1.672365in}}%
\pgfpathlineto{\pgfqpoint{1.360546in}{1.702600in}}%
\pgfpathlineto{\pgfqpoint{1.421425in}{1.733839in}}%
\pgfpathlineto{\pgfqpoint{1.481967in}{1.765729in}}%
\pgfusepath{stroke}%
\end{pgfscope}%
\begin{pgfscope}%
\pgfpathrectangle{\pgfqpoint{0.647939in}{0.492442in}}{\pgfqpoint{3.079299in}{3.079299in}}%
\pgfusepath{clip}%
\pgfsetbuttcap%
\pgfsetroundjoin%
\pgfsetlinewidth{0.301125pt}%
\definecolor{currentstroke}{rgb}{0.500000,0.500000,0.500000}%
\pgfsetstrokecolor{currentstroke}%
\pgfsetstrokeopacity{0.300000}%
\pgfsetdash{}{0pt}%
\pgfpathmoveto{\pgfqpoint{0.647939in}{1.402235in}}%
\pgfpathlineto{\pgfqpoint{0.647939in}{1.402235in}}%
\pgfpathlineto{\pgfqpoint{0.715519in}{1.412933in}}%
\pgfpathlineto{\pgfqpoint{0.782738in}{1.425685in}}%
\pgfpathlineto{\pgfqpoint{0.849503in}{1.440630in}}%
\pgfpathlineto{\pgfqpoint{0.915712in}{1.457864in}}%
\pgfpathlineto{\pgfqpoint{0.981268in}{1.477436in}}%
\pgfpathlineto{\pgfqpoint{1.046083in}{1.499338in}}%
\pgfpathlineto{\pgfqpoint{1.110090in}{1.523505in}}%
\pgfpathlineto{\pgfqpoint{1.173247in}{1.549813in}}%
\pgfusepath{stroke}%
\end{pgfscope}%
\begin{pgfscope}%
\pgfpathrectangle{\pgfqpoint{0.647939in}{0.492442in}}{\pgfqpoint{3.079299in}{3.079299in}}%
\pgfusepath{clip}%
\pgfsetbuttcap%
\pgfsetroundjoin%
\pgfsetlinewidth{0.301125pt}%
\definecolor{currentstroke}{rgb}{0.500000,0.500000,0.500000}%
\pgfsetstrokecolor{currentstroke}%
\pgfsetstrokeopacity{0.300000}%
\pgfsetdash{}{0pt}%
\pgfpathmoveto{\pgfqpoint{0.647939in}{1.332251in}}%
\pgfpathlineto{\pgfqpoint{0.647939in}{1.332251in}}%
\pgfpathlineto{\pgfqpoint{0.715488in}{1.343134in}}%
\pgfpathlineto{\pgfqpoint{0.782661in}{1.356131in}}%
\pgfpathlineto{\pgfqpoint{0.849353in}{1.371391in}}%
\pgfpathlineto{\pgfqpoint{0.915456in}{1.389024in}}%
\pgfpathlineto{\pgfqpoint{0.980860in}{1.409093in}}%
\pgfpathlineto{\pgfqpoint{1.045465in}{1.431602in}}%
\pgfpathlineto{\pgfqpoint{1.109190in}{1.456498in}}%
\pgfpathlineto{\pgfqpoint{1.171979in}{1.483667in}}%
\pgfusepath{stroke}%
\end{pgfscope}%
\begin{pgfscope}%
\pgfpathrectangle{\pgfqpoint{0.647939in}{0.492442in}}{\pgfqpoint{3.079299in}{3.079299in}}%
\pgfusepath{clip}%
\pgfsetbuttcap%
\pgfsetroundjoin%
\pgfsetlinewidth{0.301125pt}%
\definecolor{currentstroke}{rgb}{0.500000,0.500000,0.500000}%
\pgfsetstrokecolor{currentstroke}%
\pgfsetstrokeopacity{0.300000}%
\pgfsetdash{}{0pt}%
\pgfpathmoveto{\pgfqpoint{0.647939in}{1.262267in}}%
\pgfpathlineto{\pgfqpoint{0.647939in}{1.262267in}}%
\pgfpathlineto{\pgfqpoint{0.715457in}{1.273342in}}%
\pgfpathlineto{\pgfqpoint{0.782579in}{1.286591in}}%
\pgfpathlineto{\pgfqpoint{0.849195in}{1.302180in}}%
\pgfpathlineto{\pgfqpoint{0.915183in}{1.320231in}}%
\pgfpathlineto{\pgfqpoint{0.980424in}{1.340820in}}%
\pgfpathlineto{\pgfqpoint{1.044802in}{1.363965in}}%
\pgfpathlineto{\pgfqpoint{1.108219in}{1.389627in}}%
\pgfpathlineto{\pgfqpoint{1.170606in}{1.417703in}}%
\pgfpathlineto{\pgfqpoint{1.231928in}{1.448037in}}%
\pgfpathlineto{\pgfqpoint{1.292190in}{1.480427in}}%
\pgfpathlineto{\pgfqpoint{1.351446in}{1.514627in}}%
\pgfpathlineto{\pgfqpoint{1.409790in}{1.550365in}}%
\pgfpathlineto{\pgfqpoint{1.467362in}{1.587335in}}%
\pgfpathlineto{\pgfqpoint{1.524344in}{1.625210in}}%
\pgfpathlineto{\pgfqpoint{1.580963in}{1.663628in}}%
\pgfpathlineto{\pgfqpoint{1.637492in}{1.702178in}}%
\pgfpathlineto{\pgfqpoint{1.694241in}{1.740401in}}%
\pgfpathlineto{\pgfqpoint{1.751544in}{1.777773in}}%
\pgfpathlineto{\pgfqpoint{1.809770in}{1.813660in}}%
\pgfpathlineto{\pgfqpoint{1.869319in}{1.847258in}}%
\pgfpathlineto{\pgfqpoint{1.930612in}{1.877455in}}%
\pgfusepath{stroke}%
\end{pgfscope}%
\begin{pgfscope}%
\pgfpathrectangle{\pgfqpoint{0.647939in}{0.492442in}}{\pgfqpoint{3.079299in}{3.079299in}}%
\pgfusepath{clip}%
\pgfsetbuttcap%
\pgfsetroundjoin%
\pgfsetlinewidth{0.301125pt}%
\definecolor{currentstroke}{rgb}{0.500000,0.500000,0.500000}%
\pgfsetstrokecolor{currentstroke}%
\pgfsetstrokeopacity{0.300000}%
\pgfsetdash{}{0pt}%
\pgfpathmoveto{\pgfqpoint{0.647939in}{1.192283in}}%
\pgfpathlineto{\pgfqpoint{0.647939in}{1.192283in}}%
\pgfpathlineto{\pgfqpoint{0.715423in}{1.203556in}}%
\pgfpathlineto{\pgfqpoint{0.782492in}{1.217069in}}%
\pgfpathlineto{\pgfqpoint{0.849026in}{1.232999in}}%
\pgfpathlineto{\pgfqpoint{0.914893in}{1.251486in}}%
\pgfpathlineto{\pgfqpoint{0.979957in}{1.272620in}}%
\pgfpathlineto{\pgfqpoint{1.044088in}{1.296434in}}%
\pgfpathlineto{\pgfqpoint{1.107171in}{1.322901in}}%
\pgfpathlineto{\pgfqpoint{1.169116in}{1.351931in}}%
\pgfpathlineto{\pgfqpoint{1.229871in}{1.383377in}}%
\pgfpathlineto{\pgfqpoint{1.289424in}{1.417046in}}%
\pgfpathlineto{\pgfqpoint{1.347810in}{1.452704in}}%
\pgfpathlineto{\pgfqpoint{1.405110in}{1.490086in}}%
\pgfpathlineto{\pgfqpoint{1.461450in}{1.528903in}}%
\pgfpathlineto{\pgfqpoint{1.517004in}{1.568840in}}%
\pgfusepath{stroke}%
\end{pgfscope}%
\begin{pgfscope}%
\pgfpathrectangle{\pgfqpoint{0.647939in}{0.492442in}}{\pgfqpoint{3.079299in}{3.079299in}}%
\pgfusepath{clip}%
\pgfsetbuttcap%
\pgfsetroundjoin%
\pgfsetlinewidth{0.301125pt}%
\definecolor{currentstroke}{rgb}{0.500000,0.500000,0.500000}%
\pgfsetstrokecolor{currentstroke}%
\pgfsetstrokeopacity{0.300000}%
\pgfsetdash{}{0pt}%
\pgfpathmoveto{\pgfqpoint{0.647939in}{1.122299in}}%
\pgfpathlineto{\pgfqpoint{0.647939in}{1.122299in}}%
\pgfpathlineto{\pgfqpoint{0.715388in}{1.133778in}}%
\pgfpathlineto{\pgfqpoint{0.782401in}{1.147564in}}%
\pgfpathlineto{\pgfqpoint{0.848847in}{1.163850in}}%
\pgfpathlineto{\pgfqpoint{0.914582in}{1.182793in}}%
\pgfpathlineto{\pgfqpoint{0.979456in}{1.204498in}}%
\pgfusepath{stroke}%
\end{pgfscope}%
\begin{pgfscope}%
\pgfpathrectangle{\pgfqpoint{0.647939in}{0.492442in}}{\pgfqpoint{3.079299in}{3.079299in}}%
\pgfusepath{clip}%
\pgfsetbuttcap%
\pgfsetroundjoin%
\pgfsetlinewidth{0.301125pt}%
\definecolor{currentstroke}{rgb}{0.500000,0.500000,0.500000}%
\pgfsetstrokecolor{currentstroke}%
\pgfsetstrokeopacity{0.300000}%
\pgfsetdash{}{0pt}%
\pgfpathmoveto{\pgfqpoint{0.647939in}{1.052315in}}%
\pgfpathlineto{\pgfqpoint{0.647939in}{1.052315in}}%
\pgfpathlineto{\pgfqpoint{0.715351in}{1.064007in}}%
\pgfpathlineto{\pgfqpoint{0.782303in}{1.078077in}}%
\pgfpathlineto{\pgfqpoint{0.848656in}{1.094735in}}%
\pgfpathlineto{\pgfqpoint{0.914250in}{1.114154in}}%
\pgfpathlineto{\pgfqpoint{0.978918in}{1.136459in}}%
\pgfpathlineto{\pgfqpoint{1.042490in}{1.161713in}}%
\pgfpathlineto{\pgfqpoint{1.104809in}{1.189917in}}%
\pgfpathlineto{\pgfqpoint{1.165739in}{1.221004in}}%
\pgfpathlineto{\pgfqpoint{1.225184in}{1.254844in}}%
\pgfpathlineto{\pgfqpoint{1.283090in}{1.291256in}}%
\pgfpathlineto{\pgfqpoint{1.339455in}{1.330015in}}%
\pgfpathlineto{\pgfqpoint{1.394327in}{1.370870in}}%
\pgfpathlineto{\pgfqpoint{1.447822in}{1.413513in}}%
\pgfpathlineto{\pgfqpoint{1.500110in}{1.457623in}}%
\pgfpathlineto{\pgfqpoint{1.551396in}{1.502897in}}%
\pgfpathlineto{\pgfqpoint{1.601956in}{1.548983in}}%
\pgfpathlineto{\pgfqpoint{1.652102in}{1.595503in}}%
\pgfpathlineto{\pgfqpoint{1.702197in}{1.642083in}}%
\pgfusepath{stroke}%
\end{pgfscope}%
\begin{pgfscope}%
\pgfpathrectangle{\pgfqpoint{0.647939in}{0.492442in}}{\pgfqpoint{3.079299in}{3.079299in}}%
\pgfusepath{clip}%
\pgfsetbuttcap%
\pgfsetroundjoin%
\pgfsetlinewidth{0.301125pt}%
\definecolor{currentstroke}{rgb}{0.500000,0.500000,0.500000}%
\pgfsetstrokecolor{currentstroke}%
\pgfsetstrokeopacity{0.300000}%
\pgfsetdash{}{0pt}%
\pgfpathmoveto{\pgfqpoint{0.647939in}{0.982331in}}%
\pgfpathlineto{\pgfqpoint{0.647939in}{0.982331in}}%
\pgfpathlineto{\pgfqpoint{0.715312in}{0.994244in}}%
\pgfpathlineto{\pgfqpoint{0.782200in}{1.008610in}}%
\pgfpathlineto{\pgfqpoint{0.848453in}{1.025656in}}%
\pgfpathlineto{\pgfqpoint{0.913895in}{1.045574in}}%
\pgfpathlineto{\pgfqpoint{0.978340in}{1.068508in}}%
\pgfpathlineto{\pgfqpoint{1.041595in}{1.094538in}}%
\pgfpathlineto{\pgfqpoint{1.103477in}{1.123678in}}%
\pgfpathlineto{\pgfqpoint{1.163825in}{1.155871in}}%
\pgfpathlineto{\pgfqpoint{1.222515in}{1.190994in}}%
\pgfusepath{stroke}%
\end{pgfscope}%
\begin{pgfscope}%
\pgfpathrectangle{\pgfqpoint{0.647939in}{0.492442in}}{\pgfqpoint{3.079299in}{3.079299in}}%
\pgfusepath{clip}%
\pgfsetbuttcap%
\pgfsetroundjoin%
\pgfsetlinewidth{0.301125pt}%
\definecolor{currentstroke}{rgb}{0.500000,0.500000,0.500000}%
\pgfsetstrokecolor{currentstroke}%
\pgfsetstrokeopacity{0.300000}%
\pgfsetdash{}{0pt}%
\pgfpathmoveto{\pgfqpoint{0.647939in}{0.912347in}}%
\pgfpathlineto{\pgfqpoint{0.647939in}{0.912347in}}%
\pgfpathlineto{\pgfqpoint{0.715270in}{0.924489in}}%
\pgfpathlineto{\pgfqpoint{0.782091in}{0.939163in}}%
\pgfpathlineto{\pgfqpoint{0.848236in}{0.956615in}}%
\pgfpathlineto{\pgfqpoint{0.913514in}{0.977056in}}%
\pgfpathlineto{\pgfqpoint{0.977717in}{1.000650in}}%
\pgfpathlineto{\pgfqpoint{1.040627in}{1.027495in}}%
\pgfpathlineto{\pgfqpoint{1.102031in}{1.057619in}}%
\pgfusepath{stroke}%
\end{pgfscope}%
\begin{pgfscope}%
\pgfpathrectangle{\pgfqpoint{0.647939in}{0.492442in}}{\pgfqpoint{3.079299in}{3.079299in}}%
\pgfusepath{clip}%
\pgfsetbuttcap%
\pgfsetroundjoin%
\pgfsetlinewidth{0.301125pt}%
\definecolor{currentstroke}{rgb}{0.500000,0.500000,0.500000}%
\pgfsetstrokecolor{currentstroke}%
\pgfsetstrokeopacity{0.300000}%
\pgfsetdash{}{0pt}%
\pgfpathmoveto{\pgfqpoint{0.647939in}{0.842362in}}%
\pgfpathlineto{\pgfqpoint{0.647939in}{0.842362in}}%
\pgfpathlineto{\pgfqpoint{0.715226in}{0.854743in}}%
\pgfpathlineto{\pgfqpoint{0.781974in}{0.869738in}}%
\pgfpathlineto{\pgfqpoint{0.848004in}{0.887614in}}%
\pgfpathlineto{\pgfqpoint{0.913106in}{0.908604in}}%
\pgfpathlineto{\pgfqpoint{0.977046in}{0.932891in}}%
\pgfpathlineto{\pgfqpoint{1.039579in}{0.960594in}}%
\pgfpathlineto{\pgfqpoint{1.100459in}{0.991751in}}%
\pgfpathlineto{\pgfqpoint{1.159466in}{1.026320in}}%
\pgfpathlineto{\pgfqpoint{1.216419in}{1.064176in}}%
\pgfpathlineto{\pgfqpoint{1.271197in}{1.105123in}}%
\pgfpathlineto{\pgfqpoint{1.323761in}{1.148882in}}%
\pgfpathlineto{\pgfqpoint{1.374155in}{1.195109in}}%
\pgfpathlineto{\pgfqpoint{1.422504in}{1.243482in}}%
\pgfpathlineto{\pgfqpoint{1.469028in}{1.293609in}}%
\pgfpathlineto{\pgfqpoint{1.514009in}{1.345120in}}%
\pgfpathlineto{\pgfqpoint{1.557801in}{1.397647in}}%
\pgfusepath{stroke}%
\end{pgfscope}%
\begin{pgfscope}%
\pgfpathrectangle{\pgfqpoint{0.647939in}{0.492442in}}{\pgfqpoint{3.079299in}{3.079299in}}%
\pgfusepath{clip}%
\pgfsetbuttcap%
\pgfsetroundjoin%
\pgfsetlinewidth{0.301125pt}%
\definecolor{currentstroke}{rgb}{0.500000,0.500000,0.500000}%
\pgfsetstrokecolor{currentstroke}%
\pgfsetstrokeopacity{0.300000}%
\pgfsetdash{}{0pt}%
\pgfpathmoveto{\pgfqpoint{0.647939in}{0.772378in}}%
\pgfpathlineto{\pgfqpoint{0.647939in}{0.772378in}}%
\pgfpathlineto{\pgfqpoint{0.715179in}{0.785006in}}%
\pgfpathlineto{\pgfqpoint{0.781850in}{0.800336in}}%
\pgfpathlineto{\pgfqpoint{0.847756in}{0.818656in}}%
\pgfpathlineto{\pgfqpoint{0.912667in}{0.840221in}}%
\pgfpathlineto{\pgfqpoint{0.976322in}{0.865238in}}%
\pgfpathlineto{\pgfqpoint{1.038442in}{0.893841in}}%
\pgfpathlineto{\pgfqpoint{1.098748in}{0.926083in}}%
\pgfpathlineto{\pgfqpoint{1.156986in}{0.961921in}}%
\pgfpathlineto{\pgfqpoint{1.212945in}{1.001222in}}%
\pgfpathlineto{\pgfqpoint{1.266489in}{1.043764in}}%
\pgfusepath{stroke}%
\end{pgfscope}%
\begin{pgfscope}%
\pgfpathrectangle{\pgfqpoint{0.647939in}{0.492442in}}{\pgfqpoint{3.079299in}{3.079299in}}%
\pgfusepath{clip}%
\pgfsetbuttcap%
\pgfsetroundjoin%
\pgfsetlinewidth{0.301125pt}%
\definecolor{currentstroke}{rgb}{0.500000,0.500000,0.500000}%
\pgfsetstrokecolor{currentstroke}%
\pgfsetstrokeopacity{0.300000}%
\pgfsetdash{}{0pt}%
\pgfpathmoveto{\pgfqpoint{0.647939in}{0.702394in}}%
\pgfpathlineto{\pgfqpoint{0.647939in}{0.702394in}}%
\pgfpathlineto{\pgfqpoint{0.715129in}{0.715280in}}%
\pgfpathlineto{\pgfqpoint{0.781717in}{0.730959in}}%
\pgfpathlineto{\pgfqpoint{0.847490in}{0.749743in}}%
\pgfpathlineto{\pgfqpoint{0.912194in}{0.771913in}}%
\pgfpathlineto{\pgfqpoint{0.975538in}{0.797696in}}%
\pgfpathlineto{\pgfqpoint{1.037207in}{0.827246in}}%
\pgfpathlineto{\pgfqpoint{1.096884in}{0.860625in}}%
\pgfpathlineto{\pgfqpoint{1.154279in}{0.897785in}}%
\pgfpathlineto{\pgfqpoint{1.209155in}{0.938580in}}%
\pgfpathlineto{\pgfqpoint{1.261368in}{0.982733in}}%
\pgfusepath{stroke}%
\end{pgfscope}%
\begin{pgfscope}%
\pgfpathrectangle{\pgfqpoint{0.647939in}{0.492442in}}{\pgfqpoint{3.079299in}{3.079299in}}%
\pgfusepath{clip}%
\pgfsetbuttcap%
\pgfsetroundjoin%
\pgfsetlinewidth{0.301125pt}%
\definecolor{currentstroke}{rgb}{0.500000,0.500000,0.500000}%
\pgfsetstrokecolor{currentstroke}%
\pgfsetstrokeopacity{0.300000}%
\pgfsetdash{}{0pt}%
\pgfpathmoveto{\pgfqpoint{0.647939in}{0.632410in}}%
\pgfpathlineto{\pgfqpoint{0.647939in}{0.632410in}}%
\pgfpathlineto{\pgfqpoint{0.715076in}{0.645564in}}%
\pgfpathlineto{\pgfqpoint{0.781576in}{0.661608in}}%
\pgfpathlineto{\pgfqpoint{0.847205in}{0.680879in}}%
\pgfpathlineto{\pgfqpoint{0.911684in}{0.703683in}}%
\pgfpathlineto{\pgfqpoint{0.974689in}{0.730273in}}%
\pgfpathlineto{\pgfqpoint{1.035864in}{0.760818in}}%
\pgfpathlineto{\pgfqpoint{1.094852in}{0.795386in}}%
\pgfpathlineto{\pgfqpoint{1.151325in}{0.833921in}}%
\pgfpathlineto{\pgfqpoint{1.205027in}{0.876245in}}%
\pgfpathlineto{\pgfqpoint{1.255812in}{0.922014in}}%
\pgfusepath{stroke}%
\end{pgfscope}%
\begin{pgfscope}%
\pgfpathrectangle{\pgfqpoint{0.647939in}{0.492442in}}{\pgfqpoint{3.079299in}{3.079299in}}%
\pgfusepath{clip}%
\pgfsetbuttcap%
\pgfsetroundjoin%
\pgfsetlinewidth{0.301125pt}%
\definecolor{currentstroke}{rgb}{0.500000,0.500000,0.500000}%
\pgfsetstrokecolor{currentstroke}%
\pgfsetstrokeopacity{0.300000}%
\pgfsetdash{}{0pt}%
\pgfpathmoveto{\pgfqpoint{1.922097in}{3.422686in}}%
\pgfpathlineto{\pgfqpoint{1.990499in}{3.424466in}}%
\pgfpathlineto{\pgfqpoint{2.058923in}{3.424886in}}%
\pgfpathlineto{\pgfqpoint{2.127348in}{3.424299in}}%
\pgfpathlineto{\pgfqpoint{2.195767in}{3.423166in}}%
\pgfpathlineto{\pgfqpoint{2.264187in}{3.422050in}}%
\pgfpathlineto{\pgfqpoint{2.332612in}{3.421614in}}%
\pgfpathlineto{\pgfqpoint{2.401027in}{3.422592in}}%
\pgfpathlineto{\pgfqpoint{2.469369in}{3.425737in}}%
\pgfpathlineto{\pgfqpoint{2.537509in}{3.431773in}}%
\pgfusepath{stroke}%
\end{pgfscope}%
\begin{pgfscope}%
\pgfpathrectangle{\pgfqpoint{0.647939in}{0.492442in}}{\pgfqpoint{3.079299in}{3.079299in}}%
\pgfusepath{clip}%
\pgfsetbuttcap%
\pgfsetroundjoin%
\pgfsetlinewidth{0.301125pt}%
\definecolor{currentstroke}{rgb}{0.500000,0.500000,0.500000}%
\pgfsetstrokecolor{currentstroke}%
\pgfsetstrokeopacity{0.300000}%
\pgfsetdash{}{0pt}%
\pgfpathmoveto{\pgfqpoint{0.647939in}{3.135081in}}%
\pgfpathlineto{\pgfqpoint{0.652039in}{3.135502in}}%
\pgfpathlineto{\pgfqpoint{0.720043in}{3.143087in}}%
\pgfpathlineto{\pgfqpoint{0.787907in}{3.151837in}}%
\pgfpathlineto{\pgfqpoint{0.855613in}{3.161741in}}%
\pgfpathlineto{\pgfqpoint{0.923147in}{3.172750in}}%
\pgfpathlineto{\pgfqpoint{0.990509in}{3.184776in}}%
\pgfusepath{stroke}%
\end{pgfscope}%
\begin{pgfscope}%
\pgfpathrectangle{\pgfqpoint{0.647939in}{0.492442in}}{\pgfqpoint{3.079299in}{3.079299in}}%
\pgfusepath{clip}%
\pgfsetbuttcap%
\pgfsetroundjoin%
\pgfsetlinewidth{0.301125pt}%
\definecolor{currentstroke}{rgb}{0.500000,0.500000,0.500000}%
\pgfsetstrokecolor{currentstroke}%
\pgfsetstrokeopacity{0.300000}%
\pgfsetdash{}{0pt}%
\pgfpathmoveto{\pgfqpoint{0.647939in}{2.994672in}}%
\pgfpathlineto{\pgfqpoint{0.652091in}{2.995109in}}%
\pgfpathlineto{\pgfqpoint{0.720074in}{3.002883in}}%
\pgfpathlineto{\pgfqpoint{0.787907in}{3.011869in}}%
\pgfpathlineto{\pgfqpoint{0.855570in}{3.022060in}}%
\pgfpathlineto{\pgfqpoint{0.923048in}{3.033412in}}%
\pgfpathlineto{\pgfqpoint{0.990336in}{3.045839in}}%
\pgfpathlineto{\pgfqpoint{1.057445in}{3.059209in}}%
\pgfpathlineto{\pgfqpoint{1.124395in}{3.073350in}}%
\pgfpathlineto{\pgfqpoint{1.191224in}{3.088058in}}%
\pgfpathlineto{\pgfqpoint{1.257981in}{3.103095in}}%
\pgfusepath{stroke}%
\end{pgfscope}%
\begin{pgfscope}%
\pgfpathrectangle{\pgfqpoint{0.647939in}{0.492442in}}{\pgfqpoint{3.079299in}{3.079299in}}%
\pgfusepath{clip}%
\pgfsetbuttcap%
\pgfsetroundjoin%
\pgfsetlinewidth{0.301125pt}%
\definecolor{currentstroke}{rgb}{0.500000,0.500000,0.500000}%
\pgfsetstrokecolor{currentstroke}%
\pgfsetstrokeopacity{0.300000}%
\pgfsetdash{}{0pt}%
\pgfpathmoveto{\pgfqpoint{3.517286in}{1.822139in}}%
\pgfpathlineto{\pgfqpoint{3.465661in}{1.867030in}}%
\pgfpathlineto{\pgfqpoint{3.415881in}{1.913959in}}%
\pgfpathlineto{\pgfqpoint{3.368014in}{1.962836in}}%
\pgfpathlineto{\pgfqpoint{3.322170in}{2.013615in}}%
\pgfpathlineto{\pgfqpoint{3.278527in}{2.066294in}}%
\pgfpathlineto{\pgfqpoint{3.237338in}{2.120908in}}%
\pgfpathlineto{\pgfqpoint{3.198975in}{2.177536in}}%
\pgfpathlineto{\pgfqpoint{3.163941in}{2.236276in}}%
\pgfpathlineto{\pgfqpoint{3.132905in}{2.297213in}}%
\pgfpathlineto{\pgfqpoint{3.106716in}{2.360367in}}%
\pgfpathlineto{\pgfqpoint{3.086341in}{2.425605in}}%
\pgfpathlineto{\pgfqpoint{3.072730in}{2.492552in}}%
\pgfpathlineto{\pgfqpoint{3.066574in}{2.560572in}}%
\pgfpathlineto{\pgfqpoint{3.068071in}{2.628852in}}%
\pgfpathlineto{\pgfqpoint{3.076847in}{2.696604in}}%
\pgfpathlineto{\pgfqpoint{3.092117in}{2.763221in}}%
\pgfpathlineto{\pgfqpoint{3.112908in}{2.828350in}}%
\pgfpathlineto{\pgfqpoint{3.138276in}{2.891851in}}%
\pgfpathlineto{\pgfqpoint{3.167427in}{2.953721in}}%
\pgfpathlineto{\pgfqpoint{3.199731in}{3.014016in}}%
\pgfpathlineto{\pgfqpoint{3.234721in}{3.072801in}}%
\pgfusepath{stroke}%
\end{pgfscope}%
\begin{pgfscope}%
\pgfpathrectangle{\pgfqpoint{0.647939in}{0.492442in}}{\pgfqpoint{3.079299in}{3.079299in}}%
\pgfusepath{clip}%
\pgfsetbuttcap%
\pgfsetroundjoin%
\pgfsetlinewidth{0.301125pt}%
\definecolor{currentstroke}{rgb}{0.500000,0.500000,0.500000}%
\pgfsetstrokecolor{currentstroke}%
\pgfsetstrokeopacity{0.300000}%
\pgfsetdash{}{0pt}%
\pgfpathmoveto{\pgfqpoint{1.358250in}{3.100319in}}%
\pgfpathlineto{\pgfqpoint{1.425061in}{3.115107in}}%
\pgfpathlineto{\pgfqpoint{1.492019in}{3.129210in}}%
\pgfpathlineto{\pgfqpoint{1.559179in}{3.142315in}}%
\pgfpathlineto{\pgfqpoint{1.626576in}{3.154124in}}%
\pgfpathlineto{\pgfqpoint{1.694227in}{3.164373in}}%
\pgfpathlineto{\pgfqpoint{1.762121in}{3.172845in}}%
\pgfpathlineto{\pgfqpoint{1.830228in}{3.179397in}}%
\pgfpathlineto{\pgfqpoint{1.898496in}{3.183984in}}%
\pgfpathlineto{\pgfqpoint{1.966864in}{3.186682in}}%
\pgfpathlineto{\pgfqpoint{2.035280in}{3.187690in}}%
\pgfpathlineto{\pgfqpoint{2.103706in}{3.187345in}}%
\pgfpathlineto{\pgfqpoint{2.172123in}{3.186128in}}%
\pgfpathlineto{\pgfqpoint{2.240536in}{3.184676in}}%
\pgfpathlineto{\pgfqpoint{2.308957in}{3.183766in}}%
\pgfpathlineto{\pgfqpoint{2.377376in}{3.184280in}}%
\pgfpathlineto{\pgfqpoint{2.445726in}{3.187182in}}%
\pgfpathlineto{\pgfqpoint{2.513839in}{3.193444in}}%
\pgfpathlineto{\pgfqpoint{2.581419in}{3.203924in}}%
\pgfpathlineto{\pgfqpoint{2.648064in}{3.219199in}}%
\pgfpathlineto{\pgfqpoint{2.713368in}{3.239442in}}%
\pgfpathlineto{\pgfqpoint{2.777016in}{3.264421in}}%
\pgfpathlineto{\pgfqpoint{2.838862in}{3.293600in}}%
\pgfpathlineto{\pgfqpoint{2.898939in}{3.326292in}}%
\pgfpathlineto{\pgfqpoint{2.957413in}{3.361789in}}%
\pgfusepath{stroke}%
\end{pgfscope}%
\begin{pgfscope}%
\pgfpathrectangle{\pgfqpoint{0.647939in}{0.492442in}}{\pgfqpoint{3.079299in}{3.079299in}}%
\pgfusepath{clip}%
\pgfsetbuttcap%
\pgfsetroundjoin%
\pgfsetlinewidth{0.301125pt}%
\definecolor{currentstroke}{rgb}{0.500000,0.500000,0.500000}%
\pgfsetstrokecolor{currentstroke}%
\pgfsetstrokeopacity{0.300000}%
\pgfsetdash{}{0pt}%
\pgfpathmoveto{\pgfqpoint{2.454879in}{0.768694in}}%
\pgfpathlineto{\pgfqpoint{2.386485in}{0.770707in}}%
\pgfpathlineto{\pgfqpoint{2.318062in}{0.771200in}}%
\pgfpathlineto{\pgfqpoint{2.249637in}{0.770597in}}%
\pgfpathlineto{\pgfqpoint{2.181219in}{0.769403in}}%
\pgfpathlineto{\pgfqpoint{2.112801in}{0.768210in}}%
\pgfpathlineto{\pgfqpoint{2.044376in}{0.767716in}}%
\pgfpathlineto{\pgfqpoint{1.975964in}{0.768764in}}%
\pgfpathlineto{\pgfqpoint{1.907652in}{0.772378in}}%
\pgfusepath{stroke}%
\end{pgfscope}%
\begin{pgfscope}%
\pgfpathrectangle{\pgfqpoint{0.647939in}{0.492442in}}{\pgfqpoint{3.079299in}{3.079299in}}%
\pgfusepath{clip}%
\pgfsetbuttcap%
\pgfsetroundjoin%
\pgfsetlinewidth{0.301125pt}%
\definecolor{currentstroke}{rgb}{0.500000,0.500000,0.500000}%
\pgfsetstrokecolor{currentstroke}%
\pgfsetstrokeopacity{0.300000}%
\pgfsetdash{}{0pt}%
\pgfpathmoveto{\pgfqpoint{3.447302in}{1.542203in}}%
\pgfpathlineto{\pgfqpoint{3.390417in}{1.580228in}}%
\pgfpathlineto{\pgfqpoint{3.334343in}{1.619442in}}%
\pgfpathlineto{\pgfqpoint{3.278990in}{1.659666in}}%
\pgfpathlineto{\pgfqpoint{3.224265in}{1.700742in}}%
\pgfpathlineto{\pgfqpoint{3.170074in}{1.742522in}}%
\pgfusepath{stroke}%
\end{pgfscope}%
\begin{pgfscope}%
\pgfpathrectangle{\pgfqpoint{0.647939in}{0.492442in}}{\pgfqpoint{3.079299in}{3.079299in}}%
\pgfusepath{clip}%
\pgfsetbuttcap%
\pgfsetroundjoin%
\pgfsetlinewidth{0.301125pt}%
\definecolor{currentstroke}{rgb}{0.500000,0.500000,0.500000}%
\pgfsetstrokecolor{currentstroke}%
\pgfsetstrokeopacity{0.300000}%
\pgfsetdash{}{0pt}%
\pgfpathmoveto{\pgfqpoint{3.447302in}{2.382012in}}%
\pgfpathlineto{\pgfqpoint{3.424623in}{2.446502in}}%
\pgfpathlineto{\pgfqpoint{3.407564in}{2.512696in}}%
\pgfpathlineto{\pgfqpoint{3.396514in}{2.580150in}}%
\pgfpathlineto{\pgfqpoint{3.391713in}{2.648326in}}%
\pgfpathlineto{\pgfqpoint{3.393201in}{2.716651in}}%
\pgfpathlineto{\pgfqpoint{3.400814in}{2.784573in}}%
\pgfpathlineto{\pgfqpoint{3.414212in}{2.851605in}}%
\pgfpathlineto{\pgfqpoint{3.432954in}{2.917357in}}%
\pgfpathlineto{\pgfqpoint{3.456564in}{2.981533in}}%
\pgfusepath{stroke}%
\end{pgfscope}%
\begin{pgfscope}%
\pgfpathrectangle{\pgfqpoint{0.647939in}{0.492442in}}{\pgfqpoint{3.079299in}{3.079299in}}%
\pgfusepath{clip}%
\pgfsetbuttcap%
\pgfsetroundjoin%
\pgfsetlinewidth{0.301125pt}%
\definecolor{currentstroke}{rgb}{0.500000,0.500000,0.500000}%
\pgfsetstrokecolor{currentstroke}%
\pgfsetstrokeopacity{0.300000}%
\pgfsetdash{}{0pt}%
\pgfpathmoveto{\pgfqpoint{3.416676in}{2.116121in}}%
\pgfpathlineto{\pgfqpoint{3.377318in}{2.172060in}}%
\pgfpathlineto{\pgfqpoint{3.341377in}{2.230249in}}%
\pgfpathlineto{\pgfqpoint{3.309314in}{2.290651in}}%
\pgfpathlineto{\pgfqpoint{3.281680in}{2.353192in}}%
\pgfpathlineto{\pgfqpoint{3.259098in}{2.417719in}}%
\pgfpathlineto{\pgfqpoint{3.242200in}{2.483955in}}%
\pgfpathlineto{\pgfqpoint{3.231531in}{2.551462in}}%
\pgfpathlineto{\pgfqpoint{3.227425in}{2.619672in}}%
\pgfpathlineto{\pgfqpoint{3.229915in}{2.687959in}}%
\pgfpathlineto{\pgfqpoint{3.238728in}{2.755731in}}%
\pgfpathlineto{\pgfqpoint{3.253362in}{2.822505in}}%
\pgfusepath{stroke}%
\end{pgfscope}%
\begin{pgfscope}%
\pgfpathrectangle{\pgfqpoint{0.647939in}{0.492442in}}{\pgfqpoint{3.079299in}{3.079299in}}%
\pgfusepath{clip}%
\pgfsetbuttcap%
\pgfsetroundjoin%
\pgfsetlinewidth{0.301125pt}%
\definecolor{currentstroke}{rgb}{0.500000,0.500000,0.500000}%
\pgfsetstrokecolor{currentstroke}%
\pgfsetstrokeopacity{0.300000}%
\pgfsetdash{}{0pt}%
\pgfpathmoveto{\pgfqpoint{1.137828in}{1.892124in}}%
\pgfpathlineto{\pgfqpoint{1.202313in}{1.915004in}}%
\pgfpathlineto{\pgfqpoint{1.266346in}{1.939123in}}%
\pgfpathlineto{\pgfqpoint{1.330014in}{1.964197in}}%
\pgfpathlineto{\pgfqpoint{1.393433in}{1.989896in}}%
\pgfpathlineto{\pgfqpoint{1.456750in}{2.015846in}}%
\pgfpathlineto{\pgfqpoint{1.520135in}{2.041628in}}%
\pgfpathlineto{\pgfqpoint{1.583776in}{2.066769in}}%
\pgfpathlineto{\pgfqpoint{1.647861in}{2.090744in}}%
\pgfpathlineto{\pgfqpoint{1.712570in}{2.112966in}}%
\pgfpathlineto{\pgfqpoint{1.778047in}{2.132786in}}%
\pgfpathlineto{\pgfqpoint{1.844375in}{2.149493in}}%
\pgfpathlineto{\pgfqpoint{1.911546in}{2.162336in}}%
\pgfpathlineto{\pgfqpoint{1.979425in}{2.170546in}}%
\pgfpathlineto{\pgfqpoint{2.047731in}{2.173336in}}%
\pgfpathlineto{\pgfqpoint{2.115988in}{2.169898in}}%
\pgfpathlineto{\pgfqpoint{2.183315in}{2.158735in}}%
\pgfpathlineto{\pgfqpoint{2.183315in}{2.158735in}}%
\pgfusepath{stroke}%
\end{pgfscope}%
\begin{pgfscope}%
\pgfpathrectangle{\pgfqpoint{0.647939in}{0.492442in}}{\pgfqpoint{3.079299in}{3.079299in}}%
\pgfusepath{clip}%
\pgfsetbuttcap%
\pgfsetroundjoin%
\pgfsetlinewidth{0.301125pt}%
\definecolor{currentstroke}{rgb}{0.500000,0.500000,0.500000}%
\pgfsetstrokecolor{currentstroke}%
\pgfsetstrokeopacity{0.300000}%
\pgfsetdash{}{0pt}%
\pgfpathmoveto{\pgfqpoint{2.454709in}{1.047197in}}%
\pgfpathlineto{\pgfqpoint{2.386333in}{1.049664in}}%
\pgfpathlineto{\pgfqpoint{2.317912in}{1.050269in}}%
\pgfpathlineto{\pgfqpoint{2.249490in}{1.049508in}}%
\pgfpathlineto{\pgfqpoint{2.181078in}{1.047990in}}%
\pgfpathlineto{\pgfqpoint{2.112667in}{1.046455in}}%
\pgfpathlineto{\pgfqpoint{2.044245in}{1.045818in}}%
\pgfpathlineto{\pgfqpoint{1.975847in}{1.047250in}}%
\pgfpathlineto{\pgfqpoint{1.907652in}{1.052315in}}%
\pgfusepath{stroke}%
\end{pgfscope}%
\begin{pgfscope}%
\pgfpathrectangle{\pgfqpoint{0.647939in}{0.492442in}}{\pgfqpoint{3.079299in}{3.079299in}}%
\pgfusepath{clip}%
\pgfsetbuttcap%
\pgfsetroundjoin%
\pgfsetlinewidth{0.301125pt}%
\definecolor{currentstroke}{rgb}{0.500000,0.500000,0.500000}%
\pgfsetstrokecolor{currentstroke}%
\pgfsetstrokeopacity{0.300000}%
\pgfsetdash{}{0pt}%
\pgfpathmoveto{\pgfqpoint{1.459832in}{2.690168in}}%
\pgfpathlineto{\pgfqpoint{1.526174in}{2.706931in}}%
\pgfpathlineto{\pgfqpoint{1.592795in}{2.722538in}}%
\pgfpathlineto{\pgfqpoint{1.659754in}{2.736614in}}%
\pgfpathlineto{\pgfqpoint{1.727079in}{2.748804in}}%
\pgfpathlineto{\pgfqpoint{1.794762in}{2.758803in}}%
\pgfpathlineto{\pgfqpoint{1.862756in}{2.766388in}}%
\pgfpathlineto{\pgfqpoint{1.930984in}{2.771464in}}%
\pgfpathlineto{\pgfqpoint{1.999350in}{2.774099in}}%
\pgfpathlineto{\pgfqpoint{2.067769in}{2.774546in}}%
\pgfpathlineto{\pgfqpoint{2.136182in}{2.773285in}}%
\pgfpathlineto{\pgfqpoint{2.204574in}{2.771065in}}%
\pgfpathlineto{\pgfqpoint{2.272969in}{2.768958in}}%
\pgfpathlineto{\pgfqpoint{2.341383in}{2.768384in}}%
\pgfpathlineto{\pgfqpoint{2.409718in}{2.771103in}}%
\pgfpathlineto{\pgfqpoint{2.477597in}{2.779086in}}%
\pgfpathlineto{\pgfqpoint{2.544250in}{2.794060in}}%
\pgfpathlineto{\pgfqpoint{2.608655in}{2.816818in}}%
\pgfpathlineto{\pgfqpoint{2.670019in}{2.846861in}}%
\pgfpathlineto{\pgfqpoint{2.728158in}{2.882805in}}%
\pgfpathlineto{\pgfqpoint{2.783398in}{2.923095in}}%
\pgfpathlineto{\pgfqpoint{2.836295in}{2.966432in}}%
\pgfpathlineto{\pgfqpoint{2.887429in}{3.011869in}}%
\pgfusepath{stroke}%
\end{pgfscope}%
\begin{pgfscope}%
\pgfpathrectangle{\pgfqpoint{0.647939in}{0.492442in}}{\pgfqpoint{3.079299in}{3.079299in}}%
\pgfusepath{clip}%
\pgfsetbuttcap%
\pgfsetroundjoin%
\pgfsetlinewidth{0.301125pt}%
\definecolor{currentstroke}{rgb}{0.500000,0.500000,0.500000}%
\pgfsetstrokecolor{currentstroke}%
\pgfsetstrokeopacity{0.300000}%
\pgfsetdash{}{0pt}%
\pgfpathmoveto{\pgfqpoint{1.716223in}{2.987380in}}%
\pgfpathlineto{\pgfqpoint{1.784078in}{2.996152in}}%
\pgfpathlineto{\pgfqpoint{1.852172in}{3.002810in}}%
\pgfpathlineto{\pgfqpoint{1.920443in}{3.007305in}}%
\pgfpathlineto{\pgfqpoint{1.988822in}{3.009728in}}%
\pgfpathlineto{\pgfqpoint{2.057242in}{3.010332in}}%
\pgfpathlineto{\pgfqpoint{2.125664in}{3.009535in}}%
\pgfpathlineto{\pgfqpoint{2.194073in}{3.007931in}}%
\pgfpathlineto{\pgfqpoint{2.262482in}{3.006299in}}%
\pgfpathlineto{\pgfqpoint{2.330903in}{3.005619in}}%
\pgfpathlineto{\pgfqpoint{2.399301in}{3.007056in}}%
\pgfpathlineto{\pgfqpoint{2.467525in}{3.011869in}}%
\pgfusepath{stroke}%
\end{pgfscope}%
\begin{pgfscope}%
\pgfpathrectangle{\pgfqpoint{0.647939in}{0.492442in}}{\pgfqpoint{3.079299in}{3.079299in}}%
\pgfusepath{clip}%
\pgfsetbuttcap%
\pgfsetroundjoin%
\pgfsetlinewidth{0.301125pt}%
\definecolor{currentstroke}{rgb}{0.500000,0.500000,0.500000}%
\pgfsetstrokecolor{currentstroke}%
\pgfsetstrokeopacity{0.300000}%
\pgfsetdash{}{0pt}%
\pgfpathmoveto{\pgfqpoint{3.097382in}{1.822139in}}%
\pgfpathlineto{\pgfqpoint{3.044739in}{1.865852in}}%
\pgfpathlineto{\pgfqpoint{2.992454in}{1.909990in}}%
\pgfpathlineto{\pgfqpoint{2.940496in}{1.954511in}}%
\pgfpathlineto{\pgfqpoint{2.888875in}{1.999422in}}%
\pgfpathlineto{\pgfqpoint{2.837660in}{2.044791in}}%
\pgfpathlineto{\pgfqpoint{2.787027in}{2.090800in}}%
\pgfpathlineto{\pgfqpoint{2.737356in}{2.137838in}}%
\pgfpathlineto{\pgfqpoint{2.689484in}{2.186673in}}%
\pgfpathlineto{\pgfqpoint{2.645325in}{2.238808in}}%
\pgfpathlineto{\pgfqpoint{2.609808in}{2.296839in}}%
\pgfpathlineto{\pgfqpoint{2.609808in}{2.296839in}}%
\pgfpathlineto{\pgfqpoint{2.595279in}{2.342414in}}%
\pgfpathlineto{\pgfqpoint{2.594165in}{2.391693in}}%
\pgfpathlineto{\pgfqpoint{2.604080in}{2.435882in}}%
\pgfpathlineto{\pgfqpoint{2.623481in}{2.483597in}}%
\pgfpathlineto{\pgfqpoint{2.654907in}{2.541400in}}%
\pgfpathlineto{\pgfqpoint{2.691386in}{2.599104in}}%
\pgfpathlineto{\pgfqpoint{2.729958in}{2.655508in}}%
\pgfpathlineto{\pgfqpoint{2.769750in}{2.711105in}}%
\pgfusepath{stroke}%
\end{pgfscope}%
\begin{pgfscope}%
\pgfpathrectangle{\pgfqpoint{0.647939in}{0.492442in}}{\pgfqpoint{3.079299in}{3.079299in}}%
\pgfusepath{clip}%
\pgfsetbuttcap%
\pgfsetroundjoin%
\pgfsetlinewidth{0.301125pt}%
\definecolor{currentstroke}{rgb}{0.500000,0.500000,0.500000}%
\pgfsetstrokecolor{currentstroke}%
\pgfsetstrokeopacity{0.300000}%
\pgfsetdash{}{0pt}%
\pgfpathmoveto{\pgfqpoint{1.277796in}{2.801916in}}%
\pgfpathlineto{\pgfqpoint{1.344111in}{2.818792in}}%
\pgfpathlineto{\pgfqpoint{1.410481in}{2.835451in}}%
\pgfpathlineto{\pgfqpoint{1.476986in}{2.851556in}}%
\pgfpathlineto{\pgfqpoint{1.543702in}{2.866758in}}%
\pgfpathlineto{\pgfqpoint{1.610690in}{2.880705in}}%
\pgfpathlineto{\pgfqpoint{1.677988in}{2.893063in}}%
\pgfpathlineto{\pgfqpoint{1.745602in}{2.903536in}}%
\pgfusepath{stroke}%
\end{pgfscope}%
\begin{pgfscope}%
\pgfpathrectangle{\pgfqpoint{0.647939in}{0.492442in}}{\pgfqpoint{3.079299in}{3.079299in}}%
\pgfusepath{clip}%
\pgfsetbuttcap%
\pgfsetroundjoin%
\pgfsetlinewidth{0.301125pt}%
\definecolor{currentstroke}{rgb}{0.500000,0.500000,0.500000}%
\pgfsetstrokecolor{currentstroke}%
\pgfsetstrokeopacity{0.300000}%
\pgfsetdash{}{0pt}%
\pgfpathmoveto{\pgfqpoint{1.417764in}{2.312028in}}%
\pgfpathlineto{\pgfqpoint{1.482842in}{2.333175in}}%
\pgfpathlineto{\pgfqpoint{1.548151in}{2.353596in}}%
\pgfpathlineto{\pgfqpoint{1.613815in}{2.372835in}}%
\pgfpathlineto{\pgfqpoint{1.679940in}{2.390409in}}%
\pgfpathlineto{\pgfqpoint{1.746595in}{2.405832in}}%
\pgfpathlineto{\pgfqpoint{1.813797in}{2.418638in}}%
\pgfpathlineto{\pgfqpoint{1.881499in}{2.428421in}}%
\pgfpathlineto{\pgfqpoint{1.949597in}{2.434890in}}%
\pgfpathlineto{\pgfqpoint{2.017935in}{2.437939in}}%
\pgfpathlineto{\pgfqpoint{2.086345in}{2.437758in}}%
\pgfpathlineto{\pgfqpoint{2.154704in}{2.434918in}}%
\pgfpathlineto{\pgfqpoint{2.222991in}{2.430535in}}%
\pgfpathlineto{\pgfqpoint{2.291303in}{2.426675in}}%
\pgfpathlineto{\pgfqpoint{2.359639in}{2.427184in}}%
\pgfpathlineto{\pgfqpoint{2.426654in}{2.438401in}}%
\pgfpathlineto{\pgfqpoint{2.485716in}{2.462991in}}%
\pgfpathlineto{\pgfqpoint{2.532633in}{2.493837in}}%
\pgfusepath{stroke}%
\end{pgfscope}%
\begin{pgfscope}%
\pgfpathrectangle{\pgfqpoint{0.647939in}{0.492442in}}{\pgfqpoint{3.079299in}{3.079299in}}%
\pgfusepath{clip}%
\pgfsetbuttcap%
\pgfsetroundjoin%
\pgfsetlinewidth{0.301125pt}%
\definecolor{currentstroke}{rgb}{0.500000,0.500000,0.500000}%
\pgfsetstrokecolor{currentstroke}%
\pgfsetstrokeopacity{0.300000}%
\pgfsetdash{}{0pt}%
\pgfpathmoveto{\pgfqpoint{1.417764in}{1.682171in}}%
\pgfpathlineto{\pgfqpoint{1.477587in}{1.715384in}}%
\pgfpathlineto{\pgfqpoint{1.537169in}{1.749030in}}%
\pgfpathlineto{\pgfqpoint{1.596744in}{1.782687in}}%
\pgfpathlineto{\pgfqpoint{1.656579in}{1.815875in}}%
\pgfpathlineto{\pgfqpoint{1.716968in}{1.848033in}}%
\pgfpathlineto{\pgfqpoint{1.778225in}{1.878490in}}%
\pgfpathlineto{\pgfqpoint{1.840663in}{1.906410in}}%
\pgfpathlineto{\pgfqpoint{1.904578in}{1.930694in}}%
\pgfpathlineto{\pgfqpoint{1.970215in}{1.949672in}}%
\pgfpathlineto{\pgfqpoint{2.037573in}{1.959378in}}%
\pgfpathlineto{\pgfqpoint{2.037573in}{1.959378in}}%
\pgfpathlineto{\pgfqpoint{2.061765in}{1.958463in}}%
\pgfpathlineto{\pgfqpoint{2.061765in}{1.958463in}}%
\pgfpathlineto{\pgfqpoint{2.077744in}{1.951969in}}%
\pgfpathlineto{\pgfqpoint{2.077744in}{1.951969in}}%
\pgfpathlineto{\pgfqpoint{2.080400in}{1.943701in}}%
\pgfpathlineto{\pgfqpoint{2.078102in}{1.935975in}}%
\pgfusepath{stroke}%
\end{pgfscope}%
\begin{pgfscope}%
\pgfpathrectangle{\pgfqpoint{0.647939in}{0.492442in}}{\pgfqpoint{3.079299in}{3.079299in}}%
\pgfusepath{clip}%
\pgfsetbuttcap%
\pgfsetroundjoin%
\pgfsetlinewidth{0.301125pt}%
\definecolor{currentstroke}{rgb}{0.500000,0.500000,0.500000}%
\pgfsetstrokecolor{currentstroke}%
\pgfsetstrokeopacity{0.300000}%
\pgfsetdash{}{0pt}%
\pgfpathmoveto{\pgfqpoint{1.924729in}{2.707694in}}%
\pgfpathlineto{\pgfqpoint{1.993077in}{2.710730in}}%
\pgfpathlineto{\pgfqpoint{2.061493in}{2.711403in}}%
\pgfpathlineto{\pgfqpoint{2.129905in}{2.710180in}}%
\pgfpathlineto{\pgfqpoint{2.198292in}{2.707815in}}%
\pgfpathlineto{\pgfqpoint{2.266678in}{2.705413in}}%
\pgfpathlineto{\pgfqpoint{2.335088in}{2.704529in}}%
\pgfpathlineto{\pgfqpoint{2.403418in}{2.707178in}}%
\pgfpathlineto{\pgfqpoint{2.471211in}{2.715652in}}%
\pgfpathlineto{\pgfqpoint{2.537509in}{2.731932in}}%
\pgfusepath{stroke}%
\end{pgfscope}%
\begin{pgfscope}%
\pgfpathrectangle{\pgfqpoint{0.647939in}{0.492442in}}{\pgfqpoint{3.079299in}{3.079299in}}%
\pgfusepath{clip}%
\pgfsetbuttcap%
\pgfsetroundjoin%
\pgfsetlinewidth{0.301125pt}%
\definecolor{currentstroke}{rgb}{0.500000,0.500000,0.500000}%
\pgfsetstrokecolor{currentstroke}%
\pgfsetstrokeopacity{0.300000}%
\pgfsetdash{}{0pt}%
\pgfpathmoveto{\pgfqpoint{1.627716in}{2.451996in}}%
\pgfpathlineto{\pgfqpoint{1.694211in}{2.468110in}}%
\pgfpathlineto{\pgfqpoint{1.761201in}{2.482004in}}%
\pgfpathlineto{\pgfqpoint{1.828680in}{2.493267in}}%
\pgfpathlineto{\pgfqpoint{1.896584in}{2.501577in}}%
\pgfpathlineto{\pgfqpoint{1.964795in}{2.506761in}}%
\pgfpathlineto{\pgfqpoint{2.033172in}{2.508847in}}%
\pgfpathlineto{\pgfqpoint{2.101582in}{2.508133in}}%
\pgfpathlineto{\pgfqpoint{2.169946in}{2.505286in}}%
\pgfpathlineto{\pgfqpoint{2.238271in}{2.501518in}}%
\pgfpathlineto{\pgfqpoint{2.306634in}{2.498857in}}%
\pgfpathlineto{\pgfqpoint{2.374951in}{2.500536in}}%
\pgfpathlineto{\pgfqpoint{2.442265in}{2.511045in}}%
\pgfusepath{stroke}%
\end{pgfscope}%
\begin{pgfscope}%
\pgfpathrectangle{\pgfqpoint{0.647939in}{0.492442in}}{\pgfqpoint{3.079299in}{3.079299in}}%
\pgfusepath{clip}%
\pgfsetbuttcap%
\pgfsetroundjoin%
\pgfsetlinewidth{0.301125pt}%
\definecolor{currentstroke}{rgb}{0.500000,0.500000,0.500000}%
\pgfsetstrokecolor{currentstroke}%
\pgfsetstrokeopacity{0.300000}%
\pgfsetdash{}{0pt}%
\pgfpathmoveto{\pgfqpoint{2.677477in}{1.962108in}}%
\pgfpathlineto{\pgfqpoint{2.616573in}{1.993236in}}%
\pgfpathlineto{\pgfqpoint{2.554217in}{2.021313in}}%
\pgfpathlineto{\pgfqpoint{2.490221in}{2.045374in}}%
\pgfpathlineto{\pgfqpoint{2.424463in}{2.063959in}}%
\pgfpathlineto{\pgfqpoint{2.357033in}{2.074517in}}%
\pgfusepath{stroke}%
\end{pgfscope}%
\begin{pgfscope}%
\pgfpathrectangle{\pgfqpoint{0.647939in}{0.492442in}}{\pgfqpoint{3.079299in}{3.079299in}}%
\pgfusepath{clip}%
\pgfsetbuttcap%
\pgfsetroundjoin%
\pgfsetlinewidth{0.301125pt}%
\definecolor{currentstroke}{rgb}{0.500000,0.500000,0.500000}%
\pgfsetstrokecolor{currentstroke}%
\pgfsetstrokeopacity{0.300000}%
\pgfsetdash{}{0pt}%
\pgfpathmoveto{\pgfqpoint{2.731611in}{2.060247in}}%
\pgfpathlineto{\pgfqpoint{2.677477in}{2.102076in}}%
\pgfpathlineto{\pgfqpoint{2.623659in}{2.144301in}}%
\pgfpathlineto{\pgfqpoint{2.571089in}{2.188006in}}%
\pgfpathlineto{\pgfqpoint{2.522975in}{2.236359in}}%
\pgfpathlineto{\pgfqpoint{2.522975in}{2.236359in}}%
\pgfpathlineto{\pgfqpoint{2.500929in}{2.270589in}}%
\pgfpathlineto{\pgfqpoint{2.500929in}{2.270589in}}%
\pgfusepath{stroke}%
\end{pgfscope}%
\begin{pgfscope}%
\pgfpathrectangle{\pgfqpoint{0.647939in}{0.492442in}}{\pgfqpoint{3.079299in}{3.079299in}}%
\pgfusepath{clip}%
\pgfsetbuttcap%
\pgfsetroundjoin%
\pgfsetlinewidth{0.301125pt}%
\definecolor{currentstroke}{rgb}{0.500000,0.500000,0.500000}%
\pgfsetstrokecolor{currentstroke}%
\pgfsetstrokeopacity{0.300000}%
\pgfsetdash{}{0pt}%
\pgfpathmoveto{\pgfqpoint{2.852724in}{2.104772in}}%
\pgfpathlineto{\pgfqpoint{2.806566in}{2.155247in}}%
\pgfpathlineto{\pgfqpoint{2.763009in}{2.207951in}}%
\pgfpathlineto{\pgfqpoint{2.723959in}{2.264005in}}%
\pgfpathlineto{\pgfqpoint{2.696288in}{2.316901in}}%
\pgfpathlineto{\pgfqpoint{2.677477in}{2.382012in}}%
\pgfpathlineto{\pgfqpoint{2.677477in}{2.382012in}}%
\pgfpathlineto{\pgfqpoint{2.675667in}{2.436768in}}%
\pgfusepath{stroke}%
\end{pgfscope}%
\begin{pgfscope}%
\pgfpathrectangle{\pgfqpoint{0.647939in}{0.492442in}}{\pgfqpoint{3.079299in}{3.079299in}}%
\pgfusepath{clip}%
\pgfsetbuttcap%
\pgfsetroundjoin%
\pgfsetlinewidth{0.301125pt}%
\definecolor{currentstroke}{rgb}{0.500000,0.500000,0.500000}%
\pgfsetstrokecolor{currentstroke}%
\pgfsetstrokeopacity{0.300000}%
\pgfsetdash{}{0pt}%
\pgfpathmoveto{\pgfqpoint{2.590872in}{1.590601in}}%
\pgfpathlineto{\pgfqpoint{2.523613in}{1.603003in}}%
\pgfpathlineto{\pgfqpoint{2.455746in}{1.611486in}}%
\pgfpathlineto{\pgfqpoint{2.387503in}{1.616139in}}%
\pgfpathlineto{\pgfqpoint{2.319105in}{1.617343in}}%
\pgfpathlineto{\pgfqpoint{2.250706in}{1.615775in}}%
\pgfpathlineto{\pgfqpoint{2.182362in}{1.612431in}}%
\pgfpathlineto{\pgfqpoint{2.114032in}{1.608763in}}%
\pgfpathlineto{\pgfqpoint{2.045648in}{1.607102in}}%
\pgfpathlineto{\pgfqpoint{1.977636in}{1.612187in}}%
\pgfpathlineto{\pgfqpoint{1.977636in}{1.612187in}}%
\pgfusepath{stroke}%
\end{pgfscope}%
\begin{pgfscope}%
\pgfpathrectangle{\pgfqpoint{0.647939in}{0.492442in}}{\pgfqpoint{3.079299in}{3.079299in}}%
\pgfusepath{clip}%
\pgfsetbuttcap%
\pgfsetroundjoin%
\pgfsetlinewidth{0.301125pt}%
\definecolor{currentstroke}{rgb}{0.500000,0.500000,0.500000}%
\pgfsetstrokecolor{currentstroke}%
\pgfsetstrokeopacity{0.300000}%
\pgfsetdash{}{0pt}%
\pgfpathmoveto{\pgfqpoint{1.837668in}{2.312028in}}%
\pgfpathlineto{\pgfqpoint{1.905325in}{2.322082in}}%
\pgfpathlineto{\pgfqpoint{1.973431in}{2.328356in}}%
\pgfpathlineto{\pgfqpoint{2.041787in}{2.330652in}}%
\pgfpathlineto{\pgfqpoint{2.110170in}{2.329070in}}%
\pgfpathlineto{\pgfqpoint{2.178412in}{2.324221in}}%
\pgfpathlineto{\pgfqpoint{2.246518in}{2.317603in}}%
\pgfpathlineto{\pgfqpoint{2.314729in}{2.312950in}}%
\pgfpathlineto{\pgfqpoint{2.314729in}{2.312950in}}%
\pgfusepath{stroke}%
\end{pgfscope}%
\begin{pgfscope}%
\pgfpathrectangle{\pgfqpoint{0.647939in}{0.492442in}}{\pgfqpoint{3.079299in}{3.079299in}}%
\pgfusepath{clip}%
\pgfsetbuttcap%
\pgfsetroundjoin%
\pgfsetlinewidth{0.301125pt}%
\definecolor{currentstroke}{rgb}{0.500000,0.500000,0.500000}%
\pgfsetstrokecolor{currentstroke}%
\pgfsetstrokeopacity{0.300000}%
\pgfsetdash{}{0pt}%
\pgfpathmoveto{\pgfqpoint{2.472969in}{1.744877in}}%
\pgfpathlineto{\pgfqpoint{2.404951in}{1.751921in}}%
\pgfpathlineto{\pgfqpoint{2.336612in}{1.754542in}}%
\pgfpathlineto{\pgfqpoint{2.268222in}{1.753243in}}%
\pgfpathlineto{\pgfqpoint{2.199936in}{1.749005in}}%
\pgfpathlineto{\pgfqpoint{2.131732in}{1.743468in}}%
\pgfpathlineto{\pgfqpoint{2.063446in}{1.739616in}}%
\pgfpathlineto{\pgfqpoint{2.021622in}{1.740534in}}%
\pgfpathlineto{\pgfqpoint{1.977636in}{1.752155in}}%
\pgfpathlineto{\pgfqpoint{1.977636in}{1.752155in}}%
\pgfpathlineto{\pgfqpoint{1.977636in}{1.752155in}}%
\pgfpathlineto{\pgfqpoint{1.960681in}{1.767558in}}%
\pgfusepath{stroke}%
\end{pgfscope}%
\begin{pgfscope}%
\pgfpathrectangle{\pgfqpoint{0.647939in}{0.492442in}}{\pgfqpoint{3.079299in}{3.079299in}}%
\pgfusepath{clip}%
\pgfsetroundcap%
\pgfsetroundjoin%
\pgfsetlinewidth{0.301125pt}%
\definecolor{currentstroke}{rgb}{0.500000,0.500000,0.500000}%
\pgfsetstrokecolor{currentstroke}%
\pgfsetstrokeopacity{0.300000}%
\pgfsetdash{}{0pt}%
\pgfpathmoveto{\pgfqpoint{1.452664in}{1.105108in}}%
\pgfusepath{stroke}%
\end{pgfscope}%
\begin{pgfscope}%
\pgfpathrectangle{\pgfqpoint{0.647939in}{0.492442in}}{\pgfqpoint{3.079299in}{3.079299in}}%
\pgfusepath{clip}%
\pgfsetroundcap%
\pgfsetroundjoin%
\definecolor{currentfill}{rgb}{0.500000,0.500000,0.500000}%
\pgfsetfillcolor{currentfill}%
\pgfsetfillopacity{0.300000}%
\pgfsetlinewidth{0.301125pt}%
\definecolor{currentstroke}{rgb}{0.500000,0.500000,0.500000}%
\pgfsetstrokecolor{currentstroke}%
\pgfsetstrokeopacity{0.300000}%
\pgfsetdash{}{0pt}%
\pgfpathmoveto{\pgfqpoint{0.000000in}{0.000000in}}%
\pgfpathlineto{\pgfqpoint{0.000000in}{0.000000in}}%
\pgfpathclose%
\pgfusepath{stroke,fill}%
\end{pgfscope}%
\begin{pgfscope}%
\pgfpathrectangle{\pgfqpoint{0.647939in}{0.492442in}}{\pgfqpoint{3.079299in}{3.079299in}}%
\pgfusepath{clip}%
\pgfsetroundcap%
\pgfsetroundjoin%
\pgfsetlinewidth{0.301125pt}%
\definecolor{currentstroke}{rgb}{0.500000,0.500000,0.500000}%
\pgfsetstrokecolor{currentstroke}%
\pgfsetstrokeopacity{0.300000}%
\pgfsetdash{}{0pt}%
\pgfpathmoveto{\pgfqpoint{1.130114in}{0.589733in}}%
\pgfusepath{stroke}%
\end{pgfscope}%
\begin{pgfscope}%
\pgfpathrectangle{\pgfqpoint{0.647939in}{0.492442in}}{\pgfqpoint{3.079299in}{3.079299in}}%
\pgfusepath{clip}%
\pgfsetroundcap%
\pgfsetroundjoin%
\definecolor{currentfill}{rgb}{0.500000,0.500000,0.500000}%
\pgfsetfillcolor{currentfill}%
\pgfsetfillopacity{0.300000}%
\pgfsetlinewidth{0.301125pt}%
\definecolor{currentstroke}{rgb}{0.500000,0.500000,0.500000}%
\pgfsetstrokecolor{currentstroke}%
\pgfsetstrokeopacity{0.300000}%
\pgfsetdash{}{0pt}%
\pgfpathmoveto{\pgfqpoint{0.000000in}{0.000000in}}%
\pgfpathlineto{\pgfqpoint{0.000000in}{0.000000in}}%
\pgfpathclose%
\pgfusepath{stroke,fill}%
\end{pgfscope}%
\begin{pgfscope}%
\pgfpathrectangle{\pgfqpoint{0.647939in}{0.492442in}}{\pgfqpoint{3.079299in}{3.079299in}}%
\pgfusepath{clip}%
\pgfsetroundcap%
\pgfsetroundjoin%
\pgfsetlinewidth{0.301125pt}%
\definecolor{currentstroke}{rgb}{0.500000,0.500000,0.500000}%
\pgfsetstrokecolor{currentstroke}%
\pgfsetstrokeopacity{0.300000}%
\pgfsetdash{}{0pt}%
\pgfpathmoveto{\pgfqpoint{1.366520in}{0.780054in}}%
\pgfusepath{stroke}%
\end{pgfscope}%
\begin{pgfscope}%
\pgfpathrectangle{\pgfqpoint{0.647939in}{0.492442in}}{\pgfqpoint{3.079299in}{3.079299in}}%
\pgfusepath{clip}%
\pgfsetroundcap%
\pgfsetroundjoin%
\definecolor{currentfill}{rgb}{0.500000,0.500000,0.500000}%
\pgfsetfillcolor{currentfill}%
\pgfsetfillopacity{0.300000}%
\pgfsetlinewidth{0.301125pt}%
\definecolor{currentstroke}{rgb}{0.500000,0.500000,0.500000}%
\pgfsetstrokecolor{currentstroke}%
\pgfsetstrokeopacity{0.300000}%
\pgfsetdash{}{0pt}%
\pgfpathmoveto{\pgfqpoint{0.000000in}{0.000000in}}%
\pgfpathlineto{\pgfqpoint{0.000000in}{0.000000in}}%
\pgfpathclose%
\pgfusepath{stroke,fill}%
\end{pgfscope}%
\begin{pgfscope}%
\pgfpathrectangle{\pgfqpoint{0.647939in}{0.492442in}}{\pgfqpoint{3.079299in}{3.079299in}}%
\pgfusepath{clip}%
\pgfsetroundcap%
\pgfsetroundjoin%
\pgfsetlinewidth{0.301125pt}%
\definecolor{currentstroke}{rgb}{0.500000,0.500000,0.500000}%
\pgfsetstrokecolor{currentstroke}%
\pgfsetstrokeopacity{0.300000}%
\pgfsetdash{}{0pt}%
\pgfpathmoveto{\pgfqpoint{1.389240in}{0.643951in}}%
\pgfusepath{stroke}%
\end{pgfscope}%
\begin{pgfscope}%
\pgfpathrectangle{\pgfqpoint{0.647939in}{0.492442in}}{\pgfqpoint{3.079299in}{3.079299in}}%
\pgfusepath{clip}%
\pgfsetroundcap%
\pgfsetroundjoin%
\definecolor{currentfill}{rgb}{0.500000,0.500000,0.500000}%
\pgfsetfillcolor{currentfill}%
\pgfsetfillopacity{0.300000}%
\pgfsetlinewidth{0.301125pt}%
\definecolor{currentstroke}{rgb}{0.500000,0.500000,0.500000}%
\pgfsetstrokecolor{currentstroke}%
\pgfsetstrokeopacity{0.300000}%
\pgfsetdash{}{0pt}%
\pgfpathmoveto{\pgfqpoint{0.000000in}{0.000000in}}%
\pgfpathlineto{\pgfqpoint{0.000000in}{0.000000in}}%
\pgfpathclose%
\pgfusepath{stroke,fill}%
\end{pgfscope}%
\begin{pgfscope}%
\pgfpathrectangle{\pgfqpoint{0.647939in}{0.492442in}}{\pgfqpoint{3.079299in}{3.079299in}}%
\pgfusepath{clip}%
\pgfsetroundcap%
\pgfsetroundjoin%
\pgfsetlinewidth{0.301125pt}%
\definecolor{currentstroke}{rgb}{0.500000,0.500000,0.500000}%
\pgfsetstrokecolor{currentstroke}%
\pgfsetstrokeopacity{0.300000}%
\pgfsetdash{}{0pt}%
\pgfpathmoveto{\pgfqpoint{1.551825in}{0.564847in}}%
\pgfusepath{stroke}%
\end{pgfscope}%
\begin{pgfscope}%
\pgfpathrectangle{\pgfqpoint{0.647939in}{0.492442in}}{\pgfqpoint{3.079299in}{3.079299in}}%
\pgfusepath{clip}%
\pgfsetroundcap%
\pgfsetroundjoin%
\definecolor{currentfill}{rgb}{0.500000,0.500000,0.500000}%
\pgfsetfillcolor{currentfill}%
\pgfsetfillopacity{0.300000}%
\pgfsetlinewidth{0.301125pt}%
\definecolor{currentstroke}{rgb}{0.500000,0.500000,0.500000}%
\pgfsetstrokecolor{currentstroke}%
\pgfsetstrokeopacity{0.300000}%
\pgfsetdash{}{0pt}%
\pgfpathmoveto{\pgfqpoint{0.000000in}{0.000000in}}%
\pgfpathlineto{\pgfqpoint{0.000000in}{0.000000in}}%
\pgfpathclose%
\pgfusepath{stroke,fill}%
\end{pgfscope}%
\begin{pgfscope}%
\pgfpathrectangle{\pgfqpoint{0.647939in}{0.492442in}}{\pgfqpoint{3.079299in}{3.079299in}}%
\pgfusepath{clip}%
\pgfsetroundcap%
\pgfsetroundjoin%
\pgfsetlinewidth{0.301125pt}%
\definecolor{currentstroke}{rgb}{0.500000,0.500000,0.500000}%
\pgfsetstrokecolor{currentstroke}%
\pgfsetstrokeopacity{0.300000}%
\pgfsetdash{}{0pt}%
\pgfpathmoveto{\pgfqpoint{2.293551in}{0.495126in}}%
\pgfusepath{stroke}%
\end{pgfscope}%
\begin{pgfscope}%
\pgfpathrectangle{\pgfqpoint{0.647939in}{0.492442in}}{\pgfqpoint{3.079299in}{3.079299in}}%
\pgfusepath{clip}%
\pgfsetroundcap%
\pgfsetroundjoin%
\definecolor{currentfill}{rgb}{0.500000,0.500000,0.500000}%
\pgfsetfillcolor{currentfill}%
\pgfsetfillopacity{0.300000}%
\pgfsetlinewidth{0.301125pt}%
\definecolor{currentstroke}{rgb}{0.500000,0.500000,0.500000}%
\pgfsetstrokecolor{currentstroke}%
\pgfsetstrokeopacity{0.300000}%
\pgfsetdash{}{0pt}%
\pgfpathmoveto{\pgfqpoint{0.000000in}{0.000000in}}%
\pgfpathlineto{\pgfqpoint{0.000000in}{0.000000in}}%
\pgfpathclose%
\pgfusepath{stroke,fill}%
\end{pgfscope}%
\begin{pgfscope}%
\pgfpathrectangle{\pgfqpoint{0.647939in}{0.492442in}}{\pgfqpoint{3.079299in}{3.079299in}}%
\pgfusepath{clip}%
\pgfsetroundcap%
\pgfsetroundjoin%
\pgfsetlinewidth{0.301125pt}%
\definecolor{currentstroke}{rgb}{0.500000,0.500000,0.500000}%
\pgfsetstrokecolor{currentstroke}%
\pgfsetstrokeopacity{0.300000}%
\pgfsetdash{}{0pt}%
\pgfpathmoveto{\pgfqpoint{2.176532in}{0.533607in}}%
\pgfusepath{stroke}%
\end{pgfscope}%
\begin{pgfscope}%
\pgfpathrectangle{\pgfqpoint{0.647939in}{0.492442in}}{\pgfqpoint{3.079299in}{3.079299in}}%
\pgfusepath{clip}%
\pgfsetroundcap%
\pgfsetroundjoin%
\definecolor{currentfill}{rgb}{0.500000,0.500000,0.500000}%
\pgfsetfillcolor{currentfill}%
\pgfsetfillopacity{0.300000}%
\pgfsetlinewidth{0.301125pt}%
\definecolor{currentstroke}{rgb}{0.500000,0.500000,0.500000}%
\pgfsetstrokecolor{currentstroke}%
\pgfsetstrokeopacity{0.300000}%
\pgfsetdash{}{0pt}%
\pgfpathmoveto{\pgfqpoint{0.000000in}{0.000000in}}%
\pgfpathlineto{\pgfqpoint{0.000000in}{0.000000in}}%
\pgfpathclose%
\pgfusepath{stroke,fill}%
\end{pgfscope}%
\begin{pgfscope}%
\pgfpathrectangle{\pgfqpoint{0.647939in}{0.492442in}}{\pgfqpoint{3.079299in}{3.079299in}}%
\pgfusepath{clip}%
\pgfsetroundcap%
\pgfsetroundjoin%
\pgfsetlinewidth{0.301125pt}%
\definecolor{currentstroke}{rgb}{0.500000,0.500000,0.500000}%
\pgfsetstrokecolor{currentstroke}%
\pgfsetstrokeopacity{0.300000}%
\pgfsetdash{}{0pt}%
\pgfpathmoveto{\pgfqpoint{2.867702in}{0.532260in}}%
\pgfusepath{stroke}%
\end{pgfscope}%
\begin{pgfscope}%
\pgfpathrectangle{\pgfqpoint{0.647939in}{0.492442in}}{\pgfqpoint{3.079299in}{3.079299in}}%
\pgfusepath{clip}%
\pgfsetroundcap%
\pgfsetroundjoin%
\definecolor{currentfill}{rgb}{0.500000,0.500000,0.500000}%
\pgfsetfillcolor{currentfill}%
\pgfsetfillopacity{0.300000}%
\pgfsetlinewidth{0.301125pt}%
\definecolor{currentstroke}{rgb}{0.500000,0.500000,0.500000}%
\pgfsetstrokecolor{currentstroke}%
\pgfsetstrokeopacity{0.300000}%
\pgfsetdash{}{0pt}%
\pgfpathmoveto{\pgfqpoint{0.000000in}{0.000000in}}%
\pgfpathlineto{\pgfqpoint{0.000000in}{0.000000in}}%
\pgfpathclose%
\pgfusepath{stroke,fill}%
\end{pgfscope}%
\begin{pgfscope}%
\pgfpathrectangle{\pgfqpoint{0.647939in}{0.492442in}}{\pgfqpoint{3.079299in}{3.079299in}}%
\pgfusepath{clip}%
\pgfsetroundcap%
\pgfsetroundjoin%
\pgfsetlinewidth{0.301125pt}%
\definecolor{currentstroke}{rgb}{0.500000,0.500000,0.500000}%
\pgfsetstrokecolor{currentstroke}%
\pgfsetstrokeopacity{0.300000}%
\pgfsetdash{}{0pt}%
\pgfpathmoveto{\pgfqpoint{2.271265in}{0.655936in}}%
\pgfusepath{stroke}%
\end{pgfscope}%
\begin{pgfscope}%
\pgfpathrectangle{\pgfqpoint{0.647939in}{0.492442in}}{\pgfqpoint{3.079299in}{3.079299in}}%
\pgfusepath{clip}%
\pgfsetroundcap%
\pgfsetroundjoin%
\definecolor{currentfill}{rgb}{0.500000,0.500000,0.500000}%
\pgfsetfillcolor{currentfill}%
\pgfsetfillopacity{0.300000}%
\pgfsetlinewidth{0.301125pt}%
\definecolor{currentstroke}{rgb}{0.500000,0.500000,0.500000}%
\pgfsetstrokecolor{currentstroke}%
\pgfsetstrokeopacity{0.300000}%
\pgfsetdash{}{0pt}%
\pgfpathmoveto{\pgfqpoint{0.000000in}{0.000000in}}%
\pgfpathlineto{\pgfqpoint{0.000000in}{0.000000in}}%
\pgfpathclose%
\pgfusepath{stroke,fill}%
\end{pgfscope}%
\begin{pgfscope}%
\pgfpathrectangle{\pgfqpoint{0.647939in}{0.492442in}}{\pgfqpoint{3.079299in}{3.079299in}}%
\pgfusepath{clip}%
\pgfsetroundcap%
\pgfsetroundjoin%
\pgfsetlinewidth{0.301125pt}%
\definecolor{currentstroke}{rgb}{0.500000,0.500000,0.500000}%
\pgfsetstrokecolor{currentstroke}%
\pgfsetstrokeopacity{0.300000}%
\pgfsetdash{}{0pt}%
\pgfpathmoveto{\pgfqpoint{2.358074in}{0.735208in}}%
\pgfusepath{stroke}%
\end{pgfscope}%
\begin{pgfscope}%
\pgfpathrectangle{\pgfqpoint{0.647939in}{0.492442in}}{\pgfqpoint{3.079299in}{3.079299in}}%
\pgfusepath{clip}%
\pgfsetroundcap%
\pgfsetroundjoin%
\definecolor{currentfill}{rgb}{0.500000,0.500000,0.500000}%
\pgfsetfillcolor{currentfill}%
\pgfsetfillopacity{0.300000}%
\pgfsetlinewidth{0.301125pt}%
\definecolor{currentstroke}{rgb}{0.500000,0.500000,0.500000}%
\pgfsetstrokecolor{currentstroke}%
\pgfsetstrokeopacity{0.300000}%
\pgfsetdash{}{0pt}%
\pgfpathmoveto{\pgfqpoint{0.000000in}{0.000000in}}%
\pgfpathlineto{\pgfqpoint{0.000000in}{0.000000in}}%
\pgfpathclose%
\pgfusepath{stroke,fill}%
\end{pgfscope}%
\begin{pgfscope}%
\pgfpathrectangle{\pgfqpoint{0.647939in}{0.492442in}}{\pgfqpoint{3.079299in}{3.079299in}}%
\pgfusepath{clip}%
\pgfsetroundcap%
\pgfsetroundjoin%
\pgfsetlinewidth{0.301125pt}%
\definecolor{currentstroke}{rgb}{0.500000,0.500000,0.500000}%
\pgfsetstrokecolor{currentstroke}%
\pgfsetstrokeopacity{0.300000}%
\pgfsetdash{}{0pt}%
\pgfpathmoveto{\pgfqpoint{2.648974in}{0.793558in}}%
\pgfusepath{stroke}%
\end{pgfscope}%
\begin{pgfscope}%
\pgfpathrectangle{\pgfqpoint{0.647939in}{0.492442in}}{\pgfqpoint{3.079299in}{3.079299in}}%
\pgfusepath{clip}%
\pgfsetroundcap%
\pgfsetroundjoin%
\definecolor{currentfill}{rgb}{0.500000,0.500000,0.500000}%
\pgfsetfillcolor{currentfill}%
\pgfsetfillopacity{0.300000}%
\pgfsetlinewidth{0.301125pt}%
\definecolor{currentstroke}{rgb}{0.500000,0.500000,0.500000}%
\pgfsetstrokecolor{currentstroke}%
\pgfsetstrokeopacity{0.300000}%
\pgfsetdash{}{0pt}%
\pgfpathmoveto{\pgfqpoint{0.000000in}{0.000000in}}%
\pgfpathlineto{\pgfqpoint{0.000000in}{0.000000in}}%
\pgfpathclose%
\pgfusepath{stroke,fill}%
\end{pgfscope}%
\begin{pgfscope}%
\pgfpathrectangle{\pgfqpoint{0.647939in}{0.492442in}}{\pgfqpoint{3.079299in}{3.079299in}}%
\pgfusepath{clip}%
\pgfsetroundcap%
\pgfsetroundjoin%
\pgfsetlinewidth{0.301125pt}%
\definecolor{currentstroke}{rgb}{0.500000,0.500000,0.500000}%
\pgfsetstrokecolor{currentstroke}%
\pgfsetstrokeopacity{0.300000}%
\pgfsetdash{}{0pt}%
\pgfpathmoveto{\pgfqpoint{2.590407in}{0.907895in}}%
\pgfusepath{stroke}%
\end{pgfscope}%
\begin{pgfscope}%
\pgfpathrectangle{\pgfqpoint{0.647939in}{0.492442in}}{\pgfqpoint{3.079299in}{3.079299in}}%
\pgfusepath{clip}%
\pgfsetroundcap%
\pgfsetroundjoin%
\definecolor{currentfill}{rgb}{0.500000,0.500000,0.500000}%
\pgfsetfillcolor{currentfill}%
\pgfsetfillopacity{0.300000}%
\pgfsetlinewidth{0.301125pt}%
\definecolor{currentstroke}{rgb}{0.500000,0.500000,0.500000}%
\pgfsetstrokecolor{currentstroke}%
\pgfsetstrokeopacity{0.300000}%
\pgfsetdash{}{0pt}%
\pgfpathmoveto{\pgfqpoint{0.000000in}{0.000000in}}%
\pgfpathlineto{\pgfqpoint{0.000000in}{0.000000in}}%
\pgfpathclose%
\pgfusepath{stroke,fill}%
\end{pgfscope}%
\begin{pgfscope}%
\pgfpathrectangle{\pgfqpoint{0.647939in}{0.492442in}}{\pgfqpoint{3.079299in}{3.079299in}}%
\pgfusepath{clip}%
\pgfsetroundcap%
\pgfsetroundjoin%
\pgfsetlinewidth{0.301125pt}%
\definecolor{currentstroke}{rgb}{0.500000,0.500000,0.500000}%
\pgfsetstrokecolor{currentstroke}%
\pgfsetstrokeopacity{0.300000}%
\pgfsetdash{}{0pt}%
\pgfpathmoveto{\pgfqpoint{2.729957in}{0.970406in}}%
\pgfusepath{stroke}%
\end{pgfscope}%
\begin{pgfscope}%
\pgfpathrectangle{\pgfqpoint{0.647939in}{0.492442in}}{\pgfqpoint{3.079299in}{3.079299in}}%
\pgfusepath{clip}%
\pgfsetroundcap%
\pgfsetroundjoin%
\definecolor{currentfill}{rgb}{0.500000,0.500000,0.500000}%
\pgfsetfillcolor{currentfill}%
\pgfsetfillopacity{0.300000}%
\pgfsetlinewidth{0.301125pt}%
\definecolor{currentstroke}{rgb}{0.500000,0.500000,0.500000}%
\pgfsetstrokecolor{currentstroke}%
\pgfsetstrokeopacity{0.300000}%
\pgfsetdash{}{0pt}%
\pgfpathmoveto{\pgfqpoint{0.000000in}{0.000000in}}%
\pgfpathlineto{\pgfqpoint{0.000000in}{0.000000in}}%
\pgfpathclose%
\pgfusepath{stroke,fill}%
\end{pgfscope}%
\begin{pgfscope}%
\pgfpathrectangle{\pgfqpoint{0.647939in}{0.492442in}}{\pgfqpoint{3.079299in}{3.079299in}}%
\pgfusepath{clip}%
\pgfsetroundcap%
\pgfsetroundjoin%
\pgfsetlinewidth{0.301125pt}%
\definecolor{currentstroke}{rgb}{0.500000,0.500000,0.500000}%
\pgfsetstrokecolor{currentstroke}%
\pgfsetstrokeopacity{0.300000}%
\pgfsetdash{}{0pt}%
\pgfpathmoveto{\pgfqpoint{2.600130in}{1.078170in}}%
\pgfusepath{stroke}%
\end{pgfscope}%
\begin{pgfscope}%
\pgfpathrectangle{\pgfqpoint{0.647939in}{0.492442in}}{\pgfqpoint{3.079299in}{3.079299in}}%
\pgfusepath{clip}%
\pgfsetroundcap%
\pgfsetroundjoin%
\definecolor{currentfill}{rgb}{0.500000,0.500000,0.500000}%
\pgfsetfillcolor{currentfill}%
\pgfsetfillopacity{0.300000}%
\pgfsetlinewidth{0.301125pt}%
\definecolor{currentstroke}{rgb}{0.500000,0.500000,0.500000}%
\pgfsetstrokecolor{currentstroke}%
\pgfsetstrokeopacity{0.300000}%
\pgfsetdash{}{0pt}%
\pgfpathmoveto{\pgfqpoint{0.000000in}{0.000000in}}%
\pgfpathlineto{\pgfqpoint{0.000000in}{0.000000in}}%
\pgfpathclose%
\pgfusepath{stroke,fill}%
\end{pgfscope}%
\begin{pgfscope}%
\pgfpathrectangle{\pgfqpoint{0.647939in}{0.492442in}}{\pgfqpoint{3.079299in}{3.079299in}}%
\pgfusepath{clip}%
\pgfsetroundcap%
\pgfsetroundjoin%
\pgfsetlinewidth{0.301125pt}%
\definecolor{currentstroke}{rgb}{0.500000,0.500000,0.500000}%
\pgfsetstrokecolor{currentstroke}%
\pgfsetstrokeopacity{0.300000}%
\pgfsetdash{}{0pt}%
\pgfpathmoveto{\pgfqpoint{2.740422in}{1.139875in}}%
\pgfusepath{stroke}%
\end{pgfscope}%
\begin{pgfscope}%
\pgfpathrectangle{\pgfqpoint{0.647939in}{0.492442in}}{\pgfqpoint{3.079299in}{3.079299in}}%
\pgfusepath{clip}%
\pgfsetroundcap%
\pgfsetroundjoin%
\definecolor{currentfill}{rgb}{0.500000,0.500000,0.500000}%
\pgfsetfillcolor{currentfill}%
\pgfsetfillopacity{0.300000}%
\pgfsetlinewidth{0.301125pt}%
\definecolor{currentstroke}{rgb}{0.500000,0.500000,0.500000}%
\pgfsetstrokecolor{currentstroke}%
\pgfsetstrokeopacity{0.300000}%
\pgfsetdash{}{0pt}%
\pgfpathmoveto{\pgfqpoint{0.000000in}{0.000000in}}%
\pgfpathlineto{\pgfqpoint{0.000000in}{0.000000in}}%
\pgfpathclose%
\pgfusepath{stroke,fill}%
\end{pgfscope}%
\begin{pgfscope}%
\pgfpathrectangle{\pgfqpoint{0.647939in}{0.492442in}}{\pgfqpoint{3.079299in}{3.079299in}}%
\pgfusepath{clip}%
\pgfsetroundcap%
\pgfsetroundjoin%
\pgfsetlinewidth{0.301125pt}%
\definecolor{currentstroke}{rgb}{0.500000,0.500000,0.500000}%
\pgfsetstrokecolor{currentstroke}%
\pgfsetstrokeopacity{0.300000}%
\pgfsetdash{}{0pt}%
\pgfpathmoveto{\pgfqpoint{2.680248in}{1.242305in}}%
\pgfusepath{stroke}%
\end{pgfscope}%
\begin{pgfscope}%
\pgfpathrectangle{\pgfqpoint{0.647939in}{0.492442in}}{\pgfqpoint{3.079299in}{3.079299in}}%
\pgfusepath{clip}%
\pgfsetroundcap%
\pgfsetroundjoin%
\definecolor{currentfill}{rgb}{0.500000,0.500000,0.500000}%
\pgfsetfillcolor{currentfill}%
\pgfsetfillopacity{0.300000}%
\pgfsetlinewidth{0.301125pt}%
\definecolor{currentstroke}{rgb}{0.500000,0.500000,0.500000}%
\pgfsetstrokecolor{currentstroke}%
\pgfsetstrokeopacity{0.300000}%
\pgfsetdash{}{0pt}%
\pgfpathmoveto{\pgfqpoint{0.000000in}{0.000000in}}%
\pgfpathlineto{\pgfqpoint{0.000000in}{0.000000in}}%
\pgfpathclose%
\pgfusepath{stroke,fill}%
\end{pgfscope}%
\begin{pgfscope}%
\pgfpathrectangle{\pgfqpoint{0.647939in}{0.492442in}}{\pgfqpoint{3.079299in}{3.079299in}}%
\pgfusepath{clip}%
\pgfsetroundcap%
\pgfsetroundjoin%
\pgfsetlinewidth{0.301125pt}%
\definecolor{currentstroke}{rgb}{0.500000,0.500000,0.500000}%
\pgfsetstrokecolor{currentstroke}%
\pgfsetstrokeopacity{0.300000}%
\pgfsetdash{}{0pt}%
\pgfpathmoveto{\pgfqpoint{2.819805in}{1.295610in}}%
\pgfusepath{stroke}%
\end{pgfscope}%
\begin{pgfscope}%
\pgfpathrectangle{\pgfqpoint{0.647939in}{0.492442in}}{\pgfqpoint{3.079299in}{3.079299in}}%
\pgfusepath{clip}%
\pgfsetroundcap%
\pgfsetroundjoin%
\definecolor{currentfill}{rgb}{0.500000,0.500000,0.500000}%
\pgfsetfillcolor{currentfill}%
\pgfsetfillopacity{0.300000}%
\pgfsetlinewidth{0.301125pt}%
\definecolor{currentstroke}{rgb}{0.500000,0.500000,0.500000}%
\pgfsetstrokecolor{currentstroke}%
\pgfsetstrokeopacity{0.300000}%
\pgfsetdash{}{0pt}%
\pgfpathmoveto{\pgfqpoint{0.000000in}{0.000000in}}%
\pgfpathlineto{\pgfqpoint{0.000000in}{0.000000in}}%
\pgfpathclose%
\pgfusepath{stroke,fill}%
\end{pgfscope}%
\begin{pgfscope}%
\pgfpathrectangle{\pgfqpoint{0.647939in}{0.492442in}}{\pgfqpoint{3.079299in}{3.079299in}}%
\pgfusepath{clip}%
\pgfsetroundcap%
\pgfsetroundjoin%
\pgfsetlinewidth{0.301125pt}%
\definecolor{currentstroke}{rgb}{0.500000,0.500000,0.500000}%
\pgfsetstrokecolor{currentstroke}%
\pgfsetstrokeopacity{0.300000}%
\pgfsetdash{}{0pt}%
\pgfpathmoveto{\pgfqpoint{2.697384in}{1.425007in}}%
\pgfusepath{stroke}%
\end{pgfscope}%
\begin{pgfscope}%
\pgfpathrectangle{\pgfqpoint{0.647939in}{0.492442in}}{\pgfqpoint{3.079299in}{3.079299in}}%
\pgfusepath{clip}%
\pgfsetroundcap%
\pgfsetroundjoin%
\definecolor{currentfill}{rgb}{0.500000,0.500000,0.500000}%
\pgfsetfillcolor{currentfill}%
\pgfsetfillopacity{0.300000}%
\pgfsetlinewidth{0.301125pt}%
\definecolor{currentstroke}{rgb}{0.500000,0.500000,0.500000}%
\pgfsetstrokecolor{currentstroke}%
\pgfsetstrokeopacity{0.300000}%
\pgfsetdash{}{0pt}%
\pgfpathmoveto{\pgfqpoint{0.000000in}{0.000000in}}%
\pgfpathlineto{\pgfqpoint{0.000000in}{0.000000in}}%
\pgfpathclose%
\pgfusepath{stroke,fill}%
\end{pgfscope}%
\begin{pgfscope}%
\pgfpathrectangle{\pgfqpoint{0.647939in}{0.492442in}}{\pgfqpoint{3.079299in}{3.079299in}}%
\pgfusepath{clip}%
\pgfsetroundcap%
\pgfsetroundjoin%
\pgfsetlinewidth{0.301125pt}%
\definecolor{currentstroke}{rgb}{0.500000,0.500000,0.500000}%
\pgfsetstrokecolor{currentstroke}%
\pgfsetstrokeopacity{0.300000}%
\pgfsetdash{}{0pt}%
\pgfpathmoveto{\pgfqpoint{2.837592in}{1.475137in}}%
\pgfusepath{stroke}%
\end{pgfscope}%
\begin{pgfscope}%
\pgfpathrectangle{\pgfqpoint{0.647939in}{0.492442in}}{\pgfqpoint{3.079299in}{3.079299in}}%
\pgfusepath{clip}%
\pgfsetroundcap%
\pgfsetroundjoin%
\definecolor{currentfill}{rgb}{0.500000,0.500000,0.500000}%
\pgfsetfillcolor{currentfill}%
\pgfsetfillopacity{0.300000}%
\pgfsetlinewidth{0.301125pt}%
\definecolor{currentstroke}{rgb}{0.500000,0.500000,0.500000}%
\pgfsetstrokecolor{currentstroke}%
\pgfsetstrokeopacity{0.300000}%
\pgfsetdash{}{0pt}%
\pgfpathmoveto{\pgfqpoint{0.000000in}{0.000000in}}%
\pgfpathlineto{\pgfqpoint{0.000000in}{0.000000in}}%
\pgfpathclose%
\pgfusepath{stroke,fill}%
\end{pgfscope}%
\begin{pgfscope}%
\pgfpathrectangle{\pgfqpoint{0.647939in}{0.492442in}}{\pgfqpoint{3.079299in}{3.079299in}}%
\pgfusepath{clip}%
\pgfsetroundcap%
\pgfsetroundjoin%
\pgfsetlinewidth{0.301125pt}%
\definecolor{currentstroke}{rgb}{0.500000,0.500000,0.500000}%
\pgfsetstrokecolor{currentstroke}%
\pgfsetstrokeopacity{0.300000}%
\pgfsetdash{}{0pt}%
\pgfpathmoveto{\pgfqpoint{2.848997in}{1.568161in}}%
\pgfusepath{stroke}%
\end{pgfscope}%
\begin{pgfscope}%
\pgfpathrectangle{\pgfqpoint{0.647939in}{0.492442in}}{\pgfqpoint{3.079299in}{3.079299in}}%
\pgfusepath{clip}%
\pgfsetroundcap%
\pgfsetroundjoin%
\definecolor{currentfill}{rgb}{0.500000,0.500000,0.500000}%
\pgfsetfillcolor{currentfill}%
\pgfsetfillopacity{0.300000}%
\pgfsetlinewidth{0.301125pt}%
\definecolor{currentstroke}{rgb}{0.500000,0.500000,0.500000}%
\pgfsetstrokecolor{currentstroke}%
\pgfsetstrokeopacity{0.300000}%
\pgfsetdash{}{0pt}%
\pgfpathmoveto{\pgfqpoint{0.000000in}{0.000000in}}%
\pgfpathlineto{\pgfqpoint{0.000000in}{0.000000in}}%
\pgfpathclose%
\pgfusepath{stroke,fill}%
\end{pgfscope}%
\begin{pgfscope}%
\pgfpathrectangle{\pgfqpoint{0.647939in}{0.492442in}}{\pgfqpoint{3.079299in}{3.079299in}}%
\pgfusepath{clip}%
\pgfsetroundcap%
\pgfsetroundjoin%
\pgfsetlinewidth{0.301125pt}%
\definecolor{currentstroke}{rgb}{0.500000,0.500000,0.500000}%
\pgfsetstrokecolor{currentstroke}%
\pgfsetstrokeopacity{0.300000}%
\pgfsetdash{}{0pt}%
\pgfpathmoveto{\pgfqpoint{2.862716in}{1.663844in}}%
\pgfusepath{stroke}%
\end{pgfscope}%
\begin{pgfscope}%
\pgfpathrectangle{\pgfqpoint{0.647939in}{0.492442in}}{\pgfqpoint{3.079299in}{3.079299in}}%
\pgfusepath{clip}%
\pgfsetroundcap%
\pgfsetroundjoin%
\definecolor{currentfill}{rgb}{0.500000,0.500000,0.500000}%
\pgfsetfillcolor{currentfill}%
\pgfsetfillopacity{0.300000}%
\pgfsetlinewidth{0.301125pt}%
\definecolor{currentstroke}{rgb}{0.500000,0.500000,0.500000}%
\pgfsetstrokecolor{currentstroke}%
\pgfsetstrokeopacity{0.300000}%
\pgfsetdash{}{0pt}%
\pgfpathmoveto{\pgfqpoint{0.000000in}{0.000000in}}%
\pgfpathlineto{\pgfqpoint{0.000000in}{0.000000in}}%
\pgfpathclose%
\pgfusepath{stroke,fill}%
\end{pgfscope}%
\begin{pgfscope}%
\pgfpathrectangle{\pgfqpoint{0.647939in}{0.492442in}}{\pgfqpoint{3.079299in}{3.079299in}}%
\pgfusepath{clip}%
\pgfsetroundcap%
\pgfsetroundjoin%
\pgfsetlinewidth{0.301125pt}%
\definecolor{currentstroke}{rgb}{0.500000,0.500000,0.500000}%
\pgfsetstrokecolor{currentstroke}%
\pgfsetstrokeopacity{0.300000}%
\pgfsetdash{}{0pt}%
\pgfpathmoveto{\pgfqpoint{2.995887in}{1.690710in}}%
\pgfusepath{stroke}%
\end{pgfscope}%
\begin{pgfscope}%
\pgfpathrectangle{\pgfqpoint{0.647939in}{0.492442in}}{\pgfqpoint{3.079299in}{3.079299in}}%
\pgfusepath{clip}%
\pgfsetroundcap%
\pgfsetroundjoin%
\definecolor{currentfill}{rgb}{0.500000,0.500000,0.500000}%
\pgfsetfillcolor{currentfill}%
\pgfsetfillopacity{0.300000}%
\pgfsetlinewidth{0.301125pt}%
\definecolor{currentstroke}{rgb}{0.500000,0.500000,0.500000}%
\pgfsetstrokecolor{currentstroke}%
\pgfsetstrokeopacity{0.300000}%
\pgfsetdash{}{0pt}%
\pgfpathmoveto{\pgfqpoint{0.000000in}{0.000000in}}%
\pgfpathlineto{\pgfqpoint{0.000000in}{0.000000in}}%
\pgfpathclose%
\pgfusepath{stroke,fill}%
\end{pgfscope}%
\begin{pgfscope}%
\pgfpathrectangle{\pgfqpoint{0.647939in}{0.492442in}}{\pgfqpoint{3.079299in}{3.079299in}}%
\pgfusepath{clip}%
\pgfsetroundcap%
\pgfsetroundjoin%
\pgfsetlinewidth{0.301125pt}%
\definecolor{currentstroke}{rgb}{0.500000,0.500000,0.500000}%
\pgfsetstrokecolor{currentstroke}%
\pgfsetstrokeopacity{0.300000}%
\pgfsetdash{}{0pt}%
\pgfpathmoveto{\pgfqpoint{2.899985in}{1.864589in}}%
\pgfusepath{stroke}%
\end{pgfscope}%
\begin{pgfscope}%
\pgfpathrectangle{\pgfqpoint{0.647939in}{0.492442in}}{\pgfqpoint{3.079299in}{3.079299in}}%
\pgfusepath{clip}%
\pgfsetroundcap%
\pgfsetroundjoin%
\definecolor{currentfill}{rgb}{0.500000,0.500000,0.500000}%
\pgfsetfillcolor{currentfill}%
\pgfsetfillopacity{0.300000}%
\pgfsetlinewidth{0.301125pt}%
\definecolor{currentstroke}{rgb}{0.500000,0.500000,0.500000}%
\pgfsetstrokecolor{currentstroke}%
\pgfsetstrokeopacity{0.300000}%
\pgfsetdash{}{0pt}%
\pgfpathmoveto{\pgfqpoint{0.000000in}{0.000000in}}%
\pgfpathlineto{\pgfqpoint{0.000000in}{0.000000in}}%
\pgfpathclose%
\pgfusepath{stroke,fill}%
\end{pgfscope}%
\begin{pgfscope}%
\pgfpathrectangle{\pgfqpoint{0.647939in}{0.492442in}}{\pgfqpoint{3.079299in}{3.079299in}}%
\pgfusepath{clip}%
\pgfsetroundcap%
\pgfsetroundjoin%
\pgfsetlinewidth{0.301125pt}%
\definecolor{currentstroke}{rgb}{0.500000,0.500000,0.500000}%
\pgfsetstrokecolor{currentstroke}%
\pgfsetstrokeopacity{0.300000}%
\pgfsetdash{}{0pt}%
\pgfpathmoveto{\pgfqpoint{2.761177in}{2.430566in}}%
\pgfusepath{stroke}%
\end{pgfscope}%
\begin{pgfscope}%
\pgfpathrectangle{\pgfqpoint{0.647939in}{0.492442in}}{\pgfqpoint{3.079299in}{3.079299in}}%
\pgfusepath{clip}%
\pgfsetroundcap%
\pgfsetroundjoin%
\definecolor{currentfill}{rgb}{0.500000,0.500000,0.500000}%
\pgfsetfillcolor{currentfill}%
\pgfsetfillopacity{0.300000}%
\pgfsetlinewidth{0.301125pt}%
\definecolor{currentstroke}{rgb}{0.500000,0.500000,0.500000}%
\pgfsetstrokecolor{currentstroke}%
\pgfsetstrokeopacity{0.300000}%
\pgfsetdash{}{0pt}%
\pgfpathmoveto{\pgfqpoint{0.000000in}{0.000000in}}%
\pgfpathlineto{\pgfqpoint{0.000000in}{0.000000in}}%
\pgfpathclose%
\pgfusepath{stroke,fill}%
\end{pgfscope}%
\begin{pgfscope}%
\pgfpathrectangle{\pgfqpoint{0.647939in}{0.492442in}}{\pgfqpoint{3.079299in}{3.079299in}}%
\pgfusepath{clip}%
\pgfsetroundcap%
\pgfsetroundjoin%
\pgfsetlinewidth{0.301125pt}%
\definecolor{currentstroke}{rgb}{0.500000,0.500000,0.500000}%
\pgfsetstrokecolor{currentstroke}%
\pgfsetstrokeopacity{0.300000}%
\pgfsetdash{}{0pt}%
\pgfpathmoveto{\pgfqpoint{2.996946in}{2.190745in}}%
\pgfusepath{stroke}%
\end{pgfscope}%
\begin{pgfscope}%
\pgfpathrectangle{\pgfqpoint{0.647939in}{0.492442in}}{\pgfqpoint{3.079299in}{3.079299in}}%
\pgfusepath{clip}%
\pgfsetroundcap%
\pgfsetroundjoin%
\definecolor{currentfill}{rgb}{0.500000,0.500000,0.500000}%
\pgfsetfillcolor{currentfill}%
\pgfsetfillopacity{0.300000}%
\pgfsetlinewidth{0.301125pt}%
\definecolor{currentstroke}{rgb}{0.500000,0.500000,0.500000}%
\pgfsetstrokecolor{currentstroke}%
\pgfsetstrokeopacity{0.300000}%
\pgfsetdash{}{0pt}%
\pgfpathmoveto{\pgfqpoint{0.000000in}{0.000000in}}%
\pgfpathlineto{\pgfqpoint{0.000000in}{0.000000in}}%
\pgfpathclose%
\pgfusepath{stroke,fill}%
\end{pgfscope}%
\begin{pgfscope}%
\pgfpathrectangle{\pgfqpoint{0.647939in}{0.492442in}}{\pgfqpoint{3.079299in}{3.079299in}}%
\pgfusepath{clip}%
\pgfsetroundcap%
\pgfsetroundjoin%
\pgfsetlinewidth{0.301125pt}%
\definecolor{currentstroke}{rgb}{0.500000,0.500000,0.500000}%
\pgfsetstrokecolor{currentstroke}%
\pgfsetstrokeopacity{0.300000}%
\pgfsetdash{}{0pt}%
\pgfpathmoveto{\pgfqpoint{3.148824in}{2.127106in}}%
\pgfusepath{stroke}%
\end{pgfscope}%
\begin{pgfscope}%
\pgfpathrectangle{\pgfqpoint{0.647939in}{0.492442in}}{\pgfqpoint{3.079299in}{3.079299in}}%
\pgfusepath{clip}%
\pgfsetroundcap%
\pgfsetroundjoin%
\definecolor{currentfill}{rgb}{0.500000,0.500000,0.500000}%
\pgfsetfillcolor{currentfill}%
\pgfsetfillopacity{0.300000}%
\pgfsetlinewidth{0.301125pt}%
\definecolor{currentstroke}{rgb}{0.500000,0.500000,0.500000}%
\pgfsetstrokecolor{currentstroke}%
\pgfsetstrokeopacity{0.300000}%
\pgfsetdash{}{0pt}%
\pgfpathmoveto{\pgfqpoint{0.000000in}{0.000000in}}%
\pgfpathlineto{\pgfqpoint{0.000000in}{0.000000in}}%
\pgfpathclose%
\pgfusepath{stroke,fill}%
\end{pgfscope}%
\begin{pgfscope}%
\pgfpathrectangle{\pgfqpoint{0.647939in}{0.492442in}}{\pgfqpoint{3.079299in}{3.079299in}}%
\pgfusepath{clip}%
\pgfsetroundcap%
\pgfsetroundjoin%
\pgfsetlinewidth{0.301125pt}%
\definecolor{currentstroke}{rgb}{0.500000,0.500000,0.500000}%
\pgfsetstrokecolor{currentstroke}%
\pgfsetstrokeopacity{0.300000}%
\pgfsetdash{}{0pt}%
\pgfpathmoveto{\pgfqpoint{3.159400in}{2.660835in}}%
\pgfusepath{stroke}%
\end{pgfscope}%
\begin{pgfscope}%
\pgfpathrectangle{\pgfqpoint{0.647939in}{0.492442in}}{\pgfqpoint{3.079299in}{3.079299in}}%
\pgfusepath{clip}%
\pgfsetroundcap%
\pgfsetroundjoin%
\definecolor{currentfill}{rgb}{0.500000,0.500000,0.500000}%
\pgfsetfillcolor{currentfill}%
\pgfsetfillopacity{0.300000}%
\pgfsetlinewidth{0.301125pt}%
\definecolor{currentstroke}{rgb}{0.500000,0.500000,0.500000}%
\pgfsetstrokecolor{currentstroke}%
\pgfsetstrokeopacity{0.300000}%
\pgfsetdash{}{0pt}%
\pgfpathmoveto{\pgfqpoint{0.000000in}{0.000000in}}%
\pgfpathlineto{\pgfqpoint{0.000000in}{0.000000in}}%
\pgfpathclose%
\pgfusepath{stroke,fill}%
\end{pgfscope}%
\begin{pgfscope}%
\pgfpathrectangle{\pgfqpoint{0.647939in}{0.492442in}}{\pgfqpoint{3.079299in}{3.079299in}}%
\pgfusepath{clip}%
\pgfsetroundcap%
\pgfsetroundjoin%
\pgfsetlinewidth{0.301125pt}%
\definecolor{currentstroke}{rgb}{0.500000,0.500000,0.500000}%
\pgfsetstrokecolor{currentstroke}%
\pgfsetstrokeopacity{0.300000}%
\pgfsetdash{}{0pt}%
\pgfpathmoveto{\pgfqpoint{3.298070in}{2.663489in}}%
\pgfusepath{stroke}%
\end{pgfscope}%
\begin{pgfscope}%
\pgfpathrectangle{\pgfqpoint{0.647939in}{0.492442in}}{\pgfqpoint{3.079299in}{3.079299in}}%
\pgfusepath{clip}%
\pgfsetroundcap%
\pgfsetroundjoin%
\definecolor{currentfill}{rgb}{0.500000,0.500000,0.500000}%
\pgfsetfillcolor{currentfill}%
\pgfsetfillopacity{0.300000}%
\pgfsetlinewidth{0.301125pt}%
\definecolor{currentstroke}{rgb}{0.500000,0.500000,0.500000}%
\pgfsetstrokecolor{currentstroke}%
\pgfsetstrokeopacity{0.300000}%
\pgfsetdash{}{0pt}%
\pgfpathmoveto{\pgfqpoint{0.000000in}{0.000000in}}%
\pgfpathlineto{\pgfqpoint{0.000000in}{0.000000in}}%
\pgfpathclose%
\pgfusepath{stroke,fill}%
\end{pgfscope}%
\begin{pgfscope}%
\pgfpathrectangle{\pgfqpoint{0.647939in}{0.492442in}}{\pgfqpoint{3.079299in}{3.079299in}}%
\pgfusepath{clip}%
\pgfsetroundcap%
\pgfsetroundjoin%
\pgfsetlinewidth{0.301125pt}%
\definecolor{currentstroke}{rgb}{0.500000,0.500000,0.500000}%
\pgfsetstrokecolor{currentstroke}%
\pgfsetstrokeopacity{0.300000}%
\pgfsetdash{}{0pt}%
\pgfpathmoveto{\pgfqpoint{3.562975in}{2.126581in}}%
\pgfusepath{stroke}%
\end{pgfscope}%
\begin{pgfscope}%
\pgfpathrectangle{\pgfqpoint{0.647939in}{0.492442in}}{\pgfqpoint{3.079299in}{3.079299in}}%
\pgfusepath{clip}%
\pgfsetroundcap%
\pgfsetroundjoin%
\definecolor{currentfill}{rgb}{0.500000,0.500000,0.500000}%
\pgfsetfillcolor{currentfill}%
\pgfsetfillopacity{0.300000}%
\pgfsetlinewidth{0.301125pt}%
\definecolor{currentstroke}{rgb}{0.500000,0.500000,0.500000}%
\pgfsetstrokecolor{currentstroke}%
\pgfsetstrokeopacity{0.300000}%
\pgfsetdash{}{0pt}%
\pgfpathmoveto{\pgfqpoint{0.000000in}{0.000000in}}%
\pgfpathlineto{\pgfqpoint{0.000000in}{0.000000in}}%
\pgfpathclose%
\pgfusepath{stroke,fill}%
\end{pgfscope}%
\begin{pgfscope}%
\pgfpathrectangle{\pgfqpoint{0.647939in}{0.492442in}}{\pgfqpoint{3.079299in}{3.079299in}}%
\pgfusepath{clip}%
\pgfsetroundcap%
\pgfsetroundjoin%
\pgfsetlinewidth{0.301125pt}%
\definecolor{currentstroke}{rgb}{0.500000,0.500000,0.500000}%
\pgfsetstrokecolor{currentstroke}%
\pgfsetstrokeopacity{0.300000}%
\pgfsetdash{}{0pt}%
\pgfpathmoveto{\pgfqpoint{3.471066in}{2.673976in}}%
\pgfusepath{stroke}%
\end{pgfscope}%
\begin{pgfscope}%
\pgfpathrectangle{\pgfqpoint{0.647939in}{0.492442in}}{\pgfqpoint{3.079299in}{3.079299in}}%
\pgfusepath{clip}%
\pgfsetroundcap%
\pgfsetroundjoin%
\definecolor{currentfill}{rgb}{0.500000,0.500000,0.500000}%
\pgfsetfillcolor{currentfill}%
\pgfsetfillopacity{0.300000}%
\pgfsetlinewidth{0.301125pt}%
\definecolor{currentstroke}{rgb}{0.500000,0.500000,0.500000}%
\pgfsetstrokecolor{currentstroke}%
\pgfsetstrokeopacity{0.300000}%
\pgfsetdash{}{0pt}%
\pgfpathmoveto{\pgfqpoint{0.000000in}{0.000000in}}%
\pgfpathlineto{\pgfqpoint{0.000000in}{0.000000in}}%
\pgfpathclose%
\pgfusepath{stroke,fill}%
\end{pgfscope}%
\begin{pgfscope}%
\pgfpathrectangle{\pgfqpoint{0.647939in}{0.492442in}}{\pgfqpoint{3.079299in}{3.079299in}}%
\pgfusepath{clip}%
\pgfsetroundcap%
\pgfsetroundjoin%
\pgfsetlinewidth{0.301125pt}%
\definecolor{currentstroke}{rgb}{0.500000,0.500000,0.500000}%
\pgfsetstrokecolor{currentstroke}%
\pgfsetstrokeopacity{0.300000}%
\pgfsetdash{}{0pt}%
\pgfpathmoveto{\pgfqpoint{3.565392in}{2.711130in}}%
\pgfusepath{stroke}%
\end{pgfscope}%
\begin{pgfscope}%
\pgfpathrectangle{\pgfqpoint{0.647939in}{0.492442in}}{\pgfqpoint{3.079299in}{3.079299in}}%
\pgfusepath{clip}%
\pgfsetroundcap%
\pgfsetroundjoin%
\definecolor{currentfill}{rgb}{0.500000,0.500000,0.500000}%
\pgfsetfillcolor{currentfill}%
\pgfsetfillopacity{0.300000}%
\pgfsetlinewidth{0.301125pt}%
\definecolor{currentstroke}{rgb}{0.500000,0.500000,0.500000}%
\pgfsetstrokecolor{currentstroke}%
\pgfsetstrokeopacity{0.300000}%
\pgfsetdash{}{0pt}%
\pgfpathmoveto{\pgfqpoint{0.000000in}{0.000000in}}%
\pgfpathlineto{\pgfqpoint{0.000000in}{0.000000in}}%
\pgfpathclose%
\pgfusepath{stroke,fill}%
\end{pgfscope}%
\begin{pgfscope}%
\pgfpathrectangle{\pgfqpoint{0.647939in}{0.492442in}}{\pgfqpoint{3.079299in}{3.079299in}}%
\pgfusepath{clip}%
\pgfsetroundcap%
\pgfsetroundjoin%
\pgfsetlinewidth{0.301125pt}%
\definecolor{currentstroke}{rgb}{0.500000,0.500000,0.500000}%
\pgfsetstrokecolor{currentstroke}%
\pgfsetstrokeopacity{0.300000}%
\pgfsetdash{}{0pt}%
\pgfpathmoveto{\pgfqpoint{3.642110in}{2.736665in}}%
\pgfusepath{stroke}%
\end{pgfscope}%
\begin{pgfscope}%
\pgfpathrectangle{\pgfqpoint{0.647939in}{0.492442in}}{\pgfqpoint{3.079299in}{3.079299in}}%
\pgfusepath{clip}%
\pgfsetroundcap%
\pgfsetroundjoin%
\definecolor{currentfill}{rgb}{0.500000,0.500000,0.500000}%
\pgfsetfillcolor{currentfill}%
\pgfsetfillopacity{0.300000}%
\pgfsetlinewidth{0.301125pt}%
\definecolor{currentstroke}{rgb}{0.500000,0.500000,0.500000}%
\pgfsetstrokecolor{currentstroke}%
\pgfsetstrokeopacity{0.300000}%
\pgfsetdash{}{0pt}%
\pgfpathmoveto{\pgfqpoint{0.000000in}{0.000000in}}%
\pgfpathlineto{\pgfqpoint{0.000000in}{0.000000in}}%
\pgfpathclose%
\pgfusepath{stroke,fill}%
\end{pgfscope}%
\begin{pgfscope}%
\pgfpathrectangle{\pgfqpoint{0.647939in}{0.492442in}}{\pgfqpoint{3.079299in}{3.079299in}}%
\pgfusepath{clip}%
\pgfsetroundcap%
\pgfsetroundjoin%
\pgfsetlinewidth{0.301125pt}%
\definecolor{currentstroke}{rgb}{0.500000,0.500000,0.500000}%
\pgfsetstrokecolor{currentstroke}%
\pgfsetstrokeopacity{0.300000}%
\pgfsetdash{}{0pt}%
\pgfpathmoveto{\pgfqpoint{3.697530in}{2.683159in}}%
\pgfusepath{stroke}%
\end{pgfscope}%
\begin{pgfscope}%
\pgfpathrectangle{\pgfqpoint{0.647939in}{0.492442in}}{\pgfqpoint{3.079299in}{3.079299in}}%
\pgfusepath{clip}%
\pgfsetroundcap%
\pgfsetroundjoin%
\definecolor{currentfill}{rgb}{0.500000,0.500000,0.500000}%
\pgfsetfillcolor{currentfill}%
\pgfsetfillopacity{0.300000}%
\pgfsetlinewidth{0.301125pt}%
\definecolor{currentstroke}{rgb}{0.500000,0.500000,0.500000}%
\pgfsetstrokecolor{currentstroke}%
\pgfsetstrokeopacity{0.300000}%
\pgfsetdash{}{0pt}%
\pgfpathmoveto{\pgfqpoint{0.000000in}{0.000000in}}%
\pgfpathlineto{\pgfqpoint{0.000000in}{0.000000in}}%
\pgfpathclose%
\pgfusepath{stroke,fill}%
\end{pgfscope}%
\begin{pgfscope}%
\pgfpathrectangle{\pgfqpoint{0.647939in}{0.492442in}}{\pgfqpoint{3.079299in}{3.079299in}}%
\pgfusepath{clip}%
\pgfsetroundcap%
\pgfsetroundjoin%
\pgfsetlinewidth{0.301125pt}%
\definecolor{currentstroke}{rgb}{0.500000,0.500000,0.500000}%
\pgfsetstrokecolor{currentstroke}%
\pgfsetstrokeopacity{0.300000}%
\pgfsetdash{}{0pt}%
\pgfpathmoveto{\pgfqpoint{3.609518in}{3.229287in}}%
\pgfusepath{stroke}%
\end{pgfscope}%
\begin{pgfscope}%
\pgfpathrectangle{\pgfqpoint{0.647939in}{0.492442in}}{\pgfqpoint{3.079299in}{3.079299in}}%
\pgfusepath{clip}%
\pgfsetroundcap%
\pgfsetroundjoin%
\definecolor{currentfill}{rgb}{0.500000,0.500000,0.500000}%
\pgfsetfillcolor{currentfill}%
\pgfsetfillopacity{0.300000}%
\pgfsetlinewidth{0.301125pt}%
\definecolor{currentstroke}{rgb}{0.500000,0.500000,0.500000}%
\pgfsetstrokecolor{currentstroke}%
\pgfsetstrokeopacity{0.300000}%
\pgfsetdash{}{0pt}%
\pgfpathmoveto{\pgfqpoint{0.000000in}{0.000000in}}%
\pgfpathlineto{\pgfqpoint{0.000000in}{0.000000in}}%
\pgfpathclose%
\pgfusepath{stroke,fill}%
\end{pgfscope}%
\begin{pgfscope}%
\pgfpathrectangle{\pgfqpoint{0.647939in}{0.492442in}}{\pgfqpoint{3.079299in}{3.079299in}}%
\pgfusepath{clip}%
\pgfsetroundcap%
\pgfsetroundjoin%
\pgfsetlinewidth{0.301125pt}%
\definecolor{currentstroke}{rgb}{0.500000,0.500000,0.500000}%
\pgfsetstrokecolor{currentstroke}%
\pgfsetstrokeopacity{0.300000}%
\pgfsetdash{}{0pt}%
\pgfpathmoveto{\pgfqpoint{3.426506in}{3.358880in}}%
\pgfusepath{stroke}%
\end{pgfscope}%
\begin{pgfscope}%
\pgfpathrectangle{\pgfqpoint{0.647939in}{0.492442in}}{\pgfqpoint{3.079299in}{3.079299in}}%
\pgfusepath{clip}%
\pgfsetroundcap%
\pgfsetroundjoin%
\definecolor{currentfill}{rgb}{0.500000,0.500000,0.500000}%
\pgfsetfillcolor{currentfill}%
\pgfsetfillopacity{0.300000}%
\pgfsetlinewidth{0.301125pt}%
\definecolor{currentstroke}{rgb}{0.500000,0.500000,0.500000}%
\pgfsetstrokecolor{currentstroke}%
\pgfsetstrokeopacity{0.300000}%
\pgfsetdash{}{0pt}%
\pgfpathmoveto{\pgfqpoint{0.000000in}{0.000000in}}%
\pgfpathlineto{\pgfqpoint{0.000000in}{0.000000in}}%
\pgfpathclose%
\pgfusepath{stroke,fill}%
\end{pgfscope}%
\begin{pgfscope}%
\pgfpathrectangle{\pgfqpoint{0.647939in}{0.492442in}}{\pgfqpoint{3.079299in}{3.079299in}}%
\pgfusepath{clip}%
\pgfsetroundcap%
\pgfsetroundjoin%
\pgfsetlinewidth{0.301125pt}%
\definecolor{currentstroke}{rgb}{0.500000,0.500000,0.500000}%
\pgfsetstrokecolor{currentstroke}%
\pgfsetstrokeopacity{0.300000}%
\pgfsetdash{}{0pt}%
\pgfpathmoveto{\pgfqpoint{2.157963in}{2.846179in}}%
\pgfusepath{stroke}%
\end{pgfscope}%
\begin{pgfscope}%
\pgfpathrectangle{\pgfqpoint{0.647939in}{0.492442in}}{\pgfqpoint{3.079299in}{3.079299in}}%
\pgfusepath{clip}%
\pgfsetroundcap%
\pgfsetroundjoin%
\definecolor{currentfill}{rgb}{0.500000,0.500000,0.500000}%
\pgfsetfillcolor{currentfill}%
\pgfsetfillopacity{0.300000}%
\pgfsetlinewidth{0.301125pt}%
\definecolor{currentstroke}{rgb}{0.500000,0.500000,0.500000}%
\pgfsetstrokecolor{currentstroke}%
\pgfsetstrokeopacity{0.300000}%
\pgfsetdash{}{0pt}%
\pgfpathmoveto{\pgfqpoint{0.000000in}{0.000000in}}%
\pgfpathlineto{\pgfqpoint{0.000000in}{0.000000in}}%
\pgfpathclose%
\pgfusepath{stroke,fill}%
\end{pgfscope}%
\begin{pgfscope}%
\pgfpathrectangle{\pgfqpoint{0.647939in}{0.492442in}}{\pgfqpoint{3.079299in}{3.079299in}}%
\pgfusepath{clip}%
\pgfsetroundcap%
\pgfsetroundjoin%
\pgfsetlinewidth{0.301125pt}%
\definecolor{currentstroke}{rgb}{0.500000,0.500000,0.500000}%
\pgfsetstrokecolor{currentstroke}%
\pgfsetstrokeopacity{0.300000}%
\pgfsetdash{}{0pt}%
\pgfpathmoveto{\pgfqpoint{2.032977in}{3.110635in}}%
\pgfusepath{stroke}%
\end{pgfscope}%
\begin{pgfscope}%
\pgfpathrectangle{\pgfqpoint{0.647939in}{0.492442in}}{\pgfqpoint{3.079299in}{3.079299in}}%
\pgfusepath{clip}%
\pgfsetroundcap%
\pgfsetroundjoin%
\definecolor{currentfill}{rgb}{0.500000,0.500000,0.500000}%
\pgfsetfillcolor{currentfill}%
\pgfsetfillopacity{0.300000}%
\pgfsetlinewidth{0.301125pt}%
\definecolor{currentstroke}{rgb}{0.500000,0.500000,0.500000}%
\pgfsetstrokecolor{currentstroke}%
\pgfsetstrokeopacity{0.300000}%
\pgfsetdash{}{0pt}%
\pgfpathmoveto{\pgfqpoint{0.000000in}{0.000000in}}%
\pgfpathlineto{\pgfqpoint{0.000000in}{0.000000in}}%
\pgfpathclose%
\pgfusepath{stroke,fill}%
\end{pgfscope}%
\begin{pgfscope}%
\pgfpathrectangle{\pgfqpoint{0.647939in}{0.492442in}}{\pgfqpoint{3.079299in}{3.079299in}}%
\pgfusepath{clip}%
\pgfsetroundcap%
\pgfsetroundjoin%
\pgfsetlinewidth{0.301125pt}%
\definecolor{currentstroke}{rgb}{0.500000,0.500000,0.500000}%
\pgfsetstrokecolor{currentstroke}%
\pgfsetstrokeopacity{0.300000}%
\pgfsetdash{}{0pt}%
\pgfpathmoveto{\pgfqpoint{1.897830in}{3.272078in}}%
\pgfusepath{stroke}%
\end{pgfscope}%
\begin{pgfscope}%
\pgfpathrectangle{\pgfqpoint{0.647939in}{0.492442in}}{\pgfqpoint{3.079299in}{3.079299in}}%
\pgfusepath{clip}%
\pgfsetroundcap%
\pgfsetroundjoin%
\definecolor{currentfill}{rgb}{0.500000,0.500000,0.500000}%
\pgfsetfillcolor{currentfill}%
\pgfsetfillopacity{0.300000}%
\pgfsetlinewidth{0.301125pt}%
\definecolor{currentstroke}{rgb}{0.500000,0.500000,0.500000}%
\pgfsetstrokecolor{currentstroke}%
\pgfsetstrokeopacity{0.300000}%
\pgfsetdash{}{0pt}%
\pgfpathmoveto{\pgfqpoint{0.000000in}{0.000000in}}%
\pgfpathlineto{\pgfqpoint{0.000000in}{0.000000in}}%
\pgfpathclose%
\pgfusepath{stroke,fill}%
\end{pgfscope}%
\begin{pgfscope}%
\pgfpathrectangle{\pgfqpoint{0.647939in}{0.492442in}}{\pgfqpoint{3.079299in}{3.079299in}}%
\pgfusepath{clip}%
\pgfsetroundcap%
\pgfsetroundjoin%
\pgfsetlinewidth{0.301125pt}%
\definecolor{currentstroke}{rgb}{0.500000,0.500000,0.500000}%
\pgfsetstrokecolor{currentstroke}%
\pgfsetstrokeopacity{0.300000}%
\pgfsetdash{}{0pt}%
\pgfpathmoveto{\pgfqpoint{1.857756in}{3.386153in}}%
\pgfusepath{stroke}%
\end{pgfscope}%
\begin{pgfscope}%
\pgfpathrectangle{\pgfqpoint{0.647939in}{0.492442in}}{\pgfqpoint{3.079299in}{3.079299in}}%
\pgfusepath{clip}%
\pgfsetroundcap%
\pgfsetroundjoin%
\definecolor{currentfill}{rgb}{0.500000,0.500000,0.500000}%
\pgfsetfillcolor{currentfill}%
\pgfsetfillopacity{0.300000}%
\pgfsetlinewidth{0.301125pt}%
\definecolor{currentstroke}{rgb}{0.500000,0.500000,0.500000}%
\pgfsetstrokecolor{currentstroke}%
\pgfsetstrokeopacity{0.300000}%
\pgfsetdash{}{0pt}%
\pgfpathmoveto{\pgfqpoint{0.000000in}{0.000000in}}%
\pgfpathlineto{\pgfqpoint{0.000000in}{0.000000in}}%
\pgfpathclose%
\pgfusepath{stroke,fill}%
\end{pgfscope}%
\begin{pgfscope}%
\pgfpathrectangle{\pgfqpoint{0.647939in}{0.492442in}}{\pgfqpoint{3.079299in}{3.079299in}}%
\pgfusepath{clip}%
\pgfsetroundcap%
\pgfsetroundjoin%
\pgfsetlinewidth{0.301125pt}%
\definecolor{currentstroke}{rgb}{0.500000,0.500000,0.500000}%
\pgfsetstrokecolor{currentstroke}%
\pgfsetstrokeopacity{0.300000}%
\pgfsetdash{}{0pt}%
\pgfpathmoveto{\pgfqpoint{1.766223in}{3.460350in}}%
\pgfusepath{stroke}%
\end{pgfscope}%
\begin{pgfscope}%
\pgfpathrectangle{\pgfqpoint{0.647939in}{0.492442in}}{\pgfqpoint{3.079299in}{3.079299in}}%
\pgfusepath{clip}%
\pgfsetroundcap%
\pgfsetroundjoin%
\definecolor{currentfill}{rgb}{0.500000,0.500000,0.500000}%
\pgfsetfillcolor{currentfill}%
\pgfsetfillopacity{0.300000}%
\pgfsetlinewidth{0.301125pt}%
\definecolor{currentstroke}{rgb}{0.500000,0.500000,0.500000}%
\pgfsetstrokecolor{currentstroke}%
\pgfsetstrokeopacity{0.300000}%
\pgfsetdash{}{0pt}%
\pgfpathmoveto{\pgfqpoint{0.000000in}{0.000000in}}%
\pgfpathlineto{\pgfqpoint{0.000000in}{0.000000in}}%
\pgfpathclose%
\pgfusepath{stroke,fill}%
\end{pgfscope}%
\begin{pgfscope}%
\pgfpathrectangle{\pgfqpoint{0.647939in}{0.492442in}}{\pgfqpoint{3.079299in}{3.079299in}}%
\pgfusepath{clip}%
\pgfsetroundcap%
\pgfsetroundjoin%
\pgfsetlinewidth{0.301125pt}%
\definecolor{currentstroke}{rgb}{0.500000,0.500000,0.500000}%
\pgfsetstrokecolor{currentstroke}%
\pgfsetstrokeopacity{0.300000}%
\pgfsetdash{}{0pt}%
\pgfpathmoveto{\pgfqpoint{1.681800in}{3.522979in}}%
\pgfusepath{stroke}%
\end{pgfscope}%
\begin{pgfscope}%
\pgfpathrectangle{\pgfqpoint{0.647939in}{0.492442in}}{\pgfqpoint{3.079299in}{3.079299in}}%
\pgfusepath{clip}%
\pgfsetroundcap%
\pgfsetroundjoin%
\definecolor{currentfill}{rgb}{0.500000,0.500000,0.500000}%
\pgfsetfillcolor{currentfill}%
\pgfsetfillopacity{0.300000}%
\pgfsetlinewidth{0.301125pt}%
\definecolor{currentstroke}{rgb}{0.500000,0.500000,0.500000}%
\pgfsetstrokecolor{currentstroke}%
\pgfsetstrokeopacity{0.300000}%
\pgfsetdash{}{0pt}%
\pgfpathmoveto{\pgfqpoint{0.000000in}{0.000000in}}%
\pgfpathlineto{\pgfqpoint{0.000000in}{0.000000in}}%
\pgfpathclose%
\pgfusepath{stroke,fill}%
\end{pgfscope}%
\begin{pgfscope}%
\pgfpathrectangle{\pgfqpoint{0.647939in}{0.492442in}}{\pgfqpoint{3.079299in}{3.079299in}}%
\pgfusepath{clip}%
\pgfsetroundcap%
\pgfsetroundjoin%
\pgfsetlinewidth{0.301125pt}%
\definecolor{currentstroke}{rgb}{0.500000,0.500000,0.500000}%
\pgfsetstrokecolor{currentstroke}%
\pgfsetstrokeopacity{0.300000}%
\pgfsetdash{}{0pt}%
\pgfpathmoveto{\pgfqpoint{1.116492in}{3.478973in}}%
\pgfusepath{stroke}%
\end{pgfscope}%
\begin{pgfscope}%
\pgfpathrectangle{\pgfqpoint{0.647939in}{0.492442in}}{\pgfqpoint{3.079299in}{3.079299in}}%
\pgfusepath{clip}%
\pgfsetroundcap%
\pgfsetroundjoin%
\definecolor{currentfill}{rgb}{0.500000,0.500000,0.500000}%
\pgfsetfillcolor{currentfill}%
\pgfsetfillopacity{0.300000}%
\pgfsetlinewidth{0.301125pt}%
\definecolor{currentstroke}{rgb}{0.500000,0.500000,0.500000}%
\pgfsetstrokecolor{currentstroke}%
\pgfsetstrokeopacity{0.300000}%
\pgfsetdash{}{0pt}%
\pgfpathmoveto{\pgfqpoint{0.000000in}{0.000000in}}%
\pgfpathlineto{\pgfqpoint{0.000000in}{0.000000in}}%
\pgfpathclose%
\pgfusepath{stroke,fill}%
\end{pgfscope}%
\begin{pgfscope}%
\pgfpathrectangle{\pgfqpoint{0.647939in}{0.492442in}}{\pgfqpoint{3.079299in}{3.079299in}}%
\pgfusepath{clip}%
\pgfsetroundcap%
\pgfsetroundjoin%
\pgfsetlinewidth{0.301125pt}%
\definecolor{currentstroke}{rgb}{0.500000,0.500000,0.500000}%
\pgfsetstrokecolor{currentstroke}%
\pgfsetstrokeopacity{0.300000}%
\pgfsetdash{}{0pt}%
\pgfpathmoveto{\pgfqpoint{0.898313in}{3.516720in}}%
\pgfusepath{stroke}%
\end{pgfscope}%
\begin{pgfscope}%
\pgfpathrectangle{\pgfqpoint{0.647939in}{0.492442in}}{\pgfqpoint{3.079299in}{3.079299in}}%
\pgfusepath{clip}%
\pgfsetroundcap%
\pgfsetroundjoin%
\definecolor{currentfill}{rgb}{0.500000,0.500000,0.500000}%
\pgfsetfillcolor{currentfill}%
\pgfsetfillopacity{0.300000}%
\pgfsetlinewidth{0.301125pt}%
\definecolor{currentstroke}{rgb}{0.500000,0.500000,0.500000}%
\pgfsetstrokecolor{currentstroke}%
\pgfsetstrokeopacity{0.300000}%
\pgfsetdash{}{0pt}%
\pgfpathmoveto{\pgfqpoint{0.000000in}{0.000000in}}%
\pgfpathlineto{\pgfqpoint{0.000000in}{0.000000in}}%
\pgfpathclose%
\pgfusepath{stroke,fill}%
\end{pgfscope}%
\begin{pgfscope}%
\pgfpathrectangle{\pgfqpoint{0.647939in}{0.492442in}}{\pgfqpoint{3.079299in}{3.079299in}}%
\pgfusepath{clip}%
\pgfsetroundcap%
\pgfsetroundjoin%
\pgfsetlinewidth{0.301125pt}%
\definecolor{currentstroke}{rgb}{0.500000,0.500000,0.500000}%
\pgfsetstrokecolor{currentstroke}%
\pgfsetstrokeopacity{0.300000}%
\pgfsetdash{}{0pt}%
\pgfpathmoveto{\pgfqpoint{1.079979in}{2.874441in}}%
\pgfusepath{stroke}%
\end{pgfscope}%
\begin{pgfscope}%
\pgfpathrectangle{\pgfqpoint{0.647939in}{0.492442in}}{\pgfqpoint{3.079299in}{3.079299in}}%
\pgfusepath{clip}%
\pgfsetroundcap%
\pgfsetroundjoin%
\definecolor{currentfill}{rgb}{0.500000,0.500000,0.500000}%
\pgfsetfillcolor{currentfill}%
\pgfsetfillopacity{0.300000}%
\pgfsetlinewidth{0.301125pt}%
\definecolor{currentstroke}{rgb}{0.500000,0.500000,0.500000}%
\pgfsetstrokecolor{currentstroke}%
\pgfsetstrokeopacity{0.300000}%
\pgfsetdash{}{0pt}%
\pgfpathmoveto{\pgfqpoint{0.000000in}{0.000000in}}%
\pgfpathlineto{\pgfqpoint{0.000000in}{0.000000in}}%
\pgfpathclose%
\pgfusepath{stroke,fill}%
\end{pgfscope}%
\begin{pgfscope}%
\pgfpathrectangle{\pgfqpoint{0.647939in}{0.492442in}}{\pgfqpoint{3.079299in}{3.079299in}}%
\pgfusepath{clip}%
\pgfsetroundcap%
\pgfsetroundjoin%
\pgfsetlinewidth{0.301125pt}%
\definecolor{currentstroke}{rgb}{0.500000,0.500000,0.500000}%
\pgfsetstrokecolor{currentstroke}%
\pgfsetstrokeopacity{0.300000}%
\pgfsetdash{}{0pt}%
\pgfpathmoveto{\pgfqpoint{1.815709in}{2.960309in}}%
\pgfusepath{stroke}%
\end{pgfscope}%
\begin{pgfscope}%
\pgfpathrectangle{\pgfqpoint{0.647939in}{0.492442in}}{\pgfqpoint{3.079299in}{3.079299in}}%
\pgfusepath{clip}%
\pgfsetroundcap%
\pgfsetroundjoin%
\definecolor{currentfill}{rgb}{0.500000,0.500000,0.500000}%
\pgfsetfillcolor{currentfill}%
\pgfsetfillopacity{0.300000}%
\pgfsetlinewidth{0.301125pt}%
\definecolor{currentstroke}{rgb}{0.500000,0.500000,0.500000}%
\pgfsetstrokecolor{currentstroke}%
\pgfsetstrokeopacity{0.300000}%
\pgfsetdash{}{0pt}%
\pgfpathmoveto{\pgfqpoint{0.000000in}{0.000000in}}%
\pgfpathlineto{\pgfqpoint{0.000000in}{0.000000in}}%
\pgfpathclose%
\pgfusepath{stroke,fill}%
\end{pgfscope}%
\begin{pgfscope}%
\pgfpathrectangle{\pgfqpoint{0.647939in}{0.492442in}}{\pgfqpoint{3.079299in}{3.079299in}}%
\pgfusepath{clip}%
\pgfsetroundcap%
\pgfsetroundjoin%
\pgfsetlinewidth{0.301125pt}%
\definecolor{currentstroke}{rgb}{0.500000,0.500000,0.500000}%
\pgfsetstrokecolor{currentstroke}%
\pgfsetstrokeopacity{0.300000}%
\pgfsetdash{}{0pt}%
\pgfpathmoveto{\pgfqpoint{0.878513in}{2.695439in}}%
\pgfusepath{stroke}%
\end{pgfscope}%
\begin{pgfscope}%
\pgfpathrectangle{\pgfqpoint{0.647939in}{0.492442in}}{\pgfqpoint{3.079299in}{3.079299in}}%
\pgfusepath{clip}%
\pgfsetroundcap%
\pgfsetroundjoin%
\definecolor{currentfill}{rgb}{0.500000,0.500000,0.500000}%
\pgfsetfillcolor{currentfill}%
\pgfsetfillopacity{0.300000}%
\pgfsetlinewidth{0.301125pt}%
\definecolor{currentstroke}{rgb}{0.500000,0.500000,0.500000}%
\pgfsetstrokecolor{currentstroke}%
\pgfsetstrokeopacity{0.300000}%
\pgfsetdash{}{0pt}%
\pgfpathmoveto{\pgfqpoint{0.000000in}{0.000000in}}%
\pgfpathlineto{\pgfqpoint{0.000000in}{0.000000in}}%
\pgfpathclose%
\pgfusepath{stroke,fill}%
\end{pgfscope}%
\begin{pgfscope}%
\pgfpathrectangle{\pgfqpoint{0.647939in}{0.492442in}}{\pgfqpoint{3.079299in}{3.079299in}}%
\pgfusepath{clip}%
\pgfsetroundcap%
\pgfsetroundjoin%
\pgfsetlinewidth{0.301125pt}%
\definecolor{currentstroke}{rgb}{0.500000,0.500000,0.500000}%
\pgfsetstrokecolor{currentstroke}%
\pgfsetstrokeopacity{0.300000}%
\pgfsetdash{}{0pt}%
\pgfpathmoveto{\pgfqpoint{1.012401in}{2.584000in}}%
\pgfusepath{stroke}%
\end{pgfscope}%
\begin{pgfscope}%
\pgfpathrectangle{\pgfqpoint{0.647939in}{0.492442in}}{\pgfqpoint{3.079299in}{3.079299in}}%
\pgfusepath{clip}%
\pgfsetroundcap%
\pgfsetroundjoin%
\definecolor{currentfill}{rgb}{0.500000,0.500000,0.500000}%
\pgfsetfillcolor{currentfill}%
\pgfsetfillopacity{0.300000}%
\pgfsetlinewidth{0.301125pt}%
\definecolor{currentstroke}{rgb}{0.500000,0.500000,0.500000}%
\pgfsetstrokecolor{currentstroke}%
\pgfsetstrokeopacity{0.300000}%
\pgfsetdash{}{0pt}%
\pgfpathmoveto{\pgfqpoint{0.000000in}{0.000000in}}%
\pgfpathlineto{\pgfqpoint{0.000000in}{0.000000in}}%
\pgfpathclose%
\pgfusepath{stroke,fill}%
\end{pgfscope}%
\begin{pgfscope}%
\pgfpathrectangle{\pgfqpoint{0.647939in}{0.492442in}}{\pgfqpoint{3.079299in}{3.079299in}}%
\pgfusepath{clip}%
\pgfsetroundcap%
\pgfsetroundjoin%
\pgfsetlinewidth{0.301125pt}%
\definecolor{currentstroke}{rgb}{0.500000,0.500000,0.500000}%
\pgfsetstrokecolor{currentstroke}%
\pgfsetstrokeopacity{0.300000}%
\pgfsetdash{}{0pt}%
\pgfpathmoveto{\pgfqpoint{0.878276in}{2.487023in}}%
\pgfusepath{stroke}%
\end{pgfscope}%
\begin{pgfscope}%
\pgfpathrectangle{\pgfqpoint{0.647939in}{0.492442in}}{\pgfqpoint{3.079299in}{3.079299in}}%
\pgfusepath{clip}%
\pgfsetroundcap%
\pgfsetroundjoin%
\definecolor{currentfill}{rgb}{0.500000,0.500000,0.500000}%
\pgfsetfillcolor{currentfill}%
\pgfsetfillopacity{0.300000}%
\pgfsetlinewidth{0.301125pt}%
\definecolor{currentstroke}{rgb}{0.500000,0.500000,0.500000}%
\pgfsetstrokecolor{currentstroke}%
\pgfsetstrokeopacity{0.300000}%
\pgfsetdash{}{0pt}%
\pgfpathmoveto{\pgfqpoint{0.000000in}{0.000000in}}%
\pgfpathlineto{\pgfqpoint{0.000000in}{0.000000in}}%
\pgfpathclose%
\pgfusepath{stroke,fill}%
\end{pgfscope}%
\begin{pgfscope}%
\pgfpathrectangle{\pgfqpoint{0.647939in}{0.492442in}}{\pgfqpoint{3.079299in}{3.079299in}}%
\pgfusepath{clip}%
\pgfsetroundcap%
\pgfsetroundjoin%
\pgfsetlinewidth{0.301125pt}%
\definecolor{currentstroke}{rgb}{0.500000,0.500000,0.500000}%
\pgfsetstrokecolor{currentstroke}%
\pgfsetstrokeopacity{0.300000}%
\pgfsetdash{}{0pt}%
\pgfpathmoveto{\pgfqpoint{1.875993in}{2.652728in}}%
\pgfusepath{stroke}%
\end{pgfscope}%
\begin{pgfscope}%
\pgfpathrectangle{\pgfqpoint{0.647939in}{0.492442in}}{\pgfqpoint{3.079299in}{3.079299in}}%
\pgfusepath{clip}%
\pgfsetroundcap%
\pgfsetroundjoin%
\definecolor{currentfill}{rgb}{0.500000,0.500000,0.500000}%
\pgfsetfillcolor{currentfill}%
\pgfsetfillopacity{0.300000}%
\pgfsetlinewidth{0.301125pt}%
\definecolor{currentstroke}{rgb}{0.500000,0.500000,0.500000}%
\pgfsetstrokecolor{currentstroke}%
\pgfsetstrokeopacity{0.300000}%
\pgfsetdash{}{0pt}%
\pgfpathmoveto{\pgfqpoint{0.000000in}{0.000000in}}%
\pgfpathlineto{\pgfqpoint{0.000000in}{0.000000in}}%
\pgfpathclose%
\pgfusepath{stroke,fill}%
\end{pgfscope}%
\begin{pgfscope}%
\pgfpathrectangle{\pgfqpoint{0.647939in}{0.492442in}}{\pgfqpoint{3.079299in}{3.079299in}}%
\pgfusepath{clip}%
\pgfsetroundcap%
\pgfsetroundjoin%
\pgfsetlinewidth{0.301125pt}%
\definecolor{currentstroke}{rgb}{0.500000,0.500000,0.500000}%
\pgfsetstrokecolor{currentstroke}%
\pgfsetstrokeopacity{0.300000}%
\pgfsetdash{}{0pt}%
\pgfpathmoveto{\pgfqpoint{1.604862in}{2.542912in}}%
\pgfusepath{stroke}%
\end{pgfscope}%
\begin{pgfscope}%
\pgfpathrectangle{\pgfqpoint{0.647939in}{0.492442in}}{\pgfqpoint{3.079299in}{3.079299in}}%
\pgfusepath{clip}%
\pgfsetroundcap%
\pgfsetroundjoin%
\definecolor{currentfill}{rgb}{0.500000,0.500000,0.500000}%
\pgfsetfillcolor{currentfill}%
\pgfsetfillopacity{0.300000}%
\pgfsetlinewidth{0.301125pt}%
\definecolor{currentstroke}{rgb}{0.500000,0.500000,0.500000}%
\pgfsetstrokecolor{currentstroke}%
\pgfsetstrokeopacity{0.300000}%
\pgfsetdash{}{0pt}%
\pgfpathmoveto{\pgfqpoint{0.000000in}{0.000000in}}%
\pgfpathlineto{\pgfqpoint{0.000000in}{0.000000in}}%
\pgfpathclose%
\pgfusepath{stroke,fill}%
\end{pgfscope}%
\begin{pgfscope}%
\pgfpathrectangle{\pgfqpoint{0.647939in}{0.492442in}}{\pgfqpoint{3.079299in}{3.079299in}}%
\pgfusepath{clip}%
\pgfsetroundcap%
\pgfsetroundjoin%
\pgfsetlinewidth{0.301125pt}%
\definecolor{currentstroke}{rgb}{0.500000,0.500000,0.500000}%
\pgfsetstrokecolor{currentstroke}%
\pgfsetstrokeopacity{0.300000}%
\pgfsetdash{}{0pt}%
\pgfpathmoveto{\pgfqpoint{1.143860in}{2.343458in}}%
\pgfusepath{stroke}%
\end{pgfscope}%
\begin{pgfscope}%
\pgfpathrectangle{\pgfqpoint{0.647939in}{0.492442in}}{\pgfqpoint{3.079299in}{3.079299in}}%
\pgfusepath{clip}%
\pgfsetroundcap%
\pgfsetroundjoin%
\definecolor{currentfill}{rgb}{0.500000,0.500000,0.500000}%
\pgfsetfillcolor{currentfill}%
\pgfsetfillopacity{0.300000}%
\pgfsetlinewidth{0.301125pt}%
\definecolor{currentstroke}{rgb}{0.500000,0.500000,0.500000}%
\pgfsetstrokecolor{currentstroke}%
\pgfsetstrokeopacity{0.300000}%
\pgfsetdash{}{0pt}%
\pgfpathmoveto{\pgfqpoint{0.000000in}{0.000000in}}%
\pgfpathlineto{\pgfqpoint{0.000000in}{0.000000in}}%
\pgfpathclose%
\pgfusepath{stroke,fill}%
\end{pgfscope}%
\begin{pgfscope}%
\pgfpathrectangle{\pgfqpoint{0.647939in}{0.492442in}}{\pgfqpoint{3.079299in}{3.079299in}}%
\pgfusepath{clip}%
\pgfsetroundcap%
\pgfsetroundjoin%
\pgfsetlinewidth{0.301125pt}%
\definecolor{currentstroke}{rgb}{0.500000,0.500000,0.500000}%
\pgfsetstrokecolor{currentstroke}%
\pgfsetstrokeopacity{0.300000}%
\pgfsetdash{}{0pt}%
\pgfpathmoveto{\pgfqpoint{1.011337in}{2.239656in}}%
\pgfusepath{stroke}%
\end{pgfscope}%
\begin{pgfscope}%
\pgfpathrectangle{\pgfqpoint{0.647939in}{0.492442in}}{\pgfqpoint{3.079299in}{3.079299in}}%
\pgfusepath{clip}%
\pgfsetroundcap%
\pgfsetroundjoin%
\definecolor{currentfill}{rgb}{0.500000,0.500000,0.500000}%
\pgfsetfillcolor{currentfill}%
\pgfsetfillopacity{0.300000}%
\pgfsetlinewidth{0.301125pt}%
\definecolor{currentstroke}{rgb}{0.500000,0.500000,0.500000}%
\pgfsetstrokecolor{currentstroke}%
\pgfsetstrokeopacity{0.300000}%
\pgfsetdash{}{0pt}%
\pgfpathmoveto{\pgfqpoint{0.000000in}{0.000000in}}%
\pgfpathlineto{\pgfqpoint{0.000000in}{0.000000in}}%
\pgfpathclose%
\pgfusepath{stroke,fill}%
\end{pgfscope}%
\begin{pgfscope}%
\pgfpathrectangle{\pgfqpoint{0.647939in}{0.492442in}}{\pgfqpoint{3.079299in}{3.079299in}}%
\pgfusepath{clip}%
\pgfsetroundcap%
\pgfsetroundjoin%
\pgfsetlinewidth{0.301125pt}%
\definecolor{currentstroke}{rgb}{0.500000,0.500000,0.500000}%
\pgfsetstrokecolor{currentstroke}%
\pgfsetstrokeopacity{0.300000}%
\pgfsetdash{}{0pt}%
\pgfpathmoveto{\pgfqpoint{1.011088in}{2.170907in}}%
\pgfusepath{stroke}%
\end{pgfscope}%
\begin{pgfscope}%
\pgfpathrectangle{\pgfqpoint{0.647939in}{0.492442in}}{\pgfqpoint{3.079299in}{3.079299in}}%
\pgfusepath{clip}%
\pgfsetroundcap%
\pgfsetroundjoin%
\definecolor{currentfill}{rgb}{0.500000,0.500000,0.500000}%
\pgfsetfillcolor{currentfill}%
\pgfsetfillopacity{0.300000}%
\pgfsetlinewidth{0.301125pt}%
\definecolor{currentstroke}{rgb}{0.500000,0.500000,0.500000}%
\pgfsetstrokecolor{currentstroke}%
\pgfsetstrokeopacity{0.300000}%
\pgfsetdash{}{0pt}%
\pgfpathmoveto{\pgfqpoint{0.000000in}{0.000000in}}%
\pgfpathlineto{\pgfqpoint{0.000000in}{0.000000in}}%
\pgfpathclose%
\pgfusepath{stroke,fill}%
\end{pgfscope}%
\begin{pgfscope}%
\pgfpathrectangle{\pgfqpoint{0.647939in}{0.492442in}}{\pgfqpoint{3.079299in}{3.079299in}}%
\pgfusepath{clip}%
\pgfsetroundcap%
\pgfsetroundjoin%
\pgfsetlinewidth{0.301125pt}%
\definecolor{currentstroke}{rgb}{0.500000,0.500000,0.500000}%
\pgfsetstrokecolor{currentstroke}%
\pgfsetstrokeopacity{0.300000}%
\pgfsetdash{}{0pt}%
\pgfpathmoveto{\pgfqpoint{1.662501in}{2.310265in}}%
\pgfusepath{stroke}%
\end{pgfscope}%
\begin{pgfscope}%
\pgfpathrectangle{\pgfqpoint{0.647939in}{0.492442in}}{\pgfqpoint{3.079299in}{3.079299in}}%
\pgfusepath{clip}%
\pgfsetroundcap%
\pgfsetroundjoin%
\definecolor{currentfill}{rgb}{0.500000,0.500000,0.500000}%
\pgfsetfillcolor{currentfill}%
\pgfsetfillopacity{0.300000}%
\pgfsetlinewidth{0.301125pt}%
\definecolor{currentstroke}{rgb}{0.500000,0.500000,0.500000}%
\pgfsetstrokecolor{currentstroke}%
\pgfsetstrokeopacity{0.300000}%
\pgfsetdash{}{0pt}%
\pgfpathmoveto{\pgfqpoint{0.000000in}{0.000000in}}%
\pgfpathlineto{\pgfqpoint{0.000000in}{0.000000in}}%
\pgfpathclose%
\pgfusepath{stroke,fill}%
\end{pgfscope}%
\begin{pgfscope}%
\pgfpathrectangle{\pgfqpoint{0.647939in}{0.492442in}}{\pgfqpoint{3.079299in}{3.079299in}}%
\pgfusepath{clip}%
\pgfsetroundcap%
\pgfsetroundjoin%
\pgfsetlinewidth{0.301125pt}%
\definecolor{currentstroke}{rgb}{0.500000,0.500000,0.500000}%
\pgfsetstrokecolor{currentstroke}%
\pgfsetstrokeopacity{0.300000}%
\pgfsetdash{}{0pt}%
\pgfpathmoveto{\pgfqpoint{1.206891in}{2.093293in}}%
\pgfusepath{stroke}%
\end{pgfscope}%
\begin{pgfscope}%
\pgfpathrectangle{\pgfqpoint{0.647939in}{0.492442in}}{\pgfqpoint{3.079299in}{3.079299in}}%
\pgfusepath{clip}%
\pgfsetroundcap%
\pgfsetroundjoin%
\definecolor{currentfill}{rgb}{0.500000,0.500000,0.500000}%
\pgfsetfillcolor{currentfill}%
\pgfsetfillopacity{0.300000}%
\pgfsetlinewidth{0.301125pt}%
\definecolor{currentstroke}{rgb}{0.500000,0.500000,0.500000}%
\pgfsetstrokecolor{currentstroke}%
\pgfsetstrokeopacity{0.300000}%
\pgfsetdash{}{0pt}%
\pgfpathmoveto{\pgfqpoint{0.000000in}{0.000000in}}%
\pgfpathlineto{\pgfqpoint{0.000000in}{0.000000in}}%
\pgfpathclose%
\pgfusepath{stroke,fill}%
\end{pgfscope}%
\begin{pgfscope}%
\pgfpathrectangle{\pgfqpoint{0.647939in}{0.492442in}}{\pgfqpoint{3.079299in}{3.079299in}}%
\pgfusepath{clip}%
\pgfsetroundcap%
\pgfsetroundjoin%
\pgfsetlinewidth{0.301125pt}%
\definecolor{currentstroke}{rgb}{0.500000,0.500000,0.500000}%
\pgfsetstrokecolor{currentstroke}%
\pgfsetstrokeopacity{0.300000}%
\pgfsetdash{}{0pt}%
\pgfpathmoveto{\pgfqpoint{1.527032in}{2.144812in}}%
\pgfusepath{stroke}%
\end{pgfscope}%
\begin{pgfscope}%
\pgfpathrectangle{\pgfqpoint{0.647939in}{0.492442in}}{\pgfqpoint{3.079299in}{3.079299in}}%
\pgfusepath{clip}%
\pgfsetroundcap%
\pgfsetroundjoin%
\definecolor{currentfill}{rgb}{0.500000,0.500000,0.500000}%
\pgfsetfillcolor{currentfill}%
\pgfsetfillopacity{0.300000}%
\pgfsetlinewidth{0.301125pt}%
\definecolor{currentstroke}{rgb}{0.500000,0.500000,0.500000}%
\pgfsetstrokecolor{currentstroke}%
\pgfsetstrokeopacity{0.300000}%
\pgfsetdash{}{0pt}%
\pgfpathmoveto{\pgfqpoint{0.000000in}{0.000000in}}%
\pgfpathlineto{\pgfqpoint{0.000000in}{0.000000in}}%
\pgfpathclose%
\pgfusepath{stroke,fill}%
\end{pgfscope}%
\begin{pgfscope}%
\pgfpathrectangle{\pgfqpoint{0.647939in}{0.492442in}}{\pgfqpoint{3.079299in}{3.079299in}}%
\pgfusepath{clip}%
\pgfsetroundcap%
\pgfsetroundjoin%
\pgfsetlinewidth{0.301125pt}%
\definecolor{currentstroke}{rgb}{0.500000,0.500000,0.500000}%
\pgfsetstrokecolor{currentstroke}%
\pgfsetstrokeopacity{0.300000}%
\pgfsetdash{}{0pt}%
\pgfpathmoveto{\pgfqpoint{0.810360in}{1.848694in}}%
\pgfusepath{stroke}%
\end{pgfscope}%
\begin{pgfscope}%
\pgfpathrectangle{\pgfqpoint{0.647939in}{0.492442in}}{\pgfqpoint{3.079299in}{3.079299in}}%
\pgfusepath{clip}%
\pgfsetroundcap%
\pgfsetroundjoin%
\definecolor{currentfill}{rgb}{0.500000,0.500000,0.500000}%
\pgfsetfillcolor{currentfill}%
\pgfsetfillopacity{0.300000}%
\pgfsetlinewidth{0.301125pt}%
\definecolor{currentstroke}{rgb}{0.500000,0.500000,0.500000}%
\pgfsetstrokecolor{currentstroke}%
\pgfsetstrokeopacity{0.300000}%
\pgfsetdash{}{0pt}%
\pgfpathmoveto{\pgfqpoint{0.000000in}{0.000000in}}%
\pgfpathlineto{\pgfqpoint{0.000000in}{0.000000in}}%
\pgfpathclose%
\pgfusepath{stroke,fill}%
\end{pgfscope}%
\begin{pgfscope}%
\pgfpathrectangle{\pgfqpoint{0.647939in}{0.492442in}}{\pgfqpoint{3.079299in}{3.079299in}}%
\pgfusepath{clip}%
\pgfsetroundcap%
\pgfsetroundjoin%
\pgfsetlinewidth{0.301125pt}%
\definecolor{currentstroke}{rgb}{0.500000,0.500000,0.500000}%
\pgfsetstrokecolor{currentstroke}%
\pgfsetstrokeopacity{0.300000}%
\pgfsetdash{}{0pt}%
\pgfpathmoveto{\pgfqpoint{1.392939in}{1.904671in}}%
\pgfusepath{stroke}%
\end{pgfscope}%
\begin{pgfscope}%
\pgfpathrectangle{\pgfqpoint{0.647939in}{0.492442in}}{\pgfqpoint{3.079299in}{3.079299in}}%
\pgfusepath{clip}%
\pgfsetroundcap%
\pgfsetroundjoin%
\definecolor{currentfill}{rgb}{0.500000,0.500000,0.500000}%
\pgfsetfillcolor{currentfill}%
\pgfsetfillopacity{0.300000}%
\pgfsetlinewidth{0.301125pt}%
\definecolor{currentstroke}{rgb}{0.500000,0.500000,0.500000}%
\pgfsetstrokecolor{currentstroke}%
\pgfsetstrokeopacity{0.300000}%
\pgfsetdash{}{0pt}%
\pgfpathmoveto{\pgfqpoint{0.000000in}{0.000000in}}%
\pgfpathlineto{\pgfqpoint{0.000000in}{0.000000in}}%
\pgfpathclose%
\pgfusepath{stroke,fill}%
\end{pgfscope}%
\begin{pgfscope}%
\pgfpathrectangle{\pgfqpoint{0.647939in}{0.492442in}}{\pgfqpoint{3.079299in}{3.079299in}}%
\pgfusepath{clip}%
\pgfsetroundcap%
\pgfsetroundjoin%
\pgfsetlinewidth{0.301125pt}%
\definecolor{currentstroke}{rgb}{0.500000,0.500000,0.500000}%
\pgfsetstrokecolor{currentstroke}%
\pgfsetstrokeopacity{0.300000}%
\pgfsetdash{}{0pt}%
\pgfpathmoveto{\pgfqpoint{1.008864in}{1.691059in}}%
\pgfusepath{stroke}%
\end{pgfscope}%
\begin{pgfscope}%
\pgfpathrectangle{\pgfqpoint{0.647939in}{0.492442in}}{\pgfqpoint{3.079299in}{3.079299in}}%
\pgfusepath{clip}%
\pgfsetroundcap%
\pgfsetroundjoin%
\definecolor{currentfill}{rgb}{0.500000,0.500000,0.500000}%
\pgfsetfillcolor{currentfill}%
\pgfsetfillopacity{0.300000}%
\pgfsetlinewidth{0.301125pt}%
\definecolor{currentstroke}{rgb}{0.500000,0.500000,0.500000}%
\pgfsetstrokecolor{currentstroke}%
\pgfsetstrokeopacity{0.300000}%
\pgfsetdash{}{0pt}%
\pgfpathmoveto{\pgfqpoint{0.000000in}{0.000000in}}%
\pgfpathlineto{\pgfqpoint{0.000000in}{0.000000in}}%
\pgfpathclose%
\pgfusepath{stroke,fill}%
\end{pgfscope}%
\begin{pgfscope}%
\pgfpathrectangle{\pgfqpoint{0.647939in}{0.492442in}}{\pgfqpoint{3.079299in}{3.079299in}}%
\pgfusepath{clip}%
\pgfsetroundcap%
\pgfsetroundjoin%
\pgfsetlinewidth{0.301125pt}%
\definecolor{currentstroke}{rgb}{0.500000,0.500000,0.500000}%
\pgfsetstrokecolor{currentstroke}%
\pgfsetstrokeopacity{0.300000}%
\pgfsetdash{}{0pt}%
\pgfpathmoveto{\pgfqpoint{1.388044in}{1.778076in}}%
\pgfusepath{stroke}%
\end{pgfscope}%
\begin{pgfscope}%
\pgfpathrectangle{\pgfqpoint{0.647939in}{0.492442in}}{\pgfqpoint{3.079299in}{3.079299in}}%
\pgfusepath{clip}%
\pgfsetroundcap%
\pgfsetroundjoin%
\definecolor{currentfill}{rgb}{0.500000,0.500000,0.500000}%
\pgfsetfillcolor{currentfill}%
\pgfsetfillopacity{0.300000}%
\pgfsetlinewidth{0.301125pt}%
\definecolor{currentstroke}{rgb}{0.500000,0.500000,0.500000}%
\pgfsetstrokecolor{currentstroke}%
\pgfsetstrokeopacity{0.300000}%
\pgfsetdash{}{0pt}%
\pgfpathmoveto{\pgfqpoint{0.000000in}{0.000000in}}%
\pgfpathlineto{\pgfqpoint{0.000000in}{0.000000in}}%
\pgfpathclose%
\pgfusepath{stroke,fill}%
\end{pgfscope}%
\begin{pgfscope}%
\pgfpathrectangle{\pgfqpoint{0.647939in}{0.492442in}}{\pgfqpoint{3.079299in}{3.079299in}}%
\pgfusepath{clip}%
\pgfsetroundcap%
\pgfsetroundjoin%
\pgfsetlinewidth{0.301125pt}%
\definecolor{currentstroke}{rgb}{0.500000,0.500000,0.500000}%
\pgfsetstrokecolor{currentstroke}%
\pgfsetstrokeopacity{0.300000}%
\pgfsetdash{}{0pt}%
\pgfpathmoveto{\pgfqpoint{1.072736in}{1.576693in}}%
\pgfusepath{stroke}%
\end{pgfscope}%
\begin{pgfscope}%
\pgfpathrectangle{\pgfqpoint{0.647939in}{0.492442in}}{\pgfqpoint{3.079299in}{3.079299in}}%
\pgfusepath{clip}%
\pgfsetroundcap%
\pgfsetroundjoin%
\definecolor{currentfill}{rgb}{0.500000,0.500000,0.500000}%
\pgfsetfillcolor{currentfill}%
\pgfsetfillopacity{0.300000}%
\pgfsetlinewidth{0.301125pt}%
\definecolor{currentstroke}{rgb}{0.500000,0.500000,0.500000}%
\pgfsetstrokecolor{currentstroke}%
\pgfsetstrokeopacity{0.300000}%
\pgfsetdash{}{0pt}%
\pgfpathmoveto{\pgfqpoint{0.000000in}{0.000000in}}%
\pgfpathlineto{\pgfqpoint{0.000000in}{0.000000in}}%
\pgfpathclose%
\pgfusepath{stroke,fill}%
\end{pgfscope}%
\begin{pgfscope}%
\pgfpathrectangle{\pgfqpoint{0.647939in}{0.492442in}}{\pgfqpoint{3.079299in}{3.079299in}}%
\pgfusepath{clip}%
\pgfsetroundcap%
\pgfsetroundjoin%
\pgfsetlinewidth{0.301125pt}%
\definecolor{currentstroke}{rgb}{0.500000,0.500000,0.500000}%
\pgfsetstrokecolor{currentstroke}%
\pgfsetstrokeopacity{0.300000}%
\pgfsetdash{}{0pt}%
\pgfpathmoveto{\pgfqpoint{0.876368in}{1.447623in}}%
\pgfusepath{stroke}%
\end{pgfscope}%
\begin{pgfscope}%
\pgfpathrectangle{\pgfqpoint{0.647939in}{0.492442in}}{\pgfqpoint{3.079299in}{3.079299in}}%
\pgfusepath{clip}%
\pgfsetroundcap%
\pgfsetroundjoin%
\definecolor{currentfill}{rgb}{0.500000,0.500000,0.500000}%
\pgfsetfillcolor{currentfill}%
\pgfsetfillopacity{0.300000}%
\pgfsetlinewidth{0.301125pt}%
\definecolor{currentstroke}{rgb}{0.500000,0.500000,0.500000}%
\pgfsetstrokecolor{currentstroke}%
\pgfsetstrokeopacity{0.300000}%
\pgfsetdash{}{0pt}%
\pgfpathmoveto{\pgfqpoint{0.000000in}{0.000000in}}%
\pgfpathlineto{\pgfqpoint{0.000000in}{0.000000in}}%
\pgfpathclose%
\pgfusepath{stroke,fill}%
\end{pgfscope}%
\begin{pgfscope}%
\pgfpathrectangle{\pgfqpoint{0.647939in}{0.492442in}}{\pgfqpoint{3.079299in}{3.079299in}}%
\pgfusepath{clip}%
\pgfsetroundcap%
\pgfsetroundjoin%
\pgfsetlinewidth{0.301125pt}%
\definecolor{currentstroke}{rgb}{0.500000,0.500000,0.500000}%
\pgfsetstrokecolor{currentstroke}%
\pgfsetstrokeopacity{0.300000}%
\pgfsetdash{}{0pt}%
\pgfpathmoveto{\pgfqpoint{0.876175in}{1.378546in}}%
\pgfusepath{stroke}%
\end{pgfscope}%
\begin{pgfscope}%
\pgfpathrectangle{\pgfqpoint{0.647939in}{0.492442in}}{\pgfqpoint{3.079299in}{3.079299in}}%
\pgfusepath{clip}%
\pgfsetroundcap%
\pgfsetroundjoin%
\definecolor{currentfill}{rgb}{0.500000,0.500000,0.500000}%
\pgfsetfillcolor{currentfill}%
\pgfsetfillopacity{0.300000}%
\pgfsetlinewidth{0.301125pt}%
\definecolor{currentstroke}{rgb}{0.500000,0.500000,0.500000}%
\pgfsetstrokecolor{currentstroke}%
\pgfsetstrokeopacity{0.300000}%
\pgfsetdash{}{0pt}%
\pgfpathmoveto{\pgfqpoint{0.000000in}{0.000000in}}%
\pgfpathlineto{\pgfqpoint{0.000000in}{0.000000in}}%
\pgfpathclose%
\pgfusepath{stroke,fill}%
\end{pgfscope}%
\begin{pgfscope}%
\pgfpathrectangle{\pgfqpoint{0.647939in}{0.492442in}}{\pgfqpoint{3.079299in}{3.079299in}}%
\pgfusepath{clip}%
\pgfsetroundcap%
\pgfsetroundjoin%
\pgfsetlinewidth{0.301125pt}%
\definecolor{currentstroke}{rgb}{0.500000,0.500000,0.500000}%
\pgfsetstrokecolor{currentstroke}%
\pgfsetstrokeopacity{0.300000}%
\pgfsetdash{}{0pt}%
\pgfpathmoveto{\pgfqpoint{1.316234in}{1.494304in}}%
\pgfusepath{stroke}%
\end{pgfscope}%
\begin{pgfscope}%
\pgfpathrectangle{\pgfqpoint{0.647939in}{0.492442in}}{\pgfqpoint{3.079299in}{3.079299in}}%
\pgfusepath{clip}%
\pgfsetroundcap%
\pgfsetroundjoin%
\definecolor{currentfill}{rgb}{0.500000,0.500000,0.500000}%
\pgfsetfillcolor{currentfill}%
\pgfsetfillopacity{0.300000}%
\pgfsetlinewidth{0.301125pt}%
\definecolor{currentstroke}{rgb}{0.500000,0.500000,0.500000}%
\pgfsetstrokecolor{currentstroke}%
\pgfsetstrokeopacity{0.300000}%
\pgfsetdash{}{0pt}%
\pgfpathmoveto{\pgfqpoint{0.000000in}{0.000000in}}%
\pgfpathlineto{\pgfqpoint{0.000000in}{0.000000in}}%
\pgfpathclose%
\pgfusepath{stroke,fill}%
\end{pgfscope}%
\begin{pgfscope}%
\pgfpathrectangle{\pgfqpoint{0.647939in}{0.492442in}}{\pgfqpoint{3.079299in}{3.079299in}}%
\pgfusepath{clip}%
\pgfsetroundcap%
\pgfsetroundjoin%
\pgfsetlinewidth{0.301125pt}%
\definecolor{currentstroke}{rgb}{0.500000,0.500000,0.500000}%
\pgfsetstrokecolor{currentstroke}%
\pgfsetstrokeopacity{0.300000}%
\pgfsetdash{}{0pt}%
\pgfpathmoveto{\pgfqpoint{1.132306in}{1.334680in}}%
\pgfusepath{stroke}%
\end{pgfscope}%
\begin{pgfscope}%
\pgfpathrectangle{\pgfqpoint{0.647939in}{0.492442in}}{\pgfqpoint{3.079299in}{3.079299in}}%
\pgfusepath{clip}%
\pgfsetroundcap%
\pgfsetroundjoin%
\definecolor{currentfill}{rgb}{0.500000,0.500000,0.500000}%
\pgfsetfillcolor{currentfill}%
\pgfsetfillopacity{0.300000}%
\pgfsetlinewidth{0.301125pt}%
\definecolor{currentstroke}{rgb}{0.500000,0.500000,0.500000}%
\pgfsetstrokecolor{currentstroke}%
\pgfsetstrokeopacity{0.300000}%
\pgfsetdash{}{0pt}%
\pgfpathmoveto{\pgfqpoint{0.000000in}{0.000000in}}%
\pgfpathlineto{\pgfqpoint{0.000000in}{0.000000in}}%
\pgfpathclose%
\pgfusepath{stroke,fill}%
\end{pgfscope}%
\begin{pgfscope}%
\pgfpathrectangle{\pgfqpoint{0.647939in}{0.492442in}}{\pgfqpoint{3.079299in}{3.079299in}}%
\pgfusepath{clip}%
\pgfsetroundcap%
\pgfsetroundjoin%
\pgfsetlinewidth{0.301125pt}%
\definecolor{currentstroke}{rgb}{0.500000,0.500000,0.500000}%
\pgfsetstrokecolor{currentstroke}%
\pgfsetstrokeopacity{0.300000}%
\pgfsetdash{}{0pt}%
\pgfpathmoveto{\pgfqpoint{0.809362in}{1.154172in}}%
\pgfusepath{stroke}%
\end{pgfscope}%
\begin{pgfscope}%
\pgfpathrectangle{\pgfqpoint{0.647939in}{0.492442in}}{\pgfqpoint{3.079299in}{3.079299in}}%
\pgfusepath{clip}%
\pgfsetroundcap%
\pgfsetroundjoin%
\definecolor{currentfill}{rgb}{0.500000,0.500000,0.500000}%
\pgfsetfillcolor{currentfill}%
\pgfsetfillopacity{0.300000}%
\pgfsetlinewidth{0.301125pt}%
\definecolor{currentstroke}{rgb}{0.500000,0.500000,0.500000}%
\pgfsetstrokecolor{currentstroke}%
\pgfsetstrokeopacity{0.300000}%
\pgfsetdash{}{0pt}%
\pgfpathmoveto{\pgfqpoint{0.000000in}{0.000000in}}%
\pgfpathlineto{\pgfqpoint{0.000000in}{0.000000in}}%
\pgfpathclose%
\pgfusepath{stroke,fill}%
\end{pgfscope}%
\begin{pgfscope}%
\pgfpathrectangle{\pgfqpoint{0.647939in}{0.492442in}}{\pgfqpoint{3.079299in}{3.079299in}}%
\pgfusepath{clip}%
\pgfsetroundcap%
\pgfsetroundjoin%
\pgfsetlinewidth{0.301125pt}%
\definecolor{currentstroke}{rgb}{0.500000,0.500000,0.500000}%
\pgfsetstrokecolor{currentstroke}%
\pgfsetstrokeopacity{0.300000}%
\pgfsetdash{}{0pt}%
\pgfpathmoveto{\pgfqpoint{1.189860in}{1.234735in}}%
\pgfusepath{stroke}%
\end{pgfscope}%
\begin{pgfscope}%
\pgfpathrectangle{\pgfqpoint{0.647939in}{0.492442in}}{\pgfqpoint{3.079299in}{3.079299in}}%
\pgfusepath{clip}%
\pgfsetroundcap%
\pgfsetroundjoin%
\definecolor{currentfill}{rgb}{0.500000,0.500000,0.500000}%
\pgfsetfillcolor{currentfill}%
\pgfsetfillopacity{0.300000}%
\pgfsetlinewidth{0.301125pt}%
\definecolor{currentstroke}{rgb}{0.500000,0.500000,0.500000}%
\pgfsetstrokecolor{currentstroke}%
\pgfsetstrokeopacity{0.300000}%
\pgfsetdash{}{0pt}%
\pgfpathmoveto{\pgfqpoint{0.000000in}{0.000000in}}%
\pgfpathlineto{\pgfqpoint{0.000000in}{0.000000in}}%
\pgfpathclose%
\pgfusepath{stroke,fill}%
\end{pgfscope}%
\begin{pgfscope}%
\pgfpathrectangle{\pgfqpoint{0.647939in}{0.492442in}}{\pgfqpoint{3.079299in}{3.079299in}}%
\pgfusepath{clip}%
\pgfsetroundcap%
\pgfsetroundjoin%
\pgfsetlinewidth{0.301125pt}%
\definecolor{currentstroke}{rgb}{0.500000,0.500000,0.500000}%
\pgfsetstrokecolor{currentstroke}%
\pgfsetstrokeopacity{0.300000}%
\pgfsetdash{}{0pt}%
\pgfpathmoveto{\pgfqpoint{0.940044in}{1.054880in}}%
\pgfusepath{stroke}%
\end{pgfscope}%
\begin{pgfscope}%
\pgfpathrectangle{\pgfqpoint{0.647939in}{0.492442in}}{\pgfqpoint{3.079299in}{3.079299in}}%
\pgfusepath{clip}%
\pgfsetroundcap%
\pgfsetroundjoin%
\definecolor{currentfill}{rgb}{0.500000,0.500000,0.500000}%
\pgfsetfillcolor{currentfill}%
\pgfsetfillopacity{0.300000}%
\pgfsetlinewidth{0.301125pt}%
\definecolor{currentstroke}{rgb}{0.500000,0.500000,0.500000}%
\pgfsetstrokecolor{currentstroke}%
\pgfsetstrokeopacity{0.300000}%
\pgfsetdash{}{0pt}%
\pgfpathmoveto{\pgfqpoint{0.000000in}{0.000000in}}%
\pgfpathlineto{\pgfqpoint{0.000000in}{0.000000in}}%
\pgfpathclose%
\pgfusepath{stroke,fill}%
\end{pgfscope}%
\begin{pgfscope}%
\pgfpathrectangle{\pgfqpoint{0.647939in}{0.492442in}}{\pgfqpoint{3.079299in}{3.079299in}}%
\pgfusepath{clip}%
\pgfsetroundcap%
\pgfsetroundjoin%
\pgfsetlinewidth{0.301125pt}%
\definecolor{currentstroke}{rgb}{0.500000,0.500000,0.500000}%
\pgfsetstrokecolor{currentstroke}%
\pgfsetstrokeopacity{0.300000}%
\pgfsetdash{}{0pt}%
\pgfpathmoveto{\pgfqpoint{0.874723in}{0.964909in}}%
\pgfusepath{stroke}%
\end{pgfscope}%
\begin{pgfscope}%
\pgfpathrectangle{\pgfqpoint{0.647939in}{0.492442in}}{\pgfqpoint{3.079299in}{3.079299in}}%
\pgfusepath{clip}%
\pgfsetroundcap%
\pgfsetroundjoin%
\definecolor{currentfill}{rgb}{0.500000,0.500000,0.500000}%
\pgfsetfillcolor{currentfill}%
\pgfsetfillopacity{0.300000}%
\pgfsetlinewidth{0.301125pt}%
\definecolor{currentstroke}{rgb}{0.500000,0.500000,0.500000}%
\pgfsetstrokecolor{currentstroke}%
\pgfsetstrokeopacity{0.300000}%
\pgfsetdash{}{0pt}%
\pgfpathmoveto{\pgfqpoint{0.000000in}{0.000000in}}%
\pgfpathlineto{\pgfqpoint{0.000000in}{0.000000in}}%
\pgfpathclose%
\pgfusepath{stroke,fill}%
\end{pgfscope}%
\begin{pgfscope}%
\pgfpathrectangle{\pgfqpoint{0.647939in}{0.492442in}}{\pgfqpoint{3.079299in}{3.079299in}}%
\pgfusepath{clip}%
\pgfsetroundcap%
\pgfsetroundjoin%
\pgfsetlinewidth{0.301125pt}%
\definecolor{currentstroke}{rgb}{0.500000,0.500000,0.500000}%
\pgfsetstrokecolor{currentstroke}%
\pgfsetstrokeopacity{0.300000}%
\pgfsetdash{}{0pt}%
\pgfpathmoveto{\pgfqpoint{1.124402in}{1.005778in}}%
\pgfusepath{stroke}%
\end{pgfscope}%
\begin{pgfscope}%
\pgfpathrectangle{\pgfqpoint{0.647939in}{0.492442in}}{\pgfqpoint{3.079299in}{3.079299in}}%
\pgfusepath{clip}%
\pgfsetroundcap%
\pgfsetroundjoin%
\definecolor{currentfill}{rgb}{0.500000,0.500000,0.500000}%
\pgfsetfillcolor{currentfill}%
\pgfsetfillopacity{0.300000}%
\pgfsetlinewidth{0.301125pt}%
\definecolor{currentstroke}{rgb}{0.500000,0.500000,0.500000}%
\pgfsetstrokecolor{currentstroke}%
\pgfsetstrokeopacity{0.300000}%
\pgfsetdash{}{0pt}%
\pgfpathmoveto{\pgfqpoint{0.000000in}{0.000000in}}%
\pgfpathlineto{\pgfqpoint{0.000000in}{0.000000in}}%
\pgfpathclose%
\pgfusepath{stroke,fill}%
\end{pgfscope}%
\begin{pgfscope}%
\pgfpathrectangle{\pgfqpoint{0.647939in}{0.492442in}}{\pgfqpoint{3.079299in}{3.079299in}}%
\pgfusepath{clip}%
\pgfsetroundcap%
\pgfsetroundjoin%
\pgfsetlinewidth{0.301125pt}%
\definecolor{currentstroke}{rgb}{0.500000,0.500000,0.500000}%
\pgfsetstrokecolor{currentstroke}%
\pgfsetstrokeopacity{0.300000}%
\pgfsetdash{}{0pt}%
\pgfpathmoveto{\pgfqpoint{0.938495in}{0.850372in}}%
\pgfusepath{stroke}%
\end{pgfscope}%
\begin{pgfscope}%
\pgfpathrectangle{\pgfqpoint{0.647939in}{0.492442in}}{\pgfqpoint{3.079299in}{3.079299in}}%
\pgfusepath{clip}%
\pgfsetroundcap%
\pgfsetroundjoin%
\definecolor{currentfill}{rgb}{0.500000,0.500000,0.500000}%
\pgfsetfillcolor{currentfill}%
\pgfsetfillopacity{0.300000}%
\pgfsetlinewidth{0.301125pt}%
\definecolor{currentstroke}{rgb}{0.500000,0.500000,0.500000}%
\pgfsetstrokecolor{currentstroke}%
\pgfsetstrokeopacity{0.300000}%
\pgfsetdash{}{0pt}%
\pgfpathmoveto{\pgfqpoint{0.000000in}{0.000000in}}%
\pgfpathlineto{\pgfqpoint{0.000000in}{0.000000in}}%
\pgfpathclose%
\pgfusepath{stroke,fill}%
\end{pgfscope}%
\begin{pgfscope}%
\pgfpathrectangle{\pgfqpoint{0.647939in}{0.492442in}}{\pgfqpoint{3.079299in}{3.079299in}}%
\pgfusepath{clip}%
\pgfsetroundcap%
\pgfsetroundjoin%
\pgfsetlinewidth{0.301125pt}%
\definecolor{currentstroke}{rgb}{0.500000,0.500000,0.500000}%
\pgfsetstrokecolor{currentstroke}%
\pgfsetstrokeopacity{0.300000}%
\pgfsetdash{}{0pt}%
\pgfpathmoveto{\pgfqpoint{0.937896in}{0.782374in}}%
\pgfusepath{stroke}%
\end{pgfscope}%
\begin{pgfscope}%
\pgfpathrectangle{\pgfqpoint{0.647939in}{0.492442in}}{\pgfqpoint{3.079299in}{3.079299in}}%
\pgfusepath{clip}%
\pgfsetroundcap%
\pgfsetroundjoin%
\definecolor{currentfill}{rgb}{0.500000,0.500000,0.500000}%
\pgfsetfillcolor{currentfill}%
\pgfsetfillopacity{0.300000}%
\pgfsetlinewidth{0.301125pt}%
\definecolor{currentstroke}{rgb}{0.500000,0.500000,0.500000}%
\pgfsetstrokecolor{currentstroke}%
\pgfsetstrokeopacity{0.300000}%
\pgfsetdash{}{0pt}%
\pgfpathmoveto{\pgfqpoint{0.000000in}{0.000000in}}%
\pgfpathlineto{\pgfqpoint{0.000000in}{0.000000in}}%
\pgfpathclose%
\pgfusepath{stroke,fill}%
\end{pgfscope}%
\begin{pgfscope}%
\pgfpathrectangle{\pgfqpoint{0.647939in}{0.492442in}}{\pgfqpoint{3.079299in}{3.079299in}}%
\pgfusepath{clip}%
\pgfsetroundcap%
\pgfsetroundjoin%
\pgfsetlinewidth{0.301125pt}%
\definecolor{currentstroke}{rgb}{0.500000,0.500000,0.500000}%
\pgfsetstrokecolor{currentstroke}%
\pgfsetstrokeopacity{0.300000}%
\pgfsetdash{}{0pt}%
\pgfpathmoveto{\pgfqpoint{0.937249in}{0.714472in}}%
\pgfusepath{stroke}%
\end{pgfscope}%
\begin{pgfscope}%
\pgfpathrectangle{\pgfqpoint{0.647939in}{0.492442in}}{\pgfqpoint{3.079299in}{3.079299in}}%
\pgfusepath{clip}%
\pgfsetroundcap%
\pgfsetroundjoin%
\definecolor{currentfill}{rgb}{0.500000,0.500000,0.500000}%
\pgfsetfillcolor{currentfill}%
\pgfsetfillopacity{0.300000}%
\pgfsetlinewidth{0.301125pt}%
\definecolor{currentstroke}{rgb}{0.500000,0.500000,0.500000}%
\pgfsetstrokecolor{currentstroke}%
\pgfsetstrokeopacity{0.300000}%
\pgfsetdash{}{0pt}%
\pgfpathmoveto{\pgfqpoint{0.000000in}{0.000000in}}%
\pgfpathlineto{\pgfqpoint{0.000000in}{0.000000in}}%
\pgfpathclose%
\pgfusepath{stroke,fill}%
\end{pgfscope}%
\begin{pgfscope}%
\pgfpathrectangle{\pgfqpoint{0.647939in}{0.492442in}}{\pgfqpoint{3.079299in}{3.079299in}}%
\pgfusepath{clip}%
\pgfsetroundcap%
\pgfsetroundjoin%
\pgfsetlinewidth{0.301125pt}%
\definecolor{currentstroke}{rgb}{0.500000,0.500000,0.500000}%
\pgfsetstrokecolor{currentstroke}%
\pgfsetstrokeopacity{0.300000}%
\pgfsetdash{}{0pt}%
\pgfpathmoveto{\pgfqpoint{2.223529in}{3.422713in}}%
\pgfusepath{stroke}%
\end{pgfscope}%
\begin{pgfscope}%
\pgfpathrectangle{\pgfqpoint{0.647939in}{0.492442in}}{\pgfqpoint{3.079299in}{3.079299in}}%
\pgfusepath{clip}%
\pgfsetroundcap%
\pgfsetroundjoin%
\definecolor{currentfill}{rgb}{0.500000,0.500000,0.500000}%
\pgfsetfillcolor{currentfill}%
\pgfsetfillopacity{0.300000}%
\pgfsetlinewidth{0.301125pt}%
\definecolor{currentstroke}{rgb}{0.500000,0.500000,0.500000}%
\pgfsetstrokecolor{currentstroke}%
\pgfsetstrokeopacity{0.300000}%
\pgfsetdash{}{0pt}%
\pgfpathmoveto{\pgfqpoint{0.000000in}{0.000000in}}%
\pgfpathlineto{\pgfqpoint{0.000000in}{0.000000in}}%
\pgfpathclose%
\pgfusepath{stroke,fill}%
\end{pgfscope}%
\begin{pgfscope}%
\pgfpathrectangle{\pgfqpoint{0.647939in}{0.492442in}}{\pgfqpoint{3.079299in}{3.079299in}}%
\pgfusepath{clip}%
\pgfsetroundcap%
\pgfsetroundjoin%
\pgfsetlinewidth{0.301125pt}%
\definecolor{currentstroke}{rgb}{0.500000,0.500000,0.500000}%
\pgfsetstrokecolor{currentstroke}%
\pgfsetstrokeopacity{0.300000}%
\pgfsetdash{}{0pt}%
\pgfpathmoveto{\pgfqpoint{0.815379in}{3.155856in}}%
\pgfusepath{stroke}%
\end{pgfscope}%
\begin{pgfscope}%
\pgfpathrectangle{\pgfqpoint{0.647939in}{0.492442in}}{\pgfqpoint{3.079299in}{3.079299in}}%
\pgfusepath{clip}%
\pgfsetroundcap%
\pgfsetroundjoin%
\definecolor{currentfill}{rgb}{0.500000,0.500000,0.500000}%
\pgfsetfillcolor{currentfill}%
\pgfsetfillopacity{0.300000}%
\pgfsetlinewidth{0.301125pt}%
\definecolor{currentstroke}{rgb}{0.500000,0.500000,0.500000}%
\pgfsetstrokecolor{currentstroke}%
\pgfsetstrokeopacity{0.300000}%
\pgfsetdash{}{0pt}%
\pgfpathmoveto{\pgfqpoint{0.000000in}{0.000000in}}%
\pgfpathlineto{\pgfqpoint{0.000000in}{0.000000in}}%
\pgfpathclose%
\pgfusepath{stroke,fill}%
\end{pgfscope}%
\begin{pgfscope}%
\pgfpathrectangle{\pgfqpoint{0.647939in}{0.492442in}}{\pgfqpoint{3.079299in}{3.079299in}}%
\pgfusepath{clip}%
\pgfsetroundcap%
\pgfsetroundjoin%
\pgfsetlinewidth{0.301125pt}%
\definecolor{currentstroke}{rgb}{0.500000,0.500000,0.500000}%
\pgfsetstrokecolor{currentstroke}%
\pgfsetstrokeopacity{0.300000}%
\pgfsetdash{}{0pt}%
\pgfpathmoveto{\pgfqpoint{0.950351in}{3.038454in}}%
\pgfusepath{stroke}%
\end{pgfscope}%
\begin{pgfscope}%
\pgfpathrectangle{\pgfqpoint{0.647939in}{0.492442in}}{\pgfqpoint{3.079299in}{3.079299in}}%
\pgfusepath{clip}%
\pgfsetroundcap%
\pgfsetroundjoin%
\definecolor{currentfill}{rgb}{0.500000,0.500000,0.500000}%
\pgfsetfillcolor{currentfill}%
\pgfsetfillopacity{0.300000}%
\pgfsetlinewidth{0.301125pt}%
\definecolor{currentstroke}{rgb}{0.500000,0.500000,0.500000}%
\pgfsetstrokecolor{currentstroke}%
\pgfsetstrokeopacity{0.300000}%
\pgfsetdash{}{0pt}%
\pgfpathmoveto{\pgfqpoint{0.000000in}{0.000000in}}%
\pgfpathlineto{\pgfqpoint{0.000000in}{0.000000in}}%
\pgfpathclose%
\pgfusepath{stroke,fill}%
\end{pgfscope}%
\begin{pgfscope}%
\pgfpathrectangle{\pgfqpoint{0.647939in}{0.492442in}}{\pgfqpoint{3.079299in}{3.079299in}}%
\pgfusepath{clip}%
\pgfsetroundcap%
\pgfsetroundjoin%
\pgfsetlinewidth{0.301125pt}%
\definecolor{currentstroke}{rgb}{0.500000,0.500000,0.500000}%
\pgfsetstrokecolor{currentstroke}%
\pgfsetstrokeopacity{0.300000}%
\pgfsetdash{}{0pt}%
\pgfpathmoveto{\pgfqpoint{3.098428in}{2.386902in}}%
\pgfusepath{stroke}%
\end{pgfscope}%
\begin{pgfscope}%
\pgfpathrectangle{\pgfqpoint{0.647939in}{0.492442in}}{\pgfqpoint{3.079299in}{3.079299in}}%
\pgfusepath{clip}%
\pgfsetroundcap%
\pgfsetroundjoin%
\definecolor{currentfill}{rgb}{0.500000,0.500000,0.500000}%
\pgfsetfillcolor{currentfill}%
\pgfsetfillopacity{0.300000}%
\pgfsetlinewidth{0.301125pt}%
\definecolor{currentstroke}{rgb}{0.500000,0.500000,0.500000}%
\pgfsetstrokecolor{currentstroke}%
\pgfsetstrokeopacity{0.300000}%
\pgfsetdash{}{0pt}%
\pgfpathmoveto{\pgfqpoint{0.000000in}{0.000000in}}%
\pgfpathlineto{\pgfqpoint{0.000000in}{0.000000in}}%
\pgfpathclose%
\pgfusepath{stroke,fill}%
\end{pgfscope}%
\begin{pgfscope}%
\pgfpathrectangle{\pgfqpoint{0.647939in}{0.492442in}}{\pgfqpoint{3.079299in}{3.079299in}}%
\pgfusepath{clip}%
\pgfsetroundcap%
\pgfsetroundjoin%
\pgfsetlinewidth{0.301125pt}%
\definecolor{currentstroke}{rgb}{0.500000,0.500000,0.500000}%
\pgfsetstrokecolor{currentstroke}%
\pgfsetstrokeopacity{0.300000}%
\pgfsetdash{}{0pt}%
\pgfpathmoveto{\pgfqpoint{2.131467in}{3.186851in}}%
\pgfusepath{stroke}%
\end{pgfscope}%
\begin{pgfscope}%
\pgfpathrectangle{\pgfqpoint{0.647939in}{0.492442in}}{\pgfqpoint{3.079299in}{3.079299in}}%
\pgfusepath{clip}%
\pgfsetroundcap%
\pgfsetroundjoin%
\definecolor{currentfill}{rgb}{0.500000,0.500000,0.500000}%
\pgfsetfillcolor{currentfill}%
\pgfsetfillopacity{0.300000}%
\pgfsetlinewidth{0.301125pt}%
\definecolor{currentstroke}{rgb}{0.500000,0.500000,0.500000}%
\pgfsetstrokecolor{currentstroke}%
\pgfsetstrokeopacity{0.300000}%
\pgfsetdash{}{0pt}%
\pgfpathmoveto{\pgfqpoint{0.000000in}{0.000000in}}%
\pgfpathlineto{\pgfqpoint{0.000000in}{0.000000in}}%
\pgfpathclose%
\pgfusepath{stroke,fill}%
\end{pgfscope}%
\begin{pgfscope}%
\pgfpathrectangle{\pgfqpoint{0.647939in}{0.492442in}}{\pgfqpoint{3.079299in}{3.079299in}}%
\pgfusepath{clip}%
\pgfsetroundcap%
\pgfsetroundjoin%
\pgfsetlinewidth{0.301125pt}%
\definecolor{currentstroke}{rgb}{0.500000,0.500000,0.500000}%
\pgfsetstrokecolor{currentstroke}%
\pgfsetstrokeopacity{0.300000}%
\pgfsetdash{}{0pt}%
\pgfpathmoveto{\pgfqpoint{2.221876in}{0.770113in}}%
\pgfusepath{stroke}%
\end{pgfscope}%
\begin{pgfscope}%
\pgfpathrectangle{\pgfqpoint{0.647939in}{0.492442in}}{\pgfqpoint{3.079299in}{3.079299in}}%
\pgfusepath{clip}%
\pgfsetroundcap%
\pgfsetroundjoin%
\definecolor{currentfill}{rgb}{0.500000,0.500000,0.500000}%
\pgfsetfillcolor{currentfill}%
\pgfsetfillopacity{0.300000}%
\pgfsetlinewidth{0.301125pt}%
\definecolor{currentstroke}{rgb}{0.500000,0.500000,0.500000}%
\pgfsetstrokecolor{currentstroke}%
\pgfsetstrokeopacity{0.300000}%
\pgfsetdash{}{0pt}%
\pgfpathmoveto{\pgfqpoint{0.000000in}{0.000000in}}%
\pgfpathlineto{\pgfqpoint{0.000000in}{0.000000in}}%
\pgfpathclose%
\pgfusepath{stroke,fill}%
\end{pgfscope}%
\begin{pgfscope}%
\pgfpathrectangle{\pgfqpoint{0.647939in}{0.492442in}}{\pgfqpoint{3.079299in}{3.079299in}}%
\pgfusepath{clip}%
\pgfsetroundcap%
\pgfsetroundjoin%
\pgfsetlinewidth{0.301125pt}%
\definecolor{currentstroke}{rgb}{0.500000,0.500000,0.500000}%
\pgfsetstrokecolor{currentstroke}%
\pgfsetstrokeopacity{0.300000}%
\pgfsetdash{}{0pt}%
\pgfpathmoveto{\pgfqpoint{3.311883in}{1.635763in}}%
\pgfusepath{stroke}%
\end{pgfscope}%
\begin{pgfscope}%
\pgfpathrectangle{\pgfqpoint{0.647939in}{0.492442in}}{\pgfqpoint{3.079299in}{3.079299in}}%
\pgfusepath{clip}%
\pgfsetroundcap%
\pgfsetroundjoin%
\definecolor{currentfill}{rgb}{0.500000,0.500000,0.500000}%
\pgfsetfillcolor{currentfill}%
\pgfsetfillopacity{0.300000}%
\pgfsetlinewidth{0.301125pt}%
\definecolor{currentstroke}{rgb}{0.500000,0.500000,0.500000}%
\pgfsetstrokecolor{currentstroke}%
\pgfsetstrokeopacity{0.300000}%
\pgfsetdash{}{0pt}%
\pgfpathmoveto{\pgfqpoint{0.000000in}{0.000000in}}%
\pgfpathlineto{\pgfqpoint{0.000000in}{0.000000in}}%
\pgfpathclose%
\pgfusepath{stroke,fill}%
\end{pgfscope}%
\begin{pgfscope}%
\pgfpathrectangle{\pgfqpoint{0.647939in}{0.492442in}}{\pgfqpoint{3.079299in}{3.079299in}}%
\pgfusepath{clip}%
\pgfsetroundcap%
\pgfsetroundjoin%
\pgfsetlinewidth{0.301125pt}%
\definecolor{currentstroke}{rgb}{0.500000,0.500000,0.500000}%
\pgfsetstrokecolor{currentstroke}%
\pgfsetstrokeopacity{0.300000}%
\pgfsetdash{}{0pt}%
\pgfpathmoveto{\pgfqpoint{3.392319in}{2.676116in}}%
\pgfusepath{stroke}%
\end{pgfscope}%
\begin{pgfscope}%
\pgfpathrectangle{\pgfqpoint{0.647939in}{0.492442in}}{\pgfqpoint{3.079299in}{3.079299in}}%
\pgfusepath{clip}%
\pgfsetroundcap%
\pgfsetroundjoin%
\definecolor{currentfill}{rgb}{0.500000,0.500000,0.500000}%
\pgfsetfillcolor{currentfill}%
\pgfsetfillopacity{0.300000}%
\pgfsetlinewidth{0.301125pt}%
\definecolor{currentstroke}{rgb}{0.500000,0.500000,0.500000}%
\pgfsetstrokecolor{currentstroke}%
\pgfsetstrokeopacity{0.300000}%
\pgfsetdash{}{0pt}%
\pgfpathmoveto{\pgfqpoint{0.000000in}{0.000000in}}%
\pgfpathlineto{\pgfqpoint{0.000000in}{0.000000in}}%
\pgfpathclose%
\pgfusepath{stroke,fill}%
\end{pgfscope}%
\begin{pgfscope}%
\pgfpathrectangle{\pgfqpoint{0.647939in}{0.492442in}}{\pgfqpoint{3.079299in}{3.079299in}}%
\pgfusepath{clip}%
\pgfsetroundcap%
\pgfsetroundjoin%
\pgfsetlinewidth{0.301125pt}%
\definecolor{currentstroke}{rgb}{0.500000,0.500000,0.500000}%
\pgfsetstrokecolor{currentstroke}%
\pgfsetstrokeopacity{0.300000}%
\pgfsetdash{}{0pt}%
\pgfpathmoveto{\pgfqpoint{3.252225in}{2.444660in}}%
\pgfusepath{stroke}%
\end{pgfscope}%
\begin{pgfscope}%
\pgfpathrectangle{\pgfqpoint{0.647939in}{0.492442in}}{\pgfqpoint{3.079299in}{3.079299in}}%
\pgfusepath{clip}%
\pgfsetroundcap%
\pgfsetroundjoin%
\definecolor{currentfill}{rgb}{0.500000,0.500000,0.500000}%
\pgfsetfillcolor{currentfill}%
\pgfsetfillopacity{0.300000}%
\pgfsetlinewidth{0.301125pt}%
\definecolor{currentstroke}{rgb}{0.500000,0.500000,0.500000}%
\pgfsetstrokecolor{currentstroke}%
\pgfsetstrokeopacity{0.300000}%
\pgfsetdash{}{0pt}%
\pgfpathmoveto{\pgfqpoint{0.000000in}{0.000000in}}%
\pgfpathlineto{\pgfqpoint{0.000000in}{0.000000in}}%
\pgfpathclose%
\pgfusepath{stroke,fill}%
\end{pgfscope}%
\begin{pgfscope}%
\pgfpathrectangle{\pgfqpoint{0.647939in}{0.492442in}}{\pgfqpoint{3.079299in}{3.079299in}}%
\pgfusepath{clip}%
\pgfsetroundcap%
\pgfsetroundjoin%
\pgfsetlinewidth{0.301125pt}%
\definecolor{currentstroke}{rgb}{0.500000,0.500000,0.500000}%
\pgfsetstrokecolor{currentstroke}%
\pgfsetstrokeopacity{0.300000}%
\pgfsetdash{}{0pt}%
\pgfpathmoveto{\pgfqpoint{1.609779in}{2.076497in}}%
\pgfusepath{stroke}%
\end{pgfscope}%
\begin{pgfscope}%
\pgfpathrectangle{\pgfqpoint{0.647939in}{0.492442in}}{\pgfqpoint{3.079299in}{3.079299in}}%
\pgfusepath{clip}%
\pgfsetroundcap%
\pgfsetroundjoin%
\definecolor{currentfill}{rgb}{0.500000,0.500000,0.500000}%
\pgfsetfillcolor{currentfill}%
\pgfsetfillopacity{0.300000}%
\pgfsetlinewidth{0.301125pt}%
\definecolor{currentstroke}{rgb}{0.500000,0.500000,0.500000}%
\pgfsetstrokecolor{currentstroke}%
\pgfsetstrokeopacity{0.300000}%
\pgfsetdash{}{0pt}%
\pgfpathmoveto{\pgfqpoint{0.000000in}{0.000000in}}%
\pgfpathlineto{\pgfqpoint{0.000000in}{0.000000in}}%
\pgfpathclose%
\pgfusepath{stroke,fill}%
\end{pgfscope}%
\begin{pgfscope}%
\pgfpathrectangle{\pgfqpoint{0.647939in}{0.492442in}}{\pgfqpoint{3.079299in}{3.079299in}}%
\pgfusepath{clip}%
\pgfsetroundcap%
\pgfsetroundjoin%
\pgfsetlinewidth{0.301125pt}%
\definecolor{currentstroke}{rgb}{0.500000,0.500000,0.500000}%
\pgfsetstrokecolor{currentstroke}%
\pgfsetstrokeopacity{0.300000}%
\pgfsetdash{}{0pt}%
\pgfpathmoveto{\pgfqpoint{2.221731in}{1.048892in}}%
\pgfusepath{stroke}%
\end{pgfscope}%
\begin{pgfscope}%
\pgfpathrectangle{\pgfqpoint{0.647939in}{0.492442in}}{\pgfqpoint{3.079299in}{3.079299in}}%
\pgfusepath{clip}%
\pgfsetroundcap%
\pgfsetroundjoin%
\definecolor{currentfill}{rgb}{0.500000,0.500000,0.500000}%
\pgfsetfillcolor{currentfill}%
\pgfsetfillopacity{0.300000}%
\pgfsetlinewidth{0.301125pt}%
\definecolor{currentstroke}{rgb}{0.500000,0.500000,0.500000}%
\pgfsetstrokecolor{currentstroke}%
\pgfsetstrokeopacity{0.300000}%
\pgfsetdash{}{0pt}%
\pgfpathmoveto{\pgfqpoint{0.000000in}{0.000000in}}%
\pgfpathlineto{\pgfqpoint{0.000000in}{0.000000in}}%
\pgfpathclose%
\pgfusepath{stroke,fill}%
\end{pgfscope}%
\begin{pgfscope}%
\pgfpathrectangle{\pgfqpoint{0.647939in}{0.492442in}}{\pgfqpoint{3.079299in}{3.079299in}}%
\pgfusepath{clip}%
\pgfsetroundcap%
\pgfsetroundjoin%
\pgfsetlinewidth{0.301125pt}%
\definecolor{currentstroke}{rgb}{0.500000,0.500000,0.500000}%
\pgfsetstrokecolor{currentstroke}%
\pgfsetstrokeopacity{0.300000}%
\pgfsetdash{}{0pt}%
\pgfpathmoveto{\pgfqpoint{2.163932in}{2.772384in}}%
\pgfusepath{stroke}%
\end{pgfscope}%
\begin{pgfscope}%
\pgfpathrectangle{\pgfqpoint{0.647939in}{0.492442in}}{\pgfqpoint{3.079299in}{3.079299in}}%
\pgfusepath{clip}%
\pgfsetroundcap%
\pgfsetroundjoin%
\definecolor{currentfill}{rgb}{0.500000,0.500000,0.500000}%
\pgfsetfillcolor{currentfill}%
\pgfsetfillopacity{0.300000}%
\pgfsetlinewidth{0.301125pt}%
\definecolor{currentstroke}{rgb}{0.500000,0.500000,0.500000}%
\pgfsetstrokecolor{currentstroke}%
\pgfsetstrokeopacity{0.300000}%
\pgfsetdash{}{0pt}%
\pgfpathmoveto{\pgfqpoint{0.000000in}{0.000000in}}%
\pgfpathlineto{\pgfqpoint{0.000000in}{0.000000in}}%
\pgfpathclose%
\pgfusepath{stroke,fill}%
\end{pgfscope}%
\begin{pgfscope}%
\pgfpathrectangle{\pgfqpoint{0.647939in}{0.492442in}}{\pgfqpoint{3.079299in}{3.079299in}}%
\pgfusepath{clip}%
\pgfsetroundcap%
\pgfsetroundjoin%
\pgfsetlinewidth{0.301125pt}%
\definecolor{currentstroke}{rgb}{0.500000,0.500000,0.500000}%
\pgfsetstrokecolor{currentstroke}%
\pgfsetstrokeopacity{0.300000}%
\pgfsetdash{}{0pt}%
\pgfpathmoveto{\pgfqpoint{2.085005in}{3.010008in}}%
\pgfusepath{stroke}%
\end{pgfscope}%
\begin{pgfscope}%
\pgfpathrectangle{\pgfqpoint{0.647939in}{0.492442in}}{\pgfqpoint{3.079299in}{3.079299in}}%
\pgfusepath{clip}%
\pgfsetroundcap%
\pgfsetroundjoin%
\definecolor{currentfill}{rgb}{0.500000,0.500000,0.500000}%
\pgfsetfillcolor{currentfill}%
\pgfsetfillopacity{0.300000}%
\pgfsetlinewidth{0.301125pt}%
\definecolor{currentstroke}{rgb}{0.500000,0.500000,0.500000}%
\pgfsetstrokecolor{currentstroke}%
\pgfsetstrokeopacity{0.300000}%
\pgfsetdash{}{0pt}%
\pgfpathmoveto{\pgfqpoint{0.000000in}{0.000000in}}%
\pgfpathlineto{\pgfqpoint{0.000000in}{0.000000in}}%
\pgfpathclose%
\pgfusepath{stroke,fill}%
\end{pgfscope}%
\begin{pgfscope}%
\pgfpathrectangle{\pgfqpoint{0.647939in}{0.492442in}}{\pgfqpoint{3.079299in}{3.079299in}}%
\pgfusepath{clip}%
\pgfsetroundcap%
\pgfsetroundjoin%
\pgfsetlinewidth{0.301125pt}%
\definecolor{currentstroke}{rgb}{0.500000,0.500000,0.500000}%
\pgfsetstrokecolor{currentstroke}%
\pgfsetstrokeopacity{0.300000}%
\pgfsetdash{}{0pt}%
\pgfpathmoveto{\pgfqpoint{2.671523in}{2.207878in}}%
\pgfusepath{stroke}%
\end{pgfscope}%
\begin{pgfscope}%
\pgfpathrectangle{\pgfqpoint{0.647939in}{0.492442in}}{\pgfqpoint{3.079299in}{3.079299in}}%
\pgfusepath{clip}%
\pgfsetroundcap%
\pgfsetroundjoin%
\definecolor{currentfill}{rgb}{0.500000,0.500000,0.500000}%
\pgfsetfillcolor{currentfill}%
\pgfsetfillopacity{0.300000}%
\pgfsetlinewidth{0.301125pt}%
\definecolor{currentstroke}{rgb}{0.500000,0.500000,0.500000}%
\pgfsetstrokecolor{currentstroke}%
\pgfsetstrokeopacity{0.300000}%
\pgfsetdash{}{0pt}%
\pgfpathmoveto{\pgfqpoint{0.000000in}{0.000000in}}%
\pgfpathlineto{\pgfqpoint{0.000000in}{0.000000in}}%
\pgfpathclose%
\pgfusepath{stroke,fill}%
\end{pgfscope}%
\begin{pgfscope}%
\pgfpathrectangle{\pgfqpoint{0.647939in}{0.492442in}}{\pgfqpoint{3.079299in}{3.079299in}}%
\pgfusepath{clip}%
\pgfsetroundcap%
\pgfsetroundjoin%
\pgfsetlinewidth{0.301125pt}%
\definecolor{currentstroke}{rgb}{0.500000,0.500000,0.500000}%
\pgfsetstrokecolor{currentstroke}%
\pgfsetstrokeopacity{0.300000}%
\pgfsetdash{}{0pt}%
\pgfpathmoveto{\pgfqpoint{1.504057in}{2.857724in}}%
\pgfusepath{stroke}%
\end{pgfscope}%
\begin{pgfscope}%
\pgfpathrectangle{\pgfqpoint{0.647939in}{0.492442in}}{\pgfqpoint{3.079299in}{3.079299in}}%
\pgfusepath{clip}%
\pgfsetroundcap%
\pgfsetroundjoin%
\definecolor{currentfill}{rgb}{0.500000,0.500000,0.500000}%
\pgfsetfillcolor{currentfill}%
\pgfsetfillopacity{0.300000}%
\pgfsetlinewidth{0.301125pt}%
\definecolor{currentstroke}{rgb}{0.500000,0.500000,0.500000}%
\pgfsetstrokecolor{currentstroke}%
\pgfsetstrokeopacity{0.300000}%
\pgfsetdash{}{0pt}%
\pgfpathmoveto{\pgfqpoint{0.000000in}{0.000000in}}%
\pgfpathlineto{\pgfqpoint{0.000000in}{0.000000in}}%
\pgfpathclose%
\pgfusepath{stroke,fill}%
\end{pgfscope}%
\begin{pgfscope}%
\pgfpathrectangle{\pgfqpoint{0.647939in}{0.492442in}}{\pgfqpoint{3.079299in}{3.079299in}}%
\pgfusepath{clip}%
\pgfsetroundcap%
\pgfsetroundjoin%
\pgfsetlinewidth{0.301125pt}%
\definecolor{currentstroke}{rgb}{0.500000,0.500000,0.500000}%
\pgfsetstrokecolor{currentstroke}%
\pgfsetstrokeopacity{0.300000}%
\pgfsetdash{}{0pt}%
\pgfpathmoveto{\pgfqpoint{1.977326in}{2.436127in}}%
\pgfusepath{stroke}%
\end{pgfscope}%
\begin{pgfscope}%
\pgfpathrectangle{\pgfqpoint{0.647939in}{0.492442in}}{\pgfqpoint{3.079299in}{3.079299in}}%
\pgfusepath{clip}%
\pgfsetroundcap%
\pgfsetroundjoin%
\definecolor{currentfill}{rgb}{0.500000,0.500000,0.500000}%
\pgfsetfillcolor{currentfill}%
\pgfsetfillopacity{0.300000}%
\pgfsetlinewidth{0.301125pt}%
\definecolor{currentstroke}{rgb}{0.500000,0.500000,0.500000}%
\pgfsetstrokecolor{currentstroke}%
\pgfsetstrokeopacity{0.300000}%
\pgfsetdash{}{0pt}%
\pgfpathmoveto{\pgfqpoint{0.000000in}{0.000000in}}%
\pgfpathlineto{\pgfqpoint{0.000000in}{0.000000in}}%
\pgfpathclose%
\pgfusepath{stroke,fill}%
\end{pgfscope}%
\begin{pgfscope}%
\pgfpathrectangle{\pgfqpoint{0.647939in}{0.492442in}}{\pgfqpoint{3.079299in}{3.079299in}}%
\pgfusepath{clip}%
\pgfsetroundcap%
\pgfsetroundjoin%
\pgfsetlinewidth{0.301125pt}%
\definecolor{currentstroke}{rgb}{0.500000,0.500000,0.500000}%
\pgfsetstrokecolor{currentstroke}%
\pgfsetstrokeopacity{0.300000}%
\pgfsetdash{}{0pt}%
\pgfpathmoveto{\pgfqpoint{1.741824in}{1.860391in}}%
\pgfusepath{stroke}%
\end{pgfscope}%
\begin{pgfscope}%
\pgfpathrectangle{\pgfqpoint{0.647939in}{0.492442in}}{\pgfqpoint{3.079299in}{3.079299in}}%
\pgfusepath{clip}%
\pgfsetroundcap%
\pgfsetroundjoin%
\definecolor{currentfill}{rgb}{0.500000,0.500000,0.500000}%
\pgfsetfillcolor{currentfill}%
\pgfsetfillopacity{0.300000}%
\pgfsetlinewidth{0.301125pt}%
\definecolor{currentstroke}{rgb}{0.500000,0.500000,0.500000}%
\pgfsetstrokecolor{currentstroke}%
\pgfsetstrokeopacity{0.300000}%
\pgfsetdash{}{0pt}%
\pgfpathmoveto{\pgfqpoint{0.000000in}{0.000000in}}%
\pgfpathlineto{\pgfqpoint{0.000000in}{0.000000in}}%
\pgfpathclose%
\pgfusepath{stroke,fill}%
\end{pgfscope}%
\begin{pgfscope}%
\pgfpathrectangle{\pgfqpoint{0.647939in}{0.492442in}}{\pgfqpoint{3.079299in}{3.079299in}}%
\pgfusepath{clip}%
\pgfsetroundcap%
\pgfsetroundjoin%
\pgfsetlinewidth{0.301125pt}%
\definecolor{currentstroke}{rgb}{0.500000,0.500000,0.500000}%
\pgfsetstrokecolor{currentstroke}%
\pgfsetstrokeopacity{0.300000}%
\pgfsetdash{}{0pt}%
\pgfpathmoveto{\pgfqpoint{2.226040in}{2.706840in}}%
\pgfusepath{stroke}%
\end{pgfscope}%
\begin{pgfscope}%
\pgfpathrectangle{\pgfqpoint{0.647939in}{0.492442in}}{\pgfqpoint{3.079299in}{3.079299in}}%
\pgfusepath{clip}%
\pgfsetroundcap%
\pgfsetroundjoin%
\definecolor{currentfill}{rgb}{0.500000,0.500000,0.500000}%
\pgfsetfillcolor{currentfill}%
\pgfsetfillopacity{0.300000}%
\pgfsetlinewidth{0.301125pt}%
\definecolor{currentstroke}{rgb}{0.500000,0.500000,0.500000}%
\pgfsetstrokecolor{currentstroke}%
\pgfsetstrokeopacity{0.300000}%
\pgfsetdash{}{0pt}%
\pgfpathmoveto{\pgfqpoint{0.000000in}{0.000000in}}%
\pgfpathlineto{\pgfqpoint{0.000000in}{0.000000in}}%
\pgfpathclose%
\pgfusepath{stroke,fill}%
\end{pgfscope}%
\begin{pgfscope}%
\pgfpathrectangle{\pgfqpoint{0.647939in}{0.492442in}}{\pgfqpoint{3.079299in}{3.079299in}}%
\pgfusepath{clip}%
\pgfsetroundcap%
\pgfsetroundjoin%
\pgfsetlinewidth{0.301125pt}%
\definecolor{currentstroke}{rgb}{0.500000,0.500000,0.500000}%
\pgfsetstrokecolor{currentstroke}%
\pgfsetstrokeopacity{0.300000}%
\pgfsetdash{}{0pt}%
\pgfpathmoveto{\pgfqpoint{1.992539in}{2.507608in}}%
\pgfusepath{stroke}%
\end{pgfscope}%
\begin{pgfscope}%
\pgfpathrectangle{\pgfqpoint{0.647939in}{0.492442in}}{\pgfqpoint{3.079299in}{3.079299in}}%
\pgfusepath{clip}%
\pgfsetroundcap%
\pgfsetroundjoin%
\definecolor{currentfill}{rgb}{0.500000,0.500000,0.500000}%
\pgfsetfillcolor{currentfill}%
\pgfsetfillopacity{0.300000}%
\pgfsetlinewidth{0.301125pt}%
\definecolor{currentstroke}{rgb}{0.500000,0.500000,0.500000}%
\pgfsetstrokecolor{currentstroke}%
\pgfsetstrokeopacity{0.300000}%
\pgfsetdash{}{0pt}%
\pgfpathmoveto{\pgfqpoint{0.000000in}{0.000000in}}%
\pgfpathlineto{\pgfqpoint{0.000000in}{0.000000in}}%
\pgfpathclose%
\pgfusepath{stroke,fill}%
\end{pgfscope}%
\begin{pgfscope}%
\pgfpathrectangle{\pgfqpoint{0.647939in}{0.492442in}}{\pgfqpoint{3.079299in}{3.079299in}}%
\pgfusepath{clip}%
\pgfsetroundcap%
\pgfsetroundjoin%
\pgfsetlinewidth{0.301125pt}%
\definecolor{currentstroke}{rgb}{0.500000,0.500000,0.500000}%
\pgfsetstrokecolor{currentstroke}%
\pgfsetstrokeopacity{0.300000}%
\pgfsetdash{}{0pt}%
\pgfpathmoveto{\pgfqpoint{2.528187in}{2.031099in}}%
\pgfusepath{stroke}%
\end{pgfscope}%
\begin{pgfscope}%
\pgfpathrectangle{\pgfqpoint{0.647939in}{0.492442in}}{\pgfqpoint{3.079299in}{3.079299in}}%
\pgfusepath{clip}%
\pgfsetroundcap%
\pgfsetroundjoin%
\definecolor{currentfill}{rgb}{0.500000,0.500000,0.500000}%
\pgfsetfillcolor{currentfill}%
\pgfsetfillopacity{0.300000}%
\pgfsetlinewidth{0.301125pt}%
\definecolor{currentstroke}{rgb}{0.500000,0.500000,0.500000}%
\pgfsetstrokecolor{currentstroke}%
\pgfsetstrokeopacity{0.300000}%
\pgfsetdash{}{0pt}%
\pgfpathmoveto{\pgfqpoint{0.000000in}{0.000000in}}%
\pgfpathlineto{\pgfqpoint{0.000000in}{0.000000in}}%
\pgfpathclose%
\pgfusepath{stroke,fill}%
\end{pgfscope}%
\begin{pgfscope}%
\pgfpathrectangle{\pgfqpoint{0.647939in}{0.492442in}}{\pgfqpoint{3.079299in}{3.079299in}}%
\pgfusepath{clip}%
\pgfsetroundcap%
\pgfsetroundjoin%
\pgfsetlinewidth{0.301125pt}%
\definecolor{currentstroke}{rgb}{0.500000,0.500000,0.500000}%
\pgfsetstrokecolor{currentstroke}%
\pgfsetstrokeopacity{0.300000}%
\pgfsetdash{}{0pt}%
\pgfpathmoveto{\pgfqpoint{2.602277in}{2.162078in}}%
\pgfusepath{stroke}%
\end{pgfscope}%
\begin{pgfscope}%
\pgfpathrectangle{\pgfqpoint{0.647939in}{0.492442in}}{\pgfqpoint{3.079299in}{3.079299in}}%
\pgfusepath{clip}%
\pgfsetroundcap%
\pgfsetroundjoin%
\definecolor{currentfill}{rgb}{0.500000,0.500000,0.500000}%
\pgfsetfillcolor{currentfill}%
\pgfsetfillopacity{0.300000}%
\pgfsetlinewidth{0.301125pt}%
\definecolor{currentstroke}{rgb}{0.500000,0.500000,0.500000}%
\pgfsetstrokecolor{currentstroke}%
\pgfsetstrokeopacity{0.300000}%
\pgfsetdash{}{0pt}%
\pgfpathmoveto{\pgfqpoint{0.000000in}{0.000000in}}%
\pgfpathlineto{\pgfqpoint{0.000000in}{0.000000in}}%
\pgfpathclose%
\pgfusepath{stroke,fill}%
\end{pgfscope}%
\begin{pgfscope}%
\pgfpathrectangle{\pgfqpoint{0.647939in}{0.492442in}}{\pgfqpoint{3.079299in}{3.079299in}}%
\pgfusepath{clip}%
\pgfsetroundcap%
\pgfsetroundjoin%
\pgfsetlinewidth{0.301125pt}%
\definecolor{currentstroke}{rgb}{0.500000,0.500000,0.500000}%
\pgfsetstrokecolor{currentstroke}%
\pgfsetstrokeopacity{0.300000}%
\pgfsetdash{}{0pt}%
\pgfpathmoveto{\pgfqpoint{2.747126in}{2.230751in}}%
\pgfusepath{stroke}%
\end{pgfscope}%
\begin{pgfscope}%
\pgfpathrectangle{\pgfqpoint{0.647939in}{0.492442in}}{\pgfqpoint{3.079299in}{3.079299in}}%
\pgfusepath{clip}%
\pgfsetroundcap%
\pgfsetroundjoin%
\definecolor{currentfill}{rgb}{0.500000,0.500000,0.500000}%
\pgfsetfillcolor{currentfill}%
\pgfsetfillopacity{0.300000}%
\pgfsetlinewidth{0.301125pt}%
\definecolor{currentstroke}{rgb}{0.500000,0.500000,0.500000}%
\pgfsetstrokecolor{currentstroke}%
\pgfsetstrokeopacity{0.300000}%
\pgfsetdash{}{0pt}%
\pgfpathmoveto{\pgfqpoint{0.000000in}{0.000000in}}%
\pgfpathlineto{\pgfqpoint{0.000000in}{0.000000in}}%
\pgfpathclose%
\pgfusepath{stroke,fill}%
\end{pgfscope}%
\begin{pgfscope}%
\pgfpathrectangle{\pgfqpoint{0.647939in}{0.492442in}}{\pgfqpoint{3.079299in}{3.079299in}}%
\pgfusepath{clip}%
\pgfsetroundcap%
\pgfsetroundjoin%
\pgfsetlinewidth{0.301125pt}%
\definecolor{currentstroke}{rgb}{0.500000,0.500000,0.500000}%
\pgfsetstrokecolor{currentstroke}%
\pgfsetstrokeopacity{0.300000}%
\pgfsetdash{}{0pt}%
\pgfpathmoveto{\pgfqpoint{2.291351in}{1.616706in}}%
\pgfusepath{stroke}%
\end{pgfscope}%
\begin{pgfscope}%
\pgfpathrectangle{\pgfqpoint{0.647939in}{0.492442in}}{\pgfqpoint{3.079299in}{3.079299in}}%
\pgfusepath{clip}%
\pgfsetroundcap%
\pgfsetroundjoin%
\definecolor{currentfill}{rgb}{0.500000,0.500000,0.500000}%
\pgfsetfillcolor{currentfill}%
\pgfsetfillopacity{0.300000}%
\pgfsetlinewidth{0.301125pt}%
\definecolor{currentstroke}{rgb}{0.500000,0.500000,0.500000}%
\pgfsetstrokecolor{currentstroke}%
\pgfsetstrokeopacity{0.300000}%
\pgfsetdash{}{0pt}%
\pgfpathmoveto{\pgfqpoint{0.000000in}{0.000000in}}%
\pgfpathlineto{\pgfqpoint{0.000000in}{0.000000in}}%
\pgfpathclose%
\pgfusepath{stroke,fill}%
\end{pgfscope}%
\begin{pgfscope}%
\pgfpathrectangle{\pgfqpoint{0.647939in}{0.492442in}}{\pgfqpoint{3.079299in}{3.079299in}}%
\pgfusepath{clip}%
\pgfsetroundcap%
\pgfsetroundjoin%
\pgfsetlinewidth{0.301125pt}%
\definecolor{currentstroke}{rgb}{0.500000,0.500000,0.500000}%
\pgfsetstrokecolor{currentstroke}%
\pgfsetstrokeopacity{0.300000}%
\pgfsetdash{}{0pt}%
\pgfpathmoveto{\pgfqpoint{2.069534in}{2.330010in}}%
\pgfusepath{stroke}%
\end{pgfscope}%
\begin{pgfscope}%
\pgfpathrectangle{\pgfqpoint{0.647939in}{0.492442in}}{\pgfqpoint{3.079299in}{3.079299in}}%
\pgfusepath{clip}%
\pgfsetroundcap%
\pgfsetroundjoin%
\definecolor{currentfill}{rgb}{0.500000,0.500000,0.500000}%
\pgfsetfillcolor{currentfill}%
\pgfsetfillopacity{0.300000}%
\pgfsetlinewidth{0.301125pt}%
\definecolor{currentstroke}{rgb}{0.500000,0.500000,0.500000}%
\pgfsetstrokecolor{currentstroke}%
\pgfsetstrokeopacity{0.300000}%
\pgfsetdash{}{0pt}%
\pgfpathmoveto{\pgfqpoint{0.000000in}{0.000000in}}%
\pgfpathlineto{\pgfqpoint{0.000000in}{0.000000in}}%
\pgfpathclose%
\pgfusepath{stroke,fill}%
\end{pgfscope}%
\begin{pgfscope}%
\pgfpathrectangle{\pgfqpoint{0.647939in}{0.492442in}}{\pgfqpoint{3.079299in}{3.079299in}}%
\pgfusepath{clip}%
\pgfsetroundcap%
\pgfsetroundjoin%
\pgfsetlinewidth{0.301125pt}%
\definecolor{currentstroke}{rgb}{0.500000,0.500000,0.500000}%
\pgfsetstrokecolor{currentstroke}%
\pgfsetstrokeopacity{0.300000}%
\pgfsetdash{}{0pt}%
\pgfpathmoveto{\pgfqpoint{2.240514in}{1.751524in}}%
\pgfusepath{stroke}%
\end{pgfscope}%
\begin{pgfscope}%
\pgfpathrectangle{\pgfqpoint{0.647939in}{0.492442in}}{\pgfqpoint{3.079299in}{3.079299in}}%
\pgfusepath{clip}%
\pgfsetroundcap%
\pgfsetroundjoin%
\definecolor{currentfill}{rgb}{0.500000,0.500000,0.500000}%
\pgfsetfillcolor{currentfill}%
\pgfsetfillopacity{0.300000}%
\pgfsetlinewidth{0.301125pt}%
\definecolor{currentstroke}{rgb}{0.500000,0.500000,0.500000}%
\pgfsetstrokecolor{currentstroke}%
\pgfsetstrokeopacity{0.300000}%
\pgfsetdash{}{0pt}%
\pgfpathmoveto{\pgfqpoint{0.000000in}{0.000000in}}%
\pgfpathlineto{\pgfqpoint{0.000000in}{0.000000in}}%
\pgfpathclose%
\pgfusepath{stroke,fill}%
\end{pgfscope}%
\begin{pgfscope}%
\pgfpathrectangle{\pgfqpoint{0.647939in}{0.492442in}}{\pgfqpoint{3.079299in}{3.079299in}}%
\pgfusepath{clip}%
\pgfsetbuttcap%
\pgfsetroundjoin%
\pgfsetlinewidth{0.301125pt}%
\definecolor{currentstroke}{rgb}{0.500000,0.500000,0.500000}%
\pgfsetstrokecolor{currentstroke}%
\pgfsetstrokeopacity{0.300000}%
\pgfsetdash{}{0pt}%
\pgfpathmoveto{\pgfqpoint{0.647939in}{0.492442in}}%
\pgfpathlineto{\pgfqpoint{0.714902in}{0.506443in}}%
\pgfpathlineto{\pgfqpoint{0.781131in}{0.523555in}}%
\pgfpathlineto{\pgfqpoint{0.846348in}{0.544159in}}%
\pgfpathlineto{\pgfqpoint{0.910214in}{0.568606in}}%
\pgfpathlineto{\pgfqpoint{0.972333in}{0.597179in}}%
\pgfpathlineto{\pgfqpoint{1.032273in}{0.630056in}}%
\pgfpathlineto{\pgfqpoint{1.089605in}{0.667283in}}%
\pgfpathlineto{\pgfqpoint{1.143960in}{0.708748in}}%
\pgfpathlineto{\pgfqpoint{1.195083in}{0.754125in}}%
\pgfpathlineto{\pgfqpoint{1.242880in}{0.802983in}}%
\pgfpathlineto{\pgfqpoint{1.287444in}{0.854812in}}%
\pgfpathlineto{\pgfqpoint{1.329035in}{0.909068in}}%
\pgfpathlineto{\pgfqpoint{1.368047in}{0.965216in}}%
\pgfpathlineto{\pgfqpoint{1.404949in}{1.022763in}}%
\pgfpathlineto{\pgfqpoint{1.440234in}{1.081300in}}%
\pgfpathlineto{\pgfqpoint{1.507871in}{1.200068in}}%
\pgfpathlineto{\pgfqpoint{1.608325in}{1.378788in}}%
\pgfpathlineto{\pgfqpoint{1.642989in}{1.437712in}}%
\pgfpathlineto{\pgfqpoint{1.678772in}{1.495956in}}%
\pgfpathlineto{\pgfqpoint{1.715963in}{1.553263in}}%
\pgfpathlineto{\pgfqpoint{1.754859in}{1.609406in}}%
\pgfpathlineto{\pgfqpoint{1.795739in}{1.664097in}}%
\pgfpathlineto{\pgfqpoint{1.838883in}{1.716944in}}%
\pgfpathlineto{\pgfqpoint{1.884612in}{1.767550in}}%
\pgfpathlineto{\pgfqpoint{1.933078in}{1.815354in}}%
\pgfpathlineto{\pgfqpoint{1.984424in}{1.859922in}}%
\pgfpathlineto{\pgfqpoint{2.096507in}{1.946932in}}%
\pgfpathlineto{\pgfqpoint{2.102180in}{1.953541in}}%
\pgfpathlineto{\pgfqpoint{2.100843in}{1.953456in}}%
\pgfpathlineto{\pgfqpoint{2.102780in}{1.954265in}}%
\pgfpathlineto{\pgfqpoint{2.102780in}{1.954265in}}%
\pgfusepath{stroke}%
\end{pgfscope}%
\begin{pgfscope}%
\pgfpathrectangle{\pgfqpoint{0.647939in}{0.492442in}}{\pgfqpoint{3.079299in}{3.079299in}}%
\pgfusepath{clip}%
\pgfsetbuttcap%
\pgfsetroundjoin%
\pgfsetlinewidth{0.301125pt}%
\definecolor{currentstroke}{rgb}{0.500000,0.500000,0.500000}%
\pgfsetstrokecolor{currentstroke}%
\pgfsetstrokeopacity{0.300000}%
\pgfsetdash{}{0pt}%
\pgfpathmoveto{\pgfqpoint{0.997859in}{0.492442in}}%
\pgfpathlineto{\pgfqpoint{0.997859in}{0.492442in}}%
\pgfpathlineto{\pgfqpoint{1.055241in}{0.529581in}}%
\pgfpathlineto{\pgfqpoint{1.109357in}{0.571332in}}%
\pgfpathlineto{\pgfqpoint{1.159872in}{0.617350in}}%
\pgfpathlineto{\pgfqpoint{1.206655in}{0.667172in}}%
\pgfpathlineto{\pgfqpoint{1.249815in}{0.720165in}}%
\pgfusepath{stroke}%
\end{pgfscope}%
\begin{pgfscope}%
\pgfpathrectangle{\pgfqpoint{0.647939in}{0.492442in}}{\pgfqpoint{3.079299in}{3.079299in}}%
\pgfusepath{clip}%
\pgfsetbuttcap%
\pgfsetroundjoin%
\pgfsetlinewidth{0.301125pt}%
\definecolor{currentstroke}{rgb}{0.500000,0.500000,0.500000}%
\pgfsetstrokecolor{currentstroke}%
\pgfsetstrokeopacity{0.300000}%
\pgfsetdash{}{0pt}%
\pgfpathmoveto{\pgfqpoint{1.277796in}{0.492442in}}%
\pgfpathlineto{\pgfqpoint{1.277796in}{0.492442in}}%
\pgfpathlineto{\pgfqpoint{1.297498in}{0.557874in}}%
\pgfpathlineto{\pgfqpoint{1.317797in}{0.623106in}}%
\pgfpathlineto{\pgfqpoint{1.338702in}{0.688126in}}%
\pgfpathlineto{\pgfqpoint{1.360248in}{0.752979in}}%
\pgfpathlineto{\pgfqpoint{1.382477in}{0.817584in}}%
\pgfpathlineto{\pgfqpoint{1.405420in}{0.881901in}}%
\pgfpathlineto{\pgfqpoint{1.429126in}{0.945979in}}%
\pgfpathlineto{\pgfqpoint{1.453661in}{1.009779in}}%
\pgfpathlineto{\pgfqpoint{1.479090in}{1.073206in}}%
\pgfusepath{stroke}%
\end{pgfscope}%
\begin{pgfscope}%
\pgfpathrectangle{\pgfqpoint{0.647939in}{0.492442in}}{\pgfqpoint{3.079299in}{3.079299in}}%
\pgfusepath{clip}%
\pgfsetbuttcap%
\pgfsetroundjoin%
\pgfsetlinewidth{0.301125pt}%
\definecolor{currentstroke}{rgb}{0.500000,0.500000,0.500000}%
\pgfsetstrokecolor{currentstroke}%
\pgfsetstrokeopacity{0.300000}%
\pgfsetdash{}{0pt}%
\pgfpathmoveto{\pgfqpoint{1.417764in}{0.492442in}}%
\pgfpathlineto{\pgfqpoint{1.417764in}{0.492442in}}%
\pgfpathlineto{\pgfqpoint{1.399177in}{0.557669in}}%
\pgfpathlineto{\pgfqpoint{1.392411in}{0.618863in}}%
\pgfpathlineto{\pgfqpoint{1.393646in}{0.687020in}}%
\pgfpathlineto{\pgfqpoint{1.401425in}{0.754849in}}%
\pgfpathlineto{\pgfqpoint{1.413954in}{0.821901in}}%
\pgfusepath{stroke}%
\end{pgfscope}%
\begin{pgfscope}%
\pgfpathrectangle{\pgfqpoint{0.647939in}{0.492442in}}{\pgfqpoint{3.079299in}{3.079299in}}%
\pgfusepath{clip}%
\pgfsetbuttcap%
\pgfsetroundjoin%
\pgfsetlinewidth{0.301125pt}%
\definecolor{currentstroke}{rgb}{0.500000,0.500000,0.500000}%
\pgfsetstrokecolor{currentstroke}%
\pgfsetstrokeopacity{0.300000}%
\pgfsetdash{}{0pt}%
\pgfpathmoveto{\pgfqpoint{1.697700in}{0.492442in}}%
\pgfpathlineto{\pgfqpoint{1.697700in}{0.492442in}}%
\pgfpathlineto{\pgfqpoint{1.633898in}{0.516671in}}%
\pgfpathlineto{\pgfqpoint{1.574826in}{0.550521in}}%
\pgfpathlineto{\pgfqpoint{1.524221in}{0.595771in}}%
\pgfpathlineto{\pgfqpoint{1.489644in}{0.645567in}}%
\pgfpathlineto{\pgfqpoint{1.468216in}{0.697734in}}%
\pgfpathlineto{\pgfqpoint{1.456630in}{0.753723in}}%
\pgfpathlineto{\pgfqpoint{1.453505in}{0.816105in}}%
\pgfpathlineto{\pgfqpoint{1.458485in}{0.884104in}}%
\pgfusepath{stroke}%
\end{pgfscope}%
\begin{pgfscope}%
\pgfpathrectangle{\pgfqpoint{0.647939in}{0.492442in}}{\pgfqpoint{3.079299in}{3.079299in}}%
\pgfusepath{clip}%
\pgfsetbuttcap%
\pgfsetroundjoin%
\pgfsetlinewidth{0.301125pt}%
\definecolor{currentstroke}{rgb}{0.500000,0.500000,0.500000}%
\pgfsetstrokecolor{currentstroke}%
\pgfsetstrokeopacity{0.300000}%
\pgfsetdash{}{0pt}%
\pgfpathmoveto{\pgfqpoint{1.977636in}{0.492442in}}%
\pgfpathlineto{\pgfqpoint{1.977636in}{0.492442in}}%
\pgfpathlineto{\pgfqpoint{1.909323in}{0.496166in}}%
\pgfpathlineto{\pgfqpoint{1.841251in}{0.502881in}}%
\pgfpathlineto{\pgfqpoint{1.773741in}{0.513747in}}%
\pgfpathlineto{\pgfqpoint{1.707422in}{0.530252in}}%
\pgfpathlineto{\pgfqpoint{1.643538in}{0.554314in}}%
\pgfusepath{stroke}%
\end{pgfscope}%
\begin{pgfscope}%
\pgfpathrectangle{\pgfqpoint{0.647939in}{0.492442in}}{\pgfqpoint{3.079299in}{3.079299in}}%
\pgfusepath{clip}%
\pgfsetbuttcap%
\pgfsetroundjoin%
\pgfsetlinewidth{0.301125pt}%
\definecolor{currentstroke}{rgb}{0.500000,0.500000,0.500000}%
\pgfsetstrokecolor{currentstroke}%
\pgfsetstrokeopacity{0.300000}%
\pgfsetdash{}{0pt}%
\pgfpathmoveto{\pgfqpoint{2.397541in}{0.492442in}}%
\pgfpathlineto{\pgfqpoint{2.397541in}{0.492442in}}%
\pgfpathlineto{\pgfqpoint{2.329127in}{0.493732in}}%
\pgfpathlineto{\pgfqpoint{2.260700in}{0.494053in}}%
\pgfpathlineto{\pgfqpoint{2.192272in}{0.493840in}}%
\pgfpathlineto{\pgfqpoint{2.123843in}{0.493589in}}%
\pgfpathlineto{\pgfqpoint{2.055417in}{0.493865in}}%
\pgfusepath{stroke}%
\end{pgfscope}%
\begin{pgfscope}%
\pgfpathrectangle{\pgfqpoint{0.647939in}{0.492442in}}{\pgfqpoint{3.079299in}{3.079299in}}%
\pgfusepath{clip}%
\pgfsetbuttcap%
\pgfsetroundjoin%
\pgfsetlinewidth{0.301125pt}%
\definecolor{currentstroke}{rgb}{0.500000,0.500000,0.500000}%
\pgfsetstrokecolor{currentstroke}%
\pgfsetstrokeopacity{0.300000}%
\pgfsetdash{}{0pt}%
\pgfpathmoveto{\pgfqpoint{2.817445in}{0.492442in}}%
\pgfpathlineto{\pgfqpoint{2.817445in}{0.492442in}}%
\pgfpathlineto{\pgfqpoint{2.750203in}{0.505082in}}%
\pgfpathlineto{\pgfqpoint{2.682616in}{0.515725in}}%
\pgfpathlineto{\pgfqpoint{2.614739in}{0.524322in}}%
\pgfpathlineto{\pgfqpoint{2.546635in}{0.530902in}}%
\pgfpathlineto{\pgfqpoint{2.478373in}{0.535579in}}%
\pgfpathlineto{\pgfqpoint{2.410014in}{0.538555in}}%
\pgfpathlineto{\pgfqpoint{2.341606in}{0.540110in}}%
\pgfpathlineto{\pgfqpoint{2.273180in}{0.540597in}}%
\pgfpathlineto{\pgfqpoint{2.204752in}{0.540439in}}%
\pgfpathlineto{\pgfqpoint{2.136324in}{0.540138in}}%
\pgfpathlineto{\pgfqpoint{2.067897in}{0.540274in}}%
\pgfpathlineto{\pgfqpoint{1.999483in}{0.541506in}}%
\pgfpathlineto{\pgfqpoint{1.931136in}{0.544603in}}%
\pgfpathlineto{\pgfqpoint{1.862983in}{0.550491in}}%
\pgfpathlineto{\pgfqpoint{1.795308in}{0.560333in}}%
\pgfusepath{stroke}%
\end{pgfscope}%
\begin{pgfscope}%
\pgfpathrectangle{\pgfqpoint{0.647939in}{0.492442in}}{\pgfqpoint{3.079299in}{3.079299in}}%
\pgfusepath{clip}%
\pgfsetbuttcap%
\pgfsetroundjoin%
\pgfsetlinewidth{0.301125pt}%
\definecolor{currentstroke}{rgb}{0.500000,0.500000,0.500000}%
\pgfsetstrokecolor{currentstroke}%
\pgfsetstrokeopacity{0.300000}%
\pgfsetdash{}{0pt}%
\pgfpathmoveto{\pgfqpoint{3.027398in}{0.492442in}}%
\pgfpathlineto{\pgfqpoint{3.027398in}{0.492442in}}%
\pgfpathlineto{\pgfqpoint{2.961487in}{0.510820in}}%
\pgfpathlineto{\pgfqpoint{2.895184in}{0.527718in}}%
\pgfpathlineto{\pgfqpoint{2.828471in}{0.542913in}}%
\pgfpathlineto{\pgfqpoint{2.761356in}{0.556219in}}%
\pgfpathlineto{\pgfqpoint{2.693873in}{0.567502in}}%
\pgfpathlineto{\pgfqpoint{2.626074in}{0.576698in}}%
\pgfusepath{stroke}%
\end{pgfscope}%
\begin{pgfscope}%
\pgfpathrectangle{\pgfqpoint{0.647939in}{0.492442in}}{\pgfqpoint{3.079299in}{3.079299in}}%
\pgfusepath{clip}%
\pgfsetbuttcap%
\pgfsetroundjoin%
\pgfsetlinewidth{0.301125pt}%
\definecolor{currentstroke}{rgb}{0.500000,0.500000,0.500000}%
\pgfsetstrokecolor{currentstroke}%
\pgfsetstrokeopacity{0.300000}%
\pgfsetdash{}{0pt}%
\pgfpathmoveto{\pgfqpoint{3.307334in}{0.492442in}}%
\pgfpathlineto{\pgfqpoint{3.307334in}{0.492442in}}%
\pgfpathlineto{\pgfqpoint{3.242736in}{0.515017in}}%
\pgfpathlineto{\pgfqpoint{3.178029in}{0.537274in}}%
\pgfpathlineto{\pgfqpoint{3.113119in}{0.558932in}}%
\pgfpathlineto{\pgfqpoint{3.047924in}{0.579710in}}%
\pgfpathlineto{\pgfqpoint{2.982373in}{0.599332in}}%
\pgfpathlineto{\pgfqpoint{2.916413in}{0.617529in}}%
\pgfpathlineto{\pgfqpoint{2.850016in}{0.634052in}}%
\pgfpathlineto{\pgfqpoint{2.783180in}{0.648686in}}%
\pgfpathlineto{\pgfqpoint{2.715928in}{0.661266in}}%
\pgfpathlineto{\pgfqpoint{2.648308in}{0.671688in}}%
\pgfpathlineto{\pgfqpoint{2.580387in}{0.679925in}}%
\pgfpathlineto{\pgfqpoint{2.512240in}{0.686038in}}%
\pgfpathlineto{\pgfqpoint{2.443944in}{0.690188in}}%
\pgfpathlineto{\pgfqpoint{2.375564in}{0.692628in}}%
\pgfpathlineto{\pgfqpoint{2.307146in}{0.693695in}}%
\pgfpathlineto{\pgfqpoint{2.238719in}{0.693811in}}%
\pgfpathlineto{\pgfqpoint{2.170291in}{0.693479in}}%
\pgfpathlineto{\pgfqpoint{2.101863in}{0.693303in}}%
\pgfpathlineto{\pgfqpoint{2.033440in}{0.693981in}}%
\pgfpathlineto{\pgfqpoint{1.965062in}{0.696332in}}%
\pgfpathlineto{\pgfqpoint{1.896840in}{0.701361in}}%
\pgfpathlineto{\pgfqpoint{1.829047in}{0.710350in}}%
\pgfpathlineto{\pgfqpoint{1.762302in}{0.725005in}}%
\pgfpathlineto{\pgfqpoint{1.697931in}{0.747603in}}%
\pgfpathlineto{\pgfqpoint{1.638667in}{0.780935in}}%
\pgfpathlineto{\pgfqpoint{1.590581in}{0.825575in}}%
\pgfpathlineto{\pgfqpoint{1.560332in}{0.872863in}}%
\pgfpathlineto{\pgfqpoint{1.542602in}{0.922455in}}%
\pgfpathlineto{\pgfqpoint{1.534231in}{0.976323in}}%
\pgfpathlineto{\pgfqpoint{1.534325in}{1.036474in}}%
\pgfpathlineto{\pgfqpoint{1.542971in}{1.103941in}}%
\pgfpathlineto{\pgfqpoint{1.557963in}{1.170467in}}%
\pgfusepath{stroke}%
\end{pgfscope}%
\begin{pgfscope}%
\pgfpathrectangle{\pgfqpoint{0.647939in}{0.492442in}}{\pgfqpoint{3.079299in}{3.079299in}}%
\pgfusepath{clip}%
\pgfsetbuttcap%
\pgfsetroundjoin%
\pgfsetlinewidth{0.301125pt}%
\definecolor{currentstroke}{rgb}{0.500000,0.500000,0.500000}%
\pgfsetstrokecolor{currentstroke}%
\pgfsetstrokeopacity{0.300000}%
\pgfsetdash{}{0pt}%
\pgfpathmoveto{\pgfqpoint{3.517286in}{0.492442in}}%
\pgfpathlineto{\pgfqpoint{3.517286in}{0.492442in}}%
\pgfpathlineto{\pgfqpoint{3.452629in}{0.514843in}}%
\pgfpathlineto{\pgfqpoint{3.388176in}{0.537826in}}%
\pgfpathlineto{\pgfqpoint{3.323837in}{0.561128in}}%
\pgfpathlineto{\pgfqpoint{3.259515in}{0.584478in}}%
\pgfpathlineto{\pgfqpoint{3.195111in}{0.607597in}}%
\pgfpathlineto{\pgfqpoint{3.130524in}{0.630201in}}%
\pgfpathlineto{\pgfqpoint{3.065664in}{0.652003in}}%
\pgfpathlineto{\pgfqpoint{3.000450in}{0.672719in}}%
\pgfpathlineto{\pgfqpoint{2.934820in}{0.692067in}}%
\pgfpathlineto{\pgfqpoint{2.868732in}{0.709782in}}%
\pgfpathlineto{\pgfqpoint{2.802172in}{0.725626in}}%
\pgfpathlineto{\pgfqpoint{2.735154in}{0.739401in}}%
\pgfpathlineto{\pgfqpoint{2.667720in}{0.750969in}}%
\pgfpathlineto{\pgfqpoint{2.599937in}{0.760271in}}%
\pgfpathlineto{\pgfqpoint{2.531884in}{0.767335in}}%
\pgfpathlineto{\pgfqpoint{2.463644in}{0.772287in}}%
\pgfpathlineto{\pgfqpoint{2.395290in}{0.775351in}}%
\pgfpathlineto{\pgfqpoint{2.326882in}{0.776849in}}%
\pgfpathlineto{\pgfqpoint{2.258455in}{0.777205in}}%
\pgfpathlineto{\pgfqpoint{2.190027in}{0.776935in}}%
\pgfpathlineto{\pgfqpoint{2.121599in}{0.776637in}}%
\pgfpathlineto{\pgfqpoint{2.053174in}{0.777017in}}%
\pgfpathlineto{\pgfqpoint{1.984779in}{0.778921in}}%
\pgfpathlineto{\pgfqpoint{1.916518in}{0.783410in}}%
\pgfpathlineto{\pgfqpoint{1.848661in}{0.791845in}}%
\pgfpathlineto{\pgfqpoint{1.781831in}{0.806043in}}%
\pgfpathlineto{\pgfqpoint{1.717435in}{0.828493in}}%
\pgfpathlineto{\pgfqpoint{1.658525in}{0.862294in}}%
\pgfpathlineto{\pgfqpoint{1.612998in}{0.906314in}}%
\pgfusepath{stroke}%
\end{pgfscope}%
\begin{pgfscope}%
\pgfpathrectangle{\pgfqpoint{0.647939in}{0.492442in}}{\pgfqpoint{3.079299in}{3.079299in}}%
\pgfusepath{clip}%
\pgfsetbuttcap%
\pgfsetroundjoin%
\pgfsetlinewidth{0.301125pt}%
\definecolor{currentstroke}{rgb}{0.500000,0.500000,0.500000}%
\pgfsetstrokecolor{currentstroke}%
\pgfsetstrokeopacity{0.300000}%
\pgfsetdash{}{0pt}%
\pgfpathmoveto{\pgfqpoint{3.727238in}{0.492442in}}%
\pgfpathlineto{\pgfqpoint{3.727238in}{0.492442in}}%
\pgfpathlineto{\pgfqpoint{3.661626in}{0.511857in}}%
\pgfpathlineto{\pgfqpoint{3.596437in}{0.532651in}}%
\pgfpathlineto{\pgfqpoint{3.531627in}{0.554602in}}%
\pgfpathlineto{\pgfqpoint{3.467134in}{0.577471in}}%
\pgfpathlineto{\pgfqpoint{3.402881in}{0.601006in}}%
\pgfpathlineto{\pgfqpoint{3.338777in}{0.624946in}}%
\pgfpathlineto{\pgfqpoint{3.274722in}{0.649019in}}%
\pgfpathlineto{\pgfqpoint{3.210615in}{0.672949in}}%
\pgfpathlineto{\pgfqpoint{3.146349in}{0.696450in}}%
\pgfpathlineto{\pgfqpoint{3.081826in}{0.719232in}}%
\pgfpathlineto{\pgfqpoint{3.016956in}{0.741003in}}%
\pgfpathlineto{\pgfqpoint{2.951665in}{0.761470in}}%
\pgfpathlineto{\pgfqpoint{2.885900in}{0.780353in}}%
\pgfpathlineto{\pgfqpoint{2.819634in}{0.797388in}}%
\pgfpathlineto{\pgfqpoint{2.752872in}{0.812352in}}%
\pgfpathlineto{\pgfqpoint{2.685649in}{0.825076in}}%
\pgfpathlineto{\pgfqpoint{2.618026in}{0.835462in}}%
\pgfpathlineto{\pgfqpoint{2.550082in}{0.843501in}}%
\pgfpathlineto{\pgfqpoint{2.481907in}{0.849284in}}%
\pgfpathlineto{\pgfqpoint{2.413588in}{0.853016in}}%
\pgfpathlineto{\pgfqpoint{2.345193in}{0.855010in}}%
\pgfpathlineto{\pgfqpoint{2.276770in}{0.855671in}}%
\pgfpathlineto{\pgfqpoint{2.208342in}{0.855503in}}%
\pgfpathlineto{\pgfqpoint{2.139914in}{0.855113in}}%
\pgfpathlineto{\pgfqpoint{2.071487in}{0.855237in}}%
\pgfpathlineto{\pgfqpoint{2.003083in}{0.856759in}}%
\pgfpathlineto{\pgfqpoint{1.934793in}{0.860766in}}%
\pgfpathlineto{\pgfqpoint{1.866871in}{0.868682in}}%
\pgfpathlineto{\pgfqpoint{1.799944in}{0.882460in}}%
\pgfpathlineto{\pgfqpoint{1.735524in}{0.904851in}}%
\pgfpathlineto{\pgfqpoint{1.677019in}{0.939333in}}%
\pgfpathlineto{\pgfqpoint{1.634322in}{0.982744in}}%
\pgfpathlineto{\pgfqpoint{1.608848in}{1.028062in}}%
\pgfpathlineto{\pgfqpoint{1.594905in}{1.075945in}}%
\pgfusepath{stroke}%
\end{pgfscope}%
\begin{pgfscope}%
\pgfpathrectangle{\pgfqpoint{0.647939in}{0.492442in}}{\pgfqpoint{3.079299in}{3.079299in}}%
\pgfusepath{clip}%
\pgfsetbuttcap%
\pgfsetroundjoin%
\pgfsetlinewidth{0.301125pt}%
\definecolor{currentstroke}{rgb}{0.500000,0.500000,0.500000}%
\pgfsetstrokecolor{currentstroke}%
\pgfsetstrokeopacity{0.300000}%
\pgfsetdash{}{0pt}%
\pgfpathmoveto{\pgfqpoint{3.727238in}{0.562426in}}%
\pgfpathlineto{\pgfqpoint{3.727238in}{0.562426in}}%
\pgfpathlineto{\pgfqpoint{3.661802in}{0.582426in}}%
\pgfpathlineto{\pgfqpoint{3.596818in}{0.603850in}}%
\pgfpathlineto{\pgfqpoint{3.532240in}{0.626473in}}%
\pgfpathlineto{\pgfqpoint{3.468004in}{0.650053in}}%
\pgfpathlineto{\pgfqpoint{3.404030in}{0.674338in}}%
\pgfpathlineto{\pgfqpoint{3.340225in}{0.699063in}}%
\pgfpathlineto{\pgfqpoint{3.276485in}{0.723957in}}%
\pgfpathlineto{\pgfqpoint{3.212702in}{0.748739in}}%
\pgfpathlineto{\pgfqpoint{3.148766in}{0.773123in}}%
\pgfpathlineto{\pgfqpoint{3.084572in}{0.796815in}}%
\pgfpathlineto{\pgfqpoint{3.020021in}{0.819514in}}%
\pgfpathlineto{\pgfqpoint{2.955032in}{0.840922in}}%
\pgfpathlineto{\pgfqpoint{2.889544in}{0.860742in}}%
\pgfpathlineto{\pgfqpoint{2.823521in}{0.878697in}}%
\pgfpathlineto{\pgfqpoint{2.756963in}{0.894542in}}%
\pgfpathlineto{\pgfqpoint{2.689900in}{0.908084in}}%
\pgfpathlineto{\pgfqpoint{2.622395in}{0.919204in}}%
\pgfpathlineto{\pgfqpoint{2.554530in}{0.927871in}}%
\pgfpathlineto{\pgfqpoint{2.486402in}{0.934160in}}%
\pgfpathlineto{\pgfqpoint{2.418105in}{0.938264in}}%
\pgfpathlineto{\pgfqpoint{2.349718in}{0.940497in}}%
\pgfpathlineto{\pgfqpoint{2.281297in}{0.941282in}}%
\pgfpathlineto{\pgfqpoint{2.212869in}{0.941144in}}%
\pgfpathlineto{\pgfqpoint{2.144442in}{0.940722in}}%
\pgfpathlineto{\pgfqpoint{2.076015in}{0.940793in}}%
\pgfpathlineto{\pgfqpoint{2.007611in}{0.942317in}}%
\pgfpathlineto{\pgfqpoint{1.939336in}{0.946511in}}%
\pgfpathlineto{\pgfqpoint{1.871497in}{0.954997in}}%
\pgfpathlineto{\pgfqpoint{1.804892in}{0.970078in}}%
\pgfpathlineto{\pgfqpoint{1.741534in}{0.995063in}}%
\pgfpathlineto{\pgfqpoint{1.741534in}{0.995063in}}%
\pgfpathlineto{\pgfqpoint{1.694837in}{1.025731in}}%
\pgfusepath{stroke}%
\end{pgfscope}%
\begin{pgfscope}%
\pgfpathrectangle{\pgfqpoint{0.647939in}{0.492442in}}{\pgfqpoint{3.079299in}{3.079299in}}%
\pgfusepath{clip}%
\pgfsetbuttcap%
\pgfsetroundjoin%
\pgfsetlinewidth{0.301125pt}%
\definecolor{currentstroke}{rgb}{0.500000,0.500000,0.500000}%
\pgfsetstrokecolor{currentstroke}%
\pgfsetstrokeopacity{0.300000}%
\pgfsetdash{}{0pt}%
\pgfpathmoveto{\pgfqpoint{3.727238in}{0.632410in}}%
\pgfpathlineto{\pgfqpoint{3.727238in}{0.632410in}}%
\pgfpathlineto{\pgfqpoint{3.661995in}{0.653030in}}%
\pgfpathlineto{\pgfqpoint{3.597235in}{0.675121in}}%
\pgfpathlineto{\pgfqpoint{3.532911in}{0.698456in}}%
\pgfpathlineto{\pgfqpoint{3.468958in}{0.722792in}}%
\pgfpathlineto{\pgfqpoint{3.405292in}{0.747873in}}%
\pgfpathlineto{\pgfqpoint{3.341817in}{0.773435in}}%
\pgfpathlineto{\pgfqpoint{3.278427in}{0.799205in}}%
\pgfpathlineto{\pgfqpoint{3.215007in}{0.824903in}}%
\pgfpathlineto{\pgfqpoint{3.151442in}{0.850238in}}%
\pgfpathlineto{\pgfqpoint{3.087620in}{0.874915in}}%
\pgfpathlineto{\pgfqpoint{3.023435in}{0.898627in}}%
\pgfpathlineto{\pgfqpoint{2.958795in}{0.921066in}}%
\pgfpathlineto{\pgfqpoint{2.893631in}{0.941924in}}%
\pgfpathlineto{\pgfqpoint{2.827897in}{0.960907in}}%
\pgfpathlineto{\pgfqpoint{2.761583in}{0.977746in}}%
\pgfpathlineto{\pgfqpoint{2.694716in}{0.992222in}}%
\pgfpathlineto{\pgfqpoint{2.627356in}{1.004188in}}%
\pgfpathlineto{\pgfqpoint{2.559591in}{1.013589in}}%
\pgfpathlineto{\pgfqpoint{2.491523in}{1.020480in}}%
\pgfpathlineto{\pgfqpoint{2.423256in}{1.025035in}}%
\pgfpathlineto{\pgfqpoint{2.354881in}{1.027563in}}%
\pgfpathlineto{\pgfqpoint{2.286462in}{1.028502in}}%
\pgfpathlineto{\pgfqpoint{2.218034in}{1.028406in}}%
\pgfpathlineto{\pgfqpoint{2.149607in}{1.027949in}}%
\pgfpathlineto{\pgfqpoint{2.081180in}{1.027957in}}%
\pgfpathlineto{\pgfqpoint{2.012777in}{1.029473in}}%
\pgfpathlineto{\pgfqpoint{1.944518in}{1.033867in}}%
\pgfpathlineto{\pgfqpoint{1.876784in}{1.043028in}}%
\pgfpathlineto{\pgfqpoint{1.810628in}{1.059728in}}%
\pgfpathlineto{\pgfqpoint{1.748885in}{1.088014in}}%
\pgfpathlineto{\pgfqpoint{1.748885in}{1.088014in}}%
\pgfpathlineto{\pgfqpoint{1.708736in}{1.119410in}}%
\pgfpathlineto{\pgfqpoint{1.678070in}{1.161025in}}%
\pgfpathlineto{\pgfqpoint{1.661333in}{1.204374in}}%
\pgfpathlineto{\pgfqpoint{1.654351in}{1.250949in}}%
\pgfpathlineto{\pgfqpoint{1.655894in}{1.303157in}}%
\pgfpathlineto{\pgfqpoint{1.666387in}{1.362592in}}%
\pgfpathlineto{\pgfqpoint{1.685959in}{1.427806in}}%
\pgfusepath{stroke}%
\end{pgfscope}%
\begin{pgfscope}%
\pgfpathrectangle{\pgfqpoint{0.647939in}{0.492442in}}{\pgfqpoint{3.079299in}{3.079299in}}%
\pgfusepath{clip}%
\pgfsetbuttcap%
\pgfsetroundjoin%
\pgfsetlinewidth{0.301125pt}%
\definecolor{currentstroke}{rgb}{0.500000,0.500000,0.500000}%
\pgfsetstrokecolor{currentstroke}%
\pgfsetstrokeopacity{0.300000}%
\pgfsetdash{}{0pt}%
\pgfpathmoveto{\pgfqpoint{3.727238in}{0.702394in}}%
\pgfpathlineto{\pgfqpoint{3.727238in}{0.702394in}}%
\pgfpathlineto{\pgfqpoint{3.662207in}{0.723671in}}%
\pgfpathlineto{\pgfqpoint{3.597693in}{0.746471in}}%
\pgfpathlineto{\pgfqpoint{3.533649in}{0.770562in}}%
\pgfpathlineto{\pgfqpoint{3.470007in}{0.795701in}}%
\pgfpathlineto{\pgfqpoint{3.406682in}{0.821630in}}%
\pgfpathlineto{\pgfqpoint{3.343575in}{0.848085in}}%
\pgfpathlineto{\pgfqpoint{3.280574in}{0.874793in}}%
\pgfpathlineto{\pgfqpoint{3.217561in}{0.901474in}}%
\pgfpathlineto{\pgfqpoint{3.154416in}{0.927838in}}%
\pgfpathlineto{\pgfqpoint{3.091018in}{0.953585in}}%
\pgfpathlineto{\pgfqpoint{3.027253in}{0.978405in}}%
\pgfpathlineto{\pgfqpoint{2.963020in}{1.001982in}}%
\pgfpathlineto{\pgfqpoint{2.898237in}{1.023995in}}%
\pgfpathlineto{\pgfqpoint{2.832849in}{1.044133in}}%
\pgfpathlineto{\pgfqpoint{2.766834in}{1.062103in}}%
\pgfpathlineto{\pgfqpoint{2.700209in}{1.077654in}}%
\pgfpathlineto{\pgfqpoint{2.633032in}{1.090607in}}%
\pgfpathlineto{\pgfqpoint{2.565394in}{1.100875in}}%
\pgfpathlineto{\pgfqpoint{2.497405in}{1.108486in}}%
\pgfpathlineto{\pgfqpoint{2.429179in}{1.113597in}}%
\pgfpathlineto{\pgfqpoint{2.360821in}{1.116501in}}%
\pgfpathlineto{\pgfqpoint{2.292406in}{1.117641in}}%
\pgfpathlineto{\pgfqpoint{2.223978in}{1.117600in}}%
\pgfpathlineto{\pgfqpoint{2.155551in}{1.117100in}}%
\pgfpathlineto{\pgfqpoint{2.087124in}{1.117033in}}%
\pgfpathlineto{\pgfqpoint{2.018722in}{1.118534in}}%
\pgfpathlineto{\pgfqpoint{1.950481in}{1.123149in}}%
\pgfpathlineto{\pgfqpoint{1.882883in}{1.133126in}}%
\pgfpathlineto{\pgfqpoint{1.817380in}{1.151907in}}%
\pgfpathlineto{\pgfqpoint{1.817380in}{1.151907in}}%
\pgfpathlineto{\pgfqpoint{1.767101in}{1.177694in}}%
\pgfpathlineto{\pgfqpoint{1.767101in}{1.177694in}}%
\pgfusepath{stroke}%
\end{pgfscope}%
\begin{pgfscope}%
\pgfpathrectangle{\pgfqpoint{0.647939in}{0.492442in}}{\pgfqpoint{3.079299in}{3.079299in}}%
\pgfusepath{clip}%
\pgfsetbuttcap%
\pgfsetroundjoin%
\pgfsetlinewidth{0.301125pt}%
\definecolor{currentstroke}{rgb}{0.500000,0.500000,0.500000}%
\pgfsetstrokecolor{currentstroke}%
\pgfsetstrokeopacity{0.300000}%
\pgfsetdash{}{0pt}%
\pgfpathmoveto{\pgfqpoint{3.727238in}{0.772378in}}%
\pgfpathlineto{\pgfqpoint{3.727238in}{0.772378in}}%
\pgfpathlineto{\pgfqpoint{3.662440in}{0.794354in}}%
\pgfpathlineto{\pgfqpoint{3.598197in}{0.817906in}}%
\pgfpathlineto{\pgfqpoint{3.534462in}{0.842801in}}%
\pgfpathlineto{\pgfqpoint{3.471164in}{0.868794in}}%
\pgfpathlineto{\pgfqpoint{3.408217in}{0.895628in}}%
\pgfpathlineto{\pgfqpoint{3.345519in}{0.923038in}}%
\pgfpathlineto{\pgfqpoint{3.282954in}{0.950753in}}%
\pgfpathlineto{\pgfqpoint{3.220401in}{0.978494in}}%
\pgfpathlineto{\pgfqpoint{3.157732in}{1.005970in}}%
\pgfpathlineto{\pgfqpoint{3.094820in}{1.032883in}}%
\pgfpathlineto{\pgfqpoint{3.031542in}{1.058920in}}%
\pgfpathlineto{\pgfqpoint{2.967786in}{1.083757in}}%
\pgfpathlineto{\pgfqpoint{2.903457in}{1.107063in}}%
\pgfpathlineto{\pgfqpoint{2.838486in}{1.128507in}}%
\pgfpathlineto{\pgfqpoint{2.772840in}{1.147773in}}%
\pgfpathlineto{\pgfqpoint{2.706522in}{1.164578in}}%
\pgfpathlineto{\pgfqpoint{2.639582in}{1.178698in}}%
\pgfpathlineto{\pgfqpoint{2.572110in}{1.190004in}}%
\pgfpathlineto{\pgfqpoint{2.504226in}{1.198491in}}%
\pgfpathlineto{\pgfqpoint{2.436058in}{1.204290in}}%
\pgfpathlineto{\pgfqpoint{2.367724in}{1.207681in}}%
\pgfpathlineto{\pgfqpoint{2.299316in}{1.209099in}}%
\pgfpathlineto{\pgfqpoint{2.230889in}{1.209144in}}%
\pgfpathlineto{\pgfqpoint{2.162463in}{1.208592in}}%
\pgfpathlineto{\pgfqpoint{2.094036in}{1.208418in}}%
\pgfpathlineto{\pgfqpoint{2.025635in}{1.209887in}}%
\pgfpathlineto{\pgfqpoint{1.957421in}{1.214767in}}%
\pgfpathlineto{\pgfqpoint{1.890008in}{1.225775in}}%
\pgfpathlineto{\pgfqpoint{1.825509in}{1.247358in}}%
\pgfpathlineto{\pgfqpoint{1.825509in}{1.247358in}}%
\pgfpathlineto{\pgfqpoint{1.783266in}{1.273233in}}%
\pgfpathlineto{\pgfqpoint{1.783266in}{1.273233in}}%
\pgfpathlineto{\pgfqpoint{1.753698in}{1.304690in}}%
\pgfpathlineto{\pgfqpoint{1.734399in}{1.343635in}}%
\pgfusepath{stroke}%
\end{pgfscope}%
\begin{pgfscope}%
\pgfpathrectangle{\pgfqpoint{0.647939in}{0.492442in}}{\pgfqpoint{3.079299in}{3.079299in}}%
\pgfusepath{clip}%
\pgfsetbuttcap%
\pgfsetroundjoin%
\pgfsetlinewidth{0.301125pt}%
\definecolor{currentstroke}{rgb}{0.500000,0.500000,0.500000}%
\pgfsetstrokecolor{currentstroke}%
\pgfsetstrokeopacity{0.300000}%
\pgfsetdash{}{0pt}%
\pgfpathmoveto{\pgfqpoint{3.727238in}{0.842362in}}%
\pgfpathlineto{\pgfqpoint{3.727238in}{0.842362in}}%
\pgfpathlineto{\pgfqpoint{3.662698in}{0.865081in}}%
\pgfpathlineto{\pgfqpoint{3.598754in}{0.889434in}}%
\pgfpathlineto{\pgfqpoint{3.535360in}{0.915185in}}%
\pgfpathlineto{\pgfqpoint{3.472445in}{0.942089in}}%
\pgfpathlineto{\pgfqpoint{3.409919in}{0.969889in}}%
\pgfpathlineto{\pgfqpoint{3.347678in}{0.998322in}}%
\pgfpathlineto{\pgfqpoint{3.285604in}{1.027119in}}%
\pgfpathlineto{\pgfqpoint{3.223571in}{1.056004in}}%
\pgfpathlineto{\pgfqpoint{3.161446in}{1.084689in}}%
\pgfpathlineto{\pgfqpoint{3.099094in}{1.112877in}}%
\pgfpathlineto{\pgfqpoint{3.036383in}{1.140254in}}%
\pgfpathlineto{\pgfqpoint{2.973190in}{1.166493in}}%
\pgfpathlineto{\pgfqpoint{2.909406in}{1.191253in}}%
\pgfpathlineto{\pgfqpoint{2.844946in}{1.214186in}}%
\pgfpathlineto{\pgfqpoint{2.779759in}{1.234950in}}%
\pgfpathlineto{\pgfqpoint{2.713833in}{1.253226in}}%
\pgfpathlineto{\pgfqpoint{2.647207in}{1.268746in}}%
\pgfpathlineto{\pgfqpoint{2.579964in}{1.281326in}}%
\pgfpathlineto{\pgfqpoint{2.512227in}{1.290903in}}%
\pgfpathlineto{\pgfqpoint{2.444140in}{1.297570in}}%
\pgfpathlineto{\pgfqpoint{2.375842in}{1.301590in}}%
\pgfpathlineto{\pgfqpoint{2.307446in}{1.303394in}}%
\pgfpathlineto{\pgfqpoint{2.239020in}{1.303591in}}%
\pgfpathlineto{\pgfqpoint{2.170595in}{1.302989in}}%
\pgfpathlineto{\pgfqpoint{2.102168in}{1.302662in}}%
\pgfpathlineto{\pgfqpoint{2.033768in}{1.304051in}}%
\pgfpathlineto{\pgfqpoint{1.965594in}{1.309242in}}%
\pgfpathlineto{\pgfqpoint{1.898492in}{1.321647in}}%
\pgfpathlineto{\pgfqpoint{1.835770in}{1.347268in}}%
\pgfpathlineto{\pgfqpoint{1.835770in}{1.347268in}}%
\pgfpathlineto{\pgfqpoint{1.801820in}{1.373850in}}%
\pgfpathlineto{\pgfqpoint{1.777575in}{1.410576in}}%
\pgfpathlineto{\pgfqpoint{1.766973in}{1.447958in}}%
\pgfpathlineto{\pgfqpoint{1.765534in}{1.488581in}}%
\pgfpathlineto{\pgfqpoint{1.772579in}{1.534008in}}%
\pgfusepath{stroke}%
\end{pgfscope}%
\begin{pgfscope}%
\pgfpathrectangle{\pgfqpoint{0.647939in}{0.492442in}}{\pgfqpoint{3.079299in}{3.079299in}}%
\pgfusepath{clip}%
\pgfsetbuttcap%
\pgfsetroundjoin%
\pgfsetlinewidth{0.301125pt}%
\definecolor{currentstroke}{rgb}{0.500000,0.500000,0.500000}%
\pgfsetstrokecolor{currentstroke}%
\pgfsetstrokeopacity{0.300000}%
\pgfsetdash{}{0pt}%
\pgfpathmoveto{\pgfqpoint{3.727238in}{0.912347in}}%
\pgfpathlineto{\pgfqpoint{3.727238in}{0.912347in}}%
\pgfpathlineto{\pgfqpoint{3.662983in}{0.935857in}}%
\pgfpathlineto{\pgfqpoint{3.599371in}{0.961063in}}%
\pgfpathlineto{\pgfqpoint{3.536356in}{0.987728in}}%
\pgfpathlineto{\pgfqpoint{3.473866in}{1.015604in}}%
\pgfpathlineto{\pgfqpoint{3.411810in}{1.044438in}}%
\pgfpathlineto{\pgfqpoint{3.350083in}{1.073969in}}%
\pgfpathlineto{\pgfqpoint{3.288563in}{1.103931in}}%
\pgfpathlineto{\pgfqpoint{3.227122in}{1.134054in}}%
\pgfpathlineto{\pgfqpoint{3.165621in}{1.164055in}}%
\pgfpathlineto{\pgfqpoint{3.103920in}{1.193640in}}%
\pgfpathlineto{\pgfqpoint{3.041877in}{1.222499in}}%
\pgfpathlineto{\pgfqpoint{2.979357in}{1.250305in}}%
\pgfpathlineto{\pgfqpoint{2.916234in}{1.276709in}}%
\pgfpathlineto{\pgfqpoint{2.852406in}{1.301351in}}%
\pgfpathlineto{\pgfqpoint{2.787799in}{1.323862in}}%
\pgfpathlineto{\pgfqpoint{2.722383in}{1.343885in}}%
\pgfpathlineto{\pgfqpoint{2.656177in}{1.361101in}}%
\pgfpathlineto{\pgfqpoint{2.589253in}{1.375264in}}%
\pgfpathlineto{\pgfqpoint{2.521734in}{1.386242in}}%
\pgfpathlineto{\pgfqpoint{2.453773in}{1.394055in}}%
\pgfpathlineto{\pgfqpoint{2.385532in}{1.398915in}}%
\pgfpathlineto{\pgfqpoint{2.317153in}{1.401249in}}%
\pgfpathlineto{\pgfqpoint{2.248731in}{1.401682in}}%
\pgfpathlineto{\pgfqpoint{2.180306in}{1.401069in}}%
\pgfpathlineto{\pgfqpoint{2.111880in}{1.400559in}}%
\pgfpathlineto{\pgfqpoint{2.043477in}{1.401790in}}%
\pgfpathlineto{\pgfqpoint{1.975350in}{1.407326in}}%
\pgfpathlineto{\pgfqpoint{1.908796in}{1.421725in}}%
\pgfpathlineto{\pgfqpoint{1.908796in}{1.421725in}}%
\pgfpathlineto{\pgfqpoint{1.864595in}{1.441691in}}%
\pgfpathlineto{\pgfqpoint{1.864595in}{1.441691in}}%
\pgfpathlineto{\pgfqpoint{1.834752in}{1.466994in}}%
\pgfpathlineto{\pgfqpoint{1.815210in}{1.501371in}}%
\pgfpathlineto{\pgfqpoint{1.808206in}{1.536185in}}%
\pgfpathlineto{\pgfqpoint{1.809938in}{1.574198in}}%
\pgfpathlineto{\pgfqpoint{1.820281in}{1.617249in}}%
\pgfusepath{stroke}%
\end{pgfscope}%
\begin{pgfscope}%
\pgfpathrectangle{\pgfqpoint{0.647939in}{0.492442in}}{\pgfqpoint{3.079299in}{3.079299in}}%
\pgfusepath{clip}%
\pgfsetbuttcap%
\pgfsetroundjoin%
\pgfsetlinewidth{0.301125pt}%
\definecolor{currentstroke}{rgb}{0.500000,0.500000,0.500000}%
\pgfsetstrokecolor{currentstroke}%
\pgfsetstrokeopacity{0.300000}%
\pgfsetdash{}{0pt}%
\pgfpathmoveto{\pgfqpoint{3.727238in}{0.982331in}}%
\pgfpathlineto{\pgfqpoint{3.727238in}{0.982331in}}%
\pgfpathlineto{\pgfqpoint{3.663299in}{1.006686in}}%
\pgfpathlineto{\pgfqpoint{3.600057in}{1.032803in}}%
\pgfpathlineto{\pgfqpoint{3.537464in}{1.060443in}}%
\pgfpathlineto{\pgfqpoint{3.475449in}{1.089360in}}%
\pgfpathlineto{\pgfqpoint{3.413920in}{1.119302in}}%
\pgfpathlineto{\pgfqpoint{3.352772in}{1.150013in}}%
\pgfpathlineto{\pgfqpoint{3.291881in}{1.181233in}}%
\pgfpathlineto{\pgfqpoint{3.231115in}{1.212697in}}%
\pgfpathlineto{\pgfqpoint{3.170332in}{1.244128in}}%
\pgfpathlineto{\pgfqpoint{3.109388in}{1.275242in}}%
\pgfpathlineto{\pgfqpoint{3.048132in}{1.305737in}}%
\pgfpathlineto{\pgfqpoint{2.986419in}{1.335291in}}%
\pgfpathlineto{\pgfqpoint{2.924107in}{1.363557in}}%
\pgfpathlineto{\pgfqpoint{2.861074in}{1.390166in}}%
\pgfpathlineto{\pgfqpoint{2.797219in}{1.414729in}}%
\pgfpathlineto{\pgfqpoint{2.732481in}{1.436851in}}%
\pgfpathlineto{\pgfqpoint{2.666850in}{1.456153in}}%
\pgfpathlineto{\pgfqpoint{2.600381in}{1.472315in}}%
\pgfpathlineto{\pgfqpoint{2.533187in}{1.485116in}}%
\pgfpathlineto{\pgfqpoint{2.465429in}{1.494487in}}%
\pgfpathlineto{\pgfqpoint{2.397292in}{1.500555in}}%
\pgfpathlineto{\pgfqpoint{2.328949in}{1.503674in}}%
\pgfpathlineto{\pgfqpoint{2.260532in}{1.504471in}}%
\pgfpathlineto{\pgfqpoint{2.192107in}{1.503865in}}%
\pgfpathlineto{\pgfqpoint{2.123684in}{1.503129in}}%
\pgfpathlineto{\pgfqpoint{2.055281in}{1.504148in}}%
\pgfpathlineto{\pgfqpoint{1.987210in}{1.510126in}}%
\pgfpathlineto{\pgfqpoint{1.921686in}{1.527744in}}%
\pgfpathlineto{\pgfqpoint{1.921686in}{1.527744in}}%
\pgfpathlineto{\pgfqpoint{1.888478in}{1.547436in}}%
\pgfpathlineto{\pgfqpoint{1.888478in}{1.547436in}}%
\pgfusepath{stroke}%
\end{pgfscope}%
\begin{pgfscope}%
\pgfpathrectangle{\pgfqpoint{0.647939in}{0.492442in}}{\pgfqpoint{3.079299in}{3.079299in}}%
\pgfusepath{clip}%
\pgfsetbuttcap%
\pgfsetroundjoin%
\pgfsetlinewidth{0.301125pt}%
\definecolor{currentstroke}{rgb}{0.500000,0.500000,0.500000}%
\pgfsetstrokecolor{currentstroke}%
\pgfsetstrokeopacity{0.300000}%
\pgfsetdash{}{0pt}%
\pgfpathmoveto{\pgfqpoint{3.727238in}{1.052315in}}%
\pgfpathlineto{\pgfqpoint{3.727238in}{1.052315in}}%
\pgfpathlineto{\pgfqpoint{3.663651in}{1.077574in}}%
\pgfpathlineto{\pgfqpoint{3.600820in}{1.104664in}}%
\pgfpathlineto{\pgfqpoint{3.538699in}{1.133348in}}%
\pgfpathlineto{\pgfqpoint{3.477217in}{1.163379in}}%
\pgfpathlineto{\pgfqpoint{3.416283in}{1.194511in}}%
\pgfpathlineto{\pgfqpoint{3.355788in}{1.226490in}}%
\pgfpathlineto{\pgfqpoint{3.295611in}{1.259064in}}%
\pgfpathlineto{\pgfqpoint{3.235618in}{1.291976in}}%
\pgfpathlineto{\pgfqpoint{3.175667in}{1.324966in}}%
\pgfpathlineto{\pgfqpoint{3.115611in}{1.357761in}}%
\pgfpathlineto{\pgfqpoint{3.055294in}{1.390073in}}%
\pgfpathlineto{\pgfqpoint{2.994561in}{1.421592in}}%
\pgfpathlineto{\pgfqpoint{2.933257in}{1.451981in}}%
\pgfpathlineto{\pgfqpoint{2.871238in}{1.480874in}}%
\pgfpathlineto{\pgfqpoint{2.808375in}{1.507871in}}%
\pgfpathlineto{\pgfqpoint{2.744572in}{1.532550in}}%
\pgfpathlineto{\pgfqpoint{2.679776in}{1.554477in}}%
\pgfpathlineto{\pgfqpoint{2.613998in}{1.573241in}}%
\pgfpathlineto{\pgfqpoint{2.547321in}{1.588498in}}%
\pgfpathlineto{\pgfqpoint{2.479901in}{1.600040in}}%
\pgfpathlineto{\pgfqpoint{2.411949in}{1.607874in}}%
\pgfpathlineto{\pgfqpoint{2.343684in}{1.612258in}}%
\pgfpathlineto{\pgfqpoint{2.275285in}{1.613747in}}%
\pgfpathlineto{\pgfqpoint{2.206863in}{1.613249in}}%
\pgfpathlineto{\pgfqpoint{2.138443in}{1.612158in}}%
\pgfpathlineto{\pgfqpoint{2.070039in}{1.612751in}}%
\pgfpathlineto{\pgfqpoint{2.002135in}{1.619432in}}%
\pgfpathlineto{\pgfqpoint{2.002135in}{1.619432in}}%
\pgfpathlineto{\pgfqpoint{1.956850in}{1.632516in}}%
\pgfpathlineto{\pgfqpoint{1.956850in}{1.632516in}}%
\pgfpathlineto{\pgfqpoint{1.929221in}{1.650276in}}%
\pgfpathlineto{\pgfqpoint{1.929221in}{1.650276in}}%
\pgfpathlineto{\pgfqpoint{1.913714in}{1.672663in}}%
\pgfpathlineto{\pgfqpoint{1.908405in}{1.699564in}}%
\pgfusepath{stroke}%
\end{pgfscope}%
\begin{pgfscope}%
\pgfpathrectangle{\pgfqpoint{0.647939in}{0.492442in}}{\pgfqpoint{3.079299in}{3.079299in}}%
\pgfusepath{clip}%
\pgfsetbuttcap%
\pgfsetroundjoin%
\pgfsetlinewidth{0.301125pt}%
\definecolor{currentstroke}{rgb}{0.500000,0.500000,0.500000}%
\pgfsetstrokecolor{currentstroke}%
\pgfsetstrokeopacity{0.300000}%
\pgfsetdash{}{0pt}%
\pgfpathmoveto{\pgfqpoint{3.727238in}{1.122299in}}%
\pgfpathlineto{\pgfqpoint{3.727238in}{1.122299in}}%
\pgfpathlineto{\pgfqpoint{3.664045in}{1.148526in}}%
\pgfpathlineto{\pgfqpoint{3.601675in}{1.176659in}}%
\pgfpathlineto{\pgfqpoint{3.540082in}{1.206461in}}%
\pgfpathlineto{\pgfqpoint{3.479199in}{1.237688in}}%
\pgfpathlineto{\pgfqpoint{3.418933in}{1.270095in}}%
\pgfpathlineto{\pgfqpoint{3.359179in}{1.303438in}}%
\pgfpathlineto{\pgfqpoint{3.299817in}{1.337474in}}%
\pgfpathlineto{\pgfqpoint{3.240716in}{1.371964in}}%
\pgfpathlineto{\pgfqpoint{3.181738in}{1.406660in}}%
\pgfpathlineto{\pgfqpoint{3.122733in}{1.441311in}}%
\pgfpathlineto{\pgfqpoint{3.063546in}{1.475649in}}%
\pgfpathlineto{\pgfqpoint{3.004017in}{1.509387in}}%
\pgfpathlineto{\pgfqpoint{2.943981in}{1.542212in}}%
\pgfpathlineto{\pgfqpoint{2.883274in}{1.573774in}}%
\pgfpathlineto{\pgfqpoint{2.821740in}{1.603684in}}%
\pgfpathlineto{\pgfqpoint{2.759240in}{1.631511in}}%
\pgfpathlineto{\pgfqpoint{2.695671in}{1.656783in}}%
\pgfpathlineto{\pgfqpoint{2.630982in}{1.679012in}}%
\pgfpathlineto{\pgfqpoint{2.565196in}{1.697723in}}%
\pgfpathlineto{\pgfqpoint{2.498426in}{1.712522in}}%
\pgfpathlineto{\pgfqpoint{2.430872in}{1.723173in}}%
\pgfpathlineto{\pgfqpoint{2.362790in}{1.729696in}}%
\pgfpathlineto{\pgfqpoint{2.294440in}{1.732470in}}%
\pgfpathlineto{\pgfqpoint{2.226022in}{1.732360in}}%
\pgfpathlineto{\pgfqpoint{2.157610in}{1.730859in}}%
\pgfpathlineto{\pgfqpoint{2.089206in}{1.730692in}}%
\pgfpathlineto{\pgfqpoint{2.021653in}{1.738808in}}%
\pgfpathlineto{\pgfqpoint{2.021653in}{1.738808in}}%
\pgfpathlineto{\pgfqpoint{1.994181in}{1.749235in}}%
\pgfpathlineto{\pgfqpoint{1.994181in}{1.749235in}}%
\pgfpathlineto{\pgfqpoint{1.977364in}{1.764779in}}%
\pgfpathlineto{\pgfqpoint{1.971325in}{1.786940in}}%
\pgfpathlineto{\pgfqpoint{1.974272in}{1.807443in}}%
\pgfpathlineto{\pgfqpoint{1.985023in}{1.831733in}}%
\pgfusepath{stroke}%
\end{pgfscope}%
\begin{pgfscope}%
\pgfpathrectangle{\pgfqpoint{0.647939in}{0.492442in}}{\pgfqpoint{3.079299in}{3.079299in}}%
\pgfusepath{clip}%
\pgfsetbuttcap%
\pgfsetroundjoin%
\pgfsetlinewidth{0.301125pt}%
\definecolor{currentstroke}{rgb}{0.500000,0.500000,0.500000}%
\pgfsetstrokecolor{currentstroke}%
\pgfsetstrokeopacity{0.300000}%
\pgfsetdash{}{0pt}%
\pgfpathmoveto{\pgfqpoint{3.727238in}{1.192283in}}%
\pgfpathlineto{\pgfqpoint{3.727238in}{1.192283in}}%
\pgfpathlineto{\pgfqpoint{3.664487in}{1.219548in}}%
\pgfpathlineto{\pgfqpoint{3.602634in}{1.248799in}}%
\pgfpathlineto{\pgfqpoint{3.541638in}{1.279800in}}%
\pgfpathlineto{\pgfqpoint{3.481430in}{1.312308in}}%
\pgfpathlineto{\pgfqpoint{3.421923in}{1.346086in}}%
\pgfpathlineto{\pgfqpoint{3.363016in}{1.380899in}}%
\pgfpathlineto{\pgfqpoint{3.304593in}{1.416522in}}%
\pgfpathlineto{\pgfqpoint{3.246529in}{1.452729in}}%
\pgfpathlineto{\pgfqpoint{3.188692in}{1.489298in}}%
\pgfpathlineto{\pgfqpoint{3.130941in}{1.526002in}}%
\pgfpathlineto{\pgfqpoint{3.073127in}{1.562604in}}%
\pgfpathlineto{\pgfqpoint{3.015090in}{1.598852in}}%
\pgfpathlineto{\pgfqpoint{2.956666in}{1.634469in}}%
\pgfpathlineto{\pgfqpoint{2.897685in}{1.669151in}}%
\pgfpathlineto{\pgfqpoint{2.837976in}{1.702556in}}%
\pgfpathlineto{\pgfqpoint{2.777366in}{1.734291in}}%
\pgfpathlineto{\pgfqpoint{2.715697in}{1.763903in}}%
\pgfpathlineto{\pgfqpoint{2.652837in}{1.790873in}}%
\pgfpathlineto{\pgfqpoint{2.588702in}{1.814630in}}%
\pgfpathlineto{\pgfqpoint{2.523288in}{1.834568in}}%
\pgfpathlineto{\pgfqpoint{2.456700in}{1.850116in}}%
\pgfpathlineto{\pgfqpoint{2.389165in}{1.860835in}}%
\pgfpathlineto{\pgfqpoint{2.321027in}{1.866626in}}%
\pgfpathlineto{\pgfqpoint{2.252646in}{1.867896in}}%
\pgfpathlineto{\pgfqpoint{2.184263in}{1.865837in}}%
\pgfpathlineto{\pgfqpoint{2.115883in}{1.863627in}}%
\pgfpathlineto{\pgfqpoint{2.115883in}{1.863627in}}%
\pgfusepath{stroke}%
\end{pgfscope}%
\begin{pgfscope}%
\pgfpathrectangle{\pgfqpoint{0.647939in}{0.492442in}}{\pgfqpoint{3.079299in}{3.079299in}}%
\pgfusepath{clip}%
\pgfsetbuttcap%
\pgfsetroundjoin%
\pgfsetlinewidth{0.301125pt}%
\definecolor{currentstroke}{rgb}{0.500000,0.500000,0.500000}%
\pgfsetstrokecolor{currentstroke}%
\pgfsetstrokeopacity{0.300000}%
\pgfsetdash{}{0pt}%
\pgfpathmoveto{\pgfqpoint{3.727238in}{1.262267in}}%
\pgfpathlineto{\pgfqpoint{3.727238in}{1.262267in}}%
\pgfpathlineto{\pgfqpoint{3.664984in}{1.290647in}}%
\pgfpathlineto{\pgfqpoint{3.603715in}{1.321099in}}%
\pgfpathlineto{\pgfqpoint{3.543391in}{1.353386in}}%
\pgfpathlineto{\pgfqpoint{3.483948in}{1.387272in}}%
\pgfpathlineto{\pgfqpoint{3.425306in}{1.422527in}}%
\pgfpathlineto{\pgfqpoint{3.367368in}{1.458932in}}%
\pgfpathlineto{\pgfqpoint{3.310030in}{1.496276in}}%
\pgfpathlineto{\pgfqpoint{3.253180in}{1.534357in}}%
\pgfpathlineto{\pgfqpoint{3.196696in}{1.572980in}}%
\pgfpathlineto{\pgfqpoint{3.140449in}{1.611951in}}%
\pgfpathlineto{\pgfqpoint{3.084305in}{1.651068in}}%
\pgfpathlineto{\pgfqpoint{3.028128in}{1.690134in}}%
\pgfpathlineto{\pgfqpoint{2.971773in}{1.728942in}}%
\pgfpathlineto{\pgfqpoint{2.915089in}{1.767269in}}%
\pgfpathlineto{\pgfqpoint{2.857919in}{1.804866in}}%
\pgfpathlineto{\pgfqpoint{2.800101in}{1.841449in}}%
\pgfpathlineto{\pgfqpoint{2.741466in}{1.876697in}}%
\pgfpathlineto{\pgfqpoint{2.681842in}{1.910232in}}%
\pgfpathlineto{\pgfqpoint{2.621061in}{1.941605in}}%
\pgfpathlineto{\pgfqpoint{2.558967in}{1.970270in}}%
\pgfpathlineto{\pgfqpoint{2.495440in}{1.995566in}}%
\pgfpathlineto{\pgfqpoint{2.430412in}{2.016669in}}%
\pgfpathlineto{\pgfqpoint{2.363922in}{2.032485in}}%
\pgfpathlineto{\pgfqpoint{2.296184in}{2.041227in}}%
\pgfpathlineto{\pgfqpoint{2.228793in}{2.037400in}}%
\pgfpathlineto{\pgfqpoint{2.228793in}{2.037400in}}%
\pgfpathlineto{\pgfqpoint{2.205871in}{2.031965in}}%
\pgfpathlineto{\pgfqpoint{2.205871in}{2.031965in}}%
\pgfpathlineto{\pgfqpoint{2.188673in}{2.024186in}}%
\pgfpathlineto{\pgfqpoint{2.172077in}{2.011788in}}%
\pgfpathlineto{\pgfqpoint{2.172077in}{2.011788in}}%
\pgfpathlineto{\pgfqpoint{2.158912in}{2.001253in}}%
\pgfusepath{stroke}%
\end{pgfscope}%
\begin{pgfscope}%
\pgfpathrectangle{\pgfqpoint{0.647939in}{0.492442in}}{\pgfqpoint{3.079299in}{3.079299in}}%
\pgfusepath{clip}%
\pgfsetbuttcap%
\pgfsetroundjoin%
\pgfsetlinewidth{0.301125pt}%
\definecolor{currentstroke}{rgb}{0.500000,0.500000,0.500000}%
\pgfsetstrokecolor{currentstroke}%
\pgfsetstrokeopacity{0.300000}%
\pgfsetdash{}{0pt}%
\pgfpathmoveto{\pgfqpoint{3.727238in}{1.402235in}}%
\pgfpathlineto{\pgfqpoint{3.727238in}{1.402235in}}%
\pgfpathlineto{\pgfqpoint{3.666185in}{1.433108in}}%
\pgfpathlineto{\pgfqpoint{3.606326in}{1.466240in}}%
\pgfpathlineto{\pgfqpoint{3.547633in}{1.501401in}}%
\pgfpathlineto{\pgfqpoint{3.490060in}{1.538368in}}%
\pgfpathlineto{\pgfqpoint{3.433546in}{1.576938in}}%
\pgfpathlineto{\pgfqpoint{3.378023in}{1.616926in}}%
\pgfpathlineto{\pgfqpoint{3.323424in}{1.658166in}}%
\pgfpathlineto{\pgfqpoint{3.269674in}{1.700507in}}%
\pgfpathlineto{\pgfqpoint{3.216703in}{1.743820in}}%
\pgfpathlineto{\pgfqpoint{3.164461in}{1.788009in}}%
\pgfpathlineto{\pgfqpoint{3.112906in}{1.832998in}}%
\pgfpathlineto{\pgfqpoint{3.062015in}{1.878735in}}%
\pgfpathlineto{\pgfqpoint{3.011804in}{1.925218in}}%
\pgfpathlineto{\pgfqpoint{2.962332in}{1.972484in}}%
\pgfpathlineto{\pgfqpoint{2.913731in}{2.020641in}}%
\pgfpathlineto{\pgfqpoint{2.866260in}{2.069906in}}%
\pgfpathlineto{\pgfqpoint{2.820389in}{2.120649in}}%
\pgfpathlineto{\pgfqpoint{2.776967in}{2.173490in}}%
\pgfpathlineto{\pgfqpoint{2.737600in}{2.229361in}}%
\pgfpathlineto{\pgfqpoint{2.705293in}{2.289437in}}%
\pgfpathlineto{\pgfqpoint{2.684985in}{2.354266in}}%
\pgfpathlineto{\pgfqpoint{2.680906in}{2.417510in}}%
\pgfpathlineto{\pgfqpoint{2.689453in}{2.474373in}}%
\pgfpathlineto{\pgfqpoint{2.708200in}{2.532921in}}%
\pgfpathlineto{\pgfqpoint{2.736368in}{2.595083in}}%
\pgfpathlineto{\pgfqpoint{2.769211in}{2.654998in}}%
\pgfpathlineto{\pgfqpoint{2.804934in}{2.713276in}}%
\pgfpathlineto{\pgfqpoint{2.842551in}{2.770370in}}%
\pgfpathlineto{\pgfqpoint{2.881514in}{2.826583in}}%
\pgfpathlineto{\pgfqpoint{2.921502in}{2.882092in}}%
\pgfpathlineto{\pgfqpoint{2.962329in}{2.936982in}}%
\pgfpathlineto{\pgfqpoint{3.003898in}{2.991310in}}%
\pgfpathlineto{\pgfqpoint{3.046167in}{3.045108in}}%
\pgfpathlineto{\pgfqpoint{3.089126in}{3.098359in}}%
\pgfpathlineto{\pgfqpoint{3.132786in}{3.151036in}}%
\pgfpathlineto{\pgfqpoint{3.177181in}{3.203098in}}%
\pgfpathlineto{\pgfqpoint{3.222360in}{3.254482in}}%
\pgfpathlineto{\pgfqpoint{3.268377in}{3.305117in}}%
\pgfpathlineto{\pgfqpoint{3.315304in}{3.354911in}}%
\pgfpathlineto{\pgfqpoint{3.363217in}{3.403758in}}%
\pgfpathlineto{\pgfqpoint{3.412196in}{3.451534in}}%
\pgfpathlineto{\pgfqpoint{3.462333in}{3.498093in}}%
\pgfpathlineto{\pgfqpoint{3.513720in}{3.543266in}}%
\pgfpathlineto{\pgfqpoint{3.547051in}{3.571741in}}%
\pgfusepath{stroke}%
\end{pgfscope}%
\begin{pgfscope}%
\pgfpathrectangle{\pgfqpoint{0.647939in}{0.492442in}}{\pgfqpoint{3.079299in}{3.079299in}}%
\pgfusepath{clip}%
\pgfsetbuttcap%
\pgfsetroundjoin%
\pgfsetlinewidth{0.301125pt}%
\definecolor{currentstroke}{rgb}{0.500000,0.500000,0.500000}%
\pgfsetstrokecolor{currentstroke}%
\pgfsetstrokeopacity{0.300000}%
\pgfsetdash{}{0pt}%
\pgfpathmoveto{\pgfqpoint{3.727238in}{1.542203in}}%
\pgfpathlineto{\pgfqpoint{3.727238in}{1.542203in}}%
\pgfpathlineto{\pgfqpoint{3.667746in}{1.575980in}}%
\pgfpathlineto{\pgfqpoint{3.609724in}{1.612227in}}%
\pgfpathlineto{\pgfqpoint{3.553168in}{1.650725in}}%
\pgfpathlineto{\pgfqpoint{3.498065in}{1.691275in}}%
\pgfpathlineto{\pgfqpoint{3.444393in}{1.733706in}}%
\pgfpathlineto{\pgfqpoint{3.392138in}{1.777870in}}%
\pgfpathlineto{\pgfqpoint{3.341291in}{1.823649in}}%
\pgfpathlineto{\pgfqpoint{3.291875in}{1.870969in}}%
\pgfpathlineto{\pgfqpoint{3.243941in}{1.919790in}}%
\pgfpathlineto{\pgfqpoint{3.197588in}{1.970110in}}%
\pgfpathlineto{\pgfqpoint{3.152982in}{2.021983in}}%
\pgfpathlineto{\pgfqpoint{3.110379in}{2.075511in}}%
\pgfpathlineto{\pgfqpoint{3.070176in}{2.130854in}}%
\pgfpathlineto{\pgfqpoint{3.032945in}{2.188224in}}%
\pgfpathlineto{\pgfqpoint{2.999507in}{2.247862in}}%
\pgfpathlineto{\pgfqpoint{2.970985in}{2.309972in}}%
\pgfpathlineto{\pgfqpoint{2.948793in}{2.374574in}}%
\pgfpathlineto{\pgfqpoint{2.934418in}{2.441306in}}%
\pgfpathlineto{\pgfqpoint{2.928949in}{2.509330in}}%
\pgfpathlineto{\pgfqpoint{2.932570in}{2.577492in}}%
\pgfpathlineto{\pgfqpoint{2.944466in}{2.644736in}}%
\pgfpathlineto{\pgfqpoint{2.963241in}{2.710426in}}%
\pgfpathlineto{\pgfqpoint{2.987446in}{2.774346in}}%
\pgfpathlineto{\pgfqpoint{3.015856in}{2.836540in}}%
\pgfpathlineto{\pgfqpoint{3.047543in}{2.897148in}}%
\pgfpathlineto{\pgfqpoint{3.081842in}{2.956328in}}%
\pgfpathlineto{\pgfqpoint{3.118287in}{3.014219in}}%
\pgfpathlineto{\pgfqpoint{3.156557in}{3.070928in}}%
\pgfusepath{stroke}%
\end{pgfscope}%
\begin{pgfscope}%
\pgfpathrectangle{\pgfqpoint{0.647939in}{0.492442in}}{\pgfqpoint{3.079299in}{3.079299in}}%
\pgfusepath{clip}%
\pgfsetbuttcap%
\pgfsetroundjoin%
\pgfsetlinewidth{0.301125pt}%
\definecolor{currentstroke}{rgb}{0.500000,0.500000,0.500000}%
\pgfsetstrokecolor{currentstroke}%
\pgfsetstrokeopacity{0.300000}%
\pgfsetdash{}{0pt}%
\pgfpathmoveto{\pgfqpoint{3.727238in}{1.682171in}}%
\pgfpathlineto{\pgfqpoint{3.727238in}{1.682171in}}%
\pgfpathlineto{\pgfqpoint{3.669812in}{1.719343in}}%
\pgfpathlineto{\pgfqpoint{3.614229in}{1.759217in}}%
\pgfpathlineto{\pgfqpoint{3.560524in}{1.801592in}}%
\pgfpathlineto{\pgfqpoint{3.508738in}{1.846294in}}%
\pgfpathlineto{\pgfqpoint{3.458918in}{1.893178in}}%
\pgfpathlineto{\pgfqpoint{3.411132in}{1.942134in}}%
\pgfpathlineto{\pgfqpoint{3.365497in}{1.993098in}}%
\pgfpathlineto{\pgfqpoint{3.322180in}{2.046045in}}%
\pgfpathlineto{\pgfqpoint{3.281424in}{2.100984in}}%
\pgfpathlineto{\pgfqpoint{3.243562in}{2.157946in}}%
\pgfpathlineto{\pgfqpoint{3.209050in}{2.216989in}}%
\pgfpathlineto{\pgfqpoint{3.178476in}{2.278151in}}%
\pgfpathlineto{\pgfqpoint{3.152560in}{2.341412in}}%
\pgfpathlineto{\pgfqpoint{3.132111in}{2.406627in}}%
\pgfpathlineto{\pgfqpoint{3.117923in}{2.473468in}}%
\pgfpathlineto{\pgfqpoint{3.110595in}{2.541396in}}%
\pgfpathlineto{\pgfqpoint{3.110373in}{2.609723in}}%
\pgfpathlineto{\pgfqpoint{3.117057in}{2.677726in}}%
\pgfpathlineto{\pgfqpoint{3.130070in}{2.744815in}}%
\pgfpathlineto{\pgfqpoint{3.148632in}{2.810602in}}%
\pgfpathlineto{\pgfqpoint{3.171929in}{2.874882in}}%
\pgfpathlineto{\pgfqpoint{3.199225in}{2.937584in}}%
\pgfpathlineto{\pgfqpoint{3.229911in}{2.998709in}}%
\pgfpathlineto{\pgfqpoint{3.263509in}{3.058289in}}%
\pgfpathlineto{\pgfqpoint{3.299664in}{3.116358in}}%
\pgfpathlineto{\pgfqpoint{3.338116in}{3.172939in}}%
\pgfpathlineto{\pgfqpoint{3.378677in}{3.228032in}}%
\pgfpathlineto{\pgfqpoint{3.421224in}{3.281607in}}%
\pgfpathlineto{\pgfqpoint{3.465692in}{3.333596in}}%
\pgfpathlineto{\pgfqpoint{3.512048in}{3.383912in}}%
\pgfpathlineto{\pgfqpoint{3.560291in}{3.432419in}}%
\pgfpathlineto{\pgfqpoint{3.610441in}{3.478950in}}%
\pgfpathlineto{\pgfqpoint{3.662524in}{3.523305in}}%
\pgfpathlineto{\pgfqpoint{3.716570in}{3.565239in}}%
\pgfpathlineto{\pgfqpoint{3.725370in}{3.571741in}}%
\pgfusepath{stroke}%
\end{pgfscope}%
\begin{pgfscope}%
\pgfpathrectangle{\pgfqpoint{0.647939in}{0.492442in}}{\pgfqpoint{3.079299in}{3.079299in}}%
\pgfusepath{clip}%
\pgfsetbuttcap%
\pgfsetroundjoin%
\pgfsetlinewidth{0.301125pt}%
\definecolor{currentstroke}{rgb}{0.500000,0.500000,0.500000}%
\pgfsetstrokecolor{currentstroke}%
\pgfsetstrokeopacity{0.300000}%
\pgfsetdash{}{0pt}%
\pgfpathmoveto{\pgfqpoint{3.727238in}{1.822139in}}%
\pgfpathlineto{\pgfqpoint{3.727238in}{1.822139in}}%
\pgfpathlineto{\pgfqpoint{3.672595in}{1.863283in}}%
\pgfpathlineto{\pgfqpoint{3.620300in}{1.907371in}}%
\pgfpathlineto{\pgfqpoint{3.570456in}{1.954215in}}%
\pgfpathlineto{\pgfqpoint{3.523182in}{2.003651in}}%
\pgfpathlineto{\pgfqpoint{3.478629in}{2.055551in}}%
\pgfpathlineto{\pgfqpoint{3.436999in}{2.109824in}}%
\pgfpathlineto{\pgfqpoint{3.398561in}{2.166398in}}%
\pgfpathlineto{\pgfqpoint{3.363668in}{2.225216in}}%
\pgfpathlineto{\pgfqpoint{3.332757in}{2.286211in}}%
\pgfpathlineto{\pgfqpoint{3.306342in}{2.349274in}}%
\pgfpathlineto{\pgfqpoint{3.284989in}{2.414216in}}%
\pgfpathlineto{\pgfqpoint{3.269257in}{2.480736in}}%
\pgfpathlineto{\pgfqpoint{3.259600in}{2.548402in}}%
\pgfpathlineto{\pgfqpoint{3.256280in}{2.616664in}}%
\pgfpathlineto{\pgfqpoint{3.259292in}{2.684939in}}%
\pgfpathlineto{\pgfqpoint{3.268371in}{2.752681in}}%
\pgfpathlineto{\pgfqpoint{3.283062in}{2.819444in}}%
\pgfpathlineto{\pgfqpoint{3.302800in}{2.884909in}}%
\pgfpathlineto{\pgfqpoint{3.327009in}{2.948868in}}%
\pgfpathlineto{\pgfqpoint{3.355164in}{3.011201in}}%
\pgfpathlineto{\pgfqpoint{3.386818in}{3.071837in}}%
\pgfpathlineto{\pgfqpoint{3.421619in}{3.130725in}}%
\pgfpathlineto{\pgfqpoint{3.459292in}{3.187819in}}%
\pgfpathlineto{\pgfqpoint{3.499635in}{3.243059in}}%
\pgfpathlineto{\pgfqpoint{3.542503in}{3.296367in}}%
\pgfpathlineto{\pgfqpoint{3.587812in}{3.347616in}}%
\pgfpathlineto{\pgfqpoint{3.635508in}{3.396650in}}%
\pgfpathlineto{\pgfqpoint{3.685564in}{3.443270in}}%
\pgfpathlineto{\pgfqpoint{3.727238in}{3.480173in}}%
\pgfusepath{stroke}%
\end{pgfscope}%
\begin{pgfscope}%
\pgfpathrectangle{\pgfqpoint{0.647939in}{0.492442in}}{\pgfqpoint{3.079299in}{3.079299in}}%
\pgfusepath{clip}%
\pgfsetbuttcap%
\pgfsetroundjoin%
\pgfsetlinewidth{0.301125pt}%
\definecolor{currentstroke}{rgb}{0.500000,0.500000,0.500000}%
\pgfsetstrokecolor{currentstroke}%
\pgfsetstrokeopacity{0.300000}%
\pgfsetdash{}{0pt}%
\pgfpathmoveto{\pgfqpoint{3.727238in}{1.892124in}}%
\pgfpathlineto{\pgfqpoint{3.727238in}{1.892124in}}%
\pgfpathlineto{\pgfqpoint{3.674349in}{1.935491in}}%
\pgfpathlineto{\pgfqpoint{3.624125in}{1.981917in}}%
\pgfpathlineto{\pgfqpoint{3.576713in}{2.031212in}}%
\pgfpathlineto{\pgfqpoint{3.532281in}{2.083209in}}%
\pgfpathlineto{\pgfqpoint{3.491045in}{2.137774in}}%
\pgfpathlineto{\pgfqpoint{3.453286in}{2.194795in}}%
\pgfpathlineto{\pgfqpoint{3.419352in}{2.254165in}}%
\pgfpathlineto{\pgfqpoint{3.389659in}{2.315757in}}%
\pgfpathlineto{\pgfqpoint{3.364685in}{2.379400in}}%
\pgfpathlineto{\pgfqpoint{3.344931in}{2.444843in}}%
\pgfpathlineto{\pgfqpoint{3.330868in}{2.511736in}}%
\pgfpathlineto{\pgfqpoint{3.322850in}{2.579617in}}%
\pgfpathlineto{\pgfqpoint{3.321045in}{2.647941in}}%
\pgfpathlineto{\pgfqpoint{3.325389in}{2.716149in}}%
\pgfusepath{stroke}%
\end{pgfscope}%
\begin{pgfscope}%
\pgfpathrectangle{\pgfqpoint{0.647939in}{0.492442in}}{\pgfqpoint{3.079299in}{3.079299in}}%
\pgfusepath{clip}%
\pgfsetbuttcap%
\pgfsetroundjoin%
\pgfsetlinewidth{0.301125pt}%
\definecolor{currentstroke}{rgb}{0.500000,0.500000,0.500000}%
\pgfsetstrokecolor{currentstroke}%
\pgfsetstrokeopacity{0.300000}%
\pgfsetdash{}{0pt}%
\pgfpathmoveto{\pgfqpoint{3.727238in}{2.032092in}}%
\pgfpathlineto{\pgfqpoint{3.727238in}{2.032092in}}%
\pgfpathlineto{\pgfqpoint{3.678832in}{2.080395in}}%
\pgfpathlineto{\pgfqpoint{3.633887in}{2.131934in}}%
\pgfpathlineto{\pgfqpoint{3.592652in}{2.186482in}}%
\pgfpathlineto{\pgfqpoint{3.555410in}{2.243826in}}%
\pgfpathlineto{\pgfqpoint{3.522506in}{2.303761in}}%
\pgfpathlineto{\pgfqpoint{3.494329in}{2.366053in}}%
\pgfpathlineto{\pgfqpoint{3.471298in}{2.430417in}}%
\pgfpathlineto{\pgfqpoint{3.453821in}{2.496498in}}%
\pgfpathlineto{\pgfqpoint{3.442242in}{2.563856in}}%
\pgfpathlineto{\pgfqpoint{3.436780in}{2.631983in}}%
\pgfpathlineto{\pgfqpoint{3.437490in}{2.700332in}}%
\pgfpathlineto{\pgfqpoint{3.444244in}{2.768355in}}%
\pgfpathlineto{\pgfqpoint{3.456759in}{2.835558in}}%
\pgfpathlineto{\pgfqpoint{3.474651in}{2.901538in}}%
\pgfpathlineto{\pgfqpoint{3.497493in}{2.965979in}}%
\pgfpathlineto{\pgfqpoint{3.524869in}{3.028636in}}%
\pgfpathlineto{\pgfqpoint{3.556401in}{3.089313in}}%
\pgfpathlineto{\pgfqpoint{3.591771in}{3.147841in}}%
\pgfpathlineto{\pgfqpoint{3.630725in}{3.204052in}}%
\pgfpathlineto{\pgfqpoint{3.673070in}{3.257759in}}%
\pgfpathlineto{\pgfqpoint{3.718662in}{3.308735in}}%
\pgfpathlineto{\pgfqpoint{3.727238in}{3.317747in}}%
\pgfusepath{stroke}%
\end{pgfscope}%
\begin{pgfscope}%
\pgfpathrectangle{\pgfqpoint{0.647939in}{0.492442in}}{\pgfqpoint{3.079299in}{3.079299in}}%
\pgfusepath{clip}%
\pgfsetbuttcap%
\pgfsetroundjoin%
\pgfsetlinewidth{0.301125pt}%
\definecolor{currentstroke}{rgb}{0.500000,0.500000,0.500000}%
\pgfsetstrokecolor{currentstroke}%
\pgfsetstrokeopacity{0.300000}%
\pgfsetdash{}{0pt}%
\pgfpathmoveto{\pgfqpoint{3.727238in}{2.172060in}}%
\pgfpathlineto{\pgfqpoint{3.727238in}{2.172060in}}%
\pgfpathlineto{\pgfqpoint{3.685051in}{2.225861in}}%
\pgfpathlineto{\pgfqpoint{3.647350in}{2.282888in}}%
\pgfpathlineto{\pgfqpoint{3.614484in}{2.342825in}}%
\pgfpathlineto{\pgfqpoint{3.586824in}{2.405331in}}%
\pgfpathlineto{\pgfqpoint{3.564751in}{2.470016in}}%
\pgfpathlineto{\pgfqpoint{3.548614in}{2.536428in}}%
\pgfpathlineto{\pgfqpoint{3.538682in}{2.604047in}}%
\pgfpathlineto{\pgfqpoint{3.535101in}{2.672301in}}%
\pgfpathlineto{\pgfqpoint{3.537858in}{2.740592in}}%
\pgfpathlineto{\pgfqpoint{3.546788in}{2.808352in}}%
\pgfpathlineto{\pgfqpoint{3.561605in}{2.875078in}}%
\pgfpathlineto{\pgfqpoint{3.581945in}{2.940341in}}%
\pgfpathlineto{\pgfqpoint{3.607419in}{3.003784in}}%
\pgfpathlineto{\pgfqpoint{3.637658in}{3.065104in}}%
\pgfpathlineto{\pgfqpoint{3.672338in}{3.124030in}}%
\pgfpathlineto{\pgfqpoint{3.711188in}{3.180295in}}%
\pgfpathlineto{\pgfqpoint{3.727238in}{3.201854in}}%
\pgfusepath{stroke}%
\end{pgfscope}%
\begin{pgfscope}%
\pgfpathrectangle{\pgfqpoint{0.647939in}{0.492442in}}{\pgfqpoint{3.079299in}{3.079299in}}%
\pgfusepath{clip}%
\pgfsetbuttcap%
\pgfsetroundjoin%
\pgfsetlinewidth{0.301125pt}%
\definecolor{currentstroke}{rgb}{0.500000,0.500000,0.500000}%
\pgfsetstrokecolor{currentstroke}%
\pgfsetstrokeopacity{0.300000}%
\pgfsetdash{}{0pt}%
\pgfpathmoveto{\pgfqpoint{3.727238in}{2.312028in}}%
\pgfpathlineto{\pgfqpoint{3.727238in}{2.312028in}}%
\pgfpathlineto{\pgfqpoint{3.693595in}{2.371519in}}%
\pgfpathlineto{\pgfqpoint{3.665632in}{2.433876in}}%
\pgfpathlineto{\pgfqpoint{3.643716in}{2.498602in}}%
\pgfpathlineto{\pgfqpoint{3.628164in}{2.565140in}}%
\pgfpathlineto{\pgfqpoint{3.619208in}{2.632883in}}%
\pgfpathlineto{\pgfqpoint{3.616948in}{2.701185in}}%
\pgfpathlineto{\pgfqpoint{3.621336in}{2.769386in}}%
\pgfpathlineto{\pgfqpoint{3.632178in}{2.836860in}}%
\pgfpathlineto{\pgfqpoint{3.649180in}{2.903054in}}%
\pgfpathlineto{\pgfqpoint{3.671987in}{2.967485in}}%
\pgfpathlineto{\pgfqpoint{3.700229in}{3.029732in}}%
\pgfpathlineto{\pgfqpoint{3.727238in}{3.083199in}}%
\pgfusepath{stroke}%
\end{pgfscope}%
\begin{pgfscope}%
\pgfpathrectangle{\pgfqpoint{0.647939in}{0.492442in}}{\pgfqpoint{3.079299in}{3.079299in}}%
\pgfusepath{clip}%
\pgfsetbuttcap%
\pgfsetroundjoin%
\pgfsetlinewidth{0.301125pt}%
\definecolor{currentstroke}{rgb}{0.500000,0.500000,0.500000}%
\pgfsetstrokecolor{currentstroke}%
\pgfsetstrokeopacity{0.300000}%
\pgfsetdash{}{0pt}%
\pgfpathmoveto{\pgfqpoint{3.727238in}{2.521980in}}%
\pgfpathlineto{\pgfqpoint{3.727238in}{2.521980in}}%
\pgfpathlineto{\pgfqpoint{3.711523in}{2.588463in}}%
\pgfpathlineto{\pgfqpoint{3.702980in}{2.656239in}}%
\pgfpathlineto{\pgfqpoint{3.701682in}{2.724541in}}%
\pgfpathlineto{\pgfqpoint{3.707555in}{2.792615in}}%
\pgfpathlineto{\pgfqpoint{3.720386in}{2.859733in}}%
\pgfpathlineto{\pgfqpoint{3.727238in}{2.887825in}}%
\pgfusepath{stroke}%
\end{pgfscope}%
\begin{pgfscope}%
\pgfpathrectangle{\pgfqpoint{0.647939in}{0.492442in}}{\pgfqpoint{3.079299in}{3.079299in}}%
\pgfusepath{clip}%
\pgfsetbuttcap%
\pgfsetroundjoin%
\pgfsetlinewidth{0.301125pt}%
\definecolor{currentstroke}{rgb}{0.500000,0.500000,0.500000}%
\pgfsetstrokecolor{currentstroke}%
\pgfsetstrokeopacity{0.300000}%
\pgfsetdash{}{0pt}%
\pgfpathmoveto{\pgfqpoint{0.647939in}{2.560831in}}%
\pgfpathlineto{\pgfqpoint{0.657868in}{2.562006in}}%
\pgfpathlineto{\pgfqpoint{0.725733in}{2.570735in}}%
\pgfpathlineto{\pgfqpoint{0.793405in}{2.580854in}}%
\pgfpathlineto{\pgfqpoint{0.860853in}{2.592375in}}%
\pgfpathlineto{\pgfqpoint{0.928052in}{2.605269in}}%
\pgfpathlineto{\pgfqpoint{0.994992in}{2.619454in}}%
\pgfpathlineto{\pgfqpoint{1.061673in}{2.634807in}}%
\pgfpathlineto{\pgfqpoint{1.128117in}{2.651162in}}%
\pgfpathlineto{\pgfqpoint{1.194361in}{2.668311in}}%
\pgfpathlineto{\pgfqpoint{1.260461in}{2.686007in}}%
\pgfpathlineto{\pgfqpoint{1.326491in}{2.703969in}}%
\pgfpathlineto{\pgfqpoint{1.392532in}{2.721885in}}%
\pgfpathlineto{\pgfqpoint{1.458677in}{2.739416in}}%
\pgfpathlineto{\pgfqpoint{1.525012in}{2.756205in}}%
\pgfpathlineto{\pgfqpoint{1.591615in}{2.771892in}}%
\pgfpathlineto{\pgfqpoint{1.658541in}{2.786124in}}%
\pgfpathlineto{\pgfqpoint{1.725819in}{2.798576in}}%
\pgfpathlineto{\pgfqpoint{1.793443in}{2.808976in}}%
\pgfpathlineto{\pgfqpoint{1.861373in}{2.817131in}}%
\pgfpathlineto{\pgfqpoint{1.929542in}{2.822976in}}%
\pgfpathlineto{\pgfqpoint{1.997865in}{2.826605in}}%
\pgfpathlineto{\pgfqpoint{2.066266in}{2.828283in}}%
\pgfpathlineto{\pgfqpoint{2.134691in}{2.828489in}}%
\pgfpathlineto{\pgfqpoint{2.203118in}{2.827932in}}%
\pgfpathlineto{\pgfqpoint{2.271544in}{2.827604in}}%
\pgfpathlineto{\pgfqpoint{2.339953in}{2.828771in}}%
\pgfpathlineto{\pgfqpoint{2.408223in}{2.832927in}}%
\pgfpathlineto{\pgfqpoint{2.476029in}{2.841672in}}%
\pgfpathlineto{\pgfqpoint{2.542767in}{2.856390in}}%
\pgfpathlineto{\pgfqpoint{2.607660in}{2.877811in}}%
\pgfpathlineto{\pgfqpoint{2.670035in}{2.905746in}}%
\pgfpathlineto{\pgfqpoint{2.729613in}{2.939270in}}%
\pgfpathlineto{\pgfqpoint{2.786526in}{2.977174in}}%
\pgfpathlineto{\pgfqpoint{2.841150in}{3.018322in}}%
\pgfpathlineto{\pgfqpoint{2.893952in}{3.061800in}}%
\pgfpathlineto{\pgfqpoint{2.945379in}{3.106916in}}%
\pgfpathlineto{\pgfqpoint{2.995820in}{3.153138in}}%
\pgfpathlineto{\pgfqpoint{3.045605in}{3.200070in}}%
\pgfpathlineto{\pgfqpoint{3.095017in}{3.247400in}}%
\pgfpathlineto{\pgfqpoint{3.144293in}{3.294872in}}%
\pgfpathlineto{\pgfqpoint{3.193637in}{3.342274in}}%
\pgfpathlineto{\pgfqpoint{3.243236in}{3.389411in}}%
\pgfpathlineto{\pgfqpoint{3.293251in}{3.436106in}}%
\pgfpathlineto{\pgfqpoint{3.343838in}{3.482181in}}%
\pgfpathlineto{\pgfqpoint{3.395145in}{3.527452in}}%
\pgfpathlineto{\pgfqpoint{3.447302in}{3.571741in}}%
\pgfpathlineto{\pgfqpoint{3.447302in}{3.571741in}}%
\pgfusepath{stroke}%
\end{pgfscope}%
\begin{pgfscope}%
\pgfpathrectangle{\pgfqpoint{0.647939in}{0.492442in}}{\pgfqpoint{3.079299in}{3.079299in}}%
\pgfusepath{clip}%
\pgfsetbuttcap%
\pgfsetroundjoin%
\pgfsetlinewidth{0.301125pt}%
\definecolor{currentstroke}{rgb}{0.500000,0.500000,0.500000}%
\pgfsetstrokecolor{currentstroke}%
\pgfsetstrokeopacity{0.300000}%
\pgfsetdash{}{0pt}%
\pgfpathmoveto{\pgfqpoint{0.647939in}{2.856731in}}%
\pgfpathlineto{\pgfqpoint{0.663063in}{2.858449in}}%
\pgfpathlineto{\pgfqpoint{0.730976in}{2.866805in}}%
\pgfpathlineto{\pgfqpoint{0.798720in}{2.876435in}}%
\pgfpathlineto{\pgfqpoint{0.866273in}{2.887334in}}%
\pgfpathlineto{\pgfqpoint{0.933617in}{2.899451in}}%
\pgfpathlineto{\pgfqpoint{1.000750in}{2.912692in}}%
\pgfpathlineto{\pgfqpoint{1.067680in}{2.926927in}}%
\pgfpathlineto{\pgfqpoint{1.134431in}{2.941983in}}%
\pgfpathlineto{\pgfqpoint{1.201041in}{2.957651in}}%
\pgfpathlineto{\pgfqpoint{1.267565in}{2.973686in}}%
\pgfpathlineto{\pgfqpoint{1.334066in}{2.989813in}}%
\pgfpathlineto{\pgfqpoint{1.400617in}{3.005732in}}%
\pgfpathlineto{\pgfqpoint{1.467290in}{3.021129in}}%
\pgfpathlineto{\pgfqpoint{1.534150in}{3.035686in}}%
\pgfpathlineto{\pgfqpoint{1.601250in}{3.049090in}}%
\pgfpathlineto{\pgfqpoint{1.668620in}{3.061051in}}%
\pgfpathlineto{\pgfqpoint{1.736268in}{3.071315in}}%
\pgfpathlineto{\pgfqpoint{1.804174in}{3.079698in}}%
\pgfpathlineto{\pgfqpoint{1.872294in}{3.086102in}}%
\pgfpathlineto{\pgfqpoint{1.940570in}{3.090542in}}%
\pgfpathlineto{\pgfqpoint{2.008942in}{3.093164in}}%
\pgfpathlineto{\pgfqpoint{2.077359in}{3.094256in}}%
\pgfpathlineto{\pgfqpoint{2.145786in}{3.094274in}}%
\pgfpathlineto{\pgfqpoint{2.214214in}{3.093838in}}%
\pgfpathlineto{\pgfqpoint{2.282641in}{3.093715in}}%
\pgfpathlineto{\pgfqpoint{2.351055in}{3.094812in}}%
\pgfpathlineto{\pgfqpoint{2.419388in}{3.098146in}}%
\pgfpathlineto{\pgfqpoint{2.487465in}{3.104781in}}%
\pgfpathlineto{\pgfqpoint{2.554973in}{3.115676in}}%
\pgfpathlineto{\pgfqpoint{2.621485in}{3.131495in}}%
\pgfpathlineto{\pgfqpoint{2.686551in}{3.152469in}}%
\pgfpathlineto{\pgfqpoint{2.749820in}{3.178375in}}%
\pgfpathlineto{\pgfqpoint{2.811134in}{3.208644in}}%
\pgfpathlineto{\pgfqpoint{2.870533in}{3.242540in}}%
\pgfpathlineto{\pgfqpoint{2.928207in}{3.279314in}}%
\pgfpathlineto{\pgfqpoint{2.984431in}{3.318290in}}%
\pgfpathlineto{\pgfqpoint{3.039503in}{3.358888in}}%
\pgfpathlineto{\pgfqpoint{3.093724in}{3.400621in}}%
\pgfpathlineto{\pgfqpoint{3.147373in}{3.443088in}}%
\pgfpathlineto{\pgfqpoint{3.200705in}{3.485957in}}%
\pgfpathlineto{\pgfqpoint{3.253952in}{3.528932in}}%
\pgfpathlineto{\pgfqpoint{3.307334in}{3.571741in}}%
\pgfpathlineto{\pgfqpoint{3.307334in}{3.571741in}}%
\pgfusepath{stroke}%
\end{pgfscope}%
\begin{pgfscope}%
\pgfpathrectangle{\pgfqpoint{0.647939in}{0.492442in}}{\pgfqpoint{3.079299in}{3.079299in}}%
\pgfusepath{clip}%
\pgfsetbuttcap%
\pgfsetroundjoin%
\pgfsetlinewidth{0.301125pt}%
\definecolor{currentstroke}{rgb}{0.500000,0.500000,0.500000}%
\pgfsetstrokecolor{currentstroke}%
\pgfsetstrokeopacity{0.300000}%
\pgfsetdash{}{0pt}%
\pgfpathmoveto{\pgfqpoint{0.647939in}{3.039391in}}%
\pgfpathlineto{\pgfqpoint{0.661004in}{3.040821in}}%
\pgfpathlineto{\pgfqpoint{0.728954in}{3.048870in}}%
\pgfpathlineto{\pgfqpoint{0.796751in}{3.058128in}}%
\pgfpathlineto{\pgfqpoint{0.864373in}{3.068582in}}%
\pgfpathlineto{\pgfqpoint{0.931810in}{3.080179in}}%
\pgfpathlineto{\pgfqpoint{0.999059in}{3.092822in}}%
\pgfpathlineto{\pgfqpoint{1.066130in}{3.106376in}}%
\pgfpathlineto{\pgfqpoint{1.133048in}{3.120672in}}%
\pgfpathlineto{\pgfqpoint{1.199850in}{3.135502in}}%
\pgfpathlineto{\pgfqpoint{1.266586in}{3.150628in}}%
\pgfpathlineto{\pgfqpoint{1.333315in}{3.165784in}}%
\pgfpathlineto{\pgfqpoint{1.400102in}{3.180684in}}%
\pgfpathlineto{\pgfqpoint{1.467009in}{3.195028in}}%
\pgfpathlineto{\pgfqpoint{1.534092in}{3.208522in}}%
\pgfpathlineto{\pgfqpoint{1.601393in}{3.220878in}}%
\pgfpathlineto{\pgfqpoint{1.668933in}{3.231836in}}%
\pgfpathlineto{\pgfqpoint{1.736716in}{3.241178in}}%
\pgfpathlineto{\pgfqpoint{1.804717in}{3.248753in}}%
\pgfpathlineto{\pgfqpoint{1.872898in}{3.254497in}}%
\pgfpathlineto{\pgfqpoint{1.941206in}{3.258447in}}%
\pgfpathlineto{\pgfqpoint{2.009590in}{3.260757in}}%
\pgfpathlineto{\pgfqpoint{2.078010in}{3.261707in}}%
\pgfpathlineto{\pgfqpoint{2.146438in}{3.261717in}}%
\pgfpathlineto{\pgfqpoint{2.214866in}{3.261341in}}%
\pgfpathlineto{\pgfqpoint{2.283294in}{3.261241in}}%
\pgfpathlineto{\pgfqpoint{2.351712in}{3.262181in}}%
\pgfpathlineto{\pgfqpoint{2.420072in}{3.265002in}}%
\pgfpathlineto{\pgfqpoint{2.488253in}{3.270572in}}%
\pgfpathlineto{\pgfqpoint{2.556039in}{3.279690in}}%
\pgfpathlineto{\pgfqpoint{2.623126in}{3.292953in}}%
\pgfpathlineto{\pgfqpoint{2.689170in}{3.310671in}}%
\pgfpathlineto{\pgfqpoint{2.753865in}{3.332821in}}%
\pgfpathlineto{\pgfqpoint{2.817011in}{3.359082in}}%
\pgfpathlineto{\pgfqpoint{2.878550in}{3.388935in}}%
\pgfpathlineto{\pgfqpoint{2.938557in}{3.421774in}}%
\pgfpathlineto{\pgfqpoint{2.997205in}{3.456995in}}%
\pgfpathlineto{\pgfqpoint{3.054721in}{3.494043in}}%
\pgfpathlineto{\pgfqpoint{3.111358in}{3.532433in}}%
\pgfpathlineto{\pgfqpoint{3.167366in}{3.571741in}}%
\pgfpathlineto{\pgfqpoint{3.167366in}{3.571741in}}%
\pgfusepath{stroke}%
\end{pgfscope}%
\begin{pgfscope}%
\pgfpathrectangle{\pgfqpoint{0.647939in}{0.492442in}}{\pgfqpoint{3.079299in}{3.079299in}}%
\pgfusepath{clip}%
\pgfsetbuttcap%
\pgfsetroundjoin%
\pgfsetlinewidth{0.301125pt}%
\definecolor{currentstroke}{rgb}{0.500000,0.500000,0.500000}%
\pgfsetstrokecolor{currentstroke}%
\pgfsetstrokeopacity{0.300000}%
\pgfsetdash{}{0pt}%
\pgfpathmoveto{\pgfqpoint{0.647939in}{3.166739in}}%
\pgfpathlineto{\pgfqpoint{0.686681in}{3.171142in}}%
\pgfpathlineto{\pgfqpoint{0.754600in}{3.179454in}}%
\pgfpathlineto{\pgfqpoint{0.822367in}{3.188929in}}%
\pgfpathlineto{\pgfqpoint{0.889966in}{3.199535in}}%
\pgfpathlineto{\pgfqpoint{0.957390in}{3.211204in}}%
\pgfpathlineto{\pgfqpoint{1.024644in}{3.223824in}}%
\pgfpathlineto{\pgfqpoint{1.091744in}{3.237240in}}%
\pgfpathlineto{\pgfqpoint{1.158719in}{3.251268in}}%
\pgfpathlineto{\pgfqpoint{1.225610in}{3.265692in}}%
\pgfpathlineto{\pgfqpoint{1.292468in}{3.280268in}}%
\pgfpathlineto{\pgfqpoint{1.359351in}{3.294729in}}%
\pgfpathlineto{\pgfqpoint{1.426318in}{3.308797in}}%
\pgfpathlineto{\pgfqpoint{1.493423in}{3.322182in}}%
\pgfpathlineto{\pgfqpoint{1.560713in}{3.334604in}}%
\pgfpathlineto{\pgfqpoint{1.628215in}{3.345803in}}%
\pgfpathlineto{\pgfqpoint{1.695940in}{3.355554in}}%
\pgfpathlineto{\pgfqpoint{1.763878in}{3.363682in}}%
\pgfpathlineto{\pgfqpoint{1.832001in}{3.370087in}}%
\pgfpathlineto{\pgfqpoint{1.900264in}{3.374758in}}%
\pgfpathlineto{\pgfqpoint{1.968621in}{3.377791in}}%
\pgfpathlineto{\pgfqpoint{2.037027in}{3.379388in}}%
\pgfpathlineto{\pgfqpoint{2.105452in}{3.379868in}}%
\pgfpathlineto{\pgfqpoint{2.173881in}{3.379667in}}%
\pgfpathlineto{\pgfqpoint{2.242308in}{3.379341in}}%
\pgfpathlineto{\pgfqpoint{2.310735in}{3.379538in}}%
\pgfpathlineto{\pgfqpoint{2.379143in}{3.380974in}}%
\pgfpathlineto{\pgfqpoint{2.447475in}{3.384407in}}%
\pgfpathlineto{\pgfqpoint{2.515605in}{3.390585in}}%
\pgfpathlineto{\pgfqpoint{2.583332in}{3.400167in}}%
\pgfpathlineto{\pgfqpoint{2.650386in}{3.413628in}}%
\pgfpathlineto{\pgfqpoint{2.716482in}{3.431184in}}%
\pgfpathlineto{\pgfqpoint{2.781373in}{3.452779in}}%
\pgfpathlineto{\pgfqpoint{2.844901in}{3.478117in}}%
\pgfpathlineto{\pgfqpoint{2.907024in}{3.506745in}}%
\pgfpathlineto{\pgfqpoint{2.967807in}{3.538132in}}%
\pgfpathlineto{\pgfqpoint{3.027398in}{3.571741in}}%
\pgfpathlineto{\pgfqpoint{3.027398in}{3.571741in}}%
\pgfusepath{stroke}%
\end{pgfscope}%
\begin{pgfscope}%
\pgfpathrectangle{\pgfqpoint{0.647939in}{0.492442in}}{\pgfqpoint{3.079299in}{3.079299in}}%
\pgfusepath{clip}%
\pgfsetbuttcap%
\pgfsetroundjoin%
\pgfsetlinewidth{0.301125pt}%
\definecolor{currentstroke}{rgb}{0.500000,0.500000,0.500000}%
\pgfsetstrokecolor{currentstroke}%
\pgfsetstrokeopacity{0.300000}%
\pgfsetdash{}{0pt}%
\pgfpathmoveto{\pgfqpoint{0.647939in}{3.255816in}}%
\pgfpathlineto{\pgfqpoint{0.661730in}{3.257274in}}%
\pgfpathlineto{\pgfqpoint{0.729714in}{3.265039in}}%
\pgfpathlineto{\pgfqpoint{0.797558in}{3.273942in}}%
\pgfpathlineto{\pgfqpoint{0.865246in}{3.283965in}}%
\pgfpathlineto{\pgfqpoint{0.932770in}{3.295046in}}%
\pgfpathlineto{\pgfqpoint{1.000130in}{3.307083in}}%
\pgfpathlineto{\pgfqpoint{1.067338in}{3.319943in}}%
\pgfpathlineto{\pgfqpoint{1.134419in}{3.333455in}}%
\pgfpathlineto{\pgfqpoint{1.201409in}{3.347414in}}%
\pgfpathlineto{\pgfqpoint{1.268353in}{3.361590in}}%
\pgfpathlineto{\pgfqpoint{1.335306in}{3.375726in}}%
\pgfpathlineto{\pgfqpoint{1.402324in}{3.389549in}}%
\pgfpathlineto{\pgfqpoint{1.469460in}{3.402781in}}%
\pgfpathlineto{\pgfqpoint{1.536759in}{3.415151in}}%
\pgfpathlineto{\pgfqpoint{1.604253in}{3.426400in}}%
\pgfpathlineto{\pgfqpoint{1.671958in}{3.436300in}}%
\pgfpathlineto{\pgfqpoint{1.739868in}{3.444670in}}%
\pgfpathlineto{\pgfqpoint{1.807960in}{3.451395in}}%
\pgfpathlineto{\pgfqpoint{1.876196in}{3.456440in}}%
\pgfpathlineto{\pgfqpoint{1.944534in}{3.459865in}}%
\pgfpathlineto{\pgfqpoint{2.012931in}{3.461831in}}%
\pgfpathlineto{\pgfqpoint{2.081353in}{3.462612in}}%
\pgfpathlineto{\pgfqpoint{2.149781in}{3.462598in}}%
\pgfpathlineto{\pgfqpoint{2.218209in}{3.462274in}}%
\pgfpathlineto{\pgfqpoint{2.286637in}{3.462214in}}%
\pgfpathlineto{\pgfqpoint{2.355058in}{3.463065in}}%
\pgfpathlineto{\pgfqpoint{2.423435in}{3.465524in}}%
\pgfpathlineto{\pgfqpoint{2.491682in}{3.470303in}}%
\pgfpathlineto{\pgfqpoint{2.559647in}{3.478049in}}%
\pgfpathlineto{\pgfqpoint{2.627118in}{3.489272in}}%
\pgfpathlineto{\pgfqpoint{2.693845in}{3.504288in}}%
\pgfpathlineto{\pgfqpoint{2.759582in}{3.523173in}}%
\pgfpathlineto{\pgfqpoint{2.824143in}{3.545771in}}%
\pgfpathlineto{\pgfqpoint{2.887429in}{3.571741in}}%
\pgfpathlineto{\pgfqpoint{2.887429in}{3.571741in}}%
\pgfusepath{stroke}%
\end{pgfscope}%
\begin{pgfscope}%
\pgfpathrectangle{\pgfqpoint{0.647939in}{0.492442in}}{\pgfqpoint{3.079299in}{3.079299in}}%
\pgfusepath{clip}%
\pgfsetbuttcap%
\pgfsetroundjoin%
\pgfsetlinewidth{0.301125pt}%
\definecolor{currentstroke}{rgb}{0.500000,0.500000,0.500000}%
\pgfsetstrokecolor{currentstroke}%
\pgfsetstrokeopacity{0.300000}%
\pgfsetdash{}{0pt}%
\pgfpathmoveto{\pgfqpoint{0.647939in}{3.334220in}}%
\pgfpathlineto{\pgfqpoint{0.711822in}{3.341671in}}%
\pgfpathlineto{\pgfqpoint{0.779720in}{3.350156in}}%
\pgfpathlineto{\pgfqpoint{0.847472in}{3.359740in}}%
\pgfpathlineto{\pgfqpoint{0.915067in}{3.370372in}}%
\pgfpathlineto{\pgfqpoint{0.982505in}{3.381965in}}%
\pgfpathlineto{\pgfqpoint{1.049794in}{3.394395in}}%
\pgfpathlineto{\pgfqpoint{1.116956in}{3.407498in}}%
\pgfpathlineto{\pgfqpoint{1.184024in}{3.421075in}}%
\pgfpathlineto{\pgfqpoint{1.251041in}{3.434902in}}%
\pgfpathlineto{\pgfqpoint{1.318057in}{3.448735in}}%
\pgfpathlineto{\pgfqpoint{1.385125in}{3.462312in}}%
\pgfpathlineto{\pgfqpoint{1.452297in}{3.475365in}}%
\pgfpathlineto{\pgfqpoint{1.519617in}{3.487625in}}%
\pgfpathlineto{\pgfqpoint{1.587118in}{3.498837in}}%
\pgfpathlineto{\pgfqpoint{1.654818in}{3.508772in}}%
\pgfpathlineto{\pgfqpoint{1.722714in}{3.517247in}}%
\pgfpathlineto{\pgfqpoint{1.790790in}{3.524140in}}%
\pgfpathlineto{\pgfqpoint{1.859011in}{3.529398in}}%
\pgfpathlineto{\pgfqpoint{1.927337in}{3.533063in}}%
\pgfpathlineto{\pgfqpoint{1.995726in}{3.535278in}}%
\pgfpathlineto{\pgfqpoint{2.064145in}{3.536286in}}%
\pgfpathlineto{\pgfqpoint{2.132572in}{3.536428in}}%
\pgfpathlineto{\pgfqpoint{2.201001in}{3.536142in}}%
\pgfpathlineto{\pgfqpoint{2.269429in}{3.535958in}}%
\pgfpathlineto{\pgfqpoint{2.337853in}{3.536484in}}%
\pgfpathlineto{\pgfqpoint{2.406250in}{3.538368in}}%
\pgfpathlineto{\pgfqpoint{2.474556in}{3.542268in}}%
\pgfpathlineto{\pgfqpoint{2.542654in}{3.548806in}}%
\pgfpathlineto{\pgfqpoint{2.610369in}{3.558508in}}%
\pgfpathlineto{\pgfqpoint{2.677477in}{3.571741in}}%
\pgfpathlineto{\pgfqpoint{2.677477in}{3.571741in}}%
\pgfusepath{stroke}%
\end{pgfscope}%
\begin{pgfscope}%
\pgfpathrectangle{\pgfqpoint{0.647939in}{0.492442in}}{\pgfqpoint{3.079299in}{3.079299in}}%
\pgfusepath{clip}%
\pgfsetbuttcap%
\pgfsetroundjoin%
\pgfsetlinewidth{0.301125pt}%
\definecolor{currentstroke}{rgb}{0.500000,0.500000,0.500000}%
\pgfsetstrokecolor{currentstroke}%
\pgfsetstrokeopacity{0.300000}%
\pgfsetdash{}{0pt}%
\pgfpathmoveto{\pgfqpoint{1.638404in}{3.542615in}}%
\pgfpathlineto{\pgfqpoint{1.706272in}{3.551317in}}%
\pgfpathlineto{\pgfqpoint{1.774321in}{3.558478in}}%
\pgfpathlineto{\pgfqpoint{1.842518in}{3.564033in}}%
\pgfpathlineto{\pgfqpoint{1.910826in}{3.568003in}}%
\pgfpathlineto{\pgfqpoint{1.979205in}{3.570502in}}%
\pgfpathlineto{\pgfqpoint{2.047620in}{3.571741in}}%
\pgfpathlineto{\pgfqpoint{2.047620in}{3.571741in}}%
\pgfusepath{stroke}%
\end{pgfscope}%
\begin{pgfscope}%
\pgfpathrectangle{\pgfqpoint{0.647939in}{0.492442in}}{\pgfqpoint{3.079299in}{3.079299in}}%
\pgfusepath{clip}%
\pgfsetbuttcap%
\pgfsetroundjoin%
\pgfsetlinewidth{0.301125pt}%
\definecolor{currentstroke}{rgb}{0.500000,0.500000,0.500000}%
\pgfsetstrokecolor{currentstroke}%
\pgfsetstrokeopacity{0.300000}%
\pgfsetdash{}{0pt}%
\pgfpathmoveto{\pgfqpoint{0.647939in}{3.415714in}}%
\pgfpathlineto{\pgfqpoint{0.681626in}{3.419344in}}%
\pgfpathlineto{\pgfqpoint{0.749597in}{3.427223in}}%
\pgfpathlineto{\pgfqpoint{0.817434in}{3.436188in}}%
\pgfpathlineto{\pgfqpoint{0.885123in}{3.446203in}}%
\pgfpathlineto{\pgfqpoint{0.952662in}{3.457195in}}%
\pgfpathlineto{\pgfqpoint{1.020053in}{3.469056in}}%
\pgfpathlineto{\pgfqpoint{1.087315in}{3.481637in}}%
\pgfpathlineto{\pgfqpoint{1.154474in}{3.494755in}}%
\pgfpathlineto{\pgfqpoint{1.221569in}{3.508201in}}%
\pgfpathlineto{\pgfqpoint{1.288645in}{3.521741in}}%
\pgfpathlineto{\pgfqpoint{1.355751in}{3.535127in}}%
\pgfpathlineto{\pgfqpoint{1.422939in}{3.548097in}}%
\pgfpathlineto{\pgfqpoint{1.490254in}{3.560387in}}%
\pgfpathlineto{\pgfqpoint{1.557732in}{3.571741in}}%
\pgfpathlineto{\pgfqpoint{1.557732in}{3.571741in}}%
\pgfusepath{stroke}%
\end{pgfscope}%
\begin{pgfscope}%
\pgfpathrectangle{\pgfqpoint{0.647939in}{0.492442in}}{\pgfqpoint{3.079299in}{3.079299in}}%
\pgfusepath{clip}%
\pgfsetbuttcap%
\pgfsetroundjoin%
\pgfsetlinewidth{0.301125pt}%
\definecolor{currentstroke}{rgb}{0.500000,0.500000,0.500000}%
\pgfsetstrokecolor{currentstroke}%
\pgfsetstrokeopacity{0.300000}%
\pgfsetdash{}{0pt}%
\pgfpathmoveto{\pgfqpoint{0.647939in}{3.497408in}}%
\pgfpathlineto{\pgfqpoint{0.664544in}{3.499107in}}%
\pgfpathlineto{\pgfqpoint{0.732558in}{3.506608in}}%
\pgfpathlineto{\pgfqpoint{0.800445in}{3.515179in}}%
\pgfpathlineto{\pgfqpoint{0.868194in}{3.524789in}}%
\pgfpathlineto{\pgfqpoint{0.935798in}{3.535370in}}%
\pgfpathlineto{\pgfqpoint{1.003261in}{3.546819in}}%
\pgfpathlineto{\pgfqpoint{1.070596in}{3.558999in}}%
\pgfpathlineto{\pgfqpoint{1.137828in}{3.571741in}}%
\pgfpathlineto{\pgfqpoint{1.137828in}{3.571741in}}%
\pgfusepath{stroke}%
\end{pgfscope}%
\begin{pgfscope}%
\pgfpathrectangle{\pgfqpoint{0.647939in}{0.492442in}}{\pgfqpoint{3.079299in}{3.079299in}}%
\pgfusepath{clip}%
\pgfsetbuttcap%
\pgfsetroundjoin%
\pgfsetlinewidth{0.301125pt}%
\definecolor{currentstroke}{rgb}{0.500000,0.500000,0.500000}%
\pgfsetstrokecolor{currentstroke}%
\pgfsetstrokeopacity{0.300000}%
\pgfsetdash{}{0pt}%
\pgfpathmoveto{\pgfqpoint{0.647939in}{2.941885in}}%
\pgfpathlineto{\pgfqpoint{0.647939in}{2.941885in}}%
\pgfpathlineto{\pgfqpoint{0.715901in}{2.949841in}}%
\pgfpathlineto{\pgfqpoint{0.783706in}{2.959036in}}%
\pgfpathlineto{\pgfqpoint{0.851332in}{2.969466in}}%
\pgfpathlineto{\pgfqpoint{0.918763in}{2.981091in}}%
\pgfpathlineto{\pgfqpoint{0.985993in}{2.993830in}}%
\pgfpathlineto{\pgfqpoint{1.053028in}{3.007559in}}%
\pgfpathlineto{\pgfqpoint{1.119891in}{3.022110in}}%
\pgfpathlineto{\pgfqpoint{1.186617in}{3.037276in}}%
\pgfpathlineto{\pgfqpoint{1.253256in}{3.052825in}}%
\pgfpathlineto{\pgfqpoint{1.319868in}{3.068490in}}%
\pgfpathlineto{\pgfqpoint{1.386518in}{3.083988in}}%
\pgfpathlineto{\pgfqpoint{1.453276in}{3.099014in}}%
\pgfpathlineto{\pgfqpoint{1.520204in}{3.113260in}}%
\pgfpathlineto{\pgfqpoint{1.587352in}{3.126421in}}%
\pgfpathlineto{\pgfqpoint{1.654752in}{3.138211in}}%
\pgfpathlineto{\pgfqpoint{1.722414in}{3.148387in}}%
\pgfpathlineto{\pgfqpoint{1.790320in}{3.156766in}}%
\pgfpathlineto{\pgfqpoint{1.858434in}{3.163241in}}%
\pgfpathlineto{\pgfqpoint{1.926703in}{3.167817in}}%
\pgfpathlineto{\pgfqpoint{1.995068in}{3.170622in}}%
\pgfpathlineto{\pgfqpoint{2.063481in}{3.171919in}}%
\pgfpathlineto{\pgfqpoint{2.131908in}{3.172109in}}%
\pgfpathlineto{\pgfqpoint{2.200336in}{3.171734in}}%
\pgfpathlineto{\pgfqpoint{2.268764in}{3.171485in}}%
\pgfpathlineto{\pgfqpoint{2.337185in}{3.172190in}}%
\pgfpathlineto{\pgfqpoint{2.405553in}{3.174767in}}%
\pgfpathlineto{\pgfqpoint{2.473743in}{3.180172in}}%
\pgfpathlineto{\pgfqpoint{2.541524in}{3.189301in}}%
\pgfpathlineto{\pgfqpoint{2.608550in}{3.202854in}}%
\pgfpathlineto{\pgfqpoint{2.674419in}{3.221197in}}%
\pgfpathlineto{\pgfqpoint{2.738771in}{3.244296in}}%
\pgfpathlineto{\pgfqpoint{2.801390in}{3.271766in}}%
\pgfpathlineto{\pgfqpoint{2.862227in}{3.303005in}}%
\pgfusepath{stroke}%
\end{pgfscope}%
\begin{pgfscope}%
\pgfpathrectangle{\pgfqpoint{0.647939in}{0.492442in}}{\pgfqpoint{3.079299in}{3.079299in}}%
\pgfusepath{clip}%
\pgfsetbuttcap%
\pgfsetroundjoin%
\pgfsetlinewidth{0.301125pt}%
\definecolor{currentstroke}{rgb}{0.500000,0.500000,0.500000}%
\pgfsetstrokecolor{currentstroke}%
\pgfsetstrokeopacity{0.300000}%
\pgfsetdash{}{0pt}%
\pgfpathmoveto{\pgfqpoint{0.647939in}{2.731932in}}%
\pgfpathlineto{\pgfqpoint{0.647939in}{2.731932in}}%
\pgfpathlineto{\pgfqpoint{0.715864in}{2.740197in}}%
\pgfpathlineto{\pgfqpoint{0.783615in}{2.749775in}}%
\pgfpathlineto{\pgfqpoint{0.851166in}{2.760676in}}%
\pgfpathlineto{\pgfqpoint{0.918497in}{2.772866in}}%
\pgfpathlineto{\pgfqpoint{0.985596in}{2.786271in}}%
\pgfpathlineto{\pgfqpoint{1.052469in}{2.800773in}}%
\pgfpathlineto{\pgfqpoint{1.119133in}{2.816205in}}%
\pgfpathlineto{\pgfqpoint{1.185627in}{2.832359in}}%
\pgfpathlineto{\pgfqpoint{1.252003in}{2.848996in}}%
\pgfpathlineto{\pgfqpoint{1.318325in}{2.865844in}}%
\pgfpathlineto{\pgfqpoint{1.384669in}{2.882606in}}%
\pgfpathlineto{\pgfqpoint{1.451114in}{2.898961in}}%
\pgfpathlineto{\pgfqpoint{1.517735in}{2.914576in}}%
\pgfpathlineto{\pgfqpoint{1.584598in}{2.929117in}}%
\pgfpathlineto{\pgfqpoint{1.651748in}{2.942259in}}%
\pgfpathlineto{\pgfqpoint{1.719203in}{2.953715in}}%
\pgfpathlineto{\pgfqpoint{1.786955in}{2.963251in}}%
\pgfpathlineto{\pgfqpoint{1.854966in}{2.970717in}}%
\pgfpathlineto{\pgfqpoint{1.923176in}{2.976068in}}%
\pgfpathlineto{\pgfqpoint{1.991517in}{2.979409in}}%
\pgfpathlineto{\pgfqpoint{2.059922in}{2.981003in}}%
\pgfpathlineto{\pgfqpoint{2.128348in}{2.981279in}}%
\pgfpathlineto{\pgfqpoint{2.196775in}{2.980850in}}%
\pgfpathlineto{\pgfqpoint{2.265202in}{2.980518in}}%
\pgfpathlineto{\pgfqpoint{2.333621in}{2.981294in}}%
\pgfpathlineto{\pgfqpoint{2.401963in}{2.984354in}}%
\pgfpathlineto{\pgfqpoint{2.470035in}{2.990937in}}%
\pgfpathlineto{\pgfqpoint{2.537471in}{3.002186in}}%
\pgfpathlineto{\pgfqpoint{2.603747in}{3.018909in}}%
\pgfpathlineto{\pgfqpoint{2.668308in}{3.041357in}}%
\pgfpathlineto{\pgfqpoint{2.730749in}{3.069192in}}%
\pgfpathlineto{\pgfqpoint{2.790924in}{3.101666in}}%
\pgfpathlineto{\pgfqpoint{2.848941in}{3.137874in}}%
\pgfpathlineto{\pgfqpoint{2.905075in}{3.176954in}}%
\pgfpathlineto{\pgfqpoint{2.959669in}{3.218168in}}%
\pgfpathlineto{\pgfqpoint{3.013077in}{3.260928in}}%
\pgfpathlineto{\pgfqpoint{3.065626in}{3.304748in}}%
\pgfpathlineto{\pgfqpoint{3.117612in}{3.349233in}}%
\pgfusepath{stroke}%
\end{pgfscope}%
\begin{pgfscope}%
\pgfpathrectangle{\pgfqpoint{0.647939in}{0.492442in}}{\pgfqpoint{3.079299in}{3.079299in}}%
\pgfusepath{clip}%
\pgfsetbuttcap%
\pgfsetroundjoin%
\pgfsetlinewidth{0.301125pt}%
\definecolor{currentstroke}{rgb}{0.500000,0.500000,0.500000}%
\pgfsetstrokecolor{currentstroke}%
\pgfsetstrokeopacity{0.300000}%
\pgfsetdash{}{0pt}%
\pgfpathmoveto{\pgfqpoint{0.647939in}{2.661948in}}%
\pgfpathlineto{\pgfqpoint{0.647939in}{2.661948in}}%
\pgfpathlineto{\pgfqpoint{0.715850in}{2.670320in}}%
\pgfpathlineto{\pgfqpoint{0.783582in}{2.680034in}}%
\pgfpathlineto{\pgfqpoint{0.851106in}{2.691101in}}%
\pgfpathlineto{\pgfqpoint{0.918400in}{2.703492in}}%
\pgfpathlineto{\pgfqpoint{0.985452in}{2.717134in}}%
\pgfpathlineto{\pgfqpoint{1.052263in}{2.731912in}}%
\pgfpathlineto{\pgfqpoint{1.118854in}{2.747660in}}%
\pgfpathlineto{\pgfqpoint{1.185260in}{2.764170in}}%
\pgfpathlineto{\pgfqpoint{1.251535in}{2.781202in}}%
\pgfpathlineto{\pgfqpoint{1.317746in}{2.798482in}}%
\pgfpathlineto{\pgfqpoint{1.383971in}{2.815708in}}%
\pgfpathlineto{\pgfqpoint{1.450293in}{2.832554in}}%
\pgfpathlineto{\pgfqpoint{1.516793in}{2.848680in}}%
\pgfpathlineto{\pgfqpoint{1.583541in}{2.863740in}}%
\pgfpathlineto{\pgfqpoint{1.650588in}{2.877397in}}%
\pgfpathlineto{\pgfqpoint{1.717957in}{2.889346in}}%
\pgfpathlineto{\pgfqpoint{1.785643in}{2.899336in}}%
\pgfusepath{stroke}%
\end{pgfscope}%
\begin{pgfscope}%
\pgfpathrectangle{\pgfqpoint{0.647939in}{0.492442in}}{\pgfqpoint{3.079299in}{3.079299in}}%
\pgfusepath{clip}%
\pgfsetbuttcap%
\pgfsetroundjoin%
\pgfsetlinewidth{0.301125pt}%
\definecolor{currentstroke}{rgb}{0.500000,0.500000,0.500000}%
\pgfsetstrokecolor{currentstroke}%
\pgfsetstrokeopacity{0.300000}%
\pgfsetdash{}{0pt}%
\pgfpathmoveto{\pgfqpoint{0.647939in}{2.451996in}}%
\pgfpathlineto{\pgfqpoint{0.647939in}{2.451996in}}%
\pgfpathlineto{\pgfqpoint{0.715807in}{2.460709in}}%
\pgfpathlineto{\pgfqpoint{0.783475in}{2.470852in}}%
\pgfpathlineto{\pgfqpoint{0.850909in}{2.482448in}}%
\pgfpathlineto{\pgfqpoint{0.918081in}{2.495480in}}%
\pgfpathlineto{\pgfqpoint{0.984972in}{2.509886in}}%
\pgfpathlineto{\pgfqpoint{1.051580in}{2.525556in}}%
\pgfpathlineto{\pgfqpoint{1.117919in}{2.542329in}}%
\pgfpathlineto{\pgfqpoint{1.184025in}{2.560000in}}%
\pgfpathlineto{\pgfqpoint{1.249954in}{2.578327in}}%
\pgfpathlineto{\pgfqpoint{1.315776in}{2.597031in}}%
\pgfpathlineto{\pgfqpoint{1.381581in}{2.615798in}}%
\pgfpathlineto{\pgfqpoint{1.447464in}{2.634290in}}%
\pgfpathlineto{\pgfqpoint{1.513521in}{2.652143in}}%
\pgfpathlineto{\pgfqpoint{1.579843in}{2.668979in}}%
\pgfpathlineto{\pgfqpoint{1.646502in}{2.684419in}}%
\pgfpathlineto{\pgfqpoint{1.713541in}{2.698102in}}%
\pgfpathlineto{\pgfqpoint{1.780967in}{2.709710in}}%
\pgfpathlineto{\pgfqpoint{1.848749in}{2.719001in}}%
\pgfpathlineto{\pgfqpoint{1.916821in}{2.725844in}}%
\pgfpathlineto{\pgfqpoint{1.985096in}{2.730259in}}%
\pgfpathlineto{\pgfqpoint{2.053481in}{2.732474in}}%
\pgfpathlineto{\pgfqpoint{2.121904in}{2.732957in}}%
\pgfpathlineto{\pgfqpoint{2.190330in}{2.732428in}}%
\pgfpathlineto{\pgfqpoint{2.258756in}{2.731921in}}%
\pgfpathlineto{\pgfqpoint{2.327170in}{2.732824in}}%
\pgfpathlineto{\pgfqpoint{2.395441in}{2.736914in}}%
\pgfpathlineto{\pgfqpoint{2.463154in}{2.746167in}}%
\pgfpathlineto{\pgfqpoint{2.529516in}{2.762258in}}%
\pgfpathlineto{\pgfqpoint{2.593544in}{2.785914in}}%
\pgfpathlineto{\pgfqpoint{2.654520in}{2.816619in}}%
\pgfusepath{stroke}%
\end{pgfscope}%
\begin{pgfscope}%
\pgfpathrectangle{\pgfqpoint{0.647939in}{0.492442in}}{\pgfqpoint{3.079299in}{3.079299in}}%
\pgfusepath{clip}%
\pgfsetbuttcap%
\pgfsetroundjoin%
\pgfsetlinewidth{0.301125pt}%
\definecolor{currentstroke}{rgb}{0.500000,0.500000,0.500000}%
\pgfsetstrokecolor{currentstroke}%
\pgfsetstrokeopacity{0.300000}%
\pgfsetdash{}{0pt}%
\pgfpathmoveto{\pgfqpoint{0.647939in}{2.382012in}}%
\pgfpathlineto{\pgfqpoint{0.647939in}{2.382012in}}%
\pgfpathlineto{\pgfqpoint{0.715791in}{2.390845in}}%
\pgfpathlineto{\pgfqpoint{0.783437in}{2.401139in}}%
\pgfpathlineto{\pgfqpoint{0.850838in}{2.412923in}}%
\pgfpathlineto{\pgfqpoint{0.917965in}{2.426184in}}%
\pgfpathlineto{\pgfqpoint{0.984796in}{2.440862in}}%
\pgfpathlineto{\pgfqpoint{1.051327in}{2.456851in}}%
\pgfpathlineto{\pgfqpoint{1.117572in}{2.473995in}}%
\pgfpathlineto{\pgfqpoint{1.183564in}{2.492087in}}%
\pgfpathlineto{\pgfqpoint{1.249359in}{2.510886in}}%
\pgfpathlineto{\pgfqpoint{1.315031in}{2.530112in}}%
\pgfpathlineto{\pgfqpoint{1.380670in}{2.549450in}}%
\pgfpathlineto{\pgfqpoint{1.446378in}{2.568555in}}%
\pgfpathlineto{\pgfqpoint{1.512256in}{2.587057in}}%
\pgfpathlineto{\pgfqpoint{1.578403in}{2.604568in}}%
\pgfpathlineto{\pgfqpoint{1.644899in}{2.620696in}}%
\pgfpathlineto{\pgfqpoint{1.711796in}{2.635058in}}%
\pgfpathlineto{\pgfqpoint{1.779107in}{2.647311in}}%
\pgfpathlineto{\pgfqpoint{1.846806in}{2.657185in}}%
\pgfpathlineto{\pgfqpoint{1.914826in}{2.664517in}}%
\pgfpathlineto{\pgfqpoint{1.983074in}{2.669301in}}%
\pgfpathlineto{\pgfqpoint{2.051450in}{2.671742in}}%
\pgfpathlineto{\pgfqpoint{2.119872in}{2.672303in}}%
\pgfpathlineto{\pgfqpoint{2.188297in}{2.671742in}}%
\pgfpathlineto{\pgfqpoint{2.256723in}{2.671167in}}%
\pgfpathlineto{\pgfqpoint{2.325134in}{2.672114in}}%
\pgfpathlineto{\pgfqpoint{2.393372in}{2.676594in}}%
\pgfpathlineto{\pgfqpoint{2.460911in}{2.686885in}}%
\pgfusepath{stroke}%
\end{pgfscope}%
\begin{pgfscope}%
\pgfpathrectangle{\pgfqpoint{0.647939in}{0.492442in}}{\pgfqpoint{3.079299in}{3.079299in}}%
\pgfusepath{clip}%
\pgfsetbuttcap%
\pgfsetroundjoin%
\pgfsetlinewidth{0.301125pt}%
\definecolor{currentstroke}{rgb}{0.500000,0.500000,0.500000}%
\pgfsetstrokecolor{currentstroke}%
\pgfsetstrokeopacity{0.300000}%
\pgfsetdash{}{0pt}%
\pgfpathmoveto{\pgfqpoint{0.647939in}{2.312028in}}%
\pgfpathlineto{\pgfqpoint{0.647939in}{2.312028in}}%
\pgfpathlineto{\pgfqpoint{0.715775in}{2.320984in}}%
\pgfpathlineto{\pgfqpoint{0.783396in}{2.331434in}}%
\pgfpathlineto{\pgfqpoint{0.850763in}{2.343412in}}%
\pgfpathlineto{\pgfqpoint{0.917842in}{2.356908in}}%
\pgfpathlineto{\pgfqpoint{0.984611in}{2.371869in}}%
\pgfpathlineto{\pgfqpoint{1.051061in}{2.388191in}}%
\pgfpathlineto{\pgfqpoint{1.117204in}{2.405720in}}%
\pgfpathlineto{\pgfqpoint{1.183073in}{2.424253in}}%
\pgfpathlineto{\pgfqpoint{1.248725in}{2.443546in}}%
\pgfpathlineto{\pgfqpoint{1.314234in}{2.463321in}}%
\pgfpathlineto{\pgfqpoint{1.379693in}{2.483260in}}%
\pgfpathlineto{\pgfqpoint{1.445208in}{2.503014in}}%
\pgfpathlineto{\pgfqpoint{1.510889in}{2.522208in}}%
\pgfpathlineto{\pgfqpoint{1.576840in}{2.540443in}}%
\pgfusepath{stroke}%
\end{pgfscope}%
\begin{pgfscope}%
\pgfpathrectangle{\pgfqpoint{0.647939in}{0.492442in}}{\pgfqpoint{3.079299in}{3.079299in}}%
\pgfusepath{clip}%
\pgfsetbuttcap%
\pgfsetroundjoin%
\pgfsetlinewidth{0.301125pt}%
\definecolor{currentstroke}{rgb}{0.500000,0.500000,0.500000}%
\pgfsetstrokecolor{currentstroke}%
\pgfsetstrokeopacity{0.300000}%
\pgfsetdash{}{0pt}%
\pgfpathmoveto{\pgfqpoint{0.647939in}{2.242044in}}%
\pgfpathlineto{\pgfqpoint{0.647939in}{2.242044in}}%
\pgfpathlineto{\pgfqpoint{0.715758in}{2.251126in}}%
\pgfpathlineto{\pgfqpoint{0.783354in}{2.261737in}}%
\pgfpathlineto{\pgfqpoint{0.850684in}{2.273915in}}%
\pgfpathlineto{\pgfqpoint{0.917714in}{2.287656in}}%
\pgfpathlineto{\pgfqpoint{0.984416in}{2.302910in}}%
\pgfpathlineto{\pgfqpoint{1.050779in}{2.319578in}}%
\pgfpathlineto{\pgfqpoint{1.116814in}{2.337508in}}%
\pgfpathlineto{\pgfqpoint{1.182553in}{2.356501in}}%
\pgfpathlineto{\pgfqpoint{1.248049in}{2.376313in}}%
\pgfpathlineto{\pgfqpoint{1.313381in}{2.396665in}}%
\pgfusepath{stroke}%
\end{pgfscope}%
\begin{pgfscope}%
\pgfpathrectangle{\pgfqpoint{0.647939in}{0.492442in}}{\pgfqpoint{3.079299in}{3.079299in}}%
\pgfusepath{clip}%
\pgfsetbuttcap%
\pgfsetroundjoin%
\pgfsetlinewidth{0.301125pt}%
\definecolor{currentstroke}{rgb}{0.500000,0.500000,0.500000}%
\pgfsetstrokecolor{currentstroke}%
\pgfsetstrokeopacity{0.300000}%
\pgfsetdash{}{0pt}%
\pgfpathmoveto{\pgfqpoint{0.647939in}{2.172060in}}%
\pgfpathlineto{\pgfqpoint{0.647939in}{2.172060in}}%
\pgfpathlineto{\pgfqpoint{0.715740in}{2.181272in}}%
\pgfpathlineto{\pgfqpoint{0.783310in}{2.192049in}}%
\pgfpathlineto{\pgfqpoint{0.850602in}{2.204433in}}%
\pgfpathlineto{\pgfqpoint{0.917579in}{2.218427in}}%
\pgfpathlineto{\pgfqpoint{0.984210in}{2.233986in}}%
\pgfpathlineto{\pgfqpoint{1.050482in}{2.251013in}}%
\pgfpathlineto{\pgfqpoint{1.116401in}{2.269363in}}%
\pgfpathlineto{\pgfqpoint{1.181998in}{2.288836in}}%
\pgfpathlineto{\pgfqpoint{1.247328in}{2.309192in}}%
\pgfpathlineto{\pgfqpoint{1.312467in}{2.330151in}}%
\pgfusepath{stroke}%
\end{pgfscope}%
\begin{pgfscope}%
\pgfpathrectangle{\pgfqpoint{0.647939in}{0.492442in}}{\pgfqpoint{3.079299in}{3.079299in}}%
\pgfusepath{clip}%
\pgfsetbuttcap%
\pgfsetroundjoin%
\pgfsetlinewidth{0.301125pt}%
\definecolor{currentstroke}{rgb}{0.500000,0.500000,0.500000}%
\pgfsetstrokecolor{currentstroke}%
\pgfsetstrokeopacity{0.300000}%
\pgfsetdash{}{0pt}%
\pgfpathmoveto{\pgfqpoint{0.647939in}{2.102076in}}%
\pgfpathlineto{\pgfqpoint{0.647939in}{2.102076in}}%
\pgfpathlineto{\pgfqpoint{0.715722in}{2.111422in}}%
\pgfpathlineto{\pgfqpoint{0.783263in}{2.122369in}}%
\pgfpathlineto{\pgfqpoint{0.850516in}{2.134967in}}%
\pgfpathlineto{\pgfqpoint{0.917437in}{2.149224in}}%
\pgfpathlineto{\pgfqpoint{0.983993in}{2.165098in}}%
\pgfpathlineto{\pgfqpoint{1.050167in}{2.182500in}}%
\pgfpathlineto{\pgfqpoint{1.115963in}{2.201287in}}%
\pgfpathlineto{\pgfqpoint{1.181408in}{2.221263in}}%
\pgfpathlineto{\pgfqpoint{1.246556in}{2.242190in}}%
\pgfpathlineto{\pgfqpoint{1.311486in}{2.263789in}}%
\pgfpathlineto{\pgfqpoint{1.376298in}{2.285741in}}%
\pgfpathlineto{\pgfqpoint{1.441111in}{2.307691in}}%
\pgfpathlineto{\pgfqpoint{1.506055in}{2.329247in}}%
\pgfpathlineto{\pgfqpoint{1.571262in}{2.349987in}}%
\pgfpathlineto{\pgfqpoint{1.636855in}{2.369465in}}%
\pgfpathlineto{\pgfqpoint{1.702932in}{2.387217in}}%
\pgfpathlineto{\pgfqpoint{1.769554in}{2.402785in}}%
\pgfpathlineto{\pgfqpoint{1.836726in}{2.415754in}}%
\pgfpathlineto{\pgfqpoint{1.904394in}{2.425795in}}%
\pgfpathlineto{\pgfqpoint{1.972448in}{2.432726in}}%
\pgfpathlineto{\pgfqpoint{2.040749in}{2.436586in}}%
\pgfpathlineto{\pgfqpoint{2.109156in}{2.437730in}}%
\pgfpathlineto{\pgfqpoint{2.177578in}{2.436998in}}%
\pgfpathlineto{\pgfqpoint{2.245997in}{2.435917in}}%
\pgfpathlineto{\pgfqpoint{2.314382in}{2.437061in}}%
\pgfpathlineto{\pgfqpoint{2.382230in}{2.444518in}}%
\pgfpathlineto{\pgfqpoint{2.447679in}{2.462998in}}%
\pgfpathlineto{\pgfqpoint{2.506150in}{2.492547in}}%
\pgfpathlineto{\pgfqpoint{2.558097in}{2.529785in}}%
\pgfpathlineto{\pgfqpoint{2.609109in}{2.575076in}}%
\pgfpathlineto{\pgfqpoint{2.657180in}{2.623637in}}%
\pgfpathlineto{\pgfqpoint{2.703366in}{2.674015in}}%
\pgfpathlineto{\pgfqpoint{2.748407in}{2.725464in}}%
\pgfusepath{stroke}%
\end{pgfscope}%
\begin{pgfscope}%
\pgfpathrectangle{\pgfqpoint{0.647939in}{0.492442in}}{\pgfqpoint{3.079299in}{3.079299in}}%
\pgfusepath{clip}%
\pgfsetbuttcap%
\pgfsetroundjoin%
\pgfsetlinewidth{0.301125pt}%
\definecolor{currentstroke}{rgb}{0.500000,0.500000,0.500000}%
\pgfsetstrokecolor{currentstroke}%
\pgfsetstrokeopacity{0.300000}%
\pgfsetdash{}{0pt}%
\pgfpathmoveto{\pgfqpoint{0.647939in}{2.032092in}}%
\pgfpathlineto{\pgfqpoint{0.647939in}{2.032092in}}%
\pgfpathlineto{\pgfqpoint{0.715702in}{2.041576in}}%
\pgfpathlineto{\pgfqpoint{0.783215in}{2.052699in}}%
\pgfpathlineto{\pgfqpoint{0.850425in}{2.065518in}}%
\pgfpathlineto{\pgfqpoint{0.917287in}{2.080046in}}%
\pgfpathlineto{\pgfqpoint{0.983764in}{2.096248in}}%
\pgfpathlineto{\pgfqpoint{1.049834in}{2.114041in}}%
\pgfpathlineto{\pgfqpoint{1.115497in}{2.133284in}}%
\pgfpathlineto{\pgfqpoint{1.180778in}{2.153788in}}%
\pgfpathlineto{\pgfqpoint{1.245731in}{2.175314in}}%
\pgfpathlineto{\pgfqpoint{1.310432in}{2.197587in}}%
\pgfpathlineto{\pgfqpoint{1.374985in}{2.220289in}}%
\pgfpathlineto{\pgfqpoint{1.439513in}{2.243062in}}%
\pgfpathlineto{\pgfqpoint{1.504153in}{2.265515in}}%
\pgfpathlineto{\pgfqpoint{1.569046in}{2.287219in}}%
\pgfpathlineto{\pgfqpoint{1.634328in}{2.307716in}}%
\pgfpathlineto{\pgfqpoint{1.700112in}{2.326524in}}%
\pgfpathlineto{\pgfqpoint{1.766476in}{2.343157in}}%
\pgfusepath{stroke}%
\end{pgfscope}%
\begin{pgfscope}%
\pgfpathrectangle{\pgfqpoint{0.647939in}{0.492442in}}{\pgfqpoint{3.079299in}{3.079299in}}%
\pgfusepath{clip}%
\pgfsetbuttcap%
\pgfsetroundjoin%
\pgfsetlinewidth{0.301125pt}%
\definecolor{currentstroke}{rgb}{0.500000,0.500000,0.500000}%
\pgfsetstrokecolor{currentstroke}%
\pgfsetstrokeopacity{0.300000}%
\pgfsetdash{}{0pt}%
\pgfpathmoveto{\pgfqpoint{0.647939in}{1.962108in}}%
\pgfpathlineto{\pgfqpoint{0.647939in}{1.962108in}}%
\pgfpathlineto{\pgfqpoint{0.715682in}{1.971733in}}%
\pgfpathlineto{\pgfqpoint{0.783164in}{1.983039in}}%
\pgfpathlineto{\pgfqpoint{0.850330in}{1.996086in}}%
\pgfpathlineto{\pgfqpoint{0.917130in}{2.010896in}}%
\pgfpathlineto{\pgfqpoint{0.983521in}{2.027440in}}%
\pgfpathlineto{\pgfqpoint{1.049480in}{2.045638in}}%
\pgfpathlineto{\pgfqpoint{1.115001in}{2.065359in}}%
\pgfpathlineto{\pgfqpoint{1.180106in}{2.086415in}}%
\pgfpathlineto{\pgfqpoint{1.244846in}{2.108572in}}%
\pgfpathlineto{\pgfqpoint{1.309298in}{2.131555in}}%
\pgfpathlineto{\pgfqpoint{1.373567in}{2.155050in}}%
\pgfpathlineto{\pgfqpoint{1.437778in}{2.178699in}}%
\pgfpathlineto{\pgfqpoint{1.502077in}{2.202110in}}%
\pgfpathlineto{\pgfqpoint{1.566615in}{2.224851in}}%
\pgfpathlineto{\pgfqpoint{1.631539in}{2.246453in}}%
\pgfpathlineto{\pgfqpoint{1.696981in}{2.266421in}}%
\pgfusepath{stroke}%
\end{pgfscope}%
\begin{pgfscope}%
\pgfpathrectangle{\pgfqpoint{0.647939in}{0.492442in}}{\pgfqpoint{3.079299in}{3.079299in}}%
\pgfusepath{clip}%
\pgfsetbuttcap%
\pgfsetroundjoin%
\pgfsetlinewidth{0.301125pt}%
\definecolor{currentstroke}{rgb}{0.500000,0.500000,0.500000}%
\pgfsetstrokecolor{currentstroke}%
\pgfsetstrokeopacity{0.300000}%
\pgfsetdash{}{0pt}%
\pgfpathmoveto{\pgfqpoint{0.647939in}{1.892124in}}%
\pgfpathlineto{\pgfqpoint{0.647939in}{1.892124in}}%
\pgfpathlineto{\pgfqpoint{0.715661in}{1.901895in}}%
\pgfpathlineto{\pgfqpoint{0.783111in}{1.913389in}}%
\pgfpathlineto{\pgfqpoint{0.850230in}{1.926673in}}%
\pgfpathlineto{\pgfqpoint{0.916964in}{1.941774in}}%
\pgfpathlineto{\pgfqpoint{0.983265in}{1.958673in}}%
\pgfpathlineto{\pgfqpoint{1.049105in}{1.977296in}}%
\pgfusepath{stroke}%
\end{pgfscope}%
\begin{pgfscope}%
\pgfpathrectangle{\pgfqpoint{0.647939in}{0.492442in}}{\pgfqpoint{3.079299in}{3.079299in}}%
\pgfusepath{clip}%
\pgfsetbuttcap%
\pgfsetroundjoin%
\pgfsetlinewidth{0.301125pt}%
\definecolor{currentstroke}{rgb}{0.500000,0.500000,0.500000}%
\pgfsetstrokecolor{currentstroke}%
\pgfsetstrokeopacity{0.300000}%
\pgfsetdash{}{0pt}%
\pgfpathmoveto{\pgfqpoint{0.647939in}{1.752155in}}%
\pgfpathlineto{\pgfqpoint{0.647939in}{1.752155in}}%
\pgfpathlineto{\pgfqpoint{0.715615in}{1.762233in}}%
\pgfpathlineto{\pgfqpoint{0.782996in}{1.774121in}}%
\pgfpathlineto{\pgfqpoint{0.850014in}{1.787903in}}%
\pgfpathlineto{\pgfqpoint{0.916603in}{1.803624in}}%
\pgfpathlineto{\pgfqpoint{0.982706in}{1.821277in}}%
\pgfpathlineto{\pgfqpoint{1.048282in}{1.840804in}}%
\pgfpathlineto{\pgfqpoint{1.113309in}{1.862091in}}%
\pgfpathlineto{\pgfqpoint{1.177794in}{1.884969in}}%
\pgfpathlineto{\pgfqpoint{1.241778in}{1.909219in}}%
\pgfpathlineto{\pgfqpoint{1.305332in}{1.934579in}}%
\pgfpathlineto{\pgfqpoint{1.368559in}{1.960743in}}%
\pgfpathlineto{\pgfqpoint{1.431595in}{1.987369in}}%
\pgfpathlineto{\pgfqpoint{1.494599in}{2.014069in}}%
\pgfpathlineto{\pgfqpoint{1.557752in}{2.040416in}}%
\pgfpathlineto{\pgfqpoint{1.621241in}{2.065933in}}%
\pgfpathlineto{\pgfqpoint{1.685254in}{2.090094in}}%
\pgfpathlineto{\pgfqpoint{1.749958in}{2.112326in}}%
\pgfpathlineto{\pgfqpoint{1.815474in}{2.132010in}}%
\pgfpathlineto{\pgfqpoint{1.881853in}{2.148510in}}%
\pgfpathlineto{\pgfqpoint{1.949056in}{2.161209in}}%
\pgfpathlineto{\pgfqpoint{2.016928in}{2.169582in}}%
\pgfpathlineto{\pgfqpoint{2.085210in}{2.173364in}}%
\pgfpathlineto{\pgfqpoint{2.153599in}{2.172730in}}%
\pgfpathlineto{\pgfqpoint{2.221913in}{2.168955in}}%
\pgfpathlineto{\pgfqpoint{2.287843in}{2.172581in}}%
\pgfpathlineto{\pgfqpoint{2.287843in}{2.172581in}}%
\pgfpathlineto{\pgfqpoint{2.309173in}{2.179425in}}%
\pgfpathlineto{\pgfqpoint{2.329067in}{2.193621in}}%
\pgfpathlineto{\pgfqpoint{2.349178in}{2.212422in}}%
\pgfpathlineto{\pgfqpoint{2.375661in}{2.241090in}}%
\pgfpathlineto{\pgfqpoint{2.418885in}{2.291278in}}%
\pgfpathlineto{\pgfqpoint{2.462338in}{2.343020in}}%
\pgfpathlineto{\pgfqpoint{2.505386in}{2.395365in}}%
\pgfpathlineto{\pgfqpoint{2.548288in}{2.448319in}}%
\pgfpathlineto{\pgfqpoint{2.590955in}{2.501578in}}%
\pgfpathlineto{\pgfqpoint{2.633418in}{2.555015in}}%
\pgfusepath{stroke}%
\end{pgfscope}%
\begin{pgfscope}%
\pgfpathrectangle{\pgfqpoint{0.647939in}{0.492442in}}{\pgfqpoint{3.079299in}{3.079299in}}%
\pgfusepath{clip}%
\pgfsetbuttcap%
\pgfsetroundjoin%
\pgfsetlinewidth{0.301125pt}%
\definecolor{currentstroke}{rgb}{0.500000,0.500000,0.500000}%
\pgfsetstrokecolor{currentstroke}%
\pgfsetstrokeopacity{0.300000}%
\pgfsetdash{}{0pt}%
\pgfpathmoveto{\pgfqpoint{0.647939in}{1.682171in}}%
\pgfpathlineto{\pgfqpoint{0.647939in}{1.682171in}}%
\pgfpathlineto{\pgfqpoint{0.715591in}{1.692409in}}%
\pgfpathlineto{\pgfqpoint{0.782934in}{1.704504in}}%
\pgfpathlineto{\pgfqpoint{0.849897in}{1.718550in}}%
\pgfpathlineto{\pgfqpoint{0.916407in}{1.734598in}}%
\pgfpathlineto{\pgfqpoint{0.982401in}{1.752653in}}%
\pgfpathlineto{\pgfqpoint{1.047830in}{1.772662in}}%
\pgfpathlineto{\pgfqpoint{1.112666in}{1.794520in}}%
\pgfpathlineto{\pgfqpoint{1.176910in}{1.818067in}}%
\pgfpathlineto{\pgfqpoint{1.240595in}{1.843090in}}%
\pgfpathlineto{\pgfqpoint{1.303789in}{1.869331in}}%
\pgfpathlineto{\pgfqpoint{1.366594in}{1.896493in}}%
\pgfpathlineto{\pgfqpoint{1.429145in}{1.924238in}}%
\pgfpathlineto{\pgfqpoint{1.491605in}{1.952188in}}%
\pgfpathlineto{\pgfqpoint{1.554162in}{1.979920in}}%
\pgfusepath{stroke}%
\end{pgfscope}%
\begin{pgfscope}%
\pgfpathrectangle{\pgfqpoint{0.647939in}{0.492442in}}{\pgfqpoint{3.079299in}{3.079299in}}%
\pgfusepath{clip}%
\pgfsetbuttcap%
\pgfsetroundjoin%
\pgfsetlinewidth{0.301125pt}%
\definecolor{currentstroke}{rgb}{0.500000,0.500000,0.500000}%
\pgfsetstrokecolor{currentstroke}%
\pgfsetstrokeopacity{0.300000}%
\pgfsetdash{}{0pt}%
\pgfpathmoveto{\pgfqpoint{0.647939in}{1.612187in}}%
\pgfpathlineto{\pgfqpoint{0.647939in}{1.612187in}}%
\pgfpathlineto{\pgfqpoint{0.715566in}{1.622590in}}%
\pgfpathlineto{\pgfqpoint{0.782870in}{1.634899in}}%
\pgfpathlineto{\pgfqpoint{0.849773in}{1.649218in}}%
\pgfpathlineto{\pgfqpoint{0.916200in}{1.665608in}}%
\pgfpathlineto{\pgfqpoint{0.982077in}{1.684080in}}%
\pgfpathlineto{\pgfqpoint{1.047349in}{1.704594in}}%
\pgfpathlineto{\pgfqpoint{1.111979in}{1.727051in}}%
\pgfpathlineto{\pgfqpoint{1.175960in}{1.751300in}}%
\pgfpathlineto{\pgfqpoint{1.239319in}{1.777135in}}%
\pgfpathlineto{\pgfqpoint{1.302118in}{1.804307in}}%
\pgfpathlineto{\pgfqpoint{1.364455in}{1.832524in}}%
\pgfusepath{stroke}%
\end{pgfscope}%
\begin{pgfscope}%
\pgfpathrectangle{\pgfqpoint{0.647939in}{0.492442in}}{\pgfqpoint{3.079299in}{3.079299in}}%
\pgfusepath{clip}%
\pgfsetbuttcap%
\pgfsetroundjoin%
\pgfsetlinewidth{0.301125pt}%
\definecolor{currentstroke}{rgb}{0.500000,0.500000,0.500000}%
\pgfsetstrokecolor{currentstroke}%
\pgfsetstrokeopacity{0.300000}%
\pgfsetdash{}{0pt}%
\pgfpathmoveto{\pgfqpoint{0.647939in}{1.542203in}}%
\pgfpathlineto{\pgfqpoint{0.647939in}{1.542203in}}%
\pgfpathlineto{\pgfqpoint{0.715539in}{1.552777in}}%
\pgfpathlineto{\pgfqpoint{0.782801in}{1.565308in}}%
\pgfpathlineto{\pgfqpoint{0.849643in}{1.579910in}}%
\pgfpathlineto{\pgfqpoint{0.915980in}{1.596654in}}%
\pgfpathlineto{\pgfqpoint{0.981732in}{1.615563in}}%
\pgfpathlineto{\pgfqpoint{1.046835in}{1.636605in}}%
\pgfpathlineto{\pgfqpoint{1.111242in}{1.659690in}}%
\pgfpathlineto{\pgfqpoint{1.174938in}{1.684675in}}%
\pgfpathlineto{\pgfqpoint{1.237941in}{1.711364in}}%
\pgfpathlineto{\pgfqpoint{1.300306in}{1.739515in}}%
\pgfpathlineto{\pgfqpoint{1.362127in}{1.768846in}}%
\pgfpathlineto{\pgfqpoint{1.423534in}{1.799036in}}%
\pgfpathlineto{\pgfqpoint{1.484694in}{1.829726in}}%
\pgfpathlineto{\pgfqpoint{1.545803in}{1.860519in}}%
\pgfpathlineto{\pgfqpoint{1.607081in}{1.890971in}}%
\pgfpathlineto{\pgfqpoint{1.668767in}{1.920584in}}%
\pgfpathlineto{\pgfqpoint{1.731103in}{1.948790in}}%
\pgfpathlineto{\pgfqpoint{1.794320in}{1.974942in}}%
\pgfpathlineto{\pgfqpoint{1.858610in}{1.998300in}}%
\pgfpathlineto{\pgfqpoint{1.924096in}{2.018017in}}%
\pgfpathlineto{\pgfqpoint{1.990770in}{2.033156in}}%
\pgfpathlineto{\pgfqpoint{2.058453in}{2.042583in}}%
\pgfpathlineto{\pgfqpoint{2.126652in}{2.044104in}}%
\pgfpathlineto{\pgfqpoint{2.126652in}{2.044104in}}%
\pgfpathlineto{\pgfqpoint{2.151377in}{2.041397in}}%
\pgfpathlineto{\pgfqpoint{2.151377in}{2.041397in}}%
\pgfusepath{stroke}%
\end{pgfscope}%
\begin{pgfscope}%
\pgfpathrectangle{\pgfqpoint{0.647939in}{0.492442in}}{\pgfqpoint{3.079299in}{3.079299in}}%
\pgfusepath{clip}%
\pgfsetbuttcap%
\pgfsetroundjoin%
\pgfsetlinewidth{0.301125pt}%
\definecolor{currentstroke}{rgb}{0.500000,0.500000,0.500000}%
\pgfsetstrokecolor{currentstroke}%
\pgfsetstrokeopacity{0.300000}%
\pgfsetdash{}{0pt}%
\pgfpathmoveto{\pgfqpoint{0.647939in}{1.472219in}}%
\pgfpathlineto{\pgfqpoint{0.647939in}{1.472219in}}%
\pgfpathlineto{\pgfqpoint{0.715511in}{1.482969in}}%
\pgfpathlineto{\pgfqpoint{0.782729in}{1.495730in}}%
\pgfpathlineto{\pgfqpoint{0.849505in}{1.510626in}}%
\pgfpathlineto{\pgfqpoint{0.915747in}{1.527740in}}%
\pgfpathlineto{\pgfqpoint{0.981365in}{1.547105in}}%
\pgfpathlineto{\pgfqpoint{1.046285in}{1.568699in}}%
\pgfpathlineto{\pgfqpoint{1.110451in}{1.592443in}}%
\pgfpathlineto{\pgfqpoint{1.173836in}{1.618201in}}%
\pgfpathlineto{\pgfqpoint{1.236450in}{1.645787in}}%
\pgfpathlineto{\pgfqpoint{1.298339in}{1.674968in}}%
\pgfpathlineto{\pgfqpoint{1.359589in}{1.705470in}}%
\pgfpathlineto{\pgfqpoint{1.420327in}{1.736983in}}%
\pgfpathlineto{\pgfqpoint{1.480715in}{1.769164in}}%
\pgfpathlineto{\pgfqpoint{1.540950in}{1.801630in}}%
\pgfpathlineto{\pgfqpoint{1.601258in}{1.833961in}}%
\pgfusepath{stroke}%
\end{pgfscope}%
\begin{pgfscope}%
\pgfpathrectangle{\pgfqpoint{0.647939in}{0.492442in}}{\pgfqpoint{3.079299in}{3.079299in}}%
\pgfusepath{clip}%
\pgfsetbuttcap%
\pgfsetroundjoin%
\pgfsetlinewidth{0.301125pt}%
\definecolor{currentstroke}{rgb}{0.500000,0.500000,0.500000}%
\pgfsetstrokecolor{currentstroke}%
\pgfsetstrokeopacity{0.300000}%
\pgfsetdash{}{0pt}%
\pgfpathmoveto{\pgfqpoint{0.647939in}{1.402235in}}%
\pgfpathlineto{\pgfqpoint{0.647939in}{1.402235in}}%
\pgfpathlineto{\pgfqpoint{0.715481in}{1.413167in}}%
\pgfpathlineto{\pgfqpoint{0.782653in}{1.426166in}}%
\pgfpathlineto{\pgfqpoint{0.849360in}{1.441368in}}%
\pgfpathlineto{\pgfqpoint{0.915500in}{1.458867in}}%
\pgfpathlineto{\pgfqpoint{0.980974in}{1.478709in}}%
\pgfpathlineto{\pgfqpoint{1.045697in}{1.500881in}}%
\pgfpathlineto{\pgfqpoint{1.109601in}{1.525316in}}%
\pgfpathlineto{\pgfqpoint{1.172648in}{1.551888in}}%
\pgfusepath{stroke}%
\end{pgfscope}%
\begin{pgfscope}%
\pgfpathrectangle{\pgfqpoint{0.647939in}{0.492442in}}{\pgfqpoint{3.079299in}{3.079299in}}%
\pgfusepath{clip}%
\pgfsetbuttcap%
\pgfsetroundjoin%
\pgfsetlinewidth{0.301125pt}%
\definecolor{currentstroke}{rgb}{0.500000,0.500000,0.500000}%
\pgfsetstrokecolor{currentstroke}%
\pgfsetstrokeopacity{0.300000}%
\pgfsetdash{}{0pt}%
\pgfpathmoveto{\pgfqpoint{0.647939in}{1.332251in}}%
\pgfpathlineto{\pgfqpoint{0.647939in}{1.332251in}}%
\pgfpathlineto{\pgfqpoint{0.715450in}{1.343371in}}%
\pgfpathlineto{\pgfqpoint{0.782573in}{1.356617in}}%
\pgfpathlineto{\pgfqpoint{0.849205in}{1.372137in}}%
\pgfpathlineto{\pgfqpoint{0.915236in}{1.390038in}}%
\pgfpathlineto{\pgfqpoint{0.980557in}{1.410378in}}%
\pgfpathlineto{\pgfqpoint{1.045067in}{1.433156in}}%
\pgfpathlineto{\pgfqpoint{1.108687in}{1.458317in}}%
\pgfpathlineto{\pgfqpoint{1.171365in}{1.485744in}}%
\pgfpathlineto{\pgfqpoint{1.233084in}{1.515268in}}%
\pgfpathlineto{\pgfqpoint{1.293868in}{1.546674in}}%
\pgfpathlineto{\pgfqpoint{1.353785in}{1.579710in}}%
\pgfpathlineto{\pgfqpoint{1.412943in}{1.614089in}}%
\pgfpathlineto{\pgfqpoint{1.471491in}{1.649498in}}%
\pgfpathlineto{\pgfqpoint{1.529619in}{1.685594in}}%
\pgfpathlineto{\pgfqpoint{1.587551in}{1.722004in}}%
\pgfpathlineto{\pgfqpoint{1.645545in}{1.758313in}}%
\pgfpathlineto{\pgfqpoint{1.703891in}{1.794052in}}%
\pgfpathlineto{\pgfqpoint{1.762899in}{1.828685in}}%
\pgfpathlineto{\pgfqpoint{1.822881in}{1.861593in}}%
\pgfpathlineto{\pgfqpoint{1.884133in}{1.892049in}}%
\pgfusepath{stroke}%
\end{pgfscope}%
\begin{pgfscope}%
\pgfpathrectangle{\pgfqpoint{0.647939in}{0.492442in}}{\pgfqpoint{3.079299in}{3.079299in}}%
\pgfusepath{clip}%
\pgfsetbuttcap%
\pgfsetroundjoin%
\pgfsetlinewidth{0.301125pt}%
\definecolor{currentstroke}{rgb}{0.500000,0.500000,0.500000}%
\pgfsetstrokecolor{currentstroke}%
\pgfsetstrokeopacity{0.300000}%
\pgfsetdash{}{0pt}%
\pgfpathmoveto{\pgfqpoint{0.647939in}{1.262267in}}%
\pgfpathlineto{\pgfqpoint{0.647939in}{1.262267in}}%
\pgfpathlineto{\pgfqpoint{0.715417in}{1.273582in}}%
\pgfpathlineto{\pgfqpoint{0.782489in}{1.287084in}}%
\pgfpathlineto{\pgfqpoint{0.849042in}{1.302935in}}%
\pgfpathlineto{\pgfqpoint{0.914956in}{1.321255in}}%
\pgfpathlineto{\pgfqpoint{0.980111in}{1.342116in}}%
\pgfpathlineto{\pgfqpoint{1.044391in}{1.365530in}}%
\pgfpathlineto{\pgfqpoint{1.107702in}{1.391453in}}%
\pgfpathlineto{\pgfqpoint{1.169976in}{1.419778in}}%
\pgfpathlineto{\pgfqpoint{1.231181in}{1.450347in}}%
\pgfpathlineto{\pgfqpoint{1.291327in}{1.482953in}}%
\pgfpathlineto{\pgfqpoint{1.350468in}{1.517352in}}%
\pgfpathlineto{\pgfqpoint{1.408700in}{1.553271in}}%
\pgfpathlineto{\pgfqpoint{1.466164in}{1.590410in}}%
\pgfpathlineto{\pgfqpoint{1.523039in}{1.628445in}}%
\pgfusepath{stroke}%
\end{pgfscope}%
\begin{pgfscope}%
\pgfpathrectangle{\pgfqpoint{0.647939in}{0.492442in}}{\pgfqpoint{3.079299in}{3.079299in}}%
\pgfusepath{clip}%
\pgfsetbuttcap%
\pgfsetroundjoin%
\pgfsetlinewidth{0.301125pt}%
\definecolor{currentstroke}{rgb}{0.500000,0.500000,0.500000}%
\pgfsetstrokecolor{currentstroke}%
\pgfsetstrokeopacity{0.300000}%
\pgfsetdash{}{0pt}%
\pgfpathmoveto{\pgfqpoint{0.647939in}{1.192283in}}%
\pgfpathlineto{\pgfqpoint{0.647939in}{1.192283in}}%
\pgfpathlineto{\pgfqpoint{0.715382in}{1.203799in}}%
\pgfpathlineto{\pgfqpoint{0.782399in}{1.217568in}}%
\pgfpathlineto{\pgfqpoint{0.848868in}{1.233763in}}%
\pgfpathlineto{\pgfqpoint{0.914658in}{1.252521in}}%
\pgfpathlineto{\pgfqpoint{0.979633in}{1.273927in}}%
\pgfusepath{stroke}%
\end{pgfscope}%
\begin{pgfscope}%
\pgfpathrectangle{\pgfqpoint{0.647939in}{0.492442in}}{\pgfqpoint{3.079299in}{3.079299in}}%
\pgfusepath{clip}%
\pgfsetbuttcap%
\pgfsetroundjoin%
\pgfsetlinewidth{0.301125pt}%
\definecolor{currentstroke}{rgb}{0.500000,0.500000,0.500000}%
\pgfsetstrokecolor{currentstroke}%
\pgfsetstrokeopacity{0.300000}%
\pgfsetdash{}{0pt}%
\pgfpathmoveto{\pgfqpoint{0.647939in}{1.122299in}}%
\pgfpathlineto{\pgfqpoint{0.647939in}{1.122299in}}%
\pgfpathlineto{\pgfqpoint{0.715346in}{1.134024in}}%
\pgfpathlineto{\pgfqpoint{0.782304in}{1.148069in}}%
\pgfpathlineto{\pgfqpoint{0.848684in}{1.164624in}}%
\pgfpathlineto{\pgfqpoint{0.914340in}{1.183839in}}%
\pgfpathlineto{\pgfqpoint{0.979122in}{1.205816in}}%
\pgfpathlineto{\pgfqpoint{1.042883in}{1.230597in}}%
\pgfpathlineto{\pgfqpoint{1.105491in}{1.258160in}}%
\pgfpathlineto{\pgfqpoint{1.166841in}{1.288422in}}%
\pgfpathlineto{\pgfqpoint{1.226862in}{1.321239in}}%
\pgfpathlineto{\pgfqpoint{1.285530in}{1.356420in}}%
\pgfpathlineto{\pgfqpoint{1.342867in}{1.393736in}}%
\pgfpathlineto{\pgfqpoint{1.398941in}{1.432928in}}%
\pgfpathlineto{\pgfqpoint{1.453874in}{1.473711in}}%
\pgfpathlineto{\pgfqpoint{1.507840in}{1.515762in}}%
\pgfpathlineto{\pgfqpoint{1.561055in}{1.558759in}}%
\pgfpathlineto{\pgfqpoint{1.613764in}{1.602374in}}%
\pgfpathlineto{\pgfqpoint{1.666270in}{1.646236in}}%
\pgfpathlineto{\pgfqpoint{1.718899in}{1.689933in}}%
\pgfpathlineto{\pgfqpoint{1.772007in}{1.733038in}}%
\pgfpathlineto{\pgfqpoint{1.825970in}{1.775053in}}%
\pgfusepath{stroke}%
\end{pgfscope}%
\begin{pgfscope}%
\pgfpathrectangle{\pgfqpoint{0.647939in}{0.492442in}}{\pgfqpoint{3.079299in}{3.079299in}}%
\pgfusepath{clip}%
\pgfsetbuttcap%
\pgfsetroundjoin%
\pgfsetlinewidth{0.301125pt}%
\definecolor{currentstroke}{rgb}{0.500000,0.500000,0.500000}%
\pgfsetstrokecolor{currentstroke}%
\pgfsetstrokeopacity{0.300000}%
\pgfsetdash{}{0pt}%
\pgfpathmoveto{\pgfqpoint{0.647939in}{1.052315in}}%
\pgfpathlineto{\pgfqpoint{0.647939in}{1.052315in}}%
\pgfpathlineto{\pgfqpoint{0.715307in}{1.064256in}}%
\pgfpathlineto{\pgfqpoint{0.782204in}{1.078588in}}%
\pgfpathlineto{\pgfqpoint{0.848487in}{1.095518in}}%
\pgfpathlineto{\pgfqpoint{0.914000in}{1.115211in}}%
\pgfpathlineto{\pgfqpoint{0.978574in}{1.137787in}}%
\pgfpathlineto{\pgfqpoint{1.042042in}{1.163302in}}%
\pgfpathlineto{\pgfqpoint{1.104250in}{1.191749in}}%
\pgfpathlineto{\pgfqpoint{1.165070in}{1.223051in}}%
\pgfpathlineto{\pgfqpoint{1.224411in}{1.257074in}}%
\pgfpathlineto{\pgfqpoint{1.282227in}{1.293630in}}%
\pgfpathlineto{\pgfqpoint{1.338523in}{1.332492in}}%
\pgfusepath{stroke}%
\end{pgfscope}%
\begin{pgfscope}%
\pgfpathrectangle{\pgfqpoint{0.647939in}{0.492442in}}{\pgfqpoint{3.079299in}{3.079299in}}%
\pgfusepath{clip}%
\pgfsetbuttcap%
\pgfsetroundjoin%
\pgfsetlinewidth{0.301125pt}%
\definecolor{currentstroke}{rgb}{0.500000,0.500000,0.500000}%
\pgfsetstrokecolor{currentstroke}%
\pgfsetstrokeopacity{0.300000}%
\pgfsetdash{}{0pt}%
\pgfpathmoveto{\pgfqpoint{0.647939in}{0.982331in}}%
\pgfpathlineto{\pgfqpoint{0.647939in}{0.982331in}}%
\pgfpathlineto{\pgfqpoint{0.715266in}{0.994496in}}%
\pgfpathlineto{\pgfqpoint{0.782097in}{1.009128in}}%
\pgfpathlineto{\pgfqpoint{0.848278in}{1.026447in}}%
\pgfpathlineto{\pgfqpoint{0.913636in}{1.046641in}}%
\pgfpathlineto{\pgfqpoint{0.977984in}{1.069845in}}%
\pgfpathlineto{\pgfqpoint{1.041133in}{1.096132in}}%
\pgfpathlineto{\pgfqpoint{1.102905in}{1.125505in}}%
\pgfusepath{stroke}%
\end{pgfscope}%
\begin{pgfscope}%
\pgfpathrectangle{\pgfqpoint{0.647939in}{0.492442in}}{\pgfqpoint{3.079299in}{3.079299in}}%
\pgfusepath{clip}%
\pgfsetbuttcap%
\pgfsetroundjoin%
\pgfsetlinewidth{0.301125pt}%
\definecolor{currentstroke}{rgb}{0.500000,0.500000,0.500000}%
\pgfsetstrokecolor{currentstroke}%
\pgfsetstrokeopacity{0.300000}%
\pgfsetdash{}{0pt}%
\pgfpathmoveto{\pgfqpoint{0.647939in}{0.912347in}}%
\pgfpathlineto{\pgfqpoint{0.647939in}{0.912347in}}%
\pgfpathlineto{\pgfqpoint{0.715223in}{0.924745in}}%
\pgfpathlineto{\pgfqpoint{0.781984in}{0.939687in}}%
\pgfpathlineto{\pgfqpoint{0.848056in}{0.957415in}}%
\pgfpathlineto{\pgfqpoint{0.913247in}{0.978133in}}%
\pgfpathlineto{\pgfqpoint{0.977351in}{1.001996in}}%
\pgfpathlineto{\pgfqpoint{1.040153in}{1.029092in}}%
\pgfpathlineto{\pgfqpoint{1.101447in}{1.059439in}}%
\pgfusepath{stroke}%
\end{pgfscope}%
\begin{pgfscope}%
\pgfpathrectangle{\pgfqpoint{0.647939in}{0.492442in}}{\pgfqpoint{3.079299in}{3.079299in}}%
\pgfusepath{clip}%
\pgfsetbuttcap%
\pgfsetroundjoin%
\pgfsetlinewidth{0.301125pt}%
\definecolor{currentstroke}{rgb}{0.500000,0.500000,0.500000}%
\pgfsetstrokecolor{currentstroke}%
\pgfsetstrokeopacity{0.300000}%
\pgfsetdash{}{0pt}%
\pgfpathmoveto{\pgfqpoint{0.647939in}{0.842362in}}%
\pgfpathlineto{\pgfqpoint{0.647939in}{0.842362in}}%
\pgfpathlineto{\pgfqpoint{0.715177in}{0.855002in}}%
\pgfpathlineto{\pgfqpoint{0.781864in}{0.870269in}}%
\pgfpathlineto{\pgfqpoint{0.847818in}{0.888423in}}%
\pgfpathlineto{\pgfqpoint{0.912830in}{0.909690in}}%
\pgfpathlineto{\pgfqpoint{0.976668in}{0.934244in}}%
\pgfpathlineto{\pgfqpoint{1.039092in}{0.962192in}}%
\pgfpathlineto{\pgfqpoint{1.099865in}{0.993560in}}%
\pgfpathlineto{\pgfqpoint{1.158776in}{1.028294in}}%
\pgfpathlineto{\pgfqpoint{1.215657in}{1.066260in}}%
\pgfpathlineto{\pgfqpoint{1.270402in}{1.107256in}}%
\pgfpathlineto{\pgfqpoint{1.322984in}{1.150995in}}%
\pgfpathlineto{\pgfqpoint{1.373457in}{1.197140in}}%
\pgfpathlineto{\pgfqpoint{1.421950in}{1.245372in}}%
\pgfpathlineto{\pgfqpoint{1.468680in}{1.295310in}}%
\pgfpathlineto{\pgfqpoint{1.513923in}{1.346596in}}%
\pgfpathlineto{\pgfqpoint{1.558019in}{1.398872in}}%
\pgfusepath{stroke}%
\end{pgfscope}%
\begin{pgfscope}%
\pgfpathrectangle{\pgfqpoint{0.647939in}{0.492442in}}{\pgfqpoint{3.079299in}{3.079299in}}%
\pgfusepath{clip}%
\pgfsetbuttcap%
\pgfsetroundjoin%
\pgfsetlinewidth{0.301125pt}%
\definecolor{currentstroke}{rgb}{0.500000,0.500000,0.500000}%
\pgfsetstrokecolor{currentstroke}%
\pgfsetstrokeopacity{0.300000}%
\pgfsetdash{}{0pt}%
\pgfpathmoveto{\pgfqpoint{0.647939in}{0.772378in}}%
\pgfpathlineto{\pgfqpoint{0.647939in}{0.772378in}}%
\pgfpathlineto{\pgfqpoint{0.715129in}{0.785269in}}%
\pgfpathlineto{\pgfqpoint{0.781736in}{0.800874in}}%
\pgfpathlineto{\pgfqpoint{0.847563in}{0.819474in}}%
\pgfpathlineto{\pgfqpoint{0.912381in}{0.841317in}}%
\pgfpathlineto{\pgfqpoint{0.975932in}{0.866597in}}%
\pgfpathlineto{\pgfqpoint{1.037943in}{0.895438in}}%
\pgfpathlineto{\pgfqpoint{1.098145in}{0.927876in}}%
\pgfpathlineto{\pgfqpoint{1.156296in}{0.963858in}}%
\pgfpathlineto{\pgfqpoint{1.212200in}{1.003241in}}%
\pgfpathlineto{\pgfqpoint{1.265739in}{1.045792in}}%
\pgfusepath{stroke}%
\end{pgfscope}%
\begin{pgfscope}%
\pgfpathrectangle{\pgfqpoint{0.647939in}{0.492442in}}{\pgfqpoint{3.079299in}{3.079299in}}%
\pgfusepath{clip}%
\pgfsetbuttcap%
\pgfsetroundjoin%
\pgfsetlinewidth{0.301125pt}%
\definecolor{currentstroke}{rgb}{0.500000,0.500000,0.500000}%
\pgfsetstrokecolor{currentstroke}%
\pgfsetstrokeopacity{0.300000}%
\pgfsetdash{}{0pt}%
\pgfpathmoveto{\pgfqpoint{0.647939in}{0.702394in}}%
\pgfpathlineto{\pgfqpoint{0.647939in}{0.702394in}}%
\pgfpathlineto{\pgfqpoint{0.715077in}{0.715546in}}%
\pgfpathlineto{\pgfqpoint{0.781599in}{0.731503in}}%
\pgfpathlineto{\pgfqpoint{0.847291in}{0.750570in}}%
\pgfpathlineto{\pgfqpoint{0.911899in}{0.773017in}}%
\pgfpathlineto{\pgfqpoint{0.975137in}{0.799060in}}%
\pgfpathlineto{\pgfqpoint{1.036697in}{0.828838in}}%
\pgfpathlineto{\pgfqpoint{1.096274in}{0.862398in}}%
\pgfpathlineto{\pgfqpoint{1.153593in}{0.899679in}}%
\pgfpathlineto{\pgfqpoint{1.208435in}{0.940524in}}%
\pgfpathlineto{\pgfqpoint{1.260674in}{0.984652in}}%
\pgfusepath{stroke}%
\end{pgfscope}%
\begin{pgfscope}%
\pgfpathrectangle{\pgfqpoint{0.647939in}{0.492442in}}{\pgfqpoint{3.079299in}{3.079299in}}%
\pgfusepath{clip}%
\pgfsetbuttcap%
\pgfsetroundjoin%
\pgfsetlinewidth{0.301125pt}%
\definecolor{currentstroke}{rgb}{0.500000,0.500000,0.500000}%
\pgfsetstrokecolor{currentstroke}%
\pgfsetstrokeopacity{0.300000}%
\pgfsetdash{}{0pt}%
\pgfpathmoveto{\pgfqpoint{0.647939in}{0.632410in}}%
\pgfpathlineto{\pgfqpoint{0.647939in}{0.632410in}}%
\pgfpathlineto{\pgfqpoint{0.715023in}{0.645834in}}%
\pgfpathlineto{\pgfqpoint{0.781453in}{0.662159in}}%
\pgfpathlineto{\pgfqpoint{0.846999in}{0.681714in}}%
\pgfpathlineto{\pgfqpoint{0.911380in}{0.704795in}}%
\pgfpathlineto{\pgfqpoint{0.974276in}{0.731640in}}%
\pgfpathlineto{\pgfqpoint{1.035344in}{0.762403in}}%
\pgfpathlineto{\pgfqpoint{1.094238in}{0.797133in}}%
\pgfpathlineto{\pgfqpoint{1.150650in}{0.835763in}}%
\pgfusepath{stroke}%
\end{pgfscope}%
\begin{pgfscope}%
\pgfpathrectangle{\pgfqpoint{0.647939in}{0.492442in}}{\pgfqpoint{3.079299in}{3.079299in}}%
\pgfusepath{clip}%
\pgfsetbuttcap%
\pgfsetroundjoin%
\pgfsetlinewidth{0.301125pt}%
\definecolor{currentstroke}{rgb}{0.500000,0.500000,0.500000}%
\pgfsetstrokecolor{currentstroke}%
\pgfsetstrokeopacity{0.300000}%
\pgfsetdash{}{0pt}%
\pgfpathmoveto{\pgfqpoint{3.279569in}{3.256090in}}%
\pgfpathlineto{\pgfqpoint{3.324523in}{3.307670in}}%
\pgfpathlineto{\pgfqpoint{3.370724in}{3.358135in}}%
\pgfpathlineto{\pgfqpoint{3.418213in}{3.407391in}}%
\pgfpathlineto{\pgfqpoint{3.467044in}{3.455316in}}%
\pgfpathlineto{\pgfqpoint{3.517286in}{3.501757in}}%
\pgfpathlineto{\pgfqpoint{3.569005in}{3.546546in}}%
\pgfpathlineto{\pgfqpoint{3.599103in}{3.571741in}}%
\pgfusepath{stroke}%
\end{pgfscope}%
\begin{pgfscope}%
\pgfpathrectangle{\pgfqpoint{0.647939in}{0.492442in}}{\pgfqpoint{3.079299in}{3.079299in}}%
\pgfusepath{clip}%
\pgfsetbuttcap%
\pgfsetroundjoin%
\pgfsetlinewidth{0.301125pt}%
\definecolor{currentstroke}{rgb}{0.500000,0.500000,0.500000}%
\pgfsetstrokecolor{currentstroke}%
\pgfsetstrokeopacity{0.300000}%
\pgfsetdash{}{0pt}%
\pgfpathmoveto{\pgfqpoint{1.830274in}{0.599998in}}%
\pgfpathlineto{\pgfqpoint{1.763129in}{0.612845in}}%
\pgfpathlineto{\pgfqpoint{1.697700in}{0.632410in}}%
\pgfpathlineto{\pgfqpoint{1.635855in}{0.661020in}}%
\pgfpathlineto{\pgfqpoint{1.581046in}{0.701086in}}%
\pgfpathlineto{\pgfqpoint{1.541167in}{0.748458in}}%
\pgfusepath{stroke}%
\end{pgfscope}%
\begin{pgfscope}%
\pgfpathrectangle{\pgfqpoint{0.647939in}{0.492442in}}{\pgfqpoint{3.079299in}{3.079299in}}%
\pgfusepath{clip}%
\pgfsetbuttcap%
\pgfsetroundjoin%
\pgfsetlinewidth{0.301125pt}%
\definecolor{currentstroke}{rgb}{0.500000,0.500000,0.500000}%
\pgfsetstrokecolor{currentstroke}%
\pgfsetstrokeopacity{0.300000}%
\pgfsetdash{}{0pt}%
\pgfpathmoveto{\pgfqpoint{2.659170in}{0.603559in}}%
\pgfpathlineto{\pgfqpoint{2.591253in}{0.611827in}}%
\pgfpathlineto{\pgfqpoint{2.523114in}{0.618027in}}%
\pgfpathlineto{\pgfqpoint{2.454826in}{0.622299in}}%
\pgfpathlineto{\pgfqpoint{2.386450in}{0.624874in}}%
\pgfpathlineto{\pgfqpoint{2.318035in}{0.626076in}}%
\pgfpathlineto{\pgfqpoint{2.249608in}{0.626303in}}%
\pgfpathlineto{\pgfqpoint{2.181179in}{0.626028in}}%
\pgfpathlineto{\pgfqpoint{2.112751in}{0.625797in}}%
\pgfpathlineto{\pgfqpoint{2.044326in}{0.626254in}}%
\pgfpathlineto{\pgfqpoint{1.975931in}{0.628157in}}%
\pgfpathlineto{\pgfqpoint{1.907652in}{0.632410in}}%
\pgfusepath{stroke}%
\end{pgfscope}%
\begin{pgfscope}%
\pgfpathrectangle{\pgfqpoint{0.647939in}{0.492442in}}{\pgfqpoint{3.079299in}{3.079299in}}%
\pgfusepath{clip}%
\pgfsetbuttcap%
\pgfsetroundjoin%
\pgfsetlinewidth{0.301125pt}%
\definecolor{currentstroke}{rgb}{0.500000,0.500000,0.500000}%
\pgfsetstrokecolor{currentstroke}%
\pgfsetstrokeopacity{0.300000}%
\pgfsetdash{}{0pt}%
\pgfpathmoveto{\pgfqpoint{3.727238in}{1.462915in}}%
\pgfpathlineto{\pgfqpoint{3.705906in}{1.474083in}}%
\pgfpathlineto{\pgfqpoint{3.645938in}{1.507008in}}%
\pgfpathlineto{\pgfqpoint{3.587270in}{1.542203in}}%
\pgfpathlineto{\pgfqpoint{3.529876in}{1.579444in}}%
\pgfpathlineto{\pgfqpoint{3.473719in}{1.618530in}}%
\pgfpathlineto{\pgfqpoint{3.418753in}{1.659274in}}%
\pgfpathlineto{\pgfqpoint{3.364923in}{1.701510in}}%
\pgfpathlineto{\pgfqpoint{3.312184in}{1.745100in}}%
\pgfpathlineto{\pgfqpoint{3.260507in}{1.789945in}}%
\pgfpathlineto{\pgfqpoint{3.209872in}{1.835965in}}%
\pgfpathlineto{\pgfqpoint{3.160288in}{1.883112in}}%
\pgfpathlineto{\pgfqpoint{3.111805in}{1.931390in}}%
\pgfpathlineto{\pgfqpoint{3.064524in}{1.980844in}}%
\pgfpathlineto{\pgfqpoint{3.018630in}{2.031583in}}%
\pgfpathlineto{\pgfqpoint{2.974429in}{2.083798in}}%
\pgfpathlineto{\pgfqpoint{2.932415in}{2.137780in}}%
\pgfpathlineto{\pgfqpoint{2.893388in}{2.193938in}}%
\pgfpathlineto{\pgfqpoint{2.858612in}{2.252780in}}%
\pgfpathlineto{\pgfqpoint{2.830007in}{2.314786in}}%
\pgfpathlineto{\pgfqpoint{2.810145in}{2.380018in}}%
\pgfpathlineto{\pgfqpoint{2.801469in}{2.447573in}}%
\pgfpathlineto{\pgfqpoint{2.804786in}{2.515615in}}%
\pgfpathlineto{\pgfqpoint{2.818643in}{2.582388in}}%
\pgfpathlineto{\pgfqpoint{2.840485in}{2.647072in}}%
\pgfpathlineto{\pgfqpoint{2.867972in}{2.709627in}}%
\pgfusepath{stroke}%
\end{pgfscope}%
\begin{pgfscope}%
\pgfpathrectangle{\pgfqpoint{0.647939in}{0.492442in}}{\pgfqpoint{3.079299in}{3.079299in}}%
\pgfusepath{clip}%
\pgfsetbuttcap%
\pgfsetroundjoin%
\pgfsetlinewidth{0.301125pt}%
\definecolor{currentstroke}{rgb}{0.500000,0.500000,0.500000}%
\pgfsetstrokecolor{currentstroke}%
\pgfsetstrokeopacity{0.300000}%
\pgfsetdash{}{0pt}%
\pgfpathmoveto{\pgfqpoint{3.386690in}{2.518290in}}%
\pgfpathlineto{\pgfqpoint{3.378233in}{2.586115in}}%
\pgfpathlineto{\pgfqpoint{3.375928in}{2.654423in}}%
\pgfpathlineto{\pgfqpoint{3.379736in}{2.722667in}}%
\pgfpathlineto{\pgfqpoint{3.389427in}{2.790337in}}%
\pgfpathlineto{\pgfqpoint{3.404627in}{2.857000in}}%
\pgfpathlineto{\pgfqpoint{3.424884in}{2.922311in}}%
\pgfpathlineto{\pgfqpoint{3.449724in}{2.986025in}}%
\pgfpathlineto{\pgfqpoint{3.478716in}{3.047966in}}%
\pgfpathlineto{\pgfqpoint{3.511494in}{3.107994in}}%
\pgfpathlineto{\pgfqpoint{3.547756in}{3.165990in}}%
\pgfpathlineto{\pgfqpoint{3.587270in}{3.221821in}}%
\pgfusepath{stroke}%
\end{pgfscope}%
\begin{pgfscope}%
\pgfpathrectangle{\pgfqpoint{0.647939in}{0.492442in}}{\pgfqpoint{3.079299in}{3.079299in}}%
\pgfusepath{clip}%
\pgfsetbuttcap%
\pgfsetroundjoin%
\pgfsetlinewidth{0.301125pt}%
\definecolor{currentstroke}{rgb}{0.500000,0.500000,0.500000}%
\pgfsetstrokecolor{currentstroke}%
\pgfsetstrokeopacity{0.300000}%
\pgfsetdash{}{0pt}%
\pgfpathmoveto{\pgfqpoint{3.517286in}{1.752155in}}%
\pgfpathlineto{\pgfqpoint{3.464822in}{1.796065in}}%
\pgfpathlineto{\pgfqpoint{3.414030in}{1.841901in}}%
\pgfpathlineto{\pgfqpoint{3.364947in}{1.889562in}}%
\pgfpathlineto{\pgfqpoint{3.317641in}{1.938985in}}%
\pgfpathlineto{\pgfqpoint{3.272230in}{1.990154in}}%
\pgfpathlineto{\pgfqpoint{3.228899in}{2.043094in}}%
\pgfpathlineto{\pgfqpoint{3.187921in}{2.097868in}}%
\pgfpathlineto{\pgfqpoint{3.149682in}{2.154579in}}%
\pgfpathlineto{\pgfqpoint{3.114729in}{2.213361in}}%
\pgfpathlineto{\pgfqpoint{3.083789in}{2.274336in}}%
\pgfpathlineto{\pgfqpoint{3.057789in}{2.337553in}}%
\pgfpathlineto{\pgfqpoint{3.037792in}{2.402890in}}%
\pgfpathlineto{\pgfqpoint{3.024834in}{2.469950in}}%
\pgfpathlineto{\pgfqpoint{3.019629in}{2.538040in}}%
\pgfpathlineto{\pgfqpoint{3.022312in}{2.606288in}}%
\pgfpathlineto{\pgfqpoint{3.032382in}{2.673862in}}%
\pgfpathlineto{\pgfqpoint{3.048897in}{2.740175in}}%
\pgfpathlineto{\pgfqpoint{3.070790in}{2.804933in}}%
\pgfpathlineto{\pgfqpoint{3.097072in}{2.868055in}}%
\pgfusepath{stroke}%
\end{pgfscope}%
\begin{pgfscope}%
\pgfpathrectangle{\pgfqpoint{0.647939in}{0.492442in}}{\pgfqpoint{3.079299in}{3.079299in}}%
\pgfusepath{clip}%
\pgfsetbuttcap%
\pgfsetroundjoin%
\pgfsetlinewidth{0.301125pt}%
\definecolor{currentstroke}{rgb}{0.500000,0.500000,0.500000}%
\pgfsetstrokecolor{currentstroke}%
\pgfsetstrokeopacity{0.300000}%
\pgfsetdash{}{0pt}%
\pgfpathmoveto{\pgfqpoint{3.447302in}{1.472219in}}%
\pgfpathlineto{\pgfqpoint{3.390001in}{1.509615in}}%
\pgfpathlineto{\pgfqpoint{3.333442in}{1.548126in}}%
\pgfpathlineto{\pgfqpoint{3.277526in}{1.587566in}}%
\pgfpathlineto{\pgfqpoint{3.222152in}{1.627763in}}%
\pgfpathlineto{\pgfqpoint{3.167216in}{1.668559in}}%
\pgfusepath{stroke}%
\end{pgfscope}%
\begin{pgfscope}%
\pgfpathrectangle{\pgfqpoint{0.647939in}{0.492442in}}{\pgfqpoint{3.079299in}{3.079299in}}%
\pgfusepath{clip}%
\pgfsetbuttcap%
\pgfsetroundjoin%
\pgfsetlinewidth{0.301125pt}%
\definecolor{currentstroke}{rgb}{0.500000,0.500000,0.500000}%
\pgfsetstrokecolor{currentstroke}%
\pgfsetstrokeopacity{0.300000}%
\pgfsetdash{}{0pt}%
\pgfpathmoveto{\pgfqpoint{1.988792in}{3.213106in}}%
\pgfpathlineto{\pgfqpoint{2.057204in}{3.214479in}}%
\pgfpathlineto{\pgfqpoint{2.125630in}{3.214741in}}%
\pgfpathlineto{\pgfqpoint{2.194058in}{3.214410in}}%
\pgfpathlineto{\pgfqpoint{2.262486in}{3.214131in}}%
\pgfpathlineto{\pgfqpoint{2.330910in}{3.214683in}}%
\pgfpathlineto{\pgfqpoint{2.399291in}{3.216941in}}%
\pgfpathlineto{\pgfqpoint{2.467525in}{3.221821in}}%
\pgfusepath{stroke}%
\end{pgfscope}%
\begin{pgfscope}%
\pgfpathrectangle{\pgfqpoint{0.647939in}{0.492442in}}{\pgfqpoint{3.079299in}{3.079299in}}%
\pgfusepath{clip}%
\pgfsetbuttcap%
\pgfsetroundjoin%
\pgfsetlinewidth{0.301125pt}%
\definecolor{currentstroke}{rgb}{0.500000,0.500000,0.500000}%
\pgfsetstrokecolor{currentstroke}%
\pgfsetstrokeopacity{0.300000}%
\pgfsetdash{}{0pt}%
\pgfpathmoveto{\pgfqpoint{3.385089in}{2.059569in}}%
\pgfpathlineto{\pgfqpoint{3.344670in}{2.114750in}}%
\pgfpathlineto{\pgfqpoint{3.307334in}{2.172060in}}%
\pgfpathlineto{\pgfqpoint{3.273504in}{2.231501in}}%
\pgfpathlineto{\pgfqpoint{3.243713in}{2.293054in}}%
\pgfpathlineto{\pgfqpoint{3.218589in}{2.356644in}}%
\pgfpathlineto{\pgfqpoint{3.198817in}{2.422086in}}%
\pgfpathlineto{\pgfqpoint{3.185050in}{2.489033in}}%
\pgfpathlineto{\pgfqpoint{3.177774in}{2.556977in}}%
\pgfpathlineto{\pgfqpoint{3.177189in}{2.625302in}}%
\pgfpathlineto{\pgfqpoint{3.183150in}{2.693377in}}%
\pgfpathlineto{\pgfqpoint{3.195210in}{2.760657in}}%
\pgfpathlineto{\pgfqpoint{3.212733in}{2.826742in}}%
\pgfpathlineto{\pgfqpoint{3.235032in}{2.891385in}}%
\pgfpathlineto{\pgfqpoint{3.261454in}{2.954464in}}%
\pgfpathlineto{\pgfqpoint{3.291443in}{3.015936in}}%
\pgfusepath{stroke}%
\end{pgfscope}%
\begin{pgfscope}%
\pgfpathrectangle{\pgfqpoint{0.647939in}{0.492442in}}{\pgfqpoint{3.079299in}{3.079299in}}%
\pgfusepath{clip}%
\pgfsetbuttcap%
\pgfsetroundjoin%
\pgfsetlinewidth{0.301125pt}%
\definecolor{currentstroke}{rgb}{0.500000,0.500000,0.500000}%
\pgfsetstrokecolor{currentstroke}%
\pgfsetstrokeopacity{0.300000}%
\pgfsetdash{}{0pt}%
\pgfpathmoveto{\pgfqpoint{2.454569in}{0.968362in}}%
\pgfpathlineto{\pgfqpoint{2.386226in}{0.971628in}}%
\pgfpathlineto{\pgfqpoint{2.317819in}{0.973164in}}%
\pgfpathlineto{\pgfqpoint{2.249393in}{0.973454in}}%
\pgfpathlineto{\pgfqpoint{2.180965in}{0.973088in}}%
\pgfpathlineto{\pgfqpoint{2.112537in}{0.972782in}}%
\pgfpathlineto{\pgfqpoint{2.044116in}{0.973419in}}%
\pgfpathlineto{\pgfqpoint{1.975756in}{0.976115in}}%
\pgfpathlineto{\pgfqpoint{1.907652in}{0.982331in}}%
\pgfusepath{stroke}%
\end{pgfscope}%
\begin{pgfscope}%
\pgfpathrectangle{\pgfqpoint{0.647939in}{0.492442in}}{\pgfqpoint{3.079299in}{3.079299in}}%
\pgfusepath{clip}%
\pgfsetbuttcap%
\pgfsetroundjoin%
\pgfsetlinewidth{0.301125pt}%
\definecolor{currentstroke}{rgb}{0.500000,0.500000,0.500000}%
\pgfsetstrokecolor{currentstroke}%
\pgfsetstrokeopacity{0.300000}%
\pgfsetdash{}{0pt}%
\pgfpathmoveto{\pgfqpoint{2.844950in}{2.877905in}}%
\pgfpathlineto{\pgfqpoint{2.890992in}{2.928492in}}%
\pgfpathlineto{\pgfqpoint{2.936602in}{2.979483in}}%
\pgfpathlineto{\pgfqpoint{2.982008in}{3.030660in}}%
\pgfpathlineto{\pgfqpoint{3.027398in}{3.081853in}}%
\pgfpathlineto{\pgfqpoint{3.072926in}{3.132926in}}%
\pgfpathlineto{\pgfqpoint{3.118727in}{3.183757in}}%
\pgfusepath{stroke}%
\end{pgfscope}%
\begin{pgfscope}%
\pgfpathrectangle{\pgfqpoint{0.647939in}{0.492442in}}{\pgfqpoint{3.079299in}{3.079299in}}%
\pgfusepath{clip}%
\pgfsetbuttcap%
\pgfsetroundjoin%
\pgfsetlinewidth{0.301125pt}%
\definecolor{currentstroke}{rgb}{0.500000,0.500000,0.500000}%
\pgfsetstrokecolor{currentstroke}%
\pgfsetstrokeopacity{0.300000}%
\pgfsetdash{}{0pt}%
\pgfpathmoveto{\pgfqpoint{1.137828in}{1.962108in}}%
\pgfpathlineto{\pgfqpoint{1.202449in}{1.984603in}}%
\pgfpathlineto{\pgfqpoint{1.266658in}{2.008252in}}%
\pgfpathlineto{\pgfqpoint{1.330542in}{2.032770in}}%
\pgfpathlineto{\pgfqpoint{1.394216in}{2.057830in}}%
\pgfpathlineto{\pgfqpoint{1.457822in}{2.083063in}}%
\pgfpathlineto{\pgfqpoint{1.521521in}{2.108061in}}%
\pgfpathlineto{\pgfqpoint{1.585483in}{2.132374in}}%
\pgfpathlineto{\pgfqpoint{1.649876in}{2.155510in}}%
\pgfpathlineto{\pgfqpoint{1.714853in}{2.176938in}}%
\pgfpathlineto{\pgfqpoint{1.780528in}{2.196099in}}%
\pgfpathlineto{\pgfqpoint{1.846958in}{2.212422in}}%
\pgfpathlineto{\pgfqpoint{1.914119in}{2.225374in}}%
\pgfpathlineto{\pgfqpoint{1.981900in}{2.234521in}}%
\pgfpathlineto{\pgfqpoint{2.050104in}{2.239643in}}%
\pgfpathlineto{\pgfqpoint{2.118495in}{2.240927in}}%
\pgfpathlineto{\pgfqpoint{2.186892in}{2.239257in}}%
\pgfpathlineto{\pgfqpoint{2.255282in}{2.237232in}}%
\pgfusepath{stroke}%
\end{pgfscope}%
\begin{pgfscope}%
\pgfpathrectangle{\pgfqpoint{0.647939in}{0.492442in}}{\pgfqpoint{3.079299in}{3.079299in}}%
\pgfusepath{clip}%
\pgfsetbuttcap%
\pgfsetroundjoin%
\pgfsetlinewidth{0.301125pt}%
\definecolor{currentstroke}{rgb}{0.500000,0.500000,0.500000}%
\pgfsetstrokecolor{currentstroke}%
\pgfsetstrokeopacity{0.300000}%
\pgfsetdash{}{0pt}%
\pgfpathmoveto{\pgfqpoint{3.097382in}{1.752155in}}%
\pgfpathlineto{\pgfqpoint{3.043836in}{1.794758in}}%
\pgfpathlineto{\pgfqpoint{2.990495in}{1.837616in}}%
\pgfpathlineto{\pgfqpoint{2.937286in}{1.880633in}}%
\pgfpathlineto{\pgfqpoint{2.884154in}{1.923747in}}%
\pgfpathlineto{\pgfqpoint{2.831073in}{1.966923in}}%
\pgfpathlineto{\pgfqpoint{2.778069in}{2.010184in}}%
\pgfpathlineto{\pgfqpoint{2.725246in}{2.053663in}}%
\pgfpathlineto{\pgfqpoint{2.672888in}{2.097690in}}%
\pgfpathlineto{\pgfqpoint{2.621683in}{2.143041in}}%
\pgfpathlineto{\pgfqpoint{2.573464in}{2.191434in}}%
\pgfpathlineto{\pgfqpoint{2.534062in}{2.246642in}}%
\pgfpathlineto{\pgfqpoint{2.534062in}{2.246642in}}%
\pgfpathlineto{\pgfqpoint{2.520448in}{2.284855in}}%
\pgfpathlineto{\pgfqpoint{2.520037in}{2.327458in}}%
\pgfusepath{stroke}%
\end{pgfscope}%
\begin{pgfscope}%
\pgfpathrectangle{\pgfqpoint{0.647939in}{0.492442in}}{\pgfqpoint{3.079299in}{3.079299in}}%
\pgfusepath{clip}%
\pgfsetbuttcap%
\pgfsetroundjoin%
\pgfsetlinewidth{0.301125pt}%
\definecolor{currentstroke}{rgb}{0.500000,0.500000,0.500000}%
\pgfsetstrokecolor{currentstroke}%
\pgfsetstrokeopacity{0.300000}%
\pgfsetdash{}{0pt}%
\pgfpathmoveto{\pgfqpoint{1.859924in}{2.887213in}}%
\pgfpathlineto{\pgfqpoint{1.928118in}{2.892770in}}%
\pgfpathlineto{\pgfqpoint{1.996451in}{2.896215in}}%
\pgfpathlineto{\pgfqpoint{2.064855in}{2.897816in}}%
\pgfpathlineto{\pgfqpoint{2.133281in}{2.898033in}}%
\pgfpathlineto{\pgfqpoint{2.201708in}{2.897530in}}%
\pgfpathlineto{\pgfqpoint{2.270135in}{2.897214in}}%
\pgfpathlineto{\pgfqpoint{2.338548in}{2.898226in}}%
\pgfpathlineto{\pgfqpoint{2.406853in}{2.901894in}}%
\pgfpathlineto{\pgfqpoint{2.474794in}{2.909629in}}%
\pgfpathlineto{\pgfqpoint{2.541891in}{2.922698in}}%
\pgfpathlineto{\pgfqpoint{2.607493in}{2.941885in}}%
\pgfpathlineto{\pgfqpoint{2.670947in}{2.967259in}}%
\pgfpathlineto{\pgfqpoint{2.731917in}{2.998163in}}%
\pgfusepath{stroke}%
\end{pgfscope}%
\begin{pgfscope}%
\pgfpathrectangle{\pgfqpoint{0.647939in}{0.492442in}}{\pgfqpoint{3.079299in}{3.079299in}}%
\pgfusepath{clip}%
\pgfsetbuttcap%
\pgfsetroundjoin%
\pgfsetlinewidth{0.301125pt}%
\definecolor{currentstroke}{rgb}{0.500000,0.500000,0.500000}%
\pgfsetstrokecolor{currentstroke}%
\pgfsetstrokeopacity{0.300000}%
\pgfsetdash{}{0pt}%
\pgfpathmoveto{\pgfqpoint{1.987097in}{2.936832in}}%
\pgfpathlineto{\pgfqpoint{2.055498in}{2.938588in}}%
\pgfpathlineto{\pgfqpoint{2.123923in}{2.938945in}}%
\pgfpathlineto{\pgfqpoint{2.192351in}{2.938523in}}%
\pgfpathlineto{\pgfqpoint{2.260778in}{2.938147in}}%
\pgfpathlineto{\pgfqpoint{2.329198in}{2.938854in}}%
\pgfpathlineto{\pgfqpoint{2.397541in}{2.941885in}}%
\pgfusepath{stroke}%
\end{pgfscope}%
\begin{pgfscope}%
\pgfpathrectangle{\pgfqpoint{0.647939in}{0.492442in}}{\pgfqpoint{3.079299in}{3.079299in}}%
\pgfusepath{clip}%
\pgfsetbuttcap%
\pgfsetroundjoin%
\pgfsetlinewidth{0.301125pt}%
\definecolor{currentstroke}{rgb}{0.500000,0.500000,0.500000}%
\pgfsetstrokecolor{currentstroke}%
\pgfsetstrokeopacity{0.300000}%
\pgfsetdash{}{0pt}%
\pgfpathmoveto{\pgfqpoint{1.179784in}{1.144659in}}%
\pgfpathlineto{\pgfqpoint{1.237574in}{1.181261in}}%
\pgfpathlineto{\pgfqpoint{1.293571in}{1.220552in}}%
\pgfpathlineto{\pgfqpoint{1.347780in}{1.262267in}}%
\pgfpathlineto{\pgfqpoint{1.400256in}{1.306140in}}%
\pgfpathlineto{\pgfqpoint{1.451139in}{1.351857in}}%
\pgfpathlineto{\pgfqpoint{1.500628in}{1.399076in}}%
\pgfpathlineto{\pgfqpoint{1.548965in}{1.447480in}}%
\pgfusepath{stroke}%
\end{pgfscope}%
\begin{pgfscope}%
\pgfpathrectangle{\pgfqpoint{0.647939in}{0.492442in}}{\pgfqpoint{3.079299in}{3.079299in}}%
\pgfusepath{clip}%
\pgfsetbuttcap%
\pgfsetroundjoin%
\pgfsetlinewidth{0.301125pt}%
\definecolor{currentstroke}{rgb}{0.500000,0.500000,0.500000}%
\pgfsetstrokecolor{currentstroke}%
\pgfsetstrokeopacity{0.300000}%
\pgfsetdash{}{0pt}%
\pgfpathmoveto{\pgfqpoint{2.453891in}{1.241905in}}%
\pgfpathlineto{\pgfqpoint{2.385598in}{1.246035in}}%
\pgfpathlineto{\pgfqpoint{2.317205in}{1.247997in}}%
\pgfpathlineto{\pgfqpoint{2.248780in}{1.248361in}}%
\pgfpathlineto{\pgfqpoint{2.180354in}{1.247865in}}%
\pgfpathlineto{\pgfqpoint{2.111927in}{1.247456in}}%
\pgfpathlineto{\pgfqpoint{2.043513in}{1.248389in}}%
\pgfpathlineto{\pgfqpoint{1.975242in}{1.252430in}}%
\pgfpathlineto{\pgfqpoint{1.907652in}{1.262267in}}%
\pgfusepath{stroke}%
\end{pgfscope}%
\begin{pgfscope}%
\pgfpathrectangle{\pgfqpoint{0.647939in}{0.492442in}}{\pgfqpoint{3.079299in}{3.079299in}}%
\pgfusepath{clip}%
\pgfsetbuttcap%
\pgfsetroundjoin%
\pgfsetlinewidth{0.301125pt}%
\definecolor{currentstroke}{rgb}{0.500000,0.500000,0.500000}%
\pgfsetstrokecolor{currentstroke}%
\pgfsetstrokeopacity{0.300000}%
\pgfsetdash{}{0pt}%
\pgfpathmoveto{\pgfqpoint{1.417764in}{2.382012in}}%
\pgfpathlineto{\pgfqpoint{1.482975in}{2.402745in}}%
\pgfpathlineto{\pgfqpoint{1.548411in}{2.422756in}}%
\pgfpathlineto{\pgfqpoint{1.614185in}{2.441617in}}%
\pgfpathlineto{\pgfqpoint{1.680390in}{2.458890in}}%
\pgfpathlineto{\pgfqpoint{1.747085in}{2.474148in}}%
\pgfpathlineto{\pgfqpoint{1.814281in}{2.486999in}}%
\pgfpathlineto{\pgfqpoint{1.881937in}{2.497130in}}%
\pgfpathlineto{\pgfqpoint{1.949964in}{2.504359in}}%
\pgfpathlineto{\pgfqpoint{2.018240in}{2.508698in}}%
\pgfpathlineto{\pgfqpoint{2.086635in}{2.510448in}}%
\pgfpathlineto{\pgfqpoint{2.155059in}{2.510268in}}%
\pgfpathlineto{\pgfqpoint{2.223481in}{2.509281in}}%
\pgfpathlineto{\pgfqpoint{2.291901in}{2.509282in}}%
\pgfpathlineto{\pgfqpoint{2.360172in}{2.513005in}}%
\pgfpathlineto{\pgfqpoint{2.427506in}{2.524057in}}%
\pgfpathlineto{\pgfqpoint{2.492132in}{2.545412in}}%
\pgfpathlineto{\pgfqpoint{2.552362in}{2.577120in}}%
\pgfusepath{stroke}%
\end{pgfscope}%
\begin{pgfscope}%
\pgfpathrectangle{\pgfqpoint{0.647939in}{0.492442in}}{\pgfqpoint{3.079299in}{3.079299in}}%
\pgfusepath{clip}%
\pgfsetbuttcap%
\pgfsetroundjoin%
\pgfsetlinewidth{0.301125pt}%
\definecolor{currentstroke}{rgb}{0.500000,0.500000,0.500000}%
\pgfsetstrokecolor{currentstroke}%
\pgfsetstrokeopacity{0.300000}%
\pgfsetdash{}{0pt}%
\pgfpathmoveto{\pgfqpoint{2.887429in}{2.312028in}}%
\pgfpathlineto{\pgfqpoint{2.866229in}{2.376905in}}%
\pgfpathlineto{\pgfqpoint{2.854532in}{2.444082in}}%
\pgfpathlineto{\pgfqpoint{2.853450in}{2.512251in}}%
\pgfpathlineto{\pgfqpoint{2.862496in}{2.579875in}}%
\pgfpathlineto{\pgfqpoint{2.879942in}{2.645883in}}%
\pgfusepath{stroke}%
\end{pgfscope}%
\begin{pgfscope}%
\pgfpathrectangle{\pgfqpoint{0.647939in}{0.492442in}}{\pgfqpoint{3.079299in}{3.079299in}}%
\pgfusepath{clip}%
\pgfsetbuttcap%
\pgfsetroundjoin%
\pgfsetlinewidth{0.301125pt}%
\definecolor{currentstroke}{rgb}{0.500000,0.500000,0.500000}%
\pgfsetstrokecolor{currentstroke}%
\pgfsetstrokeopacity{0.300000}%
\pgfsetdash{}{0pt}%
\pgfpathmoveto{\pgfqpoint{2.817445in}{1.892124in}}%
\pgfpathlineto{\pgfqpoint{2.760945in}{1.930714in}}%
\pgfpathlineto{\pgfqpoint{2.703794in}{1.968326in}}%
\pgfpathlineto{\pgfqpoint{2.645890in}{2.004754in}}%
\pgfpathlineto{\pgfqpoint{2.587155in}{2.039817in}}%
\pgfpathlineto{\pgfqpoint{2.527579in}{2.073419in}}%
\pgfpathlineto{\pgfqpoint{2.467318in}{2.105771in}}%
\pgfpathlineto{\pgfqpoint{2.407216in}{2.138369in}}%
\pgfpathlineto{\pgfqpoint{2.407216in}{2.138369in}}%
\pgfpathlineto{\pgfqpoint{2.373393in}{2.160649in}}%
\pgfpathlineto{\pgfqpoint{2.373393in}{2.160649in}}%
\pgfusepath{stroke}%
\end{pgfscope}%
\begin{pgfscope}%
\pgfpathrectangle{\pgfqpoint{0.647939in}{0.492442in}}{\pgfqpoint{3.079299in}{3.079299in}}%
\pgfusepath{clip}%
\pgfsetbuttcap%
\pgfsetroundjoin%
\pgfsetlinewidth{0.301125pt}%
\definecolor{currentstroke}{rgb}{0.500000,0.500000,0.500000}%
\pgfsetstrokecolor{currentstroke}%
\pgfsetstrokeopacity{0.300000}%
\pgfsetdash{}{0pt}%
\pgfpathmoveto{\pgfqpoint{2.919474in}{1.940904in}}%
\pgfpathlineto{\pgfqpoint{2.868213in}{1.986225in}}%
\pgfpathlineto{\pgfqpoint{2.817445in}{2.032092in}}%
\pgfpathlineto{\pgfqpoint{2.767415in}{2.078760in}}%
\pgfpathlineto{\pgfqpoint{2.718645in}{2.126725in}}%
\pgfpathlineto{\pgfqpoint{2.672222in}{2.176928in}}%
\pgfpathlineto{\pgfqpoint{2.630665in}{2.231100in}}%
\pgfpathlineto{\pgfqpoint{2.600039in}{2.291641in}}%
\pgfpathlineto{\pgfqpoint{2.600039in}{2.291641in}}%
\pgfpathlineto{\pgfqpoint{2.590174in}{2.338549in}}%
\pgfpathlineto{\pgfqpoint{2.592905in}{2.387091in}}%
\pgfpathlineto{\pgfqpoint{2.605641in}{2.433540in}}%
\pgfpathlineto{\pgfqpoint{2.628453in}{2.485450in}}%
\pgfusepath{stroke}%
\end{pgfscope}%
\begin{pgfscope}%
\pgfpathrectangle{\pgfqpoint{0.647939in}{0.492442in}}{\pgfqpoint{3.079299in}{3.079299in}}%
\pgfusepath{clip}%
\pgfsetbuttcap%
\pgfsetroundjoin%
\pgfsetlinewidth{0.301125pt}%
\definecolor{currentstroke}{rgb}{0.500000,0.500000,0.500000}%
\pgfsetstrokecolor{currentstroke}%
\pgfsetstrokeopacity{0.300000}%
\pgfsetdash{}{0pt}%
\pgfpathmoveto{\pgfqpoint{1.873132in}{2.562061in}}%
\pgfpathlineto{\pgfqpoint{1.941181in}{2.569104in}}%
\pgfpathlineto{\pgfqpoint{2.009458in}{2.573426in}}%
\pgfpathlineto{\pgfqpoint{2.077851in}{2.575289in}}%
\pgfpathlineto{\pgfqpoint{2.146276in}{2.575292in}}%
\pgfpathlineto{\pgfqpoint{2.214699in}{2.574434in}}%
\pgfpathlineto{\pgfqpoint{2.283122in}{2.574213in}}%
\pgfpathlineto{\pgfqpoint{2.351468in}{2.576774in}}%
\pgfpathlineto{\pgfqpoint{2.419316in}{2.584892in}}%
\pgfpathlineto{\pgfqpoint{2.485559in}{2.601340in}}%
\pgfpathlineto{\pgfqpoint{2.548609in}{2.627389in}}%
\pgfpathlineto{\pgfqpoint{2.607493in}{2.661948in}}%
\pgfpathlineto{\pgfqpoint{2.662313in}{2.702711in}}%
\pgfusepath{stroke}%
\end{pgfscope}%
\begin{pgfscope}%
\pgfpathrectangle{\pgfqpoint{0.647939in}{0.492442in}}{\pgfqpoint{3.079299in}{3.079299in}}%
\pgfusepath{clip}%
\pgfsetbuttcap%
\pgfsetroundjoin%
\pgfsetlinewidth{0.301125pt}%
\definecolor{currentstroke}{rgb}{0.500000,0.500000,0.500000}%
\pgfsetstrokecolor{currentstroke}%
\pgfsetstrokeopacity{0.300000}%
\pgfsetdash{}{0pt}%
\pgfpathmoveto{\pgfqpoint{2.829380in}{2.196068in}}%
\pgfpathlineto{\pgfqpoint{2.793727in}{2.254345in}}%
\pgfpathlineto{\pgfqpoint{2.765246in}{2.316319in}}%
\pgfpathlineto{\pgfqpoint{2.747461in}{2.382012in}}%
\pgfpathlineto{\pgfqpoint{2.743498in}{2.449865in}}%
\pgfpathlineto{\pgfqpoint{2.752091in}{2.513498in}}%
\pgfusepath{stroke}%
\end{pgfscope}%
\begin{pgfscope}%
\pgfpathrectangle{\pgfqpoint{0.647939in}{0.492442in}}{\pgfqpoint{3.079299in}{3.079299in}}%
\pgfusepath{clip}%
\pgfsetbuttcap%
\pgfsetroundjoin%
\pgfsetlinewidth{0.301125pt}%
\definecolor{currentstroke}{rgb}{0.500000,0.500000,0.500000}%
\pgfsetstrokecolor{currentstroke}%
\pgfsetstrokeopacity{0.300000}%
\pgfsetdash{}{0pt}%
\pgfpathmoveto{\pgfqpoint{2.525517in}{1.523102in}}%
\pgfpathlineto{\pgfqpoint{2.457767in}{1.532513in}}%
\pgfpathlineto{\pgfqpoint{2.389623in}{1.538514in}}%
\pgfpathlineto{\pgfqpoint{2.321274in}{1.541497in}}%
\pgfpathlineto{\pgfqpoint{2.252856in}{1.542136in}}%
\pgfpathlineto{\pgfqpoint{2.184433in}{1.541401in}}%
\pgfpathlineto{\pgfqpoint{2.116009in}{1.540694in}}%
\pgfpathlineto{\pgfqpoint{2.047620in}{1.542203in}}%
\pgfusepath{stroke}%
\end{pgfscope}%
\begin{pgfscope}%
\pgfpathrectangle{\pgfqpoint{0.647939in}{0.492442in}}{\pgfqpoint{3.079299in}{3.079299in}}%
\pgfusepath{clip}%
\pgfsetbuttcap%
\pgfsetroundjoin%
\pgfsetlinewidth{0.301125pt}%
\definecolor{currentstroke}{rgb}{0.500000,0.500000,0.500000}%
\pgfsetstrokecolor{currentstroke}%
\pgfsetstrokeopacity{0.300000}%
\pgfsetdash{}{0pt}%
\pgfpathmoveto{\pgfqpoint{2.735198in}{1.703206in}}%
\pgfpathlineto{\pgfqpoint{2.671949in}{1.729263in}}%
\pgfpathlineto{\pgfqpoint{2.607493in}{1.752155in}}%
\pgfpathlineto{\pgfqpoint{2.541844in}{1.771346in}}%
\pgfpathlineto{\pgfqpoint{2.475128in}{1.786373in}}%
\pgfpathlineto{\pgfqpoint{2.407565in}{1.796936in}}%
\pgfpathlineto{\pgfqpoint{2.339450in}{1.803019in}}%
\pgfpathlineto{\pgfqpoint{2.271081in}{1.805029in}}%
\pgfpathlineto{\pgfqpoint{2.202672in}{1.803975in}}%
\pgfpathlineto{\pgfqpoint{2.134274in}{1.801987in}}%
\pgfpathlineto{\pgfqpoint{2.066098in}{1.804511in}}%
\pgfpathlineto{\pgfqpoint{2.066098in}{1.804511in}}%
\pgfpathlineto{\pgfqpoint{2.036719in}{1.810724in}}%
\pgfpathlineto{\pgfqpoint{2.036719in}{1.810724in}}%
\pgfpathlineto{\pgfqpoint{2.018214in}{1.821974in}}%
\pgfpathlineto{\pgfqpoint{2.018214in}{1.821974in}}%
\pgfusepath{stroke}%
\end{pgfscope}%
\begin{pgfscope}%
\pgfpathrectangle{\pgfqpoint{0.647939in}{0.492442in}}{\pgfqpoint{3.079299in}{3.079299in}}%
\pgfusepath{clip}%
\pgfsetbuttcap%
\pgfsetroundjoin%
\pgfsetlinewidth{0.301125pt}%
\definecolor{currentstroke}{rgb}{0.500000,0.500000,0.500000}%
\pgfsetstrokecolor{currentstroke}%
\pgfsetstrokeopacity{0.300000}%
\pgfsetdash{}{0pt}%
\pgfpathmoveto{\pgfqpoint{2.425105in}{1.658645in}}%
\pgfpathlineto{\pgfqpoint{2.356921in}{1.664111in}}%
\pgfpathlineto{\pgfqpoint{2.288544in}{1.666290in}}%
\pgfpathlineto{\pgfqpoint{2.220123in}{1.666066in}}%
\pgfpathlineto{\pgfqpoint{2.151706in}{1.664821in}}%
\pgfpathlineto{\pgfqpoint{2.083293in}{1.664889in}}%
\pgfpathlineto{\pgfqpoint{2.030698in}{1.668533in}}%
\pgfpathlineto{\pgfqpoint{1.977636in}{1.682171in}}%
\pgfpathlineto{\pgfqpoint{1.977636in}{1.682171in}}%
\pgfpathlineto{\pgfqpoint{1.977636in}{1.682171in}}%
\pgfpathlineto{\pgfqpoint{1.953232in}{1.697371in}}%
\pgfpathlineto{\pgfqpoint{1.953232in}{1.697371in}}%
\pgfusepath{stroke}%
\end{pgfscope}%
\begin{pgfscope}%
\pgfpathrectangle{\pgfqpoint{0.647939in}{0.492442in}}{\pgfqpoint{3.079299in}{3.079299in}}%
\pgfusepath{clip}%
\pgfsetbuttcap%
\pgfsetroundjoin%
\pgfsetlinewidth{0.301125pt}%
\definecolor{currentstroke}{rgb}{0.500000,0.500000,0.500000}%
\pgfsetstrokecolor{currentstroke}%
\pgfsetstrokeopacity{0.300000}%
\pgfsetdash{}{0pt}%
\pgfpathmoveto{\pgfqpoint{1.853313in}{2.346792in}}%
\pgfpathlineto{\pgfqpoint{1.920952in}{2.357015in}}%
\pgfpathlineto{\pgfqpoint{1.989012in}{2.363881in}}%
\pgfpathlineto{\pgfqpoint{2.057327in}{2.367447in}}%
\pgfpathlineto{\pgfqpoint{2.125739in}{2.368145in}}%
\pgfpathlineto{\pgfqpoint{2.194154in}{2.366977in}}%
\pgfpathlineto{\pgfqpoint{2.262571in}{2.365937in}}%
\pgfpathlineto{\pgfqpoint{2.330813in}{2.368864in}}%
\pgfpathlineto{\pgfqpoint{2.397541in}{2.382012in}}%
\pgfpathlineto{\pgfqpoint{2.397541in}{2.382012in}}%
\pgfpathlineto{\pgfqpoint{2.448495in}{2.404149in}}%
\pgfusepath{stroke}%
\end{pgfscope}%
\begin{pgfscope}%
\pgfpathrectangle{\pgfqpoint{0.647939in}{0.492442in}}{\pgfqpoint{3.079299in}{3.079299in}}%
\pgfusepath{clip}%
\pgfsetbuttcap%
\pgfsetroundjoin%
\pgfsetlinewidth{0.301125pt}%
\definecolor{currentstroke}{rgb}{0.500000,0.500000,0.500000}%
\pgfsetstrokecolor{currentstroke}%
\pgfsetstrokeopacity{0.300000}%
\pgfsetdash{}{0pt}%
\pgfpathmoveto{\pgfqpoint{1.917896in}{2.298253in}}%
\pgfpathlineto{\pgfqpoint{1.985858in}{2.306006in}}%
\pgfpathlineto{\pgfqpoint{2.054136in}{2.310159in}}%
\pgfpathlineto{\pgfqpoint{2.122541in}{2.311073in}}%
\pgfpathlineto{\pgfqpoint{2.190951in}{2.309733in}}%
\pgfpathlineto{\pgfqpoint{2.259361in}{2.308380in}}%
\pgfpathlineto{\pgfqpoint{2.327557in}{2.312028in}}%
\pgfpathlineto{\pgfqpoint{2.327557in}{2.312028in}}%
\pgfusepath{stroke}%
\end{pgfscope}%
\begin{pgfscope}%
\pgfpathrectangle{\pgfqpoint{0.647939in}{0.492442in}}{\pgfqpoint{3.079299in}{3.079299in}}%
\pgfusepath{clip}%
\pgfsetbuttcap%
\pgfsetroundjoin%
\pgfsetlinewidth{0.301125pt}%
\definecolor{currentstroke}{rgb}{0.500000,0.500000,0.500000}%
\pgfsetstrokecolor{currentstroke}%
\pgfsetstrokeopacity{0.300000}%
\pgfsetdash{}{0pt}%
\pgfpathmoveto{\pgfqpoint{1.907652in}{2.102076in}}%
\pgfpathlineto{\pgfqpoint{1.974860in}{2.114692in}}%
\pgfpathlineto{\pgfqpoint{2.042786in}{2.122483in}}%
\pgfpathlineto{\pgfqpoint{2.111110in}{2.124914in}}%
\pgfpathlineto{\pgfqpoint{2.179394in}{2.121672in}}%
\pgfpathlineto{\pgfqpoint{2.241617in}{2.112659in}}%
\pgfpathlineto{\pgfqpoint{2.241617in}{2.112659in}}%
\pgfpathlineto{\pgfqpoint{2.241617in}{2.112659in}}%
\pgfpathlineto{\pgfqpoint{2.251321in}{2.109432in}}%
\pgfpathlineto{\pgfqpoint{2.251321in}{2.109432in}}%
\pgfpathlineto{\pgfqpoint{2.253379in}{2.106725in}}%
\pgfpathlineto{\pgfqpoint{2.253446in}{2.104087in}}%
\pgfpathlineto{\pgfqpoint{2.251217in}{2.100546in}}%
\pgfpathlineto{\pgfqpoint{2.245876in}{2.094758in}}%
\pgfpathlineto{\pgfqpoint{2.233397in}{2.083430in}}%
\pgfpathlineto{\pgfqpoint{2.233397in}{2.083430in}}%
\pgfusepath{stroke}%
\end{pgfscope}%
\begin{pgfscope}%
\pgfpathrectangle{\pgfqpoint{0.647939in}{0.492442in}}{\pgfqpoint{3.079299in}{3.079299in}}%
\pgfusepath{clip}%
\pgfsetroundcap%
\pgfsetroundjoin%
\pgfsetlinewidth{0.301125pt}%
\definecolor{currentstroke}{rgb}{0.500000,0.500000,0.500000}%
\pgfsetstrokecolor{currentstroke}%
\pgfsetstrokeopacity{0.300000}%
\pgfsetdash{}{0pt}%
\pgfpathmoveto{\pgfqpoint{1.454126in}{1.105377in}}%
\pgfusepath{stroke}%
\end{pgfscope}%
\begin{pgfscope}%
\pgfpathrectangle{\pgfqpoint{0.647939in}{0.492442in}}{\pgfqpoint{3.079299in}{3.079299in}}%
\pgfusepath{clip}%
\pgfsetroundcap%
\pgfsetroundjoin%
\definecolor{currentfill}{rgb}{0.500000,0.500000,0.500000}%
\pgfsetfillcolor{currentfill}%
\pgfsetfillopacity{0.300000}%
\pgfsetlinewidth{0.301125pt}%
\definecolor{currentstroke}{rgb}{0.500000,0.500000,0.500000}%
\pgfsetstrokecolor{currentstroke}%
\pgfsetstrokeopacity{0.300000}%
\pgfsetdash{}{0pt}%
\pgfpathmoveto{\pgfqpoint{0.000000in}{0.000000in}}%
\pgfpathlineto{\pgfqpoint{0.000000in}{0.000000in}}%
\pgfpathclose%
\pgfusepath{stroke,fill}%
\end{pgfscope}%
\begin{pgfscope}%
\pgfpathrectangle{\pgfqpoint{0.647939in}{0.492442in}}{\pgfqpoint{3.079299in}{3.079299in}}%
\pgfusepath{clip}%
\pgfsetroundcap%
\pgfsetroundjoin%
\pgfsetlinewidth{0.301125pt}%
\definecolor{currentstroke}{rgb}{0.500000,0.500000,0.500000}%
\pgfsetstrokecolor{currentstroke}%
\pgfsetstrokeopacity{0.300000}%
\pgfsetdash{}{0pt}%
\pgfpathmoveto{\pgfqpoint{1.129904in}{0.590049in}}%
\pgfusepath{stroke}%
\end{pgfscope}%
\begin{pgfscope}%
\pgfpathrectangle{\pgfqpoint{0.647939in}{0.492442in}}{\pgfqpoint{3.079299in}{3.079299in}}%
\pgfusepath{clip}%
\pgfsetroundcap%
\pgfsetroundjoin%
\definecolor{currentfill}{rgb}{0.500000,0.500000,0.500000}%
\pgfsetfillcolor{currentfill}%
\pgfsetfillopacity{0.300000}%
\pgfsetlinewidth{0.301125pt}%
\definecolor{currentstroke}{rgb}{0.500000,0.500000,0.500000}%
\pgfsetstrokecolor{currentstroke}%
\pgfsetstrokeopacity{0.300000}%
\pgfsetdash{}{0pt}%
\pgfpathmoveto{\pgfqpoint{0.000000in}{0.000000in}}%
\pgfpathlineto{\pgfqpoint{0.000000in}{0.000000in}}%
\pgfpathclose%
\pgfusepath{stroke,fill}%
\end{pgfscope}%
\begin{pgfscope}%
\pgfpathrectangle{\pgfqpoint{0.647939in}{0.492442in}}{\pgfqpoint{3.079299in}{3.079299in}}%
\pgfusepath{clip}%
\pgfsetroundcap%
\pgfsetroundjoin%
\pgfsetlinewidth{0.301125pt}%
\definecolor{currentstroke}{rgb}{0.500000,0.500000,0.500000}%
\pgfsetstrokecolor{currentstroke}%
\pgfsetstrokeopacity{0.300000}%
\pgfsetdash{}{0pt}%
\pgfpathmoveto{\pgfqpoint{1.369289in}{0.779256in}}%
\pgfusepath{stroke}%
\end{pgfscope}%
\begin{pgfscope}%
\pgfpathrectangle{\pgfqpoint{0.647939in}{0.492442in}}{\pgfqpoint{3.079299in}{3.079299in}}%
\pgfusepath{clip}%
\pgfsetroundcap%
\pgfsetroundjoin%
\definecolor{currentfill}{rgb}{0.500000,0.500000,0.500000}%
\pgfsetfillcolor{currentfill}%
\pgfsetfillopacity{0.300000}%
\pgfsetlinewidth{0.301125pt}%
\definecolor{currentstroke}{rgb}{0.500000,0.500000,0.500000}%
\pgfsetstrokecolor{currentstroke}%
\pgfsetstrokeopacity{0.300000}%
\pgfsetdash{}{0pt}%
\pgfpathmoveto{\pgfqpoint{0.000000in}{0.000000in}}%
\pgfpathlineto{\pgfqpoint{0.000000in}{0.000000in}}%
\pgfpathclose%
\pgfusepath{stroke,fill}%
\end{pgfscope}%
\begin{pgfscope}%
\pgfpathrectangle{\pgfqpoint{0.647939in}{0.492442in}}{\pgfqpoint{3.079299in}{3.079299in}}%
\pgfusepath{clip}%
\pgfsetroundcap%
\pgfsetroundjoin%
\pgfsetlinewidth{0.301125pt}%
\definecolor{currentstroke}{rgb}{0.500000,0.500000,0.500000}%
\pgfsetstrokecolor{currentstroke}%
\pgfsetstrokeopacity{0.300000}%
\pgfsetdash{}{0pt}%
\pgfpathmoveto{\pgfqpoint{1.392914in}{0.646651in}}%
\pgfusepath{stroke}%
\end{pgfscope}%
\begin{pgfscope}%
\pgfpathrectangle{\pgfqpoint{0.647939in}{0.492442in}}{\pgfqpoint{3.079299in}{3.079299in}}%
\pgfusepath{clip}%
\pgfsetroundcap%
\pgfsetroundjoin%
\definecolor{currentfill}{rgb}{0.500000,0.500000,0.500000}%
\pgfsetfillcolor{currentfill}%
\pgfsetfillopacity{0.300000}%
\pgfsetlinewidth{0.301125pt}%
\definecolor{currentstroke}{rgb}{0.500000,0.500000,0.500000}%
\pgfsetstrokecolor{currentstroke}%
\pgfsetstrokeopacity{0.300000}%
\pgfsetdash{}{0pt}%
\pgfpathmoveto{\pgfqpoint{0.000000in}{0.000000in}}%
\pgfpathlineto{\pgfqpoint{0.000000in}{0.000000in}}%
\pgfpathclose%
\pgfusepath{stroke,fill}%
\end{pgfscope}%
\begin{pgfscope}%
\pgfpathrectangle{\pgfqpoint{0.647939in}{0.492442in}}{\pgfqpoint{3.079299in}{3.079299in}}%
\pgfusepath{clip}%
\pgfsetroundcap%
\pgfsetroundjoin%
\pgfsetlinewidth{0.301125pt}%
\definecolor{currentstroke}{rgb}{0.500000,0.500000,0.500000}%
\pgfsetstrokecolor{currentstroke}%
\pgfsetstrokeopacity{0.300000}%
\pgfsetdash{}{0pt}%
\pgfpathmoveto{\pgfqpoint{1.508368in}{0.618602in}}%
\pgfusepath{stroke}%
\end{pgfscope}%
\begin{pgfscope}%
\pgfpathrectangle{\pgfqpoint{0.647939in}{0.492442in}}{\pgfqpoint{3.079299in}{3.079299in}}%
\pgfusepath{clip}%
\pgfsetroundcap%
\pgfsetroundjoin%
\definecolor{currentfill}{rgb}{0.500000,0.500000,0.500000}%
\pgfsetfillcolor{currentfill}%
\pgfsetfillopacity{0.300000}%
\pgfsetlinewidth{0.301125pt}%
\definecolor{currentstroke}{rgb}{0.500000,0.500000,0.500000}%
\pgfsetstrokecolor{currentstroke}%
\pgfsetstrokeopacity{0.300000}%
\pgfsetdash{}{0pt}%
\pgfpathmoveto{\pgfqpoint{0.000000in}{0.000000in}}%
\pgfpathlineto{\pgfqpoint{0.000000in}{0.000000in}}%
\pgfpathclose%
\pgfusepath{stroke,fill}%
\end{pgfscope}%
\begin{pgfscope}%
\pgfpathrectangle{\pgfqpoint{0.647939in}{0.492442in}}{\pgfqpoint{3.079299in}{3.079299in}}%
\pgfusepath{clip}%
\pgfsetroundcap%
\pgfsetroundjoin%
\pgfsetlinewidth{0.301125pt}%
\definecolor{currentstroke}{rgb}{0.500000,0.500000,0.500000}%
\pgfsetstrokecolor{currentstroke}%
\pgfsetstrokeopacity{0.300000}%
\pgfsetdash{}{0pt}%
\pgfpathmoveto{\pgfqpoint{1.813858in}{0.507290in}}%
\pgfusepath{stroke}%
\end{pgfscope}%
\begin{pgfscope}%
\pgfpathrectangle{\pgfqpoint{0.647939in}{0.492442in}}{\pgfqpoint{3.079299in}{3.079299in}}%
\pgfusepath{clip}%
\pgfsetroundcap%
\pgfsetroundjoin%
\definecolor{currentfill}{rgb}{0.500000,0.500000,0.500000}%
\pgfsetfillcolor{currentfill}%
\pgfsetfillopacity{0.300000}%
\pgfsetlinewidth{0.301125pt}%
\definecolor{currentstroke}{rgb}{0.500000,0.500000,0.500000}%
\pgfsetstrokecolor{currentstroke}%
\pgfsetstrokeopacity{0.300000}%
\pgfsetdash{}{0pt}%
\pgfpathmoveto{\pgfqpoint{0.000000in}{0.000000in}}%
\pgfpathlineto{\pgfqpoint{0.000000in}{0.000000in}}%
\pgfpathclose%
\pgfusepath{stroke,fill}%
\end{pgfscope}%
\begin{pgfscope}%
\pgfpathrectangle{\pgfqpoint{0.647939in}{0.492442in}}{\pgfqpoint{3.079299in}{3.079299in}}%
\pgfusepath{clip}%
\pgfsetroundcap%
\pgfsetroundjoin%
\pgfsetlinewidth{0.301125pt}%
\definecolor{currentstroke}{rgb}{0.500000,0.500000,0.500000}%
\pgfsetstrokecolor{currentstroke}%
\pgfsetstrokeopacity{0.300000}%
\pgfsetdash{}{0pt}%
\pgfpathmoveto{\pgfqpoint{2.232934in}{0.493966in}}%
\pgfusepath{stroke}%
\end{pgfscope}%
\begin{pgfscope}%
\pgfpathrectangle{\pgfqpoint{0.647939in}{0.492442in}}{\pgfqpoint{3.079299in}{3.079299in}}%
\pgfusepath{clip}%
\pgfsetroundcap%
\pgfsetroundjoin%
\definecolor{currentfill}{rgb}{0.500000,0.500000,0.500000}%
\pgfsetfillcolor{currentfill}%
\pgfsetfillopacity{0.300000}%
\pgfsetlinewidth{0.301125pt}%
\definecolor{currentstroke}{rgb}{0.500000,0.500000,0.500000}%
\pgfsetstrokecolor{currentstroke}%
\pgfsetstrokeopacity{0.300000}%
\pgfsetdash{}{0pt}%
\pgfpathmoveto{\pgfqpoint{0.000000in}{0.000000in}}%
\pgfpathlineto{\pgfqpoint{0.000000in}{0.000000in}}%
\pgfpathclose%
\pgfusepath{stroke,fill}%
\end{pgfscope}%
\begin{pgfscope}%
\pgfpathrectangle{\pgfqpoint{0.647939in}{0.492442in}}{\pgfqpoint{3.079299in}{3.079299in}}%
\pgfusepath{clip}%
\pgfsetroundcap%
\pgfsetroundjoin%
\pgfsetlinewidth{0.301125pt}%
\definecolor{currentstroke}{rgb}{0.500000,0.500000,0.500000}%
\pgfsetstrokecolor{currentstroke}%
\pgfsetstrokeopacity{0.300000}%
\pgfsetdash{}{0pt}%
\pgfpathmoveto{\pgfqpoint{2.313842in}{0.540308in}}%
\pgfusepath{stroke}%
\end{pgfscope}%
\begin{pgfscope}%
\pgfpathrectangle{\pgfqpoint{0.647939in}{0.492442in}}{\pgfqpoint{3.079299in}{3.079299in}}%
\pgfusepath{clip}%
\pgfsetroundcap%
\pgfsetroundjoin%
\definecolor{currentfill}{rgb}{0.500000,0.500000,0.500000}%
\pgfsetfillcolor{currentfill}%
\pgfsetfillopacity{0.300000}%
\pgfsetlinewidth{0.301125pt}%
\definecolor{currentstroke}{rgb}{0.500000,0.500000,0.500000}%
\pgfsetstrokecolor{currentstroke}%
\pgfsetstrokeopacity{0.300000}%
\pgfsetdash{}{0pt}%
\pgfpathmoveto{\pgfqpoint{0.000000in}{0.000000in}}%
\pgfpathlineto{\pgfqpoint{0.000000in}{0.000000in}}%
\pgfpathclose%
\pgfusepath{stroke,fill}%
\end{pgfscope}%
\begin{pgfscope}%
\pgfpathrectangle{\pgfqpoint{0.647939in}{0.492442in}}{\pgfqpoint{3.079299in}{3.079299in}}%
\pgfusepath{clip}%
\pgfsetroundcap%
\pgfsetroundjoin%
\pgfsetlinewidth{0.301125pt}%
\definecolor{currentstroke}{rgb}{0.500000,0.500000,0.500000}%
\pgfsetstrokecolor{currentstroke}%
\pgfsetstrokeopacity{0.300000}%
\pgfsetdash{}{0pt}%
\pgfpathmoveto{\pgfqpoint{2.868115in}{0.533884in}}%
\pgfusepath{stroke}%
\end{pgfscope}%
\begin{pgfscope}%
\pgfpathrectangle{\pgfqpoint{0.647939in}{0.492442in}}{\pgfqpoint{3.079299in}{3.079299in}}%
\pgfusepath{clip}%
\pgfsetroundcap%
\pgfsetroundjoin%
\definecolor{currentfill}{rgb}{0.500000,0.500000,0.500000}%
\pgfsetfillcolor{currentfill}%
\pgfsetfillopacity{0.300000}%
\pgfsetlinewidth{0.301125pt}%
\definecolor{currentstroke}{rgb}{0.500000,0.500000,0.500000}%
\pgfsetstrokecolor{currentstroke}%
\pgfsetstrokeopacity{0.300000}%
\pgfsetdash{}{0pt}%
\pgfpathmoveto{\pgfqpoint{0.000000in}{0.000000in}}%
\pgfpathlineto{\pgfqpoint{0.000000in}{0.000000in}}%
\pgfpathclose%
\pgfusepath{stroke,fill}%
\end{pgfscope}%
\begin{pgfscope}%
\pgfpathrectangle{\pgfqpoint{0.647939in}{0.492442in}}{\pgfqpoint{3.079299in}{3.079299in}}%
\pgfusepath{clip}%
\pgfsetroundcap%
\pgfsetroundjoin%
\pgfsetlinewidth{0.301125pt}%
\definecolor{currentstroke}{rgb}{0.500000,0.500000,0.500000}%
\pgfsetstrokecolor{currentstroke}%
\pgfsetstrokeopacity{0.300000}%
\pgfsetdash{}{0pt}%
\pgfpathmoveto{\pgfqpoint{2.279381in}{0.693742in}}%
\pgfusepath{stroke}%
\end{pgfscope}%
\begin{pgfscope}%
\pgfpathrectangle{\pgfqpoint{0.647939in}{0.492442in}}{\pgfqpoint{3.079299in}{3.079299in}}%
\pgfusepath{clip}%
\pgfsetroundcap%
\pgfsetroundjoin%
\definecolor{currentfill}{rgb}{0.500000,0.500000,0.500000}%
\pgfsetfillcolor{currentfill}%
\pgfsetfillopacity{0.300000}%
\pgfsetlinewidth{0.301125pt}%
\definecolor{currentstroke}{rgb}{0.500000,0.500000,0.500000}%
\pgfsetstrokecolor{currentstroke}%
\pgfsetstrokeopacity{0.300000}%
\pgfsetdash{}{0pt}%
\pgfpathmoveto{\pgfqpoint{0.000000in}{0.000000in}}%
\pgfpathlineto{\pgfqpoint{0.000000in}{0.000000in}}%
\pgfpathclose%
\pgfusepath{stroke,fill}%
\end{pgfscope}%
\begin{pgfscope}%
\pgfpathrectangle{\pgfqpoint{0.647939in}{0.492442in}}{\pgfqpoint{3.079299in}{3.079299in}}%
\pgfusepath{clip}%
\pgfsetroundcap%
\pgfsetroundjoin%
\pgfsetlinewidth{0.301125pt}%
\definecolor{currentstroke}{rgb}{0.500000,0.500000,0.500000}%
\pgfsetstrokecolor{currentstroke}%
\pgfsetstrokeopacity{0.300000}%
\pgfsetdash{}{0pt}%
\pgfpathmoveto{\pgfqpoint{2.572324in}{0.763137in}}%
\pgfusepath{stroke}%
\end{pgfscope}%
\begin{pgfscope}%
\pgfpathrectangle{\pgfqpoint{0.647939in}{0.492442in}}{\pgfqpoint{3.079299in}{3.079299in}}%
\pgfusepath{clip}%
\pgfsetroundcap%
\pgfsetroundjoin%
\definecolor{currentfill}{rgb}{0.500000,0.500000,0.500000}%
\pgfsetfillcolor{currentfill}%
\pgfsetfillopacity{0.300000}%
\pgfsetlinewidth{0.301125pt}%
\definecolor{currentstroke}{rgb}{0.500000,0.500000,0.500000}%
\pgfsetstrokecolor{currentstroke}%
\pgfsetstrokeopacity{0.300000}%
\pgfsetdash{}{0pt}%
\pgfpathmoveto{\pgfqpoint{0.000000in}{0.000000in}}%
\pgfpathlineto{\pgfqpoint{0.000000in}{0.000000in}}%
\pgfpathclose%
\pgfusepath{stroke,fill}%
\end{pgfscope}%
\begin{pgfscope}%
\pgfpathrectangle{\pgfqpoint{0.647939in}{0.492442in}}{\pgfqpoint{3.079299in}{3.079299in}}%
\pgfusepath{clip}%
\pgfsetroundcap%
\pgfsetroundjoin%
\pgfsetlinewidth{0.301125pt}%
\definecolor{currentstroke}{rgb}{0.500000,0.500000,0.500000}%
\pgfsetstrokecolor{currentstroke}%
\pgfsetstrokeopacity{0.300000}%
\pgfsetdash{}{0pt}%
\pgfpathmoveto{\pgfqpoint{2.658210in}{0.829290in}}%
\pgfusepath{stroke}%
\end{pgfscope}%
\begin{pgfscope}%
\pgfpathrectangle{\pgfqpoint{0.647939in}{0.492442in}}{\pgfqpoint{3.079299in}{3.079299in}}%
\pgfusepath{clip}%
\pgfsetroundcap%
\pgfsetroundjoin%
\definecolor{currentfill}{rgb}{0.500000,0.500000,0.500000}%
\pgfsetfillcolor{currentfill}%
\pgfsetfillopacity{0.300000}%
\pgfsetlinewidth{0.301125pt}%
\definecolor{currentstroke}{rgb}{0.500000,0.500000,0.500000}%
\pgfsetstrokecolor{currentstroke}%
\pgfsetstrokeopacity{0.300000}%
\pgfsetdash{}{0pt}%
\pgfpathmoveto{\pgfqpoint{0.000000in}{0.000000in}}%
\pgfpathlineto{\pgfqpoint{0.000000in}{0.000000in}}%
\pgfpathclose%
\pgfusepath{stroke,fill}%
\end{pgfscope}%
\begin{pgfscope}%
\pgfpathrectangle{\pgfqpoint{0.647939in}{0.492442in}}{\pgfqpoint{3.079299in}{3.079299in}}%
\pgfusepath{clip}%
\pgfsetroundcap%
\pgfsetroundjoin%
\pgfsetlinewidth{0.301125pt}%
\definecolor{currentstroke}{rgb}{0.500000,0.500000,0.500000}%
\pgfsetstrokecolor{currentstroke}%
\pgfsetstrokeopacity{0.300000}%
\pgfsetdash{}{0pt}%
\pgfpathmoveto{\pgfqpoint{2.729751in}{0.900037in}}%
\pgfusepath{stroke}%
\end{pgfscope}%
\begin{pgfscope}%
\pgfpathrectangle{\pgfqpoint{0.647939in}{0.492442in}}{\pgfqpoint{3.079299in}{3.079299in}}%
\pgfusepath{clip}%
\pgfsetroundcap%
\pgfsetroundjoin%
\definecolor{currentfill}{rgb}{0.500000,0.500000,0.500000}%
\pgfsetfillcolor{currentfill}%
\pgfsetfillopacity{0.300000}%
\pgfsetlinewidth{0.301125pt}%
\definecolor{currentstroke}{rgb}{0.500000,0.500000,0.500000}%
\pgfsetstrokecolor{currentstroke}%
\pgfsetstrokeopacity{0.300000}%
\pgfsetdash{}{0pt}%
\pgfpathmoveto{\pgfqpoint{0.000000in}{0.000000in}}%
\pgfpathlineto{\pgfqpoint{0.000000in}{0.000000in}}%
\pgfpathclose%
\pgfusepath{stroke,fill}%
\end{pgfscope}%
\begin{pgfscope}%
\pgfpathrectangle{\pgfqpoint{0.647939in}{0.492442in}}{\pgfqpoint{3.079299in}{3.079299in}}%
\pgfusepath{clip}%
\pgfsetroundcap%
\pgfsetroundjoin%
\pgfsetlinewidth{0.301125pt}%
\definecolor{currentstroke}{rgb}{0.500000,0.500000,0.500000}%
\pgfsetstrokecolor{currentstroke}%
\pgfsetstrokeopacity{0.300000}%
\pgfsetdash{}{0pt}%
\pgfpathmoveto{\pgfqpoint{2.599860in}{1.008003in}}%
\pgfusepath{stroke}%
\end{pgfscope}%
\begin{pgfscope}%
\pgfpathrectangle{\pgfqpoint{0.647939in}{0.492442in}}{\pgfqpoint{3.079299in}{3.079299in}}%
\pgfusepath{clip}%
\pgfsetroundcap%
\pgfsetroundjoin%
\definecolor{currentfill}{rgb}{0.500000,0.500000,0.500000}%
\pgfsetfillcolor{currentfill}%
\pgfsetfillopacity{0.300000}%
\pgfsetlinewidth{0.301125pt}%
\definecolor{currentstroke}{rgb}{0.500000,0.500000,0.500000}%
\pgfsetstrokecolor{currentstroke}%
\pgfsetstrokeopacity{0.300000}%
\pgfsetdash{}{0pt}%
\pgfpathmoveto{\pgfqpoint{0.000000in}{0.000000in}}%
\pgfpathlineto{\pgfqpoint{0.000000in}{0.000000in}}%
\pgfpathclose%
\pgfusepath{stroke,fill}%
\end{pgfscope}%
\begin{pgfscope}%
\pgfpathrectangle{\pgfqpoint{0.647939in}{0.492442in}}{\pgfqpoint{3.079299in}{3.079299in}}%
\pgfusepath{clip}%
\pgfsetroundcap%
\pgfsetroundjoin%
\pgfsetlinewidth{0.301125pt}%
\definecolor{currentstroke}{rgb}{0.500000,0.500000,0.500000}%
\pgfsetstrokecolor{currentstroke}%
\pgfsetstrokeopacity{0.300000}%
\pgfsetdash{}{0pt}%
\pgfpathmoveto{\pgfqpoint{2.806063in}{1.051424in}}%
\pgfusepath{stroke}%
\end{pgfscope}%
\begin{pgfscope}%
\pgfpathrectangle{\pgfqpoint{0.647939in}{0.492442in}}{\pgfqpoint{3.079299in}{3.079299in}}%
\pgfusepath{clip}%
\pgfsetroundcap%
\pgfsetroundjoin%
\definecolor{currentfill}{rgb}{0.500000,0.500000,0.500000}%
\pgfsetfillcolor{currentfill}%
\pgfsetfillopacity{0.300000}%
\pgfsetlinewidth{0.301125pt}%
\definecolor{currentstroke}{rgb}{0.500000,0.500000,0.500000}%
\pgfsetstrokecolor{currentstroke}%
\pgfsetstrokeopacity{0.300000}%
\pgfsetdash{}{0pt}%
\pgfpathmoveto{\pgfqpoint{0.000000in}{0.000000in}}%
\pgfpathlineto{\pgfqpoint{0.000000in}{0.000000in}}%
\pgfpathclose%
\pgfusepath{stroke,fill}%
\end{pgfscope}%
\begin{pgfscope}%
\pgfpathrectangle{\pgfqpoint{0.647939in}{0.492442in}}{\pgfqpoint{3.079299in}{3.079299in}}%
\pgfusepath{clip}%
\pgfsetroundcap%
\pgfsetroundjoin%
\pgfsetlinewidth{0.301125pt}%
\definecolor{currentstroke}{rgb}{0.500000,0.500000,0.500000}%
\pgfsetstrokecolor{currentstroke}%
\pgfsetstrokeopacity{0.300000}%
\pgfsetdash{}{0pt}%
\pgfpathmoveto{\pgfqpoint{2.745930in}{1.154592in}}%
\pgfusepath{stroke}%
\end{pgfscope}%
\begin{pgfscope}%
\pgfpathrectangle{\pgfqpoint{0.647939in}{0.492442in}}{\pgfqpoint{3.079299in}{3.079299in}}%
\pgfusepath{clip}%
\pgfsetroundcap%
\pgfsetroundjoin%
\definecolor{currentfill}{rgb}{0.500000,0.500000,0.500000}%
\pgfsetfillcolor{currentfill}%
\pgfsetfillopacity{0.300000}%
\pgfsetlinewidth{0.301125pt}%
\definecolor{currentstroke}{rgb}{0.500000,0.500000,0.500000}%
\pgfsetstrokecolor{currentstroke}%
\pgfsetstrokeopacity{0.300000}%
\pgfsetdash{}{0pt}%
\pgfpathmoveto{\pgfqpoint{0.000000in}{0.000000in}}%
\pgfpathlineto{\pgfqpoint{0.000000in}{0.000000in}}%
\pgfpathclose%
\pgfusepath{stroke,fill}%
\end{pgfscope}%
\begin{pgfscope}%
\pgfpathrectangle{\pgfqpoint{0.647939in}{0.492442in}}{\pgfqpoint{3.079299in}{3.079299in}}%
\pgfusepath{clip}%
\pgfsetroundcap%
\pgfsetroundjoin%
\pgfsetlinewidth{0.301125pt}%
\definecolor{currentstroke}{rgb}{0.500000,0.500000,0.500000}%
\pgfsetstrokecolor{currentstroke}%
\pgfsetstrokeopacity{0.300000}%
\pgfsetdash{}{0pt}%
\pgfpathmoveto{\pgfqpoint{2.686799in}{1.259523in}}%
\pgfusepath{stroke}%
\end{pgfscope}%
\begin{pgfscope}%
\pgfpathrectangle{\pgfqpoint{0.647939in}{0.492442in}}{\pgfqpoint{3.079299in}{3.079299in}}%
\pgfusepath{clip}%
\pgfsetroundcap%
\pgfsetroundjoin%
\definecolor{currentfill}{rgb}{0.500000,0.500000,0.500000}%
\pgfsetfillcolor{currentfill}%
\pgfsetfillopacity{0.300000}%
\pgfsetlinewidth{0.301125pt}%
\definecolor{currentstroke}{rgb}{0.500000,0.500000,0.500000}%
\pgfsetstrokecolor{currentstroke}%
\pgfsetstrokeopacity{0.300000}%
\pgfsetdash{}{0pt}%
\pgfpathmoveto{\pgfqpoint{0.000000in}{0.000000in}}%
\pgfpathlineto{\pgfqpoint{0.000000in}{0.000000in}}%
\pgfpathclose%
\pgfusepath{stroke,fill}%
\end{pgfscope}%
\begin{pgfscope}%
\pgfpathrectangle{\pgfqpoint{0.647939in}{0.492442in}}{\pgfqpoint{3.079299in}{3.079299in}}%
\pgfusepath{clip}%
\pgfsetroundcap%
\pgfsetroundjoin%
\pgfsetlinewidth{0.301125pt}%
\definecolor{currentstroke}{rgb}{0.500000,0.500000,0.500000}%
\pgfsetstrokecolor{currentstroke}%
\pgfsetstrokeopacity{0.300000}%
\pgfsetdash{}{0pt}%
\pgfpathmoveto{\pgfqpoint{2.761256in}{1.331986in}}%
\pgfusepath{stroke}%
\end{pgfscope}%
\begin{pgfscope}%
\pgfpathrectangle{\pgfqpoint{0.647939in}{0.492442in}}{\pgfqpoint{3.079299in}{3.079299in}}%
\pgfusepath{clip}%
\pgfsetroundcap%
\pgfsetroundjoin%
\definecolor{currentfill}{rgb}{0.500000,0.500000,0.500000}%
\pgfsetfillcolor{currentfill}%
\pgfsetfillopacity{0.300000}%
\pgfsetlinewidth{0.301125pt}%
\definecolor{currentstroke}{rgb}{0.500000,0.500000,0.500000}%
\pgfsetstrokecolor{currentstroke}%
\pgfsetstrokeopacity{0.300000}%
\pgfsetdash{}{0pt}%
\pgfpathmoveto{\pgfqpoint{0.000000in}{0.000000in}}%
\pgfpathlineto{\pgfqpoint{0.000000in}{0.000000in}}%
\pgfpathclose%
\pgfusepath{stroke,fill}%
\end{pgfscope}%
\begin{pgfscope}%
\pgfpathrectangle{\pgfqpoint{0.647939in}{0.492442in}}{\pgfqpoint{3.079299in}{3.079299in}}%
\pgfusepath{clip}%
\pgfsetroundcap%
\pgfsetroundjoin%
\pgfsetlinewidth{0.301125pt}%
\definecolor{currentstroke}{rgb}{0.500000,0.500000,0.500000}%
\pgfsetstrokecolor{currentstroke}%
\pgfsetstrokeopacity{0.300000}%
\pgfsetdash{}{0pt}%
\pgfpathmoveto{\pgfqpoint{2.835164in}{1.400133in}}%
\pgfusepath{stroke}%
\end{pgfscope}%
\begin{pgfscope}%
\pgfpathrectangle{\pgfqpoint{0.647939in}{0.492442in}}{\pgfqpoint{3.079299in}{3.079299in}}%
\pgfusepath{clip}%
\pgfsetroundcap%
\pgfsetroundjoin%
\definecolor{currentfill}{rgb}{0.500000,0.500000,0.500000}%
\pgfsetfillcolor{currentfill}%
\pgfsetfillopacity{0.300000}%
\pgfsetlinewidth{0.301125pt}%
\definecolor{currentstroke}{rgb}{0.500000,0.500000,0.500000}%
\pgfsetstrokecolor{currentstroke}%
\pgfsetstrokeopacity{0.300000}%
\pgfsetdash{}{0pt}%
\pgfpathmoveto{\pgfqpoint{0.000000in}{0.000000in}}%
\pgfpathlineto{\pgfqpoint{0.000000in}{0.000000in}}%
\pgfpathclose%
\pgfusepath{stroke,fill}%
\end{pgfscope}%
\begin{pgfscope}%
\pgfpathrectangle{\pgfqpoint{0.647939in}{0.492442in}}{\pgfqpoint{3.079299in}{3.079299in}}%
\pgfusepath{clip}%
\pgfsetroundcap%
\pgfsetroundjoin%
\pgfsetlinewidth{0.301125pt}%
\definecolor{currentstroke}{rgb}{0.500000,0.500000,0.500000}%
\pgfsetstrokecolor{currentstroke}%
\pgfsetstrokeopacity{0.300000}%
\pgfsetdash{}{0pt}%
\pgfpathmoveto{\pgfqpoint{2.845731in}{1.491828in}}%
\pgfusepath{stroke}%
\end{pgfscope}%
\begin{pgfscope}%
\pgfpathrectangle{\pgfqpoint{0.647939in}{0.492442in}}{\pgfqpoint{3.079299in}{3.079299in}}%
\pgfusepath{clip}%
\pgfsetroundcap%
\pgfsetroundjoin%
\definecolor{currentfill}{rgb}{0.500000,0.500000,0.500000}%
\pgfsetfillcolor{currentfill}%
\pgfsetfillopacity{0.300000}%
\pgfsetlinewidth{0.301125pt}%
\definecolor{currentstroke}{rgb}{0.500000,0.500000,0.500000}%
\pgfsetstrokecolor{currentstroke}%
\pgfsetstrokeopacity{0.300000}%
\pgfsetdash{}{0pt}%
\pgfpathmoveto{\pgfqpoint{0.000000in}{0.000000in}}%
\pgfpathlineto{\pgfqpoint{0.000000in}{0.000000in}}%
\pgfpathclose%
\pgfusepath{stroke,fill}%
\end{pgfscope}%
\begin{pgfscope}%
\pgfpathrectangle{\pgfqpoint{0.647939in}{0.492442in}}{\pgfqpoint{3.079299in}{3.079299in}}%
\pgfusepath{clip}%
\pgfsetroundcap%
\pgfsetroundjoin%
\pgfsetlinewidth{0.301125pt}%
\definecolor{currentstroke}{rgb}{0.500000,0.500000,0.500000}%
\pgfsetstrokecolor{currentstroke}%
\pgfsetstrokeopacity{0.300000}%
\pgfsetdash{}{0pt}%
\pgfpathmoveto{\pgfqpoint{2.858306in}{1.585911in}}%
\pgfusepath{stroke}%
\end{pgfscope}%
\begin{pgfscope}%
\pgfpathrectangle{\pgfqpoint{0.647939in}{0.492442in}}{\pgfqpoint{3.079299in}{3.079299in}}%
\pgfusepath{clip}%
\pgfsetroundcap%
\pgfsetroundjoin%
\definecolor{currentfill}{rgb}{0.500000,0.500000,0.500000}%
\pgfsetfillcolor{currentfill}%
\pgfsetfillopacity{0.300000}%
\pgfsetlinewidth{0.301125pt}%
\definecolor{currentstroke}{rgb}{0.500000,0.500000,0.500000}%
\pgfsetstrokecolor{currentstroke}%
\pgfsetstrokeopacity{0.300000}%
\pgfsetdash{}{0pt}%
\pgfpathmoveto{\pgfqpoint{0.000000in}{0.000000in}}%
\pgfpathlineto{\pgfqpoint{0.000000in}{0.000000in}}%
\pgfpathclose%
\pgfusepath{stroke,fill}%
\end{pgfscope}%
\begin{pgfscope}%
\pgfpathrectangle{\pgfqpoint{0.647939in}{0.492442in}}{\pgfqpoint{3.079299in}{3.079299in}}%
\pgfusepath{clip}%
\pgfsetroundcap%
\pgfsetroundjoin%
\pgfsetlinewidth{0.301125pt}%
\definecolor{currentstroke}{rgb}{0.500000,0.500000,0.500000}%
\pgfsetstrokecolor{currentstroke}%
\pgfsetstrokeopacity{0.300000}%
\pgfsetdash{}{0pt}%
\pgfpathmoveto{\pgfqpoint{2.991384in}{1.613304in}}%
\pgfusepath{stroke}%
\end{pgfscope}%
\begin{pgfscope}%
\pgfpathrectangle{\pgfqpoint{0.647939in}{0.492442in}}{\pgfqpoint{3.079299in}{3.079299in}}%
\pgfusepath{clip}%
\pgfsetroundcap%
\pgfsetroundjoin%
\definecolor{currentfill}{rgb}{0.500000,0.500000,0.500000}%
\pgfsetfillcolor{currentfill}%
\pgfsetfillopacity{0.300000}%
\pgfsetlinewidth{0.301125pt}%
\definecolor{currentstroke}{rgb}{0.500000,0.500000,0.500000}%
\pgfsetstrokecolor{currentstroke}%
\pgfsetstrokeopacity{0.300000}%
\pgfsetdash{}{0pt}%
\pgfpathmoveto{\pgfqpoint{0.000000in}{0.000000in}}%
\pgfpathlineto{\pgfqpoint{0.000000in}{0.000000in}}%
\pgfpathclose%
\pgfusepath{stroke,fill}%
\end{pgfscope}%
\begin{pgfscope}%
\pgfpathrectangle{\pgfqpoint{0.647939in}{0.492442in}}{\pgfqpoint{3.079299in}{3.079299in}}%
\pgfusepath{clip}%
\pgfsetroundcap%
\pgfsetroundjoin%
\pgfsetlinewidth{0.301125pt}%
\definecolor{currentstroke}{rgb}{0.500000,0.500000,0.500000}%
\pgfsetstrokecolor{currentstroke}%
\pgfsetstrokeopacity{0.300000}%
\pgfsetdash{}{0pt}%
\pgfpathmoveto{\pgfqpoint{2.948773in}{1.744494in}}%
\pgfusepath{stroke}%
\end{pgfscope}%
\begin{pgfscope}%
\pgfpathrectangle{\pgfqpoint{0.647939in}{0.492442in}}{\pgfqpoint{3.079299in}{3.079299in}}%
\pgfusepath{clip}%
\pgfsetroundcap%
\pgfsetroundjoin%
\definecolor{currentfill}{rgb}{0.500000,0.500000,0.500000}%
\pgfsetfillcolor{currentfill}%
\pgfsetfillopacity{0.300000}%
\pgfsetlinewidth{0.301125pt}%
\definecolor{currentstroke}{rgb}{0.500000,0.500000,0.500000}%
\pgfsetstrokecolor{currentstroke}%
\pgfsetstrokeopacity{0.300000}%
\pgfsetdash{}{0pt}%
\pgfpathmoveto{\pgfqpoint{0.000000in}{0.000000in}}%
\pgfpathlineto{\pgfqpoint{0.000000in}{0.000000in}}%
\pgfpathclose%
\pgfusepath{stroke,fill}%
\end{pgfscope}%
\begin{pgfscope}%
\pgfpathrectangle{\pgfqpoint{0.647939in}{0.492442in}}{\pgfqpoint{3.079299in}{3.079299in}}%
\pgfusepath{clip}%
\pgfsetroundcap%
\pgfsetroundjoin%
\pgfsetlinewidth{0.301125pt}%
\definecolor{currentstroke}{rgb}{0.500000,0.500000,0.500000}%
\pgfsetstrokecolor{currentstroke}%
\pgfsetstrokeopacity{0.300000}%
\pgfsetdash{}{0pt}%
\pgfpathmoveto{\pgfqpoint{2.683198in}{2.381966in}}%
\pgfusepath{stroke}%
\end{pgfscope}%
\begin{pgfscope}%
\pgfpathrectangle{\pgfqpoint{0.647939in}{0.492442in}}{\pgfqpoint{3.079299in}{3.079299in}}%
\pgfusepath{clip}%
\pgfsetroundcap%
\pgfsetroundjoin%
\definecolor{currentfill}{rgb}{0.500000,0.500000,0.500000}%
\pgfsetfillcolor{currentfill}%
\pgfsetfillopacity{0.300000}%
\pgfsetlinewidth{0.301125pt}%
\definecolor{currentstroke}{rgb}{0.500000,0.500000,0.500000}%
\pgfsetstrokecolor{currentstroke}%
\pgfsetstrokeopacity{0.300000}%
\pgfsetdash{}{0pt}%
\pgfpathmoveto{\pgfqpoint{0.000000in}{0.000000in}}%
\pgfpathlineto{\pgfqpoint{0.000000in}{0.000000in}}%
\pgfpathclose%
\pgfusepath{stroke,fill}%
\end{pgfscope}%
\begin{pgfscope}%
\pgfpathrectangle{\pgfqpoint{0.647939in}{0.492442in}}{\pgfqpoint{3.079299in}{3.079299in}}%
\pgfusepath{clip}%
\pgfsetroundcap%
\pgfsetroundjoin%
\pgfsetlinewidth{0.301125pt}%
\definecolor{currentstroke}{rgb}{0.500000,0.500000,0.500000}%
\pgfsetstrokecolor{currentstroke}%
\pgfsetstrokeopacity{0.300000}%
\pgfsetdash{}{0pt}%
\pgfpathmoveto{\pgfqpoint{3.055069in}{2.154132in}}%
\pgfusepath{stroke}%
\end{pgfscope}%
\begin{pgfscope}%
\pgfpathrectangle{\pgfqpoint{0.647939in}{0.492442in}}{\pgfqpoint{3.079299in}{3.079299in}}%
\pgfusepath{clip}%
\pgfsetroundcap%
\pgfsetroundjoin%
\definecolor{currentfill}{rgb}{0.500000,0.500000,0.500000}%
\pgfsetfillcolor{currentfill}%
\pgfsetfillopacity{0.300000}%
\pgfsetlinewidth{0.301125pt}%
\definecolor{currentstroke}{rgb}{0.500000,0.500000,0.500000}%
\pgfsetstrokecolor{currentstroke}%
\pgfsetstrokeopacity{0.300000}%
\pgfsetdash{}{0pt}%
\pgfpathmoveto{\pgfqpoint{0.000000in}{0.000000in}}%
\pgfpathlineto{\pgfqpoint{0.000000in}{0.000000in}}%
\pgfpathclose%
\pgfusepath{stroke,fill}%
\end{pgfscope}%
\begin{pgfscope}%
\pgfpathrectangle{\pgfqpoint{0.647939in}{0.492442in}}{\pgfqpoint{3.079299in}{3.079299in}}%
\pgfusepath{clip}%
\pgfsetroundcap%
\pgfsetroundjoin%
\pgfsetlinewidth{0.301125pt}%
\definecolor{currentstroke}{rgb}{0.500000,0.500000,0.500000}%
\pgfsetstrokecolor{currentstroke}%
\pgfsetstrokeopacity{0.300000}%
\pgfsetdash{}{0pt}%
\pgfpathmoveto{\pgfqpoint{3.113091in}{2.637382in}}%
\pgfusepath{stroke}%
\end{pgfscope}%
\begin{pgfscope}%
\pgfpathrectangle{\pgfqpoint{0.647939in}{0.492442in}}{\pgfqpoint{3.079299in}{3.079299in}}%
\pgfusepath{clip}%
\pgfsetroundcap%
\pgfsetroundjoin%
\definecolor{currentfill}{rgb}{0.500000,0.500000,0.500000}%
\pgfsetfillcolor{currentfill}%
\pgfsetfillopacity{0.300000}%
\pgfsetlinewidth{0.301125pt}%
\definecolor{currentstroke}{rgb}{0.500000,0.500000,0.500000}%
\pgfsetstrokecolor{currentstroke}%
\pgfsetstrokeopacity{0.300000}%
\pgfsetdash{}{0pt}%
\pgfpathmoveto{\pgfqpoint{0.000000in}{0.000000in}}%
\pgfpathlineto{\pgfqpoint{0.000000in}{0.000000in}}%
\pgfpathclose%
\pgfusepath{stroke,fill}%
\end{pgfscope}%
\begin{pgfscope}%
\pgfpathrectangle{\pgfqpoint{0.647939in}{0.492442in}}{\pgfqpoint{3.079299in}{3.079299in}}%
\pgfusepath{clip}%
\pgfsetroundcap%
\pgfsetroundjoin%
\pgfsetlinewidth{0.301125pt}%
\definecolor{currentstroke}{rgb}{0.500000,0.500000,0.500000}%
\pgfsetstrokecolor{currentstroke}%
\pgfsetstrokeopacity{0.300000}%
\pgfsetdash{}{0pt}%
\pgfpathmoveto{\pgfqpoint{3.257505in}{2.644434in}}%
\pgfusepath{stroke}%
\end{pgfscope}%
\begin{pgfscope}%
\pgfpathrectangle{\pgfqpoint{0.647939in}{0.492442in}}{\pgfqpoint{3.079299in}{3.079299in}}%
\pgfusepath{clip}%
\pgfsetroundcap%
\pgfsetroundjoin%
\definecolor{currentfill}{rgb}{0.500000,0.500000,0.500000}%
\pgfsetfillcolor{currentfill}%
\pgfsetfillopacity{0.300000}%
\pgfsetlinewidth{0.301125pt}%
\definecolor{currentstroke}{rgb}{0.500000,0.500000,0.500000}%
\pgfsetstrokecolor{currentstroke}%
\pgfsetstrokeopacity{0.300000}%
\pgfsetdash{}{0pt}%
\pgfpathmoveto{\pgfqpoint{0.000000in}{0.000000in}}%
\pgfpathlineto{\pgfqpoint{0.000000in}{0.000000in}}%
\pgfpathclose%
\pgfusepath{stroke,fill}%
\end{pgfscope}%
\begin{pgfscope}%
\pgfpathrectangle{\pgfqpoint{0.647939in}{0.492442in}}{\pgfqpoint{3.079299in}{3.079299in}}%
\pgfusepath{clip}%
\pgfsetroundcap%
\pgfsetroundjoin%
\pgfsetlinewidth{0.301125pt}%
\definecolor{currentstroke}{rgb}{0.500000,0.500000,0.500000}%
\pgfsetstrokecolor{currentstroke}%
\pgfsetstrokeopacity{0.300000}%
\pgfsetdash{}{0pt}%
\pgfpathmoveto{\pgfqpoint{3.439517in}{2.218885in}}%
\pgfusepath{stroke}%
\end{pgfscope}%
\begin{pgfscope}%
\pgfpathrectangle{\pgfqpoint{0.647939in}{0.492442in}}{\pgfqpoint{3.079299in}{3.079299in}}%
\pgfusepath{clip}%
\pgfsetroundcap%
\pgfsetroundjoin%
\definecolor{currentfill}{rgb}{0.500000,0.500000,0.500000}%
\pgfsetfillcolor{currentfill}%
\pgfsetfillopacity{0.300000}%
\pgfsetlinewidth{0.301125pt}%
\definecolor{currentstroke}{rgb}{0.500000,0.500000,0.500000}%
\pgfsetstrokecolor{currentstroke}%
\pgfsetstrokeopacity{0.300000}%
\pgfsetdash{}{0pt}%
\pgfpathmoveto{\pgfqpoint{0.000000in}{0.000000in}}%
\pgfpathlineto{\pgfqpoint{0.000000in}{0.000000in}}%
\pgfpathclose%
\pgfusepath{stroke,fill}%
\end{pgfscope}%
\begin{pgfscope}%
\pgfpathrectangle{\pgfqpoint{0.647939in}{0.492442in}}{\pgfqpoint{3.079299in}{3.079299in}}%
\pgfusepath{clip}%
\pgfsetroundcap%
\pgfsetroundjoin%
\pgfsetlinewidth{0.301125pt}%
\definecolor{currentstroke}{rgb}{0.500000,0.500000,0.500000}%
\pgfsetstrokecolor{currentstroke}%
\pgfsetstrokeopacity{0.300000}%
\pgfsetdash{}{0pt}%
\pgfpathmoveto{\pgfqpoint{3.437069in}{2.659783in}}%
\pgfusepath{stroke}%
\end{pgfscope}%
\begin{pgfscope}%
\pgfpathrectangle{\pgfqpoint{0.647939in}{0.492442in}}{\pgfqpoint{3.079299in}{3.079299in}}%
\pgfusepath{clip}%
\pgfsetroundcap%
\pgfsetroundjoin%
\definecolor{currentfill}{rgb}{0.500000,0.500000,0.500000}%
\pgfsetfillcolor{currentfill}%
\pgfsetfillopacity{0.300000}%
\pgfsetlinewidth{0.301125pt}%
\definecolor{currentstroke}{rgb}{0.500000,0.500000,0.500000}%
\pgfsetstrokecolor{currentstroke}%
\pgfsetstrokeopacity{0.300000}%
\pgfsetdash{}{0pt}%
\pgfpathmoveto{\pgfqpoint{0.000000in}{0.000000in}}%
\pgfpathlineto{\pgfqpoint{0.000000in}{0.000000in}}%
\pgfpathclose%
\pgfusepath{stroke,fill}%
\end{pgfscope}%
\begin{pgfscope}%
\pgfpathrectangle{\pgfqpoint{0.647939in}{0.492442in}}{\pgfqpoint{3.079299in}{3.079299in}}%
\pgfusepath{clip}%
\pgfsetroundcap%
\pgfsetroundjoin%
\pgfsetlinewidth{0.301125pt}%
\definecolor{currentstroke}{rgb}{0.500000,0.500000,0.500000}%
\pgfsetstrokecolor{currentstroke}%
\pgfsetstrokeopacity{0.300000}%
\pgfsetdash{}{0pt}%
\pgfpathmoveto{\pgfqpoint{3.536222in}{2.700078in}}%
\pgfusepath{stroke}%
\end{pgfscope}%
\begin{pgfscope}%
\pgfpathrectangle{\pgfqpoint{0.647939in}{0.492442in}}{\pgfqpoint{3.079299in}{3.079299in}}%
\pgfusepath{clip}%
\pgfsetroundcap%
\pgfsetroundjoin%
\definecolor{currentfill}{rgb}{0.500000,0.500000,0.500000}%
\pgfsetfillcolor{currentfill}%
\pgfsetfillopacity{0.300000}%
\pgfsetlinewidth{0.301125pt}%
\definecolor{currentstroke}{rgb}{0.500000,0.500000,0.500000}%
\pgfsetstrokecolor{currentstroke}%
\pgfsetstrokeopacity{0.300000}%
\pgfsetdash{}{0pt}%
\pgfpathmoveto{\pgfqpoint{0.000000in}{0.000000in}}%
\pgfpathlineto{\pgfqpoint{0.000000in}{0.000000in}}%
\pgfpathclose%
\pgfusepath{stroke,fill}%
\end{pgfscope}%
\begin{pgfscope}%
\pgfpathrectangle{\pgfqpoint{0.647939in}{0.492442in}}{\pgfqpoint{3.079299in}{3.079299in}}%
\pgfusepath{clip}%
\pgfsetroundcap%
\pgfsetroundjoin%
\pgfsetlinewidth{0.301125pt}%
\definecolor{currentstroke}{rgb}{0.500000,0.500000,0.500000}%
\pgfsetstrokecolor{currentstroke}%
\pgfsetstrokeopacity{0.300000}%
\pgfsetdash{}{0pt}%
\pgfpathmoveto{\pgfqpoint{3.618289in}{2.660664in}}%
\pgfusepath{stroke}%
\end{pgfscope}%
\begin{pgfscope}%
\pgfpathrectangle{\pgfqpoint{0.647939in}{0.492442in}}{\pgfqpoint{3.079299in}{3.079299in}}%
\pgfusepath{clip}%
\pgfsetroundcap%
\pgfsetroundjoin%
\definecolor{currentfill}{rgb}{0.500000,0.500000,0.500000}%
\pgfsetfillcolor{currentfill}%
\pgfsetfillopacity{0.300000}%
\pgfsetlinewidth{0.301125pt}%
\definecolor{currentstroke}{rgb}{0.500000,0.500000,0.500000}%
\pgfsetstrokecolor{currentstroke}%
\pgfsetstrokeopacity{0.300000}%
\pgfsetdash{}{0pt}%
\pgfpathmoveto{\pgfqpoint{0.000000in}{0.000000in}}%
\pgfpathlineto{\pgfqpoint{0.000000in}{0.000000in}}%
\pgfpathclose%
\pgfusepath{stroke,fill}%
\end{pgfscope}%
\begin{pgfscope}%
\pgfpathrectangle{\pgfqpoint{0.647939in}{0.492442in}}{\pgfqpoint{3.079299in}{3.079299in}}%
\pgfusepath{clip}%
\pgfsetroundcap%
\pgfsetroundjoin%
\pgfsetlinewidth{0.301125pt}%
\definecolor{currentstroke}{rgb}{0.500000,0.500000,0.500000}%
\pgfsetstrokecolor{currentstroke}%
\pgfsetstrokeopacity{0.300000}%
\pgfsetdash{}{0pt}%
\pgfpathmoveto{\pgfqpoint{3.702452in}{2.684020in}}%
\pgfusepath{stroke}%
\end{pgfscope}%
\begin{pgfscope}%
\pgfpathrectangle{\pgfqpoint{0.647939in}{0.492442in}}{\pgfqpoint{3.079299in}{3.079299in}}%
\pgfusepath{clip}%
\pgfsetroundcap%
\pgfsetroundjoin%
\definecolor{currentfill}{rgb}{0.500000,0.500000,0.500000}%
\pgfsetfillcolor{currentfill}%
\pgfsetfillopacity{0.300000}%
\pgfsetlinewidth{0.301125pt}%
\definecolor{currentstroke}{rgb}{0.500000,0.500000,0.500000}%
\pgfsetstrokecolor{currentstroke}%
\pgfsetstrokeopacity{0.300000}%
\pgfsetdash{}{0pt}%
\pgfpathmoveto{\pgfqpoint{0.000000in}{0.000000in}}%
\pgfpathlineto{\pgfqpoint{0.000000in}{0.000000in}}%
\pgfpathclose%
\pgfusepath{stroke,fill}%
\end{pgfscope}%
\begin{pgfscope}%
\pgfpathrectangle{\pgfqpoint{0.647939in}{0.492442in}}{\pgfqpoint{3.079299in}{3.079299in}}%
\pgfusepath{clip}%
\pgfsetroundcap%
\pgfsetroundjoin%
\pgfsetlinewidth{0.301125pt}%
\definecolor{currentstroke}{rgb}{0.500000,0.500000,0.500000}%
\pgfsetstrokecolor{currentstroke}%
\pgfsetstrokeopacity{0.300000}%
\pgfsetdash{}{0pt}%
\pgfpathmoveto{\pgfqpoint{2.162456in}{2.828263in}}%
\pgfusepath{stroke}%
\end{pgfscope}%
\begin{pgfscope}%
\pgfpathrectangle{\pgfqpoint{0.647939in}{0.492442in}}{\pgfqpoint{3.079299in}{3.079299in}}%
\pgfusepath{clip}%
\pgfsetroundcap%
\pgfsetroundjoin%
\definecolor{currentfill}{rgb}{0.500000,0.500000,0.500000}%
\pgfsetfillcolor{currentfill}%
\pgfsetfillopacity{0.300000}%
\pgfsetlinewidth{0.301125pt}%
\definecolor{currentstroke}{rgb}{0.500000,0.500000,0.500000}%
\pgfsetstrokecolor{currentstroke}%
\pgfsetstrokeopacity{0.300000}%
\pgfsetdash{}{0pt}%
\pgfpathmoveto{\pgfqpoint{0.000000in}{0.000000in}}%
\pgfpathlineto{\pgfqpoint{0.000000in}{0.000000in}}%
\pgfpathclose%
\pgfusepath{stroke,fill}%
\end{pgfscope}%
\begin{pgfscope}%
\pgfpathrectangle{\pgfqpoint{0.647939in}{0.492442in}}{\pgfqpoint{3.079299in}{3.079299in}}%
\pgfusepath{clip}%
\pgfsetroundcap%
\pgfsetroundjoin%
\pgfsetlinewidth{0.301125pt}%
\definecolor{currentstroke}{rgb}{0.500000,0.500000,0.500000}%
\pgfsetstrokecolor{currentstroke}%
\pgfsetstrokeopacity{0.300000}%
\pgfsetdash{}{0pt}%
\pgfpathmoveto{\pgfqpoint{2.036703in}{3.093607in}}%
\pgfusepath{stroke}%
\end{pgfscope}%
\begin{pgfscope}%
\pgfpathrectangle{\pgfqpoint{0.647939in}{0.492442in}}{\pgfqpoint{3.079299in}{3.079299in}}%
\pgfusepath{clip}%
\pgfsetroundcap%
\pgfsetroundjoin%
\definecolor{currentfill}{rgb}{0.500000,0.500000,0.500000}%
\pgfsetfillcolor{currentfill}%
\pgfsetfillopacity{0.300000}%
\pgfsetlinewidth{0.301125pt}%
\definecolor{currentstroke}{rgb}{0.500000,0.500000,0.500000}%
\pgfsetstrokecolor{currentstroke}%
\pgfsetstrokeopacity{0.300000}%
\pgfsetdash{}{0pt}%
\pgfpathmoveto{\pgfqpoint{0.000000in}{0.000000in}}%
\pgfpathlineto{\pgfqpoint{0.000000in}{0.000000in}}%
\pgfpathclose%
\pgfusepath{stroke,fill}%
\end{pgfscope}%
\begin{pgfscope}%
\pgfpathrectangle{\pgfqpoint{0.647939in}{0.492442in}}{\pgfqpoint{3.079299in}{3.079299in}}%
\pgfusepath{clip}%
\pgfsetroundcap%
\pgfsetroundjoin%
\pgfsetlinewidth{0.301125pt}%
\definecolor{currentstroke}{rgb}{0.500000,0.500000,0.500000}%
\pgfsetstrokecolor{currentstroke}%
\pgfsetstrokeopacity{0.300000}%
\pgfsetdash{}{0pt}%
\pgfpathmoveto{\pgfqpoint{1.900614in}{3.256099in}}%
\pgfusepath{stroke}%
\end{pgfscope}%
\begin{pgfscope}%
\pgfpathrectangle{\pgfqpoint{0.647939in}{0.492442in}}{\pgfqpoint{3.079299in}{3.079299in}}%
\pgfusepath{clip}%
\pgfsetroundcap%
\pgfsetroundjoin%
\definecolor{currentfill}{rgb}{0.500000,0.500000,0.500000}%
\pgfsetfillcolor{currentfill}%
\pgfsetfillopacity{0.300000}%
\pgfsetlinewidth{0.301125pt}%
\definecolor{currentstroke}{rgb}{0.500000,0.500000,0.500000}%
\pgfsetstrokecolor{currentstroke}%
\pgfsetstrokeopacity{0.300000}%
\pgfsetdash{}{0pt}%
\pgfpathmoveto{\pgfqpoint{0.000000in}{0.000000in}}%
\pgfpathlineto{\pgfqpoint{0.000000in}{0.000000in}}%
\pgfpathclose%
\pgfusepath{stroke,fill}%
\end{pgfscope}%
\begin{pgfscope}%
\pgfpathrectangle{\pgfqpoint{0.647939in}{0.492442in}}{\pgfqpoint{3.079299in}{3.079299in}}%
\pgfusepath{clip}%
\pgfsetroundcap%
\pgfsetroundjoin%
\pgfsetlinewidth{0.301125pt}%
\definecolor{currentstroke}{rgb}{0.500000,0.500000,0.500000}%
\pgfsetstrokecolor{currentstroke}%
\pgfsetstrokeopacity{0.300000}%
\pgfsetdash{}{0pt}%
\pgfpathmoveto{\pgfqpoint{1.859699in}{3.371982in}}%
\pgfusepath{stroke}%
\end{pgfscope}%
\begin{pgfscope}%
\pgfpathrectangle{\pgfqpoint{0.647939in}{0.492442in}}{\pgfqpoint{3.079299in}{3.079299in}}%
\pgfusepath{clip}%
\pgfsetroundcap%
\pgfsetroundjoin%
\definecolor{currentfill}{rgb}{0.500000,0.500000,0.500000}%
\pgfsetfillcolor{currentfill}%
\pgfsetfillopacity{0.300000}%
\pgfsetlinewidth{0.301125pt}%
\definecolor{currentstroke}{rgb}{0.500000,0.500000,0.500000}%
\pgfsetstrokecolor{currentstroke}%
\pgfsetstrokeopacity{0.300000}%
\pgfsetdash{}{0pt}%
\pgfpathmoveto{\pgfqpoint{0.000000in}{0.000000in}}%
\pgfpathlineto{\pgfqpoint{0.000000in}{0.000000in}}%
\pgfpathclose%
\pgfusepath{stroke,fill}%
\end{pgfscope}%
\begin{pgfscope}%
\pgfpathrectangle{\pgfqpoint{0.647939in}{0.492442in}}{\pgfqpoint{3.079299in}{3.079299in}}%
\pgfusepath{clip}%
\pgfsetroundcap%
\pgfsetroundjoin%
\pgfsetlinewidth{0.301125pt}%
\definecolor{currentstroke}{rgb}{0.500000,0.500000,0.500000}%
\pgfsetstrokecolor{currentstroke}%
\pgfsetstrokeopacity{0.300000}%
\pgfsetdash{}{0pt}%
\pgfpathmoveto{\pgfqpoint{1.767497in}{3.447399in}}%
\pgfusepath{stroke}%
\end{pgfscope}%
\begin{pgfscope}%
\pgfpathrectangle{\pgfqpoint{0.647939in}{0.492442in}}{\pgfqpoint{3.079299in}{3.079299in}}%
\pgfusepath{clip}%
\pgfsetroundcap%
\pgfsetroundjoin%
\definecolor{currentfill}{rgb}{0.500000,0.500000,0.500000}%
\pgfsetfillcolor{currentfill}%
\pgfsetfillopacity{0.300000}%
\pgfsetlinewidth{0.301125pt}%
\definecolor{currentstroke}{rgb}{0.500000,0.500000,0.500000}%
\pgfsetstrokecolor{currentstroke}%
\pgfsetstrokeopacity{0.300000}%
\pgfsetdash{}{0pt}%
\pgfpathmoveto{\pgfqpoint{0.000000in}{0.000000in}}%
\pgfpathlineto{\pgfqpoint{0.000000in}{0.000000in}}%
\pgfpathclose%
\pgfusepath{stroke,fill}%
\end{pgfscope}%
\begin{pgfscope}%
\pgfpathrectangle{\pgfqpoint{0.647939in}{0.492442in}}{\pgfqpoint{3.079299in}{3.079299in}}%
\pgfusepath{clip}%
\pgfsetroundcap%
\pgfsetroundjoin%
\pgfsetlinewidth{0.301125pt}%
\definecolor{currentstroke}{rgb}{0.500000,0.500000,0.500000}%
\pgfsetstrokecolor{currentstroke}%
\pgfsetstrokeopacity{0.300000}%
\pgfsetdash{}{0pt}%
\pgfpathmoveto{\pgfqpoint{1.682368in}{3.512211in}}%
\pgfusepath{stroke}%
\end{pgfscope}%
\begin{pgfscope}%
\pgfpathrectangle{\pgfqpoint{0.647939in}{0.492442in}}{\pgfqpoint{3.079299in}{3.079299in}}%
\pgfusepath{clip}%
\pgfsetroundcap%
\pgfsetroundjoin%
\definecolor{currentfill}{rgb}{0.500000,0.500000,0.500000}%
\pgfsetfillcolor{currentfill}%
\pgfsetfillopacity{0.300000}%
\pgfsetlinewidth{0.301125pt}%
\definecolor{currentstroke}{rgb}{0.500000,0.500000,0.500000}%
\pgfsetstrokecolor{currentstroke}%
\pgfsetstrokeopacity{0.300000}%
\pgfsetdash{}{0pt}%
\pgfpathmoveto{\pgfqpoint{0.000000in}{0.000000in}}%
\pgfpathlineto{\pgfqpoint{0.000000in}{0.000000in}}%
\pgfpathclose%
\pgfusepath{stroke,fill}%
\end{pgfscope}%
\begin{pgfscope}%
\pgfpathrectangle{\pgfqpoint{0.647939in}{0.492442in}}{\pgfqpoint{3.079299in}{3.079299in}}%
\pgfusepath{clip}%
\pgfsetroundcap%
\pgfsetroundjoin%
\pgfsetlinewidth{0.301125pt}%
\definecolor{currentstroke}{rgb}{0.500000,0.500000,0.500000}%
\pgfsetstrokecolor{currentstroke}%
\pgfsetstrokeopacity{0.300000}%
\pgfsetdash{}{0pt}%
\pgfpathmoveto{\pgfqpoint{1.870235in}{3.565644in}}%
\pgfusepath{stroke}%
\end{pgfscope}%
\begin{pgfscope}%
\pgfpathrectangle{\pgfqpoint{0.647939in}{0.492442in}}{\pgfqpoint{3.079299in}{3.079299in}}%
\pgfusepath{clip}%
\pgfsetroundcap%
\pgfsetroundjoin%
\definecolor{currentfill}{rgb}{0.500000,0.500000,0.500000}%
\pgfsetfillcolor{currentfill}%
\pgfsetfillopacity{0.300000}%
\pgfsetlinewidth{0.301125pt}%
\definecolor{currentstroke}{rgb}{0.500000,0.500000,0.500000}%
\pgfsetstrokecolor{currentstroke}%
\pgfsetstrokeopacity{0.300000}%
\pgfsetdash{}{0pt}%
\pgfpathmoveto{\pgfqpoint{0.000000in}{0.000000in}}%
\pgfpathlineto{\pgfqpoint{0.000000in}{0.000000in}}%
\pgfpathclose%
\pgfusepath{stroke,fill}%
\end{pgfscope}%
\begin{pgfscope}%
\pgfpathrectangle{\pgfqpoint{0.647939in}{0.492442in}}{\pgfqpoint{3.079299in}{3.079299in}}%
\pgfusepath{clip}%
\pgfsetroundcap%
\pgfsetroundjoin%
\pgfsetlinewidth{0.301125pt}%
\definecolor{currentstroke}{rgb}{0.500000,0.500000,0.500000}%
\pgfsetstrokecolor{currentstroke}%
\pgfsetstrokeopacity{0.300000}%
\pgfsetdash{}{0pt}%
\pgfpathmoveto{\pgfqpoint{1.114566in}{3.486960in}}%
\pgfusepath{stroke}%
\end{pgfscope}%
\begin{pgfscope}%
\pgfpathrectangle{\pgfqpoint{0.647939in}{0.492442in}}{\pgfqpoint{3.079299in}{3.079299in}}%
\pgfusepath{clip}%
\pgfsetroundcap%
\pgfsetroundjoin%
\definecolor{currentfill}{rgb}{0.500000,0.500000,0.500000}%
\pgfsetfillcolor{currentfill}%
\pgfsetfillopacity{0.300000}%
\pgfsetlinewidth{0.301125pt}%
\definecolor{currentstroke}{rgb}{0.500000,0.500000,0.500000}%
\pgfsetstrokecolor{currentstroke}%
\pgfsetstrokeopacity{0.300000}%
\pgfsetdash{}{0pt}%
\pgfpathmoveto{\pgfqpoint{0.000000in}{0.000000in}}%
\pgfpathlineto{\pgfqpoint{0.000000in}{0.000000in}}%
\pgfpathclose%
\pgfusepath{stroke,fill}%
\end{pgfscope}%
\begin{pgfscope}%
\pgfpathrectangle{\pgfqpoint{0.647939in}{0.492442in}}{\pgfqpoint{3.079299in}{3.079299in}}%
\pgfusepath{clip}%
\pgfsetroundcap%
\pgfsetroundjoin%
\pgfsetlinewidth{0.301125pt}%
\definecolor{currentstroke}{rgb}{0.500000,0.500000,0.500000}%
\pgfsetstrokecolor{currentstroke}%
\pgfsetstrokeopacity{0.300000}%
\pgfsetdash{}{0pt}%
\pgfpathmoveto{\pgfqpoint{0.895625in}{3.529082in}}%
\pgfusepath{stroke}%
\end{pgfscope}%
\begin{pgfscope}%
\pgfpathrectangle{\pgfqpoint{0.647939in}{0.492442in}}{\pgfqpoint{3.079299in}{3.079299in}}%
\pgfusepath{clip}%
\pgfsetroundcap%
\pgfsetroundjoin%
\definecolor{currentfill}{rgb}{0.500000,0.500000,0.500000}%
\pgfsetfillcolor{currentfill}%
\pgfsetfillopacity{0.300000}%
\pgfsetlinewidth{0.301125pt}%
\definecolor{currentstroke}{rgb}{0.500000,0.500000,0.500000}%
\pgfsetstrokecolor{currentstroke}%
\pgfsetstrokeopacity{0.300000}%
\pgfsetdash{}{0pt}%
\pgfpathmoveto{\pgfqpoint{0.000000in}{0.000000in}}%
\pgfpathlineto{\pgfqpoint{0.000000in}{0.000000in}}%
\pgfpathclose%
\pgfusepath{stroke,fill}%
\end{pgfscope}%
\begin{pgfscope}%
\pgfpathrectangle{\pgfqpoint{0.647939in}{0.492442in}}{\pgfqpoint{3.079299in}{3.079299in}}%
\pgfusepath{clip}%
\pgfsetroundcap%
\pgfsetroundjoin%
\pgfsetlinewidth{0.301125pt}%
\definecolor{currentstroke}{rgb}{0.500000,0.500000,0.500000}%
\pgfsetstrokecolor{currentstroke}%
\pgfsetstrokeopacity{0.300000}%
\pgfsetdash{}{0pt}%
\pgfpathmoveto{\pgfqpoint{1.749967in}{3.151787in}}%
\pgfusepath{stroke}%
\end{pgfscope}%
\begin{pgfscope}%
\pgfpathrectangle{\pgfqpoint{0.647939in}{0.492442in}}{\pgfqpoint{3.079299in}{3.079299in}}%
\pgfusepath{clip}%
\pgfsetroundcap%
\pgfsetroundjoin%
\definecolor{currentfill}{rgb}{0.500000,0.500000,0.500000}%
\pgfsetfillcolor{currentfill}%
\pgfsetfillopacity{0.300000}%
\pgfsetlinewidth{0.301125pt}%
\definecolor{currentstroke}{rgb}{0.500000,0.500000,0.500000}%
\pgfsetstrokecolor{currentstroke}%
\pgfsetstrokeopacity{0.300000}%
\pgfsetdash{}{0pt}%
\pgfpathmoveto{\pgfqpoint{0.000000in}{0.000000in}}%
\pgfpathlineto{\pgfqpoint{0.000000in}{0.000000in}}%
\pgfpathclose%
\pgfusepath{stroke,fill}%
\end{pgfscope}%
\begin{pgfscope}%
\pgfpathrectangle{\pgfqpoint{0.647939in}{0.492442in}}{\pgfqpoint{3.079299in}{3.079299in}}%
\pgfusepath{clip}%
\pgfsetroundcap%
\pgfsetroundjoin%
\pgfsetlinewidth{0.301125pt}%
\definecolor{currentstroke}{rgb}{0.500000,0.500000,0.500000}%
\pgfsetstrokecolor{currentstroke}%
\pgfsetstrokeopacity{0.300000}%
\pgfsetdash{}{0pt}%
\pgfpathmoveto{\pgfqpoint{1.882643in}{2.972888in}}%
\pgfusepath{stroke}%
\end{pgfscope}%
\begin{pgfscope}%
\pgfpathrectangle{\pgfqpoint{0.647939in}{0.492442in}}{\pgfqpoint{3.079299in}{3.079299in}}%
\pgfusepath{clip}%
\pgfsetroundcap%
\pgfsetroundjoin%
\definecolor{currentfill}{rgb}{0.500000,0.500000,0.500000}%
\pgfsetfillcolor{currentfill}%
\pgfsetfillopacity{0.300000}%
\pgfsetlinewidth{0.301125pt}%
\definecolor{currentstroke}{rgb}{0.500000,0.500000,0.500000}%
\pgfsetstrokecolor{currentstroke}%
\pgfsetstrokeopacity{0.300000}%
\pgfsetdash{}{0pt}%
\pgfpathmoveto{\pgfqpoint{0.000000in}{0.000000in}}%
\pgfpathlineto{\pgfqpoint{0.000000in}{0.000000in}}%
\pgfpathclose%
\pgfusepath{stroke,fill}%
\end{pgfscope}%
\begin{pgfscope}%
\pgfpathrectangle{\pgfqpoint{0.647939in}{0.492442in}}{\pgfqpoint{3.079299in}{3.079299in}}%
\pgfusepath{clip}%
\pgfsetroundcap%
\pgfsetroundjoin%
\pgfsetlinewidth{0.301125pt}%
\definecolor{currentstroke}{rgb}{0.500000,0.500000,0.500000}%
\pgfsetstrokecolor{currentstroke}%
\pgfsetstrokeopacity{0.300000}%
\pgfsetdash{}{0pt}%
\pgfpathmoveto{\pgfqpoint{1.212152in}{2.771081in}}%
\pgfusepath{stroke}%
\end{pgfscope}%
\begin{pgfscope}%
\pgfpathrectangle{\pgfqpoint{0.647939in}{0.492442in}}{\pgfqpoint{3.079299in}{3.079299in}}%
\pgfusepath{clip}%
\pgfsetroundcap%
\pgfsetroundjoin%
\definecolor{currentfill}{rgb}{0.500000,0.500000,0.500000}%
\pgfsetfillcolor{currentfill}%
\pgfsetfillopacity{0.300000}%
\pgfsetlinewidth{0.301125pt}%
\definecolor{currentstroke}{rgb}{0.500000,0.500000,0.500000}%
\pgfsetstrokecolor{currentstroke}%
\pgfsetstrokeopacity{0.300000}%
\pgfsetdash{}{0pt}%
\pgfpathmoveto{\pgfqpoint{0.000000in}{0.000000in}}%
\pgfpathlineto{\pgfqpoint{0.000000in}{0.000000in}}%
\pgfpathclose%
\pgfusepath{stroke,fill}%
\end{pgfscope}%
\begin{pgfscope}%
\pgfpathrectangle{\pgfqpoint{0.647939in}{0.492442in}}{\pgfqpoint{3.079299in}{3.079299in}}%
\pgfusepath{clip}%
\pgfsetroundcap%
\pgfsetroundjoin%
\pgfsetlinewidth{0.301125pt}%
\definecolor{currentstroke}{rgb}{0.500000,0.500000,0.500000}%
\pgfsetstrokecolor{currentstroke}%
\pgfsetstrokeopacity{0.300000}%
\pgfsetdash{}{0pt}%
\pgfpathmoveto{\pgfqpoint{1.606891in}{2.675244in}}%
\pgfusepath{stroke}%
\end{pgfscope}%
\begin{pgfscope}%
\pgfpathrectangle{\pgfqpoint{0.647939in}{0.492442in}}{\pgfqpoint{3.079299in}{3.079299in}}%
\pgfusepath{clip}%
\pgfsetroundcap%
\pgfsetroundjoin%
\definecolor{currentfill}{rgb}{0.500000,0.500000,0.500000}%
\pgfsetfillcolor{currentfill}%
\pgfsetfillopacity{0.300000}%
\pgfsetlinewidth{0.301125pt}%
\definecolor{currentstroke}{rgb}{0.500000,0.500000,0.500000}%
\pgfsetstrokecolor{currentstroke}%
\pgfsetstrokeopacity{0.300000}%
\pgfsetdash{}{0pt}%
\pgfpathmoveto{\pgfqpoint{0.000000in}{0.000000in}}%
\pgfpathlineto{\pgfqpoint{0.000000in}{0.000000in}}%
\pgfpathclose%
\pgfusepath{stroke,fill}%
\end{pgfscope}%
\begin{pgfscope}%
\pgfpathrectangle{\pgfqpoint{0.647939in}{0.492442in}}{\pgfqpoint{3.079299in}{3.079299in}}%
\pgfusepath{clip}%
\pgfsetroundcap%
\pgfsetroundjoin%
\pgfsetlinewidth{0.301125pt}%
\definecolor{currentstroke}{rgb}{0.500000,0.500000,0.500000}%
\pgfsetstrokecolor{currentstroke}%
\pgfsetstrokeopacity{0.300000}%
\pgfsetdash{}{0pt}%
\pgfpathmoveto{\pgfqpoint{1.539096in}{2.594162in}}%
\pgfusepath{stroke}%
\end{pgfscope}%
\begin{pgfscope}%
\pgfpathrectangle{\pgfqpoint{0.647939in}{0.492442in}}{\pgfqpoint{3.079299in}{3.079299in}}%
\pgfusepath{clip}%
\pgfsetroundcap%
\pgfsetroundjoin%
\definecolor{currentfill}{rgb}{0.500000,0.500000,0.500000}%
\pgfsetfillcolor{currentfill}%
\pgfsetfillopacity{0.300000}%
\pgfsetlinewidth{0.301125pt}%
\definecolor{currentstroke}{rgb}{0.500000,0.500000,0.500000}%
\pgfsetstrokecolor{currentstroke}%
\pgfsetstrokeopacity{0.300000}%
\pgfsetdash{}{0pt}%
\pgfpathmoveto{\pgfqpoint{0.000000in}{0.000000in}}%
\pgfpathlineto{\pgfqpoint{0.000000in}{0.000000in}}%
\pgfpathclose%
\pgfusepath{stroke,fill}%
\end{pgfscope}%
\begin{pgfscope}%
\pgfpathrectangle{\pgfqpoint{0.647939in}{0.492442in}}{\pgfqpoint{3.079299in}{3.079299in}}%
\pgfusepath{clip}%
\pgfsetroundcap%
\pgfsetroundjoin%
\pgfsetlinewidth{0.301125pt}%
\definecolor{currentstroke}{rgb}{0.500000,0.500000,0.500000}%
\pgfsetstrokecolor{currentstroke}%
\pgfsetstrokeopacity{0.300000}%
\pgfsetdash{}{0pt}%
\pgfpathmoveto{\pgfqpoint{1.143931in}{2.413240in}}%
\pgfusepath{stroke}%
\end{pgfscope}%
\begin{pgfscope}%
\pgfpathrectangle{\pgfqpoint{0.647939in}{0.492442in}}{\pgfqpoint{3.079299in}{3.079299in}}%
\pgfusepath{clip}%
\pgfsetroundcap%
\pgfsetroundjoin%
\definecolor{currentfill}{rgb}{0.500000,0.500000,0.500000}%
\pgfsetfillcolor{currentfill}%
\pgfsetfillopacity{0.300000}%
\pgfsetlinewidth{0.301125pt}%
\definecolor{currentstroke}{rgb}{0.500000,0.500000,0.500000}%
\pgfsetstrokecolor{currentstroke}%
\pgfsetstrokeopacity{0.300000}%
\pgfsetdash{}{0pt}%
\pgfpathmoveto{\pgfqpoint{0.000000in}{0.000000in}}%
\pgfpathlineto{\pgfqpoint{0.000000in}{0.000000in}}%
\pgfpathclose%
\pgfusepath{stroke,fill}%
\end{pgfscope}%
\begin{pgfscope}%
\pgfpathrectangle{\pgfqpoint{0.647939in}{0.492442in}}{\pgfqpoint{3.079299in}{3.079299in}}%
\pgfusepath{clip}%
\pgfsetroundcap%
\pgfsetroundjoin%
\pgfsetlinewidth{0.301125pt}%
\definecolor{currentstroke}{rgb}{0.500000,0.500000,0.500000}%
\pgfsetstrokecolor{currentstroke}%
\pgfsetstrokeopacity{0.300000}%
\pgfsetdash{}{0pt}%
\pgfpathmoveto{\pgfqpoint{1.011343in}{2.309673in}}%
\pgfusepath{stroke}%
\end{pgfscope}%
\begin{pgfscope}%
\pgfpathrectangle{\pgfqpoint{0.647939in}{0.492442in}}{\pgfqpoint{3.079299in}{3.079299in}}%
\pgfusepath{clip}%
\pgfsetroundcap%
\pgfsetroundjoin%
\definecolor{currentfill}{rgb}{0.500000,0.500000,0.500000}%
\pgfsetfillcolor{currentfill}%
\pgfsetfillopacity{0.300000}%
\pgfsetlinewidth{0.301125pt}%
\definecolor{currentstroke}{rgb}{0.500000,0.500000,0.500000}%
\pgfsetstrokecolor{currentstroke}%
\pgfsetstrokeopacity{0.300000}%
\pgfsetdash{}{0pt}%
\pgfpathmoveto{\pgfqpoint{0.000000in}{0.000000in}}%
\pgfpathlineto{\pgfqpoint{0.000000in}{0.000000in}}%
\pgfpathclose%
\pgfusepath{stroke,fill}%
\end{pgfscope}%
\begin{pgfscope}%
\pgfpathrectangle{\pgfqpoint{0.647939in}{0.492442in}}{\pgfqpoint{3.079299in}{3.079299in}}%
\pgfusepath{clip}%
\pgfsetroundcap%
\pgfsetroundjoin%
\pgfsetlinewidth{0.301125pt}%
\definecolor{currentstroke}{rgb}{0.500000,0.500000,0.500000}%
\pgfsetstrokecolor{currentstroke}%
\pgfsetstrokeopacity{0.300000}%
\pgfsetdash{}{0pt}%
\pgfpathmoveto{\pgfqpoint{1.011101in}{2.240895in}}%
\pgfusepath{stroke}%
\end{pgfscope}%
\begin{pgfscope}%
\pgfpathrectangle{\pgfqpoint{0.647939in}{0.492442in}}{\pgfqpoint{3.079299in}{3.079299in}}%
\pgfusepath{clip}%
\pgfsetroundcap%
\pgfsetroundjoin%
\definecolor{currentfill}{rgb}{0.500000,0.500000,0.500000}%
\pgfsetfillcolor{currentfill}%
\pgfsetfillopacity{0.300000}%
\pgfsetlinewidth{0.301125pt}%
\definecolor{currentstroke}{rgb}{0.500000,0.500000,0.500000}%
\pgfsetstrokecolor{currentstroke}%
\pgfsetstrokeopacity{0.300000}%
\pgfsetdash{}{0pt}%
\pgfpathmoveto{\pgfqpoint{0.000000in}{0.000000in}}%
\pgfpathlineto{\pgfqpoint{0.000000in}{0.000000in}}%
\pgfpathclose%
\pgfusepath{stroke,fill}%
\end{pgfscope}%
\begin{pgfscope}%
\pgfpathrectangle{\pgfqpoint{0.647939in}{0.492442in}}{\pgfqpoint{3.079299in}{3.079299in}}%
\pgfusepath{clip}%
\pgfsetroundcap%
\pgfsetroundjoin%
\pgfsetlinewidth{0.301125pt}%
\definecolor{currentstroke}{rgb}{0.500000,0.500000,0.500000}%
\pgfsetstrokecolor{currentstroke}%
\pgfsetstrokeopacity{0.300000}%
\pgfsetdash{}{0pt}%
\pgfpathmoveto{\pgfqpoint{1.729965in}{2.393534in}}%
\pgfusepath{stroke}%
\end{pgfscope}%
\begin{pgfscope}%
\pgfpathrectangle{\pgfqpoint{0.647939in}{0.492442in}}{\pgfqpoint{3.079299in}{3.079299in}}%
\pgfusepath{clip}%
\pgfsetroundcap%
\pgfsetroundjoin%
\definecolor{currentfill}{rgb}{0.500000,0.500000,0.500000}%
\pgfsetfillcolor{currentfill}%
\pgfsetfillopacity{0.300000}%
\pgfsetlinewidth{0.301125pt}%
\definecolor{currentstroke}{rgb}{0.500000,0.500000,0.500000}%
\pgfsetstrokecolor{currentstroke}%
\pgfsetstrokeopacity{0.300000}%
\pgfsetdash{}{0pt}%
\pgfpathmoveto{\pgfqpoint{0.000000in}{0.000000in}}%
\pgfpathlineto{\pgfqpoint{0.000000in}{0.000000in}}%
\pgfpathclose%
\pgfusepath{stroke,fill}%
\end{pgfscope}%
\begin{pgfscope}%
\pgfpathrectangle{\pgfqpoint{0.647939in}{0.492442in}}{\pgfqpoint{3.079299in}{3.079299in}}%
\pgfusepath{clip}%
\pgfsetroundcap%
\pgfsetroundjoin%
\pgfsetlinewidth{0.301125pt}%
\definecolor{currentstroke}{rgb}{0.500000,0.500000,0.500000}%
\pgfsetstrokecolor{currentstroke}%
\pgfsetstrokeopacity{0.300000}%
\pgfsetdash{}{0pt}%
\pgfpathmoveto{\pgfqpoint{1.207134in}{2.162522in}}%
\pgfusepath{stroke}%
\end{pgfscope}%
\begin{pgfscope}%
\pgfpathrectangle{\pgfqpoint{0.647939in}{0.492442in}}{\pgfqpoint{3.079299in}{3.079299in}}%
\pgfusepath{clip}%
\pgfsetroundcap%
\pgfsetroundjoin%
\definecolor{currentfill}{rgb}{0.500000,0.500000,0.500000}%
\pgfsetfillcolor{currentfill}%
\pgfsetfillopacity{0.300000}%
\pgfsetlinewidth{0.301125pt}%
\definecolor{currentstroke}{rgb}{0.500000,0.500000,0.500000}%
\pgfsetstrokecolor{currentstroke}%
\pgfsetstrokeopacity{0.300000}%
\pgfsetdash{}{0pt}%
\pgfpathmoveto{\pgfqpoint{0.000000in}{0.000000in}}%
\pgfpathlineto{\pgfqpoint{0.000000in}{0.000000in}}%
\pgfpathclose%
\pgfusepath{stroke,fill}%
\end{pgfscope}%
\begin{pgfscope}%
\pgfpathrectangle{\pgfqpoint{0.647939in}{0.492442in}}{\pgfqpoint{3.079299in}{3.079299in}}%
\pgfusepath{clip}%
\pgfsetroundcap%
\pgfsetroundjoin%
\pgfsetlinewidth{0.301125pt}%
\definecolor{currentstroke}{rgb}{0.500000,0.500000,0.500000}%
\pgfsetstrokecolor{currentstroke}%
\pgfsetstrokeopacity{0.300000}%
\pgfsetdash{}{0pt}%
\pgfpathmoveto{\pgfqpoint{1.206375in}{2.095405in}}%
\pgfusepath{stroke}%
\end{pgfscope}%
\begin{pgfscope}%
\pgfpathrectangle{\pgfqpoint{0.647939in}{0.492442in}}{\pgfqpoint{3.079299in}{3.079299in}}%
\pgfusepath{clip}%
\pgfsetroundcap%
\pgfsetroundjoin%
\definecolor{currentfill}{rgb}{0.500000,0.500000,0.500000}%
\pgfsetfillcolor{currentfill}%
\pgfsetfillopacity{0.300000}%
\pgfsetlinewidth{0.301125pt}%
\definecolor{currentstroke}{rgb}{0.500000,0.500000,0.500000}%
\pgfsetstrokecolor{currentstroke}%
\pgfsetstrokeopacity{0.300000}%
\pgfsetdash{}{0pt}%
\pgfpathmoveto{\pgfqpoint{0.000000in}{0.000000in}}%
\pgfpathlineto{\pgfqpoint{0.000000in}{0.000000in}}%
\pgfpathclose%
\pgfusepath{stroke,fill}%
\end{pgfscope}%
\begin{pgfscope}%
\pgfpathrectangle{\pgfqpoint{0.647939in}{0.492442in}}{\pgfqpoint{3.079299in}{3.079299in}}%
\pgfusepath{clip}%
\pgfsetroundcap%
\pgfsetroundjoin%
\pgfsetlinewidth{0.301125pt}%
\definecolor{currentstroke}{rgb}{0.500000,0.500000,0.500000}%
\pgfsetstrokecolor{currentstroke}%
\pgfsetstrokeopacity{0.300000}%
\pgfsetdash{}{0pt}%
\pgfpathmoveto{\pgfqpoint{0.810345in}{1.918779in}}%
\pgfusepath{stroke}%
\end{pgfscope}%
\begin{pgfscope}%
\pgfpathrectangle{\pgfqpoint{0.647939in}{0.492442in}}{\pgfqpoint{3.079299in}{3.079299in}}%
\pgfusepath{clip}%
\pgfsetroundcap%
\pgfsetroundjoin%
\definecolor{currentfill}{rgb}{0.500000,0.500000,0.500000}%
\pgfsetfillcolor{currentfill}%
\pgfsetfillopacity{0.300000}%
\pgfsetlinewidth{0.301125pt}%
\definecolor{currentstroke}{rgb}{0.500000,0.500000,0.500000}%
\pgfsetstrokecolor{currentstroke}%
\pgfsetstrokeopacity{0.300000}%
\pgfsetdash{}{0pt}%
\pgfpathmoveto{\pgfqpoint{0.000000in}{0.000000in}}%
\pgfpathlineto{\pgfqpoint{0.000000in}{0.000000in}}%
\pgfpathclose%
\pgfusepath{stroke,fill}%
\end{pgfscope}%
\begin{pgfscope}%
\pgfpathrectangle{\pgfqpoint{0.647939in}{0.492442in}}{\pgfqpoint{3.079299in}{3.079299in}}%
\pgfusepath{clip}%
\pgfsetroundcap%
\pgfsetroundjoin%
\pgfsetlinewidth{0.301125pt}%
\definecolor{currentstroke}{rgb}{0.500000,0.500000,0.500000}%
\pgfsetstrokecolor{currentstroke}%
\pgfsetstrokeopacity{0.300000}%
\pgfsetdash{}{0pt}%
\pgfpathmoveto{\pgfqpoint{1.711508in}{2.099115in}}%
\pgfusepath{stroke}%
\end{pgfscope}%
\begin{pgfscope}%
\pgfpathrectangle{\pgfqpoint{0.647939in}{0.492442in}}{\pgfqpoint{3.079299in}{3.079299in}}%
\pgfusepath{clip}%
\pgfsetroundcap%
\pgfsetroundjoin%
\definecolor{currentfill}{rgb}{0.500000,0.500000,0.500000}%
\pgfsetfillcolor{currentfill}%
\pgfsetfillopacity{0.300000}%
\pgfsetlinewidth{0.301125pt}%
\definecolor{currentstroke}{rgb}{0.500000,0.500000,0.500000}%
\pgfsetstrokecolor{currentstroke}%
\pgfsetstrokeopacity{0.300000}%
\pgfsetdash{}{0pt}%
\pgfpathmoveto{\pgfqpoint{0.000000in}{0.000000in}}%
\pgfpathlineto{\pgfqpoint{0.000000in}{0.000000in}}%
\pgfpathclose%
\pgfusepath{stroke,fill}%
\end{pgfscope}%
\begin{pgfscope}%
\pgfpathrectangle{\pgfqpoint{0.647939in}{0.492442in}}{\pgfqpoint{3.079299in}{3.079299in}}%
\pgfusepath{clip}%
\pgfsetroundcap%
\pgfsetroundjoin%
\pgfsetlinewidth{0.301125pt}%
\definecolor{currentstroke}{rgb}{0.500000,0.500000,0.500000}%
\pgfsetstrokecolor{currentstroke}%
\pgfsetstrokeopacity{0.300000}%
\pgfsetdash{}{0pt}%
\pgfpathmoveto{\pgfqpoint{1.138734in}{1.804075in}}%
\pgfusepath{stroke}%
\end{pgfscope}%
\begin{pgfscope}%
\pgfpathrectangle{\pgfqpoint{0.647939in}{0.492442in}}{\pgfqpoint{3.079299in}{3.079299in}}%
\pgfusepath{clip}%
\pgfsetroundcap%
\pgfsetroundjoin%
\definecolor{currentfill}{rgb}{0.500000,0.500000,0.500000}%
\pgfsetfillcolor{currentfill}%
\pgfsetfillopacity{0.300000}%
\pgfsetlinewidth{0.301125pt}%
\definecolor{currentstroke}{rgb}{0.500000,0.500000,0.500000}%
\pgfsetstrokecolor{currentstroke}%
\pgfsetstrokeopacity{0.300000}%
\pgfsetdash{}{0pt}%
\pgfpathmoveto{\pgfqpoint{0.000000in}{0.000000in}}%
\pgfpathlineto{\pgfqpoint{0.000000in}{0.000000in}}%
\pgfpathclose%
\pgfusepath{stroke,fill}%
\end{pgfscope}%
\begin{pgfscope}%
\pgfpathrectangle{\pgfqpoint{0.647939in}{0.492442in}}{\pgfqpoint{3.079299in}{3.079299in}}%
\pgfusepath{clip}%
\pgfsetroundcap%
\pgfsetroundjoin%
\pgfsetlinewidth{0.301125pt}%
\definecolor{currentstroke}{rgb}{0.500000,0.500000,0.500000}%
\pgfsetstrokecolor{currentstroke}%
\pgfsetstrokeopacity{0.300000}%
\pgfsetdash{}{0pt}%
\pgfpathmoveto{\pgfqpoint{1.008562in}{1.692404in}}%
\pgfusepath{stroke}%
\end{pgfscope}%
\begin{pgfscope}%
\pgfpathrectangle{\pgfqpoint{0.647939in}{0.492442in}}{\pgfqpoint{3.079299in}{3.079299in}}%
\pgfusepath{clip}%
\pgfsetroundcap%
\pgfsetroundjoin%
\definecolor{currentfill}{rgb}{0.500000,0.500000,0.500000}%
\pgfsetfillcolor{currentfill}%
\pgfsetfillopacity{0.300000}%
\pgfsetlinewidth{0.301125pt}%
\definecolor{currentstroke}{rgb}{0.500000,0.500000,0.500000}%
\pgfsetstrokecolor{currentstroke}%
\pgfsetstrokeopacity{0.300000}%
\pgfsetdash{}{0pt}%
\pgfpathmoveto{\pgfqpoint{0.000000in}{0.000000in}}%
\pgfpathlineto{\pgfqpoint{0.000000in}{0.000000in}}%
\pgfpathclose%
\pgfusepath{stroke,fill}%
\end{pgfscope}%
\begin{pgfscope}%
\pgfpathrectangle{\pgfqpoint{0.647939in}{0.492442in}}{\pgfqpoint{3.079299in}{3.079299in}}%
\pgfusepath{clip}%
\pgfsetroundcap%
\pgfsetroundjoin%
\pgfsetlinewidth{0.301125pt}%
\definecolor{currentstroke}{rgb}{0.500000,0.500000,0.500000}%
\pgfsetstrokecolor{currentstroke}%
\pgfsetstrokeopacity{0.300000}%
\pgfsetdash{}{0pt}%
\pgfpathmoveto{\pgfqpoint{1.387043in}{1.781096in}}%
\pgfusepath{stroke}%
\end{pgfscope}%
\begin{pgfscope}%
\pgfpathrectangle{\pgfqpoint{0.647939in}{0.492442in}}{\pgfqpoint{3.079299in}{3.079299in}}%
\pgfusepath{clip}%
\pgfsetroundcap%
\pgfsetroundjoin%
\definecolor{currentfill}{rgb}{0.500000,0.500000,0.500000}%
\pgfsetfillcolor{currentfill}%
\pgfsetfillopacity{0.300000}%
\pgfsetlinewidth{0.301125pt}%
\definecolor{currentstroke}{rgb}{0.500000,0.500000,0.500000}%
\pgfsetstrokecolor{currentstroke}%
\pgfsetstrokeopacity{0.300000}%
\pgfsetdash{}{0pt}%
\pgfpathmoveto{\pgfqpoint{0.000000in}{0.000000in}}%
\pgfpathlineto{\pgfqpoint{0.000000in}{0.000000in}}%
\pgfpathclose%
\pgfusepath{stroke,fill}%
\end{pgfscope}%
\begin{pgfscope}%
\pgfpathrectangle{\pgfqpoint{0.647939in}{0.492442in}}{\pgfqpoint{3.079299in}{3.079299in}}%
\pgfusepath{clip}%
\pgfsetroundcap%
\pgfsetroundjoin%
\pgfsetlinewidth{0.301125pt}%
\definecolor{currentstroke}{rgb}{0.500000,0.500000,0.500000}%
\pgfsetstrokecolor{currentstroke}%
\pgfsetstrokeopacity{0.300000}%
\pgfsetdash{}{0pt}%
\pgfpathmoveto{\pgfqpoint{1.136170in}{1.602895in}}%
\pgfusepath{stroke}%
\end{pgfscope}%
\begin{pgfscope}%
\pgfpathrectangle{\pgfqpoint{0.647939in}{0.492442in}}{\pgfqpoint{3.079299in}{3.079299in}}%
\pgfusepath{clip}%
\pgfsetroundcap%
\pgfsetroundjoin%
\definecolor{currentfill}{rgb}{0.500000,0.500000,0.500000}%
\pgfsetfillcolor{currentfill}%
\pgfsetfillopacity{0.300000}%
\pgfsetlinewidth{0.301125pt}%
\definecolor{currentstroke}{rgb}{0.500000,0.500000,0.500000}%
\pgfsetstrokecolor{currentstroke}%
\pgfsetstrokeopacity{0.300000}%
\pgfsetdash{}{0pt}%
\pgfpathmoveto{\pgfqpoint{0.000000in}{0.000000in}}%
\pgfpathlineto{\pgfqpoint{0.000000in}{0.000000in}}%
\pgfpathclose%
\pgfusepath{stroke,fill}%
\end{pgfscope}%
\begin{pgfscope}%
\pgfpathrectangle{\pgfqpoint{0.647939in}{0.492442in}}{\pgfqpoint{3.079299in}{3.079299in}}%
\pgfusepath{clip}%
\pgfsetroundcap%
\pgfsetroundjoin%
\pgfsetlinewidth{0.301125pt}%
\definecolor{currentstroke}{rgb}{0.500000,0.500000,0.500000}%
\pgfsetstrokecolor{currentstroke}%
\pgfsetstrokeopacity{0.300000}%
\pgfsetdash{}{0pt}%
\pgfpathmoveto{\pgfqpoint{0.876197in}{1.448469in}}%
\pgfusepath{stroke}%
\end{pgfscope}%
\begin{pgfscope}%
\pgfpathrectangle{\pgfqpoint{0.647939in}{0.492442in}}{\pgfqpoint{3.079299in}{3.079299in}}%
\pgfusepath{clip}%
\pgfsetroundcap%
\pgfsetroundjoin%
\definecolor{currentfill}{rgb}{0.500000,0.500000,0.500000}%
\pgfsetfillcolor{currentfill}%
\pgfsetfillopacity{0.300000}%
\pgfsetlinewidth{0.301125pt}%
\definecolor{currentstroke}{rgb}{0.500000,0.500000,0.500000}%
\pgfsetstrokecolor{currentstroke}%
\pgfsetstrokeopacity{0.300000}%
\pgfsetdash{}{0pt}%
\pgfpathmoveto{\pgfqpoint{0.000000in}{0.000000in}}%
\pgfpathlineto{\pgfqpoint{0.000000in}{0.000000in}}%
\pgfpathclose%
\pgfusepath{stroke,fill}%
\end{pgfscope}%
\begin{pgfscope}%
\pgfpathrectangle{\pgfqpoint{0.647939in}{0.492442in}}{\pgfqpoint{3.079299in}{3.079299in}}%
\pgfusepath{clip}%
\pgfsetroundcap%
\pgfsetroundjoin%
\pgfsetlinewidth{0.301125pt}%
\definecolor{currentstroke}{rgb}{0.500000,0.500000,0.500000}%
\pgfsetstrokecolor{currentstroke}%
\pgfsetstrokeopacity{0.300000}%
\pgfsetdash{}{0pt}%
\pgfpathmoveto{\pgfqpoint{1.318180in}{1.560079in}}%
\pgfusepath{stroke}%
\end{pgfscope}%
\begin{pgfscope}%
\pgfpathrectangle{\pgfqpoint{0.647939in}{0.492442in}}{\pgfqpoint{3.079299in}{3.079299in}}%
\pgfusepath{clip}%
\pgfsetroundcap%
\pgfsetroundjoin%
\definecolor{currentfill}{rgb}{0.500000,0.500000,0.500000}%
\pgfsetfillcolor{currentfill}%
\pgfsetfillopacity{0.300000}%
\pgfsetlinewidth{0.301125pt}%
\definecolor{currentstroke}{rgb}{0.500000,0.500000,0.500000}%
\pgfsetstrokecolor{currentstroke}%
\pgfsetstrokeopacity{0.300000}%
\pgfsetdash{}{0pt}%
\pgfpathmoveto{\pgfqpoint{0.000000in}{0.000000in}}%
\pgfpathlineto{\pgfqpoint{0.000000in}{0.000000in}}%
\pgfpathclose%
\pgfusepath{stroke,fill}%
\end{pgfscope}%
\begin{pgfscope}%
\pgfpathrectangle{\pgfqpoint{0.647939in}{0.492442in}}{\pgfqpoint{3.079299in}{3.079299in}}%
\pgfusepath{clip}%
\pgfsetroundcap%
\pgfsetroundjoin%
\pgfsetlinewidth{0.301125pt}%
\definecolor{currentstroke}{rgb}{0.500000,0.500000,0.500000}%
\pgfsetstrokecolor{currentstroke}%
\pgfsetstrokeopacity{0.300000}%
\pgfsetdash{}{0pt}%
\pgfpathmoveto{\pgfqpoint{1.132970in}{1.402946in}}%
\pgfusepath{stroke}%
\end{pgfscope}%
\begin{pgfscope}%
\pgfpathrectangle{\pgfqpoint{0.647939in}{0.492442in}}{\pgfqpoint{3.079299in}{3.079299in}}%
\pgfusepath{clip}%
\pgfsetroundcap%
\pgfsetroundjoin%
\definecolor{currentfill}{rgb}{0.500000,0.500000,0.500000}%
\pgfsetfillcolor{currentfill}%
\pgfsetfillopacity{0.300000}%
\pgfsetlinewidth{0.301125pt}%
\definecolor{currentstroke}{rgb}{0.500000,0.500000,0.500000}%
\pgfsetstrokecolor{currentstroke}%
\pgfsetstrokeopacity{0.300000}%
\pgfsetdash{}{0pt}%
\pgfpathmoveto{\pgfqpoint{0.000000in}{0.000000in}}%
\pgfpathlineto{\pgfqpoint{0.000000in}{0.000000in}}%
\pgfpathclose%
\pgfusepath{stroke,fill}%
\end{pgfscope}%
\begin{pgfscope}%
\pgfpathrectangle{\pgfqpoint{0.647939in}{0.492442in}}{\pgfqpoint{3.079299in}{3.079299in}}%
\pgfusepath{clip}%
\pgfsetroundcap%
\pgfsetroundjoin%
\pgfsetlinewidth{0.301125pt}%
\definecolor{currentstroke}{rgb}{0.500000,0.500000,0.500000}%
\pgfsetstrokecolor{currentstroke}%
\pgfsetstrokeopacity{0.300000}%
\pgfsetdash{}{0pt}%
\pgfpathmoveto{\pgfqpoint{0.809370in}{1.224139in}}%
\pgfusepath{stroke}%
\end{pgfscope}%
\begin{pgfscope}%
\pgfpathrectangle{\pgfqpoint{0.647939in}{0.492442in}}{\pgfqpoint{3.079299in}{3.079299in}}%
\pgfusepath{clip}%
\pgfsetroundcap%
\pgfsetroundjoin%
\definecolor{currentfill}{rgb}{0.500000,0.500000,0.500000}%
\pgfsetfillcolor{currentfill}%
\pgfsetfillopacity{0.300000}%
\pgfsetlinewidth{0.301125pt}%
\definecolor{currentstroke}{rgb}{0.500000,0.500000,0.500000}%
\pgfsetstrokecolor{currentstroke}%
\pgfsetstrokeopacity{0.300000}%
\pgfsetdash{}{0pt}%
\pgfpathmoveto{\pgfqpoint{0.000000in}{0.000000in}}%
\pgfpathlineto{\pgfqpoint{0.000000in}{0.000000in}}%
\pgfpathclose%
\pgfusepath{stroke,fill}%
\end{pgfscope}%
\begin{pgfscope}%
\pgfpathrectangle{\pgfqpoint{0.647939in}{0.492442in}}{\pgfqpoint{3.079299in}{3.079299in}}%
\pgfusepath{clip}%
\pgfsetroundcap%
\pgfsetroundjoin%
\pgfsetlinewidth{0.301125pt}%
\definecolor{currentstroke}{rgb}{0.500000,0.500000,0.500000}%
\pgfsetstrokecolor{currentstroke}%
\pgfsetstrokeopacity{0.300000}%
\pgfsetdash{}{0pt}%
\pgfpathmoveto{\pgfqpoint{1.250667in}{1.335514in}}%
\pgfusepath{stroke}%
\end{pgfscope}%
\begin{pgfscope}%
\pgfpathrectangle{\pgfqpoint{0.647939in}{0.492442in}}{\pgfqpoint{3.079299in}{3.079299in}}%
\pgfusepath{clip}%
\pgfsetroundcap%
\pgfsetroundjoin%
\definecolor{currentfill}{rgb}{0.500000,0.500000,0.500000}%
\pgfsetfillcolor{currentfill}%
\pgfsetfillopacity{0.300000}%
\pgfsetlinewidth{0.301125pt}%
\definecolor{currentstroke}{rgb}{0.500000,0.500000,0.500000}%
\pgfsetstrokecolor{currentstroke}%
\pgfsetstrokeopacity{0.300000}%
\pgfsetdash{}{0pt}%
\pgfpathmoveto{\pgfqpoint{0.000000in}{0.000000in}}%
\pgfpathlineto{\pgfqpoint{0.000000in}{0.000000in}}%
\pgfpathclose%
\pgfusepath{stroke,fill}%
\end{pgfscope}%
\begin{pgfscope}%
\pgfpathrectangle{\pgfqpoint{0.647939in}{0.492442in}}{\pgfqpoint{3.079299in}{3.079299in}}%
\pgfusepath{clip}%
\pgfsetroundcap%
\pgfsetroundjoin%
\pgfsetlinewidth{0.301125pt}%
\definecolor{currentstroke}{rgb}{0.500000,0.500000,0.500000}%
\pgfsetstrokecolor{currentstroke}%
\pgfsetstrokeopacity{0.300000}%
\pgfsetdash{}{0pt}%
\pgfpathmoveto{\pgfqpoint{1.004327in}{1.148140in}}%
\pgfusepath{stroke}%
\end{pgfscope}%
\begin{pgfscope}%
\pgfpathrectangle{\pgfqpoint{0.647939in}{0.492442in}}{\pgfqpoint{3.079299in}{3.079299in}}%
\pgfusepath{clip}%
\pgfsetroundcap%
\pgfsetroundjoin%
\definecolor{currentfill}{rgb}{0.500000,0.500000,0.500000}%
\pgfsetfillcolor{currentfill}%
\pgfsetfillopacity{0.300000}%
\pgfsetlinewidth{0.301125pt}%
\definecolor{currentstroke}{rgb}{0.500000,0.500000,0.500000}%
\pgfsetstrokecolor{currentstroke}%
\pgfsetstrokeopacity{0.300000}%
\pgfsetdash{}{0pt}%
\pgfpathmoveto{\pgfqpoint{0.000000in}{0.000000in}}%
\pgfpathlineto{\pgfqpoint{0.000000in}{0.000000in}}%
\pgfpathclose%
\pgfusepath{stroke,fill}%
\end{pgfscope}%
\begin{pgfscope}%
\pgfpathrectangle{\pgfqpoint{0.647939in}{0.492442in}}{\pgfqpoint{3.079299in}{3.079299in}}%
\pgfusepath{clip}%
\pgfsetroundcap%
\pgfsetroundjoin%
\pgfsetlinewidth{0.301125pt}%
\definecolor{currentstroke}{rgb}{0.500000,0.500000,0.500000}%
\pgfsetstrokecolor{currentstroke}%
\pgfsetstrokeopacity{0.300000}%
\pgfsetdash{}{0pt}%
\pgfpathmoveto{\pgfqpoint{0.874798in}{1.034641in}}%
\pgfusepath{stroke}%
\end{pgfscope}%
\begin{pgfscope}%
\pgfpathrectangle{\pgfqpoint{0.647939in}{0.492442in}}{\pgfqpoint{3.079299in}{3.079299in}}%
\pgfusepath{clip}%
\pgfsetroundcap%
\pgfsetroundjoin%
\definecolor{currentfill}{rgb}{0.500000,0.500000,0.500000}%
\pgfsetfillcolor{currentfill}%
\pgfsetfillopacity{0.300000}%
\pgfsetlinewidth{0.301125pt}%
\definecolor{currentstroke}{rgb}{0.500000,0.500000,0.500000}%
\pgfsetstrokecolor{currentstroke}%
\pgfsetstrokeopacity{0.300000}%
\pgfsetdash{}{0pt}%
\pgfpathmoveto{\pgfqpoint{0.000000in}{0.000000in}}%
\pgfpathlineto{\pgfqpoint{0.000000in}{0.000000in}}%
\pgfpathclose%
\pgfusepath{stroke,fill}%
\end{pgfscope}%
\begin{pgfscope}%
\pgfpathrectangle{\pgfqpoint{0.647939in}{0.492442in}}{\pgfqpoint{3.079299in}{3.079299in}}%
\pgfusepath{clip}%
\pgfsetroundcap%
\pgfsetroundjoin%
\pgfsetlinewidth{0.301125pt}%
\definecolor{currentstroke}{rgb}{0.500000,0.500000,0.500000}%
\pgfsetstrokecolor{currentstroke}%
\pgfsetstrokeopacity{0.300000}%
\pgfsetdash{}{0pt}%
\pgfpathmoveto{\pgfqpoint{0.874508in}{0.965822in}}%
\pgfusepath{stroke}%
\end{pgfscope}%
\begin{pgfscope}%
\pgfpathrectangle{\pgfqpoint{0.647939in}{0.492442in}}{\pgfqpoint{3.079299in}{3.079299in}}%
\pgfusepath{clip}%
\pgfsetroundcap%
\pgfsetroundjoin%
\definecolor{currentfill}{rgb}{0.500000,0.500000,0.500000}%
\pgfsetfillcolor{currentfill}%
\pgfsetfillopacity{0.300000}%
\pgfsetlinewidth{0.301125pt}%
\definecolor{currentstroke}{rgb}{0.500000,0.500000,0.500000}%
\pgfsetstrokecolor{currentstroke}%
\pgfsetstrokeopacity{0.300000}%
\pgfsetdash{}{0pt}%
\pgfpathmoveto{\pgfqpoint{0.000000in}{0.000000in}}%
\pgfpathlineto{\pgfqpoint{0.000000in}{0.000000in}}%
\pgfpathclose%
\pgfusepath{stroke,fill}%
\end{pgfscope}%
\begin{pgfscope}%
\pgfpathrectangle{\pgfqpoint{0.647939in}{0.492442in}}{\pgfqpoint{3.079299in}{3.079299in}}%
\pgfusepath{clip}%
\pgfsetroundcap%
\pgfsetroundjoin%
\pgfsetlinewidth{0.301125pt}%
\definecolor{currentstroke}{rgb}{0.500000,0.500000,0.500000}%
\pgfsetstrokecolor{currentstroke}%
\pgfsetstrokeopacity{0.300000}%
\pgfsetdash{}{0pt}%
\pgfpathmoveto{\pgfqpoint{1.123769in}{1.007654in}}%
\pgfusepath{stroke}%
\end{pgfscope}%
\begin{pgfscope}%
\pgfpathrectangle{\pgfqpoint{0.647939in}{0.492442in}}{\pgfqpoint{3.079299in}{3.079299in}}%
\pgfusepath{clip}%
\pgfsetroundcap%
\pgfsetroundjoin%
\definecolor{currentfill}{rgb}{0.500000,0.500000,0.500000}%
\pgfsetfillcolor{currentfill}%
\pgfsetfillopacity{0.300000}%
\pgfsetlinewidth{0.301125pt}%
\definecolor{currentstroke}{rgb}{0.500000,0.500000,0.500000}%
\pgfsetstrokecolor{currentstroke}%
\pgfsetstrokeopacity{0.300000}%
\pgfsetdash{}{0pt}%
\pgfpathmoveto{\pgfqpoint{0.000000in}{0.000000in}}%
\pgfpathlineto{\pgfqpoint{0.000000in}{0.000000in}}%
\pgfpathclose%
\pgfusepath{stroke,fill}%
\end{pgfscope}%
\begin{pgfscope}%
\pgfpathrectangle{\pgfqpoint{0.647939in}{0.492442in}}{\pgfqpoint{3.079299in}{3.079299in}}%
\pgfusepath{clip}%
\pgfsetroundcap%
\pgfsetroundjoin%
\pgfsetlinewidth{0.301125pt}%
\definecolor{currentstroke}{rgb}{0.500000,0.500000,0.500000}%
\pgfsetstrokecolor{currentstroke}%
\pgfsetstrokeopacity{0.300000}%
\pgfsetdash{}{0pt}%
\pgfpathmoveto{\pgfqpoint{0.938168in}{0.851574in}}%
\pgfusepath{stroke}%
\end{pgfscope}%
\begin{pgfscope}%
\pgfpathrectangle{\pgfqpoint{0.647939in}{0.492442in}}{\pgfqpoint{3.079299in}{3.079299in}}%
\pgfusepath{clip}%
\pgfsetroundcap%
\pgfsetroundjoin%
\definecolor{currentfill}{rgb}{0.500000,0.500000,0.500000}%
\pgfsetfillcolor{currentfill}%
\pgfsetfillopacity{0.300000}%
\pgfsetlinewidth{0.301125pt}%
\definecolor{currentstroke}{rgb}{0.500000,0.500000,0.500000}%
\pgfsetstrokecolor{currentstroke}%
\pgfsetstrokeopacity{0.300000}%
\pgfsetdash{}{0pt}%
\pgfpathmoveto{\pgfqpoint{0.000000in}{0.000000in}}%
\pgfpathlineto{\pgfqpoint{0.000000in}{0.000000in}}%
\pgfpathclose%
\pgfusepath{stroke,fill}%
\end{pgfscope}%
\begin{pgfscope}%
\pgfpathrectangle{\pgfqpoint{0.647939in}{0.492442in}}{\pgfqpoint{3.079299in}{3.079299in}}%
\pgfusepath{clip}%
\pgfsetroundcap%
\pgfsetroundjoin%
\pgfsetlinewidth{0.301125pt}%
\definecolor{currentstroke}{rgb}{0.500000,0.500000,0.500000}%
\pgfsetstrokecolor{currentstroke}%
\pgfsetstrokeopacity{0.300000}%
\pgfsetdash{}{0pt}%
\pgfpathmoveto{\pgfqpoint{0.937558in}{0.783584in}}%
\pgfusepath{stroke}%
\end{pgfscope}%
\begin{pgfscope}%
\pgfpathrectangle{\pgfqpoint{0.647939in}{0.492442in}}{\pgfqpoint{3.079299in}{3.079299in}}%
\pgfusepath{clip}%
\pgfsetroundcap%
\pgfsetroundjoin%
\definecolor{currentfill}{rgb}{0.500000,0.500000,0.500000}%
\pgfsetfillcolor{currentfill}%
\pgfsetfillopacity{0.300000}%
\pgfsetlinewidth{0.301125pt}%
\definecolor{currentstroke}{rgb}{0.500000,0.500000,0.500000}%
\pgfsetstrokecolor{currentstroke}%
\pgfsetstrokeopacity{0.300000}%
\pgfsetdash{}{0pt}%
\pgfpathmoveto{\pgfqpoint{0.000000in}{0.000000in}}%
\pgfpathlineto{\pgfqpoint{0.000000in}{0.000000in}}%
\pgfpathclose%
\pgfusepath{stroke,fill}%
\end{pgfscope}%
\begin{pgfscope}%
\pgfpathrectangle{\pgfqpoint{0.647939in}{0.492442in}}{\pgfqpoint{3.079299in}{3.079299in}}%
\pgfusepath{clip}%
\pgfsetroundcap%
\pgfsetroundjoin%
\pgfsetlinewidth{0.301125pt}%
\definecolor{currentstroke}{rgb}{0.500000,0.500000,0.500000}%
\pgfsetstrokecolor{currentstroke}%
\pgfsetstrokeopacity{0.300000}%
\pgfsetdash{}{0pt}%
\pgfpathmoveto{\pgfqpoint{0.873122in}{0.691079in}}%
\pgfusepath{stroke}%
\end{pgfscope}%
\begin{pgfscope}%
\pgfpathrectangle{\pgfqpoint{0.647939in}{0.492442in}}{\pgfqpoint{3.079299in}{3.079299in}}%
\pgfusepath{clip}%
\pgfsetroundcap%
\pgfsetroundjoin%
\definecolor{currentfill}{rgb}{0.500000,0.500000,0.500000}%
\pgfsetfillcolor{currentfill}%
\pgfsetfillopacity{0.300000}%
\pgfsetlinewidth{0.301125pt}%
\definecolor{currentstroke}{rgb}{0.500000,0.500000,0.500000}%
\pgfsetstrokecolor{currentstroke}%
\pgfsetstrokeopacity{0.300000}%
\pgfsetdash{}{0pt}%
\pgfpathmoveto{\pgfqpoint{0.000000in}{0.000000in}}%
\pgfpathlineto{\pgfqpoint{0.000000in}{0.000000in}}%
\pgfpathclose%
\pgfusepath{stroke,fill}%
\end{pgfscope}%
\begin{pgfscope}%
\pgfpathrectangle{\pgfqpoint{0.647939in}{0.492442in}}{\pgfqpoint{3.079299in}{3.079299in}}%
\pgfusepath{clip}%
\pgfsetroundcap%
\pgfsetroundjoin%
\pgfsetlinewidth{0.301125pt}%
\definecolor{currentstroke}{rgb}{0.500000,0.500000,0.500000}%
\pgfsetstrokecolor{currentstroke}%
\pgfsetstrokeopacity{0.300000}%
\pgfsetdash{}{0pt}%
\pgfpathmoveto{\pgfqpoint{3.438027in}{3.426837in}}%
\pgfusepath{stroke}%
\end{pgfscope}%
\begin{pgfscope}%
\pgfpathrectangle{\pgfqpoint{0.647939in}{0.492442in}}{\pgfqpoint{3.079299in}{3.079299in}}%
\pgfusepath{clip}%
\pgfsetroundcap%
\pgfsetroundjoin%
\definecolor{currentfill}{rgb}{0.500000,0.500000,0.500000}%
\pgfsetfillcolor{currentfill}%
\pgfsetfillopacity{0.300000}%
\pgfsetlinewidth{0.301125pt}%
\definecolor{currentstroke}{rgb}{0.500000,0.500000,0.500000}%
\pgfsetstrokecolor{currentstroke}%
\pgfsetstrokeopacity{0.300000}%
\pgfsetdash{}{0pt}%
\pgfpathmoveto{\pgfqpoint{0.000000in}{0.000000in}}%
\pgfpathlineto{\pgfqpoint{0.000000in}{0.000000in}}%
\pgfpathclose%
\pgfusepath{stroke,fill}%
\end{pgfscope}%
\begin{pgfscope}%
\pgfpathrectangle{\pgfqpoint{0.647939in}{0.492442in}}{\pgfqpoint{3.079299in}{3.079299in}}%
\pgfusepath{clip}%
\pgfsetroundcap%
\pgfsetroundjoin%
\pgfsetlinewidth{0.301125pt}%
\definecolor{currentstroke}{rgb}{0.500000,0.500000,0.500000}%
\pgfsetstrokecolor{currentstroke}%
\pgfsetstrokeopacity{0.300000}%
\pgfsetdash{}{0pt}%
\pgfpathmoveto{\pgfqpoint{1.672485in}{0.644075in}}%
\pgfusepath{stroke}%
\end{pgfscope}%
\begin{pgfscope}%
\pgfpathrectangle{\pgfqpoint{0.647939in}{0.492442in}}{\pgfqpoint{3.079299in}{3.079299in}}%
\pgfusepath{clip}%
\pgfsetroundcap%
\pgfsetroundjoin%
\definecolor{currentfill}{rgb}{0.500000,0.500000,0.500000}%
\pgfsetfillcolor{currentfill}%
\pgfsetfillopacity{0.300000}%
\pgfsetlinewidth{0.301125pt}%
\definecolor{currentstroke}{rgb}{0.500000,0.500000,0.500000}%
\pgfsetstrokecolor{currentstroke}%
\pgfsetstrokeopacity{0.300000}%
\pgfsetdash{}{0pt}%
\pgfpathmoveto{\pgfqpoint{0.000000in}{0.000000in}}%
\pgfpathlineto{\pgfqpoint{0.000000in}{0.000000in}}%
\pgfpathclose%
\pgfusepath{stroke,fill}%
\end{pgfscope}%
\begin{pgfscope}%
\pgfpathrectangle{\pgfqpoint{0.647939in}{0.492442in}}{\pgfqpoint{3.079299in}{3.079299in}}%
\pgfusepath{clip}%
\pgfsetroundcap%
\pgfsetroundjoin%
\pgfsetlinewidth{0.301125pt}%
\definecolor{currentstroke}{rgb}{0.500000,0.500000,0.500000}%
\pgfsetstrokecolor{currentstroke}%
\pgfsetstrokeopacity{0.300000}%
\pgfsetdash{}{0pt}%
\pgfpathmoveto{\pgfqpoint{2.290270in}{0.626168in}}%
\pgfusepath{stroke}%
\end{pgfscope}%
\begin{pgfscope}%
\pgfpathrectangle{\pgfqpoint{0.647939in}{0.492442in}}{\pgfqpoint{3.079299in}{3.079299in}}%
\pgfusepath{clip}%
\pgfsetroundcap%
\pgfsetroundjoin%
\definecolor{currentfill}{rgb}{0.500000,0.500000,0.500000}%
\pgfsetfillcolor{currentfill}%
\pgfsetfillopacity{0.300000}%
\pgfsetlinewidth{0.301125pt}%
\definecolor{currentstroke}{rgb}{0.500000,0.500000,0.500000}%
\pgfsetstrokecolor{currentstroke}%
\pgfsetstrokeopacity{0.300000}%
\pgfsetdash{}{0pt}%
\pgfpathmoveto{\pgfqpoint{0.000000in}{0.000000in}}%
\pgfpathlineto{\pgfqpoint{0.000000in}{0.000000in}}%
\pgfpathclose%
\pgfusepath{stroke,fill}%
\end{pgfscope}%
\begin{pgfscope}%
\pgfpathrectangle{\pgfqpoint{0.647939in}{0.492442in}}{\pgfqpoint{3.079299in}{3.079299in}}%
\pgfusepath{clip}%
\pgfsetroundcap%
\pgfsetroundjoin%
\pgfsetlinewidth{0.301125pt}%
\definecolor{currentstroke}{rgb}{0.500000,0.500000,0.500000}%
\pgfsetstrokecolor{currentstroke}%
\pgfsetstrokeopacity{0.300000}%
\pgfsetdash{}{0pt}%
\pgfpathmoveto{\pgfqpoint{3.092620in}{1.951456in}}%
\pgfusepath{stroke}%
\end{pgfscope}%
\begin{pgfscope}%
\pgfpathrectangle{\pgfqpoint{0.647939in}{0.492442in}}{\pgfqpoint{3.079299in}{3.079299in}}%
\pgfusepath{clip}%
\pgfsetroundcap%
\pgfsetroundjoin%
\definecolor{currentfill}{rgb}{0.500000,0.500000,0.500000}%
\pgfsetfillcolor{currentfill}%
\pgfsetfillopacity{0.300000}%
\pgfsetlinewidth{0.301125pt}%
\definecolor{currentstroke}{rgb}{0.500000,0.500000,0.500000}%
\pgfsetstrokecolor{currentstroke}%
\pgfsetstrokeopacity{0.300000}%
\pgfsetdash{}{0pt}%
\pgfpathmoveto{\pgfqpoint{0.000000in}{0.000000in}}%
\pgfpathlineto{\pgfqpoint{0.000000in}{0.000000in}}%
\pgfpathclose%
\pgfusepath{stroke,fill}%
\end{pgfscope}%
\begin{pgfscope}%
\pgfpathrectangle{\pgfqpoint{0.647939in}{0.492442in}}{\pgfqpoint{3.079299in}{3.079299in}}%
\pgfusepath{clip}%
\pgfsetroundcap%
\pgfsetroundjoin%
\pgfsetlinewidth{0.301125pt}%
\definecolor{currentstroke}{rgb}{0.500000,0.500000,0.500000}%
\pgfsetstrokecolor{currentstroke}%
\pgfsetstrokeopacity{0.300000}%
\pgfsetdash{}{0pt}%
\pgfpathmoveto{\pgfqpoint{3.412846in}{2.883501in}}%
\pgfusepath{stroke}%
\end{pgfscope}%
\begin{pgfscope}%
\pgfpathrectangle{\pgfqpoint{0.647939in}{0.492442in}}{\pgfqpoint{3.079299in}{3.079299in}}%
\pgfusepath{clip}%
\pgfsetroundcap%
\pgfsetroundjoin%
\definecolor{currentfill}{rgb}{0.500000,0.500000,0.500000}%
\pgfsetfillcolor{currentfill}%
\pgfsetfillopacity{0.300000}%
\pgfsetlinewidth{0.301125pt}%
\definecolor{currentstroke}{rgb}{0.500000,0.500000,0.500000}%
\pgfsetstrokecolor{currentstroke}%
\pgfsetstrokeopacity{0.300000}%
\pgfsetdash{}{0pt}%
\pgfpathmoveto{\pgfqpoint{0.000000in}{0.000000in}}%
\pgfpathlineto{\pgfqpoint{0.000000in}{0.000000in}}%
\pgfpathclose%
\pgfusepath{stroke,fill}%
\end{pgfscope}%
\begin{pgfscope}%
\pgfpathrectangle{\pgfqpoint{0.647939in}{0.492442in}}{\pgfqpoint{3.079299in}{3.079299in}}%
\pgfusepath{clip}%
\pgfsetroundcap%
\pgfsetroundjoin%
\pgfsetlinewidth{0.301125pt}%
\definecolor{currentstroke}{rgb}{0.500000,0.500000,0.500000}%
\pgfsetstrokecolor{currentstroke}%
\pgfsetstrokeopacity{0.300000}%
\pgfsetdash{}{0pt}%
\pgfpathmoveto{\pgfqpoint{3.102144in}{2.238162in}}%
\pgfusepath{stroke}%
\end{pgfscope}%
\begin{pgfscope}%
\pgfpathrectangle{\pgfqpoint{0.647939in}{0.492442in}}{\pgfqpoint{3.079299in}{3.079299in}}%
\pgfusepath{clip}%
\pgfsetroundcap%
\pgfsetroundjoin%
\definecolor{currentfill}{rgb}{0.500000,0.500000,0.500000}%
\pgfsetfillcolor{currentfill}%
\pgfsetfillopacity{0.300000}%
\pgfsetlinewidth{0.301125pt}%
\definecolor{currentstroke}{rgb}{0.500000,0.500000,0.500000}%
\pgfsetstrokecolor{currentstroke}%
\pgfsetstrokeopacity{0.300000}%
\pgfsetdash{}{0pt}%
\pgfpathmoveto{\pgfqpoint{0.000000in}{0.000000in}}%
\pgfpathlineto{\pgfqpoint{0.000000in}{0.000000in}}%
\pgfpathclose%
\pgfusepath{stroke,fill}%
\end{pgfscope}%
\begin{pgfscope}%
\pgfpathrectangle{\pgfqpoint{0.647939in}{0.492442in}}{\pgfqpoint{3.079299in}{3.079299in}}%
\pgfusepath{clip}%
\pgfsetroundcap%
\pgfsetroundjoin%
\pgfsetlinewidth{0.301125pt}%
\definecolor{currentstroke}{rgb}{0.500000,0.500000,0.500000}%
\pgfsetstrokecolor{currentstroke}%
\pgfsetstrokeopacity{0.300000}%
\pgfsetdash{}{0pt}%
\pgfpathmoveto{\pgfqpoint{3.310753in}{1.564129in}}%
\pgfusepath{stroke}%
\end{pgfscope}%
\begin{pgfscope}%
\pgfpathrectangle{\pgfqpoint{0.647939in}{0.492442in}}{\pgfqpoint{3.079299in}{3.079299in}}%
\pgfusepath{clip}%
\pgfsetroundcap%
\pgfsetroundjoin%
\definecolor{currentfill}{rgb}{0.500000,0.500000,0.500000}%
\pgfsetfillcolor{currentfill}%
\pgfsetfillopacity{0.300000}%
\pgfsetlinewidth{0.301125pt}%
\definecolor{currentstroke}{rgb}{0.500000,0.500000,0.500000}%
\pgfsetstrokecolor{currentstroke}%
\pgfsetstrokeopacity{0.300000}%
\pgfsetdash{}{0pt}%
\pgfpathmoveto{\pgfqpoint{0.000000in}{0.000000in}}%
\pgfpathlineto{\pgfqpoint{0.000000in}{0.000000in}}%
\pgfpathclose%
\pgfusepath{stroke,fill}%
\end{pgfscope}%
\begin{pgfscope}%
\pgfpathrectangle{\pgfqpoint{0.647939in}{0.492442in}}{\pgfqpoint{3.079299in}{3.079299in}}%
\pgfusepath{clip}%
\pgfsetroundcap%
\pgfsetroundjoin%
\pgfsetlinewidth{0.301125pt}%
\definecolor{currentstroke}{rgb}{0.500000,0.500000,0.500000}%
\pgfsetstrokecolor{currentstroke}%
\pgfsetstrokeopacity{0.300000}%
\pgfsetdash{}{0pt}%
\pgfpathmoveto{\pgfqpoint{2.221824in}{3.214297in}}%
\pgfusepath{stroke}%
\end{pgfscope}%
\begin{pgfscope}%
\pgfpathrectangle{\pgfqpoint{0.647939in}{0.492442in}}{\pgfqpoint{3.079299in}{3.079299in}}%
\pgfusepath{clip}%
\pgfsetroundcap%
\pgfsetroundjoin%
\definecolor{currentfill}{rgb}{0.500000,0.500000,0.500000}%
\pgfsetfillcolor{currentfill}%
\pgfsetfillopacity{0.300000}%
\pgfsetlinewidth{0.301125pt}%
\definecolor{currentstroke}{rgb}{0.500000,0.500000,0.500000}%
\pgfsetstrokecolor{currentstroke}%
\pgfsetstrokeopacity{0.300000}%
\pgfsetdash{}{0pt}%
\pgfpathmoveto{\pgfqpoint{0.000000in}{0.000000in}}%
\pgfpathlineto{\pgfqpoint{0.000000in}{0.000000in}}%
\pgfpathclose%
\pgfusepath{stroke,fill}%
\end{pgfscope}%
\begin{pgfscope}%
\pgfpathrectangle{\pgfqpoint{0.647939in}{0.492442in}}{\pgfqpoint{3.079299in}{3.079299in}}%
\pgfusepath{clip}%
\pgfsetroundcap%
\pgfsetroundjoin%
\pgfsetlinewidth{0.301125pt}%
\definecolor{currentstroke}{rgb}{0.500000,0.500000,0.500000}%
\pgfsetstrokecolor{currentstroke}%
\pgfsetstrokeopacity{0.300000}%
\pgfsetdash{}{0pt}%
\pgfpathmoveto{\pgfqpoint{3.182090in}{2.516668in}}%
\pgfusepath{stroke}%
\end{pgfscope}%
\begin{pgfscope}%
\pgfpathrectangle{\pgfqpoint{0.647939in}{0.492442in}}{\pgfqpoint{3.079299in}{3.079299in}}%
\pgfusepath{clip}%
\pgfsetroundcap%
\pgfsetroundjoin%
\definecolor{currentfill}{rgb}{0.500000,0.500000,0.500000}%
\pgfsetfillcolor{currentfill}%
\pgfsetfillopacity{0.300000}%
\pgfsetlinewidth{0.301125pt}%
\definecolor{currentstroke}{rgb}{0.500000,0.500000,0.500000}%
\pgfsetstrokecolor{currentstroke}%
\pgfsetstrokeopacity{0.300000}%
\pgfsetdash{}{0pt}%
\pgfpathmoveto{\pgfqpoint{0.000000in}{0.000000in}}%
\pgfpathlineto{\pgfqpoint{0.000000in}{0.000000in}}%
\pgfpathclose%
\pgfusepath{stroke,fill}%
\end{pgfscope}%
\begin{pgfscope}%
\pgfpathrectangle{\pgfqpoint{0.647939in}{0.492442in}}{\pgfqpoint{3.079299in}{3.079299in}}%
\pgfusepath{clip}%
\pgfsetroundcap%
\pgfsetroundjoin%
\pgfsetlinewidth{0.301125pt}%
\definecolor{currentstroke}{rgb}{0.500000,0.500000,0.500000}%
\pgfsetstrokecolor{currentstroke}%
\pgfsetstrokeopacity{0.300000}%
\pgfsetdash{}{0pt}%
\pgfpathmoveto{\pgfqpoint{2.221628in}{0.973305in}}%
\pgfusepath{stroke}%
\end{pgfscope}%
\begin{pgfscope}%
\pgfpathrectangle{\pgfqpoint{0.647939in}{0.492442in}}{\pgfqpoint{3.079299in}{3.079299in}}%
\pgfusepath{clip}%
\pgfsetroundcap%
\pgfsetroundjoin%
\definecolor{currentfill}{rgb}{0.500000,0.500000,0.500000}%
\pgfsetfillcolor{currentfill}%
\pgfsetfillopacity{0.300000}%
\pgfsetlinewidth{0.301125pt}%
\definecolor{currentstroke}{rgb}{0.500000,0.500000,0.500000}%
\pgfsetstrokecolor{currentstroke}%
\pgfsetstrokeopacity{0.300000}%
\pgfsetdash{}{0pt}%
\pgfpathmoveto{\pgfqpoint{0.000000in}{0.000000in}}%
\pgfpathlineto{\pgfqpoint{0.000000in}{0.000000in}}%
\pgfpathclose%
\pgfusepath{stroke,fill}%
\end{pgfscope}%
\begin{pgfscope}%
\pgfpathrectangle{\pgfqpoint{0.647939in}{0.492442in}}{\pgfqpoint{3.079299in}{3.079299in}}%
\pgfusepath{clip}%
\pgfsetroundcap%
\pgfsetroundjoin%
\pgfsetlinewidth{0.301125pt}%
\definecolor{currentstroke}{rgb}{0.500000,0.500000,0.500000}%
\pgfsetstrokecolor{currentstroke}%
\pgfsetstrokeopacity{0.300000}%
\pgfsetdash{}{0pt}%
\pgfpathmoveto{\pgfqpoint{3.000425in}{3.051432in}}%
\pgfusepath{stroke}%
\end{pgfscope}%
\begin{pgfscope}%
\pgfpathrectangle{\pgfqpoint{0.647939in}{0.492442in}}{\pgfqpoint{3.079299in}{3.079299in}}%
\pgfusepath{clip}%
\pgfsetroundcap%
\pgfsetroundjoin%
\definecolor{currentfill}{rgb}{0.500000,0.500000,0.500000}%
\pgfsetfillcolor{currentfill}%
\pgfsetfillopacity{0.300000}%
\pgfsetlinewidth{0.301125pt}%
\definecolor{currentstroke}{rgb}{0.500000,0.500000,0.500000}%
\pgfsetstrokecolor{currentstroke}%
\pgfsetstrokeopacity{0.300000}%
\pgfsetdash{}{0pt}%
\pgfpathmoveto{\pgfqpoint{0.000000in}{0.000000in}}%
\pgfpathlineto{\pgfqpoint{0.000000in}{0.000000in}}%
\pgfpathclose%
\pgfusepath{stroke,fill}%
\end{pgfscope}%
\begin{pgfscope}%
\pgfpathrectangle{\pgfqpoint{0.647939in}{0.492442in}}{\pgfqpoint{3.079299in}{3.079299in}}%
\pgfusepath{clip}%
\pgfsetroundcap%
\pgfsetroundjoin%
\pgfsetlinewidth{0.301125pt}%
\definecolor{currentstroke}{rgb}{0.500000,0.500000,0.500000}%
\pgfsetstrokecolor{currentstroke}%
\pgfsetstrokeopacity{0.300000}%
\pgfsetdash{}{0pt}%
\pgfpathmoveto{\pgfqpoint{1.676241in}{2.164205in}}%
\pgfusepath{stroke}%
\end{pgfscope}%
\begin{pgfscope}%
\pgfpathrectangle{\pgfqpoint{0.647939in}{0.492442in}}{\pgfqpoint{3.079299in}{3.079299in}}%
\pgfusepath{clip}%
\pgfsetroundcap%
\pgfsetroundjoin%
\definecolor{currentfill}{rgb}{0.500000,0.500000,0.500000}%
\pgfsetfillcolor{currentfill}%
\pgfsetfillopacity{0.300000}%
\pgfsetlinewidth{0.301125pt}%
\definecolor{currentstroke}{rgb}{0.500000,0.500000,0.500000}%
\pgfsetstrokecolor{currentstroke}%
\pgfsetstrokeopacity{0.300000}%
\pgfsetdash{}{0pt}%
\pgfpathmoveto{\pgfqpoint{0.000000in}{0.000000in}}%
\pgfpathlineto{\pgfqpoint{0.000000in}{0.000000in}}%
\pgfpathclose%
\pgfusepath{stroke,fill}%
\end{pgfscope}%
\begin{pgfscope}%
\pgfpathrectangle{\pgfqpoint{0.647939in}{0.492442in}}{\pgfqpoint{3.079299in}{3.079299in}}%
\pgfusepath{clip}%
\pgfsetroundcap%
\pgfsetroundjoin%
\pgfsetlinewidth{0.301125pt}%
\definecolor{currentstroke}{rgb}{0.500000,0.500000,0.500000}%
\pgfsetstrokecolor{currentstroke}%
\pgfsetstrokeopacity{0.300000}%
\pgfsetdash{}{0pt}%
\pgfpathmoveto{\pgfqpoint{2.756635in}{2.027826in}}%
\pgfusepath{stroke}%
\end{pgfscope}%
\begin{pgfscope}%
\pgfpathrectangle{\pgfqpoint{0.647939in}{0.492442in}}{\pgfqpoint{3.079299in}{3.079299in}}%
\pgfusepath{clip}%
\pgfsetroundcap%
\pgfsetroundjoin%
\definecolor{currentfill}{rgb}{0.500000,0.500000,0.500000}%
\pgfsetfillcolor{currentfill}%
\pgfsetfillopacity{0.300000}%
\pgfsetlinewidth{0.301125pt}%
\definecolor{currentstroke}{rgb}{0.500000,0.500000,0.500000}%
\pgfsetstrokecolor{currentstroke}%
\pgfsetstrokeopacity{0.300000}%
\pgfsetdash{}{0pt}%
\pgfpathmoveto{\pgfqpoint{0.000000in}{0.000000in}}%
\pgfpathlineto{\pgfqpoint{0.000000in}{0.000000in}}%
\pgfpathclose%
\pgfusepath{stroke,fill}%
\end{pgfscope}%
\begin{pgfscope}%
\pgfpathrectangle{\pgfqpoint{0.647939in}{0.492442in}}{\pgfqpoint{3.079299in}{3.079299in}}%
\pgfusepath{clip}%
\pgfsetroundcap%
\pgfsetroundjoin%
\pgfsetlinewidth{0.301125pt}%
\definecolor{currentstroke}{rgb}{0.500000,0.500000,0.500000}%
\pgfsetstrokecolor{currentstroke}%
\pgfsetstrokeopacity{0.300000}%
\pgfsetdash{}{0pt}%
\pgfpathmoveto{\pgfqpoint{2.297894in}{2.897625in}}%
\pgfusepath{stroke}%
\end{pgfscope}%
\begin{pgfscope}%
\pgfpathrectangle{\pgfqpoint{0.647939in}{0.492442in}}{\pgfqpoint{3.079299in}{3.079299in}}%
\pgfusepath{clip}%
\pgfsetroundcap%
\pgfsetroundjoin%
\definecolor{currentfill}{rgb}{0.500000,0.500000,0.500000}%
\pgfsetfillcolor{currentfill}%
\pgfsetfillopacity{0.300000}%
\pgfsetlinewidth{0.301125pt}%
\definecolor{currentstroke}{rgb}{0.500000,0.500000,0.500000}%
\pgfsetstrokecolor{currentstroke}%
\pgfsetstrokeopacity{0.300000}%
\pgfsetdash{}{0pt}%
\pgfpathmoveto{\pgfqpoint{0.000000in}{0.000000in}}%
\pgfpathlineto{\pgfqpoint{0.000000in}{0.000000in}}%
\pgfpathclose%
\pgfusepath{stroke,fill}%
\end{pgfscope}%
\begin{pgfscope}%
\pgfpathrectangle{\pgfqpoint{0.647939in}{0.492442in}}{\pgfqpoint{3.079299in}{3.079299in}}%
\pgfusepath{clip}%
\pgfsetroundcap%
\pgfsetroundjoin%
\pgfsetlinewidth{0.301125pt}%
\definecolor{currentstroke}{rgb}{0.500000,0.500000,0.500000}%
\pgfsetstrokecolor{currentstroke}%
\pgfsetstrokeopacity{0.300000}%
\pgfsetdash{}{0pt}%
\pgfpathmoveto{\pgfqpoint{2.151689in}{2.938773in}}%
\pgfusepath{stroke}%
\end{pgfscope}%
\begin{pgfscope}%
\pgfpathrectangle{\pgfqpoint{0.647939in}{0.492442in}}{\pgfqpoint{3.079299in}{3.079299in}}%
\pgfusepath{clip}%
\pgfsetroundcap%
\pgfsetroundjoin%
\definecolor{currentfill}{rgb}{0.500000,0.500000,0.500000}%
\pgfsetfillcolor{currentfill}%
\pgfsetfillopacity{0.300000}%
\pgfsetlinewidth{0.301125pt}%
\definecolor{currentstroke}{rgb}{0.500000,0.500000,0.500000}%
\pgfsetstrokecolor{currentstroke}%
\pgfsetstrokeopacity{0.300000}%
\pgfsetdash{}{0pt}%
\pgfpathmoveto{\pgfqpoint{0.000000in}{0.000000in}}%
\pgfpathlineto{\pgfqpoint{0.000000in}{0.000000in}}%
\pgfpathclose%
\pgfusepath{stroke,fill}%
\end{pgfscope}%
\begin{pgfscope}%
\pgfpathrectangle{\pgfqpoint{0.647939in}{0.492442in}}{\pgfqpoint{3.079299in}{3.079299in}}%
\pgfusepath{clip}%
\pgfsetroundcap%
\pgfsetroundjoin%
\pgfsetlinewidth{0.301125pt}%
\definecolor{currentstroke}{rgb}{0.500000,0.500000,0.500000}%
\pgfsetstrokecolor{currentstroke}%
\pgfsetstrokeopacity{0.300000}%
\pgfsetdash{}{0pt}%
\pgfpathmoveto{\pgfqpoint{1.369073in}{1.280069in}}%
\pgfusepath{stroke}%
\end{pgfscope}%
\begin{pgfscope}%
\pgfpathrectangle{\pgfqpoint{0.647939in}{0.492442in}}{\pgfqpoint{3.079299in}{3.079299in}}%
\pgfusepath{clip}%
\pgfsetroundcap%
\pgfsetroundjoin%
\definecolor{currentfill}{rgb}{0.500000,0.500000,0.500000}%
\pgfsetfillcolor{currentfill}%
\pgfsetfillopacity{0.300000}%
\pgfsetlinewidth{0.301125pt}%
\definecolor{currentstroke}{rgb}{0.500000,0.500000,0.500000}%
\pgfsetstrokecolor{currentstroke}%
\pgfsetstrokeopacity{0.300000}%
\pgfsetdash{}{0pt}%
\pgfpathmoveto{\pgfqpoint{0.000000in}{0.000000in}}%
\pgfpathlineto{\pgfqpoint{0.000000in}{0.000000in}}%
\pgfpathclose%
\pgfusepath{stroke,fill}%
\end{pgfscope}%
\begin{pgfscope}%
\pgfpathrectangle{\pgfqpoint{0.647939in}{0.492442in}}{\pgfqpoint{3.079299in}{3.079299in}}%
\pgfusepath{clip}%
\pgfsetroundcap%
\pgfsetroundjoin%
\pgfsetlinewidth{0.301125pt}%
\definecolor{currentstroke}{rgb}{0.500000,0.500000,0.500000}%
\pgfsetstrokecolor{currentstroke}%
\pgfsetstrokeopacity{0.300000}%
\pgfsetdash{}{0pt}%
\pgfpathmoveto{\pgfqpoint{2.221015in}{1.248160in}}%
\pgfusepath{stroke}%
\end{pgfscope}%
\begin{pgfscope}%
\pgfpathrectangle{\pgfqpoint{0.647939in}{0.492442in}}{\pgfqpoint{3.079299in}{3.079299in}}%
\pgfusepath{clip}%
\pgfsetroundcap%
\pgfsetroundjoin%
\definecolor{currentfill}{rgb}{0.500000,0.500000,0.500000}%
\pgfsetfillcolor{currentfill}%
\pgfsetfillopacity{0.300000}%
\pgfsetlinewidth{0.301125pt}%
\definecolor{currentstroke}{rgb}{0.500000,0.500000,0.500000}%
\pgfsetstrokecolor{currentstroke}%
\pgfsetstrokeopacity{0.300000}%
\pgfsetdash{}{0pt}%
\pgfpathmoveto{\pgfqpoint{0.000000in}{0.000000in}}%
\pgfpathlineto{\pgfqpoint{0.000000in}{0.000000in}}%
\pgfpathclose%
\pgfusepath{stroke,fill}%
\end{pgfscope}%
\begin{pgfscope}%
\pgfpathrectangle{\pgfqpoint{0.647939in}{0.492442in}}{\pgfqpoint{3.079299in}{3.079299in}}%
\pgfusepath{clip}%
\pgfsetroundcap%
\pgfsetroundjoin%
\pgfsetlinewidth{0.301125pt}%
\definecolor{currentstroke}{rgb}{0.500000,0.500000,0.500000}%
\pgfsetstrokecolor{currentstroke}%
\pgfsetstrokeopacity{0.300000}%
\pgfsetdash{}{0pt}%
\pgfpathmoveto{\pgfqpoint{1.977668in}{2.506119in}}%
\pgfusepath{stroke}%
\end{pgfscope}%
\begin{pgfscope}%
\pgfpathrectangle{\pgfqpoint{0.647939in}{0.492442in}}{\pgfqpoint{3.079299in}{3.079299in}}%
\pgfusepath{clip}%
\pgfsetroundcap%
\pgfsetroundjoin%
\definecolor{currentfill}{rgb}{0.500000,0.500000,0.500000}%
\pgfsetfillcolor{currentfill}%
\pgfsetfillopacity{0.300000}%
\pgfsetlinewidth{0.301125pt}%
\definecolor{currentstroke}{rgb}{0.500000,0.500000,0.500000}%
\pgfsetstrokecolor{currentstroke}%
\pgfsetstrokeopacity{0.300000}%
\pgfsetdash{}{0pt}%
\pgfpathmoveto{\pgfqpoint{0.000000in}{0.000000in}}%
\pgfpathlineto{\pgfqpoint{0.000000in}{0.000000in}}%
\pgfpathclose%
\pgfusepath{stroke,fill}%
\end{pgfscope}%
\begin{pgfscope}%
\pgfpathrectangle{\pgfqpoint{0.647939in}{0.492442in}}{\pgfqpoint{3.079299in}{3.079299in}}%
\pgfusepath{clip}%
\pgfsetroundcap%
\pgfsetroundjoin%
\pgfsetlinewidth{0.301125pt}%
\definecolor{currentstroke}{rgb}{0.500000,0.500000,0.500000}%
\pgfsetstrokecolor{currentstroke}%
\pgfsetstrokeopacity{0.300000}%
\pgfsetdash{}{0pt}%
\pgfpathmoveto{\pgfqpoint{2.854091in}{2.471876in}}%
\pgfusepath{stroke}%
\end{pgfscope}%
\begin{pgfscope}%
\pgfpathrectangle{\pgfqpoint{0.647939in}{0.492442in}}{\pgfqpoint{3.079299in}{3.079299in}}%
\pgfusepath{clip}%
\pgfsetroundcap%
\pgfsetroundjoin%
\definecolor{currentfill}{rgb}{0.500000,0.500000,0.500000}%
\pgfsetfillcolor{currentfill}%
\pgfsetfillopacity{0.300000}%
\pgfsetlinewidth{0.301125pt}%
\definecolor{currentstroke}{rgb}{0.500000,0.500000,0.500000}%
\pgfsetstrokecolor{currentstroke}%
\pgfsetstrokeopacity{0.300000}%
\pgfsetdash{}{0pt}%
\pgfpathmoveto{\pgfqpoint{0.000000in}{0.000000in}}%
\pgfpathlineto{\pgfqpoint{0.000000in}{0.000000in}}%
\pgfpathclose%
\pgfusepath{stroke,fill}%
\end{pgfscope}%
\begin{pgfscope}%
\pgfpathrectangle{\pgfqpoint{0.647939in}{0.492442in}}{\pgfqpoint{3.079299in}{3.079299in}}%
\pgfusepath{clip}%
\pgfsetroundcap%
\pgfsetroundjoin%
\pgfsetlinewidth{0.301125pt}%
\definecolor{currentstroke}{rgb}{0.500000,0.500000,0.500000}%
\pgfsetstrokecolor{currentstroke}%
\pgfsetstrokeopacity{0.300000}%
\pgfsetdash{}{0pt}%
\pgfpathmoveto{\pgfqpoint{2.622057in}{2.018981in}}%
\pgfusepath{stroke}%
\end{pgfscope}%
\begin{pgfscope}%
\pgfpathrectangle{\pgfqpoint{0.647939in}{0.492442in}}{\pgfqpoint{3.079299in}{3.079299in}}%
\pgfusepath{clip}%
\pgfsetroundcap%
\pgfsetroundjoin%
\definecolor{currentfill}{rgb}{0.500000,0.500000,0.500000}%
\pgfsetfillcolor{currentfill}%
\pgfsetfillopacity{0.300000}%
\pgfsetlinewidth{0.301125pt}%
\definecolor{currentstroke}{rgb}{0.500000,0.500000,0.500000}%
\pgfsetstrokecolor{currentstroke}%
\pgfsetstrokeopacity{0.300000}%
\pgfsetdash{}{0pt}%
\pgfpathmoveto{\pgfqpoint{0.000000in}{0.000000in}}%
\pgfpathlineto{\pgfqpoint{0.000000in}{0.000000in}}%
\pgfpathclose%
\pgfusepath{stroke,fill}%
\end{pgfscope}%
\begin{pgfscope}%
\pgfpathrectangle{\pgfqpoint{0.647939in}{0.492442in}}{\pgfqpoint{3.079299in}{3.079299in}}%
\pgfusepath{clip}%
\pgfsetroundcap%
\pgfsetroundjoin%
\pgfsetlinewidth{0.301125pt}%
\definecolor{currentstroke}{rgb}{0.500000,0.500000,0.500000}%
\pgfsetstrokecolor{currentstroke}%
\pgfsetstrokeopacity{0.300000}%
\pgfsetdash{}{0pt}%
\pgfpathmoveto{\pgfqpoint{2.699808in}{2.147095in}}%
\pgfusepath{stroke}%
\end{pgfscope}%
\begin{pgfscope}%
\pgfpathrectangle{\pgfqpoint{0.647939in}{0.492442in}}{\pgfqpoint{3.079299in}{3.079299in}}%
\pgfusepath{clip}%
\pgfsetroundcap%
\pgfsetroundjoin%
\definecolor{currentfill}{rgb}{0.500000,0.500000,0.500000}%
\pgfsetfillcolor{currentfill}%
\pgfsetfillopacity{0.300000}%
\pgfsetlinewidth{0.301125pt}%
\definecolor{currentstroke}{rgb}{0.500000,0.500000,0.500000}%
\pgfsetstrokecolor{currentstroke}%
\pgfsetstrokeopacity{0.300000}%
\pgfsetdash{}{0pt}%
\pgfpathmoveto{\pgfqpoint{0.000000in}{0.000000in}}%
\pgfpathlineto{\pgfqpoint{0.000000in}{0.000000in}}%
\pgfpathclose%
\pgfusepath{stroke,fill}%
\end{pgfscope}%
\begin{pgfscope}%
\pgfpathrectangle{\pgfqpoint{0.647939in}{0.492442in}}{\pgfqpoint{3.079299in}{3.079299in}}%
\pgfusepath{clip}%
\pgfsetroundcap%
\pgfsetroundjoin%
\pgfsetlinewidth{0.301125pt}%
\definecolor{currentstroke}{rgb}{0.500000,0.500000,0.500000}%
\pgfsetstrokecolor{currentstroke}%
\pgfsetstrokeopacity{0.300000}%
\pgfsetdash{}{0pt}%
\pgfpathmoveto{\pgfqpoint{2.242462in}{2.574344in}}%
\pgfusepath{stroke}%
\end{pgfscope}%
\begin{pgfscope}%
\pgfpathrectangle{\pgfqpoint{0.647939in}{0.492442in}}{\pgfqpoint{3.079299in}{3.079299in}}%
\pgfusepath{clip}%
\pgfsetroundcap%
\pgfsetroundjoin%
\definecolor{currentfill}{rgb}{0.500000,0.500000,0.500000}%
\pgfsetfillcolor{currentfill}%
\pgfsetfillopacity{0.300000}%
\pgfsetlinewidth{0.301125pt}%
\definecolor{currentstroke}{rgb}{0.500000,0.500000,0.500000}%
\pgfsetstrokecolor{currentstroke}%
\pgfsetstrokeopacity{0.300000}%
\pgfsetdash{}{0pt}%
\pgfpathmoveto{\pgfqpoint{0.000000in}{0.000000in}}%
\pgfpathlineto{\pgfqpoint{0.000000in}{0.000000in}}%
\pgfpathclose%
\pgfusepath{stroke,fill}%
\end{pgfscope}%
\begin{pgfscope}%
\pgfpathrectangle{\pgfqpoint{0.647939in}{0.492442in}}{\pgfqpoint{3.079299in}{3.079299in}}%
\pgfusepath{clip}%
\pgfsetroundcap%
\pgfsetroundjoin%
\pgfsetlinewidth{0.301125pt}%
\definecolor{currentstroke}{rgb}{0.500000,0.500000,0.500000}%
\pgfsetstrokecolor{currentstroke}%
\pgfsetstrokeopacity{0.300000}%
\pgfsetdash{}{0pt}%
\pgfpathmoveto{\pgfqpoint{2.757995in}{2.343103in}}%
\pgfusepath{stroke}%
\end{pgfscope}%
\begin{pgfscope}%
\pgfpathrectangle{\pgfqpoint{0.647939in}{0.492442in}}{\pgfqpoint{3.079299in}{3.079299in}}%
\pgfusepath{clip}%
\pgfsetroundcap%
\pgfsetroundjoin%
\definecolor{currentfill}{rgb}{0.500000,0.500000,0.500000}%
\pgfsetfillcolor{currentfill}%
\pgfsetfillopacity{0.300000}%
\pgfsetlinewidth{0.301125pt}%
\definecolor{currentstroke}{rgb}{0.500000,0.500000,0.500000}%
\pgfsetstrokecolor{currentstroke}%
\pgfsetstrokeopacity{0.300000}%
\pgfsetdash{}{0pt}%
\pgfpathmoveto{\pgfqpoint{0.000000in}{0.000000in}}%
\pgfpathlineto{\pgfqpoint{0.000000in}{0.000000in}}%
\pgfpathclose%
\pgfusepath{stroke,fill}%
\end{pgfscope}%
\begin{pgfscope}%
\pgfpathrectangle{\pgfqpoint{0.647939in}{0.492442in}}{\pgfqpoint{3.079299in}{3.079299in}}%
\pgfusepath{clip}%
\pgfsetroundcap%
\pgfsetroundjoin%
\pgfsetlinewidth{0.301125pt}%
\definecolor{currentstroke}{rgb}{0.500000,0.500000,0.500000}%
\pgfsetstrokecolor{currentstroke}%
\pgfsetstrokeopacity{0.300000}%
\pgfsetdash{}{0pt}%
\pgfpathmoveto{\pgfqpoint{2.293513in}{1.541757in}}%
\pgfusepath{stroke}%
\end{pgfscope}%
\begin{pgfscope}%
\pgfpathrectangle{\pgfqpoint{0.647939in}{0.492442in}}{\pgfqpoint{3.079299in}{3.079299in}}%
\pgfusepath{clip}%
\pgfsetroundcap%
\pgfsetroundjoin%
\definecolor{currentfill}{rgb}{0.500000,0.500000,0.500000}%
\pgfsetfillcolor{currentfill}%
\pgfsetfillopacity{0.300000}%
\pgfsetlinewidth{0.301125pt}%
\definecolor{currentstroke}{rgb}{0.500000,0.500000,0.500000}%
\pgfsetstrokecolor{currentstroke}%
\pgfsetstrokeopacity{0.300000}%
\pgfsetdash{}{0pt}%
\pgfpathmoveto{\pgfqpoint{0.000000in}{0.000000in}}%
\pgfpathlineto{\pgfqpoint{0.000000in}{0.000000in}}%
\pgfpathclose%
\pgfusepath{stroke,fill}%
\end{pgfscope}%
\begin{pgfscope}%
\pgfpathrectangle{\pgfqpoint{0.647939in}{0.492442in}}{\pgfqpoint{3.079299in}{3.079299in}}%
\pgfusepath{clip}%
\pgfsetroundcap%
\pgfsetroundjoin%
\pgfsetlinewidth{0.301125pt}%
\definecolor{currentstroke}{rgb}{0.500000,0.500000,0.500000}%
\pgfsetstrokecolor{currentstroke}%
\pgfsetstrokeopacity{0.300000}%
\pgfsetdash{}{0pt}%
\pgfpathmoveto{\pgfqpoint{2.379927in}{1.799404in}}%
\pgfusepath{stroke}%
\end{pgfscope}%
\begin{pgfscope}%
\pgfpathrectangle{\pgfqpoint{0.647939in}{0.492442in}}{\pgfqpoint{3.079299in}{3.079299in}}%
\pgfusepath{clip}%
\pgfsetroundcap%
\pgfsetroundjoin%
\definecolor{currentfill}{rgb}{0.500000,0.500000,0.500000}%
\pgfsetfillcolor{currentfill}%
\pgfsetfillopacity{0.300000}%
\pgfsetlinewidth{0.301125pt}%
\definecolor{currentstroke}{rgb}{0.500000,0.500000,0.500000}%
\pgfsetstrokecolor{currentstroke}%
\pgfsetstrokeopacity{0.300000}%
\pgfsetdash{}{0pt}%
\pgfpathmoveto{\pgfqpoint{0.000000in}{0.000000in}}%
\pgfpathlineto{\pgfqpoint{0.000000in}{0.000000in}}%
\pgfpathclose%
\pgfusepath{stroke,fill}%
\end{pgfscope}%
\begin{pgfscope}%
\pgfpathrectangle{\pgfqpoint{0.647939in}{0.492442in}}{\pgfqpoint{3.079299in}{3.079299in}}%
\pgfusepath{clip}%
\pgfsetroundcap%
\pgfsetroundjoin%
\pgfsetlinewidth{0.301125pt}%
\definecolor{currentstroke}{rgb}{0.500000,0.500000,0.500000}%
\pgfsetstrokecolor{currentstroke}%
\pgfsetstrokeopacity{0.300000}%
\pgfsetdash{}{0pt}%
\pgfpathmoveto{\pgfqpoint{2.192362in}{1.665561in}}%
\pgfusepath{stroke}%
\end{pgfscope}%
\begin{pgfscope}%
\pgfpathrectangle{\pgfqpoint{0.647939in}{0.492442in}}{\pgfqpoint{3.079299in}{3.079299in}}%
\pgfusepath{clip}%
\pgfsetroundcap%
\pgfsetroundjoin%
\definecolor{currentfill}{rgb}{0.500000,0.500000,0.500000}%
\pgfsetfillcolor{currentfill}%
\pgfsetfillopacity{0.300000}%
\pgfsetlinewidth{0.301125pt}%
\definecolor{currentstroke}{rgb}{0.500000,0.500000,0.500000}%
\pgfsetstrokecolor{currentstroke}%
\pgfsetstrokeopacity{0.300000}%
\pgfsetdash{}{0pt}%
\pgfpathmoveto{\pgfqpoint{0.000000in}{0.000000in}}%
\pgfpathlineto{\pgfqpoint{0.000000in}{0.000000in}}%
\pgfpathclose%
\pgfusepath{stroke,fill}%
\end{pgfscope}%
\begin{pgfscope}%
\pgfpathrectangle{\pgfqpoint{0.647939in}{0.492442in}}{\pgfqpoint{3.079299in}{3.079299in}}%
\pgfusepath{clip}%
\pgfsetroundcap%
\pgfsetroundjoin%
\pgfsetlinewidth{0.301125pt}%
\definecolor{currentstroke}{rgb}{0.500000,0.500000,0.500000}%
\pgfsetstrokecolor{currentstroke}%
\pgfsetstrokeopacity{0.300000}%
\pgfsetdash{}{0pt}%
\pgfpathmoveto{\pgfqpoint{2.153499in}{2.367671in}}%
\pgfusepath{stroke}%
\end{pgfscope}%
\begin{pgfscope}%
\pgfpathrectangle{\pgfqpoint{0.647939in}{0.492442in}}{\pgfqpoint{3.079299in}{3.079299in}}%
\pgfusepath{clip}%
\pgfsetroundcap%
\pgfsetroundjoin%
\definecolor{currentfill}{rgb}{0.500000,0.500000,0.500000}%
\pgfsetfillcolor{currentfill}%
\pgfsetfillopacity{0.300000}%
\pgfsetlinewidth{0.301125pt}%
\definecolor{currentstroke}{rgb}{0.500000,0.500000,0.500000}%
\pgfsetstrokecolor{currentstroke}%
\pgfsetstrokeopacity{0.300000}%
\pgfsetdash{}{0pt}%
\pgfpathmoveto{\pgfqpoint{0.000000in}{0.000000in}}%
\pgfpathlineto{\pgfqpoint{0.000000in}{0.000000in}}%
\pgfpathclose%
\pgfusepath{stroke,fill}%
\end{pgfscope}%
\begin{pgfscope}%
\pgfpathrectangle{\pgfqpoint{0.647939in}{0.492442in}}{\pgfqpoint{3.079299in}{3.079299in}}%
\pgfusepath{clip}%
\pgfsetroundcap%
\pgfsetroundjoin%
\pgfsetlinewidth{0.301125pt}%
\definecolor{currentstroke}{rgb}{0.500000,0.500000,0.500000}%
\pgfsetstrokecolor{currentstroke}%
\pgfsetstrokeopacity{0.300000}%
\pgfsetdash{}{0pt}%
\pgfpathmoveto{\pgfqpoint{2.081892in}{2.310530in}}%
\pgfusepath{stroke}%
\end{pgfscope}%
\begin{pgfscope}%
\pgfpathrectangle{\pgfqpoint{0.647939in}{0.492442in}}{\pgfqpoint{3.079299in}{3.079299in}}%
\pgfusepath{clip}%
\pgfsetroundcap%
\pgfsetroundjoin%
\definecolor{currentfill}{rgb}{0.500000,0.500000,0.500000}%
\pgfsetfillcolor{currentfill}%
\pgfsetfillopacity{0.300000}%
\pgfsetlinewidth{0.301125pt}%
\definecolor{currentstroke}{rgb}{0.500000,0.500000,0.500000}%
\pgfsetstrokecolor{currentstroke}%
\pgfsetstrokeopacity{0.300000}%
\pgfsetdash{}{0pt}%
\pgfpathmoveto{\pgfqpoint{0.000000in}{0.000000in}}%
\pgfpathlineto{\pgfqpoint{0.000000in}{0.000000in}}%
\pgfpathclose%
\pgfusepath{stroke,fill}%
\end{pgfscope}%
\begin{pgfscope}%
\pgfpathrectangle{\pgfqpoint{0.647939in}{0.492442in}}{\pgfqpoint{3.079299in}{3.079299in}}%
\pgfusepath{clip}%
\pgfsetroundcap%
\pgfsetroundjoin%
\pgfsetlinewidth{0.301125pt}%
\definecolor{currentstroke}{rgb}{0.500000,0.500000,0.500000}%
\pgfsetstrokecolor{currentstroke}%
\pgfsetstrokeopacity{0.300000}%
\pgfsetdash{}{0pt}%
\pgfpathmoveto{\pgfqpoint{2.070576in}{2.123472in}}%
\pgfusepath{stroke}%
\end{pgfscope}%
\begin{pgfscope}%
\pgfpathrectangle{\pgfqpoint{0.647939in}{0.492442in}}{\pgfqpoint{3.079299in}{3.079299in}}%
\pgfusepath{clip}%
\pgfsetroundcap%
\pgfsetroundjoin%
\definecolor{currentfill}{rgb}{0.500000,0.500000,0.500000}%
\pgfsetfillcolor{currentfill}%
\pgfsetfillopacity{0.300000}%
\pgfsetlinewidth{0.301125pt}%
\definecolor{currentstroke}{rgb}{0.500000,0.500000,0.500000}%
\pgfsetstrokecolor{currentstroke}%
\pgfsetstrokeopacity{0.300000}%
\pgfsetdash{}{0pt}%
\pgfpathmoveto{\pgfqpoint{0.000000in}{0.000000in}}%
\pgfpathlineto{\pgfqpoint{0.000000in}{0.000000in}}%
\pgfpathclose%
\pgfusepath{stroke,fill}%
\end{pgfscope}%
\begin{pgfscope}%
\pgfpathrectangle{\pgfqpoint{0.647939in}{0.492442in}}{\pgfqpoint{3.079299in}{3.079299in}}%
\pgfusepath{clip}%
\pgfsetbuttcap%
\pgfsetroundjoin%
\pgfsetlinewidth{0.301125pt}%
\definecolor{currentstroke}{rgb}{0.500000,0.500000,0.500000}%
\pgfsetstrokecolor{currentstroke}%
\pgfsetstrokeopacity{0.300000}%
\pgfsetdash{}{0pt}%
\pgfpathmoveto{\pgfqpoint{0.647939in}{0.492442in}}%
\pgfpathlineto{\pgfqpoint{0.647939in}{0.492442in}}%
\pgfpathlineto{\pgfqpoint{0.714844in}{0.506716in}}%
\pgfpathlineto{\pgfqpoint{0.780999in}{0.524110in}}%
\pgfpathlineto{\pgfqpoint{0.846127in}{0.544996in}}%
\pgfpathlineto{\pgfqpoint{0.909891in}{0.569710in}}%
\pgfpathlineto{\pgfqpoint{0.971901in}{0.598518in}}%
\pgfpathlineto{\pgfqpoint{1.031741in}{0.631581in}}%
\pgfpathlineto{\pgfqpoint{1.089001in}{0.668925in}}%
\pgfpathlineto{\pgfqpoint{1.143332in}{0.710425in}}%
\pgfpathlineto{\pgfqpoint{1.194502in}{0.755752in}}%
\pgfpathlineto{\pgfqpoint{1.242429in}{0.804489in}}%
\pgfpathlineto{\pgfqpoint{1.287211in}{0.856138in}}%
\pgfpathlineto{\pgfqpoint{1.329095in}{0.910173in}}%
\pgfpathlineto{\pgfqpoint{1.368459in}{0.966078in}}%
\pgfpathlineto{\pgfqpoint{1.405749in}{1.023377in}}%
\pgfpathlineto{\pgfqpoint{1.441439in}{1.081674in}}%
\pgfpathlineto{\pgfqpoint{1.475995in}{1.140647in}}%
\pgfpathlineto{\pgfqpoint{1.509862in}{1.200018in}}%
\pgfpathlineto{\pgfqpoint{1.543452in}{1.259549in}}%
\pgfpathlineto{\pgfqpoint{1.577141in}{1.319032in}}%
\pgfpathlineto{\pgfqpoint{1.611276in}{1.378275in}}%
\pgfpathlineto{\pgfqpoint{1.646177in}{1.437070in}}%
\pgfpathlineto{\pgfqpoint{1.682140in}{1.495183in}}%
\pgfpathlineto{\pgfqpoint{1.719452in}{1.552413in}}%
\pgfpathlineto{\pgfqpoint{1.758396in}{1.608543in}}%
\pgfpathlineto{\pgfqpoint{1.799238in}{1.663308in}}%
\pgfpathlineto{\pgfqpoint{1.842226in}{1.716329in}}%
\pgfpathlineto{\pgfqpoint{1.887633in}{1.767292in}}%
\pgfpathlineto{\pgfqpoint{1.935557in}{1.815775in}}%
\pgfpathlineto{\pgfqpoint{1.986023in}{1.861526in}}%
\pgfpathlineto{\pgfqpoint{2.038338in}{1.904550in}}%
\pgfpathlineto{\pgfqpoint{2.088898in}{1.944745in}}%
\pgfpathlineto{\pgfqpoint{2.134254in}{1.985030in}}%
\pgfpathlineto{\pgfqpoint{2.134254in}{1.985030in}}%
\pgfpathlineto{\pgfqpoint{2.154712in}{2.003084in}}%
\pgfpathlineto{\pgfqpoint{2.174618in}{2.021477in}}%
\pgfpathlineto{\pgfqpoint{2.195112in}{2.043163in}}%
\pgfpathlineto{\pgfqpoint{2.214280in}{2.063549in}}%
\pgfpathlineto{\pgfqpoint{2.235423in}{2.085776in}}%
\pgfpathlineto{\pgfqpoint{2.260445in}{2.113052in}}%
\pgfpathlineto{\pgfqpoint{2.285333in}{2.138070in}}%
\pgfpathlineto{\pgfqpoint{2.313807in}{2.168210in}}%
\pgfpathlineto{\pgfqpoint{2.357324in}{2.217144in}}%
\pgfpathlineto{\pgfqpoint{2.401433in}{2.267890in}}%
\pgfpathlineto{\pgfqpoint{2.445136in}{2.319493in}}%
\pgfpathlineto{\pgfqpoint{2.488497in}{2.371778in}}%
\pgfpathlineto{\pgfqpoint{2.531559in}{2.424556in}}%
\pgfpathlineto{\pgfqpoint{2.574388in}{2.477703in}}%
\pgfpathlineto{\pgfqpoint{2.617008in}{2.531089in}}%
\pgfpathlineto{\pgfqpoint{2.659448in}{2.584610in}}%
\pgfpathlineto{\pgfqpoint{2.701820in}{2.638272in}}%
\pgfpathlineto{\pgfqpoint{2.744157in}{2.691986in}}%
\pgfpathlineto{\pgfqpoint{2.786496in}{2.745676in}}%
\pgfpathlineto{\pgfqpoint{2.828917in}{2.799329in}}%
\pgfpathlineto{\pgfqpoint{2.871475in}{2.852885in}}%
\pgfpathlineto{\pgfqpoint{2.914221in}{2.906291in}}%
\pgfpathlineto{\pgfqpoint{2.957219in}{2.959500in}}%
\pgfpathlineto{\pgfqpoint{3.000533in}{3.012458in}}%
\pgfpathlineto{\pgfqpoint{3.044222in}{3.065114in}}%
\pgfpathlineto{\pgfqpoint{3.088353in}{3.117397in}}%
\pgfpathlineto{\pgfqpoint{3.132997in}{3.169246in}}%
\pgfpathlineto{\pgfqpoint{3.178227in}{3.220587in}}%
\pgfpathlineto{\pgfqpoint{3.224118in}{3.271336in}}%
\pgfpathlineto{\pgfqpoint{3.270753in}{3.321406in}}%
\pgfpathlineto{\pgfqpoint{3.318215in}{3.370691in}}%
\pgfpathlineto{\pgfqpoint{3.366594in}{3.419076in}}%
\pgfpathlineto{\pgfqpoint{3.415983in}{3.466430in}}%
\pgfpathlineto{\pgfqpoint{3.466478in}{3.512601in}}%
\pgfpathlineto{\pgfqpoint{3.518181in}{3.557413in}}%
\pgfpathlineto{\pgfqpoint{3.535181in}{3.571741in}}%
\pgfusepath{stroke}%
\end{pgfscope}%
\begin{pgfscope}%
\pgfpathrectangle{\pgfqpoint{0.647939in}{0.492442in}}{\pgfqpoint{3.079299in}{3.079299in}}%
\pgfusepath{clip}%
\pgfsetbuttcap%
\pgfsetroundjoin%
\pgfsetlinewidth{0.301125pt}%
\definecolor{currentstroke}{rgb}{0.500000,0.500000,0.500000}%
\pgfsetstrokecolor{currentstroke}%
\pgfsetstrokeopacity{0.300000}%
\pgfsetdash{}{0pt}%
\pgfpathmoveto{\pgfqpoint{0.857891in}{0.492442in}}%
\pgfpathlineto{\pgfqpoint{0.857891in}{0.492442in}}%
\pgfpathlineto{\pgfqpoint{0.921094in}{0.518549in}}%
\pgfpathlineto{\pgfqpoint{0.982337in}{0.548949in}}%
\pgfpathlineto{\pgfqpoint{1.041161in}{0.583776in}}%
\pgfpathlineto{\pgfqpoint{1.097134in}{0.623001in}}%
\pgfpathlineto{\pgfqpoint{1.149908in}{0.666431in}}%
\pgfpathlineto{\pgfqpoint{1.199288in}{0.713694in}}%
\pgfpathlineto{\pgfqpoint{1.245262in}{0.764268in}}%
\pgfusepath{stroke}%
\end{pgfscope}%
\begin{pgfscope}%
\pgfpathrectangle{\pgfqpoint{0.647939in}{0.492442in}}{\pgfqpoint{3.079299in}{3.079299in}}%
\pgfusepath{clip}%
\pgfsetbuttcap%
\pgfsetroundjoin%
\pgfsetlinewidth{0.301125pt}%
\definecolor{currentstroke}{rgb}{0.500000,0.500000,0.500000}%
\pgfsetstrokecolor{currentstroke}%
\pgfsetstrokeopacity{0.300000}%
\pgfsetdash{}{0pt}%
\pgfpathmoveto{\pgfqpoint{1.137828in}{0.492442in}}%
\pgfpathlineto{\pgfqpoint{1.137828in}{0.492442in}}%
\pgfpathlineto{\pgfqpoint{1.182792in}{0.543881in}}%
\pgfpathlineto{\pgfqpoint{1.223922in}{0.598434in}}%
\pgfpathlineto{\pgfqpoint{1.261636in}{0.655410in}}%
\pgfpathlineto{\pgfqpoint{1.296484in}{0.714216in}}%
\pgfpathlineto{\pgfqpoint{1.329062in}{0.774338in}}%
\pgfpathlineto{\pgfqpoint{1.359942in}{0.835361in}}%
\pgfpathlineto{\pgfqpoint{1.389643in}{0.896960in}}%
\pgfpathlineto{\pgfqpoint{1.418617in}{0.958889in}}%
\pgfusepath{stroke}%
\end{pgfscope}%
\begin{pgfscope}%
\pgfpathrectangle{\pgfqpoint{0.647939in}{0.492442in}}{\pgfqpoint{3.079299in}{3.079299in}}%
\pgfusepath{clip}%
\pgfsetbuttcap%
\pgfsetroundjoin%
\pgfsetlinewidth{0.301125pt}%
\definecolor{currentstroke}{rgb}{0.500000,0.500000,0.500000}%
\pgfsetstrokecolor{currentstroke}%
\pgfsetstrokeopacity{0.300000}%
\pgfsetdash{}{0pt}%
\pgfpathmoveto{\pgfqpoint{1.347780in}{0.492442in}}%
\pgfpathlineto{\pgfqpoint{1.347780in}{0.492442in}}%
\pgfpathlineto{\pgfqpoint{1.350741in}{0.560500in}}%
\pgfpathlineto{\pgfqpoint{1.359199in}{0.628181in}}%
\pgfpathlineto{\pgfqpoint{1.371693in}{0.695313in}}%
\pgfpathlineto{\pgfqpoint{1.387236in}{0.761789in}}%
\pgfpathlineto{\pgfqpoint{1.405156in}{0.827684in}}%
\pgfusepath{stroke}%
\end{pgfscope}%
\begin{pgfscope}%
\pgfpathrectangle{\pgfqpoint{0.647939in}{0.492442in}}{\pgfqpoint{3.079299in}{3.079299in}}%
\pgfusepath{clip}%
\pgfsetbuttcap%
\pgfsetroundjoin%
\pgfsetlinewidth{0.301125pt}%
\definecolor{currentstroke}{rgb}{0.500000,0.500000,0.500000}%
\pgfsetstrokecolor{currentstroke}%
\pgfsetstrokeopacity{0.300000}%
\pgfsetdash{}{0pt}%
\pgfpathmoveto{\pgfqpoint{1.627716in}{0.492442in}}%
\pgfpathlineto{\pgfqpoint{1.627716in}{0.492442in}}%
\pgfpathlineto{\pgfqpoint{1.569739in}{0.528106in}}%
\pgfpathlineto{\pgfqpoint{1.520587in}{0.574873in}}%
\pgfpathlineto{\pgfqpoint{1.487156in}{0.625766in}}%
\pgfpathlineto{\pgfqpoint{1.466396in}{0.679160in}}%
\pgfpathlineto{\pgfqpoint{1.455297in}{0.736656in}}%
\pgfusepath{stroke}%
\end{pgfscope}%
\begin{pgfscope}%
\pgfpathrectangle{\pgfqpoint{0.647939in}{0.492442in}}{\pgfqpoint{3.079299in}{3.079299in}}%
\pgfusepath{clip}%
\pgfsetbuttcap%
\pgfsetroundjoin%
\pgfsetlinewidth{0.301125pt}%
\definecolor{currentstroke}{rgb}{0.500000,0.500000,0.500000}%
\pgfsetstrokecolor{currentstroke}%
\pgfsetstrokeopacity{0.300000}%
\pgfsetdash{}{0pt}%
\pgfpathmoveto{\pgfqpoint{1.977636in}{0.492442in}}%
\pgfpathlineto{\pgfqpoint{1.977636in}{0.492442in}}%
\pgfpathlineto{\pgfqpoint{1.909385in}{0.497158in}}%
\pgfpathlineto{\pgfqpoint{1.841436in}{0.504991in}}%
\pgfpathlineto{\pgfqpoint{1.774150in}{0.517144in}}%
\pgfpathlineto{\pgfqpoint{1.708234in}{0.535156in}}%
\pgfpathlineto{\pgfqpoint{1.645052in}{0.560977in}}%
\pgfusepath{stroke}%
\end{pgfscope}%
\begin{pgfscope}%
\pgfpathrectangle{\pgfqpoint{0.647939in}{0.492442in}}{\pgfqpoint{3.079299in}{3.079299in}}%
\pgfusepath{clip}%
\pgfsetbuttcap%
\pgfsetroundjoin%
\pgfsetlinewidth{0.301125pt}%
\definecolor{currentstroke}{rgb}{0.500000,0.500000,0.500000}%
\pgfsetstrokecolor{currentstroke}%
\pgfsetstrokeopacity{0.300000}%
\pgfsetdash{}{0pt}%
\pgfpathmoveto{\pgfqpoint{2.397541in}{0.492442in}}%
\pgfpathlineto{\pgfqpoint{2.397541in}{0.492442in}}%
\pgfpathlineto{\pgfqpoint{2.329144in}{0.494439in}}%
\pgfpathlineto{\pgfqpoint{2.260724in}{0.495483in}}%
\pgfpathlineto{\pgfqpoint{2.192297in}{0.496015in}}%
\pgfpathlineto{\pgfqpoint{2.123871in}{0.496544in}}%
\pgfpathlineto{\pgfqpoint{2.055452in}{0.497648in}}%
\pgfusepath{stroke}%
\end{pgfscope}%
\begin{pgfscope}%
\pgfpathrectangle{\pgfqpoint{0.647939in}{0.492442in}}{\pgfqpoint{3.079299in}{3.079299in}}%
\pgfusepath{clip}%
\pgfsetbuttcap%
\pgfsetroundjoin%
\pgfsetlinewidth{0.301125pt}%
\definecolor{currentstroke}{rgb}{0.500000,0.500000,0.500000}%
\pgfsetstrokecolor{currentstroke}%
\pgfsetstrokeopacity{0.300000}%
\pgfsetdash{}{0pt}%
\pgfpathmoveto{\pgfqpoint{2.747461in}{0.492442in}}%
\pgfpathlineto{\pgfqpoint{2.747461in}{0.492442in}}%
\pgfpathlineto{\pgfqpoint{2.679960in}{0.503618in}}%
\pgfpathlineto{\pgfqpoint{2.612155in}{0.512771in}}%
\pgfpathlineto{\pgfqpoint{2.544110in}{0.519932in}}%
\pgfpathlineto{\pgfqpoint{2.475892in}{0.525218in}}%
\pgfpathlineto{\pgfqpoint{2.407564in}{0.528832in}}%
\pgfpathlineto{\pgfqpoint{2.339174in}{0.531055in}}%
\pgfpathlineto{\pgfqpoint{2.270757in}{0.532244in}}%
\pgfpathlineto{\pgfqpoint{2.202331in}{0.532831in}}%
\pgfpathlineto{\pgfqpoint{2.133904in}{0.533331in}}%
\pgfpathlineto{\pgfqpoint{2.065484in}{0.534336in}}%
\pgfpathlineto{\pgfqpoint{1.997096in}{0.536522in}}%
\pgfpathlineto{\pgfqpoint{1.928806in}{0.540684in}}%
\pgfpathlineto{\pgfqpoint{1.860769in}{0.547785in}}%
\pgfpathlineto{\pgfqpoint{1.793321in}{0.559036in}}%
\pgfusepath{stroke}%
\end{pgfscope}%
\begin{pgfscope}%
\pgfpathrectangle{\pgfqpoint{0.647939in}{0.492442in}}{\pgfqpoint{3.079299in}{3.079299in}}%
\pgfusepath{clip}%
\pgfsetbuttcap%
\pgfsetroundjoin%
\pgfsetlinewidth{0.301125pt}%
\definecolor{currentstroke}{rgb}{0.500000,0.500000,0.500000}%
\pgfsetstrokecolor{currentstroke}%
\pgfsetstrokeopacity{0.300000}%
\pgfsetdash{}{0pt}%
\pgfpathmoveto{\pgfqpoint{2.957413in}{0.492442in}}%
\pgfpathlineto{\pgfqpoint{2.957413in}{0.492442in}}%
\pgfpathlineto{\pgfqpoint{2.891218in}{0.509760in}}%
\pgfpathlineto{\pgfqpoint{2.824606in}{0.525394in}}%
\pgfpathlineto{\pgfqpoint{2.757583in}{0.539159in}}%
\pgfpathlineto{\pgfqpoint{2.690184in}{0.550928in}}%
\pgfpathlineto{\pgfqpoint{2.622457in}{0.560638in}}%
\pgfpathlineto{\pgfqpoint{2.554468in}{0.568304in}}%
\pgfpathlineto{\pgfqpoint{2.486286in}{0.574024in}}%
\pgfusepath{stroke}%
\end{pgfscope}%
\begin{pgfscope}%
\pgfpathrectangle{\pgfqpoint{0.647939in}{0.492442in}}{\pgfqpoint{3.079299in}{3.079299in}}%
\pgfusepath{clip}%
\pgfsetbuttcap%
\pgfsetroundjoin%
\pgfsetlinewidth{0.301125pt}%
\definecolor{currentstroke}{rgb}{0.500000,0.500000,0.500000}%
\pgfsetstrokecolor{currentstroke}%
\pgfsetstrokeopacity{0.300000}%
\pgfsetdash{}{0pt}%
\pgfpathmoveto{\pgfqpoint{3.167366in}{0.492442in}}%
\pgfpathlineto{\pgfqpoint{3.167366in}{0.492442in}}%
\pgfpathlineto{\pgfqpoint{3.102480in}{0.514171in}}%
\pgfpathlineto{\pgfqpoint{3.037299in}{0.534995in}}%
\pgfpathlineto{\pgfqpoint{2.971757in}{0.554647in}}%
\pgfpathlineto{\pgfqpoint{2.905805in}{0.572868in}}%
\pgfpathlineto{\pgfqpoint{2.839417in}{0.589423in}}%
\pgfpathlineto{\pgfqpoint{2.772592in}{0.604112in}}%
\pgfpathlineto{\pgfqpoint{2.705356in}{0.616780in}}%
\pgfpathlineto{\pgfqpoint{2.637756in}{0.627339in}}%
\pgfpathlineto{\pgfqpoint{2.569858in}{0.635779in}}%
\pgfpathlineto{\pgfqpoint{2.501738in}{0.642180in}}%
\pgfpathlineto{\pgfqpoint{2.433466in}{0.646710in}}%
\pgfpathlineto{\pgfqpoint{2.365104in}{0.649626in}}%
\pgfpathlineto{\pgfqpoint{2.296697in}{0.651274in}}%
\pgfpathlineto{\pgfqpoint{2.228274in}{0.652094in}}%
\pgfpathlineto{\pgfqpoint{2.159847in}{0.652609in}}%
\pgfpathlineto{\pgfqpoint{2.091424in}{0.653421in}}%
\pgfpathlineto{\pgfqpoint{2.023023in}{0.655235in}}%
\pgfpathlineto{\pgfqpoint{1.954702in}{0.658887in}}%
\pgfpathlineto{\pgfqpoint{1.886610in}{0.665422in}}%
\pgfpathlineto{\pgfqpoint{1.819083in}{0.676158in}}%
\pgfpathlineto{\pgfqpoint{1.752827in}{0.692812in}}%
\pgfpathlineto{\pgfqpoint{1.689302in}{0.717628in}}%
\pgfpathlineto{\pgfqpoint{1.631396in}{0.753207in}}%
\pgfpathlineto{\pgfqpoint{1.585299in}{0.799199in}}%
\pgfpathlineto{\pgfqpoint{1.556057in}{0.847813in}}%
\pgfpathlineto{\pgfqpoint{1.538902in}{0.898888in}}%
\pgfpathlineto{\pgfqpoint{1.530976in}{0.954488in}}%
\pgfpathlineto{\pgfqpoint{1.531546in}{1.016770in}}%
\pgfpathlineto{\pgfqpoint{1.540337in}{1.084280in}}%
\pgfpathlineto{\pgfqpoint{1.555193in}{1.150881in}}%
\pgfpathlineto{\pgfqpoint{1.574592in}{1.216314in}}%
\pgfpathlineto{\pgfqpoint{1.597559in}{1.280581in}}%
\pgfpathlineto{\pgfqpoint{1.623490in}{1.343702in}}%
\pgfusepath{stroke}%
\end{pgfscope}%
\begin{pgfscope}%
\pgfpathrectangle{\pgfqpoint{0.647939in}{0.492442in}}{\pgfqpoint{3.079299in}{3.079299in}}%
\pgfusepath{clip}%
\pgfsetbuttcap%
\pgfsetroundjoin%
\pgfsetlinewidth{0.301125pt}%
\definecolor{currentstroke}{rgb}{0.500000,0.500000,0.500000}%
\pgfsetstrokecolor{currentstroke}%
\pgfsetstrokeopacity{0.300000}%
\pgfsetdash{}{0pt}%
\pgfpathmoveto{\pgfqpoint{3.377318in}{0.492442in}}%
\pgfpathlineto{\pgfqpoint{3.377318in}{0.492442in}}%
\pgfpathlineto{\pgfqpoint{3.313040in}{0.515913in}}%
\pgfpathlineto{\pgfqpoint{3.248765in}{0.539389in}}%
\pgfpathlineto{\pgfqpoint{3.184392in}{0.562597in}}%
\pgfpathlineto{\pgfqpoint{3.119827in}{0.585260in}}%
\pgfpathlineto{\pgfqpoint{3.054980in}{0.607100in}}%
\pgfpathlineto{\pgfqpoint{2.989773in}{0.627837in}}%
\pgfpathlineto{\pgfqpoint{2.924146in}{0.647201in}}%
\pgfpathlineto{\pgfqpoint{2.858063in}{0.664936in}}%
\pgfpathlineto{\pgfqpoint{2.791510in}{0.680818in}}%
\pgfpathlineto{\pgfqpoint{2.724506in}{0.694663in}}%
\pgfpathlineto{\pgfqpoint{2.657094in}{0.706352in}}%
\pgfpathlineto{\pgfqpoint{2.589337in}{0.715840in}}%
\pgfpathlineto{\pgfqpoint{2.521312in}{0.723169in}}%
\pgfpathlineto{\pgfqpoint{2.453097in}{0.728475in}}%
\pgfpathlineto{\pgfqpoint{2.384764in}{0.732002in}}%
\pgfpathlineto{\pgfqpoint{2.316370in}{0.734092in}}%
\pgfpathlineto{\pgfqpoint{2.247951in}{0.735170in}}%
\pgfpathlineto{\pgfqpoint{2.179525in}{0.735752in}}%
\pgfpathlineto{\pgfqpoint{2.111100in}{0.736448in}}%
\pgfpathlineto{\pgfqpoint{2.042691in}{0.737989in}}%
\pgfpathlineto{\pgfqpoint{1.974351in}{0.741257in}}%
\pgfpathlineto{\pgfqpoint{1.906218in}{0.747330in}}%
\pgfpathlineto{\pgfqpoint{1.838619in}{0.757593in}}%
\pgfpathlineto{\pgfqpoint{1.772271in}{0.773893in}}%
\pgfpathlineto{\pgfqpoint{1.708737in}{0.798705in}}%
\pgfpathlineto{\pgfqpoint{1.651244in}{0.834923in}}%
\pgfpathlineto{\pgfqpoint{1.607798in}{0.880267in}}%
\pgfusepath{stroke}%
\end{pgfscope}%
\begin{pgfscope}%
\pgfpathrectangle{\pgfqpoint{0.647939in}{0.492442in}}{\pgfqpoint{3.079299in}{3.079299in}}%
\pgfusepath{clip}%
\pgfsetbuttcap%
\pgfsetroundjoin%
\pgfsetlinewidth{0.301125pt}%
\definecolor{currentstroke}{rgb}{0.500000,0.500000,0.500000}%
\pgfsetstrokecolor{currentstroke}%
\pgfsetstrokeopacity{0.300000}%
\pgfsetdash{}{0pt}%
\pgfpathmoveto{\pgfqpoint{3.587270in}{0.492442in}}%
\pgfpathlineto{\pgfqpoint{3.587270in}{0.492442in}}%
\pgfpathlineto{\pgfqpoint{3.522584in}{0.514756in}}%
\pgfpathlineto{\pgfqpoint{3.458206in}{0.537946in}}%
\pgfpathlineto{\pgfqpoint{3.394056in}{0.561763in}}%
\pgfpathlineto{\pgfqpoint{3.330044in}{0.585948in}}%
\pgfpathlineto{\pgfqpoint{3.266071in}{0.610235in}}%
\pgfpathlineto{\pgfqpoint{3.202033in}{0.634352in}}%
\pgfpathlineto{\pgfqpoint{3.137828in}{0.658018in}}%
\pgfpathlineto{\pgfqpoint{3.073358in}{0.680950in}}%
\pgfpathlineto{\pgfqpoint{3.008534in}{0.702859in}}%
\pgfpathlineto{\pgfqpoint{2.943286in}{0.723461in}}%
\pgfpathlineto{\pgfqpoint{2.877562in}{0.742485in}}%
\pgfpathlineto{\pgfqpoint{2.811338in}{0.759682in}}%
\pgfpathlineto{\pgfqpoint{2.744621in}{0.774842in}}%
\pgfpathlineto{\pgfqpoint{2.677443in}{0.787806in}}%
\pgfpathlineto{\pgfqpoint{2.609865in}{0.798490in}}%
\pgfpathlineto{\pgfqpoint{2.541964in}{0.806898in}}%
\pgfpathlineto{\pgfqpoint{2.473830in}{0.813139in}}%
\pgfpathlineto{\pgfqpoint{2.405542in}{0.817425in}}%
\pgfpathlineto{\pgfqpoint{2.337169in}{0.820070in}}%
\pgfpathlineto{\pgfqpoint{2.268757in}{0.821488in}}%
\pgfpathlineto{\pgfqpoint{2.200332in}{0.822198in}}%
\pgfpathlineto{\pgfqpoint{2.131907in}{0.822832in}}%
\pgfpathlineto{\pgfqpoint{2.063493in}{0.824137in}}%
\pgfpathlineto{\pgfqpoint{1.995134in}{0.827011in}}%
\pgfpathlineto{\pgfqpoint{1.926954in}{0.832576in}}%
\pgfpathlineto{\pgfqpoint{1.859271in}{0.842309in}}%
\pgfpathlineto{\pgfqpoint{1.792840in}{0.858222in}}%
\pgfpathlineto{\pgfqpoint{1.729370in}{0.883050in}}%
\pgfpathlineto{\pgfqpoint{1.672489in}{0.920004in}}%
\pgfpathlineto{\pgfqpoint{1.631893in}{0.964724in}}%
\pgfpathlineto{\pgfqpoint{1.607549in}{1.011236in}}%
\pgfpathlineto{\pgfqpoint{1.594241in}{1.060700in}}%
\pgfusepath{stroke}%
\end{pgfscope}%
\begin{pgfscope}%
\pgfpathrectangle{\pgfqpoint{0.647939in}{0.492442in}}{\pgfqpoint{3.079299in}{3.079299in}}%
\pgfusepath{clip}%
\pgfsetbuttcap%
\pgfsetroundjoin%
\pgfsetlinewidth{0.301125pt}%
\definecolor{currentstroke}{rgb}{0.500000,0.500000,0.500000}%
\pgfsetstrokecolor{currentstroke}%
\pgfsetstrokeopacity{0.300000}%
\pgfsetdash{}{0pt}%
\pgfpathmoveto{\pgfqpoint{3.727238in}{0.562426in}}%
\pgfpathlineto{\pgfqpoint{3.727238in}{0.562426in}}%
\pgfpathlineto{\pgfqpoint{3.661999in}{0.583057in}}%
\pgfpathlineto{\pgfqpoint{3.597229in}{0.605122in}}%
\pgfpathlineto{\pgfqpoint{3.532884in}{0.628400in}}%
\pgfpathlineto{\pgfqpoint{3.468899in}{0.652651in}}%
\pgfpathlineto{\pgfqpoint{3.405191in}{0.677626in}}%
\pgfpathlineto{\pgfqpoint{3.341668in}{0.703066in}}%
\pgfpathlineto{\pgfqpoint{3.278223in}{0.728704in}}%
\pgfpathlineto{\pgfqpoint{3.214749in}{0.754266in}}%
\pgfpathlineto{\pgfqpoint{3.151132in}{0.779469in}}%
\pgfpathlineto{\pgfqpoint{3.087263in}{0.804025in}}%
\pgfpathlineto{\pgfqpoint{3.023041in}{0.827639in}}%
\pgfpathlineto{\pgfqpoint{2.958378in}{0.850013in}}%
\pgfpathlineto{\pgfqpoint{2.893208in}{0.870853in}}%
\pgfpathlineto{\pgfqpoint{2.827486in}{0.889881in}}%
\pgfpathlineto{\pgfqpoint{2.761203in}{0.906845in}}%
\pgfpathlineto{\pgfqpoint{2.694384in}{0.921542in}}%
\pgfpathlineto{\pgfqpoint{2.627082in}{0.933841in}}%
\pgfpathlineto{\pgfqpoint{2.559381in}{0.943697in}}%
\pgfpathlineto{\pgfqpoint{2.491373in}{0.951170in}}%
\pgfpathlineto{\pgfqpoint{2.423156in}{0.956437in}}%
\pgfpathlineto{\pgfqpoint{2.354816in}{0.959801in}}%
\pgfpathlineto{\pgfqpoint{2.286416in}{0.961688in}}%
\pgfpathlineto{\pgfqpoint{2.217995in}{0.962639in}}%
\pgfpathlineto{\pgfqpoint{2.149569in}{0.963313in}}%
\pgfpathlineto{\pgfqpoint{2.081153in}{0.964512in}}%
\pgfpathlineto{\pgfqpoint{2.012787in}{0.967247in}}%
\pgfpathlineto{\pgfqpoint{1.944614in}{0.972826in}}%
\pgfpathlineto{\pgfqpoint{1.877021in}{0.983015in}}%
\pgfpathlineto{\pgfqpoint{1.810996in}{1.000320in}}%
\pgfpathlineto{\pgfqpoint{1.749001in}{1.028271in}}%
\pgfpathlineto{\pgfqpoint{1.749001in}{1.028271in}}%
\pgfpathlineto{\pgfqpoint{1.705762in}{1.060533in}}%
\pgfpathlineto{\pgfqpoint{1.671810in}{1.103422in}}%
\pgfpathlineto{\pgfqpoint{1.652365in}{1.148279in}}%
\pgfpathlineto{\pgfqpoint{1.643177in}{1.196159in}}%
\pgfusepath{stroke}%
\end{pgfscope}%
\begin{pgfscope}%
\pgfpathrectangle{\pgfqpoint{0.647939in}{0.492442in}}{\pgfqpoint{3.079299in}{3.079299in}}%
\pgfusepath{clip}%
\pgfsetbuttcap%
\pgfsetroundjoin%
\pgfsetlinewidth{0.301125pt}%
\definecolor{currentstroke}{rgb}{0.500000,0.500000,0.500000}%
\pgfsetstrokecolor{currentstroke}%
\pgfsetstrokeopacity{0.300000}%
\pgfsetdash{}{0pt}%
\pgfpathmoveto{\pgfqpoint{3.727238in}{0.632410in}}%
\pgfpathlineto{\pgfqpoint{3.727238in}{0.632410in}}%
\pgfpathlineto{\pgfqpoint{3.662205in}{0.653682in}}%
\pgfpathlineto{\pgfqpoint{3.597675in}{0.676436in}}%
\pgfpathlineto{\pgfqpoint{3.533600in}{0.700448in}}%
\pgfpathlineto{\pgfqpoint{3.469916in}{0.725477in}}%
\pgfpathlineto{\pgfqpoint{3.406537in}{0.751274in}}%
\pgfpathlineto{\pgfqpoint{3.343366in}{0.777578in}}%
\pgfpathlineto{\pgfqpoint{3.280295in}{0.804122in}}%
\pgfpathlineto{\pgfqpoint{3.217210in}{0.830632in}}%
\pgfpathlineto{\pgfqpoint{3.153994in}{0.856824in}}%
\pgfpathlineto{\pgfqpoint{3.090530in}{0.882408in}}%
\pgfpathlineto{\pgfqpoint{3.026709in}{0.907084in}}%
\pgfpathlineto{\pgfqpoint{2.962434in}{0.930547in}}%
\pgfpathlineto{\pgfqpoint{2.897627in}{0.952492in}}%
\pgfpathlineto{\pgfqpoint{2.832237in}{0.972624in}}%
\pgfpathlineto{\pgfqpoint{2.766241in}{0.990669in}}%
\pgfpathlineto{\pgfqpoint{2.699657in}{1.006398in}}%
\pgfpathlineto{\pgfqpoint{2.632537in}{1.019649in}}%
\pgfpathlineto{\pgfqpoint{2.564964in}{1.030349in}}%
\pgfpathlineto{\pgfqpoint{2.497040in}{1.038538in}}%
\pgfpathlineto{\pgfqpoint{2.428872in}{1.044375in}}%
\pgfpathlineto{\pgfqpoint{2.360555in}{1.048150in}}%
\pgfpathlineto{\pgfqpoint{2.292163in}{1.050296in}}%
\pgfpathlineto{\pgfqpoint{2.223744in}{1.051383in}}%
\pgfpathlineto{\pgfqpoint{2.155319in}{1.052113in}}%
\pgfpathlineto{\pgfqpoint{2.086904in}{1.053353in}}%
\pgfpathlineto{\pgfqpoint{2.018544in}{1.056200in}}%
\pgfpathlineto{\pgfqpoint{1.950403in}{1.062128in}}%
\pgfpathlineto{\pgfqpoint{1.882970in}{1.073215in}}%
\pgfpathlineto{\pgfqpoint{1.817577in}{1.092524in}}%
\pgfpathlineto{\pgfqpoint{1.757743in}{1.124356in}}%
\pgfpathlineto{\pgfqpoint{1.757743in}{1.124356in}}%
\pgfpathlineto{\pgfqpoint{1.721337in}{1.157482in}}%
\pgfpathlineto{\pgfqpoint{1.694847in}{1.199516in}}%
\pgfpathlineto{\pgfqpoint{1.680821in}{1.243738in}}%
\pgfpathlineto{\pgfqpoint{1.676162in}{1.291720in}}%
\pgfpathlineto{\pgfqpoint{1.680121in}{1.345551in}}%
\pgfpathlineto{\pgfqpoint{1.693315in}{1.406525in}}%
\pgfusepath{stroke}%
\end{pgfscope}%
\begin{pgfscope}%
\pgfpathrectangle{\pgfqpoint{0.647939in}{0.492442in}}{\pgfqpoint{3.079299in}{3.079299in}}%
\pgfusepath{clip}%
\pgfsetbuttcap%
\pgfsetroundjoin%
\pgfsetlinewidth{0.301125pt}%
\definecolor{currentstroke}{rgb}{0.500000,0.500000,0.500000}%
\pgfsetstrokecolor{currentstroke}%
\pgfsetstrokeopacity{0.300000}%
\pgfsetdash{}{0pt}%
\pgfpathmoveto{\pgfqpoint{3.727238in}{0.702394in}}%
\pgfpathlineto{\pgfqpoint{3.727238in}{0.702394in}}%
\pgfpathlineto{\pgfqpoint{3.662432in}{0.724346in}}%
\pgfpathlineto{\pgfqpoint{3.598164in}{0.747831in}}%
\pgfpathlineto{\pgfqpoint{3.534388in}{0.772622in}}%
\pgfpathlineto{\pgfqpoint{3.471034in}{0.798479in}}%
\pgfpathlineto{\pgfqpoint{3.408018in}{0.825150in}}%
\pgfpathlineto{\pgfqpoint{3.345240in}{0.852375in}}%
\pgfpathlineto{\pgfqpoint{3.282586in}{0.879889in}}%
\pgfpathlineto{\pgfqpoint{3.219940in}{0.907418in}}%
\pgfpathlineto{\pgfqpoint{3.157177in}{0.934679in}}%
\pgfpathlineto{\pgfqpoint{3.094174in}{0.961380in}}%
\pgfpathlineto{\pgfqpoint{3.030814in}{0.987218in}}%
\pgfpathlineto{\pgfqpoint{2.966991in}{1.011883in}}%
\pgfpathlineto{\pgfqpoint{2.902614in}{1.035058in}}%
\pgfpathlineto{\pgfqpoint{2.837620in}{1.056432in}}%
\pgfpathlineto{\pgfqpoint{2.771975in}{1.075709in}}%
\pgfpathlineto{\pgfqpoint{2.705685in}{1.092629in}}%
\pgfpathlineto{\pgfqpoint{2.638796in}{1.106994in}}%
\pgfpathlineto{\pgfqpoint{2.571389in}{1.118696in}}%
\pgfpathlineto{\pgfqpoint{2.503576in}{1.127743in}}%
\pgfpathlineto{\pgfqpoint{2.435472in}{1.134274in}}%
\pgfpathlineto{\pgfqpoint{2.367188in}{1.138570in}}%
\pgfpathlineto{\pgfqpoint{2.298809in}{1.141057in}}%
\pgfpathlineto{\pgfqpoint{2.230393in}{1.142321in}}%
\pgfpathlineto{\pgfqpoint{2.161969in}{1.143119in}}%
\pgfpathlineto{\pgfqpoint{2.093555in}{1.144397in}}%
\pgfpathlineto{\pgfqpoint{2.025203in}{1.147374in}}%
\pgfpathlineto{\pgfqpoint{1.957108in}{1.153727in}}%
\pgfpathlineto{\pgfqpoint{1.889891in}{1.165940in}}%
\pgfpathlineto{\pgfqpoint{1.825435in}{1.187880in}}%
\pgfpathlineto{\pgfqpoint{1.825435in}{1.187880in}}%
\pgfpathlineto{\pgfqpoint{1.779762in}{1.215148in}}%
\pgfpathlineto{\pgfqpoint{1.779762in}{1.215148in}}%
\pgfpathlineto{\pgfqpoint{1.747298in}{1.247894in}}%
\pgfusepath{stroke}%
\end{pgfscope}%
\begin{pgfscope}%
\pgfpathrectangle{\pgfqpoint{0.647939in}{0.492442in}}{\pgfqpoint{3.079299in}{3.079299in}}%
\pgfusepath{clip}%
\pgfsetbuttcap%
\pgfsetroundjoin%
\pgfsetlinewidth{0.301125pt}%
\definecolor{currentstroke}{rgb}{0.500000,0.500000,0.500000}%
\pgfsetstrokecolor{currentstroke}%
\pgfsetstrokeopacity{0.300000}%
\pgfsetdash{}{0pt}%
\pgfpathmoveto{\pgfqpoint{3.727238in}{0.772378in}}%
\pgfpathlineto{\pgfqpoint{3.727238in}{0.772378in}}%
\pgfpathlineto{\pgfqpoint{3.662681in}{0.795052in}}%
\pgfpathlineto{\pgfqpoint{3.598703in}{0.819313in}}%
\pgfpathlineto{\pgfqpoint{3.535255in}{0.844933in}}%
\pgfpathlineto{\pgfqpoint{3.472269in}{0.871670in}}%
\pgfpathlineto{\pgfqpoint{3.409656in}{0.899274in}}%
\pgfpathlineto{\pgfqpoint{3.347314in}{0.927485in}}%
\pgfpathlineto{\pgfqpoint{3.285128in}{0.956039in}}%
\pgfpathlineto{\pgfqpoint{3.222975in}{0.984666in}}%
\pgfpathlineto{\pgfqpoint{3.160727in}{1.013084in}}%
\pgfpathlineto{\pgfqpoint{3.098254in}{1.041001in}}%
\pgfpathlineto{\pgfqpoint{3.035429in}{1.068116in}}%
\pgfpathlineto{\pgfqpoint{2.972136in}{1.094111in}}%
\pgfpathlineto{\pgfqpoint{2.908271in}{1.118663in}}%
\pgfpathlineto{\pgfqpoint{2.843756in}{1.141443in}}%
\pgfpathlineto{\pgfqpoint{2.778543in}{1.162133in}}%
\pgfpathlineto{\pgfqpoint{2.712625in}{1.180440in}}%
\pgfpathlineto{\pgfqpoint{2.646037in}{1.196129in}}%
\pgfpathlineto{\pgfqpoint{2.578855in}{1.209041in}}%
\pgfpathlineto{\pgfqpoint{2.511191in}{1.219140in}}%
\pgfpathlineto{\pgfqpoint{2.443176in}{1.226534in}}%
\pgfpathlineto{\pgfqpoint{2.374938in}{1.231487in}}%
\pgfpathlineto{\pgfqpoint{2.306579in}{1.234426in}}%
\pgfpathlineto{\pgfqpoint{2.238170in}{1.235947in}}%
\pgfpathlineto{\pgfqpoint{2.169747in}{1.236841in}}%
\pgfpathlineto{\pgfqpoint{2.101333in}{1.238151in}}%
\pgfpathlineto{\pgfqpoint{2.032990in}{1.241262in}}%
\pgfpathlineto{\pgfqpoint{1.964962in}{1.248138in}}%
\pgfpathlineto{\pgfqpoint{1.898090in}{1.261849in}}%
\pgfpathlineto{\pgfqpoint{1.835194in}{1.287447in}}%
\pgfpathlineto{\pgfqpoint{1.835194in}{1.287447in}}%
\pgfpathlineto{\pgfqpoint{1.797567in}{1.315258in}}%
\pgfpathlineto{\pgfqpoint{1.769351in}{1.353798in}}%
\pgfpathlineto{\pgfqpoint{1.755642in}{1.393346in}}%
\pgfpathlineto{\pgfqpoint{1.751668in}{1.435374in}}%
\pgfpathlineto{\pgfqpoint{1.756245in}{1.482592in}}%
\pgfpathlineto{\pgfqpoint{1.770060in}{1.536400in}}%
\pgfusepath{stroke}%
\end{pgfscope}%
\begin{pgfscope}%
\pgfpathrectangle{\pgfqpoint{0.647939in}{0.492442in}}{\pgfqpoint{3.079299in}{3.079299in}}%
\pgfusepath{clip}%
\pgfsetbuttcap%
\pgfsetroundjoin%
\pgfsetlinewidth{0.301125pt}%
\definecolor{currentstroke}{rgb}{0.500000,0.500000,0.500000}%
\pgfsetstrokecolor{currentstroke}%
\pgfsetstrokeopacity{0.300000}%
\pgfsetdash{}{0pt}%
\pgfpathmoveto{\pgfqpoint{3.727238in}{0.842362in}}%
\pgfpathlineto{\pgfqpoint{3.727238in}{0.842362in}}%
\pgfpathlineto{\pgfqpoint{3.662957in}{0.865804in}}%
\pgfpathlineto{\pgfqpoint{3.599298in}{0.890891in}}%
\pgfpathlineto{\pgfqpoint{3.536214in}{0.917394in}}%
\pgfpathlineto{\pgfqpoint{3.473635in}{0.945069in}}%
\pgfpathlineto{\pgfqpoint{3.411471in}{0.973668in}}%
\pgfpathlineto{\pgfqpoint{3.349617in}{1.002934in}}%
\pgfpathlineto{\pgfqpoint{3.287957in}{1.032607in}}%
\pgfpathlineto{\pgfqpoint{3.226364in}{1.062418in}}%
\pgfpathlineto{\pgfqpoint{3.164705in}{1.092091in}}%
\pgfpathlineto{\pgfqpoint{3.102843in}{1.121339in}}%
\pgfpathlineto{\pgfqpoint{3.040643in}{1.149859in}}%
\pgfpathlineto{\pgfqpoint{2.977977in}{1.177336in}}%
\pgfpathlineto{\pgfqpoint{2.914727in}{1.203436in}}%
\pgfpathlineto{\pgfqpoint{2.850800in}{1.227818in}}%
\pgfpathlineto{\pgfqpoint{2.786128in}{1.250141in}}%
\pgfpathlineto{\pgfqpoint{2.720685in}{1.270080in}}%
\pgfpathlineto{\pgfqpoint{2.654491in}{1.287353in}}%
\pgfpathlineto{\pgfqpoint{2.587615in}{1.301751in}}%
\pgfpathlineto{\pgfqpoint{2.520167in}{1.313176in}}%
\pgfpathlineto{\pgfqpoint{2.452284in}{1.321676in}}%
\pgfpathlineto{\pgfqpoint{2.384114in}{1.327481in}}%
\pgfpathlineto{\pgfqpoint{2.315785in}{1.331018in}}%
\pgfpathlineto{\pgfqpoint{2.247386in}{1.332900in}}%
\pgfpathlineto{\pgfqpoint{2.178966in}{1.333963in}}%
\pgfpathlineto{\pgfqpoint{2.110552in}{1.335324in}}%
\pgfpathlineto{\pgfqpoint{2.042217in}{1.338568in}}%
\pgfpathlineto{\pgfqpoint{1.974277in}{1.346087in}}%
\pgfpathlineto{\pgfqpoint{1.907982in}{1.361878in}}%
\pgfpathlineto{\pgfqpoint{1.907982in}{1.361878in}}%
\pgfpathlineto{\pgfqpoint{1.859964in}{1.383989in}}%
\pgfpathlineto{\pgfqpoint{1.859964in}{1.383989in}}%
\pgfpathlineto{\pgfqpoint{1.827296in}{1.411050in}}%
\pgfpathlineto{\pgfqpoint{1.804552in}{1.447771in}}%
\pgfusepath{stroke}%
\end{pgfscope}%
\begin{pgfscope}%
\pgfpathrectangle{\pgfqpoint{0.647939in}{0.492442in}}{\pgfqpoint{3.079299in}{3.079299in}}%
\pgfusepath{clip}%
\pgfsetbuttcap%
\pgfsetroundjoin%
\pgfsetlinewidth{0.301125pt}%
\definecolor{currentstroke}{rgb}{0.500000,0.500000,0.500000}%
\pgfsetstrokecolor{currentstroke}%
\pgfsetstrokeopacity{0.300000}%
\pgfsetdash{}{0pt}%
\pgfpathmoveto{\pgfqpoint{3.727238in}{0.912347in}}%
\pgfpathlineto{\pgfqpoint{3.727238in}{0.912347in}}%
\pgfpathlineto{\pgfqpoint{3.663262in}{0.936606in}}%
\pgfpathlineto{\pgfqpoint{3.599958in}{0.962573in}}%
\pgfpathlineto{\pgfqpoint{3.537278in}{0.990017in}}%
\pgfpathlineto{\pgfqpoint{3.475151in}{1.018694in}}%
\pgfpathlineto{\pgfqpoint{3.413488in}{1.048358in}}%
\pgfpathlineto{\pgfqpoint{3.352183in}{1.078757in}}%
\pgfpathlineto{\pgfqpoint{3.291117in}{1.109633in}}%
\pgfpathlineto{\pgfqpoint{3.230160in}{1.140725in}}%
\pgfpathlineto{\pgfqpoint{3.169175in}{1.171762in}}%
\pgfpathlineto{\pgfqpoint{3.108021in}{1.202462in}}%
\pgfpathlineto{\pgfqpoint{3.046554in}{1.232530in}}%
\pgfpathlineto{\pgfqpoint{2.984636in}{1.261652in}}%
\pgfpathlineto{\pgfqpoint{2.922135in}{1.289497in}}%
\pgfpathlineto{\pgfqpoint{2.858937in}{1.315714in}}%
\pgfpathlineto{\pgfqpoint{2.794952in}{1.339941in}}%
\pgfpathlineto{\pgfqpoint{2.730128in}{1.361818in}}%
\pgfpathlineto{\pgfqpoint{2.664465in}{1.381012in}}%
\pgfpathlineto{\pgfqpoint{2.598013in}{1.397253in}}%
\pgfpathlineto{\pgfqpoint{2.530877in}{1.410372in}}%
\pgfpathlineto{\pgfqpoint{2.463201in}{1.420344in}}%
\pgfpathlineto{\pgfqpoint{2.395147in}{1.427330in}}%
\pgfpathlineto{\pgfqpoint{2.326869in}{1.431701in}}%
\pgfpathlineto{\pgfqpoint{2.258487in}{1.434089in}}%
\pgfpathlineto{\pgfqpoint{2.190071in}{1.435398in}}%
\pgfpathlineto{\pgfqpoint{2.121660in}{1.436868in}}%
\pgfpathlineto{\pgfqpoint{2.053338in}{1.440313in}}%
\pgfpathlineto{\pgfqpoint{1.985515in}{1.448701in}}%
\pgfpathlineto{\pgfqpoint{1.920244in}{1.467661in}}%
\pgfpathlineto{\pgfqpoint{1.920244in}{1.467661in}}%
\pgfpathlineto{\pgfqpoint{1.882822in}{1.489479in}}%
\pgfpathlineto{\pgfqpoint{1.882822in}{1.489479in}}%
\pgfpathlineto{\pgfqpoint{1.858036in}{1.516566in}}%
\pgfpathlineto{\pgfqpoint{1.843752in}{1.550913in}}%
\pgfpathlineto{\pgfqpoint{1.840379in}{1.585812in}}%
\pgfpathlineto{\pgfqpoint{1.845462in}{1.624502in}}%
\pgfusepath{stroke}%
\end{pgfscope}%
\begin{pgfscope}%
\pgfpathrectangle{\pgfqpoint{0.647939in}{0.492442in}}{\pgfqpoint{3.079299in}{3.079299in}}%
\pgfusepath{clip}%
\pgfsetbuttcap%
\pgfsetroundjoin%
\pgfsetlinewidth{0.301125pt}%
\definecolor{currentstroke}{rgb}{0.500000,0.500000,0.500000}%
\pgfsetstrokecolor{currentstroke}%
\pgfsetstrokeopacity{0.300000}%
\pgfsetdash{}{0pt}%
\pgfpathmoveto{\pgfqpoint{3.727238in}{0.982331in}}%
\pgfpathlineto{\pgfqpoint{3.727238in}{0.982331in}}%
\pgfpathlineto{\pgfqpoint{3.663600in}{1.007463in}}%
\pgfpathlineto{\pgfqpoint{3.600690in}{1.034369in}}%
\pgfpathlineto{\pgfqpoint{3.538460in}{1.062817in}}%
\pgfpathlineto{\pgfqpoint{3.476840in}{1.092566in}}%
\pgfpathlineto{\pgfqpoint{3.415740in}{1.123371in}}%
\pgfpathlineto{\pgfqpoint{3.355053in}{1.154984in}}%
\pgfpathlineto{\pgfqpoint{3.294659in}{1.187154in}}%
\pgfpathlineto{\pgfqpoint{3.234427in}{1.219627in}}%
\pgfpathlineto{\pgfqpoint{3.174218in}{1.252142in}}%
\pgfpathlineto{\pgfqpoint{3.113887in}{1.284430in}}%
\pgfpathlineto{\pgfqpoint{3.053287in}{1.316207in}}%
\pgfpathlineto{\pgfqpoint{2.992269in}{1.347169in}}%
\pgfpathlineto{\pgfqpoint{2.930687in}{1.376990in}}%
\pgfpathlineto{\pgfqpoint{2.868408in}{1.405318in}}%
\pgfpathlineto{\pgfqpoint{2.805317in}{1.431781in}}%
\pgfpathlineto{\pgfqpoint{2.741331in}{1.455989in}}%
\pgfpathlineto{\pgfqpoint{2.676412in}{1.477561in}}%
\pgfpathlineto{\pgfqpoint{2.610579in}{1.496148in}}%
\pgfpathlineto{\pgfqpoint{2.543913in}{1.511481in}}%
\pgfpathlineto{\pgfqpoint{2.476561in}{1.523436in}}%
\pgfpathlineto{\pgfqpoint{2.408703in}{1.532085in}}%
\pgfpathlineto{\pgfqpoint{2.340525in}{1.537729in}}%
\pgfpathlineto{\pgfqpoint{2.272181in}{1.540950in}}%
\pgfpathlineto{\pgfqpoint{2.203776in}{1.542660in}}%
\pgfpathlineto{\pgfqpoint{2.135367in}{1.544261in}}%
\pgfpathlineto{\pgfqpoint{2.067066in}{1.547962in}}%
\pgfpathlineto{\pgfqpoint{1.999523in}{1.557761in}}%
\pgfpathlineto{\pgfqpoint{1.999523in}{1.557761in}}%
\pgfpathlineto{\pgfqpoint{1.950117in}{1.574265in}}%
\pgfpathlineto{\pgfqpoint{1.950117in}{1.574265in}}%
\pgfpathlineto{\pgfqpoint{1.919443in}{1.594945in}}%
\pgfpathlineto{\pgfqpoint{1.919443in}{1.594945in}}%
\pgfpathlineto{\pgfqpoint{1.900880in}{1.620510in}}%
\pgfpathlineto{\pgfqpoint{1.892917in}{1.651175in}}%
\pgfpathlineto{\pgfqpoint{1.894341in}{1.683220in}}%
\pgfusepath{stroke}%
\end{pgfscope}%
\begin{pgfscope}%
\pgfpathrectangle{\pgfqpoint{0.647939in}{0.492442in}}{\pgfqpoint{3.079299in}{3.079299in}}%
\pgfusepath{clip}%
\pgfsetbuttcap%
\pgfsetroundjoin%
\pgfsetlinewidth{0.301125pt}%
\definecolor{currentstroke}{rgb}{0.500000,0.500000,0.500000}%
\pgfsetstrokecolor{currentstroke}%
\pgfsetstrokeopacity{0.300000}%
\pgfsetdash{}{0pt}%
\pgfpathmoveto{\pgfqpoint{3.727238in}{1.052315in}}%
\pgfpathlineto{\pgfqpoint{3.727238in}{1.052315in}}%
\pgfpathlineto{\pgfqpoint{3.663977in}{1.078380in}}%
\pgfpathlineto{\pgfqpoint{3.601507in}{1.106290in}}%
\pgfpathlineto{\pgfqpoint{3.539780in}{1.135812in}}%
\pgfpathlineto{\pgfqpoint{3.478727in}{1.166708in}}%
\pgfpathlineto{\pgfqpoint{3.418259in}{1.198735in}}%
\pgfpathlineto{\pgfqpoint{3.358269in}{1.231651in}}%
\pgfpathlineto{\pgfqpoint{3.298638in}{1.265214in}}%
\pgfpathlineto{\pgfqpoint{3.239237in}{1.299184in}}%
\pgfpathlineto{\pgfqpoint{3.179928in}{1.333314in}}%
\pgfpathlineto{\pgfqpoint{3.120566in}{1.367351in}}%
\pgfpathlineto{\pgfqpoint{3.061000in}{1.401027in}}%
\pgfpathlineto{\pgfqpoint{3.001074in}{1.434057in}}%
\pgfpathlineto{\pgfqpoint{2.940634in}{1.466130in}}%
\pgfpathlineto{\pgfqpoint{2.879526in}{1.496907in}}%
\pgfpathlineto{\pgfqpoint{2.817611in}{1.526017in}}%
\pgfpathlineto{\pgfqpoint{2.754768in}{1.553059in}}%
\pgfpathlineto{\pgfqpoint{2.690917in}{1.577611in}}%
\pgfpathlineto{\pgfqpoint{2.626028in}{1.599257in}}%
\pgfpathlineto{\pgfqpoint{2.560140in}{1.617619in}}%
\pgfpathlineto{\pgfqpoint{2.493364in}{1.632421in}}%
\pgfpathlineto{\pgfqpoint{2.425877in}{1.643559in}}%
\pgfpathlineto{\pgfqpoint{2.357896in}{1.651179in}}%
\pgfpathlineto{\pgfqpoint{2.289636in}{1.655787in}}%
\pgfpathlineto{\pgfqpoint{2.221259in}{1.658315in}}%
\pgfpathlineto{\pgfqpoint{2.152859in}{1.660289in}}%
\pgfpathlineto{\pgfqpoint{2.084590in}{1.664431in}}%
\pgfpathlineto{\pgfqpoint{2.017620in}{1.676882in}}%
\pgfpathlineto{\pgfqpoint{2.017620in}{1.676882in}}%
\pgfpathlineto{\pgfqpoint{1.984614in}{1.691434in}}%
\pgfpathlineto{\pgfqpoint{1.984614in}{1.691434in}}%
\pgfpathlineto{\pgfqpoint{1.963813in}{1.711059in}}%
\pgfpathlineto{\pgfqpoint{1.954150in}{1.738177in}}%
\pgfpathlineto{\pgfqpoint{1.955256in}{1.764628in}}%
\pgfpathlineto{\pgfqpoint{1.963948in}{1.792550in}}%
\pgfpathlineto{\pgfqpoint{1.981313in}{1.824860in}}%
\pgfusepath{stroke}%
\end{pgfscope}%
\begin{pgfscope}%
\pgfpathrectangle{\pgfqpoint{0.647939in}{0.492442in}}{\pgfqpoint{3.079299in}{3.079299in}}%
\pgfusepath{clip}%
\pgfsetbuttcap%
\pgfsetroundjoin%
\pgfsetlinewidth{0.301125pt}%
\definecolor{currentstroke}{rgb}{0.500000,0.500000,0.500000}%
\pgfsetstrokecolor{currentstroke}%
\pgfsetstrokeopacity{0.300000}%
\pgfsetdash{}{0pt}%
\pgfpathmoveto{\pgfqpoint{3.727238in}{1.122299in}}%
\pgfpathlineto{\pgfqpoint{3.727238in}{1.122299in}}%
\pgfpathlineto{\pgfqpoint{3.664399in}{1.149363in}}%
\pgfpathlineto{\pgfqpoint{3.602420in}{1.178348in}}%
\pgfpathlineto{\pgfqpoint{3.541257in}{1.209020in}}%
\pgfpathlineto{\pgfqpoint{3.480841in}{1.241143in}}%
\pgfpathlineto{\pgfqpoint{3.421085in}{1.274479in}}%
\pgfpathlineto{\pgfqpoint{3.361887in}{1.308797in}}%
\pgfpathlineto{\pgfqpoint{3.303130in}{1.343867in}}%
\pgfpathlineto{\pgfqpoint{3.244690in}{1.379463in}}%
\pgfpathlineto{\pgfqpoint{3.186434in}{1.415359in}}%
\pgfpathlineto{\pgfqpoint{3.128219in}{1.451322in}}%
\pgfpathlineto{\pgfqpoint{3.069897in}{1.487112in}}%
\pgfpathlineto{\pgfqpoint{3.011314in}{1.522472in}}%
\pgfpathlineto{\pgfqpoint{2.952308in}{1.557118in}}%
\pgfpathlineto{\pgfqpoint{2.892717in}{1.590743in}}%
\pgfpathlineto{\pgfqpoint{2.832381in}{1.623003in}}%
\pgfpathlineto{\pgfqpoint{2.771148in}{1.653518in}}%
\pgfpathlineto{\pgfqpoint{2.708885in}{1.681861in}}%
\pgfpathlineto{\pgfqpoint{2.645496in}{1.707573in}}%
\pgfpathlineto{\pgfqpoint{2.580937in}{1.730177in}}%
\pgfpathlineto{\pgfqpoint{2.515245in}{1.749223in}}%
\pgfpathlineto{\pgfqpoint{2.448548in}{1.764376in}}%
\pgfpathlineto{\pgfqpoint{2.381067in}{1.775523in}}%
\pgfpathlineto{\pgfqpoint{2.313064in}{1.782892in}}%
\pgfpathlineto{\pgfqpoint{2.244786in}{1.787220in}}%
\pgfpathlineto{\pgfqpoint{2.176420in}{1.790120in}}%
\pgfpathlineto{\pgfqpoint{2.108249in}{1.795375in}}%
\pgfpathlineto{\pgfqpoint{2.108249in}{1.795375in}}%
\pgfpathlineto{\pgfqpoint{2.064137in}{1.805081in}}%
\pgfpathlineto{\pgfqpoint{2.064137in}{1.805081in}}%
\pgfpathlineto{\pgfqpoint{2.041927in}{1.817075in}}%
\pgfpathlineto{\pgfqpoint{2.041927in}{1.817075in}}%
\pgfpathlineto{\pgfqpoint{2.030711in}{1.833314in}}%
\pgfpathlineto{\pgfqpoint{2.029713in}{1.852657in}}%
\pgfusepath{stroke}%
\end{pgfscope}%
\begin{pgfscope}%
\pgfpathrectangle{\pgfqpoint{0.647939in}{0.492442in}}{\pgfqpoint{3.079299in}{3.079299in}}%
\pgfusepath{clip}%
\pgfsetbuttcap%
\pgfsetroundjoin%
\pgfsetlinewidth{0.301125pt}%
\definecolor{currentstroke}{rgb}{0.500000,0.500000,0.500000}%
\pgfsetstrokecolor{currentstroke}%
\pgfsetstrokeopacity{0.300000}%
\pgfsetdash{}{0pt}%
\pgfpathmoveto{\pgfqpoint{3.727238in}{1.192283in}}%
\pgfpathlineto{\pgfqpoint{3.727238in}{1.192283in}}%
\pgfpathlineto{\pgfqpoint{3.664872in}{1.220418in}}%
\pgfpathlineto{\pgfqpoint{3.603446in}{1.250554in}}%
\pgfpathlineto{\pgfqpoint{3.542917in}{1.282457in}}%
\pgfpathlineto{\pgfqpoint{3.483221in}{1.315895in}}%
\pgfpathlineto{\pgfqpoint{3.424274in}{1.350639in}}%
\pgfpathlineto{\pgfqpoint{3.365979in}{1.386468in}}%
\pgfpathlineto{\pgfqpoint{3.308228in}{1.423170in}}%
\pgfpathlineto{\pgfqpoint{3.250904in}{1.460537in}}%
\pgfpathlineto{\pgfqpoint{3.193884in}{1.498366in}}%
\pgfpathlineto{\pgfqpoint{3.137037in}{1.536454in}}%
\pgfpathlineto{\pgfqpoint{3.080223in}{1.574591in}}%
\pgfpathlineto{\pgfqpoint{3.023295in}{1.612558in}}%
\pgfpathlineto{\pgfqpoint{2.966107in}{1.650129in}}%
\pgfpathlineto{\pgfqpoint{2.908506in}{1.687057in}}%
\pgfpathlineto{\pgfqpoint{2.850328in}{1.723068in}}%
\pgfpathlineto{\pgfqpoint{2.791410in}{1.757852in}}%
\pgfpathlineto{\pgfqpoint{2.731588in}{1.791053in}}%
\pgfpathlineto{\pgfqpoint{2.670709in}{1.822268in}}%
\pgfpathlineto{\pgfqpoint{2.608643in}{1.851036in}}%
\pgfpathlineto{\pgfqpoint{2.545298in}{1.876849in}}%
\pgfpathlineto{\pgfqpoint{2.480649in}{1.899180in}}%
\pgfpathlineto{\pgfqpoint{2.414775in}{1.917555in}}%
\pgfpathlineto{\pgfqpoint{2.347864in}{1.931673in}}%
\pgfpathlineto{\pgfqpoint{2.280207in}{1.941677in}}%
\pgfpathlineto{\pgfqpoint{2.212183in}{1.949025in}}%
\pgfpathlineto{\pgfqpoint{2.212183in}{1.949025in}}%
\pgfpathlineto{\pgfqpoint{2.161792in}{1.957227in}}%
\pgfpathlineto{\pgfqpoint{2.161792in}{1.957227in}}%
\pgfusepath{stroke}%
\end{pgfscope}%
\begin{pgfscope}%
\pgfpathrectangle{\pgfqpoint{0.647939in}{0.492442in}}{\pgfqpoint{3.079299in}{3.079299in}}%
\pgfusepath{clip}%
\pgfsetbuttcap%
\pgfsetroundjoin%
\pgfsetlinewidth{0.301125pt}%
\definecolor{currentstroke}{rgb}{0.500000,0.500000,0.500000}%
\pgfsetstrokecolor{currentstroke}%
\pgfsetstrokeopacity{0.300000}%
\pgfsetdash{}{0pt}%
\pgfpathmoveto{\pgfqpoint{3.727238in}{1.332251in}}%
\pgfpathlineto{\pgfqpoint{3.727238in}{1.332251in}}%
\pgfpathlineto{\pgfqpoint{3.666007in}{1.362772in}}%
\pgfpathlineto{\pgfqpoint{3.605906in}{1.395469in}}%
\pgfpathlineto{\pgfqpoint{3.546906in}{1.430113in}}%
\pgfpathlineto{\pgfqpoint{3.488953in}{1.466485in}}%
\pgfpathlineto{\pgfqpoint{3.431982in}{1.504379in}}%
\pgfpathlineto{\pgfqpoint{3.375918in}{1.543604in}}%
\pgfpathlineto{\pgfqpoint{3.320682in}{1.583988in}}%
\pgfpathlineto{\pgfqpoint{3.266188in}{1.625370in}}%
\pgfpathlineto{\pgfqpoint{3.212353in}{1.667605in}}%
\pgfpathlineto{\pgfqpoint{3.159099in}{1.710571in}}%
\pgfpathlineto{\pgfqpoint{3.106361in}{1.754168in}}%
\pgfpathlineto{\pgfqpoint{3.054075in}{1.798308in}}%
\pgfpathlineto{\pgfqpoint{3.002200in}{1.842927in}}%
\pgfpathlineto{\pgfqpoint{2.950719in}{1.888000in}}%
\pgfpathlineto{\pgfqpoint{2.899653in}{1.933540in}}%
\pgfpathlineto{\pgfqpoint{2.849083in}{1.979625in}}%
\pgfpathlineto{\pgfqpoint{2.799193in}{2.026442in}}%
\pgfpathlineto{\pgfqpoint{2.750357in}{2.074348in}}%
\pgfpathlineto{\pgfqpoint{2.703314in}{2.123992in}}%
\pgfpathlineto{\pgfqpoint{2.659610in}{2.176545in}}%
\pgfpathlineto{\pgfqpoint{2.622652in}{2.233865in}}%
\pgfpathlineto{\pgfqpoint{2.622652in}{2.233865in}}%
\pgfpathlineto{\pgfqpoint{2.601703in}{2.286835in}}%
\pgfpathlineto{\pgfqpoint{2.595338in}{2.344210in}}%
\pgfpathlineto{\pgfqpoint{2.602261in}{2.393861in}}%
\pgfusepath{stroke}%
\end{pgfscope}%
\begin{pgfscope}%
\pgfpathrectangle{\pgfqpoint{0.647939in}{0.492442in}}{\pgfqpoint{3.079299in}{3.079299in}}%
\pgfusepath{clip}%
\pgfsetbuttcap%
\pgfsetroundjoin%
\pgfsetlinewidth{0.301125pt}%
\definecolor{currentstroke}{rgb}{0.500000,0.500000,0.500000}%
\pgfsetstrokecolor{currentstroke}%
\pgfsetstrokeopacity{0.300000}%
\pgfsetdash{}{0pt}%
\pgfpathmoveto{\pgfqpoint{3.727238in}{1.472219in}}%
\pgfpathlineto{\pgfqpoint{3.727238in}{1.472219in}}%
\pgfpathlineto{\pgfqpoint{3.667469in}{1.505505in}}%
\pgfpathlineto{\pgfqpoint{3.609083in}{1.541164in}}%
\pgfpathlineto{\pgfqpoint{3.552067in}{1.578979in}}%
\pgfpathlineto{\pgfqpoint{3.496395in}{1.618749in}}%
\pgfpathlineto{\pgfqpoint{3.442036in}{1.660298in}}%
\pgfpathlineto{\pgfqpoint{3.388959in}{1.703473in}}%
\pgfpathlineto{\pgfqpoint{3.337132in}{1.748143in}}%
\pgfpathlineto{\pgfqpoint{3.286547in}{1.794215in}}%
\pgfpathlineto{\pgfqpoint{3.237219in}{1.841629in}}%
\pgfpathlineto{\pgfqpoint{3.189190in}{1.890358in}}%
\pgfpathlineto{\pgfqpoint{3.142559in}{1.940424in}}%
\pgfpathlineto{\pgfqpoint{3.097486in}{1.991894in}}%
\pgfpathlineto{\pgfqpoint{3.054228in}{2.044896in}}%
\pgfpathlineto{\pgfqpoint{3.013188in}{2.099624in}}%
\pgfpathlineto{\pgfqpoint{2.974967in}{2.156340in}}%
\pgfpathlineto{\pgfqpoint{2.940453in}{2.215361in}}%
\pgfpathlineto{\pgfqpoint{2.910910in}{2.276986in}}%
\pgfpathlineto{\pgfqpoint{2.887992in}{2.341317in}}%
\pgfpathlineto{\pgfqpoint{2.873492in}{2.407990in}}%
\pgfpathlineto{\pgfqpoint{2.868699in}{2.476023in}}%
\pgfpathlineto{\pgfqpoint{2.873702in}{2.544066in}}%
\pgfpathlineto{\pgfqpoint{2.887354in}{2.610966in}}%
\pgfpathlineto{\pgfqpoint{2.907923in}{2.676119in}}%
\pgfpathlineto{\pgfqpoint{2.933750in}{2.739402in}}%
\pgfpathlineto{\pgfqpoint{2.963522in}{2.800954in}}%
\pgfpathlineto{\pgfqpoint{2.996292in}{2.860983in}}%
\pgfpathlineto{\pgfqpoint{3.031408in}{2.919685in}}%
\pgfpathlineto{\pgfqpoint{3.068422in}{2.977219in}}%
\pgfpathlineto{\pgfqpoint{3.107032in}{3.033697in}}%
\pgfpathlineto{\pgfqpoint{3.147049in}{3.089187in}}%
\pgfusepath{stroke}%
\end{pgfscope}%
\begin{pgfscope}%
\pgfpathrectangle{\pgfqpoint{0.647939in}{0.492442in}}{\pgfqpoint{3.079299in}{3.079299in}}%
\pgfusepath{clip}%
\pgfsetbuttcap%
\pgfsetroundjoin%
\pgfsetlinewidth{0.301125pt}%
\definecolor{currentstroke}{rgb}{0.500000,0.500000,0.500000}%
\pgfsetstrokecolor{currentstroke}%
\pgfsetstrokeopacity{0.300000}%
\pgfsetdash{}{0pt}%
\pgfpathmoveto{\pgfqpoint{3.727238in}{1.542203in}}%
\pgfpathlineto{\pgfqpoint{3.727238in}{1.542203in}}%
\pgfpathlineto{\pgfqpoint{3.668360in}{1.577038in}}%
\pgfpathlineto{\pgfqpoint{3.611019in}{1.614352in}}%
\pgfpathlineto{\pgfqpoint{3.555219in}{1.653936in}}%
\pgfpathlineto{\pgfqpoint{3.500954in}{1.695600in}}%
\pgfpathlineto{\pgfqpoint{3.448218in}{1.739186in}}%
\pgfpathlineto{\pgfqpoint{3.397006in}{1.784556in}}%
\pgfpathlineto{\pgfqpoint{3.347335in}{1.831606in}}%
\pgfpathlineto{\pgfqpoint{3.299252in}{1.880279in}}%
\pgfpathlineto{\pgfqpoint{3.252843in}{1.930547in}}%
\pgfpathlineto{\pgfqpoint{3.208248in}{1.982431in}}%
\pgfpathlineto{\pgfqpoint{3.165687in}{2.035991in}}%
\pgfpathlineto{\pgfqpoint{3.125485in}{2.091335in}}%
\pgfpathlineto{\pgfqpoint{3.088099in}{2.148611in}}%
\pgfpathlineto{\pgfqpoint{3.054179in}{2.207994in}}%
\pgfpathlineto{\pgfqpoint{3.024593in}{2.269634in}}%
\pgfpathlineto{\pgfqpoint{3.000430in}{2.333567in}}%
\pgfpathlineto{\pgfqpoint{2.982895in}{2.399591in}}%
\pgfpathlineto{\pgfqpoint{2.973050in}{2.467159in}}%
\pgfpathlineto{\pgfqpoint{2.971443in}{2.535410in}}%
\pgfpathlineto{\pgfqpoint{2.977876in}{2.603398in}}%
\pgfpathlineto{\pgfqpoint{2.991500in}{2.670352in}}%
\pgfpathlineto{\pgfqpoint{3.011158in}{2.735818in}}%
\pgfpathlineto{\pgfqpoint{3.035691in}{2.799638in}}%
\pgfpathlineto{\pgfqpoint{3.064125in}{2.861831in}}%
\pgfpathlineto{\pgfqpoint{3.095700in}{2.922503in}}%
\pgfusepath{stroke}%
\end{pgfscope}%
\begin{pgfscope}%
\pgfpathrectangle{\pgfqpoint{0.647939in}{0.492442in}}{\pgfqpoint{3.079299in}{3.079299in}}%
\pgfusepath{clip}%
\pgfsetbuttcap%
\pgfsetroundjoin%
\pgfsetlinewidth{0.301125pt}%
\definecolor{currentstroke}{rgb}{0.500000,0.500000,0.500000}%
\pgfsetstrokecolor{currentstroke}%
\pgfsetstrokeopacity{0.300000}%
\pgfsetdash{}{0pt}%
\pgfpathmoveto{\pgfqpoint{3.727238in}{1.682171in}}%
\pgfpathlineto{\pgfqpoint{3.727238in}{1.682171in}}%
\pgfpathlineto{\pgfqpoint{3.670567in}{1.720482in}}%
\pgfpathlineto{\pgfqpoint{3.615821in}{1.761498in}}%
\pgfpathlineto{\pgfqpoint{3.563047in}{1.805025in}}%
\pgfpathlineto{\pgfqpoint{3.512298in}{1.850899in}}%
\pgfpathlineto{\pgfqpoint{3.463634in}{1.898978in}}%
\pgfpathlineto{\pgfqpoint{3.417150in}{1.949169in}}%
\pgfpathlineto{\pgfqpoint{3.372983in}{2.001408in}}%
\pgfpathlineto{\pgfqpoint{3.331329in}{2.055667in}}%
\pgfpathlineto{\pgfqpoint{3.292464in}{2.111957in}}%
\pgfpathlineto{\pgfqpoint{3.256756in}{2.170296in}}%
\pgfpathlineto{\pgfqpoint{3.224683in}{2.230699in}}%
\pgfpathlineto{\pgfqpoint{3.196836in}{2.293153in}}%
\pgfpathlineto{\pgfqpoint{3.173901in}{2.357560in}}%
\pgfpathlineto{\pgfqpoint{3.156589in}{2.423689in}}%
\pgfpathlineto{\pgfqpoint{3.145530in}{2.491128in}}%
\pgfpathlineto{\pgfqpoint{3.141119in}{2.559314in}}%
\pgfpathlineto{\pgfqpoint{3.143412in}{2.627604in}}%
\pgfpathlineto{\pgfqpoint{3.152107in}{2.695390in}}%
\pgfpathlineto{\pgfqpoint{3.166631in}{2.762191in}}%
\pgfpathlineto{\pgfqpoint{3.186282in}{2.827682in}}%
\pgfpathlineto{\pgfqpoint{3.210359in}{2.891688in}}%
\pgfpathlineto{\pgfqpoint{3.238226in}{2.954146in}}%
\pgfpathlineto{\pgfqpoint{3.269362in}{3.015050in}}%
\pgfpathlineto{\pgfqpoint{3.303356in}{3.074413in}}%
\pgfpathlineto{\pgfqpoint{3.339895in}{3.132248in}}%
\pgfpathlineto{\pgfqpoint{3.378747in}{3.188555in}}%
\pgfpathlineto{\pgfqpoint{3.419750in}{3.243316in}}%
\pgfpathlineto{\pgfqpoint{3.462812in}{3.296475in}}%
\pgfpathlineto{\pgfqpoint{3.507872in}{3.347954in}}%
\pgfpathlineto{\pgfqpoint{3.554902in}{3.397635in}}%
\pgfpathlineto{\pgfqpoint{3.603907in}{3.445371in}}%
\pgfpathlineto{\pgfqpoint{3.654901in}{3.490972in}}%
\pgfpathlineto{\pgfqpoint{3.707911in}{3.534209in}}%
\pgfpathlineto{\pgfqpoint{3.727238in}{3.549238in}}%
\pgfusepath{stroke}%
\end{pgfscope}%
\begin{pgfscope}%
\pgfpathrectangle{\pgfqpoint{0.647939in}{0.492442in}}{\pgfqpoint{3.079299in}{3.079299in}}%
\pgfusepath{clip}%
\pgfsetbuttcap%
\pgfsetroundjoin%
\pgfsetlinewidth{0.301125pt}%
\definecolor{currentstroke}{rgb}{0.500000,0.500000,0.500000}%
\pgfsetstrokecolor{currentstroke}%
\pgfsetstrokeopacity{0.300000}%
\pgfsetdash{}{0pt}%
\pgfpathmoveto{\pgfqpoint{3.727238in}{1.822139in}}%
\pgfpathlineto{\pgfqpoint{3.727238in}{1.822139in}}%
\pgfpathlineto{\pgfqpoint{3.673532in}{1.864496in}}%
\pgfpathlineto{\pgfqpoint{3.622275in}{1.909785in}}%
\pgfpathlineto{\pgfqpoint{3.573583in}{1.957824in}}%
\pgfpathlineto{\pgfqpoint{3.527587in}{2.008450in}}%
\pgfpathlineto{\pgfqpoint{3.484457in}{2.061535in}}%
\pgfpathlineto{\pgfqpoint{3.444418in}{2.116986in}}%
\pgfpathlineto{\pgfqpoint{3.407755in}{2.174723in}}%
\pgfpathlineto{\pgfqpoint{3.374830in}{2.234667in}}%
\pgfpathlineto{\pgfqpoint{3.346083in}{2.296717in}}%
\pgfpathlineto{\pgfqpoint{3.322014in}{2.360719in}}%
\pgfpathlineto{\pgfqpoint{3.303140in}{2.426428in}}%
\pgfpathlineto{\pgfqpoint{3.289931in}{2.493493in}}%
\pgfpathlineto{\pgfqpoint{3.282734in}{2.561458in}}%
\pgfpathlineto{\pgfqpoint{3.281694in}{2.629795in}}%
\pgfpathlineto{\pgfqpoint{3.286721in}{2.697963in}}%
\pgfpathlineto{\pgfqpoint{3.297507in}{2.765470in}}%
\pgfpathlineto{\pgfqpoint{3.313603in}{2.831919in}}%
\pgfpathlineto{\pgfqpoint{3.334486in}{2.897028in}}%
\pgfpathlineto{\pgfqpoint{3.359640in}{2.960617in}}%
\pgfpathlineto{\pgfqpoint{3.388598in}{3.022574in}}%
\pgfpathlineto{\pgfqpoint{3.420972in}{3.082823in}}%
\pgfpathlineto{\pgfqpoint{3.456450in}{3.141303in}}%
\pgfpathlineto{\pgfqpoint{3.494796in}{3.197950in}}%
\pgfpathlineto{\pgfqpoint{3.535828in}{3.252685in}}%
\pgfpathlineto{\pgfqpoint{3.579422in}{3.305398in}}%
\pgfpathlineto{\pgfqpoint{3.625508in}{3.355947in}}%
\pgfpathlineto{\pgfqpoint{3.674039in}{3.404154in}}%
\pgfpathlineto{\pgfqpoint{3.724983in}{3.449798in}}%
\pgfpathlineto{\pgfqpoint{3.727238in}{3.451715in}}%
\pgfusepath{stroke}%
\end{pgfscope}%
\begin{pgfscope}%
\pgfpathrectangle{\pgfqpoint{0.647939in}{0.492442in}}{\pgfqpoint{3.079299in}{3.079299in}}%
\pgfusepath{clip}%
\pgfsetbuttcap%
\pgfsetroundjoin%
\pgfsetlinewidth{0.301125pt}%
\definecolor{currentstroke}{rgb}{0.500000,0.500000,0.500000}%
\pgfsetstrokecolor{currentstroke}%
\pgfsetstrokeopacity{0.300000}%
\pgfsetdash{}{0pt}%
\pgfpathmoveto{\pgfqpoint{3.727238in}{1.892124in}}%
\pgfpathlineto{\pgfqpoint{3.727238in}{1.892124in}}%
\pgfpathlineto{\pgfqpoint{3.675394in}{1.936734in}}%
\pgfpathlineto{\pgfqpoint{3.626326in}{1.984379in}}%
\pgfpathlineto{\pgfqpoint{3.580191in}{2.034871in}}%
\pgfpathlineto{\pgfqpoint{3.537176in}{2.088042in}}%
\pgfpathlineto{\pgfqpoint{3.497509in}{2.143756in}}%
\pgfpathlineto{\pgfqpoint{3.461479in}{2.201886in}}%
\pgfpathlineto{\pgfqpoint{3.429445in}{2.262304in}}%
\pgfusepath{stroke}%
\end{pgfscope}%
\begin{pgfscope}%
\pgfpathrectangle{\pgfqpoint{0.647939in}{0.492442in}}{\pgfqpoint{3.079299in}{3.079299in}}%
\pgfusepath{clip}%
\pgfsetbuttcap%
\pgfsetroundjoin%
\pgfsetlinewidth{0.301125pt}%
\definecolor{currentstroke}{rgb}{0.500000,0.500000,0.500000}%
\pgfsetstrokecolor{currentstroke}%
\pgfsetstrokeopacity{0.300000}%
\pgfsetdash{}{0pt}%
\pgfpathmoveto{\pgfqpoint{3.727238in}{2.032092in}}%
\pgfpathlineto{\pgfqpoint{3.727238in}{2.032092in}}%
\pgfpathlineto{\pgfqpoint{3.680134in}{2.081664in}}%
\pgfpathlineto{\pgfqpoint{3.636614in}{2.134410in}}%
\pgfpathlineto{\pgfqpoint{3.596936in}{2.190099in}}%
\pgfpathlineto{\pgfqpoint{3.561391in}{2.248508in}}%
\pgfpathlineto{\pgfqpoint{3.530321in}{2.309410in}}%
\pgfpathlineto{\pgfqpoint{3.504105in}{2.372548in}}%
\pgfpathlineto{\pgfqpoint{3.483135in}{2.437610in}}%
\pgfpathlineto{\pgfqpoint{3.467770in}{2.504213in}}%
\pgfpathlineto{\pgfqpoint{3.458292in}{2.571904in}}%
\pgfpathlineto{\pgfqpoint{3.454855in}{2.640173in}}%
\pgfpathlineto{\pgfqpoint{3.457453in}{2.708479in}}%
\pgfpathlineto{\pgfqpoint{3.465919in}{2.776307in}}%
\pgfpathlineto{\pgfqpoint{3.479951in}{2.843211in}}%
\pgfpathlineto{\pgfqpoint{3.499178in}{2.908819in}}%
\pgfpathlineto{\pgfqpoint{3.523204in}{2.972833in}}%
\pgfpathlineto{\pgfqpoint{3.551643in}{3.035020in}}%
\pgfpathlineto{\pgfqpoint{3.584152in}{3.095183in}}%
\pgfpathlineto{\pgfqpoint{3.620450in}{3.153140in}}%
\pgfpathlineto{\pgfqpoint{3.660308in}{3.208712in}}%
\pgfpathlineto{\pgfqpoint{3.703553in}{3.261695in}}%
\pgfpathlineto{\pgfqpoint{3.727238in}{3.288941in}}%
\pgfusepath{stroke}%
\end{pgfscope}%
\begin{pgfscope}%
\pgfpathrectangle{\pgfqpoint{0.647939in}{0.492442in}}{\pgfqpoint{3.079299in}{3.079299in}}%
\pgfusepath{clip}%
\pgfsetbuttcap%
\pgfsetroundjoin%
\pgfsetlinewidth{0.301125pt}%
\definecolor{currentstroke}{rgb}{0.500000,0.500000,0.500000}%
\pgfsetstrokecolor{currentstroke}%
\pgfsetstrokeopacity{0.300000}%
\pgfsetdash{}{0pt}%
\pgfpathmoveto{\pgfqpoint{3.727238in}{2.172060in}}%
\pgfpathlineto{\pgfqpoint{3.727238in}{2.172060in}}%
\pgfpathlineto{\pgfqpoint{3.686649in}{2.227077in}}%
\pgfpathlineto{\pgfqpoint{3.650664in}{2.285200in}}%
\pgfpathlineto{\pgfqpoint{3.619622in}{2.346103in}}%
\pgfpathlineto{\pgfqpoint{3.593883in}{2.409424in}}%
\pgfpathlineto{\pgfqpoint{3.573799in}{2.474756in}}%
\pgfpathlineto{\pgfqpoint{3.559673in}{2.541629in}}%
\pgfpathlineto{\pgfqpoint{3.551723in}{2.609516in}}%
\pgfpathlineto{\pgfqpoint{3.550041in}{2.677844in}}%
\pgfpathlineto{\pgfqpoint{3.554568in}{2.746039in}}%
\pgfpathlineto{\pgfqpoint{3.565113in}{2.813570in}}%
\pgfpathlineto{\pgfqpoint{3.581378in}{2.879963in}}%
\pgfpathlineto{\pgfqpoint{3.603014in}{2.944815in}}%
\pgfpathlineto{\pgfqpoint{3.629657in}{3.007782in}}%
\pgfpathlineto{\pgfqpoint{3.660971in}{3.068566in}}%
\pgfpathlineto{\pgfqpoint{3.696651in}{3.126895in}}%
\pgfpathlineto{\pgfqpoint{3.727238in}{3.173084in}}%
\pgfusepath{stroke}%
\end{pgfscope}%
\begin{pgfscope}%
\pgfpathrectangle{\pgfqpoint{0.647939in}{0.492442in}}{\pgfqpoint{3.079299in}{3.079299in}}%
\pgfusepath{clip}%
\pgfsetbuttcap%
\pgfsetroundjoin%
\pgfsetlinewidth{0.301125pt}%
\definecolor{currentstroke}{rgb}{0.500000,0.500000,0.500000}%
\pgfsetstrokecolor{currentstroke}%
\pgfsetstrokeopacity{0.300000}%
\pgfsetdash{}{0pt}%
\pgfpathmoveto{\pgfqpoint{3.727238in}{2.312028in}}%
\pgfpathlineto{\pgfqpoint{3.727238in}{2.312028in}}%
\pgfpathlineto{\pgfqpoint{3.695483in}{2.372548in}}%
\pgfpathlineto{\pgfqpoint{3.669475in}{2.435746in}}%
\pgfpathlineto{\pgfqpoint{3.649546in}{2.501112in}}%
\pgfpathlineto{\pgfqpoint{3.635977in}{2.568085in}}%
\pgfpathlineto{\pgfqpoint{3.628950in}{2.636061in}}%
\pgfpathlineto{\pgfqpoint{3.628517in}{2.704403in}}%
\pgfpathlineto{\pgfqpoint{3.634592in}{2.772474in}}%
\pgfpathlineto{\pgfqpoint{3.646965in}{2.839686in}}%
\pgfpathlineto{\pgfqpoint{3.665337in}{2.905516in}}%
\pgfpathlineto{\pgfqpoint{3.689367in}{2.969505in}}%
\pgfpathlineto{\pgfqpoint{3.718713in}{3.031242in}}%
\pgfpathlineto{\pgfqpoint{3.727238in}{3.047418in}}%
\pgfusepath{stroke}%
\end{pgfscope}%
\begin{pgfscope}%
\pgfpathrectangle{\pgfqpoint{0.647939in}{0.492442in}}{\pgfqpoint{3.079299in}{3.079299in}}%
\pgfusepath{clip}%
\pgfsetbuttcap%
\pgfsetroundjoin%
\pgfsetlinewidth{0.301125pt}%
\definecolor{currentstroke}{rgb}{0.500000,0.500000,0.500000}%
\pgfsetstrokecolor{currentstroke}%
\pgfsetstrokeopacity{0.300000}%
\pgfsetdash{}{0pt}%
\pgfpathmoveto{\pgfqpoint{3.727238in}{2.497431in}}%
\pgfpathlineto{\pgfqpoint{3.718610in}{2.533905in}}%
\pgfpathlineto{\pgfqpoint{3.706381in}{2.601124in}}%
\pgfpathlineto{\pgfqpoint{3.701191in}{2.669257in}}%
\pgfpathlineto{\pgfqpoint{3.703045in}{2.737567in}}%
\pgfpathlineto{\pgfqpoint{3.711811in}{2.805333in}}%
\pgfpathlineto{\pgfqpoint{3.727238in}{2.871901in}}%
\pgfpathlineto{\pgfqpoint{3.727238in}{2.871901in}}%
\pgfusepath{stroke}%
\end{pgfscope}%
\begin{pgfscope}%
\pgfpathrectangle{\pgfqpoint{0.647939in}{0.492442in}}{\pgfqpoint{3.079299in}{3.079299in}}%
\pgfusepath{clip}%
\pgfsetbuttcap%
\pgfsetroundjoin%
\pgfsetlinewidth{0.301125pt}%
\definecolor{currentstroke}{rgb}{0.500000,0.500000,0.500000}%
\pgfsetstrokecolor{currentstroke}%
\pgfsetstrokeopacity{0.300000}%
\pgfsetdash{}{0pt}%
\pgfpathmoveto{\pgfqpoint{3.319271in}{3.233919in}}%
\pgfpathlineto{\pgfqpoint{3.362855in}{3.286660in}}%
\pgfpathlineto{\pgfqpoint{3.407967in}{3.338098in}}%
\pgfpathlineto{\pgfqpoint{3.454612in}{3.388152in}}%
\pgfpathlineto{\pgfqpoint{3.502815in}{3.436705in}}%
\pgfpathlineto{\pgfqpoint{3.552623in}{3.483609in}}%
\pgfpathlineto{\pgfqpoint{3.604083in}{3.528692in}}%
\pgfpathlineto{\pgfqpoint{3.657254in}{3.571741in}}%
\pgfpathlineto{\pgfqpoint{3.657254in}{3.571741in}}%
\pgfusepath{stroke}%
\end{pgfscope}%
\begin{pgfscope}%
\pgfpathrectangle{\pgfqpoint{0.647939in}{0.492442in}}{\pgfqpoint{3.079299in}{3.079299in}}%
\pgfusepath{clip}%
\pgfsetbuttcap%
\pgfsetroundjoin%
\pgfsetlinewidth{0.301125pt}%
\definecolor{currentstroke}{rgb}{0.500000,0.500000,0.500000}%
\pgfsetstrokecolor{currentstroke}%
\pgfsetstrokeopacity{0.300000}%
\pgfsetdash{}{0pt}%
\pgfpathmoveto{\pgfqpoint{0.647939in}{2.527882in}}%
\pgfpathlineto{\pgfqpoint{0.664834in}{2.529977in}}%
\pgfpathlineto{\pgfqpoint{0.732648in}{2.539100in}}%
\pgfpathlineto{\pgfqpoint{0.800255in}{2.549642in}}%
\pgfpathlineto{\pgfqpoint{0.867625in}{2.561611in}}%
\pgfpathlineto{\pgfqpoint{0.934733in}{2.574970in}}%
\pgfpathlineto{\pgfqpoint{1.001568in}{2.589638in}}%
\pgfpathlineto{\pgfqpoint{1.068132in}{2.605490in}}%
\pgfpathlineto{\pgfqpoint{1.134447in}{2.622360in}}%
\pgfpathlineto{\pgfqpoint{1.200552in}{2.640038in}}%
\pgfpathlineto{\pgfqpoint{1.266505in}{2.658275in}}%
\pgfpathlineto{\pgfqpoint{1.332381in}{2.676790in}}%
\pgfpathlineto{\pgfqpoint{1.398268in}{2.695269in}}%
\pgfpathlineto{\pgfqpoint{1.464257in}{2.713373in}}%
\pgfpathlineto{\pgfqpoint{1.530440in}{2.730755in}}%
\pgfpathlineto{\pgfqpoint{1.596894in}{2.747060in}}%
\pgfpathlineto{\pgfqpoint{1.663679in}{2.761944in}}%
\pgfpathlineto{\pgfqpoint{1.730825in}{2.775090in}}%
\pgfpathlineto{\pgfqpoint{1.798331in}{2.786234in}}%
\pgfpathlineto{\pgfqpoint{1.866160in}{2.795203in}}%
\pgfpathlineto{\pgfqpoint{1.934245in}{2.801947in}}%
\pgfpathlineto{\pgfqpoint{2.002509in}{2.806571in}}%
\pgfpathlineto{\pgfqpoint{2.070875in}{2.809364in}}%
\pgfpathlineto{\pgfqpoint{2.139286in}{2.810828in}}%
\pgfpathlineto{\pgfqpoint{2.207709in}{2.811715in}}%
\pgfpathlineto{\pgfqpoint{2.276122in}{2.813045in}}%
\pgfpathlineto{\pgfqpoint{2.344471in}{2.816084in}}%
\pgfpathlineto{\pgfqpoint{2.412583in}{2.822301in}}%
\pgfpathlineto{\pgfqpoint{2.480078in}{2.833203in}}%
\pgfpathlineto{\pgfqpoint{2.546331in}{2.850002in}}%
\pgfpathlineto{\pgfqpoint{2.610601in}{2.873227in}}%
\pgfpathlineto{\pgfqpoint{2.672334in}{2.902556in}}%
\pgfpathlineto{\pgfqpoint{2.731353in}{2.937067in}}%
\pgfpathlineto{\pgfqpoint{2.787828in}{2.975632in}}%
\pgfpathlineto{\pgfqpoint{2.842132in}{3.017209in}}%
\pgfpathlineto{\pgfqpoint{2.894708in}{3.060964in}}%
\pgfpathlineto{\pgfqpoint{2.945979in}{3.106259in}}%
\pgfpathlineto{\pgfqpoint{2.996313in}{3.152599in}}%
\pgfpathlineto{\pgfqpoint{3.046023in}{3.199612in}}%
\pgfpathlineto{\pgfqpoint{3.095379in}{3.247001in}}%
\pgfpathlineto{\pgfqpoint{3.144609in}{3.294520in}}%
\pgfpathlineto{\pgfqpoint{3.193913in}{3.341965in}}%
\pgfpathlineto{\pgfqpoint{3.243470in}{3.389146in}}%
\pgfpathlineto{\pgfqpoint{3.293440in}{3.435890in}}%
\pgfpathlineto{\pgfqpoint{3.343974in}{3.482023in}}%
\pgfpathlineto{\pgfqpoint{3.395219in}{3.527365in}}%
\pgfpathlineto{\pgfqpoint{3.447302in}{3.571741in}}%
\pgfpathlineto{\pgfqpoint{3.447302in}{3.571741in}}%
\pgfusepath{stroke}%
\end{pgfscope}%
\begin{pgfscope}%
\pgfpathrectangle{\pgfqpoint{0.647939in}{0.492442in}}{\pgfqpoint{3.079299in}{3.079299in}}%
\pgfusepath{clip}%
\pgfsetbuttcap%
\pgfsetroundjoin%
\pgfsetlinewidth{0.301125pt}%
\definecolor{currentstroke}{rgb}{0.500000,0.500000,0.500000}%
\pgfsetstrokecolor{currentstroke}%
\pgfsetstrokeopacity{0.300000}%
\pgfsetdash{}{0pt}%
\pgfpathmoveto{\pgfqpoint{0.647939in}{2.829217in}}%
\pgfpathlineto{\pgfqpoint{0.668553in}{2.831657in}}%
\pgfpathlineto{\pgfqpoint{0.736425in}{2.840346in}}%
\pgfpathlineto{\pgfqpoint{0.804118in}{2.850332in}}%
\pgfpathlineto{\pgfqpoint{0.871609in}{2.861602in}}%
\pgfpathlineto{\pgfqpoint{0.938883in}{2.874104in}}%
\pgfpathlineto{\pgfqpoint{1.005935in}{2.887746in}}%
\pgfpathlineto{\pgfqpoint{1.072776in}{2.902394in}}%
\pgfpathlineto{\pgfqpoint{1.139429in}{2.917876in}}%
\pgfpathlineto{\pgfqpoint{1.205935in}{2.933981in}}%
\pgfpathlineto{\pgfqpoint{1.272350in}{2.950460in}}%
\pgfpathlineto{\pgfqpoint{1.338740in}{2.967037in}}%
\pgfpathlineto{\pgfqpoint{1.405180in}{2.983416in}}%
\pgfpathlineto{\pgfqpoint{1.471742in}{2.999287in}}%
\pgfpathlineto{\pgfqpoint{1.538493in}{3.014338in}}%
\pgfpathlineto{\pgfqpoint{1.605487in}{3.028258in}}%
\pgfpathlineto{\pgfqpoint{1.672758in}{3.040763in}}%
\pgfpathlineto{\pgfqpoint{1.740316in}{3.051608in}}%
\pgfpathlineto{\pgfqpoint{1.808142in}{3.060614in}}%
\pgfpathlineto{\pgfqpoint{1.876195in}{3.067694in}}%
\pgfpathlineto{\pgfqpoint{1.944420in}{3.072871in}}%
\pgfpathlineto{\pgfqpoint{2.012758in}{3.076300in}}%
\pgfpathlineto{\pgfqpoint{2.081155in}{3.078286in}}%
\pgfpathlineto{\pgfqpoint{2.149575in}{3.079300in}}%
\pgfpathlineto{\pgfqpoint{2.218001in}{3.079969in}}%
\pgfpathlineto{\pgfqpoint{2.286419in}{3.081065in}}%
\pgfpathlineto{\pgfqpoint{2.354798in}{3.083495in}}%
\pgfpathlineto{\pgfqpoint{2.423044in}{3.088268in}}%
\pgfpathlineto{\pgfqpoint{2.490956in}{3.096411in}}%
\pgfpathlineto{\pgfqpoint{2.558203in}{3.108807in}}%
\pgfpathlineto{\pgfqpoint{2.624367in}{3.126031in}}%
\pgfpathlineto{\pgfqpoint{2.689026in}{3.148239in}}%
\pgfpathlineto{\pgfqpoint{2.751875in}{3.175163in}}%
\pgfpathlineto{\pgfqpoint{2.812793in}{3.206232in}}%
\pgfpathlineto{\pgfqpoint{2.871845in}{3.240737in}}%
\pgfpathlineto{\pgfqpoint{2.929229in}{3.277966in}}%
\pgfpathlineto{\pgfqpoint{2.985215in}{3.317282in}}%
\pgfpathlineto{\pgfqpoint{3.040097in}{3.358140in}}%
\pgfpathlineto{\pgfqpoint{3.094161in}{3.400075in}}%
\pgfpathlineto{\pgfqpoint{3.147681in}{3.442706in}}%
\pgfpathlineto{\pgfqpoint{3.200901in}{3.485714in}}%
\pgfpathlineto{\pgfqpoint{3.254048in}{3.528813in}}%
\pgfpathlineto{\pgfqpoint{3.307334in}{3.571741in}}%
\pgfpathlineto{\pgfqpoint{3.307334in}{3.571741in}}%
\pgfusepath{stroke}%
\end{pgfscope}%
\begin{pgfscope}%
\pgfpathrectangle{\pgfqpoint{0.647939in}{0.492442in}}{\pgfqpoint{3.079299in}{3.079299in}}%
\pgfusepath{clip}%
\pgfsetbuttcap%
\pgfsetroundjoin%
\pgfsetlinewidth{0.301125pt}%
\definecolor{currentstroke}{rgb}{0.500000,0.500000,0.500000}%
\pgfsetstrokecolor{currentstroke}%
\pgfsetstrokeopacity{0.300000}%
\pgfsetdash{}{0pt}%
\pgfpathmoveto{\pgfqpoint{0.647939in}{3.015448in}}%
\pgfpathlineto{\pgfqpoint{0.665237in}{3.017414in}}%
\pgfpathlineto{\pgfqpoint{0.733153in}{3.025754in}}%
\pgfpathlineto{\pgfqpoint{0.800906in}{3.035322in}}%
\pgfpathlineto{\pgfqpoint{0.868478in}{3.046102in}}%
\pgfpathlineto{\pgfqpoint{0.935855in}{3.058036in}}%
\pgfpathlineto{\pgfqpoint{1.003037in}{3.071029in}}%
\pgfpathlineto{\pgfqpoint{1.070034in}{3.084947in}}%
\pgfpathlineto{\pgfqpoint{1.136871in}{3.099617in}}%
\pgfpathlineto{\pgfqpoint{1.203586in}{3.114832in}}%
\pgfpathlineto{\pgfqpoint{1.270231in}{3.130354in}}%
\pgfpathlineto{\pgfqpoint{1.336867in}{3.145915in}}%
\pgfpathlineto{\pgfqpoint{1.403560in}{3.161229in}}%
\pgfpathlineto{\pgfqpoint{1.470373in}{3.176005in}}%
\pgfpathlineto{\pgfqpoint{1.537364in}{3.189949in}}%
\pgfpathlineto{\pgfqpoint{1.604576in}{3.202779in}}%
\pgfpathlineto{\pgfqpoint{1.672034in}{3.214239in}}%
\pgfpathlineto{\pgfqpoint{1.739740in}{3.224117in}}%
\pgfpathlineto{\pgfqpoint{1.807675in}{3.232268in}}%
\pgfpathlineto{\pgfqpoint{1.875800in}{3.238634in}}%
\pgfpathlineto{\pgfqpoint{1.944067in}{3.243257in}}%
\pgfpathlineto{\pgfqpoint{2.012423in}{3.246297in}}%
\pgfpathlineto{\pgfqpoint{2.080827in}{3.248046in}}%
\pgfpathlineto{\pgfqpoint{2.149250in}{3.248932in}}%
\pgfpathlineto{\pgfqpoint{2.217676in}{3.249509in}}%
\pgfpathlineto{\pgfqpoint{2.286098in}{3.250440in}}%
\pgfpathlineto{\pgfqpoint{2.354491in}{3.252488in}}%
\pgfpathlineto{\pgfqpoint{2.422792in}{3.256486in}}%
\pgfpathlineto{\pgfqpoint{2.490862in}{3.263278in}}%
\pgfpathlineto{\pgfqpoint{2.558472in}{3.273614in}}%
\pgfpathlineto{\pgfqpoint{2.625319in}{3.288039in}}%
\pgfpathlineto{\pgfqpoint{2.691074in}{3.306813in}}%
\pgfpathlineto{\pgfqpoint{2.755454in}{3.329875in}}%
\pgfpathlineto{\pgfqpoint{2.818285in}{3.356891in}}%
\pgfpathlineto{\pgfqpoint{2.879529in}{3.387350in}}%
\pgfpathlineto{\pgfqpoint{2.939275in}{3.420666in}}%
\pgfpathlineto{\pgfqpoint{2.997698in}{3.456261in}}%
\pgfpathlineto{\pgfqpoint{3.055024in}{3.493605in}}%
\pgfpathlineto{\pgfqpoint{3.111498in}{3.532234in}}%
\pgfpathlineto{\pgfqpoint{3.167366in}{3.571741in}}%
\pgfpathlineto{\pgfqpoint{3.167366in}{3.571741in}}%
\pgfusepath{stroke}%
\end{pgfscope}%
\begin{pgfscope}%
\pgfpathrectangle{\pgfqpoint{0.647939in}{0.492442in}}{\pgfqpoint{3.079299in}{3.079299in}}%
\pgfusepath{clip}%
\pgfsetbuttcap%
\pgfsetroundjoin%
\pgfsetlinewidth{0.301125pt}%
\definecolor{currentstroke}{rgb}{0.500000,0.500000,0.500000}%
\pgfsetstrokecolor{currentstroke}%
\pgfsetstrokeopacity{0.300000}%
\pgfsetdash{}{0pt}%
\pgfpathmoveto{\pgfqpoint{0.647939in}{3.145545in}}%
\pgfpathlineto{\pgfqpoint{0.689890in}{3.150475in}}%
\pgfpathlineto{\pgfqpoint{0.757776in}{3.159055in}}%
\pgfpathlineto{\pgfqpoint{0.825502in}{3.168813in}}%
\pgfpathlineto{\pgfqpoint{0.893054in}{3.179719in}}%
\pgfpathlineto{\pgfqpoint{0.960424in}{3.191700in}}%
\pgfpathlineto{\pgfqpoint{1.027616in}{3.204641in}}%
\pgfpathlineto{\pgfqpoint{1.094648in}{3.218390in}}%
\pgfpathlineto{\pgfqpoint{1.161550in}{3.232763in}}%
\pgfpathlineto{\pgfqpoint{1.228363in}{3.247543in}}%
\pgfpathlineto{\pgfqpoint{1.295140in}{3.262487in}}%
\pgfpathlineto{\pgfqpoint{1.361940in}{3.277331in}}%
\pgfpathlineto{\pgfqpoint{1.428822in}{3.291796in}}%
\pgfpathlineto{\pgfqpoint{1.495843in}{3.305595in}}%
\pgfpathlineto{\pgfqpoint{1.563050in}{3.318454in}}%
\pgfpathlineto{\pgfqpoint{1.630474in}{3.330115in}}%
\pgfpathlineto{\pgfqpoint{1.698127in}{3.340359in}}%
\pgfpathlineto{\pgfqpoint{1.766001in}{3.349015in}}%
\pgfpathlineto{\pgfqpoint{1.834068in}{3.355987in}}%
\pgfpathlineto{\pgfqpoint{1.902286in}{3.361271in}}%
\pgfpathlineto{\pgfqpoint{1.970610in}{3.364964in}}%
\pgfpathlineto{\pgfqpoint{2.038997in}{3.367274in}}%
\pgfpathlineto{\pgfqpoint{2.107413in}{3.368521in}}%
\pgfpathlineto{\pgfqpoint{2.175839in}{3.369151in}}%
\pgfpathlineto{\pgfqpoint{2.244265in}{3.369718in}}%
\pgfpathlineto{\pgfqpoint{2.312682in}{3.370868in}}%
\pgfpathlineto{\pgfqpoint{2.381062in}{3.373308in}}%
\pgfpathlineto{\pgfqpoint{2.449332in}{3.377784in}}%
\pgfpathlineto{\pgfqpoint{2.517358in}{3.385019in}}%
\pgfpathlineto{\pgfqpoint{2.584929in}{3.395640in}}%
\pgfpathlineto{\pgfqpoint{2.651781in}{3.410078in}}%
\pgfpathlineto{\pgfqpoint{2.717638in}{3.428516in}}%
\pgfpathlineto{\pgfqpoint{2.782272in}{3.450873in}}%
\pgfpathlineto{\pgfqpoint{2.845546in}{3.476848in}}%
\pgfpathlineto{\pgfqpoint{2.907429in}{3.505994in}}%
\pgfpathlineto{\pgfqpoint{2.967996in}{3.537798in}}%
\pgfpathlineto{\pgfqpoint{3.027398in}{3.571741in}}%
\pgfpathlineto{\pgfqpoint{3.027398in}{3.571741in}}%
\pgfusepath{stroke}%
\end{pgfscope}%
\begin{pgfscope}%
\pgfpathrectangle{\pgfqpoint{0.647939in}{0.492442in}}{\pgfqpoint{3.079299in}{3.079299in}}%
\pgfusepath{clip}%
\pgfsetbuttcap%
\pgfsetroundjoin%
\pgfsetlinewidth{0.301125pt}%
\definecolor{currentstroke}{rgb}{0.500000,0.500000,0.500000}%
\pgfsetstrokecolor{currentstroke}%
\pgfsetstrokeopacity{0.300000}%
\pgfsetdash{}{0pt}%
\pgfpathmoveto{\pgfqpoint{0.647939in}{3.237013in}}%
\pgfpathlineto{\pgfqpoint{0.664125in}{3.238781in}}%
\pgfpathlineto{\pgfqpoint{0.732081in}{3.246786in}}%
\pgfpathlineto{\pgfqpoint{0.799891in}{3.255945in}}%
\pgfpathlineto{\pgfqpoint{0.867539in}{3.266237in}}%
\pgfpathlineto{\pgfqpoint{0.935016in}{3.277598in}}%
\pgfpathlineto{\pgfqpoint{1.002323in}{3.289930in}}%
\pgfpathlineto{\pgfqpoint{1.069472in}{3.303095in}}%
\pgfpathlineto{\pgfqpoint{1.136488in}{3.316925in}}%
\pgfpathlineto{\pgfqpoint{1.203407in}{3.331215in}}%
\pgfpathlineto{\pgfqpoint{1.270278in}{3.345735in}}%
\pgfpathlineto{\pgfqpoint{1.337154in}{3.360228in}}%
\pgfpathlineto{\pgfqpoint{1.404094in}{3.374423in}}%
\pgfpathlineto{\pgfqpoint{1.471152in}{3.388048in}}%
\pgfpathlineto{\pgfqpoint{1.538374in}{3.400832in}}%
\pgfpathlineto{\pgfqpoint{1.605794in}{3.412522in}}%
\pgfpathlineto{\pgfqpoint{1.673428in}{3.422892in}}%
\pgfpathlineto{\pgfqpoint{1.741274in}{3.431766in}}%
\pgfpathlineto{\pgfqpoint{1.809310in}{3.439031in}}%
\pgfpathlineto{\pgfqpoint{1.877502in}{3.444657in}}%
\pgfpathlineto{\pgfqpoint{1.945806in}{3.448705in}}%
\pgfpathlineto{\pgfqpoint{2.014180in}{3.451341in}}%
\pgfpathlineto{\pgfqpoint{2.082591in}{3.452840in}}%
\pgfpathlineto{\pgfqpoint{2.151015in}{3.453595in}}%
\pgfpathlineto{\pgfqpoint{2.219442in}{3.454090in}}%
\pgfpathlineto{\pgfqpoint{2.287865in}{3.454897in}}%
\pgfpathlineto{\pgfqpoint{2.356268in}{3.456655in}}%
\pgfpathlineto{\pgfqpoint{2.424605in}{3.460055in}}%
\pgfpathlineto{\pgfqpoint{2.492778in}{3.465789in}}%
\pgfpathlineto{\pgfqpoint{2.560630in}{3.474476in}}%
\pgfpathlineto{\pgfqpoint{2.627946in}{3.486596in}}%
\pgfpathlineto{\pgfqpoint{2.694484in}{3.502434in}}%
\pgfpathlineto{\pgfqpoint{2.760010in}{3.522045in}}%
\pgfpathlineto{\pgfqpoint{2.824353in}{3.545260in}}%
\pgfpathlineto{\pgfqpoint{2.887429in}{3.571741in}}%
\pgfpathlineto{\pgfqpoint{2.887429in}{3.571741in}}%
\pgfusepath{stroke}%
\end{pgfscope}%
\begin{pgfscope}%
\pgfpathrectangle{\pgfqpoint{0.647939in}{0.492442in}}{\pgfqpoint{3.079299in}{3.079299in}}%
\pgfusepath{clip}%
\pgfsetbuttcap%
\pgfsetroundjoin%
\pgfsetlinewidth{0.301125pt}%
\definecolor{currentstroke}{rgb}{0.500000,0.500000,0.500000}%
\pgfsetstrokecolor{currentstroke}%
\pgfsetstrokeopacity{0.300000}%
\pgfsetdash{}{0pt}%
\pgfpathmoveto{\pgfqpoint{0.647939in}{3.318567in}}%
\pgfpathlineto{\pgfqpoint{0.713332in}{3.326409in}}%
\pgfpathlineto{\pgfqpoint{0.781201in}{3.335122in}}%
\pgfpathlineto{\pgfqpoint{0.848918in}{3.344947in}}%
\pgfpathlineto{\pgfqpoint{0.916473in}{3.355833in}}%
\pgfpathlineto{\pgfqpoint{0.983865in}{3.367693in}}%
\pgfpathlineto{\pgfqpoint{1.051102in}{3.380402in}}%
\pgfpathlineto{\pgfqpoint{1.118206in}{3.393796in}}%
\pgfpathlineto{\pgfqpoint{1.185212in}{3.407676in}}%
\pgfpathlineto{\pgfqpoint{1.252163in}{3.421820in}}%
\pgfpathlineto{\pgfqpoint{1.319110in}{3.435986in}}%
\pgfpathlineto{\pgfqpoint{1.386106in}{3.449914in}}%
\pgfpathlineto{\pgfqpoint{1.453204in}{3.463338in}}%
\pgfpathlineto{\pgfqpoint{1.520451in}{3.475993in}}%
\pgfpathlineto{\pgfqpoint{1.587881in}{3.487626in}}%
\pgfpathlineto{\pgfqpoint{1.655513in}{3.498012in}}%
\pgfpathlineto{\pgfqpoint{1.723348in}{3.506972in}}%
\pgfpathlineto{\pgfqpoint{1.791369in}{3.514385in}}%
\pgfpathlineto{\pgfqpoint{1.859545in}{3.520202in}}%
\pgfpathlineto{\pgfqpoint{1.927836in}{3.524467in}}%
\pgfpathlineto{\pgfqpoint{1.996202in}{3.527324in}}%
\pgfpathlineto{\pgfqpoint{2.064607in}{3.529016in}}%
\pgfpathlineto{\pgfqpoint{2.133030in}{3.529884in}}%
\pgfpathlineto{\pgfqpoint{2.201457in}{3.530364in}}%
\pgfpathlineto{\pgfqpoint{2.269883in}{3.530985in}}%
\pgfpathlineto{\pgfqpoint{2.338295in}{3.532349in}}%
\pgfpathlineto{\pgfqpoint{2.406663in}{3.535093in}}%
\pgfpathlineto{\pgfqpoint{2.474914in}{3.539860in}}%
\pgfpathlineto{\pgfqpoint{2.542925in}{3.547248in}}%
\pgfpathlineto{\pgfqpoint{2.610519in}{3.557762in}}%
\pgfpathlineto{\pgfqpoint{2.677477in}{3.571741in}}%
\pgfpathlineto{\pgfqpoint{2.677477in}{3.571741in}}%
\pgfusepath{stroke}%
\end{pgfscope}%
\begin{pgfscope}%
\pgfpathrectangle{\pgfqpoint{0.647939in}{0.492442in}}{\pgfqpoint{3.079299in}{3.079299in}}%
\pgfusepath{clip}%
\pgfsetbuttcap%
\pgfsetroundjoin%
\pgfsetlinewidth{0.301125pt}%
\definecolor{currentstroke}{rgb}{0.500000,0.500000,0.500000}%
\pgfsetstrokecolor{currentstroke}%
\pgfsetstrokeopacity{0.300000}%
\pgfsetdash{}{0pt}%
\pgfpathmoveto{\pgfqpoint{1.714242in}{3.543561in}}%
\pgfpathlineto{\pgfqpoint{1.782254in}{3.551053in}}%
\pgfpathlineto{\pgfqpoint{1.850421in}{3.556974in}}%
\pgfpathlineto{\pgfqpoint{1.918705in}{3.561356in}}%
\pgfpathlineto{\pgfqpoint{1.987066in}{3.564324in}}%
\pgfpathlineto{\pgfqpoint{2.055469in}{3.566110in}}%
\pgfpathlineto{\pgfqpoint{2.123891in}{3.567043in}}%
\pgfpathlineto{\pgfqpoint{2.192318in}{3.567540in}}%
\pgfpathlineto{\pgfqpoint{2.260744in}{3.568100in}}%
\pgfpathlineto{\pgfqpoint{2.329161in}{3.569293in}}%
\pgfpathlineto{\pgfqpoint{2.397541in}{3.571741in}}%
\pgfpathlineto{\pgfqpoint{2.397541in}{3.571741in}}%
\pgfusepath{stroke}%
\end{pgfscope}%
\begin{pgfscope}%
\pgfpathrectangle{\pgfqpoint{0.647939in}{0.492442in}}{\pgfqpoint{3.079299in}{3.079299in}}%
\pgfusepath{clip}%
\pgfsetbuttcap%
\pgfsetroundjoin%
\pgfsetlinewidth{0.301125pt}%
\definecolor{currentstroke}{rgb}{0.500000,0.500000,0.500000}%
\pgfsetstrokecolor{currentstroke}%
\pgfsetstrokeopacity{0.300000}%
\pgfsetdash{}{0pt}%
\pgfpathmoveto{\pgfqpoint{0.647939in}{3.400735in}}%
\pgfpathlineto{\pgfqpoint{0.684711in}{3.404827in}}%
\pgfpathlineto{\pgfqpoint{0.752653in}{3.412949in}}%
\pgfpathlineto{\pgfqpoint{0.820455in}{3.422169in}}%
\pgfpathlineto{\pgfqpoint{0.888105in}{3.432451in}}%
\pgfpathlineto{\pgfqpoint{0.955597in}{3.443721in}}%
\pgfpathlineto{\pgfqpoint{1.022938in}{3.455868in}}%
\pgfpathlineto{\pgfqpoint{1.090144in}{3.468741in}}%
\pgfpathlineto{\pgfqpoint{1.157245in}{3.482157in}}%
\pgfpathlineto{\pgfqpoint{1.224277in}{3.495910in}}%
\pgfpathlineto{\pgfqpoint{1.291288in}{3.509767in}}%
\pgfpathlineto{\pgfqpoint{1.358329in}{3.523480in}}%
\pgfpathlineto{\pgfqpoint{1.425450in}{3.536790in}}%
\pgfpathlineto{\pgfqpoint{1.492700in}{3.549433in}}%
\pgfpathlineto{\pgfqpoint{1.560114in}{3.561159in}}%
\pgfpathlineto{\pgfqpoint{1.627716in}{3.571741in}}%
\pgfpathlineto{\pgfqpoint{1.627716in}{3.571741in}}%
\pgfusepath{stroke}%
\end{pgfscope}%
\begin{pgfscope}%
\pgfpathrectangle{\pgfqpoint{0.647939in}{0.492442in}}{\pgfqpoint{3.079299in}{3.079299in}}%
\pgfusepath{clip}%
\pgfsetbuttcap%
\pgfsetroundjoin%
\pgfsetlinewidth{0.301125pt}%
\definecolor{currentstroke}{rgb}{0.500000,0.500000,0.500000}%
\pgfsetstrokecolor{currentstroke}%
\pgfsetstrokeopacity{0.300000}%
\pgfsetdash{}{0pt}%
\pgfpathmoveto{\pgfqpoint{0.647939in}{3.481714in}}%
\pgfpathlineto{\pgfqpoint{0.667724in}{3.483806in}}%
\pgfpathlineto{\pgfqpoint{0.735711in}{3.491548in}}%
\pgfpathlineto{\pgfqpoint{0.803566in}{3.500371in}}%
\pgfpathlineto{\pgfqpoint{0.871277in}{3.510244in}}%
\pgfpathlineto{\pgfqpoint{0.938838in}{3.521096in}}%
\pgfpathlineto{\pgfqpoint{1.006253in}{3.532825in}}%
\pgfpathlineto{\pgfqpoint{1.073535in}{3.545294in}}%
\pgfpathlineto{\pgfqpoint{1.140709in}{3.558333in}}%
\pgfpathlineto{\pgfqpoint{1.207812in}{3.571741in}}%
\pgfpathlineto{\pgfqpoint{1.207812in}{3.571741in}}%
\pgfusepath{stroke}%
\end{pgfscope}%
\begin{pgfscope}%
\pgfpathrectangle{\pgfqpoint{0.647939in}{0.492442in}}{\pgfqpoint{3.079299in}{3.079299in}}%
\pgfusepath{clip}%
\pgfsetbuttcap%
\pgfsetroundjoin%
\pgfsetlinewidth{0.301125pt}%
\definecolor{currentstroke}{rgb}{0.500000,0.500000,0.500000}%
\pgfsetstrokecolor{currentstroke}%
\pgfsetstrokeopacity{0.300000}%
\pgfsetdash{}{0pt}%
\pgfpathmoveto{\pgfqpoint{0.647939in}{2.941885in}}%
\pgfpathlineto{\pgfqpoint{0.647939in}{2.941885in}}%
\pgfpathlineto{\pgfqpoint{0.715879in}{2.950022in}}%
\pgfpathlineto{\pgfqpoint{0.783658in}{2.959406in}}%
\pgfpathlineto{\pgfqpoint{0.851253in}{2.970036in}}%
\pgfpathlineto{\pgfqpoint{0.918647in}{2.981870in}}%
\pgfpathlineto{\pgfqpoint{0.985836in}{2.994827in}}%
\pgfpathlineto{\pgfqpoint{1.052824in}{3.008785in}}%
\pgfpathlineto{\pgfqpoint{1.119634in}{3.023576in}}%
\pgfpathlineto{\pgfqpoint{1.186302in}{3.038996in}}%
\pgfpathlineto{\pgfqpoint{1.252878in}{3.054813in}}%
\pgfpathlineto{\pgfqpoint{1.319421in}{3.070767in}}%
\pgfpathlineto{\pgfqpoint{1.385999in}{3.086575in}}%
\pgfpathlineto{\pgfqpoint{1.452679in}{3.101942in}}%
\pgfpathlineto{\pgfqpoint{1.519526in}{3.116563in}}%
\pgfpathlineto{\pgfqpoint{1.586591in}{3.130140in}}%
\pgfpathlineto{\pgfqpoint{1.653909in}{3.142397in}}%
\pgfpathlineto{\pgfqpoint{1.721490in}{3.153096in}}%
\pgfpathlineto{\pgfqpoint{1.789322in}{3.162061in}}%
\pgfpathlineto{\pgfqpoint{1.857370in}{3.169193in}}%
\pgfpathlineto{\pgfqpoint{1.925587in}{3.174496in}}%
\pgfpathlineto{\pgfqpoint{1.993916in}{3.178103in}}%
\pgfpathlineto{\pgfqpoint{2.062307in}{3.180276in}}%
\pgfpathlineto{\pgfqpoint{2.130725in}{3.181411in}}%
\pgfpathlineto{\pgfqpoint{2.199151in}{3.182041in}}%
\pgfpathlineto{\pgfqpoint{2.267574in}{3.182840in}}%
\pgfpathlineto{\pgfqpoint{2.335976in}{3.184609in}}%
\pgfusepath{stroke}%
\end{pgfscope}%
\begin{pgfscope}%
\pgfpathrectangle{\pgfqpoint{0.647939in}{0.492442in}}{\pgfqpoint{3.079299in}{3.079299in}}%
\pgfusepath{clip}%
\pgfsetbuttcap%
\pgfsetroundjoin%
\pgfsetlinewidth{0.301125pt}%
\definecolor{currentstroke}{rgb}{0.500000,0.500000,0.500000}%
\pgfsetstrokecolor{currentstroke}%
\pgfsetstrokeopacity{0.300000}%
\pgfsetdash{}{0pt}%
\pgfpathmoveto{\pgfqpoint{0.647939in}{2.731932in}}%
\pgfpathlineto{\pgfqpoint{0.647939in}{2.731932in}}%
\pgfpathlineto{\pgfqpoint{0.715841in}{2.740383in}}%
\pgfpathlineto{\pgfqpoint{0.783564in}{2.750158in}}%
\pgfpathlineto{\pgfqpoint{0.851081in}{2.761264in}}%
\pgfpathlineto{\pgfqpoint{0.918372in}{2.773670in}}%
\pgfpathlineto{\pgfqpoint{0.985427in}{2.787301in}}%
\pgfpathlineto{\pgfqpoint{1.052247in}{2.802038in}}%
\pgfpathlineto{\pgfqpoint{1.118854in}{2.817717in}}%
\pgfpathlineto{\pgfqpoint{1.185285in}{2.834131in}}%
\pgfpathlineto{\pgfqpoint{1.251590in}{2.851045in}}%
\pgfpathlineto{\pgfqpoint{1.317836in}{2.868190in}}%
\pgfpathlineto{\pgfqpoint{1.384098in}{2.885272in}}%
\pgfpathlineto{\pgfqpoint{1.450456in}{2.901979in}}%
\pgfpathlineto{\pgfqpoint{1.516984in}{2.917985in}}%
\pgfpathlineto{\pgfqpoint{1.583751in}{2.932963in}}%
\pgfpathlineto{\pgfqpoint{1.650802in}{2.946600in}}%
\pgfpathlineto{\pgfqpoint{1.718161in}{2.958617in}}%
\pgfpathlineto{\pgfqpoint{1.785821in}{2.968792in}}%
\pgfpathlineto{\pgfqpoint{1.853748in}{2.976983in}}%
\pgfpathlineto{\pgfqpoint{1.921890in}{2.983154in}}%
\pgfpathlineto{\pgfqpoint{1.990180in}{2.987410in}}%
\pgfpathlineto{\pgfqpoint{2.058555in}{2.990017in}}%
\pgfpathlineto{\pgfqpoint{2.126968in}{2.991404in}}%
\pgfpathlineto{\pgfqpoint{2.195392in}{2.992172in}}%
\pgfpathlineto{\pgfqpoint{2.263814in}{2.993103in}}%
\pgfpathlineto{\pgfqpoint{2.332206in}{2.995167in}}%
\pgfpathlineto{\pgfqpoint{2.400481in}{2.999477in}}%
\pgfpathlineto{\pgfqpoint{2.468438in}{3.007186in}}%
\pgfpathlineto{\pgfqpoint{2.535724in}{3.019332in}}%
\pgfpathlineto{\pgfqpoint{2.601857in}{3.036634in}}%
\pgfpathlineto{\pgfqpoint{2.666342in}{3.059316in}}%
\pgfpathlineto{\pgfqpoint{2.728815in}{3.087084in}}%
\pgfpathlineto{\pgfqpoint{2.789140in}{3.119276in}}%
\pgfpathlineto{\pgfqpoint{2.847406in}{3.155080in}}%
\pgfusepath{stroke}%
\end{pgfscope}%
\begin{pgfscope}%
\pgfpathrectangle{\pgfqpoint{0.647939in}{0.492442in}}{\pgfqpoint{3.079299in}{3.079299in}}%
\pgfusepath{clip}%
\pgfsetbuttcap%
\pgfsetroundjoin%
\pgfsetlinewidth{0.301125pt}%
\definecolor{currentstroke}{rgb}{0.500000,0.500000,0.500000}%
\pgfsetstrokecolor{currentstroke}%
\pgfsetstrokeopacity{0.300000}%
\pgfsetdash{}{0pt}%
\pgfpathmoveto{\pgfqpoint{0.647939in}{2.661948in}}%
\pgfpathlineto{\pgfqpoint{0.647939in}{2.661948in}}%
\pgfpathlineto{\pgfqpoint{0.715827in}{2.670509in}}%
\pgfpathlineto{\pgfqpoint{0.783530in}{2.680421in}}%
\pgfpathlineto{\pgfqpoint{0.851019in}{2.691696in}}%
\pgfpathlineto{\pgfqpoint{0.918272in}{2.704304in}}%
\pgfpathlineto{\pgfqpoint{0.985277in}{2.718175in}}%
\pgfpathlineto{\pgfqpoint{1.052036in}{2.733191in}}%
\pgfpathlineto{\pgfqpoint{1.118567in}{2.749188in}}%
\pgfpathlineto{\pgfqpoint{1.184907in}{2.765961in}}%
\pgfpathlineto{\pgfqpoint{1.251110in}{2.783271in}}%
\pgfpathlineto{\pgfqpoint{1.317242in}{2.800850in}}%
\pgfpathlineto{\pgfqpoint{1.383382in}{2.818400in}}%
\pgfpathlineto{\pgfqpoint{1.449613in}{2.835601in}}%
\pgfpathlineto{\pgfqpoint{1.516016in}{2.852123in}}%
\pgfpathlineto{\pgfqpoint{1.582662in}{2.867627in}}%
\pgfpathlineto{\pgfqpoint{1.649605in}{2.881788in}}%
\pgfpathlineto{\pgfqpoint{1.716871in}{2.894311in}}%
\pgfpathlineto{\pgfqpoint{1.784458in}{2.904958in}}%
\pgfpathlineto{\pgfqpoint{1.852333in}{2.913566in}}%
\pgfpathlineto{\pgfqpoint{1.920442in}{2.920087in}}%
\pgfpathlineto{\pgfqpoint{1.988714in}{2.924609in}}%
\pgfpathlineto{\pgfqpoint{2.057081in}{2.927397in}}%
\pgfpathlineto{\pgfqpoint{2.125491in}{2.928890in}}%
\pgfpathlineto{\pgfqpoint{2.193915in}{2.929716in}}%
\pgfpathlineto{\pgfqpoint{2.262336in}{2.930704in}}%
\pgfpathlineto{\pgfqpoint{2.330723in}{2.932891in}}%
\pgfpathlineto{\pgfqpoint{2.398977in}{2.937490in}}%
\pgfpathlineto{\pgfqpoint{2.466861in}{2.945767in}}%
\pgfpathlineto{\pgfqpoint{2.533960in}{2.958845in}}%
\pgfusepath{stroke}%
\end{pgfscope}%
\begin{pgfscope}%
\pgfpathrectangle{\pgfqpoint{0.647939in}{0.492442in}}{\pgfqpoint{3.079299in}{3.079299in}}%
\pgfusepath{clip}%
\pgfsetbuttcap%
\pgfsetroundjoin%
\pgfsetlinewidth{0.301125pt}%
\definecolor{currentstroke}{rgb}{0.500000,0.500000,0.500000}%
\pgfsetstrokecolor{currentstroke}%
\pgfsetstrokeopacity{0.300000}%
\pgfsetdash{}{0pt}%
\pgfpathmoveto{\pgfqpoint{0.647939in}{2.451996in}}%
\pgfpathlineto{\pgfqpoint{0.647939in}{2.451996in}}%
\pgfpathlineto{\pgfqpoint{0.715782in}{2.460903in}}%
\pgfpathlineto{\pgfqpoint{0.783419in}{2.471251in}}%
\pgfpathlineto{\pgfqpoint{0.850816in}{2.483063in}}%
\pgfpathlineto{\pgfqpoint{0.917943in}{2.496320in}}%
\pgfpathlineto{\pgfqpoint{0.984783in}{2.510960in}}%
\pgfpathlineto{\pgfqpoint{1.051333in}{2.526874in}}%
\pgfpathlineto{\pgfqpoint{1.117607in}{2.543904in}}%
\pgfpathlineto{\pgfqpoint{1.183640in}{2.561844in}}%
\pgfpathlineto{\pgfqpoint{1.249489in}{2.580456in}}%
\pgfpathlineto{\pgfqpoint{1.315224in}{2.599465in}}%
\pgfpathlineto{\pgfqpoint{1.380933in}{2.618564in}}%
\pgfusepath{stroke}%
\end{pgfscope}%
\begin{pgfscope}%
\pgfpathrectangle{\pgfqpoint{0.647939in}{0.492442in}}{\pgfqpoint{3.079299in}{3.079299in}}%
\pgfusepath{clip}%
\pgfsetbuttcap%
\pgfsetroundjoin%
\pgfsetlinewidth{0.301125pt}%
\definecolor{currentstroke}{rgb}{0.500000,0.500000,0.500000}%
\pgfsetstrokecolor{currentstroke}%
\pgfsetstrokeopacity{0.300000}%
\pgfsetdash{}{0pt}%
\pgfpathmoveto{\pgfqpoint{0.647939in}{2.382012in}}%
\pgfpathlineto{\pgfqpoint{0.647939in}{2.382012in}}%
\pgfpathlineto{\pgfqpoint{0.715765in}{2.391041in}}%
\pgfpathlineto{\pgfqpoint{0.783379in}{2.401543in}}%
\pgfpathlineto{\pgfqpoint{0.850742in}{2.413545in}}%
\pgfpathlineto{\pgfqpoint{0.917823in}{2.427033in}}%
\pgfpathlineto{\pgfqpoint{0.984602in}{2.441948in}}%
\pgfpathlineto{\pgfqpoint{1.051073in}{2.458184in}}%
\pgfpathlineto{\pgfqpoint{1.117250in}{2.475585in}}%
\pgfpathlineto{\pgfqpoint{1.183167in}{2.493949in}}%
\pgfpathlineto{\pgfqpoint{1.248879in}{2.513035in}}%
\pgfpathlineto{\pgfqpoint{1.314461in}{2.532567in}}%
\pgfpathlineto{\pgfqpoint{1.380002in}{2.552238in}}%
\pgfpathlineto{\pgfqpoint{1.445601in}{2.571709in}}%
\pgfpathlineto{\pgfqpoint{1.511364in}{2.590620in}}%
\pgfpathlineto{\pgfqpoint{1.577386in}{2.608598in}}%
\pgfpathlineto{\pgfqpoint{1.643750in}{2.625263in}}%
\pgfpathlineto{\pgfqpoint{1.710510in}{2.640250in}}%
\pgfpathlineto{\pgfqpoint{1.777685in}{2.653236in}}%
\pgfpathlineto{\pgfqpoint{1.845254in}{2.663971in}}%
\pgfpathlineto{\pgfqpoint{1.913159in}{2.672313in}}%
\pgfpathlineto{\pgfqpoint{1.981316in}{2.678274in}}%
\pgfpathlineto{\pgfqpoint{2.049632in}{2.682068in}}%
\pgfpathlineto{\pgfqpoint{2.118025in}{2.684167in}}%
\pgfpathlineto{\pgfqpoint{2.186443in}{2.685327in}}%
\pgfpathlineto{\pgfqpoint{2.254858in}{2.686624in}}%
\pgfpathlineto{\pgfqpoint{2.323215in}{2.689508in}}%
\pgfpathlineto{\pgfqpoint{2.391316in}{2.695792in}}%
\pgfpathlineto{\pgfqpoint{2.458655in}{2.707446in}}%
\pgfpathlineto{\pgfqpoint{2.524361in}{2.726024in}}%
\pgfpathlineto{\pgfqpoint{2.587483in}{2.752006in}}%
\pgfpathlineto{\pgfqpoint{2.647438in}{2.784681in}}%
\pgfpathlineto{\pgfqpoint{2.704232in}{2.822674in}}%
\pgfpathlineto{\pgfqpoint{2.758273in}{2.864563in}}%
\pgfusepath{stroke}%
\end{pgfscope}%
\begin{pgfscope}%
\pgfpathrectangle{\pgfqpoint{0.647939in}{0.492442in}}{\pgfqpoint{3.079299in}{3.079299in}}%
\pgfusepath{clip}%
\pgfsetbuttcap%
\pgfsetroundjoin%
\pgfsetlinewidth{0.301125pt}%
\definecolor{currentstroke}{rgb}{0.500000,0.500000,0.500000}%
\pgfsetstrokecolor{currentstroke}%
\pgfsetstrokeopacity{0.300000}%
\pgfsetdash{}{0pt}%
\pgfpathmoveto{\pgfqpoint{0.647939in}{2.312028in}}%
\pgfpathlineto{\pgfqpoint{0.647939in}{2.312028in}}%
\pgfpathlineto{\pgfqpoint{0.715748in}{2.321182in}}%
\pgfpathlineto{\pgfqpoint{0.783337in}{2.331842in}}%
\pgfpathlineto{\pgfqpoint{0.850664in}{2.344040in}}%
\pgfpathlineto{\pgfqpoint{0.917697in}{2.357767in}}%
\pgfpathlineto{\pgfqpoint{0.984411in}{2.372967in}}%
\pgfpathlineto{\pgfqpoint{1.050799in}{2.389538in}}%
\pgfpathlineto{\pgfqpoint{1.116873in}{2.407327in}}%
\pgfpathlineto{\pgfqpoint{1.182665in}{2.426132in}}%
\pgfpathlineto{\pgfqpoint{1.248231in}{2.445714in}}%
\pgfpathlineto{\pgfqpoint{1.313646in}{2.465797in}}%
\pgfpathlineto{\pgfqpoint{1.379003in}{2.486069in}}%
\pgfpathlineto{\pgfqpoint{1.444406in}{2.506191in}}%
\pgfpathlineto{\pgfqpoint{1.509965in}{2.525796in}}%
\pgfpathlineto{\pgfqpoint{1.575785in}{2.544501in}}%
\pgfusepath{stroke}%
\end{pgfscope}%
\begin{pgfscope}%
\pgfpathrectangle{\pgfqpoint{0.647939in}{0.492442in}}{\pgfqpoint{3.079299in}{3.079299in}}%
\pgfusepath{clip}%
\pgfsetbuttcap%
\pgfsetroundjoin%
\pgfsetlinewidth{0.301125pt}%
\definecolor{currentstroke}{rgb}{0.500000,0.500000,0.500000}%
\pgfsetstrokecolor{currentstroke}%
\pgfsetstrokeopacity{0.300000}%
\pgfsetdash{}{0pt}%
\pgfpathmoveto{\pgfqpoint{0.647939in}{2.242044in}}%
\pgfpathlineto{\pgfqpoint{0.647939in}{2.242044in}}%
\pgfpathlineto{\pgfqpoint{0.715731in}{2.251327in}}%
\pgfpathlineto{\pgfqpoint{0.783293in}{2.262150in}}%
\pgfpathlineto{\pgfqpoint{0.850583in}{2.274551in}}%
\pgfpathlineto{\pgfqpoint{0.917564in}{2.288524in}}%
\pgfpathlineto{\pgfqpoint{0.984210in}{2.304019in}}%
\pgfpathlineto{\pgfqpoint{1.050510in}{2.320938in}}%
\pgfpathlineto{\pgfqpoint{1.116473in}{2.339131in}}%
\pgfpathlineto{\pgfqpoint{1.182132in}{2.358397in}}%
\pgfpathlineto{\pgfqpoint{1.247540in}{2.378499in}}%
\pgfpathlineto{\pgfqpoint{1.312775in}{2.399160in}}%
\pgfusepath{stroke}%
\end{pgfscope}%
\begin{pgfscope}%
\pgfpathrectangle{\pgfqpoint{0.647939in}{0.492442in}}{\pgfqpoint{3.079299in}{3.079299in}}%
\pgfusepath{clip}%
\pgfsetbuttcap%
\pgfsetroundjoin%
\pgfsetlinewidth{0.301125pt}%
\definecolor{currentstroke}{rgb}{0.500000,0.500000,0.500000}%
\pgfsetstrokecolor{currentstroke}%
\pgfsetstrokeopacity{0.300000}%
\pgfsetdash{}{0pt}%
\pgfpathmoveto{\pgfqpoint{0.647939in}{2.172060in}}%
\pgfpathlineto{\pgfqpoint{0.647939in}{2.172060in}}%
\pgfpathlineto{\pgfqpoint{0.715712in}{2.181475in}}%
\pgfpathlineto{\pgfqpoint{0.783247in}{2.192467in}}%
\pgfpathlineto{\pgfqpoint{0.850498in}{2.205076in}}%
\pgfpathlineto{\pgfqpoint{0.917425in}{2.219305in}}%
\pgfpathlineto{\pgfqpoint{0.983998in}{2.235107in}}%
\pgfpathlineto{\pgfqpoint{1.050205in}{2.252388in}}%
\pgfpathlineto{\pgfqpoint{1.116050in}{2.271001in}}%
\pgfpathlineto{\pgfqpoint{1.181565in}{2.290749in}}%
\pgfpathlineto{\pgfqpoint{1.246803in}{2.311396in}}%
\pgfpathlineto{\pgfqpoint{1.311842in}{2.332664in}}%
\pgfpathlineto{\pgfqpoint{1.376779in}{2.354242in}}%
\pgfpathlineto{\pgfqpoint{1.441729in}{2.375783in}}%
\pgfpathlineto{\pgfqpoint{1.506813in}{2.396912in}}%
\pgfpathlineto{\pgfqpoint{1.572153in}{2.417230in}}%
\pgfpathlineto{\pgfqpoint{1.637860in}{2.436320in}}%
\pgfpathlineto{\pgfqpoint{1.704021in}{2.453763in}}%
\pgfpathlineto{\pgfqpoint{1.770685in}{2.469158in}}%
\pgfpathlineto{\pgfqpoint{1.837853in}{2.482163in}}%
\pgfpathlineto{\pgfqpoint{1.905475in}{2.492534in}}%
\pgfpathlineto{\pgfqpoint{1.973458in}{2.500184in}}%
\pgfpathlineto{\pgfqpoint{2.041686in}{2.505246in}}%
\pgfpathlineto{\pgfqpoint{2.110046in}{2.508149in}}%
\pgfpathlineto{\pgfqpoint{2.178455in}{2.509748in}}%
\pgfpathlineto{\pgfqpoint{2.246861in}{2.511425in}}%
\pgfpathlineto{\pgfqpoint{2.315160in}{2.515219in}}%
\pgfpathlineto{\pgfqpoint{2.382943in}{2.523900in}}%
\pgfpathlineto{\pgfqpoint{2.449143in}{2.540432in}}%
\pgfpathlineto{\pgfqpoint{2.512170in}{2.566423in}}%
\pgfpathlineto{\pgfqpoint{2.570952in}{2.601020in}}%
\pgfpathlineto{\pgfqpoint{2.625602in}{2.641934in}}%
\pgfusepath{stroke}%
\end{pgfscope}%
\begin{pgfscope}%
\pgfpathrectangle{\pgfqpoint{0.647939in}{0.492442in}}{\pgfqpoint{3.079299in}{3.079299in}}%
\pgfusepath{clip}%
\pgfsetbuttcap%
\pgfsetroundjoin%
\pgfsetlinewidth{0.301125pt}%
\definecolor{currentstroke}{rgb}{0.500000,0.500000,0.500000}%
\pgfsetstrokecolor{currentstroke}%
\pgfsetstrokeopacity{0.300000}%
\pgfsetdash{}{0pt}%
\pgfpathmoveto{\pgfqpoint{0.647939in}{2.102076in}}%
\pgfpathlineto{\pgfqpoint{0.647939in}{2.102076in}}%
\pgfpathlineto{\pgfqpoint{0.715693in}{2.111628in}}%
\pgfpathlineto{\pgfqpoint{0.783199in}{2.122792in}}%
\pgfpathlineto{\pgfqpoint{0.850408in}{2.135617in}}%
\pgfpathlineto{\pgfqpoint{0.917278in}{2.150111in}}%
\pgfpathlineto{\pgfqpoint{0.983775in}{2.166231in}}%
\pgfpathlineto{\pgfqpoint{1.049882in}{2.183888in}}%
\pgfpathlineto{\pgfqpoint{1.115601in}{2.202940in}}%
\pgfpathlineto{\pgfqpoint{1.180961in}{2.223193in}}%
\pgfpathlineto{\pgfqpoint{1.246015in}{2.244411in}}%
\pgfpathlineto{\pgfqpoint{1.310842in}{2.266318in}}%
\pgfpathlineto{\pgfqpoint{1.375540in}{2.288604in}}%
\pgfpathlineto{\pgfqpoint{1.440227in}{2.310919in}}%
\pgfpathlineto{\pgfqpoint{1.505033in}{2.332887in}}%
\pgfpathlineto{\pgfqpoint{1.570089in}{2.354100in}}%
\pgfpathlineto{\pgfqpoint{1.635515in}{2.374131in}}%
\pgfpathlineto{\pgfqpoint{1.701413in}{2.392543in}}%
\pgfpathlineto{\pgfqpoint{1.767844in}{2.408910in}}%
\pgfusepath{stroke}%
\end{pgfscope}%
\begin{pgfscope}%
\pgfpathrectangle{\pgfqpoint{0.647939in}{0.492442in}}{\pgfqpoint{3.079299in}{3.079299in}}%
\pgfusepath{clip}%
\pgfsetbuttcap%
\pgfsetroundjoin%
\pgfsetlinewidth{0.301125pt}%
\definecolor{currentstroke}{rgb}{0.500000,0.500000,0.500000}%
\pgfsetstrokecolor{currentstroke}%
\pgfsetstrokeopacity{0.300000}%
\pgfsetdash{}{0pt}%
\pgfpathmoveto{\pgfqpoint{0.647939in}{2.032092in}}%
\pgfpathlineto{\pgfqpoint{0.647939in}{2.032092in}}%
\pgfpathlineto{\pgfqpoint{0.715673in}{2.041784in}}%
\pgfpathlineto{\pgfqpoint{0.783149in}{2.053127in}}%
\pgfpathlineto{\pgfqpoint{0.850315in}{2.066176in}}%
\pgfpathlineto{\pgfqpoint{0.917124in}{2.080943in}}%
\pgfpathlineto{\pgfqpoint{0.983539in}{2.097394in}}%
\pgfpathlineto{\pgfqpoint{1.049539in}{2.115443in}}%
\pgfpathlineto{\pgfqpoint{1.115124in}{2.134953in}}%
\pgfpathlineto{\pgfqpoint{1.180318in}{2.155734in}}%
\pgfpathlineto{\pgfqpoint{1.245173in}{2.177551in}}%
\pgfpathlineto{\pgfqpoint{1.309768in}{2.200131in}}%
\pgfpathlineto{\pgfqpoint{1.374204in}{2.223164in}}%
\pgfpathlineto{\pgfqpoint{1.438602in}{2.246301in}}%
\pgfpathlineto{\pgfqpoint{1.503098in}{2.269162in}}%
\pgfpathlineto{\pgfqpoint{1.567833in}{2.291335in}}%
\pgfpathlineto{\pgfqpoint{1.632939in}{2.312384in}}%
\pgfpathlineto{\pgfqpoint{1.698531in}{2.331855in}}%
\pgfusepath{stroke}%
\end{pgfscope}%
\begin{pgfscope}%
\pgfpathrectangle{\pgfqpoint{0.647939in}{0.492442in}}{\pgfqpoint{3.079299in}{3.079299in}}%
\pgfusepath{clip}%
\pgfsetbuttcap%
\pgfsetroundjoin%
\pgfsetlinewidth{0.301125pt}%
\definecolor{currentstroke}{rgb}{0.500000,0.500000,0.500000}%
\pgfsetstrokecolor{currentstroke}%
\pgfsetstrokeopacity{0.300000}%
\pgfsetdash{}{0pt}%
\pgfpathmoveto{\pgfqpoint{0.647939in}{1.962108in}}%
\pgfpathlineto{\pgfqpoint{0.647939in}{1.962108in}}%
\pgfpathlineto{\pgfqpoint{0.715652in}{1.971944in}}%
\pgfpathlineto{\pgfqpoint{0.783096in}{1.983471in}}%
\pgfpathlineto{\pgfqpoint{0.850216in}{1.996751in}}%
\pgfpathlineto{\pgfqpoint{0.916962in}{2.011803in}}%
\pgfpathlineto{\pgfqpoint{0.983291in}{2.028597in}}%
\pgfpathlineto{\pgfqpoint{1.049177in}{2.047054in}}%
\pgfpathlineto{\pgfqpoint{1.114617in}{2.067043in}}%
\pgfpathlineto{\pgfqpoint{1.179632in}{2.088376in}}%
\pgfpathlineto{\pgfqpoint{1.244272in}{2.110823in}}%
\pgfpathlineto{\pgfqpoint{1.308614in}{2.134112in}}%
\pgfpathlineto{\pgfqpoint{1.372761in}{2.157935in}}%
\pgfpathlineto{\pgfqpoint{1.436839in}{2.181944in}}%
\pgfpathlineto{\pgfqpoint{1.500990in}{2.205758in}}%
\pgfpathlineto{\pgfqpoint{1.565363in}{2.228963in}}%
\pgfpathlineto{\pgfqpoint{1.630103in}{2.251112in}}%
\pgfpathlineto{\pgfqpoint{1.695341in}{2.271741in}}%
\pgfpathlineto{\pgfqpoint{1.761172in}{2.290373in}}%
\pgfpathlineto{\pgfqpoint{1.827644in}{2.306554in}}%
\pgfpathlineto{\pgfqpoint{1.894740in}{2.319897in}}%
\pgfpathlineto{\pgfqpoint{1.962374in}{2.330160in}}%
\pgfpathlineto{\pgfqpoint{2.030407in}{2.337327in}}%
\pgfpathlineto{\pgfqpoint{2.098680in}{2.341729in}}%
\pgfpathlineto{\pgfqpoint{2.167059in}{2.344234in}}%
\pgfpathlineto{\pgfqpoint{2.235445in}{2.346595in}}%
\pgfpathlineto{\pgfqpoint{2.303587in}{2.352100in}}%
\pgfpathlineto{\pgfqpoint{2.370268in}{2.365930in}}%
\pgfusepath{stroke}%
\end{pgfscope}%
\begin{pgfscope}%
\pgfpathrectangle{\pgfqpoint{0.647939in}{0.492442in}}{\pgfqpoint{3.079299in}{3.079299in}}%
\pgfusepath{clip}%
\pgfsetbuttcap%
\pgfsetroundjoin%
\pgfsetlinewidth{0.301125pt}%
\definecolor{currentstroke}{rgb}{0.500000,0.500000,0.500000}%
\pgfsetstrokecolor{currentstroke}%
\pgfsetstrokeopacity{0.300000}%
\pgfsetdash{}{0pt}%
\pgfpathmoveto{\pgfqpoint{0.647939in}{1.822139in}}%
\pgfpathlineto{\pgfqpoint{0.647939in}{1.822139in}}%
\pgfpathlineto{\pgfqpoint{0.715607in}{1.832277in}}%
\pgfpathlineto{\pgfqpoint{0.782983in}{1.844192in}}%
\pgfpathlineto{\pgfqpoint{0.850004in}{1.857958in}}%
\pgfpathlineto{\pgfqpoint{0.916610in}{1.873610in}}%
\pgfpathlineto{\pgfqpoint{0.982749in}{1.891134in}}%
\pgfpathlineto{\pgfqpoint{1.048384in}{1.910460in}}%
\pgfpathlineto{\pgfqpoint{1.113502in}{1.931470in}}%
\pgfpathlineto{\pgfqpoint{1.178115in}{1.953988in}}%
\pgfpathlineto{\pgfqpoint{1.242267in}{1.977791in}}%
\pgfpathlineto{\pgfqpoint{1.306032in}{2.002614in}}%
\pgfpathlineto{\pgfqpoint{1.369514in}{2.028156in}}%
\pgfpathlineto{\pgfqpoint{1.432844in}{2.054073in}}%
\pgfpathlineto{\pgfqpoint{1.496175in}{2.079990in}}%
\pgfpathlineto{\pgfqpoint{1.559674in}{2.105489in}}%
\pgfpathlineto{\pgfqpoint{1.623513in}{2.130119in}}%
\pgfpathlineto{\pgfqpoint{1.687855in}{2.153393in}}%
\pgfpathlineto{\pgfqpoint{1.752838in}{2.174798in}}%
\pgfpathlineto{\pgfqpoint{1.818554in}{2.193813in}}%
\pgfpathlineto{\pgfqpoint{1.885030in}{2.209949in}}%
\pgfpathlineto{\pgfqpoint{1.952214in}{2.222808in}}%
\pgfpathlineto{\pgfqpoint{2.019971in}{2.232198in}}%
\pgfpathlineto{\pgfqpoint{2.088107in}{2.238313in}}%
\pgfpathlineto{\pgfqpoint{2.156429in}{2.241987in}}%
\pgfpathlineto{\pgfqpoint{2.224772in}{2.245329in}}%
\pgfusepath{stroke}%
\end{pgfscope}%
\begin{pgfscope}%
\pgfpathrectangle{\pgfqpoint{0.647939in}{0.492442in}}{\pgfqpoint{3.079299in}{3.079299in}}%
\pgfusepath{clip}%
\pgfsetbuttcap%
\pgfsetroundjoin%
\pgfsetlinewidth{0.301125pt}%
\definecolor{currentstroke}{rgb}{0.500000,0.500000,0.500000}%
\pgfsetstrokecolor{currentstroke}%
\pgfsetstrokeopacity{0.300000}%
\pgfsetdash{}{0pt}%
\pgfpathmoveto{\pgfqpoint{0.647939in}{1.752155in}}%
\pgfpathlineto{\pgfqpoint{0.647939in}{1.752155in}}%
\pgfpathlineto{\pgfqpoint{0.715583in}{1.762451in}}%
\pgfpathlineto{\pgfqpoint{0.782922in}{1.774568in}}%
\pgfpathlineto{\pgfqpoint{0.849890in}{1.788591in}}%
\pgfpathlineto{\pgfqpoint{0.916420in}{1.804561in}}%
\pgfpathlineto{\pgfqpoint{0.982454in}{1.822471in}}%
\pgfpathlineto{\pgfqpoint{1.047950in}{1.842261in}}%
\pgfpathlineto{\pgfqpoint{1.112888in}{1.863816in}}%
\pgfpathlineto{\pgfqpoint{1.177276in}{1.886970in}}%
\pgfpathlineto{\pgfqpoint{1.241151in}{1.911504in}}%
\pgfpathlineto{\pgfqpoint{1.304587in}{1.937158in}}%
\pgfpathlineto{\pgfqpoint{1.367684in}{1.963633in}}%
\pgfpathlineto{\pgfqpoint{1.430577in}{1.990594in}}%
\pgfpathlineto{\pgfqpoint{1.493422in}{2.017667in}}%
\pgfpathlineto{\pgfqpoint{1.556395in}{2.044440in}}%
\pgfusepath{stroke}%
\end{pgfscope}%
\begin{pgfscope}%
\pgfpathrectangle{\pgfqpoint{0.647939in}{0.492442in}}{\pgfqpoint{3.079299in}{3.079299in}}%
\pgfusepath{clip}%
\pgfsetbuttcap%
\pgfsetroundjoin%
\pgfsetlinewidth{0.301125pt}%
\definecolor{currentstroke}{rgb}{0.500000,0.500000,0.500000}%
\pgfsetstrokecolor{currentstroke}%
\pgfsetstrokeopacity{0.300000}%
\pgfsetdash{}{0pt}%
\pgfpathmoveto{\pgfqpoint{0.647939in}{1.682171in}}%
\pgfpathlineto{\pgfqpoint{0.647939in}{1.682171in}}%
\pgfpathlineto{\pgfqpoint{0.715557in}{1.692630in}}%
\pgfpathlineto{\pgfqpoint{0.782858in}{1.704957in}}%
\pgfpathlineto{\pgfqpoint{0.849769in}{1.719246in}}%
\pgfpathlineto{\pgfqpoint{0.916218in}{1.735546in}}%
\pgfpathlineto{\pgfqpoint{0.982141in}{1.753859in}}%
\pgfpathlineto{\pgfqpoint{1.047488in}{1.774132in}}%
\pgfpathlineto{\pgfqpoint{1.112233in}{1.796259in}}%
\pgfpathlineto{\pgfqpoint{1.176376in}{1.820079in}}%
\pgfpathlineto{\pgfqpoint{1.239951in}{1.845381in}}%
\pgfpathlineto{\pgfqpoint{1.303024in}{1.871910in}}%
\pgfpathlineto{\pgfqpoint{1.365698in}{1.899374in}}%
\pgfusepath{stroke}%
\end{pgfscope}%
\begin{pgfscope}%
\pgfpathrectangle{\pgfqpoint{0.647939in}{0.492442in}}{\pgfqpoint{3.079299in}{3.079299in}}%
\pgfusepath{clip}%
\pgfsetbuttcap%
\pgfsetroundjoin%
\pgfsetlinewidth{0.301125pt}%
\definecolor{currentstroke}{rgb}{0.500000,0.500000,0.500000}%
\pgfsetstrokecolor{currentstroke}%
\pgfsetstrokeopacity{0.300000}%
\pgfsetdash{}{0pt}%
\pgfpathmoveto{\pgfqpoint{0.647939in}{1.612187in}}%
\pgfpathlineto{\pgfqpoint{0.647939in}{1.612187in}}%
\pgfpathlineto{\pgfqpoint{0.715531in}{1.622814in}}%
\pgfpathlineto{\pgfqpoint{0.782791in}{1.635358in}}%
\pgfpathlineto{\pgfqpoint{0.849642in}{1.649922in}}%
\pgfpathlineto{\pgfqpoint{0.916005in}{1.666566in}}%
\pgfpathlineto{\pgfqpoint{0.981809in}{1.685299in}}%
\pgfpathlineto{\pgfqpoint{1.046996in}{1.706077in}}%
\pgfpathlineto{\pgfqpoint{1.111533in}{1.728802in}}%
\pgfpathlineto{\pgfqpoint{1.175411in}{1.753320in}}%
\pgfpathlineto{\pgfqpoint{1.238657in}{1.779429in}}%
\pgfpathlineto{\pgfqpoint{1.301334in}{1.806881in}}%
\pgfpathlineto{\pgfqpoint{1.363539in}{1.835389in}}%
\pgfpathlineto{\pgfqpoint{1.425404in}{1.864631in}}%
\pgfpathlineto{\pgfqpoint{1.487092in}{1.894246in}}%
\pgfpathlineto{\pgfqpoint{1.548790in}{1.923840in}}%
\pgfpathlineto{\pgfqpoint{1.610706in}{1.952975in}}%
\pgfpathlineto{\pgfqpoint{1.673051in}{1.981170in}}%
\pgfpathlineto{\pgfqpoint{1.736036in}{2.007896in}}%
\pgfpathlineto{\pgfqpoint{1.799844in}{2.032576in}}%
\pgfpathlineto{\pgfqpoint{1.864606in}{2.054607in}}%
\pgfpathlineto{\pgfqpoint{1.930374in}{2.073399in}}%
\pgfpathlineto{\pgfqpoint{1.997086in}{2.088463in}}%
\pgfpathlineto{\pgfqpoint{2.064568in}{2.099578in}}%
\pgfpathlineto{\pgfqpoint{2.132547in}{2.107210in}}%
\pgfpathlineto{\pgfqpoint{2.200606in}{2.114209in}}%
\pgfpathlineto{\pgfqpoint{2.200606in}{2.114209in}}%
\pgfusepath{stroke}%
\end{pgfscope}%
\begin{pgfscope}%
\pgfpathrectangle{\pgfqpoint{0.647939in}{0.492442in}}{\pgfqpoint{3.079299in}{3.079299in}}%
\pgfusepath{clip}%
\pgfsetbuttcap%
\pgfsetroundjoin%
\pgfsetlinewidth{0.301125pt}%
\definecolor{currentstroke}{rgb}{0.500000,0.500000,0.500000}%
\pgfsetstrokecolor{currentstroke}%
\pgfsetstrokeopacity{0.300000}%
\pgfsetdash{}{0pt}%
\pgfpathmoveto{\pgfqpoint{0.647939in}{1.542203in}}%
\pgfpathlineto{\pgfqpoint{0.647939in}{1.542203in}}%
\pgfpathlineto{\pgfqpoint{0.715503in}{1.553003in}}%
\pgfpathlineto{\pgfqpoint{0.782720in}{1.565772in}}%
\pgfpathlineto{\pgfqpoint{0.849508in}{1.580622in}}%
\pgfpathlineto{\pgfqpoint{0.915779in}{1.597623in}}%
\pgfpathlineto{\pgfqpoint{0.981456in}{1.616794in}}%
\pgfpathlineto{\pgfqpoint{1.046471in}{1.638101in}}%
\pgfpathlineto{\pgfqpoint{1.110782in}{1.661452in}}%
\pgfpathlineto{\pgfqpoint{1.174373in}{1.686703in}}%
\pgfpathlineto{\pgfqpoint{1.237262in}{1.713658in}}%
\pgfpathlineto{\pgfqpoint{1.299505in}{1.742078in}}%
\pgfpathlineto{\pgfqpoint{1.361195in}{1.771682in}}%
\pgfpathlineto{\pgfqpoint{1.422461in}{1.802156in}}%
\pgfpathlineto{\pgfqpoint{1.483466in}{1.833153in}}%
\pgfusepath{stroke}%
\end{pgfscope}%
\begin{pgfscope}%
\pgfpathrectangle{\pgfqpoint{0.647939in}{0.492442in}}{\pgfqpoint{3.079299in}{3.079299in}}%
\pgfusepath{clip}%
\pgfsetbuttcap%
\pgfsetroundjoin%
\pgfsetlinewidth{0.301125pt}%
\definecolor{currentstroke}{rgb}{0.500000,0.500000,0.500000}%
\pgfsetstrokecolor{currentstroke}%
\pgfsetstrokeopacity{0.300000}%
\pgfsetdash{}{0pt}%
\pgfpathmoveto{\pgfqpoint{0.647939in}{1.472219in}}%
\pgfpathlineto{\pgfqpoint{0.647939in}{1.472219in}}%
\pgfpathlineto{\pgfqpoint{0.715474in}{1.483198in}}%
\pgfpathlineto{\pgfqpoint{0.782646in}{1.496199in}}%
\pgfpathlineto{\pgfqpoint{0.849366in}{1.511347in}}%
\pgfpathlineto{\pgfqpoint{0.915540in}{1.528719in}}%
\pgfpathlineto{\pgfqpoint{0.981080in}{1.548348in}}%
\pgfpathlineto{\pgfqpoint{1.045911in}{1.570207in}}%
\pgfpathlineto{\pgfqpoint{1.109978in}{1.594215in}}%
\pgfpathlineto{\pgfqpoint{1.173256in}{1.620235in}}%
\pgfusepath{stroke}%
\end{pgfscope}%
\begin{pgfscope}%
\pgfpathrectangle{\pgfqpoint{0.647939in}{0.492442in}}{\pgfqpoint{3.079299in}{3.079299in}}%
\pgfusepath{clip}%
\pgfsetbuttcap%
\pgfsetroundjoin%
\pgfsetlinewidth{0.301125pt}%
\definecolor{currentstroke}{rgb}{0.500000,0.500000,0.500000}%
\pgfsetstrokecolor{currentstroke}%
\pgfsetstrokeopacity{0.300000}%
\pgfsetdash{}{0pt}%
\pgfpathmoveto{\pgfqpoint{0.647939in}{1.402235in}}%
\pgfpathlineto{\pgfqpoint{0.647939in}{1.402235in}}%
\pgfpathlineto{\pgfqpoint{0.715443in}{1.413398in}}%
\pgfpathlineto{\pgfqpoint{0.782568in}{1.426641in}}%
\pgfpathlineto{\pgfqpoint{0.849216in}{1.442097in}}%
\pgfpathlineto{\pgfqpoint{0.915286in}{1.459857in}}%
\pgfpathlineto{\pgfqpoint{0.980680in}{1.479963in}}%
\pgfpathlineto{\pgfqpoint{1.045311in}{1.502400in}}%
\pgfpathlineto{\pgfqpoint{1.109114in}{1.527097in}}%
\pgfpathlineto{\pgfqpoint{1.172052in}{1.553925in}}%
\pgfusepath{stroke}%
\end{pgfscope}%
\begin{pgfscope}%
\pgfpathrectangle{\pgfqpoint{0.647939in}{0.492442in}}{\pgfqpoint{3.079299in}{3.079299in}}%
\pgfusepath{clip}%
\pgfsetbuttcap%
\pgfsetroundjoin%
\pgfsetlinewidth{0.301125pt}%
\definecolor{currentstroke}{rgb}{0.500000,0.500000,0.500000}%
\pgfsetstrokecolor{currentstroke}%
\pgfsetstrokeopacity{0.300000}%
\pgfsetdash{}{0pt}%
\pgfpathmoveto{\pgfqpoint{0.647939in}{1.332251in}}%
\pgfpathlineto{\pgfqpoint{0.647939in}{1.332251in}}%
\pgfpathlineto{\pgfqpoint{0.715411in}{1.343606in}}%
\pgfpathlineto{\pgfqpoint{0.782485in}{1.357098in}}%
\pgfpathlineto{\pgfqpoint{0.849057in}{1.372875in}}%
\pgfpathlineto{\pgfqpoint{0.915016in}{1.391038in}}%
\pgfpathlineto{\pgfqpoint{0.980253in}{1.411643in}}%
\pgfpathlineto{\pgfqpoint{1.044669in}{1.434686in}}%
\pgfpathlineto{\pgfqpoint{1.108186in}{1.460105in}}%
\pgfpathlineto{\pgfqpoint{1.170754in}{1.487782in}}%
\pgfpathlineto{\pgfqpoint{1.232358in}{1.517545in}}%
\pgfpathlineto{\pgfqpoint{1.293025in}{1.549179in}}%
\pgfpathlineto{\pgfqpoint{1.352822in}{1.582430in}}%
\pgfpathlineto{\pgfqpoint{1.411858in}{1.617017in}}%
\pgfpathlineto{\pgfqpoint{1.470283in}{1.652630in}}%
\pgfpathlineto{\pgfqpoint{1.528280in}{1.688935in}}%
\pgfpathlineto{\pgfqpoint{1.586069in}{1.725572in}}%
\pgfpathlineto{\pgfqpoint{1.643898in}{1.762146in}}%
\pgfpathlineto{\pgfqpoint{1.702043in}{1.798218in}}%
\pgfpathlineto{\pgfqpoint{1.760786in}{1.833301in}}%
\pgfpathlineto{\pgfqpoint{1.820400in}{1.866864in}}%
\pgfpathlineto{\pgfqpoint{1.881135in}{1.898333in}}%
\pgfpathlineto{\pgfqpoint{1.943170in}{1.927132in}}%
\pgfpathlineto{\pgfqpoint{2.006554in}{1.952831in}}%
\pgfpathlineto{\pgfqpoint{2.070976in}{1.975657in}}%
\pgfpathlineto{\pgfqpoint{2.133664in}{1.999939in}}%
\pgfpathlineto{\pgfqpoint{2.133664in}{1.999939in}}%
\pgfusepath{stroke}%
\end{pgfscope}%
\begin{pgfscope}%
\pgfpathrectangle{\pgfqpoint{0.647939in}{0.492442in}}{\pgfqpoint{3.079299in}{3.079299in}}%
\pgfusepath{clip}%
\pgfsetbuttcap%
\pgfsetroundjoin%
\pgfsetlinewidth{0.301125pt}%
\definecolor{currentstroke}{rgb}{0.500000,0.500000,0.500000}%
\pgfsetstrokecolor{currentstroke}%
\pgfsetstrokeopacity{0.300000}%
\pgfsetdash{}{0pt}%
\pgfpathmoveto{\pgfqpoint{0.647939in}{1.262267in}}%
\pgfpathlineto{\pgfqpoint{0.647939in}{1.262267in}}%
\pgfpathlineto{\pgfqpoint{0.715377in}{1.273819in}}%
\pgfpathlineto{\pgfqpoint{0.782397in}{1.287571in}}%
\pgfpathlineto{\pgfqpoint{0.848888in}{1.303681in}}%
\pgfpathlineto{\pgfqpoint{0.914729in}{1.322266in}}%
\pgfpathlineto{\pgfqpoint{0.979797in}{1.343392in}}%
\pgfpathlineto{\pgfqpoint{1.043981in}{1.367069in}}%
\pgfpathlineto{\pgfqpoint{1.107187in}{1.393246in}}%
\pgfpathlineto{\pgfqpoint{1.169351in}{1.421814in}}%
\pgfpathlineto{\pgfqpoint{1.230442in}{1.452609in}}%
\pgfpathlineto{\pgfqpoint{1.290475in}{1.485424in}}%
\pgfpathlineto{\pgfqpoint{1.349504in}{1.520015in}}%
\pgfpathlineto{\pgfqpoint{1.407629in}{1.556108in}}%
\pgfpathlineto{\pgfqpoint{1.464988in}{1.593408in}}%
\pgfpathlineto{\pgfqpoint{1.521760in}{1.631597in}}%
\pgfusepath{stroke}%
\end{pgfscope}%
\begin{pgfscope}%
\pgfpathrectangle{\pgfqpoint{0.647939in}{0.492442in}}{\pgfqpoint{3.079299in}{3.079299in}}%
\pgfusepath{clip}%
\pgfsetbuttcap%
\pgfsetroundjoin%
\pgfsetlinewidth{0.301125pt}%
\definecolor{currentstroke}{rgb}{0.500000,0.500000,0.500000}%
\pgfsetstrokecolor{currentstroke}%
\pgfsetstrokeopacity{0.300000}%
\pgfsetdash{}{0pt}%
\pgfpathmoveto{\pgfqpoint{0.647939in}{1.192283in}}%
\pgfpathlineto{\pgfqpoint{0.647939in}{1.192283in}}%
\pgfpathlineto{\pgfqpoint{0.715341in}{1.204040in}}%
\pgfpathlineto{\pgfqpoint{0.782305in}{1.218061in}}%
\pgfpathlineto{\pgfqpoint{0.848710in}{1.234518in}}%
\pgfpathlineto{\pgfqpoint{0.914423in}{1.253542in}}%
\pgfpathlineto{\pgfqpoint{0.979310in}{1.275214in}}%
\pgfusepath{stroke}%
\end{pgfscope}%
\begin{pgfscope}%
\pgfpathrectangle{\pgfqpoint{0.647939in}{0.492442in}}{\pgfqpoint{3.079299in}{3.079299in}}%
\pgfusepath{clip}%
\pgfsetbuttcap%
\pgfsetroundjoin%
\pgfsetlinewidth{0.301125pt}%
\definecolor{currentstroke}{rgb}{0.500000,0.500000,0.500000}%
\pgfsetstrokecolor{currentstroke}%
\pgfsetstrokeopacity{0.300000}%
\pgfsetdash{}{0pt}%
\pgfpathmoveto{\pgfqpoint{0.647939in}{1.122299in}}%
\pgfpathlineto{\pgfqpoint{0.647939in}{1.122299in}}%
\pgfpathlineto{\pgfqpoint{0.715303in}{1.134267in}}%
\pgfpathlineto{\pgfqpoint{0.782207in}{1.148568in}}%
\pgfpathlineto{\pgfqpoint{0.848520in}{1.165387in}}%
\pgfpathlineto{\pgfqpoint{0.914097in}{1.184870in}}%
\pgfpathlineto{\pgfqpoint{0.978788in}{1.207113in}}%
\pgfpathlineto{\pgfqpoint{1.042449in}{1.232151in}}%
\pgfpathlineto{\pgfqpoint{1.104949in}{1.259958in}}%
\pgfpathlineto{\pgfqpoint{1.166189in}{1.290442in}}%
\pgfpathlineto{\pgfqpoint{1.226103in}{1.323454in}}%
\pgfpathlineto{\pgfqpoint{1.284672in}{1.358801in}}%
\pgfpathlineto{\pgfqpoint{1.341924in}{1.396249in}}%
\pgfpathlineto{\pgfqpoint{1.397930in}{1.435540in}}%
\pgfpathlineto{\pgfqpoint{1.452817in}{1.476384in}}%
\pgfpathlineto{\pgfqpoint{1.506762in}{1.518467in}}%
\pgfpathlineto{\pgfqpoint{1.559972in}{1.561470in}}%
\pgfpathlineto{\pgfqpoint{1.612692in}{1.605077in}}%
\pgfpathlineto{\pgfqpoint{1.665210in}{1.648925in}}%
\pgfpathlineto{\pgfqpoint{1.717838in}{1.692625in}}%
\pgfpathlineto{\pgfqpoint{1.770909in}{1.735791in}}%
\pgfpathlineto{\pgfqpoint{1.824753in}{1.777972in}}%
\pgfusepath{stroke}%
\end{pgfscope}%
\begin{pgfscope}%
\pgfpathrectangle{\pgfqpoint{0.647939in}{0.492442in}}{\pgfqpoint{3.079299in}{3.079299in}}%
\pgfusepath{clip}%
\pgfsetbuttcap%
\pgfsetroundjoin%
\pgfsetlinewidth{0.301125pt}%
\definecolor{currentstroke}{rgb}{0.500000,0.500000,0.500000}%
\pgfsetstrokecolor{currentstroke}%
\pgfsetstrokeopacity{0.300000}%
\pgfsetdash{}{0pt}%
\pgfpathmoveto{\pgfqpoint{0.647939in}{1.052315in}}%
\pgfpathlineto{\pgfqpoint{0.647939in}{1.052315in}}%
\pgfpathlineto{\pgfqpoint{0.715263in}{1.064503in}}%
\pgfpathlineto{\pgfqpoint{0.782103in}{1.079094in}}%
\pgfpathlineto{\pgfqpoint{0.848318in}{1.096289in}}%
\pgfpathlineto{\pgfqpoint{0.913749in}{1.116252in}}%
\pgfpathlineto{\pgfqpoint{0.978230in}{1.139093in}}%
\pgfpathlineto{\pgfqpoint{1.041595in}{1.164863in}}%
\pgfpathlineto{\pgfqpoint{1.103695in}{1.193545in}}%
\pgfpathlineto{\pgfqpoint{1.164407in}{1.225057in}}%
\pgfpathlineto{\pgfqpoint{1.223647in}{1.259257in}}%
\pgfpathlineto{\pgfqpoint{1.281376in}{1.295952in}}%
\pgfpathlineto{\pgfqpoint{1.337605in}{1.334913in}}%
\pgfusepath{stroke}%
\end{pgfscope}%
\begin{pgfscope}%
\pgfpathrectangle{\pgfqpoint{0.647939in}{0.492442in}}{\pgfqpoint{3.079299in}{3.079299in}}%
\pgfusepath{clip}%
\pgfsetbuttcap%
\pgfsetroundjoin%
\pgfsetlinewidth{0.301125pt}%
\definecolor{currentstroke}{rgb}{0.500000,0.500000,0.500000}%
\pgfsetstrokecolor{currentstroke}%
\pgfsetstrokeopacity{0.300000}%
\pgfsetdash{}{0pt}%
\pgfpathmoveto{\pgfqpoint{0.647939in}{0.982331in}}%
\pgfpathlineto{\pgfqpoint{0.647939in}{0.982331in}}%
\pgfpathlineto{\pgfqpoint{0.715221in}{0.994746in}}%
\pgfpathlineto{\pgfqpoint{0.781994in}{1.009639in}}%
\pgfpathlineto{\pgfqpoint{0.848104in}{1.027228in}}%
\pgfpathlineto{\pgfqpoint{0.913377in}{1.047692in}}%
\pgfpathlineto{\pgfqpoint{0.977630in}{1.071160in}}%
\pgfpathlineto{\pgfqpoint{1.040674in}{1.097696in}}%
\pgfpathlineto{\pgfqpoint{1.102338in}{1.127297in}}%
\pgfusepath{stroke}%
\end{pgfscope}%
\begin{pgfscope}%
\pgfpathrectangle{\pgfqpoint{0.647939in}{0.492442in}}{\pgfqpoint{3.079299in}{3.079299in}}%
\pgfusepath{clip}%
\pgfsetbuttcap%
\pgfsetroundjoin%
\pgfsetlinewidth{0.301125pt}%
\definecolor{currentstroke}{rgb}{0.500000,0.500000,0.500000}%
\pgfsetstrokecolor{currentstroke}%
\pgfsetstrokeopacity{0.300000}%
\pgfsetdash{}{0pt}%
\pgfpathmoveto{\pgfqpoint{0.647939in}{0.912347in}}%
\pgfpathlineto{\pgfqpoint{0.647939in}{0.912347in}}%
\pgfpathlineto{\pgfqpoint{0.715176in}{0.924998in}}%
\pgfpathlineto{\pgfqpoint{0.781877in}{0.940205in}}%
\pgfpathlineto{\pgfqpoint{0.847875in}{0.958204in}}%
\pgfpathlineto{\pgfqpoint{0.912980in}{0.979193in}}%
\pgfpathlineto{\pgfqpoint{0.976985in}{1.003318in}}%
\pgfpathlineto{\pgfqpoint{1.039681in}{1.030659in}}%
\pgfpathlineto{\pgfqpoint{1.100868in}{1.061223in}}%
\pgfpathlineto{\pgfqpoint{1.160374in}{1.094940in}}%
\pgfpathlineto{\pgfqpoint{1.218068in}{1.131671in}}%
\pgfpathlineto{\pgfqpoint{1.273872in}{1.171219in}}%
\pgfpathlineto{\pgfqpoint{1.327775in}{1.213331in}}%
\pgfpathlineto{\pgfqpoint{1.379837in}{1.257692in}}%
\pgfpathlineto{\pgfqpoint{1.430179in}{1.303997in}}%
\pgfpathlineto{\pgfqpoint{1.478987in}{1.351921in}}%
\pgfpathlineto{\pgfqpoint{1.526506in}{1.401117in}}%
\pgfpathlineto{\pgfqpoint{1.573030in}{1.451257in}}%
\pgfpathlineto{\pgfqpoint{1.618889in}{1.501997in}}%
\pgfusepath{stroke}%
\end{pgfscope}%
\begin{pgfscope}%
\pgfpathrectangle{\pgfqpoint{0.647939in}{0.492442in}}{\pgfqpoint{3.079299in}{3.079299in}}%
\pgfusepath{clip}%
\pgfsetbuttcap%
\pgfsetroundjoin%
\pgfsetlinewidth{0.301125pt}%
\definecolor{currentstroke}{rgb}{0.500000,0.500000,0.500000}%
\pgfsetstrokecolor{currentstroke}%
\pgfsetstrokeopacity{0.300000}%
\pgfsetdash{}{0pt}%
\pgfpathmoveto{\pgfqpoint{0.647939in}{0.842362in}}%
\pgfpathlineto{\pgfqpoint{0.647939in}{0.842362in}}%
\pgfpathlineto{\pgfqpoint{0.715129in}{0.855259in}}%
\pgfpathlineto{\pgfqpoint{0.781753in}{0.870793in}}%
\pgfpathlineto{\pgfqpoint{0.847631in}{0.889220in}}%
\pgfpathlineto{\pgfqpoint{0.912553in}{0.910759in}}%
\pgfpathlineto{\pgfqpoint{0.976291in}{0.935574in}}%
\pgfpathlineto{\pgfqpoint{1.038608in}{0.963759in}}%
\pgfpathlineto{\pgfqpoint{1.099276in}{0.995332in}}%
\pgfpathlineto{\pgfqpoint{1.158093in}{1.030226in}}%
\pgfpathlineto{\pgfqpoint{1.214905in}{1.068299in}}%
\pgfpathlineto{\pgfqpoint{1.269616in}{1.109343in}}%
\pgfpathlineto{\pgfqpoint{1.322214in}{1.153066in}}%
\pgfusepath{stroke}%
\end{pgfscope}%
\begin{pgfscope}%
\pgfpathrectangle{\pgfqpoint{0.647939in}{0.492442in}}{\pgfqpoint{3.079299in}{3.079299in}}%
\pgfusepath{clip}%
\pgfsetbuttcap%
\pgfsetroundjoin%
\pgfsetlinewidth{0.301125pt}%
\definecolor{currentstroke}{rgb}{0.500000,0.500000,0.500000}%
\pgfsetstrokecolor{currentstroke}%
\pgfsetstrokeopacity{0.300000}%
\pgfsetdash{}{0pt}%
\pgfpathmoveto{\pgfqpoint{0.647939in}{0.772378in}}%
\pgfpathlineto{\pgfqpoint{0.647939in}{0.772378in}}%
\pgfpathlineto{\pgfqpoint{0.715078in}{0.785529in}}%
\pgfpathlineto{\pgfqpoint{0.781621in}{0.801404in}}%
\pgfpathlineto{\pgfqpoint{0.847370in}{0.820280in}}%
\pgfpathlineto{\pgfqpoint{0.912096in}{0.842394in}}%
\pgfpathlineto{\pgfqpoint{0.975544in}{0.867932in}}%
\pgfpathlineto{\pgfqpoint{1.037448in}{0.897003in}}%
\pgfpathlineto{\pgfqpoint{1.097548in}{0.929633in}}%
\pgfpathlineto{\pgfqpoint{1.155613in}{0.965754in}}%
\pgfpathlineto{\pgfqpoint{1.211464in}{1.005216in}}%
\pgfpathlineto{\pgfqpoint{1.264996in}{1.047779in}}%
\pgfusepath{stroke}%
\end{pgfscope}%
\begin{pgfscope}%
\pgfpathrectangle{\pgfqpoint{0.647939in}{0.492442in}}{\pgfqpoint{3.079299in}{3.079299in}}%
\pgfusepath{clip}%
\pgfsetbuttcap%
\pgfsetroundjoin%
\pgfsetlinewidth{0.301125pt}%
\definecolor{currentstroke}{rgb}{0.500000,0.500000,0.500000}%
\pgfsetstrokecolor{currentstroke}%
\pgfsetstrokeopacity{0.300000}%
\pgfsetdash{}{0pt}%
\pgfpathmoveto{\pgfqpoint{0.647939in}{0.702394in}}%
\pgfpathlineto{\pgfqpoint{0.647939in}{0.702394in}}%
\pgfpathlineto{\pgfqpoint{0.715025in}{0.715809in}}%
\pgfpathlineto{\pgfqpoint{0.781481in}{0.732040in}}%
\pgfpathlineto{\pgfqpoint{0.847091in}{0.751384in}}%
\pgfpathlineto{\pgfqpoint{0.911604in}{0.774102in}}%
\pgfpathlineto{\pgfqpoint{0.974737in}{0.800399in}}%
\pgfpathlineto{\pgfqpoint{1.036191in}{0.830399in}}%
\pgfpathlineto{\pgfqpoint{1.095671in}{0.864133in}}%
\pgfpathlineto{\pgfqpoint{1.152916in}{0.901532in}}%
\pgfpathlineto{\pgfqpoint{1.207724in}{0.942427in}}%
\pgfpathlineto{\pgfqpoint{1.259986in}{0.986532in}}%
\pgfusepath{stroke}%
\end{pgfscope}%
\begin{pgfscope}%
\pgfpathrectangle{\pgfqpoint{0.647939in}{0.492442in}}{\pgfqpoint{3.079299in}{3.079299in}}%
\pgfusepath{clip}%
\pgfsetbuttcap%
\pgfsetroundjoin%
\pgfsetlinewidth{0.301125pt}%
\definecolor{currentstroke}{rgb}{0.500000,0.500000,0.500000}%
\pgfsetstrokecolor{currentstroke}%
\pgfsetstrokeopacity{0.300000}%
\pgfsetdash{}{0pt}%
\pgfpathmoveto{\pgfqpoint{0.647939in}{0.632410in}}%
\pgfpathlineto{\pgfqpoint{0.647939in}{0.632410in}}%
\pgfpathlineto{\pgfqpoint{0.714969in}{0.646100in}}%
\pgfpathlineto{\pgfqpoint{0.781331in}{0.662702in}}%
\pgfpathlineto{\pgfqpoint{0.846792in}{0.682536in}}%
\pgfpathlineto{\pgfqpoint{0.911076in}{0.705887in}}%
\pgfpathlineto{\pgfqpoint{0.973866in}{0.732981in}}%
\pgfpathlineto{\pgfqpoint{1.034828in}{0.763955in}}%
\pgfpathlineto{\pgfqpoint{1.093631in}{0.798843in}}%
\pgfpathlineto{\pgfqpoint{1.149983in}{0.837566in}}%
\pgfusepath{stroke}%
\end{pgfscope}%
\begin{pgfscope}%
\pgfpathrectangle{\pgfqpoint{0.647939in}{0.492442in}}{\pgfqpoint{3.079299in}{3.079299in}}%
\pgfusepath{clip}%
\pgfsetbuttcap%
\pgfsetroundjoin%
\pgfsetlinewidth{0.301125pt}%
\definecolor{currentstroke}{rgb}{0.500000,0.500000,0.500000}%
\pgfsetstrokecolor{currentstroke}%
\pgfsetstrokeopacity{0.300000}%
\pgfsetdash{}{0pt}%
\pgfpathmoveto{\pgfqpoint{3.727238in}{1.769203in}}%
\pgfpathlineto{\pgfqpoint{3.711563in}{1.780556in}}%
\pgfpathlineto{\pgfqpoint{3.657254in}{1.822139in}}%
\pgfpathlineto{\pgfqpoint{3.605202in}{1.866520in}}%
\pgfpathlineto{\pgfqpoint{3.555495in}{1.913511in}}%
\pgfpathlineto{\pgfqpoint{3.508236in}{1.962965in}}%
\pgfpathlineto{\pgfqpoint{3.463563in}{2.014767in}}%
\pgfusepath{stroke}%
\end{pgfscope}%
\begin{pgfscope}%
\pgfpathrectangle{\pgfqpoint{0.647939in}{0.492442in}}{\pgfqpoint{3.079299in}{3.079299in}}%
\pgfusepath{clip}%
\pgfsetbuttcap%
\pgfsetroundjoin%
\pgfsetlinewidth{0.301125pt}%
\definecolor{currentstroke}{rgb}{0.500000,0.500000,0.500000}%
\pgfsetstrokecolor{currentstroke}%
\pgfsetstrokeopacity{0.300000}%
\pgfsetdash{}{0pt}%
\pgfpathmoveto{\pgfqpoint{1.688522in}{0.601660in}}%
\pgfpathlineto{\pgfqpoint{1.627716in}{0.632410in}}%
\pgfpathlineto{\pgfqpoint{1.574466in}{0.674465in}}%
\pgfpathlineto{\pgfqpoint{1.536199in}{0.723048in}}%
\pgfpathlineto{\pgfqpoint{1.512234in}{0.773883in}}%
\pgfpathlineto{\pgfqpoint{1.498652in}{0.828321in}}%
\pgfpathlineto{\pgfqpoint{1.493811in}{0.888250in}}%
\pgfpathlineto{\pgfqpoint{1.497504in}{0.956020in}}%
\pgfpathlineto{\pgfqpoint{1.508141in}{1.023338in}}%
\pgfusepath{stroke}%
\end{pgfscope}%
\begin{pgfscope}%
\pgfpathrectangle{\pgfqpoint{0.647939in}{0.492442in}}{\pgfqpoint{3.079299in}{3.079299in}}%
\pgfusepath{clip}%
\pgfsetbuttcap%
\pgfsetroundjoin%
\pgfsetlinewidth{0.301125pt}%
\definecolor{currentstroke}{rgb}{0.500000,0.500000,0.500000}%
\pgfsetstrokecolor{currentstroke}%
\pgfsetstrokeopacity{0.300000}%
\pgfsetdash{}{0pt}%
\pgfpathmoveto{\pgfqpoint{3.587270in}{1.472219in}}%
\pgfpathlineto{\pgfqpoint{3.529489in}{1.508858in}}%
\pgfpathlineto{\pgfqpoint{3.472861in}{1.547258in}}%
\pgfpathlineto{\pgfqpoint{3.417330in}{1.587232in}}%
\pgfpathlineto{\pgfqpoint{3.362833in}{1.628604in}}%
\pgfpathlineto{\pgfqpoint{3.309306in}{1.671226in}}%
\pgfusepath{stroke}%
\end{pgfscope}%
\begin{pgfscope}%
\pgfpathrectangle{\pgfqpoint{0.647939in}{0.492442in}}{\pgfqpoint{3.079299in}{3.079299in}}%
\pgfusepath{clip}%
\pgfsetbuttcap%
\pgfsetroundjoin%
\pgfsetlinewidth{0.301125pt}%
\definecolor{currentstroke}{rgb}{0.500000,0.500000,0.500000}%
\pgfsetstrokecolor{currentstroke}%
\pgfsetstrokeopacity{0.300000}%
\pgfsetdash{}{0pt}%
\pgfpathmoveto{\pgfqpoint{3.510872in}{3.108402in}}%
\pgfpathlineto{\pgfqpoint{3.547502in}{3.166168in}}%
\pgfpathlineto{\pgfqpoint{3.587270in}{3.221821in}}%
\pgfpathlineto{\pgfqpoint{3.630008in}{3.275222in}}%
\pgfpathlineto{\pgfqpoint{3.675610in}{3.326200in}}%
\pgfpathlineto{\pgfqpoint{3.723999in}{3.374539in}}%
\pgfpathlineto{\pgfqpoint{3.727238in}{3.377597in}}%
\pgfusepath{stroke}%
\end{pgfscope}%
\begin{pgfscope}%
\pgfpathrectangle{\pgfqpoint{0.647939in}{0.492442in}}{\pgfqpoint{3.079299in}{3.079299in}}%
\pgfusepath{clip}%
\pgfsetbuttcap%
\pgfsetroundjoin%
\pgfsetlinewidth{0.301125pt}%
\definecolor{currentstroke}{rgb}{0.500000,0.500000,0.500000}%
\pgfsetstrokecolor{currentstroke}%
\pgfsetstrokeopacity{0.300000}%
\pgfsetdash{}{0pt}%
\pgfpathmoveto{\pgfqpoint{3.517286in}{1.752155in}}%
\pgfpathlineto{\pgfqpoint{3.465751in}{1.797152in}}%
\pgfpathlineto{\pgfqpoint{3.415988in}{1.844100in}}%
\pgfpathlineto{\pgfqpoint{3.368048in}{1.892910in}}%
\pgfpathlineto{\pgfqpoint{3.322027in}{1.943531in}}%
\pgfpathlineto{\pgfqpoint{3.278072in}{1.995952in}}%
\pgfpathlineto{\pgfqpoint{3.236397in}{2.050203in}}%
\pgfpathlineto{\pgfqpoint{3.197321in}{2.106349in}}%
\pgfpathlineto{\pgfqpoint{3.161275in}{2.164477in}}%
\pgfpathlineto{\pgfqpoint{3.128834in}{2.224678in}}%
\pgfpathlineto{\pgfqpoint{3.100738in}{2.287008in}}%
\pgfpathlineto{\pgfqpoint{3.077867in}{2.351420in}}%
\pgfpathlineto{\pgfqpoint{3.061147in}{2.417678in}}%
\pgfpathlineto{\pgfqpoint{3.051375in}{2.485300in}}%
\pgfpathlineto{\pgfqpoint{3.048984in}{2.553570in}}%
\pgfpathlineto{\pgfqpoint{3.053892in}{2.621705in}}%
\pgfpathlineto{\pgfqpoint{3.065533in}{2.689032in}}%
\pgfpathlineto{\pgfqpoint{3.083052in}{2.755098in}}%
\pgfpathlineto{\pgfqpoint{3.105528in}{2.819670in}}%
\pgfusepath{stroke}%
\end{pgfscope}%
\begin{pgfscope}%
\pgfpathrectangle{\pgfqpoint{0.647939in}{0.492442in}}{\pgfqpoint{3.079299in}{3.079299in}}%
\pgfusepath{clip}%
\pgfsetbuttcap%
\pgfsetroundjoin%
\pgfsetlinewidth{0.301125pt}%
\definecolor{currentstroke}{rgb}{0.500000,0.500000,0.500000}%
\pgfsetstrokecolor{currentstroke}%
\pgfsetstrokeopacity{0.300000}%
\pgfsetdash{}{0pt}%
\pgfpathmoveto{\pgfqpoint{2.381647in}{3.167846in}}%
\pgfpathlineto{\pgfqpoint{2.449840in}{3.173310in}}%
\pgfpathlineto{\pgfqpoint{2.517664in}{3.182171in}}%
\pgfpathlineto{\pgfqpoint{2.584804in}{3.195168in}}%
\pgfpathlineto{\pgfqpoint{2.650884in}{3.212742in}}%
\pgfpathlineto{\pgfqpoint{2.715549in}{3.234960in}}%
\pgfpathlineto{\pgfqpoint{2.778551in}{3.261543in}}%
\pgfpathlineto{\pgfqpoint{2.839802in}{3.291967in}}%
\pgfpathlineto{\pgfqpoint{2.899364in}{3.325596in}}%
\pgfpathlineto{\pgfqpoint{2.957413in}{3.361789in}}%
\pgfusepath{stroke}%
\end{pgfscope}%
\begin{pgfscope}%
\pgfpathrectangle{\pgfqpoint{0.647939in}{0.492442in}}{\pgfqpoint{3.079299in}{3.079299in}}%
\pgfusepath{clip}%
\pgfsetbuttcap%
\pgfsetroundjoin%
\pgfsetlinewidth{0.301125pt}%
\definecolor{currentstroke}{rgb}{0.500000,0.500000,0.500000}%
\pgfsetstrokecolor{currentstroke}%
\pgfsetstrokeopacity{0.300000}%
\pgfsetdash{}{0pt}%
\pgfpathmoveto{\pgfqpoint{2.459580in}{0.763858in}}%
\pgfpathlineto{\pgfqpoint{2.391261in}{0.767632in}}%
\pgfpathlineto{\pgfqpoint{2.322873in}{0.769892in}}%
\pgfpathlineto{\pgfqpoint{2.254456in}{0.771073in}}%
\pgfpathlineto{\pgfqpoint{2.186030in}{0.771696in}}%
\pgfpathlineto{\pgfqpoint{2.117605in}{0.772378in}}%
\pgfusepath{stroke}%
\end{pgfscope}%
\begin{pgfscope}%
\pgfpathrectangle{\pgfqpoint{0.647939in}{0.492442in}}{\pgfqpoint{3.079299in}{3.079299in}}%
\pgfusepath{clip}%
\pgfsetbuttcap%
\pgfsetroundjoin%
\pgfsetlinewidth{0.301125pt}%
\definecolor{currentstroke}{rgb}{0.500000,0.500000,0.500000}%
\pgfsetstrokecolor{currentstroke}%
\pgfsetstrokeopacity{0.300000}%
\pgfsetdash{}{0pt}%
\pgfpathmoveto{\pgfqpoint{3.447302in}{1.402235in}}%
\pgfpathlineto{\pgfqpoint{3.389623in}{1.439046in}}%
\pgfpathlineto{\pgfqpoint{3.332624in}{1.476903in}}%
\pgfpathlineto{\pgfqpoint{3.276203in}{1.515618in}}%
\pgfpathlineto{\pgfqpoint{3.220250in}{1.555008in}}%
\pgfpathlineto{\pgfqpoint{3.164655in}{1.594899in}}%
\pgfpathlineto{\pgfqpoint{3.109298in}{1.635122in}}%
\pgfpathlineto{\pgfqpoint{3.054056in}{1.675502in}}%
\pgfpathlineto{\pgfqpoint{2.998809in}{1.715873in}}%
\pgfpathlineto{\pgfqpoint{2.943434in}{1.756067in}}%
\pgfpathlineto{\pgfqpoint{2.887805in}{1.795909in}}%
\pgfpathlineto{\pgfqpoint{2.831797in}{1.835215in}}%
\pgfpathlineto{\pgfqpoint{2.775290in}{1.873791in}}%
\pgfpathlineto{\pgfqpoint{2.718169in}{1.911443in}}%
\pgfpathlineto{\pgfqpoint{2.660330in}{1.947978in}}%
\pgfpathlineto{\pgfqpoint{2.601694in}{1.983226in}}%
\pgfpathlineto{\pgfqpoint{2.542247in}{2.017092in}}%
\pgfpathlineto{\pgfqpoint{2.482107in}{2.049716in}}%
\pgfpathlineto{\pgfqpoint{2.421794in}{2.081997in}}%
\pgfpathlineto{\pgfqpoint{2.364226in}{2.118389in}}%
\pgfpathlineto{\pgfqpoint{2.364226in}{2.118389in}}%
\pgfpathlineto{\pgfqpoint{2.346329in}{2.136329in}}%
\pgfpathlineto{\pgfqpoint{2.346329in}{2.136329in}}%
\pgfusepath{stroke}%
\end{pgfscope}%
\begin{pgfscope}%
\pgfpathrectangle{\pgfqpoint{0.647939in}{0.492442in}}{\pgfqpoint{3.079299in}{3.079299in}}%
\pgfusepath{clip}%
\pgfsetbuttcap%
\pgfsetroundjoin%
\pgfsetlinewidth{0.301125pt}%
\definecolor{currentstroke}{rgb}{0.500000,0.500000,0.500000}%
\pgfsetstrokecolor{currentstroke}%
\pgfsetstrokeopacity{0.300000}%
\pgfsetdash{}{0pt}%
\pgfpathmoveto{\pgfqpoint{3.447302in}{2.312028in}}%
\pgfpathlineto{\pgfqpoint{3.422882in}{2.375884in}}%
\pgfpathlineto{\pgfqpoint{3.403672in}{2.441489in}}%
\pgfpathlineto{\pgfqpoint{3.390068in}{2.508477in}}%
\pgfpathlineto{\pgfqpoint{3.382357in}{2.576399in}}%
\pgfpathlineto{\pgfqpoint{3.380670in}{2.644734in}}%
\pgfpathlineto{\pgfqpoint{3.384948in}{2.712953in}}%
\pgfpathlineto{\pgfqpoint{3.394957in}{2.780573in}}%
\pgfpathlineto{\pgfqpoint{3.410338in}{2.847184in}}%
\pgfpathlineto{\pgfqpoint{3.430658in}{2.912469in}}%
\pgfpathlineto{\pgfqpoint{3.455478in}{2.976189in}}%
\pgfusepath{stroke}%
\end{pgfscope}%
\begin{pgfscope}%
\pgfpathrectangle{\pgfqpoint{0.647939in}{0.492442in}}{\pgfqpoint{3.079299in}{3.079299in}}%
\pgfusepath{clip}%
\pgfsetbuttcap%
\pgfsetroundjoin%
\pgfsetlinewidth{0.301125pt}%
\definecolor{currentstroke}{rgb}{0.500000,0.500000,0.500000}%
\pgfsetstrokecolor{currentstroke}%
\pgfsetstrokeopacity{0.300000}%
\pgfsetdash{}{0pt}%
\pgfpathmoveto{\pgfqpoint{1.714601in}{3.258561in}}%
\pgfpathlineto{\pgfqpoint{1.782477in}{3.267187in}}%
\pgfpathlineto{\pgfqpoint{1.850553in}{3.274064in}}%
\pgfpathlineto{\pgfqpoint{1.918783in}{3.279197in}}%
\pgfpathlineto{\pgfqpoint{1.987117in}{3.282705in}}%
\pgfpathlineto{\pgfqpoint{2.055510in}{3.284834in}}%
\pgfpathlineto{\pgfqpoint{2.123928in}{3.285957in}}%
\pgfpathlineto{\pgfqpoint{2.192354in}{3.286562in}}%
\pgfpathlineto{\pgfqpoint{2.260779in}{3.287251in}}%
\pgfpathlineto{\pgfqpoint{2.329189in}{3.288734in}}%
\pgfpathlineto{\pgfqpoint{2.397541in}{3.291805in}}%
\pgfusepath{stroke}%
\end{pgfscope}%
\begin{pgfscope}%
\pgfpathrectangle{\pgfqpoint{0.647939in}{0.492442in}}{\pgfqpoint{3.079299in}{3.079299in}}%
\pgfusepath{clip}%
\pgfsetbuttcap%
\pgfsetroundjoin%
\pgfsetlinewidth{0.301125pt}%
\definecolor{currentstroke}{rgb}{0.500000,0.500000,0.500000}%
\pgfsetstrokecolor{currentstroke}%
\pgfsetstrokeopacity{0.300000}%
\pgfsetdash{}{0pt}%
\pgfpathmoveto{\pgfqpoint{2.814015in}{0.826504in}}%
\pgfpathlineto{\pgfqpoint{2.747461in}{0.842362in}}%
\pgfpathlineto{\pgfqpoint{2.680413in}{0.855976in}}%
\pgfpathlineto{\pgfqpoint{2.612931in}{0.867243in}}%
\pgfpathlineto{\pgfqpoint{2.545095in}{0.876150in}}%
\pgfpathlineto{\pgfqpoint{2.476999in}{0.882795in}}%
\pgfpathlineto{\pgfqpoint{2.408732in}{0.887388in}}%
\pgfpathlineto{\pgfqpoint{2.340368in}{0.890248in}}%
\pgfpathlineto{\pgfqpoint{2.271959in}{0.891798in}}%
\pgfpathlineto{\pgfqpoint{2.203535in}{0.892571in}}%
\pgfpathlineto{\pgfqpoint{2.135110in}{0.893228in}}%
\pgfpathlineto{\pgfqpoint{2.066697in}{0.894567in}}%
\pgfpathlineto{\pgfqpoint{1.998344in}{0.897553in}}%
\pgfpathlineto{\pgfqpoint{1.930194in}{0.903415in}}%
\pgfpathlineto{\pgfqpoint{1.862617in}{0.913796in}}%
\pgfpathlineto{\pgfqpoint{1.796532in}{0.931004in}}%
\pgfpathlineto{\pgfqpoint{1.734118in}{0.958211in}}%
\pgfpathlineto{\pgfqpoint{1.734118in}{0.958211in}}%
\pgfpathlineto{\pgfqpoint{1.688103in}{0.990829in}}%
\pgfusepath{stroke}%
\end{pgfscope}%
\begin{pgfscope}%
\pgfpathrectangle{\pgfqpoint{0.647939in}{0.492442in}}{\pgfqpoint{3.079299in}{3.079299in}}%
\pgfusepath{clip}%
\pgfsetbuttcap%
\pgfsetroundjoin%
\pgfsetlinewidth{0.301125pt}%
\definecolor{currentstroke}{rgb}{0.500000,0.500000,0.500000}%
\pgfsetstrokecolor{currentstroke}%
\pgfsetstrokeopacity{0.300000}%
\pgfsetdash{}{0pt}%
\pgfpathmoveto{\pgfqpoint{3.377318in}{2.102076in}}%
\pgfpathlineto{\pgfqpoint{3.340581in}{2.159766in}}%
\pgfpathlineto{\pgfqpoint{3.307417in}{2.219574in}}%
\pgfpathlineto{\pgfqpoint{3.278312in}{2.281449in}}%
\pgfpathlineto{\pgfqpoint{3.253829in}{2.345283in}}%
\pgfpathlineto{\pgfqpoint{3.234568in}{2.410871in}}%
\pgfpathlineto{\pgfqpoint{3.221087in}{2.477878in}}%
\pgfpathlineto{\pgfqpoint{3.213805in}{2.545838in}}%
\pgfpathlineto{\pgfqpoint{3.212898in}{2.614178in}}%
\pgfpathlineto{\pgfqpoint{3.218255in}{2.682311in}}%
\pgfpathlineto{\pgfqpoint{3.229501in}{2.749731in}}%
\pgfpathlineto{\pgfqpoint{3.246101in}{2.816049in}}%
\pgfusepath{stroke}%
\end{pgfscope}%
\begin{pgfscope}%
\pgfpathrectangle{\pgfqpoint{0.647939in}{0.492442in}}{\pgfqpoint{3.079299in}{3.079299in}}%
\pgfusepath{clip}%
\pgfsetbuttcap%
\pgfsetroundjoin%
\pgfsetlinewidth{0.301125pt}%
\definecolor{currentstroke}{rgb}{0.500000,0.500000,0.500000}%
\pgfsetstrokecolor{currentstroke}%
\pgfsetstrokeopacity{0.300000}%
\pgfsetdash{}{0pt}%
\pgfpathmoveto{\pgfqpoint{1.016936in}{3.132957in}}%
\pgfpathlineto{\pgfqpoint{1.083941in}{3.146837in}}%
\pgfpathlineto{\pgfqpoint{1.150802in}{3.161402in}}%
\pgfpathlineto{\pgfqpoint{1.217558in}{3.176436in}}%
\pgfpathlineto{\pgfqpoint{1.284264in}{3.191695in}}%
\pgfpathlineto{\pgfqpoint{1.350978in}{3.206917in}}%
\pgfpathlineto{\pgfqpoint{1.417764in}{3.221821in}}%
\pgfpathlineto{\pgfqpoint{1.484681in}{3.236118in}}%
\pgfusepath{stroke}%
\end{pgfscope}%
\begin{pgfscope}%
\pgfpathrectangle{\pgfqpoint{0.647939in}{0.492442in}}{\pgfqpoint{3.079299in}{3.079299in}}%
\pgfusepath{clip}%
\pgfsetbuttcap%
\pgfsetroundjoin%
\pgfsetlinewidth{0.301125pt}%
\definecolor{currentstroke}{rgb}{0.500000,0.500000,0.500000}%
\pgfsetstrokecolor{currentstroke}%
\pgfsetstrokeopacity{0.300000}%
\pgfsetdash{}{0pt}%
\pgfpathmoveto{\pgfqpoint{3.237350in}{1.752155in}}%
\pgfpathlineto{\pgfqpoint{3.186513in}{1.797953in}}%
\pgfpathlineto{\pgfqpoint{3.136614in}{1.844770in}}%
\pgfpathlineto{\pgfqpoint{3.087683in}{1.892598in}}%
\pgfpathlineto{\pgfqpoint{3.039802in}{1.941473in}}%
\pgfpathlineto{\pgfqpoint{2.993128in}{1.991498in}}%
\pgfpathlineto{\pgfqpoint{2.947927in}{2.042854in}}%
\pgfpathlineto{\pgfqpoint{2.904644in}{2.095826in}}%
\pgfpathlineto{\pgfqpoint{2.864005in}{2.150830in}}%
\pgfpathlineto{\pgfqpoint{2.827188in}{2.208435in}}%
\pgfpathlineto{\pgfqpoint{2.796082in}{2.269256in}}%
\pgfpathlineto{\pgfqpoint{2.773419in}{2.333592in}}%
\pgfpathlineto{\pgfqpoint{2.762150in}{2.400727in}}%
\pgfpathlineto{\pgfqpoint{2.763630in}{2.468746in}}%
\pgfpathlineto{\pgfqpoint{2.776536in}{2.535647in}}%
\pgfpathlineto{\pgfqpoint{2.798052in}{2.600424in}}%
\pgfpathlineto{\pgfqpoint{2.825523in}{2.662994in}}%
\pgfpathlineto{\pgfqpoint{2.857050in}{2.723670in}}%
\pgfpathlineto{\pgfqpoint{2.891407in}{2.782805in}}%
\pgfpathlineto{\pgfqpoint{2.927824in}{2.840698in}}%
\pgfpathlineto{\pgfqpoint{2.965822in}{2.897570in}}%
\pgfusepath{stroke}%
\end{pgfscope}%
\begin{pgfscope}%
\pgfpathrectangle{\pgfqpoint{0.647939in}{0.492442in}}{\pgfqpoint{3.079299in}{3.079299in}}%
\pgfusepath{clip}%
\pgfsetbuttcap%
\pgfsetroundjoin%
\pgfsetlinewidth{0.301125pt}%
\definecolor{currentstroke}{rgb}{0.500000,0.500000,0.500000}%
\pgfsetstrokecolor{currentstroke}%
\pgfsetstrokeopacity{0.300000}%
\pgfsetdash{}{0pt}%
\pgfpathmoveto{\pgfqpoint{2.699246in}{3.013187in}}%
\pgfpathlineto{\pgfqpoint{2.759478in}{3.045562in}}%
\pgfpathlineto{\pgfqpoint{2.817445in}{3.081853in}}%
\pgfpathlineto{\pgfqpoint{2.873407in}{3.121167in}}%
\pgfpathlineto{\pgfqpoint{2.927725in}{3.162741in}}%
\pgfpathlineto{\pgfqpoint{2.980760in}{3.205961in}}%
\pgfusepath{stroke}%
\end{pgfscope}%
\begin{pgfscope}%
\pgfpathrectangle{\pgfqpoint{0.647939in}{0.492442in}}{\pgfqpoint{3.079299in}{3.079299in}}%
\pgfusepath{clip}%
\pgfsetbuttcap%
\pgfsetroundjoin%
\pgfsetlinewidth{0.301125pt}%
\definecolor{currentstroke}{rgb}{0.500000,0.500000,0.500000}%
\pgfsetstrokecolor{currentstroke}%
\pgfsetstrokeopacity{0.300000}%
\pgfsetdash{}{0pt}%
\pgfpathmoveto{\pgfqpoint{2.666991in}{2.722543in}}%
\pgfpathlineto{\pgfqpoint{2.719214in}{2.766666in}}%
\pgfpathlineto{\pgfqpoint{2.769178in}{2.813348in}}%
\pgfpathlineto{\pgfqpoint{2.817536in}{2.861711in}}%
\pgfpathlineto{\pgfqpoint{2.864790in}{2.911168in}}%
\pgfpathlineto{\pgfqpoint{2.911318in}{2.961314in}}%
\pgfpathlineto{\pgfqpoint{2.957413in}{3.011869in}}%
\pgfusepath{stroke}%
\end{pgfscope}%
\begin{pgfscope}%
\pgfpathrectangle{\pgfqpoint{0.647939in}{0.492442in}}{\pgfqpoint{3.079299in}{3.079299in}}%
\pgfusepath{clip}%
\pgfsetbuttcap%
\pgfsetroundjoin%
\pgfsetlinewidth{0.301125pt}%
\definecolor{currentstroke}{rgb}{0.500000,0.500000,0.500000}%
\pgfsetstrokecolor{currentstroke}%
\pgfsetstrokeopacity{0.300000}%
\pgfsetdash{}{0pt}%
\pgfpathmoveto{\pgfqpoint{3.200112in}{1.731736in}}%
\pgfpathlineto{\pgfqpoint{3.148395in}{1.776539in}}%
\pgfpathlineto{\pgfqpoint{3.097382in}{1.822139in}}%
\pgfpathlineto{\pgfqpoint{3.047072in}{1.868515in}}%
\pgfpathlineto{\pgfqpoint{2.997501in}{1.915677in}}%
\pgfpathlineto{\pgfqpoint{2.948761in}{1.963696in}}%
\pgfpathlineto{\pgfqpoint{2.901038in}{2.012721in}}%
\pgfpathlineto{\pgfqpoint{2.854660in}{2.063013in}}%
\pgfpathlineto{\pgfqpoint{2.810213in}{2.115005in}}%
\pgfpathlineto{\pgfqpoint{2.768737in}{2.169372in}}%
\pgfpathlineto{\pgfqpoint{2.732120in}{2.227049in}}%
\pgfpathlineto{\pgfqpoint{2.703645in}{2.288984in}}%
\pgfpathlineto{\pgfqpoint{2.687928in}{2.355068in}}%
\pgfpathlineto{\pgfqpoint{2.687425in}{2.418096in}}%
\pgfpathlineto{\pgfqpoint{2.698496in}{2.477199in}}%
\pgfusepath{stroke}%
\end{pgfscope}%
\begin{pgfscope}%
\pgfpathrectangle{\pgfqpoint{0.647939in}{0.492442in}}{\pgfqpoint{3.079299in}{3.079299in}}%
\pgfusepath{clip}%
\pgfsetbuttcap%
\pgfsetroundjoin%
\pgfsetlinewidth{0.301125pt}%
\definecolor{currentstroke}{rgb}{0.500000,0.500000,0.500000}%
\pgfsetstrokecolor{currentstroke}%
\pgfsetstrokeopacity{0.300000}%
\pgfsetdash{}{0pt}%
\pgfpathmoveto{\pgfqpoint{2.387792in}{1.177699in}}%
\pgfpathlineto{\pgfqpoint{2.319440in}{1.180795in}}%
\pgfpathlineto{\pgfqpoint{2.251033in}{1.182435in}}%
\pgfpathlineto{\pgfqpoint{2.182610in}{1.183333in}}%
\pgfpathlineto{\pgfqpoint{2.114191in}{1.184400in}}%
\pgfpathlineto{\pgfqpoint{2.045813in}{1.186831in}}%
\pgfpathlineto{\pgfqpoint{1.977636in}{1.192283in}}%
\pgfusepath{stroke}%
\end{pgfscope}%
\begin{pgfscope}%
\pgfpathrectangle{\pgfqpoint{0.647939in}{0.492442in}}{\pgfqpoint{3.079299in}{3.079299in}}%
\pgfusepath{clip}%
\pgfsetbuttcap%
\pgfsetroundjoin%
\pgfsetlinewidth{0.301125pt}%
\definecolor{currentstroke}{rgb}{0.500000,0.500000,0.500000}%
\pgfsetstrokecolor{currentstroke}%
\pgfsetstrokeopacity{0.300000}%
\pgfsetdash{}{0pt}%
\pgfpathmoveto{\pgfqpoint{1.921334in}{2.843664in}}%
\pgfpathlineto{\pgfqpoint{1.989586in}{2.848472in}}%
\pgfpathlineto{\pgfqpoint{2.057945in}{2.851446in}}%
\pgfpathlineto{\pgfqpoint{2.126352in}{2.853045in}}%
\pgfpathlineto{\pgfqpoint{2.194775in}{2.853939in}}%
\pgfpathlineto{\pgfqpoint{2.263194in}{2.855030in}}%
\pgfpathlineto{\pgfqpoint{2.331572in}{2.857470in}}%
\pgfpathlineto{\pgfqpoint{2.399781in}{2.862620in}}%
\pgfpathlineto{\pgfqpoint{2.467525in}{2.871901in}}%
\pgfusepath{stroke}%
\end{pgfscope}%
\begin{pgfscope}%
\pgfpathrectangle{\pgfqpoint{0.647939in}{0.492442in}}{\pgfqpoint{3.079299in}{3.079299in}}%
\pgfusepath{clip}%
\pgfsetbuttcap%
\pgfsetroundjoin%
\pgfsetlinewidth{0.301125pt}%
\definecolor{currentstroke}{rgb}{0.500000,0.500000,0.500000}%
\pgfsetstrokecolor{currentstroke}%
\pgfsetstrokeopacity{0.300000}%
\pgfsetdash{}{0pt}%
\pgfpathmoveto{\pgfqpoint{1.281802in}{2.713782in}}%
\pgfpathlineto{\pgfqpoint{1.347780in}{2.731932in}}%
\pgfpathlineto{\pgfqpoint{1.413795in}{2.749944in}}%
\pgfpathlineto{\pgfqpoint{1.479937in}{2.767483in}}%
\pgfpathlineto{\pgfqpoint{1.546289in}{2.784205in}}%
\pgfpathlineto{\pgfqpoint{1.612922in}{2.799763in}}%
\pgfusepath{stroke}%
\end{pgfscope}%
\begin{pgfscope}%
\pgfpathrectangle{\pgfqpoint{0.647939in}{0.492442in}}{\pgfqpoint{3.079299in}{3.079299in}}%
\pgfusepath{clip}%
\pgfsetbuttcap%
\pgfsetroundjoin%
\pgfsetlinewidth{0.301125pt}%
\definecolor{currentstroke}{rgb}{0.500000,0.500000,0.500000}%
\pgfsetstrokecolor{currentstroke}%
\pgfsetstrokeopacity{0.300000}%
\pgfsetdash{}{0pt}%
\pgfpathmoveto{\pgfqpoint{1.417764in}{1.752155in}}%
\pgfpathlineto{\pgfqpoint{1.478184in}{1.784273in}}%
\pgfpathlineto{\pgfqpoint{1.538449in}{1.816681in}}%
\pgfpathlineto{\pgfqpoint{1.598776in}{1.848973in}}%
\pgfpathlineto{\pgfqpoint{1.659400in}{1.880698in}}%
\pgfpathlineto{\pgfqpoint{1.720567in}{1.911359in}}%
\pgfpathlineto{\pgfqpoint{1.782512in}{1.940401in}}%
\pgfpathlineto{\pgfqpoint{1.845445in}{1.967216in}}%
\pgfpathlineto{\pgfqpoint{1.909514in}{1.991166in}}%
\pgfpathlineto{\pgfqpoint{1.974770in}{2.011637in}}%
\pgfpathlineto{\pgfqpoint{2.041120in}{2.028188in}}%
\pgfusepath{stroke}%
\end{pgfscope}%
\begin{pgfscope}%
\pgfpathrectangle{\pgfqpoint{0.647939in}{0.492442in}}{\pgfqpoint{3.079299in}{3.079299in}}%
\pgfusepath{clip}%
\pgfsetbuttcap%
\pgfsetroundjoin%
\pgfsetlinewidth{0.301125pt}%
\definecolor{currentstroke}{rgb}{0.500000,0.500000,0.500000}%
\pgfsetstrokecolor{currentstroke}%
\pgfsetstrokeopacity{0.300000}%
\pgfsetdash{}{0pt}%
\pgfpathmoveto{\pgfqpoint{1.851351in}{2.701668in}}%
\pgfpathlineto{\pgfqpoint{1.919317in}{2.709510in}}%
\pgfpathlineto{\pgfqpoint{1.987512in}{2.715043in}}%
\pgfpathlineto{\pgfqpoint{2.055846in}{2.718516in}}%
\pgfpathlineto{\pgfqpoint{2.124245in}{2.720414in}}%
\pgfpathlineto{\pgfqpoint{2.192665in}{2.721482in}}%
\pgfpathlineto{\pgfqpoint{2.261080in}{2.722773in}}%
\pgfpathlineto{\pgfqpoint{2.329436in}{2.725684in}}%
\pgfpathlineto{\pgfqpoint{2.397541in}{2.731932in}}%
\pgfusepath{stroke}%
\end{pgfscope}%
\begin{pgfscope}%
\pgfpathrectangle{\pgfqpoint{0.647939in}{0.492442in}}{\pgfqpoint{3.079299in}{3.079299in}}%
\pgfusepath{clip}%
\pgfsetbuttcap%
\pgfsetroundjoin%
\pgfsetlinewidth{0.301125pt}%
\definecolor{currentstroke}{rgb}{0.500000,0.500000,0.500000}%
\pgfsetstrokecolor{currentstroke}%
\pgfsetstrokeopacity{0.300000}%
\pgfsetdash{}{0pt}%
\pgfpathmoveto{\pgfqpoint{1.487748in}{1.962108in}}%
\pgfpathlineto{\pgfqpoint{1.550192in}{1.990092in}}%
\pgfpathlineto{\pgfqpoint{1.612905in}{2.017466in}}%
\pgfpathlineto{\pgfqpoint{1.676083in}{2.043742in}}%
\pgfpathlineto{\pgfqpoint{1.739909in}{2.068389in}}%
\pgfpathlineto{\pgfqpoint{1.804531in}{2.090846in}}%
\pgfpathlineto{\pgfqpoint{1.870042in}{2.110541in}}%
\pgfusepath{stroke}%
\end{pgfscope}%
\begin{pgfscope}%
\pgfpathrectangle{\pgfqpoint{0.647939in}{0.492442in}}{\pgfqpoint{3.079299in}{3.079299in}}%
\pgfusepath{clip}%
\pgfsetbuttcap%
\pgfsetroundjoin%
\pgfsetlinewidth{0.301125pt}%
\definecolor{currentstroke}{rgb}{0.500000,0.500000,0.500000}%
\pgfsetstrokecolor{currentstroke}%
\pgfsetstrokeopacity{0.300000}%
\pgfsetdash{}{0pt}%
\pgfpathmoveto{\pgfqpoint{1.593726in}{2.530307in}}%
\pgfpathlineto{\pgfqpoint{1.659947in}{2.547525in}}%
\pgfpathlineto{\pgfqpoint{1.726601in}{2.562967in}}%
\pgfpathlineto{\pgfqpoint{1.793709in}{2.576291in}}%
\pgfpathlineto{\pgfqpoint{1.861246in}{2.587228in}}%
\pgfpathlineto{\pgfqpoint{1.929143in}{2.595641in}}%
\pgfpathlineto{\pgfqpoint{1.997302in}{2.601568in}}%
\pgfpathlineto{\pgfqpoint{2.065622in}{2.605266in}}%
\pgfpathlineto{\pgfqpoint{2.134017in}{2.607275in}}%
\pgfpathlineto{\pgfqpoint{2.202435in}{2.608475in}}%
\pgfpathlineto{\pgfqpoint{2.270839in}{2.610199in}}%
\pgfpathlineto{\pgfqpoint{2.339120in}{2.614274in}}%
\pgfpathlineto{\pgfqpoint{2.406924in}{2.622929in}}%
\pgfpathlineto{\pgfqpoint{2.473446in}{2.638381in}}%
\pgfpathlineto{\pgfqpoint{2.537509in}{2.661948in}}%
\pgfusepath{stroke}%
\end{pgfscope}%
\begin{pgfscope}%
\pgfpathrectangle{\pgfqpoint{0.647939in}{0.492442in}}{\pgfqpoint{3.079299in}{3.079299in}}%
\pgfusepath{clip}%
\pgfsetbuttcap%
\pgfsetroundjoin%
\pgfsetlinewidth{0.301125pt}%
\definecolor{currentstroke}{rgb}{0.500000,0.500000,0.500000}%
\pgfsetstrokecolor{currentstroke}%
\pgfsetstrokeopacity{0.300000}%
\pgfsetdash{}{0pt}%
\pgfpathmoveto{\pgfqpoint{2.526763in}{1.445063in}}%
\pgfpathlineto{\pgfqpoint{2.459120in}{1.455244in}}%
\pgfpathlineto{\pgfqpoint{2.391080in}{1.462357in}}%
\pgfpathlineto{\pgfqpoint{2.322806in}{1.466798in}}%
\pgfpathlineto{\pgfqpoint{2.254426in}{1.469227in}}%
\pgfpathlineto{\pgfqpoint{2.186011in}{1.470587in}}%
\pgfpathlineto{\pgfqpoint{2.117605in}{1.472219in}}%
\pgfusepath{stroke}%
\end{pgfscope}%
\begin{pgfscope}%
\pgfpathrectangle{\pgfqpoint{0.647939in}{0.492442in}}{\pgfqpoint{3.079299in}{3.079299in}}%
\pgfusepath{clip}%
\pgfsetbuttcap%
\pgfsetroundjoin%
\pgfsetlinewidth{0.301125pt}%
\definecolor{currentstroke}{rgb}{0.500000,0.500000,0.500000}%
\pgfsetstrokecolor{currentstroke}%
\pgfsetstrokeopacity{0.300000}%
\pgfsetdash{}{0pt}%
\pgfpathmoveto{\pgfqpoint{2.856076in}{1.878870in}}%
\pgfpathlineto{\pgfqpoint{2.801852in}{1.920601in}}%
\pgfpathlineto{\pgfqpoint{2.747461in}{1.962108in}}%
\pgfpathlineto{\pgfqpoint{2.692942in}{2.003437in}}%
\pgfpathlineto{\pgfqpoint{2.638442in}{2.044796in}}%
\pgfpathlineto{\pgfqpoint{2.584404in}{2.086743in}}%
\pgfpathlineto{\pgfqpoint{2.532048in}{2.130686in}}%
\pgfpathlineto{\pgfqpoint{2.485396in}{2.180279in}}%
\pgfpathlineto{\pgfqpoint{2.485396in}{2.180279in}}%
\pgfpathlineto{\pgfqpoint{2.465714in}{2.214690in}}%
\pgfpathlineto{\pgfqpoint{2.465714in}{2.214690in}}%
\pgfpathlineto{\pgfqpoint{2.459075in}{2.248201in}}%
\pgfusepath{stroke}%
\end{pgfscope}%
\begin{pgfscope}%
\pgfpathrectangle{\pgfqpoint{0.647939in}{0.492442in}}{\pgfqpoint{3.079299in}{3.079299in}}%
\pgfusepath{clip}%
\pgfsetbuttcap%
\pgfsetroundjoin%
\pgfsetlinewidth{0.301125pt}%
\definecolor{currentstroke}{rgb}{0.500000,0.500000,0.500000}%
\pgfsetstrokecolor{currentstroke}%
\pgfsetstrokeopacity{0.300000}%
\pgfsetdash{}{0pt}%
\pgfpathmoveto{\pgfqpoint{2.677477in}{1.892124in}}%
\pgfpathlineto{\pgfqpoint{2.617470in}{1.924976in}}%
\pgfpathlineto{\pgfqpoint{2.556344in}{1.955682in}}%
\pgfpathlineto{\pgfqpoint{2.494029in}{1.983891in}}%
\pgfpathlineto{\pgfqpoint{2.430537in}{2.009321in}}%
\pgfpathlineto{\pgfqpoint{2.366026in}{2.032009in}}%
\pgfpathlineto{\pgfqpoint{2.301134in}{2.053491in}}%
\pgfpathlineto{\pgfqpoint{2.301134in}{2.053491in}}%
\pgfpathlineto{\pgfqpoint{2.276189in}{2.062922in}}%
\pgfpathlineto{\pgfqpoint{2.276189in}{2.062922in}}%
\pgfusepath{stroke}%
\end{pgfscope}%
\begin{pgfscope}%
\pgfpathrectangle{\pgfqpoint{0.647939in}{0.492442in}}{\pgfqpoint{3.079299in}{3.079299in}}%
\pgfusepath{clip}%
\pgfsetbuttcap%
\pgfsetroundjoin%
\pgfsetlinewidth{0.301125pt}%
\definecolor{currentstroke}{rgb}{0.500000,0.500000,0.500000}%
\pgfsetstrokecolor{currentstroke}%
\pgfsetstrokeopacity{0.300000}%
\pgfsetdash{}{0pt}%
\pgfpathmoveto{\pgfqpoint{1.806304in}{2.394664in}}%
\pgfpathlineto{\pgfqpoint{1.873519in}{2.407406in}}%
\pgfpathlineto{\pgfqpoint{1.941210in}{2.417296in}}%
\pgfpathlineto{\pgfqpoint{2.009261in}{2.424303in}}%
\pgfpathlineto{\pgfqpoint{2.077538in}{2.428677in}}%
\pgfpathlineto{\pgfqpoint{2.145920in}{2.431090in}}%
\pgfpathlineto{\pgfqpoint{2.214328in}{2.432795in}}%
\pgfpathlineto{\pgfqpoint{2.282672in}{2.435842in}}%
\pgfpathlineto{\pgfqpoint{2.350603in}{2.443341in}}%
\pgfpathlineto{\pgfqpoint{2.416971in}{2.459113in}}%
\pgfpathlineto{\pgfqpoint{2.479759in}{2.485662in}}%
\pgfpathlineto{\pgfqpoint{2.537509in}{2.521980in}}%
\pgfpathlineto{\pgfqpoint{2.590460in}{2.564955in}}%
\pgfpathlineto{\pgfqpoint{2.640091in}{2.611882in}}%
\pgfusepath{stroke}%
\end{pgfscope}%
\begin{pgfscope}%
\pgfpathrectangle{\pgfqpoint{0.647939in}{0.492442in}}{\pgfqpoint{3.079299in}{3.079299in}}%
\pgfusepath{clip}%
\pgfsetbuttcap%
\pgfsetroundjoin%
\pgfsetlinewidth{0.301125pt}%
\definecolor{currentstroke}{rgb}{0.500000,0.500000,0.500000}%
\pgfsetstrokecolor{currentstroke}%
\pgfsetstrokeopacity{0.300000}%
\pgfsetdash{}{0pt}%
\pgfpathmoveto{\pgfqpoint{2.311067in}{1.583270in}}%
\pgfpathlineto{\pgfqpoint{2.242696in}{1.585917in}}%
\pgfpathlineto{\pgfqpoint{2.174287in}{1.587523in}}%
\pgfpathlineto{\pgfqpoint{2.105909in}{1.589943in}}%
\pgfpathlineto{\pgfqpoint{2.043433in}{1.595589in}}%
\pgfpathlineto{\pgfqpoint{1.977636in}{1.612187in}}%
\pgfpathlineto{\pgfqpoint{1.977636in}{1.612187in}}%
\pgfpathlineto{\pgfqpoint{1.945855in}{1.629793in}}%
\pgfpathlineto{\pgfqpoint{1.945855in}{1.629793in}}%
\pgfusepath{stroke}%
\end{pgfscope}%
\begin{pgfscope}%
\pgfpathrectangle{\pgfqpoint{0.647939in}{0.492442in}}{\pgfqpoint{3.079299in}{3.079299in}}%
\pgfusepath{clip}%
\pgfsetbuttcap%
\pgfsetroundjoin%
\pgfsetlinewidth{0.301125pt}%
\definecolor{currentstroke}{rgb}{0.500000,0.500000,0.500000}%
\pgfsetstrokecolor{currentstroke}%
\pgfsetstrokeopacity{0.300000}%
\pgfsetdash{}{0pt}%
\pgfpathmoveto{\pgfqpoint{2.537509in}{1.822139in}}%
\pgfpathlineto{\pgfqpoint{2.471886in}{1.841391in}}%
\pgfpathlineto{\pgfqpoint{2.405196in}{1.856523in}}%
\pgfpathlineto{\pgfqpoint{2.337681in}{1.867415in}}%
\pgfpathlineto{\pgfqpoint{2.269641in}{1.874433in}}%
\pgfpathlineto{\pgfqpoint{2.201368in}{1.878973in}}%
\pgfpathlineto{\pgfqpoint{2.133445in}{1.885966in}}%
\pgfpathlineto{\pgfqpoint{2.133445in}{1.885966in}}%
\pgfpathlineto{\pgfqpoint{2.108026in}{1.892277in}}%
\pgfpathlineto{\pgfqpoint{2.108026in}{1.892277in}}%
\pgfpathlineto{\pgfqpoint{2.092667in}{1.902223in}}%
\pgfpathlineto{\pgfqpoint{2.092667in}{1.902223in}}%
\pgfpathlineto{\pgfqpoint{2.088213in}{1.914943in}}%
\pgfusepath{stroke}%
\end{pgfscope}%
\begin{pgfscope}%
\pgfpathrectangle{\pgfqpoint{0.647939in}{0.492442in}}{\pgfqpoint{3.079299in}{3.079299in}}%
\pgfusepath{clip}%
\pgfsetroundcap%
\pgfsetroundjoin%
\pgfsetlinewidth{0.301125pt}%
\definecolor{currentstroke}{rgb}{0.500000,0.500000,0.500000}%
\pgfsetstrokecolor{currentstroke}%
\pgfsetstrokeopacity{0.300000}%
\pgfsetdash{}{0pt}%
\pgfpathmoveto{\pgfqpoint{2.109649in}{1.963176in}}%
\pgfusepath{stroke}%
\end{pgfscope}%
\begin{pgfscope}%
\pgfpathrectangle{\pgfqpoint{0.647939in}{0.492442in}}{\pgfqpoint{3.079299in}{3.079299in}}%
\pgfusepath{clip}%
\pgfsetroundcap%
\pgfsetroundjoin%
\definecolor{currentfill}{rgb}{0.500000,0.500000,0.500000}%
\pgfsetfillcolor{currentfill}%
\pgfsetfillopacity{0.300000}%
\pgfsetlinewidth{0.301125pt}%
\definecolor{currentstroke}{rgb}{0.500000,0.500000,0.500000}%
\pgfsetstrokecolor{currentstroke}%
\pgfsetstrokeopacity{0.300000}%
\pgfsetdash{}{0pt}%
\pgfpathmoveto{\pgfqpoint{0.000000in}{0.000000in}}%
\pgfpathlineto{\pgfqpoint{0.000000in}{0.000000in}}%
\pgfpathclose%
\pgfusepath{stroke,fill}%
\end{pgfscope}%
\begin{pgfscope}%
\pgfpathrectangle{\pgfqpoint{0.647939in}{0.492442in}}{\pgfqpoint{3.079299in}{3.079299in}}%
\pgfusepath{clip}%
\pgfsetroundcap%
\pgfsetroundjoin%
\pgfsetlinewidth{0.301125pt}%
\definecolor{currentstroke}{rgb}{0.500000,0.500000,0.500000}%
\pgfsetstrokecolor{currentstroke}%
\pgfsetstrokeopacity{0.300000}%
\pgfsetdash{}{0pt}%
\pgfpathmoveto{\pgfqpoint{1.063928in}{0.599730in}}%
\pgfusepath{stroke}%
\end{pgfscope}%
\begin{pgfscope}%
\pgfpathrectangle{\pgfqpoint{0.647939in}{0.492442in}}{\pgfqpoint{3.079299in}{3.079299in}}%
\pgfusepath{clip}%
\pgfsetroundcap%
\pgfsetroundjoin%
\definecolor{currentfill}{rgb}{0.500000,0.500000,0.500000}%
\pgfsetfillcolor{currentfill}%
\pgfsetfillopacity{0.300000}%
\pgfsetlinewidth{0.301125pt}%
\definecolor{currentstroke}{rgb}{0.500000,0.500000,0.500000}%
\pgfsetstrokecolor{currentstroke}%
\pgfsetstrokeopacity{0.300000}%
\pgfsetdash{}{0pt}%
\pgfpathmoveto{\pgfqpoint{0.000000in}{0.000000in}}%
\pgfpathlineto{\pgfqpoint{0.000000in}{0.000000in}}%
\pgfpathclose%
\pgfusepath{stroke,fill}%
\end{pgfscope}%
\begin{pgfscope}%
\pgfpathrectangle{\pgfqpoint{0.647939in}{0.492442in}}{\pgfqpoint{3.079299in}{3.079299in}}%
\pgfusepath{clip}%
\pgfsetroundcap%
\pgfsetroundjoin%
\pgfsetlinewidth{0.301125pt}%
\definecolor{currentstroke}{rgb}{0.500000,0.500000,0.500000}%
\pgfsetstrokecolor{currentstroke}%
\pgfsetstrokeopacity{0.300000}%
\pgfsetdash{}{0pt}%
\pgfpathmoveto{\pgfqpoint{1.309703in}{0.738611in}}%
\pgfusepath{stroke}%
\end{pgfscope}%
\begin{pgfscope}%
\pgfpathrectangle{\pgfqpoint{0.647939in}{0.492442in}}{\pgfqpoint{3.079299in}{3.079299in}}%
\pgfusepath{clip}%
\pgfsetroundcap%
\pgfsetroundjoin%
\definecolor{currentfill}{rgb}{0.500000,0.500000,0.500000}%
\pgfsetfillcolor{currentfill}%
\pgfsetfillopacity{0.300000}%
\pgfsetlinewidth{0.301125pt}%
\definecolor{currentstroke}{rgb}{0.500000,0.500000,0.500000}%
\pgfsetstrokecolor{currentstroke}%
\pgfsetstrokeopacity{0.300000}%
\pgfsetdash{}{0pt}%
\pgfpathmoveto{\pgfqpoint{0.000000in}{0.000000in}}%
\pgfpathlineto{\pgfqpoint{0.000000in}{0.000000in}}%
\pgfpathclose%
\pgfusepath{stroke,fill}%
\end{pgfscope}%
\begin{pgfscope}%
\pgfpathrectangle{\pgfqpoint{0.647939in}{0.492442in}}{\pgfqpoint{3.079299in}{3.079299in}}%
\pgfusepath{clip}%
\pgfsetroundcap%
\pgfsetroundjoin%
\pgfsetlinewidth{0.301125pt}%
\definecolor{currentstroke}{rgb}{0.500000,0.500000,0.500000}%
\pgfsetstrokecolor{currentstroke}%
\pgfsetstrokeopacity{0.300000}%
\pgfsetdash{}{0pt}%
\pgfpathmoveto{\pgfqpoint{1.364281in}{0.655486in}}%
\pgfusepath{stroke}%
\end{pgfscope}%
\begin{pgfscope}%
\pgfpathrectangle{\pgfqpoint{0.647939in}{0.492442in}}{\pgfqpoint{3.079299in}{3.079299in}}%
\pgfusepath{clip}%
\pgfsetroundcap%
\pgfsetroundjoin%
\definecolor{currentfill}{rgb}{0.500000,0.500000,0.500000}%
\pgfsetfillcolor{currentfill}%
\pgfsetfillopacity{0.300000}%
\pgfsetlinewidth{0.301125pt}%
\definecolor{currentstroke}{rgb}{0.500000,0.500000,0.500000}%
\pgfsetstrokecolor{currentstroke}%
\pgfsetstrokeopacity{0.300000}%
\pgfsetdash{}{0pt}%
\pgfpathmoveto{\pgfqpoint{0.000000in}{0.000000in}}%
\pgfpathlineto{\pgfqpoint{0.000000in}{0.000000in}}%
\pgfpathclose%
\pgfusepath{stroke,fill}%
\end{pgfscope}%
\begin{pgfscope}%
\pgfpathrectangle{\pgfqpoint{0.647939in}{0.492442in}}{\pgfqpoint{3.079299in}{3.079299in}}%
\pgfusepath{clip}%
\pgfsetroundcap%
\pgfsetroundjoin%
\pgfsetlinewidth{0.301125pt}%
\definecolor{currentstroke}{rgb}{0.500000,0.500000,0.500000}%
\pgfsetstrokecolor{currentstroke}%
\pgfsetstrokeopacity{0.300000}%
\pgfsetdash{}{0pt}%
\pgfpathmoveto{\pgfqpoint{1.505324in}{0.598107in}}%
\pgfusepath{stroke}%
\end{pgfscope}%
\begin{pgfscope}%
\pgfpathrectangle{\pgfqpoint{0.647939in}{0.492442in}}{\pgfqpoint{3.079299in}{3.079299in}}%
\pgfusepath{clip}%
\pgfsetroundcap%
\pgfsetroundjoin%
\definecolor{currentfill}{rgb}{0.500000,0.500000,0.500000}%
\pgfsetfillcolor{currentfill}%
\pgfsetfillopacity{0.300000}%
\pgfsetlinewidth{0.301125pt}%
\definecolor{currentstroke}{rgb}{0.500000,0.500000,0.500000}%
\pgfsetstrokecolor{currentstroke}%
\pgfsetstrokeopacity{0.300000}%
\pgfsetdash{}{0pt}%
\pgfpathmoveto{\pgfqpoint{0.000000in}{0.000000in}}%
\pgfpathlineto{\pgfqpoint{0.000000in}{0.000000in}}%
\pgfpathclose%
\pgfusepath{stroke,fill}%
\end{pgfscope}%
\begin{pgfscope}%
\pgfpathrectangle{\pgfqpoint{0.647939in}{0.492442in}}{\pgfqpoint{3.079299in}{3.079299in}}%
\pgfusepath{clip}%
\pgfsetroundcap%
\pgfsetroundjoin%
\pgfsetlinewidth{0.301125pt}%
\definecolor{currentstroke}{rgb}{0.500000,0.500000,0.500000}%
\pgfsetstrokecolor{currentstroke}%
\pgfsetstrokeopacity{0.300000}%
\pgfsetdash{}{0pt}%
\pgfpathmoveto{\pgfqpoint{1.814068in}{0.509934in}}%
\pgfusepath{stroke}%
\end{pgfscope}%
\begin{pgfscope}%
\pgfpathrectangle{\pgfqpoint{0.647939in}{0.492442in}}{\pgfqpoint{3.079299in}{3.079299in}}%
\pgfusepath{clip}%
\pgfsetroundcap%
\pgfsetroundjoin%
\definecolor{currentfill}{rgb}{0.500000,0.500000,0.500000}%
\pgfsetfillcolor{currentfill}%
\pgfsetfillopacity{0.300000}%
\pgfsetlinewidth{0.301125pt}%
\definecolor{currentstroke}{rgb}{0.500000,0.500000,0.500000}%
\pgfsetstrokecolor{currentstroke}%
\pgfsetstrokeopacity{0.300000}%
\pgfsetdash{}{0pt}%
\pgfpathmoveto{\pgfqpoint{0.000000in}{0.000000in}}%
\pgfpathlineto{\pgfqpoint{0.000000in}{0.000000in}}%
\pgfpathclose%
\pgfusepath{stroke,fill}%
\end{pgfscope}%
\begin{pgfscope}%
\pgfpathrectangle{\pgfqpoint{0.647939in}{0.492442in}}{\pgfqpoint{3.079299in}{3.079299in}}%
\pgfusepath{clip}%
\pgfsetroundcap%
\pgfsetroundjoin%
\pgfsetlinewidth{0.301125pt}%
\definecolor{currentstroke}{rgb}{0.500000,0.500000,0.500000}%
\pgfsetstrokecolor{currentstroke}%
\pgfsetstrokeopacity{0.300000}%
\pgfsetdash{}{0pt}%
\pgfpathmoveto{\pgfqpoint{2.232959in}{0.495699in}}%
\pgfusepath{stroke}%
\end{pgfscope}%
\begin{pgfscope}%
\pgfpathrectangle{\pgfqpoint{0.647939in}{0.492442in}}{\pgfqpoint{3.079299in}{3.079299in}}%
\pgfusepath{clip}%
\pgfsetroundcap%
\pgfsetroundjoin%
\definecolor{currentfill}{rgb}{0.500000,0.500000,0.500000}%
\pgfsetfillcolor{currentfill}%
\pgfsetfillopacity{0.300000}%
\pgfsetlinewidth{0.301125pt}%
\definecolor{currentstroke}{rgb}{0.500000,0.500000,0.500000}%
\pgfsetstrokecolor{currentstroke}%
\pgfsetstrokeopacity{0.300000}%
\pgfsetdash{}{0pt}%
\pgfpathmoveto{\pgfqpoint{0.000000in}{0.000000in}}%
\pgfpathlineto{\pgfqpoint{0.000000in}{0.000000in}}%
\pgfpathclose%
\pgfusepath{stroke,fill}%
\end{pgfscope}%
\begin{pgfscope}%
\pgfpathrectangle{\pgfqpoint{0.647939in}{0.492442in}}{\pgfqpoint{3.079299in}{3.079299in}}%
\pgfusepath{clip}%
\pgfsetroundcap%
\pgfsetroundjoin%
\pgfsetlinewidth{0.301125pt}%
\definecolor{currentstroke}{rgb}{0.500000,0.500000,0.500000}%
\pgfsetstrokecolor{currentstroke}%
\pgfsetstrokeopacity{0.300000}%
\pgfsetdash{}{0pt}%
\pgfpathmoveto{\pgfqpoint{2.311413in}{0.531538in}}%
\pgfusepath{stroke}%
\end{pgfscope}%
\begin{pgfscope}%
\pgfpathrectangle{\pgfqpoint{0.647939in}{0.492442in}}{\pgfqpoint{3.079299in}{3.079299in}}%
\pgfusepath{clip}%
\pgfsetroundcap%
\pgfsetroundjoin%
\definecolor{currentfill}{rgb}{0.500000,0.500000,0.500000}%
\pgfsetfillcolor{currentfill}%
\pgfsetfillopacity{0.300000}%
\pgfsetlinewidth{0.301125pt}%
\definecolor{currentstroke}{rgb}{0.500000,0.500000,0.500000}%
\pgfsetstrokecolor{currentstroke}%
\pgfsetstrokeopacity{0.300000}%
\pgfsetdash{}{0pt}%
\pgfpathmoveto{\pgfqpoint{0.000000in}{0.000000in}}%
\pgfpathlineto{\pgfqpoint{0.000000in}{0.000000in}}%
\pgfpathclose%
\pgfusepath{stroke,fill}%
\end{pgfscope}%
\begin{pgfscope}%
\pgfpathrectangle{\pgfqpoint{0.647939in}{0.492442in}}{\pgfqpoint{3.079299in}{3.079299in}}%
\pgfusepath{clip}%
\pgfsetroundcap%
\pgfsetroundjoin%
\pgfsetlinewidth{0.301125pt}%
\definecolor{currentstroke}{rgb}{0.500000,0.500000,0.500000}%
\pgfsetstrokecolor{currentstroke}%
\pgfsetstrokeopacity{0.300000}%
\pgfsetdash{}{0pt}%
\pgfpathmoveto{\pgfqpoint{2.730235in}{0.543934in}}%
\pgfusepath{stroke}%
\end{pgfscope}%
\begin{pgfscope}%
\pgfpathrectangle{\pgfqpoint{0.647939in}{0.492442in}}{\pgfqpoint{3.079299in}{3.079299in}}%
\pgfusepath{clip}%
\pgfsetroundcap%
\pgfsetroundjoin%
\definecolor{currentfill}{rgb}{0.500000,0.500000,0.500000}%
\pgfsetfillcolor{currentfill}%
\pgfsetfillopacity{0.300000}%
\pgfsetlinewidth{0.301125pt}%
\definecolor{currentstroke}{rgb}{0.500000,0.500000,0.500000}%
\pgfsetstrokecolor{currentstroke}%
\pgfsetstrokeopacity{0.300000}%
\pgfsetdash{}{0pt}%
\pgfpathmoveto{\pgfqpoint{0.000000in}{0.000000in}}%
\pgfpathlineto{\pgfqpoint{0.000000in}{0.000000in}}%
\pgfpathclose%
\pgfusepath{stroke,fill}%
\end{pgfscope}%
\begin{pgfscope}%
\pgfpathrectangle{\pgfqpoint{0.647939in}{0.492442in}}{\pgfqpoint{3.079299in}{3.079299in}}%
\pgfusepath{clip}%
\pgfsetroundcap%
\pgfsetroundjoin%
\pgfsetlinewidth{0.301125pt}%
\definecolor{currentstroke}{rgb}{0.500000,0.500000,0.500000}%
\pgfsetstrokecolor{currentstroke}%
\pgfsetstrokeopacity{0.300000}%
\pgfsetdash{}{0pt}%
\pgfpathmoveto{\pgfqpoint{2.063669in}{0.654157in}}%
\pgfusepath{stroke}%
\end{pgfscope}%
\begin{pgfscope}%
\pgfpathrectangle{\pgfqpoint{0.647939in}{0.492442in}}{\pgfqpoint{3.079299in}{3.079299in}}%
\pgfusepath{clip}%
\pgfsetroundcap%
\pgfsetroundjoin%
\definecolor{currentfill}{rgb}{0.500000,0.500000,0.500000}%
\pgfsetfillcolor{currentfill}%
\pgfsetfillopacity{0.300000}%
\pgfsetlinewidth{0.301125pt}%
\definecolor{currentstroke}{rgb}{0.500000,0.500000,0.500000}%
\pgfsetstrokecolor{currentstroke}%
\pgfsetstrokeopacity{0.300000}%
\pgfsetdash{}{0pt}%
\pgfpathmoveto{\pgfqpoint{0.000000in}{0.000000in}}%
\pgfpathlineto{\pgfqpoint{0.000000in}{0.000000in}}%
\pgfpathclose%
\pgfusepath{stroke,fill}%
\end{pgfscope}%
\begin{pgfscope}%
\pgfpathrectangle{\pgfqpoint{0.647939in}{0.492442in}}{\pgfqpoint{3.079299in}{3.079299in}}%
\pgfusepath{clip}%
\pgfsetroundcap%
\pgfsetroundjoin%
\pgfsetlinewidth{0.301125pt}%
\definecolor{currentstroke}{rgb}{0.500000,0.500000,0.500000}%
\pgfsetstrokecolor{currentstroke}%
\pgfsetstrokeopacity{0.300000}%
\pgfsetdash{}{0pt}%
\pgfpathmoveto{\pgfqpoint{2.493633in}{0.725322in}}%
\pgfusepath{stroke}%
\end{pgfscope}%
\begin{pgfscope}%
\pgfpathrectangle{\pgfqpoint{0.647939in}{0.492442in}}{\pgfqpoint{3.079299in}{3.079299in}}%
\pgfusepath{clip}%
\pgfsetroundcap%
\pgfsetroundjoin%
\definecolor{currentfill}{rgb}{0.500000,0.500000,0.500000}%
\pgfsetfillcolor{currentfill}%
\pgfsetfillopacity{0.300000}%
\pgfsetlinewidth{0.301125pt}%
\definecolor{currentstroke}{rgb}{0.500000,0.500000,0.500000}%
\pgfsetstrokecolor{currentstroke}%
\pgfsetstrokeopacity{0.300000}%
\pgfsetdash{}{0pt}%
\pgfpathmoveto{\pgfqpoint{0.000000in}{0.000000in}}%
\pgfpathlineto{\pgfqpoint{0.000000in}{0.000000in}}%
\pgfpathclose%
\pgfusepath{stroke,fill}%
\end{pgfscope}%
\begin{pgfscope}%
\pgfpathrectangle{\pgfqpoint{0.647939in}{0.492442in}}{\pgfqpoint{3.079299in}{3.079299in}}%
\pgfusepath{clip}%
\pgfsetroundcap%
\pgfsetroundjoin%
\pgfsetlinewidth{0.301125pt}%
\definecolor{currentstroke}{rgb}{0.500000,0.500000,0.500000}%
\pgfsetstrokecolor{currentstroke}%
\pgfsetstrokeopacity{0.300000}%
\pgfsetdash{}{0pt}%
\pgfpathmoveto{\pgfqpoint{2.582313in}{0.801901in}}%
\pgfusepath{stroke}%
\end{pgfscope}%
\begin{pgfscope}%
\pgfpathrectangle{\pgfqpoint{0.647939in}{0.492442in}}{\pgfqpoint{3.079299in}{3.079299in}}%
\pgfusepath{clip}%
\pgfsetroundcap%
\pgfsetroundjoin%
\definecolor{currentfill}{rgb}{0.500000,0.500000,0.500000}%
\pgfsetfillcolor{currentfill}%
\pgfsetfillopacity{0.300000}%
\pgfsetlinewidth{0.301125pt}%
\definecolor{currentstroke}{rgb}{0.500000,0.500000,0.500000}%
\pgfsetstrokecolor{currentstroke}%
\pgfsetstrokeopacity{0.300000}%
\pgfsetdash{}{0pt}%
\pgfpathmoveto{\pgfqpoint{0.000000in}{0.000000in}}%
\pgfpathlineto{\pgfqpoint{0.000000in}{0.000000in}}%
\pgfpathclose%
\pgfusepath{stroke,fill}%
\end{pgfscope}%
\begin{pgfscope}%
\pgfpathrectangle{\pgfqpoint{0.647939in}{0.492442in}}{\pgfqpoint{3.079299in}{3.079299in}}%
\pgfusepath{clip}%
\pgfsetroundcap%
\pgfsetroundjoin%
\pgfsetlinewidth{0.301125pt}%
\definecolor{currentstroke}{rgb}{0.500000,0.500000,0.500000}%
\pgfsetstrokecolor{currentstroke}%
\pgfsetstrokeopacity{0.300000}%
\pgfsetdash{}{0pt}%
\pgfpathmoveto{\pgfqpoint{2.667075in}{0.926533in}}%
\pgfusepath{stroke}%
\end{pgfscope}%
\begin{pgfscope}%
\pgfpathrectangle{\pgfqpoint{0.647939in}{0.492442in}}{\pgfqpoint{3.079299in}{3.079299in}}%
\pgfusepath{clip}%
\pgfsetroundcap%
\pgfsetroundjoin%
\definecolor{currentfill}{rgb}{0.500000,0.500000,0.500000}%
\pgfsetfillcolor{currentfill}%
\pgfsetfillopacity{0.300000}%
\pgfsetlinewidth{0.301125pt}%
\definecolor{currentstroke}{rgb}{0.500000,0.500000,0.500000}%
\pgfsetstrokecolor{currentstroke}%
\pgfsetstrokeopacity{0.300000}%
\pgfsetdash{}{0pt}%
\pgfpathmoveto{\pgfqpoint{0.000000in}{0.000000in}}%
\pgfpathlineto{\pgfqpoint{0.000000in}{0.000000in}}%
\pgfpathclose%
\pgfusepath{stroke,fill}%
\end{pgfscope}%
\begin{pgfscope}%
\pgfpathrectangle{\pgfqpoint{0.647939in}{0.492442in}}{\pgfqpoint{3.079299in}{3.079299in}}%
\pgfusepath{clip}%
\pgfsetroundcap%
\pgfsetroundjoin%
\pgfsetlinewidth{0.301125pt}%
\definecolor{currentstroke}{rgb}{0.500000,0.500000,0.500000}%
\pgfsetstrokecolor{currentstroke}%
\pgfsetstrokeopacity{0.300000}%
\pgfsetdash{}{0pt}%
\pgfpathmoveto{\pgfqpoint{2.605118in}{1.023990in}}%
\pgfusepath{stroke}%
\end{pgfscope}%
\begin{pgfscope}%
\pgfpathrectangle{\pgfqpoint{0.647939in}{0.492442in}}{\pgfqpoint{3.079299in}{3.079299in}}%
\pgfusepath{clip}%
\pgfsetroundcap%
\pgfsetroundjoin%
\definecolor{currentfill}{rgb}{0.500000,0.500000,0.500000}%
\pgfsetfillcolor{currentfill}%
\pgfsetfillopacity{0.300000}%
\pgfsetlinewidth{0.301125pt}%
\definecolor{currentstroke}{rgb}{0.500000,0.500000,0.500000}%
\pgfsetstrokecolor{currentstroke}%
\pgfsetstrokeopacity{0.300000}%
\pgfsetdash{}{0pt}%
\pgfpathmoveto{\pgfqpoint{0.000000in}{0.000000in}}%
\pgfpathlineto{\pgfqpoint{0.000000in}{0.000000in}}%
\pgfpathclose%
\pgfusepath{stroke,fill}%
\end{pgfscope}%
\begin{pgfscope}%
\pgfpathrectangle{\pgfqpoint{0.647939in}{0.492442in}}{\pgfqpoint{3.079299in}{3.079299in}}%
\pgfusepath{clip}%
\pgfsetroundcap%
\pgfsetroundjoin%
\pgfsetlinewidth{0.301125pt}%
\definecolor{currentstroke}{rgb}{0.500000,0.500000,0.500000}%
\pgfsetstrokecolor{currentstroke}%
\pgfsetstrokeopacity{0.300000}%
\pgfsetdash{}{0pt}%
\pgfpathmoveto{\pgfqpoint{2.745077in}{1.082575in}}%
\pgfusepath{stroke}%
\end{pgfscope}%
\begin{pgfscope}%
\pgfpathrectangle{\pgfqpoint{0.647939in}{0.492442in}}{\pgfqpoint{3.079299in}{3.079299in}}%
\pgfusepath{clip}%
\pgfsetroundcap%
\pgfsetroundjoin%
\definecolor{currentfill}{rgb}{0.500000,0.500000,0.500000}%
\pgfsetfillcolor{currentfill}%
\pgfsetfillopacity{0.300000}%
\pgfsetlinewidth{0.301125pt}%
\definecolor{currentstroke}{rgb}{0.500000,0.500000,0.500000}%
\pgfsetstrokecolor{currentstroke}%
\pgfsetstrokeopacity{0.300000}%
\pgfsetdash{}{0pt}%
\pgfpathmoveto{\pgfqpoint{0.000000in}{0.000000in}}%
\pgfpathlineto{\pgfqpoint{0.000000in}{0.000000in}}%
\pgfpathclose%
\pgfusepath{stroke,fill}%
\end{pgfscope}%
\begin{pgfscope}%
\pgfpathrectangle{\pgfqpoint{0.647939in}{0.492442in}}{\pgfqpoint{3.079299in}{3.079299in}}%
\pgfusepath{clip}%
\pgfsetroundcap%
\pgfsetroundjoin%
\pgfsetlinewidth{0.301125pt}%
\definecolor{currentstroke}{rgb}{0.500000,0.500000,0.500000}%
\pgfsetstrokecolor{currentstroke}%
\pgfsetstrokeopacity{0.300000}%
\pgfsetdash{}{0pt}%
\pgfpathmoveto{\pgfqpoint{2.685606in}{1.186806in}}%
\pgfusepath{stroke}%
\end{pgfscope}%
\begin{pgfscope}%
\pgfpathrectangle{\pgfqpoint{0.647939in}{0.492442in}}{\pgfqpoint{3.079299in}{3.079299in}}%
\pgfusepath{clip}%
\pgfsetroundcap%
\pgfsetroundjoin%
\definecolor{currentfill}{rgb}{0.500000,0.500000,0.500000}%
\pgfsetfillcolor{currentfill}%
\pgfsetfillopacity{0.300000}%
\pgfsetlinewidth{0.301125pt}%
\definecolor{currentstroke}{rgb}{0.500000,0.500000,0.500000}%
\pgfsetstrokecolor{currentstroke}%
\pgfsetstrokeopacity{0.300000}%
\pgfsetdash{}{0pt}%
\pgfpathmoveto{\pgfqpoint{0.000000in}{0.000000in}}%
\pgfpathlineto{\pgfqpoint{0.000000in}{0.000000in}}%
\pgfpathclose%
\pgfusepath{stroke,fill}%
\end{pgfscope}%
\begin{pgfscope}%
\pgfpathrectangle{\pgfqpoint{0.647939in}{0.492442in}}{\pgfqpoint{3.079299in}{3.079299in}}%
\pgfusepath{clip}%
\pgfsetroundcap%
\pgfsetroundjoin%
\pgfsetlinewidth{0.301125pt}%
\definecolor{currentstroke}{rgb}{0.500000,0.500000,0.500000}%
\pgfsetstrokecolor{currentstroke}%
\pgfsetstrokeopacity{0.300000}%
\pgfsetdash{}{0pt}%
\pgfpathmoveto{\pgfqpoint{2.759573in}{1.258232in}}%
\pgfusepath{stroke}%
\end{pgfscope}%
\begin{pgfscope}%
\pgfpathrectangle{\pgfqpoint{0.647939in}{0.492442in}}{\pgfqpoint{3.079299in}{3.079299in}}%
\pgfusepath{clip}%
\pgfsetroundcap%
\pgfsetroundjoin%
\definecolor{currentfill}{rgb}{0.500000,0.500000,0.500000}%
\pgfsetfillcolor{currentfill}%
\pgfsetfillopacity{0.300000}%
\pgfsetlinewidth{0.301125pt}%
\definecolor{currentstroke}{rgb}{0.500000,0.500000,0.500000}%
\pgfsetstrokecolor{currentstroke}%
\pgfsetstrokeopacity{0.300000}%
\pgfsetdash{}{0pt}%
\pgfpathmoveto{\pgfqpoint{0.000000in}{0.000000in}}%
\pgfpathlineto{\pgfqpoint{0.000000in}{0.000000in}}%
\pgfpathclose%
\pgfusepath{stroke,fill}%
\end{pgfscope}%
\begin{pgfscope}%
\pgfpathrectangle{\pgfqpoint{0.647939in}{0.492442in}}{\pgfqpoint{3.079299in}{3.079299in}}%
\pgfusepath{clip}%
\pgfsetroundcap%
\pgfsetroundjoin%
\pgfsetlinewidth{0.301125pt}%
\definecolor{currentstroke}{rgb}{0.500000,0.500000,0.500000}%
\pgfsetstrokecolor{currentstroke}%
\pgfsetstrokeopacity{0.300000}%
\pgfsetdash{}{0pt}%
\pgfpathmoveto{\pgfqpoint{2.768649in}{1.348818in}}%
\pgfusepath{stroke}%
\end{pgfscope}%
\begin{pgfscope}%
\pgfpathrectangle{\pgfqpoint{0.647939in}{0.492442in}}{\pgfqpoint{3.079299in}{3.079299in}}%
\pgfusepath{clip}%
\pgfsetroundcap%
\pgfsetroundjoin%
\definecolor{currentfill}{rgb}{0.500000,0.500000,0.500000}%
\pgfsetfillcolor{currentfill}%
\pgfsetfillopacity{0.300000}%
\pgfsetlinewidth{0.301125pt}%
\definecolor{currentstroke}{rgb}{0.500000,0.500000,0.500000}%
\pgfsetstrokecolor{currentstroke}%
\pgfsetstrokeopacity{0.300000}%
\pgfsetdash{}{0pt}%
\pgfpathmoveto{\pgfqpoint{0.000000in}{0.000000in}}%
\pgfpathlineto{\pgfqpoint{0.000000in}{0.000000in}}%
\pgfpathclose%
\pgfusepath{stroke,fill}%
\end{pgfscope}%
\begin{pgfscope}%
\pgfpathrectangle{\pgfqpoint{0.647939in}{0.492442in}}{\pgfqpoint{3.079299in}{3.079299in}}%
\pgfusepath{clip}%
\pgfsetroundcap%
\pgfsetroundjoin%
\pgfsetlinewidth{0.301125pt}%
\definecolor{currentstroke}{rgb}{0.500000,0.500000,0.500000}%
\pgfsetstrokecolor{currentstroke}%
\pgfsetstrokeopacity{0.300000}%
\pgfsetdash{}{0pt}%
\pgfpathmoveto{\pgfqpoint{2.842808in}{1.416056in}}%
\pgfusepath{stroke}%
\end{pgfscope}%
\begin{pgfscope}%
\pgfpathrectangle{\pgfqpoint{0.647939in}{0.492442in}}{\pgfqpoint{3.079299in}{3.079299in}}%
\pgfusepath{clip}%
\pgfsetroundcap%
\pgfsetroundjoin%
\definecolor{currentfill}{rgb}{0.500000,0.500000,0.500000}%
\pgfsetfillcolor{currentfill}%
\pgfsetfillopacity{0.300000}%
\pgfsetlinewidth{0.301125pt}%
\definecolor{currentstroke}{rgb}{0.500000,0.500000,0.500000}%
\pgfsetstrokecolor{currentstroke}%
\pgfsetstrokeopacity{0.300000}%
\pgfsetdash{}{0pt}%
\pgfpathmoveto{\pgfqpoint{0.000000in}{0.000000in}}%
\pgfpathlineto{\pgfqpoint{0.000000in}{0.000000in}}%
\pgfpathclose%
\pgfusepath{stroke,fill}%
\end{pgfscope}%
\begin{pgfscope}%
\pgfpathrectangle{\pgfqpoint{0.647939in}{0.492442in}}{\pgfqpoint{3.079299in}{3.079299in}}%
\pgfusepath{clip}%
\pgfsetroundcap%
\pgfsetroundjoin%
\pgfsetlinewidth{0.301125pt}%
\definecolor{currentstroke}{rgb}{0.500000,0.500000,0.500000}%
\pgfsetstrokecolor{currentstroke}%
\pgfsetstrokeopacity{0.300000}%
\pgfsetdash{}{0pt}%
\pgfpathmoveto{\pgfqpoint{2.854403in}{1.508719in}}%
\pgfusepath{stroke}%
\end{pgfscope}%
\begin{pgfscope}%
\pgfpathrectangle{\pgfqpoint{0.647939in}{0.492442in}}{\pgfqpoint{3.079299in}{3.079299in}}%
\pgfusepath{clip}%
\pgfsetroundcap%
\pgfsetroundjoin%
\definecolor{currentfill}{rgb}{0.500000,0.500000,0.500000}%
\pgfsetfillcolor{currentfill}%
\pgfsetfillopacity{0.300000}%
\pgfsetlinewidth{0.301125pt}%
\definecolor{currentstroke}{rgb}{0.500000,0.500000,0.500000}%
\pgfsetstrokecolor{currentstroke}%
\pgfsetstrokeopacity{0.300000}%
\pgfsetdash{}{0pt}%
\pgfpathmoveto{\pgfqpoint{0.000000in}{0.000000in}}%
\pgfpathlineto{\pgfqpoint{0.000000in}{0.000000in}}%
\pgfpathclose%
\pgfusepath{stroke,fill}%
\end{pgfscope}%
\begin{pgfscope}%
\pgfpathrectangle{\pgfqpoint{0.647939in}{0.492442in}}{\pgfqpoint{3.079299in}{3.079299in}}%
\pgfusepath{clip}%
\pgfsetroundcap%
\pgfsetroundjoin%
\pgfsetlinewidth{0.301125pt}%
\definecolor{currentstroke}{rgb}{0.500000,0.500000,0.500000}%
\pgfsetstrokecolor{currentstroke}%
\pgfsetstrokeopacity{0.300000}%
\pgfsetdash{}{0pt}%
\pgfpathmoveto{\pgfqpoint{2.928128in}{1.570762in}}%
\pgfusepath{stroke}%
\end{pgfscope}%
\begin{pgfscope}%
\pgfpathrectangle{\pgfqpoint{0.647939in}{0.492442in}}{\pgfqpoint{3.079299in}{3.079299in}}%
\pgfusepath{clip}%
\pgfsetroundcap%
\pgfsetroundjoin%
\definecolor{currentfill}{rgb}{0.500000,0.500000,0.500000}%
\pgfsetfillcolor{currentfill}%
\pgfsetfillopacity{0.300000}%
\pgfsetlinewidth{0.301125pt}%
\definecolor{currentstroke}{rgb}{0.500000,0.500000,0.500000}%
\pgfsetstrokecolor{currentstroke}%
\pgfsetstrokeopacity{0.300000}%
\pgfsetdash{}{0pt}%
\pgfpathmoveto{\pgfqpoint{0.000000in}{0.000000in}}%
\pgfpathlineto{\pgfqpoint{0.000000in}{0.000000in}}%
\pgfpathclose%
\pgfusepath{stroke,fill}%
\end{pgfscope}%
\begin{pgfscope}%
\pgfpathrectangle{\pgfqpoint{0.647939in}{0.492442in}}{\pgfqpoint{3.079299in}{3.079299in}}%
\pgfusepath{clip}%
\pgfsetroundcap%
\pgfsetroundjoin%
\pgfsetlinewidth{0.301125pt}%
\definecolor{currentstroke}{rgb}{0.500000,0.500000,0.500000}%
\pgfsetstrokecolor{currentstroke}%
\pgfsetstrokeopacity{0.300000}%
\pgfsetdash{}{0pt}%
\pgfpathmoveto{\pgfqpoint{3.000091in}{1.627803in}}%
\pgfusepath{stroke}%
\end{pgfscope}%
\begin{pgfscope}%
\pgfpathrectangle{\pgfqpoint{0.647939in}{0.492442in}}{\pgfqpoint{3.079299in}{3.079299in}}%
\pgfusepath{clip}%
\pgfsetroundcap%
\pgfsetroundjoin%
\definecolor{currentfill}{rgb}{0.500000,0.500000,0.500000}%
\pgfsetfillcolor{currentfill}%
\pgfsetfillopacity{0.300000}%
\pgfsetlinewidth{0.301125pt}%
\definecolor{currentstroke}{rgb}{0.500000,0.500000,0.500000}%
\pgfsetstrokecolor{currentstroke}%
\pgfsetstrokeopacity{0.300000}%
\pgfsetdash{}{0pt}%
\pgfpathmoveto{\pgfqpoint{0.000000in}{0.000000in}}%
\pgfpathlineto{\pgfqpoint{0.000000in}{0.000000in}}%
\pgfpathclose%
\pgfusepath{stroke,fill}%
\end{pgfscope}%
\begin{pgfscope}%
\pgfpathrectangle{\pgfqpoint{0.647939in}{0.492442in}}{\pgfqpoint{3.079299in}{3.079299in}}%
\pgfusepath{clip}%
\pgfsetroundcap%
\pgfsetroundjoin%
\pgfsetlinewidth{0.301125pt}%
\definecolor{currentstroke}{rgb}{0.500000,0.500000,0.500000}%
\pgfsetstrokecolor{currentstroke}%
\pgfsetstrokeopacity{0.300000}%
\pgfsetdash{}{0pt}%
\pgfpathmoveto{\pgfqpoint{3.085146in}{1.772078in}}%
\pgfusepath{stroke}%
\end{pgfscope}%
\begin{pgfscope}%
\pgfpathrectangle{\pgfqpoint{0.647939in}{0.492442in}}{\pgfqpoint{3.079299in}{3.079299in}}%
\pgfusepath{clip}%
\pgfsetroundcap%
\pgfsetroundjoin%
\definecolor{currentfill}{rgb}{0.500000,0.500000,0.500000}%
\pgfsetfillcolor{currentfill}%
\pgfsetfillopacity{0.300000}%
\pgfsetlinewidth{0.301125pt}%
\definecolor{currentstroke}{rgb}{0.500000,0.500000,0.500000}%
\pgfsetstrokecolor{currentstroke}%
\pgfsetstrokeopacity{0.300000}%
\pgfsetdash{}{0pt}%
\pgfpathmoveto{\pgfqpoint{0.000000in}{0.000000in}}%
\pgfpathlineto{\pgfqpoint{0.000000in}{0.000000in}}%
\pgfpathclose%
\pgfusepath{stroke,fill}%
\end{pgfscope}%
\begin{pgfscope}%
\pgfpathrectangle{\pgfqpoint{0.647939in}{0.492442in}}{\pgfqpoint{3.079299in}{3.079299in}}%
\pgfusepath{clip}%
\pgfsetroundcap%
\pgfsetroundjoin%
\pgfsetlinewidth{0.301125pt}%
\definecolor{currentstroke}{rgb}{0.500000,0.500000,0.500000}%
\pgfsetstrokecolor{currentstroke}%
\pgfsetstrokeopacity{0.300000}%
\pgfsetdash{}{0pt}%
\pgfpathmoveto{\pgfqpoint{2.997679in}{2.122637in}}%
\pgfusepath{stroke}%
\end{pgfscope}%
\begin{pgfscope}%
\pgfpathrectangle{\pgfqpoint{0.647939in}{0.492442in}}{\pgfqpoint{3.079299in}{3.079299in}}%
\pgfusepath{clip}%
\pgfsetroundcap%
\pgfsetroundjoin%
\definecolor{currentfill}{rgb}{0.500000,0.500000,0.500000}%
\pgfsetfillcolor{currentfill}%
\pgfsetfillopacity{0.300000}%
\pgfsetlinewidth{0.301125pt}%
\definecolor{currentstroke}{rgb}{0.500000,0.500000,0.500000}%
\pgfsetstrokecolor{currentstroke}%
\pgfsetstrokeopacity{0.300000}%
\pgfsetdash{}{0pt}%
\pgfpathmoveto{\pgfqpoint{0.000000in}{0.000000in}}%
\pgfpathlineto{\pgfqpoint{0.000000in}{0.000000in}}%
\pgfpathclose%
\pgfusepath{stroke,fill}%
\end{pgfscope}%
\begin{pgfscope}%
\pgfpathrectangle{\pgfqpoint{0.647939in}{0.492442in}}{\pgfqpoint{3.079299in}{3.079299in}}%
\pgfusepath{clip}%
\pgfsetroundcap%
\pgfsetroundjoin%
\pgfsetlinewidth{0.301125pt}%
\definecolor{currentstroke}{rgb}{0.500000,0.500000,0.500000}%
\pgfsetstrokecolor{currentstroke}%
\pgfsetstrokeopacity{0.300000}%
\pgfsetdash{}{0pt}%
\pgfpathmoveto{\pgfqpoint{3.110315in}{2.114575in}}%
\pgfusepath{stroke}%
\end{pgfscope}%
\begin{pgfscope}%
\pgfpathrectangle{\pgfqpoint{0.647939in}{0.492442in}}{\pgfqpoint{3.079299in}{3.079299in}}%
\pgfusepath{clip}%
\pgfsetroundcap%
\pgfsetroundjoin%
\definecolor{currentfill}{rgb}{0.500000,0.500000,0.500000}%
\pgfsetfillcolor{currentfill}%
\pgfsetfillopacity{0.300000}%
\pgfsetlinewidth{0.301125pt}%
\definecolor{currentstroke}{rgb}{0.500000,0.500000,0.500000}%
\pgfsetstrokecolor{currentstroke}%
\pgfsetstrokeopacity{0.300000}%
\pgfsetdash{}{0pt}%
\pgfpathmoveto{\pgfqpoint{0.000000in}{0.000000in}}%
\pgfpathlineto{\pgfqpoint{0.000000in}{0.000000in}}%
\pgfpathclose%
\pgfusepath{stroke,fill}%
\end{pgfscope}%
\begin{pgfscope}%
\pgfpathrectangle{\pgfqpoint{0.647939in}{0.492442in}}{\pgfqpoint{3.079299in}{3.079299in}}%
\pgfusepath{clip}%
\pgfsetroundcap%
\pgfsetroundjoin%
\pgfsetlinewidth{0.301125pt}%
\definecolor{currentstroke}{rgb}{0.500000,0.500000,0.500000}%
\pgfsetstrokecolor{currentstroke}%
\pgfsetstrokeopacity{0.300000}%
\pgfsetdash{}{0pt}%
\pgfpathmoveto{\pgfqpoint{3.142052in}{2.587090in}}%
\pgfusepath{stroke}%
\end{pgfscope}%
\begin{pgfscope}%
\pgfpathrectangle{\pgfqpoint{0.647939in}{0.492442in}}{\pgfqpoint{3.079299in}{3.079299in}}%
\pgfusepath{clip}%
\pgfsetroundcap%
\pgfsetroundjoin%
\definecolor{currentfill}{rgb}{0.500000,0.500000,0.500000}%
\pgfsetfillcolor{currentfill}%
\pgfsetfillopacity{0.300000}%
\pgfsetlinewidth{0.301125pt}%
\definecolor{currentstroke}{rgb}{0.500000,0.500000,0.500000}%
\pgfsetstrokecolor{currentstroke}%
\pgfsetstrokeopacity{0.300000}%
\pgfsetdash{}{0pt}%
\pgfpathmoveto{\pgfqpoint{0.000000in}{0.000000in}}%
\pgfpathlineto{\pgfqpoint{0.000000in}{0.000000in}}%
\pgfpathclose%
\pgfusepath{stroke,fill}%
\end{pgfscope}%
\begin{pgfscope}%
\pgfpathrectangle{\pgfqpoint{0.647939in}{0.492442in}}{\pgfqpoint{3.079299in}{3.079299in}}%
\pgfusepath{clip}%
\pgfsetroundcap%
\pgfsetroundjoin%
\pgfsetlinewidth{0.301125pt}%
\definecolor{currentstroke}{rgb}{0.500000,0.500000,0.500000}%
\pgfsetstrokecolor{currentstroke}%
\pgfsetstrokeopacity{0.300000}%
\pgfsetdash{}{0pt}%
\pgfpathmoveto{\pgfqpoint{3.283739in}{2.657521in}}%
\pgfusepath{stroke}%
\end{pgfscope}%
\begin{pgfscope}%
\pgfpathrectangle{\pgfqpoint{0.647939in}{0.492442in}}{\pgfqpoint{3.079299in}{3.079299in}}%
\pgfusepath{clip}%
\pgfsetroundcap%
\pgfsetroundjoin%
\definecolor{currentfill}{rgb}{0.500000,0.500000,0.500000}%
\pgfsetfillcolor{currentfill}%
\pgfsetfillopacity{0.300000}%
\pgfsetlinewidth{0.301125pt}%
\definecolor{currentstroke}{rgb}{0.500000,0.500000,0.500000}%
\pgfsetstrokecolor{currentstroke}%
\pgfsetstrokeopacity{0.300000}%
\pgfsetdash{}{0pt}%
\pgfpathmoveto{\pgfqpoint{0.000000in}{0.000000in}}%
\pgfpathlineto{\pgfqpoint{0.000000in}{0.000000in}}%
\pgfpathclose%
\pgfusepath{stroke,fill}%
\end{pgfscope}%
\begin{pgfscope}%
\pgfpathrectangle{\pgfqpoint{0.647939in}{0.492442in}}{\pgfqpoint{3.079299in}{3.079299in}}%
\pgfusepath{clip}%
\pgfsetroundcap%
\pgfsetroundjoin%
\pgfsetlinewidth{0.301125pt}%
\definecolor{currentstroke}{rgb}{0.500000,0.500000,0.500000}%
\pgfsetstrokecolor{currentstroke}%
\pgfsetstrokeopacity{0.300000}%
\pgfsetdash{}{0pt}%
\pgfpathmoveto{\pgfqpoint{3.562737in}{2.056445in}}%
\pgfusepath{stroke}%
\end{pgfscope}%
\begin{pgfscope}%
\pgfpathrectangle{\pgfqpoint{0.647939in}{0.492442in}}{\pgfqpoint{3.079299in}{3.079299in}}%
\pgfusepath{clip}%
\pgfsetroundcap%
\pgfsetroundjoin%
\definecolor{currentfill}{rgb}{0.500000,0.500000,0.500000}%
\pgfsetfillcolor{currentfill}%
\pgfsetfillopacity{0.300000}%
\pgfsetlinewidth{0.301125pt}%
\definecolor{currentstroke}{rgb}{0.500000,0.500000,0.500000}%
\pgfsetstrokecolor{currentstroke}%
\pgfsetstrokeopacity{0.300000}%
\pgfsetdash{}{0pt}%
\pgfpathmoveto{\pgfqpoint{0.000000in}{0.000000in}}%
\pgfpathlineto{\pgfqpoint{0.000000in}{0.000000in}}%
\pgfpathclose%
\pgfusepath{stroke,fill}%
\end{pgfscope}%
\begin{pgfscope}%
\pgfpathrectangle{\pgfqpoint{0.647939in}{0.492442in}}{\pgfqpoint{3.079299in}{3.079299in}}%
\pgfusepath{clip}%
\pgfsetroundcap%
\pgfsetroundjoin%
\pgfsetlinewidth{0.301125pt}%
\definecolor{currentstroke}{rgb}{0.500000,0.500000,0.500000}%
\pgfsetstrokecolor{currentstroke}%
\pgfsetstrokeopacity{0.300000}%
\pgfsetdash{}{0pt}%
\pgfpathmoveto{\pgfqpoint{3.455912in}{2.667956in}}%
\pgfusepath{stroke}%
\end{pgfscope}%
\begin{pgfscope}%
\pgfpathrectangle{\pgfqpoint{0.647939in}{0.492442in}}{\pgfqpoint{3.079299in}{3.079299in}}%
\pgfusepath{clip}%
\pgfsetroundcap%
\pgfsetroundjoin%
\definecolor{currentfill}{rgb}{0.500000,0.500000,0.500000}%
\pgfsetfillcolor{currentfill}%
\pgfsetfillopacity{0.300000}%
\pgfsetlinewidth{0.301125pt}%
\definecolor{currentstroke}{rgb}{0.500000,0.500000,0.500000}%
\pgfsetstrokecolor{currentstroke}%
\pgfsetstrokeopacity{0.300000}%
\pgfsetdash{}{0pt}%
\pgfpathmoveto{\pgfqpoint{0.000000in}{0.000000in}}%
\pgfpathlineto{\pgfqpoint{0.000000in}{0.000000in}}%
\pgfpathclose%
\pgfusepath{stroke,fill}%
\end{pgfscope}%
\begin{pgfscope}%
\pgfpathrectangle{\pgfqpoint{0.647939in}{0.492442in}}{\pgfqpoint{3.079299in}{3.079299in}}%
\pgfusepath{clip}%
\pgfsetroundcap%
\pgfsetroundjoin%
\pgfsetlinewidth{0.301125pt}%
\definecolor{currentstroke}{rgb}{0.500000,0.500000,0.500000}%
\pgfsetstrokecolor{currentstroke}%
\pgfsetstrokeopacity{0.300000}%
\pgfsetdash{}{0pt}%
\pgfpathmoveto{\pgfqpoint{3.551039in}{2.637308in}}%
\pgfusepath{stroke}%
\end{pgfscope}%
\begin{pgfscope}%
\pgfpathrectangle{\pgfqpoint{0.647939in}{0.492442in}}{\pgfqpoint{3.079299in}{3.079299in}}%
\pgfusepath{clip}%
\pgfsetroundcap%
\pgfsetroundjoin%
\definecolor{currentfill}{rgb}{0.500000,0.500000,0.500000}%
\pgfsetfillcolor{currentfill}%
\pgfsetfillopacity{0.300000}%
\pgfsetlinewidth{0.301125pt}%
\definecolor{currentstroke}{rgb}{0.500000,0.500000,0.500000}%
\pgfsetstrokecolor{currentstroke}%
\pgfsetstrokeopacity{0.300000}%
\pgfsetdash{}{0pt}%
\pgfpathmoveto{\pgfqpoint{0.000000in}{0.000000in}}%
\pgfpathlineto{\pgfqpoint{0.000000in}{0.000000in}}%
\pgfpathclose%
\pgfusepath{stroke,fill}%
\end{pgfscope}%
\begin{pgfscope}%
\pgfpathrectangle{\pgfqpoint{0.647939in}{0.492442in}}{\pgfqpoint{3.079299in}{3.079299in}}%
\pgfusepath{clip}%
\pgfsetroundcap%
\pgfsetroundjoin%
\pgfsetlinewidth{0.301125pt}%
\definecolor{currentstroke}{rgb}{0.500000,0.500000,0.500000}%
\pgfsetstrokecolor{currentstroke}%
\pgfsetstrokeopacity{0.300000}%
\pgfsetdash{}{0pt}%
\pgfpathmoveto{\pgfqpoint{3.628774in}{2.663859in}}%
\pgfusepath{stroke}%
\end{pgfscope}%
\begin{pgfscope}%
\pgfpathrectangle{\pgfqpoint{0.647939in}{0.492442in}}{\pgfqpoint{3.079299in}{3.079299in}}%
\pgfusepath{clip}%
\pgfsetroundcap%
\pgfsetroundjoin%
\definecolor{currentfill}{rgb}{0.500000,0.500000,0.500000}%
\pgfsetfillcolor{currentfill}%
\pgfsetfillopacity{0.300000}%
\pgfsetlinewidth{0.301125pt}%
\definecolor{currentstroke}{rgb}{0.500000,0.500000,0.500000}%
\pgfsetstrokecolor{currentstroke}%
\pgfsetstrokeopacity{0.300000}%
\pgfsetdash{}{0pt}%
\pgfpathmoveto{\pgfqpoint{0.000000in}{0.000000in}}%
\pgfpathlineto{\pgfqpoint{0.000000in}{0.000000in}}%
\pgfpathclose%
\pgfusepath{stroke,fill}%
\end{pgfscope}%
\begin{pgfscope}%
\pgfpathrectangle{\pgfqpoint{0.647939in}{0.492442in}}{\pgfqpoint{3.079299in}{3.079299in}}%
\pgfusepath{clip}%
\pgfsetroundcap%
\pgfsetroundjoin%
\pgfsetlinewidth{0.301125pt}%
\definecolor{currentstroke}{rgb}{0.500000,0.500000,0.500000}%
\pgfsetstrokecolor{currentstroke}%
\pgfsetstrokeopacity{0.300000}%
\pgfsetdash{}{0pt}%
\pgfpathmoveto{\pgfqpoint{3.701945in}{2.697041in}}%
\pgfusepath{stroke}%
\end{pgfscope}%
\begin{pgfscope}%
\pgfpathrectangle{\pgfqpoint{0.647939in}{0.492442in}}{\pgfqpoint{3.079299in}{3.079299in}}%
\pgfusepath{clip}%
\pgfsetroundcap%
\pgfsetroundjoin%
\definecolor{currentfill}{rgb}{0.500000,0.500000,0.500000}%
\pgfsetfillcolor{currentfill}%
\pgfsetfillopacity{0.300000}%
\pgfsetlinewidth{0.301125pt}%
\definecolor{currentstroke}{rgb}{0.500000,0.500000,0.500000}%
\pgfsetstrokecolor{currentstroke}%
\pgfsetstrokeopacity{0.300000}%
\pgfsetdash{}{0pt}%
\pgfpathmoveto{\pgfqpoint{0.000000in}{0.000000in}}%
\pgfpathlineto{\pgfqpoint{0.000000in}{0.000000in}}%
\pgfpathclose%
\pgfusepath{stroke,fill}%
\end{pgfscope}%
\begin{pgfscope}%
\pgfpathrectangle{\pgfqpoint{0.647939in}{0.492442in}}{\pgfqpoint{3.079299in}{3.079299in}}%
\pgfusepath{clip}%
\pgfsetroundcap%
\pgfsetroundjoin%
\pgfsetlinewidth{0.301125pt}%
\definecolor{currentstroke}{rgb}{0.500000,0.500000,0.500000}%
\pgfsetstrokecolor{currentstroke}%
\pgfsetstrokeopacity{0.300000}%
\pgfsetdash{}{0pt}%
\pgfpathmoveto{\pgfqpoint{3.474171in}{3.407853in}}%
\pgfusepath{stroke}%
\end{pgfscope}%
\begin{pgfscope}%
\pgfpathrectangle{\pgfqpoint{0.647939in}{0.492442in}}{\pgfqpoint{3.079299in}{3.079299in}}%
\pgfusepath{clip}%
\pgfsetroundcap%
\pgfsetroundjoin%
\definecolor{currentfill}{rgb}{0.500000,0.500000,0.500000}%
\pgfsetfillcolor{currentfill}%
\pgfsetfillopacity{0.300000}%
\pgfsetlinewidth{0.301125pt}%
\definecolor{currentstroke}{rgb}{0.500000,0.500000,0.500000}%
\pgfsetstrokecolor{currentstroke}%
\pgfsetstrokeopacity{0.300000}%
\pgfsetdash{}{0pt}%
\pgfpathmoveto{\pgfqpoint{0.000000in}{0.000000in}}%
\pgfpathlineto{\pgfqpoint{0.000000in}{0.000000in}}%
\pgfpathclose%
\pgfusepath{stroke,fill}%
\end{pgfscope}%
\begin{pgfscope}%
\pgfpathrectangle{\pgfqpoint{0.647939in}{0.492442in}}{\pgfqpoint{3.079299in}{3.079299in}}%
\pgfusepath{clip}%
\pgfsetroundcap%
\pgfsetroundjoin%
\pgfsetlinewidth{0.301125pt}%
\definecolor{currentstroke}{rgb}{0.500000,0.500000,0.500000}%
\pgfsetstrokecolor{currentstroke}%
\pgfsetstrokeopacity{0.300000}%
\pgfsetdash{}{0pt}%
\pgfpathmoveto{\pgfqpoint{2.167049in}{2.811187in}}%
\pgfusepath{stroke}%
\end{pgfscope}%
\begin{pgfscope}%
\pgfpathrectangle{\pgfqpoint{0.647939in}{0.492442in}}{\pgfqpoint{3.079299in}{3.079299in}}%
\pgfusepath{clip}%
\pgfsetroundcap%
\pgfsetroundjoin%
\definecolor{currentfill}{rgb}{0.500000,0.500000,0.500000}%
\pgfsetfillcolor{currentfill}%
\pgfsetfillopacity{0.300000}%
\pgfsetlinewidth{0.301125pt}%
\definecolor{currentstroke}{rgb}{0.500000,0.500000,0.500000}%
\pgfsetstrokecolor{currentstroke}%
\pgfsetstrokeopacity{0.300000}%
\pgfsetdash{}{0pt}%
\pgfpathmoveto{\pgfqpoint{0.000000in}{0.000000in}}%
\pgfpathlineto{\pgfqpoint{0.000000in}{0.000000in}}%
\pgfpathclose%
\pgfusepath{stroke,fill}%
\end{pgfscope}%
\begin{pgfscope}%
\pgfpathrectangle{\pgfqpoint{0.647939in}{0.492442in}}{\pgfqpoint{3.079299in}{3.079299in}}%
\pgfusepath{clip}%
\pgfsetroundcap%
\pgfsetroundjoin%
\pgfsetlinewidth{0.301125pt}%
\definecolor{currentstroke}{rgb}{0.500000,0.500000,0.500000}%
\pgfsetstrokecolor{currentstroke}%
\pgfsetstrokeopacity{0.300000}%
\pgfsetdash{}{0pt}%
\pgfpathmoveto{\pgfqpoint{2.040511in}{3.077106in}}%
\pgfusepath{stroke}%
\end{pgfscope}%
\begin{pgfscope}%
\pgfpathrectangle{\pgfqpoint{0.647939in}{0.492442in}}{\pgfqpoint{3.079299in}{3.079299in}}%
\pgfusepath{clip}%
\pgfsetroundcap%
\pgfsetroundjoin%
\definecolor{currentfill}{rgb}{0.500000,0.500000,0.500000}%
\pgfsetfillcolor{currentfill}%
\pgfsetfillopacity{0.300000}%
\pgfsetlinewidth{0.301125pt}%
\definecolor{currentstroke}{rgb}{0.500000,0.500000,0.500000}%
\pgfsetstrokecolor{currentstroke}%
\pgfsetstrokeopacity{0.300000}%
\pgfsetdash{}{0pt}%
\pgfpathmoveto{\pgfqpoint{0.000000in}{0.000000in}}%
\pgfpathlineto{\pgfqpoint{0.000000in}{0.000000in}}%
\pgfpathclose%
\pgfusepath{stroke,fill}%
\end{pgfscope}%
\begin{pgfscope}%
\pgfpathrectangle{\pgfqpoint{0.647939in}{0.492442in}}{\pgfqpoint{3.079299in}{3.079299in}}%
\pgfusepath{clip}%
\pgfsetroundcap%
\pgfsetroundjoin%
\pgfsetlinewidth{0.301125pt}%
\definecolor{currentstroke}{rgb}{0.500000,0.500000,0.500000}%
\pgfsetstrokecolor{currentstroke}%
\pgfsetstrokeopacity{0.300000}%
\pgfsetdash{}{0pt}%
\pgfpathmoveto{\pgfqpoint{1.903500in}{3.240509in}}%
\pgfusepath{stroke}%
\end{pgfscope}%
\begin{pgfscope}%
\pgfpathrectangle{\pgfqpoint{0.647939in}{0.492442in}}{\pgfqpoint{3.079299in}{3.079299in}}%
\pgfusepath{clip}%
\pgfsetroundcap%
\pgfsetroundjoin%
\definecolor{currentfill}{rgb}{0.500000,0.500000,0.500000}%
\pgfsetfillcolor{currentfill}%
\pgfsetfillopacity{0.300000}%
\pgfsetlinewidth{0.301125pt}%
\definecolor{currentstroke}{rgb}{0.500000,0.500000,0.500000}%
\pgfsetstrokecolor{currentstroke}%
\pgfsetstrokeopacity{0.300000}%
\pgfsetdash{}{0pt}%
\pgfpathmoveto{\pgfqpoint{0.000000in}{0.000000in}}%
\pgfpathlineto{\pgfqpoint{0.000000in}{0.000000in}}%
\pgfpathclose%
\pgfusepath{stroke,fill}%
\end{pgfscope}%
\begin{pgfscope}%
\pgfpathrectangle{\pgfqpoint{0.647939in}{0.492442in}}{\pgfqpoint{3.079299in}{3.079299in}}%
\pgfusepath{clip}%
\pgfsetroundcap%
\pgfsetroundjoin%
\pgfsetlinewidth{0.301125pt}%
\definecolor{currentstroke}{rgb}{0.500000,0.500000,0.500000}%
\pgfsetstrokecolor{currentstroke}%
\pgfsetstrokeopacity{0.300000}%
\pgfsetdash{}{0pt}%
\pgfpathmoveto{\pgfqpoint{1.861748in}{3.358131in}}%
\pgfusepath{stroke}%
\end{pgfscope}%
\begin{pgfscope}%
\pgfpathrectangle{\pgfqpoint{0.647939in}{0.492442in}}{\pgfqpoint{3.079299in}{3.079299in}}%
\pgfusepath{clip}%
\pgfsetroundcap%
\pgfsetroundjoin%
\definecolor{currentfill}{rgb}{0.500000,0.500000,0.500000}%
\pgfsetfillcolor{currentfill}%
\pgfsetfillopacity{0.300000}%
\pgfsetlinewidth{0.301125pt}%
\definecolor{currentstroke}{rgb}{0.500000,0.500000,0.500000}%
\pgfsetstrokecolor{currentstroke}%
\pgfsetstrokeopacity{0.300000}%
\pgfsetdash{}{0pt}%
\pgfpathmoveto{\pgfqpoint{0.000000in}{0.000000in}}%
\pgfpathlineto{\pgfqpoint{0.000000in}{0.000000in}}%
\pgfpathclose%
\pgfusepath{stroke,fill}%
\end{pgfscope}%
\begin{pgfscope}%
\pgfpathrectangle{\pgfqpoint{0.647939in}{0.492442in}}{\pgfqpoint{3.079299in}{3.079299in}}%
\pgfusepath{clip}%
\pgfsetroundcap%
\pgfsetroundjoin%
\pgfsetlinewidth{0.301125pt}%
\definecolor{currentstroke}{rgb}{0.500000,0.500000,0.500000}%
\pgfsetstrokecolor{currentstroke}%
\pgfsetstrokeopacity{0.300000}%
\pgfsetdash{}{0pt}%
\pgfpathmoveto{\pgfqpoint{1.768881in}{3.434714in}}%
\pgfusepath{stroke}%
\end{pgfscope}%
\begin{pgfscope}%
\pgfpathrectangle{\pgfqpoint{0.647939in}{0.492442in}}{\pgfqpoint{3.079299in}{3.079299in}}%
\pgfusepath{clip}%
\pgfsetroundcap%
\pgfsetroundjoin%
\definecolor{currentfill}{rgb}{0.500000,0.500000,0.500000}%
\pgfsetfillcolor{currentfill}%
\pgfsetfillopacity{0.300000}%
\pgfsetlinewidth{0.301125pt}%
\definecolor{currentstroke}{rgb}{0.500000,0.500000,0.500000}%
\pgfsetstrokecolor{currentstroke}%
\pgfsetstrokeopacity{0.300000}%
\pgfsetdash{}{0pt}%
\pgfpathmoveto{\pgfqpoint{0.000000in}{0.000000in}}%
\pgfpathlineto{\pgfqpoint{0.000000in}{0.000000in}}%
\pgfpathclose%
\pgfusepath{stroke,fill}%
\end{pgfscope}%
\begin{pgfscope}%
\pgfpathrectangle{\pgfqpoint{0.647939in}{0.492442in}}{\pgfqpoint{3.079299in}{3.079299in}}%
\pgfusepath{clip}%
\pgfsetroundcap%
\pgfsetroundjoin%
\pgfsetlinewidth{0.301125pt}%
\definecolor{currentstroke}{rgb}{0.500000,0.500000,0.500000}%
\pgfsetstrokecolor{currentstroke}%
\pgfsetstrokeopacity{0.300000}%
\pgfsetdash{}{0pt}%
\pgfpathmoveto{\pgfqpoint{1.683038in}{3.501647in}}%
\pgfusepath{stroke}%
\end{pgfscope}%
\begin{pgfscope}%
\pgfpathrectangle{\pgfqpoint{0.647939in}{0.492442in}}{\pgfqpoint{3.079299in}{3.079299in}}%
\pgfusepath{clip}%
\pgfsetroundcap%
\pgfsetroundjoin%
\definecolor{currentfill}{rgb}{0.500000,0.500000,0.500000}%
\pgfsetfillcolor{currentfill}%
\pgfsetfillopacity{0.300000}%
\pgfsetlinewidth{0.301125pt}%
\definecolor{currentstroke}{rgb}{0.500000,0.500000,0.500000}%
\pgfsetstrokecolor{currentstroke}%
\pgfsetstrokeopacity{0.300000}%
\pgfsetdash{}{0pt}%
\pgfpathmoveto{\pgfqpoint{0.000000in}{0.000000in}}%
\pgfpathlineto{\pgfqpoint{0.000000in}{0.000000in}}%
\pgfpathclose%
\pgfusepath{stroke,fill}%
\end{pgfscope}%
\begin{pgfscope}%
\pgfpathrectangle{\pgfqpoint{0.647939in}{0.492442in}}{\pgfqpoint{3.079299in}{3.079299in}}%
\pgfusepath{clip}%
\pgfsetroundcap%
\pgfsetroundjoin%
\pgfsetlinewidth{0.301125pt}%
\definecolor{currentstroke}{rgb}{0.500000,0.500000,0.500000}%
\pgfsetstrokecolor{currentstroke}%
\pgfsetstrokeopacity{0.300000}%
\pgfsetdash{}{0pt}%
\pgfpathmoveto{\pgfqpoint{2.083232in}{3.566489in}}%
\pgfusepath{stroke}%
\end{pgfscope}%
\begin{pgfscope}%
\pgfpathrectangle{\pgfqpoint{0.647939in}{0.492442in}}{\pgfqpoint{3.079299in}{3.079299in}}%
\pgfusepath{clip}%
\pgfsetroundcap%
\pgfsetroundjoin%
\definecolor{currentfill}{rgb}{0.500000,0.500000,0.500000}%
\pgfsetfillcolor{currentfill}%
\pgfsetfillopacity{0.300000}%
\pgfsetlinewidth{0.301125pt}%
\definecolor{currentstroke}{rgb}{0.500000,0.500000,0.500000}%
\pgfsetstrokecolor{currentstroke}%
\pgfsetstrokeopacity{0.300000}%
\pgfsetdash{}{0pt}%
\pgfpathmoveto{\pgfqpoint{0.000000in}{0.000000in}}%
\pgfpathlineto{\pgfqpoint{0.000000in}{0.000000in}}%
\pgfpathclose%
\pgfusepath{stroke,fill}%
\end{pgfscope}%
\begin{pgfscope}%
\pgfpathrectangle{\pgfqpoint{0.647939in}{0.492442in}}{\pgfqpoint{3.079299in}{3.079299in}}%
\pgfusepath{clip}%
\pgfsetroundcap%
\pgfsetroundjoin%
\pgfsetlinewidth{0.301125pt}%
\definecolor{currentstroke}{rgb}{0.500000,0.500000,0.500000}%
\pgfsetstrokecolor{currentstroke}%
\pgfsetstrokeopacity{0.300000}%
\pgfsetdash{}{0pt}%
\pgfpathmoveto{\pgfqpoint{1.117371in}{3.474185in}}%
\pgfusepath{stroke}%
\end{pgfscope}%
\begin{pgfscope}%
\pgfpathrectangle{\pgfqpoint{0.647939in}{0.492442in}}{\pgfqpoint{3.079299in}{3.079299in}}%
\pgfusepath{clip}%
\pgfsetroundcap%
\pgfsetroundjoin%
\definecolor{currentfill}{rgb}{0.500000,0.500000,0.500000}%
\pgfsetfillcolor{currentfill}%
\pgfsetfillopacity{0.300000}%
\pgfsetlinewidth{0.301125pt}%
\definecolor{currentstroke}{rgb}{0.500000,0.500000,0.500000}%
\pgfsetstrokecolor{currentstroke}%
\pgfsetstrokeopacity{0.300000}%
\pgfsetdash{}{0pt}%
\pgfpathmoveto{\pgfqpoint{0.000000in}{0.000000in}}%
\pgfpathlineto{\pgfqpoint{0.000000in}{0.000000in}}%
\pgfpathclose%
\pgfusepath{stroke,fill}%
\end{pgfscope}%
\begin{pgfscope}%
\pgfpathrectangle{\pgfqpoint{0.647939in}{0.492442in}}{\pgfqpoint{3.079299in}{3.079299in}}%
\pgfusepath{clip}%
\pgfsetroundcap%
\pgfsetroundjoin%
\pgfsetlinewidth{0.301125pt}%
\definecolor{currentstroke}{rgb}{0.500000,0.500000,0.500000}%
\pgfsetstrokecolor{currentstroke}%
\pgfsetstrokeopacity{0.300000}%
\pgfsetdash{}{0pt}%
\pgfpathmoveto{\pgfqpoint{0.898691in}{3.514647in}}%
\pgfusepath{stroke}%
\end{pgfscope}%
\begin{pgfscope}%
\pgfpathrectangle{\pgfqpoint{0.647939in}{0.492442in}}{\pgfqpoint{3.079299in}{3.079299in}}%
\pgfusepath{clip}%
\pgfsetroundcap%
\pgfsetroundjoin%
\definecolor{currentfill}{rgb}{0.500000,0.500000,0.500000}%
\pgfsetfillcolor{currentfill}%
\pgfsetfillopacity{0.300000}%
\pgfsetlinewidth{0.301125pt}%
\definecolor{currentstroke}{rgb}{0.500000,0.500000,0.500000}%
\pgfsetstrokecolor{currentstroke}%
\pgfsetstrokeopacity{0.300000}%
\pgfsetdash{}{0pt}%
\pgfpathmoveto{\pgfqpoint{0.000000in}{0.000000in}}%
\pgfpathlineto{\pgfqpoint{0.000000in}{0.000000in}}%
\pgfpathclose%
\pgfusepath{stroke,fill}%
\end{pgfscope}%
\begin{pgfscope}%
\pgfpathrectangle{\pgfqpoint{0.647939in}{0.492442in}}{\pgfqpoint{3.079299in}{3.079299in}}%
\pgfusepath{clip}%
\pgfsetroundcap%
\pgfsetroundjoin%
\pgfsetlinewidth{0.301125pt}%
\definecolor{currentstroke}{rgb}{0.500000,0.500000,0.500000}%
\pgfsetstrokecolor{currentstroke}%
\pgfsetstrokeopacity{0.300000}%
\pgfsetdash{}{0pt}%
\pgfpathmoveto{\pgfqpoint{1.479803in}{3.107874in}}%
\pgfusepath{stroke}%
\end{pgfscope}%
\begin{pgfscope}%
\pgfpathrectangle{\pgfqpoint{0.647939in}{0.492442in}}{\pgfqpoint{3.079299in}{3.079299in}}%
\pgfusepath{clip}%
\pgfsetroundcap%
\pgfsetroundjoin%
\definecolor{currentfill}{rgb}{0.500000,0.500000,0.500000}%
\pgfsetfillcolor{currentfill}%
\pgfsetfillopacity{0.300000}%
\pgfsetlinewidth{0.301125pt}%
\definecolor{currentstroke}{rgb}{0.500000,0.500000,0.500000}%
\pgfsetstrokecolor{currentstroke}%
\pgfsetstrokeopacity{0.300000}%
\pgfsetdash{}{0pt}%
\pgfpathmoveto{\pgfqpoint{0.000000in}{0.000000in}}%
\pgfpathlineto{\pgfqpoint{0.000000in}{0.000000in}}%
\pgfpathclose%
\pgfusepath{stroke,fill}%
\end{pgfscope}%
\begin{pgfscope}%
\pgfpathrectangle{\pgfqpoint{0.647939in}{0.492442in}}{\pgfqpoint{3.079299in}{3.079299in}}%
\pgfusepath{clip}%
\pgfsetroundcap%
\pgfsetroundjoin%
\pgfsetlinewidth{0.301125pt}%
\definecolor{currentstroke}{rgb}{0.500000,0.500000,0.500000}%
\pgfsetstrokecolor{currentstroke}%
\pgfsetstrokeopacity{0.300000}%
\pgfsetdash{}{0pt}%
\pgfpathmoveto{\pgfqpoint{1.745615in}{2.962746in}}%
\pgfusepath{stroke}%
\end{pgfscope}%
\begin{pgfscope}%
\pgfpathrectangle{\pgfqpoint{0.647939in}{0.492442in}}{\pgfqpoint{3.079299in}{3.079299in}}%
\pgfusepath{clip}%
\pgfsetroundcap%
\pgfsetroundjoin%
\definecolor{currentfill}{rgb}{0.500000,0.500000,0.500000}%
\pgfsetfillcolor{currentfill}%
\pgfsetfillopacity{0.300000}%
\pgfsetlinewidth{0.301125pt}%
\definecolor{currentstroke}{rgb}{0.500000,0.500000,0.500000}%
\pgfsetstrokecolor{currentstroke}%
\pgfsetstrokeopacity{0.300000}%
\pgfsetdash{}{0pt}%
\pgfpathmoveto{\pgfqpoint{0.000000in}{0.000000in}}%
\pgfpathlineto{\pgfqpoint{0.000000in}{0.000000in}}%
\pgfpathclose%
\pgfusepath{stroke,fill}%
\end{pgfscope}%
\begin{pgfscope}%
\pgfpathrectangle{\pgfqpoint{0.647939in}{0.492442in}}{\pgfqpoint{3.079299in}{3.079299in}}%
\pgfusepath{clip}%
\pgfsetroundcap%
\pgfsetroundjoin%
\pgfsetlinewidth{0.301125pt}%
\definecolor{currentstroke}{rgb}{0.500000,0.500000,0.500000}%
\pgfsetstrokecolor{currentstroke}%
\pgfsetstrokeopacity{0.300000}%
\pgfsetdash{}{0pt}%
\pgfpathmoveto{\pgfqpoint{1.543058in}{2.858414in}}%
\pgfusepath{stroke}%
\end{pgfscope}%
\begin{pgfscope}%
\pgfpathrectangle{\pgfqpoint{0.647939in}{0.492442in}}{\pgfqpoint{3.079299in}{3.079299in}}%
\pgfusepath{clip}%
\pgfsetroundcap%
\pgfsetroundjoin%
\definecolor{currentfill}{rgb}{0.500000,0.500000,0.500000}%
\pgfsetfillcolor{currentfill}%
\pgfsetfillopacity{0.300000}%
\pgfsetlinewidth{0.301125pt}%
\definecolor{currentstroke}{rgb}{0.500000,0.500000,0.500000}%
\pgfsetstrokecolor{currentstroke}%
\pgfsetstrokeopacity{0.300000}%
\pgfsetdash{}{0pt}%
\pgfpathmoveto{\pgfqpoint{0.000000in}{0.000000in}}%
\pgfpathlineto{\pgfqpoint{0.000000in}{0.000000in}}%
\pgfpathclose%
\pgfusepath{stroke,fill}%
\end{pgfscope}%
\begin{pgfscope}%
\pgfpathrectangle{\pgfqpoint{0.647939in}{0.492442in}}{\pgfqpoint{3.079299in}{3.079299in}}%
\pgfusepath{clip}%
\pgfsetroundcap%
\pgfsetroundjoin%
\pgfsetlinewidth{0.301125pt}%
\definecolor{currentstroke}{rgb}{0.500000,0.500000,0.500000}%
\pgfsetstrokecolor{currentstroke}%
\pgfsetstrokeopacity{0.300000}%
\pgfsetdash{}{0pt}%
\pgfpathmoveto{\pgfqpoint{1.011787in}{2.517418in}}%
\pgfusepath{stroke}%
\end{pgfscope}%
\begin{pgfscope}%
\pgfpathrectangle{\pgfqpoint{0.647939in}{0.492442in}}{\pgfqpoint{3.079299in}{3.079299in}}%
\pgfusepath{clip}%
\pgfsetroundcap%
\pgfsetroundjoin%
\definecolor{currentfill}{rgb}{0.500000,0.500000,0.500000}%
\pgfsetfillcolor{currentfill}%
\pgfsetfillopacity{0.300000}%
\pgfsetlinewidth{0.301125pt}%
\definecolor{currentstroke}{rgb}{0.500000,0.500000,0.500000}%
\pgfsetstrokecolor{currentstroke}%
\pgfsetstrokeopacity{0.300000}%
\pgfsetdash{}{0pt}%
\pgfpathmoveto{\pgfqpoint{0.000000in}{0.000000in}}%
\pgfpathlineto{\pgfqpoint{0.000000in}{0.000000in}}%
\pgfpathclose%
\pgfusepath{stroke,fill}%
\end{pgfscope}%
\begin{pgfscope}%
\pgfpathrectangle{\pgfqpoint{0.647939in}{0.492442in}}{\pgfqpoint{3.079299in}{3.079299in}}%
\pgfusepath{clip}%
\pgfsetroundcap%
\pgfsetroundjoin%
\pgfsetlinewidth{0.301125pt}%
\definecolor{currentstroke}{rgb}{0.500000,0.500000,0.500000}%
\pgfsetstrokecolor{currentstroke}%
\pgfsetstrokeopacity{0.300000}%
\pgfsetdash{}{0pt}%
\pgfpathmoveto{\pgfqpoint{1.670838in}{2.631344in}}%
\pgfusepath{stroke}%
\end{pgfscope}%
\begin{pgfscope}%
\pgfpathrectangle{\pgfqpoint{0.647939in}{0.492442in}}{\pgfqpoint{3.079299in}{3.079299in}}%
\pgfusepath{clip}%
\pgfsetroundcap%
\pgfsetroundjoin%
\definecolor{currentfill}{rgb}{0.500000,0.500000,0.500000}%
\pgfsetfillcolor{currentfill}%
\pgfsetfillopacity{0.300000}%
\pgfsetlinewidth{0.301125pt}%
\definecolor{currentstroke}{rgb}{0.500000,0.500000,0.500000}%
\pgfsetstrokecolor{currentstroke}%
\pgfsetstrokeopacity{0.300000}%
\pgfsetdash{}{0pt}%
\pgfpathmoveto{\pgfqpoint{0.000000in}{0.000000in}}%
\pgfpathlineto{\pgfqpoint{0.000000in}{0.000000in}}%
\pgfpathclose%
\pgfusepath{stroke,fill}%
\end{pgfscope}%
\begin{pgfscope}%
\pgfpathrectangle{\pgfqpoint{0.647939in}{0.492442in}}{\pgfqpoint{3.079299in}{3.079299in}}%
\pgfusepath{clip}%
\pgfsetroundcap%
\pgfsetroundjoin%
\pgfsetlinewidth{0.301125pt}%
\definecolor{currentstroke}{rgb}{0.500000,0.500000,0.500000}%
\pgfsetstrokecolor{currentstroke}%
\pgfsetstrokeopacity{0.300000}%
\pgfsetdash{}{0pt}%
\pgfpathmoveto{\pgfqpoint{1.143569in}{2.414957in}}%
\pgfusepath{stroke}%
\end{pgfscope}%
\begin{pgfscope}%
\pgfpathrectangle{\pgfqpoint{0.647939in}{0.492442in}}{\pgfqpoint{3.079299in}{3.079299in}}%
\pgfusepath{clip}%
\pgfsetroundcap%
\pgfsetroundjoin%
\definecolor{currentfill}{rgb}{0.500000,0.500000,0.500000}%
\pgfsetfillcolor{currentfill}%
\pgfsetfillopacity{0.300000}%
\pgfsetlinewidth{0.301125pt}%
\definecolor{currentstroke}{rgb}{0.500000,0.500000,0.500000}%
\pgfsetstrokecolor{currentstroke}%
\pgfsetstrokeopacity{0.300000}%
\pgfsetdash{}{0pt}%
\pgfpathmoveto{\pgfqpoint{0.000000in}{0.000000in}}%
\pgfpathlineto{\pgfqpoint{0.000000in}{0.000000in}}%
\pgfpathclose%
\pgfusepath{stroke,fill}%
\end{pgfscope}%
\begin{pgfscope}%
\pgfpathrectangle{\pgfqpoint{0.647939in}{0.492442in}}{\pgfqpoint{3.079299in}{3.079299in}}%
\pgfusepath{clip}%
\pgfsetroundcap%
\pgfsetroundjoin%
\pgfsetlinewidth{0.301125pt}%
\definecolor{currentstroke}{rgb}{0.500000,0.500000,0.500000}%
\pgfsetstrokecolor{currentstroke}%
\pgfsetstrokeopacity{0.300000}%
\pgfsetdash{}{0pt}%
\pgfpathmoveto{\pgfqpoint{1.011112in}{2.310884in}}%
\pgfusepath{stroke}%
\end{pgfscope}%
\begin{pgfscope}%
\pgfpathrectangle{\pgfqpoint{0.647939in}{0.492442in}}{\pgfqpoint{3.079299in}{3.079299in}}%
\pgfusepath{clip}%
\pgfsetroundcap%
\pgfsetroundjoin%
\definecolor{currentfill}{rgb}{0.500000,0.500000,0.500000}%
\pgfsetfillcolor{currentfill}%
\pgfsetfillopacity{0.300000}%
\pgfsetlinewidth{0.301125pt}%
\definecolor{currentstroke}{rgb}{0.500000,0.500000,0.500000}%
\pgfsetstrokecolor{currentstroke}%
\pgfsetstrokeopacity{0.300000}%
\pgfsetdash{}{0pt}%
\pgfpathmoveto{\pgfqpoint{0.000000in}{0.000000in}}%
\pgfpathlineto{\pgfqpoint{0.000000in}{0.000000in}}%
\pgfpathclose%
\pgfusepath{stroke,fill}%
\end{pgfscope}%
\begin{pgfscope}%
\pgfpathrectangle{\pgfqpoint{0.647939in}{0.492442in}}{\pgfqpoint{3.079299in}{3.079299in}}%
\pgfusepath{clip}%
\pgfsetroundcap%
\pgfsetroundjoin%
\pgfsetlinewidth{0.301125pt}%
\definecolor{currentstroke}{rgb}{0.500000,0.500000,0.500000}%
\pgfsetstrokecolor{currentstroke}%
\pgfsetstrokeopacity{0.300000}%
\pgfsetdash{}{0pt}%
\pgfpathmoveto{\pgfqpoint{1.598814in}{2.424976in}}%
\pgfusepath{stroke}%
\end{pgfscope}%
\begin{pgfscope}%
\pgfpathrectangle{\pgfqpoint{0.647939in}{0.492442in}}{\pgfqpoint{3.079299in}{3.079299in}}%
\pgfusepath{clip}%
\pgfsetroundcap%
\pgfsetroundjoin%
\definecolor{currentfill}{rgb}{0.500000,0.500000,0.500000}%
\pgfsetfillcolor{currentfill}%
\pgfsetfillopacity{0.300000}%
\pgfsetlinewidth{0.301125pt}%
\definecolor{currentstroke}{rgb}{0.500000,0.500000,0.500000}%
\pgfsetstrokecolor{currentstroke}%
\pgfsetstrokeopacity{0.300000}%
\pgfsetdash{}{0pt}%
\pgfpathmoveto{\pgfqpoint{0.000000in}{0.000000in}}%
\pgfpathlineto{\pgfqpoint{0.000000in}{0.000000in}}%
\pgfpathclose%
\pgfusepath{stroke,fill}%
\end{pgfscope}%
\begin{pgfscope}%
\pgfpathrectangle{\pgfqpoint{0.647939in}{0.492442in}}{\pgfqpoint{3.079299in}{3.079299in}}%
\pgfusepath{clip}%
\pgfsetroundcap%
\pgfsetroundjoin%
\pgfsetlinewidth{0.301125pt}%
\definecolor{currentstroke}{rgb}{0.500000,0.500000,0.500000}%
\pgfsetstrokecolor{currentstroke}%
\pgfsetstrokeopacity{0.300000}%
\pgfsetdash{}{0pt}%
\pgfpathmoveto{\pgfqpoint{1.207358in}{2.231803in}}%
\pgfusepath{stroke}%
\end{pgfscope}%
\begin{pgfscope}%
\pgfpathrectangle{\pgfqpoint{0.647939in}{0.492442in}}{\pgfqpoint{3.079299in}{3.079299in}}%
\pgfusepath{clip}%
\pgfsetroundcap%
\pgfsetroundjoin%
\definecolor{currentfill}{rgb}{0.500000,0.500000,0.500000}%
\pgfsetfillcolor{currentfill}%
\pgfsetfillopacity{0.300000}%
\pgfsetlinewidth{0.301125pt}%
\definecolor{currentstroke}{rgb}{0.500000,0.500000,0.500000}%
\pgfsetstrokecolor{currentstroke}%
\pgfsetstrokeopacity{0.300000}%
\pgfsetdash{}{0pt}%
\pgfpathmoveto{\pgfqpoint{0.000000in}{0.000000in}}%
\pgfpathlineto{\pgfqpoint{0.000000in}{0.000000in}}%
\pgfpathclose%
\pgfusepath{stroke,fill}%
\end{pgfscope}%
\begin{pgfscope}%
\pgfpathrectangle{\pgfqpoint{0.647939in}{0.492442in}}{\pgfqpoint{3.079299in}{3.079299in}}%
\pgfusepath{clip}%
\pgfsetroundcap%
\pgfsetroundjoin%
\pgfsetlinewidth{0.301125pt}%
\definecolor{currentstroke}{rgb}{0.500000,0.500000,0.500000}%
\pgfsetstrokecolor{currentstroke}%
\pgfsetstrokeopacity{0.300000}%
\pgfsetdash{}{0pt}%
\pgfpathmoveto{\pgfqpoint{1.206634in}{2.164586in}}%
\pgfusepath{stroke}%
\end{pgfscope}%
\begin{pgfscope}%
\pgfpathrectangle{\pgfqpoint{0.647939in}{0.492442in}}{\pgfqpoint{3.079299in}{3.079299in}}%
\pgfusepath{clip}%
\pgfsetroundcap%
\pgfsetroundjoin%
\definecolor{currentfill}{rgb}{0.500000,0.500000,0.500000}%
\pgfsetfillcolor{currentfill}%
\pgfsetfillopacity{0.300000}%
\pgfsetlinewidth{0.301125pt}%
\definecolor{currentstroke}{rgb}{0.500000,0.500000,0.500000}%
\pgfsetstrokecolor{currentstroke}%
\pgfsetstrokeopacity{0.300000}%
\pgfsetdash{}{0pt}%
\pgfpathmoveto{\pgfqpoint{0.000000in}{0.000000in}}%
\pgfpathlineto{\pgfqpoint{0.000000in}{0.000000in}}%
\pgfpathclose%
\pgfusepath{stroke,fill}%
\end{pgfscope}%
\begin{pgfscope}%
\pgfpathrectangle{\pgfqpoint{0.647939in}{0.492442in}}{\pgfqpoint{3.079299in}{3.079299in}}%
\pgfusepath{clip}%
\pgfsetroundcap%
\pgfsetroundjoin%
\pgfsetlinewidth{0.301125pt}%
\definecolor{currentstroke}{rgb}{0.500000,0.500000,0.500000}%
\pgfsetstrokecolor{currentstroke}%
\pgfsetstrokeopacity{0.300000}%
\pgfsetdash{}{0pt}%
\pgfpathmoveto{\pgfqpoint{1.462869in}{2.191607in}}%
\pgfusepath{stroke}%
\end{pgfscope}%
\begin{pgfscope}%
\pgfpathrectangle{\pgfqpoint{0.647939in}{0.492442in}}{\pgfqpoint{3.079299in}{3.079299in}}%
\pgfusepath{clip}%
\pgfsetroundcap%
\pgfsetroundjoin%
\definecolor{currentfill}{rgb}{0.500000,0.500000,0.500000}%
\pgfsetfillcolor{currentfill}%
\pgfsetfillopacity{0.300000}%
\pgfsetlinewidth{0.301125pt}%
\definecolor{currentstroke}{rgb}{0.500000,0.500000,0.500000}%
\pgfsetstrokecolor{currentstroke}%
\pgfsetstrokeopacity{0.300000}%
\pgfsetdash{}{0pt}%
\pgfpathmoveto{\pgfqpoint{0.000000in}{0.000000in}}%
\pgfpathlineto{\pgfqpoint{0.000000in}{0.000000in}}%
\pgfpathclose%
\pgfusepath{stroke,fill}%
\end{pgfscope}%
\begin{pgfscope}%
\pgfpathrectangle{\pgfqpoint{0.647939in}{0.492442in}}{\pgfqpoint{3.079299in}{3.079299in}}%
\pgfusepath{clip}%
\pgfsetroundcap%
\pgfsetroundjoin%
\pgfsetlinewidth{0.301125pt}%
\definecolor{currentstroke}{rgb}{0.500000,0.500000,0.500000}%
\pgfsetstrokecolor{currentstroke}%
\pgfsetstrokeopacity{0.300000}%
\pgfsetdash{}{0pt}%
\pgfpathmoveto{\pgfqpoint{1.395211in}{2.038672in}}%
\pgfusepath{stroke}%
\end{pgfscope}%
\begin{pgfscope}%
\pgfpathrectangle{\pgfqpoint{0.647939in}{0.492442in}}{\pgfqpoint{3.079299in}{3.079299in}}%
\pgfusepath{clip}%
\pgfsetroundcap%
\pgfsetroundjoin%
\definecolor{currentfill}{rgb}{0.500000,0.500000,0.500000}%
\pgfsetfillcolor{currentfill}%
\pgfsetfillopacity{0.300000}%
\pgfsetlinewidth{0.301125pt}%
\definecolor{currentstroke}{rgb}{0.500000,0.500000,0.500000}%
\pgfsetstrokecolor{currentstroke}%
\pgfsetstrokeopacity{0.300000}%
\pgfsetdash{}{0pt}%
\pgfpathmoveto{\pgfqpoint{0.000000in}{0.000000in}}%
\pgfpathlineto{\pgfqpoint{0.000000in}{0.000000in}}%
\pgfpathclose%
\pgfusepath{stroke,fill}%
\end{pgfscope}%
\begin{pgfscope}%
\pgfpathrectangle{\pgfqpoint{0.647939in}{0.492442in}}{\pgfqpoint{3.079299in}{3.079299in}}%
\pgfusepath{clip}%
\pgfsetroundcap%
\pgfsetroundjoin%
\pgfsetlinewidth{0.301125pt}%
\definecolor{currentstroke}{rgb}{0.500000,0.500000,0.500000}%
\pgfsetstrokecolor{currentstroke}%
\pgfsetstrokeopacity{0.300000}%
\pgfsetdash{}{0pt}%
\pgfpathmoveto{\pgfqpoint{1.139014in}{1.873211in}}%
\pgfusepath{stroke}%
\end{pgfscope}%
\begin{pgfscope}%
\pgfpathrectangle{\pgfqpoint{0.647939in}{0.492442in}}{\pgfqpoint{3.079299in}{3.079299in}}%
\pgfusepath{clip}%
\pgfsetroundcap%
\pgfsetroundjoin%
\definecolor{currentfill}{rgb}{0.500000,0.500000,0.500000}%
\pgfsetfillcolor{currentfill}%
\pgfsetfillopacity{0.300000}%
\pgfsetlinewidth{0.301125pt}%
\definecolor{currentstroke}{rgb}{0.500000,0.500000,0.500000}%
\pgfsetstrokecolor{currentstroke}%
\pgfsetstrokeopacity{0.300000}%
\pgfsetdash{}{0pt}%
\pgfpathmoveto{\pgfqpoint{0.000000in}{0.000000in}}%
\pgfpathlineto{\pgfqpoint{0.000000in}{0.000000in}}%
\pgfpathclose%
\pgfusepath{stroke,fill}%
\end{pgfscope}%
\begin{pgfscope}%
\pgfpathrectangle{\pgfqpoint{0.647939in}{0.492442in}}{\pgfqpoint{3.079299in}{3.079299in}}%
\pgfusepath{clip}%
\pgfsetroundcap%
\pgfsetroundjoin%
\pgfsetlinewidth{0.301125pt}%
\definecolor{currentstroke}{rgb}{0.500000,0.500000,0.500000}%
\pgfsetstrokecolor{currentstroke}%
\pgfsetstrokeopacity{0.300000}%
\pgfsetdash{}{0pt}%
\pgfpathmoveto{\pgfqpoint{1.008656in}{1.762085in}}%
\pgfusepath{stroke}%
\end{pgfscope}%
\begin{pgfscope}%
\pgfpathrectangle{\pgfqpoint{0.647939in}{0.492442in}}{\pgfqpoint{3.079299in}{3.079299in}}%
\pgfusepath{clip}%
\pgfsetroundcap%
\pgfsetroundjoin%
\definecolor{currentfill}{rgb}{0.500000,0.500000,0.500000}%
\pgfsetfillcolor{currentfill}%
\pgfsetfillopacity{0.300000}%
\pgfsetlinewidth{0.301125pt}%
\definecolor{currentstroke}{rgb}{0.500000,0.500000,0.500000}%
\pgfsetstrokecolor{currentstroke}%
\pgfsetstrokeopacity{0.300000}%
\pgfsetdash{}{0pt}%
\pgfpathmoveto{\pgfqpoint{0.000000in}{0.000000in}}%
\pgfpathlineto{\pgfqpoint{0.000000in}{0.000000in}}%
\pgfpathclose%
\pgfusepath{stroke,fill}%
\end{pgfscope}%
\begin{pgfscope}%
\pgfpathrectangle{\pgfqpoint{0.647939in}{0.492442in}}{\pgfqpoint{3.079299in}{3.079299in}}%
\pgfusepath{clip}%
\pgfsetroundcap%
\pgfsetroundjoin%
\pgfsetlinewidth{0.301125pt}%
\definecolor{currentstroke}{rgb}{0.500000,0.500000,0.500000}%
\pgfsetstrokecolor{currentstroke}%
\pgfsetstrokeopacity{0.300000}%
\pgfsetdash{}{0pt}%
\pgfpathmoveto{\pgfqpoint{1.388642in}{1.847254in}}%
\pgfusepath{stroke}%
\end{pgfscope}%
\begin{pgfscope}%
\pgfpathrectangle{\pgfqpoint{0.647939in}{0.492442in}}{\pgfqpoint{3.079299in}{3.079299in}}%
\pgfusepath{clip}%
\pgfsetroundcap%
\pgfsetroundjoin%
\definecolor{currentfill}{rgb}{0.500000,0.500000,0.500000}%
\pgfsetfillcolor{currentfill}%
\pgfsetfillopacity{0.300000}%
\pgfsetlinewidth{0.301125pt}%
\definecolor{currentstroke}{rgb}{0.500000,0.500000,0.500000}%
\pgfsetstrokecolor{currentstroke}%
\pgfsetstrokeopacity{0.300000}%
\pgfsetdash{}{0pt}%
\pgfpathmoveto{\pgfqpoint{0.000000in}{0.000000in}}%
\pgfpathlineto{\pgfqpoint{0.000000in}{0.000000in}}%
\pgfpathclose%
\pgfusepath{stroke,fill}%
\end{pgfscope}%
\begin{pgfscope}%
\pgfpathrectangle{\pgfqpoint{0.647939in}{0.492442in}}{\pgfqpoint{3.079299in}{3.079299in}}%
\pgfusepath{clip}%
\pgfsetroundcap%
\pgfsetroundjoin%
\pgfsetlinewidth{0.301125pt}%
\definecolor{currentstroke}{rgb}{0.500000,0.500000,0.500000}%
\pgfsetstrokecolor{currentstroke}%
\pgfsetstrokeopacity{0.300000}%
\pgfsetdash{}{0pt}%
\pgfpathmoveto{\pgfqpoint{1.072566in}{1.647576in}}%
\pgfusepath{stroke}%
\end{pgfscope}%
\begin{pgfscope}%
\pgfpathrectangle{\pgfqpoint{0.647939in}{0.492442in}}{\pgfqpoint{3.079299in}{3.079299in}}%
\pgfusepath{clip}%
\pgfsetroundcap%
\pgfsetroundjoin%
\definecolor{currentfill}{rgb}{0.500000,0.500000,0.500000}%
\pgfsetfillcolor{currentfill}%
\pgfsetfillopacity{0.300000}%
\pgfsetlinewidth{0.301125pt}%
\definecolor{currentstroke}{rgb}{0.500000,0.500000,0.500000}%
\pgfsetstrokecolor{currentstroke}%
\pgfsetstrokeopacity{0.300000}%
\pgfsetdash{}{0pt}%
\pgfpathmoveto{\pgfqpoint{0.000000in}{0.000000in}}%
\pgfpathlineto{\pgfqpoint{0.000000in}{0.000000in}}%
\pgfpathclose%
\pgfusepath{stroke,fill}%
\end{pgfscope}%
\begin{pgfscope}%
\pgfpathrectangle{\pgfqpoint{0.647939in}{0.492442in}}{\pgfqpoint{3.079299in}{3.079299in}}%
\pgfusepath{clip}%
\pgfsetroundcap%
\pgfsetroundjoin%
\pgfsetlinewidth{0.301125pt}%
\definecolor{currentstroke}{rgb}{0.500000,0.500000,0.500000}%
\pgfsetstrokecolor{currentstroke}%
\pgfsetstrokeopacity{0.300000}%
\pgfsetdash{}{0pt}%
\pgfpathmoveto{\pgfqpoint{0.876217in}{1.518396in}}%
\pgfusepath{stroke}%
\end{pgfscope}%
\begin{pgfscope}%
\pgfpathrectangle{\pgfqpoint{0.647939in}{0.492442in}}{\pgfqpoint{3.079299in}{3.079299in}}%
\pgfusepath{clip}%
\pgfsetroundcap%
\pgfsetroundjoin%
\definecolor{currentfill}{rgb}{0.500000,0.500000,0.500000}%
\pgfsetfillcolor{currentfill}%
\pgfsetfillopacity{0.300000}%
\pgfsetlinewidth{0.301125pt}%
\definecolor{currentstroke}{rgb}{0.500000,0.500000,0.500000}%
\pgfsetstrokecolor{currentstroke}%
\pgfsetstrokeopacity{0.300000}%
\pgfsetdash{}{0pt}%
\pgfpathmoveto{\pgfqpoint{0.000000in}{0.000000in}}%
\pgfpathlineto{\pgfqpoint{0.000000in}{0.000000in}}%
\pgfpathclose%
\pgfusepath{stroke,fill}%
\end{pgfscope}%
\begin{pgfscope}%
\pgfpathrectangle{\pgfqpoint{0.647939in}{0.492442in}}{\pgfqpoint{3.079299in}{3.079299in}}%
\pgfusepath{clip}%
\pgfsetroundcap%
\pgfsetroundjoin%
\pgfsetlinewidth{0.301125pt}%
\definecolor{currentstroke}{rgb}{0.500000,0.500000,0.500000}%
\pgfsetstrokecolor{currentstroke}%
\pgfsetstrokeopacity{0.300000}%
\pgfsetdash{}{0pt}%
\pgfpathmoveto{\pgfqpoint{0.876024in}{1.449303in}}%
\pgfusepath{stroke}%
\end{pgfscope}%
\begin{pgfscope}%
\pgfpathrectangle{\pgfqpoint{0.647939in}{0.492442in}}{\pgfqpoint{3.079299in}{3.079299in}}%
\pgfusepath{clip}%
\pgfsetroundcap%
\pgfsetroundjoin%
\definecolor{currentfill}{rgb}{0.500000,0.500000,0.500000}%
\pgfsetfillcolor{currentfill}%
\pgfsetfillopacity{0.300000}%
\pgfsetlinewidth{0.301125pt}%
\definecolor{currentstroke}{rgb}{0.500000,0.500000,0.500000}%
\pgfsetstrokecolor{currentstroke}%
\pgfsetstrokeopacity{0.300000}%
\pgfsetdash{}{0pt}%
\pgfpathmoveto{\pgfqpoint{0.000000in}{0.000000in}}%
\pgfpathlineto{\pgfqpoint{0.000000in}{0.000000in}}%
\pgfpathclose%
\pgfusepath{stroke,fill}%
\end{pgfscope}%
\begin{pgfscope}%
\pgfpathrectangle{\pgfqpoint{0.647939in}{0.492442in}}{\pgfqpoint{3.079299in}{3.079299in}}%
\pgfusepath{clip}%
\pgfsetroundcap%
\pgfsetroundjoin%
\pgfsetlinewidth{0.301125pt}%
\definecolor{currentstroke}{rgb}{0.500000,0.500000,0.500000}%
\pgfsetstrokecolor{currentstroke}%
\pgfsetstrokeopacity{0.300000}%
\pgfsetdash{}{0pt}%
\pgfpathmoveto{\pgfqpoint{1.376776in}{1.596464in}}%
\pgfusepath{stroke}%
\end{pgfscope}%
\begin{pgfscope}%
\pgfpathrectangle{\pgfqpoint{0.647939in}{0.492442in}}{\pgfqpoint{3.079299in}{3.079299in}}%
\pgfusepath{clip}%
\pgfsetroundcap%
\pgfsetroundjoin%
\definecolor{currentfill}{rgb}{0.500000,0.500000,0.500000}%
\pgfsetfillcolor{currentfill}%
\pgfsetfillopacity{0.300000}%
\pgfsetlinewidth{0.301125pt}%
\definecolor{currentstroke}{rgb}{0.500000,0.500000,0.500000}%
\pgfsetstrokecolor{currentstroke}%
\pgfsetstrokeopacity{0.300000}%
\pgfsetdash{}{0pt}%
\pgfpathmoveto{\pgfqpoint{0.000000in}{0.000000in}}%
\pgfpathlineto{\pgfqpoint{0.000000in}{0.000000in}}%
\pgfpathclose%
\pgfusepath{stroke,fill}%
\end{pgfscope}%
\begin{pgfscope}%
\pgfpathrectangle{\pgfqpoint{0.647939in}{0.492442in}}{\pgfqpoint{3.079299in}{3.079299in}}%
\pgfusepath{clip}%
\pgfsetroundcap%
\pgfsetroundjoin%
\pgfsetlinewidth{0.301125pt}%
\definecolor{currentstroke}{rgb}{0.500000,0.500000,0.500000}%
\pgfsetstrokecolor{currentstroke}%
\pgfsetstrokeopacity{0.300000}%
\pgfsetdash{}{0pt}%
\pgfpathmoveto{\pgfqpoint{1.132411in}{1.404838in}}%
\pgfusepath{stroke}%
\end{pgfscope}%
\begin{pgfscope}%
\pgfpathrectangle{\pgfqpoint{0.647939in}{0.492442in}}{\pgfqpoint{3.079299in}{3.079299in}}%
\pgfusepath{clip}%
\pgfsetroundcap%
\pgfsetroundjoin%
\definecolor{currentfill}{rgb}{0.500000,0.500000,0.500000}%
\pgfsetfillcolor{currentfill}%
\pgfsetfillopacity{0.300000}%
\pgfsetlinewidth{0.301125pt}%
\definecolor{currentstroke}{rgb}{0.500000,0.500000,0.500000}%
\pgfsetstrokecolor{currentstroke}%
\pgfsetstrokeopacity{0.300000}%
\pgfsetdash{}{0pt}%
\pgfpathmoveto{\pgfqpoint{0.000000in}{0.000000in}}%
\pgfpathlineto{\pgfqpoint{0.000000in}{0.000000in}}%
\pgfpathclose%
\pgfusepath{stroke,fill}%
\end{pgfscope}%
\begin{pgfscope}%
\pgfpathrectangle{\pgfqpoint{0.647939in}{0.492442in}}{\pgfqpoint{3.079299in}{3.079299in}}%
\pgfusepath{clip}%
\pgfsetroundcap%
\pgfsetroundjoin%
\pgfsetlinewidth{0.301125pt}%
\definecolor{currentstroke}{rgb}{0.500000,0.500000,0.500000}%
\pgfsetstrokecolor{currentstroke}%
\pgfsetstrokeopacity{0.300000}%
\pgfsetdash{}{0pt}%
\pgfpathmoveto{\pgfqpoint{0.809249in}{1.224738in}}%
\pgfusepath{stroke}%
\end{pgfscope}%
\begin{pgfscope}%
\pgfpathrectangle{\pgfqpoint{0.647939in}{0.492442in}}{\pgfqpoint{3.079299in}{3.079299in}}%
\pgfusepath{clip}%
\pgfsetroundcap%
\pgfsetroundjoin%
\definecolor{currentfill}{rgb}{0.500000,0.500000,0.500000}%
\pgfsetfillcolor{currentfill}%
\pgfsetfillopacity{0.300000}%
\pgfsetlinewidth{0.301125pt}%
\definecolor{currentstroke}{rgb}{0.500000,0.500000,0.500000}%
\pgfsetstrokecolor{currentstroke}%
\pgfsetstrokeopacity{0.300000}%
\pgfsetdash{}{0pt}%
\pgfpathmoveto{\pgfqpoint{0.000000in}{0.000000in}}%
\pgfpathlineto{\pgfqpoint{0.000000in}{0.000000in}}%
\pgfpathclose%
\pgfusepath{stroke,fill}%
\end{pgfscope}%
\begin{pgfscope}%
\pgfpathrectangle{\pgfqpoint{0.647939in}{0.492442in}}{\pgfqpoint{3.079299in}{3.079299in}}%
\pgfusepath{clip}%
\pgfsetroundcap%
\pgfsetroundjoin%
\pgfsetlinewidth{0.301125pt}%
\definecolor{currentstroke}{rgb}{0.500000,0.500000,0.500000}%
\pgfsetstrokecolor{currentstroke}%
\pgfsetstrokeopacity{0.300000}%
\pgfsetdash{}{0pt}%
\pgfpathmoveto{\pgfqpoint{1.307903in}{1.373996in}}%
\pgfusepath{stroke}%
\end{pgfscope}%
\begin{pgfscope}%
\pgfpathrectangle{\pgfqpoint{0.647939in}{0.492442in}}{\pgfqpoint{3.079299in}{3.079299in}}%
\pgfusepath{clip}%
\pgfsetroundcap%
\pgfsetroundjoin%
\definecolor{currentfill}{rgb}{0.500000,0.500000,0.500000}%
\pgfsetfillcolor{currentfill}%
\pgfsetfillopacity{0.300000}%
\pgfsetlinewidth{0.301125pt}%
\definecolor{currentstroke}{rgb}{0.500000,0.500000,0.500000}%
\pgfsetstrokecolor{currentstroke}%
\pgfsetstrokeopacity{0.300000}%
\pgfsetdash{}{0pt}%
\pgfpathmoveto{\pgfqpoint{0.000000in}{0.000000in}}%
\pgfpathlineto{\pgfqpoint{0.000000in}{0.000000in}}%
\pgfpathclose%
\pgfusepath{stroke,fill}%
\end{pgfscope}%
\begin{pgfscope}%
\pgfpathrectangle{\pgfqpoint{0.647939in}{0.492442in}}{\pgfqpoint{3.079299in}{3.079299in}}%
\pgfusepath{clip}%
\pgfsetroundcap%
\pgfsetroundjoin%
\pgfsetlinewidth{0.301125pt}%
\definecolor{currentstroke}{rgb}{0.500000,0.500000,0.500000}%
\pgfsetstrokecolor{currentstroke}%
\pgfsetstrokeopacity{0.300000}%
\pgfsetdash{}{0pt}%
\pgfpathmoveto{\pgfqpoint{1.003941in}{1.149549in}}%
\pgfusepath{stroke}%
\end{pgfscope}%
\begin{pgfscope}%
\pgfpathrectangle{\pgfqpoint{0.647939in}{0.492442in}}{\pgfqpoint{3.079299in}{3.079299in}}%
\pgfusepath{clip}%
\pgfsetroundcap%
\pgfsetroundjoin%
\definecolor{currentfill}{rgb}{0.500000,0.500000,0.500000}%
\pgfsetfillcolor{currentfill}%
\pgfsetfillopacity{0.300000}%
\pgfsetlinewidth{0.301125pt}%
\definecolor{currentstroke}{rgb}{0.500000,0.500000,0.500000}%
\pgfsetstrokecolor{currentstroke}%
\pgfsetstrokeopacity{0.300000}%
\pgfsetdash{}{0pt}%
\pgfpathmoveto{\pgfqpoint{0.000000in}{0.000000in}}%
\pgfpathlineto{\pgfqpoint{0.000000in}{0.000000in}}%
\pgfpathclose%
\pgfusepath{stroke,fill}%
\end{pgfscope}%
\begin{pgfscope}%
\pgfpathrectangle{\pgfqpoint{0.647939in}{0.492442in}}{\pgfqpoint{3.079299in}{3.079299in}}%
\pgfusepath{clip}%
\pgfsetroundcap%
\pgfsetroundjoin%
\pgfsetlinewidth{0.301125pt}%
\definecolor{currentstroke}{rgb}{0.500000,0.500000,0.500000}%
\pgfsetstrokecolor{currentstroke}%
\pgfsetstrokeopacity{0.300000}%
\pgfsetdash{}{0pt}%
\pgfpathmoveto{\pgfqpoint{0.874589in}{1.035531in}}%
\pgfusepath{stroke}%
\end{pgfscope}%
\begin{pgfscope}%
\pgfpathrectangle{\pgfqpoint{0.647939in}{0.492442in}}{\pgfqpoint{3.079299in}{3.079299in}}%
\pgfusepath{clip}%
\pgfsetroundcap%
\pgfsetroundjoin%
\definecolor{currentfill}{rgb}{0.500000,0.500000,0.500000}%
\pgfsetfillcolor{currentfill}%
\pgfsetfillopacity{0.300000}%
\pgfsetlinewidth{0.301125pt}%
\definecolor{currentstroke}{rgb}{0.500000,0.500000,0.500000}%
\pgfsetstrokecolor{currentstroke}%
\pgfsetstrokeopacity{0.300000}%
\pgfsetdash{}{0pt}%
\pgfpathmoveto{\pgfqpoint{0.000000in}{0.000000in}}%
\pgfpathlineto{\pgfqpoint{0.000000in}{0.000000in}}%
\pgfpathclose%
\pgfusepath{stroke,fill}%
\end{pgfscope}%
\begin{pgfscope}%
\pgfpathrectangle{\pgfqpoint{0.647939in}{0.492442in}}{\pgfqpoint{3.079299in}{3.079299in}}%
\pgfusepath{clip}%
\pgfsetroundcap%
\pgfsetroundjoin%
\pgfsetlinewidth{0.301125pt}%
\definecolor{currentstroke}{rgb}{0.500000,0.500000,0.500000}%
\pgfsetstrokecolor{currentstroke}%
\pgfsetstrokeopacity{0.300000}%
\pgfsetdash{}{0pt}%
\pgfpathmoveto{\pgfqpoint{1.183784in}{1.109844in}}%
\pgfusepath{stroke}%
\end{pgfscope}%
\begin{pgfscope}%
\pgfpathrectangle{\pgfqpoint{0.647939in}{0.492442in}}{\pgfqpoint{3.079299in}{3.079299in}}%
\pgfusepath{clip}%
\pgfsetroundcap%
\pgfsetroundjoin%
\definecolor{currentfill}{rgb}{0.500000,0.500000,0.500000}%
\pgfsetfillcolor{currentfill}%
\pgfsetfillopacity{0.300000}%
\pgfsetlinewidth{0.301125pt}%
\definecolor{currentstroke}{rgb}{0.500000,0.500000,0.500000}%
\pgfsetstrokecolor{currentstroke}%
\pgfsetstrokeopacity{0.300000}%
\pgfsetdash{}{0pt}%
\pgfpathmoveto{\pgfqpoint{0.000000in}{0.000000in}}%
\pgfpathlineto{\pgfqpoint{0.000000in}{0.000000in}}%
\pgfpathclose%
\pgfusepath{stroke,fill}%
\end{pgfscope}%
\begin{pgfscope}%
\pgfpathrectangle{\pgfqpoint{0.647939in}{0.492442in}}{\pgfqpoint{3.079299in}{3.079299in}}%
\pgfusepath{clip}%
\pgfsetroundcap%
\pgfsetroundjoin%
\pgfsetlinewidth{0.301125pt}%
\definecolor{currentstroke}{rgb}{0.500000,0.500000,0.500000}%
\pgfsetstrokecolor{currentstroke}%
\pgfsetstrokeopacity{0.300000}%
\pgfsetdash{}{0pt}%
\pgfpathmoveto{\pgfqpoint{1.001577in}{0.947010in}}%
\pgfusepath{stroke}%
\end{pgfscope}%
\begin{pgfscope}%
\pgfpathrectangle{\pgfqpoint{0.647939in}{0.492442in}}{\pgfqpoint{3.079299in}{3.079299in}}%
\pgfusepath{clip}%
\pgfsetroundcap%
\pgfsetroundjoin%
\definecolor{currentfill}{rgb}{0.500000,0.500000,0.500000}%
\pgfsetfillcolor{currentfill}%
\pgfsetfillopacity{0.300000}%
\pgfsetlinewidth{0.301125pt}%
\definecolor{currentstroke}{rgb}{0.500000,0.500000,0.500000}%
\pgfsetstrokecolor{currentstroke}%
\pgfsetstrokeopacity{0.300000}%
\pgfsetdash{}{0pt}%
\pgfpathmoveto{\pgfqpoint{0.000000in}{0.000000in}}%
\pgfpathlineto{\pgfqpoint{0.000000in}{0.000000in}}%
\pgfpathclose%
\pgfusepath{stroke,fill}%
\end{pgfscope}%
\begin{pgfscope}%
\pgfpathrectangle{\pgfqpoint{0.647939in}{0.492442in}}{\pgfqpoint{3.079299in}{3.079299in}}%
\pgfusepath{clip}%
\pgfsetroundcap%
\pgfsetroundjoin%
\pgfsetlinewidth{0.301125pt}%
\definecolor{currentstroke}{rgb}{0.500000,0.500000,0.500000}%
\pgfsetstrokecolor{currentstroke}%
\pgfsetstrokeopacity{0.300000}%
\pgfsetdash{}{0pt}%
\pgfpathmoveto{\pgfqpoint{0.937841in}{0.852756in}}%
\pgfusepath{stroke}%
\end{pgfscope}%
\begin{pgfscope}%
\pgfpathrectangle{\pgfqpoint{0.647939in}{0.492442in}}{\pgfqpoint{3.079299in}{3.079299in}}%
\pgfusepath{clip}%
\pgfsetroundcap%
\pgfsetroundjoin%
\definecolor{currentfill}{rgb}{0.500000,0.500000,0.500000}%
\pgfsetfillcolor{currentfill}%
\pgfsetfillopacity{0.300000}%
\pgfsetlinewidth{0.301125pt}%
\definecolor{currentstroke}{rgb}{0.500000,0.500000,0.500000}%
\pgfsetstrokecolor{currentstroke}%
\pgfsetstrokeopacity{0.300000}%
\pgfsetdash{}{0pt}%
\pgfpathmoveto{\pgfqpoint{0.000000in}{0.000000in}}%
\pgfpathlineto{\pgfqpoint{0.000000in}{0.000000in}}%
\pgfpathclose%
\pgfusepath{stroke,fill}%
\end{pgfscope}%
\begin{pgfscope}%
\pgfpathrectangle{\pgfqpoint{0.647939in}{0.492442in}}{\pgfqpoint{3.079299in}{3.079299in}}%
\pgfusepath{clip}%
\pgfsetroundcap%
\pgfsetroundjoin%
\pgfsetlinewidth{0.301125pt}%
\definecolor{currentstroke}{rgb}{0.500000,0.500000,0.500000}%
\pgfsetstrokecolor{currentstroke}%
\pgfsetstrokeopacity{0.300000}%
\pgfsetdash{}{0pt}%
\pgfpathmoveto{\pgfqpoint{0.937221in}{0.784772in}}%
\pgfusepath{stroke}%
\end{pgfscope}%
\begin{pgfscope}%
\pgfpathrectangle{\pgfqpoint{0.647939in}{0.492442in}}{\pgfqpoint{3.079299in}{3.079299in}}%
\pgfusepath{clip}%
\pgfsetroundcap%
\pgfsetroundjoin%
\definecolor{currentfill}{rgb}{0.500000,0.500000,0.500000}%
\pgfsetfillcolor{currentfill}%
\pgfsetfillopacity{0.300000}%
\pgfsetlinewidth{0.301125pt}%
\definecolor{currentstroke}{rgb}{0.500000,0.500000,0.500000}%
\pgfsetstrokecolor{currentstroke}%
\pgfsetstrokeopacity{0.300000}%
\pgfsetdash{}{0pt}%
\pgfpathmoveto{\pgfqpoint{0.000000in}{0.000000in}}%
\pgfpathlineto{\pgfqpoint{0.000000in}{0.000000in}}%
\pgfpathclose%
\pgfusepath{stroke,fill}%
\end{pgfscope}%
\begin{pgfscope}%
\pgfpathrectangle{\pgfqpoint{0.647939in}{0.492442in}}{\pgfqpoint{3.079299in}{3.079299in}}%
\pgfusepath{clip}%
\pgfsetroundcap%
\pgfsetroundjoin%
\pgfsetlinewidth{0.301125pt}%
\definecolor{currentstroke}{rgb}{0.500000,0.500000,0.500000}%
\pgfsetstrokecolor{currentstroke}%
\pgfsetstrokeopacity{0.300000}%
\pgfsetdash{}{0pt}%
\pgfpathmoveto{\pgfqpoint{0.872876in}{0.692011in}}%
\pgfusepath{stroke}%
\end{pgfscope}%
\begin{pgfscope}%
\pgfpathrectangle{\pgfqpoint{0.647939in}{0.492442in}}{\pgfqpoint{3.079299in}{3.079299in}}%
\pgfusepath{clip}%
\pgfsetroundcap%
\pgfsetroundjoin%
\definecolor{currentfill}{rgb}{0.500000,0.500000,0.500000}%
\pgfsetfillcolor{currentfill}%
\pgfsetfillopacity{0.300000}%
\pgfsetlinewidth{0.301125pt}%
\definecolor{currentstroke}{rgb}{0.500000,0.500000,0.500000}%
\pgfsetstrokecolor{currentstroke}%
\pgfsetstrokeopacity{0.300000}%
\pgfsetdash{}{0pt}%
\pgfpathmoveto{\pgfqpoint{0.000000in}{0.000000in}}%
\pgfpathlineto{\pgfqpoint{0.000000in}{0.000000in}}%
\pgfpathclose%
\pgfusepath{stroke,fill}%
\end{pgfscope}%
\begin{pgfscope}%
\pgfpathrectangle{\pgfqpoint{0.647939in}{0.492442in}}{\pgfqpoint{3.079299in}{3.079299in}}%
\pgfusepath{clip}%
\pgfsetroundcap%
\pgfsetroundjoin%
\pgfsetlinewidth{0.301125pt}%
\definecolor{currentstroke}{rgb}{0.500000,0.500000,0.500000}%
\pgfsetstrokecolor{currentstroke}%
\pgfsetstrokeopacity{0.300000}%
\pgfsetdash{}{0pt}%
\pgfpathmoveto{\pgfqpoint{3.585033in}{1.885587in}}%
\pgfusepath{stroke}%
\end{pgfscope}%
\begin{pgfscope}%
\pgfpathrectangle{\pgfqpoint{0.647939in}{0.492442in}}{\pgfqpoint{3.079299in}{3.079299in}}%
\pgfusepath{clip}%
\pgfsetroundcap%
\pgfsetroundjoin%
\definecolor{currentfill}{rgb}{0.500000,0.500000,0.500000}%
\pgfsetfillcolor{currentfill}%
\pgfsetfillopacity{0.300000}%
\pgfsetlinewidth{0.301125pt}%
\definecolor{currentstroke}{rgb}{0.500000,0.500000,0.500000}%
\pgfsetstrokecolor{currentstroke}%
\pgfsetstrokeopacity{0.300000}%
\pgfsetdash{}{0pt}%
\pgfpathmoveto{\pgfqpoint{0.000000in}{0.000000in}}%
\pgfpathlineto{\pgfqpoint{0.000000in}{0.000000in}}%
\pgfpathclose%
\pgfusepath{stroke,fill}%
\end{pgfscope}%
\begin{pgfscope}%
\pgfpathrectangle{\pgfqpoint{0.647939in}{0.492442in}}{\pgfqpoint{3.079299in}{3.079299in}}%
\pgfusepath{clip}%
\pgfsetroundcap%
\pgfsetroundjoin%
\pgfsetlinewidth{0.301125pt}%
\definecolor{currentstroke}{rgb}{0.500000,0.500000,0.500000}%
\pgfsetstrokecolor{currentstroke}%
\pgfsetstrokeopacity{0.300000}%
\pgfsetdash{}{0pt}%
\pgfpathmoveto{\pgfqpoint{1.524345in}{0.748193in}}%
\pgfusepath{stroke}%
\end{pgfscope}%
\begin{pgfscope}%
\pgfpathrectangle{\pgfqpoint{0.647939in}{0.492442in}}{\pgfqpoint{3.079299in}{3.079299in}}%
\pgfusepath{clip}%
\pgfsetroundcap%
\pgfsetroundjoin%
\definecolor{currentfill}{rgb}{0.500000,0.500000,0.500000}%
\pgfsetfillcolor{currentfill}%
\pgfsetfillopacity{0.300000}%
\pgfsetlinewidth{0.301125pt}%
\definecolor{currentstroke}{rgb}{0.500000,0.500000,0.500000}%
\pgfsetstrokecolor{currentstroke}%
\pgfsetstrokeopacity{0.300000}%
\pgfsetdash{}{0pt}%
\pgfpathmoveto{\pgfqpoint{0.000000in}{0.000000in}}%
\pgfpathlineto{\pgfqpoint{0.000000in}{0.000000in}}%
\pgfpathclose%
\pgfusepath{stroke,fill}%
\end{pgfscope}%
\begin{pgfscope}%
\pgfpathrectangle{\pgfqpoint{0.647939in}{0.492442in}}{\pgfqpoint{3.079299in}{3.079299in}}%
\pgfusepath{clip}%
\pgfsetroundcap%
\pgfsetroundjoin%
\pgfsetlinewidth{0.301125pt}%
\definecolor{currentstroke}{rgb}{0.500000,0.500000,0.500000}%
\pgfsetstrokecolor{currentstroke}%
\pgfsetstrokeopacity{0.300000}%
\pgfsetdash{}{0pt}%
\pgfpathmoveto{\pgfqpoint{3.450329in}{1.563478in}}%
\pgfusepath{stroke}%
\end{pgfscope}%
\begin{pgfscope}%
\pgfpathrectangle{\pgfqpoint{0.647939in}{0.492442in}}{\pgfqpoint{3.079299in}{3.079299in}}%
\pgfusepath{clip}%
\pgfsetroundcap%
\pgfsetroundjoin%
\definecolor{currentfill}{rgb}{0.500000,0.500000,0.500000}%
\pgfsetfillcolor{currentfill}%
\pgfsetfillopacity{0.300000}%
\pgfsetlinewidth{0.301125pt}%
\definecolor{currentstroke}{rgb}{0.500000,0.500000,0.500000}%
\pgfsetstrokecolor{currentstroke}%
\pgfsetstrokeopacity{0.300000}%
\pgfsetdash{}{0pt}%
\pgfpathmoveto{\pgfqpoint{0.000000in}{0.000000in}}%
\pgfpathlineto{\pgfqpoint{0.000000in}{0.000000in}}%
\pgfpathclose%
\pgfusepath{stroke,fill}%
\end{pgfscope}%
\begin{pgfscope}%
\pgfpathrectangle{\pgfqpoint{0.647939in}{0.492442in}}{\pgfqpoint{3.079299in}{3.079299in}}%
\pgfusepath{clip}%
\pgfsetroundcap%
\pgfsetroundjoin%
\pgfsetlinewidth{0.301125pt}%
\definecolor{currentstroke}{rgb}{0.500000,0.500000,0.500000}%
\pgfsetstrokecolor{currentstroke}%
\pgfsetstrokeopacity{0.300000}%
\pgfsetdash{}{0pt}%
\pgfpathmoveto{\pgfqpoint{3.604611in}{3.243489in}}%
\pgfusepath{stroke}%
\end{pgfscope}%
\begin{pgfscope}%
\pgfpathrectangle{\pgfqpoint{0.647939in}{0.492442in}}{\pgfqpoint{3.079299in}{3.079299in}}%
\pgfusepath{clip}%
\pgfsetroundcap%
\pgfsetroundjoin%
\definecolor{currentfill}{rgb}{0.500000,0.500000,0.500000}%
\pgfsetfillcolor{currentfill}%
\pgfsetfillopacity{0.300000}%
\pgfsetlinewidth{0.301125pt}%
\definecolor{currentstroke}{rgb}{0.500000,0.500000,0.500000}%
\pgfsetstrokecolor{currentstroke}%
\pgfsetstrokeopacity{0.300000}%
\pgfsetdash{}{0pt}%
\pgfpathmoveto{\pgfqpoint{0.000000in}{0.000000in}}%
\pgfpathlineto{\pgfqpoint{0.000000in}{0.000000in}}%
\pgfpathclose%
\pgfusepath{stroke,fill}%
\end{pgfscope}%
\begin{pgfscope}%
\pgfpathrectangle{\pgfqpoint{0.647939in}{0.492442in}}{\pgfqpoint{3.079299in}{3.079299in}}%
\pgfusepath{clip}%
\pgfsetroundcap%
\pgfsetroundjoin%
\pgfsetlinewidth{0.301125pt}%
\definecolor{currentstroke}{rgb}{0.500000,0.500000,0.500000}%
\pgfsetstrokecolor{currentstroke}%
\pgfsetstrokeopacity{0.300000}%
\pgfsetdash{}{0pt}%
\pgfpathmoveto{\pgfqpoint{3.148111in}{2.188904in}}%
\pgfusepath{stroke}%
\end{pgfscope}%
\begin{pgfscope}%
\pgfpathrectangle{\pgfqpoint{0.647939in}{0.492442in}}{\pgfqpoint{3.079299in}{3.079299in}}%
\pgfusepath{clip}%
\pgfsetroundcap%
\pgfsetroundjoin%
\definecolor{currentfill}{rgb}{0.500000,0.500000,0.500000}%
\pgfsetfillcolor{currentfill}%
\pgfsetfillopacity{0.300000}%
\pgfsetlinewidth{0.301125pt}%
\definecolor{currentstroke}{rgb}{0.500000,0.500000,0.500000}%
\pgfsetstrokecolor{currentstroke}%
\pgfsetstrokeopacity{0.300000}%
\pgfsetdash{}{0pt}%
\pgfpathmoveto{\pgfqpoint{0.000000in}{0.000000in}}%
\pgfpathlineto{\pgfqpoint{0.000000in}{0.000000in}}%
\pgfpathclose%
\pgfusepath{stroke,fill}%
\end{pgfscope}%
\begin{pgfscope}%
\pgfpathrectangle{\pgfqpoint{0.647939in}{0.492442in}}{\pgfqpoint{3.079299in}{3.079299in}}%
\pgfusepath{clip}%
\pgfsetroundcap%
\pgfsetroundjoin%
\pgfsetlinewidth{0.301125pt}%
\definecolor{currentstroke}{rgb}{0.500000,0.500000,0.500000}%
\pgfsetstrokecolor{currentstroke}%
\pgfsetstrokeopacity{0.300000}%
\pgfsetdash{}{0pt}%
\pgfpathmoveto{\pgfqpoint{2.677185in}{3.221779in}}%
\pgfusepath{stroke}%
\end{pgfscope}%
\begin{pgfscope}%
\pgfpathrectangle{\pgfqpoint{0.647939in}{0.492442in}}{\pgfqpoint{3.079299in}{3.079299in}}%
\pgfusepath{clip}%
\pgfsetroundcap%
\pgfsetroundjoin%
\definecolor{currentfill}{rgb}{0.500000,0.500000,0.500000}%
\pgfsetfillcolor{currentfill}%
\pgfsetfillopacity{0.300000}%
\pgfsetlinewidth{0.301125pt}%
\definecolor{currentstroke}{rgb}{0.500000,0.500000,0.500000}%
\pgfsetstrokecolor{currentstroke}%
\pgfsetstrokeopacity{0.300000}%
\pgfsetdash{}{0pt}%
\pgfpathmoveto{\pgfqpoint{0.000000in}{0.000000in}}%
\pgfpathlineto{\pgfqpoint{0.000000in}{0.000000in}}%
\pgfpathclose%
\pgfusepath{stroke,fill}%
\end{pgfscope}%
\begin{pgfscope}%
\pgfpathrectangle{\pgfqpoint{0.647939in}{0.492442in}}{\pgfqpoint{3.079299in}{3.079299in}}%
\pgfusepath{clip}%
\pgfsetroundcap%
\pgfsetroundjoin%
\pgfsetlinewidth{0.301125pt}%
\definecolor{currentstroke}{rgb}{0.500000,0.500000,0.500000}%
\pgfsetstrokecolor{currentstroke}%
\pgfsetstrokeopacity{0.300000}%
\pgfsetdash{}{0pt}%
\pgfpathmoveto{\pgfqpoint{2.295112in}{0.770372in}}%
\pgfusepath{stroke}%
\end{pgfscope}%
\begin{pgfscope}%
\pgfpathrectangle{\pgfqpoint{0.647939in}{0.492442in}}{\pgfqpoint{3.079299in}{3.079299in}}%
\pgfusepath{clip}%
\pgfsetroundcap%
\pgfsetroundjoin%
\definecolor{currentfill}{rgb}{0.500000,0.500000,0.500000}%
\pgfsetfillcolor{currentfill}%
\pgfsetfillopacity{0.300000}%
\pgfsetlinewidth{0.301125pt}%
\definecolor{currentstroke}{rgb}{0.500000,0.500000,0.500000}%
\pgfsetstrokecolor{currentstroke}%
\pgfsetstrokeopacity{0.300000}%
\pgfsetdash{}{0pt}%
\pgfpathmoveto{\pgfqpoint{0.000000in}{0.000000in}}%
\pgfpathlineto{\pgfqpoint{0.000000in}{0.000000in}}%
\pgfpathclose%
\pgfusepath{stroke,fill}%
\end{pgfscope}%
\begin{pgfscope}%
\pgfpathrectangle{\pgfqpoint{0.647939in}{0.492442in}}{\pgfqpoint{3.079299in}{3.079299in}}%
\pgfusepath{clip}%
\pgfsetroundcap%
\pgfsetroundjoin%
\pgfsetlinewidth{0.301125pt}%
\definecolor{currentstroke}{rgb}{0.500000,0.500000,0.500000}%
\pgfsetstrokecolor{currentstroke}%
\pgfsetstrokeopacity{0.300000}%
\pgfsetdash{}{0pt}%
\pgfpathmoveto{\pgfqpoint{2.920862in}{1.772234in}}%
\pgfusepath{stroke}%
\end{pgfscope}%
\begin{pgfscope}%
\pgfpathrectangle{\pgfqpoint{0.647939in}{0.492442in}}{\pgfqpoint{3.079299in}{3.079299in}}%
\pgfusepath{clip}%
\pgfsetroundcap%
\pgfsetroundjoin%
\definecolor{currentfill}{rgb}{0.500000,0.500000,0.500000}%
\pgfsetfillcolor{currentfill}%
\pgfsetfillopacity{0.300000}%
\pgfsetlinewidth{0.301125pt}%
\definecolor{currentstroke}{rgb}{0.500000,0.500000,0.500000}%
\pgfsetstrokecolor{currentstroke}%
\pgfsetstrokeopacity{0.300000}%
\pgfsetdash{}{0pt}%
\pgfpathmoveto{\pgfqpoint{0.000000in}{0.000000in}}%
\pgfpathlineto{\pgfqpoint{0.000000in}{0.000000in}}%
\pgfpathclose%
\pgfusepath{stroke,fill}%
\end{pgfscope}%
\begin{pgfscope}%
\pgfpathrectangle{\pgfqpoint{0.647939in}{0.492442in}}{\pgfqpoint{3.079299in}{3.079299in}}%
\pgfusepath{clip}%
\pgfsetroundcap%
\pgfsetroundjoin%
\pgfsetlinewidth{0.301125pt}%
\definecolor{currentstroke}{rgb}{0.500000,0.500000,0.500000}%
\pgfsetstrokecolor{currentstroke}%
\pgfsetstrokeopacity{0.300000}%
\pgfsetdash{}{0pt}%
\pgfpathmoveto{\pgfqpoint{3.382410in}{2.672482in}}%
\pgfusepath{stroke}%
\end{pgfscope}%
\begin{pgfscope}%
\pgfpathrectangle{\pgfqpoint{0.647939in}{0.492442in}}{\pgfqpoint{3.079299in}{3.079299in}}%
\pgfusepath{clip}%
\pgfsetroundcap%
\pgfsetroundjoin%
\definecolor{currentfill}{rgb}{0.500000,0.500000,0.500000}%
\pgfsetfillcolor{currentfill}%
\pgfsetfillopacity{0.300000}%
\pgfsetlinewidth{0.301125pt}%
\definecolor{currentstroke}{rgb}{0.500000,0.500000,0.500000}%
\pgfsetstrokecolor{currentstroke}%
\pgfsetstrokeopacity{0.300000}%
\pgfsetdash{}{0pt}%
\pgfpathmoveto{\pgfqpoint{0.000000in}{0.000000in}}%
\pgfpathlineto{\pgfqpoint{0.000000in}{0.000000in}}%
\pgfpathclose%
\pgfusepath{stroke,fill}%
\end{pgfscope}%
\begin{pgfscope}%
\pgfpathrectangle{\pgfqpoint{0.647939in}{0.492442in}}{\pgfqpoint{3.079299in}{3.079299in}}%
\pgfusepath{clip}%
\pgfsetroundcap%
\pgfsetroundjoin%
\pgfsetlinewidth{0.301125pt}%
\definecolor{currentstroke}{rgb}{0.500000,0.500000,0.500000}%
\pgfsetstrokecolor{currentstroke}%
\pgfsetstrokeopacity{0.300000}%
\pgfsetdash{}{0pt}%
\pgfpathmoveto{\pgfqpoint{2.083271in}{3.285290in}}%
\pgfusepath{stroke}%
\end{pgfscope}%
\begin{pgfscope}%
\pgfpathrectangle{\pgfqpoint{0.647939in}{0.492442in}}{\pgfqpoint{3.079299in}{3.079299in}}%
\pgfusepath{clip}%
\pgfsetroundcap%
\pgfsetroundjoin%
\definecolor{currentfill}{rgb}{0.500000,0.500000,0.500000}%
\pgfsetfillcolor{currentfill}%
\pgfsetfillopacity{0.300000}%
\pgfsetlinewidth{0.301125pt}%
\definecolor{currentstroke}{rgb}{0.500000,0.500000,0.500000}%
\pgfsetstrokecolor{currentstroke}%
\pgfsetstrokeopacity{0.300000}%
\pgfsetdash{}{0pt}%
\pgfpathmoveto{\pgfqpoint{0.000000in}{0.000000in}}%
\pgfpathlineto{\pgfqpoint{0.000000in}{0.000000in}}%
\pgfpathclose%
\pgfusepath{stroke,fill}%
\end{pgfscope}%
\begin{pgfscope}%
\pgfpathrectangle{\pgfqpoint{0.647939in}{0.492442in}}{\pgfqpoint{3.079299in}{3.079299in}}%
\pgfusepath{clip}%
\pgfsetroundcap%
\pgfsetroundjoin%
\pgfsetlinewidth{0.301125pt}%
\definecolor{currentstroke}{rgb}{0.500000,0.500000,0.500000}%
\pgfsetstrokecolor{currentstroke}%
\pgfsetstrokeopacity{0.300000}%
\pgfsetdash{}{0pt}%
\pgfpathmoveto{\pgfqpoint{2.244195in}{0.892111in}}%
\pgfusepath{stroke}%
\end{pgfscope}%
\begin{pgfscope}%
\pgfpathrectangle{\pgfqpoint{0.647939in}{0.492442in}}{\pgfqpoint{3.079299in}{3.079299in}}%
\pgfusepath{clip}%
\pgfsetroundcap%
\pgfsetroundjoin%
\definecolor{currentfill}{rgb}{0.500000,0.500000,0.500000}%
\pgfsetfillcolor{currentfill}%
\pgfsetfillopacity{0.300000}%
\pgfsetlinewidth{0.301125pt}%
\definecolor{currentstroke}{rgb}{0.500000,0.500000,0.500000}%
\pgfsetstrokecolor{currentstroke}%
\pgfsetstrokeopacity{0.300000}%
\pgfsetdash{}{0pt}%
\pgfpathmoveto{\pgfqpoint{0.000000in}{0.000000in}}%
\pgfpathlineto{\pgfqpoint{0.000000in}{0.000000in}}%
\pgfpathclose%
\pgfusepath{stroke,fill}%
\end{pgfscope}%
\begin{pgfscope}%
\pgfpathrectangle{\pgfqpoint{0.647939in}{0.492442in}}{\pgfqpoint{3.079299in}{3.079299in}}%
\pgfusepath{clip}%
\pgfsetroundcap%
\pgfsetroundjoin%
\pgfsetlinewidth{0.301125pt}%
\definecolor{currentstroke}{rgb}{0.500000,0.500000,0.500000}%
\pgfsetstrokecolor{currentstroke}%
\pgfsetstrokeopacity{0.300000}%
\pgfsetdash{}{0pt}%
\pgfpathmoveto{\pgfqpoint{3.229085in}{2.438125in}}%
\pgfusepath{stroke}%
\end{pgfscope}%
\begin{pgfscope}%
\pgfpathrectangle{\pgfqpoint{0.647939in}{0.492442in}}{\pgfqpoint{3.079299in}{3.079299in}}%
\pgfusepath{clip}%
\pgfsetroundcap%
\pgfsetroundjoin%
\definecolor{currentfill}{rgb}{0.500000,0.500000,0.500000}%
\pgfsetfillcolor{currentfill}%
\pgfsetfillopacity{0.300000}%
\pgfsetlinewidth{0.301125pt}%
\definecolor{currentstroke}{rgb}{0.500000,0.500000,0.500000}%
\pgfsetstrokecolor{currentstroke}%
\pgfsetstrokeopacity{0.300000}%
\pgfsetdash{}{0pt}%
\pgfpathmoveto{\pgfqpoint{0.000000in}{0.000000in}}%
\pgfpathlineto{\pgfqpoint{0.000000in}{0.000000in}}%
\pgfpathclose%
\pgfusepath{stroke,fill}%
\end{pgfscope}%
\begin{pgfscope}%
\pgfpathrectangle{\pgfqpoint{0.647939in}{0.492442in}}{\pgfqpoint{3.079299in}{3.079299in}}%
\pgfusepath{clip}%
\pgfsetroundcap%
\pgfsetroundjoin%
\pgfsetlinewidth{0.301125pt}%
\definecolor{currentstroke}{rgb}{0.500000,0.500000,0.500000}%
\pgfsetstrokecolor{currentstroke}%
\pgfsetstrokeopacity{0.300000}%
\pgfsetdash{}{0pt}%
\pgfpathmoveto{\pgfqpoint{1.244625in}{3.182627in}}%
\pgfusepath{stroke}%
\end{pgfscope}%
\begin{pgfscope}%
\pgfpathrectangle{\pgfqpoint{0.647939in}{0.492442in}}{\pgfqpoint{3.079299in}{3.079299in}}%
\pgfusepath{clip}%
\pgfsetroundcap%
\pgfsetroundjoin%
\definecolor{currentfill}{rgb}{0.500000,0.500000,0.500000}%
\pgfsetfillcolor{currentfill}%
\pgfsetfillopacity{0.300000}%
\pgfsetlinewidth{0.301125pt}%
\definecolor{currentstroke}{rgb}{0.500000,0.500000,0.500000}%
\pgfsetstrokecolor{currentstroke}%
\pgfsetstrokeopacity{0.300000}%
\pgfsetdash{}{0pt}%
\pgfpathmoveto{\pgfqpoint{0.000000in}{0.000000in}}%
\pgfpathlineto{\pgfqpoint{0.000000in}{0.000000in}}%
\pgfpathclose%
\pgfusepath{stroke,fill}%
\end{pgfscope}%
\begin{pgfscope}%
\pgfpathrectangle{\pgfqpoint{0.647939in}{0.492442in}}{\pgfqpoint{3.079299in}{3.079299in}}%
\pgfusepath{clip}%
\pgfsetroundcap%
\pgfsetroundjoin%
\pgfsetlinewidth{0.301125pt}%
\definecolor{currentstroke}{rgb}{0.500000,0.500000,0.500000}%
\pgfsetstrokecolor{currentstroke}%
\pgfsetstrokeopacity{0.300000}%
\pgfsetdash{}{0pt}%
\pgfpathmoveto{\pgfqpoint{2.814536in}{2.233173in}}%
\pgfusepath{stroke}%
\end{pgfscope}%
\begin{pgfscope}%
\pgfpathrectangle{\pgfqpoint{0.647939in}{0.492442in}}{\pgfqpoint{3.079299in}{3.079299in}}%
\pgfusepath{clip}%
\pgfsetroundcap%
\pgfsetroundjoin%
\definecolor{currentfill}{rgb}{0.500000,0.500000,0.500000}%
\pgfsetfillcolor{currentfill}%
\pgfsetfillopacity{0.300000}%
\pgfsetlinewidth{0.301125pt}%
\definecolor{currentstroke}{rgb}{0.500000,0.500000,0.500000}%
\pgfsetstrokecolor{currentstroke}%
\pgfsetstrokeopacity{0.300000}%
\pgfsetdash{}{0pt}%
\pgfpathmoveto{\pgfqpoint{0.000000in}{0.000000in}}%
\pgfpathlineto{\pgfqpoint{0.000000in}{0.000000in}}%
\pgfpathclose%
\pgfusepath{stroke,fill}%
\end{pgfscope}%
\begin{pgfscope}%
\pgfpathrectangle{\pgfqpoint{0.647939in}{0.492442in}}{\pgfqpoint{3.079299in}{3.079299in}}%
\pgfusepath{clip}%
\pgfsetroundcap%
\pgfsetroundjoin%
\pgfsetlinewidth{0.301125pt}%
\definecolor{currentstroke}{rgb}{0.500000,0.500000,0.500000}%
\pgfsetstrokecolor{currentstroke}%
\pgfsetstrokeopacity{0.300000}%
\pgfsetdash{}{0pt}%
\pgfpathmoveto{\pgfqpoint{2.840153in}{3.097805in}}%
\pgfusepath{stroke}%
\end{pgfscope}%
\begin{pgfscope}%
\pgfpathrectangle{\pgfqpoint{0.647939in}{0.492442in}}{\pgfqpoint{3.079299in}{3.079299in}}%
\pgfusepath{clip}%
\pgfsetroundcap%
\pgfsetroundjoin%
\definecolor{currentfill}{rgb}{0.500000,0.500000,0.500000}%
\pgfsetfillcolor{currentfill}%
\pgfsetfillopacity{0.300000}%
\pgfsetlinewidth{0.301125pt}%
\definecolor{currentstroke}{rgb}{0.500000,0.500000,0.500000}%
\pgfsetstrokecolor{currentstroke}%
\pgfsetstrokeopacity{0.300000}%
\pgfsetdash{}{0pt}%
\pgfpathmoveto{\pgfqpoint{0.000000in}{0.000000in}}%
\pgfpathlineto{\pgfqpoint{0.000000in}{0.000000in}}%
\pgfpathclose%
\pgfusepath{stroke,fill}%
\end{pgfscope}%
\begin{pgfscope}%
\pgfpathrectangle{\pgfqpoint{0.647939in}{0.492442in}}{\pgfqpoint{3.079299in}{3.079299in}}%
\pgfusepath{clip}%
\pgfsetroundcap%
\pgfsetroundjoin%
\pgfsetlinewidth{0.301125pt}%
\definecolor{currentstroke}{rgb}{0.500000,0.500000,0.500000}%
\pgfsetstrokecolor{currentstroke}%
\pgfsetstrokeopacity{0.300000}%
\pgfsetdash{}{0pt}%
\pgfpathmoveto{\pgfqpoint{2.836710in}{2.881779in}}%
\pgfusepath{stroke}%
\end{pgfscope}%
\begin{pgfscope}%
\pgfpathrectangle{\pgfqpoint{0.647939in}{0.492442in}}{\pgfqpoint{3.079299in}{3.079299in}}%
\pgfusepath{clip}%
\pgfsetroundcap%
\pgfsetroundjoin%
\definecolor{currentfill}{rgb}{0.500000,0.500000,0.500000}%
\pgfsetfillcolor{currentfill}%
\pgfsetfillopacity{0.300000}%
\pgfsetlinewidth{0.301125pt}%
\definecolor{currentstroke}{rgb}{0.500000,0.500000,0.500000}%
\pgfsetstrokecolor{currentstroke}%
\pgfsetstrokeopacity{0.300000}%
\pgfsetdash{}{0pt}%
\pgfpathmoveto{\pgfqpoint{0.000000in}{0.000000in}}%
\pgfpathlineto{\pgfqpoint{0.000000in}{0.000000in}}%
\pgfpathclose%
\pgfusepath{stroke,fill}%
\end{pgfscope}%
\begin{pgfscope}%
\pgfpathrectangle{\pgfqpoint{0.647939in}{0.492442in}}{\pgfqpoint{3.079299in}{3.079299in}}%
\pgfusepath{clip}%
\pgfsetroundcap%
\pgfsetroundjoin%
\pgfsetlinewidth{0.301125pt}%
\definecolor{currentstroke}{rgb}{0.500000,0.500000,0.500000}%
\pgfsetstrokecolor{currentstroke}%
\pgfsetstrokeopacity{0.300000}%
\pgfsetdash{}{0pt}%
\pgfpathmoveto{\pgfqpoint{2.882219in}{2.033127in}}%
\pgfusepath{stroke}%
\end{pgfscope}%
\begin{pgfscope}%
\pgfpathrectangle{\pgfqpoint{0.647939in}{0.492442in}}{\pgfqpoint{3.079299in}{3.079299in}}%
\pgfusepath{clip}%
\pgfsetroundcap%
\pgfsetroundjoin%
\definecolor{currentfill}{rgb}{0.500000,0.500000,0.500000}%
\pgfsetfillcolor{currentfill}%
\pgfsetfillopacity{0.300000}%
\pgfsetlinewidth{0.301125pt}%
\definecolor{currentstroke}{rgb}{0.500000,0.500000,0.500000}%
\pgfsetstrokecolor{currentstroke}%
\pgfsetstrokeopacity{0.300000}%
\pgfsetdash{}{0pt}%
\pgfpathmoveto{\pgfqpoint{0.000000in}{0.000000in}}%
\pgfpathlineto{\pgfqpoint{0.000000in}{0.000000in}}%
\pgfpathclose%
\pgfusepath{stroke,fill}%
\end{pgfscope}%
\begin{pgfscope}%
\pgfpathrectangle{\pgfqpoint{0.647939in}{0.492442in}}{\pgfqpoint{3.079299in}{3.079299in}}%
\pgfusepath{clip}%
\pgfsetroundcap%
\pgfsetroundjoin%
\pgfsetlinewidth{0.301125pt}%
\definecolor{currentstroke}{rgb}{0.500000,0.500000,0.500000}%
\pgfsetstrokecolor{currentstroke}%
\pgfsetstrokeopacity{0.300000}%
\pgfsetdash{}{0pt}%
\pgfpathmoveto{\pgfqpoint{2.223270in}{1.182800in}}%
\pgfusepath{stroke}%
\end{pgfscope}%
\begin{pgfscope}%
\pgfpathrectangle{\pgfqpoint{0.647939in}{0.492442in}}{\pgfqpoint{3.079299in}{3.079299in}}%
\pgfusepath{clip}%
\pgfsetroundcap%
\pgfsetroundjoin%
\definecolor{currentfill}{rgb}{0.500000,0.500000,0.500000}%
\pgfsetfillcolor{currentfill}%
\pgfsetfillopacity{0.300000}%
\pgfsetlinewidth{0.301125pt}%
\definecolor{currentstroke}{rgb}{0.500000,0.500000,0.500000}%
\pgfsetstrokecolor{currentstroke}%
\pgfsetstrokeopacity{0.300000}%
\pgfsetdash{}{0pt}%
\pgfpathmoveto{\pgfqpoint{0.000000in}{0.000000in}}%
\pgfpathlineto{\pgfqpoint{0.000000in}{0.000000in}}%
\pgfpathclose%
\pgfusepath{stroke,fill}%
\end{pgfscope}%
\begin{pgfscope}%
\pgfpathrectangle{\pgfqpoint{0.647939in}{0.492442in}}{\pgfqpoint{3.079299in}{3.079299in}}%
\pgfusepath{clip}%
\pgfsetroundcap%
\pgfsetroundjoin%
\pgfsetlinewidth{0.301125pt}%
\definecolor{currentstroke}{rgb}{0.500000,0.500000,0.500000}%
\pgfsetstrokecolor{currentstroke}%
\pgfsetstrokeopacity{0.300000}%
\pgfsetdash{}{0pt}%
\pgfpathmoveto{\pgfqpoint{2.154116in}{2.853408in}}%
\pgfusepath{stroke}%
\end{pgfscope}%
\begin{pgfscope}%
\pgfpathrectangle{\pgfqpoint{0.647939in}{0.492442in}}{\pgfqpoint{3.079299in}{3.079299in}}%
\pgfusepath{clip}%
\pgfsetroundcap%
\pgfsetroundjoin%
\definecolor{currentfill}{rgb}{0.500000,0.500000,0.500000}%
\pgfsetfillcolor{currentfill}%
\pgfsetfillopacity{0.300000}%
\pgfsetlinewidth{0.301125pt}%
\definecolor{currentstroke}{rgb}{0.500000,0.500000,0.500000}%
\pgfsetstrokecolor{currentstroke}%
\pgfsetstrokeopacity{0.300000}%
\pgfsetdash{}{0pt}%
\pgfpathmoveto{\pgfqpoint{0.000000in}{0.000000in}}%
\pgfpathlineto{\pgfqpoint{0.000000in}{0.000000in}}%
\pgfpathclose%
\pgfusepath{stroke,fill}%
\end{pgfscope}%
\begin{pgfscope}%
\pgfpathrectangle{\pgfqpoint{0.647939in}{0.492442in}}{\pgfqpoint{3.079299in}{3.079299in}}%
\pgfusepath{clip}%
\pgfsetroundcap%
\pgfsetroundjoin%
\pgfsetlinewidth{0.301125pt}%
\definecolor{currentstroke}{rgb}{0.500000,0.500000,0.500000}%
\pgfsetstrokecolor{currentstroke}%
\pgfsetstrokeopacity{0.300000}%
\pgfsetdash{}{0pt}%
\pgfpathmoveto{\pgfqpoint{1.440633in}{2.757060in}}%
\pgfusepath{stroke}%
\end{pgfscope}%
\begin{pgfscope}%
\pgfpathrectangle{\pgfqpoint{0.647939in}{0.492442in}}{\pgfqpoint{3.079299in}{3.079299in}}%
\pgfusepath{clip}%
\pgfsetroundcap%
\pgfsetroundjoin%
\definecolor{currentfill}{rgb}{0.500000,0.500000,0.500000}%
\pgfsetfillcolor{currentfill}%
\pgfsetfillopacity{0.300000}%
\pgfsetlinewidth{0.301125pt}%
\definecolor{currentstroke}{rgb}{0.500000,0.500000,0.500000}%
\pgfsetstrokecolor{currentstroke}%
\pgfsetstrokeopacity{0.300000}%
\pgfsetdash{}{0pt}%
\pgfpathmoveto{\pgfqpoint{0.000000in}{0.000000in}}%
\pgfpathlineto{\pgfqpoint{0.000000in}{0.000000in}}%
\pgfpathclose%
\pgfusepath{stroke,fill}%
\end{pgfscope}%
\begin{pgfscope}%
\pgfpathrectangle{\pgfqpoint{0.647939in}{0.492442in}}{\pgfqpoint{3.079299in}{3.079299in}}%
\pgfusepath{clip}%
\pgfsetroundcap%
\pgfsetroundjoin%
\pgfsetlinewidth{0.301125pt}%
\definecolor{currentstroke}{rgb}{0.500000,0.500000,0.500000}%
\pgfsetstrokecolor{currentstroke}%
\pgfsetstrokeopacity{0.300000}%
\pgfsetdash{}{0pt}%
\pgfpathmoveto{\pgfqpoint{1.684219in}{1.893140in}}%
\pgfusepath{stroke}%
\end{pgfscope}%
\begin{pgfscope}%
\pgfpathrectangle{\pgfqpoint{0.647939in}{0.492442in}}{\pgfqpoint{3.079299in}{3.079299in}}%
\pgfusepath{clip}%
\pgfsetroundcap%
\pgfsetroundjoin%
\definecolor{currentfill}{rgb}{0.500000,0.500000,0.500000}%
\pgfsetfillcolor{currentfill}%
\pgfsetfillopacity{0.300000}%
\pgfsetlinewidth{0.301125pt}%
\definecolor{currentstroke}{rgb}{0.500000,0.500000,0.500000}%
\pgfsetstrokecolor{currentstroke}%
\pgfsetstrokeopacity{0.300000}%
\pgfsetdash{}{0pt}%
\pgfpathmoveto{\pgfqpoint{0.000000in}{0.000000in}}%
\pgfpathlineto{\pgfqpoint{0.000000in}{0.000000in}}%
\pgfpathclose%
\pgfusepath{stroke,fill}%
\end{pgfscope}%
\begin{pgfscope}%
\pgfpathrectangle{\pgfqpoint{0.647939in}{0.492442in}}{\pgfqpoint{3.079299in}{3.079299in}}%
\pgfusepath{clip}%
\pgfsetroundcap%
\pgfsetroundjoin%
\pgfsetlinewidth{0.301125pt}%
\definecolor{currentstroke}{rgb}{0.500000,0.500000,0.500000}%
\pgfsetstrokecolor{currentstroke}%
\pgfsetstrokeopacity{0.300000}%
\pgfsetdash{}{0pt}%
\pgfpathmoveto{\pgfqpoint{2.083600in}{2.719286in}}%
\pgfusepath{stroke}%
\end{pgfscope}%
\begin{pgfscope}%
\pgfpathrectangle{\pgfqpoint{0.647939in}{0.492442in}}{\pgfqpoint{3.079299in}{3.079299in}}%
\pgfusepath{clip}%
\pgfsetroundcap%
\pgfsetroundjoin%
\definecolor{currentfill}{rgb}{0.500000,0.500000,0.500000}%
\pgfsetfillcolor{currentfill}%
\pgfsetfillopacity{0.300000}%
\pgfsetlinewidth{0.301125pt}%
\definecolor{currentstroke}{rgb}{0.500000,0.500000,0.500000}%
\pgfsetstrokecolor{currentstroke}%
\pgfsetstrokeopacity{0.300000}%
\pgfsetdash{}{0pt}%
\pgfpathmoveto{\pgfqpoint{0.000000in}{0.000000in}}%
\pgfpathlineto{\pgfqpoint{0.000000in}{0.000000in}}%
\pgfpathclose%
\pgfusepath{stroke,fill}%
\end{pgfscope}%
\begin{pgfscope}%
\pgfpathrectangle{\pgfqpoint{0.647939in}{0.492442in}}{\pgfqpoint{3.079299in}{3.079299in}}%
\pgfusepath{clip}%
\pgfsetroundcap%
\pgfsetroundjoin%
\pgfsetlinewidth{0.301125pt}%
\definecolor{currentstroke}{rgb}{0.500000,0.500000,0.500000}%
\pgfsetstrokecolor{currentstroke}%
\pgfsetstrokeopacity{0.300000}%
\pgfsetdash{}{0pt}%
\pgfpathmoveto{\pgfqpoint{1.638540in}{2.028128in}}%
\pgfusepath{stroke}%
\end{pgfscope}%
\begin{pgfscope}%
\pgfpathrectangle{\pgfqpoint{0.647939in}{0.492442in}}{\pgfqpoint{3.079299in}{3.079299in}}%
\pgfusepath{clip}%
\pgfsetroundcap%
\pgfsetroundjoin%
\definecolor{currentfill}{rgb}{0.500000,0.500000,0.500000}%
\pgfsetfillcolor{currentfill}%
\pgfsetfillopacity{0.300000}%
\pgfsetlinewidth{0.301125pt}%
\definecolor{currentstroke}{rgb}{0.500000,0.500000,0.500000}%
\pgfsetstrokecolor{currentstroke}%
\pgfsetstrokeopacity{0.300000}%
\pgfsetdash{}{0pt}%
\pgfpathmoveto{\pgfqpoint{0.000000in}{0.000000in}}%
\pgfpathlineto{\pgfqpoint{0.000000in}{0.000000in}}%
\pgfpathclose%
\pgfusepath{stroke,fill}%
\end{pgfscope}%
\begin{pgfscope}%
\pgfpathrectangle{\pgfqpoint{0.647939in}{0.492442in}}{\pgfqpoint{3.079299in}{3.079299in}}%
\pgfusepath{clip}%
\pgfsetroundcap%
\pgfsetroundjoin%
\pgfsetlinewidth{0.301125pt}%
\definecolor{currentstroke}{rgb}{0.500000,0.500000,0.500000}%
\pgfsetstrokecolor{currentstroke}%
\pgfsetstrokeopacity{0.300000}%
\pgfsetdash{}{0pt}%
\pgfpathmoveto{\pgfqpoint{2.025024in}{2.603069in}}%
\pgfusepath{stroke}%
\end{pgfscope}%
\begin{pgfscope}%
\pgfpathrectangle{\pgfqpoint{0.647939in}{0.492442in}}{\pgfqpoint{3.079299in}{3.079299in}}%
\pgfusepath{clip}%
\pgfsetroundcap%
\pgfsetroundjoin%
\definecolor{currentfill}{rgb}{0.500000,0.500000,0.500000}%
\pgfsetfillcolor{currentfill}%
\pgfsetfillopacity{0.300000}%
\pgfsetlinewidth{0.301125pt}%
\definecolor{currentstroke}{rgb}{0.500000,0.500000,0.500000}%
\pgfsetstrokecolor{currentstroke}%
\pgfsetstrokeopacity{0.300000}%
\pgfsetdash{}{0pt}%
\pgfpathmoveto{\pgfqpoint{0.000000in}{0.000000in}}%
\pgfpathlineto{\pgfqpoint{0.000000in}{0.000000in}}%
\pgfpathclose%
\pgfusepath{stroke,fill}%
\end{pgfscope}%
\begin{pgfscope}%
\pgfpathrectangle{\pgfqpoint{0.647939in}{0.492442in}}{\pgfqpoint{3.079299in}{3.079299in}}%
\pgfusepath{clip}%
\pgfsetroundcap%
\pgfsetroundjoin%
\pgfsetlinewidth{0.301125pt}%
\definecolor{currentstroke}{rgb}{0.500000,0.500000,0.500000}%
\pgfsetstrokecolor{currentstroke}%
\pgfsetstrokeopacity{0.300000}%
\pgfsetdash{}{0pt}%
\pgfpathmoveto{\pgfqpoint{2.295060in}{1.467783in}}%
\pgfusepath{stroke}%
\end{pgfscope}%
\begin{pgfscope}%
\pgfpathrectangle{\pgfqpoint{0.647939in}{0.492442in}}{\pgfqpoint{3.079299in}{3.079299in}}%
\pgfusepath{clip}%
\pgfsetroundcap%
\pgfsetroundjoin%
\definecolor{currentfill}{rgb}{0.500000,0.500000,0.500000}%
\pgfsetfillcolor{currentfill}%
\pgfsetfillopacity{0.300000}%
\pgfsetlinewidth{0.301125pt}%
\definecolor{currentstroke}{rgb}{0.500000,0.500000,0.500000}%
\pgfsetstrokecolor{currentstroke}%
\pgfsetstrokeopacity{0.300000}%
\pgfsetdash{}{0pt}%
\pgfpathmoveto{\pgfqpoint{0.000000in}{0.000000in}}%
\pgfpathlineto{\pgfqpoint{0.000000in}{0.000000in}}%
\pgfpathclose%
\pgfusepath{stroke,fill}%
\end{pgfscope}%
\begin{pgfscope}%
\pgfpathrectangle{\pgfqpoint{0.647939in}{0.492442in}}{\pgfqpoint{3.079299in}{3.079299in}}%
\pgfusepath{clip}%
\pgfsetroundcap%
\pgfsetroundjoin%
\pgfsetlinewidth{0.301125pt}%
\definecolor{currentstroke}{rgb}{0.500000,0.500000,0.500000}%
\pgfsetstrokecolor{currentstroke}%
\pgfsetstrokeopacity{0.300000}%
\pgfsetdash{}{0pt}%
\pgfpathmoveto{\pgfqpoint{2.616515in}{2.061816in}}%
\pgfusepath{stroke}%
\end{pgfscope}%
\begin{pgfscope}%
\pgfpathrectangle{\pgfqpoint{0.647939in}{0.492442in}}{\pgfqpoint{3.079299in}{3.079299in}}%
\pgfusepath{clip}%
\pgfsetroundcap%
\pgfsetroundjoin%
\definecolor{currentfill}{rgb}{0.500000,0.500000,0.500000}%
\pgfsetfillcolor{currentfill}%
\pgfsetfillopacity{0.300000}%
\pgfsetlinewidth{0.301125pt}%
\definecolor{currentstroke}{rgb}{0.500000,0.500000,0.500000}%
\pgfsetstrokecolor{currentstroke}%
\pgfsetstrokeopacity{0.300000}%
\pgfsetdash{}{0pt}%
\pgfpathmoveto{\pgfqpoint{0.000000in}{0.000000in}}%
\pgfpathlineto{\pgfqpoint{0.000000in}{0.000000in}}%
\pgfpathclose%
\pgfusepath{stroke,fill}%
\end{pgfscope}%
\begin{pgfscope}%
\pgfpathrectangle{\pgfqpoint{0.647939in}{0.492442in}}{\pgfqpoint{3.079299in}{3.079299in}}%
\pgfusepath{clip}%
\pgfsetroundcap%
\pgfsetroundjoin%
\pgfsetlinewidth{0.301125pt}%
\definecolor{currentstroke}{rgb}{0.500000,0.500000,0.500000}%
\pgfsetstrokecolor{currentstroke}%
\pgfsetstrokeopacity{0.300000}%
\pgfsetdash{}{0pt}%
\pgfpathmoveto{\pgfqpoint{2.468266in}{1.994209in}}%
\pgfusepath{stroke}%
\end{pgfscope}%
\begin{pgfscope}%
\pgfpathrectangle{\pgfqpoint{0.647939in}{0.492442in}}{\pgfqpoint{3.079299in}{3.079299in}}%
\pgfusepath{clip}%
\pgfsetroundcap%
\pgfsetroundjoin%
\definecolor{currentfill}{rgb}{0.500000,0.500000,0.500000}%
\pgfsetfillcolor{currentfill}%
\pgfsetfillopacity{0.300000}%
\pgfsetlinewidth{0.301125pt}%
\definecolor{currentstroke}{rgb}{0.500000,0.500000,0.500000}%
\pgfsetstrokecolor{currentstroke}%
\pgfsetstrokeopacity{0.300000}%
\pgfsetdash{}{0pt}%
\pgfpathmoveto{\pgfqpoint{0.000000in}{0.000000in}}%
\pgfpathlineto{\pgfqpoint{0.000000in}{0.000000in}}%
\pgfpathclose%
\pgfusepath{stroke,fill}%
\end{pgfscope}%
\begin{pgfscope}%
\pgfpathrectangle{\pgfqpoint{0.647939in}{0.492442in}}{\pgfqpoint{3.079299in}{3.079299in}}%
\pgfusepath{clip}%
\pgfsetroundcap%
\pgfsetroundjoin%
\pgfsetlinewidth{0.301125pt}%
\definecolor{currentstroke}{rgb}{0.500000,0.500000,0.500000}%
\pgfsetstrokecolor{currentstroke}%
\pgfsetstrokeopacity{0.300000}%
\pgfsetdash{}{0pt}%
\pgfpathmoveto{\pgfqpoint{2.242059in}{2.434032in}}%
\pgfusepath{stroke}%
\end{pgfscope}%
\begin{pgfscope}%
\pgfpathrectangle{\pgfqpoint{0.647939in}{0.492442in}}{\pgfqpoint{3.079299in}{3.079299in}}%
\pgfusepath{clip}%
\pgfsetroundcap%
\pgfsetroundjoin%
\definecolor{currentfill}{rgb}{0.500000,0.500000,0.500000}%
\pgfsetfillcolor{currentfill}%
\pgfsetfillopacity{0.300000}%
\pgfsetlinewidth{0.301125pt}%
\definecolor{currentstroke}{rgb}{0.500000,0.500000,0.500000}%
\pgfsetstrokecolor{currentstroke}%
\pgfsetstrokeopacity{0.300000}%
\pgfsetdash{}{0pt}%
\pgfpathmoveto{\pgfqpoint{0.000000in}{0.000000in}}%
\pgfpathlineto{\pgfqpoint{0.000000in}{0.000000in}}%
\pgfpathclose%
\pgfusepath{stroke,fill}%
\end{pgfscope}%
\begin{pgfscope}%
\pgfpathrectangle{\pgfqpoint{0.647939in}{0.492442in}}{\pgfqpoint{3.079299in}{3.079299in}}%
\pgfusepath{clip}%
\pgfsetroundcap%
\pgfsetroundjoin%
\pgfsetlinewidth{0.301125pt}%
\definecolor{currentstroke}{rgb}{0.500000,0.500000,0.500000}%
\pgfsetstrokecolor{currentstroke}%
\pgfsetstrokeopacity{0.300000}%
\pgfsetdash{}{0pt}%
\pgfpathmoveto{\pgfqpoint{2.146542in}{1.588505in}}%
\pgfusepath{stroke}%
\end{pgfscope}%
\begin{pgfscope}%
\pgfpathrectangle{\pgfqpoint{0.647939in}{0.492442in}}{\pgfqpoint{3.079299in}{3.079299in}}%
\pgfusepath{clip}%
\pgfsetroundcap%
\pgfsetroundjoin%
\definecolor{currentfill}{rgb}{0.500000,0.500000,0.500000}%
\pgfsetfillcolor{currentfill}%
\pgfsetfillopacity{0.300000}%
\pgfsetlinewidth{0.301125pt}%
\definecolor{currentstroke}{rgb}{0.500000,0.500000,0.500000}%
\pgfsetstrokecolor{currentstroke}%
\pgfsetstrokeopacity{0.300000}%
\pgfsetdash{}{0pt}%
\pgfpathmoveto{\pgfqpoint{0.000000in}{0.000000in}}%
\pgfpathlineto{\pgfqpoint{0.000000in}{0.000000in}}%
\pgfpathclose%
\pgfusepath{stroke,fill}%
\end{pgfscope}%
\begin{pgfscope}%
\pgfpathrectangle{\pgfqpoint{0.647939in}{0.492442in}}{\pgfqpoint{3.079299in}{3.079299in}}%
\pgfusepath{clip}%
\pgfsetroundcap%
\pgfsetroundjoin%
\pgfsetlinewidth{0.301125pt}%
\definecolor{currentstroke}{rgb}{0.500000,0.500000,0.500000}%
\pgfsetstrokecolor{currentstroke}%
\pgfsetstrokeopacity{0.300000}%
\pgfsetdash{}{0pt}%
\pgfpathmoveto{\pgfqpoint{2.310073in}{1.870262in}}%
\pgfusepath{stroke}%
\end{pgfscope}%
\begin{pgfscope}%
\pgfpathrectangle{\pgfqpoint{0.647939in}{0.492442in}}{\pgfqpoint{3.079299in}{3.079299in}}%
\pgfusepath{clip}%
\pgfsetroundcap%
\pgfsetroundjoin%
\definecolor{currentfill}{rgb}{0.500000,0.500000,0.500000}%
\pgfsetfillcolor{currentfill}%
\pgfsetfillopacity{0.300000}%
\pgfsetlinewidth{0.301125pt}%
\definecolor{currentstroke}{rgb}{0.500000,0.500000,0.500000}%
\pgfsetstrokecolor{currentstroke}%
\pgfsetstrokeopacity{0.300000}%
\pgfsetdash{}{0pt}%
\pgfpathmoveto{\pgfqpoint{0.000000in}{0.000000in}}%
\pgfpathlineto{\pgfqpoint{0.000000in}{0.000000in}}%
\pgfpathclose%
\pgfusepath{stroke,fill}%
\end{pgfscope}%
\begin{pgfscope}%
\pgfpathrectangle{\pgfqpoint{0.647939in}{0.492442in}}{\pgfqpoint{3.079299in}{3.079299in}}%
\pgfusepath{clip}%
\pgfsetbuttcap%
\pgfsetroundjoin%
\pgfsetlinewidth{0.301125pt}%
\definecolor{currentstroke}{rgb}{0.500000,0.500000,0.500000}%
\pgfsetstrokecolor{currentstroke}%
\pgfsetstrokeopacity{0.300000}%
\pgfsetdash{}{0pt}%
\pgfpathmoveto{\pgfqpoint{0.647939in}{0.492442in}}%
\pgfpathlineto{\pgfqpoint{0.647939in}{0.492442in}}%
\pgfpathlineto{\pgfqpoint{0.714786in}{0.506986in}}%
\pgfpathlineto{\pgfqpoint{0.780867in}{0.524657in}}%
\pgfpathlineto{\pgfqpoint{0.845906in}{0.545819in}}%
\pgfpathlineto{\pgfqpoint{0.909568in}{0.570794in}}%
\pgfpathlineto{\pgfqpoint{0.971473in}{0.599831in}}%
\pgfpathlineto{\pgfqpoint{1.031215in}{0.633074in}}%
\pgfpathlineto{\pgfqpoint{1.088403in}{0.670532in}}%
\pgfpathlineto{\pgfqpoint{1.142711in}{0.712067in}}%
\pgfpathlineto{\pgfqpoint{1.193923in}{0.757349in}}%
\pgfpathlineto{\pgfqpoint{1.241973in}{0.805972in}}%
\pgfpathlineto{\pgfqpoint{1.286959in}{0.857449in}}%
\pgfpathlineto{\pgfqpoint{1.329120in}{0.911272in}}%
\pgfpathlineto{\pgfqpoint{1.368819in}{0.966943in}}%
\pgfpathlineto{\pgfqpoint{1.406482in}{1.024001in}}%
\pgfpathlineto{\pgfqpoint{1.442562in}{1.082064in}}%
\pgfpathlineto{\pgfqpoint{1.477508in}{1.140818in}}%
\pgfpathlineto{\pgfqpoint{1.511750in}{1.199988in}}%
\pgfpathlineto{\pgfqpoint{1.545691in}{1.259339in}}%
\pgfpathlineto{\pgfqpoint{1.579700in}{1.318665in}}%
\pgfpathlineto{\pgfqpoint{1.614117in}{1.377758in}}%
\pgfpathlineto{\pgfqpoint{1.649254in}{1.436399in}}%
\pgfpathlineto{\pgfqpoint{1.685401in}{1.494389in}}%
\pgfpathlineto{\pgfqpoint{1.722843in}{1.551546in}}%
\pgfpathlineto{\pgfqpoint{1.761848in}{1.607660in}}%
\pgfpathlineto{\pgfqpoint{1.802661in}{1.662456in}}%
\pgfpathlineto{\pgfqpoint{1.845506in}{1.715600in}}%
\pgfpathlineto{\pgfqpoint{1.890639in}{1.766865in}}%
\pgfpathlineto{\pgfqpoint{1.938106in}{1.815887in}}%
\pgfpathlineto{\pgfqpoint{1.987863in}{1.862506in}}%
\pgfpathlineto{\pgfqpoint{2.039402in}{1.906880in}}%
\pgfpathlineto{\pgfqpoint{2.091371in}{1.949785in}}%
\pgfpathlineto{\pgfqpoint{2.142008in}{1.993131in}}%
\pgfpathlineto{\pgfqpoint{2.188796in}{2.037263in}}%
\pgfpathlineto{\pgfqpoint{2.234670in}{2.083385in}}%
\pgfpathlineto{\pgfqpoint{2.280103in}{2.130930in}}%
\pgfpathlineto{\pgfqpoint{2.325316in}{2.179547in}}%
\pgfpathlineto{\pgfqpoint{2.370251in}{2.229775in}}%
\pgfpathlineto{\pgfqpoint{2.414499in}{2.280863in}}%
\pgfpathlineto{\pgfqpoint{2.458334in}{2.332782in}}%
\pgfpathlineto{\pgfqpoint{2.501774in}{2.385266in}}%
\pgfpathlineto{\pgfqpoint{2.544890in}{2.438159in}}%
\pgfpathlineto{\pgfqpoint{2.587734in}{2.491355in}}%
\pgfpathlineto{\pgfqpoint{2.630352in}{2.544753in}}%
\pgfpathlineto{\pgfqpoint{2.672836in}{2.598320in}}%
\pgfpathlineto{\pgfqpoint{2.715219in}{2.651964in}}%
\pgfpathlineto{\pgfqpoint{2.757570in}{2.705645in}}%
\pgfpathlineto{\pgfqpoint{2.799958in}{2.759329in}}%
\pgfpathlineto{\pgfqpoint{2.842429in}{2.812945in}}%
\pgfpathlineto{\pgfqpoint{2.885041in}{2.866454in}}%
\pgfpathlineto{\pgfqpoint{2.927861in}{2.919811in}}%
\pgfpathlineto{\pgfqpoint{2.970945in}{2.972955in}}%
\pgfpathlineto{\pgfqpoint{3.014351in}{3.025838in}}%
\pgfpathlineto{\pgfqpoint{3.058149in}{3.078401in}}%
\pgfpathlineto{\pgfqpoint{3.102404in}{3.130582in}}%
\pgfpathlineto{\pgfqpoint{3.147185in}{3.182313in}}%
\pgfpathlineto{\pgfqpoint{3.192568in}{3.233518in}}%
\pgfpathlineto{\pgfqpoint{3.238628in}{3.284116in}}%
\pgfpathlineto{\pgfqpoint{3.285446in}{3.334013in}}%
\pgfpathlineto{\pgfqpoint{3.333110in}{3.383104in}}%
\pgfpathlineto{\pgfqpoint{3.381706in}{3.431271in}}%
\pgfpathlineto{\pgfqpoint{3.431331in}{3.478378in}}%
\pgfpathlineto{\pgfqpoint{3.482078in}{3.524272in}}%
\pgfpathlineto{\pgfqpoint{3.534043in}{3.568780in}}%
\pgfpathlineto{\pgfqpoint{3.537602in}{3.571741in}}%
\pgfusepath{stroke}%
\end{pgfscope}%
\begin{pgfscope}%
\pgfpathrectangle{\pgfqpoint{0.647939in}{0.492442in}}{\pgfqpoint{3.079299in}{3.079299in}}%
\pgfusepath{clip}%
\pgfsetbuttcap%
\pgfsetroundjoin%
\pgfsetlinewidth{0.301125pt}%
\definecolor{currentstroke}{rgb}{0.500000,0.500000,0.500000}%
\pgfsetstrokecolor{currentstroke}%
\pgfsetstrokeopacity{0.300000}%
\pgfsetdash{}{0pt}%
\pgfpathmoveto{\pgfqpoint{0.857891in}{0.492442in}}%
\pgfpathlineto{\pgfqpoint{0.857891in}{0.492442in}}%
\pgfpathlineto{\pgfqpoint{0.920978in}{0.518829in}}%
\pgfpathlineto{\pgfqpoint{0.982100in}{0.549473in}}%
\pgfpathlineto{\pgfqpoint{1.040815in}{0.584487in}}%
\pgfpathlineto{\pgfqpoint{1.096710in}{0.623827in}}%
\pgfpathlineto{\pgfqpoint{1.149462in}{0.667291in}}%
\pgfpathlineto{\pgfqpoint{1.198892in}{0.714506in}}%
\pgfpathlineto{\pgfqpoint{1.245000in}{0.764963in}}%
\pgfusepath{stroke}%
\end{pgfscope}%
\begin{pgfscope}%
\pgfpathrectangle{\pgfqpoint{0.647939in}{0.492442in}}{\pgfqpoint{3.079299in}{3.079299in}}%
\pgfusepath{clip}%
\pgfsetbuttcap%
\pgfsetroundjoin%
\pgfsetlinewidth{0.301125pt}%
\definecolor{currentstroke}{rgb}{0.500000,0.500000,0.500000}%
\pgfsetstrokecolor{currentstroke}%
\pgfsetstrokeopacity{0.300000}%
\pgfsetdash{}{0pt}%
\pgfpathmoveto{\pgfqpoint{1.137828in}{0.492442in}}%
\pgfpathlineto{\pgfqpoint{1.137828in}{0.492442in}}%
\pgfpathlineto{\pgfqpoint{1.182879in}{0.543812in}}%
\pgfpathlineto{\pgfqpoint{1.224186in}{0.598238in}}%
\pgfpathlineto{\pgfqpoint{1.262150in}{0.655056in}}%
\pgfpathlineto{\pgfqpoint{1.297300in}{0.713693in}}%
\pgfpathlineto{\pgfqpoint{1.330203in}{0.773651in}}%
\pgfpathlineto{\pgfqpoint{1.361415in}{0.834521in}}%
\pgfpathlineto{\pgfqpoint{1.391439in}{0.895982in}}%
\pgfpathlineto{\pgfqpoint{1.420716in}{0.957788in}}%
\pgfusepath{stroke}%
\end{pgfscope}%
\begin{pgfscope}%
\pgfpathrectangle{\pgfqpoint{0.647939in}{0.492442in}}{\pgfqpoint{3.079299in}{3.079299in}}%
\pgfusepath{clip}%
\pgfsetbuttcap%
\pgfsetroundjoin%
\pgfsetlinewidth{0.301125pt}%
\definecolor{currentstroke}{rgb}{0.500000,0.500000,0.500000}%
\pgfsetstrokecolor{currentstroke}%
\pgfsetstrokeopacity{0.300000}%
\pgfsetdash{}{0pt}%
\pgfpathmoveto{\pgfqpoint{1.347780in}{0.492442in}}%
\pgfpathlineto{\pgfqpoint{1.347780in}{0.492442in}}%
\pgfpathlineto{\pgfqpoint{1.352209in}{0.560455in}}%
\pgfpathlineto{\pgfqpoint{1.361622in}{0.628053in}}%
\pgfpathlineto{\pgfqpoint{1.374749in}{0.695082in}}%
\pgfpathlineto{\pgfqpoint{1.390713in}{0.761477in}}%
\pgfpathlineto{\pgfqpoint{1.408926in}{0.827313in}}%
\pgfusepath{stroke}%
\end{pgfscope}%
\begin{pgfscope}%
\pgfpathrectangle{\pgfqpoint{0.647939in}{0.492442in}}{\pgfqpoint{3.079299in}{3.079299in}}%
\pgfusepath{clip}%
\pgfsetbuttcap%
\pgfsetroundjoin%
\pgfsetlinewidth{0.301125pt}%
\definecolor{currentstroke}{rgb}{0.500000,0.500000,0.500000}%
\pgfsetstrokecolor{currentstroke}%
\pgfsetstrokeopacity{0.300000}%
\pgfsetdash{}{0pt}%
\pgfpathmoveto{\pgfqpoint{1.627716in}{0.492442in}}%
\pgfpathlineto{\pgfqpoint{1.627716in}{0.492442in}}%
\pgfpathlineto{\pgfqpoint{1.570912in}{0.529913in}}%
\pgfpathlineto{\pgfqpoint{1.523352in}{0.578319in}}%
\pgfpathlineto{\pgfqpoint{1.491524in}{0.630125in}}%
\pgfpathlineto{\pgfqpoint{1.471862in}{0.684619in}}%
\pgfpathlineto{\pgfqpoint{1.461615in}{0.743623in}}%
\pgfpathlineto{\pgfqpoint{1.459869in}{0.809508in}}%
\pgfpathlineto{\pgfqpoint{1.465795in}{0.877454in}}%
\pgfpathlineto{\pgfqpoint{1.477438in}{0.944643in}}%
\pgfpathlineto{\pgfqpoint{1.493333in}{1.011037in}}%
\pgfpathlineto{\pgfqpoint{1.512544in}{1.076570in}}%
\pgfpathlineto{\pgfqpoint{1.534448in}{1.141256in}}%
\pgfpathlineto{\pgfqpoint{1.558646in}{1.205126in}}%
\pgfusepath{stroke}%
\end{pgfscope}%
\begin{pgfscope}%
\pgfpathrectangle{\pgfqpoint{0.647939in}{0.492442in}}{\pgfqpoint{3.079299in}{3.079299in}}%
\pgfusepath{clip}%
\pgfsetbuttcap%
\pgfsetroundjoin%
\pgfsetlinewidth{0.301125pt}%
\definecolor{currentstroke}{rgb}{0.500000,0.500000,0.500000}%
\pgfsetstrokecolor{currentstroke}%
\pgfsetstrokeopacity{0.300000}%
\pgfsetdash{}{0pt}%
\pgfpathmoveto{\pgfqpoint{1.907652in}{0.492442in}}%
\pgfpathlineto{\pgfqpoint{1.907652in}{0.492442in}}%
\pgfpathlineto{\pgfqpoint{1.839855in}{0.501477in}}%
\pgfpathlineto{\pgfqpoint{1.772837in}{0.515011in}}%
\pgfpathlineto{\pgfqpoint{1.707386in}{0.534624in}}%
\pgfpathlineto{\pgfqpoint{1.645001in}{0.562280in}}%
\pgfpathlineto{\pgfqpoint{1.588396in}{0.600068in}}%
\pgfpathlineto{\pgfqpoint{1.542097in}{0.648475in}}%
\pgfusepath{stroke}%
\end{pgfscope}%
\begin{pgfscope}%
\pgfpathrectangle{\pgfqpoint{0.647939in}{0.492442in}}{\pgfqpoint{3.079299in}{3.079299in}}%
\pgfusepath{clip}%
\pgfsetbuttcap%
\pgfsetroundjoin%
\pgfsetlinewidth{0.301125pt}%
\definecolor{currentstroke}{rgb}{0.500000,0.500000,0.500000}%
\pgfsetstrokecolor{currentstroke}%
\pgfsetstrokeopacity{0.300000}%
\pgfsetdash{}{0pt}%
\pgfpathmoveto{\pgfqpoint{2.327557in}{0.492442in}}%
\pgfpathlineto{\pgfqpoint{2.327557in}{0.492442in}}%
\pgfpathlineto{\pgfqpoint{2.259152in}{0.494204in}}%
\pgfpathlineto{\pgfqpoint{2.190735in}{0.495494in}}%
\pgfpathlineto{\pgfqpoint{2.122319in}{0.496828in}}%
\pgfpathlineto{\pgfqpoint{2.053921in}{0.498801in}}%
\pgfpathlineto{\pgfqpoint{1.985577in}{0.502105in}}%
\pgfusepath{stroke}%
\end{pgfscope}%
\begin{pgfscope}%
\pgfpathrectangle{\pgfqpoint{0.647939in}{0.492442in}}{\pgfqpoint{3.079299in}{3.079299in}}%
\pgfusepath{clip}%
\pgfsetbuttcap%
\pgfsetroundjoin%
\pgfsetlinewidth{0.301125pt}%
\definecolor{currentstroke}{rgb}{0.500000,0.500000,0.500000}%
\pgfsetstrokecolor{currentstroke}%
\pgfsetstrokeopacity{0.300000}%
\pgfsetdash{}{0pt}%
\pgfpathmoveto{\pgfqpoint{2.747461in}{0.492442in}}%
\pgfpathlineto{\pgfqpoint{2.747461in}{0.492442in}}%
\pgfpathlineto{\pgfqpoint{2.680079in}{0.504313in}}%
\pgfpathlineto{\pgfqpoint{2.612374in}{0.514178in}}%
\pgfpathlineto{\pgfqpoint{2.544408in}{0.522061in}}%
\pgfpathlineto{\pgfqpoint{2.476251in}{0.528078in}}%
\pgfpathlineto{\pgfqpoint{2.407966in}{0.532433in}}%
\pgfpathlineto{\pgfqpoint{2.339605in}{0.535405in}}%
\pgfpathlineto{\pgfqpoint{2.271205in}{0.537357in}}%
\pgfpathlineto{\pgfqpoint{2.202790in}{0.538727in}}%
\pgfpathlineto{\pgfqpoint{2.134374in}{0.540043in}}%
\pgfpathlineto{\pgfqpoint{2.065972in}{0.541913in}}%
\pgfpathlineto{\pgfqpoint{1.997621in}{0.545038in}}%
\pgfpathlineto{\pgfqpoint{1.929404in}{0.550243in}}%
\pgfpathlineto{\pgfqpoint{1.861505in}{0.558536in}}%
\pgfpathlineto{\pgfqpoint{1.794310in}{0.571185in}}%
\pgfusepath{stroke}%
\end{pgfscope}%
\begin{pgfscope}%
\pgfpathrectangle{\pgfqpoint{0.647939in}{0.492442in}}{\pgfqpoint{3.079299in}{3.079299in}}%
\pgfusepath{clip}%
\pgfsetbuttcap%
\pgfsetroundjoin%
\pgfsetlinewidth{0.301125pt}%
\definecolor{currentstroke}{rgb}{0.500000,0.500000,0.500000}%
\pgfsetstrokecolor{currentstroke}%
\pgfsetstrokeopacity{0.300000}%
\pgfsetdash{}{0pt}%
\pgfpathmoveto{\pgfqpoint{2.957413in}{0.492442in}}%
\pgfpathlineto{\pgfqpoint{2.957413in}{0.492442in}}%
\pgfpathlineto{\pgfqpoint{2.891399in}{0.510439in}}%
\pgfpathlineto{\pgfqpoint{2.824955in}{0.526774in}}%
\pgfpathlineto{\pgfqpoint{2.758086in}{0.541264in}}%
\pgfpathlineto{\pgfqpoint{2.690820in}{0.553776in}}%
\pgfpathlineto{\pgfqpoint{2.623206in}{0.564245in}}%
\pgfpathlineto{\pgfqpoint{2.555308in}{0.572681in}}%
\pgfusepath{stroke}%
\end{pgfscope}%
\begin{pgfscope}%
\pgfpathrectangle{\pgfqpoint{0.647939in}{0.492442in}}{\pgfqpoint{3.079299in}{3.079299in}}%
\pgfusepath{clip}%
\pgfsetbuttcap%
\pgfsetroundjoin%
\pgfsetlinewidth{0.301125pt}%
\definecolor{currentstroke}{rgb}{0.500000,0.500000,0.500000}%
\pgfsetstrokecolor{currentstroke}%
\pgfsetstrokeopacity{0.300000}%
\pgfsetdash{}{0pt}%
\pgfpathmoveto{\pgfqpoint{3.167366in}{0.492442in}}%
\pgfpathlineto{\pgfqpoint{3.167366in}{0.492442in}}%
\pgfpathlineto{\pgfqpoint{3.102702in}{0.514823in}}%
\pgfpathlineto{\pgfqpoint{3.037742in}{0.536326in}}%
\pgfpathlineto{\pgfqpoint{2.972415in}{0.556683in}}%
\pgfpathlineto{\pgfqpoint{2.906670in}{0.575639in}}%
\pgfpathlineto{\pgfqpoint{2.840477in}{0.592958in}}%
\pgfpathlineto{\pgfqpoint{2.773831in}{0.608436in}}%
\pgfpathlineto{\pgfqpoint{2.706753in}{0.621916in}}%
\pgfpathlineto{\pgfqpoint{2.639287in}{0.633305in}}%
\pgfpathlineto{\pgfqpoint{2.571499in}{0.642585in}}%
\pgfpathlineto{\pgfqpoint{2.503464in}{0.649832in}}%
\pgfpathlineto{\pgfqpoint{2.435253in}{0.655210in}}%
\pgfpathlineto{\pgfqpoint{2.366933in}{0.658976in}}%
\pgfpathlineto{\pgfqpoint{2.298552in}{0.661477in}}%
\pgfpathlineto{\pgfqpoint{2.230144in}{0.663155in}}%
\pgfpathlineto{\pgfqpoint{2.161730in}{0.664548in}}%
\pgfpathlineto{\pgfqpoint{2.093324in}{0.666278in}}%
\pgfpathlineto{\pgfqpoint{2.024957in}{0.669076in}}%
\pgfpathlineto{\pgfqpoint{1.956704in}{0.673818in}}%
\pgfpathlineto{\pgfqpoint{1.888746in}{0.681597in}}%
\pgfpathlineto{\pgfqpoint{1.821476in}{0.693809in}}%
\pgfpathlineto{\pgfqpoint{1.755714in}{0.712269in}}%
\pgfpathlineto{\pgfqpoint{1.693135in}{0.739305in}}%
\pgfpathlineto{\pgfqpoint{1.636937in}{0.777448in}}%
\pgfpathlineto{\pgfqpoint{1.594282in}{0.824361in}}%
\pgfpathlineto{\pgfqpoint{1.567493in}{0.873558in}}%
\pgfpathlineto{\pgfqpoint{1.551962in}{0.925781in}}%
\pgfpathlineto{\pgfqpoint{1.545493in}{0.982970in}}%
\pgfpathlineto{\pgfqpoint{1.547592in}{1.046815in}}%
\pgfusepath{stroke}%
\end{pgfscope}%
\begin{pgfscope}%
\pgfpathrectangle{\pgfqpoint{0.647939in}{0.492442in}}{\pgfqpoint{3.079299in}{3.079299in}}%
\pgfusepath{clip}%
\pgfsetbuttcap%
\pgfsetroundjoin%
\pgfsetlinewidth{0.301125pt}%
\definecolor{currentstroke}{rgb}{0.500000,0.500000,0.500000}%
\pgfsetstrokecolor{currentstroke}%
\pgfsetstrokeopacity{0.300000}%
\pgfsetdash{}{0pt}%
\pgfpathmoveto{\pgfqpoint{3.377318in}{0.492442in}}%
\pgfpathlineto{\pgfqpoint{3.377318in}{0.492442in}}%
\pgfpathlineto{\pgfqpoint{3.313274in}{0.516544in}}%
\pgfpathlineto{\pgfqpoint{3.249240in}{0.540670in}}%
\pgfpathlineto{\pgfqpoint{3.185115in}{0.564553in}}%
\pgfpathlineto{\pgfqpoint{3.120801in}{0.587920in}}%
\pgfpathlineto{\pgfqpoint{3.056205in}{0.610493in}}%
\pgfpathlineto{\pgfqpoint{2.991247in}{0.631998in}}%
\pgfpathlineto{\pgfqpoint{2.925863in}{0.652165in}}%
\pgfpathlineto{\pgfqpoint{2.860010in}{0.670740in}}%
\pgfpathlineto{\pgfqpoint{2.793672in}{0.687493in}}%
\pgfpathlineto{\pgfqpoint{2.726860in}{0.702238in}}%
\pgfpathlineto{\pgfqpoint{2.659614in}{0.714849in}}%
\pgfpathlineto{\pgfqpoint{2.591994in}{0.725273in}}%
\pgfpathlineto{\pgfqpoint{2.524077in}{0.733544in}}%
\pgfpathlineto{\pgfqpoint{2.455942in}{0.739793in}}%
\pgfpathlineto{\pgfqpoint{2.387664in}{0.744254in}}%
\pgfpathlineto{\pgfqpoint{2.319305in}{0.747268in}}%
\pgfpathlineto{\pgfqpoint{2.250906in}{0.749272in}}%
\pgfpathlineto{\pgfqpoint{2.182495in}{0.750791in}}%
\pgfpathlineto{\pgfqpoint{2.114086in}{0.752455in}}%
\pgfpathlineto{\pgfqpoint{2.045709in}{0.755017in}}%
\pgfpathlineto{\pgfqpoint{1.977432in}{0.759400in}}%
\pgfpathlineto{\pgfqpoint{1.909428in}{0.766750in}}%
\pgfpathlineto{\pgfqpoint{1.842086in}{0.778539in}}%
\pgfpathlineto{\pgfqpoint{1.776256in}{0.796733in}}%
\pgfpathlineto{\pgfqpoint{1.713754in}{0.823909in}}%
\pgfusepath{stroke}%
\end{pgfscope}%
\begin{pgfscope}%
\pgfpathrectangle{\pgfqpoint{0.647939in}{0.492442in}}{\pgfqpoint{3.079299in}{3.079299in}}%
\pgfusepath{clip}%
\pgfsetbuttcap%
\pgfsetroundjoin%
\pgfsetlinewidth{0.301125pt}%
\definecolor{currentstroke}{rgb}{0.500000,0.500000,0.500000}%
\pgfsetstrokecolor{currentstroke}%
\pgfsetstrokeopacity{0.300000}%
\pgfsetdash{}{0pt}%
\pgfpathmoveto{\pgfqpoint{3.587270in}{0.492442in}}%
\pgfpathlineto{\pgfqpoint{3.587270in}{0.492442in}}%
\pgfpathlineto{\pgfqpoint{3.522801in}{0.515376in}}%
\pgfpathlineto{\pgfqpoint{3.458655in}{0.539198in}}%
\pgfpathlineto{\pgfqpoint{3.394750in}{0.563663in}}%
\pgfpathlineto{\pgfqpoint{3.330994in}{0.588517in}}%
\pgfpathlineto{\pgfqpoint{3.267288in}{0.613498in}}%
\pgfpathlineto{\pgfqpoint{3.203527in}{0.638337in}}%
\pgfpathlineto{\pgfqpoint{3.139606in}{0.662759in}}%
\pgfpathlineto{\pgfqpoint{3.075424in}{0.686485in}}%
\pgfpathlineto{\pgfqpoint{3.010889in}{0.709231in}}%
\pgfpathlineto{\pgfqpoint{2.945924in}{0.730715in}}%
\pgfpathlineto{\pgfqpoint{2.880475in}{0.750664in}}%
\pgfpathlineto{\pgfqpoint{2.814510in}{0.768828in}}%
\pgfpathlineto{\pgfqpoint{2.748028in}{0.784991in}}%
\pgfpathlineto{\pgfqpoint{2.681058in}{0.798988in}}%
\pgfpathlineto{\pgfqpoint{2.613655in}{0.810724in}}%
\pgfpathlineto{\pgfqpoint{2.545894in}{0.820190in}}%
\pgfpathlineto{\pgfqpoint{2.477863in}{0.827483in}}%
\pgfpathlineto{\pgfqpoint{2.409649in}{0.832811in}}%
\pgfpathlineto{\pgfqpoint{2.341324in}{0.836484in}}%
\pgfpathlineto{\pgfqpoint{2.272941in}{0.838921in}}%
\pgfpathlineto{\pgfqpoint{2.204534in}{0.840643in}}%
\pgfpathlineto{\pgfqpoint{2.136125in}{0.842298in}}%
\pgfpathlineto{\pgfqpoint{2.067740in}{0.844671in}}%
\pgfpathlineto{\pgfqpoint{1.999441in}{0.848707in}}%
\pgfpathlineto{\pgfqpoint{1.931388in}{0.855598in}}%
\pgfpathlineto{\pgfqpoint{1.863960in}{0.866918in}}%
\pgfpathlineto{\pgfqpoint{1.798047in}{0.884821in}}%
\pgfpathlineto{\pgfqpoint{1.735655in}{0.912197in}}%
\pgfpathlineto{\pgfqpoint{1.680894in}{0.952199in}}%
\pgfpathlineto{\pgfqpoint{1.644234in}{0.997710in}}%
\pgfpathlineto{\pgfqpoint{1.622486in}{1.045084in}}%
\pgfpathlineto{\pgfqpoint{1.611257in}{1.095477in}}%
\pgfpathlineto{\pgfqpoint{1.608762in}{1.151401in}}%
\pgfpathlineto{\pgfqpoint{1.614999in}{1.214236in}}%
\pgfpathlineto{\pgfqpoint{1.629411in}{1.280831in}}%
\pgfpathlineto{\pgfqpoint{1.649568in}{1.345970in}}%
\pgfpathlineto{\pgfqpoint{1.674178in}{1.409631in}}%
\pgfusepath{stroke}%
\end{pgfscope}%
\begin{pgfscope}%
\pgfpathrectangle{\pgfqpoint{0.647939in}{0.492442in}}{\pgfqpoint{3.079299in}{3.079299in}}%
\pgfusepath{clip}%
\pgfsetbuttcap%
\pgfsetroundjoin%
\pgfsetlinewidth{0.301125pt}%
\definecolor{currentstroke}{rgb}{0.500000,0.500000,0.500000}%
\pgfsetstrokecolor{currentstroke}%
\pgfsetstrokeopacity{0.300000}%
\pgfsetdash{}{0pt}%
\pgfpathmoveto{\pgfqpoint{3.727238in}{0.562426in}}%
\pgfpathlineto{\pgfqpoint{3.727238in}{0.562426in}}%
\pgfpathlineto{\pgfqpoint{3.662203in}{0.583693in}}%
\pgfpathlineto{\pgfqpoint{3.597657in}{0.606404in}}%
\pgfpathlineto{\pgfqpoint{3.533555in}{0.630340in}}%
\pgfpathlineto{\pgfqpoint{3.469829in}{0.655267in}}%
\pgfpathlineto{\pgfqpoint{3.406399in}{0.680937in}}%
\pgfpathlineto{\pgfqpoint{3.343169in}{0.707098in}}%
\pgfpathlineto{\pgfqpoint{3.280033in}{0.733488in}}%
\pgfpathlineto{\pgfqpoint{3.216881in}{0.759836in}}%
\pgfpathlineto{\pgfqpoint{3.153599in}{0.785868in}}%
\pgfpathlineto{\pgfqpoint{3.090073in}{0.811300in}}%
\pgfpathlineto{\pgfqpoint{3.026201in}{0.835842in}}%
\pgfpathlineto{\pgfqpoint{2.961887in}{0.859201in}}%
\pgfpathlineto{\pgfqpoint{2.897059in}{0.881085in}}%
\pgfpathlineto{\pgfqpoint{2.831667in}{0.901213in}}%
\pgfpathlineto{\pgfqpoint{2.765691in}{0.919331in}}%
\pgfpathlineto{\pgfqpoint{2.699145in}{0.935226in}}%
\pgfpathlineto{\pgfqpoint{2.632078in}{0.948753in}}%
\pgfpathlineto{\pgfqpoint{2.564569in}{0.959853in}}%
\pgfpathlineto{\pgfqpoint{2.496710in}{0.968572in}}%
\pgfpathlineto{\pgfqpoint{2.428601in}{0.975075in}}%
\pgfpathlineto{\pgfqpoint{2.360331in}{0.979646in}}%
\pgfpathlineto{\pgfqpoint{2.291974in}{0.982709in}}%
\pgfpathlineto{\pgfqpoint{2.223578in}{0.984819in}}%
\pgfpathlineto{\pgfqpoint{2.155174in}{0.986661in}}%
\pgfpathlineto{\pgfqpoint{2.086790in}{0.989074in}}%
\pgfpathlineto{\pgfqpoint{2.018491in}{0.993117in}}%
\pgfpathlineto{\pgfqpoint{1.950458in}{1.000185in}}%
\pgfpathlineto{\pgfqpoint{1.883175in}{1.012204in}}%
\pgfpathlineto{\pgfqpoint{1.817864in}{1.031910in}}%
\pgfpathlineto{\pgfqpoint{1.757506in}{1.063023in}}%
\pgfpathlineto{\pgfqpoint{1.757506in}{1.063023in}}%
\pgfpathlineto{\pgfqpoint{1.717707in}{1.096961in}}%
\pgfpathlineto{\pgfqpoint{1.687644in}{1.140437in}}%
\pgfpathlineto{\pgfqpoint{1.670866in}{1.185986in}}%
\pgfpathlineto{\pgfqpoint{1.663807in}{1.235131in}}%
\pgfpathlineto{\pgfqpoint{1.665414in}{1.289651in}}%
\pgfpathlineto{\pgfqpoint{1.676096in}{1.351698in}}%
\pgfusepath{stroke}%
\end{pgfscope}%
\begin{pgfscope}%
\pgfpathrectangle{\pgfqpoint{0.647939in}{0.492442in}}{\pgfqpoint{3.079299in}{3.079299in}}%
\pgfusepath{clip}%
\pgfsetbuttcap%
\pgfsetroundjoin%
\pgfsetlinewidth{0.301125pt}%
\definecolor{currentstroke}{rgb}{0.500000,0.500000,0.500000}%
\pgfsetstrokecolor{currentstroke}%
\pgfsetstrokeopacity{0.300000}%
\pgfsetdash{}{0pt}%
\pgfpathmoveto{\pgfqpoint{3.727238in}{0.632410in}}%
\pgfpathlineto{\pgfqpoint{3.727238in}{0.632410in}}%
\pgfpathlineto{\pgfqpoint{3.662424in}{0.654339in}}%
\pgfpathlineto{\pgfqpoint{3.598133in}{0.677760in}}%
\pgfpathlineto{\pgfqpoint{3.534318in}{0.702453in}}%
\pgfpathlineto{\pgfqpoint{3.470912in}{0.728181in}}%
\pgfpathlineto{\pgfqpoint{3.407831in}{0.754698in}}%
\pgfpathlineto{\pgfqpoint{3.344977in}{0.781750in}}%
\pgfpathlineto{\pgfqpoint{3.282241in}{0.809075in}}%
\pgfpathlineto{\pgfqpoint{3.219507in}{0.836405in}}%
\pgfpathlineto{\pgfqpoint{3.156657in}{0.863464in}}%
\pgfpathlineto{\pgfqpoint{3.093571in}{0.889968in}}%
\pgfpathlineto{\pgfqpoint{3.030137in}{0.915623in}}%
\pgfpathlineto{\pgfqpoint{2.966252in}{0.940129in}}%
\pgfpathlineto{\pgfqpoint{2.901832in}{0.963186in}}%
\pgfpathlineto{\pgfqpoint{2.836817in}{0.984498in}}%
\pgfpathlineto{\pgfqpoint{2.771175in}{1.003789in}}%
\pgfpathlineto{\pgfqpoint{2.704913in}{1.020820in}}%
\pgfpathlineto{\pgfqpoint{2.638071in}{1.035413in}}%
\pgfpathlineto{\pgfqpoint{2.570727in}{1.047479in}}%
\pgfpathlineto{\pgfqpoint{2.502983in}{1.057039in}}%
\pgfpathlineto{\pgfqpoint{2.434945in}{1.064238in}}%
\pgfpathlineto{\pgfqpoint{2.366716in}{1.069354in}}%
\pgfpathlineto{\pgfqpoint{2.298378in}{1.072806in}}%
\pgfpathlineto{\pgfqpoint{2.229991in}{1.075167in}}%
\pgfpathlineto{\pgfqpoint{2.161592in}{1.077173in}}%
\pgfpathlineto{\pgfqpoint{2.093214in}{1.079739in}}%
\pgfpathlineto{\pgfqpoint{2.024932in}{1.084036in}}%
\pgfpathlineto{\pgfqpoint{1.956964in}{1.091644in}}%
\pgfpathlineto{\pgfqpoint{1.889920in}{1.104838in}}%
\pgfpathlineto{\pgfqpoint{1.825477in}{1.127001in}}%
\pgfpathlineto{\pgfqpoint{1.825477in}{1.127001in}}%
\pgfpathlineto{\pgfqpoint{1.776260in}{1.155666in}}%
\pgfpathlineto{\pgfqpoint{1.776260in}{1.155666in}}%
\pgfpathlineto{\pgfqpoint{1.740962in}{1.189604in}}%
\pgfusepath{stroke}%
\end{pgfscope}%
\begin{pgfscope}%
\pgfpathrectangle{\pgfqpoint{0.647939in}{0.492442in}}{\pgfqpoint{3.079299in}{3.079299in}}%
\pgfusepath{clip}%
\pgfsetbuttcap%
\pgfsetroundjoin%
\pgfsetlinewidth{0.301125pt}%
\definecolor{currentstroke}{rgb}{0.500000,0.500000,0.500000}%
\pgfsetstrokecolor{currentstroke}%
\pgfsetstrokeopacity{0.300000}%
\pgfsetdash{}{0pt}%
\pgfpathmoveto{\pgfqpoint{3.727238in}{0.702394in}}%
\pgfpathlineto{\pgfqpoint{3.727238in}{0.702394in}}%
\pgfpathlineto{\pgfqpoint{3.662666in}{0.725025in}}%
\pgfpathlineto{\pgfqpoint{3.598655in}{0.749200in}}%
\pgfpathlineto{\pgfqpoint{3.535157in}{0.774696in}}%
\pgfpathlineto{\pgfqpoint{3.472104in}{0.801276in}}%
\pgfpathlineto{\pgfqpoint{3.409409in}{0.828693in}}%
\pgfpathlineto{\pgfqpoint{3.346973in}{0.856695in}}%
\pgfpathlineto{\pgfqpoint{3.284683in}{0.885022in}}%
\pgfpathlineto{\pgfqpoint{3.222419in}{0.913407in}}%
\pgfpathlineto{\pgfqpoint{3.160058in}{0.941575in}}%
\pgfpathlineto{\pgfqpoint{3.097474in}{0.969243in}}%
\pgfpathlineto{\pgfqpoint{3.034545in}{0.996115in}}%
\pgfpathlineto{\pgfqpoint{2.971161in}{1.021888in}}%
\pgfpathlineto{\pgfqpoint{2.907223in}{1.046252in}}%
\pgfpathlineto{\pgfqpoint{2.842660in}{1.068896in}}%
\pgfpathlineto{\pgfqpoint{2.777427in}{1.089523in}}%
\pgfpathlineto{\pgfqpoint{2.711517in}{1.107867in}}%
\pgfpathlineto{\pgfqpoint{2.644964in}{1.123715in}}%
\pgfpathlineto{\pgfqpoint{2.577840in}{1.136935in}}%
\pgfpathlineto{\pgfqpoint{2.510247in}{1.147510in}}%
\pgfpathlineto{\pgfqpoint{2.442305in}{1.155560in}}%
\pgfpathlineto{\pgfqpoint{2.374132in}{1.161353in}}%
\pgfpathlineto{\pgfqpoint{2.305823in}{1.165307in}}%
\pgfpathlineto{\pgfqpoint{2.237449in}{1.168005in}}%
\pgfpathlineto{\pgfqpoint{2.169056in}{1.170220in}}%
\pgfpathlineto{\pgfqpoint{2.100685in}{1.172961in}}%
\pgfpathlineto{\pgfqpoint{2.032425in}{1.177551in}}%
\pgfpathlineto{\pgfqpoint{1.964550in}{1.185827in}}%
\pgfpathlineto{\pgfqpoint{1.897871in}{1.200553in}}%
\pgfpathlineto{\pgfqpoint{1.834806in}{1.226028in}}%
\pgfpathlineto{\pgfqpoint{1.834806in}{1.226028in}}%
\pgfpathlineto{\pgfqpoint{1.793522in}{1.254998in}}%
\pgfpathlineto{\pgfqpoint{1.761437in}{1.295129in}}%
\pgfpathlineto{\pgfqpoint{1.744551in}{1.336747in}}%
\pgfpathlineto{\pgfqpoint{1.738050in}{1.380746in}}%
\pgfpathlineto{\pgfqpoint{1.740266in}{1.429704in}}%
\pgfpathlineto{\pgfqpoint{1.751538in}{1.484791in}}%
\pgfpathlineto{\pgfqpoint{1.773036in}{1.547266in}}%
\pgfusepath{stroke}%
\end{pgfscope}%
\begin{pgfscope}%
\pgfpathrectangle{\pgfqpoint{0.647939in}{0.492442in}}{\pgfqpoint{3.079299in}{3.079299in}}%
\pgfusepath{clip}%
\pgfsetbuttcap%
\pgfsetroundjoin%
\pgfsetlinewidth{0.301125pt}%
\definecolor{currentstroke}{rgb}{0.500000,0.500000,0.500000}%
\pgfsetstrokecolor{currentstroke}%
\pgfsetstrokeopacity{0.300000}%
\pgfsetdash{}{0pt}%
\pgfpathmoveto{\pgfqpoint{3.727238in}{0.772378in}}%
\pgfpathlineto{\pgfqpoint{3.727238in}{0.772378in}}%
\pgfpathlineto{\pgfqpoint{3.662932in}{0.795754in}}%
\pgfpathlineto{\pgfqpoint{3.599230in}{0.820729in}}%
\pgfpathlineto{\pgfqpoint{3.536082in}{0.847079in}}%
\pgfpathlineto{\pgfqpoint{3.473419in}{0.874565in}}%
\pgfpathlineto{\pgfqpoint{3.411153in}{0.902943in}}%
\pgfpathlineto{\pgfqpoint{3.349182in}{0.931961in}}%
\pgfpathlineto{\pgfqpoint{3.287392in}{0.961362in}}%
\pgfpathlineto{\pgfqpoint{3.225659in}{0.990882in}}%
\pgfpathlineto{\pgfqpoint{3.163853in}{1.020251in}}%
\pgfpathlineto{\pgfqpoint{3.101844in}{1.049187in}}%
\pgfpathlineto{\pgfqpoint{3.039503in}{1.077396in}}%
\pgfpathlineto{\pgfqpoint{2.976706in}{1.104572in}}%
\pgfpathlineto{\pgfqpoint{2.913344in}{1.130397in}}%
\pgfpathlineto{\pgfqpoint{2.849329in}{1.154550in}}%
\pgfpathlineto{\pgfqpoint{2.784601in}{1.176711in}}%
\pgfpathlineto{\pgfqpoint{2.719137in}{1.196585in}}%
\pgfpathlineto{\pgfqpoint{2.652957in}{1.213919in}}%
\pgfpathlineto{\pgfqpoint{2.586127in}{1.228538in}}%
\pgfpathlineto{\pgfqpoint{2.518746in}{1.240372in}}%
\pgfpathlineto{\pgfqpoint{2.450942in}{1.249492in}}%
\pgfpathlineto{\pgfqpoint{2.382847in}{1.256140in}}%
\pgfpathlineto{\pgfqpoint{2.314580in}{1.260740in}}%
\pgfpathlineto{\pgfqpoint{2.246226in}{1.263892in}}%
\pgfpathlineto{\pgfqpoint{2.177844in}{1.266410in}}%
\pgfpathlineto{\pgfqpoint{2.109482in}{1.269378in}}%
\pgfpathlineto{\pgfqpoint{2.041248in}{1.274307in}}%
\pgfpathlineto{\pgfqpoint{1.973497in}{1.283417in}}%
\pgfpathlineto{\pgfqpoint{1.907401in}{1.300212in}}%
\pgfpathlineto{\pgfqpoint{1.907401in}{1.300212in}}%
\pgfpathlineto{\pgfqpoint{1.855523in}{1.324250in}}%
\pgfpathlineto{\pgfqpoint{1.855523in}{1.324250in}}%
\pgfpathlineto{\pgfqpoint{1.819873in}{1.353002in}}%
\pgfusepath{stroke}%
\end{pgfscope}%
\begin{pgfscope}%
\pgfpathrectangle{\pgfqpoint{0.647939in}{0.492442in}}{\pgfqpoint{3.079299in}{3.079299in}}%
\pgfusepath{clip}%
\pgfsetbuttcap%
\pgfsetroundjoin%
\pgfsetlinewidth{0.301125pt}%
\definecolor{currentstroke}{rgb}{0.500000,0.500000,0.500000}%
\pgfsetstrokecolor{currentstroke}%
\pgfsetstrokeopacity{0.300000}%
\pgfsetdash{}{0pt}%
\pgfpathmoveto{\pgfqpoint{3.727238in}{0.842362in}}%
\pgfpathlineto{\pgfqpoint{3.727238in}{0.842362in}}%
\pgfpathlineto{\pgfqpoint{3.663227in}{0.866531in}}%
\pgfpathlineto{\pgfqpoint{3.599865in}{0.892358in}}%
\pgfpathlineto{\pgfqpoint{3.537104in}{0.919616in}}%
\pgfpathlineto{\pgfqpoint{3.474874in}{0.948068in}}%
\pgfpathlineto{\pgfqpoint{3.413086in}{0.977471in}}%
\pgfpathlineto{\pgfqpoint{3.351637in}{1.007576in}}%
\pgfpathlineto{\pgfqpoint{3.290409in}{1.038130in}}%
\pgfpathlineto{\pgfqpoint{3.229277in}{1.068876in}}%
\pgfpathlineto{\pgfqpoint{3.168107in}{1.099547in}}%
\pgfpathlineto{\pgfqpoint{3.106763in}{1.129865in}}%
\pgfpathlineto{\pgfqpoint{3.045106in}{1.159541in}}%
\pgfpathlineto{\pgfqpoint{2.983005in}{1.188273in}}%
\pgfpathlineto{\pgfqpoint{2.920337in}{1.215740in}}%
\pgfpathlineto{\pgfqpoint{2.856995in}{1.241610in}}%
\pgfpathlineto{\pgfqpoint{2.792899in}{1.265546in}}%
\pgfpathlineto{\pgfqpoint{2.728006in}{1.287218in}}%
\pgfpathlineto{\pgfqpoint{2.662319in}{1.306334in}}%
\pgfpathlineto{\pgfqpoint{2.595887in}{1.322664in}}%
\pgfpathlineto{\pgfqpoint{2.528807in}{1.336080in}}%
\pgfpathlineto{\pgfqpoint{2.461209in}{1.346593in}}%
\pgfpathlineto{\pgfqpoint{2.393240in}{1.354387in}}%
\pgfpathlineto{\pgfqpoint{2.325038in}{1.359851in}}%
\pgfpathlineto{\pgfqpoint{2.256716in}{1.363612in}}%
\pgfpathlineto{\pgfqpoint{2.188350in}{1.366547in}}%
\pgfpathlineto{\pgfqpoint{2.120004in}{1.369855in}}%
\pgfpathlineto{\pgfqpoint{2.051809in}{1.375261in}}%
\pgfpathlineto{\pgfqpoint{1.984234in}{1.385486in}}%
\pgfpathlineto{\pgfqpoint{1.919112in}{1.405281in}}%
\pgfpathlineto{\pgfqpoint{1.919112in}{1.405281in}}%
\pgfpathlineto{\pgfqpoint{1.877415in}{1.428976in}}%
\pgfpathlineto{\pgfqpoint{1.877415in}{1.428976in}}%
\pgfpathlineto{\pgfqpoint{1.849044in}{1.457823in}}%
\pgfpathlineto{\pgfqpoint{1.831105in}{1.494634in}}%
\pgfpathlineto{\pgfqpoint{1.824890in}{1.532414in}}%
\pgfpathlineto{\pgfqpoint{1.827461in}{1.573306in}}%
\pgfpathlineto{\pgfqpoint{1.838937in}{1.619762in}}%
\pgfusepath{stroke}%
\end{pgfscope}%
\begin{pgfscope}%
\pgfpathrectangle{\pgfqpoint{0.647939in}{0.492442in}}{\pgfqpoint{3.079299in}{3.079299in}}%
\pgfusepath{clip}%
\pgfsetbuttcap%
\pgfsetroundjoin%
\pgfsetlinewidth{0.301125pt}%
\definecolor{currentstroke}{rgb}{0.500000,0.500000,0.500000}%
\pgfsetstrokecolor{currentstroke}%
\pgfsetstrokeopacity{0.300000}%
\pgfsetdash{}{0pt}%
\pgfpathmoveto{\pgfqpoint{3.727238in}{0.912347in}}%
\pgfpathlineto{\pgfqpoint{3.727238in}{0.912347in}}%
\pgfpathlineto{\pgfqpoint{3.663553in}{0.937360in}}%
\pgfpathlineto{\pgfqpoint{3.600568in}{0.964093in}}%
\pgfpathlineto{\pgfqpoint{3.538237in}{0.992319in}}%
\pgfpathlineto{\pgfqpoint{3.476489in}{1.021803in}}%
\pgfpathlineto{\pgfqpoint{3.415235in}{1.052302in}}%
\pgfpathlineto{\pgfqpoint{3.354370in}{1.083573in}}%
\pgfpathlineto{\pgfqpoint{3.293776in}{1.115366in}}%
\pgfpathlineto{\pgfqpoint{3.233325in}{1.147430in}}%
\pgfpathlineto{\pgfqpoint{3.172881in}{1.179508in}}%
\pgfpathlineto{\pgfqpoint{3.112303in}{1.211331in}}%
\pgfpathlineto{\pgfqpoint{3.051450in}{1.242621in}}%
\pgfpathlineto{\pgfqpoint{2.990178in}{1.273080in}}%
\pgfpathlineto{\pgfqpoint{2.928352in}{1.302393in}}%
\pgfpathlineto{\pgfqpoint{2.865847in}{1.330225in}}%
\pgfpathlineto{\pgfqpoint{2.802562in}{1.356225in}}%
\pgfpathlineto{\pgfqpoint{2.738426in}{1.380037in}}%
\pgfpathlineto{\pgfqpoint{2.673411in}{1.401323in}}%
\pgfpathlineto{\pgfqpoint{2.607538in}{1.419785in}}%
\pgfpathlineto{\pgfqpoint{2.540892in}{1.435220in}}%
\pgfpathlineto{\pgfqpoint{2.473607in}{1.447562in}}%
\pgfpathlineto{\pgfqpoint{2.405840in}{1.456926in}}%
\pgfpathlineto{\pgfqpoint{2.337755in}{1.463647in}}%
\pgfpathlineto{\pgfqpoint{2.269492in}{1.468317in}}%
\pgfpathlineto{\pgfqpoint{2.201155in}{1.471851in}}%
\pgfpathlineto{\pgfqpoint{2.132833in}{1.475615in}}%
\pgfpathlineto{\pgfqpoint{2.064705in}{1.481693in}}%
\pgfpathlineto{\pgfqpoint{1.997482in}{1.493641in}}%
\pgfpathlineto{\pgfqpoint{1.997482in}{1.493641in}}%
\pgfpathlineto{\pgfqpoint{1.943610in}{1.513043in}}%
\pgfpathlineto{\pgfqpoint{1.943610in}{1.513043in}}%
\pgfpathlineto{\pgfqpoint{1.909579in}{1.536269in}}%
\pgfpathlineto{\pgfqpoint{1.909579in}{1.536269in}}%
\pgfpathlineto{\pgfqpoint{1.887998in}{1.564366in}}%
\pgfpathlineto{\pgfqpoint{1.877062in}{1.598376in}}%
\pgfpathlineto{\pgfqpoint{1.876192in}{1.633207in}}%
\pgfpathlineto{\pgfqpoint{1.883860in}{1.672259in}}%
\pgfpathlineto{\pgfqpoint{1.900885in}{1.716546in}}%
\pgfusepath{stroke}%
\end{pgfscope}%
\begin{pgfscope}%
\pgfpathrectangle{\pgfqpoint{0.647939in}{0.492442in}}{\pgfqpoint{3.079299in}{3.079299in}}%
\pgfusepath{clip}%
\pgfsetbuttcap%
\pgfsetroundjoin%
\pgfsetlinewidth{0.301125pt}%
\definecolor{currentstroke}{rgb}{0.500000,0.500000,0.500000}%
\pgfsetstrokecolor{currentstroke}%
\pgfsetstrokeopacity{0.300000}%
\pgfsetdash{}{0pt}%
\pgfpathmoveto{\pgfqpoint{3.727238in}{0.982331in}}%
\pgfpathlineto{\pgfqpoint{3.727238in}{0.982331in}}%
\pgfpathlineto{\pgfqpoint{3.663915in}{1.008244in}}%
\pgfpathlineto{\pgfqpoint{3.601350in}{1.035945in}}%
\pgfpathlineto{\pgfqpoint{3.539499in}{1.065205in}}%
\pgfpathlineto{\pgfqpoint{3.478289in}{1.095790in}}%
\pgfpathlineto{\pgfqpoint{3.417634in}{1.127461in}}%
\pgfpathlineto{\pgfqpoint{3.357427in}{1.159978in}}%
\pgfpathlineto{\pgfqpoint{3.297550in}{1.193101in}}%
\pgfpathlineto{\pgfqpoint{3.237877in}{1.226589in}}%
\pgfpathlineto{\pgfqpoint{3.178271in}{1.260197in}}%
\pgfpathlineto{\pgfqpoint{3.118590in}{1.293671in}}%
\pgfpathlineto{\pgfqpoint{3.058688in}{1.326746in}}%
\pgfpathlineto{\pgfqpoint{2.998416in}{1.359138in}}%
\pgfpathlineto{\pgfqpoint{2.937625in}{1.390543in}}%
\pgfpathlineto{\pgfqpoint{2.876176in}{1.420634in}}%
\pgfpathlineto{\pgfqpoint{2.813941in}{1.449056in}}%
\pgfpathlineto{\pgfqpoint{2.750820in}{1.475440in}}%
\pgfpathlineto{\pgfqpoint{2.686746in}{1.499408in}}%
\pgfpathlineto{\pgfqpoint{2.621707in}{1.520603in}}%
\pgfpathlineto{\pgfqpoint{2.555748in}{1.538727in}}%
\pgfpathlineto{\pgfqpoint{2.488979in}{1.553593in}}%
\pgfpathlineto{\pgfqpoint{2.421561in}{1.565186in}}%
\pgfpathlineto{\pgfqpoint{2.353686in}{1.573752in}}%
\pgfpathlineto{\pgfqpoint{2.285539in}{1.579853in}}%
\pgfpathlineto{\pgfqpoint{2.217266in}{1.584432in}}%
\pgfpathlineto{\pgfqpoint{2.148991in}{1.588997in}}%
\pgfpathlineto{\pgfqpoint{2.080970in}{1.596119in}}%
\pgfpathlineto{\pgfqpoint{2.014468in}{1.611013in}}%
\pgfpathlineto{\pgfqpoint{2.014468in}{1.611013in}}%
\pgfpathlineto{\pgfqpoint{1.975365in}{1.629108in}}%
\pgfpathlineto{\pgfqpoint{1.975365in}{1.629108in}}%
\pgfpathlineto{\pgfqpoint{1.950471in}{1.651761in}}%
\pgfusepath{stroke}%
\end{pgfscope}%
\begin{pgfscope}%
\pgfpathrectangle{\pgfqpoint{0.647939in}{0.492442in}}{\pgfqpoint{3.079299in}{3.079299in}}%
\pgfusepath{clip}%
\pgfsetbuttcap%
\pgfsetroundjoin%
\pgfsetlinewidth{0.301125pt}%
\definecolor{currentstroke}{rgb}{0.500000,0.500000,0.500000}%
\pgfsetstrokecolor{currentstroke}%
\pgfsetstrokeopacity{0.300000}%
\pgfsetdash{}{0pt}%
\pgfpathmoveto{\pgfqpoint{3.727238in}{1.052315in}}%
\pgfpathlineto{\pgfqpoint{3.727238in}{1.052315in}}%
\pgfpathlineto{\pgfqpoint{3.664318in}{1.079191in}}%
\pgfpathlineto{\pgfqpoint{3.602222in}{1.107925in}}%
\pgfpathlineto{\pgfqpoint{3.540905in}{1.138290in}}%
\pgfpathlineto{\pgfqpoint{3.480299in}{1.170051in}}%
\pgfpathlineto{\pgfqpoint{3.420315in}{1.202976in}}%
\pgfpathlineto{\pgfqpoint{3.360850in}{1.236831in}}%
\pgfpathlineto{\pgfqpoint{3.301790in}{1.271389in}}%
\pgfpathlineto{\pgfqpoint{3.243009in}{1.306422in}}%
\pgfpathlineto{\pgfqpoint{3.184376in}{1.341699in}}%
\pgfpathlineto{\pgfqpoint{3.125750in}{1.376987in}}%
\pgfpathlineto{\pgfqpoint{3.066984in}{1.412041in}}%
\pgfpathlineto{\pgfqpoint{3.007927in}{1.446600in}}%
\pgfpathlineto{\pgfqpoint{2.948423in}{1.480384in}}%
\pgfpathlineto{\pgfqpoint{2.888320in}{1.513086in}}%
\pgfpathlineto{\pgfqpoint{2.827468in}{1.544369in}}%
\pgfpathlineto{\pgfqpoint{2.765736in}{1.573867in}}%
\pgfpathlineto{\pgfqpoint{2.703013in}{1.601186in}}%
\pgfpathlineto{\pgfqpoint{2.639231in}{1.625922in}}%
\pgfpathlineto{\pgfqpoint{2.574378in}{1.647686in}}%
\pgfpathlineto{\pgfqpoint{2.508518in}{1.666164in}}%
\pgfpathlineto{\pgfqpoint{2.441785in}{1.681181in}}%
\pgfpathlineto{\pgfqpoint{2.374375in}{1.692793in}}%
\pgfpathlineto{\pgfqpoint{2.306506in}{1.701395in}}%
\pgfpathlineto{\pgfqpoint{2.238390in}{1.707879in}}%
\pgfpathlineto{\pgfqpoint{2.170235in}{1.713971in}}%
\pgfpathlineto{\pgfqpoint{2.102526in}{1.723291in}}%
\pgfpathlineto{\pgfqpoint{2.102526in}{1.723291in}}%
\pgfpathlineto{\pgfqpoint{2.051230in}{1.738312in}}%
\pgfpathlineto{\pgfqpoint{2.051230in}{1.738312in}}%
\pgfpathlineto{\pgfqpoint{2.024374in}{1.754957in}}%
\pgfpathlineto{\pgfqpoint{2.024374in}{1.754957in}}%
\pgfpathlineto{\pgfqpoint{2.009570in}{1.775773in}}%
\pgfpathlineto{\pgfqpoint{2.005551in}{1.801206in}}%
\pgfpathlineto{\pgfqpoint{2.010268in}{1.826319in}}%
\pgfpathlineto{\pgfqpoint{2.023038in}{1.854854in}}%
\pgfusepath{stroke}%
\end{pgfscope}%
\begin{pgfscope}%
\pgfpathrectangle{\pgfqpoint{0.647939in}{0.492442in}}{\pgfqpoint{3.079299in}{3.079299in}}%
\pgfusepath{clip}%
\pgfsetbuttcap%
\pgfsetroundjoin%
\pgfsetlinewidth{0.301125pt}%
\definecolor{currentstroke}{rgb}{0.500000,0.500000,0.500000}%
\pgfsetstrokecolor{currentstroke}%
\pgfsetstrokeopacity{0.300000}%
\pgfsetdash{}{0pt}%
\pgfpathmoveto{\pgfqpoint{3.727238in}{1.122299in}}%
\pgfpathlineto{\pgfqpoint{3.727238in}{1.122299in}}%
\pgfpathlineto{\pgfqpoint{3.664769in}{1.150205in}}%
\pgfpathlineto{\pgfqpoint{3.603197in}{1.180044in}}%
\pgfpathlineto{\pgfqpoint{3.542480in}{1.211589in}}%
\pgfpathlineto{\pgfqpoint{3.482550in}{1.244610in}}%
\pgfpathlineto{\pgfqpoint{3.423325in}{1.278878in}}%
\pgfpathlineto{\pgfqpoint{3.364705in}{1.314174in}}%
\pgfpathlineto{\pgfqpoint{3.306580in}{1.350281in}}%
\pgfpathlineto{\pgfqpoint{3.248833in}{1.386989in}}%
\pgfpathlineto{\pgfqpoint{3.191335in}{1.424090in}}%
\pgfpathlineto{\pgfqpoint{3.133956in}{1.461373in}}%
\pgfpathlineto{\pgfqpoint{3.076555in}{1.498621in}}%
\pgfpathlineto{\pgfqpoint{3.018985in}{1.535605in}}%
\pgfpathlineto{\pgfqpoint{2.961094in}{1.572083in}}%
\pgfpathlineto{\pgfqpoint{2.902731in}{1.607796in}}%
\pgfpathlineto{\pgfqpoint{2.843738in}{1.642456in}}%
\pgfpathlineto{\pgfqpoint{2.783963in}{1.675744in}}%
\pgfpathlineto{\pgfqpoint{2.723261in}{1.707302in}}%
\pgfpathlineto{\pgfqpoint{2.661506in}{1.736739in}}%
\pgfpathlineto{\pgfqpoint{2.598610in}{1.763639in}}%
\pgfpathlineto{\pgfqpoint{2.534539in}{1.787588in}}%
\pgfpathlineto{\pgfqpoint{2.469333in}{1.808231in}}%
\pgfpathlineto{\pgfqpoint{2.403118in}{1.825366in}}%
\pgfpathlineto{\pgfqpoint{2.336108in}{1.839097in}}%
\pgfpathlineto{\pgfqpoint{2.268588in}{1.850133in}}%
\pgfpathlineto{\pgfqpoint{2.200971in}{1.860608in}}%
\pgfpathlineto{\pgfqpoint{2.135434in}{1.878128in}}%
\pgfpathlineto{\pgfqpoint{2.135434in}{1.878128in}}%
\pgfpathlineto{\pgfqpoint{2.114171in}{1.890750in}}%
\pgfpathlineto{\pgfqpoint{2.114171in}{1.890750in}}%
\pgfpathlineto{\pgfqpoint{2.104450in}{1.906258in}}%
\pgfpathlineto{\pgfqpoint{2.104594in}{1.924607in}}%
\pgfusepath{stroke}%
\end{pgfscope}%
\begin{pgfscope}%
\pgfpathrectangle{\pgfqpoint{0.647939in}{0.492442in}}{\pgfqpoint{3.079299in}{3.079299in}}%
\pgfusepath{clip}%
\pgfsetbuttcap%
\pgfsetroundjoin%
\pgfsetlinewidth{0.301125pt}%
\definecolor{currentstroke}{rgb}{0.500000,0.500000,0.500000}%
\pgfsetstrokecolor{currentstroke}%
\pgfsetstrokeopacity{0.300000}%
\pgfsetdash{}{0pt}%
\pgfpathmoveto{\pgfqpoint{3.727238in}{1.262267in}}%
\pgfpathlineto{\pgfqpoint{3.727238in}{1.262267in}}%
\pgfpathlineto{\pgfqpoint{3.665844in}{1.292460in}}%
\pgfpathlineto{\pgfqpoint{3.605524in}{1.324750in}}%
\pgfpathlineto{\pgfqpoint{3.546243in}{1.358914in}}%
\pgfpathlineto{\pgfqpoint{3.487945in}{1.394733in}}%
\pgfpathlineto{\pgfqpoint{3.430561in}{1.431999in}}%
\pgfpathlineto{\pgfqpoint{3.374009in}{1.470519in}}%
\pgfpathlineto{\pgfqpoint{3.318203in}{1.510113in}}%
\pgfpathlineto{\pgfqpoint{3.263051in}{1.550614in}}%
\pgfpathlineto{\pgfqpoint{3.208455in}{1.591864in}}%
\pgfpathlineto{\pgfqpoint{3.154323in}{1.633718in}}%
\pgfpathlineto{\pgfqpoint{3.100565in}{1.676054in}}%
\pgfpathlineto{\pgfqpoint{3.047098in}{1.718756in}}%
\pgfpathlineto{\pgfqpoint{2.993841in}{1.761716in}}%
\pgfpathlineto{\pgfqpoint{2.940726in}{1.804852in}}%
\pgfpathlineto{\pgfqpoint{2.887704in}{1.848102in}}%
\pgfpathlineto{\pgfqpoint{2.834752in}{1.891437in}}%
\pgfpathlineto{\pgfqpoint{2.781894in}{1.934882in}}%
\pgfpathlineto{\pgfqpoint{2.729229in}{1.978555in}}%
\pgfpathlineto{\pgfqpoint{2.677006in}{2.022752in}}%
\pgfpathlineto{\pgfqpoint{2.625796in}{2.068095in}}%
\pgfpathlineto{\pgfqpoint{2.576893in}{2.115867in}}%
\pgfpathlineto{\pgfqpoint{2.533804in}{2.168694in}}%
\pgfpathlineto{\pgfqpoint{2.533804in}{2.168694in}}%
\pgfpathlineto{\pgfqpoint{2.510689in}{2.213337in}}%
\pgfpathlineto{\pgfqpoint{2.510689in}{2.213337in}}%
\pgfpathlineto{\pgfqpoint{2.502234in}{2.254559in}}%
\pgfpathlineto{\pgfqpoint{2.505823in}{2.297108in}}%
\pgfpathlineto{\pgfqpoint{2.518769in}{2.338640in}}%
\pgfpathlineto{\pgfqpoint{2.541644in}{2.386258in}}%
\pgfpathlineto{\pgfqpoint{2.576850in}{2.444713in}}%
\pgfusepath{stroke}%
\end{pgfscope}%
\begin{pgfscope}%
\pgfpathrectangle{\pgfqpoint{0.647939in}{0.492442in}}{\pgfqpoint{3.079299in}{3.079299in}}%
\pgfusepath{clip}%
\pgfsetbuttcap%
\pgfsetroundjoin%
\pgfsetlinewidth{0.301125pt}%
\definecolor{currentstroke}{rgb}{0.500000,0.500000,0.500000}%
\pgfsetstrokecolor{currentstroke}%
\pgfsetstrokeopacity{0.300000}%
\pgfsetdash{}{0pt}%
\pgfpathmoveto{\pgfqpoint{3.727238in}{1.402235in}}%
\pgfpathlineto{\pgfqpoint{3.727238in}{1.402235in}}%
\pgfpathlineto{\pgfqpoint{3.667217in}{1.435067in}}%
\pgfpathlineto{\pgfqpoint{3.608500in}{1.470181in}}%
\pgfpathlineto{\pgfqpoint{3.551067in}{1.507362in}}%
\pgfpathlineto{\pgfqpoint{3.494885in}{1.546410in}}%
\pgfpathlineto{\pgfqpoint{3.439911in}{1.587143in}}%
\pgfpathlineto{\pgfqpoint{3.386102in}{1.629405in}}%
\pgfpathlineto{\pgfqpoint{3.333410in}{1.673054in}}%
\pgfpathlineto{\pgfqpoint{3.281800in}{1.717977in}}%
\pgfpathlineto{\pgfqpoint{3.231260in}{1.764100in}}%
\pgfpathlineto{\pgfqpoint{3.181792in}{1.811370in}}%
\pgfpathlineto{\pgfqpoint{3.133433in}{1.859774in}}%
\pgfpathlineto{\pgfqpoint{3.086271in}{1.909343in}}%
\pgfpathlineto{\pgfqpoint{3.040458in}{1.960157in}}%
\pgfpathlineto{\pgfqpoint{2.996246in}{2.012368in}}%
\pgfpathlineto{\pgfqpoint{2.954036in}{2.066199in}}%
\pgfpathlineto{\pgfqpoint{2.914451in}{2.121973in}}%
\pgfpathlineto{\pgfqpoint{2.878450in}{2.180101in}}%
\pgfpathlineto{\pgfqpoint{2.847466in}{2.241012in}}%
\pgfpathlineto{\pgfqpoint{2.823476in}{2.304934in}}%
\pgfpathlineto{\pgfqpoint{2.808719in}{2.371508in}}%
\pgfpathlineto{\pgfqpoint{2.804765in}{2.439531in}}%
\pgfpathlineto{\pgfqpoint{2.811519in}{2.507364in}}%
\pgfpathlineto{\pgfqpoint{2.827336in}{2.573764in}}%
\pgfpathlineto{\pgfqpoint{2.850049in}{2.638207in}}%
\pgfpathlineto{\pgfqpoint{2.877766in}{2.700704in}}%
\pgfpathlineto{\pgfqpoint{2.909099in}{2.761492in}}%
\pgfpathlineto{\pgfqpoint{2.943102in}{2.820839in}}%
\pgfpathlineto{\pgfqpoint{2.979150in}{2.878976in}}%
\pgfpathlineto{\pgfqpoint{3.016832in}{2.936072in}}%
\pgfusepath{stroke}%
\end{pgfscope}%
\begin{pgfscope}%
\pgfpathrectangle{\pgfqpoint{0.647939in}{0.492442in}}{\pgfqpoint{3.079299in}{3.079299in}}%
\pgfusepath{clip}%
\pgfsetbuttcap%
\pgfsetroundjoin%
\pgfsetlinewidth{0.301125pt}%
\definecolor{currentstroke}{rgb}{0.500000,0.500000,0.500000}%
\pgfsetstrokecolor{currentstroke}%
\pgfsetstrokeopacity{0.300000}%
\pgfsetdash{}{0pt}%
\pgfpathmoveto{\pgfqpoint{3.727238in}{1.472219in}}%
\pgfpathlineto{\pgfqpoint{3.727238in}{1.472219in}}%
\pgfpathlineto{\pgfqpoint{3.668048in}{1.506524in}}%
\pgfpathlineto{\pgfqpoint{3.610303in}{1.543211in}}%
\pgfpathlineto{\pgfqpoint{3.553995in}{1.582071in}}%
\pgfpathlineto{\pgfqpoint{3.499106in}{1.622915in}}%
\pgfpathlineto{\pgfqpoint{3.445619in}{1.665577in}}%
\pgfpathlineto{\pgfqpoint{3.393510in}{1.709916in}}%
\pgfpathlineto{\pgfqpoint{3.342769in}{1.755813in}}%
\pgfpathlineto{\pgfqpoint{3.293414in}{1.803198in}}%
\pgfpathlineto{\pgfqpoint{3.245483in}{1.852022in}}%
\pgfusepath{stroke}%
\end{pgfscope}%
\begin{pgfscope}%
\pgfpathrectangle{\pgfqpoint{0.647939in}{0.492442in}}{\pgfqpoint{3.079299in}{3.079299in}}%
\pgfusepath{clip}%
\pgfsetbuttcap%
\pgfsetroundjoin%
\pgfsetlinewidth{0.301125pt}%
\definecolor{currentstroke}{rgb}{0.500000,0.500000,0.500000}%
\pgfsetstrokecolor{currentstroke}%
\pgfsetstrokeopacity{0.300000}%
\pgfsetdash{}{0pt}%
\pgfpathmoveto{\pgfqpoint{3.727238in}{1.612187in}}%
\pgfpathlineto{\pgfqpoint{3.727238in}{1.612187in}}%
\pgfpathlineto{\pgfqpoint{3.670091in}{1.649789in}}%
\pgfpathlineto{\pgfqpoint{3.614737in}{1.689986in}}%
\pgfpathlineto{\pgfqpoint{3.561207in}{1.732583in}}%
\pgfpathlineto{\pgfqpoint{3.509532in}{1.777415in}}%
\pgfpathlineto{\pgfqpoint{3.459750in}{1.824340in}}%
\pgfpathlineto{\pgfqpoint{3.411916in}{1.873250in}}%
\pgfpathlineto{\pgfqpoint{3.366129in}{1.924082in}}%
\pgfpathlineto{\pgfqpoint{3.322533in}{1.976801in}}%
\pgfpathlineto{\pgfqpoint{3.281335in}{2.031411in}}%
\pgfpathlineto{\pgfqpoint{3.242819in}{2.087939in}}%
\pgfpathlineto{\pgfqpoint{3.207382in}{2.146440in}}%
\pgfpathlineto{\pgfqpoint{3.175534in}{2.206963in}}%
\pgfpathlineto{\pgfqpoint{3.147911in}{2.269515in}}%
\pgfpathlineto{\pgfqpoint{3.125254in}{2.334016in}}%
\pgfpathlineto{\pgfqpoint{3.108332in}{2.400234in}}%
\pgfpathlineto{\pgfqpoint{3.097815in}{2.467748in}}%
\pgfpathlineto{\pgfqpoint{3.094104in}{2.535970in}}%
\pgfpathlineto{\pgfqpoint{3.097213in}{2.604229in}}%
\pgfpathlineto{\pgfqpoint{3.106765in}{2.671906in}}%
\pgfpathlineto{\pgfqpoint{3.122112in}{2.738520in}}%
\pgfpathlineto{\pgfqpoint{3.142487in}{2.803783in}}%
\pgfpathlineto{\pgfqpoint{3.167143in}{2.867565in}}%
\pgfpathlineto{\pgfqpoint{3.195428in}{2.929834in}}%
\pgfpathlineto{\pgfqpoint{3.226815in}{2.990606in}}%
\pgfpathlineto{\pgfqpoint{3.260888in}{3.049923in}}%
\pgfpathlineto{\pgfqpoint{3.297339in}{3.107814in}}%
\pgfpathlineto{\pgfqpoint{3.335948in}{3.164292in}}%
\pgfpathlineto{\pgfqpoint{3.376555in}{3.219352in}}%
\pgfpathlineto{\pgfqpoint{3.419051in}{3.272966in}}%
\pgfpathlineto{\pgfqpoint{3.463389in}{3.325070in}}%
\pgfpathlineto{\pgfqpoint{3.509542in}{3.375573in}}%
\pgfpathlineto{\pgfqpoint{3.557514in}{3.424349in}}%
\pgfpathlineto{\pgfqpoint{3.607328in}{3.471241in}}%
\pgfpathlineto{\pgfqpoint{3.659014in}{3.516059in}}%
\pgfpathlineto{\pgfqpoint{3.712606in}{3.558574in}}%
\pgfpathlineto{\pgfqpoint{3.727238in}{3.569654in}}%
\pgfusepath{stroke}%
\end{pgfscope}%
\begin{pgfscope}%
\pgfpathrectangle{\pgfqpoint{0.647939in}{0.492442in}}{\pgfqpoint{3.079299in}{3.079299in}}%
\pgfusepath{clip}%
\pgfsetbuttcap%
\pgfsetroundjoin%
\pgfsetlinewidth{0.301125pt}%
\definecolor{currentstroke}{rgb}{0.500000,0.500000,0.500000}%
\pgfsetstrokecolor{currentstroke}%
\pgfsetstrokeopacity{0.300000}%
\pgfsetdash{}{0pt}%
\pgfpathmoveto{\pgfqpoint{3.727238in}{1.752155in}}%
\pgfpathlineto{\pgfqpoint{3.727238in}{1.752155in}}%
\pgfpathlineto{\pgfqpoint{3.672808in}{1.793582in}}%
\pgfpathlineto{\pgfqpoint{3.620638in}{1.837822in}}%
\pgfpathlineto{\pgfqpoint{3.570818in}{1.884694in}}%
\pgfpathlineto{\pgfqpoint{3.523448in}{1.934041in}}%
\pgfpathlineto{\pgfqpoint{3.478656in}{1.985740in}}%
\pgfpathlineto{\pgfqpoint{3.436619in}{2.039702in}}%
\pgfpathlineto{\pgfqpoint{3.397566in}{2.095854in}}%
\pgfpathlineto{\pgfqpoint{3.361805in}{2.154152in}}%
\pgfpathlineto{\pgfqpoint{3.329717in}{2.214544in}}%
\pgfpathlineto{\pgfqpoint{3.301761in}{2.276947in}}%
\pgfpathlineto{\pgfqpoint{3.278455in}{2.341221in}}%
\pgfpathlineto{\pgfqpoint{3.260336in}{2.407135in}}%
\pgfpathlineto{\pgfqpoint{3.247888in}{2.474344in}}%
\pgfpathlineto{\pgfqpoint{3.241451in}{2.542390in}}%
\pgfpathlineto{\pgfqpoint{3.241149in}{2.610743in}}%
\pgfpathlineto{\pgfqpoint{3.246860in}{2.678857in}}%
\pgfpathlineto{\pgfqpoint{3.258238in}{2.746261in}}%
\pgfpathlineto{\pgfqpoint{3.274799in}{2.812592in}}%
\pgfpathlineto{\pgfqpoint{3.296001in}{2.877600in}}%
\pgfpathlineto{\pgfqpoint{3.321316in}{2.941128in}}%
\pgfpathlineto{\pgfqpoint{3.350275in}{3.003089in}}%
\pgfpathlineto{\pgfqpoint{3.382484in}{3.063429in}}%
\pgfpathlineto{\pgfqpoint{3.417627in}{3.122113in}}%
\pgfpathlineto{\pgfqpoint{3.455470in}{3.179096in}}%
\pgfpathlineto{\pgfqpoint{3.495844in}{3.234318in}}%
\pgfpathlineto{\pgfqpoint{3.538620in}{3.287704in}}%
\pgfpathlineto{\pgfqpoint{3.583724in}{3.339136in}}%
\pgfpathlineto{\pgfqpoint{3.631116in}{3.388467in}}%
\pgfpathlineto{\pgfqpoint{3.680776in}{3.435512in}}%
\pgfpathlineto{\pgfqpoint{3.727238in}{3.477463in}}%
\pgfusepath{stroke}%
\end{pgfscope}%
\begin{pgfscope}%
\pgfpathrectangle{\pgfqpoint{0.647939in}{0.492442in}}{\pgfqpoint{3.079299in}{3.079299in}}%
\pgfusepath{clip}%
\pgfsetbuttcap%
\pgfsetroundjoin%
\pgfsetlinewidth{0.301125pt}%
\definecolor{currentstroke}{rgb}{0.500000,0.500000,0.500000}%
\pgfsetstrokecolor{currentstroke}%
\pgfsetstrokeopacity{0.300000}%
\pgfsetdash{}{0pt}%
\pgfpathmoveto{\pgfqpoint{3.727238in}{1.822139in}}%
\pgfpathlineto{\pgfqpoint{3.727238in}{1.822139in}}%
\pgfpathlineto{\pgfqpoint{3.674501in}{1.865695in}}%
\pgfpathlineto{\pgfqpoint{3.624314in}{1.912166in}}%
\pgfpathlineto{\pgfqpoint{3.576804in}{1.961372in}}%
\pgfpathlineto{\pgfqpoint{3.532116in}{2.013153in}}%
\pgfpathlineto{\pgfqpoint{3.490436in}{2.067383in}}%
\pgfpathlineto{\pgfqpoint{3.452005in}{2.123957in}}%
\pgfpathlineto{\pgfqpoint{3.417128in}{2.182784in}}%
\pgfpathlineto{\pgfqpoint{3.386171in}{2.243759in}}%
\pgfpathlineto{\pgfqpoint{3.359562in}{2.306746in}}%
\pgfpathlineto{\pgfqpoint{3.337768in}{2.371546in}}%
\pgfpathlineto{\pgfqpoint{3.321253in}{2.437880in}}%
\pgfpathlineto{\pgfqpoint{3.310406in}{2.505367in}}%
\pgfpathlineto{\pgfqpoint{3.305475in}{2.573543in}}%
\pgfpathlineto{\pgfqpoint{3.306515in}{2.641893in}}%
\pgfpathlineto{\pgfqpoint{3.313371in}{2.709907in}}%
\pgfusepath{stroke}%
\end{pgfscope}%
\begin{pgfscope}%
\pgfpathrectangle{\pgfqpoint{0.647939in}{0.492442in}}{\pgfqpoint{3.079299in}{3.079299in}}%
\pgfusepath{clip}%
\pgfsetbuttcap%
\pgfsetroundjoin%
\pgfsetlinewidth{0.301125pt}%
\definecolor{currentstroke}{rgb}{0.500000,0.500000,0.500000}%
\pgfsetstrokecolor{currentstroke}%
\pgfsetstrokeopacity{0.300000}%
\pgfsetdash{}{0pt}%
\pgfpathmoveto{\pgfqpoint{3.727238in}{1.962108in}}%
\pgfpathlineto{\pgfqpoint{3.727238in}{1.962108in}}%
\pgfpathlineto{\pgfqpoint{3.678777in}{2.010361in}}%
\pgfpathlineto{\pgfqpoint{3.633585in}{2.061687in}}%
\pgfpathlineto{\pgfqpoint{3.591876in}{2.115882in}}%
\pgfpathlineto{\pgfqpoint{3.553896in}{2.172752in}}%
\pgfpathlineto{\pgfqpoint{3.519947in}{2.232113in}}%
\pgfpathlineto{\pgfqpoint{3.490375in}{2.293767in}}%
\pgfpathlineto{\pgfqpoint{3.465560in}{2.357479in}}%
\pgfpathlineto{\pgfqpoint{3.445891in}{2.422959in}}%
\pgfpathlineto{\pgfqpoint{3.431721in}{2.489837in}}%
\pgfpathlineto{\pgfqpoint{3.423314in}{2.557672in}}%
\pgfpathlineto{\pgfqpoint{3.420799in}{2.625977in}}%
\pgfpathlineto{\pgfqpoint{3.424146in}{2.694250in}}%
\pgfpathlineto{\pgfqpoint{3.433172in}{2.762017in}}%
\pgfpathlineto{\pgfqpoint{3.447563in}{2.828857in}}%
\pgfpathlineto{\pgfqpoint{3.466941in}{2.894429in}}%
\pgfpathlineto{\pgfqpoint{3.490909in}{2.958470in}}%
\pgfpathlineto{\pgfqpoint{3.519085in}{3.020780in}}%
\pgfpathlineto{\pgfqpoint{3.551134in}{3.081194in}}%
\pgfpathlineto{\pgfqpoint{3.586780in}{3.139564in}}%
\pgfpathlineto{\pgfqpoint{3.625802in}{3.195737in}}%
\pgfpathlineto{\pgfqpoint{3.668024in}{3.249544in}}%
\pgfpathlineto{\pgfqpoint{3.713327in}{3.300781in}}%
\pgfpathlineto{\pgfqpoint{3.727238in}{3.315612in}}%
\pgfusepath{stroke}%
\end{pgfscope}%
\begin{pgfscope}%
\pgfpathrectangle{\pgfqpoint{0.647939in}{0.492442in}}{\pgfqpoint{3.079299in}{3.079299in}}%
\pgfusepath{clip}%
\pgfsetbuttcap%
\pgfsetroundjoin%
\pgfsetlinewidth{0.301125pt}%
\definecolor{currentstroke}{rgb}{0.500000,0.500000,0.500000}%
\pgfsetstrokecolor{currentstroke}%
\pgfsetstrokeopacity{0.300000}%
\pgfsetdash{}{0pt}%
\pgfpathmoveto{\pgfqpoint{3.727238in}{2.102076in}}%
\pgfpathlineto{\pgfqpoint{3.727238in}{2.102076in}}%
\pgfpathlineto{\pgfqpoint{3.684611in}{2.155535in}}%
\pgfpathlineto{\pgfqpoint{3.646168in}{2.212073in}}%
\pgfpathlineto{\pgfqpoint{3.612213in}{2.271414in}}%
\pgfpathlineto{\pgfqpoint{3.583078in}{2.333262in}}%
\pgfpathlineto{\pgfqpoint{3.559110in}{2.397285in}}%
\pgfpathlineto{\pgfqpoint{3.540649in}{2.463100in}}%
\pgfpathlineto{\pgfqpoint{3.527978in}{2.530265in}}%
\pgfpathlineto{\pgfqpoint{3.521287in}{2.598285in}}%
\pgfpathlineto{\pgfqpoint{3.520641in}{2.666634in}}%
\pgfpathlineto{\pgfqpoint{3.525965in}{2.734786in}}%
\pgfpathlineto{\pgfqpoint{3.537051in}{2.802241in}}%
\pgfpathlineto{\pgfqpoint{3.553599in}{2.868570in}}%
\pgfpathlineto{\pgfqpoint{3.575264in}{2.933414in}}%
\pgfpathlineto{\pgfqpoint{3.601690in}{2.996472in}}%
\pgfpathlineto{\pgfqpoint{3.632548in}{3.057489in}}%
\pgfpathlineto{\pgfqpoint{3.667552in}{3.116230in}}%
\pgfpathlineto{\pgfqpoint{3.706468in}{3.172459in}}%
\pgfpathlineto{\pgfqpoint{3.727238in}{3.200413in}}%
\pgfusepath{stroke}%
\end{pgfscope}%
\begin{pgfscope}%
\pgfpathrectangle{\pgfqpoint{0.647939in}{0.492442in}}{\pgfqpoint{3.079299in}{3.079299in}}%
\pgfusepath{clip}%
\pgfsetbuttcap%
\pgfsetroundjoin%
\pgfsetlinewidth{0.301125pt}%
\definecolor{currentstroke}{rgb}{0.500000,0.500000,0.500000}%
\pgfsetstrokecolor{currentstroke}%
\pgfsetstrokeopacity{0.300000}%
\pgfsetdash{}{0pt}%
\pgfpathmoveto{\pgfqpoint{3.727238in}{2.242044in}}%
\pgfpathlineto{\pgfqpoint{3.727238in}{2.242044in}}%
\pgfpathlineto{\pgfqpoint{3.692506in}{2.300922in}}%
\pgfpathlineto{\pgfqpoint{3.663028in}{2.362597in}}%
\pgfpathlineto{\pgfqpoint{3.639142in}{2.426642in}}%
\pgfpathlineto{\pgfqpoint{3.621154in}{2.492579in}}%
\pgfpathlineto{\pgfqpoint{3.609313in}{2.559887in}}%
\pgfpathlineto{\pgfqpoint{3.603768in}{2.628001in}}%
\pgfpathlineto{\pgfqpoint{3.604541in}{2.696342in}}%
\pgfpathlineto{\pgfqpoint{3.611529in}{2.764337in}}%
\pgfpathlineto{\pgfqpoint{3.624516in}{2.831448in}}%
\pgfpathlineto{\pgfqpoint{3.643203in}{2.897202in}}%
\pgfpathlineto{\pgfqpoint{3.667256in}{2.961190in}}%
\pgfpathlineto{\pgfqpoint{3.696347in}{3.023056in}}%
\pgfpathlineto{\pgfqpoint{3.727238in}{3.082594in}}%
\pgfusepath{stroke}%
\end{pgfscope}%
\begin{pgfscope}%
\pgfpathrectangle{\pgfqpoint{0.647939in}{0.492442in}}{\pgfqpoint{3.079299in}{3.079299in}}%
\pgfusepath{clip}%
\pgfsetbuttcap%
\pgfsetroundjoin%
\pgfsetlinewidth{0.301125pt}%
\definecolor{currentstroke}{rgb}{0.500000,0.500000,0.500000}%
\pgfsetstrokecolor{currentstroke}%
\pgfsetstrokeopacity{0.300000}%
\pgfsetdash{}{0pt}%
\pgfpathmoveto{\pgfqpoint{3.727238in}{2.382012in}}%
\pgfpathlineto{\pgfqpoint{3.727238in}{2.382012in}}%
\pgfpathlineto{\pgfqpoint{3.702857in}{2.445854in}}%
\pgfpathlineto{\pgfqpoint{3.684762in}{2.511754in}}%
\pgfpathlineto{\pgfqpoint{3.673186in}{2.579107in}}%
\pgfpathlineto{\pgfqpoint{3.668258in}{2.647266in}}%
\pgfpathlineto{\pgfqpoint{3.669980in}{2.715577in}}%
\pgfusepath{stroke}%
\end{pgfscope}%
\begin{pgfscope}%
\pgfpathrectangle{\pgfqpoint{0.647939in}{0.492442in}}{\pgfqpoint{3.079299in}{3.079299in}}%
\pgfusepath{clip}%
\pgfsetbuttcap%
\pgfsetroundjoin%
\pgfsetlinewidth{0.301125pt}%
\definecolor{currentstroke}{rgb}{0.500000,0.500000,0.500000}%
\pgfsetstrokecolor{currentstroke}%
\pgfsetstrokeopacity{0.300000}%
\pgfsetdash{}{0pt}%
\pgfpathmoveto{\pgfqpoint{3.709942in}{2.533973in}}%
\pgfpathlineto{\pgfqpoint{3.699889in}{2.601558in}}%
\pgfpathlineto{\pgfqpoint{3.696683in}{2.669821in}}%
\pgfpathlineto{\pgfqpoint{3.700294in}{2.738067in}}%
\pgfpathlineto{\pgfqpoint{3.710564in}{2.805627in}}%
\pgfpathlineto{\pgfqpoint{3.727238in}{2.871901in}}%
\pgfpathlineto{\pgfqpoint{3.727238in}{2.871901in}}%
\pgfusepath{stroke}%
\end{pgfscope}%
\begin{pgfscope}%
\pgfpathrectangle{\pgfqpoint{0.647939in}{0.492442in}}{\pgfqpoint{3.079299in}{3.079299in}}%
\pgfusepath{clip}%
\pgfsetbuttcap%
\pgfsetroundjoin%
\pgfsetlinewidth{0.301125pt}%
\definecolor{currentstroke}{rgb}{0.500000,0.500000,0.500000}%
\pgfsetstrokecolor{currentstroke}%
\pgfsetstrokeopacity{0.300000}%
\pgfsetdash{}{0pt}%
\pgfpathmoveto{\pgfqpoint{0.647939in}{2.495534in}}%
\pgfpathlineto{\pgfqpoint{0.672048in}{2.498664in}}%
\pgfpathlineto{\pgfqpoint{0.739806in}{2.508193in}}%
\pgfpathlineto{\pgfqpoint{0.807345in}{2.519167in}}%
\pgfpathlineto{\pgfqpoint{0.874632in}{2.531590in}}%
\pgfpathlineto{\pgfqpoint{0.941644in}{2.545422in}}%
\pgfpathlineto{\pgfqpoint{1.008368in}{2.560583in}}%
\pgfpathlineto{\pgfqpoint{1.074809in}{2.576946in}}%
\pgfpathlineto{\pgfqpoint{1.140988in}{2.594341in}}%
\pgfpathlineto{\pgfqpoint{1.206947in}{2.612555in}}%
\pgfpathlineto{\pgfqpoint{1.272747in}{2.631337in}}%
\pgfpathlineto{\pgfqpoint{1.338467in}{2.650400in}}%
\pgfpathlineto{\pgfqpoint{1.404194in}{2.669436in}}%
\pgfpathlineto{\pgfqpoint{1.470024in}{2.688113in}}%
\pgfpathlineto{\pgfqpoint{1.536048in}{2.706089in}}%
\pgfpathlineto{\pgfqpoint{1.602347in}{2.723013in}}%
\pgfpathlineto{\pgfqpoint{1.668984in}{2.738550in}}%
\pgfpathlineto{\pgfqpoint{1.735991in}{2.752390in}}%
\pgfpathlineto{\pgfqpoint{1.803370in}{2.764283in}}%
\pgfpathlineto{\pgfqpoint{1.871085in}{2.774071in}}%
\pgfpathlineto{\pgfqpoint{1.939076in}{2.781719in}}%
\pgfpathlineto{\pgfqpoint{2.007265in}{2.787347in}}%
\pgfpathlineto{\pgfqpoint{2.075576in}{2.791266in}}%
\pgfpathlineto{\pgfqpoint{2.143949in}{2.794009in}}%
\pgfpathlineto{\pgfqpoint{2.212337in}{2.796362in}}%
\pgfpathlineto{\pgfqpoint{2.280697in}{2.799357in}}%
\pgfpathlineto{\pgfqpoint{2.348935in}{2.804255in}}%
\pgfpathlineto{\pgfqpoint{2.416833in}{2.812487in}}%
\pgfpathlineto{\pgfqpoint{2.483965in}{2.825454in}}%
\pgfpathlineto{\pgfqpoint{2.549693in}{2.844204in}}%
\pgfpathlineto{\pgfqpoint{2.613343in}{2.869079in}}%
\pgfpathlineto{\pgfqpoint{2.674466in}{2.899665in}}%
\pgfpathlineto{\pgfqpoint{2.732964in}{2.935060in}}%
\pgfpathlineto{\pgfqpoint{2.789034in}{2.974216in}}%
\pgfpathlineto{\pgfqpoint{2.843044in}{3.016181in}}%
\pgfpathlineto{\pgfqpoint{2.895414in}{3.060186in}}%
\pgfpathlineto{\pgfqpoint{2.946542in}{3.105644in}}%
\pgfpathlineto{\pgfqpoint{2.996778in}{3.152093in}}%
\pgfpathlineto{\pgfqpoint{3.046418in}{3.199180in}}%
\pgfpathlineto{\pgfqpoint{3.095722in}{3.246623in}}%
\pgfpathlineto{\pgfqpoint{3.144911in}{3.294186in}}%
\pgfpathlineto{\pgfqpoint{3.194176in}{3.341672in}}%
\pgfpathlineto{\pgfqpoint{3.243694in}{3.388894in}}%
\pgfpathlineto{\pgfqpoint{3.293621in}{3.435684in}}%
\pgfpathlineto{\pgfqpoint{3.344105in}{3.481872in}}%
\pgfpathlineto{\pgfqpoint{3.395290in}{3.527281in}}%
\pgfpathlineto{\pgfqpoint{3.447302in}{3.571741in}}%
\pgfpathlineto{\pgfqpoint{3.447302in}{3.571741in}}%
\pgfusepath{stroke}%
\end{pgfscope}%
\begin{pgfscope}%
\pgfpathrectangle{\pgfqpoint{0.647939in}{0.492442in}}{\pgfqpoint{3.079299in}{3.079299in}}%
\pgfusepath{clip}%
\pgfsetbuttcap%
\pgfsetroundjoin%
\pgfsetlinewidth{0.301125pt}%
\definecolor{currentstroke}{rgb}{0.500000,0.500000,0.500000}%
\pgfsetstrokecolor{currentstroke}%
\pgfsetstrokeopacity{0.300000}%
\pgfsetdash{}{0pt}%
\pgfpathmoveto{\pgfqpoint{0.647939in}{2.802125in}}%
\pgfpathlineto{\pgfqpoint{0.674203in}{2.805363in}}%
\pgfpathlineto{\pgfqpoint{0.742030in}{2.814392in}}%
\pgfpathlineto{\pgfqpoint{0.809669in}{2.824736in}}%
\pgfpathlineto{\pgfqpoint{0.877097in}{2.836381in}}%
\pgfpathlineto{\pgfqpoint{0.944296in}{2.849275in}}%
\pgfpathlineto{\pgfqpoint{1.011265in}{2.863323in}}%
\pgfpathlineto{\pgfqpoint{1.078012in}{2.878392in}}%
\pgfpathlineto{\pgfqpoint{1.144563in}{2.894305in}}%
\pgfpathlineto{\pgfqpoint{1.210962in}{2.910846in}}%
\pgfpathlineto{\pgfqpoint{1.277266in}{2.927766in}}%
\pgfpathlineto{\pgfqpoint{1.343543in}{2.944791in}}%
\pgfpathlineto{\pgfqpoint{1.409867in}{2.961631in}}%
\pgfpathlineto{\pgfqpoint{1.476314in}{2.977978in}}%
\pgfpathlineto{\pgfqpoint{1.542952in}{2.993521in}}%
\pgfpathlineto{\pgfqpoint{1.609838in}{3.007957in}}%
\pgfpathlineto{\pgfqpoint{1.677006in}{3.021005in}}%
\pgfpathlineto{\pgfqpoint{1.744468in}{3.032428in}}%
\pgfpathlineto{\pgfqpoint{1.812209in}{3.042058in}}%
\pgfpathlineto{\pgfqpoint{1.880189in}{3.049814in}}%
\pgfpathlineto{\pgfqpoint{1.948355in}{3.055728in}}%
\pgfpathlineto{\pgfqpoint{2.016647in}{3.059966in}}%
\pgfpathlineto{\pgfqpoint{2.085013in}{3.062851in}}%
\pgfpathlineto{\pgfqpoint{2.153411in}{3.064867in}}%
\pgfpathlineto{\pgfqpoint{2.221817in}{3.066641in}}%
\pgfpathlineto{\pgfqpoint{2.290205in}{3.068953in}}%
\pgfpathlineto{\pgfqpoint{2.358524in}{3.072705in}}%
\pgfpathlineto{\pgfqpoint{2.426654in}{3.078898in}}%
\pgfpathlineto{\pgfqpoint{2.494371in}{3.088511in}}%
\pgfpathlineto{\pgfqpoint{2.561337in}{3.102349in}}%
\pgfpathlineto{\pgfqpoint{2.627142in}{3.120909in}}%
\pgfpathlineto{\pgfqpoint{2.691396in}{3.144277in}}%
\pgfpathlineto{\pgfqpoint{2.753835in}{3.172153in}}%
\pgfpathlineto{\pgfqpoint{2.814373in}{3.203967in}}%
\pgfpathlineto{\pgfqpoint{2.873094in}{3.239037in}}%
\pgfpathlineto{\pgfqpoint{2.930202in}{3.276691in}}%
\pgfpathlineto{\pgfqpoint{2.985964in}{3.316326in}}%
\pgfpathlineto{\pgfqpoint{3.040664in}{3.357428in}}%
\pgfpathlineto{\pgfqpoint{3.094581in}{3.399555in}}%
\pgfpathlineto{\pgfqpoint{3.147976in}{3.442340in}}%
\pgfpathlineto{\pgfqpoint{3.201090in}{3.485480in}}%
\pgfpathlineto{\pgfqpoint{3.254140in}{3.528699in}}%
\pgfpathlineto{\pgfqpoint{3.307334in}{3.571741in}}%
\pgfpathlineto{\pgfqpoint{3.307334in}{3.571741in}}%
\pgfusepath{stroke}%
\end{pgfscope}%
\begin{pgfscope}%
\pgfpathrectangle{\pgfqpoint{0.647939in}{0.492442in}}{\pgfqpoint{3.079299in}{3.079299in}}%
\pgfusepath{clip}%
\pgfsetbuttcap%
\pgfsetroundjoin%
\pgfsetlinewidth{0.301125pt}%
\definecolor{currentstroke}{rgb}{0.500000,0.500000,0.500000}%
\pgfsetstrokecolor{currentstroke}%
\pgfsetstrokeopacity{0.300000}%
\pgfsetdash{}{0pt}%
\pgfpathmoveto{\pgfqpoint{0.647939in}{2.991845in}}%
\pgfpathlineto{\pgfqpoint{0.669616in}{2.994402in}}%
\pgfpathlineto{\pgfqpoint{0.737494in}{3.003039in}}%
\pgfpathlineto{\pgfqpoint{0.805202in}{3.012921in}}%
\pgfpathlineto{\pgfqpoint{0.872721in}{3.024028in}}%
\pgfpathlineto{\pgfqpoint{0.940037in}{3.036304in}}%
\pgfpathlineto{\pgfqpoint{1.007149in}{3.049651in}}%
\pgfpathlineto{\pgfqpoint{1.074068in}{3.063935in}}%
\pgfpathlineto{\pgfqpoint{1.140821in}{3.078982in}}%
\pgfpathlineto{\pgfqpoint{1.207447in}{3.094584in}}%
\pgfpathlineto{\pgfqpoint{1.273999in}{3.110499in}}%
\pgfpathlineto{\pgfqpoint{1.340541in}{3.126461in}}%
\pgfpathlineto{\pgfqpoint{1.407137in}{3.142190in}}%
\pgfpathlineto{\pgfqpoint{1.473854in}{3.157396in}}%
\pgfpathlineto{\pgfqpoint{1.540750in}{3.171789in}}%
\pgfpathlineto{\pgfqpoint{1.607870in}{3.185091in}}%
\pgfpathlineto{\pgfqpoint{1.675242in}{3.197050in}}%
\pgfpathlineto{\pgfqpoint{1.742868in}{3.207461in}}%
\pgfpathlineto{\pgfqpoint{1.810732in}{3.216185in}}%
\pgfpathlineto{\pgfqpoint{1.878797in}{3.223169in}}%
\pgfpathlineto{\pgfqpoint{1.947015in}{3.228462in}}%
\pgfpathlineto{\pgfqpoint{2.015336in}{3.232231in}}%
\pgfpathlineto{\pgfqpoint{2.083715in}{3.234778in}}%
\pgfpathlineto{\pgfqpoint{2.152121in}{3.236537in}}%
\pgfpathlineto{\pgfqpoint{2.220533in}{3.238063in}}%
\pgfpathlineto{\pgfqpoint{2.288932in}{3.240020in}}%
\pgfpathlineto{\pgfqpoint{2.357284in}{3.243165in}}%
\pgfpathlineto{\pgfqpoint{2.425506in}{3.248324in}}%
\pgfpathlineto{\pgfqpoint{2.493445in}{3.256310in}}%
\pgfpathlineto{\pgfqpoint{2.560864in}{3.267827in}}%
\pgfpathlineto{\pgfqpoint{2.627462in}{3.283370in}}%
\pgfpathlineto{\pgfqpoint{2.692924in}{3.303151in}}%
\pgfpathlineto{\pgfqpoint{2.756991in}{3.327079in}}%
\pgfpathlineto{\pgfqpoint{2.819513in}{3.354811in}}%
\pgfpathlineto{\pgfqpoint{2.880473in}{3.385842in}}%
\pgfpathlineto{\pgfqpoint{2.939967in}{3.419610in}}%
\pgfpathlineto{\pgfqpoint{2.998173in}{3.455560in}}%
\pgfpathlineto{\pgfqpoint{3.055315in}{3.493187in}}%
\pgfpathlineto{\pgfqpoint{3.111633in}{3.532043in}}%
\pgfpathlineto{\pgfqpoint{3.167366in}{3.571741in}}%
\pgfpathlineto{\pgfqpoint{3.167366in}{3.571741in}}%
\pgfusepath{stroke}%
\end{pgfscope}%
\begin{pgfscope}%
\pgfpathrectangle{\pgfqpoint{0.647939in}{0.492442in}}{\pgfqpoint{3.079299in}{3.079299in}}%
\pgfusepath{clip}%
\pgfsetbuttcap%
\pgfsetroundjoin%
\pgfsetlinewidth{0.301125pt}%
\definecolor{currentstroke}{rgb}{0.500000,0.500000,0.500000}%
\pgfsetstrokecolor{currentstroke}%
\pgfsetstrokeopacity{0.300000}%
\pgfsetdash{}{0pt}%
\pgfpathmoveto{\pgfqpoint{0.647939in}{3.124640in}}%
\pgfpathlineto{\pgfqpoint{0.693236in}{3.130140in}}%
\pgfpathlineto{\pgfqpoint{0.761087in}{3.138992in}}%
\pgfpathlineto{\pgfqpoint{0.828771in}{3.149038in}}%
\pgfpathlineto{\pgfqpoint{0.896273in}{3.160246in}}%
\pgfpathlineto{\pgfqpoint{0.963587in}{3.172541in}}%
\pgfpathlineto{\pgfqpoint{1.030716in}{3.185805in}}%
\pgfpathlineto{\pgfqpoint{1.097678in}{3.199888in}}%
\pgfpathlineto{\pgfqpoint{1.164505in}{3.214606in}}%
\pgfpathlineto{\pgfqpoint{1.231239in}{3.229743in}}%
\pgfpathlineto{\pgfqpoint{1.297932in}{3.245055in}}%
\pgfpathlineto{\pgfqpoint{1.364646in}{3.260280in}}%
\pgfpathlineto{\pgfqpoint{1.431441in}{3.275139in}}%
\pgfpathlineto{\pgfqpoint{1.498377in}{3.289349in}}%
\pgfpathlineto{\pgfqpoint{1.565500in}{3.302641in}}%
\pgfpathlineto{\pgfqpoint{1.632843in}{3.314761in}}%
\pgfpathlineto{\pgfqpoint{1.700420in}{3.325493in}}%
\pgfpathlineto{\pgfqpoint{1.768226in}{3.334672in}}%
\pgfpathlineto{\pgfqpoint{1.836233in}{3.342208in}}%
\pgfpathlineto{\pgfqpoint{1.904402in}{3.348100in}}%
\pgfpathlineto{\pgfqpoint{1.972688in}{3.352449in}}%
\pgfpathlineto{\pgfqpoint{2.041047in}{3.355466in}}%
\pgfpathlineto{\pgfqpoint{2.109445in}{3.357476in}}%
\pgfpathlineto{\pgfqpoint{2.177858in}{3.358930in}}%
\pgfpathlineto{\pgfqpoint{2.246271in}{3.360384in}}%
\pgfpathlineto{\pgfqpoint{2.314666in}{3.362476in}}%
\pgfpathlineto{\pgfqpoint{2.383003in}{3.365907in}}%
\pgfpathlineto{\pgfqpoint{2.451199in}{3.371405in}}%
\pgfpathlineto{\pgfqpoint{2.519107in}{3.379673in}}%
\pgfpathlineto{\pgfqpoint{2.586512in}{3.391301in}}%
\pgfpathlineto{\pgfqpoint{2.653153in}{3.406683in}}%
\pgfpathlineto{\pgfqpoint{2.718770in}{3.425966in}}%
\pgfpathlineto{\pgfqpoint{2.783150in}{3.449052in}}%
\pgfpathlineto{\pgfqpoint{2.846172in}{3.475635in}}%
\pgfpathlineto{\pgfqpoint{2.907822in}{3.505275in}}%
\pgfpathlineto{\pgfqpoint{2.968180in}{3.537477in}}%
\pgfpathlineto{\pgfqpoint{3.027398in}{3.571741in}}%
\pgfpathlineto{\pgfqpoint{3.027398in}{3.571741in}}%
\pgfusepath{stroke}%
\end{pgfscope}%
\begin{pgfscope}%
\pgfpathrectangle{\pgfqpoint{0.647939in}{0.492442in}}{\pgfqpoint{3.079299in}{3.079299in}}%
\pgfusepath{clip}%
\pgfsetbuttcap%
\pgfsetroundjoin%
\pgfsetlinewidth{0.301125pt}%
\definecolor{currentstroke}{rgb}{0.500000,0.500000,0.500000}%
\pgfsetstrokecolor{currentstroke}%
\pgfsetstrokeopacity{0.300000}%
\pgfsetdash{}{0pt}%
\pgfpathmoveto{\pgfqpoint{0.647939in}{3.218462in}}%
\pgfpathlineto{\pgfqpoint{0.666652in}{3.220570in}}%
\pgfpathlineto{\pgfqpoint{0.734578in}{3.228819in}}%
\pgfpathlineto{\pgfqpoint{0.802353in}{3.238238in}}%
\pgfpathlineto{\pgfqpoint{0.869959in}{3.248801in}}%
\pgfpathlineto{\pgfqpoint{0.937388in}{3.260445in}}%
\pgfpathlineto{\pgfqpoint{1.004640in}{3.273071in}}%
\pgfpathlineto{\pgfqpoint{1.071728in}{3.286544in}}%
\pgfpathlineto{\pgfqpoint{1.138677in}{3.300692in}}%
\pgfpathlineto{\pgfqpoint{1.205526in}{3.315313in}}%
\pgfpathlineto{\pgfqpoint{1.272321in}{3.330175in}}%
\pgfpathlineto{\pgfqpoint{1.339120in}{3.345021in}}%
\pgfpathlineto{\pgfqpoint{1.405980in}{3.359587in}}%
\pgfpathlineto{\pgfqpoint{1.472958in}{3.373600in}}%
\pgfpathlineto{\pgfqpoint{1.540100in}{3.386795in}}%
\pgfpathlineto{\pgfqpoint{1.607443in}{3.398921in}}%
\pgfpathlineto{\pgfqpoint{1.675005in}{3.409756in}}%
\pgfpathlineto{\pgfqpoint{1.742784in}{3.419128in}}%
\pgfpathlineto{\pgfqpoint{1.810761in}{3.426928in}}%
\pgfpathlineto{\pgfqpoint{1.878903in}{3.433128in}}%
\pgfpathlineto{\pgfqpoint{1.947168in}{3.437794in}}%
\pgfpathlineto{\pgfqpoint{2.015514in}{3.441092in}}%
\pgfpathlineto{\pgfqpoint{2.083906in}{3.443303in}}%
\pgfpathlineto{\pgfqpoint{2.152317in}{3.444820in}}%
\pgfpathlineto{\pgfqpoint{2.220734in}{3.446125in}}%
\pgfpathlineto{\pgfqpoint{2.289141in}{3.447787in}}%
\pgfpathlineto{\pgfqpoint{2.357515in}{3.450440in}}%
\pgfpathlineto{\pgfqpoint{2.425799in}{3.454764in}}%
\pgfpathlineto{\pgfqpoint{2.493887in}{3.461432in}}%
\pgfpathlineto{\pgfqpoint{2.561616in}{3.471034in}}%
\pgfpathlineto{\pgfqpoint{2.628771in}{3.484023in}}%
\pgfpathlineto{\pgfqpoint{2.695117in}{3.500654in}}%
\pgfpathlineto{\pgfqpoint{2.760432in}{3.520962in}}%
\pgfpathlineto{\pgfqpoint{2.824560in}{3.544770in}}%
\pgfpathlineto{\pgfqpoint{2.887429in}{3.571741in}}%
\pgfpathlineto{\pgfqpoint{2.887429in}{3.571741in}}%
\pgfusepath{stroke}%
\end{pgfscope}%
\begin{pgfscope}%
\pgfpathrectangle{\pgfqpoint{0.647939in}{0.492442in}}{\pgfqpoint{3.079299in}{3.079299in}}%
\pgfusepath{clip}%
\pgfsetbuttcap%
\pgfsetroundjoin%
\pgfsetlinewidth{0.301125pt}%
\definecolor{currentstroke}{rgb}{0.500000,0.500000,0.500000}%
\pgfsetstrokecolor{currentstroke}%
\pgfsetstrokeopacity{0.300000}%
\pgfsetdash{}{0pt}%
\pgfpathmoveto{\pgfqpoint{0.647939in}{3.303115in}}%
\pgfpathlineto{\pgfqpoint{0.714959in}{3.311375in}}%
\pgfpathlineto{\pgfqpoint{0.782798in}{3.320318in}}%
\pgfpathlineto{\pgfqpoint{0.850480in}{3.330385in}}%
\pgfpathlineto{\pgfqpoint{0.917993in}{3.341527in}}%
\pgfpathlineto{\pgfqpoint{0.985336in}{3.353654in}}%
\pgfpathlineto{\pgfqpoint{1.052520in}{3.366643in}}%
\pgfpathlineto{\pgfqpoint{1.119566in}{3.380327in}}%
\pgfpathlineto{\pgfqpoint{1.186509in}{3.394508in}}%
\pgfpathlineto{\pgfqpoint{1.253392in}{3.408968in}}%
\pgfpathlineto{\pgfqpoint{1.320268in}{3.423465in}}%
\pgfpathlineto{\pgfqpoint{1.387191in}{3.437742in}}%
\pgfpathlineto{\pgfqpoint{1.454214in}{3.451534in}}%
\pgfpathlineto{\pgfqpoint{1.521387in}{3.464579in}}%
\pgfpathlineto{\pgfqpoint{1.588744in}{3.476627in}}%
\pgfpathlineto{\pgfqpoint{1.656306in}{3.487458in}}%
\pgfpathlineto{\pgfqpoint{1.724076in}{3.496896in}}%
\pgfpathlineto{\pgfqpoint{1.792039in}{3.504822in}}%
\pgfpathlineto{\pgfqpoint{1.860166in}{3.511190in}}%
\pgfpathlineto{\pgfqpoint{1.928419in}{3.516047in}}%
\pgfpathlineto{\pgfqpoint{1.996755in}{3.519537in}}%
\pgfpathlineto{\pgfqpoint{2.065141in}{3.521904in}}%
\pgfpathlineto{\pgfqpoint{2.133551in}{3.523488in}}%
\pgfpathlineto{\pgfqpoint{2.201968in}{3.524724in}}%
\pgfpathlineto{\pgfqpoint{2.270382in}{3.526139in}}%
\pgfpathlineto{\pgfqpoint{2.338773in}{3.528326in}}%
\pgfpathlineto{\pgfqpoint{2.407102in}{3.531913in}}%
\pgfpathlineto{\pgfqpoint{2.475288in}{3.537524in}}%
\pgfpathlineto{\pgfqpoint{2.543205in}{3.545741in}}%
\pgfpathlineto{\pgfqpoint{2.610673in}{3.557041in}}%
\pgfpathlineto{\pgfqpoint{2.677477in}{3.571741in}}%
\pgfpathlineto{\pgfqpoint{2.677477in}{3.571741in}}%
\pgfusepath{stroke}%
\end{pgfscope}%
\begin{pgfscope}%
\pgfpathrectangle{\pgfqpoint{0.647939in}{0.492442in}}{\pgfqpoint{3.079299in}{3.079299in}}%
\pgfusepath{clip}%
\pgfsetbuttcap%
\pgfsetroundjoin%
\pgfsetlinewidth{0.301125pt}%
\definecolor{currentstroke}{rgb}{0.500000,0.500000,0.500000}%
\pgfsetstrokecolor{currentstroke}%
\pgfsetstrokeopacity{0.300000}%
\pgfsetdash{}{0pt}%
\pgfpathmoveto{\pgfqpoint{1.714520in}{3.536898in}}%
\pgfpathlineto{\pgfqpoint{1.782476in}{3.544886in}}%
\pgfpathlineto{\pgfqpoint{1.850594in}{3.551344in}}%
\pgfpathlineto{\pgfqpoint{1.918839in}{3.556301in}}%
\pgfpathlineto{\pgfqpoint{1.987170in}{3.559886in}}%
\pgfpathlineto{\pgfqpoint{2.055554in}{3.562329in}}%
\pgfpathlineto{\pgfqpoint{2.123962in}{3.563958in}}%
\pgfpathlineto{\pgfqpoint{2.192380in}{3.565188in}}%
\pgfpathlineto{\pgfqpoint{2.260796in}{3.566511in}}%
\pgfpathlineto{\pgfqpoint{2.329194in}{3.568492in}}%
\pgfpathlineto{\pgfqpoint{2.397541in}{3.571741in}}%
\pgfpathlineto{\pgfqpoint{2.397541in}{3.571741in}}%
\pgfusepath{stroke}%
\end{pgfscope}%
\begin{pgfscope}%
\pgfpathrectangle{\pgfqpoint{0.647939in}{0.492442in}}{\pgfqpoint{3.079299in}{3.079299in}}%
\pgfusepath{clip}%
\pgfsetbuttcap%
\pgfsetroundjoin%
\pgfsetlinewidth{0.301125pt}%
\definecolor{currentstroke}{rgb}{0.500000,0.500000,0.500000}%
\pgfsetstrokecolor{currentstroke}%
\pgfsetstrokeopacity{0.300000}%
\pgfsetdash{}{0pt}%
\pgfpathmoveto{\pgfqpoint{0.647939in}{3.396838in}}%
\pgfpathlineto{\pgfqpoint{0.685385in}{3.401108in}}%
\pgfpathlineto{\pgfqpoint{0.753304in}{3.409419in}}%
\pgfpathlineto{\pgfqpoint{0.821079in}{3.418838in}}%
\pgfpathlineto{\pgfqpoint{0.888697in}{3.429328in}}%
\pgfpathlineto{\pgfqpoint{0.956152in}{3.440815in}}%
\pgfpathlineto{\pgfqpoint{1.023452in}{3.453187in}}%
\pgfpathlineto{\pgfqpoint{1.090613in}{3.466295in}}%
\pgfpathlineto{\pgfqpoint{1.157663in}{3.479958in}}%
\pgfpathlineto{\pgfqpoint{1.224643in}{3.493968in}}%
\pgfpathlineto{\pgfqpoint{1.291597in}{3.508097in}}%
\pgfpathlineto{\pgfqpoint{1.358578in}{3.522098in}}%
\pgfpathlineto{\pgfqpoint{1.425638in}{3.535715in}}%
\pgfpathlineto{\pgfqpoint{1.492824in}{3.548689in}}%
\pgfpathlineto{\pgfqpoint{1.560176in}{3.560772in}}%
\pgfpathlineto{\pgfqpoint{1.627716in}{3.571741in}}%
\pgfpathlineto{\pgfqpoint{1.627716in}{3.571741in}}%
\pgfusepath{stroke}%
\end{pgfscope}%
\begin{pgfscope}%
\pgfpathrectangle{\pgfqpoint{0.647939in}{0.492442in}}{\pgfqpoint{3.079299in}{3.079299in}}%
\pgfusepath{clip}%
\pgfsetbuttcap%
\pgfsetroundjoin%
\pgfsetlinewidth{0.301125pt}%
\definecolor{currentstroke}{rgb}{0.500000,0.500000,0.500000}%
\pgfsetstrokecolor{currentstroke}%
\pgfsetstrokeopacity{0.300000}%
\pgfsetdash{}{0pt}%
\pgfpathmoveto{\pgfqpoint{0.647939in}{3.465187in}}%
\pgfpathlineto{\pgfqpoint{0.671071in}{3.467716in}}%
\pgfpathlineto{\pgfqpoint{0.739029in}{3.475703in}}%
\pgfpathlineto{\pgfqpoint{0.806851in}{3.484782in}}%
\pgfpathlineto{\pgfqpoint{0.874522in}{3.494921in}}%
\pgfpathlineto{\pgfqpoint{0.942038in}{3.506049in}}%
\pgfpathlineto{\pgfqpoint{1.009403in}{3.518063in}}%
\pgfpathlineto{\pgfqpoint{1.076630in}{3.530826in}}%
\pgfpathlineto{\pgfqpoint{1.143745in}{3.544165in}}%
\pgfpathlineto{\pgfqpoint{1.210786in}{3.557879in}}%
\pgfpathlineto{\pgfqpoint{1.277796in}{3.571741in}}%
\pgfpathlineto{\pgfqpoint{1.277796in}{3.571741in}}%
\pgfusepath{stroke}%
\end{pgfscope}%
\begin{pgfscope}%
\pgfpathrectangle{\pgfqpoint{0.647939in}{0.492442in}}{\pgfqpoint{3.079299in}{3.079299in}}%
\pgfusepath{clip}%
\pgfsetbuttcap%
\pgfsetroundjoin%
\pgfsetlinewidth{0.301125pt}%
\definecolor{currentstroke}{rgb}{0.500000,0.500000,0.500000}%
\pgfsetstrokecolor{currentstroke}%
\pgfsetstrokeopacity{0.300000}%
\pgfsetdash{}{0pt}%
\pgfpathmoveto{\pgfqpoint{0.647939in}{2.871901in}}%
\pgfpathlineto{\pgfqpoint{0.647939in}{2.871901in}}%
\pgfpathlineto{\pgfqpoint{0.715845in}{2.880321in}}%
\pgfpathlineto{\pgfqpoint{0.783579in}{2.890023in}}%
\pgfpathlineto{\pgfqpoint{0.851116in}{2.901006in}}%
\pgfpathlineto{\pgfqpoint{0.918440in}{2.913234in}}%
\pgfpathlineto{\pgfqpoint{0.985543in}{2.926627in}}%
\pgfpathlineto{\pgfqpoint{1.052430in}{2.941064in}}%
\pgfpathlineto{\pgfqpoint{1.119121in}{2.956379in}}%
\pgfusepath{stroke}%
\end{pgfscope}%
\begin{pgfscope}%
\pgfpathrectangle{\pgfqpoint{0.647939in}{0.492442in}}{\pgfqpoint{3.079299in}{3.079299in}}%
\pgfusepath{clip}%
\pgfsetbuttcap%
\pgfsetroundjoin%
\pgfsetlinewidth{0.301125pt}%
\definecolor{currentstroke}{rgb}{0.500000,0.500000,0.500000}%
\pgfsetstrokecolor{currentstroke}%
\pgfsetstrokeopacity{0.300000}%
\pgfsetdash{}{0pt}%
\pgfpathmoveto{\pgfqpoint{0.647939in}{2.731932in}}%
\pgfpathlineto{\pgfqpoint{0.647939in}{2.731932in}}%
\pgfpathlineto{\pgfqpoint{0.715817in}{2.740568in}}%
\pgfpathlineto{\pgfqpoint{0.783512in}{2.750537in}}%
\pgfpathlineto{\pgfqpoint{0.850996in}{2.761848in}}%
\pgfpathlineto{\pgfqpoint{0.918247in}{2.774467in}}%
\pgfpathlineto{\pgfqpoint{0.985256in}{2.788321in}}%
\pgfpathlineto{\pgfqpoint{1.052025in}{2.803290in}}%
\pgfpathlineto{\pgfqpoint{1.118574in}{2.819213in}}%
\pgfpathlineto{\pgfqpoint{1.184940in}{2.835883in}}%
\pgfpathlineto{\pgfqpoint{1.251175in}{2.853068in}}%
\pgfpathlineto{\pgfqpoint{1.317345in}{2.870504in}}%
\pgfpathlineto{\pgfqpoint{1.383526in}{2.887901in}}%
\pgfpathlineto{\pgfqpoint{1.449796in}{2.904951in}}%
\pgfpathlineto{\pgfqpoint{1.516232in}{2.921338in}}%
\pgfpathlineto{\pgfqpoint{1.582902in}{2.936742in}}%
\pgfpathlineto{\pgfqpoint{1.649854in}{2.950859in}}%
\pgfpathlineto{\pgfqpoint{1.717114in}{2.963420in}}%
\pgfpathlineto{\pgfqpoint{1.784679in}{2.974212in}}%
\pgfusepath{stroke}%
\end{pgfscope}%
\begin{pgfscope}%
\pgfpathrectangle{\pgfqpoint{0.647939in}{0.492442in}}{\pgfqpoint{3.079299in}{3.079299in}}%
\pgfusepath{clip}%
\pgfsetbuttcap%
\pgfsetroundjoin%
\pgfsetlinewidth{0.301125pt}%
\definecolor{currentstroke}{rgb}{0.500000,0.500000,0.500000}%
\pgfsetstrokecolor{currentstroke}%
\pgfsetstrokeopacity{0.300000}%
\pgfsetdash{}{0pt}%
\pgfpathmoveto{\pgfqpoint{0.647939in}{2.661948in}}%
\pgfpathlineto{\pgfqpoint{0.647939in}{2.661948in}}%
\pgfpathlineto{\pgfqpoint{0.715803in}{2.670696in}}%
\pgfpathlineto{\pgfqpoint{0.783477in}{2.680804in}}%
\pgfpathlineto{\pgfqpoint{0.850931in}{2.692286in}}%
\pgfpathlineto{\pgfqpoint{0.918144in}{2.705110in}}%
\pgfpathlineto{\pgfqpoint{0.985102in}{2.719205in}}%
\pgfpathlineto{\pgfqpoint{1.051807in}{2.734456in}}%
\pgfpathlineto{\pgfqpoint{1.118278in}{2.750698in}}%
\pgfusepath{stroke}%
\end{pgfscope}%
\begin{pgfscope}%
\pgfpathrectangle{\pgfqpoint{0.647939in}{0.492442in}}{\pgfqpoint{3.079299in}{3.079299in}}%
\pgfusepath{clip}%
\pgfsetbuttcap%
\pgfsetroundjoin%
\pgfsetlinewidth{0.301125pt}%
\definecolor{currentstroke}{rgb}{0.500000,0.500000,0.500000}%
\pgfsetstrokecolor{currentstroke}%
\pgfsetstrokeopacity{0.300000}%
\pgfsetdash{}{0pt}%
\pgfpathmoveto{\pgfqpoint{0.647939in}{2.591964in}}%
\pgfpathlineto{\pgfqpoint{0.647939in}{2.591964in}}%
\pgfpathlineto{\pgfqpoint{0.715788in}{2.600826in}}%
\pgfpathlineto{\pgfqpoint{0.783440in}{2.611078in}}%
\pgfpathlineto{\pgfqpoint{0.850864in}{2.622735in}}%
\pgfpathlineto{\pgfqpoint{0.918035in}{2.635771in}}%
\pgfpathlineto{\pgfqpoint{0.984940in}{2.650116in}}%
\pgfpathlineto{\pgfqpoint{1.051578in}{2.665658in}}%
\pgfpathlineto{\pgfqpoint{1.117967in}{2.682233in}}%
\pgfpathlineto{\pgfqpoint{1.184144in}{2.699640in}}%
\pgfpathlineto{\pgfqpoint{1.250162in}{2.717641in}}%
\pgfpathlineto{\pgfqpoint{1.316090in}{2.735970in}}%
\pgfpathlineto{\pgfqpoint{1.382010in}{2.754329in}}%
\pgfpathlineto{\pgfqpoint{1.448008in}{2.772403in}}%
\pgfpathlineto{\pgfqpoint{1.514172in}{2.789858in}}%
\pgfpathlineto{\pgfqpoint{1.580579in}{2.806360in}}%
\pgfpathlineto{\pgfqpoint{1.647289in}{2.821579in}}%
\pgfpathlineto{\pgfqpoint{1.714339in}{2.835216in}}%
\pgfpathlineto{\pgfqpoint{1.781733in}{2.847024in}}%
\pgfpathlineto{\pgfqpoint{1.849445in}{2.856835in}}%
\pgfpathlineto{\pgfqpoint{1.917424in}{2.864591in}}%
\pgfpathlineto{\pgfqpoint{1.985601in}{2.870368in}}%
\pgfpathlineto{\pgfqpoint{2.053905in}{2.874428in}}%
\pgfpathlineto{\pgfqpoint{2.122275in}{2.877219in}}%
\pgfpathlineto{\pgfqpoint{2.190669in}{2.879387in}}%
\pgfpathlineto{\pgfqpoint{2.259055in}{2.881784in}}%
\pgfpathlineto{\pgfqpoint{2.327378in}{2.885470in}}%
\pgfpathlineto{\pgfqpoint{2.395502in}{2.891688in}}%
\pgfpathlineto{\pgfqpoint{2.463146in}{2.901723in}}%
\pgfpathlineto{\pgfqpoint{2.529851in}{2.916665in}}%
\pgfpathlineto{\pgfqpoint{2.595045in}{2.937161in}}%
\pgfpathlineto{\pgfqpoint{2.658210in}{2.963244in}}%
\pgfpathlineto{\pgfqpoint{2.719048in}{2.994394in}}%
\pgfpathlineto{\pgfqpoint{2.777546in}{3.029776in}}%
\pgfpathlineto{\pgfqpoint{2.833912in}{3.068495in}}%
\pgfusepath{stroke}%
\end{pgfscope}%
\begin{pgfscope}%
\pgfpathrectangle{\pgfqpoint{0.647939in}{0.492442in}}{\pgfqpoint{3.079299in}{3.079299in}}%
\pgfusepath{clip}%
\pgfsetbuttcap%
\pgfsetroundjoin%
\pgfsetlinewidth{0.301125pt}%
\definecolor{currentstroke}{rgb}{0.500000,0.500000,0.500000}%
\pgfsetstrokecolor{currentstroke}%
\pgfsetstrokeopacity{0.300000}%
\pgfsetdash{}{0pt}%
\pgfpathmoveto{\pgfqpoint{0.647939in}{2.382012in}}%
\pgfpathlineto{\pgfqpoint{0.647939in}{2.382012in}}%
\pgfpathlineto{\pgfqpoint{0.715739in}{2.391236in}}%
\pgfpathlineto{\pgfqpoint{0.783320in}{2.401943in}}%
\pgfpathlineto{\pgfqpoint{0.850644in}{2.414160in}}%
\pgfpathlineto{\pgfqpoint{0.917680in}{2.427873in}}%
\pgfpathlineto{\pgfqpoint{0.984406in}{2.443022in}}%
\pgfpathlineto{\pgfqpoint{1.050817in}{2.459502in}}%
\pgfpathlineto{\pgfqpoint{1.116927in}{2.477157in}}%
\pgfpathlineto{\pgfqpoint{1.182769in}{2.495786in}}%
\pgfpathlineto{\pgfqpoint{1.248399in}{2.515153in}}%
\pgfpathlineto{\pgfqpoint{1.313891in}{2.534986in}}%
\pgfpathlineto{\pgfqpoint{1.379333in}{2.554981in}}%
\pgfpathlineto{\pgfqpoint{1.444826in}{2.574808in}}%
\pgfpathlineto{\pgfqpoint{1.510473in}{2.594117in}}%
\pgfpathlineto{\pgfqpoint{1.576371in}{2.612545in}}%
\pgfpathlineto{\pgfqpoint{1.642603in}{2.629728in}}%
\pgfpathlineto{\pgfqpoint{1.709226in}{2.645315in}}%
\pgfpathlineto{\pgfqpoint{1.776263in}{2.659002in}}%
\pgfpathlineto{\pgfqpoint{1.843698in}{2.670556in}}%
\pgfpathlineto{\pgfqpoint{1.911481in}{2.679853in}}%
\pgfpathlineto{\pgfqpoint{1.979534in}{2.686920in}}%
\pgfpathlineto{\pgfqpoint{2.047768in}{2.691977in}}%
\pgfpathlineto{\pgfqpoint{2.116104in}{2.695497in}}%
\pgfpathlineto{\pgfqpoint{2.184478in}{2.698227in}}%
\pgfpathlineto{\pgfqpoint{2.252840in}{2.701207in}}%
\pgfpathlineto{\pgfqpoint{2.321103in}{2.705799in}}%
\pgfpathlineto{\pgfqpoint{2.389047in}{2.713642in}}%
\pgfpathlineto{\pgfqpoint{2.456189in}{2.726463in}}%
\pgfpathlineto{\pgfqpoint{2.521765in}{2.745600in}}%
\pgfpathlineto{\pgfqpoint{2.584959in}{2.771493in}}%
\pgfpathlineto{\pgfqpoint{2.645250in}{2.803602in}}%
\pgfusepath{stroke}%
\end{pgfscope}%
\begin{pgfscope}%
\pgfpathrectangle{\pgfqpoint{0.647939in}{0.492442in}}{\pgfqpoint{3.079299in}{3.079299in}}%
\pgfusepath{clip}%
\pgfsetbuttcap%
\pgfsetroundjoin%
\pgfsetlinewidth{0.301125pt}%
\definecolor{currentstroke}{rgb}{0.500000,0.500000,0.500000}%
\pgfsetstrokecolor{currentstroke}%
\pgfsetstrokeopacity{0.300000}%
\pgfsetdash{}{0pt}%
\pgfpathmoveto{\pgfqpoint{0.647939in}{2.312028in}}%
\pgfpathlineto{\pgfqpoint{0.647939in}{2.312028in}}%
\pgfpathlineto{\pgfqpoint{0.715721in}{2.321380in}}%
\pgfpathlineto{\pgfqpoint{0.783276in}{2.332247in}}%
\pgfpathlineto{\pgfqpoint{0.850564in}{2.344663in}}%
\pgfpathlineto{\pgfqpoint{0.917550in}{2.358616in}}%
\pgfpathlineto{\pgfqpoint{0.984210in}{2.374053in}}%
\pgfpathlineto{\pgfqpoint{1.050536in}{2.390869in}}%
\pgfpathlineto{\pgfqpoint{1.116540in}{2.408913in}}%
\pgfpathlineto{\pgfqpoint{1.182255in}{2.427986in}}%
\pgfpathlineto{\pgfqpoint{1.247736in}{2.447851in}}%
\pgfpathlineto{\pgfqpoint{1.313058in}{2.468234in}}%
\pgfusepath{stroke}%
\end{pgfscope}%
\begin{pgfscope}%
\pgfpathrectangle{\pgfqpoint{0.647939in}{0.492442in}}{\pgfqpoint{3.079299in}{3.079299in}}%
\pgfusepath{clip}%
\pgfsetbuttcap%
\pgfsetroundjoin%
\pgfsetlinewidth{0.301125pt}%
\definecolor{currentstroke}{rgb}{0.500000,0.500000,0.500000}%
\pgfsetstrokecolor{currentstroke}%
\pgfsetstrokeopacity{0.300000}%
\pgfsetdash{}{0pt}%
\pgfpathmoveto{\pgfqpoint{0.647939in}{2.242044in}}%
\pgfpathlineto{\pgfqpoint{0.647939in}{2.242044in}}%
\pgfpathlineto{\pgfqpoint{0.715703in}{2.251527in}}%
\pgfpathlineto{\pgfqpoint{0.783231in}{2.262560in}}%
\pgfpathlineto{\pgfqpoint{0.850480in}{2.275180in}}%
\pgfpathlineto{\pgfqpoint{0.917413in}{2.289383in}}%
\pgfpathlineto{\pgfqpoint{0.984003in}{2.305117in}}%
\pgfpathlineto{\pgfqpoint{1.050239in}{2.322283in}}%
\pgfpathlineto{\pgfqpoint{1.116131in}{2.340732in}}%
\pgfpathlineto{\pgfqpoint{1.181710in}{2.360268in}}%
\pgfpathlineto{\pgfqpoint{1.247030in}{2.380653in}}%
\pgfpathlineto{\pgfqpoint{1.312169in}{2.401615in}}%
\pgfusepath{stroke}%
\end{pgfscope}%
\begin{pgfscope}%
\pgfpathrectangle{\pgfqpoint{0.647939in}{0.492442in}}{\pgfqpoint{3.079299in}{3.079299in}}%
\pgfusepath{clip}%
\pgfsetbuttcap%
\pgfsetroundjoin%
\pgfsetlinewidth{0.301125pt}%
\definecolor{currentstroke}{rgb}{0.500000,0.500000,0.500000}%
\pgfsetstrokecolor{currentstroke}%
\pgfsetstrokeopacity{0.300000}%
\pgfsetdash{}{0pt}%
\pgfpathmoveto{\pgfqpoint{0.647939in}{2.172060in}}%
\pgfpathlineto{\pgfqpoint{0.647939in}{2.172060in}}%
\pgfpathlineto{\pgfqpoint{0.715684in}{2.181677in}}%
\pgfpathlineto{\pgfqpoint{0.783183in}{2.192881in}}%
\pgfpathlineto{\pgfqpoint{0.850392in}{2.205712in}}%
\pgfpathlineto{\pgfqpoint{0.917269in}{2.220173in}}%
\pgfpathlineto{\pgfqpoint{0.983785in}{2.236216in}}%
\pgfpathlineto{\pgfqpoint{1.049926in}{2.253746in}}%
\pgfpathlineto{\pgfqpoint{1.115698in}{2.272617in}}%
\pgfpathlineto{\pgfqpoint{1.181130in}{2.292636in}}%
\pgfpathlineto{\pgfqpoint{1.246278in}{2.313566in}}%
\pgfpathlineto{\pgfqpoint{1.311217in}{2.335136in}}%
\pgfpathlineto{\pgfqpoint{1.376046in}{2.357039in}}%
\pgfpathlineto{\pgfqpoint{1.440876in}{2.378937in}}%
\pgfpathlineto{\pgfqpoint{1.505829in}{2.400466in}}%
\pgfpathlineto{\pgfqpoint{1.571026in}{2.421239in}}%
\pgfpathlineto{\pgfqpoint{1.636578in}{2.440858in}}%
\pgfpathlineto{\pgfqpoint{1.702572in}{2.458924in}}%
\pgfpathlineto{\pgfqpoint{1.769061in}{2.475063in}}%
\pgfpathlineto{\pgfqpoint{1.836052in}{2.488959in}}%
\pgfpathlineto{\pgfqpoint{1.903503in}{2.500400in}}%
\pgfpathlineto{\pgfqpoint{1.971333in}{2.509329in}}%
\pgfpathlineto{\pgfqpoint{2.039434in}{2.515908in}}%
\pgfpathlineto{\pgfqpoint{2.107697in}{2.520586in}}%
\pgfpathlineto{\pgfqpoint{2.176030in}{2.524202in}}%
\pgfpathlineto{\pgfqpoint{2.244346in}{2.528083in}}%
\pgfpathlineto{\pgfqpoint{2.312485in}{2.534099in}}%
\pgfpathlineto{\pgfqpoint{2.380020in}{2.544626in}}%
\pgfpathlineto{\pgfqpoint{2.446028in}{2.562062in}}%
\pgfpathlineto{\pgfqpoint{2.509235in}{2.587739in}}%
\pgfpathlineto{\pgfqpoint{2.568716in}{2.621177in}}%
\pgfpathlineto{\pgfqpoint{2.624407in}{2.660668in}}%
\pgfpathlineto{\pgfqpoint{2.676911in}{2.704394in}}%
\pgfusepath{stroke}%
\end{pgfscope}%
\begin{pgfscope}%
\pgfpathrectangle{\pgfqpoint{0.647939in}{0.492442in}}{\pgfqpoint{3.079299in}{3.079299in}}%
\pgfusepath{clip}%
\pgfsetbuttcap%
\pgfsetroundjoin%
\pgfsetlinewidth{0.301125pt}%
\definecolor{currentstroke}{rgb}{0.500000,0.500000,0.500000}%
\pgfsetstrokecolor{currentstroke}%
\pgfsetstrokeopacity{0.300000}%
\pgfsetdash{}{0pt}%
\pgfpathmoveto{\pgfqpoint{0.647939in}{2.102076in}}%
\pgfpathlineto{\pgfqpoint{0.647939in}{2.102076in}}%
\pgfpathlineto{\pgfqpoint{0.715664in}{2.111831in}}%
\pgfpathlineto{\pgfqpoint{0.783134in}{2.123211in}}%
\pgfpathlineto{\pgfqpoint{0.850300in}{2.136261in}}%
\pgfpathlineto{\pgfqpoint{0.917118in}{2.150988in}}%
\pgfpathlineto{\pgfqpoint{0.983556in}{2.167351in}}%
\pgfpathlineto{\pgfqpoint{1.049595in}{2.185260in}}%
\pgfpathlineto{\pgfqpoint{1.115238in}{2.204572in}}%
\pgfpathlineto{\pgfqpoint{1.180514in}{2.225096in}}%
\pgfpathlineto{\pgfqpoint{1.245475in}{2.246597in}}%
\pgfpathlineto{\pgfqpoint{1.310198in}{2.268806in}}%
\pgfpathlineto{\pgfqpoint{1.374784in}{2.291415in}}%
\pgfpathlineto{\pgfqpoint{1.439348in}{2.314086in}}%
\pgfpathlineto{\pgfqpoint{1.504018in}{2.336450in}}%
\pgfpathlineto{\pgfqpoint{1.568924in}{2.358117in}}%
\pgfpathlineto{\pgfqpoint{1.634187in}{2.378677in}}%
\pgfpathlineto{\pgfqpoint{1.699906in}{2.397717in}}%
\pgfpathlineto{\pgfqpoint{1.766149in}{2.414839in}}%
\pgfpathlineto{\pgfqpoint{1.832933in}{2.429694in}}%
\pgfpathlineto{\pgfqpoint{1.900225in}{2.442036in}}%
\pgfpathlineto{\pgfqpoint{1.967943in}{2.451769in}}%
\pgfpathlineto{\pgfqpoint{2.035973in}{2.459026in}}%
\pgfpathlineto{\pgfqpoint{2.104196in}{2.464246in}}%
\pgfpathlineto{\pgfqpoint{2.172505in}{2.468284in}}%
\pgfpathlineto{\pgfqpoint{2.240795in}{2.472578in}}%
\pgfpathlineto{\pgfqpoint{2.308866in}{2.479261in}}%
\pgfusepath{stroke}%
\end{pgfscope}%
\begin{pgfscope}%
\pgfpathrectangle{\pgfqpoint{0.647939in}{0.492442in}}{\pgfqpoint{3.079299in}{3.079299in}}%
\pgfusepath{clip}%
\pgfsetbuttcap%
\pgfsetroundjoin%
\pgfsetlinewidth{0.301125pt}%
\definecolor{currentstroke}{rgb}{0.500000,0.500000,0.500000}%
\pgfsetstrokecolor{currentstroke}%
\pgfsetstrokeopacity{0.300000}%
\pgfsetdash{}{0pt}%
\pgfpathmoveto{\pgfqpoint{0.647939in}{2.032092in}}%
\pgfpathlineto{\pgfqpoint{0.647939in}{2.032092in}}%
\pgfpathlineto{\pgfqpoint{0.715643in}{2.041990in}}%
\pgfpathlineto{\pgfqpoint{0.783082in}{2.053550in}}%
\pgfpathlineto{\pgfqpoint{0.850203in}{2.066826in}}%
\pgfpathlineto{\pgfqpoint{0.916960in}{2.081830in}}%
\pgfpathlineto{\pgfqpoint{0.983314in}{2.098526in}}%
\pgfpathlineto{\pgfqpoint{1.049244in}{2.116828in}}%
\pgfpathlineto{\pgfqpoint{1.114750in}{2.136599in}}%
\pgfpathlineto{\pgfqpoint{1.179857in}{2.157651in}}%
\pgfpathlineto{\pgfqpoint{1.244616in}{2.179752in}}%
\pgfpathlineto{\pgfqpoint{1.309105in}{2.202632in}}%
\pgfpathlineto{\pgfqpoint{1.373425in}{2.225987in}}%
\pgfpathlineto{\pgfqpoint{1.437696in}{2.249476in}}%
\pgfpathlineto{\pgfqpoint{1.502051in}{2.272730in}}%
\pgfpathlineto{\pgfqpoint{1.566630in}{2.295354in}}%
\pgfpathlineto{\pgfqpoint{1.631565in}{2.316929in}}%
\pgfpathlineto{\pgfqpoint{1.696968in}{2.337029in}}%
\pgfusepath{stroke}%
\end{pgfscope}%
\begin{pgfscope}%
\pgfpathrectangle{\pgfqpoint{0.647939in}{0.492442in}}{\pgfqpoint{3.079299in}{3.079299in}}%
\pgfusepath{clip}%
\pgfsetbuttcap%
\pgfsetroundjoin%
\pgfsetlinewidth{0.301125pt}%
\definecolor{currentstroke}{rgb}{0.500000,0.500000,0.500000}%
\pgfsetstrokecolor{currentstroke}%
\pgfsetstrokeopacity{0.300000}%
\pgfsetdash{}{0pt}%
\pgfpathmoveto{\pgfqpoint{0.647939in}{1.892124in}}%
\pgfpathlineto{\pgfqpoint{0.647939in}{1.892124in}}%
\pgfpathlineto{\pgfqpoint{0.715598in}{1.902319in}}%
\pgfpathlineto{\pgfqpoint{0.782970in}{1.914259in}}%
\pgfpathlineto{\pgfqpoint{0.849995in}{1.928010in}}%
\pgfpathlineto{\pgfqpoint{0.916617in}{1.943598in}}%
\pgfpathlineto{\pgfqpoint{0.982788in}{1.960998in}}%
\pgfpathlineto{\pgfqpoint{1.048479in}{1.980137in}}%
\pgfpathlineto{\pgfqpoint{1.113681in}{2.000887in}}%
\pgfpathlineto{\pgfqpoint{1.178410in}{2.023070in}}%
\pgfpathlineto{\pgfqpoint{1.242715in}{2.046459in}}%
\pgfpathlineto{\pgfqpoint{1.306670in}{2.070790in}}%
\pgfpathlineto{\pgfqpoint{1.370378in}{2.095762in}}%
\pgfpathlineto{\pgfqpoint{1.433967in}{2.121039in}}%
\pgfpathlineto{\pgfqpoint{1.497582in}{2.146250in}}%
\pgfpathlineto{\pgfqpoint{1.561378in}{2.170997in}}%
\pgfpathlineto{\pgfqpoint{1.625512in}{2.194849in}}%
\pgfpathlineto{\pgfqpoint{1.690126in}{2.217358in}}%
\pgfpathlineto{\pgfqpoint{1.755336in}{2.238065in}}%
\pgfpathlineto{\pgfqpoint{1.821211in}{2.256530in}}%
\pgfpathlineto{\pgfqpoint{1.887763in}{2.272373in}}%
\pgfpathlineto{\pgfqpoint{1.954933in}{2.285343in}}%
\pgfpathlineto{\pgfqpoint{2.022597in}{2.295436in}}%
\pgfpathlineto{\pgfqpoint{2.090594in}{2.303023in}}%
\pgfpathlineto{\pgfqpoint{2.158755in}{2.309040in}}%
\pgfpathlineto{\pgfqpoint{2.226888in}{2.315334in}}%
\pgfpathlineto{\pgfqpoint{2.294540in}{2.325167in}}%
\pgfpathlineto{\pgfqpoint{2.360255in}{2.343196in}}%
\pgfpathlineto{\pgfqpoint{2.421520in}{2.372534in}}%
\pgfusepath{stroke}%
\end{pgfscope}%
\begin{pgfscope}%
\pgfpathrectangle{\pgfqpoint{0.647939in}{0.492442in}}{\pgfqpoint{3.079299in}{3.079299in}}%
\pgfusepath{clip}%
\pgfsetbuttcap%
\pgfsetroundjoin%
\pgfsetlinewidth{0.301125pt}%
\definecolor{currentstroke}{rgb}{0.500000,0.500000,0.500000}%
\pgfsetstrokecolor{currentstroke}%
\pgfsetstrokeopacity{0.300000}%
\pgfsetdash{}{0pt}%
\pgfpathmoveto{\pgfqpoint{0.647939in}{1.822139in}}%
\pgfpathlineto{\pgfqpoint{0.647939in}{1.822139in}}%
\pgfpathlineto{\pgfqpoint{0.715574in}{1.832491in}}%
\pgfpathlineto{\pgfqpoint{0.782910in}{1.844630in}}%
\pgfpathlineto{\pgfqpoint{0.849883in}{1.858631in}}%
\pgfpathlineto{\pgfqpoint{0.916431in}{1.874527in}}%
\pgfpathlineto{\pgfqpoint{0.982503in}{1.892301in}}%
\pgfpathlineto{\pgfqpoint{1.048062in}{1.911884in}}%
\pgfpathlineto{\pgfqpoint{1.113094in}{1.933157in}}%
\pgfpathlineto{\pgfqpoint{1.177612in}{1.955945in}}%
\pgfpathlineto{\pgfqpoint{1.241660in}{1.980027in}}%
\pgfpathlineto{\pgfqpoint{1.305311in}{2.005142in}}%
\pgfpathlineto{\pgfqpoint{1.368668in}{2.030991in}}%
\pgfpathlineto{\pgfqpoint{1.431861in}{2.057242in}}%
\pgfpathlineto{\pgfqpoint{1.495039in}{2.083528in}}%
\pgfpathlineto{\pgfqpoint{1.558367in}{2.109450in}}%
\pgfpathlineto{\pgfqpoint{1.622012in}{2.134578in}}%
\pgfpathlineto{\pgfqpoint{1.686135in}{2.158453in}}%
\pgfpathlineto{\pgfqpoint{1.750869in}{2.180603in}}%
\pgfpathlineto{\pgfqpoint{1.816307in}{2.200563in}}%
\pgfusepath{stroke}%
\end{pgfscope}%
\begin{pgfscope}%
\pgfpathrectangle{\pgfqpoint{0.647939in}{0.492442in}}{\pgfqpoint{3.079299in}{3.079299in}}%
\pgfusepath{clip}%
\pgfsetbuttcap%
\pgfsetroundjoin%
\pgfsetlinewidth{0.301125pt}%
\definecolor{currentstroke}{rgb}{0.500000,0.500000,0.500000}%
\pgfsetstrokecolor{currentstroke}%
\pgfsetstrokeopacity{0.300000}%
\pgfsetdash{}{0pt}%
\pgfpathmoveto{\pgfqpoint{0.647939in}{1.752155in}}%
\pgfpathlineto{\pgfqpoint{0.647939in}{1.752155in}}%
\pgfpathlineto{\pgfqpoint{0.715549in}{1.762667in}}%
\pgfpathlineto{\pgfqpoint{0.782847in}{1.775012in}}%
\pgfpathlineto{\pgfqpoint{0.849765in}{1.789272in}}%
\pgfpathlineto{\pgfqpoint{0.916235in}{1.805488in}}%
\pgfpathlineto{\pgfqpoint{0.982200in}{1.823650in}}%
\pgfpathlineto{\pgfqpoint{1.047618in}{1.843698in}}%
\pgfpathlineto{\pgfqpoint{1.112468in}{1.865516in}}%
\pgfpathlineto{\pgfqpoint{1.176758in}{1.888938in}}%
\pgfpathlineto{\pgfqpoint{1.240527in}{1.913748in}}%
\pgfpathlineto{\pgfqpoint{1.303846in}{1.939688in}}%
\pgfpathlineto{\pgfqpoint{1.366817in}{1.966464in}}%
\pgfpathlineto{\pgfqpoint{1.429570in}{1.993749in}}%
\pgfpathlineto{\pgfqpoint{1.492260in}{2.021179in}}%
\pgfpathlineto{\pgfqpoint{1.555058in}{2.048358in}}%
\pgfusepath{stroke}%
\end{pgfscope}%
\begin{pgfscope}%
\pgfpathrectangle{\pgfqpoint{0.647939in}{0.492442in}}{\pgfqpoint{3.079299in}{3.079299in}}%
\pgfusepath{clip}%
\pgfsetbuttcap%
\pgfsetroundjoin%
\pgfsetlinewidth{0.301125pt}%
\definecolor{currentstroke}{rgb}{0.500000,0.500000,0.500000}%
\pgfsetstrokecolor{currentstroke}%
\pgfsetstrokeopacity{0.300000}%
\pgfsetdash{}{0pt}%
\pgfpathmoveto{\pgfqpoint{0.647939in}{1.682171in}}%
\pgfpathlineto{\pgfqpoint{0.647939in}{1.682171in}}%
\pgfpathlineto{\pgfqpoint{0.715523in}{1.692848in}}%
\pgfpathlineto{\pgfqpoint{0.782781in}{1.705405in}}%
\pgfpathlineto{\pgfqpoint{0.849640in}{1.719934in}}%
\pgfpathlineto{\pgfqpoint{0.916028in}{1.736482in}}%
\pgfpathlineto{\pgfqpoint{0.981880in}{1.755050in}}%
\pgfpathlineto{\pgfqpoint{1.047146in}{1.775582in}}%
\pgfpathlineto{\pgfqpoint{1.111800in}{1.797971in}}%
\pgfpathlineto{\pgfqpoint{1.175844in}{1.822057in}}%
\pgfpathlineto{\pgfqpoint{1.239310in}{1.847630in}}%
\pgfpathlineto{\pgfqpoint{1.302265in}{1.874439in}}%
\pgfpathlineto{\pgfqpoint{1.364810in}{1.902194in}}%
\pgfusepath{stroke}%
\end{pgfscope}%
\begin{pgfscope}%
\pgfpathrectangle{\pgfqpoint{0.647939in}{0.492442in}}{\pgfqpoint{3.079299in}{3.079299in}}%
\pgfusepath{clip}%
\pgfsetbuttcap%
\pgfsetroundjoin%
\pgfsetlinewidth{0.301125pt}%
\definecolor{currentstroke}{rgb}{0.500000,0.500000,0.500000}%
\pgfsetstrokecolor{currentstroke}%
\pgfsetstrokeopacity{0.300000}%
\pgfsetdash{}{0pt}%
\pgfpathmoveto{\pgfqpoint{0.647939in}{1.612187in}}%
\pgfpathlineto{\pgfqpoint{0.647939in}{1.612187in}}%
\pgfpathlineto{\pgfqpoint{0.715496in}{1.623035in}}%
\pgfpathlineto{\pgfqpoint{0.782712in}{1.635812in}}%
\pgfpathlineto{\pgfqpoint{0.849509in}{1.650618in}}%
\pgfpathlineto{\pgfqpoint{0.915809in}{1.667512in}}%
\pgfpathlineto{\pgfqpoint{0.981539in}{1.686501in}}%
\pgfpathlineto{\pgfqpoint{1.046643in}{1.707539in}}%
\pgfpathlineto{\pgfqpoint{1.111087in}{1.730525in}}%
\pgfpathlineto{\pgfqpoint{1.174863in}{1.755307in}}%
\pgfpathlineto{\pgfqpoint{1.238000in}{1.781681in}}%
\pgfpathlineto{\pgfqpoint{1.300557in}{1.809402in}}%
\pgfpathlineto{\pgfqpoint{1.362634in}{1.838189in}}%
\pgfpathlineto{\pgfqpoint{1.424360in}{1.867723in}}%
\pgfpathlineto{\pgfqpoint{1.485894in}{1.897655in}}%
\pgfpathlineto{\pgfqpoint{1.547420in}{1.927604in}}%
\pgfpathlineto{\pgfqpoint{1.609137in}{1.957155in}}%
\pgfpathlineto{\pgfqpoint{1.671252in}{1.985857in}}%
\pgfpathlineto{\pgfqpoint{1.733963in}{2.013224in}}%
\pgfpathlineto{\pgfqpoint{1.797444in}{2.038743in}}%
\pgfpathlineto{\pgfqpoint{1.861818in}{2.061896in}}%
\pgfpathlineto{\pgfqpoint{1.927132in}{2.082230in}}%
\pgfpathlineto{\pgfqpoint{1.993333in}{2.099455in}}%
\pgfpathlineto{\pgfqpoint{2.060251in}{2.113652in}}%
\pgfpathlineto{\pgfqpoint{2.127600in}{2.125712in}}%
\pgfpathlineto{\pgfqpoint{2.194802in}{2.138443in}}%
\pgfpathlineto{\pgfqpoint{2.259504in}{2.159055in}}%
\pgfpathlineto{\pgfqpoint{2.259504in}{2.159055in}}%
\pgfusepath{stroke}%
\end{pgfscope}%
\begin{pgfscope}%
\pgfpathrectangle{\pgfqpoint{0.647939in}{0.492442in}}{\pgfqpoint{3.079299in}{3.079299in}}%
\pgfusepath{clip}%
\pgfsetbuttcap%
\pgfsetroundjoin%
\pgfsetlinewidth{0.301125pt}%
\definecolor{currentstroke}{rgb}{0.500000,0.500000,0.500000}%
\pgfsetstrokecolor{currentstroke}%
\pgfsetstrokeopacity{0.300000}%
\pgfsetdash{}{0pt}%
\pgfpathmoveto{\pgfqpoint{0.647939in}{1.542203in}}%
\pgfpathlineto{\pgfqpoint{0.647939in}{1.542203in}}%
\pgfpathlineto{\pgfqpoint{0.715467in}{1.553227in}}%
\pgfpathlineto{\pgfqpoint{0.782639in}{1.566231in}}%
\pgfpathlineto{\pgfqpoint{0.849371in}{1.581326in}}%
\pgfpathlineto{\pgfqpoint{0.915577in}{1.598579in}}%
\pgfpathlineto{\pgfqpoint{0.981178in}{1.618008in}}%
\pgfpathlineto{\pgfqpoint{1.046108in}{1.639574in}}%
\pgfpathlineto{\pgfqpoint{1.110324in}{1.663186in}}%
\pgfpathlineto{\pgfqpoint{1.173810in}{1.688696in}}%
\pgfpathlineto{\pgfqpoint{1.236588in}{1.715910in}}%
\pgfpathlineto{\pgfqpoint{1.298711in}{1.744589in}}%
\pgfpathlineto{\pgfqpoint{1.360274in}{1.774457in}}%
\pgfpathlineto{\pgfqpoint{1.421403in}{1.805205in}}%
\pgfpathlineto{\pgfqpoint{1.482258in}{1.836494in}}%
\pgfpathlineto{\pgfqpoint{1.543024in}{1.867954in}}%
\pgfpathlineto{\pgfqpoint{1.603909in}{1.899183in}}%
\pgfusepath{stroke}%
\end{pgfscope}%
\begin{pgfscope}%
\pgfpathrectangle{\pgfqpoint{0.647939in}{0.492442in}}{\pgfqpoint{3.079299in}{3.079299in}}%
\pgfusepath{clip}%
\pgfsetbuttcap%
\pgfsetroundjoin%
\pgfsetlinewidth{0.301125pt}%
\definecolor{currentstroke}{rgb}{0.500000,0.500000,0.500000}%
\pgfsetstrokecolor{currentstroke}%
\pgfsetstrokeopacity{0.300000}%
\pgfsetdash{}{0pt}%
\pgfpathmoveto{\pgfqpoint{0.647939in}{1.472219in}}%
\pgfpathlineto{\pgfqpoint{0.647939in}{1.472219in}}%
\pgfpathlineto{\pgfqpoint{0.715436in}{1.483424in}}%
\pgfpathlineto{\pgfqpoint{0.782562in}{1.496664in}}%
\pgfpathlineto{\pgfqpoint{0.849225in}{1.512059in}}%
\pgfpathlineto{\pgfqpoint{0.915332in}{1.529686in}}%
\pgfpathlineto{\pgfqpoint{0.980794in}{1.549573in}}%
\pgfpathlineto{\pgfqpoint{1.045536in}{1.571691in}}%
\pgfpathlineto{\pgfqpoint{1.109506in}{1.595958in}}%
\pgfpathlineto{\pgfqpoint{1.172678in}{1.622233in}}%
\pgfusepath{stroke}%
\end{pgfscope}%
\begin{pgfscope}%
\pgfpathrectangle{\pgfqpoint{0.647939in}{0.492442in}}{\pgfqpoint{3.079299in}{3.079299in}}%
\pgfusepath{clip}%
\pgfsetbuttcap%
\pgfsetroundjoin%
\pgfsetlinewidth{0.301125pt}%
\definecolor{currentstroke}{rgb}{0.500000,0.500000,0.500000}%
\pgfsetstrokecolor{currentstroke}%
\pgfsetstrokeopacity{0.300000}%
\pgfsetdash{}{0pt}%
\pgfpathmoveto{\pgfqpoint{0.647939in}{1.402235in}}%
\pgfpathlineto{\pgfqpoint{0.647939in}{1.402235in}}%
\pgfpathlineto{\pgfqpoint{0.715405in}{1.413628in}}%
\pgfpathlineto{\pgfqpoint{0.782481in}{1.427111in}}%
\pgfpathlineto{\pgfqpoint{0.849070in}{1.442817in}}%
\pgfpathlineto{\pgfqpoint{0.915071in}{1.460833in}}%
\pgfpathlineto{\pgfqpoint{0.980385in}{1.481199in}}%
\pgfpathlineto{\pgfqpoint{1.044925in}{1.503895in}}%
\pgfpathlineto{\pgfqpoint{1.108629in}{1.528848in}}%
\pgfpathlineto{\pgfqpoint{1.171460in}{1.555925in}}%
\pgfpathlineto{\pgfqpoint{1.233418in}{1.584948in}}%
\pgfpathlineto{\pgfqpoint{1.294544in}{1.615691in}}%
\pgfpathlineto{\pgfqpoint{1.354914in}{1.647895in}}%
\pgfpathlineto{\pgfqpoint{1.414646in}{1.681269in}}%
\pgfpathlineto{\pgfqpoint{1.473894in}{1.715497in}}%
\pgfpathlineto{\pgfqpoint{1.532843in}{1.750239in}}%
\pgfpathlineto{\pgfqpoint{1.591707in}{1.785125in}}%
\pgfpathlineto{\pgfqpoint{1.650719in}{1.819757in}}%
\pgfpathlineto{\pgfqpoint{1.710130in}{1.853699in}}%
\pgfpathlineto{\pgfqpoint{1.770186in}{1.886478in}}%
\pgfpathlineto{\pgfqpoint{1.831115in}{1.917595in}}%
\pgfpathlineto{\pgfqpoint{1.893093in}{1.946550in}}%
\pgfpathlineto{\pgfqpoint{1.956198in}{1.972933in}}%
\pgfpathlineto{\pgfqpoint{2.020357in}{1.996624in}}%
\pgfpathlineto{\pgfqpoint{2.085223in}{2.018337in}}%
\pgfpathlineto{\pgfqpoint{2.149484in}{2.041215in}}%
\pgfpathlineto{\pgfqpoint{2.149484in}{2.041215in}}%
\pgfusepath{stroke}%
\end{pgfscope}%
\begin{pgfscope}%
\pgfpathrectangle{\pgfqpoint{0.647939in}{0.492442in}}{\pgfqpoint{3.079299in}{3.079299in}}%
\pgfusepath{clip}%
\pgfsetbuttcap%
\pgfsetroundjoin%
\pgfsetlinewidth{0.301125pt}%
\definecolor{currentstroke}{rgb}{0.500000,0.500000,0.500000}%
\pgfsetstrokecolor{currentstroke}%
\pgfsetstrokeopacity{0.300000}%
\pgfsetdash{}{0pt}%
\pgfpathmoveto{\pgfqpoint{0.647939in}{1.332251in}}%
\pgfpathlineto{\pgfqpoint{0.647939in}{1.332251in}}%
\pgfpathlineto{\pgfqpoint{0.715371in}{1.343838in}}%
\pgfpathlineto{\pgfqpoint{0.782396in}{1.357574in}}%
\pgfpathlineto{\pgfqpoint{0.848907in}{1.373603in}}%
\pgfpathlineto{\pgfqpoint{0.914794in}{1.392025in}}%
\pgfpathlineto{\pgfqpoint{0.979949in}{1.412890in}}%
\pgfpathlineto{\pgfqpoint{1.044272in}{1.436190in}}%
\pgfpathlineto{\pgfqpoint{1.107687in}{1.461862in}}%
\pgfpathlineto{\pgfqpoint{1.170147in}{1.489782in}}%
\pgfpathlineto{\pgfqpoint{1.231638in}{1.519777in}}%
\pgfpathlineto{\pgfqpoint{1.292190in}{1.551630in}}%
\pgfpathlineto{\pgfqpoint{1.351870in}{1.585090in}}%
\pgfpathlineto{\pgfqpoint{1.410790in}{1.619876in}}%
\pgfusepath{stroke}%
\end{pgfscope}%
\begin{pgfscope}%
\pgfpathrectangle{\pgfqpoint{0.647939in}{0.492442in}}{\pgfqpoint{3.079299in}{3.079299in}}%
\pgfusepath{clip}%
\pgfsetbuttcap%
\pgfsetroundjoin%
\pgfsetlinewidth{0.301125pt}%
\definecolor{currentstroke}{rgb}{0.500000,0.500000,0.500000}%
\pgfsetstrokecolor{currentstroke}%
\pgfsetstrokeopacity{0.300000}%
\pgfsetdash{}{0pt}%
\pgfpathmoveto{\pgfqpoint{0.647939in}{1.262267in}}%
\pgfpathlineto{\pgfqpoint{0.647939in}{1.262267in}}%
\pgfpathlineto{\pgfqpoint{0.715336in}{1.274054in}}%
\pgfpathlineto{\pgfqpoint{0.782305in}{1.288053in}}%
\pgfpathlineto{\pgfqpoint{0.848734in}{1.304417in}}%
\pgfpathlineto{\pgfqpoint{0.914500in}{1.323262in}}%
\pgfpathlineto{\pgfqpoint{0.979483in}{1.344649in}}%
\pgfusepath{stroke}%
\end{pgfscope}%
\begin{pgfscope}%
\pgfpathrectangle{\pgfqpoint{0.647939in}{0.492442in}}{\pgfqpoint{3.079299in}{3.079299in}}%
\pgfusepath{clip}%
\pgfsetbuttcap%
\pgfsetroundjoin%
\pgfsetlinewidth{0.301125pt}%
\definecolor{currentstroke}{rgb}{0.500000,0.500000,0.500000}%
\pgfsetstrokecolor{currentstroke}%
\pgfsetstrokeopacity{0.300000}%
\pgfsetdash{}{0pt}%
\pgfpathmoveto{\pgfqpoint{0.647939in}{1.192283in}}%
\pgfpathlineto{\pgfqpoint{0.647939in}{1.192283in}}%
\pgfpathlineto{\pgfqpoint{0.715299in}{1.204277in}}%
\pgfpathlineto{\pgfqpoint{0.782210in}{1.218548in}}%
\pgfpathlineto{\pgfqpoint{0.848550in}{1.235262in}}%
\pgfpathlineto{\pgfqpoint{0.914187in}{1.254548in}}%
\pgfpathlineto{\pgfqpoint{0.978986in}{1.276481in}}%
\pgfpathlineto{\pgfqpoint{1.042822in}{1.301077in}}%
\pgfpathlineto{\pgfqpoint{1.105585in}{1.328292in}}%
\pgfpathlineto{\pgfqpoint{1.167199in}{1.358021in}}%
\pgfpathlineto{\pgfqpoint{1.227620in}{1.390104in}}%
\pgfpathlineto{\pgfqpoint{1.286851in}{1.424339in}}%
\pgfpathlineto{\pgfqpoint{1.344936in}{1.460487in}}%
\pgfpathlineto{\pgfqpoint{1.401965in}{1.498283in}}%
\pgfpathlineto{\pgfqpoint{1.458069in}{1.537444in}}%
\pgfpathlineto{\pgfqpoint{1.513425in}{1.577659in}}%
\pgfpathlineto{\pgfqpoint{1.568249in}{1.618595in}}%
\pgfpathlineto{\pgfqpoint{1.622777in}{1.659920in}}%
\pgfpathlineto{\pgfqpoint{1.677274in}{1.701288in}}%
\pgfpathlineto{\pgfqpoint{1.732028in}{1.742314in}}%
\pgfpathlineto{\pgfqpoint{1.787325in}{1.782587in}}%
\pgfpathlineto{\pgfqpoint{1.843441in}{1.821692in}}%
\pgfpathlineto{\pgfqpoint{1.900613in}{1.859241in}}%
\pgfusepath{stroke}%
\end{pgfscope}%
\begin{pgfscope}%
\pgfpathrectangle{\pgfqpoint{0.647939in}{0.492442in}}{\pgfqpoint{3.079299in}{3.079299in}}%
\pgfusepath{clip}%
\pgfsetbuttcap%
\pgfsetroundjoin%
\pgfsetlinewidth{0.301125pt}%
\definecolor{currentstroke}{rgb}{0.500000,0.500000,0.500000}%
\pgfsetstrokecolor{currentstroke}%
\pgfsetstrokeopacity{0.300000}%
\pgfsetdash{}{0pt}%
\pgfpathmoveto{\pgfqpoint{0.647939in}{1.122299in}}%
\pgfpathlineto{\pgfqpoint{0.647939in}{1.122299in}}%
\pgfpathlineto{\pgfqpoint{0.715259in}{1.134508in}}%
\pgfpathlineto{\pgfqpoint{0.782109in}{1.149061in}}%
\pgfpathlineto{\pgfqpoint{0.848356in}{1.166139in}}%
\pgfpathlineto{\pgfqpoint{0.913854in}{1.185886in}}%
\pgfpathlineto{\pgfqpoint{0.978455in}{1.208389in}}%
\pgfpathlineto{\pgfqpoint{1.042016in}{1.233679in}}%
\pgfpathlineto{\pgfqpoint{1.104411in}{1.261723in}}%
\pgfpathlineto{\pgfqpoint{1.165543in}{1.292422in}}%
\pgfpathlineto{\pgfqpoint{1.225353in}{1.325624in}}%
\pgfpathlineto{\pgfqpoint{1.283826in}{1.361130in}}%
\pgfpathlineto{\pgfqpoint{1.340994in}{1.398706in}}%
\pgfusepath{stroke}%
\end{pgfscope}%
\begin{pgfscope}%
\pgfpathrectangle{\pgfqpoint{0.647939in}{0.492442in}}{\pgfqpoint{3.079299in}{3.079299in}}%
\pgfusepath{clip}%
\pgfsetbuttcap%
\pgfsetroundjoin%
\pgfsetlinewidth{0.301125pt}%
\definecolor{currentstroke}{rgb}{0.500000,0.500000,0.500000}%
\pgfsetstrokecolor{currentstroke}%
\pgfsetstrokeopacity{0.300000}%
\pgfsetdash{}{0pt}%
\pgfpathmoveto{\pgfqpoint{0.647939in}{1.052315in}}%
\pgfpathlineto{\pgfqpoint{0.647939in}{1.052315in}}%
\pgfpathlineto{\pgfqpoint{0.715218in}{1.064746in}}%
\pgfpathlineto{\pgfqpoint{0.782002in}{1.079592in}}%
\pgfpathlineto{\pgfqpoint{0.848149in}{1.097050in}}%
\pgfpathlineto{\pgfqpoint{0.913498in}{1.117277in}}%
\pgfpathlineto{\pgfqpoint{0.977886in}{1.140378in}}%
\pgfpathlineto{\pgfqpoint{1.041150in}{1.166396in}}%
\pgfpathlineto{\pgfqpoint{1.103143in}{1.195307in}}%
\pgfpathlineto{\pgfqpoint{1.163750in}{1.227023in}}%
\pgfpathlineto{\pgfqpoint{1.222892in}{1.261394in}}%
\pgfusepath{stroke}%
\end{pgfscope}%
\begin{pgfscope}%
\pgfpathrectangle{\pgfqpoint{0.647939in}{0.492442in}}{\pgfqpoint{3.079299in}{3.079299in}}%
\pgfusepath{clip}%
\pgfsetbuttcap%
\pgfsetroundjoin%
\pgfsetlinewidth{0.301125pt}%
\definecolor{currentstroke}{rgb}{0.500000,0.500000,0.500000}%
\pgfsetstrokecolor{currentstroke}%
\pgfsetstrokeopacity{0.300000}%
\pgfsetdash{}{0pt}%
\pgfpathmoveto{\pgfqpoint{0.647939in}{0.982331in}}%
\pgfpathlineto{\pgfqpoint{0.647939in}{0.982331in}}%
\pgfpathlineto{\pgfqpoint{0.715174in}{0.994993in}}%
\pgfpathlineto{\pgfqpoint{0.781889in}{1.010144in}}%
\pgfpathlineto{\pgfqpoint{0.847928in}{1.027997in}}%
\pgfpathlineto{\pgfqpoint{0.913118in}{1.048726in}}%
\pgfpathlineto{\pgfqpoint{0.977275in}{1.072453in}}%
\pgfpathlineto{\pgfqpoint{1.040217in}{1.099233in}}%
\pgfpathlineto{\pgfqpoint{1.101774in}{1.129054in}}%
\pgfusepath{stroke}%
\end{pgfscope}%
\begin{pgfscope}%
\pgfpathrectangle{\pgfqpoint{0.647939in}{0.492442in}}{\pgfqpoint{3.079299in}{3.079299in}}%
\pgfusepath{clip}%
\pgfsetbuttcap%
\pgfsetroundjoin%
\pgfsetlinewidth{0.301125pt}%
\definecolor{currentstroke}{rgb}{0.500000,0.500000,0.500000}%
\pgfsetstrokecolor{currentstroke}%
\pgfsetstrokeopacity{0.300000}%
\pgfsetdash{}{0pt}%
\pgfpathmoveto{\pgfqpoint{0.647939in}{0.912347in}}%
\pgfpathlineto{\pgfqpoint{0.647939in}{0.912347in}}%
\pgfpathlineto{\pgfqpoint{0.715128in}{0.925248in}}%
\pgfpathlineto{\pgfqpoint{0.781769in}{0.940716in}}%
\pgfpathlineto{\pgfqpoint{0.847694in}{0.958981in}}%
\pgfpathlineto{\pgfqpoint{0.912712in}{0.980236in}}%
\pgfpathlineto{\pgfqpoint{0.976620in}{1.004618in}}%
\pgfpathlineto{\pgfqpoint{1.039212in}{1.032198in}}%
\pgfpathlineto{\pgfqpoint{1.100294in}{1.062972in}}%
\pgfpathlineto{\pgfqpoint{1.159702in}{1.096862in}}%
\pgfpathlineto{\pgfqpoint{1.217315in}{1.133721in}}%
\pgfpathlineto{\pgfqpoint{1.273066in}{1.173349in}}%
\pgfpathlineto{\pgfqpoint{1.326952in}{1.215484in}}%
\pgfpathlineto{\pgfqpoint{1.379042in}{1.259814in}}%
\pgfpathlineto{\pgfqpoint{1.429458in}{1.306041in}}%
\pgfpathlineto{\pgfqpoint{1.478390in}{1.353842in}}%
\pgfpathlineto{\pgfqpoint{1.526078in}{1.402877in}}%
\pgfpathlineto{\pgfqpoint{1.572808in}{1.452829in}}%
\pgfpathlineto{\pgfqpoint{1.618899in}{1.503365in}}%
\pgfusepath{stroke}%
\end{pgfscope}%
\begin{pgfscope}%
\pgfpathrectangle{\pgfqpoint{0.647939in}{0.492442in}}{\pgfqpoint{3.079299in}{3.079299in}}%
\pgfusepath{clip}%
\pgfsetbuttcap%
\pgfsetroundjoin%
\pgfsetlinewidth{0.301125pt}%
\definecolor{currentstroke}{rgb}{0.500000,0.500000,0.500000}%
\pgfsetstrokecolor{currentstroke}%
\pgfsetstrokeopacity{0.300000}%
\pgfsetdash{}{0pt}%
\pgfpathmoveto{\pgfqpoint{0.647939in}{0.842362in}}%
\pgfpathlineto{\pgfqpoint{0.647939in}{0.842362in}}%
\pgfpathlineto{\pgfqpoint{0.715079in}{0.855512in}}%
\pgfpathlineto{\pgfqpoint{0.781642in}{0.871310in}}%
\pgfpathlineto{\pgfqpoint{0.847444in}{0.890006in}}%
\pgfpathlineto{\pgfqpoint{0.912277in}{0.911811in}}%
\pgfpathlineto{\pgfqpoint{0.975916in}{0.936880in}}%
\pgfpathlineto{\pgfqpoint{1.038127in}{0.965297in}}%
\pgfpathlineto{\pgfqpoint{1.098692in}{0.997068in}}%
\pgfpathlineto{\pgfqpoint{1.157418in}{1.032118in}}%
\pgfpathlineto{\pgfqpoint{1.214162in}{1.070295in}}%
\pgfpathlineto{\pgfqpoint{1.268841in}{1.111387in}}%
\pgfpathlineto{\pgfqpoint{1.321452in}{1.155095in}}%
\pgfusepath{stroke}%
\end{pgfscope}%
\begin{pgfscope}%
\pgfpathrectangle{\pgfqpoint{0.647939in}{0.492442in}}{\pgfqpoint{3.079299in}{3.079299in}}%
\pgfusepath{clip}%
\pgfsetbuttcap%
\pgfsetroundjoin%
\pgfsetlinewidth{0.301125pt}%
\definecolor{currentstroke}{rgb}{0.500000,0.500000,0.500000}%
\pgfsetstrokecolor{currentstroke}%
\pgfsetstrokeopacity{0.300000}%
\pgfsetdash{}{0pt}%
\pgfpathmoveto{\pgfqpoint{0.647939in}{0.772378in}}%
\pgfpathlineto{\pgfqpoint{0.647939in}{0.772378in}}%
\pgfpathlineto{\pgfqpoint{0.715028in}{0.785785in}}%
\pgfpathlineto{\pgfqpoint{0.781506in}{0.801927in}}%
\pgfpathlineto{\pgfqpoint{0.847177in}{0.821073in}}%
\pgfpathlineto{\pgfqpoint{0.911811in}{0.843454in}}%
\pgfpathlineto{\pgfqpoint{0.975157in}{0.869242in}}%
\pgfpathlineto{\pgfqpoint{1.036956in}{0.898538in}}%
\pgfpathlineto{\pgfqpoint{1.096956in}{0.931353in}}%
\pgfpathlineto{\pgfqpoint{1.154939in}{0.967610in}}%
\pgfpathlineto{\pgfqpoint{1.210737in}{1.007150in}}%
\pgfpathlineto{\pgfqpoint{1.264262in}{1.049726in}}%
\pgfusepath{stroke}%
\end{pgfscope}%
\begin{pgfscope}%
\pgfpathrectangle{\pgfqpoint{0.647939in}{0.492442in}}{\pgfqpoint{3.079299in}{3.079299in}}%
\pgfusepath{clip}%
\pgfsetbuttcap%
\pgfsetroundjoin%
\pgfsetlinewidth{0.301125pt}%
\definecolor{currentstroke}{rgb}{0.500000,0.500000,0.500000}%
\pgfsetstrokecolor{currentstroke}%
\pgfsetstrokeopacity{0.300000}%
\pgfsetdash{}{0pt}%
\pgfpathmoveto{\pgfqpoint{0.647939in}{0.702394in}}%
\pgfpathlineto{\pgfqpoint{0.647939in}{0.702394in}}%
\pgfpathlineto{\pgfqpoint{0.714973in}{0.716069in}}%
\pgfpathlineto{\pgfqpoint{0.781361in}{0.732569in}}%
\pgfpathlineto{\pgfqpoint{0.846891in}{0.752185in}}%
\pgfpathlineto{\pgfqpoint{0.911310in}{0.775169in}}%
\pgfpathlineto{\pgfqpoint{0.974340in}{0.801713in}}%
\pgfpathlineto{\pgfqpoint{1.035689in}{0.831928in}}%
\pgfpathlineto{\pgfqpoint{1.095073in}{0.865833in}}%
\pgfpathlineto{\pgfqpoint{1.152246in}{0.903346in}}%
\pgfpathlineto{\pgfqpoint{1.207021in}{0.944291in}}%
\pgfpathlineto{\pgfqpoint{1.259303in}{0.988373in}}%
\pgfusepath{stroke}%
\end{pgfscope}%
\begin{pgfscope}%
\pgfpathrectangle{\pgfqpoint{0.647939in}{0.492442in}}{\pgfqpoint{3.079299in}{3.079299in}}%
\pgfusepath{clip}%
\pgfsetbuttcap%
\pgfsetroundjoin%
\pgfsetlinewidth{0.301125pt}%
\definecolor{currentstroke}{rgb}{0.500000,0.500000,0.500000}%
\pgfsetstrokecolor{currentstroke}%
\pgfsetstrokeopacity{0.300000}%
\pgfsetdash{}{0pt}%
\pgfpathmoveto{\pgfqpoint{0.647939in}{0.632410in}}%
\pgfpathlineto{\pgfqpoint{0.647939in}{0.632410in}}%
\pgfpathlineto{\pgfqpoint{0.714914in}{0.646363in}}%
\pgfpathlineto{\pgfqpoint{0.781207in}{0.663237in}}%
\pgfpathlineto{\pgfqpoint{0.846585in}{0.683345in}}%
\pgfpathlineto{\pgfqpoint{0.910772in}{0.706961in}}%
\pgfpathlineto{\pgfqpoint{0.973458in}{0.734296in}}%
\pgfpathlineto{\pgfqpoint{1.034316in}{0.765475in}}%
\pgfpathlineto{\pgfqpoint{1.093030in}{0.800518in}}%
\pgfpathlineto{\pgfqpoint{1.149323in}{0.839331in}}%
\pgfusepath{stroke}%
\end{pgfscope}%
\begin{pgfscope}%
\pgfpathrectangle{\pgfqpoint{0.647939in}{0.492442in}}{\pgfqpoint{3.079299in}{3.079299in}}%
\pgfusepath{clip}%
\pgfsetbuttcap%
\pgfsetroundjoin%
\pgfsetlinewidth{0.301125pt}%
\definecolor{currentstroke}{rgb}{0.500000,0.500000,0.500000}%
\pgfsetstrokecolor{currentstroke}%
\pgfsetstrokeopacity{0.300000}%
\pgfsetdash{}{0pt}%
\pgfpathmoveto{\pgfqpoint{3.235383in}{3.203838in}}%
\pgfpathlineto{\pgfqpoint{3.279364in}{3.256252in}}%
\pgfpathlineto{\pgfqpoint{3.324487in}{3.307685in}}%
\pgfpathlineto{\pgfqpoint{3.370787in}{3.358061in}}%
\pgfpathlineto{\pgfqpoint{3.418313in}{3.407283in}}%
\pgfpathlineto{\pgfqpoint{3.467121in}{3.455231in}}%
\pgfpathlineto{\pgfqpoint{3.517286in}{3.501757in}}%
\pgfpathlineto{\pgfqpoint{3.568874in}{3.546698in}}%
\pgfpathlineto{\pgfqpoint{3.598573in}{3.571741in}}%
\pgfusepath{stroke}%
\end{pgfscope}%
\begin{pgfscope}%
\pgfpathrectangle{\pgfqpoint{0.647939in}{0.492442in}}{\pgfqpoint{3.079299in}{3.079299in}}%
\pgfusepath{clip}%
\pgfsetbuttcap%
\pgfsetroundjoin%
\pgfsetlinewidth{0.301125pt}%
\definecolor{currentstroke}{rgb}{0.500000,0.500000,0.500000}%
\pgfsetstrokecolor{currentstroke}%
\pgfsetstrokeopacity{0.300000}%
\pgfsetdash{}{0pt}%
\pgfpathmoveto{\pgfqpoint{1.969880in}{0.599637in}}%
\pgfpathlineto{\pgfqpoint{1.901795in}{0.606271in}}%
\pgfpathlineto{\pgfqpoint{1.834206in}{0.616690in}}%
\pgfpathlineto{\pgfqpoint{1.767684in}{0.632410in}}%
\pgfpathlineto{\pgfqpoint{1.703395in}{0.655392in}}%
\pgfpathlineto{\pgfqpoint{1.643578in}{0.687989in}}%
\pgfpathlineto{\pgfqpoint{1.592102in}{0.732199in}}%
\pgfpathlineto{\pgfqpoint{1.556930in}{0.781160in}}%
\pgfusepath{stroke}%
\end{pgfscope}%
\begin{pgfscope}%
\pgfpathrectangle{\pgfqpoint{0.647939in}{0.492442in}}{\pgfqpoint{3.079299in}{3.079299in}}%
\pgfusepath{clip}%
\pgfsetbuttcap%
\pgfsetroundjoin%
\pgfsetlinewidth{0.301125pt}%
\definecolor{currentstroke}{rgb}{0.500000,0.500000,0.500000}%
\pgfsetstrokecolor{currentstroke}%
\pgfsetstrokeopacity{0.300000}%
\pgfsetdash{}{0pt}%
\pgfpathmoveto{\pgfqpoint{3.587270in}{1.402235in}}%
\pgfpathlineto{\pgfqpoint{3.529137in}{1.438314in}}%
\pgfpathlineto{\pgfqpoint{3.472081in}{1.476077in}}%
\pgfpathlineto{\pgfqpoint{3.416041in}{1.515335in}}%
\pgfpathlineto{\pgfqpoint{3.360944in}{1.555908in}}%
\pgfpathlineto{\pgfqpoint{3.306716in}{1.597635in}}%
\pgfusepath{stroke}%
\end{pgfscope}%
\begin{pgfscope}%
\pgfpathrectangle{\pgfqpoint{0.647939in}{0.492442in}}{\pgfqpoint{3.079299in}{3.079299in}}%
\pgfusepath{clip}%
\pgfsetbuttcap%
\pgfsetroundjoin%
\pgfsetlinewidth{0.301125pt}%
\definecolor{currentstroke}{rgb}{0.500000,0.500000,0.500000}%
\pgfsetstrokecolor{currentstroke}%
\pgfsetstrokeopacity{0.300000}%
\pgfsetdash{}{0pt}%
\pgfpathmoveto{\pgfqpoint{2.459470in}{0.690002in}}%
\pgfpathlineto{\pgfqpoint{2.391188in}{0.694402in}}%
\pgfpathlineto{\pgfqpoint{2.322827in}{0.697375in}}%
\pgfpathlineto{\pgfqpoint{2.254428in}{0.699344in}}%
\pgfpathlineto{\pgfqpoint{2.186015in}{0.700817in}}%
\pgfpathlineto{\pgfqpoint{2.117605in}{0.702394in}}%
\pgfusepath{stroke}%
\end{pgfscope}%
\begin{pgfscope}%
\pgfpathrectangle{\pgfqpoint{0.647939in}{0.492442in}}{\pgfqpoint{3.079299in}{3.079299in}}%
\pgfusepath{clip}%
\pgfsetbuttcap%
\pgfsetroundjoin%
\pgfsetlinewidth{0.301125pt}%
\definecolor{currentstroke}{rgb}{0.500000,0.500000,0.500000}%
\pgfsetstrokecolor{currentstroke}%
\pgfsetstrokeopacity{0.300000}%
\pgfsetdash{}{0pt}%
\pgfpathmoveto{\pgfqpoint{3.237350in}{0.702394in}}%
\pgfpathlineto{\pgfqpoint{3.173878in}{0.727964in}}%
\pgfpathlineto{\pgfqpoint{3.110199in}{0.753009in}}%
\pgfpathlineto{\pgfqpoint{3.046208in}{0.777244in}}%
\pgfpathlineto{\pgfqpoint{2.981813in}{0.800379in}}%
\pgfpathlineto{\pgfqpoint{2.916937in}{0.822127in}}%
\pgfpathlineto{\pgfqpoint{2.851529in}{0.842209in}}%
\pgfpathlineto{\pgfqpoint{2.785564in}{0.860372in}}%
\pgfpathlineto{\pgfqpoint{2.719052in}{0.876403in}}%
\pgfusepath{stroke}%
\end{pgfscope}%
\begin{pgfscope}%
\pgfpathrectangle{\pgfqpoint{0.647939in}{0.492442in}}{\pgfqpoint{3.079299in}{3.079299in}}%
\pgfusepath{clip}%
\pgfsetbuttcap%
\pgfsetroundjoin%
\pgfsetlinewidth{0.301125pt}%
\definecolor{currentstroke}{rgb}{0.500000,0.500000,0.500000}%
\pgfsetstrokecolor{currentstroke}%
\pgfsetstrokeopacity{0.300000}%
\pgfsetdash{}{0pt}%
\pgfpathmoveto{\pgfqpoint{3.517286in}{1.682171in}}%
\pgfpathlineto{\pgfqpoint{3.464941in}{1.726224in}}%
\pgfpathlineto{\pgfqpoint{3.414208in}{1.772126in}}%
\pgfpathlineto{\pgfqpoint{3.365114in}{1.819777in}}%
\pgfpathlineto{\pgfqpoint{3.317712in}{1.869110in}}%
\pgfpathlineto{\pgfqpoint{3.272104in}{1.920106in}}%
\pgfpathlineto{\pgfqpoint{3.228444in}{1.972776in}}%
\pgfpathlineto{\pgfqpoint{3.186964in}{2.027174in}}%
\pgfpathlineto{\pgfqpoint{3.147992in}{2.083392in}}%
\pgfpathlineto{\pgfqpoint{3.111994in}{2.141554in}}%
\pgfpathlineto{\pgfqpoint{3.079598in}{2.201786in}}%
\pgfpathlineto{\pgfqpoint{3.051612in}{2.264172in}}%
\pgfpathlineto{\pgfqpoint{3.029001in}{2.328683in}}%
\pgfpathlineto{\pgfqpoint{3.012789in}{2.395064in}}%
\pgfpathlineto{\pgfqpoint{3.003827in}{2.462778in}}%
\pgfpathlineto{\pgfqpoint{3.002524in}{2.531059in}}%
\pgfpathlineto{\pgfqpoint{3.008694in}{2.599084in}}%
\pgfpathlineto{\pgfqpoint{3.021633in}{2.666181in}}%
\pgfpathlineto{\pgfqpoint{3.040372in}{2.731923in}}%
\pgfpathlineto{\pgfqpoint{3.063911in}{2.796124in}}%
\pgfpathlineto{\pgfqpoint{3.091375in}{2.858758in}}%
\pgfpathlineto{\pgfqpoint{3.122058in}{2.919890in}}%
\pgfpathlineto{\pgfqpoint{3.155424in}{2.979609in}}%
\pgfpathlineto{\pgfqpoint{3.191073in}{3.037998in}}%
\pgfusepath{stroke}%
\end{pgfscope}%
\begin{pgfscope}%
\pgfpathrectangle{\pgfqpoint{0.647939in}{0.492442in}}{\pgfqpoint{3.079299in}{3.079299in}}%
\pgfusepath{clip}%
\pgfsetbuttcap%
\pgfsetroundjoin%
\pgfsetlinewidth{0.301125pt}%
\definecolor{currentstroke}{rgb}{0.500000,0.500000,0.500000}%
\pgfsetstrokecolor{currentstroke}%
\pgfsetstrokeopacity{0.300000}%
\pgfsetdash{}{0pt}%
\pgfpathmoveto{\pgfqpoint{3.342730in}{2.710296in}}%
\pgfpathlineto{\pgfqpoint{3.354495in}{2.777642in}}%
\pgfpathlineto{\pgfqpoint{3.371364in}{2.843900in}}%
\pgfpathlineto{\pgfqpoint{3.392886in}{2.908806in}}%
\pgfpathlineto{\pgfqpoint{3.418607in}{2.972173in}}%
\pgfpathlineto{\pgfqpoint{3.448128in}{3.033868in}}%
\pgfpathlineto{\pgfqpoint{3.481111in}{3.093787in}}%
\pgfpathlineto{\pgfqpoint{3.517286in}{3.151837in}}%
\pgfpathlineto{\pgfqpoint{3.556439in}{3.207922in}}%
\pgfpathlineto{\pgfqpoint{3.598419in}{3.261927in}}%
\pgfpathlineto{\pgfqpoint{3.643116in}{3.313707in}}%
\pgfusepath{stroke}%
\end{pgfscope}%
\begin{pgfscope}%
\pgfpathrectangle{\pgfqpoint{0.647939in}{0.492442in}}{\pgfqpoint{3.079299in}{3.079299in}}%
\pgfusepath{clip}%
\pgfsetbuttcap%
\pgfsetroundjoin%
\pgfsetlinewidth{0.301125pt}%
\definecolor{currentstroke}{rgb}{0.500000,0.500000,0.500000}%
\pgfsetstrokecolor{currentstroke}%
\pgfsetstrokeopacity{0.300000}%
\pgfsetdash{}{0pt}%
\pgfpathmoveto{\pgfqpoint{1.101280in}{2.977722in}}%
\pgfpathlineto{\pgfqpoint{1.167880in}{2.993432in}}%
\pgfpathlineto{\pgfqpoint{1.234360in}{3.009643in}}%
\pgfpathlineto{\pgfqpoint{1.300780in}{3.026104in}}%
\pgfpathlineto{\pgfqpoint{1.367206in}{3.042534in}}%
\pgfpathlineto{\pgfqpoint{1.433713in}{3.058639in}}%
\pgfpathlineto{\pgfqpoint{1.500366in}{3.074119in}}%
\pgfpathlineto{\pgfqpoint{1.567226in}{3.088675in}}%
\pgfpathlineto{\pgfqpoint{1.634336in}{3.102027in}}%
\pgfpathlineto{\pgfqpoint{1.701718in}{3.113923in}}%
\pgfpathlineto{\pgfqpoint{1.769369in}{3.124165in}}%
\pgfpathlineto{\pgfqpoint{1.837266in}{3.132630in}}%
\pgfpathlineto{\pgfqpoint{1.905363in}{3.139296in}}%
\pgfpathlineto{\pgfqpoint{1.973606in}{3.144250in}}%
\pgfpathlineto{\pgfqpoint{2.041943in}{3.147712in}}%
\pgfpathlineto{\pgfqpoint{2.110331in}{3.150038in}}%
\pgfpathlineto{\pgfqpoint{2.178738in}{3.151740in}}%
\pgfpathlineto{\pgfqpoint{2.247145in}{3.153469in}}%
\pgfpathlineto{\pgfqpoint{2.315524in}{3.155986in}}%
\pgfpathlineto{\pgfqpoint{2.383817in}{3.160147in}}%
\pgfpathlineto{\pgfqpoint{2.451900in}{3.166849in}}%
\pgfpathlineto{\pgfqpoint{2.519550in}{3.176951in}}%
\pgfpathlineto{\pgfqpoint{2.586450in}{3.191134in}}%
\pgfpathlineto{\pgfqpoint{2.652238in}{3.209776in}}%
\pgfpathlineto{\pgfqpoint{2.716587in}{3.232903in}}%
\pgfpathlineto{\pgfqpoint{2.779278in}{3.260221in}}%
\pgfpathlineto{\pgfqpoint{2.840245in}{3.291214in}}%
\pgfpathlineto{\pgfqpoint{2.899564in}{3.325274in}}%
\pgfpathlineto{\pgfqpoint{2.957413in}{3.361789in}}%
\pgfusepath{stroke}%
\end{pgfscope}%
\begin{pgfscope}%
\pgfpathrectangle{\pgfqpoint{0.647939in}{0.492442in}}{\pgfqpoint{3.079299in}{3.079299in}}%
\pgfusepath{clip}%
\pgfsetbuttcap%
\pgfsetroundjoin%
\pgfsetlinewidth{0.301125pt}%
\definecolor{currentstroke}{rgb}{0.500000,0.500000,0.500000}%
\pgfsetstrokecolor{currentstroke}%
\pgfsetstrokeopacity{0.300000}%
\pgfsetdash{}{0pt}%
\pgfpathmoveto{\pgfqpoint{3.447302in}{1.332251in}}%
\pgfpathlineto{\pgfqpoint{3.389276in}{1.368515in}}%
\pgfpathlineto{\pgfqpoint{3.331879in}{1.405766in}}%
\pgfpathlineto{\pgfqpoint{3.275002in}{1.443808in}}%
\pgfpathlineto{\pgfqpoint{3.218531in}{1.482452in}}%
\pgfpathlineto{\pgfqpoint{3.162350in}{1.521516in}}%
\pgfpathlineto{\pgfqpoint{3.106334in}{1.560817in}}%
\pgfpathlineto{\pgfqpoint{3.050355in}{1.600167in}}%
\pgfpathlineto{\pgfqpoint{2.994277in}{1.639377in}}%
\pgfpathlineto{\pgfqpoint{2.937969in}{1.678253in}}%
\pgfpathlineto{\pgfqpoint{2.881292in}{1.716588in}}%
\pgfpathlineto{\pgfqpoint{2.824107in}{1.754156in}}%
\pgfpathlineto{\pgfqpoint{2.766272in}{1.790716in}}%
\pgfpathlineto{\pgfqpoint{2.707656in}{1.826004in}}%
\pgfpathlineto{\pgfqpoint{2.648137in}{1.859740in}}%
\pgfpathlineto{\pgfqpoint{2.587616in}{1.891637in}}%
\pgfpathlineto{\pgfqpoint{2.526038in}{1.921435in}}%
\pgfpathlineto{\pgfqpoint{2.463420in}{1.948973in}}%
\pgfpathlineto{\pgfqpoint{2.399913in}{1.974383in}}%
\pgfpathlineto{\pgfqpoint{2.335995in}{1.998717in}}%
\pgfpathlineto{\pgfqpoint{2.274271in}{2.027345in}}%
\pgfpathlineto{\pgfqpoint{2.274271in}{2.027345in}}%
\pgfpathlineto{\pgfqpoint{2.255775in}{2.041996in}}%
\pgfpathlineto{\pgfqpoint{2.255775in}{2.041996in}}%
\pgfpathlineto{\pgfqpoint{2.248267in}{2.058037in}}%
\pgfusepath{stroke}%
\end{pgfscope}%
\begin{pgfscope}%
\pgfpathrectangle{\pgfqpoint{0.647939in}{0.492442in}}{\pgfqpoint{3.079299in}{3.079299in}}%
\pgfusepath{clip}%
\pgfsetbuttcap%
\pgfsetroundjoin%
\pgfsetlinewidth{0.301125pt}%
\definecolor{currentstroke}{rgb}{0.500000,0.500000,0.500000}%
\pgfsetstrokecolor{currentstroke}%
\pgfsetstrokeopacity{0.300000}%
\pgfsetdash{}{0pt}%
\pgfpathmoveto{\pgfqpoint{3.447302in}{2.242044in}}%
\pgfpathlineto{\pgfqpoint{3.419701in}{2.304603in}}%
\pgfpathlineto{\pgfqpoint{3.396883in}{2.369052in}}%
\pgfpathlineto{\pgfqpoint{3.379271in}{2.435106in}}%
\pgfpathlineto{\pgfqpoint{3.367233in}{2.502392in}}%
\pgfpathlineto{\pgfqpoint{3.361016in}{2.570462in}}%
\pgfpathlineto{\pgfqpoint{3.360707in}{2.638821in}}%
\pgfusepath{stroke}%
\end{pgfscope}%
\begin{pgfscope}%
\pgfpathrectangle{\pgfqpoint{0.647939in}{0.492442in}}{\pgfqpoint{3.079299in}{3.079299in}}%
\pgfusepath{clip}%
\pgfsetbuttcap%
\pgfsetroundjoin%
\pgfsetlinewidth{0.301125pt}%
\definecolor{currentstroke}{rgb}{0.500000,0.500000,0.500000}%
\pgfsetstrokecolor{currentstroke}%
\pgfsetstrokeopacity{0.300000}%
\pgfsetdash{}{0pt}%
\pgfpathmoveto{\pgfqpoint{1.852696in}{3.260607in}}%
\pgfpathlineto{\pgfqpoint{1.920875in}{3.266388in}}%
\pgfpathlineto{\pgfqpoint{1.989170in}{3.270600in}}%
\pgfpathlineto{\pgfqpoint{2.057535in}{3.273494in}}%
\pgfpathlineto{\pgfqpoint{2.125935in}{3.275444in}}%
\pgfpathlineto{\pgfqpoint{2.194347in}{3.276940in}}%
\pgfpathlineto{\pgfqpoint{2.262756in}{3.278585in}}%
\pgfpathlineto{\pgfqpoint{2.331136in}{3.281089in}}%
\pgfpathlineto{\pgfqpoint{2.399430in}{3.285232in}}%
\pgfpathlineto{\pgfqpoint{2.467525in}{3.291805in}}%
\pgfusepath{stroke}%
\end{pgfscope}%
\begin{pgfscope}%
\pgfpathrectangle{\pgfqpoint{0.647939in}{0.492442in}}{\pgfqpoint{3.079299in}{3.079299in}}%
\pgfusepath{clip}%
\pgfsetbuttcap%
\pgfsetroundjoin%
\pgfsetlinewidth{0.301125pt}%
\definecolor{currentstroke}{rgb}{0.500000,0.500000,0.500000}%
\pgfsetstrokecolor{currentstroke}%
\pgfsetstrokeopacity{0.300000}%
\pgfsetdash{}{0pt}%
\pgfpathmoveto{\pgfqpoint{3.377318in}{2.032092in}}%
\pgfpathlineto{\pgfqpoint{3.338453in}{2.088378in}}%
\pgfpathlineto{\pgfqpoint{3.302756in}{2.146720in}}%
\pgfpathlineto{\pgfqpoint{3.270648in}{2.207106in}}%
\pgfpathlineto{\pgfqpoint{3.242640in}{2.269491in}}%
\pgfpathlineto{\pgfqpoint{3.219321in}{2.333764in}}%
\pgfpathlineto{\pgfqpoint{3.201305in}{2.399705in}}%
\pgfpathlineto{\pgfqpoint{3.189142in}{2.466960in}}%
\pgfpathlineto{\pgfqpoint{3.183216in}{2.535041in}}%
\pgfpathlineto{\pgfqpoint{3.183645in}{2.603382in}}%
\pgfpathlineto{\pgfqpoint{3.190253in}{2.671415in}}%
\pgfusepath{stroke}%
\end{pgfscope}%
\begin{pgfscope}%
\pgfpathrectangle{\pgfqpoint{0.647939in}{0.492442in}}{\pgfqpoint{3.079299in}{3.079299in}}%
\pgfusepath{clip}%
\pgfsetbuttcap%
\pgfsetroundjoin%
\pgfsetlinewidth{0.301125pt}%
\definecolor{currentstroke}{rgb}{0.500000,0.500000,0.500000}%
\pgfsetstrokecolor{currentstroke}%
\pgfsetstrokeopacity{0.300000}%
\pgfsetdash{}{0pt}%
\pgfpathmoveto{\pgfqpoint{2.846990in}{3.141086in}}%
\pgfpathlineto{\pgfqpoint{2.902924in}{3.180459in}}%
\pgfpathlineto{\pgfqpoint{2.957413in}{3.221821in}}%
\pgfpathlineto{\pgfqpoint{3.010779in}{3.264637in}}%
\pgfpathlineto{\pgfqpoint{3.063324in}{3.308460in}}%
\pgfpathlineto{\pgfqpoint{3.115326in}{3.352929in}}%
\pgfusepath{stroke}%
\end{pgfscope}%
\begin{pgfscope}%
\pgfpathrectangle{\pgfqpoint{0.647939in}{0.492442in}}{\pgfqpoint{3.079299in}{3.079299in}}%
\pgfusepath{clip}%
\pgfsetbuttcap%
\pgfsetroundjoin%
\pgfsetlinewidth{0.301125pt}%
\definecolor{currentstroke}{rgb}{0.500000,0.500000,0.500000}%
\pgfsetstrokecolor{currentstroke}%
\pgfsetstrokeopacity{0.300000}%
\pgfsetdash{}{0pt}%
\pgfpathmoveto{\pgfqpoint{1.087512in}{3.130308in}}%
\pgfpathlineto{\pgfqpoint{1.154296in}{3.145218in}}%
\pgfpathlineto{\pgfqpoint{1.220972in}{3.160606in}}%
\pgfpathlineto{\pgfqpoint{1.287593in}{3.176230in}}%
\pgfpathlineto{\pgfqpoint{1.354220in}{3.191827in}}%
\pgfpathlineto{\pgfqpoint{1.420918in}{3.207120in}}%
\pgfpathlineto{\pgfqpoint{1.487748in}{3.221821in}}%
\pgfpathlineto{\pgfqpoint{1.554763in}{3.235647in}}%
\pgfpathlineto{\pgfqpoint{1.622003in}{3.248331in}}%
\pgfusepath{stroke}%
\end{pgfscope}%
\begin{pgfscope}%
\pgfpathrectangle{\pgfqpoint{0.647939in}{0.492442in}}{\pgfqpoint{3.079299in}{3.079299in}}%
\pgfusepath{clip}%
\pgfsetbuttcap%
\pgfsetroundjoin%
\pgfsetlinewidth{0.301125pt}%
\definecolor{currentstroke}{rgb}{0.500000,0.500000,0.500000}%
\pgfsetstrokecolor{currentstroke}%
\pgfsetstrokeopacity{0.300000}%
\pgfsetdash{}{0pt}%
\pgfpathmoveto{\pgfqpoint{2.177251in}{0.878994in}}%
\pgfpathlineto{\pgfqpoint{2.108849in}{0.880895in}}%
\pgfpathlineto{\pgfqpoint{2.040492in}{0.883909in}}%
\pgfpathlineto{\pgfqpoint{1.972281in}{0.889131in}}%
\pgfpathlineto{\pgfqpoint{1.904470in}{0.897972in}}%
\pgfpathlineto{\pgfqpoint{1.837668in}{0.912347in}}%
\pgfusepath{stroke}%
\end{pgfscope}%
\begin{pgfscope}%
\pgfpathrectangle{\pgfqpoint{0.647939in}{0.492442in}}{\pgfqpoint{3.079299in}{3.079299in}}%
\pgfusepath{clip}%
\pgfsetbuttcap%
\pgfsetroundjoin%
\pgfsetlinewidth{0.301125pt}%
\definecolor{currentstroke}{rgb}{0.500000,0.500000,0.500000}%
\pgfsetstrokecolor{currentstroke}%
\pgfsetstrokeopacity{0.300000}%
\pgfsetdash{}{0pt}%
\pgfpathmoveto{\pgfqpoint{2.665722in}{0.872902in}}%
\pgfpathlineto{\pgfqpoint{2.598322in}{0.884650in}}%
\pgfpathlineto{\pgfqpoint{2.530556in}{0.894064in}}%
\pgfpathlineto{\pgfqpoint{2.462516in}{0.901256in}}%
\pgfpathlineto{\pgfqpoint{2.394292in}{0.906450in}}%
\pgfpathlineto{\pgfqpoint{2.325959in}{0.909990in}}%
\pgfpathlineto{\pgfqpoint{2.257573in}{0.912347in}}%
\pgfusepath{stroke}%
\end{pgfscope}%
\begin{pgfscope}%
\pgfpathrectangle{\pgfqpoint{0.647939in}{0.492442in}}{\pgfqpoint{3.079299in}{3.079299in}}%
\pgfusepath{clip}%
\pgfsetbuttcap%
\pgfsetroundjoin%
\pgfsetlinewidth{0.301125pt}%
\definecolor{currentstroke}{rgb}{0.500000,0.500000,0.500000}%
\pgfsetstrokecolor{currentstroke}%
\pgfsetstrokeopacity{0.300000}%
\pgfsetdash{}{0pt}%
\pgfpathmoveto{\pgfqpoint{3.237350in}{1.682171in}}%
\pgfpathlineto{\pgfqpoint{3.185600in}{1.726936in}}%
\pgfpathlineto{\pgfqpoint{3.134628in}{1.772583in}}%
\pgfpathlineto{\pgfqpoint{3.084424in}{1.819073in}}%
\pgfpathlineto{\pgfqpoint{3.035012in}{1.866403in}}%
\pgfpathlineto{\pgfqpoint{2.986469in}{1.914622in}}%
\pgfpathlineto{\pgfqpoint{2.938942in}{1.963838in}}%
\pgfpathlineto{\pgfqpoint{2.892692in}{2.014252in}}%
\pgfpathlineto{\pgfqpoint{2.848169in}{2.066188in}}%
\pgfpathlineto{\pgfqpoint{2.806141in}{2.120138in}}%
\pgfpathlineto{\pgfqpoint{2.767911in}{2.176810in}}%
\pgfpathlineto{\pgfqpoint{2.735694in}{2.237031in}}%
\pgfpathlineto{\pgfqpoint{2.712888in}{2.301256in}}%
\pgfpathlineto{\pgfqpoint{2.703144in}{2.368516in}}%
\pgfpathlineto{\pgfqpoint{2.706998in}{2.432739in}}%
\pgfpathlineto{\pgfqpoint{2.721749in}{2.495802in}}%
\pgfpathlineto{\pgfqpoint{2.745602in}{2.559718in}}%
\pgfpathlineto{\pgfqpoint{2.775049in}{2.621348in}}%
\pgfusepath{stroke}%
\end{pgfscope}%
\begin{pgfscope}%
\pgfpathrectangle{\pgfqpoint{0.647939in}{0.492442in}}{\pgfqpoint{3.079299in}{3.079299in}}%
\pgfusepath{clip}%
\pgfsetbuttcap%
\pgfsetroundjoin%
\pgfsetlinewidth{0.301125pt}%
\definecolor{currentstroke}{rgb}{0.500000,0.500000,0.500000}%
\pgfsetstrokecolor{currentstroke}%
\pgfsetstrokeopacity{0.300000}%
\pgfsetdash{}{0pt}%
\pgfpathmoveto{\pgfqpoint{1.809350in}{2.938766in}}%
\pgfpathlineto{\pgfqpoint{1.877251in}{2.947177in}}%
\pgfpathlineto{\pgfqpoint{1.945366in}{2.953643in}}%
\pgfpathlineto{\pgfqpoint{2.013629in}{2.958317in}}%
\pgfpathlineto{\pgfqpoint{2.081980in}{2.961520in}}%
\pgfpathlineto{\pgfqpoint{2.150371in}{2.963757in}}%
\pgfpathlineto{\pgfqpoint{2.218772in}{2.965712in}}%
\pgfpathlineto{\pgfqpoint{2.287152in}{2.968240in}}%
\pgfpathlineto{\pgfqpoint{2.355448in}{2.972352in}}%
\pgfpathlineto{\pgfqpoint{2.423514in}{2.979173in}}%
\pgfpathlineto{\pgfqpoint{2.491070in}{2.989816in}}%
\pgfpathlineto{\pgfqpoint{2.557692in}{3.005175in}}%
\pgfpathlineto{\pgfqpoint{2.622888in}{3.025728in}}%
\pgfpathlineto{\pgfqpoint{2.686226in}{3.051446in}}%
\pgfpathlineto{\pgfqpoint{2.747461in}{3.081853in}}%
\pgfusepath{stroke}%
\end{pgfscope}%
\begin{pgfscope}%
\pgfpathrectangle{\pgfqpoint{0.647939in}{0.492442in}}{\pgfqpoint{3.079299in}{3.079299in}}%
\pgfusepath{clip}%
\pgfsetbuttcap%
\pgfsetroundjoin%
\pgfsetlinewidth{0.301125pt}%
\definecolor{currentstroke}{rgb}{0.500000,0.500000,0.500000}%
\pgfsetstrokecolor{currentstroke}%
\pgfsetstrokeopacity{0.300000}%
\pgfsetdash{}{0pt}%
\pgfpathmoveto{\pgfqpoint{3.167366in}{1.962108in}}%
\pgfpathlineto{\pgfqpoint{3.124975in}{2.015805in}}%
\pgfpathlineto{\pgfqpoint{3.084891in}{2.071240in}}%
\pgfpathlineto{\pgfqpoint{3.047617in}{2.128587in}}%
\pgfpathlineto{\pgfqpoint{3.013850in}{2.188050in}}%
\pgfpathlineto{\pgfqpoint{2.984529in}{2.249803in}}%
\pgfpathlineto{\pgfqpoint{2.960837in}{2.313896in}}%
\pgfpathlineto{\pgfqpoint{2.944076in}{2.380104in}}%
\pgfpathlineto{\pgfqpoint{2.935342in}{2.447819in}}%
\pgfpathlineto{\pgfqpoint{2.935119in}{2.516095in}}%
\pgfpathlineto{\pgfqpoint{2.943045in}{2.583924in}}%
\pgfpathlineto{\pgfqpoint{2.958097in}{2.650559in}}%
\pgfpathlineto{\pgfqpoint{2.979007in}{2.715625in}}%
\pgfpathlineto{\pgfqpoint{3.004577in}{2.779030in}}%
\pgfpathlineto{\pgfqpoint{3.033830in}{2.840839in}}%
\pgfusepath{stroke}%
\end{pgfscope}%
\begin{pgfscope}%
\pgfpathrectangle{\pgfqpoint{0.647939in}{0.492442in}}{\pgfqpoint{3.079299in}{3.079299in}}%
\pgfusepath{clip}%
\pgfsetbuttcap%
\pgfsetroundjoin%
\pgfsetlinewidth{0.301125pt}%
\definecolor{currentstroke}{rgb}{0.500000,0.500000,0.500000}%
\pgfsetstrokecolor{currentstroke}%
\pgfsetstrokeopacity{0.300000}%
\pgfsetdash{}{0pt}%
\pgfpathmoveto{\pgfqpoint{1.917470in}{2.993431in}}%
\pgfpathlineto{\pgfqpoint{1.985695in}{2.998627in}}%
\pgfpathlineto{\pgfqpoint{2.054025in}{3.002248in}}%
\pgfpathlineto{\pgfqpoint{2.122407in}{3.004718in}}%
\pgfpathlineto{\pgfqpoint{2.190810in}{3.006618in}}%
\pgfpathlineto{\pgfqpoint{2.259206in}{3.008699in}}%
\pgfpathlineto{\pgfqpoint{2.327557in}{3.011869in}}%
\pgfusepath{stroke}%
\end{pgfscope}%
\begin{pgfscope}%
\pgfpathrectangle{\pgfqpoint{0.647939in}{0.492442in}}{\pgfqpoint{3.079299in}{3.079299in}}%
\pgfusepath{clip}%
\pgfsetbuttcap%
\pgfsetroundjoin%
\pgfsetlinewidth{0.301125pt}%
\definecolor{currentstroke}{rgb}{0.500000,0.500000,0.500000}%
\pgfsetstrokecolor{currentstroke}%
\pgfsetstrokeopacity{0.300000}%
\pgfsetdash{}{0pt}%
\pgfpathmoveto{\pgfqpoint{2.389394in}{1.107202in}}%
\pgfpathlineto{\pgfqpoint{2.321092in}{1.111295in}}%
\pgfpathlineto{\pgfqpoint{2.252722in}{1.114061in}}%
\pgfpathlineto{\pgfqpoint{2.184326in}{1.116189in}}%
\pgfpathlineto{\pgfqpoint{2.115939in}{1.118558in}}%
\pgfpathlineto{\pgfqpoint{2.047620in}{1.122299in}}%
\pgfusepath{stroke}%
\end{pgfscope}%
\begin{pgfscope}%
\pgfpathrectangle{\pgfqpoint{0.647939in}{0.492442in}}{\pgfqpoint{3.079299in}{3.079299in}}%
\pgfusepath{clip}%
\pgfsetbuttcap%
\pgfsetroundjoin%
\pgfsetlinewidth{0.301125pt}%
\definecolor{currentstroke}{rgb}{0.500000,0.500000,0.500000}%
\pgfsetstrokecolor{currentstroke}%
\pgfsetstrokeopacity{0.300000}%
\pgfsetdash{}{0pt}%
\pgfpathmoveto{\pgfqpoint{2.742245in}{2.796919in}}%
\pgfpathlineto{\pgfqpoint{2.791882in}{2.843972in}}%
\pgfpathlineto{\pgfqpoint{2.840135in}{2.892455in}}%
\pgfpathlineto{\pgfqpoint{2.887429in}{2.941885in}}%
\pgfpathlineto{\pgfqpoint{2.934098in}{2.991911in}}%
\pgfpathlineto{\pgfqpoint{2.980401in}{3.042281in}}%
\pgfusepath{stroke}%
\end{pgfscope}%
\begin{pgfscope}%
\pgfpathrectangle{\pgfqpoint{0.647939in}{0.492442in}}{\pgfqpoint{3.079299in}{3.079299in}}%
\pgfusepath{clip}%
\pgfsetbuttcap%
\pgfsetroundjoin%
\pgfsetlinewidth{0.301125pt}%
\definecolor{currentstroke}{rgb}{0.500000,0.500000,0.500000}%
\pgfsetstrokecolor{currentstroke}%
\pgfsetstrokeopacity{0.300000}%
\pgfsetdash{}{0pt}%
\pgfpathmoveto{\pgfqpoint{1.292176in}{1.292401in}}%
\pgfpathlineto{\pgfqpoint{1.347780in}{1.332251in}}%
\pgfpathlineto{\pgfqpoint{1.401950in}{1.374028in}}%
\pgfpathlineto{\pgfqpoint{1.454820in}{1.417446in}}%
\pgfpathlineto{\pgfqpoint{1.506581in}{1.462181in}}%
\pgfpathlineto{\pgfqpoint{1.557453in}{1.507919in}}%
\pgfpathlineto{\pgfqpoint{1.607695in}{1.554353in}}%
\pgfpathlineto{\pgfqpoint{1.657602in}{1.601138in}}%
\pgfusepath{stroke}%
\end{pgfscope}%
\begin{pgfscope}%
\pgfpathrectangle{\pgfqpoint{0.647939in}{0.492442in}}{\pgfqpoint{3.079299in}{3.079299in}}%
\pgfusepath{clip}%
\pgfsetbuttcap%
\pgfsetroundjoin%
\pgfsetlinewidth{0.301125pt}%
\definecolor{currentstroke}{rgb}{0.500000,0.500000,0.500000}%
\pgfsetstrokecolor{currentstroke}%
\pgfsetstrokeopacity{0.300000}%
\pgfsetdash{}{0pt}%
\pgfpathmoveto{\pgfqpoint{1.417764in}{2.451996in}}%
\pgfpathlineto{\pgfqpoint{1.482963in}{2.472768in}}%
\pgfpathlineto{\pgfqpoint{1.548371in}{2.492870in}}%
\pgfpathlineto{\pgfqpoint{1.614092in}{2.511918in}}%
\pgfpathlineto{\pgfqpoint{1.680210in}{2.529528in}}%
\pgfpathlineto{\pgfqpoint{1.746778in}{2.545339in}}%
\pgfpathlineto{\pgfqpoint{1.813809in}{2.559045in}}%
\pgfpathlineto{\pgfqpoint{1.881272in}{2.570424in}}%
\pgfpathlineto{\pgfqpoint{1.949098in}{2.579391in}}%
\pgfpathlineto{\pgfqpoint{2.017192in}{2.586049in}}%
\pgfpathlineto{\pgfqpoint{2.085453in}{2.590764in}}%
\pgfpathlineto{\pgfqpoint{2.153793in}{2.594207in}}%
\pgfpathlineto{\pgfqpoint{2.222147in}{2.597402in}}%
\pgfpathlineto{\pgfqpoint{2.290426in}{2.601795in}}%
\pgfpathlineto{\pgfqpoint{2.358412in}{2.609268in}}%
\pgfpathlineto{\pgfqpoint{2.425563in}{2.621970in}}%
\pgfpathlineto{\pgfqpoint{2.490935in}{2.641672in}}%
\pgfpathlineto{\pgfqpoint{2.553491in}{2.668959in}}%
\pgfpathlineto{\pgfqpoint{2.612646in}{2.703064in}}%
\pgfusepath{stroke}%
\end{pgfscope}%
\begin{pgfscope}%
\pgfpathrectangle{\pgfqpoint{0.647939in}{0.492442in}}{\pgfqpoint{3.079299in}{3.079299in}}%
\pgfusepath{clip}%
\pgfsetbuttcap%
\pgfsetroundjoin%
\pgfsetlinewidth{0.301125pt}%
\definecolor{currentstroke}{rgb}{0.500000,0.500000,0.500000}%
\pgfsetstrokecolor{currentstroke}%
\pgfsetstrokeopacity{0.300000}%
\pgfsetdash{}{0pt}%
\pgfpathmoveto{\pgfqpoint{2.975560in}{2.127537in}}%
\pgfpathlineto{\pgfqpoint{2.940917in}{2.186482in}}%
\pgfpathlineto{\pgfqpoint{2.911029in}{2.247940in}}%
\pgfpathlineto{\pgfqpoint{2.887429in}{2.312028in}}%
\pgfpathlineto{\pgfqpoint{2.871854in}{2.378477in}}%
\pgfpathlineto{\pgfqpoint{2.865604in}{2.446409in}}%
\pgfpathlineto{\pgfqpoint{2.868953in}{2.514565in}}%
\pgfpathlineto{\pgfqpoint{2.880985in}{2.581771in}}%
\pgfpathlineto{\pgfqpoint{2.900108in}{2.647352in}}%
\pgfusepath{stroke}%
\end{pgfscope}%
\begin{pgfscope}%
\pgfpathrectangle{\pgfqpoint{0.647939in}{0.492442in}}{\pgfqpoint{3.079299in}{3.079299in}}%
\pgfusepath{clip}%
\pgfsetbuttcap%
\pgfsetroundjoin%
\pgfsetlinewidth{0.301125pt}%
\definecolor{currentstroke}{rgb}{0.500000,0.500000,0.500000}%
\pgfsetstrokecolor{currentstroke}%
\pgfsetstrokeopacity{0.300000}%
\pgfsetdash{}{0pt}%
\pgfpathmoveto{\pgfqpoint{2.919278in}{1.870703in}}%
\pgfpathlineto{\pgfqpoint{2.868122in}{1.916141in}}%
\pgfpathlineto{\pgfqpoint{2.817445in}{1.962108in}}%
\pgfpathlineto{\pgfqpoint{2.767466in}{2.008828in}}%
\pgfpathlineto{\pgfqpoint{2.718630in}{2.056729in}}%
\pgfpathlineto{\pgfqpoint{2.671816in}{2.106585in}}%
\pgfpathlineto{\pgfqpoint{2.628912in}{2.159754in}}%
\pgfpathlineto{\pgfqpoint{2.594105in}{2.218252in}}%
\pgfpathlineto{\pgfqpoint{2.594105in}{2.218252in}}%
\pgfpathlineto{\pgfqpoint{2.577201in}{2.269153in}}%
\pgfpathlineto{\pgfqpoint{2.574076in}{2.323667in}}%
\pgfpathlineto{\pgfqpoint{2.582812in}{2.372545in}}%
\pgfusepath{stroke}%
\end{pgfscope}%
\begin{pgfscope}%
\pgfpathrectangle{\pgfqpoint{0.647939in}{0.492442in}}{\pgfqpoint{3.079299in}{3.079299in}}%
\pgfusepath{clip}%
\pgfsetbuttcap%
\pgfsetroundjoin%
\pgfsetlinewidth{0.301125pt}%
\definecolor{currentstroke}{rgb}{0.500000,0.500000,0.500000}%
\pgfsetstrokecolor{currentstroke}%
\pgfsetstrokeopacity{0.300000}%
\pgfsetdash{}{0pt}%
\pgfpathmoveto{\pgfqpoint{2.398362in}{2.471896in}}%
\pgfpathlineto{\pgfqpoint{2.462106in}{2.496107in}}%
\pgfpathlineto{\pgfqpoint{2.521681in}{2.529299in}}%
\pgfpathlineto{\pgfqpoint{2.576903in}{2.569471in}}%
\pgfpathlineto{\pgfqpoint{2.628507in}{2.614266in}}%
\pgfpathlineto{\pgfqpoint{2.677477in}{2.661948in}}%
\pgfusepath{stroke}%
\end{pgfscope}%
\begin{pgfscope}%
\pgfpathrectangle{\pgfqpoint{0.647939in}{0.492442in}}{\pgfqpoint{3.079299in}{3.079299in}}%
\pgfusepath{clip}%
\pgfsetbuttcap%
\pgfsetroundjoin%
\pgfsetlinewidth{0.301125pt}%
\definecolor{currentstroke}{rgb}{0.500000,0.500000,0.500000}%
\pgfsetstrokecolor{currentstroke}%
\pgfsetstrokeopacity{0.300000}%
\pgfsetdash{}{0pt}%
\pgfpathmoveto{\pgfqpoint{2.802933in}{1.852070in}}%
\pgfpathlineto{\pgfqpoint{2.747461in}{1.892124in}}%
\pgfpathlineto{\pgfqpoint{2.691612in}{1.931645in}}%
\pgfpathlineto{\pgfqpoint{2.635412in}{1.970659in}}%
\pgfpathlineto{\pgfqpoint{2.579011in}{2.009380in}}%
\pgfpathlineto{\pgfqpoint{2.522861in}{2.048467in}}%
\pgfpathlineto{\pgfqpoint{2.468339in}{2.089692in}}%
\pgfpathlineto{\pgfqpoint{2.421003in}{2.138217in}}%
\pgfpathlineto{\pgfqpoint{2.421003in}{2.138217in}}%
\pgfpathlineto{\pgfqpoint{2.405786in}{2.167717in}}%
\pgfpathlineto{\pgfqpoint{2.405786in}{2.167717in}}%
\pgfpathlineto{\pgfqpoint{2.402481in}{2.197973in}}%
\pgfusepath{stroke}%
\end{pgfscope}%
\begin{pgfscope}%
\pgfpathrectangle{\pgfqpoint{0.647939in}{0.492442in}}{\pgfqpoint{3.079299in}{3.079299in}}%
\pgfusepath{clip}%
\pgfsetbuttcap%
\pgfsetroundjoin%
\pgfsetlinewidth{0.301125pt}%
\definecolor{currentstroke}{rgb}{0.500000,0.500000,0.500000}%
\pgfsetstrokecolor{currentstroke}%
\pgfsetstrokeopacity{0.300000}%
\pgfsetdash{}{0pt}%
\pgfpathmoveto{\pgfqpoint{1.767684in}{2.312028in}}%
\pgfpathlineto{\pgfqpoint{1.834067in}{2.328581in}}%
\pgfpathlineto{\pgfqpoint{1.901046in}{2.342513in}}%
\pgfpathlineto{\pgfqpoint{1.968541in}{2.353669in}}%
\pgfpathlineto{\pgfqpoint{2.036429in}{2.362137in}}%
\pgfpathlineto{\pgfqpoint{2.104566in}{2.368362in}}%
\pgfpathlineto{\pgfqpoint{2.172814in}{2.373318in}}%
\pgfpathlineto{\pgfqpoint{2.241017in}{2.378790in}}%
\pgfpathlineto{\pgfqpoint{2.308809in}{2.387627in}}%
\pgfusepath{stroke}%
\end{pgfscope}%
\begin{pgfscope}%
\pgfpathrectangle{\pgfqpoint{0.647939in}{0.492442in}}{\pgfqpoint{3.079299in}{3.079299in}}%
\pgfusepath{clip}%
\pgfsetbuttcap%
\pgfsetroundjoin%
\pgfsetlinewidth{0.301125pt}%
\definecolor{currentstroke}{rgb}{0.500000,0.500000,0.500000}%
\pgfsetstrokecolor{currentstroke}%
\pgfsetstrokeopacity{0.300000}%
\pgfsetdash{}{0pt}%
\pgfpathmoveto{\pgfqpoint{1.640171in}{2.052446in}}%
\pgfpathlineto{\pgfqpoint{1.703589in}{2.078136in}}%
\pgfpathlineto{\pgfqpoint{1.767684in}{2.102076in}}%
\pgfpathlineto{\pgfqpoint{1.832568in}{2.123769in}}%
\pgfpathlineto{\pgfqpoint{1.898287in}{2.142766in}}%
\pgfpathlineto{\pgfqpoint{1.964801in}{2.158739in}}%
\pgfpathlineto{\pgfqpoint{2.031985in}{2.171621in}}%
\pgfpathlineto{\pgfqpoint{2.099635in}{2.181828in}}%
\pgfusepath{stroke}%
\end{pgfscope}%
\begin{pgfscope}%
\pgfpathrectangle{\pgfqpoint{0.647939in}{0.492442in}}{\pgfqpoint{3.079299in}{3.079299in}}%
\pgfusepath{clip}%
\pgfsetbuttcap%
\pgfsetroundjoin%
\pgfsetlinewidth{0.301125pt}%
\definecolor{currentstroke}{rgb}{0.500000,0.500000,0.500000}%
\pgfsetstrokecolor{currentstroke}%
\pgfsetstrokeopacity{0.300000}%
\pgfsetdash{}{0pt}%
\pgfpathmoveto{\pgfqpoint{2.530423in}{1.719890in}}%
\pgfpathlineto{\pgfqpoint{2.464414in}{1.737796in}}%
\pgfpathlineto{\pgfqpoint{2.397541in}{1.752155in}}%
\pgfpathlineto{\pgfqpoint{2.330025in}{1.763159in}}%
\pgfpathlineto{\pgfqpoint{2.262120in}{1.771541in}}%
\pgfpathlineto{\pgfqpoint{2.194099in}{1.778984in}}%
\pgfpathlineto{\pgfqpoint{2.126563in}{1.789411in}}%
\pgfpathlineto{\pgfqpoint{2.126563in}{1.789411in}}%
\pgfpathlineto{\pgfqpoint{2.082731in}{1.803063in}}%
\pgfpathlineto{\pgfqpoint{2.082731in}{1.803063in}}%
\pgfusepath{stroke}%
\end{pgfscope}%
\begin{pgfscope}%
\pgfpathrectangle{\pgfqpoint{0.647939in}{0.492442in}}{\pgfqpoint{3.079299in}{3.079299in}}%
\pgfusepath{clip}%
\pgfsetbuttcap%
\pgfsetroundjoin%
\pgfsetlinewidth{0.301125pt}%
\definecolor{currentstroke}{rgb}{0.500000,0.500000,0.500000}%
\pgfsetstrokecolor{currentstroke}%
\pgfsetstrokeopacity{0.300000}%
\pgfsetdash{}{0pt}%
\pgfpathmoveto{\pgfqpoint{2.463007in}{1.872321in}}%
\pgfpathlineto{\pgfqpoint{2.397541in}{1.892124in}}%
\pgfpathlineto{\pgfqpoint{2.331211in}{1.908851in}}%
\pgfpathlineto{\pgfqpoint{2.264410in}{1.923653in}}%
\pgfpathlineto{\pgfqpoint{2.198598in}{1.941553in}}%
\pgfpathlineto{\pgfqpoint{2.198598in}{1.941553in}}%
\pgfpathlineto{\pgfqpoint{2.175552in}{1.952282in}}%
\pgfpathlineto{\pgfqpoint{2.175552in}{1.952282in}}%
\pgfpathlineto{\pgfqpoint{2.163120in}{1.965289in}}%
\pgfpathlineto{\pgfqpoint{2.161721in}{1.982718in}}%
\pgfusepath{stroke}%
\end{pgfscope}%
\begin{pgfscope}%
\pgfpathrectangle{\pgfqpoint{0.647939in}{0.492442in}}{\pgfqpoint{3.079299in}{3.079299in}}%
\pgfusepath{clip}%
\pgfsetbuttcap%
\pgfsetroundjoin%
\pgfsetlinewidth{0.301125pt}%
\definecolor{currentstroke}{rgb}{0.500000,0.500000,0.500000}%
\pgfsetstrokecolor{currentstroke}%
\pgfsetstrokeopacity{0.300000}%
\pgfsetdash{}{0pt}%
\pgfpathmoveto{\pgfqpoint{1.984026in}{2.215995in}}%
\pgfpathlineto{\pgfqpoint{2.051628in}{2.226497in}}%
\pgfpathlineto{\pgfqpoint{2.119560in}{2.234670in}}%
\pgfpathlineto{\pgfqpoint{2.187589in}{2.242044in}}%
\pgfpathlineto{\pgfqpoint{2.255259in}{2.251823in}}%
\pgfpathlineto{\pgfqpoint{2.321000in}{2.269593in}}%
\pgfusepath{stroke}%
\end{pgfscope}%
\begin{pgfscope}%
\pgfpathrectangle{\pgfqpoint{0.647939in}{0.492442in}}{\pgfqpoint{3.079299in}{3.079299in}}%
\pgfusepath{clip}%
\pgfsetbuttcap%
\pgfsetroundjoin%
\pgfsetlinewidth{0.301125pt}%
\definecolor{currentstroke}{rgb}{0.500000,0.500000,0.500000}%
\pgfsetstrokecolor{currentstroke}%
\pgfsetstrokeopacity{0.300000}%
\pgfsetdash{}{0pt}%
\pgfpathmoveto{\pgfqpoint{2.568733in}{1.952914in}}%
\pgfpathlineto{\pgfqpoint{2.508820in}{1.985950in}}%
\pgfpathlineto{\pgfqpoint{2.448458in}{2.018164in}}%
\pgfpathlineto{\pgfqpoint{2.388629in}{2.051292in}}%
\pgfpathlineto{\pgfqpoint{2.356645in}{2.071974in}}%
\pgfpathlineto{\pgfqpoint{2.327557in}{2.102076in}}%
\pgfpathlineto{\pgfqpoint{2.327557in}{2.102076in}}%
\pgfpathlineto{\pgfqpoint{2.327557in}{2.102076in}}%
\pgfpathlineto{\pgfqpoint{2.320930in}{2.124795in}}%
\pgfusepath{stroke}%
\end{pgfscope}%
\begin{pgfscope}%
\pgfpathrectangle{\pgfqpoint{0.647939in}{0.492442in}}{\pgfqpoint{3.079299in}{3.079299in}}%
\pgfusepath{clip}%
\pgfsetroundcap%
\pgfsetroundjoin%
\pgfsetlinewidth{0.301125pt}%
\definecolor{currentstroke}{rgb}{0.500000,0.500000,0.500000}%
\pgfsetstrokecolor{currentstroke}%
\pgfsetstrokeopacity{0.300000}%
\pgfsetdash{}{0pt}%
\pgfpathmoveto{\pgfqpoint{2.112461in}{1.967839in}}%
\pgfusepath{stroke}%
\end{pgfscope}%
\begin{pgfscope}%
\pgfpathrectangle{\pgfqpoint{0.647939in}{0.492442in}}{\pgfqpoint{3.079299in}{3.079299in}}%
\pgfusepath{clip}%
\pgfsetroundcap%
\pgfsetroundjoin%
\definecolor{currentfill}{rgb}{0.500000,0.500000,0.500000}%
\pgfsetfillcolor{currentfill}%
\pgfsetfillopacity{0.300000}%
\pgfsetlinewidth{0.301125pt}%
\definecolor{currentstroke}{rgb}{0.500000,0.500000,0.500000}%
\pgfsetstrokecolor{currentstroke}%
\pgfsetstrokeopacity{0.300000}%
\pgfsetdash{}{0pt}%
\pgfpathmoveto{\pgfqpoint{0.000000in}{0.000000in}}%
\pgfpathlineto{\pgfqpoint{0.000000in}{0.000000in}}%
\pgfpathclose%
\pgfusepath{stroke,fill}%
\end{pgfscope}%
\begin{pgfscope}%
\pgfpathrectangle{\pgfqpoint{0.647939in}{0.492442in}}{\pgfqpoint{3.079299in}{3.079299in}}%
\pgfusepath{clip}%
\pgfsetroundcap%
\pgfsetroundjoin%
\pgfsetlinewidth{0.301125pt}%
\definecolor{currentstroke}{rgb}{0.500000,0.500000,0.500000}%
\pgfsetstrokecolor{currentstroke}%
\pgfsetstrokeopacity{0.300000}%
\pgfsetdash{}{0pt}%
\pgfpathmoveto{\pgfqpoint{1.063550in}{0.600488in}}%
\pgfusepath{stroke}%
\end{pgfscope}%
\begin{pgfscope}%
\pgfpathrectangle{\pgfqpoint{0.647939in}{0.492442in}}{\pgfqpoint{3.079299in}{3.079299in}}%
\pgfusepath{clip}%
\pgfsetroundcap%
\pgfsetroundjoin%
\definecolor{currentfill}{rgb}{0.500000,0.500000,0.500000}%
\pgfsetfillcolor{currentfill}%
\pgfsetfillopacity{0.300000}%
\pgfsetlinewidth{0.301125pt}%
\definecolor{currentstroke}{rgb}{0.500000,0.500000,0.500000}%
\pgfsetstrokecolor{currentstroke}%
\pgfsetstrokeopacity{0.300000}%
\pgfsetdash{}{0pt}%
\pgfpathmoveto{\pgfqpoint{0.000000in}{0.000000in}}%
\pgfpathlineto{\pgfqpoint{0.000000in}{0.000000in}}%
\pgfpathclose%
\pgfusepath{stroke,fill}%
\end{pgfscope}%
\begin{pgfscope}%
\pgfpathrectangle{\pgfqpoint{0.647939in}{0.492442in}}{\pgfqpoint{3.079299in}{3.079299in}}%
\pgfusepath{clip}%
\pgfsetroundcap%
\pgfsetroundjoin%
\pgfsetlinewidth{0.301125pt}%
\definecolor{currentstroke}{rgb}{0.500000,0.500000,0.500000}%
\pgfsetstrokecolor{currentstroke}%
\pgfsetstrokeopacity{0.300000}%
\pgfsetdash{}{0pt}%
\pgfpathmoveto{\pgfqpoint{1.310651in}{0.738021in}}%
\pgfusepath{stroke}%
\end{pgfscope}%
\begin{pgfscope}%
\pgfpathrectangle{\pgfqpoint{0.647939in}{0.492442in}}{\pgfqpoint{3.079299in}{3.079299in}}%
\pgfusepath{clip}%
\pgfsetroundcap%
\pgfsetroundjoin%
\definecolor{currentfill}{rgb}{0.500000,0.500000,0.500000}%
\pgfsetfillcolor{currentfill}%
\pgfsetfillopacity{0.300000}%
\pgfsetlinewidth{0.301125pt}%
\definecolor{currentstroke}{rgb}{0.500000,0.500000,0.500000}%
\pgfsetstrokecolor{currentstroke}%
\pgfsetstrokeopacity{0.300000}%
\pgfsetdash{}{0pt}%
\pgfpathmoveto{\pgfqpoint{0.000000in}{0.000000in}}%
\pgfpathlineto{\pgfqpoint{0.000000in}{0.000000in}}%
\pgfpathclose%
\pgfusepath{stroke,fill}%
\end{pgfscope}%
\begin{pgfscope}%
\pgfpathrectangle{\pgfqpoint{0.647939in}{0.492442in}}{\pgfqpoint{3.079299in}{3.079299in}}%
\pgfusepath{clip}%
\pgfsetroundcap%
\pgfsetroundjoin%
\pgfsetlinewidth{0.301125pt}%
\definecolor{currentstroke}{rgb}{0.500000,0.500000,0.500000}%
\pgfsetstrokecolor{currentstroke}%
\pgfsetstrokeopacity{0.300000}%
\pgfsetdash{}{0pt}%
\pgfpathmoveto{\pgfqpoint{1.366961in}{0.655316in}}%
\pgfusepath{stroke}%
\end{pgfscope}%
\begin{pgfscope}%
\pgfpathrectangle{\pgfqpoint{0.647939in}{0.492442in}}{\pgfqpoint{3.079299in}{3.079299in}}%
\pgfusepath{clip}%
\pgfsetroundcap%
\pgfsetroundjoin%
\definecolor{currentfill}{rgb}{0.500000,0.500000,0.500000}%
\pgfsetfillcolor{currentfill}%
\pgfsetfillopacity{0.300000}%
\pgfsetlinewidth{0.301125pt}%
\definecolor{currentstroke}{rgb}{0.500000,0.500000,0.500000}%
\pgfsetstrokecolor{currentstroke}%
\pgfsetstrokeopacity{0.300000}%
\pgfsetdash{}{0pt}%
\pgfpathmoveto{\pgfqpoint{0.000000in}{0.000000in}}%
\pgfpathlineto{\pgfqpoint{0.000000in}{0.000000in}}%
\pgfpathclose%
\pgfusepath{stroke,fill}%
\end{pgfscope}%
\begin{pgfscope}%
\pgfpathrectangle{\pgfqpoint{0.647939in}{0.492442in}}{\pgfqpoint{3.079299in}{3.079299in}}%
\pgfusepath{clip}%
\pgfsetroundcap%
\pgfsetroundjoin%
\pgfsetlinewidth{0.301125pt}%
\definecolor{currentstroke}{rgb}{0.500000,0.500000,0.500000}%
\pgfsetstrokecolor{currentstroke}%
\pgfsetstrokeopacity{0.300000}%
\pgfsetdash{}{0pt}%
\pgfpathmoveto{\pgfqpoint{1.462285in}{0.837211in}}%
\pgfusepath{stroke}%
\end{pgfscope}%
\begin{pgfscope}%
\pgfpathrectangle{\pgfqpoint{0.647939in}{0.492442in}}{\pgfqpoint{3.079299in}{3.079299in}}%
\pgfusepath{clip}%
\pgfsetroundcap%
\pgfsetroundjoin%
\definecolor{currentfill}{rgb}{0.500000,0.500000,0.500000}%
\pgfsetfillcolor{currentfill}%
\pgfsetfillopacity{0.300000}%
\pgfsetlinewidth{0.301125pt}%
\definecolor{currentstroke}{rgb}{0.500000,0.500000,0.500000}%
\pgfsetstrokecolor{currentstroke}%
\pgfsetstrokeopacity{0.300000}%
\pgfsetdash{}{0pt}%
\pgfpathmoveto{\pgfqpoint{0.000000in}{0.000000in}}%
\pgfpathlineto{\pgfqpoint{0.000000in}{0.000000in}}%
\pgfpathclose%
\pgfusepath{stroke,fill}%
\end{pgfscope}%
\begin{pgfscope}%
\pgfpathrectangle{\pgfqpoint{0.647939in}{0.492442in}}{\pgfqpoint{3.079299in}{3.079299in}}%
\pgfusepath{clip}%
\pgfsetroundcap%
\pgfsetroundjoin%
\pgfsetlinewidth{0.301125pt}%
\definecolor{currentstroke}{rgb}{0.500000,0.500000,0.500000}%
\pgfsetstrokecolor{currentstroke}%
\pgfsetstrokeopacity{0.300000}%
\pgfsetdash{}{0pt}%
\pgfpathmoveto{\pgfqpoint{1.746216in}{0.522988in}}%
\pgfusepath{stroke}%
\end{pgfscope}%
\begin{pgfscope}%
\pgfpathrectangle{\pgfqpoint{0.647939in}{0.492442in}}{\pgfqpoint{3.079299in}{3.079299in}}%
\pgfusepath{clip}%
\pgfsetroundcap%
\pgfsetroundjoin%
\definecolor{currentfill}{rgb}{0.500000,0.500000,0.500000}%
\pgfsetfillcolor{currentfill}%
\pgfsetfillopacity{0.300000}%
\pgfsetlinewidth{0.301125pt}%
\definecolor{currentstroke}{rgb}{0.500000,0.500000,0.500000}%
\pgfsetstrokecolor{currentstroke}%
\pgfsetstrokeopacity{0.300000}%
\pgfsetdash{}{0pt}%
\pgfpathmoveto{\pgfqpoint{0.000000in}{0.000000in}}%
\pgfpathlineto{\pgfqpoint{0.000000in}{0.000000in}}%
\pgfpathclose%
\pgfusepath{stroke,fill}%
\end{pgfscope}%
\begin{pgfscope}%
\pgfpathrectangle{\pgfqpoint{0.647939in}{0.492442in}}{\pgfqpoint{3.079299in}{3.079299in}}%
\pgfusepath{clip}%
\pgfsetroundcap%
\pgfsetroundjoin%
\pgfsetlinewidth{0.301125pt}%
\definecolor{currentstroke}{rgb}{0.500000,0.500000,0.500000}%
\pgfsetstrokecolor{currentstroke}%
\pgfsetstrokeopacity{0.300000}%
\pgfsetdash{}{0pt}%
\pgfpathmoveto{\pgfqpoint{2.162975in}{0.496036in}}%
\pgfusepath{stroke}%
\end{pgfscope}%
\begin{pgfscope}%
\pgfpathrectangle{\pgfqpoint{0.647939in}{0.492442in}}{\pgfqpoint{3.079299in}{3.079299in}}%
\pgfusepath{clip}%
\pgfsetroundcap%
\pgfsetroundjoin%
\definecolor{currentfill}{rgb}{0.500000,0.500000,0.500000}%
\pgfsetfillcolor{currentfill}%
\pgfsetfillopacity{0.300000}%
\pgfsetlinewidth{0.301125pt}%
\definecolor{currentstroke}{rgb}{0.500000,0.500000,0.500000}%
\pgfsetstrokecolor{currentstroke}%
\pgfsetstrokeopacity{0.300000}%
\pgfsetdash{}{0pt}%
\pgfpathmoveto{\pgfqpoint{0.000000in}{0.000000in}}%
\pgfpathlineto{\pgfqpoint{0.000000in}{0.000000in}}%
\pgfpathclose%
\pgfusepath{stroke,fill}%
\end{pgfscope}%
\begin{pgfscope}%
\pgfpathrectangle{\pgfqpoint{0.647939in}{0.492442in}}{\pgfqpoint{3.079299in}{3.079299in}}%
\pgfusepath{clip}%
\pgfsetroundcap%
\pgfsetroundjoin%
\pgfsetlinewidth{0.301125pt}%
\definecolor{currentstroke}{rgb}{0.500000,0.500000,0.500000}%
\pgfsetstrokecolor{currentstroke}%
\pgfsetstrokeopacity{0.300000}%
\pgfsetdash{}{0pt}%
\pgfpathmoveto{\pgfqpoint{2.311851in}{0.536197in}}%
\pgfusepath{stroke}%
\end{pgfscope}%
\begin{pgfscope}%
\pgfpathrectangle{\pgfqpoint{0.647939in}{0.492442in}}{\pgfqpoint{3.079299in}{3.079299in}}%
\pgfusepath{clip}%
\pgfsetroundcap%
\pgfsetroundjoin%
\definecolor{currentfill}{rgb}{0.500000,0.500000,0.500000}%
\pgfsetfillcolor{currentfill}%
\pgfsetfillopacity{0.300000}%
\pgfsetlinewidth{0.301125pt}%
\definecolor{currentstroke}{rgb}{0.500000,0.500000,0.500000}%
\pgfsetstrokecolor{currentstroke}%
\pgfsetstrokeopacity{0.300000}%
\pgfsetdash{}{0pt}%
\pgfpathmoveto{\pgfqpoint{0.000000in}{0.000000in}}%
\pgfpathlineto{\pgfqpoint{0.000000in}{0.000000in}}%
\pgfpathclose%
\pgfusepath{stroke,fill}%
\end{pgfscope}%
\begin{pgfscope}%
\pgfpathrectangle{\pgfqpoint{0.647939in}{0.492442in}}{\pgfqpoint{3.079299in}{3.079299in}}%
\pgfusepath{clip}%
\pgfsetroundcap%
\pgfsetroundjoin%
\pgfsetlinewidth{0.301125pt}%
\definecolor{currentstroke}{rgb}{0.500000,0.500000,0.500000}%
\pgfsetstrokecolor{currentstroke}%
\pgfsetstrokeopacity{0.300000}%
\pgfsetdash{}{0pt}%
\pgfpathmoveto{\pgfqpoint{2.797822in}{0.532653in}}%
\pgfusepath{stroke}%
\end{pgfscope}%
\begin{pgfscope}%
\pgfpathrectangle{\pgfqpoint{0.647939in}{0.492442in}}{\pgfqpoint{3.079299in}{3.079299in}}%
\pgfusepath{clip}%
\pgfsetroundcap%
\pgfsetroundjoin%
\definecolor{currentfill}{rgb}{0.500000,0.500000,0.500000}%
\pgfsetfillcolor{currentfill}%
\pgfsetfillopacity{0.300000}%
\pgfsetlinewidth{0.301125pt}%
\definecolor{currentstroke}{rgb}{0.500000,0.500000,0.500000}%
\pgfsetstrokecolor{currentstroke}%
\pgfsetstrokeopacity{0.300000}%
\pgfsetdash{}{0pt}%
\pgfpathmoveto{\pgfqpoint{0.000000in}{0.000000in}}%
\pgfpathlineto{\pgfqpoint{0.000000in}{0.000000in}}%
\pgfpathclose%
\pgfusepath{stroke,fill}%
\end{pgfscope}%
\begin{pgfscope}%
\pgfpathrectangle{\pgfqpoint{0.647939in}{0.492442in}}{\pgfqpoint{3.079299in}{3.079299in}}%
\pgfusepath{clip}%
\pgfsetroundcap%
\pgfsetroundjoin%
\pgfsetlinewidth{0.301125pt}%
\definecolor{currentstroke}{rgb}{0.500000,0.500000,0.500000}%
\pgfsetstrokecolor{currentstroke}%
\pgfsetstrokeopacity{0.300000}%
\pgfsetdash{}{0pt}%
\pgfpathmoveto{\pgfqpoint{2.270795in}{0.662158in}}%
\pgfusepath{stroke}%
\end{pgfscope}%
\begin{pgfscope}%
\pgfpathrectangle{\pgfqpoint{0.647939in}{0.492442in}}{\pgfqpoint{3.079299in}{3.079299in}}%
\pgfusepath{clip}%
\pgfsetroundcap%
\pgfsetroundjoin%
\definecolor{currentfill}{rgb}{0.500000,0.500000,0.500000}%
\pgfsetfillcolor{currentfill}%
\pgfsetfillopacity{0.300000}%
\pgfsetlinewidth{0.301125pt}%
\definecolor{currentstroke}{rgb}{0.500000,0.500000,0.500000}%
\pgfsetstrokecolor{currentstroke}%
\pgfsetstrokeopacity{0.300000}%
\pgfsetdash{}{0pt}%
\pgfpathmoveto{\pgfqpoint{0.000000in}{0.000000in}}%
\pgfpathlineto{\pgfqpoint{0.000000in}{0.000000in}}%
\pgfpathclose%
\pgfusepath{stroke,fill}%
\end{pgfscope}%
\begin{pgfscope}%
\pgfpathrectangle{\pgfqpoint{0.647939in}{0.492442in}}{\pgfqpoint{3.079299in}{3.079299in}}%
\pgfusepath{clip}%
\pgfsetroundcap%
\pgfsetroundjoin%
\pgfsetlinewidth{0.301125pt}%
\definecolor{currentstroke}{rgb}{0.500000,0.500000,0.500000}%
\pgfsetstrokecolor{currentstroke}%
\pgfsetstrokeopacity{0.300000}%
\pgfsetdash{}{0pt}%
\pgfpathmoveto{\pgfqpoint{2.564436in}{0.728629in}}%
\pgfusepath{stroke}%
\end{pgfscope}%
\begin{pgfscope}%
\pgfpathrectangle{\pgfqpoint{0.647939in}{0.492442in}}{\pgfqpoint{3.079299in}{3.079299in}}%
\pgfusepath{clip}%
\pgfsetroundcap%
\pgfsetroundjoin%
\definecolor{currentfill}{rgb}{0.500000,0.500000,0.500000}%
\pgfsetfillcolor{currentfill}%
\pgfsetfillopacity{0.300000}%
\pgfsetlinewidth{0.301125pt}%
\definecolor{currentstroke}{rgb}{0.500000,0.500000,0.500000}%
\pgfsetstrokecolor{currentstroke}%
\pgfsetstrokeopacity{0.300000}%
\pgfsetdash{}{0pt}%
\pgfpathmoveto{\pgfqpoint{0.000000in}{0.000000in}}%
\pgfpathlineto{\pgfqpoint{0.000000in}{0.000000in}}%
\pgfpathclose%
\pgfusepath{stroke,fill}%
\end{pgfscope}%
\begin{pgfscope}%
\pgfpathrectangle{\pgfqpoint{0.647939in}{0.492442in}}{\pgfqpoint{3.079299in}{3.079299in}}%
\pgfusepath{clip}%
\pgfsetroundcap%
\pgfsetroundjoin%
\pgfsetlinewidth{0.301125pt}%
\definecolor{currentstroke}{rgb}{0.500000,0.500000,0.500000}%
\pgfsetstrokecolor{currentstroke}%
\pgfsetstrokeopacity{0.300000}%
\pgfsetdash{}{0pt}%
\pgfpathmoveto{\pgfqpoint{2.381925in}{0.834301in}}%
\pgfusepath{stroke}%
\end{pgfscope}%
\begin{pgfscope}%
\pgfpathrectangle{\pgfqpoint{0.647939in}{0.492442in}}{\pgfqpoint{3.079299in}{3.079299in}}%
\pgfusepath{clip}%
\pgfsetroundcap%
\pgfsetroundjoin%
\definecolor{currentfill}{rgb}{0.500000,0.500000,0.500000}%
\pgfsetfillcolor{currentfill}%
\pgfsetfillopacity{0.300000}%
\pgfsetlinewidth{0.301125pt}%
\definecolor{currentstroke}{rgb}{0.500000,0.500000,0.500000}%
\pgfsetstrokecolor{currentstroke}%
\pgfsetstrokeopacity{0.300000}%
\pgfsetdash{}{0pt}%
\pgfpathmoveto{\pgfqpoint{0.000000in}{0.000000in}}%
\pgfpathlineto{\pgfqpoint{0.000000in}{0.000000in}}%
\pgfpathclose%
\pgfusepath{stroke,fill}%
\end{pgfscope}%
\begin{pgfscope}%
\pgfpathrectangle{\pgfqpoint{0.647939in}{0.492442in}}{\pgfqpoint{3.079299in}{3.079299in}}%
\pgfusepath{clip}%
\pgfsetroundcap%
\pgfsetroundjoin%
\pgfsetlinewidth{0.301125pt}%
\definecolor{currentstroke}{rgb}{0.500000,0.500000,0.500000}%
\pgfsetstrokecolor{currentstroke}%
\pgfsetstrokeopacity{0.300000}%
\pgfsetdash{}{0pt}%
\pgfpathmoveto{\pgfqpoint{2.604686in}{0.953257in}}%
\pgfusepath{stroke}%
\end{pgfscope}%
\begin{pgfscope}%
\pgfpathrectangle{\pgfqpoint{0.647939in}{0.492442in}}{\pgfqpoint{3.079299in}{3.079299in}}%
\pgfusepath{clip}%
\pgfsetroundcap%
\pgfsetroundjoin%
\definecolor{currentfill}{rgb}{0.500000,0.500000,0.500000}%
\pgfsetfillcolor{currentfill}%
\pgfsetfillopacity{0.300000}%
\pgfsetlinewidth{0.301125pt}%
\definecolor{currentstroke}{rgb}{0.500000,0.500000,0.500000}%
\pgfsetstrokecolor{currentstroke}%
\pgfsetstrokeopacity{0.300000}%
\pgfsetdash{}{0pt}%
\pgfpathmoveto{\pgfqpoint{0.000000in}{0.000000in}}%
\pgfpathlineto{\pgfqpoint{0.000000in}{0.000000in}}%
\pgfpathclose%
\pgfusepath{stroke,fill}%
\end{pgfscope}%
\begin{pgfscope}%
\pgfpathrectangle{\pgfqpoint{0.647939in}{0.492442in}}{\pgfqpoint{3.079299in}{3.079299in}}%
\pgfusepath{clip}%
\pgfsetroundcap%
\pgfsetroundjoin%
\pgfsetlinewidth{0.301125pt}%
\definecolor{currentstroke}{rgb}{0.500000,0.500000,0.500000}%
\pgfsetstrokecolor{currentstroke}%
\pgfsetstrokeopacity{0.300000}%
\pgfsetdash{}{0pt}%
\pgfpathmoveto{\pgfqpoint{2.744288in}{1.010699in}}%
\pgfusepath{stroke}%
\end{pgfscope}%
\begin{pgfscope}%
\pgfpathrectangle{\pgfqpoint{0.647939in}{0.492442in}}{\pgfqpoint{3.079299in}{3.079299in}}%
\pgfusepath{clip}%
\pgfsetroundcap%
\pgfsetroundjoin%
\definecolor{currentfill}{rgb}{0.500000,0.500000,0.500000}%
\pgfsetfillcolor{currentfill}%
\pgfsetfillopacity{0.300000}%
\pgfsetlinewidth{0.301125pt}%
\definecolor{currentstroke}{rgb}{0.500000,0.500000,0.500000}%
\pgfsetstrokecolor{currentstroke}%
\pgfsetstrokeopacity{0.300000}%
\pgfsetdash{}{0pt}%
\pgfpathmoveto{\pgfqpoint{0.000000in}{0.000000in}}%
\pgfpathlineto{\pgfqpoint{0.000000in}{0.000000in}}%
\pgfpathclose%
\pgfusepath{stroke,fill}%
\end{pgfscope}%
\begin{pgfscope}%
\pgfpathrectangle{\pgfqpoint{0.647939in}{0.492442in}}{\pgfqpoint{3.079299in}{3.079299in}}%
\pgfusepath{clip}%
\pgfsetroundcap%
\pgfsetroundjoin%
\pgfsetlinewidth{0.301125pt}%
\definecolor{currentstroke}{rgb}{0.500000,0.500000,0.500000}%
\pgfsetstrokecolor{currentstroke}%
\pgfsetstrokeopacity{0.300000}%
\pgfsetdash{}{0pt}%
\pgfpathmoveto{\pgfqpoint{2.617728in}{1.129079in}}%
\pgfusepath{stroke}%
\end{pgfscope}%
\begin{pgfscope}%
\pgfpathrectangle{\pgfqpoint{0.647939in}{0.492442in}}{\pgfqpoint{3.079299in}{3.079299in}}%
\pgfusepath{clip}%
\pgfsetroundcap%
\pgfsetroundjoin%
\definecolor{currentfill}{rgb}{0.500000,0.500000,0.500000}%
\pgfsetfillcolor{currentfill}%
\pgfsetfillopacity{0.300000}%
\pgfsetlinewidth{0.301125pt}%
\definecolor{currentstroke}{rgb}{0.500000,0.500000,0.500000}%
\pgfsetstrokecolor{currentstroke}%
\pgfsetstrokeopacity{0.300000}%
\pgfsetdash{}{0pt}%
\pgfpathmoveto{\pgfqpoint{0.000000in}{0.000000in}}%
\pgfpathlineto{\pgfqpoint{0.000000in}{0.000000in}}%
\pgfpathclose%
\pgfusepath{stroke,fill}%
\end{pgfscope}%
\begin{pgfscope}%
\pgfpathrectangle{\pgfqpoint{0.647939in}{0.492442in}}{\pgfqpoint{3.079299in}{3.079299in}}%
\pgfusepath{clip}%
\pgfsetroundcap%
\pgfsetroundjoin%
\pgfsetlinewidth{0.301125pt}%
\definecolor{currentstroke}{rgb}{0.500000,0.500000,0.500000}%
\pgfsetstrokecolor{currentstroke}%
\pgfsetstrokeopacity{0.300000}%
\pgfsetdash{}{0pt}%
\pgfpathmoveto{\pgfqpoint{2.823065in}{1.163542in}}%
\pgfusepath{stroke}%
\end{pgfscope}%
\begin{pgfscope}%
\pgfpathrectangle{\pgfqpoint{0.647939in}{0.492442in}}{\pgfqpoint{3.079299in}{3.079299in}}%
\pgfusepath{clip}%
\pgfsetroundcap%
\pgfsetroundjoin%
\definecolor{currentfill}{rgb}{0.500000,0.500000,0.500000}%
\pgfsetfillcolor{currentfill}%
\pgfsetfillopacity{0.300000}%
\pgfsetlinewidth{0.301125pt}%
\definecolor{currentstroke}{rgb}{0.500000,0.500000,0.500000}%
\pgfsetstrokecolor{currentstroke}%
\pgfsetstrokeopacity{0.300000}%
\pgfsetdash{}{0pt}%
\pgfpathmoveto{\pgfqpoint{0.000000in}{0.000000in}}%
\pgfpathlineto{\pgfqpoint{0.000000in}{0.000000in}}%
\pgfpathclose%
\pgfusepath{stroke,fill}%
\end{pgfscope}%
\begin{pgfscope}%
\pgfpathrectangle{\pgfqpoint{0.647939in}{0.492442in}}{\pgfqpoint{3.079299in}{3.079299in}}%
\pgfusepath{clip}%
\pgfsetroundcap%
\pgfsetroundjoin%
\pgfsetlinewidth{0.301125pt}%
\definecolor{currentstroke}{rgb}{0.500000,0.500000,0.500000}%
\pgfsetstrokecolor{currentstroke}%
\pgfsetstrokeopacity{0.300000}%
\pgfsetdash{}{0pt}%
\pgfpathmoveto{\pgfqpoint{2.766568in}{1.274340in}}%
\pgfusepath{stroke}%
\end{pgfscope}%
\begin{pgfscope}%
\pgfpathrectangle{\pgfqpoint{0.647939in}{0.492442in}}{\pgfqpoint{3.079299in}{3.079299in}}%
\pgfusepath{clip}%
\pgfsetroundcap%
\pgfsetroundjoin%
\definecolor{currentfill}{rgb}{0.500000,0.500000,0.500000}%
\pgfsetfillcolor{currentfill}%
\pgfsetfillopacity{0.300000}%
\pgfsetlinewidth{0.301125pt}%
\definecolor{currentstroke}{rgb}{0.500000,0.500000,0.500000}%
\pgfsetstrokecolor{currentstroke}%
\pgfsetstrokeopacity{0.300000}%
\pgfsetdash{}{0pt}%
\pgfpathmoveto{\pgfqpoint{0.000000in}{0.000000in}}%
\pgfpathlineto{\pgfqpoint{0.000000in}{0.000000in}}%
\pgfpathclose%
\pgfusepath{stroke,fill}%
\end{pgfscope}%
\begin{pgfscope}%
\pgfpathrectangle{\pgfqpoint{0.647939in}{0.492442in}}{\pgfqpoint{3.079299in}{3.079299in}}%
\pgfusepath{clip}%
\pgfsetroundcap%
\pgfsetroundjoin%
\pgfsetlinewidth{0.301125pt}%
\definecolor{currentstroke}{rgb}{0.500000,0.500000,0.500000}%
\pgfsetstrokecolor{currentstroke}%
\pgfsetstrokeopacity{0.300000}%
\pgfsetdash{}{0pt}%
\pgfpathmoveto{\pgfqpoint{2.776538in}{1.365887in}}%
\pgfusepath{stroke}%
\end{pgfscope}%
\begin{pgfscope}%
\pgfpathrectangle{\pgfqpoint{0.647939in}{0.492442in}}{\pgfqpoint{3.079299in}{3.079299in}}%
\pgfusepath{clip}%
\pgfsetroundcap%
\pgfsetroundjoin%
\definecolor{currentfill}{rgb}{0.500000,0.500000,0.500000}%
\pgfsetfillcolor{currentfill}%
\pgfsetfillopacity{0.300000}%
\pgfsetlinewidth{0.301125pt}%
\definecolor{currentstroke}{rgb}{0.500000,0.500000,0.500000}%
\pgfsetstrokecolor{currentstroke}%
\pgfsetstrokeopacity{0.300000}%
\pgfsetdash{}{0pt}%
\pgfpathmoveto{\pgfqpoint{0.000000in}{0.000000in}}%
\pgfpathlineto{\pgfqpoint{0.000000in}{0.000000in}}%
\pgfpathclose%
\pgfusepath{stroke,fill}%
\end{pgfscope}%
\begin{pgfscope}%
\pgfpathrectangle{\pgfqpoint{0.647939in}{0.492442in}}{\pgfqpoint{3.079299in}{3.079299in}}%
\pgfusepath{clip}%
\pgfsetroundcap%
\pgfsetroundjoin%
\pgfsetlinewidth{0.301125pt}%
\definecolor{currentstroke}{rgb}{0.500000,0.500000,0.500000}%
\pgfsetstrokecolor{currentstroke}%
\pgfsetstrokeopacity{0.300000}%
\pgfsetdash{}{0pt}%
\pgfpathmoveto{\pgfqpoint{2.850924in}{1.432167in}}%
\pgfusepath{stroke}%
\end{pgfscope}%
\begin{pgfscope}%
\pgfpathrectangle{\pgfqpoint{0.647939in}{0.492442in}}{\pgfqpoint{3.079299in}{3.079299in}}%
\pgfusepath{clip}%
\pgfsetroundcap%
\pgfsetroundjoin%
\definecolor{currentfill}{rgb}{0.500000,0.500000,0.500000}%
\pgfsetfillcolor{currentfill}%
\pgfsetfillopacity{0.300000}%
\pgfsetlinewidth{0.301125pt}%
\definecolor{currentstroke}{rgb}{0.500000,0.500000,0.500000}%
\pgfsetstrokecolor{currentstroke}%
\pgfsetstrokeopacity{0.300000}%
\pgfsetdash{}{0pt}%
\pgfpathmoveto{\pgfqpoint{0.000000in}{0.000000in}}%
\pgfpathlineto{\pgfqpoint{0.000000in}{0.000000in}}%
\pgfpathclose%
\pgfusepath{stroke,fill}%
\end{pgfscope}%
\begin{pgfscope}%
\pgfpathrectangle{\pgfqpoint{0.647939in}{0.492442in}}{\pgfqpoint{3.079299in}{3.079299in}}%
\pgfusepath{clip}%
\pgfsetroundcap%
\pgfsetroundjoin%
\pgfsetlinewidth{0.301125pt}%
\definecolor{currentstroke}{rgb}{0.500000,0.500000,0.500000}%
\pgfsetstrokecolor{currentstroke}%
\pgfsetstrokeopacity{0.300000}%
\pgfsetdash{}{0pt}%
\pgfpathmoveto{\pgfqpoint{2.863629in}{1.525780in}}%
\pgfusepath{stroke}%
\end{pgfscope}%
\begin{pgfscope}%
\pgfpathrectangle{\pgfqpoint{0.647939in}{0.492442in}}{\pgfqpoint{3.079299in}{3.079299in}}%
\pgfusepath{clip}%
\pgfsetroundcap%
\pgfsetroundjoin%
\definecolor{currentfill}{rgb}{0.500000,0.500000,0.500000}%
\pgfsetfillcolor{currentfill}%
\pgfsetfillopacity{0.300000}%
\pgfsetlinewidth{0.301125pt}%
\definecolor{currentstroke}{rgb}{0.500000,0.500000,0.500000}%
\pgfsetstrokecolor{currentstroke}%
\pgfsetstrokeopacity{0.300000}%
\pgfsetdash{}{0pt}%
\pgfpathmoveto{\pgfqpoint{0.000000in}{0.000000in}}%
\pgfpathlineto{\pgfqpoint{0.000000in}{0.000000in}}%
\pgfpathclose%
\pgfusepath{stroke,fill}%
\end{pgfscope}%
\begin{pgfscope}%
\pgfpathrectangle{\pgfqpoint{0.647939in}{0.492442in}}{\pgfqpoint{3.079299in}{3.079299in}}%
\pgfusepath{clip}%
\pgfsetroundcap%
\pgfsetroundjoin%
\pgfsetlinewidth{0.301125pt}%
\definecolor{currentstroke}{rgb}{0.500000,0.500000,0.500000}%
\pgfsetstrokecolor{currentstroke}%
\pgfsetstrokeopacity{0.300000}%
\pgfsetdash{}{0pt}%
\pgfpathmoveto{\pgfqpoint{2.937413in}{1.586574in}}%
\pgfusepath{stroke}%
\end{pgfscope}%
\begin{pgfscope}%
\pgfpathrectangle{\pgfqpoint{0.647939in}{0.492442in}}{\pgfqpoint{3.079299in}{3.079299in}}%
\pgfusepath{clip}%
\pgfsetroundcap%
\pgfsetroundjoin%
\definecolor{currentfill}{rgb}{0.500000,0.500000,0.500000}%
\pgfsetfillcolor{currentfill}%
\pgfsetfillopacity{0.300000}%
\pgfsetlinewidth{0.301125pt}%
\definecolor{currentstroke}{rgb}{0.500000,0.500000,0.500000}%
\pgfsetstrokecolor{currentstroke}%
\pgfsetstrokeopacity{0.300000}%
\pgfsetdash{}{0pt}%
\pgfpathmoveto{\pgfqpoint{0.000000in}{0.000000in}}%
\pgfpathlineto{\pgfqpoint{0.000000in}{0.000000in}}%
\pgfpathclose%
\pgfusepath{stroke,fill}%
\end{pgfscope}%
\begin{pgfscope}%
\pgfpathrectangle{\pgfqpoint{0.647939in}{0.492442in}}{\pgfqpoint{3.079299in}{3.079299in}}%
\pgfusepath{clip}%
\pgfsetroundcap%
\pgfsetroundjoin%
\pgfsetlinewidth{0.301125pt}%
\definecolor{currentstroke}{rgb}{0.500000,0.500000,0.500000}%
\pgfsetstrokecolor{currentstroke}%
\pgfsetstrokeopacity{0.300000}%
\pgfsetdash{}{0pt}%
\pgfpathmoveto{\pgfqpoint{2.972289in}{1.779219in}}%
\pgfusepath{stroke}%
\end{pgfscope}%
\begin{pgfscope}%
\pgfpathrectangle{\pgfqpoint{0.647939in}{0.492442in}}{\pgfqpoint{3.079299in}{3.079299in}}%
\pgfusepath{clip}%
\pgfsetroundcap%
\pgfsetroundjoin%
\definecolor{currentfill}{rgb}{0.500000,0.500000,0.500000}%
\pgfsetfillcolor{currentfill}%
\pgfsetfillopacity{0.300000}%
\pgfsetlinewidth{0.301125pt}%
\definecolor{currentstroke}{rgb}{0.500000,0.500000,0.500000}%
\pgfsetstrokecolor{currentstroke}%
\pgfsetstrokeopacity{0.300000}%
\pgfsetdash{}{0pt}%
\pgfpathmoveto{\pgfqpoint{0.000000in}{0.000000in}}%
\pgfpathlineto{\pgfqpoint{0.000000in}{0.000000in}}%
\pgfpathclose%
\pgfusepath{stroke,fill}%
\end{pgfscope}%
\begin{pgfscope}%
\pgfpathrectangle{\pgfqpoint{0.647939in}{0.492442in}}{\pgfqpoint{3.079299in}{3.079299in}}%
\pgfusepath{clip}%
\pgfsetroundcap%
\pgfsetroundjoin%
\pgfsetlinewidth{0.301125pt}%
\definecolor{currentstroke}{rgb}{0.500000,0.500000,0.500000}%
\pgfsetstrokecolor{currentstroke}%
\pgfsetstrokeopacity{0.300000}%
\pgfsetdash{}{0pt}%
\pgfpathmoveto{\pgfqpoint{2.979119in}{2.034210in}}%
\pgfusepath{stroke}%
\end{pgfscope}%
\begin{pgfscope}%
\pgfpathrectangle{\pgfqpoint{0.647939in}{0.492442in}}{\pgfqpoint{3.079299in}{3.079299in}}%
\pgfusepath{clip}%
\pgfsetroundcap%
\pgfsetroundjoin%
\definecolor{currentfill}{rgb}{0.500000,0.500000,0.500000}%
\pgfsetfillcolor{currentfill}%
\pgfsetfillopacity{0.300000}%
\pgfsetlinewidth{0.301125pt}%
\definecolor{currentstroke}{rgb}{0.500000,0.500000,0.500000}%
\pgfsetstrokecolor{currentstroke}%
\pgfsetstrokeopacity{0.300000}%
\pgfsetdash{}{0pt}%
\pgfpathmoveto{\pgfqpoint{0.000000in}{0.000000in}}%
\pgfpathlineto{\pgfqpoint{0.000000in}{0.000000in}}%
\pgfpathclose%
\pgfusepath{stroke,fill}%
\end{pgfscope}%
\begin{pgfscope}%
\pgfpathrectangle{\pgfqpoint{0.647939in}{0.492442in}}{\pgfqpoint{3.079299in}{3.079299in}}%
\pgfusepath{clip}%
\pgfsetroundcap%
\pgfsetroundjoin%
\pgfsetlinewidth{0.301125pt}%
\definecolor{currentstroke}{rgb}{0.500000,0.500000,0.500000}%
\pgfsetstrokecolor{currentstroke}%
\pgfsetstrokeopacity{0.300000}%
\pgfsetdash{}{0pt}%
\pgfpathmoveto{\pgfqpoint{3.477403in}{1.640226in}}%
\pgfusepath{stroke}%
\end{pgfscope}%
\begin{pgfscope}%
\pgfpathrectangle{\pgfqpoint{0.647939in}{0.492442in}}{\pgfqpoint{3.079299in}{3.079299in}}%
\pgfusepath{clip}%
\pgfsetroundcap%
\pgfsetroundjoin%
\definecolor{currentfill}{rgb}{0.500000,0.500000,0.500000}%
\pgfsetfillcolor{currentfill}%
\pgfsetfillopacity{0.300000}%
\pgfsetlinewidth{0.301125pt}%
\definecolor{currentstroke}{rgb}{0.500000,0.500000,0.500000}%
\pgfsetstrokecolor{currentstroke}%
\pgfsetstrokeopacity{0.300000}%
\pgfsetdash{}{0pt}%
\pgfpathmoveto{\pgfqpoint{0.000000in}{0.000000in}}%
\pgfpathlineto{\pgfqpoint{0.000000in}{0.000000in}}%
\pgfpathclose%
\pgfusepath{stroke,fill}%
\end{pgfscope}%
\begin{pgfscope}%
\pgfpathrectangle{\pgfqpoint{0.647939in}{0.492442in}}{\pgfqpoint{3.079299in}{3.079299in}}%
\pgfusepath{clip}%
\pgfsetroundcap%
\pgfsetroundjoin%
\pgfsetlinewidth{0.301125pt}%
\definecolor{currentstroke}{rgb}{0.500000,0.500000,0.500000}%
\pgfsetstrokecolor{currentstroke}%
\pgfsetstrokeopacity{0.300000}%
\pgfsetdash{}{0pt}%
\pgfpathmoveto{\pgfqpoint{3.095368in}{2.563733in}}%
\pgfusepath{stroke}%
\end{pgfscope}%
\begin{pgfscope}%
\pgfpathrectangle{\pgfqpoint{0.647939in}{0.492442in}}{\pgfqpoint{3.079299in}{3.079299in}}%
\pgfusepath{clip}%
\pgfsetroundcap%
\pgfsetroundjoin%
\definecolor{currentfill}{rgb}{0.500000,0.500000,0.500000}%
\pgfsetfillcolor{currentfill}%
\pgfsetfillopacity{0.300000}%
\pgfsetlinewidth{0.301125pt}%
\definecolor{currentstroke}{rgb}{0.500000,0.500000,0.500000}%
\pgfsetstrokecolor{currentstroke}%
\pgfsetstrokeopacity{0.300000}%
\pgfsetdash{}{0pt}%
\pgfpathmoveto{\pgfqpoint{0.000000in}{0.000000in}}%
\pgfpathlineto{\pgfqpoint{0.000000in}{0.000000in}}%
\pgfpathclose%
\pgfusepath{stroke,fill}%
\end{pgfscope}%
\begin{pgfscope}%
\pgfpathrectangle{\pgfqpoint{0.647939in}{0.492442in}}{\pgfqpoint{3.079299in}{3.079299in}}%
\pgfusepath{clip}%
\pgfsetroundcap%
\pgfsetroundjoin%
\pgfsetlinewidth{0.301125pt}%
\definecolor{currentstroke}{rgb}{0.500000,0.500000,0.500000}%
\pgfsetstrokecolor{currentstroke}%
\pgfsetstrokeopacity{0.300000}%
\pgfsetdash{}{0pt}%
\pgfpathmoveto{\pgfqpoint{3.241328in}{2.570192in}}%
\pgfusepath{stroke}%
\end{pgfscope}%
\begin{pgfscope}%
\pgfpathrectangle{\pgfqpoint{0.647939in}{0.492442in}}{\pgfqpoint{3.079299in}{3.079299in}}%
\pgfusepath{clip}%
\pgfsetroundcap%
\pgfsetroundjoin%
\definecolor{currentfill}{rgb}{0.500000,0.500000,0.500000}%
\pgfsetfillcolor{currentfill}%
\pgfsetfillopacity{0.300000}%
\pgfsetlinewidth{0.301125pt}%
\definecolor{currentstroke}{rgb}{0.500000,0.500000,0.500000}%
\pgfsetstrokecolor{currentstroke}%
\pgfsetstrokeopacity{0.300000}%
\pgfsetdash{}{0pt}%
\pgfpathmoveto{\pgfqpoint{0.000000in}{0.000000in}}%
\pgfpathlineto{\pgfqpoint{0.000000in}{0.000000in}}%
\pgfpathclose%
\pgfusepath{stroke,fill}%
\end{pgfscope}%
\begin{pgfscope}%
\pgfpathrectangle{\pgfqpoint{0.647939in}{0.492442in}}{\pgfqpoint{3.079299in}{3.079299in}}%
\pgfusepath{clip}%
\pgfsetroundcap%
\pgfsetroundjoin%
\pgfsetlinewidth{0.301125pt}%
\definecolor{currentstroke}{rgb}{0.500000,0.500000,0.500000}%
\pgfsetstrokecolor{currentstroke}%
\pgfsetstrokeopacity{0.300000}%
\pgfsetdash{}{0pt}%
\pgfpathmoveto{\pgfqpoint{3.404567in}{2.207525in}}%
\pgfusepath{stroke}%
\end{pgfscope}%
\begin{pgfscope}%
\pgfpathrectangle{\pgfqpoint{0.647939in}{0.492442in}}{\pgfqpoint{3.079299in}{3.079299in}}%
\pgfusepath{clip}%
\pgfsetroundcap%
\pgfsetroundjoin%
\definecolor{currentfill}{rgb}{0.500000,0.500000,0.500000}%
\pgfsetfillcolor{currentfill}%
\pgfsetfillopacity{0.300000}%
\pgfsetlinewidth{0.301125pt}%
\definecolor{currentstroke}{rgb}{0.500000,0.500000,0.500000}%
\pgfsetstrokecolor{currentstroke}%
\pgfsetstrokeopacity{0.300000}%
\pgfsetdash{}{0pt}%
\pgfpathmoveto{\pgfqpoint{0.000000in}{0.000000in}}%
\pgfpathlineto{\pgfqpoint{0.000000in}{0.000000in}}%
\pgfpathclose%
\pgfusepath{stroke,fill}%
\end{pgfscope}%
\begin{pgfscope}%
\pgfpathrectangle{\pgfqpoint{0.647939in}{0.492442in}}{\pgfqpoint{3.079299in}{3.079299in}}%
\pgfusepath{clip}%
\pgfsetroundcap%
\pgfsetroundjoin%
\pgfsetlinewidth{0.301125pt}%
\definecolor{currentstroke}{rgb}{0.500000,0.500000,0.500000}%
\pgfsetstrokecolor{currentstroke}%
\pgfsetstrokeopacity{0.300000}%
\pgfsetdash{}{0pt}%
\pgfpathmoveto{\pgfqpoint{3.422160in}{2.653746in}}%
\pgfusepath{stroke}%
\end{pgfscope}%
\begin{pgfscope}%
\pgfpathrectangle{\pgfqpoint{0.647939in}{0.492442in}}{\pgfqpoint{3.079299in}{3.079299in}}%
\pgfusepath{clip}%
\pgfsetroundcap%
\pgfsetroundjoin%
\definecolor{currentfill}{rgb}{0.500000,0.500000,0.500000}%
\pgfsetfillcolor{currentfill}%
\pgfsetfillopacity{0.300000}%
\pgfsetlinewidth{0.301125pt}%
\definecolor{currentstroke}{rgb}{0.500000,0.500000,0.500000}%
\pgfsetstrokecolor{currentstroke}%
\pgfsetstrokeopacity{0.300000}%
\pgfsetdash{}{0pt}%
\pgfpathmoveto{\pgfqpoint{0.000000in}{0.000000in}}%
\pgfpathlineto{\pgfqpoint{0.000000in}{0.000000in}}%
\pgfpathclose%
\pgfusepath{stroke,fill}%
\end{pgfscope}%
\begin{pgfscope}%
\pgfpathrectangle{\pgfqpoint{0.647939in}{0.492442in}}{\pgfqpoint{3.079299in}{3.079299in}}%
\pgfusepath{clip}%
\pgfsetroundcap%
\pgfsetroundjoin%
\pgfsetlinewidth{0.301125pt}%
\definecolor{currentstroke}{rgb}{0.500000,0.500000,0.500000}%
\pgfsetstrokecolor{currentstroke}%
\pgfsetstrokeopacity{0.300000}%
\pgfsetdash{}{0pt}%
\pgfpathmoveto{\pgfqpoint{3.521024in}{2.626085in}}%
\pgfusepath{stroke}%
\end{pgfscope}%
\begin{pgfscope}%
\pgfpathrectangle{\pgfqpoint{0.647939in}{0.492442in}}{\pgfqpoint{3.079299in}{3.079299in}}%
\pgfusepath{clip}%
\pgfsetroundcap%
\pgfsetroundjoin%
\definecolor{currentfill}{rgb}{0.500000,0.500000,0.500000}%
\pgfsetfillcolor{currentfill}%
\pgfsetfillopacity{0.300000}%
\pgfsetlinewidth{0.301125pt}%
\definecolor{currentstroke}{rgb}{0.500000,0.500000,0.500000}%
\pgfsetstrokecolor{currentstroke}%
\pgfsetstrokeopacity{0.300000}%
\pgfsetdash{}{0pt}%
\pgfpathmoveto{\pgfqpoint{0.000000in}{0.000000in}}%
\pgfpathlineto{\pgfqpoint{0.000000in}{0.000000in}}%
\pgfpathclose%
\pgfusepath{stroke,fill}%
\end{pgfscope}%
\begin{pgfscope}%
\pgfpathrectangle{\pgfqpoint{0.647939in}{0.492442in}}{\pgfqpoint{3.079299in}{3.079299in}}%
\pgfusepath{clip}%
\pgfsetroundcap%
\pgfsetroundjoin%
\pgfsetlinewidth{0.301125pt}%
\definecolor{currentstroke}{rgb}{0.500000,0.500000,0.500000}%
\pgfsetstrokecolor{currentstroke}%
\pgfsetstrokeopacity{0.300000}%
\pgfsetdash{}{0pt}%
\pgfpathmoveto{\pgfqpoint{3.604082in}{2.655798in}}%
\pgfusepath{stroke}%
\end{pgfscope}%
\begin{pgfscope}%
\pgfpathrectangle{\pgfqpoint{0.647939in}{0.492442in}}{\pgfqpoint{3.079299in}{3.079299in}}%
\pgfusepath{clip}%
\pgfsetroundcap%
\pgfsetroundjoin%
\definecolor{currentfill}{rgb}{0.500000,0.500000,0.500000}%
\pgfsetfillcolor{currentfill}%
\pgfsetfillopacity{0.300000}%
\pgfsetlinewidth{0.301125pt}%
\definecolor{currentstroke}{rgb}{0.500000,0.500000,0.500000}%
\pgfsetstrokecolor{currentstroke}%
\pgfsetstrokeopacity{0.300000}%
\pgfsetdash{}{0pt}%
\pgfpathmoveto{\pgfqpoint{0.000000in}{0.000000in}}%
\pgfpathlineto{\pgfqpoint{0.000000in}{0.000000in}}%
\pgfpathclose%
\pgfusepath{stroke,fill}%
\end{pgfscope}%
\begin{pgfscope}%
\pgfpathrectangle{\pgfqpoint{0.647939in}{0.492442in}}{\pgfqpoint{3.079299in}{3.079299in}}%
\pgfusepath{clip}%
\pgfsetroundcap%
\pgfsetroundjoin%
\pgfsetlinewidth{0.301125pt}%
\definecolor{currentstroke}{rgb}{0.500000,0.500000,0.500000}%
\pgfsetstrokecolor{currentstroke}%
\pgfsetstrokeopacity{0.300000}%
\pgfsetdash{}{0pt}%
\pgfpathmoveto{\pgfqpoint{3.680054in}{2.539149in}}%
\pgfusepath{stroke}%
\end{pgfscope}%
\begin{pgfscope}%
\pgfpathrectangle{\pgfqpoint{0.647939in}{0.492442in}}{\pgfqpoint{3.079299in}{3.079299in}}%
\pgfusepath{clip}%
\pgfsetroundcap%
\pgfsetroundjoin%
\definecolor{currentfill}{rgb}{0.500000,0.500000,0.500000}%
\pgfsetfillcolor{currentfill}%
\pgfsetfillopacity{0.300000}%
\pgfsetlinewidth{0.301125pt}%
\definecolor{currentstroke}{rgb}{0.500000,0.500000,0.500000}%
\pgfsetstrokecolor{currentstroke}%
\pgfsetstrokeopacity{0.300000}%
\pgfsetdash{}{0pt}%
\pgfpathmoveto{\pgfqpoint{0.000000in}{0.000000in}}%
\pgfpathlineto{\pgfqpoint{0.000000in}{0.000000in}}%
\pgfpathclose%
\pgfusepath{stroke,fill}%
\end{pgfscope}%
\begin{pgfscope}%
\pgfpathrectangle{\pgfqpoint{0.647939in}{0.492442in}}{\pgfqpoint{3.079299in}{3.079299in}}%
\pgfusepath{clip}%
\pgfsetroundcap%
\pgfsetroundjoin%
\pgfsetlinewidth{0.301125pt}%
\definecolor{currentstroke}{rgb}{0.500000,0.500000,0.500000}%
\pgfsetstrokecolor{currentstroke}%
\pgfsetstrokeopacity{0.300000}%
\pgfsetdash{}{0pt}%
\pgfpathmoveto{\pgfqpoint{3.698152in}{2.697579in}}%
\pgfusepath{stroke}%
\end{pgfscope}%
\begin{pgfscope}%
\pgfpathrectangle{\pgfqpoint{0.647939in}{0.492442in}}{\pgfqpoint{3.079299in}{3.079299in}}%
\pgfusepath{clip}%
\pgfsetroundcap%
\pgfsetroundjoin%
\definecolor{currentfill}{rgb}{0.500000,0.500000,0.500000}%
\pgfsetfillcolor{currentfill}%
\pgfsetfillopacity{0.300000}%
\pgfsetlinewidth{0.301125pt}%
\definecolor{currentstroke}{rgb}{0.500000,0.500000,0.500000}%
\pgfsetstrokecolor{currentstroke}%
\pgfsetstrokeopacity{0.300000}%
\pgfsetdash{}{0pt}%
\pgfpathmoveto{\pgfqpoint{0.000000in}{0.000000in}}%
\pgfpathlineto{\pgfqpoint{0.000000in}{0.000000in}}%
\pgfpathclose%
\pgfusepath{stroke,fill}%
\end{pgfscope}%
\begin{pgfscope}%
\pgfpathrectangle{\pgfqpoint{0.647939in}{0.492442in}}{\pgfqpoint{3.079299in}{3.079299in}}%
\pgfusepath{clip}%
\pgfsetroundcap%
\pgfsetroundjoin%
\pgfsetlinewidth{0.301125pt}%
\definecolor{currentstroke}{rgb}{0.500000,0.500000,0.500000}%
\pgfsetstrokecolor{currentstroke}%
\pgfsetstrokeopacity{0.300000}%
\pgfsetdash{}{0pt}%
\pgfpathmoveto{\pgfqpoint{2.171698in}{2.794964in}}%
\pgfusepath{stroke}%
\end{pgfscope}%
\begin{pgfscope}%
\pgfpathrectangle{\pgfqpoint{0.647939in}{0.492442in}}{\pgfqpoint{3.079299in}{3.079299in}}%
\pgfusepath{clip}%
\pgfsetroundcap%
\pgfsetroundjoin%
\definecolor{currentfill}{rgb}{0.500000,0.500000,0.500000}%
\pgfsetfillcolor{currentfill}%
\pgfsetfillopacity{0.300000}%
\pgfsetlinewidth{0.301125pt}%
\definecolor{currentstroke}{rgb}{0.500000,0.500000,0.500000}%
\pgfsetstrokecolor{currentstroke}%
\pgfsetstrokeopacity{0.300000}%
\pgfsetdash{}{0pt}%
\pgfpathmoveto{\pgfqpoint{0.000000in}{0.000000in}}%
\pgfpathlineto{\pgfqpoint{0.000000in}{0.000000in}}%
\pgfpathclose%
\pgfusepath{stroke,fill}%
\end{pgfscope}%
\begin{pgfscope}%
\pgfpathrectangle{\pgfqpoint{0.647939in}{0.492442in}}{\pgfqpoint{3.079299in}{3.079299in}}%
\pgfusepath{clip}%
\pgfsetroundcap%
\pgfsetroundjoin%
\pgfsetlinewidth{0.301125pt}%
\definecolor{currentstroke}{rgb}{0.500000,0.500000,0.500000}%
\pgfsetstrokecolor{currentstroke}%
\pgfsetstrokeopacity{0.300000}%
\pgfsetdash{}{0pt}%
\pgfpathmoveto{\pgfqpoint{2.044387in}{3.061136in}}%
\pgfusepath{stroke}%
\end{pgfscope}%
\begin{pgfscope}%
\pgfpathrectangle{\pgfqpoint{0.647939in}{0.492442in}}{\pgfqpoint{3.079299in}{3.079299in}}%
\pgfusepath{clip}%
\pgfsetroundcap%
\pgfsetroundjoin%
\definecolor{currentfill}{rgb}{0.500000,0.500000,0.500000}%
\pgfsetfillcolor{currentfill}%
\pgfsetfillopacity{0.300000}%
\pgfsetlinewidth{0.301125pt}%
\definecolor{currentstroke}{rgb}{0.500000,0.500000,0.500000}%
\pgfsetstrokecolor{currentstroke}%
\pgfsetstrokeopacity{0.300000}%
\pgfsetdash{}{0pt}%
\pgfpathmoveto{\pgfqpoint{0.000000in}{0.000000in}}%
\pgfpathlineto{\pgfqpoint{0.000000in}{0.000000in}}%
\pgfpathclose%
\pgfusepath{stroke,fill}%
\end{pgfscope}%
\begin{pgfscope}%
\pgfpathrectangle{\pgfqpoint{0.647939in}{0.492442in}}{\pgfqpoint{3.079299in}{3.079299in}}%
\pgfusepath{clip}%
\pgfsetroundcap%
\pgfsetroundjoin%
\pgfsetlinewidth{0.301125pt}%
\definecolor{currentstroke}{rgb}{0.500000,0.500000,0.500000}%
\pgfsetstrokecolor{currentstroke}%
\pgfsetstrokeopacity{0.300000}%
\pgfsetdash{}{0pt}%
\pgfpathmoveto{\pgfqpoint{1.906477in}{3.225316in}}%
\pgfusepath{stroke}%
\end{pgfscope}%
\begin{pgfscope}%
\pgfpathrectangle{\pgfqpoint{0.647939in}{0.492442in}}{\pgfqpoint{3.079299in}{3.079299in}}%
\pgfusepath{clip}%
\pgfsetroundcap%
\pgfsetroundjoin%
\definecolor{currentfill}{rgb}{0.500000,0.500000,0.500000}%
\pgfsetfillcolor{currentfill}%
\pgfsetfillopacity{0.300000}%
\pgfsetlinewidth{0.301125pt}%
\definecolor{currentstroke}{rgb}{0.500000,0.500000,0.500000}%
\pgfsetstrokecolor{currentstroke}%
\pgfsetstrokeopacity{0.300000}%
\pgfsetdash{}{0pt}%
\pgfpathmoveto{\pgfqpoint{0.000000in}{0.000000in}}%
\pgfpathlineto{\pgfqpoint{0.000000in}{0.000000in}}%
\pgfpathclose%
\pgfusepath{stroke,fill}%
\end{pgfscope}%
\begin{pgfscope}%
\pgfpathrectangle{\pgfqpoint{0.647939in}{0.492442in}}{\pgfqpoint{3.079299in}{3.079299in}}%
\pgfusepath{clip}%
\pgfsetroundcap%
\pgfsetroundjoin%
\pgfsetlinewidth{0.301125pt}%
\definecolor{currentstroke}{rgb}{0.500000,0.500000,0.500000}%
\pgfsetstrokecolor{currentstroke}%
\pgfsetstrokeopacity{0.300000}%
\pgfsetdash{}{0pt}%
\pgfpathmoveto{\pgfqpoint{1.863893in}{3.344599in}}%
\pgfusepath{stroke}%
\end{pgfscope}%
\begin{pgfscope}%
\pgfpathrectangle{\pgfqpoint{0.647939in}{0.492442in}}{\pgfqpoint{3.079299in}{3.079299in}}%
\pgfusepath{clip}%
\pgfsetroundcap%
\pgfsetroundjoin%
\definecolor{currentfill}{rgb}{0.500000,0.500000,0.500000}%
\pgfsetfillcolor{currentfill}%
\pgfsetfillopacity{0.300000}%
\pgfsetlinewidth{0.301125pt}%
\definecolor{currentstroke}{rgb}{0.500000,0.500000,0.500000}%
\pgfsetstrokecolor{currentstroke}%
\pgfsetstrokeopacity{0.300000}%
\pgfsetdash{}{0pt}%
\pgfpathmoveto{\pgfqpoint{0.000000in}{0.000000in}}%
\pgfpathlineto{\pgfqpoint{0.000000in}{0.000000in}}%
\pgfpathclose%
\pgfusepath{stroke,fill}%
\end{pgfscope}%
\begin{pgfscope}%
\pgfpathrectangle{\pgfqpoint{0.647939in}{0.492442in}}{\pgfqpoint{3.079299in}{3.079299in}}%
\pgfusepath{clip}%
\pgfsetroundcap%
\pgfsetroundjoin%
\pgfsetlinewidth{0.301125pt}%
\definecolor{currentstroke}{rgb}{0.500000,0.500000,0.500000}%
\pgfsetstrokecolor{currentstroke}%
\pgfsetstrokeopacity{0.300000}%
\pgfsetdash{}{0pt}%
\pgfpathmoveto{\pgfqpoint{1.770367in}{3.422293in}}%
\pgfusepath{stroke}%
\end{pgfscope}%
\begin{pgfscope}%
\pgfpathrectangle{\pgfqpoint{0.647939in}{0.492442in}}{\pgfqpoint{3.079299in}{3.079299in}}%
\pgfusepath{clip}%
\pgfsetroundcap%
\pgfsetroundjoin%
\definecolor{currentfill}{rgb}{0.500000,0.500000,0.500000}%
\pgfsetfillcolor{currentfill}%
\pgfsetfillopacity{0.300000}%
\pgfsetlinewidth{0.301125pt}%
\definecolor{currentstroke}{rgb}{0.500000,0.500000,0.500000}%
\pgfsetstrokecolor{currentstroke}%
\pgfsetstrokeopacity{0.300000}%
\pgfsetdash{}{0pt}%
\pgfpathmoveto{\pgfqpoint{0.000000in}{0.000000in}}%
\pgfpathlineto{\pgfqpoint{0.000000in}{0.000000in}}%
\pgfpathclose%
\pgfusepath{stroke,fill}%
\end{pgfscope}%
\begin{pgfscope}%
\pgfpathrectangle{\pgfqpoint{0.647939in}{0.492442in}}{\pgfqpoint{3.079299in}{3.079299in}}%
\pgfusepath{clip}%
\pgfsetroundcap%
\pgfsetroundjoin%
\pgfsetlinewidth{0.301125pt}%
\definecolor{currentstroke}{rgb}{0.500000,0.500000,0.500000}%
\pgfsetstrokecolor{currentstroke}%
\pgfsetstrokeopacity{0.300000}%
\pgfsetdash{}{0pt}%
\pgfpathmoveto{\pgfqpoint{1.683805in}{3.491287in}}%
\pgfusepath{stroke}%
\end{pgfscope}%
\begin{pgfscope}%
\pgfpathrectangle{\pgfqpoint{0.647939in}{0.492442in}}{\pgfqpoint{3.079299in}{3.079299in}}%
\pgfusepath{clip}%
\pgfsetroundcap%
\pgfsetroundjoin%
\definecolor{currentfill}{rgb}{0.500000,0.500000,0.500000}%
\pgfsetfillcolor{currentfill}%
\pgfsetfillopacity{0.300000}%
\pgfsetlinewidth{0.301125pt}%
\definecolor{currentstroke}{rgb}{0.500000,0.500000,0.500000}%
\pgfsetstrokecolor{currentstroke}%
\pgfsetstrokeopacity{0.300000}%
\pgfsetdash{}{0pt}%
\pgfpathmoveto{\pgfqpoint{0.000000in}{0.000000in}}%
\pgfpathlineto{\pgfqpoint{0.000000in}{0.000000in}}%
\pgfpathclose%
\pgfusepath{stroke,fill}%
\end{pgfscope}%
\begin{pgfscope}%
\pgfpathrectangle{\pgfqpoint{0.647939in}{0.492442in}}{\pgfqpoint{3.079299in}{3.079299in}}%
\pgfusepath{clip}%
\pgfsetroundcap%
\pgfsetroundjoin%
\pgfsetlinewidth{0.301125pt}%
\definecolor{currentstroke}{rgb}{0.500000,0.500000,0.500000}%
\pgfsetstrokecolor{currentstroke}%
\pgfsetstrokeopacity{0.300000}%
\pgfsetdash{}{0pt}%
\pgfpathmoveto{\pgfqpoint{2.083311in}{3.562990in}}%
\pgfusepath{stroke}%
\end{pgfscope}%
\begin{pgfscope}%
\pgfpathrectangle{\pgfqpoint{0.647939in}{0.492442in}}{\pgfqpoint{3.079299in}{3.079299in}}%
\pgfusepath{clip}%
\pgfsetroundcap%
\pgfsetroundjoin%
\definecolor{currentfill}{rgb}{0.500000,0.500000,0.500000}%
\pgfsetfillcolor{currentfill}%
\pgfsetfillopacity{0.300000}%
\pgfsetlinewidth{0.301125pt}%
\definecolor{currentstroke}{rgb}{0.500000,0.500000,0.500000}%
\pgfsetstrokecolor{currentstroke}%
\pgfsetstrokeopacity{0.300000}%
\pgfsetdash{}{0pt}%
\pgfpathmoveto{\pgfqpoint{0.000000in}{0.000000in}}%
\pgfpathlineto{\pgfqpoint{0.000000in}{0.000000in}}%
\pgfpathclose%
\pgfusepath{stroke,fill}%
\end{pgfscope}%
\begin{pgfscope}%
\pgfpathrectangle{\pgfqpoint{0.647939in}{0.492442in}}{\pgfqpoint{3.079299in}{3.079299in}}%
\pgfusepath{clip}%
\pgfsetroundcap%
\pgfsetroundjoin%
\pgfsetlinewidth{0.301125pt}%
\definecolor{currentstroke}{rgb}{0.500000,0.500000,0.500000}%
\pgfsetstrokecolor{currentstroke}%
\pgfsetstrokeopacity{0.300000}%
\pgfsetdash{}{0pt}%
\pgfpathmoveto{\pgfqpoint{1.117819in}{3.471839in}}%
\pgfusepath{stroke}%
\end{pgfscope}%
\begin{pgfscope}%
\pgfpathrectangle{\pgfqpoint{0.647939in}{0.492442in}}{\pgfqpoint{3.079299in}{3.079299in}}%
\pgfusepath{clip}%
\pgfsetroundcap%
\pgfsetroundjoin%
\definecolor{currentfill}{rgb}{0.500000,0.500000,0.500000}%
\pgfsetfillcolor{currentfill}%
\pgfsetfillopacity{0.300000}%
\pgfsetlinewidth{0.301125pt}%
\definecolor{currentstroke}{rgb}{0.500000,0.500000,0.500000}%
\pgfsetstrokecolor{currentstroke}%
\pgfsetstrokeopacity{0.300000}%
\pgfsetdash{}{0pt}%
\pgfpathmoveto{\pgfqpoint{0.000000in}{0.000000in}}%
\pgfpathlineto{\pgfqpoint{0.000000in}{0.000000in}}%
\pgfpathclose%
\pgfusepath{stroke,fill}%
\end{pgfscope}%
\begin{pgfscope}%
\pgfpathrectangle{\pgfqpoint{0.647939in}{0.492442in}}{\pgfqpoint{3.079299in}{3.079299in}}%
\pgfusepath{clip}%
\pgfsetroundcap%
\pgfsetroundjoin%
\pgfsetlinewidth{0.301125pt}%
\definecolor{currentstroke}{rgb}{0.500000,0.500000,0.500000}%
\pgfsetstrokecolor{currentstroke}%
\pgfsetstrokeopacity{0.300000}%
\pgfsetdash{}{0pt}%
\pgfpathmoveto{\pgfqpoint{0.969372in}{3.510924in}}%
\pgfusepath{stroke}%
\end{pgfscope}%
\begin{pgfscope}%
\pgfpathrectangle{\pgfqpoint{0.647939in}{0.492442in}}{\pgfqpoint{3.079299in}{3.079299in}}%
\pgfusepath{clip}%
\pgfsetroundcap%
\pgfsetroundjoin%
\definecolor{currentfill}{rgb}{0.500000,0.500000,0.500000}%
\pgfsetfillcolor{currentfill}%
\pgfsetfillopacity{0.300000}%
\pgfsetlinewidth{0.301125pt}%
\definecolor{currentstroke}{rgb}{0.500000,0.500000,0.500000}%
\pgfsetstrokecolor{currentstroke}%
\pgfsetstrokeopacity{0.300000}%
\pgfsetdash{}{0pt}%
\pgfpathmoveto{\pgfqpoint{0.000000in}{0.000000in}}%
\pgfpathlineto{\pgfqpoint{0.000000in}{0.000000in}}%
\pgfpathclose%
\pgfusepath{stroke,fill}%
\end{pgfscope}%
\begin{pgfscope}%
\pgfpathrectangle{\pgfqpoint{0.647939in}{0.492442in}}{\pgfqpoint{3.079299in}{3.079299in}}%
\pgfusepath{clip}%
\pgfsetroundcap%
\pgfsetroundjoin%
\pgfsetlinewidth{0.301125pt}%
\definecolor{currentstroke}{rgb}{0.500000,0.500000,0.500000}%
\pgfsetstrokecolor{currentstroke}%
\pgfsetstrokeopacity{0.300000}%
\pgfsetdash{}{0pt}%
\pgfpathmoveto{\pgfqpoint{0.878434in}{2.905968in}}%
\pgfusepath{stroke}%
\end{pgfscope}%
\begin{pgfscope}%
\pgfpathrectangle{\pgfqpoint{0.647939in}{0.492442in}}{\pgfqpoint{3.079299in}{3.079299in}}%
\pgfusepath{clip}%
\pgfsetroundcap%
\pgfsetroundjoin%
\definecolor{currentfill}{rgb}{0.500000,0.500000,0.500000}%
\pgfsetfillcolor{currentfill}%
\pgfsetfillopacity{0.300000}%
\pgfsetlinewidth{0.301125pt}%
\definecolor{currentstroke}{rgb}{0.500000,0.500000,0.500000}%
\pgfsetstrokecolor{currentstroke}%
\pgfsetstrokeopacity{0.300000}%
\pgfsetdash{}{0pt}%
\pgfpathmoveto{\pgfqpoint{0.000000in}{0.000000in}}%
\pgfpathlineto{\pgfqpoint{0.000000in}{0.000000in}}%
\pgfpathclose%
\pgfusepath{stroke,fill}%
\end{pgfscope}%
\begin{pgfscope}%
\pgfpathrectangle{\pgfqpoint{0.647939in}{0.492442in}}{\pgfqpoint{3.079299in}{3.079299in}}%
\pgfusepath{clip}%
\pgfsetroundcap%
\pgfsetroundjoin%
\pgfsetlinewidth{0.301125pt}%
\definecolor{currentstroke}{rgb}{0.500000,0.500000,0.500000}%
\pgfsetstrokecolor{currentstroke}%
\pgfsetstrokeopacity{0.300000}%
\pgfsetdash{}{0pt}%
\pgfpathmoveto{\pgfqpoint{1.211816in}{2.842856in}}%
\pgfusepath{stroke}%
\end{pgfscope}%
\begin{pgfscope}%
\pgfpathrectangle{\pgfqpoint{0.647939in}{0.492442in}}{\pgfqpoint{3.079299in}{3.079299in}}%
\pgfusepath{clip}%
\pgfsetroundcap%
\pgfsetroundjoin%
\definecolor{currentfill}{rgb}{0.500000,0.500000,0.500000}%
\pgfsetfillcolor{currentfill}%
\pgfsetfillopacity{0.300000}%
\pgfsetlinewidth{0.301125pt}%
\definecolor{currentstroke}{rgb}{0.500000,0.500000,0.500000}%
\pgfsetstrokecolor{currentstroke}%
\pgfsetstrokeopacity{0.300000}%
\pgfsetdash{}{0pt}%
\pgfpathmoveto{\pgfqpoint{0.000000in}{0.000000in}}%
\pgfpathlineto{\pgfqpoint{0.000000in}{0.000000in}}%
\pgfpathclose%
\pgfusepath{stroke,fill}%
\end{pgfscope}%
\begin{pgfscope}%
\pgfpathrectangle{\pgfqpoint{0.647939in}{0.492442in}}{\pgfqpoint{3.079299in}{3.079299in}}%
\pgfusepath{clip}%
\pgfsetroundcap%
\pgfsetroundjoin%
\pgfsetlinewidth{0.301125pt}%
\definecolor{currentstroke}{rgb}{0.500000,0.500000,0.500000}%
\pgfsetstrokecolor{currentstroke}%
\pgfsetstrokeopacity{0.300000}%
\pgfsetdash{}{0pt}%
\pgfpathmoveto{\pgfqpoint{0.878203in}{2.697489in}}%
\pgfusepath{stroke}%
\end{pgfscope}%
\begin{pgfscope}%
\pgfpathrectangle{\pgfqpoint{0.647939in}{0.492442in}}{\pgfqpoint{3.079299in}{3.079299in}}%
\pgfusepath{clip}%
\pgfsetroundcap%
\pgfsetroundjoin%
\definecolor{currentfill}{rgb}{0.500000,0.500000,0.500000}%
\pgfsetfillcolor{currentfill}%
\pgfsetfillopacity{0.300000}%
\pgfsetlinewidth{0.301125pt}%
\definecolor{currentstroke}{rgb}{0.500000,0.500000,0.500000}%
\pgfsetstrokecolor{currentstroke}%
\pgfsetstrokeopacity{0.300000}%
\pgfsetdash{}{0pt}%
\pgfpathmoveto{\pgfqpoint{0.000000in}{0.000000in}}%
\pgfpathlineto{\pgfqpoint{0.000000in}{0.000000in}}%
\pgfpathclose%
\pgfusepath{stroke,fill}%
\end{pgfscope}%
\begin{pgfscope}%
\pgfpathrectangle{\pgfqpoint{0.647939in}{0.492442in}}{\pgfqpoint{3.079299in}{3.079299in}}%
\pgfusepath{clip}%
\pgfsetroundcap%
\pgfsetroundjoin%
\pgfsetlinewidth{0.301125pt}%
\definecolor{currentstroke}{rgb}{0.500000,0.500000,0.500000}%
\pgfsetstrokecolor{currentstroke}%
\pgfsetstrokeopacity{0.300000}%
\pgfsetdash{}{0pt}%
\pgfpathmoveto{\pgfqpoint{1.741685in}{2.840007in}}%
\pgfusepath{stroke}%
\end{pgfscope}%
\begin{pgfscope}%
\pgfpathrectangle{\pgfqpoint{0.647939in}{0.492442in}}{\pgfqpoint{3.079299in}{3.079299in}}%
\pgfusepath{clip}%
\pgfsetroundcap%
\pgfsetroundjoin%
\definecolor{currentfill}{rgb}{0.500000,0.500000,0.500000}%
\pgfsetfillcolor{currentfill}%
\pgfsetfillopacity{0.300000}%
\pgfsetlinewidth{0.301125pt}%
\definecolor{currentstroke}{rgb}{0.500000,0.500000,0.500000}%
\pgfsetstrokecolor{currentstroke}%
\pgfsetstrokeopacity{0.300000}%
\pgfsetdash{}{0pt}%
\pgfpathmoveto{\pgfqpoint{0.000000in}{0.000000in}}%
\pgfpathlineto{\pgfqpoint{0.000000in}{0.000000in}}%
\pgfpathclose%
\pgfusepath{stroke,fill}%
\end{pgfscope}%
\begin{pgfscope}%
\pgfpathrectangle{\pgfqpoint{0.647939in}{0.492442in}}{\pgfqpoint{3.079299in}{3.079299in}}%
\pgfusepath{clip}%
\pgfsetroundcap%
\pgfsetroundjoin%
\pgfsetlinewidth{0.301125pt}%
\definecolor{currentstroke}{rgb}{0.500000,0.500000,0.500000}%
\pgfsetstrokecolor{currentstroke}%
\pgfsetstrokeopacity{0.300000}%
\pgfsetdash{}{0pt}%
\pgfpathmoveto{\pgfqpoint{1.603245in}{2.619517in}}%
\pgfusepath{stroke}%
\end{pgfscope}%
\begin{pgfscope}%
\pgfpathrectangle{\pgfqpoint{0.647939in}{0.492442in}}{\pgfqpoint{3.079299in}{3.079299in}}%
\pgfusepath{clip}%
\pgfsetroundcap%
\pgfsetroundjoin%
\definecolor{currentfill}{rgb}{0.500000,0.500000,0.500000}%
\pgfsetfillcolor{currentfill}%
\pgfsetfillopacity{0.300000}%
\pgfsetlinewidth{0.301125pt}%
\definecolor{currentstroke}{rgb}{0.500000,0.500000,0.500000}%
\pgfsetstrokecolor{currentstroke}%
\pgfsetstrokeopacity{0.300000}%
\pgfsetdash{}{0pt}%
\pgfpathmoveto{\pgfqpoint{0.000000in}{0.000000in}}%
\pgfpathlineto{\pgfqpoint{0.000000in}{0.000000in}}%
\pgfpathclose%
\pgfusepath{stroke,fill}%
\end{pgfscope}%
\begin{pgfscope}%
\pgfpathrectangle{\pgfqpoint{0.647939in}{0.492442in}}{\pgfqpoint{3.079299in}{3.079299in}}%
\pgfusepath{clip}%
\pgfsetroundcap%
\pgfsetroundjoin%
\pgfsetlinewidth{0.301125pt}%
\definecolor{currentstroke}{rgb}{0.500000,0.500000,0.500000}%
\pgfsetstrokecolor{currentstroke}%
\pgfsetstrokeopacity{0.300000}%
\pgfsetdash{}{0pt}%
\pgfpathmoveto{\pgfqpoint{1.011122in}{2.380876in}}%
\pgfusepath{stroke}%
\end{pgfscope}%
\begin{pgfscope}%
\pgfpathrectangle{\pgfqpoint{0.647939in}{0.492442in}}{\pgfqpoint{3.079299in}{3.079299in}}%
\pgfusepath{clip}%
\pgfsetroundcap%
\pgfsetroundjoin%
\definecolor{currentfill}{rgb}{0.500000,0.500000,0.500000}%
\pgfsetfillcolor{currentfill}%
\pgfsetfillopacity{0.300000}%
\pgfsetlinewidth{0.301125pt}%
\definecolor{currentstroke}{rgb}{0.500000,0.500000,0.500000}%
\pgfsetstrokecolor{currentstroke}%
\pgfsetstrokeopacity{0.300000}%
\pgfsetdash{}{0pt}%
\pgfpathmoveto{\pgfqpoint{0.000000in}{0.000000in}}%
\pgfpathlineto{\pgfqpoint{0.000000in}{0.000000in}}%
\pgfpathclose%
\pgfusepath{stroke,fill}%
\end{pgfscope}%
\begin{pgfscope}%
\pgfpathrectangle{\pgfqpoint{0.647939in}{0.492442in}}{\pgfqpoint{3.079299in}{3.079299in}}%
\pgfusepath{clip}%
\pgfsetroundcap%
\pgfsetroundjoin%
\pgfsetlinewidth{0.301125pt}%
\definecolor{currentstroke}{rgb}{0.500000,0.500000,0.500000}%
\pgfsetstrokecolor{currentstroke}%
\pgfsetstrokeopacity{0.300000}%
\pgfsetdash{}{0pt}%
\pgfpathmoveto{\pgfqpoint{1.010879in}{2.312082in}}%
\pgfusepath{stroke}%
\end{pgfscope}%
\begin{pgfscope}%
\pgfpathrectangle{\pgfqpoint{0.647939in}{0.492442in}}{\pgfqpoint{3.079299in}{3.079299in}}%
\pgfusepath{clip}%
\pgfsetroundcap%
\pgfsetroundjoin%
\definecolor{currentfill}{rgb}{0.500000,0.500000,0.500000}%
\pgfsetfillcolor{currentfill}%
\pgfsetfillopacity{0.300000}%
\pgfsetlinewidth{0.301125pt}%
\definecolor{currentstroke}{rgb}{0.500000,0.500000,0.500000}%
\pgfsetstrokecolor{currentstroke}%
\pgfsetstrokeopacity{0.300000}%
\pgfsetdash{}{0pt}%
\pgfpathmoveto{\pgfqpoint{0.000000in}{0.000000in}}%
\pgfpathlineto{\pgfqpoint{0.000000in}{0.000000in}}%
\pgfpathclose%
\pgfusepath{stroke,fill}%
\end{pgfscope}%
\begin{pgfscope}%
\pgfpathrectangle{\pgfqpoint{0.647939in}{0.492442in}}{\pgfqpoint{3.079299in}{3.079299in}}%
\pgfusepath{clip}%
\pgfsetroundcap%
\pgfsetroundjoin%
\pgfsetlinewidth{0.301125pt}%
\definecolor{currentstroke}{rgb}{0.500000,0.500000,0.500000}%
\pgfsetstrokecolor{currentstroke}%
\pgfsetstrokeopacity{0.300000}%
\pgfsetdash{}{0pt}%
\pgfpathmoveto{\pgfqpoint{1.663356in}{2.448188in}}%
\pgfusepath{stroke}%
\end{pgfscope}%
\begin{pgfscope}%
\pgfpathrectangle{\pgfqpoint{0.647939in}{0.492442in}}{\pgfqpoint{3.079299in}{3.079299in}}%
\pgfusepath{clip}%
\pgfsetroundcap%
\pgfsetroundjoin%
\definecolor{currentfill}{rgb}{0.500000,0.500000,0.500000}%
\pgfsetfillcolor{currentfill}%
\pgfsetfillopacity{0.300000}%
\pgfsetlinewidth{0.301125pt}%
\definecolor{currentstroke}{rgb}{0.500000,0.500000,0.500000}%
\pgfsetstrokecolor{currentstroke}%
\pgfsetstrokeopacity{0.300000}%
\pgfsetdash{}{0pt}%
\pgfpathmoveto{\pgfqpoint{0.000000in}{0.000000in}}%
\pgfpathlineto{\pgfqpoint{0.000000in}{0.000000in}}%
\pgfpathclose%
\pgfusepath{stroke,fill}%
\end{pgfscope}%
\begin{pgfscope}%
\pgfpathrectangle{\pgfqpoint{0.647939in}{0.492442in}}{\pgfqpoint{3.079299in}{3.079299in}}%
\pgfusepath{clip}%
\pgfsetroundcap%
\pgfsetroundjoin%
\pgfsetlinewidth{0.301125pt}%
\definecolor{currentstroke}{rgb}{0.500000,0.500000,0.500000}%
\pgfsetstrokecolor{currentstroke}%
\pgfsetstrokeopacity{0.300000}%
\pgfsetdash{}{0pt}%
\pgfpathmoveto{\pgfqpoint{1.465589in}{2.323160in}}%
\pgfusepath{stroke}%
\end{pgfscope}%
\begin{pgfscope}%
\pgfpathrectangle{\pgfqpoint{0.647939in}{0.492442in}}{\pgfqpoint{3.079299in}{3.079299in}}%
\pgfusepath{clip}%
\pgfsetroundcap%
\pgfsetroundjoin%
\definecolor{currentfill}{rgb}{0.500000,0.500000,0.500000}%
\pgfsetfillcolor{currentfill}%
\pgfsetfillopacity{0.300000}%
\pgfsetlinewidth{0.301125pt}%
\definecolor{currentstroke}{rgb}{0.500000,0.500000,0.500000}%
\pgfsetstrokecolor{currentstroke}%
\pgfsetstrokeopacity{0.300000}%
\pgfsetdash{}{0pt}%
\pgfpathmoveto{\pgfqpoint{0.000000in}{0.000000in}}%
\pgfpathlineto{\pgfqpoint{0.000000in}{0.000000in}}%
\pgfpathclose%
\pgfusepath{stroke,fill}%
\end{pgfscope}%
\begin{pgfscope}%
\pgfpathrectangle{\pgfqpoint{0.647939in}{0.492442in}}{\pgfqpoint{3.079299in}{3.079299in}}%
\pgfusepath{clip}%
\pgfsetroundcap%
\pgfsetroundjoin%
\pgfsetlinewidth{0.301125pt}%
\definecolor{currentstroke}{rgb}{0.500000,0.500000,0.500000}%
\pgfsetstrokecolor{currentstroke}%
\pgfsetstrokeopacity{0.300000}%
\pgfsetdash{}{0pt}%
\pgfpathmoveto{\pgfqpoint{1.206134in}{2.166619in}}%
\pgfusepath{stroke}%
\end{pgfscope}%
\begin{pgfscope}%
\pgfpathrectangle{\pgfqpoint{0.647939in}{0.492442in}}{\pgfqpoint{3.079299in}{3.079299in}}%
\pgfusepath{clip}%
\pgfsetroundcap%
\pgfsetroundjoin%
\definecolor{currentfill}{rgb}{0.500000,0.500000,0.500000}%
\pgfsetfillcolor{currentfill}%
\pgfsetfillopacity{0.300000}%
\pgfsetlinewidth{0.301125pt}%
\definecolor{currentstroke}{rgb}{0.500000,0.500000,0.500000}%
\pgfsetstrokecolor{currentstroke}%
\pgfsetstrokeopacity{0.300000}%
\pgfsetdash{}{0pt}%
\pgfpathmoveto{\pgfqpoint{0.000000in}{0.000000in}}%
\pgfpathlineto{\pgfqpoint{0.000000in}{0.000000in}}%
\pgfpathclose%
\pgfusepath{stroke,fill}%
\end{pgfscope}%
\begin{pgfscope}%
\pgfpathrectangle{\pgfqpoint{0.647939in}{0.492442in}}{\pgfqpoint{3.079299in}{3.079299in}}%
\pgfusepath{clip}%
\pgfsetroundcap%
\pgfsetroundjoin%
\pgfsetlinewidth{0.301125pt}%
\definecolor{currentstroke}{rgb}{0.500000,0.500000,0.500000}%
\pgfsetstrokecolor{currentstroke}%
\pgfsetstrokeopacity{0.300000}%
\pgfsetdash{}{0pt}%
\pgfpathmoveto{\pgfqpoint{1.523468in}{2.156291in}}%
\pgfusepath{stroke}%
\end{pgfscope}%
\begin{pgfscope}%
\pgfpathrectangle{\pgfqpoint{0.647939in}{0.492442in}}{\pgfqpoint{3.079299in}{3.079299in}}%
\pgfusepath{clip}%
\pgfsetroundcap%
\pgfsetroundjoin%
\definecolor{currentfill}{rgb}{0.500000,0.500000,0.500000}%
\pgfsetfillcolor{currentfill}%
\pgfsetfillopacity{0.300000}%
\pgfsetlinewidth{0.301125pt}%
\definecolor{currentstroke}{rgb}{0.500000,0.500000,0.500000}%
\pgfsetstrokecolor{currentstroke}%
\pgfsetstrokeopacity{0.300000}%
\pgfsetdash{}{0pt}%
\pgfpathmoveto{\pgfqpoint{0.000000in}{0.000000in}}%
\pgfpathlineto{\pgfqpoint{0.000000in}{0.000000in}}%
\pgfpathclose%
\pgfusepath{stroke,fill}%
\end{pgfscope}%
\begin{pgfscope}%
\pgfpathrectangle{\pgfqpoint{0.647939in}{0.492442in}}{\pgfqpoint{3.079299in}{3.079299in}}%
\pgfusepath{clip}%
\pgfsetroundcap%
\pgfsetroundjoin%
\pgfsetlinewidth{0.301125pt}%
\definecolor{currentstroke}{rgb}{0.500000,0.500000,0.500000}%
\pgfsetstrokecolor{currentstroke}%
\pgfsetstrokeopacity{0.300000}%
\pgfsetdash{}{0pt}%
\pgfpathmoveto{\pgfqpoint{1.267487in}{1.990217in}}%
\pgfusepath{stroke}%
\end{pgfscope}%
\begin{pgfscope}%
\pgfpathrectangle{\pgfqpoint{0.647939in}{0.492442in}}{\pgfqpoint{3.079299in}{3.079299in}}%
\pgfusepath{clip}%
\pgfsetroundcap%
\pgfsetroundjoin%
\definecolor{currentfill}{rgb}{0.500000,0.500000,0.500000}%
\pgfsetfillcolor{currentfill}%
\pgfsetfillopacity{0.300000}%
\pgfsetlinewidth{0.301125pt}%
\definecolor{currentstroke}{rgb}{0.500000,0.500000,0.500000}%
\pgfsetstrokecolor{currentstroke}%
\pgfsetstrokeopacity{0.300000}%
\pgfsetdash{}{0pt}%
\pgfpathmoveto{\pgfqpoint{0.000000in}{0.000000in}}%
\pgfpathlineto{\pgfqpoint{0.000000in}{0.000000in}}%
\pgfpathclose%
\pgfusepath{stroke,fill}%
\end{pgfscope}%
\begin{pgfscope}%
\pgfpathrectangle{\pgfqpoint{0.647939in}{0.492442in}}{\pgfqpoint{3.079299in}{3.079299in}}%
\pgfusepath{clip}%
\pgfsetroundcap%
\pgfsetroundjoin%
\pgfsetlinewidth{0.301125pt}%
\definecolor{currentstroke}{rgb}{0.500000,0.500000,0.500000}%
\pgfsetstrokecolor{currentstroke}%
\pgfsetstrokeopacity{0.300000}%
\pgfsetdash{}{0pt}%
\pgfpathmoveto{\pgfqpoint{1.138555in}{1.875020in}}%
\pgfusepath{stroke}%
\end{pgfscope}%
\begin{pgfscope}%
\pgfpathrectangle{\pgfqpoint{0.647939in}{0.492442in}}{\pgfqpoint{3.079299in}{3.079299in}}%
\pgfusepath{clip}%
\pgfsetroundcap%
\pgfsetroundjoin%
\definecolor{currentfill}{rgb}{0.500000,0.500000,0.500000}%
\pgfsetfillcolor{currentfill}%
\pgfsetfillopacity{0.300000}%
\pgfsetlinewidth{0.301125pt}%
\definecolor{currentstroke}{rgb}{0.500000,0.500000,0.500000}%
\pgfsetstrokecolor{currentstroke}%
\pgfsetstrokeopacity{0.300000}%
\pgfsetdash{}{0pt}%
\pgfpathmoveto{\pgfqpoint{0.000000in}{0.000000in}}%
\pgfpathlineto{\pgfqpoint{0.000000in}{0.000000in}}%
\pgfpathclose%
\pgfusepath{stroke,fill}%
\end{pgfscope}%
\begin{pgfscope}%
\pgfpathrectangle{\pgfqpoint{0.647939in}{0.492442in}}{\pgfqpoint{3.079299in}{3.079299in}}%
\pgfusepath{clip}%
\pgfsetroundcap%
\pgfsetroundjoin%
\pgfsetlinewidth{0.301125pt}%
\definecolor{currentstroke}{rgb}{0.500000,0.500000,0.500000}%
\pgfsetstrokecolor{currentstroke}%
\pgfsetstrokeopacity{0.300000}%
\pgfsetdash{}{0pt}%
\pgfpathmoveto{\pgfqpoint{1.008362in}{1.763381in}}%
\pgfusepath{stroke}%
\end{pgfscope}%
\begin{pgfscope}%
\pgfpathrectangle{\pgfqpoint{0.647939in}{0.492442in}}{\pgfqpoint{3.079299in}{3.079299in}}%
\pgfusepath{clip}%
\pgfsetroundcap%
\pgfsetroundjoin%
\definecolor{currentfill}{rgb}{0.500000,0.500000,0.500000}%
\pgfsetfillcolor{currentfill}%
\pgfsetfillopacity{0.300000}%
\pgfsetlinewidth{0.301125pt}%
\definecolor{currentstroke}{rgb}{0.500000,0.500000,0.500000}%
\pgfsetstrokecolor{currentstroke}%
\pgfsetstrokeopacity{0.300000}%
\pgfsetdash{}{0pt}%
\pgfpathmoveto{\pgfqpoint{0.000000in}{0.000000in}}%
\pgfpathlineto{\pgfqpoint{0.000000in}{0.000000in}}%
\pgfpathclose%
\pgfusepath{stroke,fill}%
\end{pgfscope}%
\begin{pgfscope}%
\pgfpathrectangle{\pgfqpoint{0.647939in}{0.492442in}}{\pgfqpoint{3.079299in}{3.079299in}}%
\pgfusepath{clip}%
\pgfsetroundcap%
\pgfsetroundjoin%
\pgfsetlinewidth{0.301125pt}%
\definecolor{currentstroke}{rgb}{0.500000,0.500000,0.500000}%
\pgfsetstrokecolor{currentstroke}%
\pgfsetstrokeopacity{0.300000}%
\pgfsetdash{}{0pt}%
\pgfpathmoveto{\pgfqpoint{1.449328in}{1.879868in}}%
\pgfusepath{stroke}%
\end{pgfscope}%
\begin{pgfscope}%
\pgfpathrectangle{\pgfqpoint{0.647939in}{0.492442in}}{\pgfqpoint{3.079299in}{3.079299in}}%
\pgfusepath{clip}%
\pgfsetroundcap%
\pgfsetroundjoin%
\definecolor{currentfill}{rgb}{0.500000,0.500000,0.500000}%
\pgfsetfillcolor{currentfill}%
\pgfsetfillopacity{0.300000}%
\pgfsetlinewidth{0.301125pt}%
\definecolor{currentstroke}{rgb}{0.500000,0.500000,0.500000}%
\pgfsetstrokecolor{currentstroke}%
\pgfsetstrokeopacity{0.300000}%
\pgfsetdash{}{0pt}%
\pgfpathmoveto{\pgfqpoint{0.000000in}{0.000000in}}%
\pgfpathlineto{\pgfqpoint{0.000000in}{0.000000in}}%
\pgfpathclose%
\pgfusepath{stroke,fill}%
\end{pgfscope}%
\begin{pgfscope}%
\pgfpathrectangle{\pgfqpoint{0.647939in}{0.492442in}}{\pgfqpoint{3.079299in}{3.079299in}}%
\pgfusepath{clip}%
\pgfsetroundcap%
\pgfsetroundjoin%
\pgfsetlinewidth{0.301125pt}%
\definecolor{currentstroke}{rgb}{0.500000,0.500000,0.500000}%
\pgfsetstrokecolor{currentstroke}%
\pgfsetstrokeopacity{0.300000}%
\pgfsetdash{}{0pt}%
\pgfpathmoveto{\pgfqpoint{1.136084in}{1.673537in}}%
\pgfusepath{stroke}%
\end{pgfscope}%
\begin{pgfscope}%
\pgfpathrectangle{\pgfqpoint{0.647939in}{0.492442in}}{\pgfqpoint{3.079299in}{3.079299in}}%
\pgfusepath{clip}%
\pgfsetroundcap%
\pgfsetroundjoin%
\definecolor{currentfill}{rgb}{0.500000,0.500000,0.500000}%
\pgfsetfillcolor{currentfill}%
\pgfsetfillopacity{0.300000}%
\pgfsetlinewidth{0.301125pt}%
\definecolor{currentstroke}{rgb}{0.500000,0.500000,0.500000}%
\pgfsetstrokecolor{currentstroke}%
\pgfsetstrokeopacity{0.300000}%
\pgfsetdash{}{0pt}%
\pgfpathmoveto{\pgfqpoint{0.000000in}{0.000000in}}%
\pgfpathlineto{\pgfqpoint{0.000000in}{0.000000in}}%
\pgfpathclose%
\pgfusepath{stroke,fill}%
\end{pgfscope}%
\begin{pgfscope}%
\pgfpathrectangle{\pgfqpoint{0.647939in}{0.492442in}}{\pgfqpoint{3.079299in}{3.079299in}}%
\pgfusepath{clip}%
\pgfsetroundcap%
\pgfsetroundjoin%
\pgfsetlinewidth{0.301125pt}%
\definecolor{currentstroke}{rgb}{0.500000,0.500000,0.500000}%
\pgfsetstrokecolor{currentstroke}%
\pgfsetstrokeopacity{0.300000}%
\pgfsetdash{}{0pt}%
\pgfpathmoveto{\pgfqpoint{0.876048in}{1.519211in}}%
\pgfusepath{stroke}%
\end{pgfscope}%
\begin{pgfscope}%
\pgfpathrectangle{\pgfqpoint{0.647939in}{0.492442in}}{\pgfqpoint{3.079299in}{3.079299in}}%
\pgfusepath{clip}%
\pgfsetroundcap%
\pgfsetroundjoin%
\definecolor{currentfill}{rgb}{0.500000,0.500000,0.500000}%
\pgfsetfillcolor{currentfill}%
\pgfsetfillopacity{0.300000}%
\pgfsetlinewidth{0.301125pt}%
\definecolor{currentstroke}{rgb}{0.500000,0.500000,0.500000}%
\pgfsetstrokecolor{currentstroke}%
\pgfsetstrokeopacity{0.300000}%
\pgfsetdash{}{0pt}%
\pgfpathmoveto{\pgfqpoint{0.000000in}{0.000000in}}%
\pgfpathlineto{\pgfqpoint{0.000000in}{0.000000in}}%
\pgfpathclose%
\pgfusepath{stroke,fill}%
\end{pgfscope}%
\begin{pgfscope}%
\pgfpathrectangle{\pgfqpoint{0.647939in}{0.492442in}}{\pgfqpoint{3.079299in}{3.079299in}}%
\pgfusepath{clip}%
\pgfsetroundcap%
\pgfsetroundjoin%
\pgfsetlinewidth{0.301125pt}%
\definecolor{currentstroke}{rgb}{0.500000,0.500000,0.500000}%
\pgfsetstrokecolor{currentstroke}%
\pgfsetstrokeopacity{0.300000}%
\pgfsetdash{}{0pt}%
\pgfpathmoveto{\pgfqpoint{1.379151in}{1.661436in}}%
\pgfusepath{stroke}%
\end{pgfscope}%
\begin{pgfscope}%
\pgfpathrectangle{\pgfqpoint{0.647939in}{0.492442in}}{\pgfqpoint{3.079299in}{3.079299in}}%
\pgfusepath{clip}%
\pgfsetroundcap%
\pgfsetroundjoin%
\definecolor{currentfill}{rgb}{0.500000,0.500000,0.500000}%
\pgfsetfillcolor{currentfill}%
\pgfsetfillopacity{0.300000}%
\pgfsetlinewidth{0.301125pt}%
\definecolor{currentstroke}{rgb}{0.500000,0.500000,0.500000}%
\pgfsetstrokecolor{currentstroke}%
\pgfsetstrokeopacity{0.300000}%
\pgfsetdash{}{0pt}%
\pgfpathmoveto{\pgfqpoint{0.000000in}{0.000000in}}%
\pgfpathlineto{\pgfqpoint{0.000000in}{0.000000in}}%
\pgfpathclose%
\pgfusepath{stroke,fill}%
\end{pgfscope}%
\begin{pgfscope}%
\pgfpathrectangle{\pgfqpoint{0.647939in}{0.492442in}}{\pgfqpoint{3.079299in}{3.079299in}}%
\pgfusepath{clip}%
\pgfsetroundcap%
\pgfsetroundjoin%
\pgfsetlinewidth{0.301125pt}%
\definecolor{currentstroke}{rgb}{0.500000,0.500000,0.500000}%
\pgfsetstrokecolor{currentstroke}%
\pgfsetstrokeopacity{0.300000}%
\pgfsetdash{}{0pt}%
\pgfpathmoveto{\pgfqpoint{1.070003in}{1.446607in}}%
\pgfusepath{stroke}%
\end{pgfscope}%
\begin{pgfscope}%
\pgfpathrectangle{\pgfqpoint{0.647939in}{0.492442in}}{\pgfqpoint{3.079299in}{3.079299in}}%
\pgfusepath{clip}%
\pgfsetroundcap%
\pgfsetroundjoin%
\definecolor{currentfill}{rgb}{0.500000,0.500000,0.500000}%
\pgfsetfillcolor{currentfill}%
\pgfsetfillopacity{0.300000}%
\pgfsetlinewidth{0.301125pt}%
\definecolor{currentstroke}{rgb}{0.500000,0.500000,0.500000}%
\pgfsetstrokecolor{currentstroke}%
\pgfsetstrokeopacity{0.300000}%
\pgfsetdash{}{0pt}%
\pgfpathmoveto{\pgfqpoint{0.000000in}{0.000000in}}%
\pgfpathlineto{\pgfqpoint{0.000000in}{0.000000in}}%
\pgfpathclose%
\pgfusepath{stroke,fill}%
\end{pgfscope}%
\begin{pgfscope}%
\pgfpathrectangle{\pgfqpoint{0.647939in}{0.492442in}}{\pgfqpoint{3.079299in}{3.079299in}}%
\pgfusepath{clip}%
\pgfsetroundcap%
\pgfsetroundjoin%
\pgfsetlinewidth{0.301125pt}%
\definecolor{currentstroke}{rgb}{0.500000,0.500000,0.500000}%
\pgfsetstrokecolor{currentstroke}%
\pgfsetstrokeopacity{0.300000}%
\pgfsetdash{}{0pt}%
\pgfpathmoveto{\pgfqpoint{0.809260in}{1.294693in}}%
\pgfusepath{stroke}%
\end{pgfscope}%
\begin{pgfscope}%
\pgfpathrectangle{\pgfqpoint{0.647939in}{0.492442in}}{\pgfqpoint{3.079299in}{3.079299in}}%
\pgfusepath{clip}%
\pgfsetroundcap%
\pgfsetroundjoin%
\definecolor{currentfill}{rgb}{0.500000,0.500000,0.500000}%
\pgfsetfillcolor{currentfill}%
\pgfsetfillopacity{0.300000}%
\pgfsetlinewidth{0.301125pt}%
\definecolor{currentstroke}{rgb}{0.500000,0.500000,0.500000}%
\pgfsetstrokecolor{currentstroke}%
\pgfsetstrokeopacity{0.300000}%
\pgfsetdash{}{0pt}%
\pgfpathmoveto{\pgfqpoint{0.000000in}{0.000000in}}%
\pgfpathlineto{\pgfqpoint{0.000000in}{0.000000in}}%
\pgfpathclose%
\pgfusepath{stroke,fill}%
\end{pgfscope}%
\begin{pgfscope}%
\pgfpathrectangle{\pgfqpoint{0.647939in}{0.492442in}}{\pgfqpoint{3.079299in}{3.079299in}}%
\pgfusepath{clip}%
\pgfsetroundcap%
\pgfsetroundjoin%
\pgfsetlinewidth{0.301125pt}%
\definecolor{currentstroke}{rgb}{0.500000,0.500000,0.500000}%
\pgfsetstrokecolor{currentstroke}%
\pgfsetstrokeopacity{0.300000}%
\pgfsetdash{}{0pt}%
\pgfpathmoveto{\pgfqpoint{1.310420in}{1.439007in}}%
\pgfusepath{stroke}%
\end{pgfscope}%
\begin{pgfscope}%
\pgfpathrectangle{\pgfqpoint{0.647939in}{0.492442in}}{\pgfqpoint{3.079299in}{3.079299in}}%
\pgfusepath{clip}%
\pgfsetroundcap%
\pgfsetroundjoin%
\definecolor{currentfill}{rgb}{0.500000,0.500000,0.500000}%
\pgfsetfillcolor{currentfill}%
\pgfsetfillopacity{0.300000}%
\pgfsetlinewidth{0.301125pt}%
\definecolor{currentstroke}{rgb}{0.500000,0.500000,0.500000}%
\pgfsetstrokecolor{currentstroke}%
\pgfsetstrokeopacity{0.300000}%
\pgfsetdash{}{0pt}%
\pgfpathmoveto{\pgfqpoint{0.000000in}{0.000000in}}%
\pgfpathlineto{\pgfqpoint{0.000000in}{0.000000in}}%
\pgfpathclose%
\pgfusepath{stroke,fill}%
\end{pgfscope}%
\begin{pgfscope}%
\pgfpathrectangle{\pgfqpoint{0.647939in}{0.492442in}}{\pgfqpoint{3.079299in}{3.079299in}}%
\pgfusepath{clip}%
\pgfsetroundcap%
\pgfsetroundjoin%
\pgfsetlinewidth{0.301125pt}%
\definecolor{currentstroke}{rgb}{0.500000,0.500000,0.500000}%
\pgfsetstrokecolor{currentstroke}%
\pgfsetstrokeopacity{0.300000}%
\pgfsetdash{}{0pt}%
\pgfpathmoveto{\pgfqpoint{1.004246in}{1.218651in}}%
\pgfusepath{stroke}%
\end{pgfscope}%
\begin{pgfscope}%
\pgfpathrectangle{\pgfqpoint{0.647939in}{0.492442in}}{\pgfqpoint{3.079299in}{3.079299in}}%
\pgfusepath{clip}%
\pgfsetroundcap%
\pgfsetroundjoin%
\definecolor{currentfill}{rgb}{0.500000,0.500000,0.500000}%
\pgfsetfillcolor{currentfill}%
\pgfsetfillopacity{0.300000}%
\pgfsetlinewidth{0.301125pt}%
\definecolor{currentstroke}{rgb}{0.500000,0.500000,0.500000}%
\pgfsetstrokecolor{currentstroke}%
\pgfsetstrokeopacity{0.300000}%
\pgfsetdash{}{0pt}%
\pgfpathmoveto{\pgfqpoint{0.000000in}{0.000000in}}%
\pgfpathlineto{\pgfqpoint{0.000000in}{0.000000in}}%
\pgfpathclose%
\pgfusepath{stroke,fill}%
\end{pgfscope}%
\begin{pgfscope}%
\pgfpathrectangle{\pgfqpoint{0.647939in}{0.492442in}}{\pgfqpoint{3.079299in}{3.079299in}}%
\pgfusepath{clip}%
\pgfsetroundcap%
\pgfsetroundjoin%
\pgfsetlinewidth{0.301125pt}%
\definecolor{currentstroke}{rgb}{0.500000,0.500000,0.500000}%
\pgfsetstrokecolor{currentstroke}%
\pgfsetstrokeopacity{0.300000}%
\pgfsetdash{}{0pt}%
\pgfpathmoveto{\pgfqpoint{0.939624in}{1.126651in}}%
\pgfusepath{stroke}%
\end{pgfscope}%
\begin{pgfscope}%
\pgfpathrectangle{\pgfqpoint{0.647939in}{0.492442in}}{\pgfqpoint{3.079299in}{3.079299in}}%
\pgfusepath{clip}%
\pgfsetroundcap%
\pgfsetroundjoin%
\definecolor{currentfill}{rgb}{0.500000,0.500000,0.500000}%
\pgfsetfillcolor{currentfill}%
\pgfsetfillopacity{0.300000}%
\pgfsetlinewidth{0.301125pt}%
\definecolor{currentstroke}{rgb}{0.500000,0.500000,0.500000}%
\pgfsetstrokecolor{currentstroke}%
\pgfsetstrokeopacity{0.300000}%
\pgfsetdash{}{0pt}%
\pgfpathmoveto{\pgfqpoint{0.000000in}{0.000000in}}%
\pgfpathlineto{\pgfqpoint{0.000000in}{0.000000in}}%
\pgfpathclose%
\pgfusepath{stroke,fill}%
\end{pgfscope}%
\begin{pgfscope}%
\pgfpathrectangle{\pgfqpoint{0.647939in}{0.492442in}}{\pgfqpoint{3.079299in}{3.079299in}}%
\pgfusepath{clip}%
\pgfsetroundcap%
\pgfsetroundjoin%
\pgfsetlinewidth{0.301125pt}%
\definecolor{currentstroke}{rgb}{0.500000,0.500000,0.500000}%
\pgfsetstrokecolor{currentstroke}%
\pgfsetstrokeopacity{0.300000}%
\pgfsetdash{}{0pt}%
\pgfpathmoveto{\pgfqpoint{0.874380in}{1.036408in}}%
\pgfusepath{stroke}%
\end{pgfscope}%
\begin{pgfscope}%
\pgfpathrectangle{\pgfqpoint{0.647939in}{0.492442in}}{\pgfqpoint{3.079299in}{3.079299in}}%
\pgfusepath{clip}%
\pgfsetroundcap%
\pgfsetroundjoin%
\definecolor{currentfill}{rgb}{0.500000,0.500000,0.500000}%
\pgfsetfillcolor{currentfill}%
\pgfsetfillopacity{0.300000}%
\pgfsetlinewidth{0.301125pt}%
\definecolor{currentstroke}{rgb}{0.500000,0.500000,0.500000}%
\pgfsetstrokecolor{currentstroke}%
\pgfsetstrokeopacity{0.300000}%
\pgfsetdash{}{0pt}%
\pgfpathmoveto{\pgfqpoint{0.000000in}{0.000000in}}%
\pgfpathlineto{\pgfqpoint{0.000000in}{0.000000in}}%
\pgfpathclose%
\pgfusepath{stroke,fill}%
\end{pgfscope}%
\begin{pgfscope}%
\pgfpathrectangle{\pgfqpoint{0.647939in}{0.492442in}}{\pgfqpoint{3.079299in}{3.079299in}}%
\pgfusepath{clip}%
\pgfsetroundcap%
\pgfsetroundjoin%
\pgfsetlinewidth{0.301125pt}%
\definecolor{currentstroke}{rgb}{0.500000,0.500000,0.500000}%
\pgfsetstrokecolor{currentstroke}%
\pgfsetstrokeopacity{0.300000}%
\pgfsetdash{}{0pt}%
\pgfpathmoveto{\pgfqpoint{1.183080in}{1.111818in}}%
\pgfusepath{stroke}%
\end{pgfscope}%
\begin{pgfscope}%
\pgfpathrectangle{\pgfqpoint{0.647939in}{0.492442in}}{\pgfqpoint{3.079299in}{3.079299in}}%
\pgfusepath{clip}%
\pgfsetroundcap%
\pgfsetroundjoin%
\definecolor{currentfill}{rgb}{0.500000,0.500000,0.500000}%
\pgfsetfillcolor{currentfill}%
\pgfsetfillopacity{0.300000}%
\pgfsetlinewidth{0.301125pt}%
\definecolor{currentstroke}{rgb}{0.500000,0.500000,0.500000}%
\pgfsetstrokecolor{currentstroke}%
\pgfsetstrokeopacity{0.300000}%
\pgfsetdash{}{0pt}%
\pgfpathmoveto{\pgfqpoint{0.000000in}{0.000000in}}%
\pgfpathlineto{\pgfqpoint{0.000000in}{0.000000in}}%
\pgfpathclose%
\pgfusepath{stroke,fill}%
\end{pgfscope}%
\begin{pgfscope}%
\pgfpathrectangle{\pgfqpoint{0.647939in}{0.492442in}}{\pgfqpoint{3.079299in}{3.079299in}}%
\pgfusepath{clip}%
\pgfsetroundcap%
\pgfsetroundjoin%
\pgfsetlinewidth{0.301125pt}%
\definecolor{currentstroke}{rgb}{0.500000,0.500000,0.500000}%
\pgfsetstrokecolor{currentstroke}%
\pgfsetstrokeopacity{0.300000}%
\pgfsetdash{}{0pt}%
\pgfpathmoveto{\pgfqpoint{1.001159in}{0.948410in}}%
\pgfusepath{stroke}%
\end{pgfscope}%
\begin{pgfscope}%
\pgfpathrectangle{\pgfqpoint{0.647939in}{0.492442in}}{\pgfqpoint{3.079299in}{3.079299in}}%
\pgfusepath{clip}%
\pgfsetroundcap%
\pgfsetroundjoin%
\definecolor{currentfill}{rgb}{0.500000,0.500000,0.500000}%
\pgfsetfillcolor{currentfill}%
\pgfsetfillopacity{0.300000}%
\pgfsetlinewidth{0.301125pt}%
\definecolor{currentstroke}{rgb}{0.500000,0.500000,0.500000}%
\pgfsetstrokecolor{currentstroke}%
\pgfsetstrokeopacity{0.300000}%
\pgfsetdash{}{0pt}%
\pgfpathmoveto{\pgfqpoint{0.000000in}{0.000000in}}%
\pgfpathlineto{\pgfqpoint{0.000000in}{0.000000in}}%
\pgfpathclose%
\pgfusepath{stroke,fill}%
\end{pgfscope}%
\begin{pgfscope}%
\pgfpathrectangle{\pgfqpoint{0.647939in}{0.492442in}}{\pgfqpoint{3.079299in}{3.079299in}}%
\pgfusepath{clip}%
\pgfsetroundcap%
\pgfsetroundjoin%
\pgfsetlinewidth{0.301125pt}%
\definecolor{currentstroke}{rgb}{0.500000,0.500000,0.500000}%
\pgfsetstrokecolor{currentstroke}%
\pgfsetstrokeopacity{0.300000}%
\pgfsetdash{}{0pt}%
\pgfpathmoveto{\pgfqpoint{0.937514in}{0.853918in}}%
\pgfusepath{stroke}%
\end{pgfscope}%
\begin{pgfscope}%
\pgfpathrectangle{\pgfqpoint{0.647939in}{0.492442in}}{\pgfqpoint{3.079299in}{3.079299in}}%
\pgfusepath{clip}%
\pgfsetroundcap%
\pgfsetroundjoin%
\definecolor{currentfill}{rgb}{0.500000,0.500000,0.500000}%
\pgfsetfillcolor{currentfill}%
\pgfsetfillopacity{0.300000}%
\pgfsetlinewidth{0.301125pt}%
\definecolor{currentstroke}{rgb}{0.500000,0.500000,0.500000}%
\pgfsetstrokecolor{currentstroke}%
\pgfsetstrokeopacity{0.300000}%
\pgfsetdash{}{0pt}%
\pgfpathmoveto{\pgfqpoint{0.000000in}{0.000000in}}%
\pgfpathlineto{\pgfqpoint{0.000000in}{0.000000in}}%
\pgfpathclose%
\pgfusepath{stroke,fill}%
\end{pgfscope}%
\begin{pgfscope}%
\pgfpathrectangle{\pgfqpoint{0.647939in}{0.492442in}}{\pgfqpoint{3.079299in}{3.079299in}}%
\pgfusepath{clip}%
\pgfsetroundcap%
\pgfsetroundjoin%
\pgfsetlinewidth{0.301125pt}%
\definecolor{currentstroke}{rgb}{0.500000,0.500000,0.500000}%
\pgfsetstrokecolor{currentstroke}%
\pgfsetstrokeopacity{0.300000}%
\pgfsetdash{}{0pt}%
\pgfpathmoveto{\pgfqpoint{0.936885in}{0.785939in}}%
\pgfusepath{stroke}%
\end{pgfscope}%
\begin{pgfscope}%
\pgfpathrectangle{\pgfqpoint{0.647939in}{0.492442in}}{\pgfqpoint{3.079299in}{3.079299in}}%
\pgfusepath{clip}%
\pgfsetroundcap%
\pgfsetroundjoin%
\definecolor{currentfill}{rgb}{0.500000,0.500000,0.500000}%
\pgfsetfillcolor{currentfill}%
\pgfsetfillopacity{0.300000}%
\pgfsetlinewidth{0.301125pt}%
\definecolor{currentstroke}{rgb}{0.500000,0.500000,0.500000}%
\pgfsetstrokecolor{currentstroke}%
\pgfsetstrokeopacity{0.300000}%
\pgfsetdash{}{0pt}%
\pgfpathmoveto{\pgfqpoint{0.000000in}{0.000000in}}%
\pgfpathlineto{\pgfqpoint{0.000000in}{0.000000in}}%
\pgfpathclose%
\pgfusepath{stroke,fill}%
\end{pgfscope}%
\begin{pgfscope}%
\pgfpathrectangle{\pgfqpoint{0.647939in}{0.492442in}}{\pgfqpoint{3.079299in}{3.079299in}}%
\pgfusepath{clip}%
\pgfsetroundcap%
\pgfsetroundjoin%
\pgfsetlinewidth{0.301125pt}%
\definecolor{currentstroke}{rgb}{0.500000,0.500000,0.500000}%
\pgfsetstrokecolor{currentstroke}%
\pgfsetstrokeopacity{0.300000}%
\pgfsetdash{}{0pt}%
\pgfpathmoveto{\pgfqpoint{0.872630in}{0.692927in}}%
\pgfusepath{stroke}%
\end{pgfscope}%
\begin{pgfscope}%
\pgfpathrectangle{\pgfqpoint{0.647939in}{0.492442in}}{\pgfqpoint{3.079299in}{3.079299in}}%
\pgfusepath{clip}%
\pgfsetroundcap%
\pgfsetroundjoin%
\definecolor{currentfill}{rgb}{0.500000,0.500000,0.500000}%
\pgfsetfillcolor{currentfill}%
\pgfsetfillopacity{0.300000}%
\pgfsetlinewidth{0.301125pt}%
\definecolor{currentstroke}{rgb}{0.500000,0.500000,0.500000}%
\pgfsetstrokecolor{currentstroke}%
\pgfsetstrokeopacity{0.300000}%
\pgfsetdash{}{0pt}%
\pgfpathmoveto{\pgfqpoint{0.000000in}{0.000000in}}%
\pgfpathlineto{\pgfqpoint{0.000000in}{0.000000in}}%
\pgfpathclose%
\pgfusepath{stroke,fill}%
\end{pgfscope}%
\begin{pgfscope}%
\pgfpathrectangle{\pgfqpoint{0.647939in}{0.492442in}}{\pgfqpoint{3.079299in}{3.079299in}}%
\pgfusepath{clip}%
\pgfsetroundcap%
\pgfsetroundjoin%
\pgfsetlinewidth{0.301125pt}%
\definecolor{currentstroke}{rgb}{0.500000,0.500000,0.500000}%
\pgfsetstrokecolor{currentstroke}%
\pgfsetstrokeopacity{0.300000}%
\pgfsetdash{}{0pt}%
\pgfpathmoveto{\pgfqpoint{3.390071in}{3.378033in}}%
\pgfusepath{stroke}%
\end{pgfscope}%
\begin{pgfscope}%
\pgfpathrectangle{\pgfqpoint{0.647939in}{0.492442in}}{\pgfqpoint{3.079299in}{3.079299in}}%
\pgfusepath{clip}%
\pgfsetroundcap%
\pgfsetroundjoin%
\definecolor{currentfill}{rgb}{0.500000,0.500000,0.500000}%
\pgfsetfillcolor{currentfill}%
\pgfsetfillopacity{0.300000}%
\pgfsetlinewidth{0.301125pt}%
\definecolor{currentstroke}{rgb}{0.500000,0.500000,0.500000}%
\pgfsetstrokecolor{currentstroke}%
\pgfsetstrokeopacity{0.300000}%
\pgfsetdash{}{0pt}%
\pgfpathmoveto{\pgfqpoint{0.000000in}{0.000000in}}%
\pgfpathlineto{\pgfqpoint{0.000000in}{0.000000in}}%
\pgfpathclose%
\pgfusepath{stroke,fill}%
\end{pgfscope}%
\begin{pgfscope}%
\pgfpathrectangle{\pgfqpoint{0.647939in}{0.492442in}}{\pgfqpoint{3.079299in}{3.079299in}}%
\pgfusepath{clip}%
\pgfsetroundcap%
\pgfsetroundjoin%
\pgfsetlinewidth{0.301125pt}%
\definecolor{currentstroke}{rgb}{0.500000,0.500000,0.500000}%
\pgfsetstrokecolor{currentstroke}%
\pgfsetstrokeopacity{0.300000}%
\pgfsetdash{}{0pt}%
\pgfpathmoveto{\pgfqpoint{1.741535in}{0.641758in}}%
\pgfusepath{stroke}%
\end{pgfscope}%
\begin{pgfscope}%
\pgfpathrectangle{\pgfqpoint{0.647939in}{0.492442in}}{\pgfqpoint{3.079299in}{3.079299in}}%
\pgfusepath{clip}%
\pgfsetroundcap%
\pgfsetroundjoin%
\definecolor{currentfill}{rgb}{0.500000,0.500000,0.500000}%
\pgfsetfillcolor{currentfill}%
\pgfsetfillopacity{0.300000}%
\pgfsetlinewidth{0.301125pt}%
\definecolor{currentstroke}{rgb}{0.500000,0.500000,0.500000}%
\pgfsetstrokecolor{currentstroke}%
\pgfsetstrokeopacity{0.300000}%
\pgfsetdash{}{0pt}%
\pgfpathmoveto{\pgfqpoint{0.000000in}{0.000000in}}%
\pgfpathlineto{\pgfqpoint{0.000000in}{0.000000in}}%
\pgfpathclose%
\pgfusepath{stroke,fill}%
\end{pgfscope}%
\begin{pgfscope}%
\pgfpathrectangle{\pgfqpoint{0.647939in}{0.492442in}}{\pgfqpoint{3.079299in}{3.079299in}}%
\pgfusepath{clip}%
\pgfsetroundcap%
\pgfsetroundjoin%
\pgfsetlinewidth{0.301125pt}%
\definecolor{currentstroke}{rgb}{0.500000,0.500000,0.500000}%
\pgfsetstrokecolor{currentstroke}%
\pgfsetstrokeopacity{0.300000}%
\pgfsetdash{}{0pt}%
\pgfpathmoveto{\pgfqpoint{3.449342in}{1.492006in}}%
\pgfusepath{stroke}%
\end{pgfscope}%
\begin{pgfscope}%
\pgfpathrectangle{\pgfqpoint{0.647939in}{0.492442in}}{\pgfqpoint{3.079299in}{3.079299in}}%
\pgfusepath{clip}%
\pgfsetroundcap%
\pgfsetroundjoin%
\definecolor{currentfill}{rgb}{0.500000,0.500000,0.500000}%
\pgfsetfillcolor{currentfill}%
\pgfsetfillopacity{0.300000}%
\pgfsetlinewidth{0.301125pt}%
\definecolor{currentstroke}{rgb}{0.500000,0.500000,0.500000}%
\pgfsetstrokecolor{currentstroke}%
\pgfsetstrokeopacity{0.300000}%
\pgfsetdash{}{0pt}%
\pgfpathmoveto{\pgfqpoint{0.000000in}{0.000000in}}%
\pgfpathlineto{\pgfqpoint{0.000000in}{0.000000in}}%
\pgfpathclose%
\pgfusepath{stroke,fill}%
\end{pgfscope}%
\begin{pgfscope}%
\pgfpathrectangle{\pgfqpoint{0.647939in}{0.492442in}}{\pgfqpoint{3.079299in}{3.079299in}}%
\pgfusepath{clip}%
\pgfsetroundcap%
\pgfsetroundjoin%
\pgfsetlinewidth{0.301125pt}%
\definecolor{currentstroke}{rgb}{0.500000,0.500000,0.500000}%
\pgfsetstrokecolor{currentstroke}%
\pgfsetstrokeopacity{0.300000}%
\pgfsetdash{}{0pt}%
\pgfpathmoveto{\pgfqpoint{2.295073in}{0.698174in}}%
\pgfusepath{stroke}%
\end{pgfscope}%
\begin{pgfscope}%
\pgfpathrectangle{\pgfqpoint{0.647939in}{0.492442in}}{\pgfqpoint{3.079299in}{3.079299in}}%
\pgfusepath{clip}%
\pgfsetroundcap%
\pgfsetroundjoin%
\definecolor{currentfill}{rgb}{0.500000,0.500000,0.500000}%
\pgfsetfillcolor{currentfill}%
\pgfsetfillopacity{0.300000}%
\pgfsetlinewidth{0.301125pt}%
\definecolor{currentstroke}{rgb}{0.500000,0.500000,0.500000}%
\pgfsetstrokecolor{currentstroke}%
\pgfsetstrokeopacity{0.300000}%
\pgfsetdash{}{0pt}%
\pgfpathmoveto{\pgfqpoint{0.000000in}{0.000000in}}%
\pgfpathlineto{\pgfqpoint{0.000000in}{0.000000in}}%
\pgfpathclose%
\pgfusepath{stroke,fill}%
\end{pgfscope}%
\begin{pgfscope}%
\pgfpathrectangle{\pgfqpoint{0.647939in}{0.492442in}}{\pgfqpoint{3.079299in}{3.079299in}}%
\pgfusepath{clip}%
\pgfsetroundcap%
\pgfsetroundjoin%
\pgfsetlinewidth{0.301125pt}%
\definecolor{currentstroke}{rgb}{0.500000,0.500000,0.500000}%
\pgfsetstrokecolor{currentstroke}%
\pgfsetstrokeopacity{0.300000}%
\pgfsetdash{}{0pt}%
\pgfpathmoveto{\pgfqpoint{3.020079in}{0.786631in}}%
\pgfusepath{stroke}%
\end{pgfscope}%
\begin{pgfscope}%
\pgfpathrectangle{\pgfqpoint{0.647939in}{0.492442in}}{\pgfqpoint{3.079299in}{3.079299in}}%
\pgfusepath{clip}%
\pgfsetroundcap%
\pgfsetroundjoin%
\definecolor{currentfill}{rgb}{0.500000,0.500000,0.500000}%
\pgfsetfillcolor{currentfill}%
\pgfsetfillopacity{0.300000}%
\pgfsetlinewidth{0.301125pt}%
\definecolor{currentstroke}{rgb}{0.500000,0.500000,0.500000}%
\pgfsetstrokecolor{currentstroke}%
\pgfsetstrokeopacity{0.300000}%
\pgfsetdash{}{0pt}%
\pgfpathmoveto{\pgfqpoint{0.000000in}{0.000000in}}%
\pgfpathlineto{\pgfqpoint{0.000000in}{0.000000in}}%
\pgfpathclose%
\pgfusepath{stroke,fill}%
\end{pgfscope}%
\begin{pgfscope}%
\pgfpathrectangle{\pgfqpoint{0.647939in}{0.492442in}}{\pgfqpoint{3.079299in}{3.079299in}}%
\pgfusepath{clip}%
\pgfsetroundcap%
\pgfsetroundjoin%
\pgfsetlinewidth{0.301125pt}%
\definecolor{currentstroke}{rgb}{0.500000,0.500000,0.500000}%
\pgfsetstrokecolor{currentstroke}%
\pgfsetstrokeopacity{0.300000}%
\pgfsetdash{}{0pt}%
\pgfpathmoveto{\pgfqpoint{3.042415in}{2.290411in}}%
\pgfusepath{stroke}%
\end{pgfscope}%
\begin{pgfscope}%
\pgfpathrectangle{\pgfqpoint{0.647939in}{0.492442in}}{\pgfqpoint{3.079299in}{3.079299in}}%
\pgfusepath{clip}%
\pgfsetroundcap%
\pgfsetroundjoin%
\definecolor{currentfill}{rgb}{0.500000,0.500000,0.500000}%
\pgfsetfillcolor{currentfill}%
\pgfsetfillopacity{0.300000}%
\pgfsetlinewidth{0.301125pt}%
\definecolor{currentstroke}{rgb}{0.500000,0.500000,0.500000}%
\pgfsetstrokecolor{currentstroke}%
\pgfsetstrokeopacity{0.300000}%
\pgfsetdash{}{0pt}%
\pgfpathmoveto{\pgfqpoint{0.000000in}{0.000000in}}%
\pgfpathlineto{\pgfqpoint{0.000000in}{0.000000in}}%
\pgfpathclose%
\pgfusepath{stroke,fill}%
\end{pgfscope}%
\begin{pgfscope}%
\pgfpathrectangle{\pgfqpoint{0.647939in}{0.492442in}}{\pgfqpoint{3.079299in}{3.079299in}}%
\pgfusepath{clip}%
\pgfsetroundcap%
\pgfsetroundjoin%
\pgfsetlinewidth{0.301125pt}%
\definecolor{currentstroke}{rgb}{0.500000,0.500000,0.500000}%
\pgfsetstrokecolor{currentstroke}%
\pgfsetstrokeopacity{0.300000}%
\pgfsetdash{}{0pt}%
\pgfpathmoveto{\pgfqpoint{3.461511in}{3.058181in}}%
\pgfusepath{stroke}%
\end{pgfscope}%
\begin{pgfscope}%
\pgfpathrectangle{\pgfqpoint{0.647939in}{0.492442in}}{\pgfqpoint{3.079299in}{3.079299in}}%
\pgfusepath{clip}%
\pgfsetroundcap%
\pgfsetroundjoin%
\definecolor{currentfill}{rgb}{0.500000,0.500000,0.500000}%
\pgfsetfillcolor{currentfill}%
\pgfsetfillopacity{0.300000}%
\pgfsetlinewidth{0.301125pt}%
\definecolor{currentstroke}{rgb}{0.500000,0.500000,0.500000}%
\pgfsetstrokecolor{currentstroke}%
\pgfsetstrokeopacity{0.300000}%
\pgfsetdash{}{0pt}%
\pgfpathmoveto{\pgfqpoint{0.000000in}{0.000000in}}%
\pgfpathlineto{\pgfqpoint{0.000000in}{0.000000in}}%
\pgfpathclose%
\pgfusepath{stroke,fill}%
\end{pgfscope}%
\begin{pgfscope}%
\pgfpathrectangle{\pgfqpoint{0.647939in}{0.492442in}}{\pgfqpoint{3.079299in}{3.079299in}}%
\pgfusepath{clip}%
\pgfsetroundcap%
\pgfsetroundjoin%
\pgfsetlinewidth{0.301125pt}%
\definecolor{currentstroke}{rgb}{0.500000,0.500000,0.500000}%
\pgfsetstrokecolor{currentstroke}%
\pgfsetstrokeopacity{0.300000}%
\pgfsetdash{}{0pt}%
\pgfpathmoveto{\pgfqpoint{2.001335in}{3.145655in}}%
\pgfusepath{stroke}%
\end{pgfscope}%
\begin{pgfscope}%
\pgfpathrectangle{\pgfqpoint{0.647939in}{0.492442in}}{\pgfqpoint{3.079299in}{3.079299in}}%
\pgfusepath{clip}%
\pgfsetroundcap%
\pgfsetroundjoin%
\definecolor{currentfill}{rgb}{0.500000,0.500000,0.500000}%
\pgfsetfillcolor{currentfill}%
\pgfsetfillopacity{0.300000}%
\pgfsetlinewidth{0.301125pt}%
\definecolor{currentstroke}{rgb}{0.500000,0.500000,0.500000}%
\pgfsetstrokecolor{currentstroke}%
\pgfsetstrokeopacity{0.300000}%
\pgfsetdash{}{0pt}%
\pgfpathmoveto{\pgfqpoint{0.000000in}{0.000000in}}%
\pgfpathlineto{\pgfqpoint{0.000000in}{0.000000in}}%
\pgfpathclose%
\pgfusepath{stroke,fill}%
\end{pgfscope}%
\begin{pgfscope}%
\pgfpathrectangle{\pgfqpoint{0.647939in}{0.492442in}}{\pgfqpoint{3.079299in}{3.079299in}}%
\pgfusepath{clip}%
\pgfsetroundcap%
\pgfsetroundjoin%
\pgfsetlinewidth{0.301125pt}%
\definecolor{currentstroke}{rgb}{0.500000,0.500000,0.500000}%
\pgfsetstrokecolor{currentstroke}%
\pgfsetstrokeopacity{0.300000}%
\pgfsetdash{}{0pt}%
\pgfpathmoveto{\pgfqpoint{2.858089in}{1.731832in}}%
\pgfusepath{stroke}%
\end{pgfscope}%
\begin{pgfscope}%
\pgfpathrectangle{\pgfqpoint{0.647939in}{0.492442in}}{\pgfqpoint{3.079299in}{3.079299in}}%
\pgfusepath{clip}%
\pgfsetroundcap%
\pgfsetroundjoin%
\definecolor{currentfill}{rgb}{0.500000,0.500000,0.500000}%
\pgfsetfillcolor{currentfill}%
\pgfsetfillopacity{0.300000}%
\pgfsetlinewidth{0.301125pt}%
\definecolor{currentstroke}{rgb}{0.500000,0.500000,0.500000}%
\pgfsetstrokecolor{currentstroke}%
\pgfsetstrokeopacity{0.300000}%
\pgfsetdash{}{0pt}%
\pgfpathmoveto{\pgfqpoint{0.000000in}{0.000000in}}%
\pgfpathlineto{\pgfqpoint{0.000000in}{0.000000in}}%
\pgfpathclose%
\pgfusepath{stroke,fill}%
\end{pgfscope}%
\begin{pgfscope}%
\pgfpathrectangle{\pgfqpoint{0.647939in}{0.492442in}}{\pgfqpoint{3.079299in}{3.079299in}}%
\pgfusepath{clip}%
\pgfsetroundcap%
\pgfsetroundjoin%
\pgfsetlinewidth{0.301125pt}%
\definecolor{currentstroke}{rgb}{0.500000,0.500000,0.500000}%
\pgfsetstrokecolor{currentstroke}%
\pgfsetstrokeopacity{0.300000}%
\pgfsetdash{}{0pt}%
\pgfpathmoveto{\pgfqpoint{3.389720in}{2.395919in}}%
\pgfusepath{stroke}%
\end{pgfscope}%
\begin{pgfscope}%
\pgfpathrectangle{\pgfqpoint{0.647939in}{0.492442in}}{\pgfqpoint{3.079299in}{3.079299in}}%
\pgfusepath{clip}%
\pgfsetroundcap%
\pgfsetroundjoin%
\definecolor{currentfill}{rgb}{0.500000,0.500000,0.500000}%
\pgfsetfillcolor{currentfill}%
\pgfsetfillopacity{0.300000}%
\pgfsetlinewidth{0.301125pt}%
\definecolor{currentstroke}{rgb}{0.500000,0.500000,0.500000}%
\pgfsetstrokecolor{currentstroke}%
\pgfsetstrokeopacity{0.300000}%
\pgfsetdash{}{0pt}%
\pgfpathmoveto{\pgfqpoint{0.000000in}{0.000000in}}%
\pgfpathlineto{\pgfqpoint{0.000000in}{0.000000in}}%
\pgfpathclose%
\pgfusepath{stroke,fill}%
\end{pgfscope}%
\begin{pgfscope}%
\pgfpathrectangle{\pgfqpoint{0.647939in}{0.492442in}}{\pgfqpoint{3.079299in}{3.079299in}}%
\pgfusepath{clip}%
\pgfsetroundcap%
\pgfsetroundjoin%
\pgfsetlinewidth{0.301125pt}%
\definecolor{currentstroke}{rgb}{0.500000,0.500000,0.500000}%
\pgfsetstrokecolor{currentstroke}%
\pgfsetstrokeopacity{0.300000}%
\pgfsetdash{}{0pt}%
\pgfpathmoveto{\pgfqpoint{2.153694in}{3.276051in}}%
\pgfusepath{stroke}%
\end{pgfscope}%
\begin{pgfscope}%
\pgfpathrectangle{\pgfqpoint{0.647939in}{0.492442in}}{\pgfqpoint{3.079299in}{3.079299in}}%
\pgfusepath{clip}%
\pgfsetroundcap%
\pgfsetroundjoin%
\definecolor{currentfill}{rgb}{0.500000,0.500000,0.500000}%
\pgfsetfillcolor{currentfill}%
\pgfsetfillopacity{0.300000}%
\pgfsetlinewidth{0.301125pt}%
\definecolor{currentstroke}{rgb}{0.500000,0.500000,0.500000}%
\pgfsetstrokecolor{currentstroke}%
\pgfsetstrokeopacity{0.300000}%
\pgfsetdash{}{0pt}%
\pgfpathmoveto{\pgfqpoint{0.000000in}{0.000000in}}%
\pgfpathlineto{\pgfqpoint{0.000000in}{0.000000in}}%
\pgfpathclose%
\pgfusepath{stroke,fill}%
\end{pgfscope}%
\begin{pgfscope}%
\pgfpathrectangle{\pgfqpoint{0.647939in}{0.492442in}}{\pgfqpoint{3.079299in}{3.079299in}}%
\pgfusepath{clip}%
\pgfsetroundcap%
\pgfsetroundjoin%
\pgfsetlinewidth{0.301125pt}%
\definecolor{currentstroke}{rgb}{0.500000,0.500000,0.500000}%
\pgfsetstrokecolor{currentstroke}%
\pgfsetstrokeopacity{0.300000}%
\pgfsetdash{}{0pt}%
\pgfpathmoveto{\pgfqpoint{3.233155in}{2.295633in}}%
\pgfusepath{stroke}%
\end{pgfscope}%
\begin{pgfscope}%
\pgfpathrectangle{\pgfqpoint{0.647939in}{0.492442in}}{\pgfqpoint{3.079299in}{3.079299in}}%
\pgfusepath{clip}%
\pgfsetroundcap%
\pgfsetroundjoin%
\definecolor{currentfill}{rgb}{0.500000,0.500000,0.500000}%
\pgfsetfillcolor{currentfill}%
\pgfsetfillopacity{0.300000}%
\pgfsetlinewidth{0.301125pt}%
\definecolor{currentstroke}{rgb}{0.500000,0.500000,0.500000}%
\pgfsetstrokecolor{currentstroke}%
\pgfsetstrokeopacity{0.300000}%
\pgfsetdash{}{0pt}%
\pgfpathmoveto{\pgfqpoint{0.000000in}{0.000000in}}%
\pgfpathlineto{\pgfqpoint{0.000000in}{0.000000in}}%
\pgfpathclose%
\pgfusepath{stroke,fill}%
\end{pgfscope}%
\begin{pgfscope}%
\pgfpathrectangle{\pgfqpoint{0.647939in}{0.492442in}}{\pgfqpoint{3.079299in}{3.079299in}}%
\pgfusepath{clip}%
\pgfsetroundcap%
\pgfsetroundjoin%
\pgfsetlinewidth{0.301125pt}%
\definecolor{currentstroke}{rgb}{0.500000,0.500000,0.500000}%
\pgfsetstrokecolor{currentstroke}%
\pgfsetstrokeopacity{0.300000}%
\pgfsetdash{}{0pt}%
\pgfpathmoveto{\pgfqpoint{2.979067in}{3.239194in}}%
\pgfusepath{stroke}%
\end{pgfscope}%
\begin{pgfscope}%
\pgfpathrectangle{\pgfqpoint{0.647939in}{0.492442in}}{\pgfqpoint{3.079299in}{3.079299in}}%
\pgfusepath{clip}%
\pgfsetroundcap%
\pgfsetroundjoin%
\definecolor{currentfill}{rgb}{0.500000,0.500000,0.500000}%
\pgfsetfillcolor{currentfill}%
\pgfsetfillopacity{0.300000}%
\pgfsetlinewidth{0.301125pt}%
\definecolor{currentstroke}{rgb}{0.500000,0.500000,0.500000}%
\pgfsetstrokecolor{currentstroke}%
\pgfsetstrokeopacity{0.300000}%
\pgfsetdash{}{0pt}%
\pgfpathmoveto{\pgfqpoint{0.000000in}{0.000000in}}%
\pgfpathlineto{\pgfqpoint{0.000000in}{0.000000in}}%
\pgfpathclose%
\pgfusepath{stroke,fill}%
\end{pgfscope}%
\begin{pgfscope}%
\pgfpathrectangle{\pgfqpoint{0.647939in}{0.492442in}}{\pgfqpoint{3.079299in}{3.079299in}}%
\pgfusepath{clip}%
\pgfsetroundcap%
\pgfsetroundjoin%
\pgfsetlinewidth{0.301125pt}%
\definecolor{currentstroke}{rgb}{0.500000,0.500000,0.500000}%
\pgfsetstrokecolor{currentstroke}%
\pgfsetstrokeopacity{0.300000}%
\pgfsetdash{}{0pt}%
\pgfpathmoveto{\pgfqpoint{1.314628in}{3.182559in}}%
\pgfusepath{stroke}%
\end{pgfscope}%
\begin{pgfscope}%
\pgfpathrectangle{\pgfqpoint{0.647939in}{0.492442in}}{\pgfqpoint{3.079299in}{3.079299in}}%
\pgfusepath{clip}%
\pgfsetroundcap%
\pgfsetroundjoin%
\definecolor{currentfill}{rgb}{0.500000,0.500000,0.500000}%
\pgfsetfillcolor{currentfill}%
\pgfsetfillopacity{0.300000}%
\pgfsetlinewidth{0.301125pt}%
\definecolor{currentstroke}{rgb}{0.500000,0.500000,0.500000}%
\pgfsetstrokecolor{currentstroke}%
\pgfsetstrokeopacity{0.300000}%
\pgfsetdash{}{0pt}%
\pgfpathmoveto{\pgfqpoint{0.000000in}{0.000000in}}%
\pgfpathlineto{\pgfqpoint{0.000000in}{0.000000in}}%
\pgfpathclose%
\pgfusepath{stroke,fill}%
\end{pgfscope}%
\begin{pgfscope}%
\pgfpathrectangle{\pgfqpoint{0.647939in}{0.492442in}}{\pgfqpoint{3.079299in}{3.079299in}}%
\pgfusepath{clip}%
\pgfsetroundcap%
\pgfsetroundjoin%
\pgfsetlinewidth{0.301125pt}%
\definecolor{currentstroke}{rgb}{0.500000,0.500000,0.500000}%
\pgfsetstrokecolor{currentstroke}%
\pgfsetstrokeopacity{0.300000}%
\pgfsetdash{}{0pt}%
\pgfpathmoveto{\pgfqpoint{2.012815in}{0.886028in}}%
\pgfusepath{stroke}%
\end{pgfscope}%
\begin{pgfscope}%
\pgfpathrectangle{\pgfqpoint{0.647939in}{0.492442in}}{\pgfqpoint{3.079299in}{3.079299in}}%
\pgfusepath{clip}%
\pgfsetroundcap%
\pgfsetroundjoin%
\definecolor{currentfill}{rgb}{0.500000,0.500000,0.500000}%
\pgfsetfillcolor{currentfill}%
\pgfsetfillopacity{0.300000}%
\pgfsetlinewidth{0.301125pt}%
\definecolor{currentstroke}{rgb}{0.500000,0.500000,0.500000}%
\pgfsetstrokecolor{currentstroke}%
\pgfsetstrokeopacity{0.300000}%
\pgfsetdash{}{0pt}%
\pgfpathmoveto{\pgfqpoint{0.000000in}{0.000000in}}%
\pgfpathlineto{\pgfqpoint{0.000000in}{0.000000in}}%
\pgfpathclose%
\pgfusepath{stroke,fill}%
\end{pgfscope}%
\begin{pgfscope}%
\pgfpathrectangle{\pgfqpoint{0.647939in}{0.492442in}}{\pgfqpoint{3.079299in}{3.079299in}}%
\pgfusepath{clip}%
\pgfsetroundcap%
\pgfsetroundjoin%
\pgfsetlinewidth{0.301125pt}%
\definecolor{currentstroke}{rgb}{0.500000,0.500000,0.500000}%
\pgfsetstrokecolor{currentstroke}%
\pgfsetstrokeopacity{0.300000}%
\pgfsetdash{}{0pt}%
\pgfpathmoveto{\pgfqpoint{2.434834in}{0.903363in}}%
\pgfusepath{stroke}%
\end{pgfscope}%
\begin{pgfscope}%
\pgfpathrectangle{\pgfqpoint{0.647939in}{0.492442in}}{\pgfqpoint{3.079299in}{3.079299in}}%
\pgfusepath{clip}%
\pgfsetroundcap%
\pgfsetroundjoin%
\definecolor{currentfill}{rgb}{0.500000,0.500000,0.500000}%
\pgfsetfillcolor{currentfill}%
\pgfsetfillopacity{0.300000}%
\pgfsetlinewidth{0.301125pt}%
\definecolor{currentstroke}{rgb}{0.500000,0.500000,0.500000}%
\pgfsetstrokecolor{currentstroke}%
\pgfsetstrokeopacity{0.300000}%
\pgfsetdash{}{0pt}%
\pgfpathmoveto{\pgfqpoint{0.000000in}{0.000000in}}%
\pgfpathlineto{\pgfqpoint{0.000000in}{0.000000in}}%
\pgfpathclose%
\pgfusepath{stroke,fill}%
\end{pgfscope}%
\begin{pgfscope}%
\pgfpathrectangle{\pgfqpoint{0.647939in}{0.492442in}}{\pgfqpoint{3.079299in}{3.079299in}}%
\pgfusepath{clip}%
\pgfsetroundcap%
\pgfsetroundjoin%
\pgfsetlinewidth{0.301125pt}%
\definecolor{currentstroke}{rgb}{0.500000,0.500000,0.500000}%
\pgfsetstrokecolor{currentstroke}%
\pgfsetstrokeopacity{0.300000}%
\pgfsetdash{}{0pt}%
\pgfpathmoveto{\pgfqpoint{2.831115in}{2.088079in}}%
\pgfusepath{stroke}%
\end{pgfscope}%
\begin{pgfscope}%
\pgfpathrectangle{\pgfqpoint{0.647939in}{0.492442in}}{\pgfqpoint{3.079299in}{3.079299in}}%
\pgfusepath{clip}%
\pgfsetroundcap%
\pgfsetroundjoin%
\definecolor{currentfill}{rgb}{0.500000,0.500000,0.500000}%
\pgfsetfillcolor{currentfill}%
\pgfsetfillopacity{0.300000}%
\pgfsetlinewidth{0.301125pt}%
\definecolor{currentstroke}{rgb}{0.500000,0.500000,0.500000}%
\pgfsetstrokecolor{currentstroke}%
\pgfsetstrokeopacity{0.300000}%
\pgfsetdash{}{0pt}%
\pgfpathmoveto{\pgfqpoint{0.000000in}{0.000000in}}%
\pgfpathlineto{\pgfqpoint{0.000000in}{0.000000in}}%
\pgfpathclose%
\pgfusepath{stroke,fill}%
\end{pgfscope}%
\begin{pgfscope}%
\pgfpathrectangle{\pgfqpoint{0.647939in}{0.492442in}}{\pgfqpoint{3.079299in}{3.079299in}}%
\pgfusepath{clip}%
\pgfsetroundcap%
\pgfsetroundjoin%
\pgfsetlinewidth{0.301125pt}%
\definecolor{currentstroke}{rgb}{0.500000,0.500000,0.500000}%
\pgfsetstrokecolor{currentstroke}%
\pgfsetstrokeopacity{0.300000}%
\pgfsetdash{}{0pt}%
\pgfpathmoveto{\pgfqpoint{2.246518in}{2.966738in}}%
\pgfusepath{stroke}%
\end{pgfscope}%
\begin{pgfscope}%
\pgfpathrectangle{\pgfqpoint{0.647939in}{0.492442in}}{\pgfqpoint{3.079299in}{3.079299in}}%
\pgfusepath{clip}%
\pgfsetroundcap%
\pgfsetroundjoin%
\definecolor{currentfill}{rgb}{0.500000,0.500000,0.500000}%
\pgfsetfillcolor{currentfill}%
\pgfsetfillopacity{0.300000}%
\pgfsetlinewidth{0.301125pt}%
\definecolor{currentstroke}{rgb}{0.500000,0.500000,0.500000}%
\pgfsetstrokecolor{currentstroke}%
\pgfsetstrokeopacity{0.300000}%
\pgfsetdash{}{0pt}%
\pgfpathmoveto{\pgfqpoint{0.000000in}{0.000000in}}%
\pgfpathlineto{\pgfqpoint{0.000000in}{0.000000in}}%
\pgfpathclose%
\pgfusepath{stroke,fill}%
\end{pgfscope}%
\begin{pgfscope}%
\pgfpathrectangle{\pgfqpoint{0.647939in}{0.492442in}}{\pgfqpoint{3.079299in}{3.079299in}}%
\pgfusepath{clip}%
\pgfsetroundcap%
\pgfsetroundjoin%
\pgfsetlinewidth{0.301125pt}%
\definecolor{currentstroke}{rgb}{0.500000,0.500000,0.500000}%
\pgfsetstrokecolor{currentstroke}%
\pgfsetstrokeopacity{0.300000}%
\pgfsetdash{}{0pt}%
\pgfpathmoveto{\pgfqpoint{2.954020in}{2.340825in}}%
\pgfusepath{stroke}%
\end{pgfscope}%
\begin{pgfscope}%
\pgfpathrectangle{\pgfqpoint{0.647939in}{0.492442in}}{\pgfqpoint{3.079299in}{3.079299in}}%
\pgfusepath{clip}%
\pgfsetroundcap%
\pgfsetroundjoin%
\definecolor{currentfill}{rgb}{0.500000,0.500000,0.500000}%
\pgfsetfillcolor{currentfill}%
\pgfsetfillopacity{0.300000}%
\pgfsetlinewidth{0.301125pt}%
\definecolor{currentstroke}{rgb}{0.500000,0.500000,0.500000}%
\pgfsetstrokecolor{currentstroke}%
\pgfsetstrokeopacity{0.300000}%
\pgfsetdash{}{0pt}%
\pgfpathmoveto{\pgfqpoint{0.000000in}{0.000000in}}%
\pgfpathlineto{\pgfqpoint{0.000000in}{0.000000in}}%
\pgfpathclose%
\pgfusepath{stroke,fill}%
\end{pgfscope}%
\begin{pgfscope}%
\pgfpathrectangle{\pgfqpoint{0.647939in}{0.492442in}}{\pgfqpoint{3.079299in}{3.079299in}}%
\pgfusepath{clip}%
\pgfsetroundcap%
\pgfsetroundjoin%
\pgfsetlinewidth{0.301125pt}%
\definecolor{currentstroke}{rgb}{0.500000,0.500000,0.500000}%
\pgfsetstrokecolor{currentstroke}%
\pgfsetstrokeopacity{0.300000}%
\pgfsetdash{}{0pt}%
\pgfpathmoveto{\pgfqpoint{2.150162in}{3.005489in}}%
\pgfusepath{stroke}%
\end{pgfscope}%
\begin{pgfscope}%
\pgfpathrectangle{\pgfqpoint{0.647939in}{0.492442in}}{\pgfqpoint{3.079299in}{3.079299in}}%
\pgfusepath{clip}%
\pgfsetroundcap%
\pgfsetroundjoin%
\definecolor{currentfill}{rgb}{0.500000,0.500000,0.500000}%
\pgfsetfillcolor{currentfill}%
\pgfsetfillopacity{0.300000}%
\pgfsetlinewidth{0.301125pt}%
\definecolor{currentstroke}{rgb}{0.500000,0.500000,0.500000}%
\pgfsetstrokecolor{currentstroke}%
\pgfsetstrokeopacity{0.300000}%
\pgfsetdash{}{0pt}%
\pgfpathmoveto{\pgfqpoint{0.000000in}{0.000000in}}%
\pgfpathlineto{\pgfqpoint{0.000000in}{0.000000in}}%
\pgfpathclose%
\pgfusepath{stroke,fill}%
\end{pgfscope}%
\begin{pgfscope}%
\pgfpathrectangle{\pgfqpoint{0.647939in}{0.492442in}}{\pgfqpoint{3.079299in}{3.079299in}}%
\pgfusepath{clip}%
\pgfsetroundcap%
\pgfsetroundjoin%
\pgfsetlinewidth{0.301125pt}%
\definecolor{currentstroke}{rgb}{0.500000,0.500000,0.500000}%
\pgfsetstrokecolor{currentstroke}%
\pgfsetstrokeopacity{0.300000}%
\pgfsetdash{}{0pt}%
\pgfpathmoveto{\pgfqpoint{2.224969in}{1.114924in}}%
\pgfusepath{stroke}%
\end{pgfscope}%
\begin{pgfscope}%
\pgfpathrectangle{\pgfqpoint{0.647939in}{0.492442in}}{\pgfqpoint{3.079299in}{3.079299in}}%
\pgfusepath{clip}%
\pgfsetroundcap%
\pgfsetroundjoin%
\definecolor{currentfill}{rgb}{0.500000,0.500000,0.500000}%
\pgfsetfillcolor{currentfill}%
\pgfsetfillopacity{0.300000}%
\pgfsetlinewidth{0.301125pt}%
\definecolor{currentstroke}{rgb}{0.500000,0.500000,0.500000}%
\pgfsetstrokecolor{currentstroke}%
\pgfsetstrokeopacity{0.300000}%
\pgfsetdash{}{0pt}%
\pgfpathmoveto{\pgfqpoint{0.000000in}{0.000000in}}%
\pgfpathlineto{\pgfqpoint{0.000000in}{0.000000in}}%
\pgfpathclose%
\pgfusepath{stroke,fill}%
\end{pgfscope}%
\begin{pgfscope}%
\pgfpathrectangle{\pgfqpoint{0.647939in}{0.492442in}}{\pgfqpoint{3.079299in}{3.079299in}}%
\pgfusepath{clip}%
\pgfsetroundcap%
\pgfsetroundjoin%
\pgfsetlinewidth{0.301125pt}%
\definecolor{currentstroke}{rgb}{0.500000,0.500000,0.500000}%
\pgfsetstrokecolor{currentstroke}%
\pgfsetstrokeopacity{0.300000}%
\pgfsetdash{}{0pt}%
\pgfpathmoveto{\pgfqpoint{2.859325in}{2.912512in}}%
\pgfusepath{stroke}%
\end{pgfscope}%
\begin{pgfscope}%
\pgfpathrectangle{\pgfqpoint{0.647939in}{0.492442in}}{\pgfqpoint{3.079299in}{3.079299in}}%
\pgfusepath{clip}%
\pgfsetroundcap%
\pgfsetroundjoin%
\definecolor{currentfill}{rgb}{0.500000,0.500000,0.500000}%
\pgfsetfillcolor{currentfill}%
\pgfsetfillopacity{0.300000}%
\pgfsetlinewidth{0.301125pt}%
\definecolor{currentstroke}{rgb}{0.500000,0.500000,0.500000}%
\pgfsetstrokecolor{currentstroke}%
\pgfsetstrokeopacity{0.300000}%
\pgfsetdash{}{0pt}%
\pgfpathmoveto{\pgfqpoint{0.000000in}{0.000000in}}%
\pgfpathlineto{\pgfqpoint{0.000000in}{0.000000in}}%
\pgfpathclose%
\pgfusepath{stroke,fill}%
\end{pgfscope}%
\begin{pgfscope}%
\pgfpathrectangle{\pgfqpoint{0.647939in}{0.492442in}}{\pgfqpoint{3.079299in}{3.079299in}}%
\pgfusepath{clip}%
\pgfsetroundcap%
\pgfsetroundjoin%
\pgfsetlinewidth{0.301125pt}%
\definecolor{currentstroke}{rgb}{0.500000,0.500000,0.500000}%
\pgfsetstrokecolor{currentstroke}%
\pgfsetstrokeopacity{0.300000}%
\pgfsetdash{}{0pt}%
\pgfpathmoveto{\pgfqpoint{1.475823in}{1.435598in}}%
\pgfusepath{stroke}%
\end{pgfscope}%
\begin{pgfscope}%
\pgfpathrectangle{\pgfqpoint{0.647939in}{0.492442in}}{\pgfqpoint{3.079299in}{3.079299in}}%
\pgfusepath{clip}%
\pgfsetroundcap%
\pgfsetroundjoin%
\definecolor{currentfill}{rgb}{0.500000,0.500000,0.500000}%
\pgfsetfillcolor{currentfill}%
\pgfsetfillopacity{0.300000}%
\pgfsetlinewidth{0.301125pt}%
\definecolor{currentstroke}{rgb}{0.500000,0.500000,0.500000}%
\pgfsetstrokecolor{currentstroke}%
\pgfsetstrokeopacity{0.300000}%
\pgfsetdash{}{0pt}%
\pgfpathmoveto{\pgfqpoint{0.000000in}{0.000000in}}%
\pgfpathlineto{\pgfqpoint{0.000000in}{0.000000in}}%
\pgfpathclose%
\pgfusepath{stroke,fill}%
\end{pgfscope}%
\begin{pgfscope}%
\pgfpathrectangle{\pgfqpoint{0.647939in}{0.492442in}}{\pgfqpoint{3.079299in}{3.079299in}}%
\pgfusepath{clip}%
\pgfsetroundcap%
\pgfsetroundjoin%
\pgfsetlinewidth{0.301125pt}%
\definecolor{currentstroke}{rgb}{0.500000,0.500000,0.500000}%
\pgfsetstrokecolor{currentstroke}%
\pgfsetstrokeopacity{0.300000}%
\pgfsetdash{}{0pt}%
\pgfpathmoveto{\pgfqpoint{1.976728in}{2.582093in}}%
\pgfusepath{stroke}%
\end{pgfscope}%
\begin{pgfscope}%
\pgfpathrectangle{\pgfqpoint{0.647939in}{0.492442in}}{\pgfqpoint{3.079299in}{3.079299in}}%
\pgfusepath{clip}%
\pgfsetroundcap%
\pgfsetroundjoin%
\definecolor{currentfill}{rgb}{0.500000,0.500000,0.500000}%
\pgfsetfillcolor{currentfill}%
\pgfsetfillopacity{0.300000}%
\pgfsetlinewidth{0.301125pt}%
\definecolor{currentstroke}{rgb}{0.500000,0.500000,0.500000}%
\pgfsetstrokecolor{currentstroke}%
\pgfsetstrokeopacity{0.300000}%
\pgfsetdash{}{0pt}%
\pgfpathmoveto{\pgfqpoint{0.000000in}{0.000000in}}%
\pgfpathlineto{\pgfqpoint{0.000000in}{0.000000in}}%
\pgfpathclose%
\pgfusepath{stroke,fill}%
\end{pgfscope}%
\begin{pgfscope}%
\pgfpathrectangle{\pgfqpoint{0.647939in}{0.492442in}}{\pgfqpoint{3.079299in}{3.079299in}}%
\pgfusepath{clip}%
\pgfsetroundcap%
\pgfsetroundjoin%
\pgfsetlinewidth{0.301125pt}%
\definecolor{currentstroke}{rgb}{0.500000,0.500000,0.500000}%
\pgfsetstrokecolor{currentstroke}%
\pgfsetstrokeopacity{0.300000}%
\pgfsetdash{}{0pt}%
\pgfpathmoveto{\pgfqpoint{2.881094in}{2.339055in}}%
\pgfusepath{stroke}%
\end{pgfscope}%
\begin{pgfscope}%
\pgfpathrectangle{\pgfqpoint{0.647939in}{0.492442in}}{\pgfqpoint{3.079299in}{3.079299in}}%
\pgfusepath{clip}%
\pgfsetroundcap%
\pgfsetroundjoin%
\definecolor{currentfill}{rgb}{0.500000,0.500000,0.500000}%
\pgfsetfillcolor{currentfill}%
\pgfsetfillopacity{0.300000}%
\pgfsetlinewidth{0.301125pt}%
\definecolor{currentstroke}{rgb}{0.500000,0.500000,0.500000}%
\pgfsetstrokecolor{currentstroke}%
\pgfsetstrokeopacity{0.300000}%
\pgfsetdash{}{0pt}%
\pgfpathmoveto{\pgfqpoint{0.000000in}{0.000000in}}%
\pgfpathlineto{\pgfqpoint{0.000000in}{0.000000in}}%
\pgfpathclose%
\pgfusepath{stroke,fill}%
\end{pgfscope}%
\begin{pgfscope}%
\pgfpathrectangle{\pgfqpoint{0.647939in}{0.492442in}}{\pgfqpoint{3.079299in}{3.079299in}}%
\pgfusepath{clip}%
\pgfsetroundcap%
\pgfsetroundjoin%
\pgfsetlinewidth{0.301125pt}%
\definecolor{currentstroke}{rgb}{0.500000,0.500000,0.500000}%
\pgfsetstrokecolor{currentstroke}%
\pgfsetstrokeopacity{0.300000}%
\pgfsetdash{}{0pt}%
\pgfpathmoveto{\pgfqpoint{2.699634in}{2.076958in}}%
\pgfusepath{stroke}%
\end{pgfscope}%
\begin{pgfscope}%
\pgfpathrectangle{\pgfqpoint{0.647939in}{0.492442in}}{\pgfqpoint{3.079299in}{3.079299in}}%
\pgfusepath{clip}%
\pgfsetroundcap%
\pgfsetroundjoin%
\definecolor{currentfill}{rgb}{0.500000,0.500000,0.500000}%
\pgfsetfillcolor{currentfill}%
\pgfsetfillopacity{0.300000}%
\pgfsetlinewidth{0.301125pt}%
\definecolor{currentstroke}{rgb}{0.500000,0.500000,0.500000}%
\pgfsetstrokecolor{currentstroke}%
\pgfsetstrokeopacity{0.300000}%
\pgfsetdash{}{0pt}%
\pgfpathmoveto{\pgfqpoint{0.000000in}{0.000000in}}%
\pgfpathlineto{\pgfqpoint{0.000000in}{0.000000in}}%
\pgfpathclose%
\pgfusepath{stroke,fill}%
\end{pgfscope}%
\begin{pgfscope}%
\pgfpathrectangle{\pgfqpoint{0.647939in}{0.492442in}}{\pgfqpoint{3.079299in}{3.079299in}}%
\pgfusepath{clip}%
\pgfsetroundcap%
\pgfsetroundjoin%
\pgfsetlinewidth{0.301125pt}%
\definecolor{currentstroke}{rgb}{0.500000,0.500000,0.500000}%
\pgfsetstrokecolor{currentstroke}%
\pgfsetstrokeopacity{0.300000}%
\pgfsetdash{}{0pt}%
\pgfpathmoveto{\pgfqpoint{2.544142in}{2.545638in}}%
\pgfusepath{stroke}%
\end{pgfscope}%
\begin{pgfscope}%
\pgfpathrectangle{\pgfqpoint{0.647939in}{0.492442in}}{\pgfqpoint{3.079299in}{3.079299in}}%
\pgfusepath{clip}%
\pgfsetroundcap%
\pgfsetroundjoin%
\definecolor{currentfill}{rgb}{0.500000,0.500000,0.500000}%
\pgfsetfillcolor{currentfill}%
\pgfsetfillopacity{0.300000}%
\pgfsetlinewidth{0.301125pt}%
\definecolor{currentstroke}{rgb}{0.500000,0.500000,0.500000}%
\pgfsetstrokecolor{currentstroke}%
\pgfsetstrokeopacity{0.300000}%
\pgfsetdash{}{0pt}%
\pgfpathmoveto{\pgfqpoint{0.000000in}{0.000000in}}%
\pgfpathlineto{\pgfqpoint{0.000000in}{0.000000in}}%
\pgfpathclose%
\pgfusepath{stroke,fill}%
\end{pgfscope}%
\begin{pgfscope}%
\pgfpathrectangle{\pgfqpoint{0.647939in}{0.492442in}}{\pgfqpoint{3.079299in}{3.079299in}}%
\pgfusepath{clip}%
\pgfsetroundcap%
\pgfsetroundjoin%
\pgfsetlinewidth{0.301125pt}%
\definecolor{currentstroke}{rgb}{0.500000,0.500000,0.500000}%
\pgfsetstrokecolor{currentstroke}%
\pgfsetstrokeopacity{0.300000}%
\pgfsetdash{}{0pt}%
\pgfpathmoveto{\pgfqpoint{2.612527in}{1.986371in}}%
\pgfusepath{stroke}%
\end{pgfscope}%
\begin{pgfscope}%
\pgfpathrectangle{\pgfqpoint{0.647939in}{0.492442in}}{\pgfqpoint{3.079299in}{3.079299in}}%
\pgfusepath{clip}%
\pgfsetroundcap%
\pgfsetroundjoin%
\definecolor{currentfill}{rgb}{0.500000,0.500000,0.500000}%
\pgfsetfillcolor{currentfill}%
\pgfsetfillopacity{0.300000}%
\pgfsetlinewidth{0.301125pt}%
\definecolor{currentstroke}{rgb}{0.500000,0.500000,0.500000}%
\pgfsetstrokecolor{currentstroke}%
\pgfsetstrokeopacity{0.300000}%
\pgfsetdash{}{0pt}%
\pgfpathmoveto{\pgfqpoint{0.000000in}{0.000000in}}%
\pgfpathlineto{\pgfqpoint{0.000000in}{0.000000in}}%
\pgfpathclose%
\pgfusepath{stroke,fill}%
\end{pgfscope}%
\begin{pgfscope}%
\pgfpathrectangle{\pgfqpoint{0.647939in}{0.492442in}}{\pgfqpoint{3.079299in}{3.079299in}}%
\pgfusepath{clip}%
\pgfsetroundcap%
\pgfsetroundjoin%
\pgfsetlinewidth{0.301125pt}%
\definecolor{currentstroke}{rgb}{0.500000,0.500000,0.500000}%
\pgfsetstrokecolor{currentstroke}%
\pgfsetstrokeopacity{0.300000}%
\pgfsetdash{}{0pt}%
\pgfpathmoveto{\pgfqpoint{1.996088in}{2.357105in}}%
\pgfusepath{stroke}%
\end{pgfscope}%
\begin{pgfscope}%
\pgfpathrectangle{\pgfqpoint{0.647939in}{0.492442in}}{\pgfqpoint{3.079299in}{3.079299in}}%
\pgfusepath{clip}%
\pgfsetroundcap%
\pgfsetroundjoin%
\definecolor{currentfill}{rgb}{0.500000,0.500000,0.500000}%
\pgfsetfillcolor{currentfill}%
\pgfsetfillopacity{0.300000}%
\pgfsetlinewidth{0.301125pt}%
\definecolor{currentstroke}{rgb}{0.500000,0.500000,0.500000}%
\pgfsetstrokecolor{currentstroke}%
\pgfsetstrokeopacity{0.300000}%
\pgfsetdash{}{0pt}%
\pgfpathmoveto{\pgfqpoint{0.000000in}{0.000000in}}%
\pgfpathlineto{\pgfqpoint{0.000000in}{0.000000in}}%
\pgfpathclose%
\pgfusepath{stroke,fill}%
\end{pgfscope}%
\begin{pgfscope}%
\pgfpathrectangle{\pgfqpoint{0.647939in}{0.492442in}}{\pgfqpoint{3.079299in}{3.079299in}}%
\pgfusepath{clip}%
\pgfsetroundcap%
\pgfsetroundjoin%
\pgfsetlinewidth{0.301125pt}%
\definecolor{currentstroke}{rgb}{0.500000,0.500000,0.500000}%
\pgfsetstrokecolor{currentstroke}%
\pgfsetstrokeopacity{0.300000}%
\pgfsetdash{}{0pt}%
\pgfpathmoveto{\pgfqpoint{1.859234in}{2.131477in}}%
\pgfusepath{stroke}%
\end{pgfscope}%
\begin{pgfscope}%
\pgfpathrectangle{\pgfqpoint{0.647939in}{0.492442in}}{\pgfqpoint{3.079299in}{3.079299in}}%
\pgfusepath{clip}%
\pgfsetroundcap%
\pgfsetroundjoin%
\definecolor{currentfill}{rgb}{0.500000,0.500000,0.500000}%
\pgfsetfillcolor{currentfill}%
\pgfsetfillopacity{0.300000}%
\pgfsetlinewidth{0.301125pt}%
\definecolor{currentstroke}{rgb}{0.500000,0.500000,0.500000}%
\pgfsetstrokecolor{currentstroke}%
\pgfsetstrokeopacity{0.300000}%
\pgfsetdash{}{0pt}%
\pgfpathmoveto{\pgfqpoint{0.000000in}{0.000000in}}%
\pgfpathlineto{\pgfqpoint{0.000000in}{0.000000in}}%
\pgfpathclose%
\pgfusepath{stroke,fill}%
\end{pgfscope}%
\begin{pgfscope}%
\pgfpathrectangle{\pgfqpoint{0.647939in}{0.492442in}}{\pgfqpoint{3.079299in}{3.079299in}}%
\pgfusepath{clip}%
\pgfsetroundcap%
\pgfsetroundjoin%
\pgfsetlinewidth{0.301125pt}%
\definecolor{currentstroke}{rgb}{0.500000,0.500000,0.500000}%
\pgfsetstrokecolor{currentstroke}%
\pgfsetstrokeopacity{0.300000}%
\pgfsetdash{}{0pt}%
\pgfpathmoveto{\pgfqpoint{2.302472in}{1.766560in}}%
\pgfusepath{stroke}%
\end{pgfscope}%
\begin{pgfscope}%
\pgfpathrectangle{\pgfqpoint{0.647939in}{0.492442in}}{\pgfqpoint{3.079299in}{3.079299in}}%
\pgfusepath{clip}%
\pgfsetroundcap%
\pgfsetroundjoin%
\definecolor{currentfill}{rgb}{0.500000,0.500000,0.500000}%
\pgfsetfillcolor{currentfill}%
\pgfsetfillopacity{0.300000}%
\pgfsetlinewidth{0.301125pt}%
\definecolor{currentstroke}{rgb}{0.500000,0.500000,0.500000}%
\pgfsetstrokecolor{currentstroke}%
\pgfsetstrokeopacity{0.300000}%
\pgfsetdash{}{0pt}%
\pgfpathmoveto{\pgfqpoint{0.000000in}{0.000000in}}%
\pgfpathlineto{\pgfqpoint{0.000000in}{0.000000in}}%
\pgfpathclose%
\pgfusepath{stroke,fill}%
\end{pgfscope}%
\begin{pgfscope}%
\pgfpathrectangle{\pgfqpoint{0.647939in}{0.492442in}}{\pgfqpoint{3.079299in}{3.079299in}}%
\pgfusepath{clip}%
\pgfsetroundcap%
\pgfsetroundjoin%
\pgfsetlinewidth{0.301125pt}%
\definecolor{currentstroke}{rgb}{0.500000,0.500000,0.500000}%
\pgfsetstrokecolor{currentstroke}%
\pgfsetstrokeopacity{0.300000}%
\pgfsetdash{}{0pt}%
\pgfpathmoveto{\pgfqpoint{2.304106in}{1.914857in}}%
\pgfusepath{stroke}%
\end{pgfscope}%
\begin{pgfscope}%
\pgfpathrectangle{\pgfqpoint{0.647939in}{0.492442in}}{\pgfqpoint{3.079299in}{3.079299in}}%
\pgfusepath{clip}%
\pgfsetroundcap%
\pgfsetroundjoin%
\definecolor{currentfill}{rgb}{0.500000,0.500000,0.500000}%
\pgfsetfillcolor{currentfill}%
\pgfsetfillopacity{0.300000}%
\pgfsetlinewidth{0.301125pt}%
\definecolor{currentstroke}{rgb}{0.500000,0.500000,0.500000}%
\pgfsetstrokecolor{currentstroke}%
\pgfsetstrokeopacity{0.300000}%
\pgfsetdash{}{0pt}%
\pgfpathmoveto{\pgfqpoint{0.000000in}{0.000000in}}%
\pgfpathlineto{\pgfqpoint{0.000000in}{0.000000in}}%
\pgfpathclose%
\pgfusepath{stroke,fill}%
\end{pgfscope}%
\begin{pgfscope}%
\pgfpathrectangle{\pgfqpoint{0.647939in}{0.492442in}}{\pgfqpoint{3.079299in}{3.079299in}}%
\pgfusepath{clip}%
\pgfsetroundcap%
\pgfsetroundjoin%
\pgfsetlinewidth{0.301125pt}%
\definecolor{currentstroke}{rgb}{0.500000,0.500000,0.500000}%
\pgfsetstrokecolor{currentstroke}%
\pgfsetstrokeopacity{0.300000}%
\pgfsetdash{}{0pt}%
\pgfpathmoveto{\pgfqpoint{2.147163in}{2.237662in}}%
\pgfusepath{stroke}%
\end{pgfscope}%
\begin{pgfscope}%
\pgfpathrectangle{\pgfqpoint{0.647939in}{0.492442in}}{\pgfqpoint{3.079299in}{3.079299in}}%
\pgfusepath{clip}%
\pgfsetroundcap%
\pgfsetroundjoin%
\definecolor{currentfill}{rgb}{0.500000,0.500000,0.500000}%
\pgfsetfillcolor{currentfill}%
\pgfsetfillopacity{0.300000}%
\pgfsetlinewidth{0.301125pt}%
\definecolor{currentstroke}{rgb}{0.500000,0.500000,0.500000}%
\pgfsetstrokecolor{currentstroke}%
\pgfsetstrokeopacity{0.300000}%
\pgfsetdash{}{0pt}%
\pgfpathmoveto{\pgfqpoint{0.000000in}{0.000000in}}%
\pgfpathlineto{\pgfqpoint{0.000000in}{0.000000in}}%
\pgfpathclose%
\pgfusepath{stroke,fill}%
\end{pgfscope}%
\begin{pgfscope}%
\pgfpathrectangle{\pgfqpoint{0.647939in}{0.492442in}}{\pgfqpoint{3.079299in}{3.079299in}}%
\pgfusepath{clip}%
\pgfsetroundcap%
\pgfsetroundjoin%
\pgfsetlinewidth{0.301125pt}%
\definecolor{currentstroke}{rgb}{0.500000,0.500000,0.500000}%
\pgfsetstrokecolor{currentstroke}%
\pgfsetstrokeopacity{0.300000}%
\pgfsetdash{}{0pt}%
\pgfpathmoveto{\pgfqpoint{2.424181in}{2.031606in}}%
\pgfusepath{stroke}%
\end{pgfscope}%
\begin{pgfscope}%
\pgfpathrectangle{\pgfqpoint{0.647939in}{0.492442in}}{\pgfqpoint{3.079299in}{3.079299in}}%
\pgfusepath{clip}%
\pgfsetroundcap%
\pgfsetroundjoin%
\definecolor{currentfill}{rgb}{0.500000,0.500000,0.500000}%
\pgfsetfillcolor{currentfill}%
\pgfsetfillopacity{0.300000}%
\pgfsetlinewidth{0.301125pt}%
\definecolor{currentstroke}{rgb}{0.500000,0.500000,0.500000}%
\pgfsetstrokecolor{currentstroke}%
\pgfsetstrokeopacity{0.300000}%
\pgfsetdash{}{0pt}%
\pgfpathmoveto{\pgfqpoint{0.000000in}{0.000000in}}%
\pgfpathlineto{\pgfqpoint{0.000000in}{0.000000in}}%
\pgfpathclose%
\pgfusepath{stroke,fill}%
\end{pgfscope}%
\begin{pgfscope}%
\pgfpathrectangle{\pgfqpoint{0.647939in}{0.492442in}}{\pgfqpoint{3.079299in}{3.079299in}}%
\pgfusepath{clip}%
\pgfsetbuttcap%
\pgfsetroundjoin%
\pgfsetlinewidth{0.301125pt}%
\definecolor{currentstroke}{rgb}{0.500000,0.500000,0.500000}%
\pgfsetstrokecolor{currentstroke}%
\pgfsetstrokeopacity{0.300000}%
\pgfsetdash{}{0pt}%
\pgfpathmoveto{\pgfqpoint{0.647939in}{0.492442in}}%
\pgfpathlineto{\pgfqpoint{0.647939in}{0.492442in}}%
\pgfpathlineto{\pgfqpoint{0.714728in}{0.507252in}}%
\pgfpathlineto{\pgfqpoint{0.780735in}{0.525196in}}%
\pgfpathlineto{\pgfqpoint{0.845685in}{0.546629in}}%
\pgfpathlineto{\pgfqpoint{0.909247in}{0.571858in}}%
\pgfpathlineto{\pgfqpoint{0.971048in}{0.601119in}}%
\pgfpathlineto{\pgfqpoint{1.030694in}{0.634536in}}%
\pgfpathlineto{\pgfqpoint{1.087813in}{0.672105in}}%
\pgfpathlineto{\pgfqpoint{1.142096in}{0.713675in}}%
\pgfpathlineto{\pgfqpoint{1.193348in}{0.758915in}}%
\pgfpathlineto{\pgfqpoint{1.241512in}{0.807430in}}%
\pgfpathlineto{\pgfqpoint{1.286690in}{0.858745in}}%
\pgfpathlineto{\pgfqpoint{1.329115in}{0.912365in}}%
\pgfpathlineto{\pgfqpoint{1.369132in}{0.967811in}}%
\pgfpathlineto{\pgfqpoint{1.407151in}{1.024635in}}%
\pgfpathlineto{\pgfqpoint{1.443606in}{1.082470in}}%
\pgfpathlineto{\pgfqpoint{1.478930in}{1.141007in}}%
\pgfpathlineto{\pgfqpoint{1.513540in}{1.199977in}}%
\pgfpathlineto{\pgfqpoint{1.547826in}{1.259148in}}%
\pgfpathlineto{\pgfqpoint{1.582151in}{1.318305in}}%
\pgfpathlineto{\pgfqpoint{1.616849in}{1.377227in}}%
\pgfpathlineto{\pgfqpoint{1.652223in}{1.435714in}}%
\pgfpathlineto{\pgfqpoint{1.688560in}{1.493588in}}%
\pgfpathlineto{\pgfqpoint{1.726138in}{1.550674in}}%
\pgfpathlineto{\pgfqpoint{1.765215in}{1.606766in}}%
\pgfpathlineto{\pgfqpoint{1.806015in}{1.661582in}}%
\pgfpathlineto{\pgfqpoint{1.848749in}{1.714834in}}%
\pgfpathlineto{\pgfqpoint{1.893648in}{1.766348in}}%
\pgfpathlineto{\pgfqpoint{1.940716in}{1.815797in}}%
\pgfpathlineto{\pgfqpoint{1.989883in}{1.863090in}}%
\pgfpathlineto{\pgfqpoint{2.040751in}{1.908427in}}%
\pgfpathlineto{\pgfqpoint{2.092467in}{1.952550in}}%
\pgfpathlineto{\pgfqpoint{2.143718in}{1.996700in}}%
\pgfpathlineto{\pgfqpoint{2.193219in}{2.041988in}}%
\pgfpathlineto{\pgfqpoint{2.240949in}{2.088685in}}%
\pgfpathlineto{\pgfqpoint{2.287546in}{2.136870in}}%
\pgfpathlineto{\pgfqpoint{2.333288in}{2.186356in}}%
\pgfpathlineto{\pgfqpoint{2.378329in}{2.236933in}}%
\pgfpathlineto{\pgfqpoint{2.422821in}{2.288393in}}%
\pgfpathlineto{\pgfqpoint{2.466754in}{2.340448in}}%
\pgfpathlineto{\pgfqpoint{2.510247in}{2.392974in}}%
\pgfpathlineto{\pgfqpoint{2.553417in}{2.445907in}}%
\pgfpathlineto{\pgfqpoint{2.596270in}{2.499081in}}%
\pgfpathlineto{\pgfqpoint{2.638926in}{2.552482in}}%
\pgfpathlineto{\pgfqpoint{2.681445in}{2.606044in}}%
\pgfpathlineto{\pgfqpoint{2.723860in}{2.659675in}}%
\pgfpathlineto{\pgfqpoint{2.766247in}{2.713343in}}%
\pgfpathlineto{\pgfqpoint{2.808668in}{2.766998in}}%
\pgfpathlineto{\pgfqpoint{2.851178in}{2.820589in}}%
\pgfpathlineto{\pgfqpoint{2.893836in}{2.874069in}}%
\pgfpathlineto{\pgfqpoint{2.936702in}{2.927383in}}%
\pgfpathlineto{\pgfqpoint{2.979838in}{2.980490in}}%
\pgfpathlineto{\pgfqpoint{3.023306in}{3.033329in}}%
\pgfpathlineto{\pgfqpoint{3.067169in}{3.085835in}}%
\pgfpathlineto{\pgfqpoint{3.111496in}{3.137957in}}%
\pgfpathlineto{\pgfqpoint{3.156357in}{3.189621in}}%
\pgfpathlineto{\pgfqpoint{3.201822in}{3.240752in}}%
\pgfpathlineto{\pgfqpoint{3.247973in}{3.291268in}}%
\pgfpathlineto{\pgfqpoint{3.294890in}{3.341075in}}%
\pgfpathlineto{\pgfqpoint{3.342657in}{3.390064in}}%
\pgfpathlineto{\pgfqpoint{3.391366in}{3.438118in}}%
\pgfpathlineto{\pgfqpoint{3.441107in}{3.485101in}}%
\pgfpathlineto{\pgfqpoint{3.491977in}{3.530858in}}%
\pgfpathlineto{\pgfqpoint{3.538609in}{3.571741in}}%
\pgfusepath{stroke}%
\end{pgfscope}%
\begin{pgfscope}%
\pgfpathrectangle{\pgfqpoint{0.647939in}{0.492442in}}{\pgfqpoint{3.079299in}{3.079299in}}%
\pgfusepath{clip}%
\pgfsetbuttcap%
\pgfsetroundjoin%
\pgfsetlinewidth{0.301125pt}%
\definecolor{currentstroke}{rgb}{0.500000,0.500000,0.500000}%
\pgfsetstrokecolor{currentstroke}%
\pgfsetstrokeopacity{0.300000}%
\pgfsetdash{}{0pt}%
\pgfpathmoveto{\pgfqpoint{0.927875in}{0.492442in}}%
\pgfpathlineto{\pgfqpoint{0.927875in}{0.492442in}}%
\pgfpathlineto{\pgfqpoint{0.988384in}{0.524295in}}%
\pgfpathlineto{\pgfqpoint{1.046354in}{0.560541in}}%
\pgfpathlineto{\pgfqpoint{1.101387in}{0.601080in}}%
\pgfpathlineto{\pgfqpoint{1.153194in}{0.645659in}}%
\pgfpathlineto{\pgfqpoint{1.201643in}{0.693880in}}%
\pgfpathlineto{\pgfqpoint{1.246785in}{0.745207in}}%
\pgfpathlineto{\pgfqpoint{1.288841in}{0.799091in}}%
\pgfpathlineto{\pgfqpoint{1.328165in}{0.855018in}}%
\pgfusepath{stroke}%
\end{pgfscope}%
\begin{pgfscope}%
\pgfpathrectangle{\pgfqpoint{0.647939in}{0.492442in}}{\pgfqpoint{3.079299in}{3.079299in}}%
\pgfusepath{clip}%
\pgfsetbuttcap%
\pgfsetroundjoin%
\pgfsetlinewidth{0.301125pt}%
\definecolor{currentstroke}{rgb}{0.500000,0.500000,0.500000}%
\pgfsetstrokecolor{currentstroke}%
\pgfsetstrokeopacity{0.300000}%
\pgfsetdash{}{0pt}%
\pgfpathmoveto{\pgfqpoint{1.137828in}{0.492442in}}%
\pgfpathlineto{\pgfqpoint{1.137828in}{0.492442in}}%
\pgfpathlineto{\pgfqpoint{1.182961in}{0.543746in}}%
\pgfpathlineto{\pgfqpoint{1.224437in}{0.598050in}}%
\pgfpathlineto{\pgfqpoint{1.262642in}{0.654715in}}%
\pgfpathlineto{\pgfqpoint{1.298081in}{0.713186in}}%
\pgfpathlineto{\pgfqpoint{1.331301in}{0.772979in}}%
\pgfusepath{stroke}%
\end{pgfscope}%
\begin{pgfscope}%
\pgfpathrectangle{\pgfqpoint{0.647939in}{0.492442in}}{\pgfqpoint{3.079299in}{3.079299in}}%
\pgfusepath{clip}%
\pgfsetbuttcap%
\pgfsetroundjoin%
\pgfsetlinewidth{0.301125pt}%
\definecolor{currentstroke}{rgb}{0.500000,0.500000,0.500000}%
\pgfsetstrokecolor{currentstroke}%
\pgfsetstrokeopacity{0.300000}%
\pgfsetdash{}{0pt}%
\pgfpathmoveto{\pgfqpoint{1.347780in}{0.492442in}}%
\pgfpathlineto{\pgfqpoint{1.347780in}{0.492442in}}%
\pgfpathlineto{\pgfqpoint{1.353611in}{0.560382in}}%
\pgfpathlineto{\pgfqpoint{1.363952in}{0.627886in}}%
\pgfpathlineto{\pgfqpoint{1.377706in}{0.694802in}}%
\pgfpathlineto{\pgfqpoint{1.394094in}{0.761122in}}%
\pgfpathlineto{\pgfqpoint{1.412609in}{0.826906in}}%
\pgfpathlineto{\pgfqpoint{1.432895in}{0.892136in}}%
\pgfpathlineto{\pgfqpoint{1.454726in}{0.956912in}}%
\pgfpathlineto{\pgfqpoint{1.477969in}{1.021202in}}%
\pgfpathlineto{\pgfqpoint{1.502536in}{1.084960in}}%
\pgfpathlineto{\pgfqpoint{1.528388in}{1.148208in}}%
\pgfpathlineto{\pgfqpoint{1.555532in}{1.210934in}}%
\pgfusepath{stroke}%
\end{pgfscope}%
\begin{pgfscope}%
\pgfpathrectangle{\pgfqpoint{0.647939in}{0.492442in}}{\pgfqpoint{3.079299in}{3.079299in}}%
\pgfusepath{clip}%
\pgfsetbuttcap%
\pgfsetroundjoin%
\pgfsetlinewidth{0.301125pt}%
\definecolor{currentstroke}{rgb}{0.500000,0.500000,0.500000}%
\pgfsetstrokecolor{currentstroke}%
\pgfsetstrokeopacity{0.300000}%
\pgfsetdash{}{0pt}%
\pgfpathmoveto{\pgfqpoint{1.557732in}{0.492442in}}%
\pgfpathlineto{\pgfqpoint{1.557732in}{0.492442in}}%
\pgfpathlineto{\pgfqpoint{1.512229in}{0.542739in}}%
\pgfpathlineto{\pgfqpoint{1.482038in}{0.595876in}}%
\pgfpathlineto{\pgfqpoint{1.463356in}{0.652126in}}%
\pgfpathlineto{\pgfqpoint{1.453948in}{0.713215in}}%
\pgfpathlineto{\pgfqpoint{1.453036in}{0.781314in}}%
\pgfpathlineto{\pgfqpoint{1.459396in}{0.849273in}}%
\pgfpathlineto{\pgfqpoint{1.471146in}{0.916507in}}%
\pgfusepath{stroke}%
\end{pgfscope}%
\begin{pgfscope}%
\pgfpathrectangle{\pgfqpoint{0.647939in}{0.492442in}}{\pgfqpoint{3.079299in}{3.079299in}}%
\pgfusepath{clip}%
\pgfsetbuttcap%
\pgfsetroundjoin%
\pgfsetlinewidth{0.301125pt}%
\definecolor{currentstroke}{rgb}{0.500000,0.500000,0.500000}%
\pgfsetstrokecolor{currentstroke}%
\pgfsetstrokeopacity{0.300000}%
\pgfsetdash{}{0pt}%
\pgfpathmoveto{\pgfqpoint{1.837668in}{0.492442in}}%
\pgfpathlineto{\pgfqpoint{1.837668in}{0.492442in}}%
\pgfpathlineto{\pgfqpoint{1.770946in}{0.507351in}}%
\pgfpathlineto{\pgfqpoint{1.705990in}{0.528529in}}%
\pgfpathlineto{\pgfqpoint{1.644418in}{0.557921in}}%
\pgfpathlineto{\pgfqpoint{1.589050in}{0.597479in}}%
\pgfpathlineto{\pgfqpoint{1.544650in}{0.646863in}}%
\pgfusepath{stroke}%
\end{pgfscope}%
\begin{pgfscope}%
\pgfpathrectangle{\pgfqpoint{0.647939in}{0.492442in}}{\pgfqpoint{3.079299in}{3.079299in}}%
\pgfusepath{clip}%
\pgfsetbuttcap%
\pgfsetroundjoin%
\pgfsetlinewidth{0.301125pt}%
\definecolor{currentstroke}{rgb}{0.500000,0.500000,0.500000}%
\pgfsetstrokecolor{currentstroke}%
\pgfsetstrokeopacity{0.300000}%
\pgfsetdash{}{0pt}%
\pgfpathmoveto{\pgfqpoint{2.257573in}{0.492442in}}%
\pgfpathlineto{\pgfqpoint{2.257573in}{0.492442in}}%
\pgfpathlineto{\pgfqpoint{2.189175in}{0.494506in}}%
\pgfpathlineto{\pgfqpoint{2.120781in}{0.496664in}}%
\pgfpathlineto{\pgfqpoint{2.052414in}{0.499529in}}%
\pgfpathlineto{\pgfqpoint{1.984126in}{0.503816in}}%
\pgfpathlineto{\pgfqpoint{1.916029in}{0.510384in}}%
\pgfusepath{stroke}%
\end{pgfscope}%
\begin{pgfscope}%
\pgfpathrectangle{\pgfqpoint{0.647939in}{0.492442in}}{\pgfqpoint{3.079299in}{3.079299in}}%
\pgfusepath{clip}%
\pgfsetbuttcap%
\pgfsetroundjoin%
\pgfsetlinewidth{0.301125pt}%
\definecolor{currentstroke}{rgb}{0.500000,0.500000,0.500000}%
\pgfsetstrokecolor{currentstroke}%
\pgfsetstrokeopacity{0.300000}%
\pgfsetdash{}{0pt}%
\pgfpathmoveto{\pgfqpoint{2.677477in}{0.492442in}}%
\pgfpathlineto{\pgfqpoint{2.677477in}{0.492442in}}%
\pgfpathlineto{\pgfqpoint{2.609856in}{0.502873in}}%
\pgfpathlineto{\pgfqpoint{2.541962in}{0.511353in}}%
\pgfpathlineto{\pgfqpoint{2.473863in}{0.517997in}}%
\pgfpathlineto{\pgfqpoint{2.405623in}{0.523008in}}%
\pgfpathlineto{\pgfqpoint{2.337295in}{0.526670in}}%
\pgfpathlineto{\pgfqpoint{2.268920in}{0.529349in}}%
\pgfpathlineto{\pgfqpoint{2.200525in}{0.531496in}}%
\pgfpathlineto{\pgfqpoint{2.132130in}{0.533648in}}%
\pgfpathlineto{\pgfqpoint{2.063760in}{0.536425in}}%
\pgfpathlineto{\pgfqpoint{1.995462in}{0.540548in}}%
\pgfpathlineto{\pgfqpoint{1.927340in}{0.546870in}}%
\pgfpathlineto{\pgfqpoint{1.859613in}{0.556438in}}%
\pgfpathlineto{\pgfqpoint{1.792720in}{0.570568in}}%
\pgfusepath{stroke}%
\end{pgfscope}%
\begin{pgfscope}%
\pgfpathrectangle{\pgfqpoint{0.647939in}{0.492442in}}{\pgfqpoint{3.079299in}{3.079299in}}%
\pgfusepath{clip}%
\pgfsetbuttcap%
\pgfsetroundjoin%
\pgfsetlinewidth{0.301125pt}%
\definecolor{currentstroke}{rgb}{0.500000,0.500000,0.500000}%
\pgfsetstrokecolor{currentstroke}%
\pgfsetstrokeopacity{0.300000}%
\pgfsetdash{}{0pt}%
\pgfpathmoveto{\pgfqpoint{2.887429in}{0.492442in}}%
\pgfpathlineto{\pgfqpoint{2.887429in}{0.492442in}}%
\pgfpathlineto{\pgfqpoint{2.821094in}{0.509214in}}%
\pgfpathlineto{\pgfqpoint{2.754327in}{0.524164in}}%
\pgfpathlineto{\pgfqpoint{2.687153in}{0.537165in}}%
\pgfpathlineto{\pgfqpoint{2.619622in}{0.548154in}}%
\pgfpathlineto{\pgfqpoint{2.551795in}{0.557143in}}%
\pgfpathlineto{\pgfqpoint{2.483740in}{0.564229in}}%
\pgfpathlineto{\pgfqpoint{2.415528in}{0.569604in}}%
\pgfpathlineto{\pgfqpoint{2.347216in}{0.573552in}}%
\pgfusepath{stroke}%
\end{pgfscope}%
\begin{pgfscope}%
\pgfpathrectangle{\pgfqpoint{0.647939in}{0.492442in}}{\pgfqpoint{3.079299in}{3.079299in}}%
\pgfusepath{clip}%
\pgfsetbuttcap%
\pgfsetroundjoin%
\pgfsetlinewidth{0.301125pt}%
\definecolor{currentstroke}{rgb}{0.500000,0.500000,0.500000}%
\pgfsetstrokecolor{currentstroke}%
\pgfsetstrokeopacity{0.300000}%
\pgfsetdash{}{0pt}%
\pgfpathmoveto{\pgfqpoint{3.097382in}{0.492442in}}%
\pgfpathlineto{\pgfqpoint{3.097382in}{0.492442in}}%
\pgfpathlineto{\pgfqpoint{3.032538in}{0.514292in}}%
\pgfpathlineto{\pgfqpoint{2.967323in}{0.535005in}}%
\pgfpathlineto{\pgfqpoint{2.901686in}{0.554329in}}%
\pgfpathlineto{\pgfqpoint{2.835595in}{0.572036in}}%
\pgfpathlineto{\pgfqpoint{2.769045in}{0.587926in}}%
\pgfpathlineto{\pgfqpoint{2.702056in}{0.601847in}}%
\pgfpathlineto{\pgfqpoint{2.634673in}{0.613713in}}%
\pgfpathlineto{\pgfqpoint{2.566958in}{0.623513in}}%
\pgfpathlineto{\pgfqpoint{2.498985in}{0.631323in}}%
\pgfpathlineto{\pgfqpoint{2.430825in}{0.637312in}}%
\pgfpathlineto{\pgfqpoint{2.362544in}{0.641740in}}%
\pgfpathlineto{\pgfqpoint{2.294193in}{0.644965in}}%
\pgfpathlineto{\pgfqpoint{2.225809in}{0.647440in}}%
\pgfpathlineto{\pgfqpoint{2.157418in}{0.649710in}}%
\pgfpathlineto{\pgfqpoint{2.089044in}{0.652409in}}%
\pgfpathlineto{\pgfqpoint{2.020730in}{0.656288in}}%
\pgfpathlineto{\pgfqpoint{1.952575in}{0.662256in}}%
\pgfpathlineto{\pgfqpoint{1.884798in}{0.671450in}}%
\pgfpathlineto{\pgfqpoint{1.817857in}{0.685308in}}%
\pgfpathlineto{\pgfqpoint{1.752681in}{0.705690in}}%
\pgfpathlineto{\pgfqpoint{1.691120in}{0.734914in}}%
\pgfpathlineto{\pgfqpoint{1.636559in}{0.775312in}}%
\pgfpathlineto{\pgfqpoint{1.596299in}{0.823302in}}%
\pgfpathlineto{\pgfqpoint{1.571034in}{0.873547in}}%
\pgfpathlineto{\pgfqpoint{1.556550in}{0.927181in}}%
\pgfpathlineto{\pgfqpoint{1.551050in}{0.986032in}}%
\pgfpathlineto{\pgfqpoint{1.554201in}{1.052094in}}%
\pgfusepath{stroke}%
\end{pgfscope}%
\begin{pgfscope}%
\pgfpathrectangle{\pgfqpoint{0.647939in}{0.492442in}}{\pgfqpoint{3.079299in}{3.079299in}}%
\pgfusepath{clip}%
\pgfsetbuttcap%
\pgfsetroundjoin%
\pgfsetlinewidth{0.301125pt}%
\definecolor{currentstroke}{rgb}{0.500000,0.500000,0.500000}%
\pgfsetstrokecolor{currentstroke}%
\pgfsetstrokeopacity{0.300000}%
\pgfsetdash{}{0pt}%
\pgfpathmoveto{\pgfqpoint{3.307334in}{0.492442in}}%
\pgfpathlineto{\pgfqpoint{3.307334in}{0.492442in}}%
\pgfpathlineto{\pgfqpoint{3.243436in}{0.516928in}}%
\pgfpathlineto{\pgfqpoint{3.179443in}{0.541161in}}%
\pgfpathlineto{\pgfqpoint{3.115255in}{0.564874in}}%
\pgfpathlineto{\pgfqpoint{3.050782in}{0.587796in}}%
\pgfpathlineto{\pgfqpoint{2.985943in}{0.609656in}}%
\pgfpathlineto{\pgfqpoint{2.920672in}{0.630190in}}%
\pgfpathlineto{\pgfqpoint{2.854929in}{0.649150in}}%
\pgfpathlineto{\pgfqpoint{2.788695in}{0.666313in}}%
\pgfpathlineto{\pgfqpoint{2.721982in}{0.681499in}}%
\pgfpathlineto{\pgfqpoint{2.654828in}{0.694588in}}%
\pgfpathlineto{\pgfqpoint{2.587291in}{0.705533in}}%
\pgfpathlineto{\pgfqpoint{2.519444in}{0.714372in}}%
\pgfpathlineto{\pgfqpoint{2.451368in}{0.721240in}}%
\pgfpathlineto{\pgfqpoint{2.383137in}{0.726384in}}%
\pgfpathlineto{\pgfqpoint{2.314815in}{0.730150in}}%
\pgfpathlineto{\pgfqpoint{2.246446in}{0.732975in}}%
\pgfpathlineto{\pgfqpoint{2.178060in}{0.735398in}}%
\pgfpathlineto{\pgfqpoint{2.109684in}{0.738061in}}%
\pgfpathlineto{\pgfqpoint{2.041359in}{0.741753in}}%
\pgfpathlineto{\pgfqpoint{1.973179in}{0.747428in}}%
\pgfpathlineto{\pgfqpoint{1.905357in}{0.756264in}}%
\pgfpathlineto{\pgfqpoint{1.838349in}{0.769792in}}%
\pgfpathlineto{\pgfqpoint{1.773128in}{0.790035in}}%
\pgfpathlineto{\pgfqpoint{1.711722in}{0.819569in}}%
\pgfusepath{stroke}%
\end{pgfscope}%
\begin{pgfscope}%
\pgfpathrectangle{\pgfqpoint{0.647939in}{0.492442in}}{\pgfqpoint{3.079299in}{3.079299in}}%
\pgfusepath{clip}%
\pgfsetbuttcap%
\pgfsetroundjoin%
\pgfsetlinewidth{0.301125pt}%
\definecolor{currentstroke}{rgb}{0.500000,0.500000,0.500000}%
\pgfsetstrokecolor{currentstroke}%
\pgfsetstrokeopacity{0.300000}%
\pgfsetdash{}{0pt}%
\pgfpathmoveto{\pgfqpoint{3.517286in}{0.492442in}}%
\pgfpathlineto{\pgfqpoint{3.517286in}{0.492442in}}%
\pgfpathlineto{\pgfqpoint{3.453306in}{0.516710in}}%
\pgfpathlineto{\pgfqpoint{3.389567in}{0.541604in}}%
\pgfpathlineto{\pgfqpoint{3.325975in}{0.566873in}}%
\pgfpathlineto{\pgfqpoint{3.262430in}{0.592259in}}%
\pgfpathlineto{\pgfqpoint{3.198826in}{0.617500in}}%
\pgfpathlineto{\pgfqpoint{3.135059in}{0.642322in}}%
\pgfpathlineto{\pgfqpoint{3.071028in}{0.666452in}}%
\pgfpathlineto{\pgfqpoint{3.006640in}{0.689610in}}%
\pgfpathlineto{\pgfqpoint{2.941820in}{0.711523in}}%
\pgfpathlineto{\pgfqpoint{2.876510in}{0.731925in}}%
\pgfpathlineto{\pgfqpoint{2.810680in}{0.750572in}}%
\pgfpathlineto{\pgfqpoint{2.744326in}{0.767252in}}%
\pgfpathlineto{\pgfqpoint{2.677474in}{0.781805in}}%
\pgfpathlineto{\pgfqpoint{2.610177in}{0.794141in}}%
\pgfpathlineto{\pgfqpoint{2.542509in}{0.804256in}}%
\pgfpathlineto{\pgfqpoint{2.474558in}{0.812251in}}%
\pgfpathlineto{\pgfqpoint{2.406407in}{0.818335in}}%
\pgfpathlineto{\pgfqpoint{2.338129in}{0.822821in}}%
\pgfpathlineto{\pgfqpoint{2.269783in}{0.826135in}}%
\pgfpathlineto{\pgfqpoint{2.201407in}{0.828813in}}%
\pgfpathlineto{\pgfqpoint{2.133032in}{0.831522in}}%
\pgfpathlineto{\pgfqpoint{2.064698in}{0.835067in}}%
\pgfpathlineto{\pgfqpoint{1.996492in}{0.840427in}}%
\pgfpathlineto{\pgfqpoint{1.928613in}{0.848845in}}%
\pgfpathlineto{\pgfqpoint{1.861518in}{0.861965in}}%
\pgfpathlineto{\pgfqpoint{1.796248in}{0.882023in}}%
\pgfpathlineto{\pgfqpoint{1.735064in}{0.911914in}}%
\pgfpathlineto{\pgfqpoint{1.682334in}{0.954483in}}%
\pgfpathlineto{\pgfqpoint{1.648221in}{1.001043in}}%
\pgfpathlineto{\pgfqpoint{1.628114in}{1.049509in}}%
\pgfpathlineto{\pgfqpoint{1.618066in}{1.101533in}}%
\pgfpathlineto{\pgfqpoint{1.616773in}{1.159329in}}%
\pgfpathlineto{\pgfqpoint{1.624419in}{1.224417in}}%
\pgfpathlineto{\pgfqpoint{1.639660in}{1.290812in}}%
\pgfpathlineto{\pgfqpoint{1.660396in}{1.355804in}}%
\pgfpathlineto{\pgfqpoint{1.685436in}{1.419269in}}%
\pgfusepath{stroke}%
\end{pgfscope}%
\begin{pgfscope}%
\pgfpathrectangle{\pgfqpoint{0.647939in}{0.492442in}}{\pgfqpoint{3.079299in}{3.079299in}}%
\pgfusepath{clip}%
\pgfsetbuttcap%
\pgfsetroundjoin%
\pgfsetlinewidth{0.301125pt}%
\definecolor{currentstroke}{rgb}{0.500000,0.500000,0.500000}%
\pgfsetstrokecolor{currentstroke}%
\pgfsetstrokeopacity{0.300000}%
\pgfsetdash{}{0pt}%
\pgfpathmoveto{\pgfqpoint{3.727238in}{0.492442in}}%
\pgfpathlineto{\pgfqpoint{3.727238in}{0.492442in}}%
\pgfpathlineto{\pgfqpoint{3.662201in}{0.513704in}}%
\pgfpathlineto{\pgfqpoint{3.597641in}{0.536374in}}%
\pgfpathlineto{\pgfqpoint{3.533511in}{0.560239in}}%
\pgfpathlineto{\pgfqpoint{3.469748in}{0.585067in}}%
\pgfpathlineto{\pgfqpoint{3.406269in}{0.610619in}}%
\pgfpathlineto{\pgfqpoint{3.342983in}{0.636644in}}%
\pgfpathlineto{\pgfqpoint{3.279787in}{0.662887in}}%
\pgfpathlineto{\pgfqpoint{3.216572in}{0.689084in}}%
\pgfpathlineto{\pgfqpoint{3.153227in}{0.714965in}}%
\pgfpathlineto{\pgfqpoint{3.089645in}{0.740255in}}%
\pgfpathlineto{\pgfqpoint{3.025724in}{0.764673in}}%
\pgfpathlineto{\pgfqpoint{2.961376in}{0.787935in}}%
\pgfpathlineto{\pgfqpoint{2.896528in}{0.809763in}}%
\pgfpathlineto{\pgfqpoint{2.831135in}{0.829891in}}%
\pgfpathlineto{\pgfqpoint{2.765177in}{0.848079in}}%
\pgfpathlineto{\pgfqpoint{2.698668in}{0.864130in}}%
\pgfpathlineto{\pgfqpoint{2.631653in}{0.877914in}}%
\pgfpathlineto{\pgfqpoint{2.564204in}{0.889384in}}%
\pgfpathlineto{\pgfqpoint{2.496409in}{0.898593in}}%
\pgfpathlineto{\pgfqpoint{2.428360in}{0.905706in}}%
\pgfpathlineto{\pgfqpoint{2.360142in}{0.911007in}}%
\pgfpathlineto{\pgfqpoint{2.291827in}{0.914908in}}%
\pgfpathlineto{\pgfqpoint{2.223466in}{0.917951in}}%
\pgfpathlineto{\pgfqpoint{2.155097in}{0.920800in}}%
\pgfpathlineto{\pgfqpoint{2.086759in}{0.924270in}}%
\pgfpathlineto{\pgfqpoint{2.018530in}{0.929377in}}%
\pgfpathlineto{\pgfqpoint{1.950606in}{0.937442in}}%
\pgfpathlineto{\pgfqpoint{1.883462in}{0.950251in}}%
\pgfpathlineto{\pgfqpoint{1.818211in}{0.970257in}}%
\pgfpathlineto{\pgfqpoint{1.757409in}{1.000713in}}%
\pgfpathlineto{\pgfqpoint{1.757409in}{1.000713in}}%
\pgfpathlineto{\pgfqpoint{1.714303in}{1.035421in}}%
\pgfpathlineto{\pgfqpoint{1.680700in}{1.080253in}}%
\pgfusepath{stroke}%
\end{pgfscope}%
\begin{pgfscope}%
\pgfpathrectangle{\pgfqpoint{0.647939in}{0.492442in}}{\pgfqpoint{3.079299in}{3.079299in}}%
\pgfusepath{clip}%
\pgfsetbuttcap%
\pgfsetroundjoin%
\pgfsetlinewidth{0.301125pt}%
\definecolor{currentstroke}{rgb}{0.500000,0.500000,0.500000}%
\pgfsetstrokecolor{currentstroke}%
\pgfsetstrokeopacity{0.300000}%
\pgfsetdash{}{0pt}%
\pgfpathmoveto{\pgfqpoint{3.727238in}{0.562426in}}%
\pgfpathlineto{\pgfqpoint{3.727238in}{0.562426in}}%
\pgfpathlineto{\pgfqpoint{3.662416in}{0.584333in}}%
\pgfpathlineto{\pgfqpoint{3.598103in}{0.607694in}}%
\pgfpathlineto{\pgfqpoint{3.534252in}{0.632294in}}%
\pgfpathlineto{\pgfqpoint{3.470797in}{0.657900in}}%
\pgfpathlineto{\pgfqpoint{3.407655in}{0.684271in}}%
\pgfpathlineto{\pgfqpoint{3.344730in}{0.711158in}}%
\pgfpathlineto{\pgfqpoint{3.281916in}{0.738305in}}%
\pgfpathlineto{\pgfqpoint{3.219101in}{0.765448in}}%
\pgfpathlineto{\pgfqpoint{3.156170in}{0.792318in}}%
\pgfpathlineto{\pgfqpoint{3.093007in}{0.818638in}}%
\pgfpathlineto{\pgfqpoint{3.029504in}{0.844122in}}%
\pgfpathlineto{\pgfqpoint{2.965564in}{0.868484in}}%
\pgfpathlineto{\pgfqpoint{2.901105in}{0.891433in}}%
\pgfpathlineto{\pgfqpoint{2.836071in}{0.912691in}}%
\pgfpathlineto{\pgfqpoint{2.770433in}{0.931998in}}%
\pgfpathlineto{\pgfqpoint{2.704197in}{0.949135in}}%
\pgfpathlineto{\pgfqpoint{2.637401in}{0.963942in}}%
\pgfpathlineto{\pgfqpoint{2.570118in}{0.976345in}}%
\pgfpathlineto{\pgfqpoint{2.502440in}{0.986376in}}%
\pgfpathlineto{\pgfqpoint{2.434468in}{0.994184in}}%
\pgfpathlineto{\pgfqpoint{2.366298in}{1.000048in}}%
\pgfpathlineto{\pgfqpoint{2.298009in}{1.004375in}}%
\pgfpathlineto{\pgfqpoint{2.229663in}{1.007727in}}%
\pgfpathlineto{\pgfqpoint{2.161305in}{1.010817in}}%
\pgfpathlineto{\pgfqpoint{2.092979in}{1.014529in}}%
\pgfpathlineto{\pgfqpoint{2.024781in}{1.019985in}}%
\pgfpathlineto{\pgfqpoint{1.956939in}{1.028673in}}%
\pgfpathlineto{\pgfqpoint{1.890048in}{1.042674in}}%
\pgfpathlineto{\pgfqpoint{1.825602in}{1.064972in}}%
\pgfpathlineto{\pgfqpoint{1.767174in}{1.099454in}}%
\pgfpathlineto{\pgfqpoint{1.767174in}{1.099454in}}%
\pgfpathlineto{\pgfqpoint{1.730894in}{1.135033in}}%
\pgfpathlineto{\pgfqpoint{1.704642in}{1.179057in}}%
\pgfpathlineto{\pgfqpoint{1.690534in}{1.225405in}}%
\pgfpathlineto{\pgfqpoint{1.685784in}{1.275504in}}%
\pgfpathlineto{\pgfqpoint{1.689771in}{1.331863in}}%
\pgfusepath{stroke}%
\end{pgfscope}%
\begin{pgfscope}%
\pgfpathrectangle{\pgfqpoint{0.647939in}{0.492442in}}{\pgfqpoint{3.079299in}{3.079299in}}%
\pgfusepath{clip}%
\pgfsetbuttcap%
\pgfsetroundjoin%
\pgfsetlinewidth{0.301125pt}%
\definecolor{currentstroke}{rgb}{0.500000,0.500000,0.500000}%
\pgfsetstrokecolor{currentstroke}%
\pgfsetstrokeopacity{0.300000}%
\pgfsetdash{}{0pt}%
\pgfpathmoveto{\pgfqpoint{3.727238in}{0.632410in}}%
\pgfpathlineto{\pgfqpoint{3.727238in}{0.632410in}}%
\pgfpathlineto{\pgfqpoint{3.662651in}{0.655000in}}%
\pgfpathlineto{\pgfqpoint{3.598609in}{0.679093in}}%
\pgfpathlineto{\pgfqpoint{3.535064in}{0.704472in}}%
\pgfpathlineto{\pgfqpoint{3.471948in}{0.730903in}}%
\pgfpathlineto{\pgfqpoint{3.409177in}{0.758145in}}%
\pgfpathlineto{\pgfqpoint{3.346653in}{0.785950in}}%
\pgfpathlineto{\pgfqpoint{3.284266in}{0.814062in}}%
\pgfpathlineto{\pgfqpoint{3.221899in}{0.842220in}}%
\pgfpathlineto{\pgfqpoint{3.159433in}{0.870155in}}%
\pgfpathlineto{\pgfqpoint{3.096746in}{0.897590in}}%
\pgfpathlineto{\pgfqpoint{3.033722in}{0.924239in}}%
\pgfpathlineto{\pgfqpoint{2.970255in}{0.949808in}}%
\pgfpathlineto{\pgfqpoint{2.906252in}{0.974001in}}%
\pgfpathlineto{\pgfqpoint{2.841646in}{0.996523in}}%
\pgfpathlineto{\pgfqpoint{2.776395in}{1.017099in}}%
\pgfpathlineto{\pgfqpoint{2.710495in}{1.035481in}}%
\pgfpathlineto{\pgfqpoint{2.643977in}{1.051481in}}%
\pgfpathlineto{\pgfqpoint{2.576908in}{1.064986in}}%
\pgfpathlineto{\pgfqpoint{2.509383in}{1.075997in}}%
\pgfpathlineto{\pgfqpoint{2.441513in}{1.084644in}}%
\pgfpathlineto{\pgfqpoint{2.373407in}{1.091194in}}%
\pgfpathlineto{\pgfqpoint{2.305156in}{1.096060in}}%
\pgfpathlineto{\pgfqpoint{2.236831in}{1.099811in}}%
\pgfpathlineto{\pgfqpoint{2.168486in}{1.103199in}}%
\pgfpathlineto{\pgfqpoint{2.100178in}{1.107196in}}%
\pgfpathlineto{\pgfqpoint{2.032016in}{1.113062in}}%
\pgfpathlineto{\pgfqpoint{1.964290in}{1.122513in}}%
\pgfpathlineto{\pgfqpoint{1.897774in}{1.138041in}}%
\pgfpathlineto{\pgfqpoint{1.834552in}{1.163345in}}%
\pgfpathlineto{\pgfqpoint{1.834552in}{1.163345in}}%
\pgfpathlineto{\pgfqpoint{1.789595in}{1.193481in}}%
\pgfpathlineto{\pgfqpoint{1.753701in}{1.235154in}}%
\pgfpathlineto{\pgfqpoint{1.733790in}{1.278472in}}%
\pgfusepath{stroke}%
\end{pgfscope}%
\begin{pgfscope}%
\pgfpathrectangle{\pgfqpoint{0.647939in}{0.492442in}}{\pgfqpoint{3.079299in}{3.079299in}}%
\pgfusepath{clip}%
\pgfsetbuttcap%
\pgfsetroundjoin%
\pgfsetlinewidth{0.301125pt}%
\definecolor{currentstroke}{rgb}{0.500000,0.500000,0.500000}%
\pgfsetstrokecolor{currentstroke}%
\pgfsetstrokeopacity{0.300000}%
\pgfsetdash{}{0pt}%
\pgfpathmoveto{\pgfqpoint{3.727238in}{0.702394in}}%
\pgfpathlineto{\pgfqpoint{3.727238in}{0.702394in}}%
\pgfpathlineto{\pgfqpoint{3.662910in}{0.725708in}}%
\pgfpathlineto{\pgfqpoint{3.599165in}{0.750577in}}%
\pgfpathlineto{\pgfqpoint{3.535957in}{0.776783in}}%
\pgfpathlineto{\pgfqpoint{3.473215in}{0.804090in}}%
\pgfpathlineto{\pgfqpoint{3.410855in}{0.832260in}}%
\pgfpathlineto{\pgfqpoint{3.348775in}{0.861044in}}%
\pgfpathlineto{\pgfqpoint{3.286864in}{0.890189in}}%
\pgfpathlineto{\pgfqpoint{3.225001in}{0.919437in}}%
\pgfpathlineto{\pgfqpoint{3.163062in}{0.948522in}}%
\pgfpathlineto{\pgfqpoint{3.100919in}{0.977169in}}%
\pgfpathlineto{\pgfqpoint{3.038449in}{1.005090in}}%
\pgfpathlineto{\pgfqpoint{2.975534in}{1.031992in}}%
\pgfpathlineto{\pgfqpoint{2.912072in}{1.057569in}}%
\pgfpathlineto{\pgfqpoint{2.847979in}{1.081517in}}%
\pgfpathlineto{\pgfqpoint{2.783202in}{1.103538in}}%
\pgfpathlineto{\pgfqpoint{2.717722in}{1.123360in}}%
\pgfpathlineto{\pgfqpoint{2.651558in}{1.140760in}}%
\pgfpathlineto{\pgfqpoint{2.584771in}{1.155587in}}%
\pgfpathlineto{\pgfqpoint{2.517455in}{1.167797in}}%
\pgfpathlineto{\pgfqpoint{2.449726in}{1.177478in}}%
\pgfpathlineto{\pgfqpoint{2.381708in}{1.184882in}}%
\pgfpathlineto{\pgfqpoint{2.313510in}{1.190426in}}%
\pgfpathlineto{\pgfqpoint{2.245217in}{1.194698in}}%
\pgfpathlineto{\pgfqpoint{2.176893in}{1.198488in}}%
\pgfpathlineto{\pgfqpoint{2.108605in}{1.202843in}}%
\pgfpathlineto{\pgfqpoint{2.040490in}{1.209199in}}%
\pgfpathlineto{\pgfqpoint{1.972919in}{1.219599in}}%
\pgfpathlineto{\pgfqpoint{1.906977in}{1.237136in}}%
\pgfpathlineto{\pgfqpoint{1.845844in}{1.266529in}}%
\pgfpathlineto{\pgfqpoint{1.845844in}{1.266529in}}%
\pgfpathlineto{\pgfqpoint{1.808826in}{1.297431in}}%
\pgfpathlineto{\pgfqpoint{1.781634in}{1.338171in}}%
\pgfpathlineto{\pgfqpoint{1.768164in}{1.380136in}}%
\pgfpathlineto{\pgfqpoint{1.764372in}{1.425281in}}%
\pgfpathlineto{\pgfqpoint{1.769339in}{1.475235in}}%
\pgfpathlineto{\pgfqpoint{1.783706in}{1.531729in}}%
\pgfpathlineto{\pgfqpoint{1.808582in}{1.594969in}}%
\pgfusepath{stroke}%
\end{pgfscope}%
\begin{pgfscope}%
\pgfpathrectangle{\pgfqpoint{0.647939in}{0.492442in}}{\pgfqpoint{3.079299in}{3.079299in}}%
\pgfusepath{clip}%
\pgfsetbuttcap%
\pgfsetroundjoin%
\pgfsetlinewidth{0.301125pt}%
\definecolor{currentstroke}{rgb}{0.500000,0.500000,0.500000}%
\pgfsetstrokecolor{currentstroke}%
\pgfsetstrokeopacity{0.300000}%
\pgfsetdash{}{0pt}%
\pgfpathmoveto{\pgfqpoint{3.727238in}{0.772378in}}%
\pgfpathlineto{\pgfqpoint{3.727238in}{0.772378in}}%
\pgfpathlineto{\pgfqpoint{3.663194in}{0.796461in}}%
\pgfpathlineto{\pgfqpoint{3.599778in}{0.822155in}}%
\pgfpathlineto{\pgfqpoint{3.536941in}{0.849238in}}%
\pgfpathlineto{\pgfqpoint{3.474614in}{0.877478in}}%
\pgfpathlineto{\pgfqpoint{3.412710in}{0.906635in}}%
\pgfpathlineto{\pgfqpoint{3.351126in}{0.936465in}}%
\pgfpathlineto{\pgfqpoint{3.289749in}{0.966718in}}%
\pgfpathlineto{\pgfqpoint{3.228455in}{0.997140in}}%
\pgfpathlineto{\pgfqpoint{3.167115in}{1.027469in}}%
\pgfpathlineto{\pgfqpoint{3.105597in}{1.057434in}}%
\pgfpathlineto{\pgfqpoint{3.043769in}{1.086752in}}%
\pgfpathlineto{\pgfqpoint{2.981503in}{1.115127in}}%
\pgfpathlineto{\pgfqpoint{2.918685in}{1.142253in}}%
\pgfpathlineto{\pgfqpoint{2.855217in}{1.167815in}}%
\pgfpathlineto{\pgfqpoint{2.791027in}{1.191496in}}%
\pgfpathlineto{\pgfqpoint{2.726077in}{1.212997in}}%
\pgfpathlineto{\pgfqpoint{2.660372in}{1.232056in}}%
\pgfpathlineto{\pgfqpoint{2.593961in}{1.248480in}}%
\pgfpathlineto{\pgfqpoint{2.526935in}{1.262173in}}%
\pgfpathlineto{\pgfqpoint{2.459412in}{1.273175in}}%
\pgfpathlineto{\pgfqpoint{2.391526in}{1.281688in}}%
\pgfpathlineto{\pgfqpoint{2.323406in}{1.288110in}}%
\pgfpathlineto{\pgfqpoint{2.255159in}{1.293061in}}%
\pgfpathlineto{\pgfqpoint{2.186867in}{1.297387in}}%
\pgfpathlineto{\pgfqpoint{2.118614in}{1.302242in}}%
\pgfpathlineto{\pgfqpoint{2.050565in}{1.309254in}}%
\pgfpathlineto{\pgfqpoint{1.983213in}{1.320899in}}%
\pgfpathlineto{\pgfqpoint{1.918178in}{1.341230in}}%
\pgfpathlineto{\pgfqpoint{1.918178in}{1.341230in}}%
\pgfusepath{stroke}%
\end{pgfscope}%
\begin{pgfscope}%
\pgfpathrectangle{\pgfqpoint{0.647939in}{0.492442in}}{\pgfqpoint{3.079299in}{3.079299in}}%
\pgfusepath{clip}%
\pgfsetbuttcap%
\pgfsetroundjoin%
\pgfsetlinewidth{0.301125pt}%
\definecolor{currentstroke}{rgb}{0.500000,0.500000,0.500000}%
\pgfsetstrokecolor{currentstroke}%
\pgfsetstrokeopacity{0.300000}%
\pgfsetdash{}{0pt}%
\pgfpathmoveto{\pgfqpoint{3.727238in}{0.842362in}}%
\pgfpathlineto{\pgfqpoint{3.727238in}{0.842362in}}%
\pgfpathlineto{\pgfqpoint{3.663508in}{0.867263in}}%
\pgfpathlineto{\pgfqpoint{3.600454in}{0.893833in}}%
\pgfpathlineto{\pgfqpoint{3.538029in}{0.921851in}}%
\pgfpathlineto{\pgfqpoint{3.476162in}{0.951084in}}%
\pgfpathlineto{\pgfqpoint{3.414766in}{0.981296in}}%
\pgfpathlineto{\pgfqpoint{3.353737in}{1.012245in}}%
\pgfpathlineto{\pgfqpoint{3.292960in}{1.043686in}}%
\pgfpathlineto{\pgfqpoint{3.232308in}{1.075369in}}%
\pgfpathlineto{\pgfqpoint{3.171651in}{1.107040in}}%
\pgfpathlineto{\pgfqpoint{3.110850in}{1.138435in}}%
\pgfpathlineto{\pgfqpoint{3.049769in}{1.169279in}}%
\pgfpathlineto{\pgfqpoint{2.988272in}{1.199282in}}%
\pgfpathlineto{\pgfqpoint{2.926232in}{1.228139in}}%
\pgfpathlineto{\pgfqpoint{2.863533in}{1.255530in}}%
\pgfpathlineto{\pgfqpoint{2.800084in}{1.281127in}}%
\pgfpathlineto{\pgfqpoint{2.735822in}{1.304604in}}%
\pgfpathlineto{\pgfqpoint{2.670728in}{1.325658in}}%
\pgfpathlineto{\pgfqpoint{2.604830in}{1.344038in}}%
\pgfpathlineto{\pgfqpoint{2.538210in}{1.359591in}}%
\pgfpathlineto{\pgfqpoint{2.470989in}{1.372292in}}%
\pgfpathlineto{\pgfqpoint{2.403310in}{1.382290in}}%
\pgfpathlineto{\pgfqpoint{2.335320in}{1.389942in}}%
\pgfpathlineto{\pgfqpoint{2.267150in}{1.395842in}}%
\pgfpathlineto{\pgfqpoint{2.198909in}{1.400896in}}%
\pgfpathlineto{\pgfqpoint{2.130707in}{1.406419in}}%
\pgfpathlineto{\pgfqpoint{2.062769in}{1.414347in}}%
\pgfpathlineto{\pgfqpoint{1.995813in}{1.427830in}}%
\pgfpathlineto{\pgfqpoint{1.932562in}{1.452496in}}%
\pgfpathlineto{\pgfqpoint{1.932562in}{1.452496in}}%
\pgfpathlineto{\pgfqpoint{1.896696in}{1.478384in}}%
\pgfpathlineto{\pgfqpoint{1.896696in}{1.478384in}}%
\pgfpathlineto{\pgfqpoint{1.873342in}{1.509112in}}%
\pgfpathlineto{\pgfqpoint{1.860639in}{1.545783in}}%
\pgfusepath{stroke}%
\end{pgfscope}%
\begin{pgfscope}%
\pgfpathrectangle{\pgfqpoint{0.647939in}{0.492442in}}{\pgfqpoint{3.079299in}{3.079299in}}%
\pgfusepath{clip}%
\pgfsetbuttcap%
\pgfsetroundjoin%
\pgfsetlinewidth{0.301125pt}%
\definecolor{currentstroke}{rgb}{0.500000,0.500000,0.500000}%
\pgfsetstrokecolor{currentstroke}%
\pgfsetstrokeopacity{0.300000}%
\pgfsetdash{}{0pt}%
\pgfpathmoveto{\pgfqpoint{3.727238in}{0.912347in}}%
\pgfpathlineto{\pgfqpoint{3.727238in}{0.912347in}}%
\pgfpathlineto{\pgfqpoint{3.663856in}{0.938118in}}%
\pgfpathlineto{\pgfqpoint{3.601204in}{0.965621in}}%
\pgfpathlineto{\pgfqpoint{3.539236in}{0.994635in}}%
\pgfpathlineto{\pgfqpoint{3.477881in}{1.024928in}}%
\pgfpathlineto{\pgfqpoint{3.417052in}{1.056265in}}%
\pgfpathlineto{\pgfqpoint{3.356645in}{1.088410in}}%
\pgfpathlineto{\pgfqpoint{3.296543in}{1.121122in}}%
\pgfpathlineto{\pgfqpoint{3.236620in}{1.154164in}}%
\pgfpathlineto{\pgfqpoint{3.176745in}{1.187289in}}%
\pgfpathlineto{\pgfqpoint{3.116777in}{1.220247in}}%
\pgfpathlineto{\pgfqpoint{3.056576in}{1.252773in}}%
\pgfpathlineto{\pgfqpoint{2.995997in}{1.284589in}}%
\pgfpathlineto{\pgfqpoint{2.934903in}{1.315397in}}%
\pgfpathlineto{\pgfqpoint{2.873161in}{1.344881in}}%
\pgfpathlineto{\pgfqpoint{2.810658in}{1.372708in}}%
\pgfpathlineto{\pgfqpoint{2.747305in}{1.398535in}}%
\pgfpathlineto{\pgfqpoint{2.683052in}{1.422023in}}%
\pgfpathlineto{\pgfqpoint{2.617896in}{1.442867in}}%
\pgfpathlineto{\pgfqpoint{2.551889in}{1.460831in}}%
\pgfpathlineto{\pgfqpoint{2.485137in}{1.475795in}}%
\pgfpathlineto{\pgfqpoint{2.417789in}{1.487819in}}%
\pgfpathlineto{\pgfqpoint{2.350020in}{1.497206in}}%
\pgfpathlineto{\pgfqpoint{2.281991in}{1.504538in}}%
\pgfpathlineto{\pgfqpoint{2.213846in}{1.510753in}}%
\pgfpathlineto{\pgfqpoint{2.145735in}{1.517305in}}%
\pgfpathlineto{\pgfqpoint{2.077972in}{1.526573in}}%
\pgfpathlineto{\pgfqpoint{2.011770in}{1.542960in}}%
\pgfpathlineto{\pgfqpoint{2.011770in}{1.542960in}}%
\pgfpathlineto{\pgfqpoint{1.966611in}{1.564077in}}%
\pgfpathlineto{\pgfqpoint{1.966611in}{1.564077in}}%
\pgfpathlineto{\pgfqpoint{1.937647in}{1.589176in}}%
\pgfpathlineto{\pgfqpoint{1.919318in}{1.623654in}}%
\pgfpathlineto{\pgfqpoint{1.914305in}{1.657446in}}%
\pgfpathlineto{\pgfqpoint{1.918187in}{1.693090in}}%
\pgfusepath{stroke}%
\end{pgfscope}%
\begin{pgfscope}%
\pgfpathrectangle{\pgfqpoint{0.647939in}{0.492442in}}{\pgfqpoint{3.079299in}{3.079299in}}%
\pgfusepath{clip}%
\pgfsetbuttcap%
\pgfsetroundjoin%
\pgfsetlinewidth{0.301125pt}%
\definecolor{currentstroke}{rgb}{0.500000,0.500000,0.500000}%
\pgfsetstrokecolor{currentstroke}%
\pgfsetstrokeopacity{0.300000}%
\pgfsetdash{}{0pt}%
\pgfpathmoveto{\pgfqpoint{3.727238in}{0.982331in}}%
\pgfpathlineto{\pgfqpoint{3.727238in}{0.982331in}}%
\pgfpathlineto{\pgfqpoint{3.664242in}{1.009030in}}%
\pgfpathlineto{\pgfqpoint{3.602038in}{1.037529in}}%
\pgfpathlineto{\pgfqpoint{3.540579in}{1.067605in}}%
\pgfpathlineto{\pgfqpoint{3.479796in}{1.099028in}}%
\pgfpathlineto{\pgfqpoint{3.419602in}{1.131567in}}%
\pgfpathlineto{\pgfqpoint{3.359894in}{1.164991in}}%
\pgfpathlineto{\pgfqpoint{3.300557in}{1.199072in}}%
\pgfpathlineto{\pgfqpoint{3.241469in}{1.233581in}}%
\pgfpathlineto{\pgfqpoint{3.182496in}{1.268289in}}%
\pgfpathlineto{\pgfqpoint{3.123501in}{1.302959in}}%
\pgfpathlineto{\pgfqpoint{3.064341in}{1.337344in}}%
\pgfpathlineto{\pgfqpoint{3.004868in}{1.371184in}}%
\pgfpathlineto{\pgfqpoint{2.944936in}{1.404200in}}%
\pgfpathlineto{\pgfqpoint{2.884399in}{1.436089in}}%
\pgfpathlineto{\pgfqpoint{2.823121in}{1.466527in}}%
\pgfpathlineto{\pgfqpoint{2.760985in}{1.495168in}}%
\pgfpathlineto{\pgfqpoint{2.697904in}{1.521651in}}%
\pgfpathlineto{\pgfqpoint{2.633830in}{1.545626in}}%
\pgfpathlineto{\pgfqpoint{2.568776in}{1.566780in}}%
\pgfpathlineto{\pgfqpoint{2.502812in}{1.584892in}}%
\pgfpathlineto{\pgfqpoint{2.436072in}{1.599896in}}%
\pgfpathlineto{\pgfqpoint{2.368732in}{1.611955in}}%
\pgfpathlineto{\pgfqpoint{2.300991in}{1.621560in}}%
\pgfpathlineto{\pgfqpoint{2.233046in}{1.629663in}}%
\pgfpathlineto{\pgfqpoint{2.165132in}{1.637993in}}%
\pgfpathlineto{\pgfqpoint{2.097800in}{1.649771in}}%
\pgfpathlineto{\pgfqpoint{2.033833in}{1.672155in}}%
\pgfpathlineto{\pgfqpoint{2.033833in}{1.672155in}}%
\pgfpathlineto{\pgfqpoint{2.004705in}{1.692756in}}%
\pgfpathlineto{\pgfqpoint{2.004705in}{1.692756in}}%
\pgfpathlineto{\pgfqpoint{1.987547in}{1.717822in}}%
\pgfpathlineto{\pgfqpoint{1.981432in}{1.747766in}}%
\pgfpathlineto{\pgfqpoint{1.984727in}{1.777514in}}%
\pgfpathlineto{\pgfqpoint{1.996364in}{1.810806in}}%
\pgfpathlineto{\pgfqpoint{2.017332in}{1.848512in}}%
\pgfusepath{stroke}%
\end{pgfscope}%
\begin{pgfscope}%
\pgfpathrectangle{\pgfqpoint{0.647939in}{0.492442in}}{\pgfqpoint{3.079299in}{3.079299in}}%
\pgfusepath{clip}%
\pgfsetbuttcap%
\pgfsetroundjoin%
\pgfsetlinewidth{0.301125pt}%
\definecolor{currentstroke}{rgb}{0.500000,0.500000,0.500000}%
\pgfsetstrokecolor{currentstroke}%
\pgfsetstrokeopacity{0.300000}%
\pgfsetdash{}{0pt}%
\pgfpathmoveto{\pgfqpoint{3.727238in}{1.052315in}}%
\pgfpathlineto{\pgfqpoint{3.727238in}{1.052315in}}%
\pgfpathlineto{\pgfqpoint{3.664672in}{1.080006in}}%
\pgfpathlineto{\pgfqpoint{3.602966in}{1.109568in}}%
\pgfpathlineto{\pgfqpoint{3.542075in}{1.140777in}}%
\pgfpathlineto{\pgfqpoint{3.481932in}{1.173406in}}%
\pgfpathlineto{\pgfqpoint{3.422451in}{1.207231in}}%
\pgfpathlineto{\pgfqpoint{3.363533in}{1.242030in}}%
\pgfpathlineto{\pgfqpoint{3.305069in}{1.277586in}}%
\pgfpathlineto{\pgfqpoint{3.246939in}{1.313685in}}%
\pgfpathlineto{\pgfqpoint{3.189015in}{1.350116in}}%
\pgfpathlineto{\pgfqpoint{3.131165in}{1.386663in}}%
\pgfpathlineto{\pgfqpoint{3.073248in}{1.423104in}}%
\pgfpathlineto{\pgfqpoint{3.015119in}{1.459205in}}%
\pgfpathlineto{\pgfqpoint{2.956628in}{1.494715in}}%
\pgfpathlineto{\pgfqpoint{2.897627in}{1.529365in}}%
\pgfpathlineto{\pgfqpoint{2.837969in}{1.562864in}}%
\pgfpathlineto{\pgfqpoint{2.777514in}{1.594896in}}%
\pgfpathlineto{\pgfqpoint{2.716136in}{1.625114in}}%
\pgfpathlineto{\pgfqpoint{2.653739in}{1.653158in}}%
\pgfpathlineto{\pgfqpoint{2.590267in}{1.678666in}}%
\pgfpathlineto{\pgfqpoint{2.525723in}{1.701318in}}%
\pgfpathlineto{\pgfqpoint{2.460182in}{1.720898in}}%
\pgfpathlineto{\pgfqpoint{2.393792in}{1.737385in}}%
\pgfpathlineto{\pgfqpoint{2.326767in}{1.751107in}}%
\pgfpathlineto{\pgfqpoint{2.259388in}{1.763029in}}%
\pgfpathlineto{\pgfqpoint{2.192083in}{1.775311in}}%
\pgfpathlineto{\pgfqpoint{2.126356in}{1.793378in}}%
\pgfpathlineto{\pgfqpoint{2.126356in}{1.793378in}}%
\pgfpathlineto{\pgfqpoint{2.094006in}{1.809841in}}%
\pgfpathlineto{\pgfqpoint{2.094006in}{1.809841in}}%
\pgfusepath{stroke}%
\end{pgfscope}%
\begin{pgfscope}%
\pgfpathrectangle{\pgfqpoint{0.647939in}{0.492442in}}{\pgfqpoint{3.079299in}{3.079299in}}%
\pgfusepath{clip}%
\pgfsetbuttcap%
\pgfsetroundjoin%
\pgfsetlinewidth{0.301125pt}%
\definecolor{currentstroke}{rgb}{0.500000,0.500000,0.500000}%
\pgfsetstrokecolor{currentstroke}%
\pgfsetstrokeopacity{0.300000}%
\pgfsetdash{}{0pt}%
\pgfpathmoveto{\pgfqpoint{3.727238in}{1.122299in}}%
\pgfpathlineto{\pgfqpoint{3.727238in}{1.122299in}}%
\pgfpathlineto{\pgfqpoint{3.665153in}{1.151050in}}%
\pgfpathlineto{\pgfqpoint{3.604005in}{1.181747in}}%
\pgfpathlineto{\pgfqpoint{3.543751in}{1.214167in}}%
\pgfpathlineto{\pgfqpoint{3.484327in}{1.248087in}}%
\pgfpathlineto{\pgfqpoint{3.425653in}{1.283291in}}%
\pgfpathlineto{\pgfqpoint{3.367635in}{1.319567in}}%
\pgfpathlineto{\pgfqpoint{3.310169in}{1.356713in}}%
\pgfpathlineto{\pgfqpoint{3.253145in}{1.394536in}}%
\pgfpathlineto{\pgfqpoint{3.196445in}{1.432844in}}%
\pgfpathlineto{\pgfqpoint{3.139946in}{1.471446in}}%
\pgfpathlineto{\pgfqpoint{3.083517in}{1.510150in}}%
\pgfpathlineto{\pgfqpoint{3.027025in}{1.548762in}}%
\pgfpathlineto{\pgfqpoint{2.970336in}{1.587083in}}%
\pgfpathlineto{\pgfqpoint{2.913314in}{1.624903in}}%
\pgfpathlineto{\pgfqpoint{2.855816in}{1.661995in}}%
\pgfpathlineto{\pgfqpoint{2.797703in}{1.698112in}}%
\pgfpathlineto{\pgfqpoint{2.738839in}{1.732989in}}%
\pgfpathlineto{\pgfqpoint{2.679104in}{1.766344in}}%
\pgfpathlineto{\pgfqpoint{2.618396in}{1.797887in}}%
\pgfpathlineto{\pgfqpoint{2.556650in}{1.827341in}}%
\pgfpathlineto{\pgfqpoint{2.493862in}{1.854492in}}%
\pgfpathlineto{\pgfqpoint{2.430112in}{1.879299in}}%
\pgfpathlineto{\pgfqpoint{2.365622in}{1.902121in}}%
\pgfpathlineto{\pgfqpoint{2.300921in}{1.924367in}}%
\pgfpathlineto{\pgfqpoint{2.238231in}{1.951231in}}%
\pgfpathlineto{\pgfqpoint{2.238231in}{1.951231in}}%
\pgfpathlineto{\pgfqpoint{2.214782in}{1.967565in}}%
\pgfpathlineto{\pgfqpoint{2.214782in}{1.967565in}}%
\pgfpathlineto{\pgfqpoint{2.202080in}{1.986703in}}%
\pgfusepath{stroke}%
\end{pgfscope}%
\begin{pgfscope}%
\pgfpathrectangle{\pgfqpoint{0.647939in}{0.492442in}}{\pgfqpoint{3.079299in}{3.079299in}}%
\pgfusepath{clip}%
\pgfsetbuttcap%
\pgfsetroundjoin%
\pgfsetlinewidth{0.301125pt}%
\definecolor{currentstroke}{rgb}{0.500000,0.500000,0.500000}%
\pgfsetstrokecolor{currentstroke}%
\pgfsetstrokeopacity{0.300000}%
\pgfsetdash{}{0pt}%
\pgfpathmoveto{\pgfqpoint{3.727238in}{1.262267in}}%
\pgfpathlineto{\pgfqpoint{3.727238in}{1.262267in}}%
\pgfpathlineto{\pgfqpoint{3.666300in}{1.293371in}}%
\pgfpathlineto{\pgfqpoint{3.606484in}{1.326583in}}%
\pgfpathlineto{\pgfqpoint{3.547755in}{1.361687in}}%
\pgfpathlineto{\pgfqpoint{3.490064in}{1.398473in}}%
\pgfpathlineto{\pgfqpoint{3.433346in}{1.436745in}}%
\pgfpathlineto{\pgfqpoint{3.377529in}{1.476321in}}%
\pgfpathlineto{\pgfqpoint{3.322536in}{1.517037in}}%
\pgfpathlineto{\pgfqpoint{3.268287in}{1.558738in}}%
\pgfpathlineto{\pgfqpoint{3.214699in}{1.601287in}}%
\pgfpathlineto{\pgfqpoint{3.161703in}{1.644572in}}%
\pgfpathlineto{\pgfqpoint{3.109237in}{1.688496in}}%
\pgfpathlineto{\pgfqpoint{3.057244in}{1.732980in}}%
\pgfpathlineto{\pgfqpoint{3.005690in}{1.777970in}}%
\pgfpathlineto{\pgfqpoint{2.954566in}{1.823447in}}%
\pgfpathlineto{\pgfqpoint{2.903899in}{1.869430in}}%
\pgfpathlineto{\pgfqpoint{2.853776in}{1.916004in}}%
\pgfpathlineto{\pgfqpoint{2.804385in}{1.963348in}}%
\pgfpathlineto{\pgfqpoint{2.756080in}{2.011792in}}%
\pgfpathlineto{\pgfqpoint{2.709532in}{2.061907in}}%
\pgfpathlineto{\pgfqpoint{2.666038in}{2.114653in}}%
\pgfpathlineto{\pgfqpoint{2.628209in}{2.171488in}}%
\pgfpathlineto{\pgfqpoint{2.601099in}{2.233847in}}%
\pgfpathlineto{\pgfqpoint{2.601099in}{2.233847in}}%
\pgfpathlineto{\pgfqpoint{2.591440in}{2.287262in}}%
\pgfpathlineto{\pgfqpoint{2.594325in}{2.341857in}}%
\pgfpathlineto{\pgfqpoint{2.607628in}{2.395092in}}%
\pgfpathlineto{\pgfqpoint{2.631371in}{2.453556in}}%
\pgfpathlineto{\pgfqpoint{2.662993in}{2.514093in}}%
\pgfusepath{stroke}%
\end{pgfscope}%
\begin{pgfscope}%
\pgfpathrectangle{\pgfqpoint{0.647939in}{0.492442in}}{\pgfqpoint{3.079299in}{3.079299in}}%
\pgfusepath{clip}%
\pgfsetbuttcap%
\pgfsetroundjoin%
\pgfsetlinewidth{0.301125pt}%
\definecolor{currentstroke}{rgb}{0.500000,0.500000,0.500000}%
\pgfsetstrokecolor{currentstroke}%
\pgfsetstrokeopacity{0.300000}%
\pgfsetdash{}{0pt}%
\pgfpathmoveto{\pgfqpoint{3.727238in}{1.402235in}}%
\pgfpathlineto{\pgfqpoint{3.727238in}{1.402235in}}%
\pgfpathlineto{\pgfqpoint{3.667765in}{1.436049in}}%
\pgfpathlineto{\pgfqpoint{3.609652in}{1.472153in}}%
\pgfpathlineto{\pgfqpoint{3.552886in}{1.510342in}}%
\pgfpathlineto{\pgfqpoint{3.497438in}{1.550425in}}%
\pgfpathlineto{\pgfqpoint{3.443277in}{1.592233in}}%
\pgfpathlineto{\pgfqpoint{3.390369in}{1.635616in}}%
\pgfpathlineto{\pgfqpoint{3.338682in}{1.680449in}}%
\pgfpathlineto{\pgfqpoint{3.288206in}{1.726640in}}%
\pgfpathlineto{\pgfqpoint{3.238949in}{1.774128in}}%
\pgfpathlineto{\pgfqpoint{3.190945in}{1.822882in}}%
\pgfpathlineto{\pgfqpoint{3.144277in}{1.872915in}}%
\pgfpathlineto{\pgfqpoint{3.099083in}{1.924283in}}%
\pgfpathlineto{\pgfqpoint{3.055586in}{1.977091in}}%
\pgfpathlineto{\pgfqpoint{3.014131in}{2.031506in}}%
\pgfpathlineto{\pgfqpoint{2.975225in}{2.087762in}}%
\pgfpathlineto{\pgfqpoint{2.939615in}{2.146147in}}%
\pgfpathlineto{\pgfqpoint{2.908367in}{2.206955in}}%
\pgfpathlineto{\pgfqpoint{2.882907in}{2.270366in}}%
\pgfpathlineto{\pgfqpoint{2.864898in}{2.336226in}}%
\pgfpathlineto{\pgfqpoint{2.855807in}{2.403845in}}%
\pgfpathlineto{\pgfqpoint{2.856237in}{2.472061in}}%
\pgfpathlineto{\pgfqpoint{2.865618in}{2.539671in}}%
\pgfpathlineto{\pgfqpoint{2.882540in}{2.605852in}}%
\pgfpathlineto{\pgfqpoint{2.905373in}{2.670277in}}%
\pgfpathlineto{\pgfqpoint{2.932705in}{2.732954in}}%
\pgfpathlineto{\pgfqpoint{2.963456in}{2.794045in}}%
\pgfpathlineto{\pgfqpoint{2.996853in}{2.853741in}}%
\pgfpathlineto{\pgfqpoint{3.032357in}{2.912213in}}%
\pgfpathlineto{\pgfqpoint{3.069597in}{2.969598in}}%
\pgfpathlineto{\pgfqpoint{3.108327in}{3.025993in}}%
\pgfpathlineto{\pgfqpoint{3.148387in}{3.081453in}}%
\pgfpathlineto{\pgfqpoint{3.189678in}{3.136009in}}%
\pgfpathlineto{\pgfqpoint{3.232136in}{3.189663in}}%
\pgfpathlineto{\pgfqpoint{3.275742in}{3.242387in}}%
\pgfpathlineto{\pgfqpoint{3.320502in}{3.294136in}}%
\pgfusepath{stroke}%
\end{pgfscope}%
\begin{pgfscope}%
\pgfpathrectangle{\pgfqpoint{0.647939in}{0.492442in}}{\pgfqpoint{3.079299in}{3.079299in}}%
\pgfusepath{clip}%
\pgfsetbuttcap%
\pgfsetroundjoin%
\pgfsetlinewidth{0.301125pt}%
\definecolor{currentstroke}{rgb}{0.500000,0.500000,0.500000}%
\pgfsetstrokecolor{currentstroke}%
\pgfsetstrokeopacity{0.300000}%
\pgfsetdash{}{0pt}%
\pgfpathmoveto{\pgfqpoint{3.727238in}{1.542203in}}%
\pgfpathlineto{\pgfqpoint{3.727238in}{1.542203in}}%
\pgfpathlineto{\pgfqpoint{3.669663in}{1.579150in}}%
\pgfpathlineto{\pgfqpoint{3.613763in}{1.618588in}}%
\pgfpathlineto{\pgfqpoint{3.559557in}{1.660324in}}%
\pgfpathlineto{\pgfqpoint{3.507057in}{1.704190in}}%
\pgfpathlineto{\pgfqpoint{3.456281in}{1.750042in}}%
\pgfpathlineto{\pgfqpoint{3.407257in}{1.797764in}}%
\pgfpathlineto{\pgfqpoint{3.360049in}{1.847281in}}%
\pgfpathlineto{\pgfqpoint{3.314752in}{1.898552in}}%
\pgfpathlineto{\pgfqpoint{3.271515in}{1.951569in}}%
\pgfpathlineto{\pgfqpoint{3.230560in}{2.006364in}}%
\pgfpathlineto{\pgfqpoint{3.192191in}{2.062993in}}%
\pgfpathlineto{\pgfqpoint{3.156823in}{2.121534in}}%
\pgfpathlineto{\pgfqpoint{3.125005in}{2.182068in}}%
\pgfpathlineto{\pgfqpoint{3.097429in}{2.244635in}}%
\pgfpathlineto{\pgfqpoint{3.074904in}{2.309173in}}%
\pgfpathlineto{\pgfqpoint{3.058272in}{2.375455in}}%
\pgfpathlineto{\pgfqpoint{3.048254in}{2.443040in}}%
\pgfpathlineto{\pgfqpoint{3.045260in}{2.511297in}}%
\pgfpathlineto{\pgfqpoint{3.049256in}{2.579511in}}%
\pgfpathlineto{\pgfqpoint{3.059778in}{2.647036in}}%
\pgfpathlineto{\pgfqpoint{3.076069in}{2.713419in}}%
\pgfpathlineto{\pgfqpoint{3.097294in}{2.778409in}}%
\pgfpathlineto{\pgfqpoint{3.122662in}{2.841910in}}%
\pgfpathlineto{\pgfqpoint{3.151496in}{2.903925in}}%
\pgfpathlineto{\pgfqpoint{3.183259in}{2.964502in}}%
\pgfpathlineto{\pgfqpoint{3.217537in}{3.023698in}}%
\pgfpathlineto{\pgfqpoint{3.254023in}{3.081565in}}%
\pgfpathlineto{\pgfqpoint{3.292494in}{3.138139in}}%
\pgfusepath{stroke}%
\end{pgfscope}%
\begin{pgfscope}%
\pgfpathrectangle{\pgfqpoint{0.647939in}{0.492442in}}{\pgfqpoint{3.079299in}{3.079299in}}%
\pgfusepath{clip}%
\pgfsetbuttcap%
\pgfsetroundjoin%
\pgfsetlinewidth{0.301125pt}%
\definecolor{currentstroke}{rgb}{0.500000,0.500000,0.500000}%
\pgfsetstrokecolor{currentstroke}%
\pgfsetstrokeopacity{0.300000}%
\pgfsetdash{}{0pt}%
\pgfpathmoveto{\pgfqpoint{3.727238in}{1.682171in}}%
\pgfpathlineto{\pgfqpoint{3.727238in}{1.682171in}}%
\pgfpathlineto{\pgfqpoint{3.672163in}{1.722741in}}%
\pgfpathlineto{\pgfqpoint{3.619182in}{1.766011in}}%
\pgfpathlineto{\pgfqpoint{3.568362in}{1.811801in}}%
\pgfpathlineto{\pgfqpoint{3.519776in}{1.859958in}}%
\pgfpathlineto{\pgfqpoint{3.473520in}{1.910355in}}%
\pgfpathlineto{\pgfqpoint{3.429727in}{1.962905in}}%
\pgfpathlineto{\pgfqpoint{3.388579in}{2.017551in}}%
\pgfpathlineto{\pgfqpoint{3.350323in}{2.074255in}}%
\pgfpathlineto{\pgfqpoint{3.315278in}{2.132992in}}%
\pgfpathlineto{\pgfqpoint{3.283849in}{2.193731in}}%
\pgfpathlineto{\pgfqpoint{3.256522in}{2.256413in}}%
\pgfpathlineto{\pgfqpoint{3.233846in}{2.320915in}}%
\pgfpathlineto{\pgfqpoint{3.216378in}{2.387010in}}%
\pgfpathlineto{\pgfqpoint{3.204618in}{2.454350in}}%
\pgfpathlineto{\pgfqpoint{3.198905in}{2.522461in}}%
\pgfpathlineto{\pgfqpoint{3.199338in}{2.590806in}}%
\pgfpathlineto{\pgfqpoint{3.205756in}{2.658854in}}%
\pgfpathlineto{\pgfqpoint{3.217776in}{2.726150in}}%
\pgfpathlineto{\pgfqpoint{3.234874in}{2.792354in}}%
\pgfpathlineto{\pgfqpoint{3.256481in}{2.857238in}}%
\pgfpathlineto{\pgfqpoint{3.282054in}{2.920672in}}%
\pgfpathlineto{\pgfqpoint{3.311117in}{2.982591in}}%
\pgfpathlineto{\pgfqpoint{3.343272in}{3.042968in}}%
\pgfpathlineto{\pgfqpoint{3.378203in}{3.101787in}}%
\pgfpathlineto{\pgfqpoint{3.415668in}{3.159025in}}%
\pgfpathlineto{\pgfqpoint{3.455497in}{3.214644in}}%
\pgfpathlineto{\pgfqpoint{3.497579in}{3.268578in}}%
\pgfpathlineto{\pgfqpoint{3.541833in}{3.320750in}}%
\pgfpathlineto{\pgfqpoint{3.588220in}{3.371031in}}%
\pgfpathlineto{\pgfqpoint{3.636727in}{3.419269in}}%
\pgfpathlineto{\pgfqpoint{3.687353in}{3.465276in}}%
\pgfpathlineto{\pgfqpoint{3.727238in}{3.499881in}}%
\pgfusepath{stroke}%
\end{pgfscope}%
\begin{pgfscope}%
\pgfpathrectangle{\pgfqpoint{0.647939in}{0.492442in}}{\pgfqpoint{3.079299in}{3.079299in}}%
\pgfusepath{clip}%
\pgfsetbuttcap%
\pgfsetroundjoin%
\pgfsetlinewidth{0.301125pt}%
\definecolor{currentstroke}{rgb}{0.500000,0.500000,0.500000}%
\pgfsetstrokecolor{currentstroke}%
\pgfsetstrokeopacity{0.300000}%
\pgfsetdash{}{0pt}%
\pgfpathmoveto{\pgfqpoint{3.727238in}{1.752155in}}%
\pgfpathlineto{\pgfqpoint{3.727238in}{1.752155in}}%
\pgfpathlineto{\pgfqpoint{3.673709in}{1.794739in}}%
\pgfpathlineto{\pgfqpoint{3.622533in}{1.840123in}}%
\pgfpathlineto{\pgfqpoint{3.573810in}{1.888134in}}%
\pgfpathlineto{\pgfqpoint{3.527652in}{1.938615in}}%
\pgfpathlineto{\pgfqpoint{3.484204in}{1.991444in}}%
\pgfpathlineto{\pgfqpoint{3.443659in}{2.046534in}}%
\pgfusepath{stroke}%
\end{pgfscope}%
\begin{pgfscope}%
\pgfpathrectangle{\pgfqpoint{0.647939in}{0.492442in}}{\pgfqpoint{3.079299in}{3.079299in}}%
\pgfusepath{clip}%
\pgfsetbuttcap%
\pgfsetroundjoin%
\pgfsetlinewidth{0.301125pt}%
\definecolor{currentstroke}{rgb}{0.500000,0.500000,0.500000}%
\pgfsetstrokecolor{currentstroke}%
\pgfsetstrokeopacity{0.300000}%
\pgfsetdash{}{0pt}%
\pgfpathmoveto{\pgfqpoint{3.727238in}{1.822139in}}%
\pgfpathlineto{\pgfqpoint{3.727238in}{1.822139in}}%
\pgfpathlineto{\pgfqpoint{3.675501in}{1.866878in}}%
\pgfpathlineto{\pgfqpoint{3.626416in}{1.914509in}}%
\pgfpathlineto{\pgfqpoint{3.580118in}{1.964854in}}%
\pgfpathlineto{\pgfqpoint{3.536764in}{2.017758in}}%
\pgfpathlineto{\pgfqpoint{3.496557in}{2.073089in}}%
\pgfpathlineto{\pgfqpoint{3.459748in}{2.130730in}}%
\pgfpathlineto{\pgfqpoint{3.426650in}{2.190573in}}%
\pgfpathlineto{\pgfqpoint{3.397631in}{2.252490in}}%
\pgfpathlineto{\pgfqpoint{3.373106in}{2.316314in}}%
\pgfpathlineto{\pgfqpoint{3.353508in}{2.381812in}}%
\pgfpathlineto{\pgfqpoint{3.339239in}{2.448670in}}%
\pgfpathlineto{\pgfqpoint{3.330606in}{2.516486in}}%
\pgfpathlineto{\pgfqpoint{3.327776in}{2.584786in}}%
\pgfpathlineto{\pgfqpoint{3.330727in}{2.653078in}}%
\pgfpathlineto{\pgfqpoint{3.339264in}{2.720903in}}%
\pgfpathlineto{\pgfqpoint{3.353044in}{2.787869in}}%
\pgfpathlineto{\pgfqpoint{3.371652in}{2.853669in}}%
\pgfpathlineto{\pgfqpoint{3.394649in}{2.918076in}}%
\pgfpathlineto{\pgfqpoint{3.421612in}{2.980933in}}%
\pgfpathlineto{\pgfqpoint{3.452171in}{3.042128in}}%
\pgfpathlineto{\pgfqpoint{3.486021in}{3.101567in}}%
\pgfpathlineto{\pgfqpoint{3.522921in}{3.159163in}}%
\pgfpathlineto{\pgfqpoint{3.562682in}{3.214821in}}%
\pgfpathlineto{\pgfqpoint{3.605169in}{3.268428in}}%
\pgfpathlineto{\pgfqpoint{3.650297in}{3.319834in}}%
\pgfpathlineto{\pgfqpoint{3.698001in}{3.368856in}}%
\pgfpathlineto{\pgfqpoint{3.727238in}{3.397387in}}%
\pgfusepath{stroke}%
\end{pgfscope}%
\begin{pgfscope}%
\pgfpathrectangle{\pgfqpoint{0.647939in}{0.492442in}}{\pgfqpoint{3.079299in}{3.079299in}}%
\pgfusepath{clip}%
\pgfsetbuttcap%
\pgfsetroundjoin%
\pgfsetlinewidth{0.301125pt}%
\definecolor{currentstroke}{rgb}{0.500000,0.500000,0.500000}%
\pgfsetstrokecolor{currentstroke}%
\pgfsetstrokeopacity{0.300000}%
\pgfsetdash{}{0pt}%
\pgfpathmoveto{\pgfqpoint{3.727238in}{1.962108in}}%
\pgfpathlineto{\pgfqpoint{3.727238in}{1.962108in}}%
\pgfpathlineto{\pgfqpoint{3.680009in}{2.011566in}}%
\pgfpathlineto{\pgfqpoint{3.636161in}{2.064045in}}%
\pgfpathlineto{\pgfqpoint{3.595916in}{2.119332in}}%
\pgfpathlineto{\pgfqpoint{3.559526in}{2.177229in}}%
\pgfpathlineto{\pgfqpoint{3.527295in}{2.237538in}}%
\pgfpathlineto{\pgfqpoint{3.499562in}{2.300039in}}%
\pgfpathlineto{\pgfqpoint{3.476690in}{2.364472in}}%
\pgfpathlineto{\pgfqpoint{3.459032in}{2.430516in}}%
\pgfpathlineto{\pgfqpoint{3.446887in}{2.497786in}}%
\pgfpathlineto{\pgfqpoint{3.440460in}{2.565838in}}%
\pgfpathlineto{\pgfqpoint{3.439823in}{2.634192in}}%
\pgfpathlineto{\pgfqpoint{3.444897in}{2.702369in}}%
\pgfpathlineto{\pgfqpoint{3.455469in}{2.769914in}}%
\pgfpathlineto{\pgfqpoint{3.471231in}{2.836442in}}%
\pgfpathlineto{\pgfqpoint{3.491817in}{2.901644in}}%
\pgfpathlineto{\pgfqpoint{3.516855in}{2.965274in}}%
\pgfpathlineto{\pgfqpoint{3.545998in}{3.027138in}}%
\pgfpathlineto{\pgfqpoint{3.578946in}{3.087066in}}%
\pgfpathlineto{\pgfqpoint{3.615453in}{3.144898in}}%
\pgfpathlineto{\pgfqpoint{3.655321in}{3.200471in}}%
\pgfpathlineto{\pgfqpoint{3.698392in}{3.253601in}}%
\pgfpathlineto{\pgfqpoint{3.727238in}{3.287123in}}%
\pgfusepath{stroke}%
\end{pgfscope}%
\begin{pgfscope}%
\pgfpathrectangle{\pgfqpoint{0.647939in}{0.492442in}}{\pgfqpoint{3.079299in}{3.079299in}}%
\pgfusepath{clip}%
\pgfsetbuttcap%
\pgfsetroundjoin%
\pgfsetlinewidth{0.301125pt}%
\definecolor{currentstroke}{rgb}{0.500000,0.500000,0.500000}%
\pgfsetstrokecolor{currentstroke}%
\pgfsetstrokeopacity{0.300000}%
\pgfsetdash{}{0pt}%
\pgfpathmoveto{\pgfqpoint{3.727238in}{2.102076in}}%
\pgfpathlineto{\pgfqpoint{3.727238in}{2.102076in}}%
\pgfpathlineto{\pgfqpoint{3.686111in}{2.156696in}}%
\pgfpathlineto{\pgfqpoint{3.649274in}{2.214292in}}%
\pgfpathlineto{\pgfqpoint{3.617029in}{2.274577in}}%
\pgfpathlineto{\pgfqpoint{3.589697in}{2.337239in}}%
\pgfpathlineto{\pgfqpoint{3.567607in}{2.401930in}}%
\pgfpathlineto{\pgfqpoint{3.551061in}{2.468251in}}%
\pgfpathlineto{\pgfqpoint{3.540298in}{2.535750in}}%
\pgfpathlineto{\pgfqpoint{3.535462in}{2.603933in}}%
\pgfpathlineto{\pgfqpoint{3.536571in}{2.672283in}}%
\pgfpathlineto{\pgfqpoint{3.543512in}{2.740290in}}%
\pgfpathlineto{\pgfqpoint{3.556066in}{2.807487in}}%
\pgfpathlineto{\pgfqpoint{3.573934in}{2.873474in}}%
\pgfpathlineto{\pgfqpoint{3.596786in}{2.937912in}}%
\pgfpathlineto{\pgfqpoint{3.624294in}{3.000508in}}%
\pgfpathlineto{\pgfqpoint{3.656154in}{3.061010in}}%
\pgfpathlineto{\pgfqpoint{3.692101in}{3.119179in}}%
\pgfpathlineto{\pgfqpoint{3.727238in}{3.171963in}}%
\pgfusepath{stroke}%
\end{pgfscope}%
\begin{pgfscope}%
\pgfpathrectangle{\pgfqpoint{0.647939in}{0.492442in}}{\pgfqpoint{3.079299in}{3.079299in}}%
\pgfusepath{clip}%
\pgfsetbuttcap%
\pgfsetroundjoin%
\pgfsetlinewidth{0.301125pt}%
\definecolor{currentstroke}{rgb}{0.500000,0.500000,0.500000}%
\pgfsetstrokecolor{currentstroke}%
\pgfsetstrokeopacity{0.300000}%
\pgfsetdash{}{0pt}%
\pgfpathmoveto{\pgfqpoint{3.727238in}{2.242044in}}%
\pgfpathlineto{\pgfqpoint{3.727238in}{2.242044in}}%
\pgfpathlineto{\pgfqpoint{3.694271in}{2.301928in}}%
\pgfpathlineto{\pgfqpoint{3.666627in}{2.364446in}}%
\pgfpathlineto{\pgfqpoint{3.644619in}{2.429158in}}%
\pgfpathlineto{\pgfqpoint{3.628521in}{2.495581in}}%
\pgfpathlineto{\pgfqpoint{3.618543in}{2.563191in}}%
\pgfpathlineto{\pgfqpoint{3.614788in}{2.631431in}}%
\pgfpathlineto{\pgfqpoint{3.617246in}{2.699739in}}%
\pgfpathlineto{\pgfqpoint{3.625785in}{2.767560in}}%
\pgfpathlineto{\pgfqpoint{3.640171in}{2.834385in}}%
\pgfpathlineto{\pgfqpoint{3.660117in}{2.899769in}}%
\pgfpathlineto{\pgfqpoint{3.685307in}{2.963320in}}%
\pgfpathlineto{\pgfqpoint{3.715433in}{3.024689in}}%
\pgfpathlineto{\pgfqpoint{3.727238in}{3.046557in}}%
\pgfusepath{stroke}%
\end{pgfscope}%
\begin{pgfscope}%
\pgfpathrectangle{\pgfqpoint{0.647939in}{0.492442in}}{\pgfqpoint{3.079299in}{3.079299in}}%
\pgfusepath{clip}%
\pgfsetbuttcap%
\pgfsetroundjoin%
\pgfsetlinewidth{0.301125pt}%
\definecolor{currentstroke}{rgb}{0.500000,0.500000,0.500000}%
\pgfsetstrokecolor{currentstroke}%
\pgfsetstrokeopacity{0.300000}%
\pgfsetdash{}{0pt}%
\pgfpathmoveto{\pgfqpoint{3.727238in}{2.451996in}}%
\pgfpathlineto{\pgfqpoint{3.727238in}{2.451996in}}%
\pgfpathlineto{\pgfqpoint{3.710921in}{2.518362in}}%
\pgfpathlineto{\pgfqpoint{3.701215in}{2.586003in}}%
\pgfpathlineto{\pgfqpoint{3.698204in}{2.654267in}}%
\pgfpathlineto{\pgfqpoint{3.701849in}{2.722502in}}%
\pgfpathlineto{\pgfqpoint{3.712005in}{2.790084in}}%
\pgfusepath{stroke}%
\end{pgfscope}%
\begin{pgfscope}%
\pgfpathrectangle{\pgfqpoint{0.647939in}{0.492442in}}{\pgfqpoint{3.079299in}{3.079299in}}%
\pgfusepath{clip}%
\pgfsetbuttcap%
\pgfsetroundjoin%
\pgfsetlinewidth{0.301125pt}%
\definecolor{currentstroke}{rgb}{0.500000,0.500000,0.500000}%
\pgfsetstrokecolor{currentstroke}%
\pgfsetstrokeopacity{0.300000}%
\pgfsetdash{}{0pt}%
\pgfpathmoveto{\pgfqpoint{3.345743in}{3.182036in}}%
\pgfpathlineto{\pgfqpoint{3.387171in}{3.236482in}}%
\pgfpathlineto{\pgfqpoint{3.430402in}{3.289510in}}%
\pgfpathlineto{\pgfqpoint{3.475400in}{3.341047in}}%
\pgfpathlineto{\pgfqpoint{3.522157in}{3.390992in}}%
\pgfpathlineto{\pgfqpoint{3.570686in}{3.439215in}}%
\pgfpathlineto{\pgfqpoint{3.621020in}{3.485550in}}%
\pgfpathlineto{\pgfqpoint{3.673194in}{3.529799in}}%
\pgfpathlineto{\pgfqpoint{3.727238in}{3.571741in}}%
\pgfpathlineto{\pgfqpoint{3.727238in}{3.571741in}}%
\pgfusepath{stroke}%
\end{pgfscope}%
\begin{pgfscope}%
\pgfpathrectangle{\pgfqpoint{0.647939in}{0.492442in}}{\pgfqpoint{3.079299in}{3.079299in}}%
\pgfusepath{clip}%
\pgfsetbuttcap%
\pgfsetroundjoin%
\pgfsetlinewidth{0.301125pt}%
\definecolor{currentstroke}{rgb}{0.500000,0.500000,0.500000}%
\pgfsetstrokecolor{currentstroke}%
\pgfsetstrokeopacity{0.300000}%
\pgfsetdash{}{0pt}%
\pgfpathmoveto{\pgfqpoint{0.647939in}{2.463789in}}%
\pgfpathlineto{\pgfqpoint{0.679481in}{2.468074in}}%
\pgfpathlineto{\pgfqpoint{0.747179in}{2.478017in}}%
\pgfpathlineto{\pgfqpoint{0.814645in}{2.489430in}}%
\pgfpathlineto{\pgfqpoint{0.881845in}{2.502315in}}%
\pgfpathlineto{\pgfqpoint{0.948754in}{2.516633in}}%
\pgfpathlineto{\pgfqpoint{1.015362in}{2.532298in}}%
\pgfpathlineto{\pgfqpoint{1.081672in}{2.549182in}}%
\pgfpathlineto{\pgfqpoint{1.147710in}{2.567108in}}%
\pgfpathlineto{\pgfqpoint{1.213519in}{2.585857in}}%
\pgfpathlineto{\pgfqpoint{1.279163in}{2.605177in}}%
\pgfpathlineto{\pgfqpoint{1.344721in}{2.624789in}}%
\pgfpathlineto{\pgfqpoint{1.410284in}{2.644385in}}%
\pgfpathlineto{\pgfqpoint{1.475948in}{2.663637in}}%
\pgfpathlineto{\pgfqpoint{1.541807in}{2.682207in}}%
\pgfpathlineto{\pgfqpoint{1.607945in}{2.699752in}}%
\pgfpathlineto{\pgfqpoint{1.674427in}{2.715940in}}%
\pgfpathlineto{\pgfqpoint{1.741287in}{2.730475in}}%
\pgfpathlineto{\pgfqpoint{1.808529in}{2.743121in}}%
\pgfpathlineto{\pgfqpoint{1.876121in}{2.753732in}}%
\pgfpathlineto{\pgfqpoint{1.944003in}{2.762287in}}%
\pgfpathlineto{\pgfqpoint{2.012102in}{2.768926in}}%
\pgfpathlineto{\pgfqpoint{2.080340in}{2.773983in}}%
\pgfpathlineto{\pgfqpoint{2.148649in}{2.778025in}}%
\pgfpathlineto{\pgfqpoint{2.216970in}{2.781855in}}%
\pgfpathlineto{\pgfqpoint{2.285236in}{2.786510in}}%
\pgfpathlineto{\pgfqpoint{2.353317in}{2.793244in}}%
\pgfpathlineto{\pgfqpoint{2.420950in}{2.803431in}}%
\pgfpathlineto{\pgfqpoint{2.487669in}{2.818370in}}%
\pgfpathlineto{\pgfqpoint{2.552850in}{2.838935in}}%
\pgfpathlineto{\pgfqpoint{2.615896in}{2.865313in}}%
\pgfpathlineto{\pgfqpoint{2.676444in}{2.897033in}}%
\pgfpathlineto{\pgfqpoint{2.734457in}{2.933223in}}%
\pgfpathlineto{\pgfqpoint{2.790154in}{2.972912in}}%
\pgfpathlineto{\pgfqpoint{2.843895in}{3.015228in}}%
\pgfpathlineto{\pgfqpoint{2.896075in}{3.059461in}}%
\pgfpathlineto{\pgfqpoint{2.947071in}{3.105068in}}%
\pgfpathlineto{\pgfqpoint{2.997216in}{3.151616in}}%
\pgfpathlineto{\pgfqpoint{3.046793in}{3.198771in}}%
\pgfpathlineto{\pgfqpoint{3.096049in}{3.246265in}}%
\pgfpathlineto{\pgfqpoint{3.145198in}{3.293868in}}%
\pgfpathlineto{\pgfqpoint{3.194428in}{3.341391in}}%
\pgfpathlineto{\pgfqpoint{3.243909in}{3.388653in}}%
\pgfpathlineto{\pgfqpoint{3.293795in}{3.435487in}}%
\pgfpathlineto{\pgfqpoint{3.344231in}{3.481727in}}%
\pgfpathlineto{\pgfqpoint{3.395359in}{3.527201in}}%
\pgfpathlineto{\pgfqpoint{3.447302in}{3.571741in}}%
\pgfpathlineto{\pgfqpoint{3.447302in}{3.571741in}}%
\pgfusepath{stroke}%
\end{pgfscope}%
\begin{pgfscope}%
\pgfpathrectangle{\pgfqpoint{0.647939in}{0.492442in}}{\pgfqpoint{3.079299in}{3.079299in}}%
\pgfusepath{clip}%
\pgfsetbuttcap%
\pgfsetroundjoin%
\pgfsetlinewidth{0.301125pt}%
\definecolor{currentstroke}{rgb}{0.500000,0.500000,0.500000}%
\pgfsetstrokecolor{currentstroke}%
\pgfsetstrokeopacity{0.300000}%
\pgfsetdash{}{0pt}%
\pgfpathmoveto{\pgfqpoint{0.647939in}{2.775431in}}%
\pgfpathlineto{\pgfqpoint{0.680005in}{2.779545in}}%
\pgfpathlineto{\pgfqpoint{0.747786in}{2.788918in}}%
\pgfpathlineto{\pgfqpoint{0.815368in}{2.799625in}}%
\pgfpathlineto{\pgfqpoint{0.882727in}{2.811652in}}%
\pgfpathlineto{\pgfqpoint{0.949849in}{2.824944in}}%
\pgfpathlineto{\pgfqpoint{1.016730in}{2.839406in}}%
\pgfpathlineto{\pgfqpoint{1.083380in}{2.854900in}}%
\pgfpathlineto{\pgfqpoint{1.149827in}{2.871243in}}%
\pgfpathlineto{\pgfqpoint{1.216116in}{2.888218in}}%
\pgfpathlineto{\pgfqpoint{1.282306in}{2.905579in}}%
\pgfpathlineto{\pgfqpoint{1.348465in}{2.923055in}}%
\pgfpathlineto{\pgfqpoint{1.414671in}{2.940356in}}%
\pgfpathlineto{\pgfqpoint{1.480999in}{2.957178in}}%
\pgfpathlineto{\pgfqpoint{1.547520in}{2.973213in}}%
\pgfpathlineto{\pgfqpoint{1.614293in}{2.988163in}}%
\pgfpathlineto{\pgfqpoint{1.681354in}{3.001753in}}%
\pgfpathlineto{\pgfqpoint{1.748716in}{3.013754in}}%
\pgfpathlineto{\pgfqpoint{1.816365in}{3.024006in}}%
\pgfpathlineto{\pgfqpoint{1.884265in}{3.032437in}}%
\pgfpathlineto{\pgfqpoint{1.952364in}{3.039089in}}%
\pgfpathlineto{\pgfqpoint{2.020602in}{3.044143in}}%
\pgfpathlineto{\pgfqpoint{2.088923in}{3.047936in}}%
\pgfpathlineto{\pgfqpoint{2.157285in}{3.050953in}}%
\pgfpathlineto{\pgfqpoint{2.225653in}{3.053832in}}%
\pgfpathlineto{\pgfqpoint{2.293989in}{3.057353in}}%
\pgfpathlineto{\pgfqpoint{2.362222in}{3.062423in}}%
\pgfpathlineto{\pgfqpoint{2.430208in}{3.070017in}}%
\pgfpathlineto{\pgfqpoint{2.497706in}{3.081056in}}%
\pgfpathlineto{\pgfqpoint{2.564373in}{3.096278in}}%
\pgfpathlineto{\pgfqpoint{2.629811in}{3.116104in}}%
\pgfpathlineto{\pgfqpoint{2.693664in}{3.140562in}}%
\pgfpathlineto{\pgfqpoint{2.755705in}{3.169326in}}%
\pgfpathlineto{\pgfqpoint{2.815879in}{3.201835in}}%
\pgfpathlineto{\pgfqpoint{2.874285in}{3.237434in}}%
\pgfpathlineto{\pgfqpoint{2.931131in}{3.275485in}}%
\pgfpathlineto{\pgfqpoint{2.986680in}{3.315419in}}%
\pgfpathlineto{\pgfqpoint{3.041207in}{3.356750in}}%
\pgfpathlineto{\pgfqpoint{3.094983in}{3.399058in}}%
\pgfpathlineto{\pgfqpoint{3.148260in}{3.441991in}}%
\pgfpathlineto{\pgfqpoint{3.201271in}{3.485257in}}%
\pgfpathlineto{\pgfqpoint{3.254229in}{3.528589in}}%
\pgfpathlineto{\pgfqpoint{3.307334in}{3.571741in}}%
\pgfpathlineto{\pgfqpoint{3.307334in}{3.571741in}}%
\pgfusepath{stroke}%
\end{pgfscope}%
\begin{pgfscope}%
\pgfpathrectangle{\pgfqpoint{0.647939in}{0.492442in}}{\pgfqpoint{3.079299in}{3.079299in}}%
\pgfusepath{clip}%
\pgfsetbuttcap%
\pgfsetroundjoin%
\pgfsetlinewidth{0.301125pt}%
\definecolor{currentstroke}{rgb}{0.500000,0.500000,0.500000}%
\pgfsetstrokecolor{currentstroke}%
\pgfsetstrokeopacity{0.300000}%
\pgfsetdash{}{0pt}%
\pgfpathmoveto{\pgfqpoint{0.647939in}{2.968573in}}%
\pgfpathlineto{\pgfqpoint{0.674131in}{2.971776in}}%
\pgfpathlineto{\pgfqpoint{0.741970in}{2.980715in}}%
\pgfpathlineto{\pgfqpoint{0.809631in}{2.990914in}}%
\pgfpathlineto{\pgfqpoint{0.877094in}{3.002352in}}%
\pgfpathlineto{\pgfqpoint{0.944346in}{3.014973in}}%
\pgfpathlineto{\pgfqpoint{1.011386in}{3.028679in}}%
\pgfpathlineto{\pgfqpoint{1.078225in}{3.043332in}}%
\pgfpathlineto{\pgfqpoint{1.144892in}{3.058759in}}%
\pgfpathlineto{\pgfqpoint{1.211426in}{3.074746in}}%
\pgfpathlineto{\pgfqpoint{1.277884in}{3.091052in}}%
\pgfpathlineto{\pgfqpoint{1.344327in}{3.107415in}}%
\pgfpathlineto{\pgfqpoint{1.410825in}{3.123557in}}%
\pgfpathlineto{\pgfqpoint{1.477443in}{3.139192in}}%
\pgfpathlineto{\pgfqpoint{1.544241in}{3.154031in}}%
\pgfpathlineto{\pgfqpoint{1.611267in}{3.167803in}}%
\pgfpathlineto{\pgfqpoint{1.678549in}{3.180258in}}%
\pgfpathlineto{\pgfqpoint{1.746092in}{3.191199in}}%
\pgfpathlineto{\pgfqpoint{1.813880in}{3.200493in}}%
\pgfpathlineto{\pgfqpoint{1.881879in}{3.208092in}}%
\pgfpathlineto{\pgfqpoint{1.950043in}{3.214053in}}%
\pgfpathlineto{\pgfqpoint{2.018320in}{3.218551in}}%
\pgfpathlineto{\pgfqpoint{2.086665in}{3.221895in}}%
\pgfpathlineto{\pgfqpoint{2.155043in}{3.224525in}}%
\pgfpathlineto{\pgfqpoint{2.223427in}{3.226995in}}%
\pgfpathlineto{\pgfqpoint{2.291790in}{3.229969in}}%
\pgfpathlineto{\pgfqpoint{2.360083in}{3.234201in}}%
\pgfpathlineto{\pgfqpoint{2.428208in}{3.240503in}}%
\pgfpathlineto{\pgfqpoint{2.495999in}{3.249655in}}%
\pgfpathlineto{\pgfqpoint{2.563213in}{3.262316in}}%
\pgfpathlineto{\pgfqpoint{2.629553in}{3.278933in}}%
\pgfpathlineto{\pgfqpoint{2.694720in}{3.299675in}}%
\pgfpathlineto{\pgfqpoint{2.758478in}{3.324425in}}%
\pgfpathlineto{\pgfqpoint{2.820699in}{3.352834in}}%
\pgfpathlineto{\pgfqpoint{2.881382in}{3.384407in}}%
\pgfpathlineto{\pgfqpoint{2.940633in}{3.418603in}}%
\pgfpathlineto{\pgfqpoint{2.998631in}{3.454891in}}%
\pgfpathlineto{\pgfqpoint{3.055596in}{3.492786in}}%
\pgfpathlineto{\pgfqpoint{3.111763in}{3.531860in}}%
\pgfpathlineto{\pgfqpoint{3.167366in}{3.571741in}}%
\pgfpathlineto{\pgfqpoint{3.167366in}{3.571741in}}%
\pgfusepath{stroke}%
\end{pgfscope}%
\begin{pgfscope}%
\pgfpathrectangle{\pgfqpoint{0.647939in}{0.492442in}}{\pgfqpoint{3.079299in}{3.079299in}}%
\pgfusepath{clip}%
\pgfsetbuttcap%
\pgfsetroundjoin%
\pgfsetlinewidth{0.301125pt}%
\definecolor{currentstroke}{rgb}{0.500000,0.500000,0.500000}%
\pgfsetstrokecolor{currentstroke}%
\pgfsetstrokeopacity{0.300000}%
\pgfsetdash{}{0pt}%
\pgfpathmoveto{\pgfqpoint{0.647939in}{3.104008in}}%
\pgfpathlineto{\pgfqpoint{0.696713in}{3.110125in}}%
\pgfpathlineto{\pgfqpoint{0.764527in}{3.119252in}}%
\pgfpathlineto{\pgfqpoint{0.832168in}{3.129590in}}%
\pgfpathlineto{\pgfqpoint{0.899619in}{3.141102in}}%
\pgfpathlineto{\pgfqpoint{0.966874in}{3.153710in}}%
\pgfpathlineto{\pgfqpoint{1.033938in}{3.167300in}}%
\pgfpathlineto{\pgfqpoint{1.100829in}{3.181719in}}%
\pgfpathlineto{\pgfqpoint{1.167578in}{3.196784in}}%
\pgfpathlineto{\pgfqpoint{1.234229in}{3.212277in}}%
\pgfpathlineto{\pgfqpoint{1.300837in}{3.227958in}}%
\pgfpathlineto{\pgfqpoint{1.367463in}{3.243561in}}%
\pgfpathlineto{\pgfqpoint{1.434171in}{3.258810in}}%
\pgfpathlineto{\pgfqpoint{1.501019in}{3.273430in}}%
\pgfpathlineto{\pgfqpoint{1.568055in}{3.287151in}}%
\pgfpathlineto{\pgfqpoint{1.635315in}{3.299727in}}%
\pgfpathlineto{\pgfqpoint{1.702814in}{3.310943in}}%
\pgfpathlineto{\pgfqpoint{1.770547in}{3.320642in}}%
\pgfpathlineto{\pgfqpoint{1.838490in}{3.328737in}}%
\pgfpathlineto{\pgfqpoint{1.906604in}{3.335232in}}%
\pgfpathlineto{\pgfqpoint{1.974846in}{3.340232in}}%
\pgfpathlineto{\pgfqpoint{2.043171in}{3.343951in}}%
\pgfpathlineto{\pgfqpoint{2.111542in}{3.346721in}}%
\pgfpathlineto{\pgfqpoint{2.179933in}{3.348996in}}%
\pgfpathlineto{\pgfqpoint{2.248322in}{3.351328in}}%
\pgfpathlineto{\pgfqpoint{2.316682in}{3.354352in}}%
\pgfpathlineto{\pgfqpoint{2.384962in}{3.358759in}}%
\pgfpathlineto{\pgfqpoint{2.453070in}{3.365262in}}%
\pgfpathlineto{\pgfqpoint{2.520847in}{3.374539in}}%
\pgfpathlineto{\pgfqpoint{2.588076in}{3.387144in}}%
\pgfpathlineto{\pgfqpoint{2.654503in}{3.403434in}}%
\pgfpathlineto{\pgfqpoint{2.719877in}{3.423529in}}%
\pgfpathlineto{\pgfqpoint{2.784005in}{3.447311in}}%
\pgfpathlineto{\pgfqpoint{2.846781in}{3.474474in}}%
\pgfpathlineto{\pgfqpoint{2.908203in}{3.504588in}}%
\pgfpathlineto{\pgfqpoint{2.968358in}{3.537170in}}%
\pgfpathlineto{\pgfqpoint{3.027398in}{3.571741in}}%
\pgfpathlineto{\pgfqpoint{3.027398in}{3.571741in}}%
\pgfusepath{stroke}%
\end{pgfscope}%
\begin{pgfscope}%
\pgfpathrectangle{\pgfqpoint{0.647939in}{0.492442in}}{\pgfqpoint{3.079299in}{3.079299in}}%
\pgfusepath{clip}%
\pgfsetbuttcap%
\pgfsetroundjoin%
\pgfsetlinewidth{0.301125pt}%
\definecolor{currentstroke}{rgb}{0.500000,0.500000,0.500000}%
\pgfsetstrokecolor{currentstroke}%
\pgfsetstrokeopacity{0.300000}%
\pgfsetdash{}{0pt}%
\pgfpathmoveto{\pgfqpoint{0.647939in}{3.200154in}}%
\pgfpathlineto{\pgfqpoint{0.669303in}{3.202638in}}%
\pgfpathlineto{\pgfqpoint{0.737200in}{3.211134in}}%
\pgfpathlineto{\pgfqpoint{0.804938in}{3.220813in}}%
\pgfpathlineto{\pgfqpoint{0.872501in}{3.231649in}}%
\pgfpathlineto{\pgfqpoint{0.939879in}{3.243578in}}%
\pgfpathlineto{\pgfqpoint{1.007075in}{3.256501in}}%
\pgfpathlineto{\pgfqpoint{1.074100in}{3.270282in}}%
\pgfpathlineto{\pgfqpoint{1.140981in}{3.284749in}}%
\pgfpathlineto{\pgfqpoint{1.207756in}{3.299701in}}%
\pgfpathlineto{\pgfqpoint{1.274475in}{3.314903in}}%
\pgfpathlineto{\pgfqpoint{1.341195in}{3.330102in}}%
\pgfpathlineto{\pgfqpoint{1.407974in}{3.345035in}}%
\pgfpathlineto{\pgfqpoint{1.474870in}{3.359434in}}%
\pgfpathlineto{\pgfqpoint{1.541931in}{3.373036in}}%
\pgfpathlineto{\pgfqpoint{1.609195in}{3.385593in}}%
\pgfpathlineto{\pgfqpoint{1.676681in}{3.396889in}}%
\pgfpathlineto{\pgfqpoint{1.744391in}{3.406753in}}%
\pgfpathlineto{\pgfqpoint{1.812306in}{3.415082in}}%
\pgfpathlineto{\pgfqpoint{1.880394in}{3.421851in}}%
\pgfpathlineto{\pgfqpoint{1.948614in}{3.427127in}}%
\pgfpathlineto{\pgfqpoint{2.016926in}{3.431081in}}%
\pgfpathlineto{\pgfqpoint{2.085291in}{3.433998in}}%
\pgfpathlineto{\pgfqpoint{2.153682in}{3.436269in}}%
\pgfpathlineto{\pgfqpoint{2.222078in}{3.438375in}}%
\pgfpathlineto{\pgfqpoint{2.290460in}{3.440881in}}%
\pgfpathlineto{\pgfqpoint{2.358794in}{3.444415in}}%
\pgfpathlineto{\pgfqpoint{2.427015in}{3.449646in}}%
\pgfpathlineto{\pgfqpoint{2.495007in}{3.457226in}}%
\pgfpathlineto{\pgfqpoint{2.562603in}{3.467719in}}%
\pgfpathlineto{\pgfqpoint{2.629592in}{3.481549in}}%
\pgfpathlineto{\pgfqpoint{2.695742in}{3.498945in}}%
\pgfpathlineto{\pgfqpoint{2.760847in}{3.519923in}}%
\pgfpathlineto{\pgfqpoint{2.824763in}{3.544299in}}%
\pgfpathlineto{\pgfqpoint{2.887429in}{3.571741in}}%
\pgfpathlineto{\pgfqpoint{2.887429in}{3.571741in}}%
\pgfusepath{stroke}%
\end{pgfscope}%
\begin{pgfscope}%
\pgfpathrectangle{\pgfqpoint{0.647939in}{0.492442in}}{\pgfqpoint{3.079299in}{3.079299in}}%
\pgfusepath{clip}%
\pgfsetbuttcap%
\pgfsetroundjoin%
\pgfsetlinewidth{0.301125pt}%
\definecolor{currentstroke}{rgb}{0.500000,0.500000,0.500000}%
\pgfsetstrokecolor{currentstroke}%
\pgfsetstrokeopacity{0.300000}%
\pgfsetdash{}{0pt}%
\pgfpathmoveto{\pgfqpoint{0.647939in}{3.288459in}}%
\pgfpathlineto{\pgfqpoint{0.648741in}{3.288547in}}%
\pgfpathlineto{\pgfqpoint{0.716695in}{3.296568in}}%
\pgfpathlineto{\pgfqpoint{0.784503in}{3.305742in}}%
\pgfpathlineto{\pgfqpoint{0.852148in}{3.316054in}}%
\pgfpathlineto{\pgfqpoint{0.919618in}{3.327452in}}%
\pgfpathlineto{\pgfqpoint{0.986913in}{3.339849in}}%
\pgfpathlineto{\pgfqpoint{1.054042in}{3.353116in}}%
\pgfpathlineto{\pgfqpoint{1.121029in}{3.367088in}}%
\pgfpathlineto{\pgfqpoint{1.187907in}{3.381571in}}%
\pgfpathlineto{\pgfqpoint{1.254722in}{3.396345in}}%
\pgfpathlineto{\pgfqpoint{1.321525in}{3.411170in}}%
\pgfpathlineto{\pgfqpoint{1.388373in}{3.425792in}}%
\pgfpathlineto{\pgfqpoint{1.455321in}{3.439949in}}%
\pgfpathlineto{\pgfqpoint{1.522418in}{3.453379in}}%
\pgfpathlineto{\pgfqpoint{1.589700in}{3.465838in}}%
\pgfpathlineto{\pgfqpoint{1.657191in}{3.477108in}}%
\pgfpathlineto{\pgfqpoint{1.724894in}{3.487017in}}%
\pgfpathlineto{\pgfqpoint{1.792796in}{3.495449in}}%
\pgfpathlineto{\pgfqpoint{1.860870in}{3.502361in}}%
\pgfpathlineto{\pgfqpoint{1.929079in}{3.507802in}}%
\pgfpathlineto{\pgfqpoint{1.997381in}{3.511917in}}%
\pgfpathlineto{\pgfqpoint{2.065740in}{3.514950in}}%
\pgfpathlineto{\pgfqpoint{2.134130in}{3.517241in}}%
\pgfpathlineto{\pgfqpoint{2.202530in}{3.519222in}}%
\pgfpathlineto{\pgfqpoint{2.270923in}{3.521416in}}%
\pgfpathlineto{\pgfqpoint{2.339284in}{3.524412in}}%
\pgfpathlineto{\pgfqpoint{2.407564in}{3.528824in}}%
\pgfpathlineto{\pgfqpoint{2.475678in}{3.535261in}}%
\pgfpathlineto{\pgfqpoint{2.543493in}{3.544283in}}%
\pgfpathlineto{\pgfqpoint{2.610829in}{3.556345in}}%
\pgfpathlineto{\pgfqpoint{2.677477in}{3.571741in}}%
\pgfpathlineto{\pgfqpoint{2.677477in}{3.571741in}}%
\pgfusepath{stroke}%
\end{pgfscope}%
\begin{pgfscope}%
\pgfpathrectangle{\pgfqpoint{0.647939in}{0.492442in}}{\pgfqpoint{3.079299in}{3.079299in}}%
\pgfusepath{clip}%
\pgfsetbuttcap%
\pgfsetroundjoin%
\pgfsetlinewidth{0.301125pt}%
\definecolor{currentstroke}{rgb}{0.500000,0.500000,0.500000}%
\pgfsetstrokecolor{currentstroke}%
\pgfsetstrokeopacity{0.300000}%
\pgfsetdash{}{0pt}%
\pgfpathmoveto{\pgfqpoint{1.852653in}{3.539765in}}%
\pgfpathlineto{\pgfqpoint{1.920856in}{3.545268in}}%
\pgfpathlineto{\pgfqpoint{1.989154in}{3.549441in}}%
\pgfpathlineto{\pgfqpoint{2.057512in}{3.552517in}}%
\pgfpathlineto{\pgfqpoint{2.125901in}{3.554823in}}%
\pgfpathlineto{\pgfqpoint{2.194302in}{3.556775in}}%
\pgfpathlineto{\pgfqpoint{2.262699in}{3.558864in}}%
\pgfpathlineto{\pgfqpoint{2.331069in}{3.561653in}}%
\pgfpathlineto{\pgfqpoint{2.399370in}{3.565746in}}%
\pgfpathlineto{\pgfqpoint{2.467525in}{3.571741in}}%
\pgfpathlineto{\pgfqpoint{2.467525in}{3.571741in}}%
\pgfusepath{stroke}%
\end{pgfscope}%
\begin{pgfscope}%
\pgfpathrectangle{\pgfqpoint{0.647939in}{0.492442in}}{\pgfqpoint{3.079299in}{3.079299in}}%
\pgfusepath{clip}%
\pgfsetbuttcap%
\pgfsetroundjoin%
\pgfsetlinewidth{0.301125pt}%
\definecolor{currentstroke}{rgb}{0.500000,0.500000,0.500000}%
\pgfsetstrokecolor{currentstroke}%
\pgfsetstrokeopacity{0.300000}%
\pgfsetdash{}{0pt}%
\pgfpathmoveto{\pgfqpoint{0.647939in}{3.372467in}}%
\pgfpathlineto{\pgfqpoint{0.690696in}{3.377525in}}%
\pgfpathlineto{\pgfqpoint{0.758578in}{3.386131in}}%
\pgfpathlineto{\pgfqpoint{0.826309in}{3.395859in}}%
\pgfpathlineto{\pgfqpoint{0.893876in}{3.406672in}}%
\pgfpathlineto{\pgfqpoint{0.961275in}{3.418491in}}%
\pgfpathlineto{\pgfqpoint{1.028512in}{3.431199in}}%
\pgfpathlineto{\pgfqpoint{1.095605in}{3.444647in}}%
\pgfpathlineto{\pgfqpoint{1.162585in}{3.458655in}}%
\pgfpathlineto{\pgfqpoint{1.229490in}{3.473015in}}%
\pgfpathlineto{\pgfqpoint{1.296368in}{3.487498in}}%
\pgfpathlineto{\pgfqpoint{1.363273in}{3.501857in}}%
\pgfpathlineto{\pgfqpoint{1.430259in}{3.515835in}}%
\pgfpathlineto{\pgfqpoint{1.497374in}{3.529174in}}%
\pgfpathlineto{\pgfqpoint{1.564656in}{3.541634in}}%
\pgfpathlineto{\pgfqpoint{1.632132in}{3.552995in}}%
\pgfpathlineto{\pgfqpoint{1.699811in}{3.563074in}}%
\pgfpathlineto{\pgfqpoint{1.767684in}{3.571741in}}%
\pgfpathlineto{\pgfqpoint{1.767684in}{3.571741in}}%
\pgfusepath{stroke}%
\end{pgfscope}%
\begin{pgfscope}%
\pgfpathrectangle{\pgfqpoint{0.647939in}{0.492442in}}{\pgfqpoint{3.079299in}{3.079299in}}%
\pgfusepath{clip}%
\pgfsetbuttcap%
\pgfsetroundjoin%
\pgfsetlinewidth{0.301125pt}%
\definecolor{currentstroke}{rgb}{0.500000,0.500000,0.500000}%
\pgfsetstrokecolor{currentstroke}%
\pgfsetstrokeopacity{0.300000}%
\pgfsetdash{}{0pt}%
\pgfpathmoveto{\pgfqpoint{0.647939in}{3.448050in}}%
\pgfpathlineto{\pgfqpoint{0.674545in}{3.451055in}}%
\pgfpathlineto{\pgfqpoint{0.742474in}{3.459290in}}%
\pgfpathlineto{\pgfqpoint{0.810260in}{3.468630in}}%
\pgfpathlineto{\pgfqpoint{0.877891in}{3.479039in}}%
\pgfpathlineto{\pgfqpoint{0.945360in}{3.490447in}}%
\pgfpathlineto{\pgfqpoint{1.012672in}{3.502751in}}%
\pgfpathlineto{\pgfqpoint{1.079842in}{3.515810in}}%
\pgfpathlineto{\pgfqpoint{1.146896in}{3.529453in}}%
\pgfpathlineto{\pgfqpoint{1.213873in}{3.543475in}}%
\pgfpathlineto{\pgfqpoint{1.280817in}{3.557651in}}%
\pgfpathlineto{\pgfqpoint{1.347780in}{3.571741in}}%
\pgfpathlineto{\pgfqpoint{1.347780in}{3.571741in}}%
\pgfusepath{stroke}%
\end{pgfscope}%
\begin{pgfscope}%
\pgfpathrectangle{\pgfqpoint{0.647939in}{0.492442in}}{\pgfqpoint{3.079299in}{3.079299in}}%
\pgfusepath{clip}%
\pgfsetbuttcap%
\pgfsetroundjoin%
\pgfsetlinewidth{0.301125pt}%
\definecolor{currentstroke}{rgb}{0.500000,0.500000,0.500000}%
\pgfsetstrokecolor{currentstroke}%
\pgfsetstrokeopacity{0.300000}%
\pgfsetdash{}{0pt}%
\pgfpathmoveto{\pgfqpoint{0.647939in}{3.520607in}}%
\pgfpathlineto{\pgfqpoint{0.659461in}{3.521851in}}%
\pgfpathlineto{\pgfqpoint{0.727430in}{3.529745in}}%
\pgfpathlineto{\pgfqpoint{0.795264in}{3.538726in}}%
\pgfpathlineto{\pgfqpoint{0.862950in}{3.548767in}}%
\pgfpathlineto{\pgfqpoint{0.930481in}{3.559806in}}%
\pgfpathlineto{\pgfqpoint{0.997859in}{3.571741in}}%
\pgfpathlineto{\pgfqpoint{0.997859in}{3.571741in}}%
\pgfusepath{stroke}%
\end{pgfscope}%
\begin{pgfscope}%
\pgfpathrectangle{\pgfqpoint{0.647939in}{0.492442in}}{\pgfqpoint{3.079299in}{3.079299in}}%
\pgfusepath{clip}%
\pgfsetbuttcap%
\pgfsetroundjoin%
\pgfsetlinewidth{0.301125pt}%
\definecolor{currentstroke}{rgb}{0.500000,0.500000,0.500000}%
\pgfsetstrokecolor{currentstroke}%
\pgfsetstrokeopacity{0.300000}%
\pgfsetdash{}{0pt}%
\pgfpathmoveto{\pgfqpoint{0.647939in}{2.871901in}}%
\pgfpathlineto{\pgfqpoint{0.647939in}{2.871901in}}%
\pgfpathlineto{\pgfqpoint{0.715822in}{2.880501in}}%
\pgfpathlineto{\pgfqpoint{0.783529in}{2.890392in}}%
\pgfpathlineto{\pgfqpoint{0.851034in}{2.901573in}}%
\pgfpathlineto{\pgfqpoint{0.918320in}{2.914008in}}%
\pgfpathlineto{\pgfqpoint{0.985379in}{2.927616in}}%
\pgfpathlineto{\pgfqpoint{1.052217in}{2.942278in}}%
\pgfpathlineto{\pgfqpoint{1.118854in}{2.957829in}}%
\pgfpathlineto{\pgfqpoint{1.185327in}{2.974068in}}%
\pgfpathlineto{\pgfqpoint{1.251687in}{2.990765in}}%
\pgfpathlineto{\pgfqpoint{1.317997in}{3.007665in}}%
\pgfpathlineto{\pgfqpoint{1.384325in}{3.024489in}}%
\pgfpathlineto{\pgfqpoint{1.450744in}{3.040946in}}%
\pgfpathlineto{\pgfqpoint{1.517325in}{3.056739in}}%
\pgfpathlineto{\pgfqpoint{1.584124in}{3.071572in}}%
\pgfpathlineto{\pgfqpoint{1.651183in}{3.085174in}}%
\pgfpathlineto{\pgfqpoint{1.718522in}{3.097307in}}%
\pgfpathlineto{\pgfqpoint{1.786136in}{3.107796in}}%
\pgfpathlineto{\pgfqpoint{1.853996in}{3.116541in}}%
\pgfpathlineto{\pgfqpoint{1.922060in}{3.123542in}}%
\pgfpathlineto{\pgfqpoint{1.990273in}{3.128924in}}%
\pgfpathlineto{\pgfqpoint{2.058580in}{3.132950in}}%
\pgfpathlineto{\pgfqpoint{2.126939in}{3.136016in}}%
\pgfpathlineto{\pgfqpoint{2.195317in}{3.138658in}}%
\pgfpathlineto{\pgfqpoint{2.263684in}{3.141542in}}%
\pgfpathlineto{\pgfqpoint{2.331997in}{3.145459in}}%
\pgfpathlineto{\pgfqpoint{2.400165in}{3.151281in}}%
\pgfpathlineto{\pgfqpoint{2.468028in}{3.159877in}}%
\pgfpathlineto{\pgfqpoint{2.535335in}{3.172023in}}%
\pgfpathlineto{\pgfqpoint{2.601758in}{3.188283in}}%
\pgfpathlineto{\pgfqpoint{2.666954in}{3.208916in}}%
\pgfpathlineto{\pgfqpoint{2.730636in}{3.233844in}}%
\pgfpathlineto{\pgfqpoint{2.792642in}{3.262707in}}%
\pgfpathlineto{\pgfqpoint{2.852956in}{3.294972in}}%
\pgfusepath{stroke}%
\end{pgfscope}%
\begin{pgfscope}%
\pgfpathrectangle{\pgfqpoint{0.647939in}{0.492442in}}{\pgfqpoint{3.079299in}{3.079299in}}%
\pgfusepath{clip}%
\pgfsetbuttcap%
\pgfsetroundjoin%
\pgfsetlinewidth{0.301125pt}%
\definecolor{currentstroke}{rgb}{0.500000,0.500000,0.500000}%
\pgfsetstrokecolor{currentstroke}%
\pgfsetstrokeopacity{0.300000}%
\pgfsetdash{}{0pt}%
\pgfpathmoveto{\pgfqpoint{0.647939in}{2.661948in}}%
\pgfpathlineto{\pgfqpoint{0.647939in}{2.661948in}}%
\pgfpathlineto{\pgfqpoint{0.715779in}{2.670881in}}%
\pgfpathlineto{\pgfqpoint{0.783423in}{2.681185in}}%
\pgfpathlineto{\pgfqpoint{0.850842in}{2.692871in}}%
\pgfpathlineto{\pgfqpoint{0.918014in}{2.705908in}}%
\pgfpathlineto{\pgfqpoint{0.984924in}{2.720225in}}%
\pgfpathlineto{\pgfqpoint{1.051576in}{2.735707in}}%
\pgfpathlineto{\pgfqpoint{1.117988in}{2.752192in}}%
\pgfpathlineto{\pgfqpoint{1.184197in}{2.769478in}}%
\pgfpathlineto{\pgfqpoint{1.250255in}{2.787331in}}%
\pgfpathlineto{\pgfqpoint{1.316230in}{2.805490in}}%
\pgfpathlineto{\pgfqpoint{1.382200in}{2.823667in}}%
\pgfpathlineto{\pgfqpoint{1.448249in}{2.841555in}}%
\pgfpathlineto{\pgfqpoint{1.514459in}{2.858836in}}%
\pgfpathlineto{\pgfqpoint{1.580902in}{2.875192in}}%
\pgfpathlineto{\pgfqpoint{1.647634in}{2.890316in}}%
\pgfpathlineto{\pgfqpoint{1.714689in}{2.903933in}}%
\pgfpathlineto{\pgfqpoint{1.782069in}{2.915824in}}%
\pgfpathlineto{\pgfqpoint{1.849751in}{2.925848in}}%
\pgfpathlineto{\pgfqpoint{1.917689in}{2.933968in}}%
\pgfpathlineto{\pgfqpoint{1.985820in}{2.940282in}}%
\pgfpathlineto{\pgfqpoint{2.054079in}{2.945050in}}%
\pgfpathlineto{\pgfqpoint{2.122408in}{2.948708in}}%
\pgfpathlineto{\pgfqpoint{2.190765in}{2.951858in}}%
\pgfpathlineto{\pgfqpoint{2.259107in}{2.955280in}}%
\pgfpathlineto{\pgfqpoint{2.327373in}{2.959918in}}%
\pgfpathlineto{\pgfqpoint{2.395433in}{2.966847in}}%
\pgfpathlineto{\pgfqpoint{2.463045in}{2.977158in}}%
\pgfpathlineto{\pgfqpoint{2.529838in}{2.991780in}}%
\pgfpathlineto{\pgfqpoint{2.595352in}{3.011304in}}%
\pgfpathlineto{\pgfqpoint{2.659149in}{3.035856in}}%
\pgfpathlineto{\pgfqpoint{2.720937in}{3.065111in}}%
\pgfpathlineto{\pgfqpoint{2.780637in}{3.098444in}}%
\pgfpathlineto{\pgfqpoint{2.838363in}{3.135114in}}%
\pgfpathlineto{\pgfqpoint{2.894354in}{3.174406in}}%
\pgfpathlineto{\pgfqpoint{2.948905in}{3.215691in}}%
\pgfpathlineto{\pgfqpoint{3.002325in}{3.258439in}}%
\pgfpathlineto{\pgfqpoint{3.054910in}{3.302212in}}%
\pgfpathlineto{\pgfqpoint{3.106924in}{3.346666in}}%
\pgfusepath{stroke}%
\end{pgfscope}%
\begin{pgfscope}%
\pgfpathrectangle{\pgfqpoint{0.647939in}{0.492442in}}{\pgfqpoint{3.079299in}{3.079299in}}%
\pgfusepath{clip}%
\pgfsetbuttcap%
\pgfsetroundjoin%
\pgfsetlinewidth{0.301125pt}%
\definecolor{currentstroke}{rgb}{0.500000,0.500000,0.500000}%
\pgfsetstrokecolor{currentstroke}%
\pgfsetstrokeopacity{0.300000}%
\pgfsetdash{}{0pt}%
\pgfpathmoveto{\pgfqpoint{0.647939in}{2.591964in}}%
\pgfpathlineto{\pgfqpoint{0.647939in}{2.591964in}}%
\pgfpathlineto{\pgfqpoint{0.715763in}{2.601014in}}%
\pgfpathlineto{\pgfqpoint{0.783385in}{2.611463in}}%
\pgfpathlineto{\pgfqpoint{0.850773in}{2.623326in}}%
\pgfpathlineto{\pgfqpoint{0.917902in}{2.636577in}}%
\pgfpathlineto{\pgfqpoint{0.984758in}{2.651147in}}%
\pgfpathlineto{\pgfqpoint{1.051341in}{2.666922in}}%
\pgfpathlineto{\pgfqpoint{1.117669in}{2.683741in}}%
\pgfpathlineto{\pgfqpoint{1.183777in}{2.701404in}}%
\pgfpathlineto{\pgfqpoint{1.249721in}{2.719677in}}%
\pgfpathlineto{\pgfqpoint{1.315568in}{2.738295in}}%
\pgfpathlineto{\pgfqpoint{1.381400in}{2.756966in}}%
\pgfpathlineto{\pgfqpoint{1.447304in}{2.775381in}}%
\pgfpathlineto{\pgfqpoint{1.513367in}{2.793216in}}%
\pgfpathlineto{\pgfqpoint{1.579667in}{2.810142in}}%
\pgfpathlineto{\pgfqpoint{1.646266in}{2.825842in}}%
\pgfusepath{stroke}%
\end{pgfscope}%
\begin{pgfscope}%
\pgfpathrectangle{\pgfqpoint{0.647939in}{0.492442in}}{\pgfqpoint{3.079299in}{3.079299in}}%
\pgfusepath{clip}%
\pgfsetbuttcap%
\pgfsetroundjoin%
\pgfsetlinewidth{0.301125pt}%
\definecolor{currentstroke}{rgb}{0.500000,0.500000,0.500000}%
\pgfsetstrokecolor{currentstroke}%
\pgfsetstrokeopacity{0.300000}%
\pgfsetdash{}{0pt}%
\pgfpathmoveto{\pgfqpoint{0.647939in}{2.382012in}}%
\pgfpathlineto{\pgfqpoint{0.647939in}{2.382012in}}%
\pgfpathlineto{\pgfqpoint{0.715712in}{2.391430in}}%
\pgfpathlineto{\pgfqpoint{0.783261in}{2.402341in}}%
\pgfpathlineto{\pgfqpoint{0.850546in}{2.414771in}}%
\pgfpathlineto{\pgfqpoint{0.917535in}{2.428705in}}%
\pgfpathlineto{\pgfqpoint{0.984209in}{2.444085in}}%
\pgfpathlineto{\pgfqpoint{1.050560in}{2.460804in}}%
\pgfpathlineto{\pgfqpoint{1.116602in}{2.478709in}}%
\pgfpathlineto{\pgfqpoint{1.182370in}{2.497600in}}%
\pgfpathlineto{\pgfqpoint{1.247918in}{2.517243in}}%
\pgfpathlineto{\pgfqpoint{1.313320in}{2.537369in}}%
\pgfusepath{stroke}%
\end{pgfscope}%
\begin{pgfscope}%
\pgfpathrectangle{\pgfqpoint{0.647939in}{0.492442in}}{\pgfqpoint{3.079299in}{3.079299in}}%
\pgfusepath{clip}%
\pgfsetbuttcap%
\pgfsetroundjoin%
\pgfsetlinewidth{0.301125pt}%
\definecolor{currentstroke}{rgb}{0.500000,0.500000,0.500000}%
\pgfsetstrokecolor{currentstroke}%
\pgfsetstrokeopacity{0.300000}%
\pgfsetdash{}{0pt}%
\pgfpathmoveto{\pgfqpoint{0.647939in}{2.312028in}}%
\pgfpathlineto{\pgfqpoint{0.647939in}{2.312028in}}%
\pgfpathlineto{\pgfqpoint{0.715694in}{2.321575in}}%
\pgfpathlineto{\pgfqpoint{0.783216in}{2.332649in}}%
\pgfpathlineto{\pgfqpoint{0.850463in}{2.345280in}}%
\pgfpathlineto{\pgfqpoint{0.917401in}{2.359457in}}%
\pgfpathlineto{\pgfqpoint{0.984007in}{2.375127in}}%
\pgfpathlineto{\pgfqpoint{1.050272in}{2.392185in}}%
\pgfpathlineto{\pgfqpoint{1.116206in}{2.410480in}}%
\pgfpathlineto{\pgfqpoint{1.181845in}{2.429816in}}%
\pgfpathlineto{\pgfqpoint{1.247241in}{2.449957in}}%
\pgfpathlineto{\pgfqpoint{1.312470in}{2.470635in}}%
\pgfusepath{stroke}%
\end{pgfscope}%
\begin{pgfscope}%
\pgfpathrectangle{\pgfqpoint{0.647939in}{0.492442in}}{\pgfqpoint{3.079299in}{3.079299in}}%
\pgfusepath{clip}%
\pgfsetbuttcap%
\pgfsetroundjoin%
\pgfsetlinewidth{0.301125pt}%
\definecolor{currentstroke}{rgb}{0.500000,0.500000,0.500000}%
\pgfsetstrokecolor{currentstroke}%
\pgfsetstrokeopacity{0.300000}%
\pgfsetdash{}{0pt}%
\pgfpathmoveto{\pgfqpoint{0.647939in}{2.242044in}}%
\pgfpathlineto{\pgfqpoint{0.647939in}{2.242044in}}%
\pgfpathlineto{\pgfqpoint{0.715675in}{2.251724in}}%
\pgfpathlineto{\pgfqpoint{0.783169in}{2.262965in}}%
\pgfpathlineto{\pgfqpoint{0.850376in}{2.275803in}}%
\pgfpathlineto{\pgfqpoint{0.917261in}{2.290233in}}%
\pgfpathlineto{\pgfqpoint{0.983795in}{2.306202in}}%
\pgfpathlineto{\pgfqpoint{1.049967in}{2.323611in}}%
\pgfpathlineto{\pgfqpoint{1.115788in}{2.342314in}}%
\pgfpathlineto{\pgfqpoint{1.181287in}{2.362113in}}%
\pgfpathlineto{\pgfqpoint{1.246520in}{2.382776in}}%
\pgfpathlineto{\pgfqpoint{1.311563in}{2.404032in}}%
\pgfpathlineto{\pgfqpoint{1.376510in}{2.425582in}}%
\pgfpathlineto{\pgfqpoint{1.441468in}{2.447098in}}%
\pgfpathlineto{\pgfqpoint{1.506552in}{2.468228in}}%
\pgfpathlineto{\pgfqpoint{1.571874in}{2.488606in}}%
\pgfpathlineto{\pgfqpoint{1.637535in}{2.507858in}}%
\pgfpathlineto{\pgfqpoint{1.703612in}{2.525620in}}%
\pgfpathlineto{\pgfqpoint{1.770149in}{2.541561in}}%
\pgfpathlineto{\pgfqpoint{1.837151in}{2.555416in}}%
\pgfpathlineto{\pgfqpoint{1.904575in}{2.567028in}}%
\pgfpathlineto{\pgfqpoint{1.972350in}{2.576391in}}%
\pgfpathlineto{\pgfqpoint{2.040378in}{2.583706in}}%
\pgfpathlineto{\pgfqpoint{2.108564in}{2.589432in}}%
\pgfpathlineto{\pgfqpoint{2.176814in}{2.594360in}}%
\pgfpathlineto{\pgfqpoint{2.245035in}{2.599657in}}%
\pgfpathlineto{\pgfqpoint{2.313065in}{2.606863in}}%
\pgfpathlineto{\pgfqpoint{2.380559in}{2.617820in}}%
\pgfpathlineto{\pgfqpoint{2.446859in}{2.634354in}}%
\pgfpathlineto{\pgfqpoint{2.511057in}{2.657664in}}%
\pgfpathlineto{\pgfqpoint{2.572338in}{2.687815in}}%
\pgfpathlineto{\pgfqpoint{2.630373in}{2.723853in}}%
\pgfpathlineto{\pgfqpoint{2.685373in}{2.764417in}}%
\pgfpathlineto{\pgfqpoint{2.737840in}{2.808250in}}%
\pgfpathlineto{\pgfqpoint{2.788336in}{2.854376in}}%
\pgfpathlineto{\pgfqpoint{2.837361in}{2.902075in}}%
\pgfpathlineto{\pgfqpoint{2.885333in}{2.950846in}}%
\pgfusepath{stroke}%
\end{pgfscope}%
\begin{pgfscope}%
\pgfpathrectangle{\pgfqpoint{0.647939in}{0.492442in}}{\pgfqpoint{3.079299in}{3.079299in}}%
\pgfusepath{clip}%
\pgfsetbuttcap%
\pgfsetroundjoin%
\pgfsetlinewidth{0.301125pt}%
\definecolor{currentstroke}{rgb}{0.500000,0.500000,0.500000}%
\pgfsetstrokecolor{currentstroke}%
\pgfsetstrokeopacity{0.300000}%
\pgfsetdash{}{0pt}%
\pgfpathmoveto{\pgfqpoint{0.647939in}{2.172060in}}%
\pgfpathlineto{\pgfqpoint{0.647939in}{2.172060in}}%
\pgfpathlineto{\pgfqpoint{0.715655in}{2.181877in}}%
\pgfpathlineto{\pgfqpoint{0.783119in}{2.193291in}}%
\pgfpathlineto{\pgfqpoint{0.850286in}{2.206343in}}%
\pgfpathlineto{\pgfqpoint{0.917113in}{2.221032in}}%
\pgfpathlineto{\pgfqpoint{0.983571in}{2.237312in}}%
\pgfpathlineto{\pgfqpoint{1.049646in}{2.255087in}}%
\pgfpathlineto{\pgfqpoint{1.115344in}{2.274213in}}%
\pgfpathlineto{\pgfqpoint{1.180695in}{2.294497in}}%
\pgfpathlineto{\pgfqpoint{1.245753in}{2.315705in}}%
\pgfpathlineto{\pgfqpoint{1.310594in}{2.337569in}}%
\pgfpathlineto{\pgfqpoint{1.375315in}{2.359788in}}%
\pgfpathlineto{\pgfqpoint{1.440027in}{2.382032in}}%
\pgfpathlineto{\pgfqpoint{1.504851in}{2.403947in}}%
\pgfpathlineto{\pgfqpoint{1.569907in}{2.425159in}}%
\pgfpathlineto{\pgfqpoint{1.635305in}{2.445285in}}%
\pgfpathlineto{\pgfqpoint{1.701133in}{2.463947in}}%
\pgfpathlineto{\pgfqpoint{1.767448in}{2.480792in}}%
\pgfusepath{stroke}%
\end{pgfscope}%
\begin{pgfscope}%
\pgfpathrectangle{\pgfqpoint{0.647939in}{0.492442in}}{\pgfqpoint{3.079299in}{3.079299in}}%
\pgfusepath{clip}%
\pgfsetbuttcap%
\pgfsetroundjoin%
\pgfsetlinewidth{0.301125pt}%
\definecolor{currentstroke}{rgb}{0.500000,0.500000,0.500000}%
\pgfsetstrokecolor{currentstroke}%
\pgfsetstrokeopacity{0.300000}%
\pgfsetdash{}{0pt}%
\pgfpathmoveto{\pgfqpoint{0.647939in}{2.102076in}}%
\pgfpathlineto{\pgfqpoint{0.647939in}{2.102076in}}%
\pgfpathlineto{\pgfqpoint{0.715634in}{2.112034in}}%
\pgfpathlineto{\pgfqpoint{0.783068in}{2.123625in}}%
\pgfpathlineto{\pgfqpoint{0.850191in}{2.136898in}}%
\pgfpathlineto{\pgfqpoint{0.916958in}{2.151857in}}%
\pgfpathlineto{\pgfqpoint{0.983336in}{2.168459in}}%
\pgfpathlineto{\pgfqpoint{1.049307in}{2.186614in}}%
\pgfpathlineto{\pgfqpoint{1.114874in}{2.206182in}}%
\pgfpathlineto{\pgfqpoint{1.180066in}{2.226971in}}%
\pgfpathlineto{\pgfqpoint{1.244934in}{2.248751in}}%
\pgfpathlineto{\pgfqpoint{1.309556in}{2.271253in}}%
\pgfpathlineto{\pgfqpoint{1.374031in}{2.294177in}}%
\pgfpathlineto{\pgfqpoint{1.438473in}{2.317191in}}%
\pgfpathlineto{\pgfqpoint{1.503009in}{2.339940in}}%
\pgfpathlineto{\pgfqpoint{1.567768in}{2.362043in}}%
\pgfpathlineto{\pgfqpoint{1.632870in}{2.383109in}}%
\pgfpathlineto{\pgfqpoint{1.698414in}{2.402747in}}%
\pgfusepath{stroke}%
\end{pgfscope}%
\begin{pgfscope}%
\pgfpathrectangle{\pgfqpoint{0.647939in}{0.492442in}}{\pgfqpoint{3.079299in}{3.079299in}}%
\pgfusepath{clip}%
\pgfsetbuttcap%
\pgfsetroundjoin%
\pgfsetlinewidth{0.301125pt}%
\definecolor{currentstroke}{rgb}{0.500000,0.500000,0.500000}%
\pgfsetstrokecolor{currentstroke}%
\pgfsetstrokeopacity{0.300000}%
\pgfsetdash{}{0pt}%
\pgfpathmoveto{\pgfqpoint{0.647939in}{1.962108in}}%
\pgfpathlineto{\pgfqpoint{0.647939in}{1.962108in}}%
\pgfpathlineto{\pgfqpoint{0.715590in}{1.972359in}}%
\pgfpathlineto{\pgfqpoint{0.782958in}{1.984324in}}%
\pgfpathlineto{\pgfqpoint{0.849986in}{1.998060in}}%
\pgfpathlineto{\pgfqpoint{0.916622in}{2.013586in}}%
\pgfpathlineto{\pgfqpoint{0.982825in}{2.030871in}}%
\pgfpathlineto{\pgfqpoint{1.048568in}{2.049832in}}%
\pgfpathlineto{\pgfqpoint{1.113847in}{2.070340in}}%
\pgfpathlineto{\pgfqpoint{1.178683in}{2.092210in}}%
\pgfpathlineto{\pgfqpoint{1.243126in}{2.115216in}}%
\pgfpathlineto{\pgfqpoint{1.307252in}{2.139093in}}%
\pgfpathlineto{\pgfqpoint{1.371163in}{2.163543in}}%
\pgfpathlineto{\pgfqpoint{1.434981in}{2.188236in}}%
\pgfpathlineto{\pgfqpoint{1.498844in}{2.212812in}}%
\pgfpathlineto{\pgfqpoint{1.562896in}{2.236886in}}%
\pgfpathlineto{\pgfqpoint{1.627281in}{2.260053in}}%
\pgfpathlineto{\pgfqpoint{1.692123in}{2.281897in}}%
\pgfpathlineto{\pgfqpoint{1.757520in}{2.302009in}}%
\pgfpathlineto{\pgfqpoint{1.823526in}{2.320014in}}%
\pgfpathlineto{\pgfqpoint{1.890138in}{2.335617in}}%
\pgfpathlineto{\pgfqpoint{1.957295in}{2.348673in}}%
\pgfpathlineto{\pgfqpoint{2.024885in}{2.359280in}}%
\pgfpathlineto{\pgfqpoint{2.092766in}{2.367876in}}%
\pgfpathlineto{\pgfqpoint{2.160782in}{2.375369in}}%
\pgfpathlineto{\pgfqpoint{2.228741in}{2.383327in}}%
\pgfpathlineto{\pgfqpoint{2.296273in}{2.394141in}}%
\pgfpathlineto{\pgfqpoint{2.362481in}{2.410880in}}%
\pgfpathlineto{\pgfqpoint{2.425835in}{2.436029in}}%
\pgfpathlineto{\pgfqpoint{2.485033in}{2.469793in}}%
\pgfpathlineto{\pgfqpoint{2.539924in}{2.510354in}}%
\pgfusepath{stroke}%
\end{pgfscope}%
\begin{pgfscope}%
\pgfpathrectangle{\pgfqpoint{0.647939in}{0.492442in}}{\pgfqpoint{3.079299in}{3.079299in}}%
\pgfusepath{clip}%
\pgfsetbuttcap%
\pgfsetroundjoin%
\pgfsetlinewidth{0.301125pt}%
\definecolor{currentstroke}{rgb}{0.500000,0.500000,0.500000}%
\pgfsetstrokecolor{currentstroke}%
\pgfsetstrokeopacity{0.300000}%
\pgfsetdash{}{0pt}%
\pgfpathmoveto{\pgfqpoint{0.647939in}{1.892124in}}%
\pgfpathlineto{\pgfqpoint{0.647939in}{1.892124in}}%
\pgfpathlineto{\pgfqpoint{0.715566in}{1.902528in}}%
\pgfpathlineto{\pgfqpoint{0.782899in}{1.914688in}}%
\pgfpathlineto{\pgfqpoint{0.849876in}{1.928669in}}%
\pgfpathlineto{\pgfqpoint{0.916442in}{1.944494in}}%
\pgfpathlineto{\pgfqpoint{0.982548in}{1.962140in}}%
\pgfpathlineto{\pgfqpoint{1.048165in}{1.981530in}}%
\pgfpathlineto{\pgfqpoint{1.113284in}{2.002537in}}%
\pgfpathlineto{\pgfqpoint{1.177922in}{2.024985in}}%
\pgfpathlineto{\pgfqpoint{1.242126in}{2.048649in}}%
\pgfusepath{stroke}%
\end{pgfscope}%
\begin{pgfscope}%
\pgfpathrectangle{\pgfqpoint{0.647939in}{0.492442in}}{\pgfqpoint{3.079299in}{3.079299in}}%
\pgfusepath{clip}%
\pgfsetbuttcap%
\pgfsetroundjoin%
\pgfsetlinewidth{0.301125pt}%
\definecolor{currentstroke}{rgb}{0.500000,0.500000,0.500000}%
\pgfsetstrokecolor{currentstroke}%
\pgfsetstrokeopacity{0.300000}%
\pgfsetdash{}{0pt}%
\pgfpathmoveto{\pgfqpoint{0.647939in}{1.822139in}}%
\pgfpathlineto{\pgfqpoint{0.647939in}{1.822139in}}%
\pgfpathlineto{\pgfqpoint{0.715541in}{1.832702in}}%
\pgfpathlineto{\pgfqpoint{0.782837in}{1.845064in}}%
\pgfpathlineto{\pgfqpoint{0.849761in}{1.859297in}}%
\pgfpathlineto{\pgfqpoint{0.916251in}{1.875433in}}%
\pgfpathlineto{\pgfqpoint{0.982255in}{1.893453in}}%
\pgfpathlineto{\pgfqpoint{1.047738in}{1.913289in}}%
\pgfpathlineto{\pgfqpoint{1.112686in}{1.934819in}}%
\pgfpathlineto{\pgfqpoint{1.177110in}{1.957871in}}%
\pgfpathlineto{\pgfqpoint{1.241055in}{1.982226in}}%
\pgfpathlineto{\pgfqpoint{1.304594in}{2.007623in}}%
\pgfpathlineto{\pgfqpoint{1.367828in}{2.033771in}}%
\pgfpathlineto{\pgfqpoint{1.430886in}{2.060344in}}%
\pgfpathlineto{\pgfqpoint{1.493915in}{2.086986in}}%
\pgfpathlineto{\pgfqpoint{1.557076in}{2.113312in}}%
\pgfpathlineto{\pgfqpoint{1.620534in}{2.138911in}}%
\pgfpathlineto{\pgfqpoint{1.684444in}{2.163352in}}%
\pgfpathlineto{\pgfqpoint{1.748937in}{2.186197in}}%
\pgfpathlineto{\pgfqpoint{1.814106in}{2.207025in}}%
\pgfpathlineto{\pgfqpoint{1.879985in}{2.225473in}}%
\pgfpathlineto{\pgfqpoint{1.946540in}{2.241303in}}%
\pgfpathlineto{\pgfqpoint{2.013669in}{2.254502in}}%
\pgfpathlineto{\pgfqpoint{2.081210in}{2.265438in}}%
\pgfpathlineto{\pgfqpoint{2.148952in}{2.275097in}}%
\pgfpathlineto{\pgfqpoint{2.216594in}{2.285368in}}%
\pgfpathlineto{\pgfqpoint{2.283491in}{2.299367in}}%
\pgfpathlineto{\pgfqpoint{2.348106in}{2.321017in}}%
\pgfpathlineto{\pgfqpoint{2.408350in}{2.352491in}}%
\pgfpathlineto{\pgfqpoint{2.463500in}{2.392346in}}%
\pgfusepath{stroke}%
\end{pgfscope}%
\begin{pgfscope}%
\pgfpathrectangle{\pgfqpoint{0.647939in}{0.492442in}}{\pgfqpoint{3.079299in}{3.079299in}}%
\pgfusepath{clip}%
\pgfsetbuttcap%
\pgfsetroundjoin%
\pgfsetlinewidth{0.301125pt}%
\definecolor{currentstroke}{rgb}{0.500000,0.500000,0.500000}%
\pgfsetstrokecolor{currentstroke}%
\pgfsetstrokeopacity{0.300000}%
\pgfsetdash{}{0pt}%
\pgfpathmoveto{\pgfqpoint{0.647939in}{1.752155in}}%
\pgfpathlineto{\pgfqpoint{0.647939in}{1.752155in}}%
\pgfpathlineto{\pgfqpoint{0.715516in}{1.762881in}}%
\pgfpathlineto{\pgfqpoint{0.782772in}{1.775450in}}%
\pgfpathlineto{\pgfqpoint{0.849639in}{1.789945in}}%
\pgfpathlineto{\pgfqpoint{0.916049in}{1.806403in}}%
\pgfpathlineto{\pgfqpoint{0.981946in}{1.824814in}}%
\pgfpathlineto{\pgfqpoint{1.047285in}{1.845115in}}%
\pgfpathlineto{\pgfqpoint{1.112048in}{1.867191in}}%
\pgfpathlineto{\pgfqpoint{1.176242in}{1.890875in}}%
\pgfpathlineto{\pgfqpoint{1.239906in}{1.915954in}}%
\pgfpathlineto{\pgfqpoint{1.303110in}{1.942171in}}%
\pgfpathlineto{\pgfqpoint{1.365956in}{1.969238in}}%
\pgfpathlineto{\pgfqpoint{1.428573in}{1.996834in}}%
\pgfpathlineto{\pgfqpoint{1.491112in}{2.024607in}}%
\pgfpathlineto{\pgfqpoint{1.553741in}{2.052174in}}%
\pgfpathlineto{\pgfqpoint{1.616636in}{2.079126in}}%
\pgfpathlineto{\pgfqpoint{1.679969in}{2.105030in}}%
\pgfpathlineto{\pgfqpoint{1.743888in}{2.129437in}}%
\pgfusepath{stroke}%
\end{pgfscope}%
\begin{pgfscope}%
\pgfpathrectangle{\pgfqpoint{0.647939in}{0.492442in}}{\pgfqpoint{3.079299in}{3.079299in}}%
\pgfusepath{clip}%
\pgfsetbuttcap%
\pgfsetroundjoin%
\pgfsetlinewidth{0.301125pt}%
\definecolor{currentstroke}{rgb}{0.500000,0.500000,0.500000}%
\pgfsetstrokecolor{currentstroke}%
\pgfsetstrokeopacity{0.300000}%
\pgfsetdash{}{0pt}%
\pgfpathmoveto{\pgfqpoint{0.647939in}{1.682171in}}%
\pgfpathlineto{\pgfqpoint{0.647939in}{1.682171in}}%
\pgfpathlineto{\pgfqpoint{0.715488in}{1.693065in}}%
\pgfpathlineto{\pgfqpoint{0.782704in}{1.705849in}}%
\pgfpathlineto{\pgfqpoint{0.849511in}{1.720615in}}%
\pgfpathlineto{\pgfqpoint{0.915837in}{1.737407in}}%
\pgfpathlineto{\pgfqpoint{0.981617in}{1.756225in}}%
\pgfpathlineto{\pgfqpoint{1.046803in}{1.777011in}}%
\pgfpathlineto{\pgfqpoint{1.111368in}{1.799657in}}%
\pgfpathlineto{\pgfqpoint{1.175313in}{1.824002in}}%
\pgfpathlineto{\pgfqpoint{1.238672in}{1.849840in}}%
\pgfpathlineto{\pgfqpoint{1.301511in}{1.876919in}}%
\pgfpathlineto{\pgfqpoint{1.363930in}{1.904956in}}%
\pgfusepath{stroke}%
\end{pgfscope}%
\begin{pgfscope}%
\pgfpathrectangle{\pgfqpoint{0.647939in}{0.492442in}}{\pgfqpoint{3.079299in}{3.079299in}}%
\pgfusepath{clip}%
\pgfsetbuttcap%
\pgfsetroundjoin%
\pgfsetlinewidth{0.301125pt}%
\definecolor{currentstroke}{rgb}{0.500000,0.500000,0.500000}%
\pgfsetstrokecolor{currentstroke}%
\pgfsetstrokeopacity{0.300000}%
\pgfsetdash{}{0pt}%
\pgfpathmoveto{\pgfqpoint{0.647939in}{1.612187in}}%
\pgfpathlineto{\pgfqpoint{0.647939in}{1.612187in}}%
\pgfpathlineto{\pgfqpoint{0.715460in}{1.623254in}}%
\pgfpathlineto{\pgfqpoint{0.782632in}{1.636261in}}%
\pgfpathlineto{\pgfqpoint{0.849376in}{1.651307in}}%
\pgfpathlineto{\pgfqpoint{0.915612in}{1.668447in}}%
\pgfpathlineto{\pgfqpoint{0.981269in}{1.687687in}}%
\pgfpathlineto{\pgfqpoint{1.046290in}{1.708980in}}%
\pgfpathlineto{\pgfqpoint{1.110642in}{1.732222in}}%
\pgfpathlineto{\pgfqpoint{1.174318in}{1.757260in}}%
\pgfpathlineto{\pgfqpoint{1.237346in}{1.783891in}}%
\pgfpathlineto{\pgfqpoint{1.299787in}{1.811874in}}%
\pgfpathlineto{\pgfqpoint{1.361739in}{1.840928in}}%
\pgfpathlineto{\pgfqpoint{1.423329in}{1.870743in}}%
\pgfpathlineto{\pgfqpoint{1.484714in}{1.900978in}}%
\pgfpathlineto{\pgfqpoint{1.546074in}{1.931266in}}%
\pgfpathlineto{\pgfqpoint{1.607601in}{1.961210in}}%
\pgfpathlineto{\pgfqpoint{1.669495in}{1.990385in}}%
\pgfpathlineto{\pgfqpoint{1.731946in}{2.018342in}}%
\pgfpathlineto{\pgfqpoint{1.795118in}{2.044621in}}%
\pgfpathlineto{\pgfqpoint{1.859127in}{2.068777in}}%
\pgfpathlineto{\pgfqpoint{1.924016in}{2.090442in}}%
\pgfpathlineto{\pgfqpoint{1.989733in}{2.109453in}}%
\pgfpathlineto{\pgfqpoint{2.056109in}{2.126036in}}%
\pgfpathlineto{\pgfqpoint{2.122841in}{2.141160in}}%
\pgfpathlineto{\pgfqpoint{2.189332in}{2.157195in}}%
\pgfpathlineto{\pgfqpoint{2.254060in}{2.178678in}}%
\pgfpathlineto{\pgfqpoint{2.313987in}{2.210306in}}%
\pgfusepath{stroke}%
\end{pgfscope}%
\begin{pgfscope}%
\pgfpathrectangle{\pgfqpoint{0.647939in}{0.492442in}}{\pgfqpoint{3.079299in}{3.079299in}}%
\pgfusepath{clip}%
\pgfsetbuttcap%
\pgfsetroundjoin%
\pgfsetlinewidth{0.301125pt}%
\definecolor{currentstroke}{rgb}{0.500000,0.500000,0.500000}%
\pgfsetstrokecolor{currentstroke}%
\pgfsetstrokeopacity{0.300000}%
\pgfsetdash{}{0pt}%
\pgfpathmoveto{\pgfqpoint{0.647939in}{1.542203in}}%
\pgfpathlineto{\pgfqpoint{0.647939in}{1.542203in}}%
\pgfpathlineto{\pgfqpoint{0.715430in}{1.553448in}}%
\pgfpathlineto{\pgfqpoint{0.782557in}{1.566685in}}%
\pgfpathlineto{\pgfqpoint{0.849234in}{1.582022in}}%
\pgfpathlineto{\pgfqpoint{0.915374in}{1.599524in}}%
\pgfpathlineto{\pgfqpoint{0.980900in}{1.619205in}}%
\pgfpathlineto{\pgfqpoint{1.045744in}{1.641026in}}%
\pgfpathlineto{\pgfqpoint{1.109866in}{1.664892in}}%
\pgfpathlineto{\pgfqpoint{1.173250in}{1.690655in}}%
\pgfusepath{stroke}%
\end{pgfscope}%
\begin{pgfscope}%
\pgfpathrectangle{\pgfqpoint{0.647939in}{0.492442in}}{\pgfqpoint{3.079299in}{3.079299in}}%
\pgfusepath{clip}%
\pgfsetbuttcap%
\pgfsetroundjoin%
\pgfsetlinewidth{0.301125pt}%
\definecolor{currentstroke}{rgb}{0.500000,0.500000,0.500000}%
\pgfsetstrokecolor{currentstroke}%
\pgfsetstrokeopacity{0.300000}%
\pgfsetdash{}{0pt}%
\pgfpathmoveto{\pgfqpoint{0.647939in}{1.472219in}}%
\pgfpathlineto{\pgfqpoint{0.647939in}{1.472219in}}%
\pgfpathlineto{\pgfqpoint{0.715399in}{1.483648in}}%
\pgfpathlineto{\pgfqpoint{0.782477in}{1.497123in}}%
\pgfpathlineto{\pgfqpoint{0.849083in}{1.512762in}}%
\pgfpathlineto{\pgfqpoint{0.915123in}{1.530640in}}%
\pgfpathlineto{\pgfqpoint{0.980507in}{1.550780in}}%
\pgfpathlineto{\pgfqpoint{1.045162in}{1.573153in}}%
\pgfpathlineto{\pgfqpoint{1.109035in}{1.597672in}}%
\pgfpathlineto{\pgfqpoint{1.172104in}{1.624196in}}%
\pgfusepath{stroke}%
\end{pgfscope}%
\begin{pgfscope}%
\pgfpathrectangle{\pgfqpoint{0.647939in}{0.492442in}}{\pgfqpoint{3.079299in}{3.079299in}}%
\pgfusepath{clip}%
\pgfsetbuttcap%
\pgfsetroundjoin%
\pgfsetlinewidth{0.301125pt}%
\definecolor{currentstroke}{rgb}{0.500000,0.500000,0.500000}%
\pgfsetstrokecolor{currentstroke}%
\pgfsetstrokeopacity{0.300000}%
\pgfsetdash{}{0pt}%
\pgfpathmoveto{\pgfqpoint{0.647939in}{1.402235in}}%
\pgfpathlineto{\pgfqpoint{0.647939in}{1.402235in}}%
\pgfpathlineto{\pgfqpoint{0.715366in}{1.413855in}}%
\pgfpathlineto{\pgfqpoint{0.782394in}{1.427576in}}%
\pgfpathlineto{\pgfqpoint{0.848925in}{1.443529in}}%
\pgfpathlineto{\pgfqpoint{0.914856in}{1.461797in}}%
\pgfpathlineto{\pgfqpoint{0.980089in}{1.482417in}}%
\pgfpathlineto{\pgfqpoint{1.044540in}{1.505367in}}%
\pgfpathlineto{\pgfqpoint{1.108145in}{1.530569in}}%
\pgfpathlineto{\pgfqpoint{1.170871in}{1.557890in}}%
\pgfpathlineto{\pgfqpoint{1.232718in}{1.587148in}}%
\pgfpathlineto{\pgfqpoint{1.293728in}{1.618118in}}%
\pgfpathlineto{\pgfqpoint{1.353980in}{1.650542in}}%
\pgfpathlineto{\pgfqpoint{1.413589in}{1.684135in}}%
\pgfpathlineto{\pgfqpoint{1.472708in}{1.718585in}}%
\pgfpathlineto{\pgfqpoint{1.531519in}{1.753561in}}%
\pgfpathlineto{\pgfqpoint{1.590229in}{1.788706in}}%
\pgfpathlineto{\pgfqpoint{1.649063in}{1.823641in}}%
\pgfpathlineto{\pgfqpoint{1.708260in}{1.857959in}}%
\pgfpathlineto{\pgfqpoint{1.768051in}{1.891227in}}%
\pgfpathlineto{\pgfqpoint{1.828644in}{1.923008in}}%
\pgfpathlineto{\pgfqpoint{1.890190in}{1.952893in}}%
\pgfpathlineto{\pgfqpoint{1.952743in}{1.980593in}}%
\pgfpathlineto{\pgfqpoint{2.016206in}{2.006123in}}%
\pgfpathlineto{\pgfqpoint{2.080219in}{2.030220in}}%
\pgfpathlineto{\pgfqpoint{2.143855in}{2.055180in}}%
\pgfpathlineto{\pgfqpoint{2.204724in}{2.085616in}}%
\pgfusepath{stroke}%
\end{pgfscope}%
\begin{pgfscope}%
\pgfpathrectangle{\pgfqpoint{0.647939in}{0.492442in}}{\pgfqpoint{3.079299in}{3.079299in}}%
\pgfusepath{clip}%
\pgfsetbuttcap%
\pgfsetroundjoin%
\pgfsetlinewidth{0.301125pt}%
\definecolor{currentstroke}{rgb}{0.500000,0.500000,0.500000}%
\pgfsetstrokecolor{currentstroke}%
\pgfsetstrokeopacity{0.300000}%
\pgfsetdash{}{0pt}%
\pgfpathmoveto{\pgfqpoint{0.647939in}{1.332251in}}%
\pgfpathlineto{\pgfqpoint{0.647939in}{1.332251in}}%
\pgfpathlineto{\pgfqpoint{0.715331in}{1.344067in}}%
\pgfpathlineto{\pgfqpoint{0.782306in}{1.358044in}}%
\pgfpathlineto{\pgfqpoint{0.848757in}{1.374322in}}%
\pgfpathlineto{\pgfqpoint{0.914572in}{1.392998in}}%
\pgfpathlineto{\pgfqpoint{0.979644in}{1.414118in}}%
\pgfusepath{stroke}%
\end{pgfscope}%
\begin{pgfscope}%
\pgfpathrectangle{\pgfqpoint{0.647939in}{0.492442in}}{\pgfqpoint{3.079299in}{3.079299in}}%
\pgfusepath{clip}%
\pgfsetbuttcap%
\pgfsetroundjoin%
\pgfsetlinewidth{0.301125pt}%
\definecolor{currentstroke}{rgb}{0.500000,0.500000,0.500000}%
\pgfsetstrokecolor{currentstroke}%
\pgfsetstrokeopacity{0.300000}%
\pgfsetdash{}{0pt}%
\pgfpathmoveto{\pgfqpoint{0.647939in}{1.262267in}}%
\pgfpathlineto{\pgfqpoint{0.647939in}{1.262267in}}%
\pgfpathlineto{\pgfqpoint{0.715295in}{1.274286in}}%
\pgfpathlineto{\pgfqpoint{0.782213in}{1.288528in}}%
\pgfpathlineto{\pgfqpoint{0.848579in}{1.305145in}}%
\pgfpathlineto{\pgfqpoint{0.914271in}{1.324245in}}%
\pgfpathlineto{\pgfqpoint{0.979170in}{1.345887in}}%
\pgfpathlineto{\pgfqpoint{1.043164in}{1.370072in}}%
\pgfpathlineto{\pgfqpoint{1.106165in}{1.396739in}}%
\pgfpathlineto{\pgfqpoint{1.168113in}{1.425770in}}%
\pgfpathlineto{\pgfqpoint{1.228985in}{1.456998in}}%
\pgfpathlineto{\pgfqpoint{1.288799in}{1.490211in}}%
\pgfpathlineto{\pgfqpoint{1.347615in}{1.525163in}}%
\pgfpathlineto{\pgfqpoint{1.405534in}{1.561585in}}%
\pgfpathlineto{\pgfqpoint{1.462695in}{1.599188in}}%
\pgfpathlineto{\pgfqpoint{1.519275in}{1.637665in}}%
\pgfpathlineto{\pgfqpoint{1.575486in}{1.676684in}}%
\pgfpathlineto{\pgfqpoint{1.631563in}{1.715891in}}%
\pgfpathlineto{\pgfqpoint{1.687752in}{1.754929in}}%
\pgfpathlineto{\pgfqpoint{1.744313in}{1.793424in}}%
\pgfpathlineto{\pgfqpoint{1.801502in}{1.830984in}}%
\pgfusepath{stroke}%
\end{pgfscope}%
\begin{pgfscope}%
\pgfpathrectangle{\pgfqpoint{0.647939in}{0.492442in}}{\pgfqpoint{3.079299in}{3.079299in}}%
\pgfusepath{clip}%
\pgfsetbuttcap%
\pgfsetroundjoin%
\pgfsetlinewidth{0.301125pt}%
\definecolor{currentstroke}{rgb}{0.500000,0.500000,0.500000}%
\pgfsetstrokecolor{currentstroke}%
\pgfsetstrokeopacity{0.300000}%
\pgfsetdash{}{0pt}%
\pgfpathmoveto{\pgfqpoint{0.647939in}{1.192283in}}%
\pgfpathlineto{\pgfqpoint{0.647939in}{1.192283in}}%
\pgfpathlineto{\pgfqpoint{0.715256in}{1.204513in}}%
\pgfpathlineto{\pgfqpoint{0.782115in}{1.219029in}}%
\pgfpathlineto{\pgfqpoint{0.848390in}{1.235997in}}%
\pgfpathlineto{\pgfqpoint{0.913951in}{1.255540in}}%
\pgfpathlineto{\pgfqpoint{0.978663in}{1.277728in}}%
\pgfpathlineto{\pgfqpoint{1.042402in}{1.302573in}}%
\pgfpathlineto{\pgfqpoint{1.105063in}{1.330025in}}%
\pgfpathlineto{\pgfqpoint{1.166570in}{1.359973in}}%
\pgfpathlineto{\pgfqpoint{1.226885in}{1.392256in}}%
\pgfpathlineto{\pgfqpoint{1.286014in}{1.426667in}}%
\pgfpathlineto{\pgfqpoint{1.344005in}{1.462967in}}%
\pgfpathlineto{\pgfqpoint{1.400949in}{1.500891in}}%
\pgfpathlineto{\pgfqpoint{1.456979in}{1.540159in}}%
\pgfusepath{stroke}%
\end{pgfscope}%
\begin{pgfscope}%
\pgfpathrectangle{\pgfqpoint{0.647939in}{0.492442in}}{\pgfqpoint{3.079299in}{3.079299in}}%
\pgfusepath{clip}%
\pgfsetbuttcap%
\pgfsetroundjoin%
\pgfsetlinewidth{0.301125pt}%
\definecolor{currentstroke}{rgb}{0.500000,0.500000,0.500000}%
\pgfsetstrokecolor{currentstroke}%
\pgfsetstrokeopacity{0.300000}%
\pgfsetdash{}{0pt}%
\pgfpathmoveto{\pgfqpoint{0.647939in}{1.122299in}}%
\pgfpathlineto{\pgfqpoint{0.647939in}{1.122299in}}%
\pgfpathlineto{\pgfqpoint{0.715216in}{1.134746in}}%
\pgfpathlineto{\pgfqpoint{0.782011in}{1.149548in}}%
\pgfpathlineto{\pgfqpoint{0.848190in}{1.166882in}}%
\pgfpathlineto{\pgfqpoint{0.913610in}{1.186887in}}%
\pgfpathlineto{\pgfqpoint{0.978122in}{1.209645in}}%
\pgfpathlineto{\pgfqpoint{1.041585in}{1.235182in}}%
\pgfpathlineto{\pgfqpoint{1.103876in}{1.263455in}}%
\pgfpathlineto{\pgfqpoint{1.164902in}{1.294363in}}%
\pgfpathlineto{\pgfqpoint{1.224610in}{1.327749in}}%
\pgfpathlineto{\pgfqpoint{1.282990in}{1.363410in}}%
\pgfpathlineto{\pgfqpoint{1.340078in}{1.401110in}}%
\pgfusepath{stroke}%
\end{pgfscope}%
\begin{pgfscope}%
\pgfpathrectangle{\pgfqpoint{0.647939in}{0.492442in}}{\pgfqpoint{3.079299in}{3.079299in}}%
\pgfusepath{clip}%
\pgfsetbuttcap%
\pgfsetroundjoin%
\pgfsetlinewidth{0.301125pt}%
\definecolor{currentstroke}{rgb}{0.500000,0.500000,0.500000}%
\pgfsetstrokecolor{currentstroke}%
\pgfsetstrokeopacity{0.300000}%
\pgfsetdash{}{0pt}%
\pgfpathmoveto{\pgfqpoint{0.647939in}{1.052315in}}%
\pgfpathlineto{\pgfqpoint{0.647939in}{1.052315in}}%
\pgfpathlineto{\pgfqpoint{0.715173in}{1.064988in}}%
\pgfpathlineto{\pgfqpoint{0.781901in}{1.080085in}}%
\pgfpathlineto{\pgfqpoint{0.847978in}{1.097801in}}%
\pgfpathlineto{\pgfqpoint{0.913247in}{1.118287in}}%
\pgfpathlineto{\pgfqpoint{0.977543in}{1.141643in}}%
\pgfpathlineto{\pgfqpoint{1.040706in}{1.167903in}}%
\pgfpathlineto{\pgfqpoint{1.102596in}{1.197037in}}%
\pgfusepath{stroke}%
\end{pgfscope}%
\begin{pgfscope}%
\pgfpathrectangle{\pgfqpoint{0.647939in}{0.492442in}}{\pgfqpoint{3.079299in}{3.079299in}}%
\pgfusepath{clip}%
\pgfsetbuttcap%
\pgfsetroundjoin%
\pgfsetlinewidth{0.301125pt}%
\definecolor{currentstroke}{rgb}{0.500000,0.500000,0.500000}%
\pgfsetstrokecolor{currentstroke}%
\pgfsetstrokeopacity{0.300000}%
\pgfsetdash{}{0pt}%
\pgfpathmoveto{\pgfqpoint{0.647939in}{0.982331in}}%
\pgfpathlineto{\pgfqpoint{0.647939in}{0.982331in}}%
\pgfpathlineto{\pgfqpoint{0.715128in}{0.995237in}}%
\pgfpathlineto{\pgfqpoint{0.781784in}{1.010642in}}%
\pgfpathlineto{\pgfqpoint{0.847752in}{1.028755in}}%
\pgfpathlineto{\pgfqpoint{0.912859in}{1.049745in}}%
\pgfpathlineto{\pgfqpoint{0.976922in}{1.073725in}}%
\pgfpathlineto{\pgfqpoint{1.039762in}{1.100743in}}%
\pgfpathlineto{\pgfqpoint{1.101216in}{1.130778in}}%
\pgfpathlineto{\pgfqpoint{1.161149in}{1.163740in}}%
\pgfpathlineto{\pgfqpoint{1.219472in}{1.199477in}}%
\pgfpathlineto{\pgfqpoint{1.276140in}{1.237787in}}%
\pgfpathlineto{\pgfqpoint{1.331167in}{1.278432in}}%
\pgfpathlineto{\pgfqpoint{1.384628in}{1.321113in}}%
\pgfpathlineto{\pgfqpoint{1.436653in}{1.365531in}}%
\pgfpathlineto{\pgfqpoint{1.487408in}{1.411403in}}%
\pgfpathlineto{\pgfqpoint{1.537121in}{1.458404in}}%
\pgfpathlineto{\pgfqpoint{1.586050in}{1.506213in}}%
\pgfpathlineto{\pgfqpoint{1.634483in}{1.554525in}}%
\pgfpathlineto{\pgfqpoint{1.682734in}{1.603010in}}%
\pgfpathlineto{\pgfqpoint{1.731130in}{1.651343in}}%
\pgfpathlineto{\pgfqpoint{1.780007in}{1.699179in}}%
\pgfpathlineto{\pgfqpoint{1.829683in}{1.746174in}}%
\pgfusepath{stroke}%
\end{pgfscope}%
\begin{pgfscope}%
\pgfpathrectangle{\pgfqpoint{0.647939in}{0.492442in}}{\pgfqpoint{3.079299in}{3.079299in}}%
\pgfusepath{clip}%
\pgfsetbuttcap%
\pgfsetroundjoin%
\pgfsetlinewidth{0.301125pt}%
\definecolor{currentstroke}{rgb}{0.500000,0.500000,0.500000}%
\pgfsetstrokecolor{currentstroke}%
\pgfsetstrokeopacity{0.300000}%
\pgfsetdash{}{0pt}%
\pgfpathmoveto{\pgfqpoint{0.647939in}{0.912347in}}%
\pgfpathlineto{\pgfqpoint{0.647939in}{0.912347in}}%
\pgfpathlineto{\pgfqpoint{0.715080in}{0.925495in}}%
\pgfpathlineto{\pgfqpoint{0.781661in}{0.941220in}}%
\pgfpathlineto{\pgfqpoint{0.847512in}{0.959748in}}%
\pgfpathlineto{\pgfqpoint{0.912445in}{0.981263in}}%
\pgfpathlineto{\pgfqpoint{0.976257in}{1.005896in}}%
\pgfpathlineto{\pgfqpoint{1.038746in}{1.033708in}}%
\pgfpathlineto{\pgfqpoint{1.099725in}{1.064686in}}%
\pgfpathlineto{\pgfqpoint{1.159038in}{1.098744in}}%
\pgfpathlineto{\pgfqpoint{1.216572in}{1.135729in}}%
\pgfpathlineto{\pgfqpoint{1.272270in}{1.175434in}}%
\pgfpathlineto{\pgfqpoint{1.326139in}{1.217593in}}%
\pgfusepath{stroke}%
\end{pgfscope}%
\begin{pgfscope}%
\pgfpathrectangle{\pgfqpoint{0.647939in}{0.492442in}}{\pgfqpoint{3.079299in}{3.079299in}}%
\pgfusepath{clip}%
\pgfsetbuttcap%
\pgfsetroundjoin%
\pgfsetlinewidth{0.301125pt}%
\definecolor{currentstroke}{rgb}{0.500000,0.500000,0.500000}%
\pgfsetstrokecolor{currentstroke}%
\pgfsetstrokeopacity{0.300000}%
\pgfsetdash{}{0pt}%
\pgfpathmoveto{\pgfqpoint{0.647939in}{0.842362in}}%
\pgfpathlineto{\pgfqpoint{0.647939in}{0.842362in}}%
\pgfpathlineto{\pgfqpoint{0.715030in}{0.855762in}}%
\pgfpathlineto{\pgfqpoint{0.781530in}{0.871820in}}%
\pgfpathlineto{\pgfqpoint{0.847256in}{0.890780in}}%
\pgfpathlineto{\pgfqpoint{0.912001in}{0.912846in}}%
\pgfpathlineto{\pgfqpoint{0.975542in}{0.938163in}}%
\pgfpathlineto{\pgfqpoint{1.037650in}{0.966806in}}%
\pgfpathlineto{\pgfqpoint{1.098113in}{0.998770in}}%
\pgfpathlineto{\pgfqpoint{1.156750in}{1.033971in}}%
\pgfpathlineto{\pgfqpoint{1.213428in}{1.072249in}}%
\pgfpathlineto{\pgfqpoint{1.268075in}{1.113386in}}%
\pgfpathlineto{\pgfqpoint{1.320697in}{1.157081in}}%
\pgfusepath{stroke}%
\end{pgfscope}%
\begin{pgfscope}%
\pgfpathrectangle{\pgfqpoint{0.647939in}{0.492442in}}{\pgfqpoint{3.079299in}{3.079299in}}%
\pgfusepath{clip}%
\pgfsetbuttcap%
\pgfsetroundjoin%
\pgfsetlinewidth{0.301125pt}%
\definecolor{currentstroke}{rgb}{0.500000,0.500000,0.500000}%
\pgfsetstrokecolor{currentstroke}%
\pgfsetstrokeopacity{0.300000}%
\pgfsetdash{}{0pt}%
\pgfpathmoveto{\pgfqpoint{0.647939in}{0.772378in}}%
\pgfpathlineto{\pgfqpoint{0.647939in}{0.772378in}}%
\pgfpathlineto{\pgfqpoint{0.714976in}{0.786039in}}%
\pgfpathlineto{\pgfqpoint{0.781390in}{0.802443in}}%
\pgfpathlineto{\pgfqpoint{0.846983in}{0.821855in}}%
\pgfpathlineto{\pgfqpoint{0.911526in}{0.844496in}}%
\pgfpathlineto{\pgfqpoint{0.974773in}{0.870530in}}%
\pgfpathlineto{\pgfqpoint{1.036468in}{0.900043in}}%
\pgfpathlineto{\pgfqpoint{1.096370in}{0.933038in}}%
\pgfpathlineto{\pgfqpoint{1.154272in}{0.969427in}}%
\pgfpathlineto{\pgfqpoint{1.210019in}{1.009043in}}%
\pgfpathlineto{\pgfqpoint{1.263534in}{1.051634in}}%
\pgfusepath{stroke}%
\end{pgfscope}%
\begin{pgfscope}%
\pgfpathrectangle{\pgfqpoint{0.647939in}{0.492442in}}{\pgfqpoint{3.079299in}{3.079299in}}%
\pgfusepath{clip}%
\pgfsetbuttcap%
\pgfsetroundjoin%
\pgfsetlinewidth{0.301125pt}%
\definecolor{currentstroke}{rgb}{0.500000,0.500000,0.500000}%
\pgfsetstrokecolor{currentstroke}%
\pgfsetstrokeopacity{0.300000}%
\pgfsetdash{}{0pt}%
\pgfpathmoveto{\pgfqpoint{0.647939in}{0.702394in}}%
\pgfpathlineto{\pgfqpoint{0.647939in}{0.702394in}}%
\pgfpathlineto{\pgfqpoint{0.714920in}{0.716325in}}%
\pgfpathlineto{\pgfqpoint{0.781242in}{0.733091in}}%
\pgfpathlineto{\pgfqpoint{0.846691in}{0.752974in}}%
\pgfpathlineto{\pgfqpoint{0.911017in}{0.776218in}}%
\pgfpathlineto{\pgfqpoint{0.973945in}{0.803003in}}%
\pgfpathlineto{\pgfqpoint{1.035191in}{0.833428in}}%
\pgfpathlineto{\pgfqpoint{1.094482in}{0.867498in}}%
\pgfpathlineto{\pgfqpoint{1.151584in}{0.905121in}}%
\pgfpathlineto{\pgfqpoint{1.206326in}{0.946115in}}%
\pgfpathlineto{\pgfqpoint{1.258626in}{0.990178in}}%
\pgfusepath{stroke}%
\end{pgfscope}%
\begin{pgfscope}%
\pgfpathrectangle{\pgfqpoint{0.647939in}{0.492442in}}{\pgfqpoint{3.079299in}{3.079299in}}%
\pgfusepath{clip}%
\pgfsetbuttcap%
\pgfsetroundjoin%
\pgfsetlinewidth{0.301125pt}%
\definecolor{currentstroke}{rgb}{0.500000,0.500000,0.500000}%
\pgfsetstrokecolor{currentstroke}%
\pgfsetstrokeopacity{0.300000}%
\pgfsetdash{}{0pt}%
\pgfpathmoveto{\pgfqpoint{0.647939in}{0.632410in}}%
\pgfpathlineto{\pgfqpoint{0.647939in}{0.632410in}}%
\pgfpathlineto{\pgfqpoint{0.714859in}{0.646623in}}%
\pgfpathlineto{\pgfqpoint{0.781084in}{0.663764in}}%
\pgfpathlineto{\pgfqpoint{0.846379in}{0.684141in}}%
\pgfpathlineto{\pgfqpoint{0.910470in}{0.708016in}}%
\pgfpathlineto{\pgfqpoint{0.973053in}{0.735587in}}%
\pgfpathlineto{\pgfqpoint{1.033809in}{0.766966in}}%
\pgfpathlineto{\pgfqpoint{1.092436in}{0.802157in}}%
\pgfpathlineto{\pgfqpoint{1.148671in}{0.841059in}}%
\pgfusepath{stroke}%
\end{pgfscope}%
\begin{pgfscope}%
\pgfpathrectangle{\pgfqpoint{0.647939in}{0.492442in}}{\pgfqpoint{3.079299in}{3.079299in}}%
\pgfusepath{clip}%
\pgfsetbuttcap%
\pgfsetroundjoin%
\pgfsetlinewidth{0.301125pt}%
\definecolor{currentstroke}{rgb}{0.500000,0.500000,0.500000}%
\pgfsetstrokecolor{currentstroke}%
\pgfsetstrokeopacity{0.300000}%
\pgfsetdash{}{0pt}%
\pgfpathmoveto{\pgfqpoint{3.727238in}{1.322831in}}%
\pgfpathlineto{\pgfqpoint{3.705898in}{1.334083in}}%
\pgfpathlineto{\pgfqpoint{3.645978in}{1.367103in}}%
\pgfpathlineto{\pgfqpoint{3.587270in}{1.402235in}}%
\pgfpathlineto{\pgfqpoint{3.529740in}{1.439267in}}%
\pgfpathlineto{\pgfqpoint{3.473343in}{1.478007in}}%
\pgfpathlineto{\pgfqpoint{3.418024in}{1.518273in}}%
\pgfpathlineto{\pgfqpoint{3.363719in}{1.559900in}}%
\pgfpathlineto{\pgfqpoint{3.310369in}{1.602743in}}%
\pgfusepath{stroke}%
\end{pgfscope}%
\begin{pgfscope}%
\pgfpathrectangle{\pgfqpoint{0.647939in}{0.492442in}}{\pgfqpoint{3.079299in}{3.079299in}}%
\pgfusepath{clip}%
\pgfsetbuttcap%
\pgfsetroundjoin%
\pgfsetlinewidth{0.301125pt}%
\definecolor{currentstroke}{rgb}{0.500000,0.500000,0.500000}%
\pgfsetstrokecolor{currentstroke}%
\pgfsetstrokeopacity{0.300000}%
\pgfsetdash{}{0pt}%
\pgfpathmoveto{\pgfqpoint{1.879036in}{0.603972in}}%
\pgfpathlineto{\pgfqpoint{1.811981in}{0.617307in}}%
\pgfpathlineto{\pgfqpoint{1.746488in}{0.636733in}}%
\pgfpathlineto{\pgfqpoint{1.684120in}{0.664351in}}%
\pgfpathlineto{\pgfqpoint{1.627716in}{0.702394in}}%
\pgfpathlineto{\pgfqpoint{1.581706in}{0.752077in}}%
\pgfpathlineto{\pgfqpoint{1.553162in}{0.802869in}}%
\pgfusepath{stroke}%
\end{pgfscope}%
\begin{pgfscope}%
\pgfpathrectangle{\pgfqpoint{0.647939in}{0.492442in}}{\pgfqpoint{3.079299in}{3.079299in}}%
\pgfusepath{clip}%
\pgfsetbuttcap%
\pgfsetroundjoin%
\pgfsetlinewidth{0.301125pt}%
\definecolor{currentstroke}{rgb}{0.500000,0.500000,0.500000}%
\pgfsetstrokecolor{currentstroke}%
\pgfsetstrokeopacity{0.300000}%
\pgfsetdash{}{0pt}%
\pgfpathmoveto{\pgfqpoint{3.517286in}{1.612187in}}%
\pgfpathlineto{\pgfqpoint{3.464211in}{1.655362in}}%
\pgfpathlineto{\pgfqpoint{3.412611in}{1.700289in}}%
\pgfpathlineto{\pgfqpoint{3.362489in}{1.746860in}}%
\pgfpathlineto{\pgfqpoint{3.313868in}{1.794996in}}%
\pgfpathlineto{\pgfqpoint{3.266808in}{1.844658in}}%
\pgfpathlineto{\pgfqpoint{3.221410in}{1.895843in}}%
\pgfpathlineto{\pgfqpoint{3.177836in}{1.948587in}}%
\pgfusepath{stroke}%
\end{pgfscope}%
\begin{pgfscope}%
\pgfpathrectangle{\pgfqpoint{0.647939in}{0.492442in}}{\pgfqpoint{3.079299in}{3.079299in}}%
\pgfusepath{clip}%
\pgfsetbuttcap%
\pgfsetroundjoin%
\pgfsetlinewidth{0.301125pt}%
\definecolor{currentstroke}{rgb}{0.500000,0.500000,0.500000}%
\pgfsetstrokecolor{currentstroke}%
\pgfsetstrokeopacity{0.300000}%
\pgfsetdash{}{0pt}%
\pgfpathmoveto{\pgfqpoint{3.517286in}{1.752155in}}%
\pgfpathlineto{\pgfqpoint{3.467686in}{1.799274in}}%
\pgfpathlineto{\pgfqpoint{3.420053in}{1.848380in}}%
\pgfpathlineto{\pgfqpoint{3.374478in}{1.899400in}}%
\pgfpathlineto{\pgfqpoint{3.331100in}{1.952300in}}%
\pgfpathlineto{\pgfqpoint{3.290122in}{2.007077in}}%
\pgfusepath{stroke}%
\end{pgfscope}%
\begin{pgfscope}%
\pgfpathrectangle{\pgfqpoint{0.647939in}{0.492442in}}{\pgfqpoint{3.079299in}{3.079299in}}%
\pgfusepath{clip}%
\pgfsetbuttcap%
\pgfsetroundjoin%
\pgfsetlinewidth{0.301125pt}%
\definecolor{currentstroke}{rgb}{0.500000,0.500000,0.500000}%
\pgfsetstrokecolor{currentstroke}%
\pgfsetstrokeopacity{0.300000}%
\pgfsetdash{}{0pt}%
\pgfpathmoveto{\pgfqpoint{3.447302in}{1.332251in}}%
\pgfpathlineto{\pgfqpoint{3.389862in}{1.369435in}}%
\pgfpathlineto{\pgfqpoint{3.333100in}{1.407646in}}%
\pgfpathlineto{\pgfqpoint{3.276915in}{1.446703in}}%
\pgfpathlineto{\pgfqpoint{3.221204in}{1.486434in}}%
\pgfpathlineto{\pgfqpoint{3.165861in}{1.526677in}}%
\pgfusepath{stroke}%
\end{pgfscope}%
\begin{pgfscope}%
\pgfpathrectangle{\pgfqpoint{0.647939in}{0.492442in}}{\pgfqpoint{3.079299in}{3.079299in}}%
\pgfusepath{clip}%
\pgfsetbuttcap%
\pgfsetroundjoin%
\pgfsetlinewidth{0.301125pt}%
\definecolor{currentstroke}{rgb}{0.500000,0.500000,0.500000}%
\pgfsetstrokecolor{currentstroke}%
\pgfsetstrokeopacity{0.300000}%
\pgfsetdash{}{0pt}%
\pgfpathmoveto{\pgfqpoint{1.921248in}{3.259385in}}%
\pgfpathlineto{\pgfqpoint{1.989495in}{3.264315in}}%
\pgfpathlineto{\pgfqpoint{2.057823in}{3.267981in}}%
\pgfpathlineto{\pgfqpoint{2.126195in}{3.270757in}}%
\pgfpathlineto{\pgfqpoint{2.194583in}{3.273130in}}%
\pgfpathlineto{\pgfqpoint{2.262963in}{3.275697in}}%
\pgfpathlineto{\pgfqpoint{2.331301in}{3.279157in}}%
\pgfpathlineto{\pgfqpoint{2.399529in}{3.284272in}}%
\pgfpathlineto{\pgfqpoint{2.467525in}{3.291805in}}%
\pgfusepath{stroke}%
\end{pgfscope}%
\begin{pgfscope}%
\pgfpathrectangle{\pgfqpoint{0.647939in}{0.492442in}}{\pgfqpoint{3.079299in}{3.079299in}}%
\pgfusepath{clip}%
\pgfsetbuttcap%
\pgfsetroundjoin%
\pgfsetlinewidth{0.301125pt}%
\definecolor{currentstroke}{rgb}{0.500000,0.500000,0.500000}%
\pgfsetstrokecolor{currentstroke}%
\pgfsetstrokeopacity{0.300000}%
\pgfsetdash{}{0pt}%
\pgfpathmoveto{\pgfqpoint{0.792998in}{3.058720in}}%
\pgfpathlineto{\pgfqpoint{0.860537in}{3.069699in}}%
\pgfpathlineto{\pgfqpoint{0.927875in}{3.081853in}}%
\pgfpathlineto{\pgfqpoint{0.995010in}{3.095088in}}%
\pgfpathlineto{\pgfqpoint{1.061951in}{3.109273in}}%
\pgfpathlineto{\pgfqpoint{1.128721in}{3.124240in}}%
\pgfpathlineto{\pgfqpoint{1.195359in}{3.139790in}}%
\pgfpathlineto{\pgfqpoint{1.261914in}{3.155695in}}%
\pgfpathlineto{\pgfqpoint{1.328445in}{3.171700in}}%
\pgfpathlineto{\pgfqpoint{1.395016in}{3.187535in}}%
\pgfpathlineto{\pgfqpoint{1.461693in}{3.202915in}}%
\pgfpathlineto{\pgfqpoint{1.528536in}{3.217557in}}%
\pgfpathlineto{\pgfqpoint{1.595589in}{3.231193in}}%
\pgfpathlineto{\pgfqpoint{1.662883in}{3.243580in}}%
\pgfpathlineto{\pgfqpoint{1.730428in}{3.254514in}}%
\pgfusepath{stroke}%
\end{pgfscope}%
\begin{pgfscope}%
\pgfpathrectangle{\pgfqpoint{0.647939in}{0.492442in}}{\pgfqpoint{3.079299in}{3.079299in}}%
\pgfusepath{clip}%
\pgfsetbuttcap%
\pgfsetroundjoin%
\pgfsetlinewidth{0.301125pt}%
\definecolor{currentstroke}{rgb}{0.500000,0.500000,0.500000}%
\pgfsetstrokecolor{currentstroke}%
\pgfsetstrokeopacity{0.300000}%
\pgfsetdash{}{0pt}%
\pgfpathmoveto{\pgfqpoint{3.237350in}{1.682171in}}%
\pgfpathlineto{\pgfqpoint{3.186574in}{1.728036in}}%
\pgfpathlineto{\pgfqpoint{3.136698in}{1.774878in}}%
\pgfpathlineto{\pgfqpoint{3.087746in}{1.822682in}}%
\pgfpathlineto{\pgfqpoint{3.039782in}{1.871477in}}%
\pgfpathlineto{\pgfqpoint{2.992940in}{1.921349in}}%
\pgfpathlineto{\pgfqpoint{2.947445in}{1.972446in}}%
\pgfpathlineto{\pgfqpoint{2.903668in}{2.025014in}}%
\pgfpathlineto{\pgfqpoint{2.862200in}{2.079415in}}%
\pgfpathlineto{\pgfqpoint{2.824006in}{2.136139in}}%
\pgfpathlineto{\pgfqpoint{2.790608in}{2.195760in}}%
\pgfpathlineto{\pgfqpoint{2.764263in}{2.258729in}}%
\pgfpathlineto{\pgfqpoint{2.747760in}{2.324858in}}%
\pgfpathlineto{\pgfqpoint{2.743229in}{2.392805in}}%
\pgfpathlineto{\pgfqpoint{2.750586in}{2.460523in}}%
\pgfpathlineto{\pgfqpoint{2.767676in}{2.526561in}}%
\pgfpathlineto{\pgfqpoint{2.791834in}{2.590438in}}%
\pgfpathlineto{\pgfqpoint{2.820891in}{2.652299in}}%
\pgfpathlineto{\pgfqpoint{2.853355in}{2.712475in}}%
\pgfpathlineto{\pgfqpoint{2.888260in}{2.771295in}}%
\pgfusepath{stroke}%
\end{pgfscope}%
\begin{pgfscope}%
\pgfpathrectangle{\pgfqpoint{0.647939in}{0.492442in}}{\pgfqpoint{3.079299in}{3.079299in}}%
\pgfusepath{clip}%
\pgfsetbuttcap%
\pgfsetroundjoin%
\pgfsetlinewidth{0.301125pt}%
\definecolor{currentstroke}{rgb}{0.500000,0.500000,0.500000}%
\pgfsetstrokecolor{currentstroke}%
\pgfsetstrokeopacity{0.300000}%
\pgfsetdash{}{0pt}%
\pgfpathmoveto{\pgfqpoint{3.237350in}{2.102076in}}%
\pgfpathlineto{\pgfqpoint{3.204348in}{2.161979in}}%
\pgfpathlineto{\pgfqpoint{3.175246in}{2.223859in}}%
\pgfpathlineto{\pgfqpoint{3.150690in}{2.287667in}}%
\pgfpathlineto{\pgfqpoint{3.131381in}{2.353241in}}%
\pgfpathlineto{\pgfqpoint{3.117984in}{2.420266in}}%
\pgfpathlineto{\pgfqpoint{3.110993in}{2.488250in}}%
\pgfpathlineto{\pgfqpoint{3.110598in}{2.556584in}}%
\pgfpathlineto{\pgfqpoint{3.116635in}{2.624653in}}%
\pgfpathlineto{\pgfqpoint{3.128635in}{2.691941in}}%
\pgfpathlineto{\pgfqpoint{3.145941in}{2.758081in}}%
\pgfpathlineto{\pgfqpoint{3.167851in}{2.822859in}}%
\pgfusepath{stroke}%
\end{pgfscope}%
\begin{pgfscope}%
\pgfpathrectangle{\pgfqpoint{0.647939in}{0.492442in}}{\pgfqpoint{3.079299in}{3.079299in}}%
\pgfusepath{clip}%
\pgfsetbuttcap%
\pgfsetroundjoin%
\pgfsetlinewidth{0.301125pt}%
\definecolor{currentstroke}{rgb}{0.500000,0.500000,0.500000}%
\pgfsetstrokecolor{currentstroke}%
\pgfsetstrokeopacity{0.300000}%
\pgfsetdash{}{0pt}%
\pgfpathmoveto{\pgfqpoint{3.097382in}{1.612187in}}%
\pgfpathlineto{\pgfqpoint{3.043252in}{1.654047in}}%
\pgfpathlineto{\pgfqpoint{2.989231in}{1.696045in}}%
\pgfpathlineto{\pgfqpoint{2.935228in}{1.738065in}}%
\pgfpathlineto{\pgfqpoint{2.881167in}{1.780009in}}%
\pgfpathlineto{\pgfqpoint{2.826987in}{1.821797in}}%
\pgfpathlineto{\pgfqpoint{2.772648in}{1.863378in}}%
\pgfpathlineto{\pgfqpoint{2.718163in}{1.904761in}}%
\pgfpathlineto{\pgfqpoint{2.663615in}{1.946055in}}%
\pgfpathlineto{\pgfqpoint{2.609236in}{1.987577in}}%
\pgfpathlineto{\pgfqpoint{2.555627in}{2.030062in}}%
\pgfpathlineto{\pgfqpoint{2.504305in}{2.075201in}}%
\pgfpathlineto{\pgfqpoint{2.459992in}{2.126755in}}%
\pgfpathlineto{\pgfqpoint{2.459992in}{2.126755in}}%
\pgfpathlineto{\pgfqpoint{2.441589in}{2.163900in}}%
\pgfpathlineto{\pgfqpoint{2.441589in}{2.163900in}}%
\pgfpathlineto{\pgfqpoint{2.436028in}{2.200711in}}%
\pgfpathlineto{\pgfqpoint{2.441544in}{2.237903in}}%
\pgfpathlineto{\pgfqpoint{2.455722in}{2.275846in}}%
\pgfusepath{stroke}%
\end{pgfscope}%
\begin{pgfscope}%
\pgfpathrectangle{\pgfqpoint{0.647939in}{0.492442in}}{\pgfqpoint{3.079299in}{3.079299in}}%
\pgfusepath{clip}%
\pgfsetbuttcap%
\pgfsetroundjoin%
\pgfsetlinewidth{0.301125pt}%
\definecolor{currentstroke}{rgb}{0.500000,0.500000,0.500000}%
\pgfsetstrokecolor{currentstroke}%
\pgfsetstrokeopacity{0.300000}%
\pgfsetdash{}{0pt}%
\pgfpathmoveto{\pgfqpoint{3.199952in}{1.661568in}}%
\pgfpathlineto{\pgfqpoint{3.148326in}{1.706477in}}%
\pgfpathlineto{\pgfqpoint{3.097382in}{1.752155in}}%
\pgfpathlineto{\pgfqpoint{3.047114in}{1.798578in}}%
\pgfpathlineto{\pgfqpoint{2.997551in}{1.845751in}}%
\pgfpathlineto{\pgfqpoint{2.948771in}{1.893732in}}%
\pgfpathlineto{\pgfqpoint{2.900931in}{1.942647in}}%
\pgfpathlineto{\pgfqpoint{2.854312in}{1.992717in}}%
\pgfpathlineto{\pgfqpoint{2.809395in}{2.044314in}}%
\pgfpathlineto{\pgfqpoint{2.767024in}{2.097995in}}%
\pgfpathlineto{\pgfqpoint{2.728674in}{2.154562in}}%
\pgfpathlineto{\pgfqpoint{2.696883in}{2.214949in}}%
\pgfpathlineto{\pgfqpoint{2.675508in}{2.279559in}}%
\pgfpathlineto{\pgfqpoint{2.668357in}{2.347038in}}%
\pgfpathlineto{\pgfqpoint{2.674596in}{2.408682in}}%
\pgfpathlineto{\pgfqpoint{2.691370in}{2.470652in}}%
\pgfusepath{stroke}%
\end{pgfscope}%
\begin{pgfscope}%
\pgfpathrectangle{\pgfqpoint{0.647939in}{0.492442in}}{\pgfqpoint{3.079299in}{3.079299in}}%
\pgfusepath{clip}%
\pgfsetbuttcap%
\pgfsetroundjoin%
\pgfsetlinewidth{0.301125pt}%
\definecolor{currentstroke}{rgb}{0.500000,0.500000,0.500000}%
\pgfsetstrokecolor{currentstroke}%
\pgfsetstrokeopacity{0.300000}%
\pgfsetdash{}{0pt}%
\pgfpathmoveto{\pgfqpoint{3.097382in}{2.032092in}}%
\pgfpathlineto{\pgfqpoint{3.059578in}{2.089095in}}%
\pgfpathlineto{\pgfqpoint{3.025030in}{2.148118in}}%
\pgfpathlineto{\pgfqpoint{2.994570in}{2.209333in}}%
\pgfpathlineto{\pgfqpoint{2.969258in}{2.272825in}}%
\pgfpathlineto{\pgfqpoint{2.950304in}{2.338461in}}%
\pgfpathlineto{\pgfqpoint{2.938836in}{2.405776in}}%
\pgfpathlineto{\pgfqpoint{2.935535in}{2.473968in}}%
\pgfpathlineto{\pgfqpoint{2.940349in}{2.542086in}}%
\pgfpathlineto{\pgfqpoint{2.952523in}{2.609315in}}%
\pgfpathlineto{\pgfqpoint{2.970923in}{2.675138in}}%
\pgfpathlineto{\pgfqpoint{2.994364in}{2.739357in}}%
\pgfpathlineto{\pgfqpoint{3.021822in}{2.801983in}}%
\pgfusepath{stroke}%
\end{pgfscope}%
\begin{pgfscope}%
\pgfpathrectangle{\pgfqpoint{0.647939in}{0.492442in}}{\pgfqpoint{3.079299in}{3.079299in}}%
\pgfusepath{clip}%
\pgfsetbuttcap%
\pgfsetroundjoin%
\pgfsetlinewidth{0.301125pt}%
\definecolor{currentstroke}{rgb}{0.500000,0.500000,0.500000}%
\pgfsetstrokecolor{currentstroke}%
\pgfsetstrokeopacity{0.300000}%
\pgfsetdash{}{0pt}%
\pgfpathmoveto{\pgfqpoint{1.729220in}{2.822722in}}%
\pgfpathlineto{\pgfqpoint{1.796553in}{2.834877in}}%
\pgfpathlineto{\pgfqpoint{1.864209in}{2.845083in}}%
\pgfpathlineto{\pgfqpoint{1.932132in}{2.853322in}}%
\pgfpathlineto{\pgfqpoint{2.000255in}{2.859718in}}%
\pgfpathlineto{\pgfqpoint{2.068508in}{2.864565in}}%
\pgfpathlineto{\pgfqpoint{2.136831in}{2.868346in}}%
\pgfpathlineto{\pgfqpoint{2.205175in}{2.871751in}}%
\pgfpathlineto{\pgfqpoint{2.273488in}{2.875689in}}%
\pgfpathlineto{\pgfqpoint{2.341679in}{2.881247in}}%
\pgfpathlineto{\pgfqpoint{2.409568in}{2.889619in}}%
\pgfpathlineto{\pgfqpoint{2.476826in}{2.901976in}}%
\pgfpathlineto{\pgfqpoint{2.542981in}{2.919246in}}%
\pgfpathlineto{\pgfqpoint{2.607493in}{2.941885in}}%
\pgfpathlineto{\pgfqpoint{2.669903in}{2.969788in}}%
\pgfpathlineto{\pgfqpoint{2.730027in}{3.002353in}}%
\pgfusepath{stroke}%
\end{pgfscope}%
\begin{pgfscope}%
\pgfpathrectangle{\pgfqpoint{0.647939in}{0.492442in}}{\pgfqpoint{3.079299in}{3.079299in}}%
\pgfusepath{clip}%
\pgfsetbuttcap%
\pgfsetroundjoin%
\pgfsetlinewidth{0.301125pt}%
\definecolor{currentstroke}{rgb}{0.500000,0.500000,0.500000}%
\pgfsetstrokecolor{currentstroke}%
\pgfsetstrokeopacity{0.300000}%
\pgfsetdash{}{0pt}%
\pgfpathmoveto{\pgfqpoint{1.212030in}{2.643044in}}%
\pgfpathlineto{\pgfqpoint{1.277796in}{2.661948in}}%
\pgfpathlineto{\pgfqpoint{1.343486in}{2.681114in}}%
\pgfpathlineto{\pgfqpoint{1.409188in}{2.700236in}}%
\pgfpathlineto{\pgfqpoint{1.474995in}{2.718994in}}%
\pgfpathlineto{\pgfqpoint{1.540996in}{2.737053in}}%
\pgfusepath{stroke}%
\end{pgfscope}%
\begin{pgfscope}%
\pgfpathrectangle{\pgfqpoint{0.647939in}{0.492442in}}{\pgfqpoint{3.079299in}{3.079299in}}%
\pgfusepath{clip}%
\pgfsetbuttcap%
\pgfsetroundjoin%
\pgfsetlinewidth{0.301125pt}%
\definecolor{currentstroke}{rgb}{0.500000,0.500000,0.500000}%
\pgfsetstrokecolor{currentstroke}%
\pgfsetstrokeopacity{0.300000}%
\pgfsetdash{}{0pt}%
\pgfpathmoveto{\pgfqpoint{2.073824in}{2.703617in}}%
\pgfpathlineto{\pgfqpoint{2.142106in}{2.708077in}}%
\pgfpathlineto{\pgfqpoint{2.210408in}{2.712235in}}%
\pgfpathlineto{\pgfqpoint{2.278650in}{2.717222in}}%
\pgfpathlineto{\pgfqpoint{2.346680in}{2.724416in}}%
\pgfpathlineto{\pgfqpoint{2.414185in}{2.735350in}}%
\pgfpathlineto{\pgfqpoint{2.480623in}{2.751445in}}%
\pgfpathlineto{\pgfqpoint{2.545280in}{2.773605in}}%
\pgfpathlineto{\pgfqpoint{2.607493in}{2.801916in}}%
\pgfpathlineto{\pgfqpoint{2.666906in}{2.835715in}}%
\pgfpathlineto{\pgfqpoint{2.723603in}{2.873911in}}%
\pgfusepath{stroke}%
\end{pgfscope}%
\begin{pgfscope}%
\pgfpathrectangle{\pgfqpoint{0.647939in}{0.492442in}}{\pgfqpoint{3.079299in}{3.079299in}}%
\pgfusepath{clip}%
\pgfsetbuttcap%
\pgfsetroundjoin%
\pgfsetlinewidth{0.301125pt}%
\definecolor{currentstroke}{rgb}{0.500000,0.500000,0.500000}%
\pgfsetstrokecolor{currentstroke}%
\pgfsetstrokeopacity{0.300000}%
\pgfsetdash{}{0pt}%
\pgfpathmoveto{\pgfqpoint{1.417764in}{2.521980in}}%
\pgfpathlineto{\pgfqpoint{1.483075in}{2.542399in}}%
\pgfpathlineto{\pgfqpoint{1.548589in}{2.562153in}}%
\pgfpathlineto{\pgfqpoint{1.614402in}{2.580880in}}%
\pgfpathlineto{\pgfqpoint{1.680590in}{2.598226in}}%
\pgfpathlineto{\pgfqpoint{1.747199in}{2.613870in}}%
\pgfpathlineto{\pgfqpoint{1.814237in}{2.627545in}}%
\pgfpathlineto{\pgfqpoint{1.881676in}{2.639079in}}%
\pgfpathlineto{\pgfqpoint{1.949453in}{2.648430in}}%
\pgfpathlineto{\pgfqpoint{2.017483in}{2.655735in}}%
\pgfpathlineto{\pgfqpoint{2.085676in}{2.661369in}}%
\pgfpathlineto{\pgfqpoint{2.153949in}{2.665970in}}%
\pgfpathlineto{\pgfqpoint{2.222229in}{2.670462in}}%
\pgfpathlineto{\pgfqpoint{2.290419in}{2.676086in}}%
\pgfpathlineto{\pgfqpoint{2.358321in}{2.684372in}}%
\pgfusepath{stroke}%
\end{pgfscope}%
\begin{pgfscope}%
\pgfpathrectangle{\pgfqpoint{0.647939in}{0.492442in}}{\pgfqpoint{3.079299in}{3.079299in}}%
\pgfusepath{clip}%
\pgfsetbuttcap%
\pgfsetroundjoin%
\pgfsetlinewidth{0.301125pt}%
\definecolor{currentstroke}{rgb}{0.500000,0.500000,0.500000}%
\pgfsetstrokecolor{currentstroke}%
\pgfsetstrokeopacity{0.300000}%
\pgfsetdash{}{0pt}%
\pgfpathmoveto{\pgfqpoint{1.363694in}{1.360316in}}%
\pgfpathlineto{\pgfqpoint{1.417764in}{1.402235in}}%
\pgfpathlineto{\pgfqpoint{1.470657in}{1.445621in}}%
\pgfpathlineto{\pgfqpoint{1.522568in}{1.490185in}}%
\pgfpathlineto{\pgfqpoint{1.573723in}{1.535614in}}%
\pgfpathlineto{\pgfqpoint{1.624391in}{1.581585in}}%
\pgfusepath{stroke}%
\end{pgfscope}%
\begin{pgfscope}%
\pgfpathrectangle{\pgfqpoint{0.647939in}{0.492442in}}{\pgfqpoint{3.079299in}{3.079299in}}%
\pgfusepath{clip}%
\pgfsetbuttcap%
\pgfsetroundjoin%
\pgfsetlinewidth{0.301125pt}%
\definecolor{currentstroke}{rgb}{0.500000,0.500000,0.500000}%
\pgfsetstrokecolor{currentstroke}%
\pgfsetstrokeopacity{0.300000}%
\pgfsetdash{}{0pt}%
\pgfpathmoveto{\pgfqpoint{1.391950in}{1.234596in}}%
\pgfpathlineto{\pgfqpoint{1.440586in}{1.282696in}}%
\pgfpathlineto{\pgfqpoint{1.487748in}{1.332251in}}%
\pgfpathlineto{\pgfqpoint{1.533699in}{1.382926in}}%
\pgfpathlineto{\pgfqpoint{1.578758in}{1.434389in}}%
\pgfpathlineto{\pgfqpoint{1.623266in}{1.486332in}}%
\pgfusepath{stroke}%
\end{pgfscope}%
\begin{pgfscope}%
\pgfpathrectangle{\pgfqpoint{0.647939in}{0.492442in}}{\pgfqpoint{3.079299in}{3.079299in}}%
\pgfusepath{clip}%
\pgfsetbuttcap%
\pgfsetroundjoin%
\pgfsetlinewidth{0.301125pt}%
\definecolor{currentstroke}{rgb}{0.500000,0.500000,0.500000}%
\pgfsetstrokecolor{currentstroke}%
\pgfsetstrokeopacity{0.300000}%
\pgfsetdash{}{0pt}%
\pgfpathmoveto{\pgfqpoint{2.817445in}{1.752155in}}%
\pgfpathlineto{\pgfqpoint{2.760550in}{1.790160in}}%
\pgfpathlineto{\pgfqpoint{2.703029in}{1.827213in}}%
\pgfpathlineto{\pgfqpoint{2.644813in}{1.863160in}}%
\pgfpathlineto{\pgfqpoint{2.585874in}{1.897901in}}%
\pgfpathlineto{\pgfqpoint{2.526265in}{1.931468in}}%
\pgfpathlineto{\pgfqpoint{2.466202in}{1.964206in}}%
\pgfpathlineto{\pgfqpoint{2.406336in}{1.997272in}}%
\pgfpathlineto{\pgfqpoint{2.349184in}{2.034447in}}%
\pgfpathlineto{\pgfqpoint{2.349184in}{2.034447in}}%
\pgfpathlineto{\pgfqpoint{2.322899in}{2.059691in}}%
\pgfpathlineto{\pgfqpoint{2.322899in}{2.059691in}}%
\pgfpathlineto{\pgfqpoint{2.310333in}{2.084263in}}%
\pgfpathlineto{\pgfqpoint{2.310325in}{2.113018in}}%
\pgfusepath{stroke}%
\end{pgfscope}%
\begin{pgfscope}%
\pgfpathrectangle{\pgfqpoint{0.647939in}{0.492442in}}{\pgfqpoint{3.079299in}{3.079299in}}%
\pgfusepath{clip}%
\pgfsetbuttcap%
\pgfsetroundjoin%
\pgfsetlinewidth{0.301125pt}%
\definecolor{currentstroke}{rgb}{0.500000,0.500000,0.500000}%
\pgfsetstrokecolor{currentstroke}%
\pgfsetstrokeopacity{0.300000}%
\pgfsetdash{}{0pt}%
\pgfpathmoveto{\pgfqpoint{1.804785in}{2.447175in}}%
\pgfpathlineto{\pgfqpoint{1.871741in}{2.461238in}}%
\pgfpathlineto{\pgfqpoint{1.939152in}{2.472918in}}%
\pgfpathlineto{\pgfqpoint{2.006927in}{2.482280in}}%
\pgfpathlineto{\pgfqpoint{2.074951in}{2.489648in}}%
\pgfpathlineto{\pgfqpoint{2.143110in}{2.495697in}}%
\pgfpathlineto{\pgfqpoint{2.211287in}{2.501551in}}%
\pgfpathlineto{\pgfqpoint{2.279314in}{2.508836in}}%
\pgfpathlineto{\pgfqpoint{2.346831in}{2.519654in}}%
\pgfpathlineto{\pgfqpoint{2.413099in}{2.536277in}}%
\pgfpathlineto{\pgfqpoint{2.476999in}{2.560334in}}%
\pgfpathlineto{\pgfqpoint{2.537509in}{2.591964in}}%
\pgfpathlineto{\pgfqpoint{2.594239in}{2.629941in}}%
\pgfpathlineto{\pgfqpoint{2.647674in}{2.672509in}}%
\pgfusepath{stroke}%
\end{pgfscope}%
\begin{pgfscope}%
\pgfpathrectangle{\pgfqpoint{0.647939in}{0.492442in}}{\pgfqpoint{3.079299in}{3.079299in}}%
\pgfusepath{clip}%
\pgfsetbuttcap%
\pgfsetroundjoin%
\pgfsetlinewidth{0.301125pt}%
\definecolor{currentstroke}{rgb}{0.500000,0.500000,0.500000}%
\pgfsetstrokecolor{currentstroke}%
\pgfsetstrokeopacity{0.300000}%
\pgfsetdash{}{0pt}%
\pgfpathmoveto{\pgfqpoint{1.263775in}{1.702200in}}%
\pgfpathlineto{\pgfqpoint{1.325284in}{1.732174in}}%
\pgfpathlineto{\pgfqpoint{1.386239in}{1.763263in}}%
\pgfpathlineto{\pgfqpoint{1.446780in}{1.795153in}}%
\pgfpathlineto{\pgfqpoint{1.507075in}{1.827508in}}%
\pgfpathlineto{\pgfqpoint{1.567317in}{1.859962in}}%
\pgfpathlineto{\pgfqpoint{1.627716in}{1.892124in}}%
\pgfusepath{stroke}%
\end{pgfscope}%
\begin{pgfscope}%
\pgfpathrectangle{\pgfqpoint{0.647939in}{0.492442in}}{\pgfqpoint{3.079299in}{3.079299in}}%
\pgfusepath{clip}%
\pgfsetbuttcap%
\pgfsetroundjoin%
\pgfsetlinewidth{0.301125pt}%
\definecolor{currentstroke}{rgb}{0.500000,0.500000,0.500000}%
\pgfsetstrokecolor{currentstroke}%
\pgfsetstrokeopacity{0.300000}%
\pgfsetdash{}{0pt}%
\pgfpathmoveto{\pgfqpoint{2.464268in}{1.667112in}}%
\pgfpathlineto{\pgfqpoint{2.397541in}{1.682171in}}%
\pgfpathlineto{\pgfqpoint{2.330233in}{1.694426in}}%
\pgfpathlineto{\pgfqpoint{2.262583in}{1.704689in}}%
\pgfpathlineto{\pgfqpoint{2.194883in}{1.714625in}}%
\pgfpathlineto{\pgfqpoint{2.127794in}{1.727689in}}%
\pgfpathlineto{\pgfqpoint{2.127794in}{1.727689in}}%
\pgfpathlineto{\pgfqpoint{2.075604in}{1.746089in}}%
\pgfpathlineto{\pgfqpoint{2.075604in}{1.746089in}}%
\pgfpathlineto{\pgfqpoint{2.048391in}{1.764864in}}%
\pgfpathlineto{\pgfqpoint{2.048391in}{1.764864in}}%
\pgfusepath{stroke}%
\end{pgfscope}%
\begin{pgfscope}%
\pgfpathrectangle{\pgfqpoint{0.647939in}{0.492442in}}{\pgfqpoint{3.079299in}{3.079299in}}%
\pgfusepath{clip}%
\pgfsetbuttcap%
\pgfsetroundjoin%
\pgfsetlinewidth{0.301125pt}%
\definecolor{currentstroke}{rgb}{0.500000,0.500000,0.500000}%
\pgfsetstrokecolor{currentstroke}%
\pgfsetstrokeopacity{0.300000}%
\pgfsetdash{}{0pt}%
\pgfpathmoveto{\pgfqpoint{2.397423in}{1.796540in}}%
\pgfpathlineto{\pgfqpoint{2.330966in}{1.812781in}}%
\pgfpathlineto{\pgfqpoint{2.264143in}{1.827499in}}%
\pgfpathlineto{\pgfqpoint{2.200727in}{1.842815in}}%
\pgfpathlineto{\pgfqpoint{2.161191in}{1.856222in}}%
\pgfpathlineto{\pgfqpoint{2.135897in}{1.870117in}}%
\pgfpathlineto{\pgfqpoint{2.117605in}{1.892124in}}%
\pgfpathlineto{\pgfqpoint{2.117605in}{1.892124in}}%
\pgfpathlineto{\pgfqpoint{2.117605in}{1.892124in}}%
\pgfpathlineto{\pgfqpoint{2.117605in}{1.892124in}}%
\pgfpathlineto{\pgfqpoint{2.114999in}{1.916823in}}%
\pgfusepath{stroke}%
\end{pgfscope}%
\begin{pgfscope}%
\pgfpathrectangle{\pgfqpoint{0.647939in}{0.492442in}}{\pgfqpoint{3.079299in}{3.079299in}}%
\pgfusepath{clip}%
\pgfsetbuttcap%
\pgfsetroundjoin%
\pgfsetlinewidth{0.301125pt}%
\definecolor{currentstroke}{rgb}{0.500000,0.500000,0.500000}%
\pgfsetstrokecolor{currentstroke}%
\pgfsetstrokeopacity{0.300000}%
\pgfsetdash{}{0pt}%
\pgfpathmoveto{\pgfqpoint{2.544084in}{1.877867in}}%
\pgfpathlineto{\pgfqpoint{2.482364in}{1.907374in}}%
\pgfpathlineto{\pgfqpoint{2.419935in}{1.935350in}}%
\pgfpathlineto{\pgfqpoint{2.357328in}{1.962910in}}%
\pgfpathlineto{\pgfqpoint{2.308801in}{1.986666in}}%
\pgfpathlineto{\pgfqpoint{2.280956in}{2.004322in}}%
\pgfpathlineto{\pgfqpoint{2.257573in}{2.032092in}}%
\pgfpathlineto{\pgfqpoint{2.257573in}{2.032092in}}%
\pgfpathlineto{\pgfqpoint{2.257573in}{2.032092in}}%
\pgfpathlineto{\pgfqpoint{2.254432in}{2.056105in}}%
\pgfusepath{stroke}%
\end{pgfscope}%
\begin{pgfscope}%
\pgfpathrectangle{\pgfqpoint{0.647939in}{0.492442in}}{\pgfqpoint{3.079299in}{3.079299in}}%
\pgfusepath{clip}%
\pgfsetroundcap%
\pgfsetroundjoin%
\pgfsetlinewidth{0.301125pt}%
\definecolor{currentstroke}{rgb}{0.500000,0.500000,0.500000}%
\pgfsetstrokecolor{currentstroke}%
\pgfsetstrokeopacity{0.300000}%
\pgfsetdash{}{0pt}%
\pgfpathmoveto{\pgfqpoint{2.113513in}{1.970680in}}%
\pgfusepath{stroke}%
\end{pgfscope}%
\begin{pgfscope}%
\pgfpathrectangle{\pgfqpoint{0.647939in}{0.492442in}}{\pgfqpoint{3.079299in}{3.079299in}}%
\pgfusepath{clip}%
\pgfsetroundcap%
\pgfsetroundjoin%
\definecolor{currentfill}{rgb}{0.500000,0.500000,0.500000}%
\pgfsetfillcolor{currentfill}%
\pgfsetfillopacity{0.300000}%
\pgfsetlinewidth{0.301125pt}%
\definecolor{currentstroke}{rgb}{0.500000,0.500000,0.500000}%
\pgfsetstrokecolor{currentstroke}%
\pgfsetstrokeopacity{0.300000}%
\pgfsetdash{}{0pt}%
\pgfpathmoveto{\pgfqpoint{0.000000in}{0.000000in}}%
\pgfpathlineto{\pgfqpoint{0.000000in}{0.000000in}}%
\pgfpathclose%
\pgfusepath{stroke,fill}%
\end{pgfscope}%
\begin{pgfscope}%
\pgfpathrectangle{\pgfqpoint{0.647939in}{0.492442in}}{\pgfqpoint{3.079299in}{3.079299in}}%
\pgfusepath{clip}%
\pgfsetroundcap%
\pgfsetroundjoin%
\pgfsetlinewidth{0.301125pt}%
\definecolor{currentstroke}{rgb}{0.500000,0.500000,0.500000}%
\pgfsetstrokecolor{currentstroke}%
\pgfsetstrokeopacity{0.300000}%
\pgfsetdash{}{0pt}%
\pgfpathmoveto{\pgfqpoint{1.122459in}{0.619212in}}%
\pgfusepath{stroke}%
\end{pgfscope}%
\begin{pgfscope}%
\pgfpathrectangle{\pgfqpoint{0.647939in}{0.492442in}}{\pgfqpoint{3.079299in}{3.079299in}}%
\pgfusepath{clip}%
\pgfsetroundcap%
\pgfsetroundjoin%
\definecolor{currentfill}{rgb}{0.500000,0.500000,0.500000}%
\pgfsetfillcolor{currentfill}%
\pgfsetfillopacity{0.300000}%
\pgfsetlinewidth{0.301125pt}%
\definecolor{currentstroke}{rgb}{0.500000,0.500000,0.500000}%
\pgfsetstrokecolor{currentstroke}%
\pgfsetstrokeopacity{0.300000}%
\pgfsetdash{}{0pt}%
\pgfpathmoveto{\pgfqpoint{0.000000in}{0.000000in}}%
\pgfpathlineto{\pgfqpoint{0.000000in}{0.000000in}}%
\pgfpathclose%
\pgfusepath{stroke,fill}%
\end{pgfscope}%
\begin{pgfscope}%
\pgfpathrectangle{\pgfqpoint{0.647939in}{0.492442in}}{\pgfqpoint{3.079299in}{3.079299in}}%
\pgfusepath{clip}%
\pgfsetroundcap%
\pgfsetroundjoin%
\pgfsetlinewidth{0.301125pt}%
\definecolor{currentstroke}{rgb}{0.500000,0.500000,0.500000}%
\pgfsetstrokecolor{currentstroke}%
\pgfsetstrokeopacity{0.300000}%
\pgfsetdash{}{0pt}%
\pgfpathmoveto{\pgfqpoint{1.239977in}{0.621098in}}%
\pgfusepath{stroke}%
\end{pgfscope}%
\begin{pgfscope}%
\pgfpathrectangle{\pgfqpoint{0.647939in}{0.492442in}}{\pgfqpoint{3.079299in}{3.079299in}}%
\pgfusepath{clip}%
\pgfsetroundcap%
\pgfsetroundjoin%
\definecolor{currentfill}{rgb}{0.500000,0.500000,0.500000}%
\pgfsetfillcolor{currentfill}%
\pgfsetfillopacity{0.300000}%
\pgfsetlinewidth{0.301125pt}%
\definecolor{currentstroke}{rgb}{0.500000,0.500000,0.500000}%
\pgfsetstrokecolor{currentstroke}%
\pgfsetstrokeopacity{0.300000}%
\pgfsetdash{}{0pt}%
\pgfpathmoveto{\pgfqpoint{0.000000in}{0.000000in}}%
\pgfpathlineto{\pgfqpoint{0.000000in}{0.000000in}}%
\pgfpathclose%
\pgfusepath{stroke,fill}%
\end{pgfscope}%
\begin{pgfscope}%
\pgfpathrectangle{\pgfqpoint{0.647939in}{0.492442in}}{\pgfqpoint{3.079299in}{3.079299in}}%
\pgfusepath{clip}%
\pgfsetroundcap%
\pgfsetroundjoin%
\pgfsetlinewidth{0.301125pt}%
\definecolor{currentstroke}{rgb}{0.500000,0.500000,0.500000}%
\pgfsetstrokecolor{currentstroke}%
\pgfsetstrokeopacity{0.300000}%
\pgfsetdash{}{0pt}%
\pgfpathmoveto{\pgfqpoint{1.420860in}{0.853437in}}%
\pgfusepath{stroke}%
\end{pgfscope}%
\begin{pgfscope}%
\pgfpathrectangle{\pgfqpoint{0.647939in}{0.492442in}}{\pgfqpoint{3.079299in}{3.079299in}}%
\pgfusepath{clip}%
\pgfsetroundcap%
\pgfsetroundjoin%
\definecolor{currentfill}{rgb}{0.500000,0.500000,0.500000}%
\pgfsetfillcolor{currentfill}%
\pgfsetfillopacity{0.300000}%
\pgfsetlinewidth{0.301125pt}%
\definecolor{currentstroke}{rgb}{0.500000,0.500000,0.500000}%
\pgfsetstrokecolor{currentstroke}%
\pgfsetstrokeopacity{0.300000}%
\pgfsetdash{}{0pt}%
\pgfpathmoveto{\pgfqpoint{0.000000in}{0.000000in}}%
\pgfpathlineto{\pgfqpoint{0.000000in}{0.000000in}}%
\pgfpathclose%
\pgfusepath{stroke,fill}%
\end{pgfscope}%
\begin{pgfscope}%
\pgfpathrectangle{\pgfqpoint{0.647939in}{0.492442in}}{\pgfqpoint{3.079299in}{3.079299in}}%
\pgfusepath{clip}%
\pgfsetroundcap%
\pgfsetroundjoin%
\pgfsetlinewidth{0.301125pt}%
\definecolor{currentstroke}{rgb}{0.500000,0.500000,0.500000}%
\pgfsetstrokecolor{currentstroke}%
\pgfsetstrokeopacity{0.300000}%
\pgfsetdash{}{0pt}%
\pgfpathmoveto{\pgfqpoint{1.459125in}{0.679598in}}%
\pgfusepath{stroke}%
\end{pgfscope}%
\begin{pgfscope}%
\pgfpathrectangle{\pgfqpoint{0.647939in}{0.492442in}}{\pgfqpoint{3.079299in}{3.079299in}}%
\pgfusepath{clip}%
\pgfsetroundcap%
\pgfsetroundjoin%
\definecolor{currentfill}{rgb}{0.500000,0.500000,0.500000}%
\pgfsetfillcolor{currentfill}%
\pgfsetfillopacity{0.300000}%
\pgfsetlinewidth{0.301125pt}%
\definecolor{currentstroke}{rgb}{0.500000,0.500000,0.500000}%
\pgfsetstrokecolor{currentstroke}%
\pgfsetstrokeopacity{0.300000}%
\pgfsetdash{}{0pt}%
\pgfpathmoveto{\pgfqpoint{0.000000in}{0.000000in}}%
\pgfpathlineto{\pgfqpoint{0.000000in}{0.000000in}}%
\pgfpathclose%
\pgfusepath{stroke,fill}%
\end{pgfscope}%
\begin{pgfscope}%
\pgfpathrectangle{\pgfqpoint{0.647939in}{0.492442in}}{\pgfqpoint{3.079299in}{3.079299in}}%
\pgfusepath{clip}%
\pgfsetroundcap%
\pgfsetroundjoin%
\pgfsetlinewidth{0.301125pt}%
\definecolor{currentstroke}{rgb}{0.500000,0.500000,0.500000}%
\pgfsetstrokecolor{currentstroke}%
\pgfsetstrokeopacity{0.300000}%
\pgfsetdash{}{0pt}%
\pgfpathmoveto{\pgfqpoint{1.680947in}{0.540484in}}%
\pgfusepath{stroke}%
\end{pgfscope}%
\begin{pgfscope}%
\pgfpathrectangle{\pgfqpoint{0.647939in}{0.492442in}}{\pgfqpoint{3.079299in}{3.079299in}}%
\pgfusepath{clip}%
\pgfsetroundcap%
\pgfsetroundjoin%
\definecolor{currentfill}{rgb}{0.500000,0.500000,0.500000}%
\pgfsetfillcolor{currentfill}%
\pgfsetfillopacity{0.300000}%
\pgfsetlinewidth{0.301125pt}%
\definecolor{currentstroke}{rgb}{0.500000,0.500000,0.500000}%
\pgfsetstrokecolor{currentstroke}%
\pgfsetstrokeopacity{0.300000}%
\pgfsetdash{}{0pt}%
\pgfpathmoveto{\pgfqpoint{0.000000in}{0.000000in}}%
\pgfpathlineto{\pgfqpoint{0.000000in}{0.000000in}}%
\pgfpathclose%
\pgfusepath{stroke,fill}%
\end{pgfscope}%
\begin{pgfscope}%
\pgfpathrectangle{\pgfqpoint{0.647939in}{0.492442in}}{\pgfqpoint{3.079299in}{3.079299in}}%
\pgfusepath{clip}%
\pgfsetroundcap%
\pgfsetroundjoin%
\pgfsetlinewidth{0.301125pt}%
\definecolor{currentstroke}{rgb}{0.500000,0.500000,0.500000}%
\pgfsetstrokecolor{currentstroke}%
\pgfsetstrokeopacity{0.300000}%
\pgfsetdash{}{0pt}%
\pgfpathmoveto{\pgfqpoint{2.093040in}{0.497827in}}%
\pgfusepath{stroke}%
\end{pgfscope}%
\begin{pgfscope}%
\pgfpathrectangle{\pgfqpoint{0.647939in}{0.492442in}}{\pgfqpoint{3.079299in}{3.079299in}}%
\pgfusepath{clip}%
\pgfsetroundcap%
\pgfsetroundjoin%
\definecolor{currentfill}{rgb}{0.500000,0.500000,0.500000}%
\pgfsetfillcolor{currentfill}%
\pgfsetfillopacity{0.300000}%
\pgfsetlinewidth{0.301125pt}%
\definecolor{currentstroke}{rgb}{0.500000,0.500000,0.500000}%
\pgfsetstrokecolor{currentstroke}%
\pgfsetstrokeopacity{0.300000}%
\pgfsetdash{}{0pt}%
\pgfpathmoveto{\pgfqpoint{0.000000in}{0.000000in}}%
\pgfpathlineto{\pgfqpoint{0.000000in}{0.000000in}}%
\pgfpathclose%
\pgfusepath{stroke,fill}%
\end{pgfscope}%
\begin{pgfscope}%
\pgfpathrectangle{\pgfqpoint{0.647939in}{0.492442in}}{\pgfqpoint{3.079299in}{3.079299in}}%
\pgfusepath{clip}%
\pgfsetroundcap%
\pgfsetroundjoin%
\pgfsetlinewidth{0.301125pt}%
\definecolor{currentstroke}{rgb}{0.500000,0.500000,0.500000}%
\pgfsetstrokecolor{currentstroke}%
\pgfsetstrokeopacity{0.300000}%
\pgfsetdash{}{0pt}%
\pgfpathmoveto{\pgfqpoint{2.241168in}{0.530220in}}%
\pgfusepath{stroke}%
\end{pgfscope}%
\begin{pgfscope}%
\pgfpathrectangle{\pgfqpoint{0.647939in}{0.492442in}}{\pgfqpoint{3.079299in}{3.079299in}}%
\pgfusepath{clip}%
\pgfsetroundcap%
\pgfsetroundjoin%
\definecolor{currentfill}{rgb}{0.500000,0.500000,0.500000}%
\pgfsetfillcolor{currentfill}%
\pgfsetfillopacity{0.300000}%
\pgfsetlinewidth{0.301125pt}%
\definecolor{currentstroke}{rgb}{0.500000,0.500000,0.500000}%
\pgfsetstrokecolor{currentstroke}%
\pgfsetstrokeopacity{0.300000}%
\pgfsetdash{}{0pt}%
\pgfpathmoveto{\pgfqpoint{0.000000in}{0.000000in}}%
\pgfpathlineto{\pgfqpoint{0.000000in}{0.000000in}}%
\pgfpathclose%
\pgfusepath{stroke,fill}%
\end{pgfscope}%
\begin{pgfscope}%
\pgfpathrectangle{\pgfqpoint{0.647939in}{0.492442in}}{\pgfqpoint{3.079299in}{3.079299in}}%
\pgfusepath{clip}%
\pgfsetroundcap%
\pgfsetroundjoin%
\pgfsetlinewidth{0.301125pt}%
\definecolor{currentstroke}{rgb}{0.500000,0.500000,0.500000}%
\pgfsetstrokecolor{currentstroke}%
\pgfsetstrokeopacity{0.300000}%
\pgfsetdash{}{0pt}%
\pgfpathmoveto{\pgfqpoint{2.592100in}{0.551801in}}%
\pgfusepath{stroke}%
\end{pgfscope}%
\begin{pgfscope}%
\pgfpathrectangle{\pgfqpoint{0.647939in}{0.492442in}}{\pgfqpoint{3.079299in}{3.079299in}}%
\pgfusepath{clip}%
\pgfsetroundcap%
\pgfsetroundjoin%
\definecolor{currentfill}{rgb}{0.500000,0.500000,0.500000}%
\pgfsetfillcolor{currentfill}%
\pgfsetfillopacity{0.300000}%
\pgfsetlinewidth{0.301125pt}%
\definecolor{currentstroke}{rgb}{0.500000,0.500000,0.500000}%
\pgfsetstrokecolor{currentstroke}%
\pgfsetstrokeopacity{0.300000}%
\pgfsetdash{}{0pt}%
\pgfpathmoveto{\pgfqpoint{0.000000in}{0.000000in}}%
\pgfpathlineto{\pgfqpoint{0.000000in}{0.000000in}}%
\pgfpathclose%
\pgfusepath{stroke,fill}%
\end{pgfscope}%
\begin{pgfscope}%
\pgfpathrectangle{\pgfqpoint{0.647939in}{0.492442in}}{\pgfqpoint{3.079299in}{3.079299in}}%
\pgfusepath{clip}%
\pgfsetroundcap%
\pgfsetroundjoin%
\pgfsetlinewidth{0.301125pt}%
\definecolor{currentstroke}{rgb}{0.500000,0.500000,0.500000}%
\pgfsetstrokecolor{currentstroke}%
\pgfsetstrokeopacity{0.300000}%
\pgfsetdash{}{0pt}%
\pgfpathmoveto{\pgfqpoint{2.198059in}{0.648361in}}%
\pgfusepath{stroke}%
\end{pgfscope}%
\begin{pgfscope}%
\pgfpathrectangle{\pgfqpoint{0.647939in}{0.492442in}}{\pgfqpoint{3.079299in}{3.079299in}}%
\pgfusepath{clip}%
\pgfsetroundcap%
\pgfsetroundjoin%
\definecolor{currentfill}{rgb}{0.500000,0.500000,0.500000}%
\pgfsetfillcolor{currentfill}%
\pgfsetfillopacity{0.300000}%
\pgfsetlinewidth{0.301125pt}%
\definecolor{currentstroke}{rgb}{0.500000,0.500000,0.500000}%
\pgfsetstrokecolor{currentstroke}%
\pgfsetstrokeopacity{0.300000}%
\pgfsetdash{}{0pt}%
\pgfpathmoveto{\pgfqpoint{0.000000in}{0.000000in}}%
\pgfpathlineto{\pgfqpoint{0.000000in}{0.000000in}}%
\pgfpathclose%
\pgfusepath{stroke,fill}%
\end{pgfscope}%
\begin{pgfscope}%
\pgfpathrectangle{\pgfqpoint{0.647939in}{0.492442in}}{\pgfqpoint{3.079299in}{3.079299in}}%
\pgfusepath{clip}%
\pgfsetroundcap%
\pgfsetroundjoin%
\pgfsetlinewidth{0.301125pt}%
\definecolor{currentstroke}{rgb}{0.500000,0.500000,0.500000}%
\pgfsetstrokecolor{currentstroke}%
\pgfsetstrokeopacity{0.300000}%
\pgfsetdash{}{0pt}%
\pgfpathmoveto{\pgfqpoint{2.559761in}{0.709119in}}%
\pgfusepath{stroke}%
\end{pgfscope}%
\begin{pgfscope}%
\pgfpathrectangle{\pgfqpoint{0.647939in}{0.492442in}}{\pgfqpoint{3.079299in}{3.079299in}}%
\pgfusepath{clip}%
\pgfsetroundcap%
\pgfsetroundjoin%
\definecolor{currentfill}{rgb}{0.500000,0.500000,0.500000}%
\pgfsetfillcolor{currentfill}%
\pgfsetfillopacity{0.300000}%
\pgfsetlinewidth{0.301125pt}%
\definecolor{currentstroke}{rgb}{0.500000,0.500000,0.500000}%
\pgfsetstrokecolor{currentstroke}%
\pgfsetstrokeopacity{0.300000}%
\pgfsetdash{}{0pt}%
\pgfpathmoveto{\pgfqpoint{0.000000in}{0.000000in}}%
\pgfpathlineto{\pgfqpoint{0.000000in}{0.000000in}}%
\pgfpathclose%
\pgfusepath{stroke,fill}%
\end{pgfscope}%
\begin{pgfscope}%
\pgfpathrectangle{\pgfqpoint{0.647939in}{0.492442in}}{\pgfqpoint{3.079299in}{3.079299in}}%
\pgfusepath{clip}%
\pgfsetroundcap%
\pgfsetroundjoin%
\pgfsetlinewidth{0.301125pt}%
\definecolor{currentstroke}{rgb}{0.500000,0.500000,0.500000}%
\pgfsetstrokecolor{currentstroke}%
\pgfsetstrokeopacity{0.300000}%
\pgfsetdash{}{0pt}%
\pgfpathmoveto{\pgfqpoint{2.378702in}{0.820155in}}%
\pgfusepath{stroke}%
\end{pgfscope}%
\begin{pgfscope}%
\pgfpathrectangle{\pgfqpoint{0.647939in}{0.492442in}}{\pgfqpoint{3.079299in}{3.079299in}}%
\pgfusepath{clip}%
\pgfsetroundcap%
\pgfsetroundjoin%
\definecolor{currentfill}{rgb}{0.500000,0.500000,0.500000}%
\pgfsetfillcolor{currentfill}%
\pgfsetfillopacity{0.300000}%
\pgfsetlinewidth{0.301125pt}%
\definecolor{currentstroke}{rgb}{0.500000,0.500000,0.500000}%
\pgfsetstrokecolor{currentstroke}%
\pgfsetstrokeopacity{0.300000}%
\pgfsetdash{}{0pt}%
\pgfpathmoveto{\pgfqpoint{0.000000in}{0.000000in}}%
\pgfpathlineto{\pgfqpoint{0.000000in}{0.000000in}}%
\pgfpathclose%
\pgfusepath{stroke,fill}%
\end{pgfscope}%
\begin{pgfscope}%
\pgfpathrectangle{\pgfqpoint{0.647939in}{0.492442in}}{\pgfqpoint{3.079299in}{3.079299in}}%
\pgfusepath{clip}%
\pgfsetroundcap%
\pgfsetroundjoin%
\pgfsetlinewidth{0.301125pt}%
\definecolor{currentstroke}{rgb}{0.500000,0.500000,0.500000}%
\pgfsetstrokecolor{currentstroke}%
\pgfsetstrokeopacity{0.300000}%
\pgfsetdash{}{0pt}%
\pgfpathmoveto{\pgfqpoint{2.738190in}{0.854592in}}%
\pgfusepath{stroke}%
\end{pgfscope}%
\begin{pgfscope}%
\pgfpathrectangle{\pgfqpoint{0.647939in}{0.492442in}}{\pgfqpoint{3.079299in}{3.079299in}}%
\pgfusepath{clip}%
\pgfsetroundcap%
\pgfsetroundjoin%
\definecolor{currentfill}{rgb}{0.500000,0.500000,0.500000}%
\pgfsetfillcolor{currentfill}%
\pgfsetfillopacity{0.300000}%
\pgfsetlinewidth{0.301125pt}%
\definecolor{currentstroke}{rgb}{0.500000,0.500000,0.500000}%
\pgfsetstrokecolor{currentstroke}%
\pgfsetstrokeopacity{0.300000}%
\pgfsetdash{}{0pt}%
\pgfpathmoveto{\pgfqpoint{0.000000in}{0.000000in}}%
\pgfpathlineto{\pgfqpoint{0.000000in}{0.000000in}}%
\pgfpathclose%
\pgfusepath{stroke,fill}%
\end{pgfscope}%
\begin{pgfscope}%
\pgfpathrectangle{\pgfqpoint{0.647939in}{0.492442in}}{\pgfqpoint{3.079299in}{3.079299in}}%
\pgfusepath{clip}%
\pgfsetroundcap%
\pgfsetroundjoin%
\pgfsetlinewidth{0.301125pt}%
\definecolor{currentstroke}{rgb}{0.500000,0.500000,0.500000}%
\pgfsetstrokecolor{currentstroke}%
\pgfsetstrokeopacity{0.300000}%
\pgfsetdash{}{0pt}%
\pgfpathmoveto{\pgfqpoint{2.677094in}{0.955143in}}%
\pgfusepath{stroke}%
\end{pgfscope}%
\begin{pgfscope}%
\pgfpathrectangle{\pgfqpoint{0.647939in}{0.492442in}}{\pgfqpoint{3.079299in}{3.079299in}}%
\pgfusepath{clip}%
\pgfsetroundcap%
\pgfsetroundjoin%
\definecolor{currentfill}{rgb}{0.500000,0.500000,0.500000}%
\pgfsetfillcolor{currentfill}%
\pgfsetfillopacity{0.300000}%
\pgfsetlinewidth{0.301125pt}%
\definecolor{currentstroke}{rgb}{0.500000,0.500000,0.500000}%
\pgfsetstrokecolor{currentstroke}%
\pgfsetstrokeopacity{0.300000}%
\pgfsetdash{}{0pt}%
\pgfpathmoveto{\pgfqpoint{0.000000in}{0.000000in}}%
\pgfpathlineto{\pgfqpoint{0.000000in}{0.000000in}}%
\pgfpathclose%
\pgfusepath{stroke,fill}%
\end{pgfscope}%
\begin{pgfscope}%
\pgfpathrectangle{\pgfqpoint{0.647939in}{0.492442in}}{\pgfqpoint{3.079299in}{3.079299in}}%
\pgfusepath{clip}%
\pgfsetroundcap%
\pgfsetroundjoin%
\pgfsetlinewidth{0.301125pt}%
\definecolor{currentstroke}{rgb}{0.500000,0.500000,0.500000}%
\pgfsetstrokecolor{currentstroke}%
\pgfsetstrokeopacity{0.300000}%
\pgfsetdash{}{0pt}%
\pgfpathmoveto{\pgfqpoint{2.749656in}{1.024558in}}%
\pgfusepath{stroke}%
\end{pgfscope}%
\begin{pgfscope}%
\pgfpathrectangle{\pgfqpoint{0.647939in}{0.492442in}}{\pgfqpoint{3.079299in}{3.079299in}}%
\pgfusepath{clip}%
\pgfsetroundcap%
\pgfsetroundjoin%
\definecolor{currentfill}{rgb}{0.500000,0.500000,0.500000}%
\pgfsetfillcolor{currentfill}%
\pgfsetfillopacity{0.300000}%
\pgfsetlinewidth{0.301125pt}%
\definecolor{currentstroke}{rgb}{0.500000,0.500000,0.500000}%
\pgfsetstrokecolor{currentstroke}%
\pgfsetstrokeopacity{0.300000}%
\pgfsetdash{}{0pt}%
\pgfpathmoveto{\pgfqpoint{0.000000in}{0.000000in}}%
\pgfpathlineto{\pgfqpoint{0.000000in}{0.000000in}}%
\pgfpathclose%
\pgfusepath{stroke,fill}%
\end{pgfscope}%
\begin{pgfscope}%
\pgfpathrectangle{\pgfqpoint{0.647939in}{0.492442in}}{\pgfqpoint{3.079299in}{3.079299in}}%
\pgfusepath{clip}%
\pgfsetroundcap%
\pgfsetroundjoin%
\pgfsetlinewidth{0.301125pt}%
\definecolor{currentstroke}{rgb}{0.500000,0.500000,0.500000}%
\pgfsetstrokecolor{currentstroke}%
\pgfsetstrokeopacity{0.300000}%
\pgfsetdash{}{0pt}%
\pgfpathmoveto{\pgfqpoint{2.624458in}{1.146776in}}%
\pgfusepath{stroke}%
\end{pgfscope}%
\begin{pgfscope}%
\pgfpathrectangle{\pgfqpoint{0.647939in}{0.492442in}}{\pgfqpoint{3.079299in}{3.079299in}}%
\pgfusepath{clip}%
\pgfsetroundcap%
\pgfsetroundjoin%
\definecolor{currentfill}{rgb}{0.500000,0.500000,0.500000}%
\pgfsetfillcolor{currentfill}%
\pgfsetfillopacity{0.300000}%
\pgfsetlinewidth{0.301125pt}%
\definecolor{currentstroke}{rgb}{0.500000,0.500000,0.500000}%
\pgfsetstrokecolor{currentstroke}%
\pgfsetstrokeopacity{0.300000}%
\pgfsetdash{}{0pt}%
\pgfpathmoveto{\pgfqpoint{0.000000in}{0.000000in}}%
\pgfpathlineto{\pgfqpoint{0.000000in}{0.000000in}}%
\pgfpathclose%
\pgfusepath{stroke,fill}%
\end{pgfscope}%
\begin{pgfscope}%
\pgfpathrectangle{\pgfqpoint{0.647939in}{0.492442in}}{\pgfqpoint{3.079299in}{3.079299in}}%
\pgfusepath{clip}%
\pgfsetroundcap%
\pgfsetroundjoin%
\pgfsetlinewidth{0.301125pt}%
\definecolor{currentstroke}{rgb}{0.500000,0.500000,0.500000}%
\pgfsetstrokecolor{currentstroke}%
\pgfsetstrokeopacity{0.300000}%
\pgfsetdash{}{0pt}%
\pgfpathmoveto{\pgfqpoint{2.892932in}{1.152625in}}%
\pgfusepath{stroke}%
\end{pgfscope}%
\begin{pgfscope}%
\pgfpathrectangle{\pgfqpoint{0.647939in}{0.492442in}}{\pgfqpoint{3.079299in}{3.079299in}}%
\pgfusepath{clip}%
\pgfsetroundcap%
\pgfsetroundjoin%
\definecolor{currentfill}{rgb}{0.500000,0.500000,0.500000}%
\pgfsetfillcolor{currentfill}%
\pgfsetfillopacity{0.300000}%
\pgfsetlinewidth{0.301125pt}%
\definecolor{currentstroke}{rgb}{0.500000,0.500000,0.500000}%
\pgfsetstrokecolor{currentstroke}%
\pgfsetstrokeopacity{0.300000}%
\pgfsetdash{}{0pt}%
\pgfpathmoveto{\pgfqpoint{0.000000in}{0.000000in}}%
\pgfpathlineto{\pgfqpoint{0.000000in}{0.000000in}}%
\pgfpathclose%
\pgfusepath{stroke,fill}%
\end{pgfscope}%
\begin{pgfscope}%
\pgfpathrectangle{\pgfqpoint{0.647939in}{0.492442in}}{\pgfqpoint{3.079299in}{3.079299in}}%
\pgfusepath{clip}%
\pgfsetroundcap%
\pgfsetroundjoin%
\pgfsetlinewidth{0.301125pt}%
\definecolor{currentstroke}{rgb}{0.500000,0.500000,0.500000}%
\pgfsetstrokecolor{currentstroke}%
\pgfsetstrokeopacity{0.300000}%
\pgfsetdash{}{0pt}%
\pgfpathmoveto{\pgfqpoint{2.837788in}{1.265916in}}%
\pgfusepath{stroke}%
\end{pgfscope}%
\begin{pgfscope}%
\pgfpathrectangle{\pgfqpoint{0.647939in}{0.492442in}}{\pgfqpoint{3.079299in}{3.079299in}}%
\pgfusepath{clip}%
\pgfsetroundcap%
\pgfsetroundjoin%
\definecolor{currentfill}{rgb}{0.500000,0.500000,0.500000}%
\pgfsetfillcolor{currentfill}%
\pgfsetfillopacity{0.300000}%
\pgfsetlinewidth{0.301125pt}%
\definecolor{currentstroke}{rgb}{0.500000,0.500000,0.500000}%
\pgfsetstrokecolor{currentstroke}%
\pgfsetstrokeopacity{0.300000}%
\pgfsetdash{}{0pt}%
\pgfpathmoveto{\pgfqpoint{0.000000in}{0.000000in}}%
\pgfpathlineto{\pgfqpoint{0.000000in}{0.000000in}}%
\pgfpathclose%
\pgfusepath{stroke,fill}%
\end{pgfscope}%
\begin{pgfscope}%
\pgfpathrectangle{\pgfqpoint{0.647939in}{0.492442in}}{\pgfqpoint{3.079299in}{3.079299in}}%
\pgfusepath{clip}%
\pgfsetroundcap%
\pgfsetroundjoin%
\pgfsetlinewidth{0.301125pt}%
\definecolor{currentstroke}{rgb}{0.500000,0.500000,0.500000}%
\pgfsetstrokecolor{currentstroke}%
\pgfsetstrokeopacity{0.300000}%
\pgfsetdash{}{0pt}%
\pgfpathmoveto{\pgfqpoint{2.847800in}{1.356173in}}%
\pgfusepath{stroke}%
\end{pgfscope}%
\begin{pgfscope}%
\pgfpathrectangle{\pgfqpoint{0.647939in}{0.492442in}}{\pgfqpoint{3.079299in}{3.079299in}}%
\pgfusepath{clip}%
\pgfsetroundcap%
\pgfsetroundjoin%
\definecolor{currentfill}{rgb}{0.500000,0.500000,0.500000}%
\pgfsetfillcolor{currentfill}%
\pgfsetfillopacity{0.300000}%
\pgfsetlinewidth{0.301125pt}%
\definecolor{currentstroke}{rgb}{0.500000,0.500000,0.500000}%
\pgfsetstrokecolor{currentstroke}%
\pgfsetstrokeopacity{0.300000}%
\pgfsetdash{}{0pt}%
\pgfpathmoveto{\pgfqpoint{0.000000in}{0.000000in}}%
\pgfpathlineto{\pgfqpoint{0.000000in}{0.000000in}}%
\pgfpathclose%
\pgfusepath{stroke,fill}%
\end{pgfscope}%
\begin{pgfscope}%
\pgfpathrectangle{\pgfqpoint{0.647939in}{0.492442in}}{\pgfqpoint{3.079299in}{3.079299in}}%
\pgfusepath{clip}%
\pgfsetroundcap%
\pgfsetroundjoin%
\pgfsetlinewidth{0.301125pt}%
\definecolor{currentstroke}{rgb}{0.500000,0.500000,0.500000}%
\pgfsetstrokecolor{currentstroke}%
\pgfsetstrokeopacity{0.300000}%
\pgfsetdash{}{0pt}%
\pgfpathmoveto{\pgfqpoint{2.859534in}{1.448440in}}%
\pgfusepath{stroke}%
\end{pgfscope}%
\begin{pgfscope}%
\pgfpathrectangle{\pgfqpoint{0.647939in}{0.492442in}}{\pgfqpoint{3.079299in}{3.079299in}}%
\pgfusepath{clip}%
\pgfsetroundcap%
\pgfsetroundjoin%
\definecolor{currentfill}{rgb}{0.500000,0.500000,0.500000}%
\pgfsetfillcolor{currentfill}%
\pgfsetfillopacity{0.300000}%
\pgfsetlinewidth{0.301125pt}%
\definecolor{currentstroke}{rgb}{0.500000,0.500000,0.500000}%
\pgfsetstrokecolor{currentstroke}%
\pgfsetstrokeopacity{0.300000}%
\pgfsetdash{}{0pt}%
\pgfpathmoveto{\pgfqpoint{0.000000in}{0.000000in}}%
\pgfpathlineto{\pgfqpoint{0.000000in}{0.000000in}}%
\pgfpathclose%
\pgfusepath{stroke,fill}%
\end{pgfscope}%
\begin{pgfscope}%
\pgfpathrectangle{\pgfqpoint{0.647939in}{0.492442in}}{\pgfqpoint{3.079299in}{3.079299in}}%
\pgfusepath{clip}%
\pgfsetroundcap%
\pgfsetroundjoin%
\pgfsetlinewidth{0.301125pt}%
\definecolor{currentstroke}{rgb}{0.500000,0.500000,0.500000}%
\pgfsetstrokecolor{currentstroke}%
\pgfsetstrokeopacity{0.300000}%
\pgfsetdash{}{0pt}%
\pgfpathmoveto{\pgfqpoint{2.932688in}{1.508774in}}%
\pgfusepath{stroke}%
\end{pgfscope}%
\begin{pgfscope}%
\pgfpathrectangle{\pgfqpoint{0.647939in}{0.492442in}}{\pgfqpoint{3.079299in}{3.079299in}}%
\pgfusepath{clip}%
\pgfsetroundcap%
\pgfsetroundjoin%
\definecolor{currentfill}{rgb}{0.500000,0.500000,0.500000}%
\pgfsetfillcolor{currentfill}%
\pgfsetfillopacity{0.300000}%
\pgfsetlinewidth{0.301125pt}%
\definecolor{currentstroke}{rgb}{0.500000,0.500000,0.500000}%
\pgfsetstrokecolor{currentstroke}%
\pgfsetstrokeopacity{0.300000}%
\pgfsetdash{}{0pt}%
\pgfpathmoveto{\pgfqpoint{0.000000in}{0.000000in}}%
\pgfpathlineto{\pgfqpoint{0.000000in}{0.000000in}}%
\pgfpathclose%
\pgfusepath{stroke,fill}%
\end{pgfscope}%
\begin{pgfscope}%
\pgfpathrectangle{\pgfqpoint{0.647939in}{0.492442in}}{\pgfqpoint{3.079299in}{3.079299in}}%
\pgfusepath{clip}%
\pgfsetroundcap%
\pgfsetroundjoin%
\pgfsetlinewidth{0.301125pt}%
\definecolor{currentstroke}{rgb}{0.500000,0.500000,0.500000}%
\pgfsetstrokecolor{currentstroke}%
\pgfsetstrokeopacity{0.300000}%
\pgfsetdash{}{0pt}%
\pgfpathmoveto{\pgfqpoint{3.004023in}{1.564311in}}%
\pgfusepath{stroke}%
\end{pgfscope}%
\begin{pgfscope}%
\pgfpathrectangle{\pgfqpoint{0.647939in}{0.492442in}}{\pgfqpoint{3.079299in}{3.079299in}}%
\pgfusepath{clip}%
\pgfsetroundcap%
\pgfsetroundjoin%
\definecolor{currentfill}{rgb}{0.500000,0.500000,0.500000}%
\pgfsetfillcolor{currentfill}%
\pgfsetfillopacity{0.300000}%
\pgfsetlinewidth{0.301125pt}%
\definecolor{currentstroke}{rgb}{0.500000,0.500000,0.500000}%
\pgfsetstrokecolor{currentstroke}%
\pgfsetstrokeopacity{0.300000}%
\pgfsetdash{}{0pt}%
\pgfpathmoveto{\pgfqpoint{0.000000in}{0.000000in}}%
\pgfpathlineto{\pgfqpoint{0.000000in}{0.000000in}}%
\pgfpathclose%
\pgfusepath{stroke,fill}%
\end{pgfscope}%
\begin{pgfscope}%
\pgfpathrectangle{\pgfqpoint{0.647939in}{0.492442in}}{\pgfqpoint{3.079299in}{3.079299in}}%
\pgfusepath{clip}%
\pgfsetroundcap%
\pgfsetroundjoin%
\pgfsetlinewidth{0.301125pt}%
\definecolor{currentstroke}{rgb}{0.500000,0.500000,0.500000}%
\pgfsetstrokecolor{currentstroke}%
\pgfsetstrokeopacity{0.300000}%
\pgfsetdash{}{0pt}%
\pgfpathmoveto{\pgfqpoint{2.984946in}{1.796423in}}%
\pgfusepath{stroke}%
\end{pgfscope}%
\begin{pgfscope}%
\pgfpathrectangle{\pgfqpoint{0.647939in}{0.492442in}}{\pgfqpoint{3.079299in}{3.079299in}}%
\pgfusepath{clip}%
\pgfsetroundcap%
\pgfsetroundjoin%
\definecolor{currentfill}{rgb}{0.500000,0.500000,0.500000}%
\pgfsetfillcolor{currentfill}%
\pgfsetfillopacity{0.300000}%
\pgfsetlinewidth{0.301125pt}%
\definecolor{currentstroke}{rgb}{0.500000,0.500000,0.500000}%
\pgfsetstrokecolor{currentstroke}%
\pgfsetstrokeopacity{0.300000}%
\pgfsetdash{}{0pt}%
\pgfpathmoveto{\pgfqpoint{0.000000in}{0.000000in}}%
\pgfpathlineto{\pgfqpoint{0.000000in}{0.000000in}}%
\pgfpathclose%
\pgfusepath{stroke,fill}%
\end{pgfscope}%
\begin{pgfscope}%
\pgfpathrectangle{\pgfqpoint{0.647939in}{0.492442in}}{\pgfqpoint{3.079299in}{3.079299in}}%
\pgfusepath{clip}%
\pgfsetroundcap%
\pgfsetroundjoin%
\pgfsetlinewidth{0.301125pt}%
\definecolor{currentstroke}{rgb}{0.500000,0.500000,0.500000}%
\pgfsetstrokecolor{currentstroke}%
\pgfsetstrokeopacity{0.300000}%
\pgfsetdash{}{0pt}%
\pgfpathmoveto{\pgfqpoint{2.898011in}{2.232746in}}%
\pgfusepath{stroke}%
\end{pgfscope}%
\begin{pgfscope}%
\pgfpathrectangle{\pgfqpoint{0.647939in}{0.492442in}}{\pgfqpoint{3.079299in}{3.079299in}}%
\pgfusepath{clip}%
\pgfsetroundcap%
\pgfsetroundjoin%
\definecolor{currentfill}{rgb}{0.500000,0.500000,0.500000}%
\pgfsetfillcolor{currentfill}%
\pgfsetfillopacity{0.300000}%
\pgfsetlinewidth{0.301125pt}%
\definecolor{currentstroke}{rgb}{0.500000,0.500000,0.500000}%
\pgfsetstrokecolor{currentstroke}%
\pgfsetstrokeopacity{0.300000}%
\pgfsetdash{}{0pt}%
\pgfpathmoveto{\pgfqpoint{0.000000in}{0.000000in}}%
\pgfpathlineto{\pgfqpoint{0.000000in}{0.000000in}}%
\pgfpathclose%
\pgfusepath{stroke,fill}%
\end{pgfscope}%
\begin{pgfscope}%
\pgfpathrectangle{\pgfqpoint{0.647939in}{0.492442in}}{\pgfqpoint{3.079299in}{3.079299in}}%
\pgfusepath{clip}%
\pgfsetroundcap%
\pgfsetroundjoin%
\pgfsetlinewidth{0.301125pt}%
\definecolor{currentstroke}{rgb}{0.500000,0.500000,0.500000}%
\pgfsetstrokecolor{currentstroke}%
\pgfsetstrokeopacity{0.300000}%
\pgfsetdash{}{0pt}%
\pgfpathmoveto{\pgfqpoint{3.113789in}{2.207517in}}%
\pgfusepath{stroke}%
\end{pgfscope}%
\begin{pgfscope}%
\pgfpathrectangle{\pgfqpoint{0.647939in}{0.492442in}}{\pgfqpoint{3.079299in}{3.079299in}}%
\pgfusepath{clip}%
\pgfsetroundcap%
\pgfsetroundjoin%
\definecolor{currentfill}{rgb}{0.500000,0.500000,0.500000}%
\pgfsetfillcolor{currentfill}%
\pgfsetfillopacity{0.300000}%
\pgfsetlinewidth{0.301125pt}%
\definecolor{currentstroke}{rgb}{0.500000,0.500000,0.500000}%
\pgfsetstrokecolor{currentstroke}%
\pgfsetstrokeopacity{0.300000}%
\pgfsetdash{}{0pt}%
\pgfpathmoveto{\pgfqpoint{0.000000in}{0.000000in}}%
\pgfpathlineto{\pgfqpoint{0.000000in}{0.000000in}}%
\pgfpathclose%
\pgfusepath{stroke,fill}%
\end{pgfscope}%
\begin{pgfscope}%
\pgfpathrectangle{\pgfqpoint{0.647939in}{0.492442in}}{\pgfqpoint{3.079299in}{3.079299in}}%
\pgfusepath{clip}%
\pgfsetroundcap%
\pgfsetroundjoin%
\pgfsetlinewidth{0.301125pt}%
\definecolor{currentstroke}{rgb}{0.500000,0.500000,0.500000}%
\pgfsetstrokecolor{currentstroke}%
\pgfsetstrokeopacity{0.300000}%
\pgfsetdash{}{0pt}%
\pgfpathmoveto{\pgfqpoint{3.199081in}{2.550260in}}%
\pgfusepath{stroke}%
\end{pgfscope}%
\begin{pgfscope}%
\pgfpathrectangle{\pgfqpoint{0.647939in}{0.492442in}}{\pgfqpoint{3.079299in}{3.079299in}}%
\pgfusepath{clip}%
\pgfsetroundcap%
\pgfsetroundjoin%
\definecolor{currentfill}{rgb}{0.500000,0.500000,0.500000}%
\pgfsetfillcolor{currentfill}%
\pgfsetfillopacity{0.300000}%
\pgfsetlinewidth{0.301125pt}%
\definecolor{currentstroke}{rgb}{0.500000,0.500000,0.500000}%
\pgfsetstrokecolor{currentstroke}%
\pgfsetstrokeopacity{0.300000}%
\pgfsetdash{}{0pt}%
\pgfpathmoveto{\pgfqpoint{0.000000in}{0.000000in}}%
\pgfpathlineto{\pgfqpoint{0.000000in}{0.000000in}}%
\pgfpathclose%
\pgfusepath{stroke,fill}%
\end{pgfscope}%
\begin{pgfscope}%
\pgfpathrectangle{\pgfqpoint{0.647939in}{0.492442in}}{\pgfqpoint{3.079299in}{3.079299in}}%
\pgfusepath{clip}%
\pgfsetroundcap%
\pgfsetroundjoin%
\pgfsetlinewidth{0.301125pt}%
\definecolor{currentstroke}{rgb}{0.500000,0.500000,0.500000}%
\pgfsetstrokecolor{currentstroke}%
\pgfsetstrokeopacity{0.300000}%
\pgfsetdash{}{0pt}%
\pgfpathmoveto{\pgfqpoint{3.602763in}{1.859604in}}%
\pgfusepath{stroke}%
\end{pgfscope}%
\begin{pgfscope}%
\pgfpathrectangle{\pgfqpoint{0.647939in}{0.492442in}}{\pgfqpoint{3.079299in}{3.079299in}}%
\pgfusepath{clip}%
\pgfsetroundcap%
\pgfsetroundjoin%
\definecolor{currentfill}{rgb}{0.500000,0.500000,0.500000}%
\pgfsetfillcolor{currentfill}%
\pgfsetfillopacity{0.300000}%
\pgfsetlinewidth{0.301125pt}%
\definecolor{currentstroke}{rgb}{0.500000,0.500000,0.500000}%
\pgfsetstrokecolor{currentstroke}%
\pgfsetstrokeopacity{0.300000}%
\pgfsetdash{}{0pt}%
\pgfpathmoveto{\pgfqpoint{0.000000in}{0.000000in}}%
\pgfpathlineto{\pgfqpoint{0.000000in}{0.000000in}}%
\pgfpathclose%
\pgfusepath{stroke,fill}%
\end{pgfscope}%
\begin{pgfscope}%
\pgfpathrectangle{\pgfqpoint{0.647939in}{0.492442in}}{\pgfqpoint{3.079299in}{3.079299in}}%
\pgfusepath{clip}%
\pgfsetroundcap%
\pgfsetroundjoin%
\pgfsetlinewidth{0.301125pt}%
\definecolor{currentstroke}{rgb}{0.500000,0.500000,0.500000}%
\pgfsetstrokecolor{currentstroke}%
\pgfsetstrokeopacity{0.300000}%
\pgfsetdash{}{0pt}%
\pgfpathmoveto{\pgfqpoint{3.328976in}{2.612563in}}%
\pgfusepath{stroke}%
\end{pgfscope}%
\begin{pgfscope}%
\pgfpathrectangle{\pgfqpoint{0.647939in}{0.492442in}}{\pgfqpoint{3.079299in}{3.079299in}}%
\pgfusepath{clip}%
\pgfsetroundcap%
\pgfsetroundjoin%
\definecolor{currentfill}{rgb}{0.500000,0.500000,0.500000}%
\pgfsetfillcolor{currentfill}%
\pgfsetfillopacity{0.300000}%
\pgfsetlinewidth{0.301125pt}%
\definecolor{currentstroke}{rgb}{0.500000,0.500000,0.500000}%
\pgfsetstrokecolor{currentstroke}%
\pgfsetstrokeopacity{0.300000}%
\pgfsetdash{}{0pt}%
\pgfpathmoveto{\pgfqpoint{0.000000in}{0.000000in}}%
\pgfpathlineto{\pgfqpoint{0.000000in}{0.000000in}}%
\pgfpathclose%
\pgfusepath{stroke,fill}%
\end{pgfscope}%
\begin{pgfscope}%
\pgfpathrectangle{\pgfqpoint{0.647939in}{0.492442in}}{\pgfqpoint{3.079299in}{3.079299in}}%
\pgfusepath{clip}%
\pgfsetroundcap%
\pgfsetroundjoin%
\pgfsetlinewidth{0.301125pt}%
\definecolor{currentstroke}{rgb}{0.500000,0.500000,0.500000}%
\pgfsetstrokecolor{currentstroke}%
\pgfsetstrokeopacity{0.300000}%
\pgfsetdash{}{0pt}%
\pgfpathmoveto{\pgfqpoint{3.440201in}{2.593640in}}%
\pgfusepath{stroke}%
\end{pgfscope}%
\begin{pgfscope}%
\pgfpathrectangle{\pgfqpoint{0.647939in}{0.492442in}}{\pgfqpoint{3.079299in}{3.079299in}}%
\pgfusepath{clip}%
\pgfsetroundcap%
\pgfsetroundjoin%
\definecolor{currentfill}{rgb}{0.500000,0.500000,0.500000}%
\pgfsetfillcolor{currentfill}%
\pgfsetfillopacity{0.300000}%
\pgfsetlinewidth{0.301125pt}%
\definecolor{currentstroke}{rgb}{0.500000,0.500000,0.500000}%
\pgfsetstrokecolor{currentstroke}%
\pgfsetstrokeopacity{0.300000}%
\pgfsetdash{}{0pt}%
\pgfpathmoveto{\pgfqpoint{0.000000in}{0.000000in}}%
\pgfpathlineto{\pgfqpoint{0.000000in}{0.000000in}}%
\pgfpathclose%
\pgfusepath{stroke,fill}%
\end{pgfscope}%
\begin{pgfscope}%
\pgfpathrectangle{\pgfqpoint{0.647939in}{0.492442in}}{\pgfqpoint{3.079299in}{3.079299in}}%
\pgfusepath{clip}%
\pgfsetroundcap%
\pgfsetroundjoin%
\pgfsetlinewidth{0.301125pt}%
\definecolor{currentstroke}{rgb}{0.500000,0.500000,0.500000}%
\pgfsetstrokecolor{currentstroke}%
\pgfsetstrokeopacity{0.300000}%
\pgfsetdash{}{0pt}%
\pgfpathmoveto{\pgfqpoint{3.535913in}{2.631734in}}%
\pgfusepath{stroke}%
\end{pgfscope}%
\begin{pgfscope}%
\pgfpathrectangle{\pgfqpoint{0.647939in}{0.492442in}}{\pgfqpoint{3.079299in}{3.079299in}}%
\pgfusepath{clip}%
\pgfsetroundcap%
\pgfsetroundjoin%
\definecolor{currentfill}{rgb}{0.500000,0.500000,0.500000}%
\pgfsetfillcolor{currentfill}%
\pgfsetfillopacity{0.300000}%
\pgfsetlinewidth{0.301125pt}%
\definecolor{currentstroke}{rgb}{0.500000,0.500000,0.500000}%
\pgfsetstrokecolor{currentstroke}%
\pgfsetstrokeopacity{0.300000}%
\pgfsetdash{}{0pt}%
\pgfpathmoveto{\pgfqpoint{0.000000in}{0.000000in}}%
\pgfpathlineto{\pgfqpoint{0.000000in}{0.000000in}}%
\pgfpathclose%
\pgfusepath{stroke,fill}%
\end{pgfscope}%
\begin{pgfscope}%
\pgfpathrectangle{\pgfqpoint{0.647939in}{0.492442in}}{\pgfqpoint{3.079299in}{3.079299in}}%
\pgfusepath{clip}%
\pgfsetroundcap%
\pgfsetroundjoin%
\pgfsetlinewidth{0.301125pt}%
\definecolor{currentstroke}{rgb}{0.500000,0.500000,0.500000}%
\pgfsetstrokecolor{currentstroke}%
\pgfsetstrokeopacity{0.300000}%
\pgfsetdash{}{0pt}%
\pgfpathmoveto{\pgfqpoint{3.615788in}{2.659215in}}%
\pgfusepath{stroke}%
\end{pgfscope}%
\begin{pgfscope}%
\pgfpathrectangle{\pgfqpoint{0.647939in}{0.492442in}}{\pgfqpoint{3.079299in}{3.079299in}}%
\pgfusepath{clip}%
\pgfsetroundcap%
\pgfsetroundjoin%
\definecolor{currentfill}{rgb}{0.500000,0.500000,0.500000}%
\pgfsetfillcolor{currentfill}%
\pgfsetfillopacity{0.300000}%
\pgfsetlinewidth{0.301125pt}%
\definecolor{currentstroke}{rgb}{0.500000,0.500000,0.500000}%
\pgfsetstrokecolor{currentstroke}%
\pgfsetstrokeopacity{0.300000}%
\pgfsetdash{}{0pt}%
\pgfpathmoveto{\pgfqpoint{0.000000in}{0.000000in}}%
\pgfpathlineto{\pgfqpoint{0.000000in}{0.000000in}}%
\pgfpathclose%
\pgfusepath{stroke,fill}%
\end{pgfscope}%
\begin{pgfscope}%
\pgfpathrectangle{\pgfqpoint{0.647939in}{0.492442in}}{\pgfqpoint{3.079299in}{3.079299in}}%
\pgfusepath{clip}%
\pgfsetroundcap%
\pgfsetroundjoin%
\pgfsetlinewidth{0.301125pt}%
\definecolor{currentstroke}{rgb}{0.500000,0.500000,0.500000}%
\pgfsetstrokecolor{currentstroke}%
\pgfsetstrokeopacity{0.300000}%
\pgfsetdash{}{0pt}%
\pgfpathmoveto{\pgfqpoint{3.699990in}{2.613768in}}%
\pgfusepath{stroke}%
\end{pgfscope}%
\begin{pgfscope}%
\pgfpathrectangle{\pgfqpoint{0.647939in}{0.492442in}}{\pgfqpoint{3.079299in}{3.079299in}}%
\pgfusepath{clip}%
\pgfsetroundcap%
\pgfsetroundjoin%
\definecolor{currentfill}{rgb}{0.500000,0.500000,0.500000}%
\pgfsetfillcolor{currentfill}%
\pgfsetfillopacity{0.300000}%
\pgfsetlinewidth{0.301125pt}%
\definecolor{currentstroke}{rgb}{0.500000,0.500000,0.500000}%
\pgfsetstrokecolor{currentstroke}%
\pgfsetstrokeopacity{0.300000}%
\pgfsetdash{}{0pt}%
\pgfpathmoveto{\pgfqpoint{0.000000in}{0.000000in}}%
\pgfpathlineto{\pgfqpoint{0.000000in}{0.000000in}}%
\pgfpathclose%
\pgfusepath{stroke,fill}%
\end{pgfscope}%
\begin{pgfscope}%
\pgfpathrectangle{\pgfqpoint{0.647939in}{0.492442in}}{\pgfqpoint{3.079299in}{3.079299in}}%
\pgfusepath{clip}%
\pgfsetroundcap%
\pgfsetroundjoin%
\pgfsetlinewidth{0.301125pt}%
\definecolor{currentstroke}{rgb}{0.500000,0.500000,0.500000}%
\pgfsetstrokecolor{currentstroke}%
\pgfsetstrokeopacity{0.300000}%
\pgfsetdash{}{0pt}%
\pgfpathmoveto{\pgfqpoint{3.494372in}{3.361313in}}%
\pgfusepath{stroke}%
\end{pgfscope}%
\begin{pgfscope}%
\pgfpathrectangle{\pgfqpoint{0.647939in}{0.492442in}}{\pgfqpoint{3.079299in}{3.079299in}}%
\pgfusepath{clip}%
\pgfsetroundcap%
\pgfsetroundjoin%
\definecolor{currentfill}{rgb}{0.500000,0.500000,0.500000}%
\pgfsetfillcolor{currentfill}%
\pgfsetfillopacity{0.300000}%
\pgfsetlinewidth{0.301125pt}%
\definecolor{currentstroke}{rgb}{0.500000,0.500000,0.500000}%
\pgfsetstrokecolor{currentstroke}%
\pgfsetstrokeopacity{0.300000}%
\pgfsetdash{}{0pt}%
\pgfpathmoveto{\pgfqpoint{0.000000in}{0.000000in}}%
\pgfpathlineto{\pgfqpoint{0.000000in}{0.000000in}}%
\pgfpathclose%
\pgfusepath{stroke,fill}%
\end{pgfscope}%
\begin{pgfscope}%
\pgfpathrectangle{\pgfqpoint{0.647939in}{0.492442in}}{\pgfqpoint{3.079299in}{3.079299in}}%
\pgfusepath{clip}%
\pgfsetroundcap%
\pgfsetroundjoin%
\pgfsetlinewidth{0.301125pt}%
\definecolor{currentstroke}{rgb}{0.500000,0.500000,0.500000}%
\pgfsetstrokecolor{currentstroke}%
\pgfsetstrokeopacity{0.300000}%
\pgfsetdash{}{0pt}%
\pgfpathmoveto{\pgfqpoint{2.176371in}{2.779579in}}%
\pgfusepath{stroke}%
\end{pgfscope}%
\begin{pgfscope}%
\pgfpathrectangle{\pgfqpoint{0.647939in}{0.492442in}}{\pgfqpoint{3.079299in}{3.079299in}}%
\pgfusepath{clip}%
\pgfsetroundcap%
\pgfsetroundjoin%
\definecolor{currentfill}{rgb}{0.500000,0.500000,0.500000}%
\pgfsetfillcolor{currentfill}%
\pgfsetfillopacity{0.300000}%
\pgfsetlinewidth{0.301125pt}%
\definecolor{currentstroke}{rgb}{0.500000,0.500000,0.500000}%
\pgfsetstrokecolor{currentstroke}%
\pgfsetstrokeopacity{0.300000}%
\pgfsetdash{}{0pt}%
\pgfpathmoveto{\pgfqpoint{0.000000in}{0.000000in}}%
\pgfpathlineto{\pgfqpoint{0.000000in}{0.000000in}}%
\pgfpathclose%
\pgfusepath{stroke,fill}%
\end{pgfscope}%
\begin{pgfscope}%
\pgfpathrectangle{\pgfqpoint{0.647939in}{0.492442in}}{\pgfqpoint{3.079299in}{3.079299in}}%
\pgfusepath{clip}%
\pgfsetroundcap%
\pgfsetroundjoin%
\pgfsetlinewidth{0.301125pt}%
\definecolor{currentstroke}{rgb}{0.500000,0.500000,0.500000}%
\pgfsetstrokecolor{currentstroke}%
\pgfsetstrokeopacity{0.300000}%
\pgfsetdash{}{0pt}%
\pgfpathmoveto{\pgfqpoint{2.048324in}{3.045682in}}%
\pgfusepath{stroke}%
\end{pgfscope}%
\begin{pgfscope}%
\pgfpathrectangle{\pgfqpoint{0.647939in}{0.492442in}}{\pgfqpoint{3.079299in}{3.079299in}}%
\pgfusepath{clip}%
\pgfsetroundcap%
\pgfsetroundjoin%
\definecolor{currentfill}{rgb}{0.500000,0.500000,0.500000}%
\pgfsetfillcolor{currentfill}%
\pgfsetfillopacity{0.300000}%
\pgfsetlinewidth{0.301125pt}%
\definecolor{currentstroke}{rgb}{0.500000,0.500000,0.500000}%
\pgfsetstrokecolor{currentstroke}%
\pgfsetstrokeopacity{0.300000}%
\pgfsetdash{}{0pt}%
\pgfpathmoveto{\pgfqpoint{0.000000in}{0.000000in}}%
\pgfpathlineto{\pgfqpoint{0.000000in}{0.000000in}}%
\pgfpathclose%
\pgfusepath{stroke,fill}%
\end{pgfscope}%
\begin{pgfscope}%
\pgfpathrectangle{\pgfqpoint{0.647939in}{0.492442in}}{\pgfqpoint{3.079299in}{3.079299in}}%
\pgfusepath{clip}%
\pgfsetroundcap%
\pgfsetroundjoin%
\pgfsetlinewidth{0.301125pt}%
\definecolor{currentstroke}{rgb}{0.500000,0.500000,0.500000}%
\pgfsetstrokecolor{currentstroke}%
\pgfsetstrokeopacity{0.300000}%
\pgfsetdash{}{0pt}%
\pgfpathmoveto{\pgfqpoint{1.909537in}{3.210511in}}%
\pgfusepath{stroke}%
\end{pgfscope}%
\begin{pgfscope}%
\pgfpathrectangle{\pgfqpoint{0.647939in}{0.492442in}}{\pgfqpoint{3.079299in}{3.079299in}}%
\pgfusepath{clip}%
\pgfsetroundcap%
\pgfsetroundjoin%
\definecolor{currentfill}{rgb}{0.500000,0.500000,0.500000}%
\pgfsetfillcolor{currentfill}%
\pgfsetfillopacity{0.300000}%
\pgfsetlinewidth{0.301125pt}%
\definecolor{currentstroke}{rgb}{0.500000,0.500000,0.500000}%
\pgfsetstrokecolor{currentstroke}%
\pgfsetstrokeopacity{0.300000}%
\pgfsetdash{}{0pt}%
\pgfpathmoveto{\pgfqpoint{0.000000in}{0.000000in}}%
\pgfpathlineto{\pgfqpoint{0.000000in}{0.000000in}}%
\pgfpathclose%
\pgfusepath{stroke,fill}%
\end{pgfscope}%
\begin{pgfscope}%
\pgfpathrectangle{\pgfqpoint{0.647939in}{0.492442in}}{\pgfqpoint{3.079299in}{3.079299in}}%
\pgfusepath{clip}%
\pgfsetroundcap%
\pgfsetroundjoin%
\pgfsetlinewidth{0.301125pt}%
\definecolor{currentstroke}{rgb}{0.500000,0.500000,0.500000}%
\pgfsetstrokecolor{currentstroke}%
\pgfsetstrokeopacity{0.300000}%
\pgfsetdash{}{0pt}%
\pgfpathmoveto{\pgfqpoint{1.866128in}{3.331372in}}%
\pgfusepath{stroke}%
\end{pgfscope}%
\begin{pgfscope}%
\pgfpathrectangle{\pgfqpoint{0.647939in}{0.492442in}}{\pgfqpoint{3.079299in}{3.079299in}}%
\pgfusepath{clip}%
\pgfsetroundcap%
\pgfsetroundjoin%
\definecolor{currentfill}{rgb}{0.500000,0.500000,0.500000}%
\pgfsetfillcolor{currentfill}%
\pgfsetfillopacity{0.300000}%
\pgfsetlinewidth{0.301125pt}%
\definecolor{currentstroke}{rgb}{0.500000,0.500000,0.500000}%
\pgfsetstrokecolor{currentstroke}%
\pgfsetstrokeopacity{0.300000}%
\pgfsetdash{}{0pt}%
\pgfpathmoveto{\pgfqpoint{0.000000in}{0.000000in}}%
\pgfpathlineto{\pgfqpoint{0.000000in}{0.000000in}}%
\pgfpathclose%
\pgfusepath{stroke,fill}%
\end{pgfscope}%
\begin{pgfscope}%
\pgfpathrectangle{\pgfqpoint{0.647939in}{0.492442in}}{\pgfqpoint{3.079299in}{3.079299in}}%
\pgfusepath{clip}%
\pgfsetroundcap%
\pgfsetroundjoin%
\pgfsetlinewidth{0.301125pt}%
\definecolor{currentstroke}{rgb}{0.500000,0.500000,0.500000}%
\pgfsetstrokecolor{currentstroke}%
\pgfsetstrokeopacity{0.300000}%
\pgfsetdash{}{0pt}%
\pgfpathmoveto{\pgfqpoint{1.771948in}{3.410132in}}%
\pgfusepath{stroke}%
\end{pgfscope}%
\begin{pgfscope}%
\pgfpathrectangle{\pgfqpoint{0.647939in}{0.492442in}}{\pgfqpoint{3.079299in}{3.079299in}}%
\pgfusepath{clip}%
\pgfsetroundcap%
\pgfsetroundjoin%
\definecolor{currentfill}{rgb}{0.500000,0.500000,0.500000}%
\pgfsetfillcolor{currentfill}%
\pgfsetfillopacity{0.300000}%
\pgfsetlinewidth{0.301125pt}%
\definecolor{currentstroke}{rgb}{0.500000,0.500000,0.500000}%
\pgfsetstrokecolor{currentstroke}%
\pgfsetstrokeopacity{0.300000}%
\pgfsetdash{}{0pt}%
\pgfpathmoveto{\pgfqpoint{0.000000in}{0.000000in}}%
\pgfpathlineto{\pgfqpoint{0.000000in}{0.000000in}}%
\pgfpathclose%
\pgfusepath{stroke,fill}%
\end{pgfscope}%
\begin{pgfscope}%
\pgfpathrectangle{\pgfqpoint{0.647939in}{0.492442in}}{\pgfqpoint{3.079299in}{3.079299in}}%
\pgfusepath{clip}%
\pgfsetroundcap%
\pgfsetroundjoin%
\pgfsetlinewidth{0.301125pt}%
\definecolor{currentstroke}{rgb}{0.500000,0.500000,0.500000}%
\pgfsetstrokecolor{currentstroke}%
\pgfsetstrokeopacity{0.300000}%
\pgfsetdash{}{0pt}%
\pgfpathmoveto{\pgfqpoint{1.617085in}{3.470411in}}%
\pgfusepath{stroke}%
\end{pgfscope}%
\begin{pgfscope}%
\pgfpathrectangle{\pgfqpoint{0.647939in}{0.492442in}}{\pgfqpoint{3.079299in}{3.079299in}}%
\pgfusepath{clip}%
\pgfsetroundcap%
\pgfsetroundjoin%
\definecolor{currentfill}{rgb}{0.500000,0.500000,0.500000}%
\pgfsetfillcolor{currentfill}%
\pgfsetfillopacity{0.300000}%
\pgfsetlinewidth{0.301125pt}%
\definecolor{currentstroke}{rgb}{0.500000,0.500000,0.500000}%
\pgfsetstrokecolor{currentstroke}%
\pgfsetstrokeopacity{0.300000}%
\pgfsetdash{}{0pt}%
\pgfpathmoveto{\pgfqpoint{0.000000in}{0.000000in}}%
\pgfpathlineto{\pgfqpoint{0.000000in}{0.000000in}}%
\pgfpathclose%
\pgfusepath{stroke,fill}%
\end{pgfscope}%
\begin{pgfscope}%
\pgfpathrectangle{\pgfqpoint{0.647939in}{0.492442in}}{\pgfqpoint{3.079299in}{3.079299in}}%
\pgfusepath{clip}%
\pgfsetroundcap%
\pgfsetroundjoin%
\pgfsetlinewidth{0.301125pt}%
\definecolor{currentstroke}{rgb}{0.500000,0.500000,0.500000}%
\pgfsetstrokecolor{currentstroke}%
\pgfsetstrokeopacity{0.300000}%
\pgfsetdash{}{0pt}%
\pgfpathmoveto{\pgfqpoint{2.153656in}{3.555615in}}%
\pgfusepath{stroke}%
\end{pgfscope}%
\begin{pgfscope}%
\pgfpathrectangle{\pgfqpoint{0.647939in}{0.492442in}}{\pgfqpoint{3.079299in}{3.079299in}}%
\pgfusepath{clip}%
\pgfsetroundcap%
\pgfsetroundjoin%
\definecolor{currentfill}{rgb}{0.500000,0.500000,0.500000}%
\pgfsetfillcolor{currentfill}%
\pgfsetfillopacity{0.300000}%
\pgfsetlinewidth{0.301125pt}%
\definecolor{currentstroke}{rgb}{0.500000,0.500000,0.500000}%
\pgfsetstrokecolor{currentstroke}%
\pgfsetstrokeopacity{0.300000}%
\pgfsetdash{}{0pt}%
\pgfpathmoveto{\pgfqpoint{0.000000in}{0.000000in}}%
\pgfpathlineto{\pgfqpoint{0.000000in}{0.000000in}}%
\pgfpathclose%
\pgfusepath{stroke,fill}%
\end{pgfscope}%
\begin{pgfscope}%
\pgfpathrectangle{\pgfqpoint{0.647939in}{0.492442in}}{\pgfqpoint{3.079299in}{3.079299in}}%
\pgfusepath{clip}%
\pgfsetroundcap%
\pgfsetroundjoin%
\pgfsetlinewidth{0.301125pt}%
\definecolor{currentstroke}{rgb}{0.500000,0.500000,0.500000}%
\pgfsetstrokecolor{currentstroke}%
\pgfsetstrokeopacity{0.300000}%
\pgfsetdash{}{0pt}%
\pgfpathmoveto{\pgfqpoint{1.189732in}{3.464482in}}%
\pgfusepath{stroke}%
\end{pgfscope}%
\begin{pgfscope}%
\pgfpathrectangle{\pgfqpoint{0.647939in}{0.492442in}}{\pgfqpoint{3.079299in}{3.079299in}}%
\pgfusepath{clip}%
\pgfsetroundcap%
\pgfsetroundjoin%
\definecolor{currentfill}{rgb}{0.500000,0.500000,0.500000}%
\pgfsetfillcolor{currentfill}%
\pgfsetfillopacity{0.300000}%
\pgfsetlinewidth{0.301125pt}%
\definecolor{currentstroke}{rgb}{0.500000,0.500000,0.500000}%
\pgfsetstrokecolor{currentstroke}%
\pgfsetstrokeopacity{0.300000}%
\pgfsetdash{}{0pt}%
\pgfpathmoveto{\pgfqpoint{0.000000in}{0.000000in}}%
\pgfpathlineto{\pgfqpoint{0.000000in}{0.000000in}}%
\pgfpathclose%
\pgfusepath{stroke,fill}%
\end{pgfscope}%
\begin{pgfscope}%
\pgfpathrectangle{\pgfqpoint{0.647939in}{0.492442in}}{\pgfqpoint{3.079299in}{3.079299in}}%
\pgfusepath{clip}%
\pgfsetroundcap%
\pgfsetroundjoin%
\pgfsetlinewidth{0.301125pt}%
\definecolor{currentstroke}{rgb}{0.500000,0.500000,0.500000}%
\pgfsetstrokecolor{currentstroke}%
\pgfsetstrokeopacity{0.300000}%
\pgfsetdash{}{0pt}%
\pgfpathmoveto{\pgfqpoint{0.972672in}{3.495439in}}%
\pgfusepath{stroke}%
\end{pgfscope}%
\begin{pgfscope}%
\pgfpathrectangle{\pgfqpoint{0.647939in}{0.492442in}}{\pgfqpoint{3.079299in}{3.079299in}}%
\pgfusepath{clip}%
\pgfsetroundcap%
\pgfsetroundjoin%
\definecolor{currentfill}{rgb}{0.500000,0.500000,0.500000}%
\pgfsetfillcolor{currentfill}%
\pgfsetfillopacity{0.300000}%
\pgfsetlinewidth{0.301125pt}%
\definecolor{currentstroke}{rgb}{0.500000,0.500000,0.500000}%
\pgfsetstrokecolor{currentstroke}%
\pgfsetstrokeopacity{0.300000}%
\pgfsetdash{}{0pt}%
\pgfpathmoveto{\pgfqpoint{0.000000in}{0.000000in}}%
\pgfpathlineto{\pgfqpoint{0.000000in}{0.000000in}}%
\pgfpathclose%
\pgfusepath{stroke,fill}%
\end{pgfscope}%
\begin{pgfscope}%
\pgfpathrectangle{\pgfqpoint{0.647939in}{0.492442in}}{\pgfqpoint{3.079299in}{3.079299in}}%
\pgfusepath{clip}%
\pgfsetroundcap%
\pgfsetroundjoin%
\pgfsetlinewidth{0.301125pt}%
\definecolor{currentstroke}{rgb}{0.500000,0.500000,0.500000}%
\pgfsetstrokecolor{currentstroke}%
\pgfsetstrokeopacity{0.300000}%
\pgfsetdash{}{0pt}%
\pgfpathmoveto{\pgfqpoint{0.822729in}{3.542800in}}%
\pgfusepath{stroke}%
\end{pgfscope}%
\begin{pgfscope}%
\pgfpathrectangle{\pgfqpoint{0.647939in}{0.492442in}}{\pgfqpoint{3.079299in}{3.079299in}}%
\pgfusepath{clip}%
\pgfsetroundcap%
\pgfsetroundjoin%
\definecolor{currentfill}{rgb}{0.500000,0.500000,0.500000}%
\pgfsetfillcolor{currentfill}%
\pgfsetfillopacity{0.300000}%
\pgfsetlinewidth{0.301125pt}%
\definecolor{currentstroke}{rgb}{0.500000,0.500000,0.500000}%
\pgfsetstrokecolor{currentstroke}%
\pgfsetstrokeopacity{0.300000}%
\pgfsetdash{}{0pt}%
\pgfpathmoveto{\pgfqpoint{0.000000in}{0.000000in}}%
\pgfpathlineto{\pgfqpoint{0.000000in}{0.000000in}}%
\pgfpathclose%
\pgfusepath{stroke,fill}%
\end{pgfscope}%
\begin{pgfscope}%
\pgfpathrectangle{\pgfqpoint{0.647939in}{0.492442in}}{\pgfqpoint{3.079299in}{3.079299in}}%
\pgfusepath{clip}%
\pgfsetroundcap%
\pgfsetroundjoin%
\pgfsetlinewidth{0.301125pt}%
\definecolor{currentstroke}{rgb}{0.500000,0.500000,0.500000}%
\pgfsetstrokecolor{currentstroke}%
\pgfsetstrokeopacity{0.300000}%
\pgfsetdash{}{0pt}%
\pgfpathmoveto{\pgfqpoint{1.745957in}{3.101563in}}%
\pgfusepath{stroke}%
\end{pgfscope}%
\begin{pgfscope}%
\pgfpathrectangle{\pgfqpoint{0.647939in}{0.492442in}}{\pgfqpoint{3.079299in}{3.079299in}}%
\pgfusepath{clip}%
\pgfsetroundcap%
\pgfsetroundjoin%
\definecolor{currentfill}{rgb}{0.500000,0.500000,0.500000}%
\pgfsetfillcolor{currentfill}%
\pgfsetfillopacity{0.300000}%
\pgfsetlinewidth{0.301125pt}%
\definecolor{currentstroke}{rgb}{0.500000,0.500000,0.500000}%
\pgfsetstrokecolor{currentstroke}%
\pgfsetstrokeopacity{0.300000}%
\pgfsetdash{}{0pt}%
\pgfpathmoveto{\pgfqpoint{0.000000in}{0.000000in}}%
\pgfpathlineto{\pgfqpoint{0.000000in}{0.000000in}}%
\pgfpathclose%
\pgfusepath{stroke,fill}%
\end{pgfscope}%
\begin{pgfscope}%
\pgfpathrectangle{\pgfqpoint{0.647939in}{0.492442in}}{\pgfqpoint{3.079299in}{3.079299in}}%
\pgfusepath{clip}%
\pgfsetroundcap%
\pgfsetroundjoin%
\pgfsetlinewidth{0.301125pt}%
\definecolor{currentstroke}{rgb}{0.500000,0.500000,0.500000}%
\pgfsetstrokecolor{currentstroke}%
\pgfsetstrokeopacity{0.300000}%
\pgfsetdash{}{0pt}%
\pgfpathmoveto{\pgfqpoint{1.877318in}{2.929143in}}%
\pgfusepath{stroke}%
\end{pgfscope}%
\begin{pgfscope}%
\pgfpathrectangle{\pgfqpoint{0.647939in}{0.492442in}}{\pgfqpoint{3.079299in}{3.079299in}}%
\pgfusepath{clip}%
\pgfsetroundcap%
\pgfsetroundjoin%
\definecolor{currentfill}{rgb}{0.500000,0.500000,0.500000}%
\pgfsetfillcolor{currentfill}%
\pgfsetfillopacity{0.300000}%
\pgfsetlinewidth{0.301125pt}%
\definecolor{currentstroke}{rgb}{0.500000,0.500000,0.500000}%
\pgfsetstrokecolor{currentstroke}%
\pgfsetstrokeopacity{0.300000}%
\pgfsetdash{}{0pt}%
\pgfpathmoveto{\pgfqpoint{0.000000in}{0.000000in}}%
\pgfpathlineto{\pgfqpoint{0.000000in}{0.000000in}}%
\pgfpathclose%
\pgfusepath{stroke,fill}%
\end{pgfscope}%
\begin{pgfscope}%
\pgfpathrectangle{\pgfqpoint{0.647939in}{0.492442in}}{\pgfqpoint{3.079299in}{3.079299in}}%
\pgfusepath{clip}%
\pgfsetroundcap%
\pgfsetroundjoin%
\pgfsetlinewidth{0.301125pt}%
\definecolor{currentstroke}{rgb}{0.500000,0.500000,0.500000}%
\pgfsetstrokecolor{currentstroke}%
\pgfsetstrokeopacity{0.300000}%
\pgfsetdash{}{0pt}%
\pgfpathmoveto{\pgfqpoint{1.144493in}{2.690908in}}%
\pgfusepath{stroke}%
\end{pgfscope}%
\begin{pgfscope}%
\pgfpathrectangle{\pgfqpoint{0.647939in}{0.492442in}}{\pgfqpoint{3.079299in}{3.079299in}}%
\pgfusepath{clip}%
\pgfsetroundcap%
\pgfsetroundjoin%
\definecolor{currentfill}{rgb}{0.500000,0.500000,0.500000}%
\pgfsetfillcolor{currentfill}%
\pgfsetfillopacity{0.300000}%
\pgfsetlinewidth{0.301125pt}%
\definecolor{currentstroke}{rgb}{0.500000,0.500000,0.500000}%
\pgfsetstrokecolor{currentstroke}%
\pgfsetstrokeopacity{0.300000}%
\pgfsetdash{}{0pt}%
\pgfpathmoveto{\pgfqpoint{0.000000in}{0.000000in}}%
\pgfpathlineto{\pgfqpoint{0.000000in}{0.000000in}}%
\pgfpathclose%
\pgfusepath{stroke,fill}%
\end{pgfscope}%
\begin{pgfscope}%
\pgfpathrectangle{\pgfqpoint{0.647939in}{0.492442in}}{\pgfqpoint{3.079299in}{3.079299in}}%
\pgfusepath{clip}%
\pgfsetroundcap%
\pgfsetroundjoin%
\pgfsetlinewidth{0.301125pt}%
\definecolor{currentstroke}{rgb}{0.500000,0.500000,0.500000}%
\pgfsetstrokecolor{currentstroke}%
\pgfsetstrokeopacity{0.300000}%
\pgfsetdash{}{0pt}%
\pgfpathmoveto{\pgfqpoint{1.011132in}{2.450869in}}%
\pgfusepath{stroke}%
\end{pgfscope}%
\begin{pgfscope}%
\pgfpathrectangle{\pgfqpoint{0.647939in}{0.492442in}}{\pgfqpoint{3.079299in}{3.079299in}}%
\pgfusepath{clip}%
\pgfsetroundcap%
\pgfsetroundjoin%
\definecolor{currentfill}{rgb}{0.500000,0.500000,0.500000}%
\pgfsetfillcolor{currentfill}%
\pgfsetfillopacity{0.300000}%
\pgfsetlinewidth{0.301125pt}%
\definecolor{currentstroke}{rgb}{0.500000,0.500000,0.500000}%
\pgfsetstrokecolor{currentstroke}%
\pgfsetstrokeopacity{0.300000}%
\pgfsetdash{}{0pt}%
\pgfpathmoveto{\pgfqpoint{0.000000in}{0.000000in}}%
\pgfpathlineto{\pgfqpoint{0.000000in}{0.000000in}}%
\pgfpathclose%
\pgfusepath{stroke,fill}%
\end{pgfscope}%
\begin{pgfscope}%
\pgfpathrectangle{\pgfqpoint{0.647939in}{0.492442in}}{\pgfqpoint{3.079299in}{3.079299in}}%
\pgfusepath{clip}%
\pgfsetroundcap%
\pgfsetroundjoin%
\pgfsetlinewidth{0.301125pt}%
\definecolor{currentstroke}{rgb}{0.500000,0.500000,0.500000}%
\pgfsetstrokecolor{currentstroke}%
\pgfsetstrokeopacity{0.300000}%
\pgfsetdash{}{0pt}%
\pgfpathmoveto{\pgfqpoint{1.010895in}{2.382048in}}%
\pgfusepath{stroke}%
\end{pgfscope}%
\begin{pgfscope}%
\pgfpathrectangle{\pgfqpoint{0.647939in}{0.492442in}}{\pgfqpoint{3.079299in}{3.079299in}}%
\pgfusepath{clip}%
\pgfsetroundcap%
\pgfsetroundjoin%
\definecolor{currentfill}{rgb}{0.500000,0.500000,0.500000}%
\pgfsetfillcolor{currentfill}%
\pgfsetfillopacity{0.300000}%
\pgfsetlinewidth{0.301125pt}%
\definecolor{currentstroke}{rgb}{0.500000,0.500000,0.500000}%
\pgfsetstrokecolor{currentstroke}%
\pgfsetstrokeopacity{0.300000}%
\pgfsetdash{}{0pt}%
\pgfpathmoveto{\pgfqpoint{0.000000in}{0.000000in}}%
\pgfpathlineto{\pgfqpoint{0.000000in}{0.000000in}}%
\pgfpathclose%
\pgfusepath{stroke,fill}%
\end{pgfscope}%
\begin{pgfscope}%
\pgfpathrectangle{\pgfqpoint{0.647939in}{0.492442in}}{\pgfqpoint{3.079299in}{3.079299in}}%
\pgfusepath{clip}%
\pgfsetroundcap%
\pgfsetroundjoin%
\pgfsetlinewidth{0.301125pt}%
\definecolor{currentstroke}{rgb}{0.500000,0.500000,0.500000}%
\pgfsetstrokecolor{currentstroke}%
\pgfsetstrokeopacity{0.300000}%
\pgfsetdash{}{0pt}%
\pgfpathmoveto{\pgfqpoint{1.797336in}{2.547183in}}%
\pgfusepath{stroke}%
\end{pgfscope}%
\begin{pgfscope}%
\pgfpathrectangle{\pgfqpoint{0.647939in}{0.492442in}}{\pgfqpoint{3.079299in}{3.079299in}}%
\pgfusepath{clip}%
\pgfsetroundcap%
\pgfsetroundjoin%
\definecolor{currentfill}{rgb}{0.500000,0.500000,0.500000}%
\pgfsetfillcolor{currentfill}%
\pgfsetfillopacity{0.300000}%
\pgfsetlinewidth{0.301125pt}%
\definecolor{currentstroke}{rgb}{0.500000,0.500000,0.500000}%
\pgfsetstrokecolor{currentstroke}%
\pgfsetstrokeopacity{0.300000}%
\pgfsetdash{}{0pt}%
\pgfpathmoveto{\pgfqpoint{0.000000in}{0.000000in}}%
\pgfpathlineto{\pgfqpoint{0.000000in}{0.000000in}}%
\pgfpathclose%
\pgfusepath{stroke,fill}%
\end{pgfscope}%
\begin{pgfscope}%
\pgfpathrectangle{\pgfqpoint{0.647939in}{0.492442in}}{\pgfqpoint{3.079299in}{3.079299in}}%
\pgfusepath{clip}%
\pgfsetroundcap%
\pgfsetroundjoin%
\pgfsetlinewidth{0.301125pt}%
\definecolor{currentstroke}{rgb}{0.500000,0.500000,0.500000}%
\pgfsetstrokecolor{currentstroke}%
\pgfsetstrokeopacity{0.300000}%
\pgfsetdash{}{0pt}%
\pgfpathmoveto{\pgfqpoint{1.207093in}{2.303102in}}%
\pgfusepath{stroke}%
\end{pgfscope}%
\begin{pgfscope}%
\pgfpathrectangle{\pgfqpoint{0.647939in}{0.492442in}}{\pgfqpoint{3.079299in}{3.079299in}}%
\pgfusepath{clip}%
\pgfsetroundcap%
\pgfsetroundjoin%
\definecolor{currentfill}{rgb}{0.500000,0.500000,0.500000}%
\pgfsetfillcolor{currentfill}%
\pgfsetfillopacity{0.300000}%
\pgfsetlinewidth{0.301125pt}%
\definecolor{currentstroke}{rgb}{0.500000,0.500000,0.500000}%
\pgfsetstrokecolor{currentstroke}%
\pgfsetstrokeopacity{0.300000}%
\pgfsetdash{}{0pt}%
\pgfpathmoveto{\pgfqpoint{0.000000in}{0.000000in}}%
\pgfpathlineto{\pgfqpoint{0.000000in}{0.000000in}}%
\pgfpathclose%
\pgfusepath{stroke,fill}%
\end{pgfscope}%
\begin{pgfscope}%
\pgfpathrectangle{\pgfqpoint{0.647939in}{0.492442in}}{\pgfqpoint{3.079299in}{3.079299in}}%
\pgfusepath{clip}%
\pgfsetroundcap%
\pgfsetroundjoin%
\pgfsetlinewidth{0.301125pt}%
\definecolor{currentstroke}{rgb}{0.500000,0.500000,0.500000}%
\pgfsetstrokecolor{currentstroke}%
\pgfsetstrokeopacity{0.300000}%
\pgfsetdash{}{0pt}%
\pgfpathmoveto{\pgfqpoint{1.206387in}{2.235809in}}%
\pgfusepath{stroke}%
\end{pgfscope}%
\begin{pgfscope}%
\pgfpathrectangle{\pgfqpoint{0.647939in}{0.492442in}}{\pgfqpoint{3.079299in}{3.079299in}}%
\pgfusepath{clip}%
\pgfsetroundcap%
\pgfsetroundjoin%
\definecolor{currentfill}{rgb}{0.500000,0.500000,0.500000}%
\pgfsetfillcolor{currentfill}%
\pgfsetfillopacity{0.300000}%
\pgfsetlinewidth{0.301125pt}%
\definecolor{currentstroke}{rgb}{0.500000,0.500000,0.500000}%
\pgfsetstrokecolor{currentstroke}%
\pgfsetstrokeopacity{0.300000}%
\pgfsetdash{}{0pt}%
\pgfpathmoveto{\pgfqpoint{0.000000in}{0.000000in}}%
\pgfpathlineto{\pgfqpoint{0.000000in}{0.000000in}}%
\pgfpathclose%
\pgfusepath{stroke,fill}%
\end{pgfscope}%
\begin{pgfscope}%
\pgfpathrectangle{\pgfqpoint{0.647939in}{0.492442in}}{\pgfqpoint{3.079299in}{3.079299in}}%
\pgfusepath{clip}%
\pgfsetroundcap%
\pgfsetroundjoin%
\pgfsetlinewidth{0.301125pt}%
\definecolor{currentstroke}{rgb}{0.500000,0.500000,0.500000}%
\pgfsetstrokecolor{currentstroke}%
\pgfsetstrokeopacity{0.300000}%
\pgfsetdash{}{0pt}%
\pgfpathmoveto{\pgfqpoint{1.589021in}{2.246287in}}%
\pgfusepath{stroke}%
\end{pgfscope}%
\begin{pgfscope}%
\pgfpathrectangle{\pgfqpoint{0.647939in}{0.492442in}}{\pgfqpoint{3.079299in}{3.079299in}}%
\pgfusepath{clip}%
\pgfsetroundcap%
\pgfsetroundjoin%
\definecolor{currentfill}{rgb}{0.500000,0.500000,0.500000}%
\pgfsetfillcolor{currentfill}%
\pgfsetfillopacity{0.300000}%
\pgfsetlinewidth{0.301125pt}%
\definecolor{currentstroke}{rgb}{0.500000,0.500000,0.500000}%
\pgfsetstrokecolor{currentstroke}%
\pgfsetstrokeopacity{0.300000}%
\pgfsetdash{}{0pt}%
\pgfpathmoveto{\pgfqpoint{0.000000in}{0.000000in}}%
\pgfpathlineto{\pgfqpoint{0.000000in}{0.000000in}}%
\pgfpathclose%
\pgfusepath{stroke,fill}%
\end{pgfscope}%
\begin{pgfscope}%
\pgfpathrectangle{\pgfqpoint{0.647939in}{0.492442in}}{\pgfqpoint{3.079299in}{3.079299in}}%
\pgfusepath{clip}%
\pgfsetroundcap%
\pgfsetroundjoin%
\pgfsetlinewidth{0.301125pt}%
\definecolor{currentstroke}{rgb}{0.500000,0.500000,0.500000}%
\pgfsetstrokecolor{currentstroke}%
\pgfsetstrokeopacity{0.300000}%
\pgfsetdash{}{0pt}%
\pgfpathmoveto{\pgfqpoint{0.943265in}{1.951654in}}%
\pgfusepath{stroke}%
\end{pgfscope}%
\begin{pgfscope}%
\pgfpathrectangle{\pgfqpoint{0.647939in}{0.492442in}}{\pgfqpoint{3.079299in}{3.079299in}}%
\pgfusepath{clip}%
\pgfsetroundcap%
\pgfsetroundjoin%
\definecolor{currentfill}{rgb}{0.500000,0.500000,0.500000}%
\pgfsetfillcolor{currentfill}%
\pgfsetfillopacity{0.300000}%
\pgfsetlinewidth{0.301125pt}%
\definecolor{currentstroke}{rgb}{0.500000,0.500000,0.500000}%
\pgfsetstrokecolor{currentstroke}%
\pgfsetstrokeopacity{0.300000}%
\pgfsetdash{}{0pt}%
\pgfpathmoveto{\pgfqpoint{0.000000in}{0.000000in}}%
\pgfpathlineto{\pgfqpoint{0.000000in}{0.000000in}}%
\pgfpathclose%
\pgfusepath{stroke,fill}%
\end{pgfscope}%
\begin{pgfscope}%
\pgfpathrectangle{\pgfqpoint{0.647939in}{0.492442in}}{\pgfqpoint{3.079299in}{3.079299in}}%
\pgfusepath{clip}%
\pgfsetroundcap%
\pgfsetroundjoin%
\pgfsetlinewidth{0.301125pt}%
\definecolor{currentstroke}{rgb}{0.500000,0.500000,0.500000}%
\pgfsetstrokecolor{currentstroke}%
\pgfsetstrokeopacity{0.300000}%
\pgfsetdash{}{0pt}%
\pgfpathmoveto{\pgfqpoint{1.519543in}{2.097668in}}%
\pgfusepath{stroke}%
\end{pgfscope}%
\begin{pgfscope}%
\pgfpathrectangle{\pgfqpoint{0.647939in}{0.492442in}}{\pgfqpoint{3.079299in}{3.079299in}}%
\pgfusepath{clip}%
\pgfsetroundcap%
\pgfsetroundjoin%
\definecolor{currentfill}{rgb}{0.500000,0.500000,0.500000}%
\pgfsetfillcolor{currentfill}%
\pgfsetfillopacity{0.300000}%
\pgfsetlinewidth{0.301125pt}%
\definecolor{currentstroke}{rgb}{0.500000,0.500000,0.500000}%
\pgfsetstrokecolor{currentstroke}%
\pgfsetstrokeopacity{0.300000}%
\pgfsetdash{}{0pt}%
\pgfpathmoveto{\pgfqpoint{0.000000in}{0.000000in}}%
\pgfpathlineto{\pgfqpoint{0.000000in}{0.000000in}}%
\pgfpathclose%
\pgfusepath{stroke,fill}%
\end{pgfscope}%
\begin{pgfscope}%
\pgfpathrectangle{\pgfqpoint{0.647939in}{0.492442in}}{\pgfqpoint{3.079299in}{3.079299in}}%
\pgfusepath{clip}%
\pgfsetroundcap%
\pgfsetroundjoin%
\pgfsetlinewidth{0.301125pt}%
\definecolor{currentstroke}{rgb}{0.500000,0.500000,0.500000}%
\pgfsetstrokecolor{currentstroke}%
\pgfsetstrokeopacity{0.300000}%
\pgfsetdash{}{0pt}%
\pgfpathmoveto{\pgfqpoint{1.202074in}{1.901051in}}%
\pgfusepath{stroke}%
\end{pgfscope}%
\begin{pgfscope}%
\pgfpathrectangle{\pgfqpoint{0.647939in}{0.492442in}}{\pgfqpoint{3.079299in}{3.079299in}}%
\pgfusepath{clip}%
\pgfsetroundcap%
\pgfsetroundjoin%
\definecolor{currentfill}{rgb}{0.500000,0.500000,0.500000}%
\pgfsetfillcolor{currentfill}%
\pgfsetfillopacity{0.300000}%
\pgfsetlinewidth{0.301125pt}%
\definecolor{currentstroke}{rgb}{0.500000,0.500000,0.500000}%
\pgfsetstrokecolor{currentstroke}%
\pgfsetstrokeopacity{0.300000}%
\pgfsetdash{}{0pt}%
\pgfpathmoveto{\pgfqpoint{0.000000in}{0.000000in}}%
\pgfpathlineto{\pgfqpoint{0.000000in}{0.000000in}}%
\pgfpathclose%
\pgfusepath{stroke,fill}%
\end{pgfscope}%
\begin{pgfscope}%
\pgfpathrectangle{\pgfqpoint{0.647939in}{0.492442in}}{\pgfqpoint{3.079299in}{3.079299in}}%
\pgfusepath{clip}%
\pgfsetroundcap%
\pgfsetroundjoin%
\pgfsetlinewidth{0.301125pt}%
\definecolor{currentstroke}{rgb}{0.500000,0.500000,0.500000}%
\pgfsetstrokecolor{currentstroke}%
\pgfsetstrokeopacity{0.300000}%
\pgfsetdash{}{0pt}%
\pgfpathmoveto{\pgfqpoint{1.008067in}{1.764659in}}%
\pgfusepath{stroke}%
\end{pgfscope}%
\begin{pgfscope}%
\pgfpathrectangle{\pgfqpoint{0.647939in}{0.492442in}}{\pgfqpoint{3.079299in}{3.079299in}}%
\pgfusepath{clip}%
\pgfsetroundcap%
\pgfsetroundjoin%
\definecolor{currentfill}{rgb}{0.500000,0.500000,0.500000}%
\pgfsetfillcolor{currentfill}%
\pgfsetfillopacity{0.300000}%
\pgfsetlinewidth{0.301125pt}%
\definecolor{currentstroke}{rgb}{0.500000,0.500000,0.500000}%
\pgfsetstrokecolor{currentstroke}%
\pgfsetstrokeopacity{0.300000}%
\pgfsetdash{}{0pt}%
\pgfpathmoveto{\pgfqpoint{0.000000in}{0.000000in}}%
\pgfpathlineto{\pgfqpoint{0.000000in}{0.000000in}}%
\pgfpathclose%
\pgfusepath{stroke,fill}%
\end{pgfscope}%
\begin{pgfscope}%
\pgfpathrectangle{\pgfqpoint{0.647939in}{0.492442in}}{\pgfqpoint{3.079299in}{3.079299in}}%
\pgfusepath{clip}%
\pgfsetroundcap%
\pgfsetroundjoin%
\pgfsetlinewidth{0.301125pt}%
\definecolor{currentstroke}{rgb}{0.500000,0.500000,0.500000}%
\pgfsetstrokecolor{currentstroke}%
\pgfsetstrokeopacity{0.300000}%
\pgfsetdash{}{0pt}%
\pgfpathmoveto{\pgfqpoint{1.448237in}{1.883011in}}%
\pgfusepath{stroke}%
\end{pgfscope}%
\begin{pgfscope}%
\pgfpathrectangle{\pgfqpoint{0.647939in}{0.492442in}}{\pgfqpoint{3.079299in}{3.079299in}}%
\pgfusepath{clip}%
\pgfsetroundcap%
\pgfsetroundjoin%
\definecolor{currentfill}{rgb}{0.500000,0.500000,0.500000}%
\pgfsetfillcolor{currentfill}%
\pgfsetfillopacity{0.300000}%
\pgfsetlinewidth{0.301125pt}%
\definecolor{currentstroke}{rgb}{0.500000,0.500000,0.500000}%
\pgfsetstrokecolor{currentstroke}%
\pgfsetstrokeopacity{0.300000}%
\pgfsetdash{}{0pt}%
\pgfpathmoveto{\pgfqpoint{0.000000in}{0.000000in}}%
\pgfpathlineto{\pgfqpoint{0.000000in}{0.000000in}}%
\pgfpathclose%
\pgfusepath{stroke,fill}%
\end{pgfscope}%
\begin{pgfscope}%
\pgfpathrectangle{\pgfqpoint{0.647939in}{0.492442in}}{\pgfqpoint{3.079299in}{3.079299in}}%
\pgfusepath{clip}%
\pgfsetroundcap%
\pgfsetroundjoin%
\pgfsetlinewidth{0.301125pt}%
\definecolor{currentstroke}{rgb}{0.500000,0.500000,0.500000}%
\pgfsetstrokecolor{currentstroke}%
\pgfsetstrokeopacity{0.300000}%
\pgfsetdash{}{0pt}%
\pgfpathmoveto{\pgfqpoint{0.876071in}{1.589124in}}%
\pgfusepath{stroke}%
\end{pgfscope}%
\begin{pgfscope}%
\pgfpathrectangle{\pgfqpoint{0.647939in}{0.492442in}}{\pgfqpoint{3.079299in}{3.079299in}}%
\pgfusepath{clip}%
\pgfsetroundcap%
\pgfsetroundjoin%
\definecolor{currentfill}{rgb}{0.500000,0.500000,0.500000}%
\pgfsetfillcolor{currentfill}%
\pgfsetfillopacity{0.300000}%
\pgfsetlinewidth{0.301125pt}%
\definecolor{currentstroke}{rgb}{0.500000,0.500000,0.500000}%
\pgfsetstrokecolor{currentstroke}%
\pgfsetstrokeopacity{0.300000}%
\pgfsetdash{}{0pt}%
\pgfpathmoveto{\pgfqpoint{0.000000in}{0.000000in}}%
\pgfpathlineto{\pgfqpoint{0.000000in}{0.000000in}}%
\pgfpathclose%
\pgfusepath{stroke,fill}%
\end{pgfscope}%
\begin{pgfscope}%
\pgfpathrectangle{\pgfqpoint{0.647939in}{0.492442in}}{\pgfqpoint{3.079299in}{3.079299in}}%
\pgfusepath{clip}%
\pgfsetroundcap%
\pgfsetroundjoin%
\pgfsetlinewidth{0.301125pt}%
\definecolor{currentstroke}{rgb}{0.500000,0.500000,0.500000}%
\pgfsetstrokecolor{currentstroke}%
\pgfsetstrokeopacity{0.300000}%
\pgfsetdash{}{0pt}%
\pgfpathmoveto{\pgfqpoint{0.875880in}{1.520016in}}%
\pgfusepath{stroke}%
\end{pgfscope}%
\begin{pgfscope}%
\pgfpathrectangle{\pgfqpoint{0.647939in}{0.492442in}}{\pgfqpoint{3.079299in}{3.079299in}}%
\pgfusepath{clip}%
\pgfsetroundcap%
\pgfsetroundjoin%
\definecolor{currentfill}{rgb}{0.500000,0.500000,0.500000}%
\pgfsetfillcolor{currentfill}%
\pgfsetfillopacity{0.300000}%
\pgfsetlinewidth{0.301125pt}%
\definecolor{currentstroke}{rgb}{0.500000,0.500000,0.500000}%
\pgfsetstrokecolor{currentstroke}%
\pgfsetstrokeopacity{0.300000}%
\pgfsetdash{}{0pt}%
\pgfpathmoveto{\pgfqpoint{0.000000in}{0.000000in}}%
\pgfpathlineto{\pgfqpoint{0.000000in}{0.000000in}}%
\pgfpathclose%
\pgfusepath{stroke,fill}%
\end{pgfscope}%
\begin{pgfscope}%
\pgfpathrectangle{\pgfqpoint{0.647939in}{0.492442in}}{\pgfqpoint{3.079299in}{3.079299in}}%
\pgfusepath{clip}%
\pgfsetroundcap%
\pgfsetroundjoin%
\pgfsetlinewidth{0.301125pt}%
\definecolor{currentstroke}{rgb}{0.500000,0.500000,0.500000}%
\pgfsetstrokecolor{currentstroke}%
\pgfsetstrokeopacity{0.300000}%
\pgfsetdash{}{0pt}%
\pgfpathmoveto{\pgfqpoint{1.437578in}{1.698113in}}%
\pgfusepath{stroke}%
\end{pgfscope}%
\begin{pgfscope}%
\pgfpathrectangle{\pgfqpoint{0.647939in}{0.492442in}}{\pgfqpoint{3.079299in}{3.079299in}}%
\pgfusepath{clip}%
\pgfsetroundcap%
\pgfsetroundjoin%
\definecolor{currentfill}{rgb}{0.500000,0.500000,0.500000}%
\pgfsetfillcolor{currentfill}%
\pgfsetfillopacity{0.300000}%
\pgfsetlinewidth{0.301125pt}%
\definecolor{currentstroke}{rgb}{0.500000,0.500000,0.500000}%
\pgfsetstrokecolor{currentstroke}%
\pgfsetstrokeopacity{0.300000}%
\pgfsetdash{}{0pt}%
\pgfpathmoveto{\pgfqpoint{0.000000in}{0.000000in}}%
\pgfpathlineto{\pgfqpoint{0.000000in}{0.000000in}}%
\pgfpathclose%
\pgfusepath{stroke,fill}%
\end{pgfscope}%
\begin{pgfscope}%
\pgfpathrectangle{\pgfqpoint{0.647939in}{0.492442in}}{\pgfqpoint{3.079299in}{3.079299in}}%
\pgfusepath{clip}%
\pgfsetroundcap%
\pgfsetroundjoin%
\pgfsetlinewidth{0.301125pt}%
\definecolor{currentstroke}{rgb}{0.500000,0.500000,0.500000}%
\pgfsetstrokecolor{currentstroke}%
\pgfsetstrokeopacity{0.300000}%
\pgfsetdash{}{0pt}%
\pgfpathmoveto{\pgfqpoint{0.809269in}{1.364649in}}%
\pgfusepath{stroke}%
\end{pgfscope}%
\begin{pgfscope}%
\pgfpathrectangle{\pgfqpoint{0.647939in}{0.492442in}}{\pgfqpoint{3.079299in}{3.079299in}}%
\pgfusepath{clip}%
\pgfsetroundcap%
\pgfsetroundjoin%
\definecolor{currentfill}{rgb}{0.500000,0.500000,0.500000}%
\pgfsetfillcolor{currentfill}%
\pgfsetfillopacity{0.300000}%
\pgfsetlinewidth{0.301125pt}%
\definecolor{currentstroke}{rgb}{0.500000,0.500000,0.500000}%
\pgfsetstrokecolor{currentstroke}%
\pgfsetstrokeopacity{0.300000}%
\pgfsetdash{}{0pt}%
\pgfpathmoveto{\pgfqpoint{0.000000in}{0.000000in}}%
\pgfpathlineto{\pgfqpoint{0.000000in}{0.000000in}}%
\pgfpathclose%
\pgfusepath{stroke,fill}%
\end{pgfscope}%
\begin{pgfscope}%
\pgfpathrectangle{\pgfqpoint{0.647939in}{0.492442in}}{\pgfqpoint{3.079299in}{3.079299in}}%
\pgfusepath{clip}%
\pgfsetroundcap%
\pgfsetroundjoin%
\pgfsetlinewidth{0.301125pt}%
\definecolor{currentstroke}{rgb}{0.500000,0.500000,0.500000}%
\pgfsetstrokecolor{currentstroke}%
\pgfsetstrokeopacity{0.300000}%
\pgfsetdash{}{0pt}%
\pgfpathmoveto{\pgfqpoint{1.253255in}{1.470474in}}%
\pgfusepath{stroke}%
\end{pgfscope}%
\begin{pgfscope}%
\pgfpathrectangle{\pgfqpoint{0.647939in}{0.492442in}}{\pgfqpoint{3.079299in}{3.079299in}}%
\pgfusepath{clip}%
\pgfsetroundcap%
\pgfsetroundjoin%
\definecolor{currentfill}{rgb}{0.500000,0.500000,0.500000}%
\pgfsetfillcolor{currentfill}%
\pgfsetfillopacity{0.300000}%
\pgfsetlinewidth{0.301125pt}%
\definecolor{currentstroke}{rgb}{0.500000,0.500000,0.500000}%
\pgfsetstrokecolor{currentstroke}%
\pgfsetstrokeopacity{0.300000}%
\pgfsetdash{}{0pt}%
\pgfpathmoveto{\pgfqpoint{0.000000in}{0.000000in}}%
\pgfpathlineto{\pgfqpoint{0.000000in}{0.000000in}}%
\pgfpathclose%
\pgfusepath{stroke,fill}%
\end{pgfscope}%
\begin{pgfscope}%
\pgfpathrectangle{\pgfqpoint{0.647939in}{0.492442in}}{\pgfqpoint{3.079299in}{3.079299in}}%
\pgfusepath{clip}%
\pgfsetroundcap%
\pgfsetroundjoin%
\pgfsetlinewidth{0.301125pt}%
\definecolor{currentstroke}{rgb}{0.500000,0.500000,0.500000}%
\pgfsetstrokecolor{currentstroke}%
\pgfsetstrokeopacity{0.300000}%
\pgfsetdash{}{0pt}%
\pgfpathmoveto{\pgfqpoint{1.067827in}{1.313712in}}%
\pgfusepath{stroke}%
\end{pgfscope}%
\begin{pgfscope}%
\pgfpathrectangle{\pgfqpoint{0.647939in}{0.492442in}}{\pgfqpoint{3.079299in}{3.079299in}}%
\pgfusepath{clip}%
\pgfsetroundcap%
\pgfsetroundjoin%
\definecolor{currentfill}{rgb}{0.500000,0.500000,0.500000}%
\pgfsetfillcolor{currentfill}%
\pgfsetfillopacity{0.300000}%
\pgfsetlinewidth{0.301125pt}%
\definecolor{currentstroke}{rgb}{0.500000,0.500000,0.500000}%
\pgfsetstrokecolor{currentstroke}%
\pgfsetstrokeopacity{0.300000}%
\pgfsetdash{}{0pt}%
\pgfpathmoveto{\pgfqpoint{0.000000in}{0.000000in}}%
\pgfpathlineto{\pgfqpoint{0.000000in}{0.000000in}}%
\pgfpathclose%
\pgfusepath{stroke,fill}%
\end{pgfscope}%
\begin{pgfscope}%
\pgfpathrectangle{\pgfqpoint{0.647939in}{0.492442in}}{\pgfqpoint{3.079299in}{3.079299in}}%
\pgfusepath{clip}%
\pgfsetroundcap%
\pgfsetroundjoin%
\pgfsetlinewidth{0.301125pt}%
\definecolor{currentstroke}{rgb}{0.500000,0.500000,0.500000}%
\pgfsetstrokecolor{currentstroke}%
\pgfsetstrokeopacity{0.300000}%
\pgfsetdash{}{0pt}%
\pgfpathmoveto{\pgfqpoint{1.003873in}{1.220007in}}%
\pgfusepath{stroke}%
\end{pgfscope}%
\begin{pgfscope}%
\pgfpathrectangle{\pgfqpoint{0.647939in}{0.492442in}}{\pgfqpoint{3.079299in}{3.079299in}}%
\pgfusepath{clip}%
\pgfsetroundcap%
\pgfsetroundjoin%
\definecolor{currentfill}{rgb}{0.500000,0.500000,0.500000}%
\pgfsetfillcolor{currentfill}%
\pgfsetfillopacity{0.300000}%
\pgfsetlinewidth{0.301125pt}%
\definecolor{currentstroke}{rgb}{0.500000,0.500000,0.500000}%
\pgfsetstrokecolor{currentstroke}%
\pgfsetstrokeopacity{0.300000}%
\pgfsetdash{}{0pt}%
\pgfpathmoveto{\pgfqpoint{0.000000in}{0.000000in}}%
\pgfpathlineto{\pgfqpoint{0.000000in}{0.000000in}}%
\pgfpathclose%
\pgfusepath{stroke,fill}%
\end{pgfscope}%
\begin{pgfscope}%
\pgfpathrectangle{\pgfqpoint{0.647939in}{0.492442in}}{\pgfqpoint{3.079299in}{3.079299in}}%
\pgfusepath{clip}%
\pgfsetroundcap%
\pgfsetroundjoin%
\pgfsetlinewidth{0.301125pt}%
\definecolor{currentstroke}{rgb}{0.500000,0.500000,0.500000}%
\pgfsetstrokecolor{currentstroke}%
\pgfsetstrokeopacity{0.300000}%
\pgfsetdash{}{0pt}%
\pgfpathmoveto{\pgfqpoint{0.874462in}{1.106114in}}%
\pgfusepath{stroke}%
\end{pgfscope}%
\begin{pgfscope}%
\pgfpathrectangle{\pgfqpoint{0.647939in}{0.492442in}}{\pgfqpoint{3.079299in}{3.079299in}}%
\pgfusepath{clip}%
\pgfsetroundcap%
\pgfsetroundjoin%
\definecolor{currentfill}{rgb}{0.500000,0.500000,0.500000}%
\pgfsetfillcolor{currentfill}%
\pgfsetfillopacity{0.300000}%
\pgfsetlinewidth{0.301125pt}%
\definecolor{currentstroke}{rgb}{0.500000,0.500000,0.500000}%
\pgfsetstrokecolor{currentstroke}%
\pgfsetstrokeopacity{0.300000}%
\pgfsetdash{}{0pt}%
\pgfpathmoveto{\pgfqpoint{0.000000in}{0.000000in}}%
\pgfpathlineto{\pgfqpoint{0.000000in}{0.000000in}}%
\pgfpathclose%
\pgfusepath{stroke,fill}%
\end{pgfscope}%
\begin{pgfscope}%
\pgfpathrectangle{\pgfqpoint{0.647939in}{0.492442in}}{\pgfqpoint{3.079299in}{3.079299in}}%
\pgfusepath{clip}%
\pgfsetroundcap%
\pgfsetroundjoin%
\pgfsetlinewidth{0.301125pt}%
\definecolor{currentstroke}{rgb}{0.500000,0.500000,0.500000}%
\pgfsetstrokecolor{currentstroke}%
\pgfsetstrokeopacity{0.300000}%
\pgfsetdash{}{0pt}%
\pgfpathmoveto{\pgfqpoint{1.298468in}{1.254279in}}%
\pgfusepath{stroke}%
\end{pgfscope}%
\begin{pgfscope}%
\pgfpathrectangle{\pgfqpoint{0.647939in}{0.492442in}}{\pgfqpoint{3.079299in}{3.079299in}}%
\pgfusepath{clip}%
\pgfsetroundcap%
\pgfsetroundjoin%
\definecolor{currentfill}{rgb}{0.500000,0.500000,0.500000}%
\pgfsetfillcolor{currentfill}%
\pgfsetfillopacity{0.300000}%
\pgfsetlinewidth{0.301125pt}%
\definecolor{currentstroke}{rgb}{0.500000,0.500000,0.500000}%
\pgfsetstrokecolor{currentstroke}%
\pgfsetstrokeopacity{0.300000}%
\pgfsetdash{}{0pt}%
\pgfpathmoveto{\pgfqpoint{0.000000in}{0.000000in}}%
\pgfpathlineto{\pgfqpoint{0.000000in}{0.000000in}}%
\pgfpathclose%
\pgfusepath{stroke,fill}%
\end{pgfscope}%
\begin{pgfscope}%
\pgfpathrectangle{\pgfqpoint{0.647939in}{0.492442in}}{\pgfqpoint{3.079299in}{3.079299in}}%
\pgfusepath{clip}%
\pgfsetroundcap%
\pgfsetroundjoin%
\pgfsetlinewidth{0.301125pt}%
\definecolor{currentstroke}{rgb}{0.500000,0.500000,0.500000}%
\pgfsetstrokecolor{currentstroke}%
\pgfsetstrokeopacity{0.300000}%
\pgfsetdash{}{0pt}%
\pgfpathmoveto{\pgfqpoint{1.001612in}{1.017181in}}%
\pgfusepath{stroke}%
\end{pgfscope}%
\begin{pgfscope}%
\pgfpathrectangle{\pgfqpoint{0.647939in}{0.492442in}}{\pgfqpoint{3.079299in}{3.079299in}}%
\pgfusepath{clip}%
\pgfsetroundcap%
\pgfsetroundjoin%
\definecolor{currentfill}{rgb}{0.500000,0.500000,0.500000}%
\pgfsetfillcolor{currentfill}%
\pgfsetfillopacity{0.300000}%
\pgfsetlinewidth{0.301125pt}%
\definecolor{currentstroke}{rgb}{0.500000,0.500000,0.500000}%
\pgfsetstrokecolor{currentstroke}%
\pgfsetstrokeopacity{0.300000}%
\pgfsetdash{}{0pt}%
\pgfpathmoveto{\pgfqpoint{0.000000in}{0.000000in}}%
\pgfpathlineto{\pgfqpoint{0.000000in}{0.000000in}}%
\pgfpathclose%
\pgfusepath{stroke,fill}%
\end{pgfscope}%
\begin{pgfscope}%
\pgfpathrectangle{\pgfqpoint{0.647939in}{0.492442in}}{\pgfqpoint{3.079299in}{3.079299in}}%
\pgfusepath{clip}%
\pgfsetroundcap%
\pgfsetroundjoin%
\pgfsetlinewidth{0.301125pt}%
\definecolor{currentstroke}{rgb}{0.500000,0.500000,0.500000}%
\pgfsetstrokecolor{currentstroke}%
\pgfsetstrokeopacity{0.300000}%
\pgfsetdash{}{0pt}%
\pgfpathmoveto{\pgfqpoint{1.000743in}{0.949785in}}%
\pgfusepath{stroke}%
\end{pgfscope}%
\begin{pgfscope}%
\pgfpathrectangle{\pgfqpoint{0.647939in}{0.492442in}}{\pgfqpoint{3.079299in}{3.079299in}}%
\pgfusepath{clip}%
\pgfsetroundcap%
\pgfsetroundjoin%
\definecolor{currentfill}{rgb}{0.500000,0.500000,0.500000}%
\pgfsetfillcolor{currentfill}%
\pgfsetfillopacity{0.300000}%
\pgfsetlinewidth{0.301125pt}%
\definecolor{currentstroke}{rgb}{0.500000,0.500000,0.500000}%
\pgfsetstrokecolor{currentstroke}%
\pgfsetstrokeopacity{0.300000}%
\pgfsetdash{}{0pt}%
\pgfpathmoveto{\pgfqpoint{0.000000in}{0.000000in}}%
\pgfpathlineto{\pgfqpoint{0.000000in}{0.000000in}}%
\pgfpathclose%
\pgfusepath{stroke,fill}%
\end{pgfscope}%
\begin{pgfscope}%
\pgfpathrectangle{\pgfqpoint{0.647939in}{0.492442in}}{\pgfqpoint{3.079299in}{3.079299in}}%
\pgfusepath{clip}%
\pgfsetroundcap%
\pgfsetroundjoin%
\pgfsetlinewidth{0.301125pt}%
\definecolor{currentstroke}{rgb}{0.500000,0.500000,0.500000}%
\pgfsetstrokecolor{currentstroke}%
\pgfsetstrokeopacity{0.300000}%
\pgfsetdash{}{0pt}%
\pgfpathmoveto{\pgfqpoint{0.937189in}{0.855059in}}%
\pgfusepath{stroke}%
\end{pgfscope}%
\begin{pgfscope}%
\pgfpathrectangle{\pgfqpoint{0.647939in}{0.492442in}}{\pgfqpoint{3.079299in}{3.079299in}}%
\pgfusepath{clip}%
\pgfsetroundcap%
\pgfsetroundjoin%
\definecolor{currentfill}{rgb}{0.500000,0.500000,0.500000}%
\pgfsetfillcolor{currentfill}%
\pgfsetfillopacity{0.300000}%
\pgfsetlinewidth{0.301125pt}%
\definecolor{currentstroke}{rgb}{0.500000,0.500000,0.500000}%
\pgfsetstrokecolor{currentstroke}%
\pgfsetstrokeopacity{0.300000}%
\pgfsetdash{}{0pt}%
\pgfpathmoveto{\pgfqpoint{0.000000in}{0.000000in}}%
\pgfpathlineto{\pgfqpoint{0.000000in}{0.000000in}}%
\pgfpathclose%
\pgfusepath{stroke,fill}%
\end{pgfscope}%
\begin{pgfscope}%
\pgfpathrectangle{\pgfqpoint{0.647939in}{0.492442in}}{\pgfqpoint{3.079299in}{3.079299in}}%
\pgfusepath{clip}%
\pgfsetroundcap%
\pgfsetroundjoin%
\pgfsetlinewidth{0.301125pt}%
\definecolor{currentstroke}{rgb}{0.500000,0.500000,0.500000}%
\pgfsetstrokecolor{currentstroke}%
\pgfsetstrokeopacity{0.300000}%
\pgfsetdash{}{0pt}%
\pgfpathmoveto{\pgfqpoint{0.936551in}{0.787086in}}%
\pgfusepath{stroke}%
\end{pgfscope}%
\begin{pgfscope}%
\pgfpathrectangle{\pgfqpoint{0.647939in}{0.492442in}}{\pgfqpoint{3.079299in}{3.079299in}}%
\pgfusepath{clip}%
\pgfsetroundcap%
\pgfsetroundjoin%
\definecolor{currentfill}{rgb}{0.500000,0.500000,0.500000}%
\pgfsetfillcolor{currentfill}%
\pgfsetfillopacity{0.300000}%
\pgfsetlinewidth{0.301125pt}%
\definecolor{currentstroke}{rgb}{0.500000,0.500000,0.500000}%
\pgfsetstrokecolor{currentstroke}%
\pgfsetstrokeopacity{0.300000}%
\pgfsetdash{}{0pt}%
\pgfpathmoveto{\pgfqpoint{0.000000in}{0.000000in}}%
\pgfpathlineto{\pgfqpoint{0.000000in}{0.000000in}}%
\pgfpathclose%
\pgfusepath{stroke,fill}%
\end{pgfscope}%
\begin{pgfscope}%
\pgfpathrectangle{\pgfqpoint{0.647939in}{0.492442in}}{\pgfqpoint{3.079299in}{3.079299in}}%
\pgfusepath{clip}%
\pgfsetroundcap%
\pgfsetroundjoin%
\pgfsetlinewidth{0.301125pt}%
\definecolor{currentstroke}{rgb}{0.500000,0.500000,0.500000}%
\pgfsetstrokecolor{currentstroke}%
\pgfsetstrokeopacity{0.300000}%
\pgfsetdash{}{0pt}%
\pgfpathmoveto{\pgfqpoint{0.872384in}{0.693828in}}%
\pgfusepath{stroke}%
\end{pgfscope}%
\begin{pgfscope}%
\pgfpathrectangle{\pgfqpoint{0.647939in}{0.492442in}}{\pgfqpoint{3.079299in}{3.079299in}}%
\pgfusepath{clip}%
\pgfsetroundcap%
\pgfsetroundjoin%
\definecolor{currentfill}{rgb}{0.500000,0.500000,0.500000}%
\pgfsetfillcolor{currentfill}%
\pgfsetfillopacity{0.300000}%
\pgfsetlinewidth{0.301125pt}%
\definecolor{currentstroke}{rgb}{0.500000,0.500000,0.500000}%
\pgfsetstrokecolor{currentstroke}%
\pgfsetstrokeopacity{0.300000}%
\pgfsetdash{}{0pt}%
\pgfpathmoveto{\pgfqpoint{0.000000in}{0.000000in}}%
\pgfpathlineto{\pgfqpoint{0.000000in}{0.000000in}}%
\pgfpathclose%
\pgfusepath{stroke,fill}%
\end{pgfscope}%
\begin{pgfscope}%
\pgfpathrectangle{\pgfqpoint{0.647939in}{0.492442in}}{\pgfqpoint{3.079299in}{3.079299in}}%
\pgfusepath{clip}%
\pgfsetroundcap%
\pgfsetroundjoin%
\pgfsetlinewidth{0.301125pt}%
\definecolor{currentstroke}{rgb}{0.500000,0.500000,0.500000}%
\pgfsetstrokecolor{currentstroke}%
\pgfsetstrokeopacity{0.300000}%
\pgfsetdash{}{0pt}%
\pgfpathmoveto{\pgfqpoint{3.506856in}{1.454986in}}%
\pgfusepath{stroke}%
\end{pgfscope}%
\begin{pgfscope}%
\pgfpathrectangle{\pgfqpoint{0.647939in}{0.492442in}}{\pgfqpoint{3.079299in}{3.079299in}}%
\pgfusepath{clip}%
\pgfsetroundcap%
\pgfsetroundjoin%
\definecolor{currentfill}{rgb}{0.500000,0.500000,0.500000}%
\pgfsetfillcolor{currentfill}%
\pgfsetfillopacity{0.300000}%
\pgfsetlinewidth{0.301125pt}%
\definecolor{currentstroke}{rgb}{0.500000,0.500000,0.500000}%
\pgfsetstrokecolor{currentstroke}%
\pgfsetstrokeopacity{0.300000}%
\pgfsetdash{}{0pt}%
\pgfpathmoveto{\pgfqpoint{0.000000in}{0.000000in}}%
\pgfpathlineto{\pgfqpoint{0.000000in}{0.000000in}}%
\pgfpathclose%
\pgfusepath{stroke,fill}%
\end{pgfscope}%
\begin{pgfscope}%
\pgfpathrectangle{\pgfqpoint{0.647939in}{0.492442in}}{\pgfqpoint{3.079299in}{3.079299in}}%
\pgfusepath{clip}%
\pgfsetroundcap%
\pgfsetroundjoin%
\pgfsetlinewidth{0.301125pt}%
\definecolor{currentstroke}{rgb}{0.500000,0.500000,0.500000}%
\pgfsetstrokecolor{currentstroke}%
\pgfsetstrokeopacity{0.300000}%
\pgfsetdash{}{0pt}%
\pgfpathmoveto{\pgfqpoint{1.721060in}{0.647993in}}%
\pgfusepath{stroke}%
\end{pgfscope}%
\begin{pgfscope}%
\pgfpathrectangle{\pgfqpoint{0.647939in}{0.492442in}}{\pgfqpoint{3.079299in}{3.079299in}}%
\pgfusepath{clip}%
\pgfsetroundcap%
\pgfsetroundjoin%
\definecolor{currentfill}{rgb}{0.500000,0.500000,0.500000}%
\pgfsetfillcolor{currentfill}%
\pgfsetfillopacity{0.300000}%
\pgfsetlinewidth{0.301125pt}%
\definecolor{currentstroke}{rgb}{0.500000,0.500000,0.500000}%
\pgfsetstrokecolor{currentstroke}%
\pgfsetstrokeopacity{0.300000}%
\pgfsetdash{}{0pt}%
\pgfpathmoveto{\pgfqpoint{0.000000in}{0.000000in}}%
\pgfpathlineto{\pgfqpoint{0.000000in}{0.000000in}}%
\pgfpathclose%
\pgfusepath{stroke,fill}%
\end{pgfscope}%
\begin{pgfscope}%
\pgfpathrectangle{\pgfqpoint{0.647939in}{0.492442in}}{\pgfqpoint{3.079299in}{3.079299in}}%
\pgfusepath{clip}%
\pgfsetroundcap%
\pgfsetroundjoin%
\pgfsetlinewidth{0.301125pt}%
\definecolor{currentstroke}{rgb}{0.500000,0.500000,0.500000}%
\pgfsetstrokecolor{currentstroke}%
\pgfsetstrokeopacity{0.300000}%
\pgfsetdash{}{0pt}%
\pgfpathmoveto{\pgfqpoint{3.342761in}{1.766392in}}%
\pgfusepath{stroke}%
\end{pgfscope}%
\begin{pgfscope}%
\pgfpathrectangle{\pgfqpoint{0.647939in}{0.492442in}}{\pgfqpoint{3.079299in}{3.079299in}}%
\pgfusepath{clip}%
\pgfsetroundcap%
\pgfsetroundjoin%
\definecolor{currentfill}{rgb}{0.500000,0.500000,0.500000}%
\pgfsetfillcolor{currentfill}%
\pgfsetfillopacity{0.300000}%
\pgfsetlinewidth{0.301125pt}%
\definecolor{currentstroke}{rgb}{0.500000,0.500000,0.500000}%
\pgfsetstrokecolor{currentstroke}%
\pgfsetstrokeopacity{0.300000}%
\pgfsetdash{}{0pt}%
\pgfpathmoveto{\pgfqpoint{0.000000in}{0.000000in}}%
\pgfpathlineto{\pgfqpoint{0.000000in}{0.000000in}}%
\pgfpathclose%
\pgfusepath{stroke,fill}%
\end{pgfscope}%
\begin{pgfscope}%
\pgfpathrectangle{\pgfqpoint{0.647939in}{0.492442in}}{\pgfqpoint{3.079299in}{3.079299in}}%
\pgfusepath{clip}%
\pgfsetroundcap%
\pgfsetroundjoin%
\pgfsetlinewidth{0.301125pt}%
\definecolor{currentstroke}{rgb}{0.500000,0.500000,0.500000}%
\pgfsetstrokecolor{currentstroke}%
\pgfsetstrokeopacity{0.300000}%
\pgfsetdash{}{0pt}%
\pgfpathmoveto{\pgfqpoint{3.401561in}{1.869082in}}%
\pgfusepath{stroke}%
\end{pgfscope}%
\begin{pgfscope}%
\pgfpathrectangle{\pgfqpoint{0.647939in}{0.492442in}}{\pgfqpoint{3.079299in}{3.079299in}}%
\pgfusepath{clip}%
\pgfsetroundcap%
\pgfsetroundjoin%
\definecolor{currentfill}{rgb}{0.500000,0.500000,0.500000}%
\pgfsetfillcolor{currentfill}%
\pgfsetfillopacity{0.300000}%
\pgfsetlinewidth{0.301125pt}%
\definecolor{currentstroke}{rgb}{0.500000,0.500000,0.500000}%
\pgfsetstrokecolor{currentstroke}%
\pgfsetstrokeopacity{0.300000}%
\pgfsetdash{}{0pt}%
\pgfpathmoveto{\pgfqpoint{0.000000in}{0.000000in}}%
\pgfpathlineto{\pgfqpoint{0.000000in}{0.000000in}}%
\pgfpathclose%
\pgfusepath{stroke,fill}%
\end{pgfscope}%
\begin{pgfscope}%
\pgfpathrectangle{\pgfqpoint{0.647939in}{0.492442in}}{\pgfqpoint{3.079299in}{3.079299in}}%
\pgfusepath{clip}%
\pgfsetroundcap%
\pgfsetroundjoin%
\pgfsetlinewidth{0.301125pt}%
\definecolor{currentstroke}{rgb}{0.500000,0.500000,0.500000}%
\pgfsetstrokecolor{currentstroke}%
\pgfsetstrokeopacity{0.300000}%
\pgfsetdash{}{0pt}%
\pgfpathmoveto{\pgfqpoint{3.310302in}{1.423494in}}%
\pgfusepath{stroke}%
\end{pgfscope}%
\begin{pgfscope}%
\pgfpathrectangle{\pgfqpoint{0.647939in}{0.492442in}}{\pgfqpoint{3.079299in}{3.079299in}}%
\pgfusepath{clip}%
\pgfsetroundcap%
\pgfsetroundjoin%
\definecolor{currentfill}{rgb}{0.500000,0.500000,0.500000}%
\pgfsetfillcolor{currentfill}%
\pgfsetfillopacity{0.300000}%
\pgfsetlinewidth{0.301125pt}%
\definecolor{currentstroke}{rgb}{0.500000,0.500000,0.500000}%
\pgfsetstrokecolor{currentstroke}%
\pgfsetstrokeopacity{0.300000}%
\pgfsetdash{}{0pt}%
\pgfpathmoveto{\pgfqpoint{0.000000in}{0.000000in}}%
\pgfpathlineto{\pgfqpoint{0.000000in}{0.000000in}}%
\pgfpathclose%
\pgfusepath{stroke,fill}%
\end{pgfscope}%
\begin{pgfscope}%
\pgfpathrectangle{\pgfqpoint{0.647939in}{0.492442in}}{\pgfqpoint{3.079299in}{3.079299in}}%
\pgfusepath{clip}%
\pgfsetroundcap%
\pgfsetroundjoin%
\pgfsetlinewidth{0.301125pt}%
\definecolor{currentstroke}{rgb}{0.500000,0.500000,0.500000}%
\pgfsetstrokecolor{currentstroke}%
\pgfsetstrokeopacity{0.300000}%
\pgfsetdash{}{0pt}%
\pgfpathmoveto{\pgfqpoint{2.153944in}{3.271720in}}%
\pgfusepath{stroke}%
\end{pgfscope}%
\begin{pgfscope}%
\pgfpathrectangle{\pgfqpoint{0.647939in}{0.492442in}}{\pgfqpoint{3.079299in}{3.079299in}}%
\pgfusepath{clip}%
\pgfsetroundcap%
\pgfsetroundjoin%
\definecolor{currentfill}{rgb}{0.500000,0.500000,0.500000}%
\pgfsetfillcolor{currentfill}%
\pgfsetfillopacity{0.300000}%
\pgfsetlinewidth{0.301125pt}%
\definecolor{currentstroke}{rgb}{0.500000,0.500000,0.500000}%
\pgfsetstrokecolor{currentstroke}%
\pgfsetstrokeopacity{0.300000}%
\pgfsetdash{}{0pt}%
\pgfpathmoveto{\pgfqpoint{0.000000in}{0.000000in}}%
\pgfpathlineto{\pgfqpoint{0.000000in}{0.000000in}}%
\pgfpathclose%
\pgfusepath{stroke,fill}%
\end{pgfscope}%
\begin{pgfscope}%
\pgfpathrectangle{\pgfqpoint{0.647939in}{0.492442in}}{\pgfqpoint{3.079299in}{3.079299in}}%
\pgfusepath{clip}%
\pgfsetroundcap%
\pgfsetroundjoin%
\pgfsetlinewidth{0.301125pt}%
\definecolor{currentstroke}{rgb}{0.500000,0.500000,0.500000}%
\pgfsetstrokecolor{currentstroke}%
\pgfsetstrokeopacity{0.300000}%
\pgfsetdash{}{0pt}%
\pgfpathmoveto{\pgfqpoint{1.222365in}{3.146244in}}%
\pgfusepath{stroke}%
\end{pgfscope}%
\begin{pgfscope}%
\pgfpathrectangle{\pgfqpoint{0.647939in}{0.492442in}}{\pgfqpoint{3.079299in}{3.079299in}}%
\pgfusepath{clip}%
\pgfsetroundcap%
\pgfsetroundjoin%
\definecolor{currentfill}{rgb}{0.500000,0.500000,0.500000}%
\pgfsetfillcolor{currentfill}%
\pgfsetfillopacity{0.300000}%
\pgfsetlinewidth{0.301125pt}%
\definecolor{currentstroke}{rgb}{0.500000,0.500000,0.500000}%
\pgfsetstrokecolor{currentstroke}%
\pgfsetstrokeopacity{0.300000}%
\pgfsetdash{}{0pt}%
\pgfpathmoveto{\pgfqpoint{0.000000in}{0.000000in}}%
\pgfpathlineto{\pgfqpoint{0.000000in}{0.000000in}}%
\pgfpathclose%
\pgfusepath{stroke,fill}%
\end{pgfscope}%
\begin{pgfscope}%
\pgfpathrectangle{\pgfqpoint{0.647939in}{0.492442in}}{\pgfqpoint{3.079299in}{3.079299in}}%
\pgfusepath{clip}%
\pgfsetroundcap%
\pgfsetroundjoin%
\pgfsetlinewidth{0.301125pt}%
\definecolor{currentstroke}{rgb}{0.500000,0.500000,0.500000}%
\pgfsetstrokecolor{currentstroke}%
\pgfsetstrokeopacity{0.300000}%
\pgfsetdash{}{0pt}%
\pgfpathmoveto{\pgfqpoint{2.810422in}{2.160389in}}%
\pgfusepath{stroke}%
\end{pgfscope}%
\begin{pgfscope}%
\pgfpathrectangle{\pgfqpoint{0.647939in}{0.492442in}}{\pgfqpoint{3.079299in}{3.079299in}}%
\pgfusepath{clip}%
\pgfsetroundcap%
\pgfsetroundjoin%
\definecolor{currentfill}{rgb}{0.500000,0.500000,0.500000}%
\pgfsetfillcolor{currentfill}%
\pgfsetfillopacity{0.300000}%
\pgfsetlinewidth{0.301125pt}%
\definecolor{currentstroke}{rgb}{0.500000,0.500000,0.500000}%
\pgfsetstrokecolor{currentstroke}%
\pgfsetstrokeopacity{0.300000}%
\pgfsetdash{}{0pt}%
\pgfpathmoveto{\pgfqpoint{0.000000in}{0.000000in}}%
\pgfpathlineto{\pgfqpoint{0.000000in}{0.000000in}}%
\pgfpathclose%
\pgfusepath{stroke,fill}%
\end{pgfscope}%
\begin{pgfscope}%
\pgfpathrectangle{\pgfqpoint{0.647939in}{0.492442in}}{\pgfqpoint{3.079299in}{3.079299in}}%
\pgfusepath{clip}%
\pgfsetroundcap%
\pgfsetroundjoin%
\pgfsetlinewidth{0.301125pt}%
\definecolor{currentstroke}{rgb}{0.500000,0.500000,0.500000}%
\pgfsetstrokecolor{currentstroke}%
\pgfsetstrokeopacity{0.300000}%
\pgfsetdash{}{0pt}%
\pgfpathmoveto{\pgfqpoint{3.115141in}{2.447918in}}%
\pgfusepath{stroke}%
\end{pgfscope}%
\begin{pgfscope}%
\pgfpathrectangle{\pgfqpoint{0.647939in}{0.492442in}}{\pgfqpoint{3.079299in}{3.079299in}}%
\pgfusepath{clip}%
\pgfsetroundcap%
\pgfsetroundjoin%
\definecolor{currentfill}{rgb}{0.500000,0.500000,0.500000}%
\pgfsetfillcolor{currentfill}%
\pgfsetfillopacity{0.300000}%
\pgfsetlinewidth{0.301125pt}%
\definecolor{currentstroke}{rgb}{0.500000,0.500000,0.500000}%
\pgfsetstrokecolor{currentstroke}%
\pgfsetstrokeopacity{0.300000}%
\pgfsetdash{}{0pt}%
\pgfpathmoveto{\pgfqpoint{0.000000in}{0.000000in}}%
\pgfpathlineto{\pgfqpoint{0.000000in}{0.000000in}}%
\pgfpathclose%
\pgfusepath{stroke,fill}%
\end{pgfscope}%
\begin{pgfscope}%
\pgfpathrectangle{\pgfqpoint{0.647939in}{0.492442in}}{\pgfqpoint{3.079299in}{3.079299in}}%
\pgfusepath{clip}%
\pgfsetroundcap%
\pgfsetroundjoin%
\pgfsetlinewidth{0.301125pt}%
\definecolor{currentstroke}{rgb}{0.500000,0.500000,0.500000}%
\pgfsetstrokecolor{currentstroke}%
\pgfsetstrokeopacity{0.300000}%
\pgfsetdash{}{0pt}%
\pgfpathmoveto{\pgfqpoint{2.696029in}{1.921516in}}%
\pgfusepath{stroke}%
\end{pgfscope}%
\begin{pgfscope}%
\pgfpathrectangle{\pgfqpoint{0.647939in}{0.492442in}}{\pgfqpoint{3.079299in}{3.079299in}}%
\pgfusepath{clip}%
\pgfsetroundcap%
\pgfsetroundjoin%
\definecolor{currentfill}{rgb}{0.500000,0.500000,0.500000}%
\pgfsetfillcolor{currentfill}%
\pgfsetfillopacity{0.300000}%
\pgfsetlinewidth{0.301125pt}%
\definecolor{currentstroke}{rgb}{0.500000,0.500000,0.500000}%
\pgfsetstrokecolor{currentstroke}%
\pgfsetstrokeopacity{0.300000}%
\pgfsetdash{}{0pt}%
\pgfpathmoveto{\pgfqpoint{0.000000in}{0.000000in}}%
\pgfpathlineto{\pgfqpoint{0.000000in}{0.000000in}}%
\pgfpathclose%
\pgfusepath{stroke,fill}%
\end{pgfscope}%
\begin{pgfscope}%
\pgfpathrectangle{\pgfqpoint{0.647939in}{0.492442in}}{\pgfqpoint{3.079299in}{3.079299in}}%
\pgfusepath{clip}%
\pgfsetroundcap%
\pgfsetroundjoin%
\pgfsetlinewidth{0.301125pt}%
\definecolor{currentstroke}{rgb}{0.500000,0.500000,0.500000}%
\pgfsetstrokecolor{currentstroke}%
\pgfsetstrokeopacity{0.300000}%
\pgfsetdash{}{0pt}%
\pgfpathmoveto{\pgfqpoint{2.836086in}{2.013653in}}%
\pgfusepath{stroke}%
\end{pgfscope}%
\begin{pgfscope}%
\pgfpathrectangle{\pgfqpoint{0.647939in}{0.492442in}}{\pgfqpoint{3.079299in}{3.079299in}}%
\pgfusepath{clip}%
\pgfsetroundcap%
\pgfsetroundjoin%
\definecolor{currentfill}{rgb}{0.500000,0.500000,0.500000}%
\pgfsetfillcolor{currentfill}%
\pgfsetfillopacity{0.300000}%
\pgfsetlinewidth{0.301125pt}%
\definecolor{currentstroke}{rgb}{0.500000,0.500000,0.500000}%
\pgfsetstrokecolor{currentstroke}%
\pgfsetstrokeopacity{0.300000}%
\pgfsetdash{}{0pt}%
\pgfpathmoveto{\pgfqpoint{0.000000in}{0.000000in}}%
\pgfpathlineto{\pgfqpoint{0.000000in}{0.000000in}}%
\pgfpathclose%
\pgfusepath{stroke,fill}%
\end{pgfscope}%
\begin{pgfscope}%
\pgfpathrectangle{\pgfqpoint{0.647939in}{0.492442in}}{\pgfqpoint{3.079299in}{3.079299in}}%
\pgfusepath{clip}%
\pgfsetroundcap%
\pgfsetroundjoin%
\pgfsetlinewidth{0.301125pt}%
\definecolor{currentstroke}{rgb}{0.500000,0.500000,0.500000}%
\pgfsetstrokecolor{currentstroke}%
\pgfsetstrokeopacity{0.300000}%
\pgfsetdash{}{0pt}%
\pgfpathmoveto{\pgfqpoint{2.945639in}{2.365840in}}%
\pgfusepath{stroke}%
\end{pgfscope}%
\begin{pgfscope}%
\pgfpathrectangle{\pgfqpoint{0.647939in}{0.492442in}}{\pgfqpoint{3.079299in}{3.079299in}}%
\pgfusepath{clip}%
\pgfsetroundcap%
\pgfsetroundjoin%
\definecolor{currentfill}{rgb}{0.500000,0.500000,0.500000}%
\pgfsetfillcolor{currentfill}%
\pgfsetfillopacity{0.300000}%
\pgfsetlinewidth{0.301125pt}%
\definecolor{currentstroke}{rgb}{0.500000,0.500000,0.500000}%
\pgfsetstrokecolor{currentstroke}%
\pgfsetstrokeopacity{0.300000}%
\pgfsetdash{}{0pt}%
\pgfpathmoveto{\pgfqpoint{0.000000in}{0.000000in}}%
\pgfpathlineto{\pgfqpoint{0.000000in}{0.000000in}}%
\pgfpathclose%
\pgfusepath{stroke,fill}%
\end{pgfscope}%
\begin{pgfscope}%
\pgfpathrectangle{\pgfqpoint{0.647939in}{0.492442in}}{\pgfqpoint{3.079299in}{3.079299in}}%
\pgfusepath{clip}%
\pgfsetroundcap%
\pgfsetroundjoin%
\pgfsetlinewidth{0.301125pt}%
\definecolor{currentstroke}{rgb}{0.500000,0.500000,0.500000}%
\pgfsetstrokecolor{currentstroke}%
\pgfsetstrokeopacity{0.300000}%
\pgfsetdash{}{0pt}%
\pgfpathmoveto{\pgfqpoint{2.232893in}{2.873349in}}%
\pgfusepath{stroke}%
\end{pgfscope}%
\begin{pgfscope}%
\pgfpathrectangle{\pgfqpoint{0.647939in}{0.492442in}}{\pgfqpoint{3.079299in}{3.079299in}}%
\pgfusepath{clip}%
\pgfsetroundcap%
\pgfsetroundjoin%
\definecolor{currentfill}{rgb}{0.500000,0.500000,0.500000}%
\pgfsetfillcolor{currentfill}%
\pgfsetfillopacity{0.300000}%
\pgfsetlinewidth{0.301125pt}%
\definecolor{currentstroke}{rgb}{0.500000,0.500000,0.500000}%
\pgfsetstrokecolor{currentstroke}%
\pgfsetstrokeopacity{0.300000}%
\pgfsetdash{}{0pt}%
\pgfpathmoveto{\pgfqpoint{0.000000in}{0.000000in}}%
\pgfpathlineto{\pgfqpoint{0.000000in}{0.000000in}}%
\pgfpathclose%
\pgfusepath{stroke,fill}%
\end{pgfscope}%
\begin{pgfscope}%
\pgfpathrectangle{\pgfqpoint{0.647939in}{0.492442in}}{\pgfqpoint{3.079299in}{3.079299in}}%
\pgfusepath{clip}%
\pgfsetroundcap%
\pgfsetroundjoin%
\pgfsetlinewidth{0.301125pt}%
\definecolor{currentstroke}{rgb}{0.500000,0.500000,0.500000}%
\pgfsetstrokecolor{currentstroke}%
\pgfsetstrokeopacity{0.300000}%
\pgfsetdash{}{0pt}%
\pgfpathmoveto{\pgfqpoint{1.370145in}{2.688873in}}%
\pgfusepath{stroke}%
\end{pgfscope}%
\begin{pgfscope}%
\pgfpathrectangle{\pgfqpoint{0.647939in}{0.492442in}}{\pgfqpoint{3.079299in}{3.079299in}}%
\pgfusepath{clip}%
\pgfsetroundcap%
\pgfsetroundjoin%
\definecolor{currentfill}{rgb}{0.500000,0.500000,0.500000}%
\pgfsetfillcolor{currentfill}%
\pgfsetfillopacity{0.300000}%
\pgfsetlinewidth{0.301125pt}%
\definecolor{currentstroke}{rgb}{0.500000,0.500000,0.500000}%
\pgfsetstrokecolor{currentstroke}%
\pgfsetstrokeopacity{0.300000}%
\pgfsetdash{}{0pt}%
\pgfpathmoveto{\pgfqpoint{0.000000in}{0.000000in}}%
\pgfpathlineto{\pgfqpoint{0.000000in}{0.000000in}}%
\pgfpathclose%
\pgfusepath{stroke,fill}%
\end{pgfscope}%
\begin{pgfscope}%
\pgfpathrectangle{\pgfqpoint{0.647939in}{0.492442in}}{\pgfqpoint{3.079299in}{3.079299in}}%
\pgfusepath{clip}%
\pgfsetroundcap%
\pgfsetroundjoin%
\pgfsetlinewidth{0.301125pt}%
\definecolor{currentstroke}{rgb}{0.500000,0.500000,0.500000}%
\pgfsetstrokecolor{currentstroke}%
\pgfsetstrokeopacity{0.300000}%
\pgfsetdash{}{0pt}%
\pgfpathmoveto{\pgfqpoint{2.374071in}{2.728853in}}%
\pgfusepath{stroke}%
\end{pgfscope}%
\begin{pgfscope}%
\pgfpathrectangle{\pgfqpoint{0.647939in}{0.492442in}}{\pgfqpoint{3.079299in}{3.079299in}}%
\pgfusepath{clip}%
\pgfsetroundcap%
\pgfsetroundjoin%
\definecolor{currentfill}{rgb}{0.500000,0.500000,0.500000}%
\pgfsetfillcolor{currentfill}%
\pgfsetfillopacity{0.300000}%
\pgfsetlinewidth{0.301125pt}%
\definecolor{currentstroke}{rgb}{0.500000,0.500000,0.500000}%
\pgfsetstrokecolor{currentstroke}%
\pgfsetstrokeopacity{0.300000}%
\pgfsetdash{}{0pt}%
\pgfpathmoveto{\pgfqpoint{0.000000in}{0.000000in}}%
\pgfpathlineto{\pgfqpoint{0.000000in}{0.000000in}}%
\pgfpathclose%
\pgfusepath{stroke,fill}%
\end{pgfscope}%
\begin{pgfscope}%
\pgfpathrectangle{\pgfqpoint{0.647939in}{0.492442in}}{\pgfqpoint{3.079299in}{3.079299in}}%
\pgfusepath{clip}%
\pgfsetroundcap%
\pgfsetroundjoin%
\pgfsetlinewidth{0.301125pt}%
\definecolor{currentstroke}{rgb}{0.500000,0.500000,0.500000}%
\pgfsetstrokecolor{currentstroke}%
\pgfsetstrokeopacity{0.300000}%
\pgfsetdash{}{0pt}%
\pgfpathmoveto{\pgfqpoint{1.841601in}{2.632225in}}%
\pgfusepath{stroke}%
\end{pgfscope}%
\begin{pgfscope}%
\pgfpathrectangle{\pgfqpoint{0.647939in}{0.492442in}}{\pgfqpoint{3.079299in}{3.079299in}}%
\pgfusepath{clip}%
\pgfsetroundcap%
\pgfsetroundjoin%
\definecolor{currentfill}{rgb}{0.500000,0.500000,0.500000}%
\pgfsetfillcolor{currentfill}%
\pgfsetfillopacity{0.300000}%
\pgfsetlinewidth{0.301125pt}%
\definecolor{currentstroke}{rgb}{0.500000,0.500000,0.500000}%
\pgfsetstrokecolor{currentstroke}%
\pgfsetstrokeopacity{0.300000}%
\pgfsetdash{}{0pt}%
\pgfpathmoveto{\pgfqpoint{0.000000in}{0.000000in}}%
\pgfpathlineto{\pgfqpoint{0.000000in}{0.000000in}}%
\pgfpathclose%
\pgfusepath{stroke,fill}%
\end{pgfscope}%
\begin{pgfscope}%
\pgfpathrectangle{\pgfqpoint{0.647939in}{0.492442in}}{\pgfqpoint{3.079299in}{3.079299in}}%
\pgfusepath{clip}%
\pgfsetroundcap%
\pgfsetroundjoin%
\pgfsetlinewidth{0.301125pt}%
\definecolor{currentstroke}{rgb}{0.500000,0.500000,0.500000}%
\pgfsetstrokecolor{currentstroke}%
\pgfsetstrokeopacity{0.300000}%
\pgfsetdash{}{0pt}%
\pgfpathmoveto{\pgfqpoint{1.491721in}{1.463703in}}%
\pgfusepath{stroke}%
\end{pgfscope}%
\begin{pgfscope}%
\pgfpathrectangle{\pgfqpoint{0.647939in}{0.492442in}}{\pgfqpoint{3.079299in}{3.079299in}}%
\pgfusepath{clip}%
\pgfsetroundcap%
\pgfsetroundjoin%
\definecolor{currentfill}{rgb}{0.500000,0.500000,0.500000}%
\pgfsetfillcolor{currentfill}%
\pgfsetfillopacity{0.300000}%
\pgfsetlinewidth{0.301125pt}%
\definecolor{currentstroke}{rgb}{0.500000,0.500000,0.500000}%
\pgfsetstrokecolor{currentstroke}%
\pgfsetstrokeopacity{0.300000}%
\pgfsetdash{}{0pt}%
\pgfpathmoveto{\pgfqpoint{0.000000in}{0.000000in}}%
\pgfpathlineto{\pgfqpoint{0.000000in}{0.000000in}}%
\pgfpathclose%
\pgfusepath{stroke,fill}%
\end{pgfscope}%
\begin{pgfscope}%
\pgfpathrectangle{\pgfqpoint{0.647939in}{0.492442in}}{\pgfqpoint{3.079299in}{3.079299in}}%
\pgfusepath{clip}%
\pgfsetroundcap%
\pgfsetroundjoin%
\pgfsetlinewidth{0.301125pt}%
\definecolor{currentstroke}{rgb}{0.500000,0.500000,0.500000}%
\pgfsetstrokecolor{currentstroke}%
\pgfsetstrokeopacity{0.300000}%
\pgfsetdash{}{0pt}%
\pgfpathmoveto{\pgfqpoint{1.506393in}{1.352813in}}%
\pgfusepath{stroke}%
\end{pgfscope}%
\begin{pgfscope}%
\pgfpathrectangle{\pgfqpoint{0.647939in}{0.492442in}}{\pgfqpoint{3.079299in}{3.079299in}}%
\pgfusepath{clip}%
\pgfsetroundcap%
\pgfsetroundjoin%
\definecolor{currentfill}{rgb}{0.500000,0.500000,0.500000}%
\pgfsetfillcolor{currentfill}%
\pgfsetfillopacity{0.300000}%
\pgfsetlinewidth{0.301125pt}%
\definecolor{currentstroke}{rgb}{0.500000,0.500000,0.500000}%
\pgfsetstrokecolor{currentstroke}%
\pgfsetstrokeopacity{0.300000}%
\pgfsetdash{}{0pt}%
\pgfpathmoveto{\pgfqpoint{0.000000in}{0.000000in}}%
\pgfpathlineto{\pgfqpoint{0.000000in}{0.000000in}}%
\pgfpathclose%
\pgfusepath{stroke,fill}%
\end{pgfscope}%
\begin{pgfscope}%
\pgfpathrectangle{\pgfqpoint{0.647939in}{0.492442in}}{\pgfqpoint{3.079299in}{3.079299in}}%
\pgfusepath{clip}%
\pgfsetroundcap%
\pgfsetroundjoin%
\pgfsetlinewidth{0.301125pt}%
\definecolor{currentstroke}{rgb}{0.500000,0.500000,0.500000}%
\pgfsetstrokecolor{currentstroke}%
\pgfsetstrokeopacity{0.300000}%
\pgfsetdash{}{0pt}%
\pgfpathmoveto{\pgfqpoint{2.561687in}{1.911521in}}%
\pgfusepath{stroke}%
\end{pgfscope}%
\begin{pgfscope}%
\pgfpathrectangle{\pgfqpoint{0.647939in}{0.492442in}}{\pgfqpoint{3.079299in}{3.079299in}}%
\pgfusepath{clip}%
\pgfsetroundcap%
\pgfsetroundjoin%
\definecolor{currentfill}{rgb}{0.500000,0.500000,0.500000}%
\pgfsetfillcolor{currentfill}%
\pgfsetfillopacity{0.300000}%
\pgfsetlinewidth{0.301125pt}%
\definecolor{currentstroke}{rgb}{0.500000,0.500000,0.500000}%
\pgfsetstrokecolor{currentstroke}%
\pgfsetstrokeopacity{0.300000}%
\pgfsetdash{}{0pt}%
\pgfpathmoveto{\pgfqpoint{0.000000in}{0.000000in}}%
\pgfpathlineto{\pgfqpoint{0.000000in}{0.000000in}}%
\pgfpathclose%
\pgfusepath{stroke,fill}%
\end{pgfscope}%
\begin{pgfscope}%
\pgfpathrectangle{\pgfqpoint{0.647939in}{0.492442in}}{\pgfqpoint{3.079299in}{3.079299in}}%
\pgfusepath{clip}%
\pgfsetroundcap%
\pgfsetroundjoin%
\pgfsetlinewidth{0.301125pt}%
\definecolor{currentstroke}{rgb}{0.500000,0.500000,0.500000}%
\pgfsetstrokecolor{currentstroke}%
\pgfsetstrokeopacity{0.300000}%
\pgfsetdash{}{0pt}%
\pgfpathmoveto{\pgfqpoint{2.238890in}{2.504507in}}%
\pgfusepath{stroke}%
\end{pgfscope}%
\begin{pgfscope}%
\pgfpathrectangle{\pgfqpoint{0.647939in}{0.492442in}}{\pgfqpoint{3.079299in}{3.079299in}}%
\pgfusepath{clip}%
\pgfsetroundcap%
\pgfsetroundjoin%
\definecolor{currentfill}{rgb}{0.500000,0.500000,0.500000}%
\pgfsetfillcolor{currentfill}%
\pgfsetfillopacity{0.300000}%
\pgfsetlinewidth{0.301125pt}%
\definecolor{currentstroke}{rgb}{0.500000,0.500000,0.500000}%
\pgfsetstrokecolor{currentstroke}%
\pgfsetstrokeopacity{0.300000}%
\pgfsetdash{}{0pt}%
\pgfpathmoveto{\pgfqpoint{0.000000in}{0.000000in}}%
\pgfpathlineto{\pgfqpoint{0.000000in}{0.000000in}}%
\pgfpathclose%
\pgfusepath{stroke,fill}%
\end{pgfscope}%
\begin{pgfscope}%
\pgfpathrectangle{\pgfqpoint{0.647939in}{0.492442in}}{\pgfqpoint{3.079299in}{3.079299in}}%
\pgfusepath{clip}%
\pgfsetroundcap%
\pgfsetroundjoin%
\pgfsetlinewidth{0.301125pt}%
\definecolor{currentstroke}{rgb}{0.500000,0.500000,0.500000}%
\pgfsetstrokecolor{currentstroke}%
\pgfsetstrokeopacity{0.300000}%
\pgfsetdash{}{0pt}%
\pgfpathmoveto{\pgfqpoint{1.471245in}{1.808282in}}%
\pgfusepath{stroke}%
\end{pgfscope}%
\begin{pgfscope}%
\pgfpathrectangle{\pgfqpoint{0.647939in}{0.492442in}}{\pgfqpoint{3.079299in}{3.079299in}}%
\pgfusepath{clip}%
\pgfsetroundcap%
\pgfsetroundjoin%
\definecolor{currentfill}{rgb}{0.500000,0.500000,0.500000}%
\pgfsetfillcolor{currentfill}%
\pgfsetfillopacity{0.300000}%
\pgfsetlinewidth{0.301125pt}%
\definecolor{currentstroke}{rgb}{0.500000,0.500000,0.500000}%
\pgfsetstrokecolor{currentstroke}%
\pgfsetstrokeopacity{0.300000}%
\pgfsetdash{}{0pt}%
\pgfpathmoveto{\pgfqpoint{0.000000in}{0.000000in}}%
\pgfpathlineto{\pgfqpoint{0.000000in}{0.000000in}}%
\pgfpathclose%
\pgfusepath{stroke,fill}%
\end{pgfscope}%
\begin{pgfscope}%
\pgfpathrectangle{\pgfqpoint{0.647939in}{0.492442in}}{\pgfqpoint{3.079299in}{3.079299in}}%
\pgfusepath{clip}%
\pgfsetroundcap%
\pgfsetroundjoin%
\pgfsetlinewidth{0.301125pt}%
\definecolor{currentstroke}{rgb}{0.500000,0.500000,0.500000}%
\pgfsetstrokecolor{currentstroke}%
\pgfsetstrokeopacity{0.300000}%
\pgfsetdash{}{0pt}%
\pgfpathmoveto{\pgfqpoint{2.235113in}{1.708720in}}%
\pgfusepath{stroke}%
\end{pgfscope}%
\begin{pgfscope}%
\pgfpathrectangle{\pgfqpoint{0.647939in}{0.492442in}}{\pgfqpoint{3.079299in}{3.079299in}}%
\pgfusepath{clip}%
\pgfsetroundcap%
\pgfsetroundjoin%
\definecolor{currentfill}{rgb}{0.500000,0.500000,0.500000}%
\pgfsetfillcolor{currentfill}%
\pgfsetfillopacity{0.300000}%
\pgfsetlinewidth{0.301125pt}%
\definecolor{currentstroke}{rgb}{0.500000,0.500000,0.500000}%
\pgfsetstrokecolor{currentstroke}%
\pgfsetstrokeopacity{0.300000}%
\pgfsetdash{}{0pt}%
\pgfpathmoveto{\pgfqpoint{0.000000in}{0.000000in}}%
\pgfpathlineto{\pgfqpoint{0.000000in}{0.000000in}}%
\pgfpathclose%
\pgfusepath{stroke,fill}%
\end{pgfscope}%
\begin{pgfscope}%
\pgfpathrectangle{\pgfqpoint{0.647939in}{0.492442in}}{\pgfqpoint{3.079299in}{3.079299in}}%
\pgfusepath{clip}%
\pgfsetroundcap%
\pgfsetroundjoin%
\pgfsetlinewidth{0.301125pt}%
\definecolor{currentstroke}{rgb}{0.500000,0.500000,0.500000}%
\pgfsetstrokecolor{currentstroke}%
\pgfsetstrokeopacity{0.300000}%
\pgfsetdash{}{0pt}%
\pgfpathmoveto{\pgfqpoint{2.237111in}{1.834028in}}%
\pgfusepath{stroke}%
\end{pgfscope}%
\begin{pgfscope}%
\pgfpathrectangle{\pgfqpoint{0.647939in}{0.492442in}}{\pgfqpoint{3.079299in}{3.079299in}}%
\pgfusepath{clip}%
\pgfsetroundcap%
\pgfsetroundjoin%
\definecolor{currentfill}{rgb}{0.500000,0.500000,0.500000}%
\pgfsetfillcolor{currentfill}%
\pgfsetfillopacity{0.300000}%
\pgfsetlinewidth{0.301125pt}%
\definecolor{currentstroke}{rgb}{0.500000,0.500000,0.500000}%
\pgfsetstrokecolor{currentstroke}%
\pgfsetstrokeopacity{0.300000}%
\pgfsetdash{}{0pt}%
\pgfpathmoveto{\pgfqpoint{0.000000in}{0.000000in}}%
\pgfpathlineto{\pgfqpoint{0.000000in}{0.000000in}}%
\pgfpathclose%
\pgfusepath{stroke,fill}%
\end{pgfscope}%
\begin{pgfscope}%
\pgfpathrectangle{\pgfqpoint{0.647939in}{0.492442in}}{\pgfqpoint{3.079299in}{3.079299in}}%
\pgfusepath{clip}%
\pgfsetroundcap%
\pgfsetroundjoin%
\pgfsetlinewidth{0.301125pt}%
\definecolor{currentstroke}{rgb}{0.500000,0.500000,0.500000}%
\pgfsetstrokecolor{currentstroke}%
\pgfsetstrokeopacity{0.300000}%
\pgfsetdash{}{0pt}%
\pgfpathmoveto{\pgfqpoint{2.394532in}{1.946533in}}%
\pgfusepath{stroke}%
\end{pgfscope}%
\begin{pgfscope}%
\pgfpathrectangle{\pgfqpoint{0.647939in}{0.492442in}}{\pgfqpoint{3.079299in}{3.079299in}}%
\pgfusepath{clip}%
\pgfsetroundcap%
\pgfsetroundjoin%
\definecolor{currentfill}{rgb}{0.500000,0.500000,0.500000}%
\pgfsetfillcolor{currentfill}%
\pgfsetfillopacity{0.300000}%
\pgfsetlinewidth{0.301125pt}%
\definecolor{currentstroke}{rgb}{0.500000,0.500000,0.500000}%
\pgfsetstrokecolor{currentstroke}%
\pgfsetstrokeopacity{0.300000}%
\pgfsetdash{}{0pt}%
\pgfpathmoveto{\pgfqpoint{0.000000in}{0.000000in}}%
\pgfpathlineto{\pgfqpoint{0.000000in}{0.000000in}}%
\pgfpathclose%
\pgfusepath{stroke,fill}%
\end{pgfscope}%
\begin{pgfscope}%
\pgfpathrectangle{\pgfqpoint{0.647939in}{0.492442in}}{\pgfqpoint{3.079299in}{3.079299in}}%
\pgfusepath{clip}%
\pgfsetbuttcap%
\pgfsetroundjoin%
\pgfsetlinewidth{0.301125pt}%
\definecolor{currentstroke}{rgb}{0.500000,0.500000,0.500000}%
\pgfsetstrokecolor{currentstroke}%
\pgfsetstrokeopacity{0.300000}%
\pgfsetdash{}{0pt}%
\pgfpathmoveto{\pgfqpoint{0.647939in}{0.492442in}}%
\pgfpathlineto{\pgfqpoint{0.647939in}{0.492442in}}%
\pgfpathlineto{\pgfqpoint{0.714669in}{0.507514in}}%
\pgfpathlineto{\pgfqpoint{0.780603in}{0.525727in}}%
\pgfpathlineto{\pgfqpoint{0.845465in}{0.547425in}}%
\pgfpathlineto{\pgfqpoint{0.908928in}{0.572903in}}%
\pgfpathlineto{\pgfqpoint{0.970626in}{0.602381in}}%
\pgfpathlineto{\pgfqpoint{1.030178in}{0.635968in}}%
\pgfpathlineto{\pgfqpoint{1.087229in}{0.673646in}}%
\pgfpathlineto{\pgfqpoint{1.141488in}{0.715251in}}%
\pgfpathlineto{\pgfqpoint{1.192775in}{0.760453in}}%
\pgfpathlineto{\pgfqpoint{1.241047in}{0.808866in}}%
\pgfpathlineto{\pgfqpoint{1.286406in}{0.860026in}}%
\pgfpathlineto{\pgfqpoint{1.329080in}{0.913452in}}%
\pgfpathlineto{\pgfqpoint{1.369401in}{0.968681in}}%
\pgfpathlineto{\pgfqpoint{1.407761in}{1.025279in}}%
\pgfpathlineto{\pgfqpoint{1.444577in}{1.082890in}}%
\pgfpathlineto{\pgfqpoint{1.480268in}{1.141214in}}%
\pgfpathlineto{\pgfqpoint{1.515235in}{1.199986in}}%
\pgfpathlineto{\pgfqpoint{1.549860in}{1.258976in}}%
\pgfpathlineto{\pgfqpoint{1.584499in}{1.317952in}}%
\pgfpathlineto{\pgfqpoint{1.619474in}{1.376697in}}%
\pgfpathlineto{\pgfqpoint{1.655087in}{1.435037in}}%
\pgfpathlineto{\pgfqpoint{1.691618in}{1.492798in}}%
\pgfpathlineto{\pgfqpoint{1.729339in}{1.549810in}}%
\pgfpathlineto{\pgfqpoint{1.768496in}{1.605874in}}%
\pgfpathlineto{\pgfqpoint{1.809299in}{1.660696in}}%
\pgfpathlineto{\pgfqpoint{1.851949in}{1.714041in}}%
\pgfpathlineto{\pgfqpoint{1.896633in}{1.765743in}}%
\pgfpathlineto{\pgfqpoint{1.943355in}{1.815541in}}%
\pgfpathlineto{\pgfqpoint{1.992032in}{1.863387in}}%
\pgfpathlineto{\pgfqpoint{2.042324in}{1.909476in}}%
\pgfpathlineto{\pgfqpoint{2.093559in}{1.954433in}}%
\pgfpathlineto{\pgfqpoint{2.144738in}{1.999218in}}%
\pgfpathlineto{\pgfqpoint{2.194896in}{2.044867in}}%
\pgfpathlineto{\pgfqpoint{2.243546in}{2.091830in}}%
\pgfpathlineto{\pgfqpoint{2.290939in}{2.140235in}}%
\pgfpathlineto{\pgfqpoint{2.337246in}{2.189899in}}%
\pgfpathlineto{\pgfqpoint{2.382589in}{2.240467in}}%
\pgfpathlineto{\pgfqpoint{2.427274in}{2.291886in}}%
\pgfpathlineto{\pgfqpoint{2.471350in}{2.343891in}}%
\pgfpathlineto{\pgfqpoint{2.514972in}{2.396391in}}%
\pgfpathlineto{\pgfqpoint{2.558236in}{2.449291in}}%
\pgfpathlineto{\pgfqpoint{2.601169in}{2.502443in}}%
\pgfpathlineto{\pgfqpoint{2.643879in}{2.555808in}}%
\pgfpathlineto{\pgfqpoint{2.686434in}{2.609325in}}%
\pgfpathlineto{\pgfqpoint{2.728896in}{2.662936in}}%
\pgfpathlineto{\pgfqpoint{2.771324in}{2.716581in}}%
\pgfpathlineto{\pgfqpoint{2.813777in}{2.770205in}}%
\pgfpathlineto{\pgfqpoint{2.856322in}{2.823779in}}%
\pgfpathlineto{\pgfqpoint{2.899016in}{2.877240in}}%
\pgfpathlineto{\pgfqpoint{2.941914in}{2.930529in}}%
\pgfpathlineto{\pgfqpoint{2.985081in}{2.983609in}}%
\pgfpathlineto{\pgfqpoint{3.028581in}{3.036420in}}%
\pgfpathlineto{\pgfqpoint{3.072474in}{3.088904in}}%
\pgfpathlineto{\pgfqpoint{3.116832in}{3.140999in}}%
\pgfpathlineto{\pgfqpoint{3.161725in}{3.192635in}}%
\pgfpathlineto{\pgfqpoint{3.207224in}{3.243737in}}%
\pgfpathlineto{\pgfqpoint{3.253408in}{3.294222in}}%
\pgfpathlineto{\pgfqpoint{3.300358in}{3.343997in}}%
\pgfpathlineto{\pgfqpoint{3.348158in}{3.392955in}}%
\pgfpathlineto{\pgfqpoint{3.396897in}{3.440977in}}%
\pgfpathlineto{\pgfqpoint{3.446665in}{3.487932in}}%
\pgfpathlineto{\pgfqpoint{3.497558in}{3.533664in}}%
\pgfpathlineto{\pgfqpoint{3.541035in}{3.571741in}}%
\pgfusepath{stroke}%
\end{pgfscope}%
\begin{pgfscope}%
\pgfpathrectangle{\pgfqpoint{0.647939in}{0.492442in}}{\pgfqpoint{3.079299in}{3.079299in}}%
\pgfusepath{clip}%
\pgfsetbuttcap%
\pgfsetroundjoin%
\pgfsetlinewidth{0.301125pt}%
\definecolor{currentstroke}{rgb}{0.500000,0.500000,0.500000}%
\pgfsetstrokecolor{currentstroke}%
\pgfsetstrokeopacity{0.300000}%
\pgfsetdash{}{0pt}%
\pgfpathmoveto{\pgfqpoint{0.927875in}{0.492442in}}%
\pgfpathlineto{\pgfqpoint{0.927875in}{0.492442in}}%
\pgfpathlineto{\pgfqpoint{0.988260in}{0.524531in}}%
\pgfpathlineto{\pgfqpoint{1.046120in}{0.560953in}}%
\pgfpathlineto{\pgfqpoint{1.101079in}{0.601598in}}%
\pgfpathlineto{\pgfqpoint{1.152866in}{0.646205in}}%
\pgfpathlineto{\pgfqpoint{1.201366in}{0.694380in}}%
\pgfpathlineto{\pgfqpoint{1.246635in}{0.745599in}}%
\pgfpathlineto{\pgfqpoint{1.288892in}{0.799332in}}%
\pgfpathlineto{\pgfqpoint{1.328476in}{0.855080in}}%
\pgfusepath{stroke}%
\end{pgfscope}%
\begin{pgfscope}%
\pgfpathrectangle{\pgfqpoint{0.647939in}{0.492442in}}{\pgfqpoint{3.079299in}{3.079299in}}%
\pgfusepath{clip}%
\pgfsetbuttcap%
\pgfsetroundjoin%
\pgfsetlinewidth{0.301125pt}%
\definecolor{currentstroke}{rgb}{0.500000,0.500000,0.500000}%
\pgfsetstrokecolor{currentstroke}%
\pgfsetstrokeopacity{0.300000}%
\pgfsetdash{}{0pt}%
\pgfpathmoveto{\pgfqpoint{1.137828in}{0.492442in}}%
\pgfpathlineto{\pgfqpoint{1.137828in}{0.492442in}}%
\pgfpathlineto{\pgfqpoint{1.183040in}{0.543683in}}%
\pgfpathlineto{\pgfqpoint{1.224677in}{0.597870in}}%
\pgfpathlineto{\pgfqpoint{1.263113in}{0.654385in}}%
\pgfpathlineto{\pgfqpoint{1.298830in}{0.712695in}}%
\pgfpathlineto{\pgfqpoint{1.332357in}{0.772316in}}%
\pgfusepath{stroke}%
\end{pgfscope}%
\begin{pgfscope}%
\pgfpathrectangle{\pgfqpoint{0.647939in}{0.492442in}}{\pgfqpoint{3.079299in}{3.079299in}}%
\pgfusepath{clip}%
\pgfsetbuttcap%
\pgfsetroundjoin%
\pgfsetlinewidth{0.301125pt}%
\definecolor{currentstroke}{rgb}{0.500000,0.500000,0.500000}%
\pgfsetstrokecolor{currentstroke}%
\pgfsetstrokeopacity{0.300000}%
\pgfsetdash{}{0pt}%
\pgfpathmoveto{\pgfqpoint{1.347780in}{0.492442in}}%
\pgfpathlineto{\pgfqpoint{1.347780in}{0.492442in}}%
\pgfpathlineto{\pgfqpoint{1.354951in}{0.560283in}}%
\pgfpathlineto{\pgfqpoint{1.366193in}{0.627670in}}%
\pgfpathlineto{\pgfqpoint{1.380566in}{0.694445in}}%
\pgfpathlineto{\pgfqpoint{1.397382in}{0.760666in}}%
\pgfpathlineto{\pgfqpoint{1.416201in}{0.826357in}}%
\pgfpathlineto{\pgfqpoint{1.436712in}{0.891519in}}%
\pgfpathlineto{\pgfqpoint{1.458709in}{0.956240in}}%
\pgfpathlineto{\pgfqpoint{1.482083in}{1.020496in}}%
\pgfpathlineto{\pgfqpoint{1.506761in}{1.084234in}}%
\pgfpathlineto{\pgfqpoint{1.532710in}{1.147447in}}%
\pgfpathlineto{\pgfqpoint{1.559937in}{1.210124in}}%
\pgfusepath{stroke}%
\end{pgfscope}%
\begin{pgfscope}%
\pgfpathrectangle{\pgfqpoint{0.647939in}{0.492442in}}{\pgfqpoint{3.079299in}{3.079299in}}%
\pgfusepath{clip}%
\pgfsetbuttcap%
\pgfsetroundjoin%
\pgfsetlinewidth{0.301125pt}%
\definecolor{currentstroke}{rgb}{0.500000,0.500000,0.500000}%
\pgfsetstrokecolor{currentstroke}%
\pgfsetstrokeopacity{0.300000}%
\pgfsetdash{}{0pt}%
\pgfpathmoveto{\pgfqpoint{1.557732in}{0.492442in}}%
\pgfpathlineto{\pgfqpoint{1.557732in}{0.492442in}}%
\pgfpathlineto{\pgfqpoint{1.513992in}{0.544286in}}%
\pgfpathlineto{\pgfqpoint{1.485348in}{0.598445in}}%
\pgfpathlineto{\pgfqpoint{1.467794in}{0.655938in}}%
\pgfpathlineto{\pgfqpoint{1.459347in}{0.718536in}}%
\pgfpathlineto{\pgfqpoint{1.459281in}{0.786721in}}%
\pgfpathlineto{\pgfqpoint{1.466180in}{0.854610in}}%
\pgfpathlineto{\pgfqpoint{1.478258in}{0.921763in}}%
\pgfusepath{stroke}%
\end{pgfscope}%
\begin{pgfscope}%
\pgfpathrectangle{\pgfqpoint{0.647939in}{0.492442in}}{\pgfqpoint{3.079299in}{3.079299in}}%
\pgfusepath{clip}%
\pgfsetbuttcap%
\pgfsetroundjoin%
\pgfsetlinewidth{0.301125pt}%
\definecolor{currentstroke}{rgb}{0.500000,0.500000,0.500000}%
\pgfsetstrokecolor{currentstroke}%
\pgfsetstrokeopacity{0.300000}%
\pgfsetdash{}{0pt}%
\pgfpathmoveto{\pgfqpoint{1.837668in}{0.492442in}}%
\pgfpathlineto{\pgfqpoint{1.837668in}{0.492442in}}%
\pgfpathlineto{\pgfqpoint{1.771269in}{0.508710in}}%
\pgfpathlineto{\pgfqpoint{1.706851in}{0.531445in}}%
\pgfpathlineto{\pgfqpoint{1.646155in}{0.562579in}}%
\pgfpathlineto{\pgfqpoint{1.592139in}{0.603906in}}%
\pgfpathlineto{\pgfqpoint{1.549743in}{0.654217in}}%
\pgfusepath{stroke}%
\end{pgfscope}%
\begin{pgfscope}%
\pgfpathrectangle{\pgfqpoint{0.647939in}{0.492442in}}{\pgfqpoint{3.079299in}{3.079299in}}%
\pgfusepath{clip}%
\pgfsetbuttcap%
\pgfsetroundjoin%
\pgfsetlinewidth{0.301125pt}%
\definecolor{currentstroke}{rgb}{0.500000,0.500000,0.500000}%
\pgfsetstrokecolor{currentstroke}%
\pgfsetstrokeopacity{0.300000}%
\pgfsetdash{}{0pt}%
\pgfpathmoveto{\pgfqpoint{2.257573in}{0.492442in}}%
\pgfpathlineto{\pgfqpoint{2.257573in}{0.492442in}}%
\pgfpathlineto{\pgfqpoint{2.189204in}{0.495303in}}%
\pgfpathlineto{\pgfqpoint{2.120841in}{0.498299in}}%
\pgfpathlineto{\pgfqpoint{2.052518in}{0.502059in}}%
\pgfpathlineto{\pgfqpoint{1.984299in}{0.507321in}}%
\pgfpathlineto{\pgfqpoint{1.916316in}{0.514972in}}%
\pgfusepath{stroke}%
\end{pgfscope}%
\begin{pgfscope}%
\pgfpathrectangle{\pgfqpoint{0.647939in}{0.492442in}}{\pgfqpoint{3.079299in}{3.079299in}}%
\pgfusepath{clip}%
\pgfsetbuttcap%
\pgfsetroundjoin%
\pgfsetlinewidth{0.301125pt}%
\definecolor{currentstroke}{rgb}{0.500000,0.500000,0.500000}%
\pgfsetstrokecolor{currentstroke}%
\pgfsetstrokeopacity{0.300000}%
\pgfsetdash{}{0pt}%
\pgfpathmoveto{\pgfqpoint{2.677477in}{0.492442in}}%
\pgfpathlineto{\pgfqpoint{2.677477in}{0.492442in}}%
\pgfpathlineto{\pgfqpoint{2.609971in}{0.503593in}}%
\pgfpathlineto{\pgfqpoint{2.542173in}{0.512808in}}%
\pgfpathlineto{\pgfqpoint{2.474151in}{0.520201in}}%
\pgfpathlineto{\pgfqpoint{2.405971in}{0.525975in}}%
\pgfpathlineto{\pgfqpoint{2.337689in}{0.530415in}}%
\pgfpathlineto{\pgfqpoint{2.269350in}{0.533891in}}%
\pgfpathlineto{\pgfqpoint{2.200986in}{0.536860in}}%
\pgfpathlineto{\pgfqpoint{2.132623in}{0.539871in}}%
\pgfpathlineto{\pgfqpoint{2.064297in}{0.543566in}}%
\pgfpathlineto{\pgfqpoint{1.996067in}{0.548686in}}%
\pgfpathlineto{\pgfqpoint{1.928060in}{0.556122in}}%
\pgfpathlineto{\pgfqpoint{1.860527in}{0.566962in}}%
\pgfpathlineto{\pgfqpoint{1.793971in}{0.582580in}}%
\pgfpathlineto{\pgfqpoint{1.729357in}{0.604703in}}%
\pgfpathlineto{\pgfqpoint{1.668480in}{0.635384in}}%
\pgfpathlineto{\pgfqpoint{1.614397in}{0.676595in}}%
\pgfpathlineto{\pgfqpoint{1.572914in}{0.726226in}}%
\pgfpathlineto{\pgfqpoint{1.546240in}{0.778294in}}%
\pgfpathlineto{\pgfqpoint{1.530435in}{0.833819in}}%
\pgfpathlineto{\pgfqpoint{1.523672in}{0.894496in}}%
\pgfpathlineto{\pgfqpoint{1.525555in}{0.962616in}}%
\pgfpathlineto{\pgfqpoint{1.534709in}{1.030177in}}%
\pgfusepath{stroke}%
\end{pgfscope}%
\begin{pgfscope}%
\pgfpathrectangle{\pgfqpoint{0.647939in}{0.492442in}}{\pgfqpoint{3.079299in}{3.079299in}}%
\pgfusepath{clip}%
\pgfsetbuttcap%
\pgfsetroundjoin%
\pgfsetlinewidth{0.301125pt}%
\definecolor{currentstroke}{rgb}{0.500000,0.500000,0.500000}%
\pgfsetstrokecolor{currentstroke}%
\pgfsetstrokeopacity{0.300000}%
\pgfsetdash{}{0pt}%
\pgfpathmoveto{\pgfqpoint{2.887429in}{0.492442in}}%
\pgfpathlineto{\pgfqpoint{2.887429in}{0.492442in}}%
\pgfpathlineto{\pgfqpoint{2.821275in}{0.509914in}}%
\pgfpathlineto{\pgfqpoint{2.754674in}{0.525589in}}%
\pgfpathlineto{\pgfqpoint{2.687649in}{0.539338in}}%
\pgfpathlineto{\pgfqpoint{2.620247in}{0.551094in}}%
\pgfpathlineto{\pgfqpoint{2.552528in}{0.560866in}}%
\pgfpathlineto{\pgfqpoint{2.484562in}{0.568749in}}%
\pgfpathlineto{\pgfqpoint{2.416418in}{0.574930in}}%
\pgfusepath{stroke}%
\end{pgfscope}%
\begin{pgfscope}%
\pgfpathrectangle{\pgfqpoint{0.647939in}{0.492442in}}{\pgfqpoint{3.079299in}{3.079299in}}%
\pgfusepath{clip}%
\pgfsetbuttcap%
\pgfsetroundjoin%
\pgfsetlinewidth{0.301125pt}%
\definecolor{currentstroke}{rgb}{0.500000,0.500000,0.500000}%
\pgfsetstrokecolor{currentstroke}%
\pgfsetstrokeopacity{0.300000}%
\pgfsetdash{}{0pt}%
\pgfpathmoveto{\pgfqpoint{3.097382in}{0.492442in}}%
\pgfpathlineto{\pgfqpoint{3.097382in}{0.492442in}}%
\pgfpathlineto{\pgfqpoint{3.032768in}{0.514963in}}%
\pgfpathlineto{\pgfqpoint{2.967779in}{0.536375in}}%
\pgfpathlineto{\pgfqpoint{2.902361in}{0.556430in}}%
\pgfpathlineto{\pgfqpoint{2.836479in}{0.574897in}}%
\pgfpathlineto{\pgfqpoint{2.770122in}{0.591577in}}%
\pgfpathlineto{\pgfqpoint{2.703308in}{0.606315in}}%
\pgfpathlineto{\pgfqpoint{2.636078in}{0.619019in}}%
\pgfpathlineto{\pgfqpoint{2.568492in}{0.629674in}}%
\pgfpathlineto{\pgfqpoint{2.500624in}{0.638351in}}%
\pgfpathlineto{\pgfqpoint{2.432547in}{0.645215in}}%
\pgfpathlineto{\pgfqpoint{2.364329in}{0.650526in}}%
\pgfpathlineto{\pgfqpoint{2.296025in}{0.654640in}}%
\pgfpathlineto{\pgfqpoint{2.227680in}{0.658018in}}%
\pgfpathlineto{\pgfqpoint{2.159327in}{0.661220in}}%
\pgfpathlineto{\pgfqpoint{2.090998in}{0.664900in}}%
\pgfpathlineto{\pgfqpoint{2.022754in}{0.669837in}}%
\pgfpathlineto{\pgfqpoint{1.954713in}{0.676979in}}%
\pgfpathlineto{\pgfqpoint{1.887134in}{0.687516in}}%
\pgfpathlineto{\pgfqpoint{1.820551in}{0.702971in}}%
\pgfpathlineto{\pgfqpoint{1.756029in}{0.725285in}}%
\pgfpathlineto{\pgfqpoint{1.695654in}{0.756815in}}%
\pgfpathlineto{\pgfqpoint{1.643063in}{0.799708in}}%
\pgfpathlineto{\pgfqpoint{1.605893in}{0.848593in}}%
\pgfusepath{stroke}%
\end{pgfscope}%
\begin{pgfscope}%
\pgfpathrectangle{\pgfqpoint{0.647939in}{0.492442in}}{\pgfqpoint{3.079299in}{3.079299in}}%
\pgfusepath{clip}%
\pgfsetbuttcap%
\pgfsetroundjoin%
\pgfsetlinewidth{0.301125pt}%
\definecolor{currentstroke}{rgb}{0.500000,0.500000,0.500000}%
\pgfsetstrokecolor{currentstroke}%
\pgfsetstrokeopacity{0.300000}%
\pgfsetdash{}{0pt}%
\pgfpathmoveto{\pgfqpoint{3.307334in}{0.492442in}}%
\pgfpathlineto{\pgfqpoint{3.307334in}{0.492442in}}%
\pgfpathlineto{\pgfqpoint{3.243686in}{0.517572in}}%
\pgfpathlineto{\pgfqpoint{3.179950in}{0.542473in}}%
\pgfpathlineto{\pgfqpoint{3.116024in}{0.566881in}}%
\pgfpathlineto{\pgfqpoint{3.051814in}{0.590530in}}%
\pgfpathlineto{\pgfqpoint{2.987236in}{0.613153in}}%
\pgfpathlineto{\pgfqpoint{2.922223in}{0.634487in}}%
\pgfpathlineto{\pgfqpoint{2.856727in}{0.654284in}}%
\pgfpathlineto{\pgfqpoint{2.790725in}{0.672322in}}%
\pgfpathlineto{\pgfqpoint{2.724225in}{0.688416in}}%
\pgfpathlineto{\pgfqpoint{2.657259in}{0.702440in}}%
\pgfpathlineto{\pgfqpoint{2.589884in}{0.714340in}}%
\pgfpathlineto{\pgfqpoint{2.522171in}{0.724148in}}%
\pgfpathlineto{\pgfqpoint{2.454201in}{0.731993in}}%
\pgfpathlineto{\pgfqpoint{2.386051in}{0.738112in}}%
\pgfpathlineto{\pgfqpoint{2.317789in}{0.742851in}}%
\pgfpathlineto{\pgfqpoint{2.249467in}{0.746658in}}%
\pgfpathlineto{\pgfqpoint{2.181124in}{0.750083in}}%
\pgfpathlineto{\pgfqpoint{2.112797in}{0.753789in}}%
\pgfpathlineto{\pgfqpoint{2.044541in}{0.758587in}}%
\pgfpathlineto{\pgfqpoint{1.976475in}{0.765477in}}%
\pgfpathlineto{\pgfqpoint{1.908853in}{0.775712in}}%
\pgfpathlineto{\pgfqpoint{1.842215in}{0.790909in}}%
\pgfpathlineto{\pgfqpoint{1.777696in}{0.813201in}}%
\pgfpathlineto{\pgfqpoint{1.717611in}{0.845211in}}%
\pgfpathlineto{\pgfqpoint{1.666130in}{0.889233in}}%
\pgfpathlineto{\pgfqpoint{1.632014in}{0.937376in}}%
\pgfpathlineto{\pgfqpoint{1.611420in}{0.987698in}}%
\pgfpathlineto{\pgfqpoint{1.600814in}{1.041907in}}%
\pgfpathlineto{\pgfqpoint{1.599010in}{1.101525in}}%
\pgfpathlineto{\pgfqpoint{1.606109in}{1.169022in}}%
\pgfpathlineto{\pgfqpoint{1.620256in}{1.235755in}}%
\pgfpathlineto{\pgfqpoint{1.639691in}{1.301189in}}%
\pgfpathlineto{\pgfqpoint{1.663288in}{1.365284in}}%
\pgfpathlineto{\pgfqpoint{1.690337in}{1.427978in}}%
\pgfusepath{stroke}%
\end{pgfscope}%
\begin{pgfscope}%
\pgfpathrectangle{\pgfqpoint{0.647939in}{0.492442in}}{\pgfqpoint{3.079299in}{3.079299in}}%
\pgfusepath{clip}%
\pgfsetbuttcap%
\pgfsetroundjoin%
\pgfsetlinewidth{0.301125pt}%
\definecolor{currentstroke}{rgb}{0.500000,0.500000,0.500000}%
\pgfsetstrokecolor{currentstroke}%
\pgfsetstrokeopacity{0.300000}%
\pgfsetdash{}{0pt}%
\pgfpathmoveto{\pgfqpoint{3.517286in}{0.492442in}}%
\pgfpathlineto{\pgfqpoint{3.517286in}{0.492442in}}%
\pgfpathlineto{\pgfqpoint{3.453548in}{0.517339in}}%
\pgfpathlineto{\pgfqpoint{3.390065in}{0.542877in}}%
\pgfpathlineto{\pgfqpoint{3.326741in}{0.568810in}}%
\pgfpathlineto{\pgfqpoint{3.263475in}{0.594885in}}%
\pgfpathlineto{\pgfqpoint{3.200160in}{0.620842in}}%
\pgfpathlineto{\pgfqpoint{3.136691in}{0.646416in}}%
\pgfpathlineto{\pgfqpoint{3.072963in}{0.671338in}}%
\pgfpathlineto{\pgfqpoint{3.008882in}{0.695332in}}%
\pgfpathlineto{\pgfqpoint{2.944366in}{0.718126in}}%
\pgfpathlineto{\pgfqpoint{2.879353in}{0.739457in}}%
\pgfpathlineto{\pgfqpoint{2.813806in}{0.759079in}}%
\pgfpathlineto{\pgfqpoint{2.747716in}{0.776777in}}%
\pgfpathlineto{\pgfqpoint{2.681103in}{0.792386in}}%
\pgfpathlineto{\pgfqpoint{2.614014in}{0.805805in}}%
\pgfpathlineto{\pgfqpoint{2.546520in}{0.817018in}}%
\pgfpathlineto{\pgfqpoint{2.478707in}{0.826115in}}%
\pgfpathlineto{\pgfqpoint{2.410663in}{0.833298in}}%
\pgfpathlineto{\pgfqpoint{2.342466in}{0.838880in}}%
\pgfpathlineto{\pgfqpoint{2.274181in}{0.843287in}}%
\pgfpathlineto{\pgfqpoint{2.205857in}{0.847062in}}%
\pgfpathlineto{\pgfqpoint{2.137535in}{0.850887in}}%
\pgfpathlineto{\pgfqpoint{2.069273in}{0.855603in}}%
\pgfpathlineto{\pgfqpoint{2.001182in}{0.862245in}}%
\pgfpathlineto{\pgfqpoint{1.933507in}{0.872133in}}%
\pgfpathlineto{\pgfqpoint{1.866797in}{0.887019in}}%
\pgfpathlineto{\pgfqpoint{1.802266in}{0.909272in}}%
\pgfpathlineto{\pgfqpoint{1.742527in}{0.941873in}}%
\pgfpathlineto{\pgfqpoint{1.742527in}{0.941873in}}%
\pgfpathlineto{\pgfqpoint{1.699632in}{0.978746in}}%
\pgfpathlineto{\pgfqpoint{1.666430in}{1.025188in}}%
\pgfusepath{stroke}%
\end{pgfscope}%
\begin{pgfscope}%
\pgfpathrectangle{\pgfqpoint{0.647939in}{0.492442in}}{\pgfqpoint{3.079299in}{3.079299in}}%
\pgfusepath{clip}%
\pgfsetbuttcap%
\pgfsetroundjoin%
\pgfsetlinewidth{0.301125pt}%
\definecolor{currentstroke}{rgb}{0.500000,0.500000,0.500000}%
\pgfsetstrokecolor{currentstroke}%
\pgfsetstrokeopacity{0.300000}%
\pgfsetdash{}{0pt}%
\pgfpathmoveto{\pgfqpoint{3.727238in}{0.492442in}}%
\pgfpathlineto{\pgfqpoint{3.727238in}{0.492442in}}%
\pgfpathlineto{\pgfqpoint{3.662409in}{0.514328in}}%
\pgfpathlineto{\pgfqpoint{3.598075in}{0.537632in}}%
\pgfpathlineto{\pgfqpoint{3.534189in}{0.562143in}}%
\pgfpathlineto{\pgfqpoint{3.470687in}{0.587634in}}%
\pgfpathlineto{\pgfqpoint{3.407488in}{0.613867in}}%
\pgfpathlineto{\pgfqpoint{3.344497in}{0.640598in}}%
\pgfpathlineto{\pgfqpoint{3.281611in}{0.667576in}}%
\pgfpathlineto{\pgfqpoint{3.218720in}{0.694544in}}%
\pgfpathlineto{\pgfqpoint{3.155713in}{0.721236in}}%
\pgfpathlineto{\pgfqpoint{3.092479in}{0.747383in}}%
\pgfpathlineto{\pgfqpoint{3.028912in}{0.772710in}}%
\pgfpathlineto{\pgfqpoint{2.964920in}{0.796937in}}%
\pgfpathlineto{\pgfqpoint{2.900427in}{0.819788in}}%
\pgfpathlineto{\pgfqpoint{2.835376in}{0.840998in}}%
\pgfpathlineto{\pgfqpoint{2.769743in}{0.860324in}}%
\pgfpathlineto{\pgfqpoint{2.703532in}{0.877562in}}%
\pgfpathlineto{\pgfqpoint{2.636780in}{0.892568in}}%
\pgfpathlineto{\pgfqpoint{2.569554in}{0.905284in}}%
\pgfpathlineto{\pgfqpoint{2.501941in}{0.915749in}}%
\pgfpathlineto{\pgfqpoint{2.434034in}{0.924117in}}%
\pgfpathlineto{\pgfqpoint{2.365925in}{0.930663in}}%
\pgfpathlineto{\pgfqpoint{2.297691in}{0.935787in}}%
\pgfpathlineto{\pgfqpoint{2.229395in}{0.940036in}}%
\pgfpathlineto{\pgfqpoint{2.161088in}{0.944104in}}%
\pgfpathlineto{\pgfqpoint{2.092826in}{0.948841in}}%
\pgfpathlineto{\pgfqpoint{2.024716in}{0.955320in}}%
\pgfpathlineto{\pgfqpoint{1.956999in}{0.964941in}}%
\pgfpathlineto{\pgfqpoint{1.890242in}{0.979615in}}%
\pgfpathlineto{\pgfqpoint{1.825783in}{1.001990in}}%
\pgfpathlineto{\pgfqpoint{1.766625in}{1.035450in}}%
\pgfpathlineto{\pgfqpoint{1.766625in}{1.035450in}}%
\pgfusepath{stroke}%
\end{pgfscope}%
\begin{pgfscope}%
\pgfpathrectangle{\pgfqpoint{0.647939in}{0.492442in}}{\pgfqpoint{3.079299in}{3.079299in}}%
\pgfusepath{clip}%
\pgfsetbuttcap%
\pgfsetroundjoin%
\pgfsetlinewidth{0.301125pt}%
\definecolor{currentstroke}{rgb}{0.500000,0.500000,0.500000}%
\pgfsetstrokecolor{currentstroke}%
\pgfsetstrokeopacity{0.300000}%
\pgfsetdash{}{0pt}%
\pgfpathmoveto{\pgfqpoint{3.727238in}{0.562426in}}%
\pgfpathlineto{\pgfqpoint{3.727238in}{0.562426in}}%
\pgfpathlineto{\pgfqpoint{3.662637in}{0.584977in}}%
\pgfpathlineto{\pgfqpoint{3.598566in}{0.608993in}}%
\pgfpathlineto{\pgfqpoint{3.534976in}{0.634260in}}%
\pgfpathlineto{\pgfqpoint{3.471801in}{0.660550in}}%
\pgfpathlineto{\pgfqpoint{3.408959in}{0.687627in}}%
\pgfpathlineto{\pgfqpoint{3.346352in}{0.715245in}}%
\pgfpathlineto{\pgfqpoint{3.283873in}{0.743155in}}%
\pgfpathlineto{\pgfqpoint{3.221411in}{0.771100in}}%
\pgfpathlineto{\pgfqpoint{3.158847in}{0.798817in}}%
\pgfpathlineto{\pgfqpoint{3.096066in}{0.826035in}}%
\pgfpathlineto{\pgfqpoint{3.032955in}{0.852476in}}%
\pgfpathlineto{\pgfqpoint{2.969412in}{0.877857in}}%
\pgfpathlineto{\pgfqpoint{2.905351in}{0.901894in}}%
\pgfpathlineto{\pgfqpoint{2.840706in}{0.924309in}}%
\pgfpathlineto{\pgfqpoint{2.775441in}{0.944840in}}%
\pgfpathlineto{\pgfqpoint{2.709551in}{0.963264in}}%
\pgfpathlineto{\pgfqpoint{2.643067in}{0.979407in}}%
\pgfpathlineto{\pgfqpoint{2.576049in}{0.993177in}}%
\pgfpathlineto{\pgfqpoint{2.508590in}{1.004589in}}%
\pgfpathlineto{\pgfqpoint{2.440791in}{1.013781in}}%
\pgfpathlineto{\pgfqpoint{2.372753in}{1.021018in}}%
\pgfpathlineto{\pgfqpoint{2.304565in}{1.026708in}}%
\pgfpathlineto{\pgfqpoint{2.236298in}{1.031404in}}%
\pgfpathlineto{\pgfqpoint{2.168013in}{1.035837in}}%
\pgfpathlineto{\pgfqpoint{2.099778in}{1.040942in}}%
\pgfpathlineto{\pgfqpoint{2.031721in}{1.047917in}}%
\pgfpathlineto{\pgfqpoint{1.964137in}{1.058364in}}%
\pgfpathlineto{\pgfqpoint{1.897759in}{1.074535in}}%
\pgfpathlineto{\pgfqpoint{1.834396in}{1.099650in}}%
\pgfpathlineto{\pgfqpoint{1.778333in}{1.137659in}}%
\pgfpathlineto{\pgfqpoint{1.741363in}{1.181844in}}%
\pgfpathlineto{\pgfqpoint{1.720687in}{1.227096in}}%
\pgfpathlineto{\pgfqpoint{1.710861in}{1.274892in}}%
\pgfusepath{stroke}%
\end{pgfscope}%
\begin{pgfscope}%
\pgfpathrectangle{\pgfqpoint{0.647939in}{0.492442in}}{\pgfqpoint{3.079299in}{3.079299in}}%
\pgfusepath{clip}%
\pgfsetbuttcap%
\pgfsetroundjoin%
\pgfsetlinewidth{0.301125pt}%
\definecolor{currentstroke}{rgb}{0.500000,0.500000,0.500000}%
\pgfsetstrokecolor{currentstroke}%
\pgfsetstrokeopacity{0.300000}%
\pgfsetdash{}{0pt}%
\pgfpathmoveto{\pgfqpoint{3.727238in}{0.632410in}}%
\pgfpathlineto{\pgfqpoint{3.727238in}{0.632410in}}%
\pgfpathlineto{\pgfqpoint{3.662888in}{0.655665in}}%
\pgfpathlineto{\pgfqpoint{3.599104in}{0.680434in}}%
\pgfpathlineto{\pgfqpoint{3.535839in}{0.706503in}}%
\pgfpathlineto{\pgfqpoint{3.473024in}{0.733641in}}%
\pgfpathlineto{\pgfqpoint{3.410575in}{0.761614in}}%
\pgfpathlineto{\pgfqpoint{3.348394in}{0.790177in}}%
\pgfpathlineto{\pgfqpoint{3.286370in}{0.819082in}}%
\pgfpathlineto{\pgfqpoint{3.224387in}{0.848075in}}%
\pgfpathlineto{\pgfqpoint{3.162324in}{0.876895in}}%
\pgfpathlineto{\pgfqpoint{3.100058in}{0.905272in}}%
\pgfpathlineto{\pgfqpoint{3.037470in}{0.932929in}}%
\pgfpathlineto{\pgfqpoint{2.974448in}{0.959579in}}%
\pgfpathlineto{\pgfqpoint{2.910895in}{0.984931in}}%
\pgfpathlineto{\pgfqpoint{2.846734in}{1.008695in}}%
\pgfpathlineto{\pgfqpoint{2.781915in}{1.030594in}}%
\pgfpathlineto{\pgfqpoint{2.716421in}{1.050377in}}%
\pgfpathlineto{\pgfqpoint{2.650273in}{1.067843in}}%
\pgfpathlineto{\pgfqpoint{2.583529in}{1.082867in}}%
\pgfpathlineto{\pgfqpoint{2.516274in}{1.095422in}}%
\pgfpathlineto{\pgfqpoint{2.448619in}{1.105614in}}%
\pgfpathlineto{\pgfqpoint{2.380677in}{1.113698in}}%
\pgfpathlineto{\pgfqpoint{2.312552in}{1.120086in}}%
\pgfpathlineto{\pgfqpoint{2.244327in}{1.125352in}}%
\pgfpathlineto{\pgfqpoint{2.176075in}{1.130260in}}%
\pgfpathlineto{\pgfqpoint{2.107875in}{1.135820in}}%
\pgfpathlineto{\pgfqpoint{2.039883in}{1.143389in}}%
\pgfpathlineto{\pgfqpoint{1.972479in}{1.154852in}}%
\pgfpathlineto{\pgfqpoint{1.906656in}{1.172949in}}%
\pgfpathlineto{\pgfqpoint{1.845061in}{1.201684in}}%
\pgfpathlineto{\pgfqpoint{1.845061in}{1.201684in}}%
\pgfpathlineto{\pgfqpoint{1.804049in}{1.233589in}}%
\pgfpathlineto{\pgfqpoint{1.772570in}{1.276074in}}%
\pgfpathlineto{\pgfqpoint{1.755836in}{1.319816in}}%
\pgfpathlineto{\pgfqpoint{1.749312in}{1.366285in}}%
\pgfpathlineto{\pgfqpoint{1.751645in}{1.417809in}}%
\pgfpathlineto{\pgfqpoint{1.763223in}{1.475768in}}%
\pgfpathlineto{\pgfqpoint{1.784855in}{1.540278in}}%
\pgfpathlineto{\pgfqpoint{1.812867in}{1.602338in}}%
\pgfusepath{stroke}%
\end{pgfscope}%
\begin{pgfscope}%
\pgfpathrectangle{\pgfqpoint{0.647939in}{0.492442in}}{\pgfqpoint{3.079299in}{3.079299in}}%
\pgfusepath{clip}%
\pgfsetbuttcap%
\pgfsetroundjoin%
\pgfsetlinewidth{0.301125pt}%
\definecolor{currentstroke}{rgb}{0.500000,0.500000,0.500000}%
\pgfsetstrokecolor{currentstroke}%
\pgfsetstrokeopacity{0.300000}%
\pgfsetdash{}{0pt}%
\pgfpathmoveto{\pgfqpoint{3.727238in}{0.702394in}}%
\pgfpathlineto{\pgfqpoint{3.727238in}{0.702394in}}%
\pgfpathlineto{\pgfqpoint{3.663163in}{0.726396in}}%
\pgfpathlineto{\pgfqpoint{3.599696in}{0.751964in}}%
\pgfpathlineto{\pgfqpoint{3.536788in}{0.778882in}}%
\pgfpathlineto{\pgfqpoint{3.474370in}{0.806922in}}%
\pgfpathlineto{\pgfqpoint{3.412357in}{0.835847in}}%
\pgfpathlineto{\pgfqpoint{3.350648in}{0.865418in}}%
\pgfpathlineto{\pgfqpoint{3.289132in}{0.895388in}}%
\pgfpathlineto{\pgfqpoint{3.227688in}{0.925508in}}%
\pgfpathlineto{\pgfqpoint{3.166192in}{0.955518in}}%
\pgfpathlineto{\pgfqpoint{3.104514in}{0.985154in}}%
\pgfpathlineto{\pgfqpoint{3.042530in}{1.014140in}}%
\pgfpathlineto{\pgfqpoint{2.980117in}{1.042189in}}%
\pgfpathlineto{\pgfqpoint{2.917166in}{1.069005in}}%
\pgfpathlineto{\pgfqpoint{2.853588in}{1.094289in}}%
\pgfpathlineto{\pgfqpoint{2.789315in}{1.117747in}}%
\pgfpathlineto{\pgfqpoint{2.724316in}{1.139102in}}%
\pgfpathlineto{\pgfqpoint{2.658599in}{1.158123in}}%
\pgfpathlineto{\pgfqpoint{2.592210in}{1.174642in}}%
\pgfpathlineto{\pgfqpoint{2.525235in}{1.188595in}}%
\pgfpathlineto{\pgfqpoint{2.457784in}{1.200042in}}%
\pgfpathlineto{\pgfqpoint{2.389979in}{1.209197in}}%
\pgfpathlineto{\pgfqpoint{2.321943in}{1.216467in}}%
\pgfpathlineto{\pgfqpoint{2.253779in}{1.222457in}}%
\pgfpathlineto{\pgfqpoint{2.185574in}{1.227985in}}%
\pgfpathlineto{\pgfqpoint{2.117427in}{1.234154in}}%
\pgfpathlineto{\pgfqpoint{2.049527in}{1.242495in}}%
\pgfpathlineto{\pgfqpoint{1.982370in}{1.255261in}}%
\pgfpathlineto{\pgfqpoint{1.917388in}{1.275946in}}%
\pgfpathlineto{\pgfqpoint{1.917388in}{1.275946in}}%
\pgfpathlineto{\pgfqpoint{1.866776in}{1.303305in}}%
\pgfpathlineto{\pgfqpoint{1.866776in}{1.303305in}}%
\pgfpathlineto{\pgfqpoint{1.831617in}{1.335200in}}%
\pgfpathlineto{\pgfqpoint{1.806583in}{1.376326in}}%
\pgfusepath{stroke}%
\end{pgfscope}%
\begin{pgfscope}%
\pgfpathrectangle{\pgfqpoint{0.647939in}{0.492442in}}{\pgfqpoint{3.079299in}{3.079299in}}%
\pgfusepath{clip}%
\pgfsetbuttcap%
\pgfsetroundjoin%
\pgfsetlinewidth{0.301125pt}%
\definecolor{currentstroke}{rgb}{0.500000,0.500000,0.500000}%
\pgfsetstrokecolor{currentstroke}%
\pgfsetstrokeopacity{0.300000}%
\pgfsetdash{}{0pt}%
\pgfpathmoveto{\pgfqpoint{3.727238in}{0.772378in}}%
\pgfpathlineto{\pgfqpoint{3.727238in}{0.772378in}}%
\pgfpathlineto{\pgfqpoint{3.663466in}{0.797172in}}%
\pgfpathlineto{\pgfqpoint{3.600348in}{0.823589in}}%
\pgfpathlineto{\pgfqpoint{3.537834in}{0.851410in}}%
\pgfpathlineto{\pgfqpoint{3.475856in}{0.880408in}}%
\pgfpathlineto{\pgfqpoint{3.414328in}{0.910348in}}%
\pgfpathlineto{\pgfqpoint{3.353146in}{0.940995in}}%
\pgfpathlineto{\pgfqpoint{3.292200in}{0.972106in}}%
\pgfpathlineto{\pgfqpoint{3.231364in}{1.003434in}}%
\pgfpathlineto{\pgfqpoint{3.170510in}{1.034727in}}%
\pgfpathlineto{\pgfqpoint{3.109507in}{1.065725in}}%
\pgfpathlineto{\pgfqpoint{3.048221in}{1.096159in}}%
\pgfpathlineto{\pgfqpoint{2.986523in}{1.125747in}}%
\pgfpathlineto{\pgfqpoint{2.924292in}{1.154192in}}%
\pgfpathlineto{\pgfqpoint{2.861423in}{1.181192in}}%
\pgfpathlineto{\pgfqpoint{2.797831in}{1.206437in}}%
\pgfpathlineto{\pgfqpoint{2.733463in}{1.229627in}}%
\pgfpathlineto{\pgfqpoint{2.668306in}{1.250491in}}%
\pgfpathlineto{\pgfqpoint{2.602393in}{1.268820in}}%
\pgfpathlineto{\pgfqpoint{2.535800in}{1.284496in}}%
\pgfpathlineto{\pgfqpoint{2.468640in}{1.297529in}}%
\pgfpathlineto{\pgfqpoint{2.401044in}{1.308094in}}%
\pgfpathlineto{\pgfqpoint{2.333149in}{1.316559in}}%
\pgfpathlineto{\pgfqpoint{2.265077in}{1.323516in}}%
\pgfpathlineto{\pgfqpoint{2.196943in}{1.329854in}}%
\pgfpathlineto{\pgfqpoint{2.128874in}{1.336825in}}%
\pgfpathlineto{\pgfqpoint{2.061122in}{1.346222in}}%
\pgfpathlineto{\pgfqpoint{1.994384in}{1.360839in}}%
\pgfpathlineto{\pgfqpoint{1.930900in}{1.385313in}}%
\pgfpathlineto{\pgfqpoint{1.930900in}{1.385313in}}%
\pgfpathlineto{\pgfqpoint{1.890329in}{1.412634in}}%
\pgfpathlineto{\pgfqpoint{1.890329in}{1.412634in}}%
\pgfpathlineto{\pgfqpoint{1.862760in}{1.444804in}}%
\pgfpathlineto{\pgfqpoint{1.845840in}{1.483982in}}%
\pgfpathlineto{\pgfqpoint{1.840126in}{1.524391in}}%
\pgfpathlineto{\pgfqpoint{1.843337in}{1.568704in}}%
\pgfusepath{stroke}%
\end{pgfscope}%
\begin{pgfscope}%
\pgfpathrectangle{\pgfqpoint{0.647939in}{0.492442in}}{\pgfqpoint{3.079299in}{3.079299in}}%
\pgfusepath{clip}%
\pgfsetbuttcap%
\pgfsetroundjoin%
\pgfsetlinewidth{0.301125pt}%
\definecolor{currentstroke}{rgb}{0.500000,0.500000,0.500000}%
\pgfsetstrokecolor{currentstroke}%
\pgfsetstrokeopacity{0.300000}%
\pgfsetdash{}{0pt}%
\pgfpathmoveto{\pgfqpoint{3.727238in}{0.842362in}}%
\pgfpathlineto{\pgfqpoint{3.727238in}{0.842362in}}%
\pgfpathlineto{\pgfqpoint{3.663801in}{0.867999in}}%
\pgfpathlineto{\pgfqpoint{3.601067in}{0.895317in}}%
\pgfpathlineto{\pgfqpoint{3.538990in}{0.924099in}}%
\pgfpathlineto{\pgfqpoint{3.477501in}{0.954117in}}%
\pgfpathlineto{\pgfqpoint{3.416511in}{0.985141in}}%
\pgfpathlineto{\pgfqpoint{3.355919in}{1.016936in}}%
\pgfpathlineto{\pgfqpoint{3.295609in}{1.049265in}}%
\pgfpathlineto{\pgfqpoint{3.235459in}{1.081890in}}%
\pgfpathlineto{\pgfqpoint{3.175338in}{1.114569in}}%
\pgfpathlineto{\pgfqpoint{3.115110in}{1.147051in}}%
\pgfpathlineto{\pgfqpoint{3.054640in}{1.179075in}}%
\pgfpathlineto{\pgfqpoint{2.993789in}{1.210368in}}%
\pgfpathlineto{\pgfqpoint{2.932426in}{1.240641in}}%
\pgfpathlineto{\pgfqpoint{2.870430in}{1.269589in}}%
\pgfpathlineto{\pgfqpoint{2.807697in}{1.296897in}}%
\pgfpathlineto{\pgfqpoint{2.744151in}{1.322247in}}%
\pgfpathlineto{\pgfqpoint{2.679752in}{1.345337in}}%
\pgfpathlineto{\pgfqpoint{2.614505in}{1.365904in}}%
\pgfpathlineto{\pgfqpoint{2.548466in}{1.383763in}}%
\pgfpathlineto{\pgfqpoint{2.481737in}{1.398848in}}%
\pgfpathlineto{\pgfqpoint{2.414459in}{1.411275in}}%
\pgfpathlineto{\pgfqpoint{2.346790in}{1.421377in}}%
\pgfpathlineto{\pgfqpoint{2.278879in}{1.429742in}}%
\pgfpathlineto{\pgfqpoint{2.210868in}{1.437290in}}%
\pgfpathlineto{\pgfqpoint{2.142926in}{1.445406in}}%
\pgfpathlineto{\pgfqpoint{2.075403in}{1.456283in}}%
\pgfpathlineto{\pgfqpoint{2.009408in}{1.473657in}}%
\pgfpathlineto{\pgfqpoint{2.009408in}{1.473657in}}%
\pgfpathlineto{\pgfqpoint{1.958511in}{1.497348in}}%
\pgfpathlineto{\pgfqpoint{1.958511in}{1.497348in}}%
\pgfpathlineto{\pgfqpoint{1.925518in}{1.524544in}}%
\pgfpathlineto{\pgfqpoint{1.902813in}{1.561978in}}%
\pgfpathlineto{\pgfqpoint{1.894590in}{1.598727in}}%
\pgfpathlineto{\pgfqpoint{1.895845in}{1.637790in}}%
\pgfpathlineto{\pgfqpoint{1.905927in}{1.680852in}}%
\pgfusepath{stroke}%
\end{pgfscope}%
\begin{pgfscope}%
\pgfpathrectangle{\pgfqpoint{0.647939in}{0.492442in}}{\pgfqpoint{3.079299in}{3.079299in}}%
\pgfusepath{clip}%
\pgfsetbuttcap%
\pgfsetroundjoin%
\pgfsetlinewidth{0.301125pt}%
\definecolor{currentstroke}{rgb}{0.500000,0.500000,0.500000}%
\pgfsetstrokecolor{currentstroke}%
\pgfsetstrokeopacity{0.300000}%
\pgfsetdash{}{0pt}%
\pgfpathmoveto{\pgfqpoint{3.727238in}{0.912347in}}%
\pgfpathlineto{\pgfqpoint{3.727238in}{0.912347in}}%
\pgfpathlineto{\pgfqpoint{3.664171in}{0.938880in}}%
\pgfpathlineto{\pgfqpoint{3.601866in}{0.967158in}}%
\pgfpathlineto{\pgfqpoint{3.540274in}{0.996962in}}%
\pgfpathlineto{\pgfqpoint{3.479327in}{1.028067in}}%
\pgfpathlineto{\pgfqpoint{3.418939in}{1.060245in}}%
\pgfpathlineto{\pgfqpoint{3.359006in}{1.093265in}}%
\pgfpathlineto{\pgfqpoint{3.299417in}{1.126902in}}%
\pgfpathlineto{\pgfqpoint{3.240047in}{1.160926in}}%
\pgfpathlineto{\pgfqpoint{3.180768in}{1.195107in}}%
\pgfpathlineto{\pgfqpoint{3.121443in}{1.229208in}}%
\pgfpathlineto{\pgfqpoint{3.061933in}{1.262983in}}%
\pgfpathlineto{\pgfqpoint{3.002095in}{1.296173in}}%
\pgfpathlineto{\pgfqpoint{2.941789in}{1.328501in}}%
\pgfpathlineto{\pgfqpoint{2.880880in}{1.359674in}}%
\pgfpathlineto{\pgfqpoint{2.819244in}{1.389379in}}%
\pgfpathlineto{\pgfqpoint{2.756778in}{1.417290in}}%
\pgfpathlineto{\pgfqpoint{2.693412in}{1.443083in}}%
\pgfpathlineto{\pgfqpoint{2.629116in}{1.466452in}}%
\pgfpathlineto{\pgfqpoint{2.563912in}{1.487146in}}%
\pgfpathlineto{\pgfqpoint{2.497879in}{1.505017in}}%
\pgfpathlineto{\pgfqpoint{2.431147in}{1.520074in}}%
\pgfpathlineto{\pgfqpoint{2.363879in}{1.532552in}}%
\pgfpathlineto{\pgfqpoint{2.296257in}{1.542991in}}%
\pgfpathlineto{\pgfqpoint{2.228476in}{1.552381in}}%
\pgfpathlineto{\pgfqpoint{2.160786in}{1.562361in}}%
\pgfpathlineto{\pgfqpoint{2.093746in}{1.575745in}}%
\pgfpathlineto{\pgfqpoint{2.029486in}{1.598057in}}%
\pgfpathlineto{\pgfqpoint{2.029486in}{1.598057in}}%
\pgfpathlineto{\pgfqpoint{1.993318in}{1.621013in}}%
\pgfpathlineto{\pgfqpoint{1.993318in}{1.621013in}}%
\pgfpathlineto{\pgfqpoint{1.970443in}{1.648151in}}%
\pgfpathlineto{\pgfqpoint{1.958623in}{1.682129in}}%
\pgfpathlineto{\pgfqpoint{1.957953in}{1.715886in}}%
\pgfpathlineto{\pgfqpoint{1.965884in}{1.752208in}}%
\pgfusepath{stroke}%
\end{pgfscope}%
\begin{pgfscope}%
\pgfpathrectangle{\pgfqpoint{0.647939in}{0.492442in}}{\pgfqpoint{3.079299in}{3.079299in}}%
\pgfusepath{clip}%
\pgfsetbuttcap%
\pgfsetroundjoin%
\pgfsetlinewidth{0.301125pt}%
\definecolor{currentstroke}{rgb}{0.500000,0.500000,0.500000}%
\pgfsetstrokecolor{currentstroke}%
\pgfsetstrokeopacity{0.300000}%
\pgfsetdash{}{0pt}%
\pgfpathmoveto{\pgfqpoint{3.727238in}{0.982331in}}%
\pgfpathlineto{\pgfqpoint{3.727238in}{0.982331in}}%
\pgfpathlineto{\pgfqpoint{3.664583in}{1.009820in}}%
\pgfpathlineto{\pgfqpoint{3.602752in}{1.039121in}}%
\pgfpathlineto{\pgfqpoint{3.541701in}{1.070015in}}%
\pgfpathlineto{\pgfqpoint{3.481360in}{1.102278in}}%
\pgfpathlineto{\pgfqpoint{3.421645in}{1.135687in}}%
\pgfpathlineto{\pgfqpoint{3.362455in}{1.170021in}}%
\pgfpathlineto{\pgfqpoint{3.303682in}{1.205064in}}%
\pgfpathlineto{\pgfqpoint{3.245204in}{1.240600in}}%
\pgfpathlineto{\pgfqpoint{3.186897in}{1.276413in}}%
\pgfpathlineto{\pgfqpoint{3.128626in}{1.312286in}}%
\pgfpathlineto{\pgfqpoint{3.070253in}{1.347991in}}%
\pgfpathlineto{\pgfqpoint{3.011636in}{1.383294in}}%
\pgfpathlineto{\pgfqpoint{2.952629in}{1.417938in}}%
\pgfpathlineto{\pgfqpoint{2.893088in}{1.451654in}}%
\pgfpathlineto{\pgfqpoint{2.832876in}{1.484152in}}%
\pgfpathlineto{\pgfqpoint{2.771868in}{1.515122in}}%
\pgfpathlineto{\pgfqpoint{2.709957in}{1.544237in}}%
\pgfpathlineto{\pgfqpoint{2.647070in}{1.571171in}}%
\pgfpathlineto{\pgfqpoint{2.583178in}{1.595618in}}%
\pgfpathlineto{\pgfqpoint{2.518310in}{1.617340in}}%
\pgfpathlineto{\pgfqpoint{2.452557in}{1.636225in}}%
\pgfpathlineto{\pgfqpoint{2.386080in}{1.652392in}}%
\pgfpathlineto{\pgfqpoint{2.319093in}{1.666320in}}%
\pgfpathlineto{\pgfqpoint{2.251856in}{1.679025in}}%
\pgfpathlineto{\pgfqpoint{2.184778in}{1.692490in}}%
\pgfpathlineto{\pgfqpoint{2.119032in}{1.710929in}}%
\pgfpathlineto{\pgfqpoint{2.119032in}{1.710929in}}%
\pgfpathlineto{\pgfqpoint{2.075628in}{1.731878in}}%
\pgfpathlineto{\pgfqpoint{2.075628in}{1.731878in}}%
\pgfpathlineto{\pgfqpoint{2.050540in}{1.754568in}}%
\pgfpathlineto{\pgfqpoint{2.050540in}{1.754568in}}%
\pgfpathlineto{\pgfqpoint{2.037818in}{1.780490in}}%
\pgfpathlineto{\pgfqpoint{2.035604in}{1.809346in}}%
\pgfpathlineto{\pgfqpoint{2.042025in}{1.839208in}}%
\pgfusepath{stroke}%
\end{pgfscope}%
\begin{pgfscope}%
\pgfpathrectangle{\pgfqpoint{0.647939in}{0.492442in}}{\pgfqpoint{3.079299in}{3.079299in}}%
\pgfusepath{clip}%
\pgfsetbuttcap%
\pgfsetroundjoin%
\pgfsetlinewidth{0.301125pt}%
\definecolor{currentstroke}{rgb}{0.500000,0.500000,0.500000}%
\pgfsetstrokecolor{currentstroke}%
\pgfsetstrokeopacity{0.300000}%
\pgfsetdash{}{0pt}%
\pgfpathmoveto{\pgfqpoint{3.727238in}{1.052315in}}%
\pgfpathlineto{\pgfqpoint{3.727238in}{1.052315in}}%
\pgfpathlineto{\pgfqpoint{3.665041in}{1.080824in}}%
\pgfpathlineto{\pgfqpoint{3.603740in}{1.111217in}}%
\pgfpathlineto{\pgfqpoint{3.543291in}{1.143272in}}%
\pgfpathlineto{\pgfqpoint{3.483628in}{1.176772in}}%
\pgfpathlineto{\pgfqpoint{3.424670in}{1.211500in}}%
\pgfpathlineto{\pgfqpoint{3.366322in}{1.247244in}}%
\pgfpathlineto{\pgfqpoint{3.308478in}{1.283800in}}%
\pgfpathlineto{\pgfqpoint{3.251027in}{1.320970in}}%
\pgfpathlineto{\pgfqpoint{3.193847in}{1.358558in}}%
\pgfpathlineto{\pgfqpoint{3.136814in}{1.396368in}}%
\pgfpathlineto{\pgfqpoint{3.079794in}{1.434198in}}%
\pgfpathlineto{\pgfqpoint{3.022652in}{1.471840in}}%
\pgfpathlineto{\pgfqpoint{2.965251in}{1.509084in}}%
\pgfpathlineto{\pgfqpoint{2.907451in}{1.545704in}}%
\pgfpathlineto{\pgfqpoint{2.849112in}{1.581457in}}%
\pgfpathlineto{\pgfqpoint{2.790098in}{1.616083in}}%
\pgfpathlineto{\pgfqpoint{2.730281in}{1.649299in}}%
\pgfpathlineto{\pgfqpoint{2.669552in}{1.680812in}}%
\pgfpathlineto{\pgfqpoint{2.607832in}{1.710333in}}%
\pgfpathlineto{\pgfqpoint{2.545086in}{1.737603in}}%
\pgfpathlineto{\pgfqpoint{2.481346in}{1.762459in}}%
\pgfpathlineto{\pgfqpoint{2.416729in}{1.784935in}}%
\pgfpathlineto{\pgfqpoint{2.351460in}{1.805453in}}%
\pgfpathlineto{\pgfqpoint{2.285960in}{1.825238in}}%
\pgfpathlineto{\pgfqpoint{2.221344in}{1.847517in}}%
\pgfpathlineto{\pgfqpoint{2.221344in}{1.847517in}}%
\pgfpathlineto{\pgfqpoint{2.176164in}{1.870809in}}%
\pgfpathlineto{\pgfqpoint{2.176164in}{1.870809in}}%
\pgfpathlineto{\pgfqpoint{2.154351in}{1.891454in}}%
\pgfpathlineto{\pgfqpoint{2.154351in}{1.891454in}}%
\pgfpathlineto{\pgfqpoint{2.144596in}{1.914593in}}%
\pgfusepath{stroke}%
\end{pgfscope}%
\begin{pgfscope}%
\pgfpathrectangle{\pgfqpoint{0.647939in}{0.492442in}}{\pgfqpoint{3.079299in}{3.079299in}}%
\pgfusepath{clip}%
\pgfsetbuttcap%
\pgfsetroundjoin%
\pgfsetlinewidth{0.301125pt}%
\definecolor{currentstroke}{rgb}{0.500000,0.500000,0.500000}%
\pgfsetstrokecolor{currentstroke}%
\pgfsetstrokeopacity{0.300000}%
\pgfsetdash{}{0pt}%
\pgfpathmoveto{\pgfqpoint{3.727238in}{1.192283in}}%
\pgfpathlineto{\pgfqpoint{3.727238in}{1.192283in}}%
\pgfpathlineto{\pgfqpoint{3.666129in}{1.223050in}}%
\pgfpathlineto{\pgfqpoint{3.606085in}{1.255852in}}%
\pgfpathlineto{\pgfqpoint{3.547072in}{1.290475in}}%
\pgfpathlineto{\pgfqpoint{3.489034in}{1.326713in}}%
\pgfpathlineto{\pgfqpoint{3.431902in}{1.364366in}}%
\pgfpathlineto{\pgfqpoint{3.375599in}{1.403249in}}%
\pgfpathlineto{\pgfqpoint{3.320043in}{1.443192in}}%
\pgfpathlineto{\pgfqpoint{3.265143in}{1.484035in}}%
\pgfpathlineto{\pgfqpoint{3.210808in}{1.525627in}}%
\pgfpathlineto{\pgfqpoint{3.156952in}{1.567836in}}%
\pgfpathlineto{\pgfqpoint{3.103493in}{1.610549in}}%
\pgfpathlineto{\pgfqpoint{3.050353in}{1.653658in}}%
\pgfpathlineto{\pgfqpoint{2.997462in}{1.697069in}}%
\pgfpathlineto{\pgfqpoint{2.944765in}{1.740715in}}%
\pgfpathlineto{\pgfqpoint{2.892228in}{1.784553in}}%
\pgfpathlineto{\pgfqpoint{2.839847in}{1.828576in}}%
\pgfpathlineto{\pgfqpoint{2.787667in}{1.872834in}}%
\pgfpathlineto{\pgfqpoint{2.735817in}{1.917474in}}%
\pgfpathlineto{\pgfqpoint{2.684581in}{1.962811in}}%
\pgfpathlineto{\pgfqpoint{2.634536in}{2.009442in}}%
\pgfpathlineto{\pgfqpoint{2.586906in}{2.058488in}}%
\pgfpathlineto{\pgfqpoint{2.544491in}{2.111976in}}%
\pgfpathlineto{\pgfqpoint{2.513957in}{2.172487in}}%
\pgfpathlineto{\pgfqpoint{2.513957in}{2.172487in}}%
\pgfpathlineto{\pgfqpoint{2.504810in}{2.218668in}}%
\pgfpathlineto{\pgfqpoint{2.507839in}{2.266153in}}%
\pgfpathlineto{\pgfqpoint{2.520748in}{2.312721in}}%
\pgfusepath{stroke}%
\end{pgfscope}%
\begin{pgfscope}%
\pgfpathrectangle{\pgfqpoint{0.647939in}{0.492442in}}{\pgfqpoint{3.079299in}{3.079299in}}%
\pgfusepath{clip}%
\pgfsetbuttcap%
\pgfsetroundjoin%
\pgfsetlinewidth{0.301125pt}%
\definecolor{currentstroke}{rgb}{0.500000,0.500000,0.500000}%
\pgfsetstrokecolor{currentstroke}%
\pgfsetstrokeopacity{0.300000}%
\pgfsetdash{}{0pt}%
\pgfpathmoveto{\pgfqpoint{3.727238in}{1.332251in}}%
\pgfpathlineto{\pgfqpoint{3.727238in}{1.332251in}}%
\pgfpathlineto{\pgfqpoint{3.667505in}{1.365608in}}%
\pgfpathlineto{\pgfqpoint{3.609058in}{1.401172in}}%
\pgfpathlineto{\pgfqpoint{3.551876in}{1.438737in}}%
\pgfpathlineto{\pgfqpoint{3.495923in}{1.478113in}}%
\pgfpathlineto{\pgfqpoint{3.441157in}{1.519127in}}%
\pgfpathlineto{\pgfqpoint{3.387534in}{1.561626in}}%
\pgfpathlineto{\pgfqpoint{3.335008in}{1.605474in}}%
\pgfpathlineto{\pgfqpoint{3.283542in}{1.650562in}}%
\pgfpathlineto{\pgfqpoint{3.233122in}{1.696817in}}%
\pgfpathlineto{\pgfqpoint{3.183746in}{1.744183in}}%
\pgfpathlineto{\pgfqpoint{3.135446in}{1.792646in}}%
\pgfpathlineto{\pgfqpoint{3.088296in}{1.842226in}}%
\pgfpathlineto{\pgfqpoint{3.042426in}{1.892994in}}%
\pgfpathlineto{\pgfqpoint{2.998059in}{1.945075in}}%
\pgfpathlineto{\pgfqpoint{2.955536in}{1.998662in}}%
\pgfpathlineto{\pgfqpoint{2.915386in}{2.054040in}}%
\pgfpathlineto{\pgfqpoint{2.878409in}{2.111573in}}%
\pgfpathlineto{\pgfqpoint{2.845800in}{2.171662in}}%
\pgfpathlineto{\pgfqpoint{2.819227in}{2.234608in}}%
\pgfpathlineto{\pgfqpoint{2.800720in}{2.300313in}}%
\pgfpathlineto{\pgfqpoint{2.792086in}{2.367953in}}%
\pgfpathlineto{\pgfqpoint{2.793934in}{2.436101in}}%
\pgfpathlineto{\pgfqpoint{2.805324in}{2.503368in}}%
\pgfpathlineto{\pgfqpoint{2.824394in}{2.568941in}}%
\pgfpathlineto{\pgfqpoint{2.849218in}{2.632618in}}%
\pgfpathlineto{\pgfqpoint{2.878256in}{2.694515in}}%
\pgfpathlineto{\pgfqpoint{2.910399in}{2.754875in}}%
\pgfpathlineto{\pgfqpoint{2.944893in}{2.813937in}}%
\pgfpathlineto{\pgfqpoint{2.981227in}{2.871897in}}%
\pgfpathlineto{\pgfqpoint{3.019066in}{2.928891in}}%
\pgfusepath{stroke}%
\end{pgfscope}%
\begin{pgfscope}%
\pgfpathrectangle{\pgfqpoint{0.647939in}{0.492442in}}{\pgfqpoint{3.079299in}{3.079299in}}%
\pgfusepath{clip}%
\pgfsetbuttcap%
\pgfsetroundjoin%
\pgfsetlinewidth{0.301125pt}%
\definecolor{currentstroke}{rgb}{0.500000,0.500000,0.500000}%
\pgfsetstrokecolor{currentstroke}%
\pgfsetstrokeopacity{0.300000}%
\pgfsetdash{}{0pt}%
\pgfpathmoveto{\pgfqpoint{3.727238in}{1.402235in}}%
\pgfpathlineto{\pgfqpoint{3.727238in}{1.402235in}}%
\pgfpathlineto{\pgfqpoint{3.668334in}{1.437030in}}%
\pgfpathlineto{\pgfqpoint{3.610848in}{1.474124in}}%
\pgfpathlineto{\pgfqpoint{3.554772in}{1.513318in}}%
\pgfpathlineto{\pgfqpoint{3.500084in}{1.554431in}}%
\pgfpathlineto{\pgfqpoint{3.446765in}{1.597305in}}%
\pgfpathlineto{\pgfqpoint{3.394789in}{1.641801in}}%
\pgfpathlineto{\pgfqpoint{3.344143in}{1.687804in}}%
\pgfpathlineto{\pgfqpoint{3.294838in}{1.735241in}}%
\pgfpathlineto{\pgfqpoint{3.246904in}{1.784064in}}%
\pgfusepath{stroke}%
\end{pgfscope}%
\begin{pgfscope}%
\pgfpathrectangle{\pgfqpoint{0.647939in}{0.492442in}}{\pgfqpoint{3.079299in}{3.079299in}}%
\pgfusepath{clip}%
\pgfsetbuttcap%
\pgfsetroundjoin%
\pgfsetlinewidth{0.301125pt}%
\definecolor{currentstroke}{rgb}{0.500000,0.500000,0.500000}%
\pgfsetstrokecolor{currentstroke}%
\pgfsetstrokeopacity{0.300000}%
\pgfsetdash{}{0pt}%
\pgfpathmoveto{\pgfqpoint{3.727238in}{1.542203in}}%
\pgfpathlineto{\pgfqpoint{3.727238in}{1.542203in}}%
\pgfpathlineto{\pgfqpoint{3.670352in}{1.580201in}}%
\pgfpathlineto{\pgfqpoint{3.615212in}{1.620693in}}%
\pgfpathlineto{\pgfqpoint{3.561842in}{1.663493in}}%
\pgfpathlineto{\pgfqpoint{3.510269in}{1.708444in}}%
\pgfpathlineto{\pgfqpoint{3.460518in}{1.755405in}}%
\pgfpathlineto{\pgfqpoint{3.412638in}{1.804271in}}%
\pgfpathlineto{\pgfqpoint{3.366711in}{1.854977in}}%
\pgfpathlineto{\pgfqpoint{3.322859in}{1.907485in}}%
\pgfpathlineto{\pgfqpoint{3.281259in}{1.961793in}}%
\pgfpathlineto{\pgfqpoint{3.242157in}{2.017924in}}%
\pgfpathlineto{\pgfqpoint{3.205896in}{2.075927in}}%
\pgfpathlineto{\pgfqpoint{3.172925in}{2.135851in}}%
\pgfpathlineto{\pgfqpoint{3.143801in}{2.197725in}}%
\pgfpathlineto{\pgfqpoint{3.119191in}{2.261517in}}%
\pgfpathlineto{\pgfqpoint{3.099825in}{2.327080in}}%
\pgfpathlineto{\pgfqpoint{3.086397in}{2.394099in}}%
\pgfpathlineto{\pgfqpoint{3.079418in}{2.462077in}}%
\pgfpathlineto{\pgfqpoint{3.079085in}{2.530406in}}%
\pgfpathlineto{\pgfqpoint{3.085221in}{2.598466in}}%
\pgfpathlineto{\pgfqpoint{3.097331in}{2.665738in}}%
\pgfpathlineto{\pgfqpoint{3.114731in}{2.731859in}}%
\pgfpathlineto{\pgfqpoint{3.136692in}{2.796623in}}%
\pgfpathlineto{\pgfqpoint{3.162537in}{2.859946in}}%
\pgfpathlineto{\pgfqpoint{3.191696in}{2.921821in}}%
\pgfpathlineto{\pgfqpoint{3.223703in}{2.982278in}}%
\pgfpathlineto{\pgfqpoint{3.258199in}{3.041357in}}%
\pgfpathlineto{\pgfqpoint{3.294913in}{3.099085in}}%
\pgfpathlineto{\pgfqpoint{3.333657in}{3.155470in}}%
\pgfpathlineto{\pgfqpoint{3.374299in}{3.210504in}}%
\pgfpathlineto{\pgfqpoint{3.416746in}{3.264160in}}%
\pgfpathlineto{\pgfqpoint{3.460958in}{3.316372in}}%
\pgfpathlineto{\pgfqpoint{3.506919in}{3.367050in}}%
\pgfpathlineto{\pgfqpoint{3.554636in}{3.416078in}}%
\pgfpathlineto{\pgfqpoint{3.604132in}{3.463309in}}%
\pgfpathlineto{\pgfqpoint{3.655442in}{3.508559in}}%
\pgfpathlineto{\pgfqpoint{3.708603in}{3.551616in}}%
\pgfpathlineto{\pgfqpoint{3.727238in}{3.566054in}}%
\pgfusepath{stroke}%
\end{pgfscope}%
\begin{pgfscope}%
\pgfpathrectangle{\pgfqpoint{0.647939in}{0.492442in}}{\pgfqpoint{3.079299in}{3.079299in}}%
\pgfusepath{clip}%
\pgfsetbuttcap%
\pgfsetroundjoin%
\pgfsetlinewidth{0.301125pt}%
\definecolor{currentstroke}{rgb}{0.500000,0.500000,0.500000}%
\pgfsetstrokecolor{currentstroke}%
\pgfsetstrokeopacity{0.300000}%
\pgfsetdash{}{0pt}%
\pgfpathmoveto{\pgfqpoint{3.727238in}{1.682171in}}%
\pgfpathlineto{\pgfqpoint{3.727238in}{1.682171in}}%
\pgfpathlineto{\pgfqpoint{3.673004in}{1.723857in}}%
\pgfpathlineto{\pgfqpoint{3.620948in}{1.768235in}}%
\pgfpathlineto{\pgfqpoint{3.571149in}{1.815133in}}%
\pgfpathlineto{\pgfqpoint{3.523690in}{1.864399in}}%
\pgfpathlineto{\pgfqpoint{3.478683in}{1.915913in}}%
\pgfpathlineto{\pgfqpoint{3.436278in}{1.969587in}}%
\pgfpathlineto{\pgfqpoint{3.396674in}{2.025356in}}%
\pgfpathlineto{\pgfqpoint{3.360136in}{2.083179in}}%
\pgfpathlineto{\pgfqpoint{3.327001in}{2.143013in}}%
\pgfpathlineto{\pgfqpoint{3.297676in}{2.204799in}}%
\pgfpathlineto{\pgfqpoint{3.272632in}{2.268432in}}%
\pgfpathlineto{\pgfqpoint{3.252378in}{2.333737in}}%
\pgfpathlineto{\pgfqpoint{3.237401in}{2.400438in}}%
\pgfpathlineto{\pgfqpoint{3.228095in}{2.468155in}}%
\pgfpathlineto{\pgfqpoint{3.224682in}{2.536420in}}%
\pgfpathlineto{\pgfqpoint{3.227164in}{2.604732in}}%
\pgfpathlineto{\pgfqpoint{3.235320in}{2.672610in}}%
\pgfpathlineto{\pgfqpoint{3.248761in}{2.739646in}}%
\pgfpathlineto{\pgfqpoint{3.266993in}{2.805547in}}%
\pgfpathlineto{\pgfqpoint{3.289501in}{2.870120in}}%
\pgfpathlineto{\pgfqpoint{3.315806in}{2.933250in}}%
\pgfpathlineto{\pgfqpoint{3.345489in}{2.994869in}}%
\pgfpathlineto{\pgfqpoint{3.378198in}{3.054942in}}%
\pgfpathlineto{\pgfqpoint{3.413658in}{3.113439in}}%
\pgfpathlineto{\pgfqpoint{3.451659in}{3.170322in}}%
\pgfpathlineto{\pgfqpoint{3.492058in}{3.225531in}}%
\pgfpathlineto{\pgfqpoint{3.534748in}{3.278986in}}%
\pgfpathlineto{\pgfqpoint{3.579663in}{3.330585in}}%
\pgfpathlineto{\pgfqpoint{3.626774in}{3.380187in}}%
\pgfpathlineto{\pgfqpoint{3.676066in}{3.427620in}}%
\pgfusepath{stroke}%
\end{pgfscope}%
\begin{pgfscope}%
\pgfpathrectangle{\pgfqpoint{0.647939in}{0.492442in}}{\pgfqpoint{3.079299in}{3.079299in}}%
\pgfusepath{clip}%
\pgfsetbuttcap%
\pgfsetroundjoin%
\pgfsetlinewidth{0.301125pt}%
\definecolor{currentstroke}{rgb}{0.500000,0.500000,0.500000}%
\pgfsetstrokecolor{currentstroke}%
\pgfsetstrokeopacity{0.300000}%
\pgfsetdash{}{0pt}%
\pgfpathmoveto{\pgfqpoint{3.727238in}{1.752155in}}%
\pgfpathlineto{\pgfqpoint{3.727238in}{1.752155in}}%
\pgfpathlineto{\pgfqpoint{3.674640in}{1.795883in}}%
\pgfpathlineto{\pgfqpoint{3.624487in}{1.842394in}}%
\pgfpathlineto{\pgfqpoint{3.576888in}{1.891518in}}%
\pgfpathlineto{\pgfqpoint{3.531968in}{1.943103in}}%
\pgfpathlineto{\pgfqpoint{3.489888in}{1.997026in}}%
\pgfpathlineto{\pgfqpoint{3.450856in}{2.053193in}}%
\pgfpathlineto{\pgfqpoint{3.415136in}{2.111522in}}%
\pgfpathlineto{\pgfqpoint{3.383052in}{2.171924in}}%
\pgfpathlineto{\pgfqpoint{3.354991in}{2.234292in}}%
\pgfpathlineto{\pgfqpoint{3.331380in}{2.298470in}}%
\pgfpathlineto{\pgfqpoint{3.312664in}{2.364232in}}%
\pgfpathlineto{\pgfqpoint{3.299250in}{2.431266in}}%
\pgfpathlineto{\pgfqpoint{3.291441in}{2.499175in}}%
\pgfpathlineto{\pgfqpoint{3.289382in}{2.567499in}}%
\pgfpathlineto{\pgfqpoint{3.293026in}{2.635764in}}%
\pgfpathlineto{\pgfqpoint{3.302151in}{2.703522in}}%
\pgfpathlineto{\pgfqpoint{3.316397in}{2.770396in}}%
\pgfpathlineto{\pgfqpoint{3.335324in}{2.836103in}}%
\pgfpathlineto{\pgfqpoint{3.358478in}{2.900449in}}%
\pgfpathlineto{\pgfqpoint{3.385436in}{2.963303in}}%
\pgfpathlineto{\pgfqpoint{3.415828in}{3.024575in}}%
\pgfusepath{stroke}%
\end{pgfscope}%
\begin{pgfscope}%
\pgfpathrectangle{\pgfqpoint{0.647939in}{0.492442in}}{\pgfqpoint{3.079299in}{3.079299in}}%
\pgfusepath{clip}%
\pgfsetbuttcap%
\pgfsetroundjoin%
\pgfsetlinewidth{0.301125pt}%
\definecolor{currentstroke}{rgb}{0.500000,0.500000,0.500000}%
\pgfsetstrokecolor{currentstroke}%
\pgfsetstrokeopacity{0.300000}%
\pgfsetdash{}{0pt}%
\pgfpathmoveto{\pgfqpoint{3.727238in}{1.822139in}}%
\pgfpathlineto{\pgfqpoint{3.727238in}{1.822139in}}%
\pgfpathlineto{\pgfqpoint{3.676533in}{1.868042in}}%
\pgfpathlineto{\pgfqpoint{3.628577in}{1.916810in}}%
\pgfpathlineto{\pgfqpoint{3.583518in}{1.968265in}}%
\pgfpathlineto{\pgfqpoint{3.541526in}{2.022252in}}%
\pgfpathlineto{\pgfqpoint{3.502812in}{2.078637in}}%
\pgfpathlineto{\pgfqpoint{3.467636in}{2.137291in}}%
\pgfpathlineto{\pgfqpoint{3.436315in}{2.198085in}}%
\pgfusepath{stroke}%
\end{pgfscope}%
\begin{pgfscope}%
\pgfpathrectangle{\pgfqpoint{0.647939in}{0.492442in}}{\pgfqpoint{3.079299in}{3.079299in}}%
\pgfusepath{clip}%
\pgfsetbuttcap%
\pgfsetroundjoin%
\pgfsetlinewidth{0.301125pt}%
\definecolor{currentstroke}{rgb}{0.500000,0.500000,0.500000}%
\pgfsetstrokecolor{currentstroke}%
\pgfsetstrokeopacity{0.300000}%
\pgfsetdash{}{0pt}%
\pgfpathmoveto{\pgfqpoint{3.727238in}{1.892124in}}%
\pgfpathlineto{\pgfqpoint{3.727238in}{1.892124in}}%
\pgfpathlineto{\pgfqpoint{3.678726in}{1.940332in}}%
\pgfpathlineto{\pgfqpoint{3.633312in}{1.991468in}}%
\pgfpathlineto{\pgfqpoint{3.591180in}{2.045342in}}%
\pgfpathlineto{\pgfqpoint{3.552546in}{2.101774in}}%
\pgfpathlineto{\pgfqpoint{3.517676in}{2.160600in}}%
\pgfpathlineto{\pgfqpoint{3.486874in}{2.221650in}}%
\pgfpathlineto{\pgfqpoint{3.460482in}{2.284726in}}%
\pgfpathlineto{\pgfqpoint{3.438864in}{2.349589in}}%
\pgfpathlineto{\pgfqpoint{3.422370in}{2.415936in}}%
\pgfpathlineto{\pgfqpoint{3.411291in}{2.483400in}}%
\pgfpathlineto{\pgfqpoint{3.405815in}{2.551546in}}%
\pgfpathlineto{\pgfqpoint{3.405993in}{2.619905in}}%
\pgfpathlineto{\pgfqpoint{3.411725in}{2.688026in}}%
\pgfpathlineto{\pgfqpoint{3.422781in}{2.755493in}}%
\pgfpathlineto{\pgfqpoint{3.438843in}{2.821956in}}%
\pgfpathlineto{\pgfqpoint{3.459544in}{2.887133in}}%
\pgfpathlineto{\pgfqpoint{3.484510in}{2.950804in}}%
\pgfpathlineto{\pgfqpoint{3.513396in}{3.012800in}}%
\pgfpathlineto{\pgfqpoint{3.545906in}{3.072979in}}%
\pgfpathlineto{\pgfqpoint{3.581796in}{3.131206in}}%
\pgfpathlineto{\pgfqpoint{3.620871in}{3.187343in}}%
\pgfpathlineto{\pgfqpoint{3.662982in}{3.241239in}}%
\pgfpathlineto{\pgfqpoint{3.708022in}{3.292713in}}%
\pgfpathlineto{\pgfqpoint{3.727238in}{3.313475in}}%
\pgfusepath{stroke}%
\end{pgfscope}%
\begin{pgfscope}%
\pgfpathrectangle{\pgfqpoint{0.647939in}{0.492442in}}{\pgfqpoint{3.079299in}{3.079299in}}%
\pgfusepath{clip}%
\pgfsetbuttcap%
\pgfsetroundjoin%
\pgfsetlinewidth{0.301125pt}%
\definecolor{currentstroke}{rgb}{0.500000,0.500000,0.500000}%
\pgfsetstrokecolor{currentstroke}%
\pgfsetstrokeopacity{0.300000}%
\pgfsetdash{}{0pt}%
\pgfpathmoveto{\pgfqpoint{3.727238in}{2.032092in}}%
\pgfpathlineto{\pgfqpoint{3.727238in}{2.032092in}}%
\pgfpathlineto{\pgfqpoint{3.684220in}{2.085244in}}%
\pgfpathlineto{\pgfqpoint{3.645119in}{2.141342in}}%
\pgfpathlineto{\pgfqpoint{3.610205in}{2.200136in}}%
\pgfpathlineto{\pgfqpoint{3.579773in}{2.261366in}}%
\pgfpathlineto{\pgfqpoint{3.554137in}{2.324750in}}%
\pgfpathlineto{\pgfqpoint{3.533615in}{2.389968in}}%
\pgfpathlineto{\pgfqpoint{3.518499in}{2.456642in}}%
\pgfpathlineto{\pgfqpoint{3.509008in}{2.524339in}}%
\pgfpathlineto{\pgfqpoint{3.505264in}{2.592592in}}%
\pgfpathlineto{\pgfqpoint{3.507262in}{2.660920in}}%
\pgfpathlineto{\pgfqpoint{3.514873in}{2.728859in}}%
\pgfpathlineto{\pgfqpoint{3.527872in}{2.795984in}}%
\pgfpathlineto{\pgfqpoint{3.545955in}{2.861925in}}%
\pgfpathlineto{\pgfqpoint{3.568795in}{2.926375in}}%
\pgfpathlineto{\pgfqpoint{3.596070in}{2.989080in}}%
\pgfpathlineto{\pgfqpoint{3.627486in}{3.049821in}}%
\pgfpathlineto{\pgfqpoint{3.662788in}{3.108393in}}%
\pgfpathlineto{\pgfqpoint{3.701766in}{3.164589in}}%
\pgfpathlineto{\pgfqpoint{3.727238in}{3.198935in}}%
\pgfusepath{stroke}%
\end{pgfscope}%
\begin{pgfscope}%
\pgfpathrectangle{\pgfqpoint{0.647939in}{0.492442in}}{\pgfqpoint{3.079299in}{3.079299in}}%
\pgfusepath{clip}%
\pgfsetbuttcap%
\pgfsetroundjoin%
\pgfsetlinewidth{0.301125pt}%
\definecolor{currentstroke}{rgb}{0.500000,0.500000,0.500000}%
\pgfsetstrokecolor{currentstroke}%
\pgfsetstrokeopacity{0.300000}%
\pgfsetdash{}{0pt}%
\pgfpathmoveto{\pgfqpoint{3.727238in}{2.172060in}}%
\pgfpathlineto{\pgfqpoint{3.727238in}{2.172060in}}%
\pgfpathlineto{\pgfqpoint{3.691553in}{2.230378in}}%
\pgfpathlineto{\pgfqpoint{3.660750in}{2.291410in}}%
\pgfpathlineto{\pgfqpoint{3.635137in}{2.354792in}}%
\pgfpathlineto{\pgfqpoint{3.615002in}{2.420121in}}%
\pgfpathlineto{\pgfqpoint{3.600600in}{2.486946in}}%
\pgfpathlineto{\pgfqpoint{3.592114in}{2.554771in}}%
\pgfpathlineto{\pgfqpoint{3.589623in}{2.623075in}}%
\pgfpathlineto{\pgfqpoint{3.593091in}{2.691338in}}%
\pgfpathlineto{\pgfqpoint{3.602371in}{2.759064in}}%
\pgfpathlineto{\pgfqpoint{3.617234in}{2.825796in}}%
\pgfpathlineto{\pgfqpoint{3.637390in}{2.891129in}}%
\pgfpathlineto{\pgfqpoint{3.662534in}{2.954711in}}%
\pgfpathlineto{\pgfqpoint{3.692369in}{3.016234in}}%
\pgfpathlineto{\pgfqpoint{3.726628in}{3.075412in}}%
\pgfpathlineto{\pgfqpoint{3.727238in}{3.076384in}}%
\pgfusepath{stroke}%
\end{pgfscope}%
\begin{pgfscope}%
\pgfpathrectangle{\pgfqpoint{0.647939in}{0.492442in}}{\pgfqpoint{3.079299in}{3.079299in}}%
\pgfusepath{clip}%
\pgfsetbuttcap%
\pgfsetroundjoin%
\pgfsetlinewidth{0.301125pt}%
\definecolor{currentstroke}{rgb}{0.500000,0.500000,0.500000}%
\pgfsetstrokecolor{currentstroke}%
\pgfsetstrokeopacity{0.300000}%
\pgfsetdash{}{0pt}%
\pgfpathmoveto{\pgfqpoint{3.727238in}{2.312028in}}%
\pgfpathlineto{\pgfqpoint{3.727238in}{2.312028in}}%
\pgfpathlineto{\pgfqpoint{3.701093in}{2.375177in}}%
\pgfpathlineto{\pgfqpoint{3.680785in}{2.440433in}}%
\pgfpathlineto{\pgfqpoint{3.666555in}{2.507278in}}%
\pgfpathlineto{\pgfqpoint{3.658562in}{2.575158in}}%
\pgfpathlineto{\pgfqpoint{3.656866in}{2.643487in}}%
\pgfpathlineto{\pgfqpoint{3.661412in}{2.711683in}}%
\pgfpathlineto{\pgfqpoint{3.672040in}{2.779199in}}%
\pgfusepath{stroke}%
\end{pgfscope}%
\begin{pgfscope}%
\pgfpathrectangle{\pgfqpoint{0.647939in}{0.492442in}}{\pgfqpoint{3.079299in}{3.079299in}}%
\pgfusepath{clip}%
\pgfsetbuttcap%
\pgfsetroundjoin%
\pgfsetlinewidth{0.301125pt}%
\definecolor{currentstroke}{rgb}{0.500000,0.500000,0.500000}%
\pgfsetstrokecolor{currentstroke}%
\pgfsetstrokeopacity{0.300000}%
\pgfsetdash{}{0pt}%
\pgfpathmoveto{\pgfqpoint{3.727238in}{2.451996in}}%
\pgfpathlineto{\pgfqpoint{3.727238in}{2.451996in}}%
\pgfpathlineto{\pgfqpoint{3.712833in}{2.518805in}}%
\pgfpathlineto{\pgfqpoint{3.704952in}{2.586686in}}%
\pgfpathlineto{\pgfqpoint{3.703642in}{2.655007in}}%
\pgfpathlineto{\pgfqpoint{3.708840in}{2.723147in}}%
\pgfpathlineto{\pgfqpoint{3.720380in}{2.790511in}}%
\pgfpathlineto{\pgfqpoint{3.727238in}{2.821826in}}%
\pgfusepath{stroke}%
\end{pgfscope}%
\begin{pgfscope}%
\pgfpathrectangle{\pgfqpoint{0.647939in}{0.492442in}}{\pgfqpoint{3.079299in}{3.079299in}}%
\pgfusepath{clip}%
\pgfsetbuttcap%
\pgfsetroundjoin%
\pgfsetlinewidth{0.301125pt}%
\definecolor{currentstroke}{rgb}{0.500000,0.500000,0.500000}%
\pgfsetstrokecolor{currentstroke}%
\pgfsetstrokeopacity{0.300000}%
\pgfsetdash{}{0pt}%
\pgfpathmoveto{\pgfqpoint{0.647939in}{2.432638in}}%
\pgfpathlineto{\pgfqpoint{0.687113in}{2.438200in}}%
\pgfpathlineto{\pgfqpoint{0.754747in}{2.448564in}}%
\pgfpathlineto{\pgfqpoint{0.822135in}{2.460426in}}%
\pgfpathlineto{\pgfqpoint{0.889242in}{2.473787in}}%
\pgfpathlineto{\pgfqpoint{0.956043in}{2.488602in}}%
\pgfpathlineto{\pgfqpoint{1.022528in}{2.504781in}}%
\pgfpathlineto{\pgfqpoint{1.088703in}{2.522187in}}%
\pgfpathlineto{\pgfqpoint{1.154595in}{2.540640in}}%
\pgfpathlineto{\pgfqpoint{1.220249in}{2.559923in}}%
\pgfpathlineto{\pgfqpoint{1.285731in}{2.579787in}}%
\pgfpathlineto{\pgfqpoint{1.351121in}{2.599951in}}%
\pgfpathlineto{\pgfqpoint{1.416513in}{2.620110in}}%
\pgfpathlineto{\pgfqpoint{1.482005in}{2.639939in}}%
\pgfpathlineto{\pgfqpoint{1.547694in}{2.659104in}}%
\pgfpathlineto{\pgfqpoint{1.613665in}{2.677268in}}%
\pgfpathlineto{\pgfqpoint{1.679985in}{2.694107in}}%
\pgfpathlineto{\pgfqpoint{1.746690in}{2.709338in}}%
\pgfpathlineto{\pgfqpoint{1.813786in}{2.722736in}}%
\pgfpathlineto{\pgfqpoint{1.881244in}{2.734171in}}%
\pgfpathlineto{\pgfqpoint{1.949007in}{2.743637in}}%
\pgfpathlineto{\pgfqpoint{2.017000in}{2.751293in}}%
\pgfpathlineto{\pgfqpoint{2.085144in}{2.757504in}}%
\pgfpathlineto{\pgfqpoint{2.153362in}{2.762857in}}%
\pgfpathlineto{\pgfqpoint{2.221585in}{2.768164in}}%
\pgfpathlineto{\pgfqpoint{2.289717in}{2.774467in}}%
\pgfpathlineto{\pgfqpoint{2.357596in}{2.782999in}}%
\pgfpathlineto{\pgfqpoint{2.424913in}{2.795084in}}%
\pgfpathlineto{\pgfqpoint{2.491182in}{2.811895in}}%
\pgfpathlineto{\pgfqpoint{2.555810in}{2.834138in}}%
\pgfpathlineto{\pgfqpoint{2.618272in}{2.861886in}}%
\pgfpathlineto{\pgfqpoint{2.678279in}{2.894630in}}%
\pgfpathlineto{\pgfqpoint{2.735842in}{2.931536in}}%
\pgfpathlineto{\pgfqpoint{2.791196in}{2.971708in}}%
\pgfpathlineto{\pgfqpoint{2.844689in}{3.014342in}}%
\pgfpathlineto{\pgfqpoint{2.896694in}{3.058784in}}%
\pgfpathlineto{\pgfqpoint{2.947569in}{3.104527in}}%
\pgfpathlineto{\pgfqpoint{2.997630in}{3.151167in}}%
\pgfpathlineto{\pgfqpoint{3.047147in}{3.198385in}}%
\pgfpathlineto{\pgfqpoint{3.096359in}{3.245924in}}%
\pgfpathlineto{\pgfqpoint{3.145472in}{3.293566in}}%
\pgfpathlineto{\pgfqpoint{3.194668in}{3.341124in}}%
\pgfpathlineto{\pgfqpoint{3.244114in}{3.388422in}}%
\pgfpathlineto{\pgfqpoint{3.293961in}{3.435297in}}%
\pgfpathlineto{\pgfqpoint{3.344352in}{3.481588in}}%
\pgfpathlineto{\pgfqpoint{3.395425in}{3.527124in}}%
\pgfpathlineto{\pgfqpoint{3.447302in}{3.571741in}}%
\pgfpathlineto{\pgfqpoint{3.447302in}{3.571741in}}%
\pgfusepath{stroke}%
\end{pgfscope}%
\begin{pgfscope}%
\pgfpathrectangle{\pgfqpoint{0.647939in}{0.492442in}}{\pgfqpoint{3.079299in}{3.079299in}}%
\pgfusepath{clip}%
\pgfsetbuttcap%
\pgfsetroundjoin%
\pgfsetlinewidth{0.301125pt}%
\definecolor{currentstroke}{rgb}{0.500000,0.500000,0.500000}%
\pgfsetstrokecolor{currentstroke}%
\pgfsetstrokeopacity{0.300000}%
\pgfsetdash{}{0pt}%
\pgfpathmoveto{\pgfqpoint{0.647939in}{2.749158in}}%
\pgfpathlineto{\pgfqpoint{0.685945in}{2.754226in}}%
\pgfpathlineto{\pgfqpoint{0.753675in}{2.763948in}}%
\pgfpathlineto{\pgfqpoint{0.821198in}{2.775025in}}%
\pgfpathlineto{\pgfqpoint{0.888487in}{2.787441in}}%
\pgfpathlineto{\pgfqpoint{0.955527in}{2.801138in}}%
\pgfpathlineto{\pgfqpoint{1.022316in}{2.816018in}}%
\pgfpathlineto{\pgfqpoint{1.088866in}{2.831935in}}%
\pgfpathlineto{\pgfqpoint{1.155207in}{2.848706in}}%
\pgfpathlineto{\pgfqpoint{1.221383in}{2.866116in}}%
\pgfpathlineto{\pgfqpoint{1.287455in}{2.883921in}}%
\pgfpathlineto{\pgfqpoint{1.353493in}{2.901849in}}%
\pgfpathlineto{\pgfqpoint{1.419576in}{2.919612in}}%
\pgfpathlineto{\pgfqpoint{1.485782in}{2.936909in}}%
\pgfpathlineto{\pgfqpoint{1.552183in}{2.953435in}}%
\pgfpathlineto{\pgfqpoint{1.618839in}{2.968896in}}%
\pgfpathlineto{\pgfqpoint{1.685789in}{2.983025in}}%
\pgfpathlineto{\pgfqpoint{1.753046in}{2.995602in}}%
\pgfpathlineto{\pgfqpoint{1.820598in}{3.006475in}}%
\pgfpathlineto{\pgfqpoint{1.888412in}{3.015580in}}%
\pgfpathlineto{\pgfqpoint{1.956435in}{3.022972in}}%
\pgfpathlineto{\pgfqpoint{2.024608in}{3.028846in}}%
\pgfpathlineto{\pgfqpoint{2.092873in}{3.033546in}}%
\pgfpathlineto{\pgfqpoint{2.161183in}{3.037564in}}%
\pgfpathlineto{\pgfqpoint{2.229496in}{3.041543in}}%
\pgfpathlineto{\pgfqpoint{2.297759in}{3.046268in}}%
\pgfpathlineto{\pgfqpoint{2.365881in}{3.052641in}}%
\pgfpathlineto{\pgfqpoint{2.433699in}{3.061603in}}%
\pgfpathlineto{\pgfqpoint{2.500957in}{3.074022in}}%
\pgfpathlineto{\pgfqpoint{2.567310in}{3.090568in}}%
\pgfpathlineto{\pgfqpoint{2.632377in}{3.111592in}}%
\pgfpathlineto{\pgfqpoint{2.695836in}{3.137074in}}%
\pgfpathlineto{\pgfqpoint{2.757491in}{3.166669in}}%
\pgfpathlineto{\pgfqpoint{2.817314in}{3.199825in}}%
\pgfpathlineto{\pgfqpoint{2.875420in}{3.235918in}}%
\pgfpathlineto{\pgfqpoint{2.932018in}{3.274341in}}%
\pgfpathlineto{\pgfqpoint{2.987365in}{3.314557in}}%
\pgfpathlineto{\pgfqpoint{3.041728in}{3.356104in}}%
\pgfpathlineto{\pgfqpoint{3.095369in}{3.398583in}}%
\pgfpathlineto{\pgfqpoint{3.148534in}{3.441657in}}%
\pgfpathlineto{\pgfqpoint{3.201446in}{3.485043in}}%
\pgfpathlineto{\pgfqpoint{3.254314in}{3.528484in}}%
\pgfpathlineto{\pgfqpoint{3.307334in}{3.571741in}}%
\pgfpathlineto{\pgfqpoint{3.307334in}{3.571741in}}%
\pgfusepath{stroke}%
\end{pgfscope}%
\begin{pgfscope}%
\pgfpathrectangle{\pgfqpoint{0.647939in}{0.492442in}}{\pgfqpoint{3.079299in}{3.079299in}}%
\pgfusepath{clip}%
\pgfsetbuttcap%
\pgfsetroundjoin%
\pgfsetlinewidth{0.301125pt}%
\definecolor{currentstroke}{rgb}{0.500000,0.500000,0.500000}%
\pgfsetstrokecolor{currentstroke}%
\pgfsetstrokeopacity{0.300000}%
\pgfsetdash{}{0pt}%
\pgfpathmoveto{\pgfqpoint{0.647939in}{2.945634in}}%
\pgfpathlineto{\pgfqpoint{0.678773in}{2.949541in}}%
\pgfpathlineto{\pgfqpoint{0.746572in}{2.958784in}}%
\pgfpathlineto{\pgfqpoint{0.814184in}{2.969302in}}%
\pgfpathlineto{\pgfqpoint{0.881589in}{2.981077in}}%
\pgfpathlineto{\pgfqpoint{0.948774in}{2.994047in}}%
\pgfpathlineto{\pgfqpoint{1.015739in}{3.008115in}}%
\pgfpathlineto{\pgfqpoint{1.082496in}{3.023141in}}%
\pgfpathlineto{\pgfqpoint{1.149073in}{3.038946in}}%
\pgfpathlineto{\pgfqpoint{1.215515in}{3.055317in}}%
\pgfpathlineto{\pgfqpoint{1.281875in}{3.072014in}}%
\pgfpathlineto{\pgfqpoint{1.348218in}{3.088778in}}%
\pgfpathlineto{\pgfqpoint{1.414614in}{3.105333in}}%
\pgfpathlineto{\pgfqpoint{1.481131in}{3.121394in}}%
\pgfpathlineto{\pgfqpoint{1.547829in}{3.136679in}}%
\pgfpathlineto{\pgfqpoint{1.614757in}{3.150916in}}%
\pgfpathlineto{\pgfqpoint{1.681946in}{3.163865in}}%
\pgfpathlineto{\pgfqpoint{1.749401in}{3.175332in}}%
\pgfpathlineto{\pgfqpoint{1.817110in}{3.185192in}}%
\pgfpathlineto{\pgfqpoint{1.885038in}{3.193404in}}%
\pgfpathlineto{\pgfqpoint{1.953141in}{3.200029in}}%
\pgfpathlineto{\pgfqpoint{2.021367in}{3.205254in}}%
\pgfpathlineto{\pgfqpoint{2.089669in}{3.209395in}}%
\pgfpathlineto{\pgfqpoint{2.158008in}{3.212891in}}%
\pgfpathlineto{\pgfqpoint{2.226352in}{3.216298in}}%
\pgfpathlineto{\pgfqpoint{2.294663in}{3.220279in}}%
\pgfpathlineto{\pgfqpoint{2.362880in}{3.225587in}}%
\pgfpathlineto{\pgfqpoint{2.430891in}{3.233013in}}%
\pgfpathlineto{\pgfqpoint{2.498519in}{3.243300in}}%
\pgfpathlineto{\pgfqpoint{2.565516in}{3.257066in}}%
\pgfpathlineto{\pgfqpoint{2.631592in}{3.274713in}}%
\pgfpathlineto{\pgfqpoint{2.696463in}{3.296372in}}%
\pgfpathlineto{\pgfqpoint{2.759916in}{3.321904in}}%
\pgfpathlineto{\pgfqpoint{2.821844in}{3.350953in}}%
\pgfpathlineto{\pgfqpoint{2.882259in}{3.383040in}}%
\pgfpathlineto{\pgfqpoint{2.941275in}{3.417643in}}%
\pgfpathlineto{\pgfqpoint{2.999073in}{3.454251in}}%
\pgfpathlineto{\pgfqpoint{3.055868in}{3.492402in}}%
\pgfpathlineto{\pgfqpoint{3.111889in}{3.531685in}}%
\pgfpathlineto{\pgfqpoint{3.167366in}{3.571741in}}%
\pgfpathlineto{\pgfqpoint{3.167366in}{3.571741in}}%
\pgfusepath{stroke}%
\end{pgfscope}%
\begin{pgfscope}%
\pgfpathrectangle{\pgfqpoint{0.647939in}{0.492442in}}{\pgfqpoint{3.079299in}{3.079299in}}%
\pgfusepath{clip}%
\pgfsetbuttcap%
\pgfsetroundjoin%
\pgfsetlinewidth{0.301125pt}%
\definecolor{currentstroke}{rgb}{0.500000,0.500000,0.500000}%
\pgfsetstrokecolor{currentstroke}%
\pgfsetstrokeopacity{0.300000}%
\pgfsetdash{}{0pt}%
\pgfpathmoveto{\pgfqpoint{0.647939in}{3.083656in}}%
\pgfpathlineto{\pgfqpoint{0.700311in}{3.090438in}}%
\pgfpathlineto{\pgfqpoint{0.768087in}{3.099844in}}%
\pgfpathlineto{\pgfqpoint{0.835682in}{3.110476in}}%
\pgfpathlineto{\pgfqpoint{0.903080in}{3.122294in}}%
\pgfpathlineto{\pgfqpoint{0.970274in}{3.135219in}}%
\pgfpathlineto{\pgfqpoint{1.037271in}{3.149135in}}%
\pgfpathlineto{\pgfqpoint{1.104088in}{3.163893in}}%
\pgfpathlineto{\pgfqpoint{1.170758in}{3.179305in}}%
\pgfpathlineto{\pgfqpoint{1.237326in}{3.195156in}}%
\pgfpathlineto{\pgfqpoint{1.303846in}{3.211202in}}%
\pgfpathlineto{\pgfqpoint{1.370384in}{3.227180in}}%
\pgfpathlineto{\pgfqpoint{1.437001in}{3.242817in}}%
\pgfpathlineto{\pgfqpoint{1.503759in}{3.257843in}}%
\pgfpathlineto{\pgfqpoint{1.570707in}{3.271991in}}%
\pgfpathlineto{\pgfqpoint{1.637881in}{3.285018in}}%
\pgfpathlineto{\pgfqpoint{1.705299in}{3.296714in}}%
\pgfpathlineto{\pgfqpoint{1.772956in}{3.306928in}}%
\pgfpathlineto{\pgfqpoint{1.840831in}{3.315577in}}%
\pgfpathlineto{\pgfqpoint{1.908886in}{3.322670in}}%
\pgfpathlineto{\pgfqpoint{1.977077in}{3.328316in}}%
\pgfpathlineto{\pgfqpoint{2.045361in}{3.332732in}}%
\pgfpathlineto{\pgfqpoint{2.113698in}{3.336258in}}%
\pgfpathlineto{\pgfqpoint{2.182057in}{3.339346in}}%
\pgfpathlineto{\pgfqpoint{2.250411in}{3.342548in}}%
\pgfpathlineto{\pgfqpoint{2.318723in}{3.346492in}}%
\pgfpathlineto{\pgfqpoint{2.386935in}{3.351859in}}%
\pgfpathlineto{\pgfqpoint{2.454941in}{3.359347in}}%
\pgfpathlineto{\pgfqpoint{2.522576in}{3.369607in}}%
\pgfpathlineto{\pgfqpoint{2.589621in}{3.383159in}}%
\pgfpathlineto{\pgfqpoint{2.655828in}{3.400324in}}%
\pgfpathlineto{\pgfqpoint{2.720960in}{3.421197in}}%
\pgfpathlineto{\pgfqpoint{2.784838in}{3.445647in}}%
\pgfpathlineto{\pgfqpoint{2.847374in}{3.473364in}}%
\pgfpathlineto{\pgfqpoint{2.908574in}{3.503929in}}%
\pgfpathlineto{\pgfqpoint{2.968530in}{3.536876in}}%
\pgfpathlineto{\pgfqpoint{3.027398in}{3.571741in}}%
\pgfpathlineto{\pgfqpoint{3.027398in}{3.571741in}}%
\pgfusepath{stroke}%
\end{pgfscope}%
\begin{pgfscope}%
\pgfpathrectangle{\pgfqpoint{0.647939in}{0.492442in}}{\pgfqpoint{3.079299in}{3.079299in}}%
\pgfusepath{clip}%
\pgfsetbuttcap%
\pgfsetroundjoin%
\pgfsetlinewidth{0.301125pt}%
\definecolor{currentstroke}{rgb}{0.500000,0.500000,0.500000}%
\pgfsetstrokecolor{currentstroke}%
\pgfsetstrokeopacity{0.300000}%
\pgfsetdash{}{0pt}%
\pgfpathmoveto{\pgfqpoint{0.647939in}{3.182086in}}%
\pgfpathlineto{\pgfqpoint{0.672073in}{3.184979in}}%
\pgfpathlineto{\pgfqpoint{0.739937in}{3.193725in}}%
\pgfpathlineto{\pgfqpoint{0.807637in}{3.203667in}}%
\pgfpathlineto{\pgfqpoint{0.875155in}{3.214777in}}%
\pgfpathlineto{\pgfqpoint{0.942482in}{3.226993in}}%
\pgfpathlineto{\pgfqpoint{1.009620in}{3.240215in}}%
\pgfpathlineto{\pgfqpoint{1.076581in}{3.254305in}}%
\pgfpathlineto{\pgfqpoint{1.143392in}{3.269092in}}%
\pgfpathlineto{\pgfqpoint{1.210093in}{3.284372in}}%
\pgfpathlineto{\pgfqpoint{1.276734in}{3.299911in}}%
\pgfpathlineto{\pgfqpoint{1.343372in}{3.315461in}}%
\pgfpathlineto{\pgfqpoint{1.410069in}{3.330759in}}%
\pgfpathlineto{\pgfqpoint{1.476881in}{3.345541in}}%
\pgfpathlineto{\pgfqpoint{1.543859in}{3.359547in}}%
\pgfpathlineto{\pgfqpoint{1.611042in}{3.372531in}}%
\pgfpathlineto{\pgfqpoint{1.678451in}{3.384282in}}%
\pgfpathlineto{\pgfqpoint{1.746088in}{3.394633in}}%
\pgfpathlineto{\pgfqpoint{1.813937in}{3.403485in}}%
\pgfpathlineto{\pgfqpoint{1.881967in}{3.410816in}}%
\pgfpathlineto{\pgfqpoint{1.950138in}{3.416697in}}%
\pgfpathlineto{\pgfqpoint{2.018409in}{3.421301in}}%
\pgfpathlineto{\pgfqpoint{2.086741in}{3.424917in}}%
\pgfpathlineto{\pgfqpoint{2.155103in}{3.427935in}}%
\pgfpathlineto{\pgfqpoint{2.223471in}{3.430832in}}%
\pgfpathlineto{\pgfqpoint{2.291817in}{3.434171in}}%
\pgfpathlineto{\pgfqpoint{2.360100in}{3.438574in}}%
\pgfpathlineto{\pgfqpoint{2.428246in}{3.444696in}}%
\pgfpathlineto{\pgfqpoint{2.496133in}{3.453167in}}%
\pgfpathlineto{\pgfqpoint{2.563590in}{3.464526in}}%
\pgfpathlineto{\pgfqpoint{2.630407in}{3.479169in}}%
\pgfpathlineto{\pgfqpoint{2.696360in}{3.497302in}}%
\pgfpathlineto{\pgfqpoint{2.761256in}{3.518925in}}%
\pgfpathlineto{\pgfqpoint{2.824962in}{3.543848in}}%
\pgfpathlineto{\pgfqpoint{2.887429in}{3.571741in}}%
\pgfpathlineto{\pgfqpoint{2.887429in}{3.571741in}}%
\pgfusepath{stroke}%
\end{pgfscope}%
\begin{pgfscope}%
\pgfpathrectangle{\pgfqpoint{0.647939in}{0.492442in}}{\pgfqpoint{3.079299in}{3.079299in}}%
\pgfusepath{clip}%
\pgfsetbuttcap%
\pgfsetroundjoin%
\pgfsetlinewidth{0.301125pt}%
\definecolor{currentstroke}{rgb}{0.500000,0.500000,0.500000}%
\pgfsetstrokecolor{currentstroke}%
\pgfsetstrokeopacity{0.300000}%
\pgfsetdash{}{0pt}%
\pgfpathmoveto{\pgfqpoint{0.647939in}{3.273441in}}%
\pgfpathlineto{\pgfqpoint{0.650608in}{3.273741in}}%
\pgfpathlineto{\pgfqpoint{0.718536in}{3.281983in}}%
\pgfpathlineto{\pgfqpoint{0.786312in}{3.291390in}}%
\pgfpathlineto{\pgfqpoint{0.853918in}{3.301948in}}%
\pgfpathlineto{\pgfqpoint{0.921345in}{3.313604in}}%
\pgfpathlineto{\pgfqpoint{0.988589in}{3.326270in}}%
\pgfpathlineto{\pgfqpoint{1.055663in}{3.339816in}}%
\pgfpathlineto{\pgfqpoint{1.122588in}{3.354077in}}%
\pgfpathlineto{\pgfqpoint{1.189400in}{3.368860in}}%
\pgfpathlineto{\pgfqpoint{1.256145in}{3.383946in}}%
\pgfpathlineto{\pgfqpoint{1.322876in}{3.399098in}}%
\pgfpathlineto{\pgfqpoint{1.389648in}{3.414062in}}%
\pgfpathlineto{\pgfqpoint{1.456519in}{3.428579in}}%
\pgfpathlineto{\pgfqpoint{1.523538in}{3.442391in}}%
\pgfpathlineto{\pgfqpoint{1.590744in}{3.455255in}}%
\pgfpathlineto{\pgfqpoint{1.658161in}{3.466959in}}%
\pgfpathlineto{\pgfqpoint{1.725794in}{3.477333in}}%
\pgfpathlineto{\pgfqpoint{1.793633in}{3.486265in}}%
\pgfpathlineto{\pgfqpoint{1.861651in}{3.493713in}}%
\pgfpathlineto{\pgfqpoint{1.929811in}{3.499730in}}%
\pgfpathlineto{\pgfqpoint{1.998073in}{3.504462in}}%
\pgfpathlineto{\pgfqpoint{2.066401in}{3.508152in}}%
\pgfpathlineto{\pgfqpoint{2.134764in}{3.511139in}}%
\pgfpathlineto{\pgfqpoint{2.203139in}{3.513853in}}%
\pgfpathlineto{\pgfqpoint{2.271503in}{3.516815in}}%
\pgfpathlineto{\pgfqpoint{2.339824in}{3.520605in}}%
\pgfpathlineto{\pgfqpoint{2.408047in}{3.525824in}}%
\pgfpathlineto{\pgfqpoint{2.476080in}{3.533067in}}%
\pgfpathlineto{\pgfqpoint{2.543787in}{3.542872in}}%
\pgfpathlineto{\pgfqpoint{2.610987in}{3.555672in}}%
\pgfpathlineto{\pgfqpoint{2.677477in}{3.571741in}}%
\pgfpathlineto{\pgfqpoint{2.677477in}{3.571741in}}%
\pgfusepath{stroke}%
\end{pgfscope}%
\begin{pgfscope}%
\pgfpathrectangle{\pgfqpoint{0.647939in}{0.492442in}}{\pgfqpoint{3.079299in}{3.079299in}}%
\pgfusepath{clip}%
\pgfsetbuttcap%
\pgfsetroundjoin%
\pgfsetlinewidth{0.301125pt}%
\definecolor{currentstroke}{rgb}{0.500000,0.500000,0.500000}%
\pgfsetstrokecolor{currentstroke}%
\pgfsetstrokeopacity{0.300000}%
\pgfsetdash{}{0pt}%
\pgfpathmoveto{\pgfqpoint{1.921160in}{3.539614in}}%
\pgfpathlineto{\pgfqpoint{1.989420in}{3.544387in}}%
\pgfpathlineto{\pgfqpoint{2.057746in}{3.548102in}}%
\pgfpathlineto{\pgfqpoint{2.126109in}{3.551085in}}%
\pgfpathlineto{\pgfqpoint{2.194486in}{3.553745in}}%
\pgfpathlineto{\pgfqpoint{2.262856in}{3.556573in}}%
\pgfpathlineto{\pgfqpoint{2.331191in}{3.560122in}}%
\pgfpathlineto{\pgfqpoint{2.399441in}{3.564983in}}%
\pgfpathlineto{\pgfqpoint{2.467525in}{3.571741in}}%
\pgfpathlineto{\pgfqpoint{2.467525in}{3.571741in}}%
\pgfusepath{stroke}%
\end{pgfscope}%
\begin{pgfscope}%
\pgfpathrectangle{\pgfqpoint{0.647939in}{0.492442in}}{\pgfqpoint{3.079299in}{3.079299in}}%
\pgfusepath{clip}%
\pgfsetbuttcap%
\pgfsetroundjoin%
\pgfsetlinewidth{0.301125pt}%
\definecolor{currentstroke}{rgb}{0.500000,0.500000,0.500000}%
\pgfsetstrokecolor{currentstroke}%
\pgfsetstrokeopacity{0.300000}%
\pgfsetdash{}{0pt}%
\pgfpathmoveto{\pgfqpoint{0.647939in}{3.359223in}}%
\pgfpathlineto{\pgfqpoint{0.693610in}{3.364785in}}%
\pgfpathlineto{\pgfqpoint{0.761461in}{3.373632in}}%
\pgfpathlineto{\pgfqpoint{0.829155in}{3.383613in}}%
\pgfpathlineto{\pgfqpoint{0.896680in}{3.394689in}}%
\pgfpathlineto{\pgfqpoint{0.964030in}{3.406778in}}%
\pgfpathlineto{\pgfqpoint{1.031214in}{3.419763in}}%
\pgfpathlineto{\pgfqpoint{1.098250in}{3.433495in}}%
\pgfpathlineto{\pgfqpoint{1.165168in}{3.447794in}}%
\pgfpathlineto{\pgfqpoint{1.232008in}{3.462454in}}%
\pgfpathlineto{\pgfqpoint{1.298819in}{3.477245in}}%
\pgfpathlineto{\pgfqpoint{1.365655in}{3.491920in}}%
\pgfpathlineto{\pgfqpoint{1.432572in}{3.506226in}}%
\pgfpathlineto{\pgfqpoint{1.499617in}{3.519908in}}%
\pgfpathlineto{\pgfqpoint{1.566832in}{3.532730in}}%
\pgfpathlineto{\pgfqpoint{1.634243in}{3.544474in}}%
\pgfpathlineto{\pgfqpoint{1.701859in}{3.554964in}}%
\pgfpathlineto{\pgfqpoint{1.769675in}{3.564072in}}%
\pgfpathlineto{\pgfqpoint{1.837668in}{3.571741in}}%
\pgfpathlineto{\pgfqpoint{1.837668in}{3.571741in}}%
\pgfusepath{stroke}%
\end{pgfscope}%
\begin{pgfscope}%
\pgfpathrectangle{\pgfqpoint{0.647939in}{0.492442in}}{\pgfqpoint{3.079299in}{3.079299in}}%
\pgfusepath{clip}%
\pgfsetbuttcap%
\pgfsetroundjoin%
\pgfsetlinewidth{0.301125pt}%
\definecolor{currentstroke}{rgb}{0.500000,0.500000,0.500000}%
\pgfsetstrokecolor{currentstroke}%
\pgfsetstrokeopacity{0.300000}%
\pgfsetdash{}{0pt}%
\pgfpathmoveto{\pgfqpoint{0.647939in}{3.430531in}}%
\pgfpathlineto{\pgfqpoint{0.678104in}{3.434048in}}%
\pgfpathlineto{\pgfqpoint{0.746002in}{3.442535in}}%
\pgfpathlineto{\pgfqpoint{0.813752in}{3.452137in}}%
\pgfpathlineto{\pgfqpoint{0.881339in}{3.462819in}}%
\pgfpathlineto{\pgfqpoint{0.948760in}{3.474511in}}%
\pgfpathlineto{\pgfqpoint{1.016018in}{3.487107in}}%
\pgfpathlineto{\pgfqpoint{1.083129in}{3.500467in}}%
\pgfpathlineto{\pgfqpoint{1.150121in}{3.514414in}}%
\pgfpathlineto{\pgfqpoint{1.217032in}{3.528744in}}%
\pgfpathlineto{\pgfqpoint{1.283909in}{3.543235in}}%
\pgfpathlineto{\pgfqpoint{1.350802in}{3.557650in}}%
\pgfpathlineto{\pgfqpoint{1.417764in}{3.571741in}}%
\pgfpathlineto{\pgfqpoint{1.417764in}{3.571741in}}%
\pgfusepath{stroke}%
\end{pgfscope}%
\begin{pgfscope}%
\pgfpathrectangle{\pgfqpoint{0.647939in}{0.492442in}}{\pgfqpoint{3.079299in}{3.079299in}}%
\pgfusepath{clip}%
\pgfsetbuttcap%
\pgfsetroundjoin%
\pgfsetlinewidth{0.301125pt}%
\definecolor{currentstroke}{rgb}{0.500000,0.500000,0.500000}%
\pgfsetstrokecolor{currentstroke}%
\pgfsetstrokeopacity{0.300000}%
\pgfsetdash{}{0pt}%
\pgfpathmoveto{\pgfqpoint{0.647939in}{3.506000in}}%
\pgfpathlineto{\pgfqpoint{0.662448in}{3.507615in}}%
\pgfpathlineto{\pgfqpoint{0.730390in}{3.515741in}}%
\pgfpathlineto{\pgfqpoint{0.798192in}{3.524966in}}%
\pgfpathlineto{\pgfqpoint{0.865839in}{3.535262in}}%
\pgfpathlineto{\pgfqpoint{0.933327in}{3.546564in}}%
\pgfpathlineto{\pgfqpoint{1.000657in}{3.558768in}}%
\pgfpathlineto{\pgfqpoint{1.067843in}{3.571741in}}%
\pgfpathlineto{\pgfqpoint{1.067843in}{3.571741in}}%
\pgfusepath{stroke}%
\end{pgfscope}%
\begin{pgfscope}%
\pgfpathrectangle{\pgfqpoint{0.647939in}{0.492442in}}{\pgfqpoint{3.079299in}{3.079299in}}%
\pgfusepath{clip}%
\pgfsetbuttcap%
\pgfsetroundjoin%
\pgfsetlinewidth{0.301125pt}%
\definecolor{currentstroke}{rgb}{0.500000,0.500000,0.500000}%
\pgfsetstrokecolor{currentstroke}%
\pgfsetstrokeopacity{0.300000}%
\pgfsetdash{}{0pt}%
\pgfpathmoveto{\pgfqpoint{0.647939in}{2.871901in}}%
\pgfpathlineto{\pgfqpoint{0.647939in}{2.871901in}}%
\pgfpathlineto{\pgfqpoint{0.715799in}{2.880679in}}%
\pgfpathlineto{\pgfqpoint{0.783478in}{2.890758in}}%
\pgfpathlineto{\pgfqpoint{0.850950in}{2.902136in}}%
\pgfpathlineto{\pgfqpoint{0.918198in}{2.914775in}}%
\pgfpathlineto{\pgfqpoint{0.985214in}{2.928597in}}%
\pgfpathlineto{\pgfqpoint{1.052002in}{2.943481in}}%
\pgfpathlineto{\pgfqpoint{1.118584in}{2.959264in}}%
\pgfpathlineto{\pgfqpoint{1.184998in}{2.975747in}}%
\pgfpathlineto{\pgfqpoint{1.251292in}{2.992703in}}%
\pgfpathlineto{\pgfqpoint{1.317531in}{3.009878in}}%
\pgfpathlineto{\pgfqpoint{1.383783in}{3.026999in}}%
\pgfpathlineto{\pgfqpoint{1.450122in}{3.043779in}}%
\pgfpathlineto{\pgfqpoint{1.516617in}{3.059926in}}%
\pgfpathlineto{\pgfqpoint{1.583328in}{3.075153in}}%
\pgfpathlineto{\pgfqpoint{1.650297in}{3.089194in}}%
\pgfpathlineto{\pgfqpoint{1.717546in}{3.101819in}}%
\pgfpathlineto{\pgfqpoint{1.785072in}{3.112857in}}%
\pgfusepath{stroke}%
\end{pgfscope}%
\begin{pgfscope}%
\pgfpathrectangle{\pgfqpoint{0.647939in}{0.492442in}}{\pgfqpoint{3.079299in}{3.079299in}}%
\pgfusepath{clip}%
\pgfsetbuttcap%
\pgfsetroundjoin%
\pgfsetlinewidth{0.301125pt}%
\definecolor{currentstroke}{rgb}{0.500000,0.500000,0.500000}%
\pgfsetstrokecolor{currentstroke}%
\pgfsetstrokeopacity{0.300000}%
\pgfsetdash{}{0pt}%
\pgfpathmoveto{\pgfqpoint{0.647939in}{2.661948in}}%
\pgfpathlineto{\pgfqpoint{0.647939in}{2.661948in}}%
\pgfpathlineto{\pgfqpoint{0.715754in}{2.671065in}}%
\pgfpathlineto{\pgfqpoint{0.783369in}{2.681563in}}%
\pgfpathlineto{\pgfqpoint{0.850753in}{2.693450in}}%
\pgfpathlineto{\pgfqpoint{0.917882in}{2.706699in}}%
\pgfpathlineto{\pgfqpoint{0.984746in}{2.721236in}}%
\pgfpathlineto{\pgfqpoint{1.051344in}{2.736946in}}%
\pgfpathlineto{\pgfqpoint{1.117696in}{2.753669in}}%
\pgfpathlineto{\pgfqpoint{1.183839in}{2.771205in}}%
\pgfpathlineto{\pgfqpoint{1.249825in}{2.789323in}}%
\pgfpathlineto{\pgfqpoint{1.315722in}{2.807764in}}%
\pgfpathlineto{\pgfqpoint{1.381608in}{2.826244in}}%
\pgfpathlineto{\pgfqpoint{1.447567in}{2.844464in}}%
\pgfpathlineto{\pgfqpoint{1.513679in}{2.862111in}}%
\pgfpathlineto{\pgfqpoint{1.580021in}{2.878875in}}%
\pgfpathlineto{\pgfqpoint{1.646648in}{2.894459in}}%
\pgfpathlineto{\pgfqpoint{1.713595in}{2.908596in}}%
\pgfpathlineto{\pgfqpoint{1.780869in}{2.921077in}}%
\pgfpathlineto{\pgfqpoint{1.848450in}{2.931768in}}%
\pgfpathlineto{\pgfqpoint{1.916294in}{2.940641in}}%
\pgfpathlineto{\pgfqpoint{1.984342in}{2.947793in}}%
\pgfpathlineto{\pgfqpoint{2.052531in}{2.953482in}}%
\pgfpathlineto{\pgfqpoint{2.120800in}{2.958137in}}%
\pgfpathlineto{\pgfqpoint{2.189099in}{2.962349in}}%
\pgfpathlineto{\pgfqpoint{2.257377in}{2.966871in}}%
\pgfpathlineto{\pgfqpoint{2.325560in}{2.972607in}}%
\pgfpathlineto{\pgfqpoint{2.393508in}{2.980566in}}%
\pgfpathlineto{\pgfqpoint{2.460983in}{2.991765in}}%
\pgfpathlineto{\pgfqpoint{2.527631in}{3.007060in}}%
\pgfpathlineto{\pgfqpoint{2.593028in}{3.026991in}}%
\pgfpathlineto{\pgfqpoint{2.656778in}{3.051679in}}%
\pgfpathlineto{\pgfqpoint{2.718615in}{3.080844in}}%
\pgfpathlineto{\pgfqpoint{2.778458in}{3.113927in}}%
\pgfpathlineto{\pgfqpoint{2.836404in}{3.150252in}}%
\pgfusepath{stroke}%
\end{pgfscope}%
\begin{pgfscope}%
\pgfpathrectangle{\pgfqpoint{0.647939in}{0.492442in}}{\pgfqpoint{3.079299in}{3.079299in}}%
\pgfusepath{clip}%
\pgfsetbuttcap%
\pgfsetroundjoin%
\pgfsetlinewidth{0.301125pt}%
\definecolor{currentstroke}{rgb}{0.500000,0.500000,0.500000}%
\pgfsetstrokecolor{currentstroke}%
\pgfsetstrokeopacity{0.300000}%
\pgfsetdash{}{0pt}%
\pgfpathmoveto{\pgfqpoint{0.647939in}{2.591964in}}%
\pgfpathlineto{\pgfqpoint{0.647939in}{2.591964in}}%
\pgfpathlineto{\pgfqpoint{0.715738in}{2.601200in}}%
\pgfpathlineto{\pgfqpoint{0.783329in}{2.611844in}}%
\pgfpathlineto{\pgfqpoint{0.850681in}{2.623912in}}%
\pgfpathlineto{\pgfqpoint{0.917768in}{2.637376in}}%
\pgfpathlineto{\pgfqpoint{0.984575in}{2.652167in}}%
\pgfpathlineto{\pgfqpoint{1.051103in}{2.668173in}}%
\pgfpathlineto{\pgfqpoint{1.117369in}{2.685233in}}%
\pgfpathlineto{\pgfqpoint{1.183410in}{2.703148in}}%
\pgfpathlineto{\pgfqpoint{1.249279in}{2.721686in}}%
\pgfpathlineto{\pgfqpoint{1.315045in}{2.740587in}}%
\pgfpathlineto{\pgfqpoint{1.380790in}{2.759565in}}%
\pgfpathlineto{\pgfqpoint{1.446600in}{2.778314in}}%
\pgfpathlineto{\pgfqpoint{1.512562in}{2.796517in}}%
\pgfpathlineto{\pgfqpoint{1.578755in}{2.813856in}}%
\pgfpathlineto{\pgfqpoint{1.645243in}{2.830022in}}%
\pgfpathlineto{\pgfqpoint{1.712066in}{2.844737in}}%
\pgfpathlineto{\pgfqpoint{1.779234in}{2.857775in}}%
\pgfpathlineto{\pgfqpoint{1.846730in}{2.868988in}}%
\pgfpathlineto{\pgfqpoint{1.914510in}{2.878332in}}%
\pgfpathlineto{\pgfqpoint{1.982513in}{2.885897in}}%
\pgfpathlineto{\pgfqpoint{2.050672in}{2.891936in}}%
\pgfpathlineto{\pgfqpoint{2.118920in}{2.896885in}}%
\pgfpathlineto{\pgfqpoint{2.187202in}{2.901365in}}%
\pgfpathlineto{\pgfqpoint{2.255460in}{2.906172in}}%
\pgfusepath{stroke}%
\end{pgfscope}%
\begin{pgfscope}%
\pgfpathrectangle{\pgfqpoint{0.647939in}{0.492442in}}{\pgfqpoint{3.079299in}{3.079299in}}%
\pgfusepath{clip}%
\pgfsetbuttcap%
\pgfsetroundjoin%
\pgfsetlinewidth{0.301125pt}%
\definecolor{currentstroke}{rgb}{0.500000,0.500000,0.500000}%
\pgfsetstrokecolor{currentstroke}%
\pgfsetstrokeopacity{0.300000}%
\pgfsetdash{}{0pt}%
\pgfpathmoveto{\pgfqpoint{0.647939in}{2.521980in}}%
\pgfpathlineto{\pgfqpoint{0.647939in}{2.521980in}}%
\pgfpathlineto{\pgfqpoint{0.715721in}{2.531337in}}%
\pgfpathlineto{\pgfqpoint{0.783288in}{2.542133in}}%
\pgfpathlineto{\pgfqpoint{0.850606in}{2.554386in}}%
\pgfpathlineto{\pgfqpoint{0.917647in}{2.568073in}}%
\pgfpathlineto{\pgfqpoint{0.984396in}{2.583127in}}%
\pgfpathlineto{\pgfqpoint{1.050849in}{2.599438in}}%
\pgfpathlineto{\pgfqpoint{1.117024in}{2.616848in}}%
\pgfpathlineto{\pgfqpoint{1.182956in}{2.635157in}}%
\pgfpathlineto{\pgfqpoint{1.248700in}{2.654134in}}%
\pgfpathlineto{\pgfqpoint{1.314327in}{2.673517in}}%
\pgfpathlineto{\pgfqpoint{1.379918in}{2.693016in}}%
\pgfpathlineto{\pgfqpoint{1.445567in}{2.712322in}}%
\pgfpathlineto{\pgfqpoint{1.511364in}{2.731113in}}%
\pgfpathlineto{\pgfqpoint{1.577395in}{2.749061in}}%
\pgfpathlineto{\pgfqpoint{1.643729in}{2.765850in}}%
\pgfpathlineto{\pgfqpoint{1.710412in}{2.781185in}}%
\pgfpathlineto{\pgfqpoint{1.777460in}{2.794825in}}%
\pgfpathlineto{\pgfqpoint{1.844859in}{2.806605in}}%
\pgfpathlineto{\pgfqpoint{1.912565in}{2.816467in}}%
\pgfpathlineto{\pgfqpoint{1.980515in}{2.824489in}}%
\pgfpathlineto{\pgfqpoint{2.048638in}{2.830916in}}%
\pgfpathlineto{\pgfqpoint{2.116861in}{2.836194in}}%
\pgfusepath{stroke}%
\end{pgfscope}%
\begin{pgfscope}%
\pgfpathrectangle{\pgfqpoint{0.647939in}{0.492442in}}{\pgfqpoint{3.079299in}{3.079299in}}%
\pgfusepath{clip}%
\pgfsetbuttcap%
\pgfsetroundjoin%
\pgfsetlinewidth{0.301125pt}%
\definecolor{currentstroke}{rgb}{0.500000,0.500000,0.500000}%
\pgfsetstrokecolor{currentstroke}%
\pgfsetstrokeopacity{0.300000}%
\pgfsetdash{}{0pt}%
\pgfpathmoveto{\pgfqpoint{0.647939in}{2.382012in}}%
\pgfpathlineto{\pgfqpoint{0.647939in}{2.382012in}}%
\pgfpathlineto{\pgfqpoint{0.715685in}{2.391622in}}%
\pgfpathlineto{\pgfqpoint{0.783201in}{2.402734in}}%
\pgfpathlineto{\pgfqpoint{0.850446in}{2.415375in}}%
\pgfpathlineto{\pgfqpoint{0.917390in}{2.429529in}}%
\pgfpathlineto{\pgfqpoint{0.984010in}{2.445137in}}%
\pgfpathlineto{\pgfqpoint{1.050302in}{2.462092in}}%
\pgfpathlineto{\pgfqpoint{1.116277in}{2.480243in}}%
\pgfpathlineto{\pgfqpoint{1.181970in}{2.499391in}}%
\pgfpathlineto{\pgfqpoint{1.247437in}{2.519303in}}%
\pgfpathlineto{\pgfqpoint{1.312749in}{2.539716in}}%
\pgfusepath{stroke}%
\end{pgfscope}%
\begin{pgfscope}%
\pgfpathrectangle{\pgfqpoint{0.647939in}{0.492442in}}{\pgfqpoint{3.079299in}{3.079299in}}%
\pgfusepath{clip}%
\pgfsetbuttcap%
\pgfsetroundjoin%
\pgfsetlinewidth{0.301125pt}%
\definecolor{currentstroke}{rgb}{0.500000,0.500000,0.500000}%
\pgfsetstrokecolor{currentstroke}%
\pgfsetstrokeopacity{0.300000}%
\pgfsetdash{}{0pt}%
\pgfpathmoveto{\pgfqpoint{0.647939in}{2.312028in}}%
\pgfpathlineto{\pgfqpoint{0.647939in}{2.312028in}}%
\pgfpathlineto{\pgfqpoint{0.715666in}{2.321769in}}%
\pgfpathlineto{\pgfqpoint{0.783154in}{2.333047in}}%
\pgfpathlineto{\pgfqpoint{0.850361in}{2.345891in}}%
\pgfpathlineto{\pgfqpoint{0.917252in}{2.360290in}}%
\pgfpathlineto{\pgfqpoint{0.983803in}{2.376189in}}%
\pgfpathlineto{\pgfqpoint{1.050006in}{2.393485in}}%
\pgfpathlineto{\pgfqpoint{1.115872in}{2.412028in}}%
\pgfpathlineto{\pgfqpoint{1.181433in}{2.431622in}}%
\pgfpathlineto{\pgfqpoint{1.246745in}{2.452034in}}%
\pgfpathlineto{\pgfqpoint{1.311883in}{2.472999in}}%
\pgfpathlineto{\pgfqpoint{1.376937in}{2.494223in}}%
\pgfpathlineto{\pgfqpoint{1.442010in}{2.515388in}}%
\pgfpathlineto{\pgfqpoint{1.507211in}{2.536155in}}%
\pgfpathlineto{\pgfqpoint{1.572643in}{2.556175in}}%
\pgfpathlineto{\pgfqpoint{1.638400in}{2.575098in}}%
\pgfpathlineto{\pgfqpoint{1.704549in}{2.592591in}}%
\pgfpathlineto{\pgfqpoint{1.771130in}{2.608359in}}%
\pgfpathlineto{\pgfqpoint{1.838140in}{2.622177in}}%
\pgfpathlineto{\pgfqpoint{1.905541in}{2.633931in}}%
\pgfpathlineto{\pgfqpoint{1.973267in}{2.643652in}}%
\pgfpathlineto{\pgfqpoint{2.041232in}{2.651562in}}%
\pgfpathlineto{\pgfqpoint{2.109344in}{2.658114in}}%
\pgfpathlineto{\pgfqpoint{2.177515in}{2.664044in}}%
\pgfpathlineto{\pgfqpoint{2.245646in}{2.670390in}}%
\pgfpathlineto{\pgfqpoint{2.313584in}{2.678453in}}%
\pgfpathlineto{\pgfqpoint{2.381039in}{2.689733in}}%
\pgfpathlineto{\pgfqpoint{2.447511in}{2.705699in}}%
\pgfpathlineto{\pgfqpoint{2.512313in}{2.727406in}}%
\pgfpathlineto{\pgfqpoint{2.574767in}{2.755140in}}%
\pgfpathlineto{\pgfqpoint{2.634474in}{2.788400in}}%
\pgfpathlineto{\pgfqpoint{2.691422in}{2.826215in}}%
\pgfpathlineto{\pgfqpoint{2.745901in}{2.867527in}}%
\pgfusepath{stroke}%
\end{pgfscope}%
\begin{pgfscope}%
\pgfpathrectangle{\pgfqpoint{0.647939in}{0.492442in}}{\pgfqpoint{3.079299in}{3.079299in}}%
\pgfusepath{clip}%
\pgfsetbuttcap%
\pgfsetroundjoin%
\pgfsetlinewidth{0.301125pt}%
\definecolor{currentstroke}{rgb}{0.500000,0.500000,0.500000}%
\pgfsetstrokecolor{currentstroke}%
\pgfsetstrokeopacity{0.300000}%
\pgfsetdash{}{0pt}%
\pgfpathmoveto{\pgfqpoint{0.647939in}{2.242044in}}%
\pgfpathlineto{\pgfqpoint{0.647939in}{2.242044in}}%
\pgfpathlineto{\pgfqpoint{0.715647in}{2.251920in}}%
\pgfpathlineto{\pgfqpoint{0.783105in}{2.263368in}}%
\pgfpathlineto{\pgfqpoint{0.850272in}{2.276421in}}%
\pgfpathlineto{\pgfqpoint{0.917107in}{2.291074in}}%
\pgfpathlineto{\pgfqpoint{0.983585in}{2.307275in}}%
\pgfpathlineto{\pgfqpoint{1.049694in}{2.324924in}}%
\pgfpathlineto{\pgfqpoint{1.115443in}{2.343875in}}%
\pgfpathlineto{\pgfqpoint{1.180864in}{2.363934in}}%
\pgfpathlineto{\pgfqpoint{1.246010in}{2.384868in}}%
\pgfpathlineto{\pgfqpoint{1.310959in}{2.406412in}}%
\pgfpathlineto{\pgfqpoint{1.375802in}{2.428271in}}%
\pgfpathlineto{\pgfqpoint{1.440647in}{2.450125in}}%
\pgfpathlineto{\pgfqpoint{1.505608in}{2.471630in}}%
\pgfusepath{stroke}%
\end{pgfscope}%
\begin{pgfscope}%
\pgfpathrectangle{\pgfqpoint{0.647939in}{0.492442in}}{\pgfqpoint{3.079299in}{3.079299in}}%
\pgfusepath{clip}%
\pgfsetbuttcap%
\pgfsetroundjoin%
\pgfsetlinewidth{0.301125pt}%
\definecolor{currentstroke}{rgb}{0.500000,0.500000,0.500000}%
\pgfsetstrokecolor{currentstroke}%
\pgfsetstrokeopacity{0.300000}%
\pgfsetdash{}{0pt}%
\pgfpathmoveto{\pgfqpoint{0.647939in}{2.172060in}}%
\pgfpathlineto{\pgfqpoint{0.647939in}{2.172060in}}%
\pgfpathlineto{\pgfqpoint{0.715626in}{2.182075in}}%
\pgfpathlineto{\pgfqpoint{0.783055in}{2.193697in}}%
\pgfpathlineto{\pgfqpoint{0.850178in}{2.206967in}}%
\pgfpathlineto{\pgfqpoint{0.916955in}{2.221882in}}%
\pgfpathlineto{\pgfqpoint{0.983356in}{2.238396in}}%
\pgfpathlineto{\pgfqpoint{1.049365in}{2.256412in}}%
\pgfpathlineto{\pgfqpoint{1.114990in}{2.275789in}}%
\pgfpathlineto{\pgfqpoint{1.180260in}{2.296332in}}%
\pgfpathlineto{\pgfqpoint{1.245228in}{2.317812in}}%
\pgfpathlineto{\pgfqpoint{1.309971in}{2.339963in}}%
\pgfpathlineto{\pgfqpoint{1.374586in}{2.362491in}}%
\pgfpathlineto{\pgfqpoint{1.439181in}{2.385071in}}%
\pgfpathlineto{\pgfqpoint{1.503878in}{2.407359in}}%
\pgfpathlineto{\pgfqpoint{1.568795in}{2.428994in}}%
\pgfpathlineto{\pgfqpoint{1.634041in}{2.449607in}}%
\pgfpathlineto{\pgfqpoint{1.699706in}{2.468838in}}%
\pgfpathlineto{\pgfqpoint{1.765848in}{2.486354in}}%
\pgfpathlineto{\pgfqpoint{1.832481in}{2.501885in}}%
\pgfpathlineto{\pgfqpoint{1.899579in}{2.515264in}}%
\pgfpathlineto{\pgfqpoint{1.967072in}{2.526482in}}%
\pgfpathlineto{\pgfqpoint{2.034865in}{2.535733in}}%
\pgfpathlineto{\pgfqpoint{2.102851in}{2.543476in}}%
\pgfpathlineto{\pgfqpoint{2.170919in}{2.550495in}}%
\pgfpathlineto{\pgfqpoint{2.238935in}{2.557969in}}%
\pgfpathlineto{\pgfqpoint{2.306680in}{2.567473in}}%
\pgfpathlineto{\pgfqpoint{2.373735in}{2.580829in}}%
\pgfpathlineto{\pgfqpoint{2.439384in}{2.599748in}}%
\pgfpathlineto{\pgfqpoint{2.502734in}{2.625233in}}%
\pgfpathlineto{\pgfqpoint{2.563082in}{2.657174in}}%
\pgfpathlineto{\pgfqpoint{2.620234in}{2.694581in}}%
\pgfpathlineto{\pgfqpoint{2.674473in}{2.736170in}}%
\pgfpathlineto{\pgfqpoint{2.726303in}{2.780775in}}%
\pgfpathlineto{\pgfqpoint{2.776253in}{2.827488in}}%
\pgfpathlineto{\pgfqpoint{2.824813in}{2.875667in}}%
\pgfpathlineto{\pgfqpoint{2.872375in}{2.924840in}}%
\pgfpathlineto{\pgfqpoint{2.919255in}{2.974667in}}%
\pgfusepath{stroke}%
\end{pgfscope}%
\begin{pgfscope}%
\pgfpathrectangle{\pgfqpoint{0.647939in}{0.492442in}}{\pgfqpoint{3.079299in}{3.079299in}}%
\pgfusepath{clip}%
\pgfsetbuttcap%
\pgfsetroundjoin%
\pgfsetlinewidth{0.301125pt}%
\definecolor{currentstroke}{rgb}{0.500000,0.500000,0.500000}%
\pgfsetstrokecolor{currentstroke}%
\pgfsetstrokeopacity{0.300000}%
\pgfsetdash{}{0pt}%
\pgfpathmoveto{\pgfqpoint{0.647939in}{2.032092in}}%
\pgfpathlineto{\pgfqpoint{0.647939in}{2.032092in}}%
\pgfpathlineto{\pgfqpoint{0.715582in}{2.042397in}}%
\pgfpathlineto{\pgfqpoint{0.782946in}{2.054385in}}%
\pgfpathlineto{\pgfqpoint{0.849977in}{2.068108in}}%
\pgfpathlineto{\pgfqpoint{0.916628in}{2.083576in}}%
\pgfpathlineto{\pgfqpoint{0.982859in}{2.100750in}}%
\pgfpathlineto{\pgfqpoint{1.048651in}{2.119545in}}%
\pgfpathlineto{\pgfqpoint{1.114001in}{2.139825in}}%
\pgfpathlineto{\pgfqpoint{1.178935in}{2.161403in}}%
\pgfpathlineto{\pgfqpoint{1.243505in}{2.184051in}}%
\pgfpathlineto{\pgfqpoint{1.307786in}{2.207509in}}%
\pgfpathlineto{\pgfqpoint{1.371879in}{2.231479in}}%
\pgfpathlineto{\pgfqpoint{1.435900in}{2.255640in}}%
\pgfpathlineto{\pgfqpoint{1.499981in}{2.279641in}}%
\pgfpathlineto{\pgfqpoint{1.564257in}{2.303112in}}%
\pgfpathlineto{\pgfqpoint{1.628857in}{2.325671in}}%
\pgfpathlineto{\pgfqpoint{1.693893in}{2.346934in}}%
\pgfpathlineto{\pgfqpoint{1.759447in}{2.366532in}}%
\pgfpathlineto{\pgfqpoint{1.825559in}{2.384148in}}%
\pgfpathlineto{\pgfqpoint{1.892221in}{2.399550in}}%
\pgfpathlineto{\pgfqpoint{1.959369in}{2.412669in}}%
\pgfpathlineto{\pgfqpoint{2.026900in}{2.423664in}}%
\pgfpathlineto{\pgfqpoint{2.094687in}{2.432989in}}%
\pgfpathlineto{\pgfqpoint{2.162584in}{2.441496in}}%
\pgfpathlineto{\pgfqpoint{2.230410in}{2.450527in}}%
\pgfpathlineto{\pgfqpoint{2.297850in}{2.461966in}}%
\pgfpathlineto{\pgfqpoint{2.364269in}{2.478080in}}%
\pgfpathlineto{\pgfqpoint{2.428637in}{2.500827in}}%
\pgfpathlineto{\pgfqpoint{2.489878in}{2.530909in}}%
\pgfusepath{stroke}%
\end{pgfscope}%
\begin{pgfscope}%
\pgfpathrectangle{\pgfqpoint{0.647939in}{0.492442in}}{\pgfqpoint{3.079299in}{3.079299in}}%
\pgfusepath{clip}%
\pgfsetbuttcap%
\pgfsetroundjoin%
\pgfsetlinewidth{0.301125pt}%
\definecolor{currentstroke}{rgb}{0.500000,0.500000,0.500000}%
\pgfsetstrokecolor{currentstroke}%
\pgfsetstrokeopacity{0.300000}%
\pgfsetdash{}{0pt}%
\pgfpathmoveto{\pgfqpoint{0.647939in}{1.962108in}}%
\pgfpathlineto{\pgfqpoint{0.647939in}{1.962108in}}%
\pgfpathlineto{\pgfqpoint{0.715558in}{1.972564in}}%
\pgfpathlineto{\pgfqpoint{0.782888in}{1.984744in}}%
\pgfpathlineto{\pgfqpoint{0.849869in}{1.998705in}}%
\pgfpathlineto{\pgfqpoint{0.916451in}{2.014464in}}%
\pgfpathlineto{\pgfqpoint{0.982591in}{2.031987in}}%
\pgfpathlineto{\pgfqpoint{1.048262in}{2.051195in}}%
\pgfpathlineto{\pgfqpoint{1.113461in}{2.071954in}}%
\pgfpathlineto{\pgfqpoint{1.178209in}{2.094084in}}%
\pgfpathlineto{\pgfqpoint{1.242555in}{2.117359in}}%
\pgfusepath{stroke}%
\end{pgfscope}%
\begin{pgfscope}%
\pgfpathrectangle{\pgfqpoint{0.647939in}{0.492442in}}{\pgfqpoint{3.079299in}{3.079299in}}%
\pgfusepath{clip}%
\pgfsetbuttcap%
\pgfsetroundjoin%
\pgfsetlinewidth{0.301125pt}%
\definecolor{currentstroke}{rgb}{0.500000,0.500000,0.500000}%
\pgfsetstrokecolor{currentstroke}%
\pgfsetstrokeopacity{0.300000}%
\pgfsetdash{}{0pt}%
\pgfpathmoveto{\pgfqpoint{0.647939in}{1.892124in}}%
\pgfpathlineto{\pgfqpoint{0.647939in}{1.892124in}}%
\pgfpathlineto{\pgfqpoint{0.715534in}{1.902736in}}%
\pgfpathlineto{\pgfqpoint{0.782827in}{1.915113in}}%
\pgfpathlineto{\pgfqpoint{0.849756in}{1.929321in}}%
\pgfpathlineto{\pgfqpoint{0.916265in}{1.945381in}}%
\pgfpathlineto{\pgfqpoint{0.982307in}{1.963267in}}%
\pgfpathlineto{\pgfqpoint{1.047851in}{1.982904in}}%
\pgfpathlineto{\pgfqpoint{1.112887in}{2.004164in}}%
\pgfpathlineto{\pgfqpoint{1.177435in}{2.026871in}}%
\pgfpathlineto{\pgfqpoint{1.241539in}{2.050802in}}%
\pgfpathlineto{\pgfqpoint{1.305276in}{2.075700in}}%
\pgfpathlineto{\pgfqpoint{1.368746in}{2.101272in}}%
\pgfpathlineto{\pgfqpoint{1.432073in}{2.127197in}}%
\pgfpathlineto{\pgfqpoint{1.495398in}{2.153127in}}%
\pgfpathlineto{\pgfqpoint{1.558872in}{2.178689in}}%
\pgfpathlineto{\pgfqpoint{1.622646in}{2.203491in}}%
\pgfpathlineto{\pgfqpoint{1.686856in}{2.227131in}}%
\pgfpathlineto{\pgfqpoint{1.751616in}{2.249212in}}%
\pgfpathlineto{\pgfqpoint{1.816997in}{2.269371in}}%
\pgfpathlineto{\pgfqpoint{1.883017in}{2.287319in}}%
\pgfpathlineto{\pgfqpoint{1.949633in}{2.302907in}}%
\pgfpathlineto{\pgfqpoint{2.016746in}{2.316210in}}%
\pgfpathlineto{\pgfqpoint{2.084204in}{2.327667in}}%
\pgfpathlineto{\pgfqpoint{2.151815in}{2.338208in}}%
\pgfpathlineto{\pgfqpoint{2.219314in}{2.349398in}}%
\pgfpathlineto{\pgfqpoint{2.286216in}{2.363545in}}%
\pgfpathlineto{\pgfqpoint{2.351559in}{2.383399in}}%
\pgfpathlineto{\pgfqpoint{2.413933in}{2.410976in}}%
\pgfpathlineto{\pgfqpoint{2.472264in}{2.446268in}}%
\pgfpathlineto{\pgfqpoint{2.526536in}{2.487624in}}%
\pgfusepath{stroke}%
\end{pgfscope}%
\begin{pgfscope}%
\pgfpathrectangle{\pgfqpoint{0.647939in}{0.492442in}}{\pgfqpoint{3.079299in}{3.079299in}}%
\pgfusepath{clip}%
\pgfsetbuttcap%
\pgfsetroundjoin%
\pgfsetlinewidth{0.301125pt}%
\definecolor{currentstroke}{rgb}{0.500000,0.500000,0.500000}%
\pgfsetstrokecolor{currentstroke}%
\pgfsetstrokeopacity{0.300000}%
\pgfsetdash{}{0pt}%
\pgfpathmoveto{\pgfqpoint{0.647939in}{1.822139in}}%
\pgfpathlineto{\pgfqpoint{0.647939in}{1.822139in}}%
\pgfpathlineto{\pgfqpoint{0.715508in}{1.832912in}}%
\pgfpathlineto{\pgfqpoint{0.782763in}{1.845493in}}%
\pgfpathlineto{\pgfqpoint{0.849637in}{1.859956in}}%
\pgfpathlineto{\pgfqpoint{0.916069in}{1.876328in}}%
\pgfpathlineto{\pgfqpoint{0.982007in}{1.894592in}}%
\pgfpathlineto{\pgfqpoint{1.047415in}{1.914676in}}%
\pgfpathlineto{\pgfqpoint{1.112277in}{1.936458in}}%
\pgfpathlineto{\pgfqpoint{1.176609in}{1.959768in}}%
\pgfpathlineto{\pgfqpoint{1.240452in}{1.984388in}}%
\pgfpathlineto{\pgfqpoint{1.303880in}{2.010060in}}%
\pgfpathlineto{\pgfqpoint{1.366994in}{2.036498in}}%
\pgfpathlineto{\pgfqpoint{1.429920in}{2.063382in}}%
\pgfpathlineto{\pgfqpoint{1.492803in}{2.090366in}}%
\pgfpathlineto{\pgfqpoint{1.555802in}{2.117077in}}%
\pgfusepath{stroke}%
\end{pgfscope}%
\begin{pgfscope}%
\pgfpathrectangle{\pgfqpoint{0.647939in}{0.492442in}}{\pgfqpoint{3.079299in}{3.079299in}}%
\pgfusepath{clip}%
\pgfsetbuttcap%
\pgfsetroundjoin%
\pgfsetlinewidth{0.301125pt}%
\definecolor{currentstroke}{rgb}{0.500000,0.500000,0.500000}%
\pgfsetstrokecolor{currentstroke}%
\pgfsetstrokeopacity{0.300000}%
\pgfsetdash{}{0pt}%
\pgfpathmoveto{\pgfqpoint{0.647939in}{1.752155in}}%
\pgfpathlineto{\pgfqpoint{0.647939in}{1.752155in}}%
\pgfpathlineto{\pgfqpoint{0.715481in}{1.763093in}}%
\pgfpathlineto{\pgfqpoint{0.782696in}{1.775885in}}%
\pgfpathlineto{\pgfqpoint{0.849512in}{1.790611in}}%
\pgfpathlineto{\pgfqpoint{0.915863in}{1.807308in}}%
\pgfpathlineto{\pgfqpoint{0.981690in}{1.825963in}}%
\pgfpathlineto{\pgfqpoint{1.046952in}{1.846513in}}%
\pgfpathlineto{\pgfqpoint{1.111628in}{1.868841in}}%
\pgfpathlineto{\pgfqpoint{1.175727in}{1.892781in}}%
\pgfpathlineto{\pgfqpoint{1.239287in}{1.918121in}}%
\pgfpathlineto{\pgfqpoint{1.302379in}{1.944608in}}%
\pgfpathlineto{\pgfqpoint{1.365103in}{1.971957in}}%
\pgfusepath{stroke}%
\end{pgfscope}%
\begin{pgfscope}%
\pgfpathrectangle{\pgfqpoint{0.647939in}{0.492442in}}{\pgfqpoint{3.079299in}{3.079299in}}%
\pgfusepath{clip}%
\pgfsetbuttcap%
\pgfsetroundjoin%
\pgfsetlinewidth{0.301125pt}%
\definecolor{currentstroke}{rgb}{0.500000,0.500000,0.500000}%
\pgfsetstrokecolor{currentstroke}%
\pgfsetstrokeopacity{0.300000}%
\pgfsetdash{}{0pt}%
\pgfpathmoveto{\pgfqpoint{0.647939in}{1.682171in}}%
\pgfpathlineto{\pgfqpoint{0.647939in}{1.682171in}}%
\pgfpathlineto{\pgfqpoint{0.715453in}{1.693279in}}%
\pgfpathlineto{\pgfqpoint{0.782626in}{1.706289in}}%
\pgfpathlineto{\pgfqpoint{0.849380in}{1.721288in}}%
\pgfpathlineto{\pgfqpoint{0.915645in}{1.738321in}}%
\pgfpathlineto{\pgfqpoint{0.981354in}{1.757385in}}%
\pgfpathlineto{\pgfqpoint{1.046460in}{1.778420in}}%
\pgfpathlineto{\pgfqpoint{1.110936in}{1.801317in}}%
\pgfpathlineto{\pgfqpoint{1.174784in}{1.825916in}}%
\pgfpathlineto{\pgfqpoint{1.238038in}{1.852010in}}%
\pgfpathlineto{\pgfqpoint{1.300763in}{1.879351in}}%
\pgfpathlineto{\pgfqpoint{1.363059in}{1.907661in}}%
\pgfpathlineto{\pgfqpoint{1.425054in}{1.936626in}}%
\pgfpathlineto{\pgfqpoint{1.486900in}{1.965909in}}%
\pgfpathlineto{\pgfqpoint{1.548767in}{1.995146in}}%
\pgfpathlineto{\pgfqpoint{1.610836in}{2.023951in}}%
\pgfpathlineto{\pgfqpoint{1.673287in}{2.051914in}}%
\pgfpathlineto{\pgfqpoint{1.736285in}{2.078616in}}%
\pgfpathlineto{\pgfqpoint{1.799963in}{2.103645in}}%
\pgfpathlineto{\pgfqpoint{1.864403in}{2.126631in}}%
\pgfpathlineto{\pgfqpoint{1.929615in}{2.147317in}}%
\pgfpathlineto{\pgfqpoint{1.995525in}{2.165662in}}%
\pgfpathlineto{\pgfqpoint{2.061967in}{2.181996in}}%
\pgfpathlineto{\pgfqpoint{2.128666in}{2.197267in}}%
\pgfpathlineto{\pgfqpoint{2.195154in}{2.213377in}}%
\pgfpathlineto{\pgfqpoint{2.260490in}{2.233404in}}%
\pgfpathlineto{\pgfqpoint{2.322929in}{2.260767in}}%
\pgfusepath{stroke}%
\end{pgfscope}%
\begin{pgfscope}%
\pgfpathrectangle{\pgfqpoint{0.647939in}{0.492442in}}{\pgfqpoint{3.079299in}{3.079299in}}%
\pgfusepath{clip}%
\pgfsetbuttcap%
\pgfsetroundjoin%
\pgfsetlinewidth{0.301125pt}%
\definecolor{currentstroke}{rgb}{0.500000,0.500000,0.500000}%
\pgfsetstrokecolor{currentstroke}%
\pgfsetstrokeopacity{0.300000}%
\pgfsetdash{}{0pt}%
\pgfpathmoveto{\pgfqpoint{0.647939in}{1.612187in}}%
\pgfpathlineto{\pgfqpoint{0.647939in}{1.612187in}}%
\pgfpathlineto{\pgfqpoint{0.715424in}{1.623471in}}%
\pgfpathlineto{\pgfqpoint{0.782552in}{1.636705in}}%
\pgfpathlineto{\pgfqpoint{0.849242in}{1.651987in}}%
\pgfpathlineto{\pgfqpoint{0.915414in}{1.669370in}}%
\pgfpathlineto{\pgfqpoint{0.980998in}{1.688858in}}%
\pgfpathlineto{\pgfqpoint{1.045937in}{1.710400in}}%
\pgfpathlineto{\pgfqpoint{1.110198in}{1.733892in}}%
\pgfpathlineto{\pgfqpoint{1.173775in}{1.759180in}}%
\pgfpathlineto{\pgfqpoint{1.236696in}{1.786063in}}%
\pgfpathlineto{\pgfqpoint{1.299023in}{1.814299in}}%
\pgfpathlineto{\pgfqpoint{1.360852in}{1.843612in}}%
\pgfpathlineto{\pgfqpoint{1.422310in}{1.873697in}}%
\pgfpathlineto{\pgfqpoint{1.483552in}{1.904223in}}%
\pgfusepath{stroke}%
\end{pgfscope}%
\begin{pgfscope}%
\pgfpathrectangle{\pgfqpoint{0.647939in}{0.492442in}}{\pgfqpoint{3.079299in}{3.079299in}}%
\pgfusepath{clip}%
\pgfsetbuttcap%
\pgfsetroundjoin%
\pgfsetlinewidth{0.301125pt}%
\definecolor{currentstroke}{rgb}{0.500000,0.500000,0.500000}%
\pgfsetstrokecolor{currentstroke}%
\pgfsetstrokeopacity{0.300000}%
\pgfsetdash{}{0pt}%
\pgfpathmoveto{\pgfqpoint{0.647939in}{1.542203in}}%
\pgfpathlineto{\pgfqpoint{0.647939in}{1.542203in}}%
\pgfpathlineto{\pgfqpoint{0.715393in}{1.553668in}}%
\pgfpathlineto{\pgfqpoint{0.782474in}{1.567135in}}%
\pgfpathlineto{\pgfqpoint{0.849095in}{1.582710in}}%
\pgfpathlineto{\pgfqpoint{0.915171in}{1.600456in}}%
\pgfpathlineto{\pgfqpoint{0.980621in}{1.620386in}}%
\pgfpathlineto{\pgfqpoint{1.045381in}{1.642456in}}%
\pgfpathlineto{\pgfqpoint{1.109409in}{1.666571in}}%
\pgfpathlineto{\pgfqpoint{1.172693in}{1.692581in}}%
\pgfusepath{stroke}%
\end{pgfscope}%
\begin{pgfscope}%
\pgfpathrectangle{\pgfqpoint{0.647939in}{0.492442in}}{\pgfqpoint{3.079299in}{3.079299in}}%
\pgfusepath{clip}%
\pgfsetbuttcap%
\pgfsetroundjoin%
\pgfsetlinewidth{0.301125pt}%
\definecolor{currentstroke}{rgb}{0.500000,0.500000,0.500000}%
\pgfsetstrokecolor{currentstroke}%
\pgfsetstrokeopacity{0.300000}%
\pgfsetdash{}{0pt}%
\pgfpathmoveto{\pgfqpoint{0.647939in}{1.472219in}}%
\pgfpathlineto{\pgfqpoint{0.647939in}{1.472219in}}%
\pgfpathlineto{\pgfqpoint{0.715361in}{1.483870in}}%
\pgfpathlineto{\pgfqpoint{0.782392in}{1.497578in}}%
\pgfpathlineto{\pgfqpoint{0.848941in}{1.513458in}}%
\pgfpathlineto{\pgfqpoint{0.914913in}{1.531582in}}%
\pgfpathlineto{\pgfqpoint{0.980220in}{1.551971in}}%
\pgfpathlineto{\pgfqpoint{1.044788in}{1.574594in}}%
\pgfpathlineto{\pgfqpoint{1.108566in}{1.599359in}}%
\pgfpathlineto{\pgfqpoint{1.171532in}{1.626125in}}%
\pgfpathlineto{\pgfqpoint{1.233699in}{1.654702in}}%
\pgfpathlineto{\pgfqpoint{1.295116in}{1.684860in}}%
\pgfpathlineto{\pgfqpoint{1.355869in}{1.716337in}}%
\pgfpathlineto{\pgfqpoint{1.416081in}{1.748840in}}%
\pgfpathlineto{\pgfqpoint{1.475904in}{1.782056in}}%
\pgfpathlineto{\pgfqpoint{1.535517in}{1.815649in}}%
\pgfpathlineto{\pgfqpoint{1.595118in}{1.849261in}}%
\pgfpathlineto{\pgfqpoint{1.654921in}{1.882513in}}%
\pgfpathlineto{\pgfqpoint{1.715136in}{1.915008in}}%
\pgfpathlineto{\pgfqpoint{1.775964in}{1.946335in}}%
\pgfpathlineto{\pgfqpoint{1.837568in}{1.976098in}}%
\pgfpathlineto{\pgfqpoint{1.900049in}{2.003964in}}%
\pgfpathlineto{\pgfqpoint{1.963410in}{2.029758in}}%
\pgfpathlineto{\pgfqpoint{2.027518in}{2.053647in}}%
\pgfpathlineto{\pgfqpoint{2.092028in}{2.076443in}}%
\pgfpathlineto{\pgfqpoint{2.156199in}{2.100070in}}%
\pgfpathlineto{\pgfqpoint{2.218500in}{2.127815in}}%
\pgfusepath{stroke}%
\end{pgfscope}%
\begin{pgfscope}%
\pgfpathrectangle{\pgfqpoint{0.647939in}{0.492442in}}{\pgfqpoint{3.079299in}{3.079299in}}%
\pgfusepath{clip}%
\pgfsetbuttcap%
\pgfsetroundjoin%
\pgfsetlinewidth{0.301125pt}%
\definecolor{currentstroke}{rgb}{0.500000,0.500000,0.500000}%
\pgfsetstrokecolor{currentstroke}%
\pgfsetstrokeopacity{0.300000}%
\pgfsetdash{}{0pt}%
\pgfpathmoveto{\pgfqpoint{0.647939in}{1.402235in}}%
\pgfpathlineto{\pgfqpoint{0.647939in}{1.402235in}}%
\pgfpathlineto{\pgfqpoint{0.715326in}{1.414079in}}%
\pgfpathlineto{\pgfqpoint{0.782306in}{1.428036in}}%
\pgfpathlineto{\pgfqpoint{0.848778in}{1.444231in}}%
\pgfpathlineto{\pgfqpoint{0.914639in}{1.462748in}}%
\pgfpathlineto{\pgfqpoint{0.979793in}{1.483618in}}%
\pgfpathlineto{\pgfqpoint{1.044155in}{1.506816in}}%
\pgfpathlineto{\pgfqpoint{1.107663in}{1.532262in}}%
\pgfpathlineto{\pgfqpoint{1.170285in}{1.559820in}}%
\pgfpathlineto{\pgfqpoint{1.232024in}{1.589306in}}%
\pgfpathlineto{\pgfqpoint{1.292921in}{1.620496in}}%
\pgfusepath{stroke}%
\end{pgfscope}%
\begin{pgfscope}%
\pgfpathrectangle{\pgfqpoint{0.647939in}{0.492442in}}{\pgfqpoint{3.079299in}{3.079299in}}%
\pgfusepath{clip}%
\pgfsetbuttcap%
\pgfsetroundjoin%
\pgfsetlinewidth{0.301125pt}%
\definecolor{currentstroke}{rgb}{0.500000,0.500000,0.500000}%
\pgfsetstrokecolor{currentstroke}%
\pgfsetstrokeopacity{0.300000}%
\pgfsetdash{}{0pt}%
\pgfpathmoveto{\pgfqpoint{0.647939in}{1.332251in}}%
\pgfpathlineto{\pgfqpoint{0.647939in}{1.332251in}}%
\pgfpathlineto{\pgfqpoint{0.715291in}{1.344294in}}%
\pgfpathlineto{\pgfqpoint{0.782215in}{1.358509in}}%
\pgfpathlineto{\pgfqpoint{0.848605in}{1.375033in}}%
\pgfpathlineto{\pgfqpoint{0.914350in}{1.393958in}}%
\pgfpathlineto{\pgfqpoint{0.979340in}{1.415329in}}%
\pgfusepath{stroke}%
\end{pgfscope}%
\begin{pgfscope}%
\pgfpathrectangle{\pgfqpoint{0.647939in}{0.492442in}}{\pgfqpoint{3.079299in}{3.079299in}}%
\pgfusepath{clip}%
\pgfsetbuttcap%
\pgfsetroundjoin%
\pgfsetlinewidth{0.301125pt}%
\definecolor{currentstroke}{rgb}{0.500000,0.500000,0.500000}%
\pgfsetstrokecolor{currentstroke}%
\pgfsetstrokeopacity{0.300000}%
\pgfsetdash{}{0pt}%
\pgfpathmoveto{\pgfqpoint{0.647939in}{1.262267in}}%
\pgfpathlineto{\pgfqpoint{0.647939in}{1.262267in}}%
\pgfpathlineto{\pgfqpoint{0.715253in}{1.274516in}}%
\pgfpathlineto{\pgfqpoint{0.782119in}{1.288999in}}%
\pgfpathlineto{\pgfqpoint{0.848423in}{1.305863in}}%
\pgfpathlineto{\pgfqpoint{0.914042in}{1.325214in}}%
\pgfpathlineto{\pgfqpoint{0.978856in}{1.347107in}}%
\pgfpathlineto{\pgfqpoint{1.042757in}{1.371537in}}%
\pgfpathlineto{\pgfqpoint{1.105658in}{1.398440in}}%
\pgfpathlineto{\pgfqpoint{1.167501in}{1.427694in}}%
\pgfpathlineto{\pgfqpoint{1.228267in}{1.459128in}}%
\pgfpathlineto{\pgfqpoint{1.287975in}{1.492529in}}%
\pgfpathlineto{\pgfqpoint{1.346689in}{1.527653in}}%
\pgfpathlineto{\pgfqpoint{1.404509in}{1.564231in}}%
\pgfpathlineto{\pgfqpoint{1.461578in}{1.601976in}}%
\pgfpathlineto{\pgfqpoint{1.518067in}{1.640587in}}%
\pgfpathlineto{\pgfqpoint{1.574187in}{1.679736in}}%
\pgfpathlineto{\pgfqpoint{1.630163in}{1.719085in}}%
\pgfpathlineto{\pgfqpoint{1.686235in}{1.758294in}}%
\pgfpathlineto{\pgfqpoint{1.742648in}{1.797012in}}%
\pgfpathlineto{\pgfqpoint{1.799637in}{1.834876in}}%
\pgfpathlineto{\pgfqpoint{1.857394in}{1.871541in}}%
\pgfpathlineto{\pgfqpoint{1.916040in}{1.906747in}}%
\pgfpathlineto{\pgfqpoint{1.975564in}{1.940429in}}%
\pgfpathlineto{\pgfqpoint{2.035726in}{1.972928in}}%
\pgfusepath{stroke}%
\end{pgfscope}%
\begin{pgfscope}%
\pgfpathrectangle{\pgfqpoint{0.647939in}{0.492442in}}{\pgfqpoint{3.079299in}{3.079299in}}%
\pgfusepath{clip}%
\pgfsetbuttcap%
\pgfsetroundjoin%
\pgfsetlinewidth{0.301125pt}%
\definecolor{currentstroke}{rgb}{0.500000,0.500000,0.500000}%
\pgfsetstrokecolor{currentstroke}%
\pgfsetstrokeopacity{0.300000}%
\pgfsetdash{}{0pt}%
\pgfpathmoveto{\pgfqpoint{0.647939in}{1.192283in}}%
\pgfpathlineto{\pgfqpoint{0.647939in}{1.192283in}}%
\pgfpathlineto{\pgfqpoint{0.715213in}{1.204745in}}%
\pgfpathlineto{\pgfqpoint{0.782018in}{1.219505in}}%
\pgfpathlineto{\pgfqpoint{0.848230in}{1.236723in}}%
\pgfpathlineto{\pgfqpoint{0.913715in}{1.256518in}}%
\pgfpathlineto{\pgfqpoint{0.978340in}{1.278957in}}%
\pgfpathlineto{\pgfqpoint{1.041984in}{1.304045in}}%
\pgfpathlineto{\pgfqpoint{1.104543in}{1.331727in}}%
\pgfpathlineto{\pgfqpoint{1.165946in}{1.361889in}}%
\pgfpathlineto{\pgfqpoint{1.226158in}{1.394365in}}%
\pgfpathlineto{\pgfqpoint{1.285188in}{1.428946in}}%
\pgfpathlineto{\pgfqpoint{1.343087in}{1.465393in}}%
\pgfpathlineto{\pgfqpoint{1.399948in}{1.503441in}}%
\pgfpathlineto{\pgfqpoint{1.455907in}{1.542812in}}%
\pgfusepath{stroke}%
\end{pgfscope}%
\begin{pgfscope}%
\pgfpathrectangle{\pgfqpoint{0.647939in}{0.492442in}}{\pgfqpoint{3.079299in}{3.079299in}}%
\pgfusepath{clip}%
\pgfsetbuttcap%
\pgfsetroundjoin%
\pgfsetlinewidth{0.301125pt}%
\definecolor{currentstroke}{rgb}{0.500000,0.500000,0.500000}%
\pgfsetstrokecolor{currentstroke}%
\pgfsetstrokeopacity{0.300000}%
\pgfsetdash{}{0pt}%
\pgfpathmoveto{\pgfqpoint{0.647939in}{1.122299in}}%
\pgfpathlineto{\pgfqpoint{0.647939in}{1.122299in}}%
\pgfpathlineto{\pgfqpoint{0.715172in}{1.134982in}}%
\pgfpathlineto{\pgfqpoint{0.781912in}{1.150029in}}%
\pgfpathlineto{\pgfqpoint{0.848025in}{1.167615in}}%
\pgfpathlineto{\pgfqpoint{0.913366in}{1.187874in}}%
\pgfpathlineto{\pgfqpoint{0.977789in}{1.210882in}}%
\pgfpathlineto{\pgfqpoint{1.041155in}{1.236658in}}%
\pgfpathlineto{\pgfqpoint{1.103344in}{1.265156in}}%
\pgfpathlineto{\pgfqpoint{1.164267in}{1.296268in}}%
\pgfpathlineto{\pgfqpoint{1.223876in}{1.329832in}}%
\pgfpathlineto{\pgfqpoint{1.282165in}{1.365642in}}%
\pgfpathlineto{\pgfqpoint{1.339174in}{1.403462in}}%
\pgfusepath{stroke}%
\end{pgfscope}%
\begin{pgfscope}%
\pgfpathrectangle{\pgfqpoint{0.647939in}{0.492442in}}{\pgfqpoint{3.079299in}{3.079299in}}%
\pgfusepath{clip}%
\pgfsetbuttcap%
\pgfsetroundjoin%
\pgfsetlinewidth{0.301125pt}%
\definecolor{currentstroke}{rgb}{0.500000,0.500000,0.500000}%
\pgfsetstrokecolor{currentstroke}%
\pgfsetstrokeopacity{0.300000}%
\pgfsetdash{}{0pt}%
\pgfpathmoveto{\pgfqpoint{0.647939in}{1.052315in}}%
\pgfpathlineto{\pgfqpoint{0.647939in}{1.052315in}}%
\pgfpathlineto{\pgfqpoint{0.715127in}{1.065226in}}%
\pgfpathlineto{\pgfqpoint{0.781798in}{1.080572in}}%
\pgfpathlineto{\pgfqpoint{0.847807in}{1.098542in}}%
\pgfpathlineto{\pgfqpoint{0.912995in}{1.119283in}}%
\pgfpathlineto{\pgfqpoint{0.977200in}{1.142887in}}%
\pgfpathlineto{\pgfqpoint{1.040265in}{1.169383in}}%
\pgfpathlineto{\pgfqpoint{1.102053in}{1.198734in}}%
\pgfusepath{stroke}%
\end{pgfscope}%
\begin{pgfscope}%
\pgfpathrectangle{\pgfqpoint{0.647939in}{0.492442in}}{\pgfqpoint{3.079299in}{3.079299in}}%
\pgfusepath{clip}%
\pgfsetbuttcap%
\pgfsetroundjoin%
\pgfsetlinewidth{0.301125pt}%
\definecolor{currentstroke}{rgb}{0.500000,0.500000,0.500000}%
\pgfsetstrokecolor{currentstroke}%
\pgfsetstrokeopacity{0.300000}%
\pgfsetdash{}{0pt}%
\pgfpathmoveto{\pgfqpoint{0.647939in}{0.982331in}}%
\pgfpathlineto{\pgfqpoint{0.647939in}{0.982331in}}%
\pgfpathlineto{\pgfqpoint{0.715081in}{0.995478in}}%
\pgfpathlineto{\pgfqpoint{0.781679in}{1.011134in}}%
\pgfpathlineto{\pgfqpoint{0.847576in}{1.029504in}}%
\pgfpathlineto{\pgfqpoint{0.912600in}{1.050749in}}%
\pgfpathlineto{\pgfqpoint{0.976570in}{1.074975in}}%
\pgfpathlineto{\pgfqpoint{1.039310in}{1.102225in}}%
\pgfpathlineto{\pgfqpoint{1.100661in}{1.132469in}}%
\pgfpathlineto{\pgfqpoint{1.160497in}{1.165609in}}%
\pgfpathlineto{\pgfqpoint{1.218733in}{1.201489in}}%
\pgfpathlineto{\pgfqpoint{1.275334in}{1.239901in}}%
\pgfpathlineto{\pgfqpoint{1.330320in}{1.280602in}}%
\pgfpathlineto{\pgfqpoint{1.383771in}{1.323297in}}%
\pgfpathlineto{\pgfqpoint{1.435819in}{1.367689in}}%
\pgfpathlineto{\pgfqpoint{1.486632in}{1.413497in}}%
\pgfpathlineto{\pgfqpoint{1.536436in}{1.460401in}}%
\pgfpathlineto{\pgfqpoint{1.585482in}{1.508092in}}%
\pgfpathlineto{\pgfqpoint{1.634050in}{1.556270in}}%
\pgfpathlineto{\pgfqpoint{1.682443in}{1.604618in}}%
\pgfpathlineto{\pgfqpoint{1.730969in}{1.652825in}}%
\pgfpathlineto{\pgfqpoint{1.779944in}{1.700570in}}%
\pgfpathlineto{\pgfqpoint{1.829659in}{1.747533in}}%
\pgfpathlineto{\pgfqpoint{1.880361in}{1.793416in}}%
\pgfpathlineto{\pgfqpoint{1.932197in}{1.837972in}}%
\pgfusepath{stroke}%
\end{pgfscope}%
\begin{pgfscope}%
\pgfpathrectangle{\pgfqpoint{0.647939in}{0.492442in}}{\pgfqpoint{3.079299in}{3.079299in}}%
\pgfusepath{clip}%
\pgfsetbuttcap%
\pgfsetroundjoin%
\pgfsetlinewidth{0.301125pt}%
\definecolor{currentstroke}{rgb}{0.500000,0.500000,0.500000}%
\pgfsetstrokecolor{currentstroke}%
\pgfsetstrokeopacity{0.300000}%
\pgfsetdash{}{0pt}%
\pgfpathmoveto{\pgfqpoint{0.647939in}{0.912347in}}%
\pgfpathlineto{\pgfqpoint{0.647939in}{0.912347in}}%
\pgfpathlineto{\pgfqpoint{0.715032in}{0.925739in}}%
\pgfpathlineto{\pgfqpoint{0.781552in}{0.941718in}}%
\pgfpathlineto{\pgfqpoint{0.847330in}{0.960503in}}%
\pgfpathlineto{\pgfqpoint{0.912177in}{0.982275in}}%
\pgfpathlineto{\pgfqpoint{0.975894in}{1.007153in}}%
\pgfpathlineto{\pgfqpoint{1.038282in}{1.035191in}}%
\pgfpathlineto{\pgfqpoint{1.099161in}{1.066368in}}%
\pgfpathlineto{\pgfqpoint{1.158380in}{1.100589in}}%
\pgfpathlineto{\pgfqpoint{1.215838in}{1.137695in}}%
\pgfpathlineto{\pgfqpoint{1.271484in}{1.177476in}}%
\pgfpathlineto{\pgfqpoint{1.325335in}{1.219660in}}%
\pgfusepath{stroke}%
\end{pgfscope}%
\begin{pgfscope}%
\pgfpathrectangle{\pgfqpoint{0.647939in}{0.492442in}}{\pgfqpoint{3.079299in}{3.079299in}}%
\pgfusepath{clip}%
\pgfsetbuttcap%
\pgfsetroundjoin%
\pgfsetlinewidth{0.301125pt}%
\definecolor{currentstroke}{rgb}{0.500000,0.500000,0.500000}%
\pgfsetstrokecolor{currentstroke}%
\pgfsetstrokeopacity{0.300000}%
\pgfsetdash{}{0pt}%
\pgfpathmoveto{\pgfqpoint{0.647939in}{0.842362in}}%
\pgfpathlineto{\pgfqpoint{0.647939in}{0.842362in}}%
\pgfpathlineto{\pgfqpoint{0.714980in}{0.856009in}}%
\pgfpathlineto{\pgfqpoint{0.781417in}{0.872323in}}%
\pgfpathlineto{\pgfqpoint{0.847068in}{0.891543in}}%
\pgfpathlineto{\pgfqpoint{0.911726in}{0.913864in}}%
\pgfpathlineto{\pgfqpoint{0.975169in}{0.939424in}}%
\pgfpathlineto{\pgfqpoint{1.037176in}{0.968287in}}%
\pgfpathlineto{\pgfqpoint{1.097540in}{1.000439in}}%
\pgfpathlineto{\pgfqpoint{1.156090in}{1.035786in}}%
\pgfpathlineto{\pgfqpoint{1.212703in}{1.074163in}}%
\pgfpathlineto{\pgfqpoint{1.267318in}{1.115345in}}%
\pgfusepath{stroke}%
\end{pgfscope}%
\begin{pgfscope}%
\pgfpathrectangle{\pgfqpoint{0.647939in}{0.492442in}}{\pgfqpoint{3.079299in}{3.079299in}}%
\pgfusepath{clip}%
\pgfsetbuttcap%
\pgfsetroundjoin%
\pgfsetlinewidth{0.301125pt}%
\definecolor{currentstroke}{rgb}{0.500000,0.500000,0.500000}%
\pgfsetstrokecolor{currentstroke}%
\pgfsetstrokeopacity{0.300000}%
\pgfsetdash{}{0pt}%
\pgfpathmoveto{\pgfqpoint{0.647939in}{0.772378in}}%
\pgfpathlineto{\pgfqpoint{0.647939in}{0.772378in}}%
\pgfpathlineto{\pgfqpoint{0.714925in}{0.786289in}}%
\pgfpathlineto{\pgfqpoint{0.781274in}{0.802952in}}%
\pgfpathlineto{\pgfqpoint{0.846789in}{0.822624in}}%
\pgfpathlineto{\pgfqpoint{0.911242in}{0.845521in}}%
\pgfpathlineto{\pgfqpoint{0.974390in}{0.871794in}}%
\pgfpathlineto{\pgfqpoint{1.035983in}{0.901521in}}%
\pgfpathlineto{\pgfqpoint{1.095790in}{0.934691in}}%
\pgfpathlineto{\pgfqpoint{1.153612in}{0.971208in}}%
\pgfpathlineto{\pgfqpoint{1.209309in}{1.010898in}}%
\pgfpathlineto{\pgfqpoint{1.262815in}{1.053504in}}%
\pgfpathlineto{\pgfqpoint{1.314141in}{1.098701in}}%
\pgfpathlineto{\pgfqpoint{1.363377in}{1.146179in}}%
\pgfpathlineto{\pgfqpoint{1.410691in}{1.195576in}}%
\pgfpathlineto{\pgfqpoint{1.456315in}{1.246538in}}%
\pgfpathlineto{\pgfqpoint{1.500535in}{1.298727in}}%
\pgfusepath{stroke}%
\end{pgfscope}%
\begin{pgfscope}%
\pgfpathrectangle{\pgfqpoint{0.647939in}{0.492442in}}{\pgfqpoint{3.079299in}{3.079299in}}%
\pgfusepath{clip}%
\pgfsetbuttcap%
\pgfsetroundjoin%
\pgfsetlinewidth{0.301125pt}%
\definecolor{currentstroke}{rgb}{0.500000,0.500000,0.500000}%
\pgfsetstrokecolor{currentstroke}%
\pgfsetstrokeopacity{0.300000}%
\pgfsetdash{}{0pt}%
\pgfpathmoveto{\pgfqpoint{0.647939in}{0.702394in}}%
\pgfpathlineto{\pgfqpoint{0.647939in}{0.702394in}}%
\pgfpathlineto{\pgfqpoint{0.714866in}{0.716578in}}%
\pgfpathlineto{\pgfqpoint{0.781122in}{0.733605in}}%
\pgfpathlineto{\pgfqpoint{0.846490in}{0.753751in}}%
\pgfpathlineto{\pgfqpoint{0.910724in}{0.777249in}}%
\pgfpathlineto{\pgfqpoint{0.973552in}{0.804269in}}%
\pgfpathlineto{\pgfqpoint{1.034697in}{0.834898in}}%
\pgfpathlineto{\pgfqpoint{1.093897in}{0.869130in}}%
\pgfpathlineto{\pgfqpoint{1.150930in}{0.906861in}}%
\pgfusepath{stroke}%
\end{pgfscope}%
\begin{pgfscope}%
\pgfpathrectangle{\pgfqpoint{0.647939in}{0.492442in}}{\pgfqpoint{3.079299in}{3.079299in}}%
\pgfusepath{clip}%
\pgfsetbuttcap%
\pgfsetroundjoin%
\pgfsetlinewidth{0.301125pt}%
\definecolor{currentstroke}{rgb}{0.500000,0.500000,0.500000}%
\pgfsetstrokecolor{currentstroke}%
\pgfsetstrokeopacity{0.300000}%
\pgfsetdash{}{0pt}%
\pgfpathmoveto{\pgfqpoint{0.647939in}{0.632410in}}%
\pgfpathlineto{\pgfqpoint{0.647939in}{0.632410in}}%
\pgfpathlineto{\pgfqpoint{0.714804in}{0.646879in}}%
\pgfpathlineto{\pgfqpoint{0.780960in}{0.664284in}}%
\pgfpathlineto{\pgfqpoint{0.846172in}{0.684924in}}%
\pgfpathlineto{\pgfqpoint{0.910168in}{0.709052in}}%
\pgfpathlineto{\pgfqpoint{0.972650in}{0.736855in}}%
\pgfpathlineto{\pgfqpoint{1.033307in}{0.768427in}}%
\pgfpathlineto{\pgfqpoint{1.091848in}{0.803764in}}%
\pgfpathlineto{\pgfqpoint{1.148026in}{0.842751in}}%
\pgfpathlineto{\pgfqpoint{1.201672in}{0.885160in}}%
\pgfpathlineto{\pgfqpoint{1.252717in}{0.930656in}}%
\pgfpathlineto{\pgfqpoint{1.301191in}{0.978892in}}%
\pgfpathlineto{\pgfqpoint{1.347241in}{1.029453in}}%
\pgfusepath{stroke}%
\end{pgfscope}%
\begin{pgfscope}%
\pgfpathrectangle{\pgfqpoint{0.647939in}{0.492442in}}{\pgfqpoint{3.079299in}{3.079299in}}%
\pgfusepath{clip}%
\pgfsetbuttcap%
\pgfsetroundjoin%
\pgfsetlinewidth{0.301125pt}%
\definecolor{currentstroke}{rgb}{0.500000,0.500000,0.500000}%
\pgfsetstrokecolor{currentstroke}%
\pgfsetstrokeopacity{0.300000}%
\pgfsetdash{}{0pt}%
\pgfpathmoveto{\pgfqpoint{3.394860in}{3.307385in}}%
\pgfpathlineto{\pgfqpoint{3.440675in}{3.358200in}}%
\pgfpathlineto{\pgfqpoint{3.487989in}{3.407622in}}%
\pgfpathlineto{\pgfqpoint{3.536836in}{3.455527in}}%
\pgfpathlineto{\pgfqpoint{3.587270in}{3.501757in}}%
\pgfpathlineto{\pgfqpoint{3.639341in}{3.546133in}}%
\pgfpathlineto{\pgfqpoint{3.670549in}{3.571741in}}%
\pgfusepath{stroke}%
\end{pgfscope}%
\begin{pgfscope}%
\pgfpathrectangle{\pgfqpoint{0.647939in}{0.492442in}}{\pgfqpoint{3.079299in}{3.079299in}}%
\pgfusepath{clip}%
\pgfsetbuttcap%
\pgfsetroundjoin%
\pgfsetlinewidth{0.301125pt}%
\definecolor{currentstroke}{rgb}{0.500000,0.500000,0.500000}%
\pgfsetstrokecolor{currentstroke}%
\pgfsetstrokeopacity{0.300000}%
\pgfsetdash{}{0pt}%
\pgfpathmoveto{\pgfqpoint{2.040742in}{0.604277in}}%
\pgfpathlineto{\pgfqpoint{1.972598in}{0.610383in}}%
\pgfpathlineto{\pgfqpoint{1.904783in}{0.619335in}}%
\pgfpathlineto{\pgfqpoint{1.837668in}{0.632410in}}%
\pgfpathlineto{\pgfqpoint{1.771982in}{0.651238in}}%
\pgfpathlineto{\pgfqpoint{1.709148in}{0.677869in}}%
\pgfusepath{stroke}%
\end{pgfscope}%
\begin{pgfscope}%
\pgfpathrectangle{\pgfqpoint{0.647939in}{0.492442in}}{\pgfqpoint{3.079299in}{3.079299in}}%
\pgfusepath{clip}%
\pgfsetbuttcap%
\pgfsetroundjoin%
\pgfsetlinewidth{0.301125pt}%
\definecolor{currentstroke}{rgb}{0.500000,0.500000,0.500000}%
\pgfsetstrokecolor{currentstroke}%
\pgfsetstrokeopacity{0.300000}%
\pgfsetdash{}{0pt}%
\pgfpathmoveto{\pgfqpoint{3.587270in}{1.332251in}}%
\pgfpathlineto{\pgfqpoint{3.529386in}{1.368730in}}%
\pgfpathlineto{\pgfqpoint{3.472563in}{1.406844in}}%
\pgfpathlineto{\pgfqpoint{3.416741in}{1.446411in}}%
\pgfpathlineto{\pgfqpoint{3.361848in}{1.487260in}}%
\pgfpathlineto{\pgfqpoint{3.307812in}{1.529237in}}%
\pgfusepath{stroke}%
\end{pgfscope}%
\begin{pgfscope}%
\pgfpathrectangle{\pgfqpoint{0.647939in}{0.492442in}}{\pgfqpoint{3.079299in}{3.079299in}}%
\pgfusepath{clip}%
\pgfsetbuttcap%
\pgfsetroundjoin%
\pgfsetlinewidth{0.301125pt}%
\definecolor{currentstroke}{rgb}{0.500000,0.500000,0.500000}%
\pgfsetstrokecolor{currentstroke}%
\pgfsetstrokeopacity{0.300000}%
\pgfsetdash{}{0pt}%
\pgfpathmoveto{\pgfqpoint{3.517286in}{1.612187in}}%
\pgfpathlineto{\pgfqpoint{3.465050in}{1.656372in}}%
\pgfpathlineto{\pgfqpoint{3.414371in}{1.702334in}}%
\pgfpathlineto{\pgfqpoint{3.365266in}{1.749975in}}%
\pgfpathlineto{\pgfqpoint{3.317777in}{1.799228in}}%
\pgfpathlineto{\pgfqpoint{3.271989in}{1.850064in}}%
\pgfpathlineto{\pgfqpoint{3.228033in}{1.902489in}}%
\pgfpathlineto{\pgfqpoint{3.186105in}{1.956549in}}%
\pgfpathlineto{\pgfqpoint{3.146487in}{2.012319in}}%
\pgfpathlineto{\pgfqpoint{3.109581in}{2.069911in}}%
\pgfpathlineto{\pgfqpoint{3.075927in}{2.129448in}}%
\pgfpathlineto{\pgfqpoint{3.046224in}{2.191037in}}%
\pgfpathlineto{\pgfqpoint{3.021339in}{2.254705in}}%
\pgfpathlineto{\pgfqpoint{3.002238in}{2.320320in}}%
\pgfpathlineto{\pgfqpoint{2.989832in}{2.387508in}}%
\pgfpathlineto{\pgfqpoint{2.984735in}{2.455634in}}%
\pgfpathlineto{\pgfqpoint{2.987054in}{2.523907in}}%
\pgfpathlineto{\pgfqpoint{2.996355in}{2.591590in}}%
\pgfpathlineto{\pgfqpoint{3.011813in}{2.658160in}}%
\pgfpathlineto{\pgfqpoint{3.032473in}{2.723327in}}%
\pgfpathlineto{\pgfqpoint{3.057429in}{2.786989in}}%
\pgfpathlineto{\pgfqpoint{3.085908in}{2.849170in}}%
\pgfpathlineto{\pgfqpoint{3.117300in}{2.909941in}}%
\pgfusepath{stroke}%
\end{pgfscope}%
\begin{pgfscope}%
\pgfpathrectangle{\pgfqpoint{0.647939in}{0.492442in}}{\pgfqpoint{3.079299in}{3.079299in}}%
\pgfusepath{clip}%
\pgfsetbuttcap%
\pgfsetroundjoin%
\pgfsetlinewidth{0.301125pt}%
\definecolor{currentstroke}{rgb}{0.500000,0.500000,0.500000}%
\pgfsetstrokecolor{currentstroke}%
\pgfsetstrokeopacity{0.300000}%
\pgfsetdash{}{0pt}%
\pgfpathmoveto{\pgfqpoint{3.447302in}{1.262267in}}%
\pgfpathlineto{\pgfqpoint{3.389516in}{1.298911in}}%
\pgfpathlineto{\pgfqpoint{3.332354in}{1.336525in}}%
\pgfpathlineto{\pgfqpoint{3.275716in}{1.374921in}}%
\pgfpathlineto{\pgfqpoint{3.219492in}{1.413923in}}%
\pgfpathlineto{\pgfqpoint{3.163571in}{1.453358in}}%
\pgfpathlineto{\pgfqpoint{3.107836in}{1.493056in}}%
\pgfpathlineto{\pgfqpoint{3.052166in}{1.532844in}}%
\pgfpathlineto{\pgfqpoint{2.996441in}{1.572552in}}%
\pgfpathlineto{\pgfqpoint{2.940538in}{1.612011in}}%
\pgfpathlineto{\pgfqpoint{2.884339in}{1.651046in}}%
\pgfpathlineto{\pgfqpoint{2.827725in}{1.689472in}}%
\pgfpathlineto{\pgfqpoint{2.770583in}{1.727107in}}%
\pgfpathlineto{\pgfqpoint{2.712816in}{1.763769in}}%
\pgfpathlineto{\pgfqpoint{2.654343in}{1.799294in}}%
\pgfpathlineto{\pgfqpoint{2.595126in}{1.833560in}}%
\pgfpathlineto{\pgfqpoint{2.535189in}{1.866548in}}%
\pgfpathlineto{\pgfqpoint{2.474677in}{1.898467in}}%
\pgfpathlineto{\pgfqpoint{2.413996in}{1.930058in}}%
\pgfpathlineto{\pgfqpoint{2.354336in}{1.963446in}}%
\pgfpathlineto{\pgfqpoint{2.301098in}{2.005237in}}%
\pgfpathlineto{\pgfqpoint{2.301098in}{2.005237in}}%
\pgfpathlineto{\pgfqpoint{2.284399in}{2.030274in}}%
\pgfpathlineto{\pgfqpoint{2.284399in}{2.030274in}}%
\pgfpathlineto{\pgfqpoint{2.279227in}{2.057156in}}%
\pgfusepath{stroke}%
\end{pgfscope}%
\begin{pgfscope}%
\pgfpathrectangle{\pgfqpoint{0.647939in}{0.492442in}}{\pgfqpoint{3.079299in}{3.079299in}}%
\pgfusepath{clip}%
\pgfsetbuttcap%
\pgfsetroundjoin%
\pgfsetlinewidth{0.301125pt}%
\definecolor{currentstroke}{rgb}{0.500000,0.500000,0.500000}%
\pgfsetstrokecolor{currentstroke}%
\pgfsetstrokeopacity{0.300000}%
\pgfsetdash{}{0pt}%
\pgfpathmoveto{\pgfqpoint{1.989892in}{3.258186in}}%
\pgfpathlineto{\pgfqpoint{2.058176in}{3.262611in}}%
\pgfpathlineto{\pgfqpoint{2.126510in}{3.266197in}}%
\pgfpathlineto{\pgfqpoint{2.194862in}{3.269428in}}%
\pgfpathlineto{\pgfqpoint{2.263203in}{3.272896in}}%
\pgfpathlineto{\pgfqpoint{2.331487in}{3.277287in}}%
\pgfpathlineto{\pgfqpoint{2.399638in}{3.283345in}}%
\pgfpathlineto{\pgfqpoint{2.467525in}{3.291805in}}%
\pgfusepath{stroke}%
\end{pgfscope}%
\begin{pgfscope}%
\pgfpathrectangle{\pgfqpoint{0.647939in}{0.492442in}}{\pgfqpoint{3.079299in}{3.079299in}}%
\pgfusepath{clip}%
\pgfsetbuttcap%
\pgfsetroundjoin%
\pgfsetlinewidth{0.301125pt}%
\definecolor{currentstroke}{rgb}{0.500000,0.500000,0.500000}%
\pgfsetstrokecolor{currentstroke}%
\pgfsetstrokeopacity{0.300000}%
\pgfsetdash{}{0pt}%
\pgfpathmoveto{\pgfqpoint{1.806837in}{3.075277in}}%
\pgfpathlineto{\pgfqpoint{1.874659in}{3.084319in}}%
\pgfpathlineto{\pgfqpoint{1.942685in}{3.091684in}}%
\pgfpathlineto{\pgfqpoint{2.010859in}{3.097533in}}%
\pgfpathlineto{\pgfqpoint{2.079129in}{3.102164in}}%
\pgfpathlineto{\pgfqpoint{2.147448in}{3.106028in}}%
\pgfpathlineto{\pgfqpoint{2.215777in}{3.109720in}}%
\pgfpathlineto{\pgfqpoint{2.284073in}{3.113949in}}%
\pgfpathlineto{\pgfqpoint{2.352268in}{3.119526in}}%
\pgfpathlineto{\pgfqpoint{2.420238in}{3.127314in}}%
\pgfpathlineto{\pgfqpoint{2.487779in}{3.138157in}}%
\pgfpathlineto{\pgfqpoint{2.554595in}{3.152757in}}%
\pgfpathlineto{\pgfqpoint{2.620347in}{3.171546in}}%
\pgfpathlineto{\pgfqpoint{2.684712in}{3.194636in}}%
\pgfpathlineto{\pgfqpoint{2.747461in}{3.221821in}}%
\pgfusepath{stroke}%
\end{pgfscope}%
\begin{pgfscope}%
\pgfpathrectangle{\pgfqpoint{0.647939in}{0.492442in}}{\pgfqpoint{3.079299in}{3.079299in}}%
\pgfusepath{clip}%
\pgfsetbuttcap%
\pgfsetroundjoin%
\pgfsetlinewidth{0.301125pt}%
\definecolor{currentstroke}{rgb}{0.500000,0.500000,0.500000}%
\pgfsetstrokecolor{currentstroke}%
\pgfsetstrokeopacity{0.300000}%
\pgfsetdash{}{0pt}%
\pgfpathmoveto{\pgfqpoint{0.893673in}{3.055737in}}%
\pgfpathlineto{\pgfqpoint{0.960860in}{3.068703in}}%
\pgfpathlineto{\pgfqpoint{1.027838in}{3.082708in}}%
\pgfpathlineto{\pgfqpoint{1.094625in}{3.097603in}}%
\pgfpathlineto{\pgfqpoint{1.161250in}{3.113206in}}%
\pgfpathlineto{\pgfqpoint{1.227758in}{3.129305in}}%
\pgfpathlineto{\pgfqpoint{1.294204in}{3.145658in}}%
\pgfpathlineto{\pgfqpoint{1.360651in}{3.162006in}}%
\pgfpathlineto{\pgfqpoint{1.427166in}{3.178074in}}%
\pgfpathlineto{\pgfqpoint{1.493814in}{3.193579in}}%
\pgfpathlineto{\pgfqpoint{1.560651in}{3.208246in}}%
\pgfpathlineto{\pgfqpoint{1.627716in}{3.221821in}}%
\pgfpathlineto{\pgfqpoint{1.695033in}{3.234081in}}%
\pgfpathlineto{\pgfqpoint{1.762603in}{3.244855in}}%
\pgfpathlineto{\pgfqpoint{1.830408in}{3.254038in}}%
\pgfusepath{stroke}%
\end{pgfscope}%
\begin{pgfscope}%
\pgfpathrectangle{\pgfqpoint{0.647939in}{0.492442in}}{\pgfqpoint{3.079299in}{3.079299in}}%
\pgfusepath{clip}%
\pgfsetbuttcap%
\pgfsetroundjoin%
\pgfsetlinewidth{0.301125pt}%
\definecolor{currentstroke}{rgb}{0.500000,0.500000,0.500000}%
\pgfsetstrokecolor{currentstroke}%
\pgfsetstrokeopacity{0.300000}%
\pgfsetdash{}{0pt}%
\pgfpathmoveto{\pgfqpoint{3.119395in}{2.983920in}}%
\pgfpathlineto{\pgfqpoint{3.157216in}{3.040932in}}%
\pgfpathlineto{\pgfqpoint{3.196573in}{3.096898in}}%
\pgfpathlineto{\pgfqpoint{3.237350in}{3.151837in}}%
\pgfpathlineto{\pgfqpoint{3.279478in}{3.205746in}}%
\pgfpathlineto{\pgfqpoint{3.322920in}{3.258607in}}%
\pgfusepath{stroke}%
\end{pgfscope}%
\begin{pgfscope}%
\pgfpathrectangle{\pgfqpoint{0.647939in}{0.492442in}}{\pgfqpoint{3.079299in}{3.079299in}}%
\pgfusepath{clip}%
\pgfsetbuttcap%
\pgfsetroundjoin%
\pgfsetlinewidth{0.301125pt}%
\definecolor{currentstroke}{rgb}{0.500000,0.500000,0.500000}%
\pgfsetstrokecolor{currentstroke}%
\pgfsetstrokeopacity{0.300000}%
\pgfsetdash{}{0pt}%
\pgfpathmoveto{\pgfqpoint{1.916211in}{3.129365in}}%
\pgfpathlineto{\pgfqpoint{1.984352in}{3.135593in}}%
\pgfpathlineto{\pgfqpoint{2.052602in}{3.140502in}}%
\pgfpathlineto{\pgfqpoint{2.120914in}{3.144481in}}%
\pgfpathlineto{\pgfqpoint{2.189250in}{3.148046in}}%
\pgfpathlineto{\pgfqpoint{2.257573in}{3.151837in}}%
\pgfusepath{stroke}%
\end{pgfscope}%
\begin{pgfscope}%
\pgfpathrectangle{\pgfqpoint{0.647939in}{0.492442in}}{\pgfqpoint{3.079299in}{3.079299in}}%
\pgfusepath{clip}%
\pgfsetbuttcap%
\pgfsetroundjoin%
\pgfsetlinewidth{0.301125pt}%
\definecolor{currentstroke}{rgb}{0.500000,0.500000,0.500000}%
\pgfsetstrokecolor{currentstroke}%
\pgfsetstrokeopacity{0.300000}%
\pgfsetdash{}{0pt}%
\pgfpathmoveto{\pgfqpoint{3.237350in}{1.612187in}}%
\pgfpathlineto{\pgfqpoint{3.185694in}{1.657061in}}%
\pgfpathlineto{\pgfqpoint{3.134792in}{1.702787in}}%
\pgfpathlineto{\pgfqpoint{3.084631in}{1.749325in}}%
\pgfpathlineto{\pgfqpoint{3.035227in}{1.796665in}}%
\pgfpathlineto{\pgfqpoint{2.986647in}{1.844847in}}%
\pgfpathlineto{\pgfqpoint{2.939013in}{1.893964in}}%
\pgfpathlineto{\pgfqpoint{2.892549in}{1.944184in}}%
\pgfpathlineto{\pgfqpoint{2.847635in}{1.995784in}}%
\pgfpathlineto{\pgfqpoint{2.804896in}{2.049191in}}%
\pgfpathlineto{\pgfqpoint{2.765386in}{2.105002in}}%
\pgfpathlineto{\pgfqpoint{2.730863in}{2.163962in}}%
\pgfpathlineto{\pgfqpoint{2.704065in}{2.226690in}}%
\pgfpathlineto{\pgfqpoint{2.688451in}{2.292941in}}%
\pgfpathlineto{\pgfqpoint{2.686348in}{2.360910in}}%
\pgfpathlineto{\pgfqpoint{2.696263in}{2.425376in}}%
\pgfpathlineto{\pgfqpoint{2.716193in}{2.490605in}}%
\pgfpathlineto{\pgfqpoint{2.742805in}{2.553498in}}%
\pgfpathlineto{\pgfqpoint{2.773834in}{2.614392in}}%
\pgfusepath{stroke}%
\end{pgfscope}%
\begin{pgfscope}%
\pgfpathrectangle{\pgfqpoint{0.647939in}{0.492442in}}{\pgfqpoint{3.079299in}{3.079299in}}%
\pgfusepath{clip}%
\pgfsetbuttcap%
\pgfsetroundjoin%
\pgfsetlinewidth{0.301125pt}%
\definecolor{currentstroke}{rgb}{0.500000,0.500000,0.500000}%
\pgfsetstrokecolor{currentstroke}%
\pgfsetstrokeopacity{0.300000}%
\pgfsetdash{}{0pt}%
\pgfpathmoveto{\pgfqpoint{3.237350in}{2.172060in}}%
\pgfpathlineto{\pgfqpoint{3.210493in}{2.234954in}}%
\pgfpathlineto{\pgfqpoint{3.188306in}{2.299632in}}%
\pgfpathlineto{\pgfqpoint{3.171385in}{2.365868in}}%
\pgfpathlineto{\pgfqpoint{3.160246in}{2.433303in}}%
\pgfpathlineto{\pgfqpoint{3.155225in}{2.501461in}}%
\pgfpathlineto{\pgfqpoint{3.156397in}{2.569796in}}%
\pgfpathlineto{\pgfqpoint{3.163556in}{2.637777in}}%
\pgfpathlineto{\pgfqpoint{3.176267in}{2.704953in}}%
\pgfusepath{stroke}%
\end{pgfscope}%
\begin{pgfscope}%
\pgfpathrectangle{\pgfqpoint{0.647939in}{0.492442in}}{\pgfqpoint{3.079299in}{3.079299in}}%
\pgfusepath{clip}%
\pgfsetbuttcap%
\pgfsetroundjoin%
\pgfsetlinewidth{0.301125pt}%
\definecolor{currentstroke}{rgb}{0.500000,0.500000,0.500000}%
\pgfsetstrokecolor{currentstroke}%
\pgfsetstrokeopacity{0.300000}%
\pgfsetdash{}{0pt}%
\pgfpathmoveto{\pgfqpoint{2.701011in}{3.010158in}}%
\pgfpathlineto{\pgfqpoint{2.760247in}{3.044345in}}%
\pgfpathlineto{\pgfqpoint{2.817445in}{3.081853in}}%
\pgfpathlineto{\pgfqpoint{2.872855in}{3.121954in}}%
\pgfpathlineto{\pgfqpoint{2.926796in}{3.164025in}}%
\pgfpathlineto{\pgfqpoint{2.979582in}{3.207553in}}%
\pgfusepath{stroke}%
\end{pgfscope}%
\begin{pgfscope}%
\pgfpathrectangle{\pgfqpoint{0.647939in}{0.492442in}}{\pgfqpoint{3.079299in}{3.079299in}}%
\pgfusepath{clip}%
\pgfsetbuttcap%
\pgfsetroundjoin%
\pgfsetlinewidth{0.301125pt}%
\definecolor{currentstroke}{rgb}{0.500000,0.500000,0.500000}%
\pgfsetstrokecolor{currentstroke}%
\pgfsetstrokeopacity{0.300000}%
\pgfsetdash{}{0pt}%
\pgfpathmoveto{\pgfqpoint{3.167366in}{1.892124in}}%
\pgfpathlineto{\pgfqpoint{3.124582in}{1.945511in}}%
\pgfpathlineto{\pgfqpoint{3.083915in}{2.000519in}}%
\pgfpathlineto{\pgfqpoint{3.045791in}{2.057311in}}%
\pgfpathlineto{\pgfqpoint{3.010802in}{2.116077in}}%
\pgfpathlineto{\pgfqpoint{2.979754in}{2.177001in}}%
\pgfpathlineto{\pgfqpoint{2.953689in}{2.240193in}}%
\pgfpathlineto{\pgfqpoint{2.933818in}{2.305563in}}%
\pgfpathlineto{\pgfqpoint{2.921308in}{2.372698in}}%
\pgfpathlineto{\pgfqpoint{2.916923in}{2.440829in}}%
\pgfpathlineto{\pgfqpoint{2.920705in}{2.509009in}}%
\pgfpathlineto{\pgfqpoint{2.931975in}{2.576390in}}%
\pgfpathlineto{\pgfqpoint{2.949610in}{2.642423in}}%
\pgfpathlineto{\pgfqpoint{2.972406in}{2.706878in}}%
\pgfusepath{stroke}%
\end{pgfscope}%
\begin{pgfscope}%
\pgfpathrectangle{\pgfqpoint{0.647939in}{0.492442in}}{\pgfqpoint{3.079299in}{3.079299in}}%
\pgfusepath{clip}%
\pgfsetbuttcap%
\pgfsetroundjoin%
\pgfsetlinewidth{0.301125pt}%
\definecolor{currentstroke}{rgb}{0.500000,0.500000,0.500000}%
\pgfsetstrokecolor{currentstroke}%
\pgfsetstrokeopacity{0.300000}%
\pgfsetdash{}{0pt}%
\pgfpathmoveto{\pgfqpoint{1.207812in}{1.542203in}}%
\pgfpathlineto{\pgfqpoint{1.268736in}{1.573343in}}%
\pgfpathlineto{\pgfqpoint{1.328807in}{1.606100in}}%
\pgfpathlineto{\pgfqpoint{1.388125in}{1.640204in}}%
\pgfpathlineto{\pgfqpoint{1.446826in}{1.675363in}}%
\pgfpathlineto{\pgfqpoint{1.505076in}{1.711265in}}%
\pgfpathlineto{\pgfqpoint{1.563068in}{1.747581in}}%
\pgfusepath{stroke}%
\end{pgfscope}%
\begin{pgfscope}%
\pgfpathrectangle{\pgfqpoint{0.647939in}{0.492442in}}{\pgfqpoint{3.079299in}{3.079299in}}%
\pgfusepath{clip}%
\pgfsetbuttcap%
\pgfsetroundjoin%
\pgfsetlinewidth{0.301125pt}%
\definecolor{currentstroke}{rgb}{0.500000,0.500000,0.500000}%
\pgfsetstrokecolor{currentstroke}%
\pgfsetstrokeopacity{0.300000}%
\pgfsetdash{}{0pt}%
\pgfpathmoveto{\pgfqpoint{2.353439in}{2.839395in}}%
\pgfpathlineto{\pgfqpoint{2.420967in}{2.850292in}}%
\pgfpathlineto{\pgfqpoint{2.487638in}{2.865490in}}%
\pgfpathlineto{\pgfqpoint{2.552927in}{2.885762in}}%
\pgfpathlineto{\pgfqpoint{2.616320in}{2.911334in}}%
\pgfpathlineto{\pgfqpoint{2.677477in}{2.941885in}}%
\pgfusepath{stroke}%
\end{pgfscope}%
\begin{pgfscope}%
\pgfpathrectangle{\pgfqpoint{0.647939in}{0.492442in}}{\pgfqpoint{3.079299in}{3.079299in}}%
\pgfusepath{clip}%
\pgfsetbuttcap%
\pgfsetroundjoin%
\pgfsetlinewidth{0.301125pt}%
\definecolor{currentstroke}{rgb}{0.500000,0.500000,0.500000}%
\pgfsetstrokecolor{currentstroke}%
\pgfsetstrokeopacity{0.300000}%
\pgfsetdash{}{0pt}%
\pgfpathmoveto{\pgfqpoint{1.487748in}{2.032092in}}%
\pgfpathlineto{\pgfqpoint{1.550253in}{2.059939in}}%
\pgfpathlineto{\pgfqpoint{1.612999in}{2.087236in}}%
\pgfpathlineto{\pgfqpoint{1.676152in}{2.113574in}}%
\pgfpathlineto{\pgfqpoint{1.739860in}{2.138535in}}%
\pgfpathlineto{\pgfqpoint{1.804231in}{2.161721in}}%
\pgfpathlineto{\pgfqpoint{1.869323in}{2.182787in}}%
\pgfusepath{stroke}%
\end{pgfscope}%
\begin{pgfscope}%
\pgfpathrectangle{\pgfqpoint{0.647939in}{0.492442in}}{\pgfqpoint{3.079299in}{3.079299in}}%
\pgfusepath{clip}%
\pgfsetbuttcap%
\pgfsetroundjoin%
\pgfsetlinewidth{0.301125pt}%
\definecolor{currentstroke}{rgb}{0.500000,0.500000,0.500000}%
\pgfsetstrokecolor{currentstroke}%
\pgfsetstrokeopacity{0.300000}%
\pgfsetdash{}{0pt}%
\pgfpathmoveto{\pgfqpoint{2.802565in}{1.781583in}}%
\pgfpathlineto{\pgfqpoint{2.747461in}{1.822139in}}%
\pgfpathlineto{\pgfqpoint{2.692087in}{1.862329in}}%
\pgfpathlineto{\pgfqpoint{2.636512in}{1.902237in}}%
\pgfpathlineto{\pgfqpoint{2.580951in}{1.942155in}}%
\pgfpathlineto{\pgfqpoint{2.525929in}{1.982804in}}%
\pgfpathlineto{\pgfqpoint{2.472790in}{2.025817in}}%
\pgfpathlineto{\pgfqpoint{2.425682in}{2.074943in}}%
\pgfpathlineto{\pgfqpoint{2.425682in}{2.074943in}}%
\pgfpathlineto{\pgfqpoint{2.404427in}{2.111394in}}%
\pgfpathlineto{\pgfqpoint{2.404427in}{2.111394in}}%
\pgfpathlineto{\pgfqpoint{2.396536in}{2.147090in}}%
\pgfpathlineto{\pgfqpoint{2.400176in}{2.183772in}}%
\pgfusepath{stroke}%
\end{pgfscope}%
\begin{pgfscope}%
\pgfpathrectangle{\pgfqpoint{0.647939in}{0.492442in}}{\pgfqpoint{3.079299in}{3.079299in}}%
\pgfusepath{clip}%
\pgfsetbuttcap%
\pgfsetroundjoin%
\pgfsetlinewidth{0.301125pt}%
\definecolor{currentstroke}{rgb}{0.500000,0.500000,0.500000}%
\pgfsetstrokecolor{currentstroke}%
\pgfsetstrokeopacity{0.300000}%
\pgfsetdash{}{0pt}%
\pgfpathmoveto{\pgfqpoint{2.747461in}{2.032092in}}%
\pgfpathlineto{\pgfqpoint{2.705061in}{2.085733in}}%
\pgfpathlineto{\pgfqpoint{2.667545in}{2.142837in}}%
\pgfpathlineto{\pgfqpoint{2.638445in}{2.204494in}}%
\pgfpathlineto{\pgfqpoint{2.622710in}{2.270549in}}%
\pgfpathlineto{\pgfqpoint{2.622532in}{2.331861in}}%
\pgfpathlineto{\pgfqpoint{2.633703in}{2.389842in}}%
\pgfusepath{stroke}%
\end{pgfscope}%
\begin{pgfscope}%
\pgfpathrectangle{\pgfqpoint{0.647939in}{0.492442in}}{\pgfqpoint{3.079299in}{3.079299in}}%
\pgfusepath{clip}%
\pgfsetbuttcap%
\pgfsetroundjoin%
\pgfsetlinewidth{0.301125pt}%
\definecolor{currentstroke}{rgb}{0.500000,0.500000,0.500000}%
\pgfsetstrokecolor{currentstroke}%
\pgfsetstrokeopacity{0.300000}%
\pgfsetdash{}{0pt}%
\pgfpathmoveto{\pgfqpoint{1.627716in}{1.962108in}}%
\pgfpathlineto{\pgfqpoint{1.689365in}{1.991796in}}%
\pgfpathlineto{\pgfqpoint{1.751588in}{2.020254in}}%
\pgfpathlineto{\pgfqpoint{1.814535in}{2.047070in}}%
\pgfpathlineto{\pgfqpoint{1.878295in}{2.071879in}}%
\pgfpathlineto{\pgfqpoint{1.942882in}{2.094438in}}%
\pgfpathlineto{\pgfqpoint{2.008210in}{2.114756in}}%
\pgfpathlineto{\pgfqpoint{2.074071in}{2.133296in}}%
\pgfusepath{stroke}%
\end{pgfscope}%
\begin{pgfscope}%
\pgfpathrectangle{\pgfqpoint{0.647939in}{0.492442in}}{\pgfqpoint{3.079299in}{3.079299in}}%
\pgfusepath{clip}%
\pgfsetbuttcap%
\pgfsetroundjoin%
\pgfsetlinewidth{0.301125pt}%
\definecolor{currentstroke}{rgb}{0.500000,0.500000,0.500000}%
\pgfsetstrokecolor{currentstroke}%
\pgfsetstrokeopacity{0.300000}%
\pgfsetdash{}{0pt}%
\pgfpathmoveto{\pgfqpoint{2.588532in}{1.526743in}}%
\pgfpathlineto{\pgfqpoint{2.523141in}{1.546828in}}%
\pgfpathlineto{\pgfqpoint{2.456936in}{1.564053in}}%
\pgfpathlineto{\pgfqpoint{2.390073in}{1.578533in}}%
\pgfpathlineto{\pgfqpoint{2.322743in}{1.590700in}}%
\pgfpathlineto{\pgfqpoint{2.255161in}{1.601416in}}%
\pgfpathlineto{\pgfqpoint{2.187589in}{1.612187in}}%
\pgfpathlineto{\pgfqpoint{2.120565in}{1.625736in}}%
\pgfpathlineto{\pgfqpoint{2.056127in}{1.647695in}}%
\pgfpathlineto{\pgfqpoint{2.056127in}{1.647695in}}%
\pgfpathlineto{\pgfqpoint{2.020980in}{1.669605in}}%
\pgfpathlineto{\pgfqpoint{2.020980in}{1.669605in}}%
\pgfusepath{stroke}%
\end{pgfscope}%
\begin{pgfscope}%
\pgfpathrectangle{\pgfqpoint{0.647939in}{0.492442in}}{\pgfqpoint{3.079299in}{3.079299in}}%
\pgfusepath{clip}%
\pgfsetbuttcap%
\pgfsetroundjoin%
\pgfsetlinewidth{0.301125pt}%
\definecolor{currentstroke}{rgb}{0.500000,0.500000,0.500000}%
\pgfsetstrokecolor{currentstroke}%
\pgfsetstrokeopacity{0.300000}%
\pgfsetdash{}{0pt}%
\pgfpathmoveto{\pgfqpoint{2.537509in}{1.822139in}}%
\pgfpathlineto{\pgfqpoint{2.475537in}{1.851119in}}%
\pgfpathlineto{\pgfqpoint{2.412936in}{1.878714in}}%
\pgfpathlineto{\pgfqpoint{2.350219in}{1.906031in}}%
\pgfpathlineto{\pgfqpoint{2.289027in}{1.936370in}}%
\pgfpathlineto{\pgfqpoint{2.289027in}{1.936370in}}%
\pgfpathlineto{\pgfqpoint{2.252532in}{1.962378in}}%
\pgfpathlineto{\pgfqpoint{2.252532in}{1.962378in}}%
\pgfpathlineto{\pgfqpoint{2.234865in}{1.985625in}}%
\pgfpathlineto{\pgfqpoint{2.234865in}{1.985625in}}%
\pgfusepath{stroke}%
\end{pgfscope}%
\begin{pgfscope}%
\pgfpathrectangle{\pgfqpoint{0.647939in}{0.492442in}}{\pgfqpoint{3.079299in}{3.079299in}}%
\pgfusepath{clip}%
\pgfsetbuttcap%
\pgfsetroundjoin%
\pgfsetlinewidth{0.301125pt}%
\definecolor{currentstroke}{rgb}{0.500000,0.500000,0.500000}%
\pgfsetstrokecolor{currentstroke}%
\pgfsetstrokeopacity{0.300000}%
\pgfsetdash{}{0pt}%
\pgfpathmoveto{\pgfqpoint{1.911192in}{2.225739in}}%
\pgfpathlineto{\pgfqpoint{1.977636in}{2.242044in}}%
\pgfpathlineto{\pgfqpoint{2.044584in}{2.256160in}}%
\pgfpathlineto{\pgfqpoint{2.111841in}{2.268748in}}%
\pgfpathlineto{\pgfqpoint{2.179131in}{2.281159in}}%
\pgfpathlineto{\pgfqpoint{2.245978in}{2.295630in}}%
\pgfpathlineto{\pgfqpoint{2.311411in}{2.315178in}}%
\pgfusepath{stroke}%
\end{pgfscope}%
\begin{pgfscope}%
\pgfpathrectangle{\pgfqpoint{0.647939in}{0.492442in}}{\pgfqpoint{3.079299in}{3.079299in}}%
\pgfusepath{clip}%
\pgfsetroundcap%
\pgfsetroundjoin%
\pgfsetlinewidth{0.301125pt}%
\definecolor{currentstroke}{rgb}{0.500000,0.500000,0.500000}%
\pgfsetstrokecolor{currentstroke}%
\pgfsetstrokeopacity{0.300000}%
\pgfsetdash{}{0pt}%
\pgfpathmoveto{\pgfqpoint{2.114475in}{1.972736in}}%
\pgfusepath{stroke}%
\end{pgfscope}%
\begin{pgfscope}%
\pgfpathrectangle{\pgfqpoint{0.647939in}{0.492442in}}{\pgfqpoint{3.079299in}{3.079299in}}%
\pgfusepath{clip}%
\pgfsetroundcap%
\pgfsetroundjoin%
\definecolor{currentfill}{rgb}{0.500000,0.500000,0.500000}%
\pgfsetfillcolor{currentfill}%
\pgfsetfillopacity{0.300000}%
\pgfsetlinewidth{0.301125pt}%
\definecolor{currentstroke}{rgb}{0.500000,0.500000,0.500000}%
\pgfsetstrokecolor{currentstroke}%
\pgfsetstrokeopacity{0.300000}%
\pgfsetdash{}{0pt}%
\pgfpathmoveto{\pgfqpoint{0.000000in}{0.000000in}}%
\pgfpathlineto{\pgfqpoint{0.000000in}{0.000000in}}%
\pgfpathclose%
\pgfusepath{stroke,fill}%
\end{pgfscope}%
\begin{pgfscope}%
\pgfpathrectangle{\pgfqpoint{0.647939in}{0.492442in}}{\pgfqpoint{3.079299in}{3.079299in}}%
\pgfusepath{clip}%
\pgfsetroundcap%
\pgfsetroundjoin%
\pgfsetlinewidth{0.301125pt}%
\definecolor{currentstroke}{rgb}{0.500000,0.500000,0.500000}%
\pgfsetstrokecolor{currentstroke}%
\pgfsetstrokeopacity{0.300000}%
\pgfsetdash{}{0pt}%
\pgfpathmoveto{\pgfqpoint{1.122143in}{0.619742in}}%
\pgfusepath{stroke}%
\end{pgfscope}%
\begin{pgfscope}%
\pgfpathrectangle{\pgfqpoint{0.647939in}{0.492442in}}{\pgfqpoint{3.079299in}{3.079299in}}%
\pgfusepath{clip}%
\pgfsetroundcap%
\pgfsetroundjoin%
\definecolor{currentfill}{rgb}{0.500000,0.500000,0.500000}%
\pgfsetfillcolor{currentfill}%
\pgfsetfillopacity{0.300000}%
\pgfsetlinewidth{0.301125pt}%
\definecolor{currentstroke}{rgb}{0.500000,0.500000,0.500000}%
\pgfsetstrokecolor{currentstroke}%
\pgfsetstrokeopacity{0.300000}%
\pgfsetdash{}{0pt}%
\pgfpathmoveto{\pgfqpoint{0.000000in}{0.000000in}}%
\pgfpathlineto{\pgfqpoint{0.000000in}{0.000000in}}%
\pgfpathclose%
\pgfusepath{stroke,fill}%
\end{pgfscope}%
\begin{pgfscope}%
\pgfpathrectangle{\pgfqpoint{0.647939in}{0.492442in}}{\pgfqpoint{3.079299in}{3.079299in}}%
\pgfusepath{clip}%
\pgfsetroundcap%
\pgfsetroundjoin%
\pgfsetlinewidth{0.301125pt}%
\definecolor{currentstroke}{rgb}{0.500000,0.500000,0.500000}%
\pgfsetstrokecolor{currentstroke}%
\pgfsetstrokeopacity{0.300000}%
\pgfsetdash{}{0pt}%
\pgfpathmoveto{\pgfqpoint{1.240311in}{0.620857in}}%
\pgfusepath{stroke}%
\end{pgfscope}%
\begin{pgfscope}%
\pgfpathrectangle{\pgfqpoint{0.647939in}{0.492442in}}{\pgfqpoint{3.079299in}{3.079299in}}%
\pgfusepath{clip}%
\pgfsetroundcap%
\pgfsetroundjoin%
\definecolor{currentfill}{rgb}{0.500000,0.500000,0.500000}%
\pgfsetfillcolor{currentfill}%
\pgfsetfillopacity{0.300000}%
\pgfsetlinewidth{0.301125pt}%
\definecolor{currentstroke}{rgb}{0.500000,0.500000,0.500000}%
\pgfsetstrokecolor{currentstroke}%
\pgfsetstrokeopacity{0.300000}%
\pgfsetdash{}{0pt}%
\pgfpathmoveto{\pgfqpoint{0.000000in}{0.000000in}}%
\pgfpathlineto{\pgfqpoint{0.000000in}{0.000000in}}%
\pgfpathclose%
\pgfusepath{stroke,fill}%
\end{pgfscope}%
\begin{pgfscope}%
\pgfpathrectangle{\pgfqpoint{0.647939in}{0.492442in}}{\pgfqpoint{3.079299in}{3.079299in}}%
\pgfusepath{clip}%
\pgfsetroundcap%
\pgfsetroundjoin%
\pgfsetlinewidth{0.301125pt}%
\definecolor{currentstroke}{rgb}{0.500000,0.500000,0.500000}%
\pgfsetstrokecolor{currentstroke}%
\pgfsetstrokeopacity{0.300000}%
\pgfsetdash{}{0pt}%
\pgfpathmoveto{\pgfqpoint{1.424543in}{0.852861in}}%
\pgfusepath{stroke}%
\end{pgfscope}%
\begin{pgfscope}%
\pgfpathrectangle{\pgfqpoint{0.647939in}{0.492442in}}{\pgfqpoint{3.079299in}{3.079299in}}%
\pgfusepath{clip}%
\pgfsetroundcap%
\pgfsetroundjoin%
\definecolor{currentfill}{rgb}{0.500000,0.500000,0.500000}%
\pgfsetfillcolor{currentfill}%
\pgfsetfillopacity{0.300000}%
\pgfsetlinewidth{0.301125pt}%
\definecolor{currentstroke}{rgb}{0.500000,0.500000,0.500000}%
\pgfsetstrokecolor{currentstroke}%
\pgfsetstrokeopacity{0.300000}%
\pgfsetdash{}{0pt}%
\pgfpathmoveto{\pgfqpoint{0.000000in}{0.000000in}}%
\pgfpathlineto{\pgfqpoint{0.000000in}{0.000000in}}%
\pgfpathclose%
\pgfusepath{stroke,fill}%
\end{pgfscope}%
\begin{pgfscope}%
\pgfpathrectangle{\pgfqpoint{0.647939in}{0.492442in}}{\pgfqpoint{3.079299in}{3.079299in}}%
\pgfusepath{clip}%
\pgfsetroundcap%
\pgfsetroundjoin%
\pgfsetlinewidth{0.301125pt}%
\definecolor{currentstroke}{rgb}{0.500000,0.500000,0.500000}%
\pgfsetstrokecolor{currentstroke}%
\pgfsetstrokeopacity{0.300000}%
\pgfsetdash{}{0pt}%
\pgfpathmoveto{\pgfqpoint{1.464078in}{0.683478in}}%
\pgfusepath{stroke}%
\end{pgfscope}%
\begin{pgfscope}%
\pgfpathrectangle{\pgfqpoint{0.647939in}{0.492442in}}{\pgfqpoint{3.079299in}{3.079299in}}%
\pgfusepath{clip}%
\pgfsetroundcap%
\pgfsetroundjoin%
\definecolor{currentfill}{rgb}{0.500000,0.500000,0.500000}%
\pgfsetfillcolor{currentfill}%
\pgfsetfillopacity{0.300000}%
\pgfsetlinewidth{0.301125pt}%
\definecolor{currentstroke}{rgb}{0.500000,0.500000,0.500000}%
\pgfsetstrokecolor{currentstroke}%
\pgfsetstrokeopacity{0.300000}%
\pgfsetdash{}{0pt}%
\pgfpathmoveto{\pgfqpoint{0.000000in}{0.000000in}}%
\pgfpathlineto{\pgfqpoint{0.000000in}{0.000000in}}%
\pgfpathclose%
\pgfusepath{stroke,fill}%
\end{pgfscope}%
\begin{pgfscope}%
\pgfpathrectangle{\pgfqpoint{0.647939in}{0.492442in}}{\pgfqpoint{3.079299in}{3.079299in}}%
\pgfusepath{clip}%
\pgfsetroundcap%
\pgfsetroundjoin%
\pgfsetlinewidth{0.301125pt}%
\definecolor{currentstroke}{rgb}{0.500000,0.500000,0.500000}%
\pgfsetstrokecolor{currentstroke}%
\pgfsetstrokeopacity{0.300000}%
\pgfsetdash{}{0pt}%
\pgfpathmoveto{\pgfqpoint{1.682163in}{0.544108in}}%
\pgfusepath{stroke}%
\end{pgfscope}%
\begin{pgfscope}%
\pgfpathrectangle{\pgfqpoint{0.647939in}{0.492442in}}{\pgfqpoint{3.079299in}{3.079299in}}%
\pgfusepath{clip}%
\pgfsetroundcap%
\pgfsetroundjoin%
\definecolor{currentfill}{rgb}{0.500000,0.500000,0.500000}%
\pgfsetfillcolor{currentfill}%
\pgfsetfillopacity{0.300000}%
\pgfsetlinewidth{0.301125pt}%
\definecolor{currentstroke}{rgb}{0.500000,0.500000,0.500000}%
\pgfsetstrokecolor{currentstroke}%
\pgfsetstrokeopacity{0.300000}%
\pgfsetdash{}{0pt}%
\pgfpathmoveto{\pgfqpoint{0.000000in}{0.000000in}}%
\pgfpathlineto{\pgfqpoint{0.000000in}{0.000000in}}%
\pgfpathclose%
\pgfusepath{stroke,fill}%
\end{pgfscope}%
\begin{pgfscope}%
\pgfpathrectangle{\pgfqpoint{0.647939in}{0.492442in}}{\pgfqpoint{3.079299in}{3.079299in}}%
\pgfusepath{clip}%
\pgfsetroundcap%
\pgfsetroundjoin%
\pgfsetlinewidth{0.301125pt}%
\definecolor{currentstroke}{rgb}{0.500000,0.500000,0.500000}%
\pgfsetstrokecolor{currentstroke}%
\pgfsetstrokeopacity{0.300000}%
\pgfsetdash{}{0pt}%
\pgfpathmoveto{\pgfqpoint{2.093118in}{0.499825in}}%
\pgfusepath{stroke}%
\end{pgfscope}%
\begin{pgfscope}%
\pgfpathrectangle{\pgfqpoint{0.647939in}{0.492442in}}{\pgfqpoint{3.079299in}{3.079299in}}%
\pgfusepath{clip}%
\pgfsetroundcap%
\pgfsetroundjoin%
\definecolor{currentfill}{rgb}{0.500000,0.500000,0.500000}%
\pgfsetfillcolor{currentfill}%
\pgfsetfillopacity{0.300000}%
\pgfsetlinewidth{0.301125pt}%
\definecolor{currentstroke}{rgb}{0.500000,0.500000,0.500000}%
\pgfsetstrokecolor{currentstroke}%
\pgfsetstrokeopacity{0.300000}%
\pgfsetdash{}{0pt}%
\pgfpathmoveto{\pgfqpoint{0.000000in}{0.000000in}}%
\pgfpathlineto{\pgfqpoint{0.000000in}{0.000000in}}%
\pgfpathclose%
\pgfusepath{stroke,fill}%
\end{pgfscope}%
\begin{pgfscope}%
\pgfpathrectangle{\pgfqpoint{0.647939in}{0.492442in}}{\pgfqpoint{3.079299in}{3.079299in}}%
\pgfusepath{clip}%
\pgfsetroundcap%
\pgfsetroundjoin%
\pgfsetlinewidth{0.301125pt}%
\definecolor{currentstroke}{rgb}{0.500000,0.500000,0.500000}%
\pgfsetstrokecolor{currentstroke}%
\pgfsetstrokeopacity{0.300000}%
\pgfsetdash{}{0pt}%
\pgfpathmoveto{\pgfqpoint{1.968472in}{0.551703in}}%
\pgfusepath{stroke}%
\end{pgfscope}%
\begin{pgfscope}%
\pgfpathrectangle{\pgfqpoint{0.647939in}{0.492442in}}{\pgfqpoint{3.079299in}{3.079299in}}%
\pgfusepath{clip}%
\pgfsetroundcap%
\pgfsetroundjoin%
\definecolor{currentfill}{rgb}{0.500000,0.500000,0.500000}%
\pgfsetfillcolor{currentfill}%
\pgfsetfillopacity{0.300000}%
\pgfsetlinewidth{0.301125pt}%
\definecolor{currentstroke}{rgb}{0.500000,0.500000,0.500000}%
\pgfsetstrokecolor{currentstroke}%
\pgfsetstrokeopacity{0.300000}%
\pgfsetdash{}{0pt}%
\pgfpathmoveto{\pgfqpoint{0.000000in}{0.000000in}}%
\pgfpathlineto{\pgfqpoint{0.000000in}{0.000000in}}%
\pgfpathclose%
\pgfusepath{stroke,fill}%
\end{pgfscope}%
\begin{pgfscope}%
\pgfpathrectangle{\pgfqpoint{0.647939in}{0.492442in}}{\pgfqpoint{3.079299in}{3.079299in}}%
\pgfusepath{clip}%
\pgfsetroundcap%
\pgfsetroundjoin%
\pgfsetlinewidth{0.301125pt}%
\definecolor{currentstroke}{rgb}{0.500000,0.500000,0.500000}%
\pgfsetstrokecolor{currentstroke}%
\pgfsetstrokeopacity{0.300000}%
\pgfsetdash{}{0pt}%
\pgfpathmoveto{\pgfqpoint{2.660300in}{0.544108in}}%
\pgfusepath{stroke}%
\end{pgfscope}%
\begin{pgfscope}%
\pgfpathrectangle{\pgfqpoint{0.647939in}{0.492442in}}{\pgfqpoint{3.079299in}{3.079299in}}%
\pgfusepath{clip}%
\pgfsetroundcap%
\pgfsetroundjoin%
\definecolor{currentfill}{rgb}{0.500000,0.500000,0.500000}%
\pgfsetfillcolor{currentfill}%
\pgfsetfillopacity{0.300000}%
\pgfsetlinewidth{0.301125pt}%
\definecolor{currentstroke}{rgb}{0.500000,0.500000,0.500000}%
\pgfsetstrokecolor{currentstroke}%
\pgfsetstrokeopacity{0.300000}%
\pgfsetdash{}{0pt}%
\pgfpathmoveto{\pgfqpoint{0.000000in}{0.000000in}}%
\pgfpathlineto{\pgfqpoint{0.000000in}{0.000000in}}%
\pgfpathclose%
\pgfusepath{stroke,fill}%
\end{pgfscope}%
\begin{pgfscope}%
\pgfpathrectangle{\pgfqpoint{0.647939in}{0.492442in}}{\pgfqpoint{3.079299in}{3.079299in}}%
\pgfusepath{clip}%
\pgfsetroundcap%
\pgfsetroundjoin%
\pgfsetlinewidth{0.301125pt}%
\definecolor{currentstroke}{rgb}{0.500000,0.500000,0.500000}%
\pgfsetstrokecolor{currentstroke}%
\pgfsetstrokeopacity{0.300000}%
\pgfsetdash{}{0pt}%
\pgfpathmoveto{\pgfqpoint{2.336614in}{0.652196in}}%
\pgfusepath{stroke}%
\end{pgfscope}%
\begin{pgfscope}%
\pgfpathrectangle{\pgfqpoint{0.647939in}{0.492442in}}{\pgfqpoint{3.079299in}{3.079299in}}%
\pgfusepath{clip}%
\pgfsetroundcap%
\pgfsetroundjoin%
\definecolor{currentfill}{rgb}{0.500000,0.500000,0.500000}%
\pgfsetfillcolor{currentfill}%
\pgfsetfillopacity{0.300000}%
\pgfsetlinewidth{0.301125pt}%
\definecolor{currentstroke}{rgb}{0.500000,0.500000,0.500000}%
\pgfsetstrokecolor{currentstroke}%
\pgfsetstrokeopacity{0.300000}%
\pgfsetdash{}{0pt}%
\pgfpathmoveto{\pgfqpoint{0.000000in}{0.000000in}}%
\pgfpathlineto{\pgfqpoint{0.000000in}{0.000000in}}%
\pgfpathclose%
\pgfusepath{stroke,fill}%
\end{pgfscope}%
\begin{pgfscope}%
\pgfpathrectangle{\pgfqpoint{0.647939in}{0.492442in}}{\pgfqpoint{3.079299in}{3.079299in}}%
\pgfusepath{clip}%
\pgfsetroundcap%
\pgfsetroundjoin%
\pgfsetlinewidth{0.301125pt}%
\definecolor{currentstroke}{rgb}{0.500000,0.500000,0.500000}%
\pgfsetstrokecolor{currentstroke}%
\pgfsetstrokeopacity{0.300000}%
\pgfsetdash{}{0pt}%
\pgfpathmoveto{\pgfqpoint{2.221736in}{0.748048in}}%
\pgfusepath{stroke}%
\end{pgfscope}%
\begin{pgfscope}%
\pgfpathrectangle{\pgfqpoint{0.647939in}{0.492442in}}{\pgfqpoint{3.079299in}{3.079299in}}%
\pgfusepath{clip}%
\pgfsetroundcap%
\pgfsetroundjoin%
\definecolor{currentfill}{rgb}{0.500000,0.500000,0.500000}%
\pgfsetfillcolor{currentfill}%
\pgfsetfillopacity{0.300000}%
\pgfsetlinewidth{0.301125pt}%
\definecolor{currentstroke}{rgb}{0.500000,0.500000,0.500000}%
\pgfsetstrokecolor{currentstroke}%
\pgfsetstrokeopacity{0.300000}%
\pgfsetdash{}{0pt}%
\pgfpathmoveto{\pgfqpoint{0.000000in}{0.000000in}}%
\pgfpathlineto{\pgfqpoint{0.000000in}{0.000000in}}%
\pgfpathclose%
\pgfusepath{stroke,fill}%
\end{pgfscope}%
\begin{pgfscope}%
\pgfpathrectangle{\pgfqpoint{0.647939in}{0.492442in}}{\pgfqpoint{3.079299in}{3.079299in}}%
\pgfusepath{clip}%
\pgfsetroundcap%
\pgfsetroundjoin%
\pgfsetlinewidth{0.301125pt}%
\definecolor{currentstroke}{rgb}{0.500000,0.500000,0.500000}%
\pgfsetstrokecolor{currentstroke}%
\pgfsetstrokeopacity{0.300000}%
\pgfsetdash{}{0pt}%
\pgfpathmoveto{\pgfqpoint{2.586627in}{0.810355in}}%
\pgfusepath{stroke}%
\end{pgfscope}%
\begin{pgfscope}%
\pgfpathrectangle{\pgfqpoint{0.647939in}{0.492442in}}{\pgfqpoint{3.079299in}{3.079299in}}%
\pgfusepath{clip}%
\pgfsetroundcap%
\pgfsetroundjoin%
\definecolor{currentfill}{rgb}{0.500000,0.500000,0.500000}%
\pgfsetfillcolor{currentfill}%
\pgfsetfillopacity{0.300000}%
\pgfsetlinewidth{0.301125pt}%
\definecolor{currentstroke}{rgb}{0.500000,0.500000,0.500000}%
\pgfsetstrokecolor{currentstroke}%
\pgfsetstrokeopacity{0.300000}%
\pgfsetdash{}{0pt}%
\pgfpathmoveto{\pgfqpoint{0.000000in}{0.000000in}}%
\pgfpathlineto{\pgfqpoint{0.000000in}{0.000000in}}%
\pgfpathclose%
\pgfusepath{stroke,fill}%
\end{pgfscope}%
\begin{pgfscope}%
\pgfpathrectangle{\pgfqpoint{0.647939in}{0.492442in}}{\pgfqpoint{3.079299in}{3.079299in}}%
\pgfusepath{clip}%
\pgfsetroundcap%
\pgfsetroundjoin%
\pgfsetlinewidth{0.301125pt}%
\definecolor{currentstroke}{rgb}{0.500000,0.500000,0.500000}%
\pgfsetstrokecolor{currentstroke}%
\pgfsetstrokeopacity{0.300000}%
\pgfsetdash{}{0pt}%
\pgfpathmoveto{\pgfqpoint{2.808745in}{0.848840in}}%
\pgfusepath{stroke}%
\end{pgfscope}%
\begin{pgfscope}%
\pgfpathrectangle{\pgfqpoint{0.647939in}{0.492442in}}{\pgfqpoint{3.079299in}{3.079299in}}%
\pgfusepath{clip}%
\pgfsetroundcap%
\pgfsetroundjoin%
\definecolor{currentfill}{rgb}{0.500000,0.500000,0.500000}%
\pgfsetfillcolor{currentfill}%
\pgfsetfillopacity{0.300000}%
\pgfsetlinewidth{0.301125pt}%
\definecolor{currentstroke}{rgb}{0.500000,0.500000,0.500000}%
\pgfsetstrokecolor{currentstroke}%
\pgfsetstrokeopacity{0.300000}%
\pgfsetdash{}{0pt}%
\pgfpathmoveto{\pgfqpoint{0.000000in}{0.000000in}}%
\pgfpathlineto{\pgfqpoint{0.000000in}{0.000000in}}%
\pgfpathclose%
\pgfusepath{stroke,fill}%
\end{pgfscope}%
\begin{pgfscope}%
\pgfpathrectangle{\pgfqpoint{0.647939in}{0.492442in}}{\pgfqpoint{3.079299in}{3.079299in}}%
\pgfusepath{clip}%
\pgfsetroundcap%
\pgfsetroundjoin%
\pgfsetlinewidth{0.301125pt}%
\definecolor{currentstroke}{rgb}{0.500000,0.500000,0.500000}%
\pgfsetstrokecolor{currentstroke}%
\pgfsetstrokeopacity{0.300000}%
\pgfsetdash{}{0pt}%
\pgfpathmoveto{\pgfqpoint{2.682574in}{0.969814in}}%
\pgfusepath{stroke}%
\end{pgfscope}%
\begin{pgfscope}%
\pgfpathrectangle{\pgfqpoint{0.647939in}{0.492442in}}{\pgfqpoint{3.079299in}{3.079299in}}%
\pgfusepath{clip}%
\pgfsetroundcap%
\pgfsetroundjoin%
\definecolor{currentfill}{rgb}{0.500000,0.500000,0.500000}%
\pgfsetfillcolor{currentfill}%
\pgfsetfillopacity{0.300000}%
\pgfsetlinewidth{0.301125pt}%
\definecolor{currentstroke}{rgb}{0.500000,0.500000,0.500000}%
\pgfsetstrokecolor{currentstroke}%
\pgfsetstrokeopacity{0.300000}%
\pgfsetdash{}{0pt}%
\pgfpathmoveto{\pgfqpoint{0.000000in}{0.000000in}}%
\pgfpathlineto{\pgfqpoint{0.000000in}{0.000000in}}%
\pgfpathclose%
\pgfusepath{stroke,fill}%
\end{pgfscope}%
\begin{pgfscope}%
\pgfpathrectangle{\pgfqpoint{0.647939in}{0.492442in}}{\pgfqpoint{3.079299in}{3.079299in}}%
\pgfusepath{clip}%
\pgfsetroundcap%
\pgfsetroundjoin%
\pgfsetlinewidth{0.301125pt}%
\definecolor{currentstroke}{rgb}{0.500000,0.500000,0.500000}%
\pgfsetstrokecolor{currentstroke}%
\pgfsetstrokeopacity{0.300000}%
\pgfsetdash{}{0pt}%
\pgfpathmoveto{\pgfqpoint{2.623191in}{1.073939in}}%
\pgfusepath{stroke}%
\end{pgfscope}%
\begin{pgfscope}%
\pgfpathrectangle{\pgfqpoint{0.647939in}{0.492442in}}{\pgfqpoint{3.079299in}{3.079299in}}%
\pgfusepath{clip}%
\pgfsetroundcap%
\pgfsetroundjoin%
\definecolor{currentfill}{rgb}{0.500000,0.500000,0.500000}%
\pgfsetfillcolor{currentfill}%
\pgfsetfillopacity{0.300000}%
\pgfsetlinewidth{0.301125pt}%
\definecolor{currentstroke}{rgb}{0.500000,0.500000,0.500000}%
\pgfsetstrokecolor{currentstroke}%
\pgfsetstrokeopacity{0.300000}%
\pgfsetdash{}{0pt}%
\pgfpathmoveto{\pgfqpoint{0.000000in}{0.000000in}}%
\pgfpathlineto{\pgfqpoint{0.000000in}{0.000000in}}%
\pgfpathclose%
\pgfusepath{stroke,fill}%
\end{pgfscope}%
\begin{pgfscope}%
\pgfpathrectangle{\pgfqpoint{0.647939in}{0.492442in}}{\pgfqpoint{3.079299in}{3.079299in}}%
\pgfusepath{clip}%
\pgfsetroundcap%
\pgfsetroundjoin%
\pgfsetlinewidth{0.301125pt}%
\definecolor{currentstroke}{rgb}{0.500000,0.500000,0.500000}%
\pgfsetstrokecolor{currentstroke}%
\pgfsetstrokeopacity{0.300000}%
\pgfsetdash{}{0pt}%
\pgfpathmoveto{\pgfqpoint{2.762941in}{1.126412in}}%
\pgfusepath{stroke}%
\end{pgfscope}%
\begin{pgfscope}%
\pgfpathrectangle{\pgfqpoint{0.647939in}{0.492442in}}{\pgfqpoint{3.079299in}{3.079299in}}%
\pgfusepath{clip}%
\pgfsetroundcap%
\pgfsetroundjoin%
\definecolor{currentfill}{rgb}{0.500000,0.500000,0.500000}%
\pgfsetfillcolor{currentfill}%
\pgfsetfillopacity{0.300000}%
\pgfsetlinewidth{0.301125pt}%
\definecolor{currentstroke}{rgb}{0.500000,0.500000,0.500000}%
\pgfsetstrokecolor{currentstroke}%
\pgfsetstrokeopacity{0.300000}%
\pgfsetdash{}{0pt}%
\pgfpathmoveto{\pgfqpoint{0.000000in}{0.000000in}}%
\pgfpathlineto{\pgfqpoint{0.000000in}{0.000000in}}%
\pgfpathclose%
\pgfusepath{stroke,fill}%
\end{pgfscope}%
\begin{pgfscope}%
\pgfpathrectangle{\pgfqpoint{0.647939in}{0.492442in}}{\pgfqpoint{3.079299in}{3.079299in}}%
\pgfusepath{clip}%
\pgfsetroundcap%
\pgfsetroundjoin%
\pgfsetlinewidth{0.301125pt}%
\definecolor{currentstroke}{rgb}{0.500000,0.500000,0.500000}%
\pgfsetstrokecolor{currentstroke}%
\pgfsetstrokeopacity{0.300000}%
\pgfsetdash{}{0pt}%
\pgfpathmoveto{\pgfqpoint{2.771713in}{1.215846in}}%
\pgfusepath{stroke}%
\end{pgfscope}%
\begin{pgfscope}%
\pgfpathrectangle{\pgfqpoint{0.647939in}{0.492442in}}{\pgfqpoint{3.079299in}{3.079299in}}%
\pgfusepath{clip}%
\pgfsetroundcap%
\pgfsetroundjoin%
\definecolor{currentfill}{rgb}{0.500000,0.500000,0.500000}%
\pgfsetfillcolor{currentfill}%
\pgfsetfillopacity{0.300000}%
\pgfsetlinewidth{0.301125pt}%
\definecolor{currentstroke}{rgb}{0.500000,0.500000,0.500000}%
\pgfsetstrokecolor{currentstroke}%
\pgfsetstrokeopacity{0.300000}%
\pgfsetdash{}{0pt}%
\pgfpathmoveto{\pgfqpoint{0.000000in}{0.000000in}}%
\pgfpathlineto{\pgfqpoint{0.000000in}{0.000000in}}%
\pgfpathclose%
\pgfusepath{stroke,fill}%
\end{pgfscope}%
\begin{pgfscope}%
\pgfpathrectangle{\pgfqpoint{0.647939in}{0.492442in}}{\pgfqpoint{3.079299in}{3.079299in}}%
\pgfusepath{clip}%
\pgfsetroundcap%
\pgfsetroundjoin%
\pgfsetlinewidth{0.301125pt}%
\definecolor{currentstroke}{rgb}{0.500000,0.500000,0.500000}%
\pgfsetstrokecolor{currentstroke}%
\pgfsetstrokeopacity{0.300000}%
\pgfsetdash{}{0pt}%
\pgfpathmoveto{\pgfqpoint{2.781912in}{1.307183in}}%
\pgfusepath{stroke}%
\end{pgfscope}%
\begin{pgfscope}%
\pgfpathrectangle{\pgfqpoint{0.647939in}{0.492442in}}{\pgfqpoint{3.079299in}{3.079299in}}%
\pgfusepath{clip}%
\pgfsetroundcap%
\pgfsetroundjoin%
\definecolor{currentfill}{rgb}{0.500000,0.500000,0.500000}%
\pgfsetfillcolor{currentfill}%
\pgfsetfillopacity{0.300000}%
\pgfsetlinewidth{0.301125pt}%
\definecolor{currentstroke}{rgb}{0.500000,0.500000,0.500000}%
\pgfsetstrokecolor{currentstroke}%
\pgfsetstrokeopacity{0.300000}%
\pgfsetdash{}{0pt}%
\pgfpathmoveto{\pgfqpoint{0.000000in}{0.000000in}}%
\pgfpathlineto{\pgfqpoint{0.000000in}{0.000000in}}%
\pgfpathclose%
\pgfusepath{stroke,fill}%
\end{pgfscope}%
\begin{pgfscope}%
\pgfpathrectangle{\pgfqpoint{0.647939in}{0.492442in}}{\pgfqpoint{3.079299in}{3.079299in}}%
\pgfusepath{clip}%
\pgfsetroundcap%
\pgfsetroundjoin%
\pgfsetlinewidth{0.301125pt}%
\definecolor{currentstroke}{rgb}{0.500000,0.500000,0.500000}%
\pgfsetstrokecolor{currentstroke}%
\pgfsetstrokeopacity{0.300000}%
\pgfsetdash{}{0pt}%
\pgfpathmoveto{\pgfqpoint{2.855870in}{1.371727in}}%
\pgfusepath{stroke}%
\end{pgfscope}%
\begin{pgfscope}%
\pgfpathrectangle{\pgfqpoint{0.647939in}{0.492442in}}{\pgfqpoint{3.079299in}{3.079299in}}%
\pgfusepath{clip}%
\pgfsetroundcap%
\pgfsetroundjoin%
\definecolor{currentfill}{rgb}{0.500000,0.500000,0.500000}%
\pgfsetfillcolor{currentfill}%
\pgfsetfillopacity{0.300000}%
\pgfsetlinewidth{0.301125pt}%
\definecolor{currentstroke}{rgb}{0.500000,0.500000,0.500000}%
\pgfsetstrokecolor{currentstroke}%
\pgfsetstrokeopacity{0.300000}%
\pgfsetdash{}{0pt}%
\pgfpathmoveto{\pgfqpoint{0.000000in}{0.000000in}}%
\pgfpathlineto{\pgfqpoint{0.000000in}{0.000000in}}%
\pgfpathclose%
\pgfusepath{stroke,fill}%
\end{pgfscope}%
\begin{pgfscope}%
\pgfpathrectangle{\pgfqpoint{0.647939in}{0.492442in}}{\pgfqpoint{3.079299in}{3.079299in}}%
\pgfusepath{clip}%
\pgfsetroundcap%
\pgfsetroundjoin%
\pgfsetlinewidth{0.301125pt}%
\definecolor{currentstroke}{rgb}{0.500000,0.500000,0.500000}%
\pgfsetstrokecolor{currentstroke}%
\pgfsetstrokeopacity{0.300000}%
\pgfsetdash{}{0pt}%
\pgfpathmoveto{\pgfqpoint{2.868656in}{1.464841in}}%
\pgfusepath{stroke}%
\end{pgfscope}%
\begin{pgfscope}%
\pgfpathrectangle{\pgfqpoint{0.647939in}{0.492442in}}{\pgfqpoint{3.079299in}{3.079299in}}%
\pgfusepath{clip}%
\pgfsetroundcap%
\pgfsetroundjoin%
\definecolor{currentfill}{rgb}{0.500000,0.500000,0.500000}%
\pgfsetfillcolor{currentfill}%
\pgfsetfillopacity{0.300000}%
\pgfsetlinewidth{0.301125pt}%
\definecolor{currentstroke}{rgb}{0.500000,0.500000,0.500000}%
\pgfsetstrokecolor{currentstroke}%
\pgfsetstrokeopacity{0.300000}%
\pgfsetdash{}{0pt}%
\pgfpathmoveto{\pgfqpoint{0.000000in}{0.000000in}}%
\pgfpathlineto{\pgfqpoint{0.000000in}{0.000000in}}%
\pgfpathclose%
\pgfusepath{stroke,fill}%
\end{pgfscope}%
\begin{pgfscope}%
\pgfpathrectangle{\pgfqpoint{0.647939in}{0.492442in}}{\pgfqpoint{3.079299in}{3.079299in}}%
\pgfusepath{clip}%
\pgfsetroundcap%
\pgfsetroundjoin%
\pgfsetlinewidth{0.301125pt}%
\definecolor{currentstroke}{rgb}{0.500000,0.500000,0.500000}%
\pgfsetstrokecolor{currentstroke}%
\pgfsetstrokeopacity{0.300000}%
\pgfsetdash{}{0pt}%
\pgfpathmoveto{\pgfqpoint{2.941798in}{1.523943in}}%
\pgfusepath{stroke}%
\end{pgfscope}%
\begin{pgfscope}%
\pgfpathrectangle{\pgfqpoint{0.647939in}{0.492442in}}{\pgfqpoint{3.079299in}{3.079299in}}%
\pgfusepath{clip}%
\pgfsetroundcap%
\pgfsetroundjoin%
\definecolor{currentfill}{rgb}{0.500000,0.500000,0.500000}%
\pgfsetfillcolor{currentfill}%
\pgfsetfillopacity{0.300000}%
\pgfsetlinewidth{0.301125pt}%
\definecolor{currentstroke}{rgb}{0.500000,0.500000,0.500000}%
\pgfsetstrokecolor{currentstroke}%
\pgfsetstrokeopacity{0.300000}%
\pgfsetdash{}{0pt}%
\pgfpathmoveto{\pgfqpoint{0.000000in}{0.000000in}}%
\pgfpathlineto{\pgfqpoint{0.000000in}{0.000000in}}%
\pgfpathclose%
\pgfusepath{stroke,fill}%
\end{pgfscope}%
\begin{pgfscope}%
\pgfpathrectangle{\pgfqpoint{0.647939in}{0.492442in}}{\pgfqpoint{3.079299in}{3.079299in}}%
\pgfusepath{clip}%
\pgfsetroundcap%
\pgfsetroundjoin%
\pgfsetlinewidth{0.301125pt}%
\definecolor{currentstroke}{rgb}{0.500000,0.500000,0.500000}%
\pgfsetstrokecolor{currentstroke}%
\pgfsetstrokeopacity{0.300000}%
\pgfsetdash{}{0pt}%
\pgfpathmoveto{\pgfqpoint{3.028892in}{1.671272in}}%
\pgfusepath{stroke}%
\end{pgfscope}%
\begin{pgfscope}%
\pgfpathrectangle{\pgfqpoint{0.647939in}{0.492442in}}{\pgfqpoint{3.079299in}{3.079299in}}%
\pgfusepath{clip}%
\pgfsetroundcap%
\pgfsetroundjoin%
\definecolor{currentfill}{rgb}{0.500000,0.500000,0.500000}%
\pgfsetfillcolor{currentfill}%
\pgfsetfillopacity{0.300000}%
\pgfsetlinewidth{0.301125pt}%
\definecolor{currentstroke}{rgb}{0.500000,0.500000,0.500000}%
\pgfsetstrokecolor{currentstroke}%
\pgfsetstrokeopacity{0.300000}%
\pgfsetdash{}{0pt}%
\pgfpathmoveto{\pgfqpoint{0.000000in}{0.000000in}}%
\pgfpathlineto{\pgfqpoint{0.000000in}{0.000000in}}%
\pgfpathclose%
\pgfusepath{stroke,fill}%
\end{pgfscope}%
\begin{pgfscope}%
\pgfpathrectangle{\pgfqpoint{0.647939in}{0.492442in}}{\pgfqpoint{3.079299in}{3.079299in}}%
\pgfusepath{clip}%
\pgfsetroundcap%
\pgfsetroundjoin%
\pgfsetlinewidth{0.301125pt}%
\definecolor{currentstroke}{rgb}{0.500000,0.500000,0.500000}%
\pgfsetstrokecolor{currentstroke}%
\pgfsetstrokeopacity{0.300000}%
\pgfsetdash{}{0pt}%
\pgfpathmoveto{\pgfqpoint{2.980805in}{1.966818in}}%
\pgfusepath{stroke}%
\end{pgfscope}%
\begin{pgfscope}%
\pgfpathrectangle{\pgfqpoint{0.647939in}{0.492442in}}{\pgfqpoint{3.079299in}{3.079299in}}%
\pgfusepath{clip}%
\pgfsetroundcap%
\pgfsetroundjoin%
\definecolor{currentfill}{rgb}{0.500000,0.500000,0.500000}%
\pgfsetfillcolor{currentfill}%
\pgfsetfillopacity{0.300000}%
\pgfsetlinewidth{0.301125pt}%
\definecolor{currentstroke}{rgb}{0.500000,0.500000,0.500000}%
\pgfsetstrokecolor{currentstroke}%
\pgfsetstrokeopacity{0.300000}%
\pgfsetdash{}{0pt}%
\pgfpathmoveto{\pgfqpoint{0.000000in}{0.000000in}}%
\pgfpathlineto{\pgfqpoint{0.000000in}{0.000000in}}%
\pgfpathclose%
\pgfusepath{stroke,fill}%
\end{pgfscope}%
\begin{pgfscope}%
\pgfpathrectangle{\pgfqpoint{0.647939in}{0.492442in}}{\pgfqpoint{3.079299in}{3.079299in}}%
\pgfusepath{clip}%
\pgfsetroundcap%
\pgfsetroundjoin%
\pgfsetlinewidth{0.301125pt}%
\definecolor{currentstroke}{rgb}{0.500000,0.500000,0.500000}%
\pgfsetstrokecolor{currentstroke}%
\pgfsetstrokeopacity{0.300000}%
\pgfsetdash{}{0pt}%
\pgfpathmoveto{\pgfqpoint{3.478449in}{1.571828in}}%
\pgfusepath{stroke}%
\end{pgfscope}%
\begin{pgfscope}%
\pgfpathrectangle{\pgfqpoint{0.647939in}{0.492442in}}{\pgfqpoint{3.079299in}{3.079299in}}%
\pgfusepath{clip}%
\pgfsetroundcap%
\pgfsetroundjoin%
\definecolor{currentfill}{rgb}{0.500000,0.500000,0.500000}%
\pgfsetfillcolor{currentfill}%
\pgfsetfillopacity{0.300000}%
\pgfsetlinewidth{0.301125pt}%
\definecolor{currentstroke}{rgb}{0.500000,0.500000,0.500000}%
\pgfsetstrokecolor{currentstroke}%
\pgfsetstrokeopacity{0.300000}%
\pgfsetdash{}{0pt}%
\pgfpathmoveto{\pgfqpoint{0.000000in}{0.000000in}}%
\pgfpathlineto{\pgfqpoint{0.000000in}{0.000000in}}%
\pgfpathclose%
\pgfusepath{stroke,fill}%
\end{pgfscope}%
\begin{pgfscope}%
\pgfpathrectangle{\pgfqpoint{0.647939in}{0.492442in}}{\pgfqpoint{3.079299in}{3.079299in}}%
\pgfusepath{clip}%
\pgfsetroundcap%
\pgfsetroundjoin%
\pgfsetlinewidth{0.301125pt}%
\definecolor{currentstroke}{rgb}{0.500000,0.500000,0.500000}%
\pgfsetstrokecolor{currentstroke}%
\pgfsetstrokeopacity{0.300000}%
\pgfsetdash{}{0pt}%
\pgfpathmoveto{\pgfqpoint{3.081581in}{2.558089in}}%
\pgfusepath{stroke}%
\end{pgfscope}%
\begin{pgfscope}%
\pgfpathrectangle{\pgfqpoint{0.647939in}{0.492442in}}{\pgfqpoint{3.079299in}{3.079299in}}%
\pgfusepath{clip}%
\pgfsetroundcap%
\pgfsetroundjoin%
\definecolor{currentfill}{rgb}{0.500000,0.500000,0.500000}%
\pgfsetfillcolor{currentfill}%
\pgfsetfillopacity{0.300000}%
\pgfsetlinewidth{0.301125pt}%
\definecolor{currentstroke}{rgb}{0.500000,0.500000,0.500000}%
\pgfsetstrokecolor{currentstroke}%
\pgfsetstrokeopacity{0.300000}%
\pgfsetdash{}{0pt}%
\pgfpathmoveto{\pgfqpoint{0.000000in}{0.000000in}}%
\pgfpathlineto{\pgfqpoint{0.000000in}{0.000000in}}%
\pgfpathclose%
\pgfusepath{stroke,fill}%
\end{pgfscope}%
\begin{pgfscope}%
\pgfpathrectangle{\pgfqpoint{0.647939in}{0.492442in}}{\pgfqpoint{3.079299in}{3.079299in}}%
\pgfusepath{clip}%
\pgfsetroundcap%
\pgfsetroundjoin%
\pgfsetlinewidth{0.301125pt}%
\definecolor{currentstroke}{rgb}{0.500000,0.500000,0.500000}%
\pgfsetstrokecolor{currentstroke}%
\pgfsetstrokeopacity{0.300000}%
\pgfsetdash{}{0pt}%
\pgfpathmoveto{\pgfqpoint{3.225692in}{2.564206in}}%
\pgfusepath{stroke}%
\end{pgfscope}%
\begin{pgfscope}%
\pgfpathrectangle{\pgfqpoint{0.647939in}{0.492442in}}{\pgfqpoint{3.079299in}{3.079299in}}%
\pgfusepath{clip}%
\pgfsetroundcap%
\pgfsetroundjoin%
\definecolor{currentfill}{rgb}{0.500000,0.500000,0.500000}%
\pgfsetfillcolor{currentfill}%
\pgfsetfillopacity{0.300000}%
\pgfsetlinewidth{0.301125pt}%
\definecolor{currentstroke}{rgb}{0.500000,0.500000,0.500000}%
\pgfsetstrokecolor{currentstroke}%
\pgfsetstrokeopacity{0.300000}%
\pgfsetdash{}{0pt}%
\pgfpathmoveto{\pgfqpoint{0.000000in}{0.000000in}}%
\pgfpathlineto{\pgfqpoint{0.000000in}{0.000000in}}%
\pgfpathclose%
\pgfusepath{stroke,fill}%
\end{pgfscope}%
\begin{pgfscope}%
\pgfpathrectangle{\pgfqpoint{0.647939in}{0.492442in}}{\pgfqpoint{3.079299in}{3.079299in}}%
\pgfusepath{clip}%
\pgfsetroundcap%
\pgfsetroundjoin%
\pgfsetlinewidth{0.301125pt}%
\definecolor{currentstroke}{rgb}{0.500000,0.500000,0.500000}%
\pgfsetstrokecolor{currentstroke}%
\pgfsetstrokeopacity{0.300000}%
\pgfsetdash{}{0pt}%
\pgfpathmoveto{\pgfqpoint{3.323768in}{2.325218in}}%
\pgfusepath{stroke}%
\end{pgfscope}%
\begin{pgfscope}%
\pgfpathrectangle{\pgfqpoint{0.647939in}{0.492442in}}{\pgfqpoint{3.079299in}{3.079299in}}%
\pgfusepath{clip}%
\pgfsetroundcap%
\pgfsetroundjoin%
\definecolor{currentfill}{rgb}{0.500000,0.500000,0.500000}%
\pgfsetfillcolor{currentfill}%
\pgfsetfillopacity{0.300000}%
\pgfsetlinewidth{0.301125pt}%
\definecolor{currentstroke}{rgb}{0.500000,0.500000,0.500000}%
\pgfsetstrokecolor{currentstroke}%
\pgfsetstrokeopacity{0.300000}%
\pgfsetdash{}{0pt}%
\pgfpathmoveto{\pgfqpoint{0.000000in}{0.000000in}}%
\pgfpathlineto{\pgfqpoint{0.000000in}{0.000000in}}%
\pgfpathclose%
\pgfusepath{stroke,fill}%
\end{pgfscope}%
\begin{pgfscope}%
\pgfpathrectangle{\pgfqpoint{0.647939in}{0.492442in}}{\pgfqpoint{3.079299in}{3.079299in}}%
\pgfusepath{clip}%
\pgfsetroundcap%
\pgfsetroundjoin%
\pgfsetlinewidth{0.301125pt}%
\definecolor{currentstroke}{rgb}{0.500000,0.500000,0.500000}%
\pgfsetstrokecolor{currentstroke}%
\pgfsetstrokeopacity{0.300000}%
\pgfsetdash{}{0pt}%
\pgfpathmoveto{\pgfqpoint{3.566479in}{1.990171in}}%
\pgfusepath{stroke}%
\end{pgfscope}%
\begin{pgfscope}%
\pgfpathrectangle{\pgfqpoint{0.647939in}{0.492442in}}{\pgfqpoint{3.079299in}{3.079299in}}%
\pgfusepath{clip}%
\pgfsetroundcap%
\pgfsetroundjoin%
\definecolor{currentfill}{rgb}{0.500000,0.500000,0.500000}%
\pgfsetfillcolor{currentfill}%
\pgfsetfillopacity{0.300000}%
\pgfsetlinewidth{0.301125pt}%
\definecolor{currentstroke}{rgb}{0.500000,0.500000,0.500000}%
\pgfsetstrokecolor{currentstroke}%
\pgfsetstrokeopacity{0.300000}%
\pgfsetdash{}{0pt}%
\pgfpathmoveto{\pgfqpoint{0.000000in}{0.000000in}}%
\pgfpathlineto{\pgfqpoint{0.000000in}{0.000000in}}%
\pgfpathclose%
\pgfusepath{stroke,fill}%
\end{pgfscope}%
\begin{pgfscope}%
\pgfpathrectangle{\pgfqpoint{0.647939in}{0.492442in}}{\pgfqpoint{3.079299in}{3.079299in}}%
\pgfusepath{clip}%
\pgfsetroundcap%
\pgfsetroundjoin%
\pgfsetlinewidth{0.301125pt}%
\definecolor{currentstroke}{rgb}{0.500000,0.500000,0.500000}%
\pgfsetstrokecolor{currentstroke}%
\pgfsetstrokeopacity{0.300000}%
\pgfsetdash{}{0pt}%
\pgfpathmoveto{\pgfqpoint{3.405887in}{2.579350in}}%
\pgfusepath{stroke}%
\end{pgfscope}%
\begin{pgfscope}%
\pgfpathrectangle{\pgfqpoint{0.647939in}{0.492442in}}{\pgfqpoint{3.079299in}{3.079299in}}%
\pgfusepath{clip}%
\pgfsetroundcap%
\pgfsetroundjoin%
\definecolor{currentfill}{rgb}{0.500000,0.500000,0.500000}%
\pgfsetfillcolor{currentfill}%
\pgfsetfillopacity{0.300000}%
\pgfsetlinewidth{0.301125pt}%
\definecolor{currentstroke}{rgb}{0.500000,0.500000,0.500000}%
\pgfsetstrokecolor{currentstroke}%
\pgfsetstrokeopacity{0.300000}%
\pgfsetdash{}{0pt}%
\pgfpathmoveto{\pgfqpoint{0.000000in}{0.000000in}}%
\pgfpathlineto{\pgfqpoint{0.000000in}{0.000000in}}%
\pgfpathclose%
\pgfusepath{stroke,fill}%
\end{pgfscope}%
\begin{pgfscope}%
\pgfpathrectangle{\pgfqpoint{0.647939in}{0.492442in}}{\pgfqpoint{3.079299in}{3.079299in}}%
\pgfusepath{clip}%
\pgfsetroundcap%
\pgfsetroundjoin%
\pgfsetlinewidth{0.301125pt}%
\definecolor{currentstroke}{rgb}{0.500000,0.500000,0.500000}%
\pgfsetstrokecolor{currentstroke}%
\pgfsetstrokeopacity{0.300000}%
\pgfsetdash{}{0pt}%
\pgfpathmoveto{\pgfqpoint{3.506077in}{2.620384in}}%
\pgfusepath{stroke}%
\end{pgfscope}%
\begin{pgfscope}%
\pgfpathrectangle{\pgfqpoint{0.647939in}{0.492442in}}{\pgfqpoint{3.079299in}{3.079299in}}%
\pgfusepath{clip}%
\pgfsetroundcap%
\pgfsetroundjoin%
\definecolor{currentfill}{rgb}{0.500000,0.500000,0.500000}%
\pgfsetfillcolor{currentfill}%
\pgfsetfillopacity{0.300000}%
\pgfsetlinewidth{0.301125pt}%
\definecolor{currentstroke}{rgb}{0.500000,0.500000,0.500000}%
\pgfsetstrokecolor{currentstroke}%
\pgfsetstrokeopacity{0.300000}%
\pgfsetdash{}{0pt}%
\pgfpathmoveto{\pgfqpoint{0.000000in}{0.000000in}}%
\pgfpathlineto{\pgfqpoint{0.000000in}{0.000000in}}%
\pgfpathclose%
\pgfusepath{stroke,fill}%
\end{pgfscope}%
\begin{pgfscope}%
\pgfpathrectangle{\pgfqpoint{0.647939in}{0.492442in}}{\pgfqpoint{3.079299in}{3.079299in}}%
\pgfusepath{clip}%
\pgfsetroundcap%
\pgfsetroundjoin%
\pgfsetlinewidth{0.301125pt}%
\definecolor{currentstroke}{rgb}{0.500000,0.500000,0.500000}%
\pgfsetstrokecolor{currentstroke}%
\pgfsetstrokeopacity{0.300000}%
\pgfsetdash{}{0pt}%
\pgfpathmoveto{\pgfqpoint{3.591033in}{2.650840in}}%
\pgfusepath{stroke}%
\end{pgfscope}%
\begin{pgfscope}%
\pgfpathrectangle{\pgfqpoint{0.647939in}{0.492442in}}{\pgfqpoint{3.079299in}{3.079299in}}%
\pgfusepath{clip}%
\pgfsetroundcap%
\pgfsetroundjoin%
\definecolor{currentfill}{rgb}{0.500000,0.500000,0.500000}%
\pgfsetfillcolor{currentfill}%
\pgfsetfillopacity{0.300000}%
\pgfsetlinewidth{0.301125pt}%
\definecolor{currentstroke}{rgb}{0.500000,0.500000,0.500000}%
\pgfsetstrokecolor{currentstroke}%
\pgfsetstrokeopacity{0.300000}%
\pgfsetdash{}{0pt}%
\pgfpathmoveto{\pgfqpoint{0.000000in}{0.000000in}}%
\pgfpathlineto{\pgfqpoint{0.000000in}{0.000000in}}%
\pgfpathclose%
\pgfusepath{stroke,fill}%
\end{pgfscope}%
\begin{pgfscope}%
\pgfpathrectangle{\pgfqpoint{0.647939in}{0.492442in}}{\pgfqpoint{3.079299in}{3.079299in}}%
\pgfusepath{clip}%
\pgfsetroundcap%
\pgfsetroundjoin%
\pgfsetlinewidth{0.301125pt}%
\definecolor{currentstroke}{rgb}{0.500000,0.500000,0.500000}%
\pgfsetstrokecolor{currentstroke}%
\pgfsetstrokeopacity{0.300000}%
\pgfsetdash{}{0pt}%
\pgfpathmoveto{\pgfqpoint{3.663304in}{2.534888in}}%
\pgfusepath{stroke}%
\end{pgfscope}%
\begin{pgfscope}%
\pgfpathrectangle{\pgfqpoint{0.647939in}{0.492442in}}{\pgfqpoint{3.079299in}{3.079299in}}%
\pgfusepath{clip}%
\pgfsetroundcap%
\pgfsetroundjoin%
\definecolor{currentfill}{rgb}{0.500000,0.500000,0.500000}%
\pgfsetfillcolor{currentfill}%
\pgfsetfillopacity{0.300000}%
\pgfsetlinewidth{0.301125pt}%
\definecolor{currentstroke}{rgb}{0.500000,0.500000,0.500000}%
\pgfsetstrokecolor{currentstroke}%
\pgfsetstrokeopacity{0.300000}%
\pgfsetdash{}{0pt}%
\pgfpathmoveto{\pgfqpoint{0.000000in}{0.000000in}}%
\pgfpathlineto{\pgfqpoint{0.000000in}{0.000000in}}%
\pgfpathclose%
\pgfusepath{stroke,fill}%
\end{pgfscope}%
\begin{pgfscope}%
\pgfpathrectangle{\pgfqpoint{0.647939in}{0.492442in}}{\pgfqpoint{3.079299in}{3.079299in}}%
\pgfusepath{clip}%
\pgfsetroundcap%
\pgfsetroundjoin%
\pgfsetlinewidth{0.301125pt}%
\definecolor{currentstroke}{rgb}{0.500000,0.500000,0.500000}%
\pgfsetstrokecolor{currentstroke}%
\pgfsetstrokeopacity{0.300000}%
\pgfsetdash{}{0pt}%
\pgfpathmoveto{\pgfqpoint{3.704419in}{2.614475in}}%
\pgfusepath{stroke}%
\end{pgfscope}%
\begin{pgfscope}%
\pgfpathrectangle{\pgfqpoint{0.647939in}{0.492442in}}{\pgfqpoint{3.079299in}{3.079299in}}%
\pgfusepath{clip}%
\pgfsetroundcap%
\pgfsetroundjoin%
\definecolor{currentfill}{rgb}{0.500000,0.500000,0.500000}%
\pgfsetfillcolor{currentfill}%
\pgfsetfillopacity{0.300000}%
\pgfsetlinewidth{0.301125pt}%
\definecolor{currentstroke}{rgb}{0.500000,0.500000,0.500000}%
\pgfsetstrokecolor{currentstroke}%
\pgfsetstrokeopacity{0.300000}%
\pgfsetdash{}{0pt}%
\pgfpathmoveto{\pgfqpoint{0.000000in}{0.000000in}}%
\pgfpathlineto{\pgfqpoint{0.000000in}{0.000000in}}%
\pgfpathclose%
\pgfusepath{stroke,fill}%
\end{pgfscope}%
\begin{pgfscope}%
\pgfpathrectangle{\pgfqpoint{0.647939in}{0.492442in}}{\pgfqpoint{3.079299in}{3.079299in}}%
\pgfusepath{clip}%
\pgfsetroundcap%
\pgfsetroundjoin%
\pgfsetlinewidth{0.301125pt}%
\definecolor{currentstroke}{rgb}{0.500000,0.500000,0.500000}%
\pgfsetstrokecolor{currentstroke}%
\pgfsetstrokeopacity{0.300000}%
\pgfsetdash{}{0pt}%
\pgfpathmoveto{\pgfqpoint{2.181044in}{2.765011in}}%
\pgfusepath{stroke}%
\end{pgfscope}%
\begin{pgfscope}%
\pgfpathrectangle{\pgfqpoint{0.647939in}{0.492442in}}{\pgfqpoint{3.079299in}{3.079299in}}%
\pgfusepath{clip}%
\pgfsetroundcap%
\pgfsetroundjoin%
\definecolor{currentfill}{rgb}{0.500000,0.500000,0.500000}%
\pgfsetfillcolor{currentfill}%
\pgfsetfillopacity{0.300000}%
\pgfsetlinewidth{0.301125pt}%
\definecolor{currentstroke}{rgb}{0.500000,0.500000,0.500000}%
\pgfsetstrokecolor{currentstroke}%
\pgfsetstrokeopacity{0.300000}%
\pgfsetdash{}{0pt}%
\pgfpathmoveto{\pgfqpoint{0.000000in}{0.000000in}}%
\pgfpathlineto{\pgfqpoint{0.000000in}{0.000000in}}%
\pgfpathclose%
\pgfusepath{stroke,fill}%
\end{pgfscope}%
\begin{pgfscope}%
\pgfpathrectangle{\pgfqpoint{0.647939in}{0.492442in}}{\pgfqpoint{3.079299in}{3.079299in}}%
\pgfusepath{clip}%
\pgfsetroundcap%
\pgfsetroundjoin%
\pgfsetlinewidth{0.301125pt}%
\definecolor{currentstroke}{rgb}{0.500000,0.500000,0.500000}%
\pgfsetstrokecolor{currentstroke}%
\pgfsetstrokeopacity{0.300000}%
\pgfsetdash{}{0pt}%
\pgfpathmoveto{\pgfqpoint{2.052307in}{3.030753in}}%
\pgfusepath{stroke}%
\end{pgfscope}%
\begin{pgfscope}%
\pgfpathrectangle{\pgfqpoint{0.647939in}{0.492442in}}{\pgfqpoint{3.079299in}{3.079299in}}%
\pgfusepath{clip}%
\pgfsetroundcap%
\pgfsetroundjoin%
\definecolor{currentfill}{rgb}{0.500000,0.500000,0.500000}%
\pgfsetfillcolor{currentfill}%
\pgfsetfillopacity{0.300000}%
\pgfsetlinewidth{0.301125pt}%
\definecolor{currentstroke}{rgb}{0.500000,0.500000,0.500000}%
\pgfsetstrokecolor{currentstroke}%
\pgfsetstrokeopacity{0.300000}%
\pgfsetdash{}{0pt}%
\pgfpathmoveto{\pgfqpoint{0.000000in}{0.000000in}}%
\pgfpathlineto{\pgfqpoint{0.000000in}{0.000000in}}%
\pgfpathclose%
\pgfusepath{stroke,fill}%
\end{pgfscope}%
\begin{pgfscope}%
\pgfpathrectangle{\pgfqpoint{0.647939in}{0.492442in}}{\pgfqpoint{3.079299in}{3.079299in}}%
\pgfusepath{clip}%
\pgfsetroundcap%
\pgfsetroundjoin%
\pgfsetlinewidth{0.301125pt}%
\definecolor{currentstroke}{rgb}{0.500000,0.500000,0.500000}%
\pgfsetstrokecolor{currentstroke}%
\pgfsetstrokeopacity{0.300000}%
\pgfsetdash{}{0pt}%
\pgfpathmoveto{\pgfqpoint{1.912672in}{3.196092in}}%
\pgfusepath{stroke}%
\end{pgfscope}%
\begin{pgfscope}%
\pgfpathrectangle{\pgfqpoint{0.647939in}{0.492442in}}{\pgfqpoint{3.079299in}{3.079299in}}%
\pgfusepath{clip}%
\pgfsetroundcap%
\pgfsetroundjoin%
\definecolor{currentfill}{rgb}{0.500000,0.500000,0.500000}%
\pgfsetfillcolor{currentfill}%
\pgfsetfillopacity{0.300000}%
\pgfsetlinewidth{0.301125pt}%
\definecolor{currentstroke}{rgb}{0.500000,0.500000,0.500000}%
\pgfsetstrokecolor{currentstroke}%
\pgfsetstrokeopacity{0.300000}%
\pgfsetdash{}{0pt}%
\pgfpathmoveto{\pgfqpoint{0.000000in}{0.000000in}}%
\pgfpathlineto{\pgfqpoint{0.000000in}{0.000000in}}%
\pgfpathclose%
\pgfusepath{stroke,fill}%
\end{pgfscope}%
\begin{pgfscope}%
\pgfpathrectangle{\pgfqpoint{0.647939in}{0.492442in}}{\pgfqpoint{3.079299in}{3.079299in}}%
\pgfusepath{clip}%
\pgfsetroundcap%
\pgfsetroundjoin%
\pgfsetlinewidth{0.301125pt}%
\definecolor{currentstroke}{rgb}{0.500000,0.500000,0.500000}%
\pgfsetstrokecolor{currentstroke}%
\pgfsetstrokeopacity{0.300000}%
\pgfsetdash{}{0pt}%
\pgfpathmoveto{\pgfqpoint{1.868445in}{3.318455in}}%
\pgfusepath{stroke}%
\end{pgfscope}%
\begin{pgfscope}%
\pgfpathrectangle{\pgfqpoint{0.647939in}{0.492442in}}{\pgfqpoint{3.079299in}{3.079299in}}%
\pgfusepath{clip}%
\pgfsetroundcap%
\pgfsetroundjoin%
\definecolor{currentfill}{rgb}{0.500000,0.500000,0.500000}%
\pgfsetfillcolor{currentfill}%
\pgfsetfillopacity{0.300000}%
\pgfsetlinewidth{0.301125pt}%
\definecolor{currentstroke}{rgb}{0.500000,0.500000,0.500000}%
\pgfsetstrokecolor{currentstroke}%
\pgfsetstrokeopacity{0.300000}%
\pgfsetdash{}{0pt}%
\pgfpathmoveto{\pgfqpoint{0.000000in}{0.000000in}}%
\pgfpathlineto{\pgfqpoint{0.000000in}{0.000000in}}%
\pgfpathclose%
\pgfusepath{stroke,fill}%
\end{pgfscope}%
\begin{pgfscope}%
\pgfpathrectangle{\pgfqpoint{0.647939in}{0.492442in}}{\pgfqpoint{3.079299in}{3.079299in}}%
\pgfusepath{clip}%
\pgfsetroundcap%
\pgfsetroundjoin%
\pgfsetlinewidth{0.301125pt}%
\definecolor{currentstroke}{rgb}{0.500000,0.500000,0.500000}%
\pgfsetstrokecolor{currentstroke}%
\pgfsetstrokeopacity{0.300000}%
\pgfsetdash{}{0pt}%
\pgfpathmoveto{\pgfqpoint{1.773618in}{3.398225in}}%
\pgfusepath{stroke}%
\end{pgfscope}%
\begin{pgfscope}%
\pgfpathrectangle{\pgfqpoint{0.647939in}{0.492442in}}{\pgfqpoint{3.079299in}{3.079299in}}%
\pgfusepath{clip}%
\pgfsetroundcap%
\pgfsetroundjoin%
\definecolor{currentfill}{rgb}{0.500000,0.500000,0.500000}%
\pgfsetfillcolor{currentfill}%
\pgfsetfillopacity{0.300000}%
\pgfsetlinewidth{0.301125pt}%
\definecolor{currentstroke}{rgb}{0.500000,0.500000,0.500000}%
\pgfsetstrokecolor{currentstroke}%
\pgfsetstrokeopacity{0.300000}%
\pgfsetdash{}{0pt}%
\pgfpathmoveto{\pgfqpoint{0.000000in}{0.000000in}}%
\pgfpathlineto{\pgfqpoint{0.000000in}{0.000000in}}%
\pgfpathclose%
\pgfusepath{stroke,fill}%
\end{pgfscope}%
\begin{pgfscope}%
\pgfpathrectangle{\pgfqpoint{0.647939in}{0.492442in}}{\pgfqpoint{3.079299in}{3.079299in}}%
\pgfusepath{clip}%
\pgfsetroundcap%
\pgfsetroundjoin%
\pgfsetlinewidth{0.301125pt}%
\definecolor{currentstroke}{rgb}{0.500000,0.500000,0.500000}%
\pgfsetstrokecolor{currentstroke}%
\pgfsetstrokeopacity{0.300000}%
\pgfsetdash{}{0pt}%
\pgfpathmoveto{\pgfqpoint{1.618099in}{3.460004in}}%
\pgfusepath{stroke}%
\end{pgfscope}%
\begin{pgfscope}%
\pgfpathrectangle{\pgfqpoint{0.647939in}{0.492442in}}{\pgfqpoint{3.079299in}{3.079299in}}%
\pgfusepath{clip}%
\pgfsetroundcap%
\pgfsetroundjoin%
\definecolor{currentfill}{rgb}{0.500000,0.500000,0.500000}%
\pgfsetfillcolor{currentfill}%
\pgfsetfillopacity{0.300000}%
\pgfsetlinewidth{0.301125pt}%
\definecolor{currentstroke}{rgb}{0.500000,0.500000,0.500000}%
\pgfsetstrokecolor{currentstroke}%
\pgfsetstrokeopacity{0.300000}%
\pgfsetdash{}{0pt}%
\pgfpathmoveto{\pgfqpoint{0.000000in}{0.000000in}}%
\pgfpathlineto{\pgfqpoint{0.000000in}{0.000000in}}%
\pgfpathclose%
\pgfusepath{stroke,fill}%
\end{pgfscope}%
\begin{pgfscope}%
\pgfpathrectangle{\pgfqpoint{0.647939in}{0.492442in}}{\pgfqpoint{3.079299in}{3.079299in}}%
\pgfusepath{clip}%
\pgfsetroundcap%
\pgfsetroundjoin%
\pgfsetlinewidth{0.301125pt}%
\definecolor{currentstroke}{rgb}{0.500000,0.500000,0.500000}%
\pgfsetstrokecolor{currentstroke}%
\pgfsetstrokeopacity{0.300000}%
\pgfsetdash{}{0pt}%
\pgfpathmoveto{\pgfqpoint{2.153854in}{3.552164in}}%
\pgfusepath{stroke}%
\end{pgfscope}%
\begin{pgfscope}%
\pgfpathrectangle{\pgfqpoint{0.647939in}{0.492442in}}{\pgfqpoint{3.079299in}{3.079299in}}%
\pgfusepath{clip}%
\pgfsetroundcap%
\pgfsetroundjoin%
\definecolor{currentfill}{rgb}{0.500000,0.500000,0.500000}%
\pgfsetfillcolor{currentfill}%
\pgfsetfillopacity{0.300000}%
\pgfsetlinewidth{0.301125pt}%
\definecolor{currentstroke}{rgb}{0.500000,0.500000,0.500000}%
\pgfsetstrokecolor{currentstroke}%
\pgfsetstrokeopacity{0.300000}%
\pgfsetdash{}{0pt}%
\pgfpathmoveto{\pgfqpoint{0.000000in}{0.000000in}}%
\pgfpathlineto{\pgfqpoint{0.000000in}{0.000000in}}%
\pgfpathclose%
\pgfusepath{stroke,fill}%
\end{pgfscope}%
\begin{pgfscope}%
\pgfpathrectangle{\pgfqpoint{0.647939in}{0.492442in}}{\pgfqpoint{3.079299in}{3.079299in}}%
\pgfusepath{clip}%
\pgfsetroundcap%
\pgfsetroundjoin%
\pgfsetlinewidth{0.301125pt}%
\definecolor{currentstroke}{rgb}{0.500000,0.500000,0.500000}%
\pgfsetstrokecolor{currentstroke}%
\pgfsetstrokeopacity{0.300000}%
\pgfsetdash{}{0pt}%
\pgfpathmoveto{\pgfqpoint{1.259117in}{3.468455in}}%
\pgfusepath{stroke}%
\end{pgfscope}%
\begin{pgfscope}%
\pgfpathrectangle{\pgfqpoint{0.647939in}{0.492442in}}{\pgfqpoint{3.079299in}{3.079299in}}%
\pgfusepath{clip}%
\pgfsetroundcap%
\pgfsetroundjoin%
\definecolor{currentfill}{rgb}{0.500000,0.500000,0.500000}%
\pgfsetfillcolor{currentfill}%
\pgfsetfillopacity{0.300000}%
\pgfsetlinewidth{0.301125pt}%
\definecolor{currentstroke}{rgb}{0.500000,0.500000,0.500000}%
\pgfsetstrokecolor{currentstroke}%
\pgfsetstrokeopacity{0.300000}%
\pgfsetdash{}{0pt}%
\pgfpathmoveto{\pgfqpoint{0.000000in}{0.000000in}}%
\pgfpathlineto{\pgfqpoint{0.000000in}{0.000000in}}%
\pgfpathclose%
\pgfusepath{stroke,fill}%
\end{pgfscope}%
\begin{pgfscope}%
\pgfpathrectangle{\pgfqpoint{0.647939in}{0.492442in}}{\pgfqpoint{3.079299in}{3.079299in}}%
\pgfusepath{clip}%
\pgfsetroundcap%
\pgfsetroundjoin%
\pgfsetlinewidth{0.301125pt}%
\definecolor{currentstroke}{rgb}{0.500000,0.500000,0.500000}%
\pgfsetstrokecolor{currentstroke}%
\pgfsetstrokeopacity{0.300000}%
\pgfsetdash{}{0pt}%
\pgfpathmoveto{\pgfqpoint{1.043249in}{3.492528in}}%
\pgfusepath{stroke}%
\end{pgfscope}%
\begin{pgfscope}%
\pgfpathrectangle{\pgfqpoint{0.647939in}{0.492442in}}{\pgfqpoint{3.079299in}{3.079299in}}%
\pgfusepath{clip}%
\pgfsetroundcap%
\pgfsetroundjoin%
\definecolor{currentfill}{rgb}{0.500000,0.500000,0.500000}%
\pgfsetfillcolor{currentfill}%
\pgfsetfillopacity{0.300000}%
\pgfsetlinewidth{0.301125pt}%
\definecolor{currentstroke}{rgb}{0.500000,0.500000,0.500000}%
\pgfsetstrokecolor{currentstroke}%
\pgfsetstrokeopacity{0.300000}%
\pgfsetdash{}{0pt}%
\pgfpathmoveto{\pgfqpoint{0.000000in}{0.000000in}}%
\pgfpathlineto{\pgfqpoint{0.000000in}{0.000000in}}%
\pgfpathclose%
\pgfusepath{stroke,fill}%
\end{pgfscope}%
\begin{pgfscope}%
\pgfpathrectangle{\pgfqpoint{0.647939in}{0.492442in}}{\pgfqpoint{3.079299in}{3.079299in}}%
\pgfusepath{clip}%
\pgfsetroundcap%
\pgfsetroundjoin%
\pgfsetlinewidth{0.301125pt}%
\definecolor{currentstroke}{rgb}{0.500000,0.500000,0.500000}%
\pgfsetstrokecolor{currentstroke}%
\pgfsetstrokeopacity{0.300000}%
\pgfsetdash{}{0pt}%
\pgfpathmoveto{\pgfqpoint{0.825641in}{3.529144in}}%
\pgfusepath{stroke}%
\end{pgfscope}%
\begin{pgfscope}%
\pgfpathrectangle{\pgfqpoint{0.647939in}{0.492442in}}{\pgfqpoint{3.079299in}{3.079299in}}%
\pgfusepath{clip}%
\pgfsetroundcap%
\pgfsetroundjoin%
\definecolor{currentfill}{rgb}{0.500000,0.500000,0.500000}%
\pgfsetfillcolor{currentfill}%
\pgfsetfillopacity{0.300000}%
\pgfsetlinewidth{0.301125pt}%
\definecolor{currentstroke}{rgb}{0.500000,0.500000,0.500000}%
\pgfsetstrokecolor{currentstroke}%
\pgfsetstrokeopacity{0.300000}%
\pgfsetdash{}{0pt}%
\pgfpathmoveto{\pgfqpoint{0.000000in}{0.000000in}}%
\pgfpathlineto{\pgfqpoint{0.000000in}{0.000000in}}%
\pgfpathclose%
\pgfusepath{stroke,fill}%
\end{pgfscope}%
\begin{pgfscope}%
\pgfpathrectangle{\pgfqpoint{0.647939in}{0.492442in}}{\pgfqpoint{3.079299in}{3.079299in}}%
\pgfusepath{clip}%
\pgfsetroundcap%
\pgfsetroundjoin%
\pgfsetlinewidth{0.301125pt}%
\definecolor{currentstroke}{rgb}{0.500000,0.500000,0.500000}%
\pgfsetstrokecolor{currentstroke}%
\pgfsetstrokeopacity{0.300000}%
\pgfsetdash{}{0pt}%
\pgfpathmoveto{\pgfqpoint{1.211897in}{2.982627in}}%
\pgfusepath{stroke}%
\end{pgfscope}%
\begin{pgfscope}%
\pgfpathrectangle{\pgfqpoint{0.647939in}{0.492442in}}{\pgfqpoint{3.079299in}{3.079299in}}%
\pgfusepath{clip}%
\pgfsetroundcap%
\pgfsetroundjoin%
\definecolor{currentfill}{rgb}{0.500000,0.500000,0.500000}%
\pgfsetfillcolor{currentfill}%
\pgfsetfillopacity{0.300000}%
\pgfsetlinewidth{0.301125pt}%
\definecolor{currentstroke}{rgb}{0.500000,0.500000,0.500000}%
\pgfsetstrokecolor{currentstroke}%
\pgfsetstrokeopacity{0.300000}%
\pgfsetdash{}{0pt}%
\pgfpathmoveto{\pgfqpoint{0.000000in}{0.000000in}}%
\pgfpathlineto{\pgfqpoint{0.000000in}{0.000000in}}%
\pgfpathclose%
\pgfusepath{stroke,fill}%
\end{pgfscope}%
\begin{pgfscope}%
\pgfpathrectangle{\pgfqpoint{0.647939in}{0.492442in}}{\pgfqpoint{3.079299in}{3.079299in}}%
\pgfusepath{clip}%
\pgfsetroundcap%
\pgfsetroundjoin%
\pgfsetlinewidth{0.301125pt}%
\definecolor{currentstroke}{rgb}{0.500000,0.500000,0.500000}%
\pgfsetstrokecolor{currentstroke}%
\pgfsetstrokeopacity{0.300000}%
\pgfsetdash{}{0pt}%
\pgfpathmoveto{\pgfqpoint{1.740892in}{2.913660in}}%
\pgfusepath{stroke}%
\end{pgfscope}%
\begin{pgfscope}%
\pgfpathrectangle{\pgfqpoint{0.647939in}{0.492442in}}{\pgfqpoint{3.079299in}{3.079299in}}%
\pgfusepath{clip}%
\pgfsetroundcap%
\pgfsetroundjoin%
\definecolor{currentfill}{rgb}{0.500000,0.500000,0.500000}%
\pgfsetfillcolor{currentfill}%
\pgfsetfillopacity{0.300000}%
\pgfsetlinewidth{0.301125pt}%
\definecolor{currentstroke}{rgb}{0.500000,0.500000,0.500000}%
\pgfsetstrokecolor{currentstroke}%
\pgfsetstrokeopacity{0.300000}%
\pgfsetdash{}{0pt}%
\pgfpathmoveto{\pgfqpoint{0.000000in}{0.000000in}}%
\pgfpathlineto{\pgfqpoint{0.000000in}{0.000000in}}%
\pgfpathclose%
\pgfusepath{stroke,fill}%
\end{pgfscope}%
\begin{pgfscope}%
\pgfpathrectangle{\pgfqpoint{0.647939in}{0.492442in}}{\pgfqpoint{3.079299in}{3.079299in}}%
\pgfusepath{clip}%
\pgfsetroundcap%
\pgfsetroundjoin%
\pgfsetlinewidth{0.301125pt}%
\definecolor{currentstroke}{rgb}{0.500000,0.500000,0.500000}%
\pgfsetstrokecolor{currentstroke}%
\pgfsetstrokeopacity{0.300000}%
\pgfsetdash{}{0pt}%
\pgfpathmoveto{\pgfqpoint{1.407493in}{2.767172in}}%
\pgfusepath{stroke}%
\end{pgfscope}%
\begin{pgfscope}%
\pgfpathrectangle{\pgfqpoint{0.647939in}{0.492442in}}{\pgfqpoint{3.079299in}{3.079299in}}%
\pgfusepath{clip}%
\pgfsetroundcap%
\pgfsetroundjoin%
\definecolor{currentfill}{rgb}{0.500000,0.500000,0.500000}%
\pgfsetfillcolor{currentfill}%
\pgfsetfillopacity{0.300000}%
\pgfsetlinewidth{0.301125pt}%
\definecolor{currentstroke}{rgb}{0.500000,0.500000,0.500000}%
\pgfsetstrokecolor{currentstroke}%
\pgfsetstrokeopacity{0.300000}%
\pgfsetdash{}{0pt}%
\pgfpathmoveto{\pgfqpoint{0.000000in}{0.000000in}}%
\pgfpathlineto{\pgfqpoint{0.000000in}{0.000000in}}%
\pgfpathclose%
\pgfusepath{stroke,fill}%
\end{pgfscope}%
\begin{pgfscope}%
\pgfpathrectangle{\pgfqpoint{0.647939in}{0.492442in}}{\pgfqpoint{3.079299in}{3.079299in}}%
\pgfusepath{clip}%
\pgfsetroundcap%
\pgfsetroundjoin%
\pgfsetlinewidth{0.301125pt}%
\definecolor{currentstroke}{rgb}{0.500000,0.500000,0.500000}%
\pgfsetstrokecolor{currentstroke}%
\pgfsetstrokeopacity{0.300000}%
\pgfsetdash{}{0pt}%
\pgfpathmoveto{\pgfqpoint{1.340941in}{2.681429in}}%
\pgfusepath{stroke}%
\end{pgfscope}%
\begin{pgfscope}%
\pgfpathrectangle{\pgfqpoint{0.647939in}{0.492442in}}{\pgfqpoint{3.079299in}{3.079299in}}%
\pgfusepath{clip}%
\pgfsetroundcap%
\pgfsetroundjoin%
\definecolor{currentfill}{rgb}{0.500000,0.500000,0.500000}%
\pgfsetfillcolor{currentfill}%
\pgfsetfillopacity{0.300000}%
\pgfsetlinewidth{0.301125pt}%
\definecolor{currentstroke}{rgb}{0.500000,0.500000,0.500000}%
\pgfsetstrokecolor{currentstroke}%
\pgfsetstrokeopacity{0.300000}%
\pgfsetdash{}{0pt}%
\pgfpathmoveto{\pgfqpoint{0.000000in}{0.000000in}}%
\pgfpathlineto{\pgfqpoint{0.000000in}{0.000000in}}%
\pgfpathclose%
\pgfusepath{stroke,fill}%
\end{pgfscope}%
\begin{pgfscope}%
\pgfpathrectangle{\pgfqpoint{0.647939in}{0.492442in}}{\pgfqpoint{3.079299in}{3.079299in}}%
\pgfusepath{clip}%
\pgfsetroundcap%
\pgfsetroundjoin%
\pgfsetlinewidth{0.301125pt}%
\definecolor{currentstroke}{rgb}{0.500000,0.500000,0.500000}%
\pgfsetstrokecolor{currentstroke}%
\pgfsetstrokeopacity{0.300000}%
\pgfsetdash{}{0pt}%
\pgfpathmoveto{\pgfqpoint{1.010909in}{2.452017in}}%
\pgfusepath{stroke}%
\end{pgfscope}%
\begin{pgfscope}%
\pgfpathrectangle{\pgfqpoint{0.647939in}{0.492442in}}{\pgfqpoint{3.079299in}{3.079299in}}%
\pgfusepath{clip}%
\pgfsetroundcap%
\pgfsetroundjoin%
\definecolor{currentfill}{rgb}{0.500000,0.500000,0.500000}%
\pgfsetfillcolor{currentfill}%
\pgfsetfillopacity{0.300000}%
\pgfsetlinewidth{0.301125pt}%
\definecolor{currentstroke}{rgb}{0.500000,0.500000,0.500000}%
\pgfsetstrokecolor{currentstroke}%
\pgfsetstrokeopacity{0.300000}%
\pgfsetdash{}{0pt}%
\pgfpathmoveto{\pgfqpoint{0.000000in}{0.000000in}}%
\pgfpathlineto{\pgfqpoint{0.000000in}{0.000000in}}%
\pgfpathclose%
\pgfusepath{stroke,fill}%
\end{pgfscope}%
\begin{pgfscope}%
\pgfpathrectangle{\pgfqpoint{0.647939in}{0.492442in}}{\pgfqpoint{3.079299in}{3.079299in}}%
\pgfusepath{clip}%
\pgfsetroundcap%
\pgfsetroundjoin%
\pgfsetlinewidth{0.301125pt}%
\definecolor{currentstroke}{rgb}{0.500000,0.500000,0.500000}%
\pgfsetstrokecolor{currentstroke}%
\pgfsetstrokeopacity{0.300000}%
\pgfsetdash{}{0pt}%
\pgfpathmoveto{\pgfqpoint{1.665241in}{2.582196in}}%
\pgfusepath{stroke}%
\end{pgfscope}%
\begin{pgfscope}%
\pgfpathrectangle{\pgfqpoint{0.647939in}{0.492442in}}{\pgfqpoint{3.079299in}{3.079299in}}%
\pgfusepath{clip}%
\pgfsetroundcap%
\pgfsetroundjoin%
\definecolor{currentfill}{rgb}{0.500000,0.500000,0.500000}%
\pgfsetfillcolor{currentfill}%
\pgfsetfillopacity{0.300000}%
\pgfsetlinewidth{0.301125pt}%
\definecolor{currentstroke}{rgb}{0.500000,0.500000,0.500000}%
\pgfsetstrokecolor{currentstroke}%
\pgfsetstrokeopacity{0.300000}%
\pgfsetdash{}{0pt}%
\pgfpathmoveto{\pgfqpoint{0.000000in}{0.000000in}}%
\pgfpathlineto{\pgfqpoint{0.000000in}{0.000000in}}%
\pgfpathclose%
\pgfusepath{stroke,fill}%
\end{pgfscope}%
\begin{pgfscope}%
\pgfpathrectangle{\pgfqpoint{0.647939in}{0.492442in}}{\pgfqpoint{3.079299in}{3.079299in}}%
\pgfusepath{clip}%
\pgfsetroundcap%
\pgfsetroundjoin%
\pgfsetlinewidth{0.301125pt}%
\definecolor{currentstroke}{rgb}{0.500000,0.500000,0.500000}%
\pgfsetstrokecolor{currentstroke}%
\pgfsetstrokeopacity{0.300000}%
\pgfsetdash{}{0pt}%
\pgfpathmoveto{\pgfqpoint{1.076373in}{2.332614in}}%
\pgfusepath{stroke}%
\end{pgfscope}%
\begin{pgfscope}%
\pgfpathrectangle{\pgfqpoint{0.647939in}{0.492442in}}{\pgfqpoint{3.079299in}{3.079299in}}%
\pgfusepath{clip}%
\pgfsetroundcap%
\pgfsetroundjoin%
\definecolor{currentfill}{rgb}{0.500000,0.500000,0.500000}%
\pgfsetfillcolor{currentfill}%
\pgfsetfillopacity{0.300000}%
\pgfsetlinewidth{0.301125pt}%
\definecolor{currentstroke}{rgb}{0.500000,0.500000,0.500000}%
\pgfsetstrokecolor{currentstroke}%
\pgfsetstrokeopacity{0.300000}%
\pgfsetdash{}{0pt}%
\pgfpathmoveto{\pgfqpoint{0.000000in}{0.000000in}}%
\pgfpathlineto{\pgfqpoint{0.000000in}{0.000000in}}%
\pgfpathclose%
\pgfusepath{stroke,fill}%
\end{pgfscope}%
\begin{pgfscope}%
\pgfpathrectangle{\pgfqpoint{0.647939in}{0.492442in}}{\pgfqpoint{3.079299in}{3.079299in}}%
\pgfusepath{clip}%
\pgfsetroundcap%
\pgfsetroundjoin%
\pgfsetlinewidth{0.301125pt}%
\definecolor{currentstroke}{rgb}{0.500000,0.500000,0.500000}%
\pgfsetstrokecolor{currentstroke}%
\pgfsetstrokeopacity{0.300000}%
\pgfsetdash{}{0pt}%
\pgfpathmoveto{\pgfqpoint{1.792885in}{2.492656in}}%
\pgfusepath{stroke}%
\end{pgfscope}%
\begin{pgfscope}%
\pgfpathrectangle{\pgfqpoint{0.647939in}{0.492442in}}{\pgfqpoint{3.079299in}{3.079299in}}%
\pgfusepath{clip}%
\pgfsetroundcap%
\pgfsetroundjoin%
\definecolor{currentfill}{rgb}{0.500000,0.500000,0.500000}%
\pgfsetfillcolor{currentfill}%
\pgfsetfillopacity{0.300000}%
\pgfsetlinewidth{0.301125pt}%
\definecolor{currentstroke}{rgb}{0.500000,0.500000,0.500000}%
\pgfsetstrokecolor{currentstroke}%
\pgfsetstrokeopacity{0.300000}%
\pgfsetdash{}{0pt}%
\pgfpathmoveto{\pgfqpoint{0.000000in}{0.000000in}}%
\pgfpathlineto{\pgfqpoint{0.000000in}{0.000000in}}%
\pgfpathclose%
\pgfusepath{stroke,fill}%
\end{pgfscope}%
\begin{pgfscope}%
\pgfpathrectangle{\pgfqpoint{0.647939in}{0.492442in}}{\pgfqpoint{3.079299in}{3.079299in}}%
\pgfusepath{clip}%
\pgfsetroundcap%
\pgfsetroundjoin%
\pgfsetlinewidth{0.301125pt}%
\definecolor{currentstroke}{rgb}{0.500000,0.500000,0.500000}%
\pgfsetstrokecolor{currentstroke}%
\pgfsetstrokeopacity{0.300000}%
\pgfsetdash{}{0pt}%
\pgfpathmoveto{\pgfqpoint{1.526062in}{2.289164in}}%
\pgfusepath{stroke}%
\end{pgfscope}%
\begin{pgfscope}%
\pgfpathrectangle{\pgfqpoint{0.647939in}{0.492442in}}{\pgfqpoint{3.079299in}{3.079299in}}%
\pgfusepath{clip}%
\pgfsetroundcap%
\pgfsetroundjoin%
\definecolor{currentfill}{rgb}{0.500000,0.500000,0.500000}%
\pgfsetfillcolor{currentfill}%
\pgfsetfillopacity{0.300000}%
\pgfsetlinewidth{0.301125pt}%
\definecolor{currentstroke}{rgb}{0.500000,0.500000,0.500000}%
\pgfsetstrokecolor{currentstroke}%
\pgfsetstrokeopacity{0.300000}%
\pgfsetdash{}{0pt}%
\pgfpathmoveto{\pgfqpoint{0.000000in}{0.000000in}}%
\pgfpathlineto{\pgfqpoint{0.000000in}{0.000000in}}%
\pgfpathclose%
\pgfusepath{stroke,fill}%
\end{pgfscope}%
\begin{pgfscope}%
\pgfpathrectangle{\pgfqpoint{0.647939in}{0.492442in}}{\pgfqpoint{3.079299in}{3.079299in}}%
\pgfusepath{clip}%
\pgfsetroundcap%
\pgfsetroundjoin%
\pgfsetlinewidth{0.301125pt}%
\definecolor{currentstroke}{rgb}{0.500000,0.500000,0.500000}%
\pgfsetstrokecolor{currentstroke}%
\pgfsetstrokeopacity{0.300000}%
\pgfsetdash{}{0pt}%
\pgfpathmoveto{\pgfqpoint{0.943288in}{2.021574in}}%
\pgfusepath{stroke}%
\end{pgfscope}%
\begin{pgfscope}%
\pgfpathrectangle{\pgfqpoint{0.647939in}{0.492442in}}{\pgfqpoint{3.079299in}{3.079299in}}%
\pgfusepath{clip}%
\pgfsetroundcap%
\pgfsetroundjoin%
\definecolor{currentfill}{rgb}{0.500000,0.500000,0.500000}%
\pgfsetfillcolor{currentfill}%
\pgfsetfillopacity{0.300000}%
\pgfsetlinewidth{0.301125pt}%
\definecolor{currentstroke}{rgb}{0.500000,0.500000,0.500000}%
\pgfsetstrokecolor{currentstroke}%
\pgfsetstrokeopacity{0.300000}%
\pgfsetdash{}{0pt}%
\pgfpathmoveto{\pgfqpoint{0.000000in}{0.000000in}}%
\pgfpathlineto{\pgfqpoint{0.000000in}{0.000000in}}%
\pgfpathclose%
\pgfusepath{stroke,fill}%
\end{pgfscope}%
\begin{pgfscope}%
\pgfpathrectangle{\pgfqpoint{0.647939in}{0.492442in}}{\pgfqpoint{3.079299in}{3.079299in}}%
\pgfusepath{clip}%
\pgfsetroundcap%
\pgfsetroundjoin%
\pgfsetlinewidth{0.301125pt}%
\definecolor{currentstroke}{rgb}{0.500000,0.500000,0.500000}%
\pgfsetstrokecolor{currentstroke}%
\pgfsetstrokeopacity{0.300000}%
\pgfsetdash{}{0pt}%
\pgfpathmoveto{\pgfqpoint{1.584749in}{2.188752in}}%
\pgfusepath{stroke}%
\end{pgfscope}%
\begin{pgfscope}%
\pgfpathrectangle{\pgfqpoint{0.647939in}{0.492442in}}{\pgfqpoint{3.079299in}{3.079299in}}%
\pgfusepath{clip}%
\pgfsetroundcap%
\pgfsetroundjoin%
\definecolor{currentfill}{rgb}{0.500000,0.500000,0.500000}%
\pgfsetfillcolor{currentfill}%
\pgfsetfillopacity{0.300000}%
\pgfsetlinewidth{0.301125pt}%
\definecolor{currentstroke}{rgb}{0.500000,0.500000,0.500000}%
\pgfsetstrokecolor{currentstroke}%
\pgfsetstrokeopacity{0.300000}%
\pgfsetdash{}{0pt}%
\pgfpathmoveto{\pgfqpoint{0.000000in}{0.000000in}}%
\pgfpathlineto{\pgfqpoint{0.000000in}{0.000000in}}%
\pgfpathclose%
\pgfusepath{stroke,fill}%
\end{pgfscope}%
\begin{pgfscope}%
\pgfpathrectangle{\pgfqpoint{0.647939in}{0.492442in}}{\pgfqpoint{3.079299in}{3.079299in}}%
\pgfusepath{clip}%
\pgfsetroundcap%
\pgfsetroundjoin%
\pgfsetlinewidth{0.301125pt}%
\definecolor{currentstroke}{rgb}{0.500000,0.500000,0.500000}%
\pgfsetstrokecolor{currentstroke}%
\pgfsetstrokeopacity{0.300000}%
\pgfsetdash{}{0pt}%
\pgfpathmoveto{\pgfqpoint{1.138381in}{1.945917in}}%
\pgfusepath{stroke}%
\end{pgfscope}%
\begin{pgfscope}%
\pgfpathrectangle{\pgfqpoint{0.647939in}{0.492442in}}{\pgfqpoint{3.079299in}{3.079299in}}%
\pgfusepath{clip}%
\pgfsetroundcap%
\pgfsetroundjoin%
\definecolor{currentfill}{rgb}{0.500000,0.500000,0.500000}%
\pgfsetfillcolor{currentfill}%
\pgfsetfillopacity{0.300000}%
\pgfsetlinewidth{0.301125pt}%
\definecolor{currentstroke}{rgb}{0.500000,0.500000,0.500000}%
\pgfsetstrokecolor{currentstroke}%
\pgfsetstrokeopacity{0.300000}%
\pgfsetdash{}{0pt}%
\pgfpathmoveto{\pgfqpoint{0.000000in}{0.000000in}}%
\pgfpathlineto{\pgfqpoint{0.000000in}{0.000000in}}%
\pgfpathclose%
\pgfusepath{stroke,fill}%
\end{pgfscope}%
\begin{pgfscope}%
\pgfpathrectangle{\pgfqpoint{0.647939in}{0.492442in}}{\pgfqpoint{3.079299in}{3.079299in}}%
\pgfusepath{clip}%
\pgfsetroundcap%
\pgfsetroundjoin%
\pgfsetlinewidth{0.301125pt}%
\definecolor{currentstroke}{rgb}{0.500000,0.500000,0.500000}%
\pgfsetstrokecolor{currentstroke}%
\pgfsetstrokeopacity{0.300000}%
\pgfsetdash{}{0pt}%
\pgfpathmoveto{\pgfqpoint{1.008171in}{1.834302in}}%
\pgfusepath{stroke}%
\end{pgfscope}%
\begin{pgfscope}%
\pgfpathrectangle{\pgfqpoint{0.647939in}{0.492442in}}{\pgfqpoint{3.079299in}{3.079299in}}%
\pgfusepath{clip}%
\pgfsetroundcap%
\pgfsetroundjoin%
\definecolor{currentfill}{rgb}{0.500000,0.500000,0.500000}%
\pgfsetfillcolor{currentfill}%
\pgfsetfillopacity{0.300000}%
\pgfsetlinewidth{0.301125pt}%
\definecolor{currentstroke}{rgb}{0.500000,0.500000,0.500000}%
\pgfsetstrokecolor{currentstroke}%
\pgfsetstrokeopacity{0.300000}%
\pgfsetdash{}{0pt}%
\pgfpathmoveto{\pgfqpoint{0.000000in}{0.000000in}}%
\pgfpathlineto{\pgfqpoint{0.000000in}{0.000000in}}%
\pgfpathclose%
\pgfusepath{stroke,fill}%
\end{pgfscope}%
\begin{pgfscope}%
\pgfpathrectangle{\pgfqpoint{0.647939in}{0.492442in}}{\pgfqpoint{3.079299in}{3.079299in}}%
\pgfusepath{clip}%
\pgfsetroundcap%
\pgfsetroundjoin%
\pgfsetlinewidth{0.301125pt}%
\definecolor{currentstroke}{rgb}{0.500000,0.500000,0.500000}%
\pgfsetstrokecolor{currentstroke}%
\pgfsetstrokeopacity{0.300000}%
\pgfsetdash{}{0pt}%
\pgfpathmoveto{\pgfqpoint{1.450149in}{1.948508in}}%
\pgfusepath{stroke}%
\end{pgfscope}%
\begin{pgfscope}%
\pgfpathrectangle{\pgfqpoint{0.647939in}{0.492442in}}{\pgfqpoint{3.079299in}{3.079299in}}%
\pgfusepath{clip}%
\pgfsetroundcap%
\pgfsetroundjoin%
\definecolor{currentfill}{rgb}{0.500000,0.500000,0.500000}%
\pgfsetfillcolor{currentfill}%
\pgfsetfillopacity{0.300000}%
\pgfsetlinewidth{0.301125pt}%
\definecolor{currentstroke}{rgb}{0.500000,0.500000,0.500000}%
\pgfsetstrokecolor{currentstroke}%
\pgfsetstrokeopacity{0.300000}%
\pgfsetdash{}{0pt}%
\pgfpathmoveto{\pgfqpoint{0.000000in}{0.000000in}}%
\pgfpathlineto{\pgfqpoint{0.000000in}{0.000000in}}%
\pgfpathclose%
\pgfusepath{stroke,fill}%
\end{pgfscope}%
\begin{pgfscope}%
\pgfpathrectangle{\pgfqpoint{0.647939in}{0.492442in}}{\pgfqpoint{3.079299in}{3.079299in}}%
\pgfusepath{clip}%
\pgfsetroundcap%
\pgfsetroundjoin%
\pgfsetlinewidth{0.301125pt}%
\definecolor{currentstroke}{rgb}{0.500000,0.500000,0.500000}%
\pgfsetstrokecolor{currentstroke}%
\pgfsetstrokeopacity{0.300000}%
\pgfsetdash{}{0pt}%
\pgfpathmoveto{\pgfqpoint{1.072012in}{1.719932in}}%
\pgfusepath{stroke}%
\end{pgfscope}%
\begin{pgfscope}%
\pgfpathrectangle{\pgfqpoint{0.647939in}{0.492442in}}{\pgfqpoint{3.079299in}{3.079299in}}%
\pgfusepath{clip}%
\pgfsetroundcap%
\pgfsetroundjoin%
\definecolor{currentfill}{rgb}{0.500000,0.500000,0.500000}%
\pgfsetfillcolor{currentfill}%
\pgfsetfillopacity{0.300000}%
\pgfsetlinewidth{0.301125pt}%
\definecolor{currentstroke}{rgb}{0.500000,0.500000,0.500000}%
\pgfsetstrokecolor{currentstroke}%
\pgfsetstrokeopacity{0.300000}%
\pgfsetdash{}{0pt}%
\pgfpathmoveto{\pgfqpoint{0.000000in}{0.000000in}}%
\pgfpathlineto{\pgfqpoint{0.000000in}{0.000000in}}%
\pgfpathclose%
\pgfusepath{stroke,fill}%
\end{pgfscope}%
\begin{pgfscope}%
\pgfpathrectangle{\pgfqpoint{0.647939in}{0.492442in}}{\pgfqpoint{3.079299in}{3.079299in}}%
\pgfusepath{clip}%
\pgfsetroundcap%
\pgfsetroundjoin%
\pgfsetlinewidth{0.301125pt}%
\definecolor{currentstroke}{rgb}{0.500000,0.500000,0.500000}%
\pgfsetstrokecolor{currentstroke}%
\pgfsetstrokeopacity{0.300000}%
\pgfsetdash{}{0pt}%
\pgfpathmoveto{\pgfqpoint{0.875906in}{1.589911in}}%
\pgfusepath{stroke}%
\end{pgfscope}%
\begin{pgfscope}%
\pgfpathrectangle{\pgfqpoint{0.647939in}{0.492442in}}{\pgfqpoint{3.079299in}{3.079299in}}%
\pgfusepath{clip}%
\pgfsetroundcap%
\pgfsetroundjoin%
\definecolor{currentfill}{rgb}{0.500000,0.500000,0.500000}%
\pgfsetfillcolor{currentfill}%
\pgfsetfillopacity{0.300000}%
\pgfsetlinewidth{0.301125pt}%
\definecolor{currentstroke}{rgb}{0.500000,0.500000,0.500000}%
\pgfsetstrokecolor{currentstroke}%
\pgfsetstrokeopacity{0.300000}%
\pgfsetdash{}{0pt}%
\pgfpathmoveto{\pgfqpoint{0.000000in}{0.000000in}}%
\pgfpathlineto{\pgfqpoint{0.000000in}{0.000000in}}%
\pgfpathclose%
\pgfusepath{stroke,fill}%
\end{pgfscope}%
\begin{pgfscope}%
\pgfpathrectangle{\pgfqpoint{0.647939in}{0.492442in}}{\pgfqpoint{3.079299in}{3.079299in}}%
\pgfusepath{clip}%
\pgfsetroundcap%
\pgfsetroundjoin%
\pgfsetlinewidth{0.301125pt}%
\definecolor{currentstroke}{rgb}{0.500000,0.500000,0.500000}%
\pgfsetstrokecolor{currentstroke}%
\pgfsetstrokeopacity{0.300000}%
\pgfsetdash{}{0pt}%
\pgfpathmoveto{\pgfqpoint{1.440355in}{1.762318in}}%
\pgfusepath{stroke}%
\end{pgfscope}%
\begin{pgfscope}%
\pgfpathrectangle{\pgfqpoint{0.647939in}{0.492442in}}{\pgfqpoint{3.079299in}{3.079299in}}%
\pgfusepath{clip}%
\pgfsetroundcap%
\pgfsetroundjoin%
\definecolor{currentfill}{rgb}{0.500000,0.500000,0.500000}%
\pgfsetfillcolor{currentfill}%
\pgfsetfillopacity{0.300000}%
\pgfsetlinewidth{0.301125pt}%
\definecolor{currentstroke}{rgb}{0.500000,0.500000,0.500000}%
\pgfsetstrokecolor{currentstroke}%
\pgfsetstrokeopacity{0.300000}%
\pgfsetdash{}{0pt}%
\pgfpathmoveto{\pgfqpoint{0.000000in}{0.000000in}}%
\pgfpathlineto{\pgfqpoint{0.000000in}{0.000000in}}%
\pgfpathclose%
\pgfusepath{stroke,fill}%
\end{pgfscope}%
\begin{pgfscope}%
\pgfpathrectangle{\pgfqpoint{0.647939in}{0.492442in}}{\pgfqpoint{3.079299in}{3.079299in}}%
\pgfusepath{clip}%
\pgfsetroundcap%
\pgfsetroundjoin%
\pgfsetlinewidth{0.301125pt}%
\definecolor{currentstroke}{rgb}{0.500000,0.500000,0.500000}%
\pgfsetstrokecolor{currentstroke}%
\pgfsetstrokeopacity{0.300000}%
\pgfsetdash{}{0pt}%
\pgfpathmoveto{\pgfqpoint{1.005909in}{1.493031in}}%
\pgfusepath{stroke}%
\end{pgfscope}%
\begin{pgfscope}%
\pgfpathrectangle{\pgfqpoint{0.647939in}{0.492442in}}{\pgfqpoint{3.079299in}{3.079299in}}%
\pgfusepath{clip}%
\pgfsetroundcap%
\pgfsetroundjoin%
\definecolor{currentfill}{rgb}{0.500000,0.500000,0.500000}%
\pgfsetfillcolor{currentfill}%
\pgfsetfillopacity{0.300000}%
\pgfsetlinewidth{0.301125pt}%
\definecolor{currentstroke}{rgb}{0.500000,0.500000,0.500000}%
\pgfsetstrokecolor{currentstroke}%
\pgfsetstrokeopacity{0.300000}%
\pgfsetdash{}{0pt}%
\pgfpathmoveto{\pgfqpoint{0.000000in}{0.000000in}}%
\pgfpathlineto{\pgfqpoint{0.000000in}{0.000000in}}%
\pgfpathclose%
\pgfusepath{stroke,fill}%
\end{pgfscope}%
\begin{pgfscope}%
\pgfpathrectangle{\pgfqpoint{0.647939in}{0.492442in}}{\pgfqpoint{3.079299in}{3.079299in}}%
\pgfusepath{clip}%
\pgfsetroundcap%
\pgfsetroundjoin%
\pgfsetlinewidth{0.301125pt}%
\definecolor{currentstroke}{rgb}{0.500000,0.500000,0.500000}%
\pgfsetstrokecolor{currentstroke}%
\pgfsetstrokeopacity{0.300000}%
\pgfsetdash{}{0pt}%
\pgfpathmoveto{\pgfqpoint{0.809154in}{1.365214in}}%
\pgfusepath{stroke}%
\end{pgfscope}%
\begin{pgfscope}%
\pgfpathrectangle{\pgfqpoint{0.647939in}{0.492442in}}{\pgfqpoint{3.079299in}{3.079299in}}%
\pgfusepath{clip}%
\pgfsetroundcap%
\pgfsetroundjoin%
\definecolor{currentfill}{rgb}{0.500000,0.500000,0.500000}%
\pgfsetfillcolor{currentfill}%
\pgfsetfillopacity{0.300000}%
\pgfsetlinewidth{0.301125pt}%
\definecolor{currentstroke}{rgb}{0.500000,0.500000,0.500000}%
\pgfsetstrokecolor{currentstroke}%
\pgfsetstrokeopacity{0.300000}%
\pgfsetdash{}{0pt}%
\pgfpathmoveto{\pgfqpoint{0.000000in}{0.000000in}}%
\pgfpathlineto{\pgfqpoint{0.000000in}{0.000000in}}%
\pgfpathclose%
\pgfusepath{stroke,fill}%
\end{pgfscope}%
\begin{pgfscope}%
\pgfpathrectangle{\pgfqpoint{0.647939in}{0.492442in}}{\pgfqpoint{3.079299in}{3.079299in}}%
\pgfusepath{clip}%
\pgfsetroundcap%
\pgfsetroundjoin%
\pgfsetlinewidth{0.301125pt}%
\definecolor{currentstroke}{rgb}{0.500000,0.500000,0.500000}%
\pgfsetstrokecolor{currentstroke}%
\pgfsetstrokeopacity{0.300000}%
\pgfsetdash{}{0pt}%
\pgfpathmoveto{\pgfqpoint{1.370150in}{1.542495in}}%
\pgfusepath{stroke}%
\end{pgfscope}%
\begin{pgfscope}%
\pgfpathrectangle{\pgfqpoint{0.647939in}{0.492442in}}{\pgfqpoint{3.079299in}{3.079299in}}%
\pgfusepath{clip}%
\pgfsetroundcap%
\pgfsetroundjoin%
\definecolor{currentfill}{rgb}{0.500000,0.500000,0.500000}%
\pgfsetfillcolor{currentfill}%
\pgfsetfillopacity{0.300000}%
\pgfsetlinewidth{0.301125pt}%
\definecolor{currentstroke}{rgb}{0.500000,0.500000,0.500000}%
\pgfsetstrokecolor{currentstroke}%
\pgfsetstrokeopacity{0.300000}%
\pgfsetdash{}{0pt}%
\pgfpathmoveto{\pgfqpoint{0.000000in}{0.000000in}}%
\pgfpathlineto{\pgfqpoint{0.000000in}{0.000000in}}%
\pgfpathclose%
\pgfusepath{stroke,fill}%
\end{pgfscope}%
\begin{pgfscope}%
\pgfpathrectangle{\pgfqpoint{0.647939in}{0.492442in}}{\pgfqpoint{3.079299in}{3.079299in}}%
\pgfusepath{clip}%
\pgfsetroundcap%
\pgfsetroundjoin%
\pgfsetlinewidth{0.301125pt}%
\definecolor{currentstroke}{rgb}{0.500000,0.500000,0.500000}%
\pgfsetstrokecolor{currentstroke}%
\pgfsetstrokeopacity{0.300000}%
\pgfsetdash{}{0pt}%
\pgfpathmoveto{\pgfqpoint{1.067368in}{1.315277in}}%
\pgfusepath{stroke}%
\end{pgfscope}%
\begin{pgfscope}%
\pgfpathrectangle{\pgfqpoint{0.647939in}{0.492442in}}{\pgfqpoint{3.079299in}{3.079299in}}%
\pgfusepath{clip}%
\pgfsetroundcap%
\pgfsetroundjoin%
\definecolor{currentfill}{rgb}{0.500000,0.500000,0.500000}%
\pgfsetfillcolor{currentfill}%
\pgfsetfillopacity{0.300000}%
\pgfsetlinewidth{0.301125pt}%
\definecolor{currentstroke}{rgb}{0.500000,0.500000,0.500000}%
\pgfsetstrokecolor{currentstroke}%
\pgfsetstrokeopacity{0.300000}%
\pgfsetdash{}{0pt}%
\pgfpathmoveto{\pgfqpoint{0.000000in}{0.000000in}}%
\pgfpathlineto{\pgfqpoint{0.000000in}{0.000000in}}%
\pgfpathclose%
\pgfusepath{stroke,fill}%
\end{pgfscope}%
\begin{pgfscope}%
\pgfpathrectangle{\pgfqpoint{0.647939in}{0.492442in}}{\pgfqpoint{3.079299in}{3.079299in}}%
\pgfusepath{clip}%
\pgfsetroundcap%
\pgfsetroundjoin%
\pgfsetlinewidth{0.301125pt}%
\definecolor{currentstroke}{rgb}{0.500000,0.500000,0.500000}%
\pgfsetstrokecolor{currentstroke}%
\pgfsetstrokeopacity{0.300000}%
\pgfsetdash{}{0pt}%
\pgfpathmoveto{\pgfqpoint{1.003501in}{1.221341in}}%
\pgfusepath{stroke}%
\end{pgfscope}%
\begin{pgfscope}%
\pgfpathrectangle{\pgfqpoint{0.647939in}{0.492442in}}{\pgfqpoint{3.079299in}{3.079299in}}%
\pgfusepath{clip}%
\pgfsetroundcap%
\pgfsetroundjoin%
\definecolor{currentfill}{rgb}{0.500000,0.500000,0.500000}%
\pgfsetfillcolor{currentfill}%
\pgfsetfillopacity{0.300000}%
\pgfsetlinewidth{0.301125pt}%
\definecolor{currentstroke}{rgb}{0.500000,0.500000,0.500000}%
\pgfsetstrokecolor{currentstroke}%
\pgfsetstrokeopacity{0.300000}%
\pgfsetdash{}{0pt}%
\pgfpathmoveto{\pgfqpoint{0.000000in}{0.000000in}}%
\pgfpathlineto{\pgfqpoint{0.000000in}{0.000000in}}%
\pgfpathclose%
\pgfusepath{stroke,fill}%
\end{pgfscope}%
\begin{pgfscope}%
\pgfpathrectangle{\pgfqpoint{0.647939in}{0.492442in}}{\pgfqpoint{3.079299in}{3.079299in}}%
\pgfusepath{clip}%
\pgfsetroundcap%
\pgfsetroundjoin%
\pgfsetlinewidth{0.301125pt}%
\definecolor{currentstroke}{rgb}{0.500000,0.500000,0.500000}%
\pgfsetstrokecolor{currentstroke}%
\pgfsetstrokeopacity{0.300000}%
\pgfsetdash{}{0pt}%
\pgfpathmoveto{\pgfqpoint{0.874258in}{1.106958in}}%
\pgfusepath{stroke}%
\end{pgfscope}%
\begin{pgfscope}%
\pgfpathrectangle{\pgfqpoint{0.647939in}{0.492442in}}{\pgfqpoint{3.079299in}{3.079299in}}%
\pgfusepath{clip}%
\pgfsetroundcap%
\pgfsetroundjoin%
\definecolor{currentfill}{rgb}{0.500000,0.500000,0.500000}%
\pgfsetfillcolor{currentfill}%
\pgfsetfillopacity{0.300000}%
\pgfsetlinewidth{0.301125pt}%
\definecolor{currentstroke}{rgb}{0.500000,0.500000,0.500000}%
\pgfsetstrokecolor{currentstroke}%
\pgfsetstrokeopacity{0.300000}%
\pgfsetdash{}{0pt}%
\pgfpathmoveto{\pgfqpoint{0.000000in}{0.000000in}}%
\pgfpathlineto{\pgfqpoint{0.000000in}{0.000000in}}%
\pgfpathclose%
\pgfusepath{stroke,fill}%
\end{pgfscope}%
\begin{pgfscope}%
\pgfpathrectangle{\pgfqpoint{0.647939in}{0.492442in}}{\pgfqpoint{3.079299in}{3.079299in}}%
\pgfusepath{clip}%
\pgfsetroundcap%
\pgfsetroundjoin%
\pgfsetlinewidth{0.301125pt}%
\definecolor{currentstroke}{rgb}{0.500000,0.500000,0.500000}%
\pgfsetstrokecolor{currentstroke}%
\pgfsetstrokeopacity{0.300000}%
\pgfsetdash{}{0pt}%
\pgfpathmoveto{\pgfqpoint{1.352008in}{1.297926in}}%
\pgfusepath{stroke}%
\end{pgfscope}%
\begin{pgfscope}%
\pgfpathrectangle{\pgfqpoint{0.647939in}{0.492442in}}{\pgfqpoint{3.079299in}{3.079299in}}%
\pgfusepath{clip}%
\pgfsetroundcap%
\pgfsetroundjoin%
\definecolor{currentfill}{rgb}{0.500000,0.500000,0.500000}%
\pgfsetfillcolor{currentfill}%
\pgfsetfillopacity{0.300000}%
\pgfsetlinewidth{0.301125pt}%
\definecolor{currentstroke}{rgb}{0.500000,0.500000,0.500000}%
\pgfsetstrokecolor{currentstroke}%
\pgfsetstrokeopacity{0.300000}%
\pgfsetdash{}{0pt}%
\pgfpathmoveto{\pgfqpoint{0.000000in}{0.000000in}}%
\pgfpathlineto{\pgfqpoint{0.000000in}{0.000000in}}%
\pgfpathclose%
\pgfusepath{stroke,fill}%
\end{pgfscope}%
\begin{pgfscope}%
\pgfpathrectangle{\pgfqpoint{0.647939in}{0.492442in}}{\pgfqpoint{3.079299in}{3.079299in}}%
\pgfusepath{clip}%
\pgfsetroundcap%
\pgfsetroundjoin%
\pgfsetlinewidth{0.301125pt}%
\definecolor{currentstroke}{rgb}{0.500000,0.500000,0.500000}%
\pgfsetstrokecolor{currentstroke}%
\pgfsetstrokeopacity{0.300000}%
\pgfsetdash{}{0pt}%
\pgfpathmoveto{\pgfqpoint{1.001209in}{1.018530in}}%
\pgfusepath{stroke}%
\end{pgfscope}%
\begin{pgfscope}%
\pgfpathrectangle{\pgfqpoint{0.647939in}{0.492442in}}{\pgfqpoint{3.079299in}{3.079299in}}%
\pgfusepath{clip}%
\pgfsetroundcap%
\pgfsetroundjoin%
\definecolor{currentfill}{rgb}{0.500000,0.500000,0.500000}%
\pgfsetfillcolor{currentfill}%
\pgfsetfillopacity{0.300000}%
\pgfsetlinewidth{0.301125pt}%
\definecolor{currentstroke}{rgb}{0.500000,0.500000,0.500000}%
\pgfsetstrokecolor{currentstroke}%
\pgfsetstrokeopacity{0.300000}%
\pgfsetdash{}{0pt}%
\pgfpathmoveto{\pgfqpoint{0.000000in}{0.000000in}}%
\pgfpathlineto{\pgfqpoint{0.000000in}{0.000000in}}%
\pgfpathclose%
\pgfusepath{stroke,fill}%
\end{pgfscope}%
\begin{pgfscope}%
\pgfpathrectangle{\pgfqpoint{0.647939in}{0.492442in}}{\pgfqpoint{3.079299in}{3.079299in}}%
\pgfusepath{clip}%
\pgfsetroundcap%
\pgfsetroundjoin%
\pgfsetlinewidth{0.301125pt}%
\definecolor{currentstroke}{rgb}{0.500000,0.500000,0.500000}%
\pgfsetstrokecolor{currentstroke}%
\pgfsetstrokeopacity{0.300000}%
\pgfsetdash{}{0pt}%
\pgfpathmoveto{\pgfqpoint{0.937469in}{0.924235in}}%
\pgfusepath{stroke}%
\end{pgfscope}%
\begin{pgfscope}%
\pgfpathrectangle{\pgfqpoint{0.647939in}{0.492442in}}{\pgfqpoint{3.079299in}{3.079299in}}%
\pgfusepath{clip}%
\pgfsetroundcap%
\pgfsetroundjoin%
\definecolor{currentfill}{rgb}{0.500000,0.500000,0.500000}%
\pgfsetfillcolor{currentfill}%
\pgfsetfillopacity{0.300000}%
\pgfsetlinewidth{0.301125pt}%
\definecolor{currentstroke}{rgb}{0.500000,0.500000,0.500000}%
\pgfsetstrokecolor{currentstroke}%
\pgfsetstrokeopacity{0.300000}%
\pgfsetdash{}{0pt}%
\pgfpathmoveto{\pgfqpoint{0.000000in}{0.000000in}}%
\pgfpathlineto{\pgfqpoint{0.000000in}{0.000000in}}%
\pgfpathclose%
\pgfusepath{stroke,fill}%
\end{pgfscope}%
\begin{pgfscope}%
\pgfpathrectangle{\pgfqpoint{0.647939in}{0.492442in}}{\pgfqpoint{3.079299in}{3.079299in}}%
\pgfusepath{clip}%
\pgfsetroundcap%
\pgfsetroundjoin%
\pgfsetlinewidth{0.301125pt}%
\definecolor{currentstroke}{rgb}{0.500000,0.500000,0.500000}%
\pgfsetstrokecolor{currentstroke}%
\pgfsetstrokeopacity{0.300000}%
\pgfsetdash{}{0pt}%
\pgfpathmoveto{\pgfqpoint{1.119252in}{0.949508in}}%
\pgfusepath{stroke}%
\end{pgfscope}%
\begin{pgfscope}%
\pgfpathrectangle{\pgfqpoint{0.647939in}{0.492442in}}{\pgfqpoint{3.079299in}{3.079299in}}%
\pgfusepath{clip}%
\pgfsetroundcap%
\pgfsetroundjoin%
\definecolor{currentfill}{rgb}{0.500000,0.500000,0.500000}%
\pgfsetfillcolor{currentfill}%
\pgfsetfillopacity{0.300000}%
\pgfsetlinewidth{0.301125pt}%
\definecolor{currentstroke}{rgb}{0.500000,0.500000,0.500000}%
\pgfsetstrokecolor{currentstroke}%
\pgfsetstrokeopacity{0.300000}%
\pgfsetdash{}{0pt}%
\pgfpathmoveto{\pgfqpoint{0.000000in}{0.000000in}}%
\pgfpathlineto{\pgfqpoint{0.000000in}{0.000000in}}%
\pgfpathclose%
\pgfusepath{stroke,fill}%
\end{pgfscope}%
\begin{pgfscope}%
\pgfpathrectangle{\pgfqpoint{0.647939in}{0.492442in}}{\pgfqpoint{3.079299in}{3.079299in}}%
\pgfusepath{clip}%
\pgfsetroundcap%
\pgfsetroundjoin%
\pgfsetlinewidth{0.301125pt}%
\definecolor{currentstroke}{rgb}{0.500000,0.500000,0.500000}%
\pgfsetstrokecolor{currentstroke}%
\pgfsetstrokeopacity{0.300000}%
\pgfsetdash{}{0pt}%
\pgfpathmoveto{\pgfqpoint{0.872554in}{0.763285in}}%
\pgfusepath{stroke}%
\end{pgfscope}%
\begin{pgfscope}%
\pgfpathrectangle{\pgfqpoint{0.647939in}{0.492442in}}{\pgfqpoint{3.079299in}{3.079299in}}%
\pgfusepath{clip}%
\pgfsetroundcap%
\pgfsetroundjoin%
\definecolor{currentfill}{rgb}{0.500000,0.500000,0.500000}%
\pgfsetfillcolor{currentfill}%
\pgfsetfillopacity{0.300000}%
\pgfsetlinewidth{0.301125pt}%
\definecolor{currentstroke}{rgb}{0.500000,0.500000,0.500000}%
\pgfsetstrokecolor{currentstroke}%
\pgfsetstrokeopacity{0.300000}%
\pgfsetdash{}{0pt}%
\pgfpathmoveto{\pgfqpoint{0.000000in}{0.000000in}}%
\pgfpathlineto{\pgfqpoint{0.000000in}{0.000000in}}%
\pgfpathclose%
\pgfusepath{stroke,fill}%
\end{pgfscope}%
\begin{pgfscope}%
\pgfpathrectangle{\pgfqpoint{0.647939in}{0.492442in}}{\pgfqpoint{3.079299in}{3.079299in}}%
\pgfusepath{clip}%
\pgfsetroundcap%
\pgfsetroundjoin%
\pgfsetlinewidth{0.301125pt}%
\definecolor{currentstroke}{rgb}{0.500000,0.500000,0.500000}%
\pgfsetstrokecolor{currentstroke}%
\pgfsetstrokeopacity{0.300000}%
\pgfsetdash{}{0pt}%
\pgfpathmoveto{\pgfqpoint{0.997262in}{0.749666in}}%
\pgfusepath{stroke}%
\end{pgfscope}%
\begin{pgfscope}%
\pgfpathrectangle{\pgfqpoint{0.647939in}{0.492442in}}{\pgfqpoint{3.079299in}{3.079299in}}%
\pgfusepath{clip}%
\pgfsetroundcap%
\pgfsetroundjoin%
\definecolor{currentfill}{rgb}{0.500000,0.500000,0.500000}%
\pgfsetfillcolor{currentfill}%
\pgfsetfillopacity{0.300000}%
\pgfsetlinewidth{0.301125pt}%
\definecolor{currentstroke}{rgb}{0.500000,0.500000,0.500000}%
\pgfsetstrokecolor{currentstroke}%
\pgfsetstrokeopacity{0.300000}%
\pgfsetdash{}{0pt}%
\pgfpathmoveto{\pgfqpoint{0.000000in}{0.000000in}}%
\pgfpathlineto{\pgfqpoint{0.000000in}{0.000000in}}%
\pgfpathclose%
\pgfusepath{stroke,fill}%
\end{pgfscope}%
\begin{pgfscope}%
\pgfpathrectangle{\pgfqpoint{0.647939in}{0.492442in}}{\pgfqpoint{3.079299in}{3.079299in}}%
\pgfusepath{clip}%
\pgfsetroundcap%
\pgfsetroundjoin%
\pgfsetlinewidth{0.301125pt}%
\definecolor{currentstroke}{rgb}{0.500000,0.500000,0.500000}%
\pgfsetstrokecolor{currentstroke}%
\pgfsetstrokeopacity{0.300000}%
\pgfsetdash{}{0pt}%
\pgfpathmoveto{\pgfqpoint{3.507809in}{3.427060in}}%
\pgfusepath{stroke}%
\end{pgfscope}%
\begin{pgfscope}%
\pgfpathrectangle{\pgfqpoint{0.647939in}{0.492442in}}{\pgfqpoint{3.079299in}{3.079299in}}%
\pgfusepath{clip}%
\pgfsetroundcap%
\pgfsetroundjoin%
\definecolor{currentfill}{rgb}{0.500000,0.500000,0.500000}%
\pgfsetfillcolor{currentfill}%
\pgfsetfillopacity{0.300000}%
\pgfsetlinewidth{0.301125pt}%
\definecolor{currentstroke}{rgb}{0.500000,0.500000,0.500000}%
\pgfsetstrokecolor{currentstroke}%
\pgfsetstrokeopacity{0.300000}%
\pgfsetdash{}{0pt}%
\pgfpathmoveto{\pgfqpoint{0.000000in}{0.000000in}}%
\pgfpathlineto{\pgfqpoint{0.000000in}{0.000000in}}%
\pgfpathclose%
\pgfusepath{stroke,fill}%
\end{pgfscope}%
\begin{pgfscope}%
\pgfpathrectangle{\pgfqpoint{0.647939in}{0.492442in}}{\pgfqpoint{3.079299in}{3.079299in}}%
\pgfusepath{clip}%
\pgfsetroundcap%
\pgfsetroundjoin%
\pgfsetlinewidth{0.301125pt}%
\definecolor{currentstroke}{rgb}{0.500000,0.500000,0.500000}%
\pgfsetstrokecolor{currentstroke}%
\pgfsetstrokeopacity{0.300000}%
\pgfsetdash{}{0pt}%
\pgfpathmoveto{\pgfqpoint{1.877551in}{0.624640in}}%
\pgfusepath{stroke}%
\end{pgfscope}%
\begin{pgfscope}%
\pgfpathrectangle{\pgfqpoint{0.647939in}{0.492442in}}{\pgfqpoint{3.079299in}{3.079299in}}%
\pgfusepath{clip}%
\pgfsetroundcap%
\pgfsetroundjoin%
\definecolor{currentfill}{rgb}{0.500000,0.500000,0.500000}%
\pgfsetfillcolor{currentfill}%
\pgfsetfillopacity{0.300000}%
\pgfsetlinewidth{0.301125pt}%
\definecolor{currentstroke}{rgb}{0.500000,0.500000,0.500000}%
\pgfsetstrokecolor{currentstroke}%
\pgfsetstrokeopacity{0.300000}%
\pgfsetdash{}{0pt}%
\pgfpathmoveto{\pgfqpoint{0.000000in}{0.000000in}}%
\pgfpathlineto{\pgfqpoint{0.000000in}{0.000000in}}%
\pgfpathclose%
\pgfusepath{stroke,fill}%
\end{pgfscope}%
\begin{pgfscope}%
\pgfpathrectangle{\pgfqpoint{0.647939in}{0.492442in}}{\pgfqpoint{3.079299in}{3.079299in}}%
\pgfusepath{clip}%
\pgfsetroundcap%
\pgfsetroundjoin%
\pgfsetlinewidth{0.301125pt}%
\definecolor{currentstroke}{rgb}{0.500000,0.500000,0.500000}%
\pgfsetstrokecolor{currentstroke}%
\pgfsetstrokeopacity{0.300000}%
\pgfsetdash{}{0pt}%
\pgfpathmoveto{\pgfqpoint{3.449912in}{1.422899in}}%
\pgfusepath{stroke}%
\end{pgfscope}%
\begin{pgfscope}%
\pgfpathrectangle{\pgfqpoint{0.647939in}{0.492442in}}{\pgfqpoint{3.079299in}{3.079299in}}%
\pgfusepath{clip}%
\pgfsetroundcap%
\pgfsetroundjoin%
\definecolor{currentfill}{rgb}{0.500000,0.500000,0.500000}%
\pgfsetfillcolor{currentfill}%
\pgfsetfillopacity{0.300000}%
\pgfsetlinewidth{0.301125pt}%
\definecolor{currentstroke}{rgb}{0.500000,0.500000,0.500000}%
\pgfsetstrokecolor{currentstroke}%
\pgfsetstrokeopacity{0.300000}%
\pgfsetdash{}{0pt}%
\pgfpathmoveto{\pgfqpoint{0.000000in}{0.000000in}}%
\pgfpathlineto{\pgfqpoint{0.000000in}{0.000000in}}%
\pgfpathclose%
\pgfusepath{stroke,fill}%
\end{pgfscope}%
\begin{pgfscope}%
\pgfpathrectangle{\pgfqpoint{0.647939in}{0.492442in}}{\pgfqpoint{3.079299in}{3.079299in}}%
\pgfusepath{clip}%
\pgfsetroundcap%
\pgfsetroundjoin%
\pgfsetlinewidth{0.301125pt}%
\definecolor{currentstroke}{rgb}{0.500000,0.500000,0.500000}%
\pgfsetstrokecolor{currentstroke}%
\pgfsetstrokeopacity{0.300000}%
\pgfsetdash{}{0pt}%
\pgfpathmoveto{\pgfqpoint{3.063875in}{2.154439in}}%
\pgfusepath{stroke}%
\end{pgfscope}%
\begin{pgfscope}%
\pgfpathrectangle{\pgfqpoint{0.647939in}{0.492442in}}{\pgfqpoint{3.079299in}{3.079299in}}%
\pgfusepath{clip}%
\pgfsetroundcap%
\pgfsetroundjoin%
\definecolor{currentfill}{rgb}{0.500000,0.500000,0.500000}%
\pgfsetfillcolor{currentfill}%
\pgfsetfillopacity{0.300000}%
\pgfsetlinewidth{0.301125pt}%
\definecolor{currentstroke}{rgb}{0.500000,0.500000,0.500000}%
\pgfsetstrokecolor{currentstroke}%
\pgfsetstrokeopacity{0.300000}%
\pgfsetdash{}{0pt}%
\pgfpathmoveto{\pgfqpoint{0.000000in}{0.000000in}}%
\pgfpathlineto{\pgfqpoint{0.000000in}{0.000000in}}%
\pgfpathclose%
\pgfusepath{stroke,fill}%
\end{pgfscope}%
\begin{pgfscope}%
\pgfpathrectangle{\pgfqpoint{0.647939in}{0.492442in}}{\pgfqpoint{3.079299in}{3.079299in}}%
\pgfusepath{clip}%
\pgfsetroundcap%
\pgfsetroundjoin%
\pgfsetlinewidth{0.301125pt}%
\definecolor{currentstroke}{rgb}{0.500000,0.500000,0.500000}%
\pgfsetstrokecolor{currentstroke}%
\pgfsetstrokeopacity{0.300000}%
\pgfsetdash{}{0pt}%
\pgfpathmoveto{\pgfqpoint{2.861367in}{1.666638in}}%
\pgfusepath{stroke}%
\end{pgfscope}%
\begin{pgfscope}%
\pgfpathrectangle{\pgfqpoint{0.647939in}{0.492442in}}{\pgfqpoint{3.079299in}{3.079299in}}%
\pgfusepath{clip}%
\pgfsetroundcap%
\pgfsetroundjoin%
\definecolor{currentfill}{rgb}{0.500000,0.500000,0.500000}%
\pgfsetfillcolor{currentfill}%
\pgfsetfillopacity{0.300000}%
\pgfsetlinewidth{0.301125pt}%
\definecolor{currentstroke}{rgb}{0.500000,0.500000,0.500000}%
\pgfsetstrokecolor{currentstroke}%
\pgfsetstrokeopacity{0.300000}%
\pgfsetdash{}{0pt}%
\pgfpathmoveto{\pgfqpoint{0.000000in}{0.000000in}}%
\pgfpathlineto{\pgfqpoint{0.000000in}{0.000000in}}%
\pgfpathclose%
\pgfusepath{stroke,fill}%
\end{pgfscope}%
\begin{pgfscope}%
\pgfpathrectangle{\pgfqpoint{0.647939in}{0.492442in}}{\pgfqpoint{3.079299in}{3.079299in}}%
\pgfusepath{clip}%
\pgfsetroundcap%
\pgfsetroundjoin%
\pgfsetlinewidth{0.301125pt}%
\definecolor{currentstroke}{rgb}{0.500000,0.500000,0.500000}%
\pgfsetstrokecolor{currentstroke}%
\pgfsetstrokeopacity{0.300000}%
\pgfsetdash{}{0pt}%
\pgfpathmoveto{\pgfqpoint{2.222592in}{3.270835in}}%
\pgfusepath{stroke}%
\end{pgfscope}%
\begin{pgfscope}%
\pgfpathrectangle{\pgfqpoint{0.647939in}{0.492442in}}{\pgfqpoint{3.079299in}{3.079299in}}%
\pgfusepath{clip}%
\pgfsetroundcap%
\pgfsetroundjoin%
\definecolor{currentfill}{rgb}{0.500000,0.500000,0.500000}%
\pgfsetfillcolor{currentfill}%
\pgfsetfillopacity{0.300000}%
\pgfsetlinewidth{0.301125pt}%
\definecolor{currentstroke}{rgb}{0.500000,0.500000,0.500000}%
\pgfsetstrokecolor{currentstroke}%
\pgfsetstrokeopacity{0.300000}%
\pgfsetdash{}{0pt}%
\pgfpathmoveto{\pgfqpoint{0.000000in}{0.000000in}}%
\pgfpathlineto{\pgfqpoint{0.000000in}{0.000000in}}%
\pgfpathclose%
\pgfusepath{stroke,fill}%
\end{pgfscope}%
\begin{pgfscope}%
\pgfpathrectangle{\pgfqpoint{0.647939in}{0.492442in}}{\pgfqpoint{3.079299in}{3.079299in}}%
\pgfusepath{clip}%
\pgfsetroundcap%
\pgfsetroundjoin%
\pgfsetlinewidth{0.301125pt}%
\definecolor{currentstroke}{rgb}{0.500000,0.500000,0.500000}%
\pgfsetstrokecolor{currentstroke}%
\pgfsetstrokeopacity{0.300000}%
\pgfsetdash{}{0pt}%
\pgfpathmoveto{\pgfqpoint{2.243489in}{3.111436in}}%
\pgfusepath{stroke}%
\end{pgfscope}%
\begin{pgfscope}%
\pgfpathrectangle{\pgfqpoint{0.647939in}{0.492442in}}{\pgfqpoint{3.079299in}{3.079299in}}%
\pgfusepath{clip}%
\pgfsetroundcap%
\pgfsetroundjoin%
\definecolor{currentfill}{rgb}{0.500000,0.500000,0.500000}%
\pgfsetfillcolor{currentfill}%
\pgfsetfillopacity{0.300000}%
\pgfsetlinewidth{0.301125pt}%
\definecolor{currentstroke}{rgb}{0.500000,0.500000,0.500000}%
\pgfsetstrokecolor{currentstroke}%
\pgfsetstrokeopacity{0.300000}%
\pgfsetdash{}{0pt}%
\pgfpathmoveto{\pgfqpoint{0.000000in}{0.000000in}}%
\pgfpathlineto{\pgfqpoint{0.000000in}{0.000000in}}%
\pgfpathclose%
\pgfusepath{stroke,fill}%
\end{pgfscope}%
\begin{pgfscope}%
\pgfpathrectangle{\pgfqpoint{0.647939in}{0.492442in}}{\pgfqpoint{3.079299in}{3.079299in}}%
\pgfusepath{clip}%
\pgfsetroundcap%
\pgfsetroundjoin%
\pgfsetlinewidth{0.301125pt}%
\definecolor{currentstroke}{rgb}{0.500000,0.500000,0.500000}%
\pgfsetstrokecolor{currentstroke}%
\pgfsetstrokeopacity{0.300000}%
\pgfsetdash{}{0pt}%
\pgfpathmoveto{\pgfqpoint{1.321165in}{3.152291in}}%
\pgfusepath{stroke}%
\end{pgfscope}%
\begin{pgfscope}%
\pgfpathrectangle{\pgfqpoint{0.647939in}{0.492442in}}{\pgfqpoint{3.079299in}{3.079299in}}%
\pgfusepath{clip}%
\pgfsetroundcap%
\pgfsetroundjoin%
\definecolor{currentfill}{rgb}{0.500000,0.500000,0.500000}%
\pgfsetfillcolor{currentfill}%
\pgfsetfillopacity{0.300000}%
\pgfsetlinewidth{0.301125pt}%
\definecolor{currentstroke}{rgb}{0.500000,0.500000,0.500000}%
\pgfsetstrokecolor{currentstroke}%
\pgfsetstrokeopacity{0.300000}%
\pgfsetdash{}{0pt}%
\pgfpathmoveto{\pgfqpoint{0.000000in}{0.000000in}}%
\pgfpathlineto{\pgfqpoint{0.000000in}{0.000000in}}%
\pgfpathclose%
\pgfusepath{stroke,fill}%
\end{pgfscope}%
\begin{pgfscope}%
\pgfpathrectangle{\pgfqpoint{0.647939in}{0.492442in}}{\pgfqpoint{3.079299in}{3.079299in}}%
\pgfusepath{clip}%
\pgfsetroundcap%
\pgfsetroundjoin%
\pgfsetlinewidth{0.301125pt}%
\definecolor{currentstroke}{rgb}{0.500000,0.500000,0.500000}%
\pgfsetstrokecolor{currentstroke}%
\pgfsetstrokeopacity{0.300000}%
\pgfsetdash{}{0pt}%
\pgfpathmoveto{\pgfqpoint{3.213119in}{3.119190in}}%
\pgfusepath{stroke}%
\end{pgfscope}%
\begin{pgfscope}%
\pgfpathrectangle{\pgfqpoint{0.647939in}{0.492442in}}{\pgfqpoint{3.079299in}{3.079299in}}%
\pgfusepath{clip}%
\pgfsetroundcap%
\pgfsetroundjoin%
\definecolor{currentfill}{rgb}{0.500000,0.500000,0.500000}%
\pgfsetfillcolor{currentfill}%
\pgfsetfillopacity{0.300000}%
\pgfsetlinewidth{0.301125pt}%
\definecolor{currentstroke}{rgb}{0.500000,0.500000,0.500000}%
\pgfsetstrokecolor{currentstroke}%
\pgfsetstrokeopacity{0.300000}%
\pgfsetdash{}{0pt}%
\pgfpathmoveto{\pgfqpoint{0.000000in}{0.000000in}}%
\pgfpathlineto{\pgfqpoint{0.000000in}{0.000000in}}%
\pgfpathclose%
\pgfusepath{stroke,fill}%
\end{pgfscope}%
\begin{pgfscope}%
\pgfpathrectangle{\pgfqpoint{0.647939in}{0.492442in}}{\pgfqpoint{3.079299in}{3.079299in}}%
\pgfusepath{clip}%
\pgfsetroundcap%
\pgfsetroundjoin%
\pgfsetlinewidth{0.301125pt}%
\definecolor{currentstroke}{rgb}{0.500000,0.500000,0.500000}%
\pgfsetstrokecolor{currentstroke}%
\pgfsetstrokeopacity{0.300000}%
\pgfsetdash{}{0pt}%
\pgfpathmoveto{\pgfqpoint{2.080320in}{3.142117in}}%
\pgfusepath{stroke}%
\end{pgfscope}%
\begin{pgfscope}%
\pgfpathrectangle{\pgfqpoint{0.647939in}{0.492442in}}{\pgfqpoint{3.079299in}{3.079299in}}%
\pgfusepath{clip}%
\pgfsetroundcap%
\pgfsetroundjoin%
\definecolor{currentfill}{rgb}{0.500000,0.500000,0.500000}%
\pgfsetfillcolor{currentfill}%
\pgfsetfillopacity{0.300000}%
\pgfsetlinewidth{0.301125pt}%
\definecolor{currentstroke}{rgb}{0.500000,0.500000,0.500000}%
\pgfsetstrokecolor{currentstroke}%
\pgfsetstrokeopacity{0.300000}%
\pgfsetdash{}{0pt}%
\pgfpathmoveto{\pgfqpoint{0.000000in}{0.000000in}}%
\pgfpathlineto{\pgfqpoint{0.000000in}{0.000000in}}%
\pgfpathclose%
\pgfusepath{stroke,fill}%
\end{pgfscope}%
\begin{pgfscope}%
\pgfpathrectangle{\pgfqpoint{0.647939in}{0.492442in}}{\pgfqpoint{3.079299in}{3.079299in}}%
\pgfusepath{clip}%
\pgfsetroundcap%
\pgfsetroundjoin%
\pgfsetlinewidth{0.301125pt}%
\definecolor{currentstroke}{rgb}{0.500000,0.500000,0.500000}%
\pgfsetstrokecolor{currentstroke}%
\pgfsetstrokeopacity{0.300000}%
\pgfsetdash{}{0pt}%
\pgfpathmoveto{\pgfqpoint{2.830293in}{2.017455in}}%
\pgfusepath{stroke}%
\end{pgfscope}%
\begin{pgfscope}%
\pgfpathrectangle{\pgfqpoint{0.647939in}{0.492442in}}{\pgfqpoint{3.079299in}{3.079299in}}%
\pgfusepath{clip}%
\pgfsetroundcap%
\pgfsetroundjoin%
\definecolor{currentfill}{rgb}{0.500000,0.500000,0.500000}%
\pgfsetfillcolor{currentfill}%
\pgfsetfillopacity{0.300000}%
\pgfsetlinewidth{0.301125pt}%
\definecolor{currentstroke}{rgb}{0.500000,0.500000,0.500000}%
\pgfsetstrokecolor{currentstroke}%
\pgfsetstrokeopacity{0.300000}%
\pgfsetdash{}{0pt}%
\pgfpathmoveto{\pgfqpoint{0.000000in}{0.000000in}}%
\pgfpathlineto{\pgfqpoint{0.000000in}{0.000000in}}%
\pgfpathclose%
\pgfusepath{stroke,fill}%
\end{pgfscope}%
\begin{pgfscope}%
\pgfpathrectangle{\pgfqpoint{0.647939in}{0.492442in}}{\pgfqpoint{3.079299in}{3.079299in}}%
\pgfusepath{clip}%
\pgfsetroundcap%
\pgfsetroundjoin%
\pgfsetlinewidth{0.301125pt}%
\definecolor{currentstroke}{rgb}{0.500000,0.500000,0.500000}%
\pgfsetstrokecolor{currentstroke}%
\pgfsetstrokeopacity{0.300000}%
\pgfsetdash{}{0pt}%
\pgfpathmoveto{\pgfqpoint{3.166854in}{2.393296in}}%
\pgfusepath{stroke}%
\end{pgfscope}%
\begin{pgfscope}%
\pgfpathrectangle{\pgfqpoint{0.647939in}{0.492442in}}{\pgfqpoint{3.079299in}{3.079299in}}%
\pgfusepath{clip}%
\pgfsetroundcap%
\pgfsetroundjoin%
\definecolor{currentfill}{rgb}{0.500000,0.500000,0.500000}%
\pgfsetfillcolor{currentfill}%
\pgfsetfillopacity{0.300000}%
\pgfsetlinewidth{0.301125pt}%
\definecolor{currentstroke}{rgb}{0.500000,0.500000,0.500000}%
\pgfsetstrokecolor{currentstroke}%
\pgfsetstrokeopacity{0.300000}%
\pgfsetdash{}{0pt}%
\pgfpathmoveto{\pgfqpoint{0.000000in}{0.000000in}}%
\pgfpathlineto{\pgfqpoint{0.000000in}{0.000000in}}%
\pgfpathclose%
\pgfusepath{stroke,fill}%
\end{pgfscope}%
\begin{pgfscope}%
\pgfpathrectangle{\pgfqpoint{0.647939in}{0.492442in}}{\pgfqpoint{3.079299in}{3.079299in}}%
\pgfusepath{clip}%
\pgfsetroundcap%
\pgfsetroundjoin%
\pgfsetlinewidth{0.301125pt}%
\definecolor{currentstroke}{rgb}{0.500000,0.500000,0.500000}%
\pgfsetstrokecolor{currentstroke}%
\pgfsetstrokeopacity{0.300000}%
\pgfsetdash{}{0pt}%
\pgfpathmoveto{\pgfqpoint{2.839929in}{3.098124in}}%
\pgfusepath{stroke}%
\end{pgfscope}%
\begin{pgfscope}%
\pgfpathrectangle{\pgfqpoint{0.647939in}{0.492442in}}{\pgfqpoint{3.079299in}{3.079299in}}%
\pgfusepath{clip}%
\pgfsetroundcap%
\pgfsetroundjoin%
\definecolor{currentfill}{rgb}{0.500000,0.500000,0.500000}%
\pgfsetfillcolor{currentfill}%
\pgfsetfillopacity{0.300000}%
\pgfsetlinewidth{0.301125pt}%
\definecolor{currentstroke}{rgb}{0.500000,0.500000,0.500000}%
\pgfsetstrokecolor{currentstroke}%
\pgfsetstrokeopacity{0.300000}%
\pgfsetdash{}{0pt}%
\pgfpathmoveto{\pgfqpoint{0.000000in}{0.000000in}}%
\pgfpathlineto{\pgfqpoint{0.000000in}{0.000000in}}%
\pgfpathclose%
\pgfusepath{stroke,fill}%
\end{pgfscope}%
\begin{pgfscope}%
\pgfpathrectangle{\pgfqpoint{0.647939in}{0.492442in}}{\pgfqpoint{3.079299in}{3.079299in}}%
\pgfusepath{clip}%
\pgfsetroundcap%
\pgfsetroundjoin%
\pgfsetlinewidth{0.301125pt}%
\definecolor{currentstroke}{rgb}{0.500000,0.500000,0.500000}%
\pgfsetstrokecolor{currentstroke}%
\pgfsetstrokeopacity{0.300000}%
\pgfsetdash{}{0pt}%
\pgfpathmoveto{\pgfqpoint{2.945607in}{2.266782in}}%
\pgfusepath{stroke}%
\end{pgfscope}%
\begin{pgfscope}%
\pgfpathrectangle{\pgfqpoint{0.647939in}{0.492442in}}{\pgfqpoint{3.079299in}{3.079299in}}%
\pgfusepath{clip}%
\pgfsetroundcap%
\pgfsetroundjoin%
\definecolor{currentfill}{rgb}{0.500000,0.500000,0.500000}%
\pgfsetfillcolor{currentfill}%
\pgfsetfillopacity{0.300000}%
\pgfsetlinewidth{0.301125pt}%
\definecolor{currentstroke}{rgb}{0.500000,0.500000,0.500000}%
\pgfsetstrokecolor{currentstroke}%
\pgfsetstrokeopacity{0.300000}%
\pgfsetdash{}{0pt}%
\pgfpathmoveto{\pgfqpoint{0.000000in}{0.000000in}}%
\pgfpathlineto{\pgfqpoint{0.000000in}{0.000000in}}%
\pgfpathclose%
\pgfusepath{stroke,fill}%
\end{pgfscope}%
\begin{pgfscope}%
\pgfpathrectangle{\pgfqpoint{0.647939in}{0.492442in}}{\pgfqpoint{3.079299in}{3.079299in}}%
\pgfusepath{clip}%
\pgfsetroundcap%
\pgfsetroundjoin%
\pgfsetlinewidth{0.301125pt}%
\definecolor{currentstroke}{rgb}{0.500000,0.500000,0.500000}%
\pgfsetstrokecolor{currentstroke}%
\pgfsetstrokeopacity{0.300000}%
\pgfsetdash{}{0pt}%
\pgfpathmoveto{\pgfqpoint{1.411943in}{1.654470in}}%
\pgfusepath{stroke}%
\end{pgfscope}%
\begin{pgfscope}%
\pgfpathrectangle{\pgfqpoint{0.647939in}{0.492442in}}{\pgfqpoint{3.079299in}{3.079299in}}%
\pgfusepath{clip}%
\pgfsetroundcap%
\pgfsetroundjoin%
\definecolor{currentfill}{rgb}{0.500000,0.500000,0.500000}%
\pgfsetfillcolor{currentfill}%
\pgfsetfillopacity{0.300000}%
\pgfsetlinewidth{0.301125pt}%
\definecolor{currentstroke}{rgb}{0.500000,0.500000,0.500000}%
\pgfsetstrokecolor{currentstroke}%
\pgfsetstrokeopacity{0.300000}%
\pgfsetdash{}{0pt}%
\pgfpathmoveto{\pgfqpoint{0.000000in}{0.000000in}}%
\pgfpathlineto{\pgfqpoint{0.000000in}{0.000000in}}%
\pgfpathclose%
\pgfusepath{stroke,fill}%
\end{pgfscope}%
\begin{pgfscope}%
\pgfpathrectangle{\pgfqpoint{0.647939in}{0.492442in}}{\pgfqpoint{3.079299in}{3.079299in}}%
\pgfusepath{clip}%
\pgfsetroundcap%
\pgfsetroundjoin%
\pgfsetlinewidth{0.301125pt}%
\definecolor{currentstroke}{rgb}{0.500000,0.500000,0.500000}%
\pgfsetstrokecolor{currentstroke}%
\pgfsetstrokeopacity{0.300000}%
\pgfsetdash{}{0pt}%
\pgfpathmoveto{\pgfqpoint{2.514194in}{2.873736in}}%
\pgfusepath{stroke}%
\end{pgfscope}%
\begin{pgfscope}%
\pgfpathrectangle{\pgfqpoint{0.647939in}{0.492442in}}{\pgfqpoint{3.079299in}{3.079299in}}%
\pgfusepath{clip}%
\pgfsetroundcap%
\pgfsetroundjoin%
\definecolor{currentfill}{rgb}{0.500000,0.500000,0.500000}%
\pgfsetfillcolor{currentfill}%
\pgfsetfillopacity{0.300000}%
\pgfsetlinewidth{0.301125pt}%
\definecolor{currentstroke}{rgb}{0.500000,0.500000,0.500000}%
\pgfsetstrokecolor{currentstroke}%
\pgfsetstrokeopacity{0.300000}%
\pgfsetdash{}{0pt}%
\pgfpathmoveto{\pgfqpoint{0.000000in}{0.000000in}}%
\pgfpathlineto{\pgfqpoint{0.000000in}{0.000000in}}%
\pgfpathclose%
\pgfusepath{stroke,fill}%
\end{pgfscope}%
\begin{pgfscope}%
\pgfpathrectangle{\pgfqpoint{0.647939in}{0.492442in}}{\pgfqpoint{3.079299in}{3.079299in}}%
\pgfusepath{clip}%
\pgfsetroundcap%
\pgfsetroundjoin%
\pgfsetlinewidth{0.301125pt}%
\definecolor{currentstroke}{rgb}{0.500000,0.500000,0.500000}%
\pgfsetstrokecolor{currentstroke}%
\pgfsetstrokeopacity{0.300000}%
\pgfsetdash{}{0pt}%
\pgfpathmoveto{\pgfqpoint{1.638624in}{2.097923in}}%
\pgfusepath{stroke}%
\end{pgfscope}%
\begin{pgfscope}%
\pgfpathrectangle{\pgfqpoint{0.647939in}{0.492442in}}{\pgfqpoint{3.079299in}{3.079299in}}%
\pgfusepath{clip}%
\pgfsetroundcap%
\pgfsetroundjoin%
\definecolor{currentfill}{rgb}{0.500000,0.500000,0.500000}%
\pgfsetfillcolor{currentfill}%
\pgfsetfillopacity{0.300000}%
\pgfsetlinewidth{0.301125pt}%
\definecolor{currentstroke}{rgb}{0.500000,0.500000,0.500000}%
\pgfsetstrokecolor{currentstroke}%
\pgfsetstrokeopacity{0.300000}%
\pgfsetdash{}{0pt}%
\pgfpathmoveto{\pgfqpoint{0.000000in}{0.000000in}}%
\pgfpathlineto{\pgfqpoint{0.000000in}{0.000000in}}%
\pgfpathclose%
\pgfusepath{stroke,fill}%
\end{pgfscope}%
\begin{pgfscope}%
\pgfpathrectangle{\pgfqpoint{0.647939in}{0.492442in}}{\pgfqpoint{3.079299in}{3.079299in}}%
\pgfusepath{clip}%
\pgfsetroundcap%
\pgfsetroundjoin%
\pgfsetlinewidth{0.301125pt}%
\definecolor{currentstroke}{rgb}{0.500000,0.500000,0.500000}%
\pgfsetstrokecolor{currentstroke}%
\pgfsetstrokeopacity{0.300000}%
\pgfsetdash{}{0pt}%
\pgfpathmoveto{\pgfqpoint{2.558625in}{1.958649in}}%
\pgfusepath{stroke}%
\end{pgfscope}%
\begin{pgfscope}%
\pgfpathrectangle{\pgfqpoint{0.647939in}{0.492442in}}{\pgfqpoint{3.079299in}{3.079299in}}%
\pgfusepath{clip}%
\pgfsetroundcap%
\pgfsetroundjoin%
\definecolor{currentfill}{rgb}{0.500000,0.500000,0.500000}%
\pgfsetfillcolor{currentfill}%
\pgfsetfillopacity{0.300000}%
\pgfsetlinewidth{0.301125pt}%
\definecolor{currentstroke}{rgb}{0.500000,0.500000,0.500000}%
\pgfsetstrokecolor{currentstroke}%
\pgfsetstrokeopacity{0.300000}%
\pgfsetdash{}{0pt}%
\pgfpathmoveto{\pgfqpoint{0.000000in}{0.000000in}}%
\pgfpathlineto{\pgfqpoint{0.000000in}{0.000000in}}%
\pgfpathclose%
\pgfusepath{stroke,fill}%
\end{pgfscope}%
\begin{pgfscope}%
\pgfpathrectangle{\pgfqpoint{0.647939in}{0.492442in}}{\pgfqpoint{3.079299in}{3.079299in}}%
\pgfusepath{clip}%
\pgfsetroundcap%
\pgfsetroundjoin%
\pgfsetlinewidth{0.301125pt}%
\definecolor{currentstroke}{rgb}{0.500000,0.500000,0.500000}%
\pgfsetstrokecolor{currentstroke}%
\pgfsetstrokeopacity{0.300000}%
\pgfsetdash{}{0pt}%
\pgfpathmoveto{\pgfqpoint{2.655681in}{2.167976in}}%
\pgfusepath{stroke}%
\end{pgfscope}%
\begin{pgfscope}%
\pgfpathrectangle{\pgfqpoint{0.647939in}{0.492442in}}{\pgfqpoint{3.079299in}{3.079299in}}%
\pgfusepath{clip}%
\pgfsetroundcap%
\pgfsetroundjoin%
\definecolor{currentfill}{rgb}{0.500000,0.500000,0.500000}%
\pgfsetfillcolor{currentfill}%
\pgfsetfillopacity{0.300000}%
\pgfsetlinewidth{0.301125pt}%
\definecolor{currentstroke}{rgb}{0.500000,0.500000,0.500000}%
\pgfsetstrokecolor{currentstroke}%
\pgfsetstrokeopacity{0.300000}%
\pgfsetdash{}{0pt}%
\pgfpathmoveto{\pgfqpoint{0.000000in}{0.000000in}}%
\pgfpathlineto{\pgfqpoint{0.000000in}{0.000000in}}%
\pgfpathclose%
\pgfusepath{stroke,fill}%
\end{pgfscope}%
\begin{pgfscope}%
\pgfpathrectangle{\pgfqpoint{0.647939in}{0.492442in}}{\pgfqpoint{3.079299in}{3.079299in}}%
\pgfusepath{clip}%
\pgfsetroundcap%
\pgfsetroundjoin%
\pgfsetlinewidth{0.301125pt}%
\definecolor{currentstroke}{rgb}{0.500000,0.500000,0.500000}%
\pgfsetstrokecolor{currentstroke}%
\pgfsetstrokeopacity{0.300000}%
\pgfsetdash{}{0pt}%
\pgfpathmoveto{\pgfqpoint{1.840406in}{2.057136in}}%
\pgfusepath{stroke}%
\end{pgfscope}%
\begin{pgfscope}%
\pgfpathrectangle{\pgfqpoint{0.647939in}{0.492442in}}{\pgfqpoint{3.079299in}{3.079299in}}%
\pgfusepath{clip}%
\pgfsetroundcap%
\pgfsetroundjoin%
\definecolor{currentfill}{rgb}{0.500000,0.500000,0.500000}%
\pgfsetfillcolor{currentfill}%
\pgfsetfillopacity{0.300000}%
\pgfsetlinewidth{0.301125pt}%
\definecolor{currentstroke}{rgb}{0.500000,0.500000,0.500000}%
\pgfsetstrokecolor{currentstroke}%
\pgfsetstrokeopacity{0.300000}%
\pgfsetdash{}{0pt}%
\pgfpathmoveto{\pgfqpoint{0.000000in}{0.000000in}}%
\pgfpathlineto{\pgfqpoint{0.000000in}{0.000000in}}%
\pgfpathclose%
\pgfusepath{stroke,fill}%
\end{pgfscope}%
\begin{pgfscope}%
\pgfpathrectangle{\pgfqpoint{0.647939in}{0.492442in}}{\pgfqpoint{3.079299in}{3.079299in}}%
\pgfusepath{clip}%
\pgfsetroundcap%
\pgfsetroundjoin%
\pgfsetlinewidth{0.301125pt}%
\definecolor{currentstroke}{rgb}{0.500000,0.500000,0.500000}%
\pgfsetstrokecolor{currentstroke}%
\pgfsetstrokeopacity{0.300000}%
\pgfsetdash{}{0pt}%
\pgfpathmoveto{\pgfqpoint{2.295321in}{1.595048in}}%
\pgfusepath{stroke}%
\end{pgfscope}%
\begin{pgfscope}%
\pgfpathrectangle{\pgfqpoint{0.647939in}{0.492442in}}{\pgfqpoint{3.079299in}{3.079299in}}%
\pgfusepath{clip}%
\pgfsetroundcap%
\pgfsetroundjoin%
\definecolor{currentfill}{rgb}{0.500000,0.500000,0.500000}%
\pgfsetfillcolor{currentfill}%
\pgfsetfillopacity{0.300000}%
\pgfsetlinewidth{0.301125pt}%
\definecolor{currentstroke}{rgb}{0.500000,0.500000,0.500000}%
\pgfsetstrokecolor{currentstroke}%
\pgfsetstrokeopacity{0.300000}%
\pgfsetdash{}{0pt}%
\pgfpathmoveto{\pgfqpoint{0.000000in}{0.000000in}}%
\pgfpathlineto{\pgfqpoint{0.000000in}{0.000000in}}%
\pgfpathclose%
\pgfusepath{stroke,fill}%
\end{pgfscope}%
\begin{pgfscope}%
\pgfpathrectangle{\pgfqpoint{0.647939in}{0.492442in}}{\pgfqpoint{3.079299in}{3.079299in}}%
\pgfusepath{clip}%
\pgfsetroundcap%
\pgfsetroundjoin%
\pgfsetlinewidth{0.301125pt}%
\definecolor{currentstroke}{rgb}{0.500000,0.500000,0.500000}%
\pgfsetstrokecolor{currentstroke}%
\pgfsetstrokeopacity{0.300000}%
\pgfsetdash{}{0pt}%
\pgfpathmoveto{\pgfqpoint{2.387488in}{1.889798in}}%
\pgfusepath{stroke}%
\end{pgfscope}%
\begin{pgfscope}%
\pgfpathrectangle{\pgfqpoint{0.647939in}{0.492442in}}{\pgfqpoint{3.079299in}{3.079299in}}%
\pgfusepath{clip}%
\pgfsetroundcap%
\pgfsetroundjoin%
\definecolor{currentfill}{rgb}{0.500000,0.500000,0.500000}%
\pgfsetfillcolor{currentfill}%
\pgfsetfillopacity{0.300000}%
\pgfsetlinewidth{0.301125pt}%
\definecolor{currentstroke}{rgb}{0.500000,0.500000,0.500000}%
\pgfsetstrokecolor{currentstroke}%
\pgfsetstrokeopacity{0.300000}%
\pgfsetdash{}{0pt}%
\pgfpathmoveto{\pgfqpoint{0.000000in}{0.000000in}}%
\pgfpathlineto{\pgfqpoint{0.000000in}{0.000000in}}%
\pgfpathclose%
\pgfusepath{stroke,fill}%
\end{pgfscope}%
\begin{pgfscope}%
\pgfpathrectangle{\pgfqpoint{0.647939in}{0.492442in}}{\pgfqpoint{3.079299in}{3.079299in}}%
\pgfusepath{clip}%
\pgfsetroundcap%
\pgfsetroundjoin%
\pgfsetlinewidth{0.301125pt}%
\definecolor{currentstroke}{rgb}{0.500000,0.500000,0.500000}%
\pgfsetstrokecolor{currentstroke}%
\pgfsetstrokeopacity{0.300000}%
\pgfsetdash{}{0pt}%
\pgfpathmoveto{\pgfqpoint{2.071874in}{2.261268in}}%
\pgfusepath{stroke}%
\end{pgfscope}%
\begin{pgfscope}%
\pgfpathrectangle{\pgfqpoint{0.647939in}{0.492442in}}{\pgfqpoint{3.079299in}{3.079299in}}%
\pgfusepath{clip}%
\pgfsetroundcap%
\pgfsetroundjoin%
\definecolor{currentfill}{rgb}{0.500000,0.500000,0.500000}%
\pgfsetfillcolor{currentfill}%
\pgfsetfillopacity{0.300000}%
\pgfsetlinewidth{0.301125pt}%
\definecolor{currentstroke}{rgb}{0.500000,0.500000,0.500000}%
\pgfsetstrokecolor{currentstroke}%
\pgfsetstrokeopacity{0.300000}%
\pgfsetdash{}{0pt}%
\pgfpathmoveto{\pgfqpoint{0.000000in}{0.000000in}}%
\pgfpathlineto{\pgfqpoint{0.000000in}{0.000000in}}%
\pgfpathclose%
\pgfusepath{stroke,fill}%
\end{pgfscope}%
\begin{pgfscope}%
\pgfpathrectangle{\pgfqpoint{0.647939in}{0.492442in}}{\pgfqpoint{3.079299in}{3.079299in}}%
\pgfusepath{clip}%
\pgfsetbuttcap%
\pgfsetroundjoin%
\pgfsetlinewidth{0.301125pt}%
\definecolor{currentstroke}{rgb}{0.500000,0.500000,0.500000}%
\pgfsetstrokecolor{currentstroke}%
\pgfsetstrokeopacity{0.300000}%
\pgfsetdash{}{0pt}%
\pgfpathmoveto{\pgfqpoint{0.647939in}{0.492442in}}%
\pgfpathlineto{\pgfqpoint{0.647939in}{0.492442in}}%
\pgfpathlineto{\pgfqpoint{0.714610in}{0.507773in}}%
\pgfpathlineto{\pgfqpoint{0.780470in}{0.526250in}}%
\pgfpathlineto{\pgfqpoint{0.845245in}{0.548208in}}%
\pgfpathlineto{\pgfqpoint{0.908610in}{0.573929in}}%
\pgfpathlineto{\pgfqpoint{0.970207in}{0.603619in}}%
\pgfpathlineto{\pgfqpoint{1.029668in}{0.637372in}}%
\pgfpathlineto{\pgfqpoint{1.086652in}{0.675155in}}%
\pgfpathlineto{\pgfqpoint{1.140886in}{0.716795in}}%
\pgfpathlineto{\pgfqpoint{1.192207in}{0.761962in}}%
\pgfpathlineto{\pgfqpoint{1.240580in}{0.810280in}}%
\pgfpathlineto{\pgfqpoint{1.286110in}{0.861292in}}%
\pgfpathlineto{\pgfqpoint{1.329020in}{0.914532in}}%
\pgfpathlineto{\pgfqpoint{1.369629in}{0.969552in}}%
\pgfpathlineto{\pgfqpoint{1.408315in}{1.025931in}}%
\pgfpathlineto{\pgfqpoint{1.445480in}{1.083325in}}%
\pgfpathlineto{\pgfqpoint{1.481525in}{1.141439in}}%
\pgfpathlineto{\pgfqpoint{1.516840in}{1.200014in}}%
\pgfpathlineto{\pgfqpoint{1.551797in}{1.258808in}}%
\pgfpathlineto{\pgfqpoint{1.586744in}{1.317591in}}%
\pgfpathlineto{\pgfqpoint{1.621995in}{1.376167in}}%
\pgfpathlineto{\pgfqpoint{1.657846in}{1.434363in}}%
\pgfpathlineto{\pgfqpoint{1.694573in}{1.492012in}}%
\pgfpathlineto{\pgfqpoint{1.732440in}{1.548947in}}%
\pgfpathlineto{\pgfqpoint{1.771679in}{1.604949in}}%
\pgfpathlineto{\pgfqpoint{1.812492in}{1.659758in}}%
\pgfpathlineto{\pgfqpoint{1.855074in}{1.713183in}}%
\pgfpathlineto{\pgfqpoint{1.899578in}{1.765051in}}%
\pgfpathlineto{\pgfqpoint{1.946004in}{1.815149in}}%
\pgfpathlineto{\pgfqpoint{1.994264in}{1.863457in}}%
\pgfpathlineto{\pgfqpoint{2.044057in}{1.910160in}}%
\pgfpathlineto{\pgfqpoint{2.094799in}{1.955774in}}%
\pgfpathlineto{\pgfqpoint{2.145681in}{2.001083in}}%
\pgfpathlineto{\pgfqpoint{2.195953in}{2.047019in}}%
\pgfpathlineto{\pgfqpoint{2.245014in}{2.094073in}}%
\pgfpathlineto{\pgfqpoint{2.292867in}{2.142448in}}%
\pgfpathlineto{\pgfqpoint{2.339553in}{2.192012in}}%
\pgfpathlineto{\pgfqpoint{2.385223in}{2.242498in}}%
\pgfpathlineto{\pgfqpoint{2.430147in}{2.293824in}}%
\pgfpathlineto{\pgfqpoint{2.474410in}{2.345751in}}%
\pgfpathlineto{\pgfqpoint{2.518165in}{2.398169in}}%
\pgfpathlineto{\pgfqpoint{2.561522in}{2.450987in}}%
\pgfpathlineto{\pgfqpoint{2.604556in}{2.504097in}}%
\pgfpathlineto{\pgfqpoint{2.647342in}{2.557419in}}%
\pgfpathlineto{\pgfqpoint{2.689953in}{2.610890in}}%
\pgfpathlineto{\pgfqpoint{2.732469in}{2.664470in}}%
\pgfpathlineto{\pgfqpoint{2.774944in}{2.718087in}}%
\pgfpathlineto{\pgfqpoint{2.817435in}{2.771684in}}%
\pgfpathlineto{\pgfqpoint{2.860015in}{2.825231in}}%
\pgfpathlineto{\pgfqpoint{2.902739in}{2.878666in}}%
\pgfpathlineto{\pgfqpoint{2.945663in}{2.931936in}}%
\pgfpathlineto{\pgfqpoint{2.988855in}{2.984997in}}%
\pgfpathlineto{\pgfqpoint{3.032377in}{3.037789in}}%
\pgfpathlineto{\pgfqpoint{3.076290in}{3.090258in}}%
\pgfpathlineto{\pgfqpoint{3.120666in}{3.142338in}}%
\pgfpathlineto{\pgfqpoint{3.165573in}{3.193962in}}%
\pgfpathlineto{\pgfqpoint{3.211085in}{3.245054in}}%
\pgfpathlineto{\pgfqpoint{3.257279in}{3.295530in}}%
\pgfpathlineto{\pgfqpoint{3.304236in}{3.345299in}}%
\pgfpathlineto{\pgfqpoint{3.352038in}{3.394254in}}%
\pgfpathlineto{\pgfqpoint{3.400776in}{3.442279in}}%
\pgfpathlineto{\pgfqpoint{3.450540in}{3.489238in}}%
\pgfpathlineto{\pgfqpoint{3.501421in}{3.534983in}}%
\pgfpathlineto{\pgfqpoint{3.543360in}{3.571741in}}%
\pgfusepath{stroke}%
\end{pgfscope}%
\begin{pgfscope}%
\pgfpathrectangle{\pgfqpoint{0.647939in}{0.492442in}}{\pgfqpoint{3.079299in}{3.079299in}}%
\pgfusepath{clip}%
\pgfsetbuttcap%
\pgfsetroundjoin%
\pgfsetlinewidth{0.301125pt}%
\definecolor{currentstroke}{rgb}{0.500000,0.500000,0.500000}%
\pgfsetstrokecolor{currentstroke}%
\pgfsetstrokeopacity{0.300000}%
\pgfsetdash{}{0pt}%
\pgfpathmoveto{\pgfqpoint{0.927875in}{0.492442in}}%
\pgfpathlineto{\pgfqpoint{0.927875in}{0.492442in}}%
\pgfpathlineto{\pgfqpoint{0.988139in}{0.524760in}}%
\pgfpathlineto{\pgfqpoint{1.045892in}{0.561354in}}%
\pgfpathlineto{\pgfqpoint{1.100777in}{0.602102in}}%
\pgfpathlineto{\pgfqpoint{1.152544in}{0.646737in}}%
\pgfpathlineto{\pgfqpoint{1.201091in}{0.694868in}}%
\pgfpathlineto{\pgfqpoint{1.246482in}{0.745985in}}%
\pgfpathlineto{\pgfqpoint{1.288929in}{0.799572in}}%
\pgfpathlineto{\pgfqpoint{1.328762in}{0.855148in}}%
\pgfusepath{stroke}%
\end{pgfscope}%
\begin{pgfscope}%
\pgfpathrectangle{\pgfqpoint{0.647939in}{0.492442in}}{\pgfqpoint{3.079299in}{3.079299in}}%
\pgfusepath{clip}%
\pgfsetbuttcap%
\pgfsetroundjoin%
\pgfsetlinewidth{0.301125pt}%
\definecolor{currentstroke}{rgb}{0.500000,0.500000,0.500000}%
\pgfsetstrokecolor{currentstroke}%
\pgfsetstrokeopacity{0.300000}%
\pgfsetdash{}{0pt}%
\pgfpathmoveto{\pgfqpoint{1.137828in}{0.492442in}}%
\pgfpathlineto{\pgfqpoint{1.137828in}{0.492442in}}%
\pgfpathlineto{\pgfqpoint{1.183115in}{0.543622in}}%
\pgfpathlineto{\pgfqpoint{1.224907in}{0.597696in}}%
\pgfpathlineto{\pgfqpoint{1.263563in}{0.654067in}}%
\pgfpathlineto{\pgfqpoint{1.299549in}{0.712219in}}%
\pgfpathlineto{\pgfqpoint{1.333373in}{0.771672in}}%
\pgfusepath{stroke}%
\end{pgfscope}%
\begin{pgfscope}%
\pgfpathrectangle{\pgfqpoint{0.647939in}{0.492442in}}{\pgfqpoint{3.079299in}{3.079299in}}%
\pgfusepath{clip}%
\pgfsetbuttcap%
\pgfsetroundjoin%
\pgfsetlinewidth{0.301125pt}%
\definecolor{currentstroke}{rgb}{0.500000,0.500000,0.500000}%
\pgfsetstrokecolor{currentstroke}%
\pgfsetstrokeopacity{0.300000}%
\pgfsetdash{}{0pt}%
\pgfpathmoveto{\pgfqpoint{1.347780in}{0.492442in}}%
\pgfpathlineto{\pgfqpoint{1.347780in}{0.492442in}}%
\pgfpathlineto{\pgfqpoint{1.356229in}{0.560163in}}%
\pgfpathlineto{\pgfqpoint{1.368353in}{0.627397in}}%
\pgfpathlineto{\pgfqpoint{1.383333in}{0.694037in}}%
\pgfpathlineto{\pgfqpoint{1.400580in}{0.760140in}}%
\pgfpathlineto{\pgfqpoint{1.419707in}{0.825742in}}%
\pgfpathlineto{\pgfqpoint{1.440448in}{0.890845in}}%
\pgfpathlineto{\pgfqpoint{1.462625in}{0.955483in}}%
\pgfpathlineto{\pgfqpoint{1.486133in}{1.019691in}}%
\pgfpathlineto{\pgfqpoint{1.510925in}{1.083417in}}%
\pgfpathlineto{\pgfqpoint{1.536974in}{1.146607in}}%
\pgfpathlineto{\pgfqpoint{1.564289in}{1.209251in}}%
\pgfpathlineto{\pgfqpoint{1.592908in}{1.271313in}}%
\pgfpathlineto{\pgfqpoint{1.622893in}{1.332734in}}%
\pgfusepath{stroke}%
\end{pgfscope}%
\begin{pgfscope}%
\pgfpathrectangle{\pgfqpoint{0.647939in}{0.492442in}}{\pgfqpoint{3.079299in}{3.079299in}}%
\pgfusepath{clip}%
\pgfsetbuttcap%
\pgfsetroundjoin%
\pgfsetlinewidth{0.301125pt}%
\definecolor{currentstroke}{rgb}{0.500000,0.500000,0.500000}%
\pgfsetstrokecolor{currentstroke}%
\pgfsetstrokeopacity{0.300000}%
\pgfsetdash{}{0pt}%
\pgfpathmoveto{\pgfqpoint{1.557732in}{0.492442in}}%
\pgfpathlineto{\pgfqpoint{1.557732in}{0.492442in}}%
\pgfpathlineto{\pgfqpoint{1.515783in}{0.545760in}}%
\pgfpathlineto{\pgfqpoint{1.488625in}{0.600993in}}%
\pgfpathlineto{\pgfqpoint{1.472190in}{0.659795in}}%
\pgfpathlineto{\pgfqpoint{1.464736in}{0.723972in}}%
\pgfpathlineto{\pgfqpoint{1.465535in}{0.792119in}}%
\pgfpathlineto{\pgfqpoint{1.472969in}{0.859909in}}%
\pgfpathlineto{\pgfqpoint{1.485374in}{0.927042in}}%
\pgfusepath{stroke}%
\end{pgfscope}%
\begin{pgfscope}%
\pgfpathrectangle{\pgfqpoint{0.647939in}{0.492442in}}{\pgfqpoint{3.079299in}{3.079299in}}%
\pgfusepath{clip}%
\pgfsetbuttcap%
\pgfsetroundjoin%
\pgfsetlinewidth{0.301125pt}%
\definecolor{currentstroke}{rgb}{0.500000,0.500000,0.500000}%
\pgfsetstrokecolor{currentstroke}%
\pgfsetstrokeopacity{0.300000}%
\pgfsetdash{}{0pt}%
\pgfpathmoveto{\pgfqpoint{1.767684in}{0.492442in}}%
\pgfpathlineto{\pgfqpoint{1.767684in}{0.492442in}}%
\pgfpathlineto{\pgfqpoint{1.703844in}{0.516727in}}%
\pgfpathlineto{\pgfqpoint{1.644030in}{0.549490in}}%
\pgfpathlineto{\pgfqpoint{1.591207in}{0.592350in}}%
\pgfpathlineto{\pgfqpoint{1.550207in}{0.643652in}}%
\pgfpathlineto{\pgfqpoint{1.523364in}{0.697743in}}%
\pgfusepath{stroke}%
\end{pgfscope}%
\begin{pgfscope}%
\pgfpathrectangle{\pgfqpoint{0.647939in}{0.492442in}}{\pgfqpoint{3.079299in}{3.079299in}}%
\pgfusepath{clip}%
\pgfsetbuttcap%
\pgfsetroundjoin%
\pgfsetlinewidth{0.301125pt}%
\definecolor{currentstroke}{rgb}{0.500000,0.500000,0.500000}%
\pgfsetstrokecolor{currentstroke}%
\pgfsetstrokeopacity{0.300000}%
\pgfsetdash{}{0pt}%
\pgfpathmoveto{\pgfqpoint{2.047620in}{0.492442in}}%
\pgfpathlineto{\pgfqpoint{2.047620in}{0.492442in}}%
\pgfpathlineto{\pgfqpoint{1.979496in}{0.498792in}}%
\pgfpathlineto{\pgfqpoint{1.911669in}{0.507692in}}%
\pgfpathlineto{\pgfqpoint{1.844443in}{0.520247in}}%
\pgfpathlineto{\pgfqpoint{1.778393in}{0.537828in}}%
\pgfpathlineto{\pgfqpoint{1.714579in}{0.562131in}}%
\pgfusepath{stroke}%
\end{pgfscope}%
\begin{pgfscope}%
\pgfpathrectangle{\pgfqpoint{0.647939in}{0.492442in}}{\pgfqpoint{3.079299in}{3.079299in}}%
\pgfusepath{clip}%
\pgfsetbuttcap%
\pgfsetroundjoin%
\pgfsetlinewidth{0.301125pt}%
\definecolor{currentstroke}{rgb}{0.500000,0.500000,0.500000}%
\pgfsetstrokecolor{currentstroke}%
\pgfsetstrokeopacity{0.300000}%
\pgfsetdash{}{0pt}%
\pgfpathmoveto{\pgfqpoint{2.467525in}{0.492442in}}%
\pgfpathlineto{\pgfqpoint{2.467525in}{0.492442in}}%
\pgfpathlineto{\pgfqpoint{2.399390in}{0.498730in}}%
\pgfpathlineto{\pgfqpoint{2.331148in}{0.503750in}}%
\pgfpathlineto{\pgfqpoint{2.262845in}{0.507883in}}%
\pgfpathlineto{\pgfqpoint{2.194518in}{0.511603in}}%
\pgfpathlineto{\pgfqpoint{2.126198in}{0.515464in}}%
\pgfusepath{stroke}%
\end{pgfscope}%
\begin{pgfscope}%
\pgfpathrectangle{\pgfqpoint{0.647939in}{0.492442in}}{\pgfqpoint{3.079299in}{3.079299in}}%
\pgfusepath{clip}%
\pgfsetbuttcap%
\pgfsetroundjoin%
\pgfsetlinewidth{0.301125pt}%
\definecolor{currentstroke}{rgb}{0.500000,0.500000,0.500000}%
\pgfsetstrokecolor{currentstroke}%
\pgfsetstrokeopacity{0.300000}%
\pgfsetdash{}{0pt}%
\pgfpathmoveto{\pgfqpoint{2.747461in}{0.492442in}}%
\pgfpathlineto{\pgfqpoint{2.747461in}{0.492442in}}%
\pgfpathlineto{\pgfqpoint{2.680492in}{0.506456in}}%
\pgfpathlineto{\pgfqpoint{2.613143in}{0.518519in}}%
\pgfpathlineto{\pgfqpoint{2.545476in}{0.528650in}}%
\pgfpathlineto{\pgfqpoint{2.477559in}{0.536955in}}%
\pgfpathlineto{\pgfqpoint{2.409461in}{0.543635in}}%
\pgfpathlineto{\pgfqpoint{2.341244in}{0.548974in}}%
\pgfpathlineto{\pgfqpoint{2.272956in}{0.553345in}}%
\pgfpathlineto{\pgfqpoint{2.204637in}{0.557211in}}%
\pgfpathlineto{\pgfqpoint{2.136321in}{0.561132in}}%
\pgfpathlineto{\pgfqpoint{2.068052in}{0.565773in}}%
\pgfpathlineto{\pgfqpoint{1.999908in}{0.571913in}}%
\pgfpathlineto{\pgfqpoint{1.932036in}{0.580482in}}%
\pgfpathlineto{\pgfqpoint{1.864732in}{0.592627in}}%
\pgfpathlineto{\pgfqpoint{1.798563in}{0.609793in}}%
\pgfpathlineto{\pgfqpoint{1.734626in}{0.633788in}}%
\pgfpathlineto{\pgfqpoint{1.674938in}{0.666712in}}%
\pgfpathlineto{\pgfqpoint{1.622846in}{0.710351in}}%
\pgfpathlineto{\pgfqpoint{1.584632in}{0.760763in}}%
\pgfpathlineto{\pgfqpoint{1.560237in}{0.813799in}}%
\pgfpathlineto{\pgfqpoint{1.546321in}{0.870407in}}%
\pgfpathlineto{\pgfqpoint{1.541286in}{0.932723in}}%
\pgfpathlineto{\pgfqpoint{1.544785in}{1.000735in}}%
\pgfpathlineto{\pgfqpoint{1.555157in}{1.068190in}}%
\pgfusepath{stroke}%
\end{pgfscope}%
\begin{pgfscope}%
\pgfpathrectangle{\pgfqpoint{0.647939in}{0.492442in}}{\pgfqpoint{3.079299in}{3.079299in}}%
\pgfusepath{clip}%
\pgfsetbuttcap%
\pgfsetroundjoin%
\pgfsetlinewidth{0.301125pt}%
\definecolor{currentstroke}{rgb}{0.500000,0.500000,0.500000}%
\pgfsetstrokecolor{currentstroke}%
\pgfsetstrokeopacity{0.300000}%
\pgfsetdash{}{0pt}%
\pgfpathmoveto{\pgfqpoint{2.957413in}{0.492442in}}%
\pgfpathlineto{\pgfqpoint{2.957413in}{0.492442in}}%
\pgfpathlineto{\pgfqpoint{2.891999in}{0.512511in}}%
\pgfpathlineto{\pgfqpoint{2.826122in}{0.531001in}}%
\pgfpathlineto{\pgfqpoint{2.759775in}{0.547724in}}%
\pgfpathlineto{\pgfqpoint{2.692978in}{0.562540in}}%
\pgfpathlineto{\pgfqpoint{2.625773in}{0.575375in}}%
\pgfusepath{stroke}%
\end{pgfscope}%
\begin{pgfscope}%
\pgfpathrectangle{\pgfqpoint{0.647939in}{0.492442in}}{\pgfqpoint{3.079299in}{3.079299in}}%
\pgfusepath{clip}%
\pgfsetbuttcap%
\pgfsetroundjoin%
\pgfsetlinewidth{0.301125pt}%
\definecolor{currentstroke}{rgb}{0.500000,0.500000,0.500000}%
\pgfsetstrokecolor{currentstroke}%
\pgfsetstrokeopacity{0.300000}%
\pgfsetdash{}{0pt}%
\pgfpathmoveto{\pgfqpoint{3.167366in}{0.492442in}}%
\pgfpathlineto{\pgfqpoint{3.167366in}{0.492442in}}%
\pgfpathlineto{\pgfqpoint{3.103422in}{0.516806in}}%
\pgfpathlineto{\pgfqpoint{3.039183in}{0.540375in}}%
\pgfpathlineto{\pgfqpoint{2.974568in}{0.562892in}}%
\pgfpathlineto{\pgfqpoint{2.909516in}{0.584104in}}%
\pgfpathlineto{\pgfqpoint{2.843983in}{0.603777in}}%
\pgfpathlineto{\pgfqpoint{2.777952in}{0.621704in}}%
\pgfpathlineto{\pgfqpoint{2.711434in}{0.637718in}}%
\pgfpathlineto{\pgfqpoint{2.644461in}{0.651713in}}%
\pgfpathlineto{\pgfqpoint{2.577090in}{0.663651in}}%
\pgfpathlineto{\pgfqpoint{2.509394in}{0.673587in}}%
\pgfpathlineto{\pgfqpoint{2.441451in}{0.681677in}}%
\pgfpathlineto{\pgfqpoint{2.373336in}{0.688172in}}%
\pgfpathlineto{\pgfqpoint{2.305112in}{0.693426in}}%
\pgfpathlineto{\pgfqpoint{2.236829in}{0.697893in}}%
\pgfpathlineto{\pgfqpoint{2.168533in}{0.702140in}}%
\pgfpathlineto{\pgfqpoint{2.100268in}{0.706858in}}%
\pgfpathlineto{\pgfqpoint{2.032111in}{0.712872in}}%
\pgfpathlineto{\pgfqpoint{1.964207in}{0.721187in}}%
\pgfpathlineto{\pgfqpoint{1.896856in}{0.733064in}}%
\pgfpathlineto{\pgfqpoint{1.830668in}{0.750132in}}%
\pgfpathlineto{\pgfqpoint{1.766893in}{0.774493in}}%
\pgfpathlineto{\pgfqpoint{1.707936in}{0.808611in}}%
\pgfpathlineto{\pgfqpoint{1.657917in}{0.854381in}}%
\pgfpathlineto{\pgfqpoint{1.624793in}{0.903815in}}%
\pgfpathlineto{\pgfqpoint{1.604623in}{0.955710in}}%
\pgfpathlineto{\pgfqpoint{1.594290in}{1.011683in}}%
\pgfpathlineto{\pgfqpoint{1.592763in}{1.073558in}}%
\pgfpathlineto{\pgfqpoint{1.599879in}{1.141263in}}%
\pgfpathlineto{\pgfqpoint{1.613707in}{1.208084in}}%
\pgfusepath{stroke}%
\end{pgfscope}%
\begin{pgfscope}%
\pgfpathrectangle{\pgfqpoint{0.647939in}{0.492442in}}{\pgfqpoint{3.079299in}{3.079299in}}%
\pgfusepath{clip}%
\pgfsetbuttcap%
\pgfsetroundjoin%
\pgfsetlinewidth{0.301125pt}%
\definecolor{currentstroke}{rgb}{0.500000,0.500000,0.500000}%
\pgfsetstrokecolor{currentstroke}%
\pgfsetstrokeopacity{0.300000}%
\pgfsetdash{}{0pt}%
\pgfpathmoveto{\pgfqpoint{3.377318in}{0.492442in}}%
\pgfpathlineto{\pgfqpoint{3.377318in}{0.492442in}}%
\pgfpathlineto{\pgfqpoint{3.314027in}{0.518456in}}%
\pgfpathlineto{\pgfqpoint{3.250772in}{0.544558in}}%
\pgfpathlineto{\pgfqpoint{3.187449in}{0.570494in}}%
\pgfpathlineto{\pgfqpoint{3.123954in}{0.596003in}}%
\pgfpathlineto{\pgfqpoint{3.060187in}{0.620821in}}%
\pgfpathlineto{\pgfqpoint{2.996057in}{0.644684in}}%
\pgfpathlineto{\pgfqpoint{2.931489in}{0.667329in}}%
\pgfpathlineto{\pgfqpoint{2.866426in}{0.688503in}}%
\pgfpathlineto{\pgfqpoint{2.800835in}{0.707976in}}%
\pgfpathlineto{\pgfqpoint{2.734710in}{0.725550in}}%
\pgfpathlineto{\pgfqpoint{2.668075in}{0.741078in}}%
\pgfpathlineto{\pgfqpoint{2.600982in}{0.754484in}}%
\pgfpathlineto{\pgfqpoint{2.533502in}{0.765782in}}%
\pgfpathlineto{\pgfqpoint{2.465716in}{0.775078in}}%
\pgfpathlineto{\pgfqpoint{2.397707in}{0.782590in}}%
\pgfpathlineto{\pgfqpoint{2.329549in}{0.788642in}}%
\pgfpathlineto{\pgfqpoint{2.261308in}{0.793684in}}%
\pgfpathlineto{\pgfqpoint{2.193034in}{0.798288in}}%
\pgfpathlineto{\pgfqpoint{2.124779in}{0.803146in}}%
\pgfpathlineto{\pgfqpoint{2.056615in}{0.809102in}}%
\pgfpathlineto{\pgfqpoint{1.988683in}{0.817207in}}%
\pgfpathlineto{\pgfqpoint{1.921282in}{0.828807in}}%
\pgfpathlineto{\pgfqpoint{1.855055in}{0.845679in}}%
\pgfpathlineto{\pgfqpoint{1.791358in}{0.870144in}}%
\pgfpathlineto{\pgfqpoint{1.732917in}{0.904948in}}%
\pgfpathlineto{\pgfqpoint{1.685129in}{0.951107in}}%
\pgfusepath{stroke}%
\end{pgfscope}%
\begin{pgfscope}%
\pgfpathrectangle{\pgfqpoint{0.647939in}{0.492442in}}{\pgfqpoint{3.079299in}{3.079299in}}%
\pgfusepath{clip}%
\pgfsetbuttcap%
\pgfsetroundjoin%
\pgfsetlinewidth{0.301125pt}%
\definecolor{currentstroke}{rgb}{0.500000,0.500000,0.500000}%
\pgfsetstrokecolor{currentstroke}%
\pgfsetstrokeopacity{0.300000}%
\pgfsetdash{}{0pt}%
\pgfpathmoveto{\pgfqpoint{3.517286in}{0.492442in}}%
\pgfpathlineto{\pgfqpoint{3.517286in}{0.492442in}}%
\pgfpathlineto{\pgfqpoint{3.453799in}{0.517971in}}%
\pgfpathlineto{\pgfqpoint{3.390580in}{0.544157in}}%
\pgfpathlineto{\pgfqpoint{3.327533in}{0.570757in}}%
\pgfpathlineto{\pgfqpoint{3.264557in}{0.597524in}}%
\pgfpathlineto{\pgfqpoint{3.201544in}{0.624204in}}%
\pgfpathlineto{\pgfqpoint{3.138385in}{0.650535in}}%
\pgfpathlineto{\pgfqpoint{3.074975in}{0.676255in}}%
\pgfpathlineto{\pgfqpoint{3.011216in}{0.701094in}}%
\pgfpathlineto{\pgfqpoint{2.947022in}{0.724783in}}%
\pgfpathlineto{\pgfqpoint{2.882327in}{0.747059in}}%
\pgfpathlineto{\pgfqpoint{2.817085in}{0.767676in}}%
\pgfpathlineto{\pgfqpoint{2.751283in}{0.786417in}}%
\pgfusepath{stroke}%
\end{pgfscope}%
\begin{pgfscope}%
\pgfpathrectangle{\pgfqpoint{0.647939in}{0.492442in}}{\pgfqpoint{3.079299in}{3.079299in}}%
\pgfusepath{clip}%
\pgfsetbuttcap%
\pgfsetroundjoin%
\pgfsetlinewidth{0.301125pt}%
\definecolor{currentstroke}{rgb}{0.500000,0.500000,0.500000}%
\pgfsetstrokecolor{currentstroke}%
\pgfsetstrokeopacity{0.300000}%
\pgfsetdash{}{0pt}%
\pgfpathmoveto{\pgfqpoint{3.727238in}{0.492442in}}%
\pgfpathlineto{\pgfqpoint{3.727238in}{0.492442in}}%
\pgfpathlineto{\pgfqpoint{3.662624in}{0.514956in}}%
\pgfpathlineto{\pgfqpoint{3.598525in}{0.538898in}}%
\pgfpathlineto{\pgfqpoint{3.534893in}{0.564059in}}%
\pgfpathlineto{\pgfqpoint{3.471663in}{0.590216in}}%
\pgfpathlineto{\pgfqpoint{3.408752in}{0.617135in}}%
\pgfpathlineto{\pgfqpoint{3.346068in}{0.644578in}}%
\pgfpathlineto{\pgfqpoint{3.283505in}{0.672297in}}%
\pgfpathlineto{\pgfqpoint{3.220953in}{0.700041in}}%
\pgfpathlineto{\pgfqpoint{3.158299in}{0.727553in}}%
\pgfpathlineto{\pgfqpoint{3.095430in}{0.754568in}}%
\pgfpathlineto{\pgfqpoint{3.032239in}{0.780816in}}%
\pgfpathlineto{\pgfqpoint{2.968627in}{0.806023in}}%
\pgfpathlineto{\pgfqpoint{2.904512in}{0.829918in}}%
\pgfpathlineto{\pgfqpoint{2.839833in}{0.852236in}}%
\pgfpathlineto{\pgfqpoint{2.774555in}{0.872731in}}%
\pgfpathlineto{\pgfqpoint{2.708676in}{0.891196in}}%
\pgfpathlineto{\pgfqpoint{2.642224in}{0.907476in}}%
\pgfpathlineto{\pgfqpoint{2.575257in}{0.921494in}}%
\pgfpathlineto{\pgfqpoint{2.507860in}{0.933277in}}%
\pgfpathlineto{\pgfqpoint{2.440129in}{0.942968in}}%
\pgfpathlineto{\pgfqpoint{2.372160in}{0.950831in}}%
\pgfpathlineto{\pgfqpoint{2.304038in}{0.957267in}}%
\pgfpathlineto{\pgfqpoint{2.235835in}{0.962815in}}%
\pgfpathlineto{\pgfqpoint{2.167617in}{0.968183in}}%
\pgfpathlineto{\pgfqpoint{2.099463in}{0.974272in}}%
\pgfpathlineto{\pgfqpoint{2.031511in}{0.982217in}}%
\pgfpathlineto{\pgfqpoint{1.964061in}{0.993515in}}%
\pgfpathlineto{\pgfqpoint{1.897799in}{1.010210in}}%
\pgfpathlineto{\pgfqpoint{1.834310in}{1.035132in}}%
\pgfpathlineto{\pgfqpoint{1.777129in}{1.071728in}}%
\pgfpathlineto{\pgfqpoint{1.735624in}{1.116847in}}%
\pgfpathlineto{\pgfqpoint{1.711136in}{1.163606in}}%
\pgfpathlineto{\pgfqpoint{1.698346in}{1.212504in}}%
\pgfpathlineto{\pgfqpoint{1.694726in}{1.265938in}}%
\pgfpathlineto{\pgfqpoint{1.700015in}{1.325622in}}%
\pgfpathlineto{\pgfqpoint{1.714713in}{1.392152in}}%
\pgfpathlineto{\pgfqpoint{1.736088in}{1.456935in}}%
\pgfpathlineto{\pgfqpoint{1.762619in}{1.519881in}}%
\pgfusepath{stroke}%
\end{pgfscope}%
\begin{pgfscope}%
\pgfpathrectangle{\pgfqpoint{0.647939in}{0.492442in}}{\pgfqpoint{3.079299in}{3.079299in}}%
\pgfusepath{clip}%
\pgfsetbuttcap%
\pgfsetroundjoin%
\pgfsetlinewidth{0.301125pt}%
\definecolor{currentstroke}{rgb}{0.500000,0.500000,0.500000}%
\pgfsetstrokecolor{currentstroke}%
\pgfsetstrokeopacity{0.300000}%
\pgfsetdash{}{0pt}%
\pgfpathmoveto{\pgfqpoint{3.727238in}{0.562426in}}%
\pgfpathlineto{\pgfqpoint{3.727238in}{0.562426in}}%
\pgfpathlineto{\pgfqpoint{3.662868in}{0.585625in}}%
\pgfpathlineto{\pgfqpoint{3.599047in}{0.610299in}}%
\pgfpathlineto{\pgfqpoint{3.535728in}{0.636238in}}%
\pgfpathlineto{\pgfqpoint{3.472844in}{0.663216in}}%
\pgfpathlineto{\pgfqpoint{3.410312in}{0.691002in}}%
\pgfpathlineto{\pgfqpoint{3.348035in}{0.719357in}}%
\pgfpathlineto{\pgfqpoint{3.285906in}{0.748036in}}%
\pgfpathlineto{\pgfqpoint{3.223812in}{0.776790in}}%
\pgfpathlineto{\pgfqpoint{3.161634in}{0.805362in}}%
\pgfpathlineto{\pgfqpoint{3.099254in}{0.833488in}}%
\pgfpathlineto{\pgfqpoint{3.036558in}{0.860899in}}%
\pgfpathlineto{\pgfqpoint{2.973438in}{0.887317in}}%
\pgfpathlineto{\pgfqpoint{2.909803in}{0.912463in}}%
\pgfpathlineto{\pgfqpoint{2.845581in}{0.936062in}}%
\pgfpathlineto{\pgfqpoint{2.780725in}{0.957854in}}%
\pgfpathlineto{\pgfqpoint{2.715221in}{0.977607in}}%
\pgfpathlineto{\pgfqpoint{2.649090in}{0.995141in}}%
\pgfpathlineto{\pgfqpoint{2.582386in}{1.010350in}}%
\pgfpathlineto{\pgfqpoint{2.515190in}{1.023225in}}%
\pgfpathlineto{\pgfqpoint{2.447606in}{1.033882in}}%
\pgfpathlineto{\pgfqpoint{2.379739in}{1.042582in}}%
\pgfpathlineto{\pgfqpoint{2.311688in}{1.049728in}}%
\pgfpathlineto{\pgfqpoint{2.243537in}{1.055880in}}%
\pgfpathlineto{\pgfqpoint{2.175363in}{1.061779in}}%
\pgfpathlineto{\pgfqpoint{2.107258in}{1.068391in}}%
\pgfpathlineto{\pgfqpoint{2.039390in}{1.077003in}}%
\pgfpathlineto{\pgfqpoint{1.972136in}{1.089350in}}%
\pgfpathlineto{\pgfqpoint{1.906409in}{1.107880in}}%
\pgfpathlineto{\pgfqpoint{1.844440in}{1.136036in}}%
\pgfpathlineto{\pgfqpoint{1.844440in}{1.136036in}}%
\pgfpathlineto{\pgfqpoint{1.799553in}{1.168951in}}%
\pgfpathlineto{\pgfqpoint{1.764062in}{1.212874in}}%
\pgfpathlineto{\pgfqpoint{1.744052in}{1.258304in}}%
\pgfpathlineto{\pgfqpoint{1.734843in}{1.306296in}}%
\pgfpathlineto{\pgfqpoint{1.734632in}{1.359146in}}%
\pgfpathlineto{\pgfqpoint{1.743511in}{1.418198in}}%
\pgfusepath{stroke}%
\end{pgfscope}%
\begin{pgfscope}%
\pgfpathrectangle{\pgfqpoint{0.647939in}{0.492442in}}{\pgfqpoint{3.079299in}{3.079299in}}%
\pgfusepath{clip}%
\pgfsetbuttcap%
\pgfsetroundjoin%
\pgfsetlinewidth{0.301125pt}%
\definecolor{currentstroke}{rgb}{0.500000,0.500000,0.500000}%
\pgfsetstrokecolor{currentstroke}%
\pgfsetstrokeopacity{0.300000}%
\pgfsetdash{}{0pt}%
\pgfpathmoveto{\pgfqpoint{3.727238in}{0.632410in}}%
\pgfpathlineto{\pgfqpoint{3.727238in}{0.632410in}}%
\pgfpathlineto{\pgfqpoint{3.663134in}{0.656335in}}%
\pgfpathlineto{\pgfqpoint{3.599619in}{0.681784in}}%
\pgfpathlineto{\pgfqpoint{3.536643in}{0.708546in}}%
\pgfpathlineto{\pgfqpoint{3.474141in}{0.736396in}}%
\pgfpathlineto{\pgfqpoint{3.412026in}{0.765103in}}%
\pgfpathlineto{\pgfqpoint{3.350201in}{0.794429in}}%
\pgfpathlineto{\pgfqpoint{3.288556in}{0.824133in}}%
\pgfpathlineto{\pgfqpoint{3.226974in}{0.853968in}}%
\pgfpathlineto{\pgfqpoint{3.165334in}{0.883681in}}%
\pgfpathlineto{\pgfqpoint{3.103511in}{0.913011in}}%
\pgfpathlineto{\pgfqpoint{3.041383in}{0.941689in}}%
\pgfpathlineto{\pgfqpoint{2.978836in}{0.969437in}}%
\pgfpathlineto{\pgfqpoint{2.915766in}{0.995971in}}%
\pgfpathlineto{\pgfqpoint{2.852089in}{1.021006in}}%
\pgfpathlineto{\pgfqpoint{2.787744in}{1.044266in}}%
\pgfpathlineto{\pgfqpoint{2.722704in}{1.065498in}}%
\pgfpathlineto{\pgfqpoint{2.656978in}{1.084493in}}%
\pgfpathlineto{\pgfqpoint{2.590612in}{1.101110in}}%
\pgfpathlineto{\pgfqpoint{2.523686in}{1.115305in}}%
\pgfpathlineto{\pgfqpoint{2.456303in}{1.127158in}}%
\pgfpathlineto{\pgfqpoint{2.388578in}{1.136895in}}%
\pgfpathlineto{\pgfqpoint{2.320626in}{1.144921in}}%
\pgfpathlineto{\pgfqpoint{2.252548in}{1.151827in}}%
\pgfpathlineto{\pgfqpoint{2.184436in}{1.158401in}}%
\pgfpathlineto{\pgfqpoint{2.116401in}{1.165698in}}%
\pgfpathlineto{\pgfqpoint{2.048646in}{1.175152in}}%
\pgfpathlineto{\pgfqpoint{1.981659in}{1.188820in}}%
\pgfpathlineto{\pgfqpoint{1.916710in}{1.209743in}}%
\pgfpathlineto{\pgfqpoint{1.857109in}{1.242223in}}%
\pgfpathlineto{\pgfqpoint{1.857109in}{1.242223in}}%
\pgfpathlineto{\pgfqpoint{1.820231in}{1.275887in}}%
\pgfusepath{stroke}%
\end{pgfscope}%
\begin{pgfscope}%
\pgfpathrectangle{\pgfqpoint{0.647939in}{0.492442in}}{\pgfqpoint{3.079299in}{3.079299in}}%
\pgfusepath{clip}%
\pgfsetbuttcap%
\pgfsetroundjoin%
\pgfsetlinewidth{0.301125pt}%
\definecolor{currentstroke}{rgb}{0.500000,0.500000,0.500000}%
\pgfsetstrokecolor{currentstroke}%
\pgfsetstrokeopacity{0.300000}%
\pgfsetdash{}{0pt}%
\pgfpathmoveto{\pgfqpoint{3.727238in}{0.702394in}}%
\pgfpathlineto{\pgfqpoint{3.727238in}{0.702394in}}%
\pgfpathlineto{\pgfqpoint{3.663427in}{0.727087in}}%
\pgfpathlineto{\pgfqpoint{3.600247in}{0.753358in}}%
\pgfpathlineto{\pgfqpoint{3.537651in}{0.780993in}}%
\pgfpathlineto{\pgfqpoint{3.475569in}{0.809769in}}%
\pgfpathlineto{\pgfqpoint{3.413917in}{0.839455in}}%
\pgfpathlineto{\pgfqpoint{3.352594in}{0.869817in}}%
\pgfpathlineto{\pgfqpoint{3.291489in}{0.900617in}}%
\pgfpathlineto{\pgfqpoint{3.230483in}{0.931613in}}%
\pgfpathlineto{\pgfqpoint{3.169449in}{0.962554in}}%
\pgfpathlineto{\pgfqpoint{3.108260in}{0.993186in}}%
\pgfpathlineto{\pgfqpoint{3.046788in}{1.023242in}}%
\pgfpathlineto{\pgfqpoint{2.984908in}{1.052450in}}%
\pgfpathlineto{\pgfqpoint{2.922508in}{1.080523in}}%
\pgfpathlineto{\pgfqpoint{2.859488in}{1.107172in}}%
\pgfpathlineto{\pgfqpoint{2.795772in}{1.132105in}}%
\pgfpathlineto{\pgfqpoint{2.731315in}{1.155048in}}%
\pgfpathlineto{\pgfqpoint{2.666107in}{1.175758in}}%
\pgfpathlineto{\pgfqpoint{2.600185in}{1.194058in}}%
\pgfpathlineto{\pgfqpoint{2.533621in}{1.209859in}}%
\pgfpathlineto{\pgfqpoint{2.466519in}{1.223201in}}%
\pgfpathlineto{\pgfqpoint{2.399003in}{1.234275in}}%
\pgfpathlineto{\pgfqpoint{2.331199in}{1.243456in}}%
\pgfpathlineto{\pgfqpoint{2.263227in}{1.251336in}}%
\pgfpathlineto{\pgfqpoint{2.195204in}{1.258775in}}%
\pgfpathlineto{\pgfqpoint{2.127271in}{1.266956in}}%
\pgfpathlineto{\pgfqpoint{2.059692in}{1.277542in}}%
\pgfpathlineto{\pgfqpoint{1.993131in}{1.293022in}}%
\pgfpathlineto{\pgfqpoint{1.929465in}{1.317293in}}%
\pgfpathlineto{\pgfqpoint{1.929465in}{1.317293in}}%
\pgfpathlineto{\pgfqpoint{1.884092in}{1.346235in}}%
\pgfpathlineto{\pgfqpoint{1.884092in}{1.346235in}}%
\pgfpathlineto{\pgfqpoint{1.852716in}{1.379767in}}%
\pgfpathlineto{\pgfqpoint{1.831831in}{1.421177in}}%
\pgfpathlineto{\pgfqpoint{1.822978in}{1.463977in}}%
\pgfpathlineto{\pgfqpoint{1.823377in}{1.510001in}}%
\pgfpathlineto{\pgfqpoint{1.832840in}{1.561615in}}%
\pgfpathlineto{\pgfqpoint{1.852365in}{1.619527in}}%
\pgfpathlineto{\pgfqpoint{1.881615in}{1.680940in}}%
\pgfusepath{stroke}%
\end{pgfscope}%
\begin{pgfscope}%
\pgfpathrectangle{\pgfqpoint{0.647939in}{0.492442in}}{\pgfqpoint{3.079299in}{3.079299in}}%
\pgfusepath{clip}%
\pgfsetbuttcap%
\pgfsetroundjoin%
\pgfsetlinewidth{0.301125pt}%
\definecolor{currentstroke}{rgb}{0.500000,0.500000,0.500000}%
\pgfsetstrokecolor{currentstroke}%
\pgfsetstrokeopacity{0.300000}%
\pgfsetdash{}{0pt}%
\pgfpathmoveto{\pgfqpoint{3.727238in}{0.772378in}}%
\pgfpathlineto{\pgfqpoint{3.727238in}{0.772378in}}%
\pgfpathlineto{\pgfqpoint{3.663749in}{0.797888in}}%
\pgfpathlineto{\pgfqpoint{3.600939in}{0.825030in}}%
\pgfpathlineto{\pgfqpoint{3.538761in}{0.853593in}}%
\pgfpathlineto{\pgfqpoint{3.477146in}{0.883352in}}%
\pgfpathlineto{\pgfqpoint{3.416007in}{0.914081in}}%
\pgfpathlineto{\pgfqpoint{3.355244in}{0.945547in}}%
\pgfpathlineto{\pgfqpoint{3.294743in}{0.977517in}}%
\pgfpathlineto{\pgfqpoint{3.234384in}{1.009754in}}%
\pgfpathlineto{\pgfqpoint{3.174039in}{1.042017in}}%
\pgfpathlineto{\pgfqpoint{3.113577in}{1.074058in}}%
\pgfpathlineto{\pgfqpoint{3.052864in}{1.105622in}}%
\pgfpathlineto{\pgfqpoint{2.991771in}{1.136438in}}%
\pgfpathlineto{\pgfqpoint{2.930172in}{1.166227in}}%
\pgfpathlineto{\pgfqpoint{2.867955in}{1.194696in}}%
\pgfpathlineto{\pgfqpoint{2.805025in}{1.221548in}}%
\pgfpathlineto{\pgfqpoint{2.741316in}{1.246488in}}%
\pgfpathlineto{\pgfqpoint{2.676797in}{1.269245in}}%
\pgfpathlineto{\pgfqpoint{2.611479in}{1.289593in}}%
\pgfpathlineto{\pgfqpoint{2.545419in}{1.307387in}}%
\pgfpathlineto{\pgfqpoint{2.478718in}{1.322610in}}%
\pgfpathlineto{\pgfqpoint{2.411509in}{1.335412in}}%
\pgfpathlineto{\pgfqpoint{2.343935in}{1.346141in}}%
\pgfpathlineto{\pgfqpoint{2.276136in}{1.355387in}}%
\pgfpathlineto{\pgfqpoint{2.208257in}{1.364042in}}%
\pgfpathlineto{\pgfqpoint{2.140479in}{1.373422in}}%
\pgfpathlineto{\pgfqpoint{2.073165in}{1.385533in}}%
\pgfpathlineto{\pgfqpoint{2.007316in}{1.403589in}}%
\pgfpathlineto{\pgfqpoint{1.946116in}{1.432892in}}%
\pgfpathlineto{\pgfqpoint{1.946116in}{1.432892in}}%
\pgfpathlineto{\pgfqpoint{1.911026in}{1.462363in}}%
\pgfpathlineto{\pgfqpoint{1.886055in}{1.501902in}}%
\pgfpathlineto{\pgfqpoint{1.875504in}{1.541594in}}%
\pgfpathlineto{\pgfqpoint{1.874800in}{1.583297in}}%
\pgfpathlineto{\pgfqpoint{1.883057in}{1.629532in}}%
\pgfusepath{stroke}%
\end{pgfscope}%
\begin{pgfscope}%
\pgfpathrectangle{\pgfqpoint{0.647939in}{0.492442in}}{\pgfqpoint{3.079299in}{3.079299in}}%
\pgfusepath{clip}%
\pgfsetbuttcap%
\pgfsetroundjoin%
\pgfsetlinewidth{0.301125pt}%
\definecolor{currentstroke}{rgb}{0.500000,0.500000,0.500000}%
\pgfsetstrokecolor{currentstroke}%
\pgfsetstrokeopacity{0.300000}%
\pgfsetdash{}{0pt}%
\pgfpathmoveto{\pgfqpoint{3.727238in}{0.842362in}}%
\pgfpathlineto{\pgfqpoint{3.727238in}{0.842362in}}%
\pgfpathlineto{\pgfqpoint{3.664105in}{0.868739in}}%
\pgfpathlineto{\pgfqpoint{3.601704in}{0.896808in}}%
\pgfpathlineto{\pgfqpoint{3.539989in}{0.926357in}}%
\pgfpathlineto{\pgfqpoint{3.478890in}{0.957163in}}%
\pgfpathlineto{\pgfqpoint{3.418321in}{0.989002in}}%
\pgfpathlineto{\pgfqpoint{3.358182in}{1.021644in}}%
\pgfpathlineto{\pgfqpoint{3.298359in}{1.054866in}}%
\pgfpathlineto{\pgfqpoint{3.238732in}{1.088438in}}%
\pgfpathlineto{\pgfqpoint{3.179174in}{1.122131in}}%
\pgfpathlineto{\pgfqpoint{3.119551in}{1.155709in}}%
\pgfpathlineto{\pgfqpoint{3.059727in}{1.188925in}}%
\pgfpathlineto{\pgfqpoint{2.999566in}{1.221524in}}%
\pgfpathlineto{\pgfqpoint{2.938934in}{1.253235in}}%
\pgfpathlineto{\pgfqpoint{2.877704in}{1.283771in}}%
\pgfpathlineto{\pgfqpoint{2.815764in}{1.312834in}}%
\pgfpathlineto{\pgfqpoint{2.753024in}{1.340121in}}%
\pgfpathlineto{\pgfqpoint{2.689425in}{1.365336in}}%
\pgfpathlineto{\pgfqpoint{2.624951in}{1.388215in}}%
\pgfpathlineto{\pgfqpoint{2.559635in}{1.408557in}}%
\pgfpathlineto{\pgfqpoint{2.493556in}{1.426269in}}%
\pgfpathlineto{\pgfqpoint{2.426840in}{1.441418in}}%
\pgfpathlineto{\pgfqpoint{2.359641in}{1.454285in}}%
\pgfpathlineto{\pgfqpoint{2.292133in}{1.465451in}}%
\pgfpathlineto{\pgfqpoint{2.224506in}{1.475892in}}%
\pgfpathlineto{\pgfqpoint{2.157013in}{1.487125in}}%
\pgfpathlineto{\pgfqpoint{2.090197in}{1.501638in}}%
\pgfpathlineto{\pgfqpoint{2.025757in}{1.523864in}}%
\pgfpathlineto{\pgfqpoint{2.025757in}{1.523864in}}%
\pgfpathlineto{\pgfqpoint{1.982758in}{1.549333in}}%
\pgfpathlineto{\pgfqpoint{1.982758in}{1.549333in}}%
\pgfpathlineto{\pgfqpoint{1.954899in}{1.578399in}}%
\pgfusepath{stroke}%
\end{pgfscope}%
\begin{pgfscope}%
\pgfpathrectangle{\pgfqpoint{0.647939in}{0.492442in}}{\pgfqpoint{3.079299in}{3.079299in}}%
\pgfusepath{clip}%
\pgfsetbuttcap%
\pgfsetroundjoin%
\pgfsetlinewidth{0.301125pt}%
\definecolor{currentstroke}{rgb}{0.500000,0.500000,0.500000}%
\pgfsetstrokecolor{currentstroke}%
\pgfsetstrokeopacity{0.300000}%
\pgfsetdash{}{0pt}%
\pgfpathmoveto{\pgfqpoint{3.727238in}{0.912347in}}%
\pgfpathlineto{\pgfqpoint{3.727238in}{0.912347in}}%
\pgfpathlineto{\pgfqpoint{3.664499in}{0.939646in}}%
\pgfpathlineto{\pgfqpoint{3.602552in}{0.968701in}}%
\pgfpathlineto{\pgfqpoint{3.541351in}{0.999299in}}%
\pgfpathlineto{\pgfqpoint{3.480827in}{1.031218in}}%
\pgfpathlineto{\pgfqpoint{3.420896in}{1.064238in}}%
\pgfpathlineto{\pgfqpoint{3.361457in}{1.098138in}}%
\pgfpathlineto{\pgfqpoint{3.302401in}{1.132702in}}%
\pgfpathlineto{\pgfqpoint{3.243608in}{1.167714in}}%
\pgfpathlineto{\pgfqpoint{3.184954in}{1.202956in}}%
\pgfpathlineto{\pgfqpoint{3.126305in}{1.238209in}}%
\pgfpathlineto{\pgfqpoint{3.067526in}{1.273243in}}%
\pgfpathlineto{\pgfqpoint{3.008479in}{1.307821in}}%
\pgfpathlineto{\pgfqpoint{2.949023in}{1.341690in}}%
\pgfpathlineto{\pgfqpoint{2.889021in}{1.374579in}}%
\pgfpathlineto{\pgfqpoint{2.828346in}{1.406204in}}%
\pgfpathlineto{\pgfqpoint{2.766884in}{1.436266in}}%
\pgfpathlineto{\pgfqpoint{2.704547in}{1.464462in}}%
\pgfpathlineto{\pgfqpoint{2.641280in}{1.490499in}}%
\pgfpathlineto{\pgfqpoint{2.577074in}{1.514124in}}%
\pgfpathlineto{\pgfqpoint{2.511977in}{1.535168in}}%
\pgfpathlineto{\pgfqpoint{2.446097in}{1.553614in}}%
\pgfpathlineto{\pgfqpoint{2.379591in}{1.569669in}}%
\pgfpathlineto{\pgfqpoint{2.312656in}{1.583862in}}%
\pgfpathlineto{\pgfqpoint{2.245542in}{1.597205in}}%
\pgfpathlineto{\pgfqpoint{2.178637in}{1.611514in}}%
\pgfpathlineto{\pgfqpoint{2.112915in}{1.630218in}}%
\pgfpathlineto{\pgfqpoint{2.052298in}{1.660509in}}%
\pgfpathlineto{\pgfqpoint{2.052298in}{1.660509in}}%
\pgfpathlineto{\pgfqpoint{2.024966in}{1.686388in}}%
\pgfpathlineto{\pgfqpoint{2.024966in}{1.686388in}}%
\pgfpathlineto{\pgfqpoint{2.009785in}{1.716045in}}%
\pgfpathlineto{\pgfqpoint{2.005413in}{1.749238in}}%
\pgfpathlineto{\pgfqpoint{2.010238in}{1.783337in}}%
\pgfpathlineto{\pgfqpoint{2.023973in}{1.821541in}}%
\pgfusepath{stroke}%
\end{pgfscope}%
\begin{pgfscope}%
\pgfpathrectangle{\pgfqpoint{0.647939in}{0.492442in}}{\pgfqpoint{3.079299in}{3.079299in}}%
\pgfusepath{clip}%
\pgfsetbuttcap%
\pgfsetroundjoin%
\pgfsetlinewidth{0.301125pt}%
\definecolor{currentstroke}{rgb}{0.500000,0.500000,0.500000}%
\pgfsetstrokecolor{currentstroke}%
\pgfsetstrokeopacity{0.300000}%
\pgfsetdash{}{0pt}%
\pgfpathmoveto{\pgfqpoint{3.727238in}{0.982331in}}%
\pgfpathlineto{\pgfqpoint{3.727238in}{0.982331in}}%
\pgfpathlineto{\pgfqpoint{3.664937in}{1.010613in}}%
\pgfpathlineto{\pgfqpoint{3.603494in}{1.040719in}}%
\pgfpathlineto{\pgfqpoint{3.542865in}{1.072433in}}%
\pgfpathlineto{\pgfqpoint{3.482982in}{1.105538in}}%
\pgfpathlineto{\pgfqpoint{3.423764in}{1.139820in}}%
\pgfpathlineto{\pgfqpoint{3.365114in}{1.175068in}}%
\pgfpathlineto{\pgfqpoint{3.306927in}{1.211075in}}%
\pgfpathlineto{\pgfqpoint{3.249088in}{1.247640in}}%
\pgfpathlineto{\pgfqpoint{3.191477in}{1.284563in}}%
\pgfpathlineto{\pgfqpoint{3.133967in}{1.321643in}}%
\pgfpathlineto{\pgfqpoint{3.076425in}{1.358675in}}%
\pgfpathlineto{\pgfqpoint{3.018716in}{1.395443in}}%
\pgfpathlineto{\pgfqpoint{2.960700in}{1.431724in}}%
\pgfpathlineto{\pgfqpoint{2.902244in}{1.467286in}}%
\pgfpathlineto{\pgfqpoint{2.843210in}{1.501877in}}%
\pgfpathlineto{\pgfqpoint{2.783472in}{1.535232in}}%
\pgfpathlineto{\pgfqpoint{2.722916in}{1.567072in}}%
\pgfpathlineto{\pgfqpoint{2.661451in}{1.597116in}}%
\pgfpathlineto{\pgfqpoint{2.599023in}{1.625099in}}%
\pgfpathlineto{\pgfqpoint{2.535629in}{1.650812in}}%
\pgfpathlineto{\pgfqpoint{2.471324in}{1.674162in}}%
\pgfpathlineto{\pgfqpoint{2.406245in}{1.695265in}}%
\pgfpathlineto{\pgfqpoint{2.340616in}{1.714607in}}%
\pgfpathlineto{\pgfqpoint{2.274804in}{1.733337in}}%
\pgfpathlineto{\pgfqpoint{2.209601in}{1.753975in}}%
\pgfpathlineto{\pgfqpoint{2.147897in}{1.782607in}}%
\pgfpathlineto{\pgfqpoint{2.147897in}{1.782607in}}%
\pgfpathlineto{\pgfqpoint{2.119357in}{1.805701in}}%
\pgfpathlineto{\pgfqpoint{2.119357in}{1.805701in}}%
\pgfpathlineto{\pgfqpoint{2.103426in}{1.831787in}}%
\pgfpathlineto{\pgfqpoint{2.099087in}{1.862509in}}%
\pgfpathlineto{\pgfqpoint{2.104345in}{1.891737in}}%
\pgfpathlineto{\pgfqpoint{2.118045in}{1.924277in}}%
\pgfusepath{stroke}%
\end{pgfscope}%
\begin{pgfscope}%
\pgfpathrectangle{\pgfqpoint{0.647939in}{0.492442in}}{\pgfqpoint{3.079299in}{3.079299in}}%
\pgfusepath{clip}%
\pgfsetbuttcap%
\pgfsetroundjoin%
\pgfsetlinewidth{0.301125pt}%
\definecolor{currentstroke}{rgb}{0.500000,0.500000,0.500000}%
\pgfsetstrokecolor{currentstroke}%
\pgfsetstrokeopacity{0.300000}%
\pgfsetdash{}{0pt}%
\pgfpathmoveto{\pgfqpoint{3.727238in}{1.122299in}}%
\pgfpathlineto{\pgfqpoint{3.727238in}{1.122299in}}%
\pgfpathlineto{\pgfqpoint{3.665970in}{1.152750in}}%
\pgfpathlineto{\pgfqpoint{3.605717in}{1.185169in}}%
\pgfpathlineto{\pgfqpoint{3.546442in}{1.219344in}}%
\pgfpathlineto{\pgfqpoint{3.488087in}{1.255069in}}%
\pgfpathlineto{\pgfqpoint{3.430579in}{1.292144in}}%
\pgfpathlineto{\pgfqpoint{3.373836in}{1.330384in}}%
\pgfpathlineto{\pgfqpoint{3.317770in}{1.369611in}}%
\pgfpathlineto{\pgfqpoint{3.262288in}{1.409660in}}%
\pgfpathlineto{\pgfqpoint{3.207290in}{1.450371in}}%
\pgfpathlineto{\pgfqpoint{3.152676in}{1.491596in}}%
\pgfpathlineto{\pgfqpoint{3.098352in}{1.533203in}}%
\pgfpathlineto{\pgfqpoint{3.044224in}{1.575065in}}%
\pgfpathlineto{\pgfqpoint{2.990201in}{1.617062in}}%
\pgfpathlineto{\pgfqpoint{2.936198in}{1.659080in}}%
\pgfpathlineto{\pgfqpoint{2.882141in}{1.701030in}}%
\pgfpathlineto{\pgfqpoint{2.827972in}{1.742836in}}%
\pgfpathlineto{\pgfqpoint{2.773662in}{1.784452in}}%
\pgfpathlineto{\pgfqpoint{2.719220in}{1.825894in}}%
\pgfpathlineto{\pgfqpoint{2.664726in}{1.867268in}}%
\pgfpathlineto{\pgfqpoint{2.610398in}{1.908852in}}%
\pgfpathlineto{\pgfqpoint{2.556728in}{1.951274in}}%
\pgfpathlineto{\pgfqpoint{2.504859in}{1.995828in}}%
\pgfpathlineto{\pgfqpoint{2.457722in}{2.045155in}}%
\pgfpathlineto{\pgfqpoint{2.457722in}{2.045155in}}%
\pgfpathlineto{\pgfqpoint{2.429245in}{2.089402in}}%
\pgfpathlineto{\pgfqpoint{2.429245in}{2.089402in}}%
\pgfpathlineto{\pgfqpoint{2.416836in}{2.129388in}}%
\pgfpathlineto{\pgfqpoint{2.416978in}{2.172161in}}%
\pgfusepath{stroke}%
\end{pgfscope}%
\begin{pgfscope}%
\pgfpathrectangle{\pgfqpoint{0.647939in}{0.492442in}}{\pgfqpoint{3.079299in}{3.079299in}}%
\pgfusepath{clip}%
\pgfsetbuttcap%
\pgfsetroundjoin%
\pgfsetlinewidth{0.301125pt}%
\definecolor{currentstroke}{rgb}{0.500000,0.500000,0.500000}%
\pgfsetstrokecolor{currentstroke}%
\pgfsetstrokeopacity{0.300000}%
\pgfsetdash{}{0pt}%
\pgfpathmoveto{\pgfqpoint{3.727238in}{1.262267in}}%
\pgfpathlineto{\pgfqpoint{3.727238in}{1.262267in}}%
\pgfpathlineto{\pgfqpoint{3.667269in}{1.295198in}}%
\pgfpathlineto{\pgfqpoint{3.608516in}{1.330257in}}%
\pgfpathlineto{\pgfqpoint{3.550954in}{1.367241in}}%
\pgfpathlineto{\pgfqpoint{3.494543in}{1.405959in}}%
\pgfpathlineto{\pgfqpoint{3.439231in}{1.446235in}}%
\pgfpathlineto{\pgfqpoint{3.384966in}{1.487912in}}%
\pgfpathlineto{\pgfqpoint{3.331690in}{1.530847in}}%
\pgfpathlineto{\pgfqpoint{3.279347in}{1.574917in}}%
\pgfpathlineto{\pgfqpoint{3.227903in}{1.620031in}}%
\pgfpathlineto{\pgfqpoint{3.177329in}{1.666119in}}%
\pgfpathlineto{\pgfqpoint{3.127615in}{1.713133in}}%
\pgfpathlineto{\pgfqpoint{3.078788in}{1.761067in}}%
\pgfpathlineto{\pgfqpoint{3.030910in}{1.809946in}}%
\pgfpathlineto{\pgfqpoint{2.984103in}{1.859851in}}%
\pgfpathlineto{\pgfqpoint{2.938575in}{1.910922in}}%
\pgfpathlineto{\pgfqpoint{2.894666in}{1.963382in}}%
\pgfpathlineto{\pgfqpoint{2.852912in}{2.017567in}}%
\pgfpathlineto{\pgfqpoint{2.814171in}{2.073927in}}%
\pgfpathlineto{\pgfqpoint{2.779782in}{2.132998in}}%
\pgfpathlineto{\pgfqpoint{2.751728in}{2.195255in}}%
\pgfpathlineto{\pgfqpoint{2.732550in}{2.260699in}}%
\pgfpathlineto{\pgfqpoint{2.724525in}{2.328359in}}%
\pgfpathlineto{\pgfqpoint{2.728228in}{2.396388in}}%
\pgfpathlineto{\pgfqpoint{2.742118in}{2.463157in}}%
\pgfpathlineto{\pgfqpoint{2.763731in}{2.527922in}}%
\pgfpathlineto{\pgfqpoint{2.790822in}{2.590661in}}%
\pgfpathlineto{\pgfqpoint{2.821751in}{2.651633in}}%
\pgfpathlineto{\pgfqpoint{2.855420in}{2.711161in}}%
\pgfusepath{stroke}%
\end{pgfscope}%
\begin{pgfscope}%
\pgfpathrectangle{\pgfqpoint{0.647939in}{0.492442in}}{\pgfqpoint{3.079299in}{3.079299in}}%
\pgfusepath{clip}%
\pgfsetbuttcap%
\pgfsetroundjoin%
\pgfsetlinewidth{0.301125pt}%
\definecolor{currentstroke}{rgb}{0.500000,0.500000,0.500000}%
\pgfsetstrokecolor{currentstroke}%
\pgfsetstrokeopacity{0.300000}%
\pgfsetdash{}{0pt}%
\pgfpathmoveto{\pgfqpoint{3.727238in}{1.332251in}}%
\pgfpathlineto{\pgfqpoint{3.727238in}{1.332251in}}%
\pgfpathlineto{\pgfqpoint{3.668045in}{1.366555in}}%
\pgfpathlineto{\pgfqpoint{3.610190in}{1.403074in}}%
\pgfpathlineto{\pgfqpoint{3.553658in}{1.441610in}}%
\pgfpathlineto{\pgfqpoint{3.498419in}{1.481981in}}%
\pgfpathlineto{\pgfqpoint{3.444441in}{1.524025in}}%
\pgfpathlineto{\pgfqpoint{3.391689in}{1.567599in}}%
\pgfpathlineto{\pgfqpoint{3.340129in}{1.612577in}}%
\pgfpathlineto{\pgfqpoint{3.289748in}{1.658874in}}%
\pgfpathlineto{\pgfqpoint{3.240549in}{1.706425in}}%
\pgfpathlineto{\pgfqpoint{3.192559in}{1.755195in}}%
\pgfpathlineto{\pgfqpoint{3.145850in}{1.805190in}}%
\pgfpathlineto{\pgfqpoint{3.100538in}{1.856454in}}%
\pgfpathlineto{\pgfqpoint{3.056819in}{1.909078in}}%
\pgfpathlineto{\pgfqpoint{3.014984in}{1.963206in}}%
\pgfpathlineto{\pgfqpoint{2.975468in}{2.019046in}}%
\pgfpathlineto{\pgfqpoint{2.938908in}{2.076853in}}%
\pgfpathlineto{\pgfqpoint{2.906206in}{2.136905in}}%
\pgfpathlineto{\pgfqpoint{2.878586in}{2.199426in}}%
\pgfpathlineto{\pgfqpoint{2.857541in}{2.264418in}}%
\pgfpathlineto{\pgfqpoint{2.844575in}{2.331451in}}%
\pgfpathlineto{\pgfqpoint{2.840661in}{2.399578in}}%
\pgfpathlineto{\pgfqpoint{2.845750in}{2.467630in}}%
\pgfpathlineto{\pgfqpoint{2.858831in}{2.534650in}}%
\pgfpathlineto{\pgfqpoint{2.878436in}{2.600104in}}%
\pgfpathlineto{\pgfqpoint{2.903115in}{2.663852in}}%
\pgfpathlineto{\pgfqpoint{2.931679in}{2.725980in}}%
\pgfpathlineto{\pgfqpoint{2.963240in}{2.786655in}}%
\pgfpathlineto{\pgfqpoint{2.997160in}{2.846051in}}%
\pgfpathlineto{\pgfqpoint{3.032994in}{2.904319in}}%
\pgfpathlineto{\pgfqpoint{3.070431in}{2.961577in}}%
\pgfpathlineto{\pgfqpoint{3.109264in}{3.017904in}}%
\pgfpathlineto{\pgfqpoint{3.149360in}{3.073344in}}%
\pgfpathlineto{\pgfqpoint{3.190631in}{3.127915in}}%
\pgfpathlineto{\pgfqpoint{3.233025in}{3.181618in}}%
\pgfpathlineto{\pgfqpoint{3.276535in}{3.234422in}}%
\pgfpathlineto{\pgfqpoint{3.321166in}{3.286284in}}%
\pgfusepath{stroke}%
\end{pgfscope}%
\begin{pgfscope}%
\pgfpathrectangle{\pgfqpoint{0.647939in}{0.492442in}}{\pgfqpoint{3.079299in}{3.079299in}}%
\pgfusepath{clip}%
\pgfsetbuttcap%
\pgfsetroundjoin%
\pgfsetlinewidth{0.301125pt}%
\definecolor{currentstroke}{rgb}{0.500000,0.500000,0.500000}%
\pgfsetstrokecolor{currentstroke}%
\pgfsetstrokeopacity{0.300000}%
\pgfsetdash{}{0pt}%
\pgfpathmoveto{\pgfqpoint{3.727238in}{1.472219in}}%
\pgfpathlineto{\pgfqpoint{3.727238in}{1.472219in}}%
\pgfpathlineto{\pgfqpoint{3.669925in}{1.509572in}}%
\pgfpathlineto{\pgfqpoint{3.614247in}{1.549324in}}%
\pgfpathlineto{\pgfqpoint{3.560217in}{1.591291in}}%
\pgfpathlineto{\pgfqpoint{3.507844in}{1.635310in}}%
\pgfpathlineto{\pgfqpoint{3.457139in}{1.681242in}}%
\pgfpathlineto{\pgfqpoint{3.408124in}{1.728972in}}%
\pgfpathlineto{\pgfqpoint{3.360850in}{1.778428in}}%
\pgfpathlineto{\pgfqpoint{3.315396in}{1.829562in}}%
\pgfpathlineto{\pgfqpoint{3.271892in}{1.882363in}}%
\pgfpathlineto{\pgfqpoint{3.230527in}{1.936851in}}%
\pgfpathlineto{\pgfqpoint{3.191565in}{1.993078in}}%
\pgfpathlineto{\pgfqpoint{3.155362in}{2.051115in}}%
\pgfpathlineto{\pgfqpoint{3.122398in}{2.111047in}}%
\pgfpathlineto{\pgfqpoint{3.093282in}{2.172930in}}%
\pgfpathlineto{\pgfqpoint{3.068746in}{2.236753in}}%
\pgfpathlineto{\pgfqpoint{3.049588in}{2.302371in}}%
\pgfpathlineto{\pgfqpoint{3.036564in}{2.369456in}}%
\pgfpathlineto{\pgfqpoint{3.030213in}{2.437483in}}%
\pgfpathlineto{\pgfqpoint{3.030710in}{2.505805in}}%
\pgfpathlineto{\pgfqpoint{3.037803in}{2.573774in}}%
\pgfpathlineto{\pgfqpoint{3.050900in}{2.640865in}}%
\pgfpathlineto{\pgfqpoint{3.069236in}{2.706730in}}%
\pgfpathlineto{\pgfqpoint{3.092021in}{2.771202in}}%
\pgfpathlineto{\pgfqpoint{3.118546in}{2.834238in}}%
\pgfpathlineto{\pgfqpoint{3.148220in}{2.895864in}}%
\pgfpathlineto{\pgfqpoint{3.180577in}{2.956132in}}%
\pgfpathlineto{\pgfqpoint{3.215261in}{3.015098in}}%
\pgfpathlineto{\pgfqpoint{3.252008in}{3.072806in}}%
\pgfpathlineto{\pgfqpoint{3.290623in}{3.129284in}}%
\pgfpathlineto{\pgfqpoint{3.330980in}{3.184529in}}%
\pgfpathlineto{\pgfqpoint{3.373003in}{3.238518in}}%
\pgfpathlineto{\pgfqpoint{3.416640in}{3.291215in}}%
\pgfpathlineto{\pgfqpoint{3.461880in}{3.342542in}}%
\pgfpathlineto{\pgfqpoint{3.508731in}{3.392401in}}%
\pgfpathlineto{\pgfqpoint{3.557219in}{3.440667in}}%
\pgfpathlineto{\pgfqpoint{3.607385in}{3.487186in}}%
\pgfpathlineto{\pgfqpoint{3.659271in}{3.531776in}}%
\pgfpathlineto{\pgfqpoint{3.707647in}{3.571741in}}%
\pgfusepath{stroke}%
\end{pgfscope}%
\begin{pgfscope}%
\pgfpathrectangle{\pgfqpoint{0.647939in}{0.492442in}}{\pgfqpoint{3.079299in}{3.079299in}}%
\pgfusepath{clip}%
\pgfsetbuttcap%
\pgfsetroundjoin%
\pgfsetlinewidth{0.301125pt}%
\definecolor{currentstroke}{rgb}{0.500000,0.500000,0.500000}%
\pgfsetstrokecolor{currentstroke}%
\pgfsetstrokeopacity{0.300000}%
\pgfsetdash{}{0pt}%
\pgfpathmoveto{\pgfqpoint{3.727238in}{1.612187in}}%
\pgfpathlineto{\pgfqpoint{3.727238in}{1.612187in}}%
\pgfpathlineto{\pgfqpoint{3.672372in}{1.653043in}}%
\pgfpathlineto{\pgfqpoint{3.619531in}{1.696486in}}%
\pgfpathlineto{\pgfqpoint{3.568773in}{1.742347in}}%
\pgfpathlineto{\pgfqpoint{3.520158in}{1.790477in}}%
\pgfpathlineto{\pgfqpoint{3.473768in}{1.840753in}}%
\pgfpathlineto{\pgfqpoint{3.429716in}{1.893090in}}%
\pgfpathlineto{\pgfqpoint{3.388160in}{1.947429in}}%
\pgfpathlineto{\pgfqpoint{3.349313in}{2.003731in}}%
\pgfpathlineto{\pgfqpoint{3.313454in}{2.061976in}}%
\pgfpathlineto{\pgfqpoint{3.280936in}{2.122144in}}%
\pgfpathlineto{\pgfqpoint{3.252194in}{2.184197in}}%
\pgfpathlineto{\pgfqpoint{3.227730in}{2.248048in}}%
\pgfpathlineto{\pgfqpoint{3.208077in}{2.313530in}}%
\pgfpathlineto{\pgfqpoint{3.193740in}{2.380369in}}%
\pgfpathlineto{\pgfqpoint{3.185121in}{2.448179in}}%
\pgfpathlineto{\pgfqpoint{3.182433in}{2.516483in}}%
\pgfpathlineto{\pgfqpoint{3.185648in}{2.584764in}}%
\pgfpathlineto{\pgfqpoint{3.194502in}{2.652547in}}%
\pgfpathlineto{\pgfqpoint{3.208561in}{2.719453in}}%
\pgfpathlineto{\pgfqpoint{3.227305in}{2.785212in}}%
\pgfpathlineto{\pgfqpoint{3.250198in}{2.849654in}}%
\pgfpathlineto{\pgfqpoint{3.276748in}{2.912686in}}%
\pgfpathlineto{\pgfqpoint{3.306525in}{2.974266in}}%
\pgfpathlineto{\pgfqpoint{3.339178in}{3.034374in}}%
\pgfpathlineto{\pgfqpoint{3.374432in}{3.092997in}}%
\pgfpathlineto{\pgfqpoint{3.412076in}{3.150117in}}%
\pgfpathlineto{\pgfqpoint{3.451961in}{3.205698in}}%
\pgfpathlineto{\pgfqpoint{3.493993in}{3.259677in}}%
\pgfpathlineto{\pgfqpoint{3.538103in}{3.311971in}}%
\pgfpathlineto{\pgfqpoint{3.584260in}{3.362465in}}%
\pgfpathlineto{\pgfqpoint{3.632457in}{3.411015in}}%
\pgfpathlineto{\pgfqpoint{3.682698in}{3.457443in}}%
\pgfpathlineto{\pgfqpoint{3.727238in}{3.496821in}}%
\pgfusepath{stroke}%
\end{pgfscope}%
\begin{pgfscope}%
\pgfpathrectangle{\pgfqpoint{0.647939in}{0.492442in}}{\pgfqpoint{3.079299in}{3.079299in}}%
\pgfusepath{clip}%
\pgfsetbuttcap%
\pgfsetroundjoin%
\pgfsetlinewidth{0.301125pt}%
\definecolor{currentstroke}{rgb}{0.500000,0.500000,0.500000}%
\pgfsetstrokecolor{currentstroke}%
\pgfsetstrokeopacity{0.300000}%
\pgfsetdash{}{0pt}%
\pgfpathmoveto{\pgfqpoint{3.727238in}{1.682171in}}%
\pgfpathlineto{\pgfqpoint{3.727238in}{1.682171in}}%
\pgfpathlineto{\pgfqpoint{3.673873in}{1.724962in}}%
\pgfpathlineto{\pgfqpoint{3.622770in}{1.770434in}}%
\pgfpathlineto{\pgfqpoint{3.574018in}{1.818418in}}%
\pgfpathlineto{\pgfqpoint{3.527712in}{1.868766in}}%
\pgfpathlineto{\pgfqpoint{3.483975in}{1.921362in}}%
\pgfpathlineto{\pgfqpoint{3.442977in}{1.976117in}}%
\pgfusepath{stroke}%
\end{pgfscope}%
\begin{pgfscope}%
\pgfpathrectangle{\pgfqpoint{0.647939in}{0.492442in}}{\pgfqpoint{3.079299in}{3.079299in}}%
\pgfusepath{clip}%
\pgfsetbuttcap%
\pgfsetroundjoin%
\pgfsetlinewidth{0.301125pt}%
\definecolor{currentstroke}{rgb}{0.500000,0.500000,0.500000}%
\pgfsetstrokecolor{currentstroke}%
\pgfsetstrokeopacity{0.300000}%
\pgfsetdash{}{0pt}%
\pgfpathmoveto{\pgfqpoint{3.727238in}{1.752155in}}%
\pgfpathlineto{\pgfqpoint{3.727238in}{1.752155in}}%
\pgfpathlineto{\pgfqpoint{3.675600in}{1.797011in}}%
\pgfpathlineto{\pgfqpoint{3.626497in}{1.844629in}}%
\pgfpathlineto{\pgfqpoint{3.580049in}{1.894842in}}%
\pgfpathlineto{\pgfqpoint{3.536394in}{1.947499in}}%
\pgfpathlineto{\pgfqpoint{3.495703in}{2.002477in}}%
\pgfpathlineto{\pgfqpoint{3.458197in}{2.059671in}}%
\pgfpathlineto{\pgfqpoint{3.424149in}{2.118985in}}%
\pgfpathlineto{\pgfqpoint{3.393888in}{2.180313in}}%
\pgfpathlineto{\pgfqpoint{3.367790in}{2.243519in}}%
\pgfpathlineto{\pgfqpoint{3.346259in}{2.308415in}}%
\pgfpathlineto{\pgfqpoint{3.329696in}{2.374746in}}%
\pgfpathlineto{\pgfqpoint{3.318436in}{2.442175in}}%
\pgfpathlineto{\pgfqpoint{3.312707in}{2.510297in}}%
\pgfpathlineto{\pgfqpoint{3.312580in}{2.578663in}}%
\pgfpathlineto{\pgfqpoint{3.317955in}{2.646818in}}%
\pgfpathlineto{\pgfqpoint{3.328577in}{2.714356in}}%
\pgfpathlineto{\pgfqpoint{3.344094in}{2.780946in}}%
\pgfpathlineto{\pgfqpoint{3.364095in}{2.846338in}}%
\pgfpathlineto{\pgfqpoint{3.388163in}{2.910351in}}%
\pgfpathlineto{\pgfqpoint{3.415920in}{2.972860in}}%
\pgfpathlineto{\pgfqpoint{3.447035in}{3.033772in}}%
\pgfpathlineto{\pgfqpoint{3.481236in}{3.093009in}}%
\pgfpathlineto{\pgfqpoint{3.518313in}{3.150490in}}%
\pgfpathlineto{\pgfqpoint{3.558107in}{3.206129in}}%
\pgfpathlineto{\pgfqpoint{3.600496in}{3.259819in}}%
\pgfpathlineto{\pgfqpoint{3.645403in}{3.311421in}}%
\pgfpathlineto{\pgfqpoint{3.692779in}{3.360763in}}%
\pgfpathlineto{\pgfqpoint{3.727238in}{3.394928in}}%
\pgfusepath{stroke}%
\end{pgfscope}%
\begin{pgfscope}%
\pgfpathrectangle{\pgfqpoint{0.647939in}{0.492442in}}{\pgfqpoint{3.079299in}{3.079299in}}%
\pgfusepath{clip}%
\pgfsetbuttcap%
\pgfsetroundjoin%
\pgfsetlinewidth{0.301125pt}%
\definecolor{currentstroke}{rgb}{0.500000,0.500000,0.500000}%
\pgfsetstrokecolor{currentstroke}%
\pgfsetstrokeopacity{0.300000}%
\pgfsetdash{}{0pt}%
\pgfpathmoveto{\pgfqpoint{3.727238in}{1.892124in}}%
\pgfpathlineto{\pgfqpoint{3.727238in}{1.892124in}}%
\pgfpathlineto{\pgfqpoint{3.679896in}{1.941481in}}%
\pgfpathlineto{\pgfqpoint{3.635754in}{1.993717in}}%
\pgfpathlineto{\pgfqpoint{3.595001in}{2.048641in}}%
\pgfpathlineto{\pgfqpoint{3.557862in}{2.106068in}}%
\pgfpathlineto{\pgfqpoint{3.524603in}{2.165821in}}%
\pgfpathlineto{\pgfqpoint{3.495529in}{2.227713in}}%
\pgfpathlineto{\pgfqpoint{3.470969in}{2.291527in}}%
\pgfpathlineto{\pgfqpoint{3.451262in}{2.357001in}}%
\pgfpathlineto{\pgfqpoint{3.436713in}{2.423807in}}%
\pgfpathlineto{\pgfqpoint{3.427565in}{2.491558in}}%
\pgfpathlineto{\pgfqpoint{3.423946in}{2.559823in}}%
\pgfpathlineto{\pgfqpoint{3.425858in}{2.628158in}}%
\pgfpathlineto{\pgfqpoint{3.433168in}{2.696134in}}%
\pgfpathlineto{\pgfqpoint{3.445632in}{2.763365in}}%
\pgfpathlineto{\pgfqpoint{3.462935in}{2.829521in}}%
\pgfpathlineto{\pgfqpoint{3.484730in}{2.894339in}}%
\pgfpathlineto{\pgfqpoint{3.510670in}{2.957616in}}%
\pgfpathlineto{\pgfqpoint{3.540446in}{3.019185in}}%
\pgfpathlineto{\pgfqpoint{3.573788in}{3.078903in}}%
\pgfpathlineto{\pgfqpoint{3.610483in}{3.136625in}}%
\pgfpathlineto{\pgfqpoint{3.650358in}{3.192199in}}%
\pgfpathlineto{\pgfqpoint{3.693275in}{3.245456in}}%
\pgfpathlineto{\pgfqpoint{3.727238in}{3.285284in}}%
\pgfusepath{stroke}%
\end{pgfscope}%
\begin{pgfscope}%
\pgfpathrectangle{\pgfqpoint{0.647939in}{0.492442in}}{\pgfqpoint{3.079299in}{3.079299in}}%
\pgfusepath{clip}%
\pgfsetbuttcap%
\pgfsetroundjoin%
\pgfsetlinewidth{0.301125pt}%
\definecolor{currentstroke}{rgb}{0.500000,0.500000,0.500000}%
\pgfsetstrokecolor{currentstroke}%
\pgfsetstrokeopacity{0.300000}%
\pgfsetdash{}{0pt}%
\pgfpathmoveto{\pgfqpoint{3.727238in}{2.032092in}}%
\pgfpathlineto{\pgfqpoint{3.727238in}{2.032092in}}%
\pgfpathlineto{\pgfqpoint{3.685631in}{2.086356in}}%
\pgfpathlineto{\pgfqpoint{3.648040in}{2.143475in}}%
\pgfpathlineto{\pgfqpoint{3.614731in}{2.203192in}}%
\pgfpathlineto{\pgfqpoint{3.585994in}{2.265238in}}%
\pgfpathlineto{\pgfqpoint{3.562129in}{2.329313in}}%
\pgfpathlineto{\pgfqpoint{3.543433in}{2.395076in}}%
\pgfpathlineto{\pgfqpoint{3.530154in}{2.462137in}}%
\pgfpathlineto{\pgfqpoint{3.522470in}{2.530061in}}%
\pgfpathlineto{\pgfqpoint{3.520458in}{2.598389in}}%
\pgfpathlineto{\pgfqpoint{3.524078in}{2.666657in}}%
\pgfpathlineto{\pgfqpoint{3.533179in}{2.734418in}}%
\pgfpathlineto{\pgfqpoint{3.547521in}{2.801269in}}%
\pgfpathlineto{\pgfqpoint{3.566814in}{2.866864in}}%
\pgfpathlineto{\pgfqpoint{3.590748in}{2.930915in}}%
\pgfpathlineto{\pgfqpoint{3.619024in}{2.993176in}}%
\pgfpathlineto{\pgfqpoint{3.651371in}{3.053425in}}%
\pgfpathlineto{\pgfqpoint{3.687557in}{3.111453in}}%
\pgfusepath{stroke}%
\end{pgfscope}%
\begin{pgfscope}%
\pgfpathrectangle{\pgfqpoint{0.647939in}{0.492442in}}{\pgfqpoint{3.079299in}{3.079299in}}%
\pgfusepath{clip}%
\pgfsetbuttcap%
\pgfsetroundjoin%
\pgfsetlinewidth{0.301125pt}%
\definecolor{currentstroke}{rgb}{0.500000,0.500000,0.500000}%
\pgfsetstrokecolor{currentstroke}%
\pgfsetstrokeopacity{0.300000}%
\pgfsetdash{}{0pt}%
\pgfpathmoveto{\pgfqpoint{3.727238in}{2.172060in}}%
\pgfpathlineto{\pgfqpoint{3.727238in}{2.172060in}}%
\pgfpathlineto{\pgfqpoint{3.693207in}{2.231359in}}%
\pgfpathlineto{\pgfqpoint{3.664129in}{2.293233in}}%
\pgfpathlineto{\pgfqpoint{3.640288in}{2.357307in}}%
\pgfpathlineto{\pgfqpoint{3.621952in}{2.423165in}}%
\pgfpathlineto{\pgfqpoint{3.609343in}{2.490350in}}%
\pgfpathlineto{\pgfqpoint{3.602603in}{2.558370in}}%
\pgfpathlineto{\pgfqpoint{3.601775in}{2.626717in}}%
\pgfpathlineto{\pgfqpoint{3.606791in}{2.694889in}}%
\pgfpathlineto{\pgfqpoint{3.617496in}{2.762411in}}%
\pgfpathlineto{\pgfqpoint{3.633649in}{2.828845in}}%
\pgfpathlineto{\pgfqpoint{3.654969in}{2.893806in}}%
\pgfpathlineto{\pgfqpoint{3.681171in}{2.956959in}}%
\pgfpathlineto{\pgfqpoint{3.711977in}{3.018001in}}%
\pgfpathlineto{\pgfqpoint{3.727238in}{3.045693in}}%
\pgfusepath{stroke}%
\end{pgfscope}%
\begin{pgfscope}%
\pgfpathrectangle{\pgfqpoint{0.647939in}{0.492442in}}{\pgfqpoint{3.079299in}{3.079299in}}%
\pgfusepath{clip}%
\pgfsetbuttcap%
\pgfsetroundjoin%
\pgfsetlinewidth{0.301125pt}%
\definecolor{currentstroke}{rgb}{0.500000,0.500000,0.500000}%
\pgfsetstrokecolor{currentstroke}%
\pgfsetstrokeopacity{0.300000}%
\pgfsetdash{}{0pt}%
\pgfpathmoveto{\pgfqpoint{3.727238in}{2.312028in}}%
\pgfpathlineto{\pgfqpoint{3.727238in}{2.312028in}}%
\pgfpathlineto{\pgfqpoint{3.702926in}{2.375906in}}%
\pgfpathlineto{\pgfqpoint{3.684452in}{2.441707in}}%
\pgfpathlineto{\pgfqpoint{3.672021in}{2.508914in}}%
\pgfpathlineto{\pgfqpoint{3.665759in}{2.576979in}}%
\pgfpathlineto{\pgfqpoint{3.665693in}{2.645332in}}%
\pgfpathlineto{\pgfqpoint{3.671741in}{2.713413in}}%
\pgfusepath{stroke}%
\end{pgfscope}%
\begin{pgfscope}%
\pgfpathrectangle{\pgfqpoint{0.647939in}{0.492442in}}{\pgfqpoint{3.079299in}{3.079299in}}%
\pgfusepath{clip}%
\pgfsetbuttcap%
\pgfsetroundjoin%
\pgfsetlinewidth{0.301125pt}%
\definecolor{currentstroke}{rgb}{0.500000,0.500000,0.500000}%
\pgfsetstrokecolor{currentstroke}%
\pgfsetstrokeopacity{0.300000}%
\pgfsetdash{}{0pt}%
\pgfpathmoveto{\pgfqpoint{3.727238in}{2.432168in}}%
\pgfpathlineto{\pgfqpoint{3.720474in}{2.463196in}}%
\pgfpathlineto{\pgfqpoint{3.709109in}{2.530583in}}%
\pgfpathlineto{\pgfqpoint{3.704154in}{2.598742in}}%
\pgfpathlineto{\pgfqpoint{3.705607in}{2.667074in}}%
\pgfpathlineto{\pgfqpoint{3.713366in}{2.734986in}}%
\pgfpathlineto{\pgfqpoint{3.727238in}{2.801916in}}%
\pgfpathlineto{\pgfqpoint{3.727238in}{2.801916in}}%
\pgfusepath{stroke}%
\end{pgfscope}%
\begin{pgfscope}%
\pgfpathrectangle{\pgfqpoint{0.647939in}{0.492442in}}{\pgfqpoint{3.079299in}{3.079299in}}%
\pgfusepath{clip}%
\pgfsetbuttcap%
\pgfsetroundjoin%
\pgfsetlinewidth{0.301125pt}%
\definecolor{currentstroke}{rgb}{0.500000,0.500000,0.500000}%
\pgfsetstrokecolor{currentstroke}%
\pgfsetstrokeopacity{0.300000}%
\pgfsetdash{}{0pt}%
\pgfpathmoveto{\pgfqpoint{0.647939in}{2.402073in}}%
\pgfpathlineto{\pgfqpoint{0.694920in}{2.409040in}}%
\pgfpathlineto{\pgfqpoint{0.762486in}{2.419837in}}%
\pgfpathlineto{\pgfqpoint{0.829791in}{2.432161in}}%
\pgfpathlineto{\pgfqpoint{0.896799in}{2.446009in}}%
\pgfpathlineto{\pgfqpoint{0.963487in}{2.461327in}}%
\pgfpathlineto{\pgfqpoint{1.029845in}{2.478019in}}%
\pgfpathlineto{\pgfqpoint{1.095880in}{2.495947in}}%
\pgfpathlineto{\pgfqpoint{1.161621in}{2.514932in}}%
\pgfpathlineto{\pgfqpoint{1.227113in}{2.534757in}}%
\pgfpathlineto{\pgfqpoint{1.292426in}{2.555169in}}%
\pgfpathlineto{\pgfqpoint{1.357642in}{2.575889in}}%
\pgfpathlineto{\pgfqpoint{1.422857in}{2.596613in}}%
\pgfpathlineto{\pgfqpoint{1.488172in}{2.617021in}}%
\pgfpathlineto{\pgfqpoint{1.553685in}{2.636778in}}%
\pgfpathlineto{\pgfqpoint{1.619483in}{2.655557in}}%
\pgfpathlineto{\pgfqpoint{1.685635in}{2.673046in}}%
\pgfpathlineto{\pgfqpoint{1.752178in}{2.688971in}}%
\pgfpathlineto{\pgfqpoint{1.819120in}{2.703122in}}%
\pgfpathlineto{\pgfqpoint{1.886433in}{2.715381in}}%
\pgfpathlineto{\pgfqpoint{1.954064in}{2.725759in}}%
\pgfpathlineto{\pgfqpoint{2.021934in}{2.734444in}}%
\pgfpathlineto{\pgfqpoint{2.089962in}{2.741821in}}%
\pgfpathlineto{\pgfqpoint{2.158065in}{2.748486in}}%
\pgfpathlineto{\pgfqpoint{2.226157in}{2.755263in}}%
\pgfpathlineto{\pgfqpoint{2.294119in}{2.763191in}}%
\pgfpathlineto{\pgfqpoint{2.361750in}{2.773487in}}%
\pgfpathlineto{\pgfqpoint{2.428710in}{2.787401in}}%
\pgfpathlineto{\pgfqpoint{2.494507in}{2.805969in}}%
\pgfpathlineto{\pgfqpoint{2.558586in}{2.829761in}}%
\pgfpathlineto{\pgfqpoint{2.620486in}{2.858757in}}%
\pgfpathlineto{\pgfqpoint{2.679985in}{2.892428in}}%
\pgfpathlineto{\pgfqpoint{2.737131in}{2.929982in}}%
\pgfpathlineto{\pgfqpoint{2.792168in}{2.970592in}}%
\pgfpathlineto{\pgfqpoint{2.845432in}{3.013517in}}%
\pgfpathlineto{\pgfqpoint{2.897276in}{3.058149in}}%
\pgfpathlineto{\pgfqpoint{2.948039in}{3.104019in}}%
\pgfpathlineto{\pgfqpoint{2.998021in}{3.150743in}}%
\pgfpathlineto{\pgfqpoint{3.047484in}{3.198018in}}%
\pgfpathlineto{\pgfqpoint{3.096654in}{3.245601in}}%
\pgfpathlineto{\pgfqpoint{3.145734in}{3.293278in}}%
\pgfpathlineto{\pgfqpoint{3.194898in}{3.340869in}}%
\pgfpathlineto{\pgfqpoint{3.244312in}{3.388201in}}%
\pgfpathlineto{\pgfqpoint{3.294122in}{3.435116in}}%
\pgfpathlineto{\pgfqpoint{3.344468in}{3.481455in}}%
\pgfpathlineto{\pgfqpoint{3.395488in}{3.527050in}}%
\pgfpathlineto{\pgfqpoint{3.447302in}{3.571741in}}%
\pgfpathlineto{\pgfqpoint{3.447302in}{3.571741in}}%
\pgfusepath{stroke}%
\end{pgfscope}%
\begin{pgfscope}%
\pgfpathrectangle{\pgfqpoint{0.647939in}{0.492442in}}{\pgfqpoint{3.079299in}{3.079299in}}%
\pgfusepath{clip}%
\pgfsetbuttcap%
\pgfsetroundjoin%
\pgfsetlinewidth{0.301125pt}%
\definecolor{currentstroke}{rgb}{0.500000,0.500000,0.500000}%
\pgfsetstrokecolor{currentstroke}%
\pgfsetstrokeopacity{0.300000}%
\pgfsetdash{}{0pt}%
\pgfpathmoveto{\pgfqpoint{0.647939in}{2.723291in}}%
\pgfpathlineto{\pgfqpoint{0.692014in}{2.729397in}}%
\pgfpathlineto{\pgfqpoint{0.759692in}{2.739475in}}%
\pgfpathlineto{\pgfqpoint{0.827151in}{2.750929in}}%
\pgfpathlineto{\pgfqpoint{0.894366in}{2.763741in}}%
\pgfpathlineto{\pgfqpoint{0.961322in}{2.777848in}}%
\pgfpathlineto{\pgfqpoint{1.028017in}{2.793144in}}%
\pgfpathlineto{\pgfqpoint{1.094464in}{2.809484in}}%
\pgfpathlineto{\pgfqpoint{1.160694in}{2.826686in}}%
\pgfpathlineto{\pgfqpoint{1.226753in}{2.844535in}}%
\pgfpathlineto{\pgfqpoint{1.292703in}{2.862786in}}%
\pgfpathlineto{\pgfqpoint{1.358616in}{2.881167in}}%
\pgfpathlineto{\pgfqpoint{1.424573in}{2.899393in}}%
\pgfpathlineto{\pgfqpoint{1.490654in}{2.917163in}}%
\pgfpathlineto{\pgfqpoint{1.556932in}{2.934175in}}%
\pgfpathlineto{\pgfqpoint{1.623468in}{2.950144in}}%
\pgfpathlineto{\pgfqpoint{1.690302in}{2.964811in}}%
\pgfpathlineto{\pgfqpoint{1.757449in}{2.977963in}}%
\pgfpathlineto{\pgfqpoint{1.824900in}{2.989454in}}%
\pgfpathlineto{\pgfqpoint{1.892621in}{2.999232in}}%
\pgfpathlineto{\pgfqpoint{1.960560in}{3.007366in}}%
\pgfpathlineto{\pgfqpoint{2.028657in}{3.014062in}}%
\pgfpathlineto{\pgfqpoint{2.096854in}{3.019668in}}%
\pgfpathlineto{\pgfqpoint{2.165098in}{3.024684in}}%
\pgfpathlineto{\pgfqpoint{2.233338in}{3.029756in}}%
\pgfpathlineto{\pgfqpoint{2.301507in}{3.035678in}}%
\pgfpathlineto{\pgfqpoint{2.369495in}{3.043338in}}%
\pgfpathlineto{\pgfqpoint{2.437123in}{3.053635in}}%
\pgfpathlineto{\pgfqpoint{2.504122in}{3.067385in}}%
\pgfpathlineto{\pgfqpoint{2.570150in}{3.085194in}}%
\pgfpathlineto{\pgfqpoint{2.634845in}{3.107353in}}%
\pgfpathlineto{\pgfqpoint{2.697915in}{3.133797in}}%
\pgfpathlineto{\pgfqpoint{2.759197in}{3.164168in}}%
\pgfpathlineto{\pgfqpoint{2.818685in}{3.197930in}}%
\pgfpathlineto{\pgfqpoint{2.876505in}{3.234484in}}%
\pgfpathlineto{\pgfqpoint{2.932866in}{3.273256in}}%
\pgfpathlineto{\pgfqpoint{2.988020in}{3.313736in}}%
\pgfpathlineto{\pgfqpoint{3.042227in}{3.355488in}}%
\pgfpathlineto{\pgfqpoint{3.095739in}{3.398130in}}%
\pgfpathlineto{\pgfqpoint{3.148796in}{3.441336in}}%
\pgfpathlineto{\pgfqpoint{3.201615in}{3.484837in}}%
\pgfpathlineto{\pgfqpoint{3.254397in}{3.528383in}}%
\pgfpathlineto{\pgfqpoint{3.307334in}{3.571741in}}%
\pgfpathlineto{\pgfqpoint{3.307334in}{3.571741in}}%
\pgfusepath{stroke}%
\end{pgfscope}%
\begin{pgfscope}%
\pgfpathrectangle{\pgfqpoint{0.647939in}{0.492442in}}{\pgfqpoint{3.079299in}{3.079299in}}%
\pgfusepath{clip}%
\pgfsetbuttcap%
\pgfsetroundjoin%
\pgfsetlinewidth{0.301125pt}%
\definecolor{currentstroke}{rgb}{0.500000,0.500000,0.500000}%
\pgfsetstrokecolor{currentstroke}%
\pgfsetstrokeopacity{0.300000}%
\pgfsetdash{}{0pt}%
\pgfpathmoveto{\pgfqpoint{0.647939in}{2.923024in}}%
\pgfpathlineto{\pgfqpoint{0.683534in}{2.927693in}}%
\pgfpathlineto{\pgfqpoint{0.751289in}{2.937243in}}%
\pgfpathlineto{\pgfqpoint{0.818850in}{2.948086in}}%
\pgfpathlineto{\pgfqpoint{0.886194in}{2.960200in}}%
\pgfpathlineto{\pgfqpoint{0.953310in}{2.973525in}}%
\pgfpathlineto{\pgfqpoint{1.020197in}{2.987958in}}%
\pgfpathlineto{\pgfqpoint{1.086869in}{3.003357in}}%
\pgfpathlineto{\pgfqpoint{1.153356in}{3.019538in}}%
\pgfpathlineto{\pgfqpoint{1.219702in}{3.036292in}}%
\pgfpathlineto{\pgfqpoint{1.285963in}{3.053381in}}%
\pgfpathlineto{\pgfqpoint{1.352203in}{3.070546in}}%
\pgfpathlineto{\pgfqpoint{1.418495in}{3.087513in}}%
\pgfpathlineto{\pgfqpoint{1.484907in}{3.104000in}}%
\pgfpathlineto{\pgfqpoint{1.551503in}{3.119727in}}%
\pgfpathlineto{\pgfqpoint{1.618331in}{3.134427in}}%
\pgfpathlineto{\pgfqpoint{1.685423in}{3.147865in}}%
\pgfpathlineto{\pgfqpoint{1.752788in}{3.159856in}}%
\pgfpathlineto{\pgfqpoint{1.820413in}{3.170279in}}%
\pgfpathlineto{\pgfqpoint{1.888265in}{3.179100in}}%
\pgfpathlineto{\pgfqpoint{1.956301in}{3.186388in}}%
\pgfpathlineto{\pgfqpoint{2.024467in}{3.192339in}}%
\pgfpathlineto{\pgfqpoint{2.092717in}{3.197272in}}%
\pgfpathlineto{\pgfqpoint{2.161006in}{3.201628in}}%
\pgfpathlineto{\pgfqpoint{2.229298in}{3.205965in}}%
\pgfpathlineto{\pgfqpoint{2.297543in}{3.210944in}}%
\pgfpathlineto{\pgfqpoint{2.365668in}{3.217314in}}%
\pgfpathlineto{\pgfqpoint{2.433550in}{3.225840in}}%
\pgfpathlineto{\pgfqpoint{2.501000in}{3.237231in}}%
\pgfpathlineto{\pgfqpoint{2.567770in}{3.252064in}}%
\pgfpathlineto{\pgfqpoint{2.633577in}{3.270699in}}%
\pgfpathlineto{\pgfqpoint{2.698154in}{3.293232in}}%
\pgfpathlineto{\pgfqpoint{2.761307in}{3.319505in}}%
\pgfpathlineto{\pgfqpoint{2.822949in}{3.349163in}}%
\pgfpathlineto{\pgfqpoint{2.883105in}{3.381737in}}%
\pgfpathlineto{\pgfqpoint{2.941895in}{3.416725in}}%
\pgfpathlineto{\pgfqpoint{2.999499in}{3.453639in}}%
\pgfpathlineto{\pgfqpoint{3.056130in}{3.492034in}}%
\pgfpathlineto{\pgfqpoint{3.112011in}{3.531516in}}%
\pgfpathlineto{\pgfqpoint{3.167366in}{3.571741in}}%
\pgfpathlineto{\pgfqpoint{3.167366in}{3.571741in}}%
\pgfusepath{stroke}%
\end{pgfscope}%
\begin{pgfscope}%
\pgfpathrectangle{\pgfqpoint{0.647939in}{0.492442in}}{\pgfqpoint{3.079299in}{3.079299in}}%
\pgfusepath{clip}%
\pgfsetbuttcap%
\pgfsetroundjoin%
\pgfsetlinewidth{0.301125pt}%
\definecolor{currentstroke}{rgb}{0.500000,0.500000,0.500000}%
\pgfsetstrokecolor{currentstroke}%
\pgfsetstrokeopacity{0.300000}%
\pgfsetdash{}{0pt}%
\pgfpathmoveto{\pgfqpoint{0.647939in}{3.063585in}}%
\pgfpathlineto{\pgfqpoint{0.704021in}{3.071079in}}%
\pgfpathlineto{\pgfqpoint{0.771757in}{3.080768in}}%
\pgfpathlineto{\pgfqpoint{0.839305in}{3.091696in}}%
\pgfpathlineto{\pgfqpoint{0.906648in}{3.103821in}}%
\pgfpathlineto{\pgfqpoint{0.973781in}{3.117064in}}%
\pgfpathlineto{\pgfqpoint{1.040708in}{3.131311in}}%
\pgfpathlineto{\pgfqpoint{1.107449in}{3.146408in}}%
\pgfpathlineto{\pgfqpoint{1.174038in}{3.162168in}}%
\pgfpathlineto{\pgfqpoint{1.240519in}{3.178375in}}%
\pgfpathlineto{\pgfqpoint{1.306952in}{3.194784in}}%
\pgfpathlineto{\pgfqpoint{1.373398in}{3.211135in}}%
\pgfpathlineto{\pgfqpoint{1.439924in}{3.227159in}}%
\pgfpathlineto{\pgfqpoint{1.506589in}{3.242588in}}%
\pgfpathlineto{\pgfqpoint{1.573446in}{3.257159in}}%
\pgfpathlineto{\pgfqpoint{1.640532in}{3.270634in}}%
\pgfpathlineto{\pgfqpoint{1.707866in}{3.282807in}}%
\pgfpathlineto{\pgfqpoint{1.775444in}{3.293531in}}%
\pgfpathlineto{\pgfqpoint{1.843247in}{3.302730in}}%
\pgfpathlineto{\pgfqpoint{1.911238in}{3.310417in}}%
\pgfpathlineto{\pgfqpoint{1.979374in}{3.316703in}}%
\pgfpathlineto{\pgfqpoint{2.047609in}{3.321810in}}%
\pgfpathlineto{\pgfqpoint{2.115904in}{3.326084in}}%
\pgfpathlineto{\pgfqpoint{2.184222in}{3.329979in}}%
\pgfpathlineto{\pgfqpoint{2.252529in}{3.334041in}}%
\pgfpathlineto{\pgfqpoint{2.320784in}{3.338892in}}%
\pgfpathlineto{\pgfqpoint{2.388915in}{3.345202in}}%
\pgfpathlineto{\pgfqpoint{2.456807in}{3.353654in}}%
\pgfpathlineto{\pgfqpoint{2.524290in}{3.364871in}}%
\pgfpathlineto{\pgfqpoint{2.591144in}{3.379338in}}%
\pgfpathlineto{\pgfqpoint{2.657127in}{3.397346in}}%
\pgfpathlineto{\pgfqpoint{2.722017in}{3.418965in}}%
\pgfpathlineto{\pgfqpoint{2.785650in}{3.444053in}}%
\pgfpathlineto{\pgfqpoint{2.847950in}{3.472300in}}%
\pgfpathlineto{\pgfqpoint{2.908933in}{3.503297in}}%
\pgfpathlineto{\pgfqpoint{2.968698in}{3.536593in}}%
\pgfpathlineto{\pgfqpoint{3.027398in}{3.571741in}}%
\pgfpathlineto{\pgfqpoint{3.027398in}{3.571741in}}%
\pgfusepath{stroke}%
\end{pgfscope}%
\begin{pgfscope}%
\pgfpathrectangle{\pgfqpoint{0.647939in}{0.492442in}}{\pgfqpoint{3.079299in}{3.079299in}}%
\pgfusepath{clip}%
\pgfsetbuttcap%
\pgfsetroundjoin%
\pgfsetlinewidth{0.301125pt}%
\definecolor{currentstroke}{rgb}{0.500000,0.500000,0.500000}%
\pgfsetstrokecolor{currentstroke}%
\pgfsetstrokeopacity{0.300000}%
\pgfsetdash{}{0pt}%
\pgfpathmoveto{\pgfqpoint{0.647939in}{3.164260in}}%
\pgfpathlineto{\pgfqpoint{0.674952in}{3.167597in}}%
\pgfpathlineto{\pgfqpoint{0.742784in}{3.176594in}}%
\pgfpathlineto{\pgfqpoint{0.810444in}{3.186801in}}%
\pgfpathlineto{\pgfqpoint{0.877916in}{3.198188in}}%
\pgfpathlineto{\pgfqpoint{0.945190in}{3.210693in}}%
\pgfpathlineto{\pgfqpoint{1.012267in}{3.224214in}}%
\pgfpathlineto{\pgfqpoint{1.079162in}{3.238615in}}%
\pgfpathlineto{\pgfqpoint{1.145902in}{3.253722in}}%
\pgfpathlineto{\pgfqpoint{1.212527in}{3.269329in}}%
\pgfpathlineto{\pgfqpoint{1.279088in}{3.285205in}}%
\pgfpathlineto{\pgfqpoint{1.345645in}{3.301103in}}%
\pgfpathlineto{\pgfqpoint{1.412257in}{3.316764in}}%
\pgfpathlineto{\pgfqpoint{1.478984in}{3.331926in}}%
\pgfpathlineto{\pgfqpoint{1.545878in}{3.346331in}}%
\pgfpathlineto{\pgfqpoint{1.612978in}{3.359739in}}%
\pgfpathlineto{\pgfqpoint{1.680307in}{3.371939in}}%
\pgfpathlineto{\pgfqpoint{1.747868in}{3.382772in}}%
\pgfpathlineto{\pgfqpoint{1.815648in}{3.392141in}}%
\pgfpathlineto{\pgfqpoint{1.883616in}{3.400029in}}%
\pgfpathlineto{\pgfqpoint{1.951733in}{3.406507in}}%
\pgfpathlineto{\pgfqpoint{2.019958in}{3.411756in}}%
\pgfpathlineto{\pgfqpoint{2.088250in}{3.416064in}}%
\pgfpathlineto{\pgfqpoint{2.156576in}{3.419818in}}%
\pgfpathlineto{\pgfqpoint{2.224905in}{3.423498in}}%
\pgfpathlineto{\pgfqpoint{2.293206in}{3.427657in}}%
\pgfpathlineto{\pgfqpoint{2.361429in}{3.432914in}}%
\pgfpathlineto{\pgfqpoint{2.429491in}{3.439909in}}%
\pgfpathlineto{\pgfqpoint{2.497263in}{3.449249in}}%
\pgfpathlineto{\pgfqpoint{2.564574in}{3.461448in}}%
\pgfpathlineto{\pgfqpoint{2.631214in}{3.476879in}}%
\pgfpathlineto{\pgfqpoint{2.696969in}{3.495723in}}%
\pgfpathlineto{\pgfqpoint{2.761657in}{3.517966in}}%
\pgfpathlineto{\pgfqpoint{2.825156in}{3.543414in}}%
\pgfpathlineto{\pgfqpoint{2.887429in}{3.571741in}}%
\pgfpathlineto{\pgfqpoint{2.887429in}{3.571741in}}%
\pgfusepath{stroke}%
\end{pgfscope}%
\begin{pgfscope}%
\pgfpathrectangle{\pgfqpoint{0.647939in}{0.492442in}}{\pgfqpoint{3.079299in}{3.079299in}}%
\pgfusepath{clip}%
\pgfsetbuttcap%
\pgfsetroundjoin%
\pgfsetlinewidth{0.301125pt}%
\definecolor{currentstroke}{rgb}{0.500000,0.500000,0.500000}%
\pgfsetstrokecolor{currentstroke}%
\pgfsetstrokeopacity{0.300000}%
\pgfsetdash{}{0pt}%
\pgfpathmoveto{\pgfqpoint{0.647939in}{3.258616in}}%
\pgfpathlineto{\pgfqpoint{0.652573in}{3.259153in}}%
\pgfpathlineto{\pgfqpoint{0.720474in}{3.267616in}}%
\pgfpathlineto{\pgfqpoint{0.788217in}{3.277259in}}%
\pgfpathlineto{\pgfqpoint{0.855784in}{3.288064in}}%
\pgfpathlineto{\pgfqpoint{0.923165in}{3.299979in}}%
\pgfpathlineto{\pgfqpoint{0.990358in}{3.312915in}}%
\pgfpathlineto{\pgfqpoint{1.057375in}{3.326740in}}%
\pgfpathlineto{\pgfqpoint{1.124238in}{3.341289in}}%
\pgfpathlineto{\pgfqpoint{1.190984in}{3.356371in}}%
\pgfpathlineto{\pgfqpoint{1.257657in}{3.371769in}}%
\pgfpathlineto{\pgfqpoint{1.324313in}{3.387245in}}%
\pgfpathlineto{\pgfqpoint{1.391009in}{3.402549in}}%
\pgfpathlineto{\pgfqpoint{1.457801in}{3.417422in}}%
\pgfpathlineto{\pgfqpoint{1.524741in}{3.431611in}}%
\pgfpathlineto{\pgfqpoint{1.591870in}{3.444875in}}%
\pgfpathlineto{\pgfqpoint{1.659211in}{3.457007in}}%
\pgfpathlineto{\pgfqpoint{1.726773in}{3.467840in}}%
\pgfpathlineto{\pgfqpoint{1.794545in}{3.477264in}}%
\pgfpathlineto{\pgfqpoint{1.862503in}{3.485242in}}%
\pgfpathlineto{\pgfqpoint{1.930611in}{3.491827in}}%
\pgfpathlineto{\pgfqpoint{1.998828in}{3.497167in}}%
\pgfpathlineto{\pgfqpoint{2.067118in}{3.501504in}}%
\pgfpathlineto{\pgfqpoint{2.135448in}{3.505178in}}%
\pgfpathlineto{\pgfqpoint{2.203790in}{3.508616in}}%
\pgfpathlineto{\pgfqpoint{2.272117in}{3.512334in}}%
\pgfpathlineto{\pgfqpoint{2.340391in}{3.516903in}}%
\pgfpathlineto{\pgfqpoint{2.408548in}{3.522912in}}%
\pgfpathlineto{\pgfqpoint{2.476493in}{3.530940in}}%
\pgfpathlineto{\pgfqpoint{2.544086in}{3.541506in}}%
\pgfpathlineto{\pgfqpoint{2.611147in}{3.555022in}}%
\pgfpathlineto{\pgfqpoint{2.677477in}{3.571741in}}%
\pgfpathlineto{\pgfqpoint{2.677477in}{3.571741in}}%
\pgfusepath{stroke}%
\end{pgfscope}%
\begin{pgfscope}%
\pgfpathrectangle{\pgfqpoint{0.647939in}{0.492442in}}{\pgfqpoint{3.079299in}{3.079299in}}%
\pgfusepath{clip}%
\pgfsetbuttcap%
\pgfsetroundjoin%
\pgfsetlinewidth{0.301125pt}%
\definecolor{currentstroke}{rgb}{0.500000,0.500000,0.500000}%
\pgfsetstrokecolor{currentstroke}%
\pgfsetstrokeopacity{0.300000}%
\pgfsetdash{}{0pt}%
\pgfpathmoveto{\pgfqpoint{1.989731in}{3.539439in}}%
\pgfpathlineto{\pgfqpoint{2.058020in}{3.543782in}}%
\pgfpathlineto{\pgfqpoint{2.126351in}{3.547429in}}%
\pgfpathlineto{\pgfqpoint{2.194698in}{3.550786in}}%
\pgfpathlineto{\pgfqpoint{2.263034in}{3.554338in}}%
\pgfpathlineto{\pgfqpoint{2.331326in}{3.558629in}}%
\pgfpathlineto{\pgfqpoint{2.399519in}{3.564241in}}%
\pgfpathlineto{\pgfqpoint{2.467525in}{3.571741in}}%
\pgfpathlineto{\pgfqpoint{2.467525in}{3.571741in}}%
\pgfusepath{stroke}%
\end{pgfscope}%
\begin{pgfscope}%
\pgfpathrectangle{\pgfqpoint{0.647939in}{0.492442in}}{\pgfqpoint{3.079299in}{3.079299in}}%
\pgfusepath{clip}%
\pgfsetbuttcap%
\pgfsetroundjoin%
\pgfsetlinewidth{0.301125pt}%
\definecolor{currentstroke}{rgb}{0.500000,0.500000,0.500000}%
\pgfsetstrokecolor{currentstroke}%
\pgfsetstrokeopacity{0.300000}%
\pgfsetdash{}{0pt}%
\pgfpathmoveto{\pgfqpoint{0.647939in}{3.346443in}}%
\pgfpathlineto{\pgfqpoint{0.696498in}{3.352527in}}%
\pgfpathlineto{\pgfqpoint{0.764318in}{3.361614in}}%
\pgfpathlineto{\pgfqpoint{0.831975in}{3.371847in}}%
\pgfpathlineto{\pgfqpoint{0.899456in}{3.383183in}}%
\pgfpathlineto{\pgfqpoint{0.966758in}{3.395539in}}%
\pgfpathlineto{\pgfqpoint{1.033888in}{3.408799in}}%
\pgfpathlineto{\pgfqpoint{1.100866in}{3.422814in}}%
\pgfpathlineto{\pgfqpoint{1.167721in}{3.437402in}}%
\pgfpathlineto{\pgfqpoint{1.234495in}{3.452359in}}%
\pgfpathlineto{\pgfqpoint{1.301238in}{3.467455in}}%
\pgfpathlineto{\pgfqpoint{1.368005in}{3.482444in}}%
\pgfpathlineto{\pgfqpoint{1.434851in}{3.497074in}}%
\pgfpathlineto{\pgfqpoint{1.501826in}{3.511096in}}%
\pgfpathlineto{\pgfqpoint{1.568972in}{3.524275in}}%
\pgfpathlineto{\pgfqpoint{1.636315in}{3.536400in}}%
\pgfpathlineto{\pgfqpoint{1.703867in}{3.547296in}}%
\pgfpathlineto{\pgfqpoint{1.771623in}{3.556840in}}%
\pgfpathlineto{\pgfqpoint{1.839562in}{3.564980in}}%
\pgfpathlineto{\pgfqpoint{1.907652in}{3.571741in}}%
\pgfpathlineto{\pgfqpoint{1.907652in}{3.571741in}}%
\pgfusepath{stroke}%
\end{pgfscope}%
\begin{pgfscope}%
\pgfpathrectangle{\pgfqpoint{0.647939in}{0.492442in}}{\pgfqpoint{3.079299in}{3.079299in}}%
\pgfusepath{clip}%
\pgfsetbuttcap%
\pgfsetroundjoin%
\pgfsetlinewidth{0.301125pt}%
\definecolor{currentstroke}{rgb}{0.500000,0.500000,0.500000}%
\pgfsetstrokecolor{currentstroke}%
\pgfsetstrokeopacity{0.300000}%
\pgfsetdash{}{0pt}%
\pgfpathmoveto{\pgfqpoint{0.647939in}{3.427914in}}%
\pgfpathlineto{\pgfqpoint{0.678581in}{3.431567in}}%
\pgfpathlineto{\pgfqpoint{0.746456in}{3.440234in}}%
\pgfpathlineto{\pgfqpoint{0.814179in}{3.450023in}}%
\pgfpathlineto{\pgfqpoint{0.881735in}{3.460901in}}%
\pgfpathlineto{\pgfqpoint{0.949119in}{3.472797in}}%
\pgfpathlineto{\pgfqpoint{1.016337in}{3.485606in}}%
\pgfpathlineto{\pgfqpoint{1.083404in}{3.499187in}}%
\pgfpathlineto{\pgfqpoint{1.150347in}{3.513364in}}%
\pgfpathlineto{\pgfqpoint{1.217207in}{3.527935in}}%
\pgfpathlineto{\pgfqpoint{1.284028in}{3.542680in}}%
\pgfpathlineto{\pgfqpoint{1.350863in}{3.557363in}}%
\pgfpathlineto{\pgfqpoint{1.417764in}{3.571741in}}%
\pgfpathlineto{\pgfqpoint{1.417764in}{3.571741in}}%
\pgfusepath{stroke}%
\end{pgfscope}%
\begin{pgfscope}%
\pgfpathrectangle{\pgfqpoint{0.647939in}{0.492442in}}{\pgfqpoint{3.079299in}{3.079299in}}%
\pgfusepath{clip}%
\pgfsetbuttcap%
\pgfsetroundjoin%
\pgfsetlinewidth{0.301125pt}%
\definecolor{currentstroke}{rgb}{0.500000,0.500000,0.500000}%
\pgfsetstrokecolor{currentstroke}%
\pgfsetstrokeopacity{0.300000}%
\pgfsetdash{}{0pt}%
\pgfpathmoveto{\pgfqpoint{0.647939in}{3.504811in}}%
\pgfpathlineto{\pgfqpoint{0.662631in}{3.506483in}}%
\pgfpathlineto{\pgfqpoint{0.730553in}{3.514777in}}%
\pgfpathlineto{\pgfqpoint{0.798330in}{3.524179in}}%
\pgfpathlineto{\pgfqpoint{0.865949in}{3.534660in}}%
\pgfpathlineto{\pgfqpoint{0.933404in}{3.546154in}}%
\pgfpathlineto{\pgfqpoint{1.000698in}{3.558559in}}%
\pgfpathlineto{\pgfqpoint{1.067843in}{3.571741in}}%
\pgfpathlineto{\pgfqpoint{1.067843in}{3.571741in}}%
\pgfusepath{stroke}%
\end{pgfscope}%
\begin{pgfscope}%
\pgfpathrectangle{\pgfqpoint{0.647939in}{0.492442in}}{\pgfqpoint{3.079299in}{3.079299in}}%
\pgfusepath{clip}%
\pgfsetbuttcap%
\pgfsetroundjoin%
\pgfsetlinewidth{0.301125pt}%
\definecolor{currentstroke}{rgb}{0.500000,0.500000,0.500000}%
\pgfsetstrokecolor{currentstroke}%
\pgfsetstrokeopacity{0.300000}%
\pgfsetdash{}{0pt}%
\pgfpathmoveto{\pgfqpoint{0.647939in}{2.801916in}}%
\pgfpathlineto{\pgfqpoint{0.647939in}{2.801916in}}%
\pgfpathlineto{\pgfqpoint{0.715761in}{2.810984in}}%
\pgfpathlineto{\pgfqpoint{0.783391in}{2.821386in}}%
\pgfpathlineto{\pgfqpoint{0.850801in}{2.833126in}}%
\pgfpathlineto{\pgfqpoint{0.917972in}{2.846167in}}%
\pgfpathlineto{\pgfqpoint{0.984894in}{2.860432in}}%
\pgfpathlineto{\pgfqpoint{1.051572in}{2.875804in}}%
\pgfpathlineto{\pgfqpoint{1.118025in}{2.892121in}}%
\pgfpathlineto{\pgfqpoint{1.184291in}{2.909184in}}%
\pgfpathlineto{\pgfqpoint{1.250422in}{2.926767in}}%
\pgfpathlineto{\pgfqpoint{1.316481in}{2.944619in}}%
\pgfpathlineto{\pgfqpoint{1.382542in}{2.962466in}}%
\pgfpathlineto{\pgfqpoint{1.448680in}{2.980021in}}%
\pgfpathlineto{\pgfqpoint{1.514969in}{2.996993in}}%
\pgfpathlineto{\pgfqpoint{1.581475in}{3.013093in}}%
\pgfpathlineto{\pgfqpoint{1.648246in}{3.028050in}}%
\pgfpathlineto{\pgfqpoint{1.715308in}{3.041633in}}%
\pgfpathlineto{\pgfqpoint{1.782665in}{3.053663in}}%
\pgfpathlineto{\pgfqpoint{1.850296in}{3.064041in}}%
\pgfpathlineto{\pgfqpoint{1.918161in}{3.072764in}}%
\pgfpathlineto{\pgfqpoint{1.986208in}{3.079948in}}%
\pgfpathlineto{\pgfqpoint{2.054379in}{3.085850in}}%
\pgfpathlineto{\pgfqpoint{2.122623in}{3.090869in}}%
\pgfpathlineto{\pgfqpoint{2.190892in}{3.095536in}}%
\pgfpathlineto{\pgfqpoint{2.259139in}{3.100511in}}%
\pgfpathlineto{\pgfqpoint{2.327295in}{3.106563in}}%
\pgfpathlineto{\pgfqpoint{2.395246in}{3.114538in}}%
\pgfpathlineto{\pgfqpoint{2.462805in}{3.125273in}}%
\pgfpathlineto{\pgfqpoint{2.529706in}{3.139484in}}%
\pgfpathlineto{\pgfqpoint{2.595630in}{3.157667in}}%
\pgfusepath{stroke}%
\end{pgfscope}%
\begin{pgfscope}%
\pgfpathrectangle{\pgfqpoint{0.647939in}{0.492442in}}{\pgfqpoint{3.079299in}{3.079299in}}%
\pgfusepath{clip}%
\pgfsetbuttcap%
\pgfsetroundjoin%
\pgfsetlinewidth{0.301125pt}%
\definecolor{currentstroke}{rgb}{0.500000,0.500000,0.500000}%
\pgfsetstrokecolor{currentstroke}%
\pgfsetstrokeopacity{0.300000}%
\pgfsetdash{}{0pt}%
\pgfpathmoveto{\pgfqpoint{0.647939in}{2.591964in}}%
\pgfpathlineto{\pgfqpoint{0.647939in}{2.591964in}}%
\pgfpathlineto{\pgfqpoint{0.715712in}{2.601384in}}%
\pgfpathlineto{\pgfqpoint{0.783273in}{2.612223in}}%
\pgfpathlineto{\pgfqpoint{0.850588in}{2.624493in}}%
\pgfpathlineto{\pgfqpoint{0.917632in}{2.638168in}}%
\pgfpathlineto{\pgfqpoint{0.984390in}{2.653178in}}%
\pgfpathlineto{\pgfqpoint{1.050863in}{2.669410in}}%
\pgfpathlineto{\pgfqpoint{1.117068in}{2.686707in}}%
\pgfpathlineto{\pgfqpoint{1.183041in}{2.704870in}}%
\pgfpathlineto{\pgfqpoint{1.248836in}{2.723670in}}%
\pgfpathlineto{\pgfqpoint{1.314522in}{2.742849in}}%
\pgfpathlineto{\pgfqpoint{1.380179in}{2.762126in}}%
\pgfpathlineto{\pgfqpoint{1.445895in}{2.781200in}}%
\pgfpathlineto{\pgfqpoint{1.511757in}{2.799763in}}%
\pgfpathlineto{\pgfqpoint{1.577844in}{2.817503in}}%
\pgfpathlineto{\pgfqpoint{1.644221in}{2.834121in}}%
\pgfpathlineto{\pgfqpoint{1.710929in}{2.849349in}}%
\pgfpathlineto{\pgfqpoint{1.777983in}{2.862969in}}%
\pgfpathlineto{\pgfqpoint{1.845366in}{2.874841in}}%
\pgfpathlineto{\pgfqpoint{1.913040in}{2.884930in}}%
\pgfpathlineto{\pgfqpoint{1.980946in}{2.893328in}}%
\pgfpathlineto{\pgfqpoint{2.049017in}{2.900283in}}%
\pgfpathlineto{\pgfqpoint{2.117187in}{2.906223in}}%
\pgfpathlineto{\pgfqpoint{2.185392in}{2.911754in}}%
\pgfpathlineto{\pgfqpoint{2.253565in}{2.917645in}}%
\pgfpathlineto{\pgfqpoint{2.321610in}{2.924815in}}%
\pgfpathlineto{\pgfqpoint{2.389366in}{2.934274in}}%
\pgfpathlineto{\pgfqpoint{2.456563in}{2.947037in}}%
\pgfpathlineto{\pgfqpoint{2.522822in}{2.963941in}}%
\pgfpathlineto{\pgfqpoint{2.587710in}{2.985472in}}%
\pgfpathlineto{\pgfqpoint{2.650848in}{3.011689in}}%
\pgfpathlineto{\pgfqpoint{2.712004in}{3.042261in}}%
\pgfpathlineto{\pgfqpoint{2.771137in}{3.076606in}}%
\pgfpathlineto{\pgfqpoint{2.828374in}{3.114052in}}%
\pgfpathlineto{\pgfqpoint{2.883945in}{3.153946in}}%
\pgfpathlineto{\pgfqpoint{2.938131in}{3.195709in}}%
\pgfpathlineto{\pgfqpoint{2.991224in}{3.238858in}}%
\pgfpathlineto{\pgfqpoint{3.043501in}{3.283001in}}%
\pgfpathlineto{\pgfqpoint{3.095218in}{3.327802in}}%
\pgfusepath{stroke}%
\end{pgfscope}%
\begin{pgfscope}%
\pgfpathrectangle{\pgfqpoint{0.647939in}{0.492442in}}{\pgfqpoint{3.079299in}{3.079299in}}%
\pgfusepath{clip}%
\pgfsetbuttcap%
\pgfsetroundjoin%
\pgfsetlinewidth{0.301125pt}%
\definecolor{currentstroke}{rgb}{0.500000,0.500000,0.500000}%
\pgfsetstrokecolor{currentstroke}%
\pgfsetstrokeopacity{0.300000}%
\pgfsetdash{}{0pt}%
\pgfpathmoveto{\pgfqpoint{0.647939in}{2.521980in}}%
\pgfpathlineto{\pgfqpoint{0.647939in}{2.521980in}}%
\pgfpathlineto{\pgfqpoint{0.715695in}{2.531524in}}%
\pgfpathlineto{\pgfqpoint{0.783230in}{2.542516in}}%
\pgfpathlineto{\pgfqpoint{0.850511in}{2.554973in}}%
\pgfpathlineto{\pgfqpoint{0.917508in}{2.568873in}}%
\pgfpathlineto{\pgfqpoint{0.984206in}{2.584148in}}%
\pgfpathlineto{\pgfqpoint{1.050603in}{2.600688in}}%
\pgfpathlineto{\pgfqpoint{1.116715in}{2.618336in}}%
\pgfpathlineto{\pgfqpoint{1.182577in}{2.636895in}}%
\pgfpathlineto{\pgfqpoint{1.248245in}{2.656135in}}%
\pgfpathlineto{\pgfqpoint{1.313788in}{2.675796in}}%
\pgfpathlineto{\pgfqpoint{1.379290in}{2.695595in}}%
\pgfpathlineto{\pgfqpoint{1.444841in}{2.715229in}}%
\pgfpathlineto{\pgfqpoint{1.510534in}{2.734382in}}%
\pgfpathlineto{\pgfqpoint{1.576453in}{2.752735in}}%
\pgfusepath{stroke}%
\end{pgfscope}%
\begin{pgfscope}%
\pgfpathrectangle{\pgfqpoint{0.647939in}{0.492442in}}{\pgfqpoint{3.079299in}{3.079299in}}%
\pgfusepath{clip}%
\pgfsetbuttcap%
\pgfsetroundjoin%
\pgfsetlinewidth{0.301125pt}%
\definecolor{currentstroke}{rgb}{0.500000,0.500000,0.500000}%
\pgfsetstrokecolor{currentstroke}%
\pgfsetstrokeopacity{0.300000}%
\pgfsetdash{}{0pt}%
\pgfpathmoveto{\pgfqpoint{0.647939in}{2.312028in}}%
\pgfpathlineto{\pgfqpoint{0.647939in}{2.312028in}}%
\pgfpathlineto{\pgfqpoint{0.715638in}{2.321962in}}%
\pgfpathlineto{\pgfqpoint{0.783092in}{2.333441in}}%
\pgfpathlineto{\pgfqpoint{0.850258in}{2.346496in}}%
\pgfpathlineto{\pgfqpoint{0.917101in}{2.361114in}}%
\pgfpathlineto{\pgfqpoint{0.983598in}{2.377240in}}%
\pgfpathlineto{\pgfqpoint{1.049739in}{2.394771in}}%
\pgfpathlineto{\pgfqpoint{1.115536in}{2.413557in}}%
\pgfpathlineto{\pgfqpoint{1.181021in}{2.433404in}}%
\pgfpathlineto{\pgfqpoint{1.246250in}{2.454081in}}%
\pgfpathlineto{\pgfqpoint{1.311296in}{2.475328in}}%
\pgfpathlineto{\pgfqpoint{1.376250in}{2.496854in}}%
\pgfpathlineto{\pgfqpoint{1.441216in}{2.518348in}}%
\pgfpathlineto{\pgfqpoint{1.506299in}{2.539479in}}%
\pgfpathlineto{\pgfqpoint{1.571605in}{2.559909in}}%
\pgfpathlineto{\pgfqpoint{1.637226in}{2.579300in}}%
\pgfpathlineto{\pgfqpoint{1.703231in}{2.597332in}}%
\pgfpathlineto{\pgfqpoint{1.769661in}{2.613724in}}%
\pgfpathlineto{\pgfqpoint{1.836518in}{2.628267in}}%
\pgfpathlineto{\pgfqpoint{1.903770in}{2.640860in}}%
\pgfpathlineto{\pgfqpoint{1.971352in}{2.651543in}}%
\pgfpathlineto{\pgfqpoint{2.039181in}{2.660543in}}%
\pgfpathlineto{\pgfqpoint{2.107166in}{2.668305in}}%
\pgfpathlineto{\pgfqpoint{2.175212in}{2.675535in}}%
\pgfpathlineto{\pgfqpoint{2.243205in}{2.683219in}}%
\pgfpathlineto{\pgfqpoint{2.310978in}{2.692576in}}%
\pgfpathlineto{\pgfqpoint{2.378240in}{2.704975in}}%
\pgfpathlineto{\pgfqpoint{2.444524in}{2.721728in}}%
\pgfpathlineto{\pgfqpoint{2.509218in}{2.743774in}}%
\pgfusepath{stroke}%
\end{pgfscope}%
\begin{pgfscope}%
\pgfpathrectangle{\pgfqpoint{0.647939in}{0.492442in}}{\pgfqpoint{3.079299in}{3.079299in}}%
\pgfusepath{clip}%
\pgfsetbuttcap%
\pgfsetroundjoin%
\pgfsetlinewidth{0.301125pt}%
\definecolor{currentstroke}{rgb}{0.500000,0.500000,0.500000}%
\pgfsetstrokecolor{currentstroke}%
\pgfsetstrokeopacity{0.300000}%
\pgfsetdash{}{0pt}%
\pgfpathmoveto{\pgfqpoint{0.647939in}{2.242044in}}%
\pgfpathlineto{\pgfqpoint{0.647939in}{2.242044in}}%
\pgfpathlineto{\pgfqpoint{0.715618in}{2.252115in}}%
\pgfpathlineto{\pgfqpoint{0.783042in}{2.263766in}}%
\pgfpathlineto{\pgfqpoint{0.850166in}{2.277033in}}%
\pgfpathlineto{\pgfqpoint{0.916953in}{2.291907in}}%
\pgfpathlineto{\pgfqpoint{0.983374in}{2.308336in}}%
\pgfpathlineto{\pgfqpoint{1.049420in}{2.326222in}}%
\pgfpathlineto{\pgfqpoint{1.115098in}{2.345418in}}%
\pgfpathlineto{\pgfqpoint{1.180440in}{2.365730in}}%
\pgfpathlineto{\pgfqpoint{1.245501in}{2.386930in}}%
\pgfpathlineto{\pgfqpoint{1.310355in}{2.408756in}}%
\pgfpathlineto{\pgfqpoint{1.375095in}{2.430916in}}%
\pgfpathlineto{\pgfqpoint{1.439829in}{2.453098in}}%
\pgfpathlineto{\pgfqpoint{1.504669in}{2.474967in}}%
\pgfpathlineto{\pgfqpoint{1.569725in}{2.496178in}}%
\pgfpathlineto{\pgfqpoint{1.635099in}{2.516386in}}%
\pgfpathlineto{\pgfqpoint{1.700869in}{2.535256in}}%
\pgfpathlineto{\pgfqpoint{1.767085in}{2.552493in}}%
\pgfpathlineto{\pgfqpoint{1.833756in}{2.567865in}}%
\pgfpathlineto{\pgfqpoint{1.900854in}{2.581250in}}%
\pgfpathlineto{\pgfqpoint{1.968315in}{2.592671in}}%
\pgfpathlineto{\pgfqpoint{2.036051in}{2.602345in}}%
\pgfpathlineto{\pgfqpoint{2.103963in}{2.610723in}}%
\pgfpathlineto{\pgfqpoint{2.171944in}{2.618530in}}%
\pgfpathlineto{\pgfqpoint{2.239867in}{2.626811in}}%
\pgfusepath{stroke}%
\end{pgfscope}%
\begin{pgfscope}%
\pgfpathrectangle{\pgfqpoint{0.647939in}{0.492442in}}{\pgfqpoint{3.079299in}{3.079299in}}%
\pgfusepath{clip}%
\pgfsetbuttcap%
\pgfsetroundjoin%
\pgfsetlinewidth{0.301125pt}%
\definecolor{currentstroke}{rgb}{0.500000,0.500000,0.500000}%
\pgfsetstrokecolor{currentstroke}%
\pgfsetstrokeopacity{0.300000}%
\pgfsetdash{}{0pt}%
\pgfpathmoveto{\pgfqpoint{0.647939in}{2.102076in}}%
\pgfpathlineto{\pgfqpoint{0.647939in}{2.102076in}}%
\pgfpathlineto{\pgfqpoint{0.715574in}{2.112433in}}%
\pgfpathlineto{\pgfqpoint{0.782934in}{2.124444in}}%
\pgfpathlineto{\pgfqpoint{0.849969in}{2.138154in}}%
\pgfpathlineto{\pgfqpoint{0.916632in}{2.153566in}}%
\pgfpathlineto{\pgfqpoint{0.982891in}{2.170636in}}%
\pgfpathlineto{\pgfqpoint{1.048728in}{2.189273in}}%
\pgfpathlineto{\pgfqpoint{1.114145in}{2.209338in}}%
\pgfpathlineto{\pgfqpoint{1.179170in}{2.230643in}}%
\pgfpathlineto{\pgfqpoint{1.243855in}{2.252960in}}%
\pgfpathlineto{\pgfqpoint{1.308277in}{2.276027in}}%
\pgfpathlineto{\pgfqpoint{1.372533in}{2.299556in}}%
\pgfpathlineto{\pgfqpoint{1.436737in}{2.323229in}}%
\pgfpathlineto{\pgfqpoint{1.501012in}{2.346707in}}%
\pgfpathlineto{\pgfqpoint{1.565483in}{2.369635in}}%
\pgfpathlineto{\pgfqpoint{1.630270in}{2.391652in}}%
\pgfpathlineto{\pgfqpoint{1.695472in}{2.412402in}}%
\pgfpathlineto{\pgfqpoint{1.761158in}{2.431555in}}%
\pgfpathlineto{\pgfqpoint{1.827360in}{2.448838in}}%
\pgfpathlineto{\pgfqpoint{1.894061in}{2.464073in}}%
\pgfpathlineto{\pgfqpoint{1.961202in}{2.477243in}}%
\pgfpathlineto{\pgfqpoint{2.028686in}{2.488536in}}%
\pgfpathlineto{\pgfqpoint{2.096395in}{2.498412in}}%
\pgfpathlineto{\pgfqpoint{2.164197in}{2.507655in}}%
\pgfpathlineto{\pgfqpoint{2.231920in}{2.517432in}}%
\pgfpathlineto{\pgfqpoint{2.299293in}{2.529296in}}%
\pgfpathlineto{\pgfqpoint{2.365830in}{2.545036in}}%
\pgfpathlineto{\pgfqpoint{2.430780in}{2.566241in}}%
\pgfpathlineto{\pgfqpoint{2.493292in}{2.593727in}}%
\pgfpathlineto{\pgfqpoint{2.552780in}{2.627268in}}%
\pgfpathlineto{\pgfqpoint{2.609157in}{2.665880in}}%
\pgfpathlineto{\pgfqpoint{2.662732in}{2.708355in}}%
\pgfusepath{stroke}%
\end{pgfscope}%
\begin{pgfscope}%
\pgfpathrectangle{\pgfqpoint{0.647939in}{0.492442in}}{\pgfqpoint{3.079299in}{3.079299in}}%
\pgfusepath{clip}%
\pgfsetbuttcap%
\pgfsetroundjoin%
\pgfsetlinewidth{0.301125pt}%
\definecolor{currentstroke}{rgb}{0.500000,0.500000,0.500000}%
\pgfsetstrokecolor{currentstroke}%
\pgfsetstrokeopacity{0.300000}%
\pgfsetdash{}{0pt}%
\pgfpathmoveto{\pgfqpoint{0.647939in}{2.032092in}}%
\pgfpathlineto{\pgfqpoint{0.647939in}{2.032092in}}%
\pgfpathlineto{\pgfqpoint{0.715551in}{2.042598in}}%
\pgfpathlineto{\pgfqpoint{0.782877in}{2.054797in}}%
\pgfpathlineto{\pgfqpoint{0.849863in}{2.068739in}}%
\pgfpathlineto{\pgfqpoint{0.916460in}{2.084434in}}%
\pgfpathlineto{\pgfqpoint{0.982631in}{2.101843in}}%
\pgfpathlineto{\pgfqpoint{1.048353in}{2.120878in}}%
\pgfpathlineto{\pgfqpoint{1.113626in}{2.141405in}}%
\pgfpathlineto{\pgfqpoint{1.178475in}{2.163238in}}%
\pgfpathlineto{\pgfqpoint{1.242951in}{2.186151in}}%
\pgfusepath{stroke}%
\end{pgfscope}%
\begin{pgfscope}%
\pgfpathrectangle{\pgfqpoint{0.647939in}{0.492442in}}{\pgfqpoint{3.079299in}{3.079299in}}%
\pgfusepath{clip}%
\pgfsetbuttcap%
\pgfsetroundjoin%
\pgfsetlinewidth{0.301125pt}%
\definecolor{currentstroke}{rgb}{0.500000,0.500000,0.500000}%
\pgfsetstrokecolor{currentstroke}%
\pgfsetstrokeopacity{0.300000}%
\pgfsetdash{}{0pt}%
\pgfpathmoveto{\pgfqpoint{0.647939in}{1.962108in}}%
\pgfpathlineto{\pgfqpoint{0.647939in}{1.962108in}}%
\pgfpathlineto{\pgfqpoint{0.715527in}{1.972767in}}%
\pgfpathlineto{\pgfqpoint{0.782817in}{1.985160in}}%
\pgfpathlineto{\pgfqpoint{0.849752in}{1.999343in}}%
\pgfpathlineto{\pgfqpoint{0.916279in}{2.015331in}}%
\pgfpathlineto{\pgfqpoint{0.982355in}{2.033091in}}%
\pgfpathlineto{\pgfqpoint{1.047956in}{2.052540in}}%
\pgfpathlineto{\pgfqpoint{1.113075in}{2.073547in}}%
\pgfpathlineto{\pgfqpoint{1.177735in}{2.095931in}}%
\pgfpathlineto{\pgfqpoint{1.241985in}{2.119470in}}%
\pgfusepath{stroke}%
\end{pgfscope}%
\begin{pgfscope}%
\pgfpathrectangle{\pgfqpoint{0.647939in}{0.492442in}}{\pgfqpoint{3.079299in}{3.079299in}}%
\pgfusepath{clip}%
\pgfsetbuttcap%
\pgfsetroundjoin%
\pgfsetlinewidth{0.301125pt}%
\definecolor{currentstroke}{rgb}{0.500000,0.500000,0.500000}%
\pgfsetstrokecolor{currentstroke}%
\pgfsetstrokeopacity{0.300000}%
\pgfsetdash{}{0pt}%
\pgfpathmoveto{\pgfqpoint{0.647939in}{1.892124in}}%
\pgfpathlineto{\pgfqpoint{0.647939in}{1.892124in}}%
\pgfpathlineto{\pgfqpoint{0.715501in}{1.902941in}}%
\pgfpathlineto{\pgfqpoint{0.782754in}{1.915534in}}%
\pgfpathlineto{\pgfqpoint{0.849636in}{1.929965in}}%
\pgfpathlineto{\pgfqpoint{0.916088in}{1.946257in}}%
\pgfpathlineto{\pgfqpoint{0.982065in}{1.964381in}}%
\pgfpathlineto{\pgfqpoint{1.047535in}{1.984261in}}%
\pgfpathlineto{\pgfqpoint{1.112490in}{2.005768in}}%
\pgfpathlineto{\pgfqpoint{1.176948in}{2.028729in}}%
\pgfpathlineto{\pgfqpoint{1.240954in}{2.052922in}}%
\pgfpathlineto{\pgfqpoint{1.304584in}{2.078091in}}%
\pgfpathlineto{\pgfqpoint{1.367938in}{2.103950in}}%
\pgfpathlineto{\pgfqpoint{1.431138in}{2.130184in}}%
\pgfpathlineto{\pgfqpoint{1.494323in}{2.156453in}}%
\pgfpathlineto{\pgfqpoint{1.557641in}{2.182398in}}%
\pgfpathlineto{\pgfqpoint{1.621241in}{2.207641in}}%
\pgfpathlineto{\pgfqpoint{1.685259in}{2.231802in}}%
\pgfpathlineto{\pgfqpoint{1.749803in}{2.254509in}}%
\pgfpathlineto{\pgfqpoint{1.814946in}{2.275428in}}%
\pgfpathlineto{\pgfqpoint{1.880708in}{2.294305in}}%
\pgfpathlineto{\pgfqpoint{1.947052in}{2.311023in}}%
\pgfpathlineto{\pgfqpoint{2.013882in}{2.325688in}}%
\pgfpathlineto{\pgfqpoint{2.081053in}{2.338730in}}%
\pgfpathlineto{\pgfqpoint{2.148364in}{2.351038in}}%
\pgfpathlineto{\pgfqpoint{2.215538in}{2.364043in}}%
\pgfpathlineto{\pgfqpoint{2.282101in}{2.379733in}}%
\pgfpathlineto{\pgfqpoint{2.347229in}{2.400375in}}%
\pgfpathlineto{\pgfqpoint{2.409786in}{2.427670in}}%
\pgfpathlineto{\pgfqpoint{2.468831in}{2.461853in}}%
\pgfpathlineto{\pgfqpoint{2.524166in}{2.501803in}}%
\pgfusepath{stroke}%
\end{pgfscope}%
\begin{pgfscope}%
\pgfpathrectangle{\pgfqpoint{0.647939in}{0.492442in}}{\pgfqpoint{3.079299in}{3.079299in}}%
\pgfusepath{clip}%
\pgfsetbuttcap%
\pgfsetroundjoin%
\pgfsetlinewidth{0.301125pt}%
\definecolor{currentstroke}{rgb}{0.500000,0.500000,0.500000}%
\pgfsetstrokecolor{currentstroke}%
\pgfsetstrokeopacity{0.300000}%
\pgfsetdash{}{0pt}%
\pgfpathmoveto{\pgfqpoint{0.647939in}{1.822139in}}%
\pgfpathlineto{\pgfqpoint{0.647939in}{1.822139in}}%
\pgfpathlineto{\pgfqpoint{0.715475in}{1.833120in}}%
\pgfpathlineto{\pgfqpoint{0.782688in}{1.845919in}}%
\pgfpathlineto{\pgfqpoint{0.849513in}{1.860607in}}%
\pgfpathlineto{\pgfqpoint{0.915887in}{1.877213in}}%
\pgfpathlineto{\pgfqpoint{0.981758in}{1.895716in}}%
\pgfpathlineto{\pgfqpoint{1.047090in}{1.916044in}}%
\pgfpathlineto{\pgfqpoint{1.111869in}{1.938074in}}%
\pgfpathlineto{\pgfqpoint{1.176109in}{1.961635in}}%
\pgfpathlineto{\pgfqpoint{1.239852in}{1.986513in}}%
\pgfpathlineto{\pgfqpoint{1.303171in}{2.012453in}}%
\pgfpathlineto{\pgfqpoint{1.366167in}{2.039172in}}%
\pgfpathlineto{\pgfqpoint{1.428963in}{2.066356in}}%
\pgfpathlineto{\pgfqpoint{1.491704in}{2.093669in}}%
\pgfpathlineto{\pgfqpoint{1.554545in}{2.120751in}}%
\pgfpathlineto{\pgfqpoint{1.617643in}{2.147225in}}%
\pgfpathlineto{\pgfqpoint{1.681147in}{2.172704in}}%
\pgfpathlineto{\pgfqpoint{1.745184in}{2.196808in}}%
\pgfusepath{stroke}%
\end{pgfscope}%
\begin{pgfscope}%
\pgfpathrectangle{\pgfqpoint{0.647939in}{0.492442in}}{\pgfqpoint{3.079299in}{3.079299in}}%
\pgfusepath{clip}%
\pgfsetbuttcap%
\pgfsetroundjoin%
\pgfsetlinewidth{0.301125pt}%
\definecolor{currentstroke}{rgb}{0.500000,0.500000,0.500000}%
\pgfsetstrokecolor{currentstroke}%
\pgfsetstrokeopacity{0.300000}%
\pgfsetdash{}{0pt}%
\pgfpathmoveto{\pgfqpoint{0.647939in}{1.752155in}}%
\pgfpathlineto{\pgfqpoint{0.647939in}{1.752155in}}%
\pgfpathlineto{\pgfqpoint{0.715447in}{1.763303in}}%
\pgfpathlineto{\pgfqpoint{0.782619in}{1.776315in}}%
\pgfpathlineto{\pgfqpoint{0.849384in}{1.791270in}}%
\pgfpathlineto{\pgfqpoint{0.915675in}{1.808202in}}%
\pgfpathlineto{\pgfqpoint{0.981434in}{1.827098in}}%
\pgfpathlineto{\pgfqpoint{1.046618in}{1.847892in}}%
\pgfpathlineto{\pgfqpoint{1.111208in}{1.870467in}}%
\pgfpathlineto{\pgfqpoint{1.175213in}{1.894657in}}%
\pgfpathlineto{\pgfqpoint{1.238671in}{1.920251in}}%
\pgfpathlineto{\pgfqpoint{1.301652in}{1.947000in}}%
\pgfpathlineto{\pgfqpoint{1.364257in}{1.974623in}}%
\pgfusepath{stroke}%
\end{pgfscope}%
\begin{pgfscope}%
\pgfpathrectangle{\pgfqpoint{0.647939in}{0.492442in}}{\pgfqpoint{3.079299in}{3.079299in}}%
\pgfusepath{clip}%
\pgfsetbuttcap%
\pgfsetroundjoin%
\pgfsetlinewidth{0.301125pt}%
\definecolor{currentstroke}{rgb}{0.500000,0.500000,0.500000}%
\pgfsetstrokecolor{currentstroke}%
\pgfsetstrokeopacity{0.300000}%
\pgfsetdash{}{0pt}%
\pgfpathmoveto{\pgfqpoint{0.647939in}{1.682171in}}%
\pgfpathlineto{\pgfqpoint{0.647939in}{1.682171in}}%
\pgfpathlineto{\pgfqpoint{0.715418in}{1.693492in}}%
\pgfpathlineto{\pgfqpoint{0.782547in}{1.706723in}}%
\pgfpathlineto{\pgfqpoint{0.849249in}{1.721953in}}%
\pgfpathlineto{\pgfqpoint{0.915452in}{1.739224in}}%
\pgfpathlineto{\pgfqpoint{0.981091in}{1.758529in}}%
\pgfpathlineto{\pgfqpoint{1.046117in}{1.779810in}}%
\pgfpathlineto{\pgfqpoint{1.110505in}{1.802953in}}%
\pgfpathlineto{\pgfqpoint{1.174257in}{1.827799in}}%
\pgfpathlineto{\pgfqpoint{1.237407in}{1.854142in}}%
\pgfpathlineto{\pgfqpoint{1.300021in}{1.881738in}}%
\pgfpathlineto{\pgfqpoint{1.362198in}{1.910308in}}%
\pgfpathlineto{\pgfqpoint{1.424063in}{1.939549in}}%
\pgfpathlineto{\pgfqpoint{1.485767in}{1.969129in}}%
\pgfpathlineto{\pgfqpoint{1.547476in}{1.998698in}}%
\pgfpathlineto{\pgfqpoint{1.609367in}{2.027884in}}%
\pgfpathlineto{\pgfqpoint{1.671613in}{2.056299in}}%
\pgfpathlineto{\pgfqpoint{1.734375in}{2.083552in}}%
\pgfpathlineto{\pgfqpoint{1.797781in}{2.109266in}}%
\pgfpathlineto{\pgfqpoint{1.861910in}{2.133113in}}%
\pgfpathlineto{\pgfqpoint{1.926771in}{2.154883in}}%
\pgfpathlineto{\pgfqpoint{1.992292in}{2.174583in}}%
\pgfpathlineto{\pgfqpoint{2.058309in}{2.192565in}}%
\pgfpathlineto{\pgfqpoint{2.124547in}{2.209725in}}%
\pgfpathlineto{\pgfqpoint{2.190554in}{2.227709in}}%
\pgfpathlineto{\pgfqpoint{2.255517in}{2.248973in}}%
\pgfpathlineto{\pgfqpoint{2.318110in}{2.276156in}}%
\pgfusepath{stroke}%
\end{pgfscope}%
\begin{pgfscope}%
\pgfpathrectangle{\pgfqpoint{0.647939in}{0.492442in}}{\pgfqpoint{3.079299in}{3.079299in}}%
\pgfusepath{clip}%
\pgfsetbuttcap%
\pgfsetroundjoin%
\pgfsetlinewidth{0.301125pt}%
\definecolor{currentstroke}{rgb}{0.500000,0.500000,0.500000}%
\pgfsetstrokecolor{currentstroke}%
\pgfsetstrokeopacity{0.300000}%
\pgfsetdash{}{0pt}%
\pgfpathmoveto{\pgfqpoint{0.647939in}{1.612187in}}%
\pgfpathlineto{\pgfqpoint{0.647939in}{1.612187in}}%
\pgfpathlineto{\pgfqpoint{0.715388in}{1.623685in}}%
\pgfpathlineto{\pgfqpoint{0.782470in}{1.637145in}}%
\pgfpathlineto{\pgfqpoint{0.849106in}{1.652660in}}%
\pgfpathlineto{\pgfqpoint{0.915216in}{1.670282in}}%
\pgfpathlineto{\pgfqpoint{0.980727in}{1.690013in}}%
\pgfpathlineto{\pgfqpoint{1.045584in}{1.711800in}}%
\pgfpathlineto{\pgfqpoint{1.109754in}{1.735537in}}%
\pgfpathlineto{\pgfqpoint{1.173233in}{1.761070in}}%
\pgfusepath{stroke}%
\end{pgfscope}%
\begin{pgfscope}%
\pgfpathrectangle{\pgfqpoint{0.647939in}{0.492442in}}{\pgfqpoint{3.079299in}{3.079299in}}%
\pgfusepath{clip}%
\pgfsetbuttcap%
\pgfsetroundjoin%
\pgfsetlinewidth{0.301125pt}%
\definecolor{currentstroke}{rgb}{0.500000,0.500000,0.500000}%
\pgfsetstrokecolor{currentstroke}%
\pgfsetstrokeopacity{0.300000}%
\pgfsetdash{}{0pt}%
\pgfpathmoveto{\pgfqpoint{0.647939in}{1.542203in}}%
\pgfpathlineto{\pgfqpoint{0.647939in}{1.542203in}}%
\pgfpathlineto{\pgfqpoint{0.715356in}{1.553885in}}%
\pgfpathlineto{\pgfqpoint{0.782390in}{1.567579in}}%
\pgfpathlineto{\pgfqpoint{0.848956in}{1.583390in}}%
\pgfpathlineto{\pgfqpoint{0.914966in}{1.601377in}}%
\pgfpathlineto{\pgfqpoint{0.980342in}{1.621551in}}%
\pgfpathlineto{\pgfqpoint{1.045018in}{1.643866in}}%
\pgfpathlineto{\pgfqpoint{1.108954in}{1.668224in}}%
\pgfpathlineto{\pgfqpoint{1.172138in}{1.694475in}}%
\pgfusepath{stroke}%
\end{pgfscope}%
\begin{pgfscope}%
\pgfpathrectangle{\pgfqpoint{0.647939in}{0.492442in}}{\pgfqpoint{3.079299in}{3.079299in}}%
\pgfusepath{clip}%
\pgfsetbuttcap%
\pgfsetroundjoin%
\pgfsetlinewidth{0.301125pt}%
\definecolor{currentstroke}{rgb}{0.500000,0.500000,0.500000}%
\pgfsetstrokecolor{currentstroke}%
\pgfsetstrokeopacity{0.300000}%
\pgfsetdash{}{0pt}%
\pgfpathmoveto{\pgfqpoint{0.647939in}{1.472219in}}%
\pgfpathlineto{\pgfqpoint{0.647939in}{1.472219in}}%
\pgfpathlineto{\pgfqpoint{0.715322in}{1.484090in}}%
\pgfpathlineto{\pgfqpoint{0.782306in}{1.498028in}}%
\pgfpathlineto{\pgfqpoint{0.848798in}{1.514145in}}%
\pgfpathlineto{\pgfqpoint{0.914702in}{1.532511in}}%
\pgfpathlineto{\pgfqpoint{0.979932in}{1.553146in}}%
\pgfpathlineto{\pgfqpoint{1.044414in}{1.576012in}}%
\pgfpathlineto{\pgfqpoint{1.108098in}{1.601019in}}%
\pgfpathlineto{\pgfqpoint{1.170963in}{1.628021in}}%
\pgfpathlineto{\pgfqpoint{1.233023in}{1.656829in}}%
\pgfpathlineto{\pgfqpoint{1.294329in}{1.687213in}}%
\pgfpathlineto{\pgfqpoint{1.354966in}{1.718912in}}%
\pgfpathlineto{\pgfqpoint{1.415056in}{1.751639in}}%
\pgfpathlineto{\pgfqpoint{1.474750in}{1.785087in}}%
\pgfpathlineto{\pgfqpoint{1.534222in}{1.818929in}}%
\pgfpathlineto{\pgfqpoint{1.593667in}{1.852818in}}%
\pgfpathlineto{\pgfqpoint{1.653289in}{1.886394in}}%
\pgfpathlineto{\pgfqpoint{1.713292in}{1.919282in}}%
\pgfpathlineto{\pgfqpoint{1.773865in}{1.951104in}}%
\pgfpathlineto{\pgfqpoint{1.835161in}{1.981506in}}%
\pgfpathlineto{\pgfqpoint{1.897268in}{2.010205in}}%
\pgfpathlineto{\pgfqpoint{1.960183in}{2.037083in}}%
\pgfpathlineto{\pgfqpoint{2.023767in}{2.062333in}}%
\pgfpathlineto{\pgfqpoint{2.087696in}{2.086701in}}%
\pgfpathlineto{\pgfqpoint{2.151343in}{2.111763in}}%
\pgfpathlineto{\pgfqpoint{2.213562in}{2.139983in}}%
\pgfpathlineto{\pgfqpoint{2.272774in}{2.173774in}}%
\pgfusepath{stroke}%
\end{pgfscope}%
\begin{pgfscope}%
\pgfpathrectangle{\pgfqpoint{0.647939in}{0.492442in}}{\pgfqpoint{3.079299in}{3.079299in}}%
\pgfusepath{clip}%
\pgfsetbuttcap%
\pgfsetroundjoin%
\pgfsetlinewidth{0.301125pt}%
\definecolor{currentstroke}{rgb}{0.500000,0.500000,0.500000}%
\pgfsetstrokecolor{currentstroke}%
\pgfsetstrokeopacity{0.300000}%
\pgfsetdash{}{0pt}%
\pgfpathmoveto{\pgfqpoint{0.647939in}{1.402235in}}%
\pgfpathlineto{\pgfqpoint{0.647939in}{1.402235in}}%
\pgfpathlineto{\pgfqpoint{0.715287in}{1.414302in}}%
\pgfpathlineto{\pgfqpoint{0.782218in}{1.428491in}}%
\pgfpathlineto{\pgfqpoint{0.848630in}{1.444926in}}%
\pgfpathlineto{\pgfqpoint{0.914423in}{1.463687in}}%
\pgfpathlineto{\pgfqpoint{0.979497in}{1.484802in}}%
\pgfusepath{stroke}%
\end{pgfscope}%
\begin{pgfscope}%
\pgfpathrectangle{\pgfqpoint{0.647939in}{0.492442in}}{\pgfqpoint{3.079299in}{3.079299in}}%
\pgfusepath{clip}%
\pgfsetbuttcap%
\pgfsetroundjoin%
\pgfsetlinewidth{0.301125pt}%
\definecolor{currentstroke}{rgb}{0.500000,0.500000,0.500000}%
\pgfsetstrokecolor{currentstroke}%
\pgfsetstrokeopacity{0.300000}%
\pgfsetdash{}{0pt}%
\pgfpathmoveto{\pgfqpoint{0.647939in}{1.332251in}}%
\pgfpathlineto{\pgfqpoint{0.647939in}{1.332251in}}%
\pgfpathlineto{\pgfqpoint{0.715250in}{1.344519in}}%
\pgfpathlineto{\pgfqpoint{0.782124in}{1.358969in}}%
\pgfpathlineto{\pgfqpoint{0.848454in}{1.375734in}}%
\pgfpathlineto{\pgfqpoint{0.914126in}{1.394906in}}%
\pgfpathlineto{\pgfqpoint{0.979035in}{1.416522in}}%
\pgfpathlineto{\pgfqpoint{1.043085in}{1.440564in}}%
\pgfpathlineto{\pgfqpoint{1.106203in}{1.466956in}}%
\pgfpathlineto{\pgfqpoint{1.168349in}{1.495567in}}%
\pgfpathlineto{\pgfqpoint{1.229516in}{1.526217in}}%
\pgfpathlineto{\pgfqpoint{1.289739in}{1.558687in}}%
\pgfpathlineto{\pgfqpoint{1.349089in}{1.592728in}}%
\pgfpathlineto{\pgfqpoint{1.407678in}{1.628066in}}%
\pgfpathlineto{\pgfqpoint{1.465650in}{1.664412in}}%
\pgfpathlineto{\pgfqpoint{1.523178in}{1.701459in}}%
\pgfpathlineto{\pgfqpoint{1.580462in}{1.738885in}}%
\pgfpathlineto{\pgfqpoint{1.637722in}{1.776349in}}%
\pgfpathlineto{\pgfqpoint{1.695181in}{1.813500in}}%
\pgfpathlineto{\pgfqpoint{1.753058in}{1.849989in}}%
\pgfpathlineto{\pgfqpoint{1.811552in}{1.885479in}}%
\pgfpathlineto{\pgfqpoint{1.870814in}{1.919670in}}%
\pgfpathlineto{\pgfqpoint{1.930912in}{1.952371in}}%
\pgfpathlineto{\pgfqpoint{1.991776in}{1.983624in}}%
\pgfusepath{stroke}%
\end{pgfscope}%
\begin{pgfscope}%
\pgfpathrectangle{\pgfqpoint{0.647939in}{0.492442in}}{\pgfqpoint{3.079299in}{3.079299in}}%
\pgfusepath{clip}%
\pgfsetbuttcap%
\pgfsetroundjoin%
\pgfsetlinewidth{0.301125pt}%
\definecolor{currentstroke}{rgb}{0.500000,0.500000,0.500000}%
\pgfsetstrokecolor{currentstroke}%
\pgfsetstrokeopacity{0.300000}%
\pgfsetdash{}{0pt}%
\pgfpathmoveto{\pgfqpoint{0.647939in}{1.262267in}}%
\pgfpathlineto{\pgfqpoint{0.647939in}{1.262267in}}%
\pgfpathlineto{\pgfqpoint{0.715211in}{1.274744in}}%
\pgfpathlineto{\pgfqpoint{0.782026in}{1.289463in}}%
\pgfpathlineto{\pgfqpoint{0.848266in}{1.306571in}}%
\pgfpathlineto{\pgfqpoint{0.913812in}{1.326170in}}%
\pgfpathlineto{\pgfqpoint{0.978542in}{1.348309in}}%
\pgfpathlineto{\pgfqpoint{1.042351in}{1.372979in}}%
\pgfpathlineto{\pgfqpoint{1.105153in}{1.400111in}}%
\pgfpathlineto{\pgfqpoint{1.166893in}{1.429582in}}%
\pgfpathlineto{\pgfqpoint{1.227555in}{1.461216in}}%
\pgfpathlineto{\pgfqpoint{1.287160in}{1.494801in}}%
\pgfpathlineto{\pgfqpoint{1.345775in}{1.530090in}}%
\pgfpathlineto{\pgfqpoint{1.403501in}{1.566818in}}%
\pgfpathlineto{\pgfqpoint{1.460478in}{1.604700in}}%
\pgfpathlineto{\pgfqpoint{1.516882in}{1.643439in}}%
\pgfusepath{stroke}%
\end{pgfscope}%
\begin{pgfscope}%
\pgfpathrectangle{\pgfqpoint{0.647939in}{0.492442in}}{\pgfqpoint{3.079299in}{3.079299in}}%
\pgfusepath{clip}%
\pgfsetbuttcap%
\pgfsetroundjoin%
\pgfsetlinewidth{0.301125pt}%
\definecolor{currentstroke}{rgb}{0.500000,0.500000,0.500000}%
\pgfsetstrokecolor{currentstroke}%
\pgfsetstrokeopacity{0.300000}%
\pgfsetdash{}{0pt}%
\pgfpathmoveto{\pgfqpoint{0.647939in}{1.192283in}}%
\pgfpathlineto{\pgfqpoint{0.647939in}{1.192283in}}%
\pgfpathlineto{\pgfqpoint{0.715170in}{1.204976in}}%
\pgfpathlineto{\pgfqpoint{0.781922in}{1.219975in}}%
\pgfpathlineto{\pgfqpoint{0.848068in}{1.237439in}}%
\pgfpathlineto{\pgfqpoint{0.913478in}{1.257483in}}%
\pgfpathlineto{\pgfqpoint{0.978017in}{1.280167in}}%
\pgfpathlineto{\pgfqpoint{1.041567in}{1.305492in}}%
\pgfpathlineto{\pgfqpoint{1.104026in}{1.333399in}}%
\pgfpathlineto{\pgfqpoint{1.165326in}{1.363769in}}%
\pgfpathlineto{\pgfqpoint{1.225438in}{1.396432in}}%
\pgfpathlineto{\pgfqpoint{1.284371in}{1.431179in}}%
\pgfpathlineto{\pgfqpoint{1.342180in}{1.467767in}}%
\pgfpathlineto{\pgfqpoint{1.398962in}{1.505936in}}%
\pgfpathlineto{\pgfqpoint{1.454853in}{1.545404in}}%
\pgfusepath{stroke}%
\end{pgfscope}%
\begin{pgfscope}%
\pgfpathrectangle{\pgfqpoint{0.647939in}{0.492442in}}{\pgfqpoint{3.079299in}{3.079299in}}%
\pgfusepath{clip}%
\pgfsetbuttcap%
\pgfsetroundjoin%
\pgfsetlinewidth{0.301125pt}%
\definecolor{currentstroke}{rgb}{0.500000,0.500000,0.500000}%
\pgfsetstrokecolor{currentstroke}%
\pgfsetstrokeopacity{0.300000}%
\pgfsetdash{}{0pt}%
\pgfpathmoveto{\pgfqpoint{0.647939in}{1.122299in}}%
\pgfpathlineto{\pgfqpoint{0.647939in}{1.122299in}}%
\pgfpathlineto{\pgfqpoint{0.715127in}{1.135215in}}%
\pgfpathlineto{\pgfqpoint{0.781812in}{1.150504in}}%
\pgfpathlineto{\pgfqpoint{0.847858in}{1.168339in}}%
\pgfpathlineto{\pgfqpoint{0.913122in}{1.188846in}}%
\pgfpathlineto{\pgfqpoint{0.977457in}{1.212099in}}%
\pgfpathlineto{\pgfqpoint{1.040727in}{1.238110in}}%
\pgfpathlineto{\pgfqpoint{1.102815in}{1.266827in}}%
\pgfusepath{stroke}%
\end{pgfscope}%
\begin{pgfscope}%
\pgfpathrectangle{\pgfqpoint{0.647939in}{0.492442in}}{\pgfqpoint{3.079299in}{3.079299in}}%
\pgfusepath{clip}%
\pgfsetbuttcap%
\pgfsetroundjoin%
\pgfsetlinewidth{0.301125pt}%
\definecolor{currentstroke}{rgb}{0.500000,0.500000,0.500000}%
\pgfsetstrokecolor{currentstroke}%
\pgfsetstrokeopacity{0.300000}%
\pgfsetdash{}{0pt}%
\pgfpathmoveto{\pgfqpoint{0.647939in}{1.052315in}}%
\pgfpathlineto{\pgfqpoint{0.647939in}{1.052315in}}%
\pgfpathlineto{\pgfqpoint{0.715082in}{1.065462in}}%
\pgfpathlineto{\pgfqpoint{0.781696in}{1.081052in}}%
\pgfpathlineto{\pgfqpoint{0.847636in}{1.099272in}}%
\pgfpathlineto{\pgfqpoint{0.912744in}{1.120263in}}%
\pgfpathlineto{\pgfqpoint{0.976859in}{1.144111in}}%
\pgfpathlineto{\pgfqpoint{1.039826in}{1.170838in}}%
\pgfpathlineto{\pgfqpoint{1.101513in}{1.200401in}}%
\pgfpathlineto{\pgfqpoint{1.161817in}{1.232691in}}%
\pgfpathlineto{\pgfqpoint{1.220677in}{1.267546in}}%
\pgfpathlineto{\pgfqpoint{1.278079in}{1.304757in}}%
\pgfpathlineto{\pgfqpoint{1.334056in}{1.344086in}}%
\pgfpathlineto{\pgfqpoint{1.388695in}{1.385258in}}%
\pgfpathlineto{\pgfqpoint{1.442134in}{1.427976in}}%
\pgfpathlineto{\pgfqpoint{1.494539in}{1.471956in}}%
\pgfpathlineto{\pgfqpoint{1.546115in}{1.516911in}}%
\pgfpathlineto{\pgfqpoint{1.597103in}{1.562530in}}%
\pgfpathlineto{\pgfqpoint{1.647758in}{1.608516in}}%
\pgfpathlineto{\pgfqpoint{1.698353in}{1.654566in}}%
\pgfpathlineto{\pgfqpoint{1.749167in}{1.700367in}}%
\pgfpathlineto{\pgfqpoint{1.800466in}{1.745620in}}%
\pgfpathlineto{\pgfqpoint{1.852482in}{1.790033in}}%
\pgfpathlineto{\pgfqpoint{1.905387in}{1.833372in}}%
\pgfusepath{stroke}%
\end{pgfscope}%
\begin{pgfscope}%
\pgfpathrectangle{\pgfqpoint{0.647939in}{0.492442in}}{\pgfqpoint{3.079299in}{3.079299in}}%
\pgfusepath{clip}%
\pgfsetbuttcap%
\pgfsetroundjoin%
\pgfsetlinewidth{0.301125pt}%
\definecolor{currentstroke}{rgb}{0.500000,0.500000,0.500000}%
\pgfsetstrokecolor{currentstroke}%
\pgfsetstrokeopacity{0.300000}%
\pgfsetdash{}{0pt}%
\pgfpathmoveto{\pgfqpoint{0.647939in}{0.982331in}}%
\pgfpathlineto{\pgfqpoint{0.647939in}{0.982331in}}%
\pgfpathlineto{\pgfqpoint{0.715034in}{0.995717in}}%
\pgfpathlineto{\pgfqpoint{0.781573in}{1.011620in}}%
\pgfpathlineto{\pgfqpoint{0.847399in}{1.030241in}}%
\pgfpathlineto{\pgfqpoint{0.912341in}{1.051737in}}%
\pgfpathlineto{\pgfqpoint{0.976219in}{1.076206in}}%
\pgfpathlineto{\pgfqpoint{1.038860in}{1.103682in}}%
\pgfpathlineto{\pgfqpoint{1.100111in}{1.134129in}}%
\pgfpathlineto{\pgfqpoint{1.159852in}{1.167442in}}%
\pgfpathlineto{\pgfqpoint{1.218003in}{1.203459in}}%
\pgfpathlineto{\pgfqpoint{1.274539in}{1.241971in}}%
\pgfpathlineto{\pgfqpoint{1.329482in}{1.282729in}}%
\pgfpathlineto{\pgfqpoint{1.382922in}{1.325438in}}%
\pgfusepath{stroke}%
\end{pgfscope}%
\begin{pgfscope}%
\pgfpathrectangle{\pgfqpoint{0.647939in}{0.492442in}}{\pgfqpoint{3.079299in}{3.079299in}}%
\pgfusepath{clip}%
\pgfsetbuttcap%
\pgfsetroundjoin%
\pgfsetlinewidth{0.301125pt}%
\definecolor{currentstroke}{rgb}{0.500000,0.500000,0.500000}%
\pgfsetstrokecolor{currentstroke}%
\pgfsetstrokeopacity{0.300000}%
\pgfsetdash{}{0pt}%
\pgfpathmoveto{\pgfqpoint{0.647939in}{0.912347in}}%
\pgfpathlineto{\pgfqpoint{0.647939in}{0.912347in}}%
\pgfpathlineto{\pgfqpoint{0.714983in}{0.925980in}}%
\pgfpathlineto{\pgfqpoint{0.781442in}{0.942209in}}%
\pgfpathlineto{\pgfqpoint{0.847148in}{0.961248in}}%
\pgfpathlineto{\pgfqpoint{0.911910in}{0.983270in}}%
\pgfpathlineto{\pgfqpoint{0.975533in}{1.008388in}}%
\pgfpathlineto{\pgfqpoint{1.037822in}{1.036648in}}%
\pgfpathlineto{\pgfqpoint{1.098601in}{1.068018in}}%
\pgfpathlineto{\pgfqpoint{1.157730in}{1.102398in}}%
\pgfpathlineto{\pgfqpoint{1.215113in}{1.139622in}}%
\pgfpathlineto{\pgfqpoint{1.270708in}{1.179476in}}%
\pgfpathlineto{\pgfqpoint{1.324539in}{1.221686in}}%
\pgfusepath{stroke}%
\end{pgfscope}%
\begin{pgfscope}%
\pgfpathrectangle{\pgfqpoint{0.647939in}{0.492442in}}{\pgfqpoint{3.079299in}{3.079299in}}%
\pgfusepath{clip}%
\pgfsetbuttcap%
\pgfsetroundjoin%
\pgfsetlinewidth{0.301125pt}%
\definecolor{currentstroke}{rgb}{0.500000,0.500000,0.500000}%
\pgfsetstrokecolor{currentstroke}%
\pgfsetstrokeopacity{0.300000}%
\pgfsetdash{}{0pt}%
\pgfpathmoveto{\pgfqpoint{0.647939in}{0.842362in}}%
\pgfpathlineto{\pgfqpoint{0.647939in}{0.842362in}}%
\pgfpathlineto{\pgfqpoint{0.714929in}{0.856253in}}%
\pgfpathlineto{\pgfqpoint{0.781304in}{0.872820in}}%
\pgfpathlineto{\pgfqpoint{0.846880in}{0.892295in}}%
\pgfpathlineto{\pgfqpoint{0.911451in}{0.914867in}}%
\pgfpathlineto{\pgfqpoint{0.974798in}{0.940664in}}%
\pgfpathlineto{\pgfqpoint{1.036705in}{0.969741in}}%
\pgfpathlineto{\pgfqpoint{1.096973in}{1.002076in}}%
\pgfpathlineto{\pgfqpoint{1.155438in}{1.037565in}}%
\pgfpathlineto{\pgfqpoint{1.211987in}{1.076039in}}%
\pgfpathlineto{\pgfqpoint{1.266569in}{1.117264in}}%
\pgfusepath{stroke}%
\end{pgfscope}%
\begin{pgfscope}%
\pgfpathrectangle{\pgfqpoint{0.647939in}{0.492442in}}{\pgfqpoint{3.079299in}{3.079299in}}%
\pgfusepath{clip}%
\pgfsetbuttcap%
\pgfsetroundjoin%
\pgfsetlinewidth{0.301125pt}%
\definecolor{currentstroke}{rgb}{0.500000,0.500000,0.500000}%
\pgfsetstrokecolor{currentstroke}%
\pgfsetstrokeopacity{0.300000}%
\pgfsetdash{}{0pt}%
\pgfpathmoveto{\pgfqpoint{0.647939in}{0.772378in}}%
\pgfpathlineto{\pgfqpoint{0.647939in}{0.772378in}}%
\pgfpathlineto{\pgfqpoint{0.714873in}{0.786536in}}%
\pgfpathlineto{\pgfqpoint{0.781158in}{0.803454in}}%
\pgfpathlineto{\pgfqpoint{0.846595in}{0.823383in}}%
\pgfpathlineto{\pgfqpoint{0.910959in}{0.846530in}}%
\pgfpathlineto{\pgfqpoint{0.974009in}{0.873037in}}%
\pgfpathlineto{\pgfqpoint{1.035503in}{0.902970in}}%
\pgfpathlineto{\pgfqpoint{1.095215in}{0.936310in}}%
\pgfpathlineto{\pgfqpoint{1.152961in}{0.972952in}}%
\pgfpathlineto{\pgfqpoint{1.208608in}{1.012716in}}%
\pgfpathlineto{\pgfqpoint{1.262102in}{1.055338in}}%
\pgfpathlineto{\pgfqpoint{1.313464in}{1.100496in}}%
\pgfpathlineto{\pgfqpoint{1.362786in}{1.147888in}}%
\pgfpathlineto{\pgfqpoint{1.410239in}{1.197153in}}%
\pgfpathlineto{\pgfqpoint{1.456049in}{1.247950in}}%
\pgfpathlineto{\pgfqpoint{1.500497in}{1.299948in}}%
\pgfusepath{stroke}%
\end{pgfscope}%
\begin{pgfscope}%
\pgfpathrectangle{\pgfqpoint{0.647939in}{0.492442in}}{\pgfqpoint{3.079299in}{3.079299in}}%
\pgfusepath{clip}%
\pgfsetbuttcap%
\pgfsetroundjoin%
\pgfsetlinewidth{0.301125pt}%
\definecolor{currentstroke}{rgb}{0.500000,0.500000,0.500000}%
\pgfsetstrokecolor{currentstroke}%
\pgfsetstrokeopacity{0.300000}%
\pgfsetdash{}{0pt}%
\pgfpathmoveto{\pgfqpoint{0.647939in}{0.702394in}}%
\pgfpathlineto{\pgfqpoint{0.647939in}{0.702394in}}%
\pgfpathlineto{\pgfqpoint{0.714813in}{0.716828in}}%
\pgfpathlineto{\pgfqpoint{0.781001in}{0.734112in}}%
\pgfpathlineto{\pgfqpoint{0.846290in}{0.754516in}}%
\pgfpathlineto{\pgfqpoint{0.910432in}{0.778263in}}%
\pgfpathlineto{\pgfqpoint{0.973161in}{0.805513in}}%
\pgfpathlineto{\pgfqpoint{1.034207in}{0.836341in}}%
\pgfpathlineto{\pgfqpoint{1.093318in}{0.870729in}}%
\pgfpathlineto{\pgfqpoint{1.150283in}{0.908565in}}%
\pgfusepath{stroke}%
\end{pgfscope}%
\begin{pgfscope}%
\pgfpathrectangle{\pgfqpoint{0.647939in}{0.492442in}}{\pgfqpoint{3.079299in}{3.079299in}}%
\pgfusepath{clip}%
\pgfsetbuttcap%
\pgfsetroundjoin%
\pgfsetlinewidth{0.301125pt}%
\definecolor{currentstroke}{rgb}{0.500000,0.500000,0.500000}%
\pgfsetstrokecolor{currentstroke}%
\pgfsetstrokeopacity{0.300000}%
\pgfsetdash{}{0pt}%
\pgfpathmoveto{\pgfqpoint{0.647939in}{0.562426in}}%
\pgfpathlineto{\pgfqpoint{0.647939in}{0.562426in}}%
\pgfpathlineto{\pgfqpoint{0.714681in}{0.577446in}}%
\pgfpathlineto{\pgfqpoint{0.780659in}{0.595509in}}%
\pgfpathlineto{\pgfqpoint{0.845617in}{0.616925in}}%
\pgfpathlineto{\pgfqpoint{0.909262in}{0.641959in}}%
\pgfpathlineto{\pgfqpoint{0.971267in}{0.670798in}}%
\pgfpathlineto{\pgfqpoint{1.031299in}{0.703534in}}%
\pgfpathlineto{\pgfqpoint{1.089049in}{0.740144in}}%
\pgfpathlineto{\pgfqpoint{1.144262in}{0.780490in}}%
\pgfusepath{stroke}%
\end{pgfscope}%
\begin{pgfscope}%
\pgfpathrectangle{\pgfqpoint{0.647939in}{0.492442in}}{\pgfqpoint{3.079299in}{3.079299in}}%
\pgfusepath{clip}%
\pgfsetbuttcap%
\pgfsetroundjoin%
\pgfsetlinewidth{0.301125pt}%
\definecolor{currentstroke}{rgb}{0.500000,0.500000,0.500000}%
\pgfsetstrokecolor{currentstroke}%
\pgfsetstrokeopacity{0.300000}%
\pgfsetdash{}{0pt}%
\pgfpathmoveto{\pgfqpoint{0.647939in}{3.002821in}}%
\pgfpathlineto{\pgfqpoint{0.650065in}{3.003076in}}%
\pgfpathlineto{\pgfqpoint{0.717923in}{3.011869in}}%
\pgfpathlineto{\pgfqpoint{0.785606in}{3.021924in}}%
\pgfpathlineto{\pgfqpoint{0.853090in}{3.033234in}}%
\pgfpathlineto{\pgfqpoint{0.920360in}{3.045757in}}%
\pgfpathlineto{\pgfqpoint{0.987411in}{3.059408in}}%
\pgfpathlineto{\pgfqpoint{1.054250in}{3.074062in}}%
\pgfpathlineto{\pgfqpoint{1.120900in}{3.089556in}}%
\pgfpathlineto{\pgfqpoint{1.187399in}{3.105693in}}%
\pgfpathlineto{\pgfqpoint{1.253794in}{3.122250in}}%
\pgfpathlineto{\pgfqpoint{1.320146in}{3.138982in}}%
\pgfpathlineto{\pgfqpoint{1.386520in}{3.155624in}}%
\pgfpathlineto{\pgfqpoint{1.452984in}{3.171903in}}%
\pgfusepath{stroke}%
\end{pgfscope}%
\begin{pgfscope}%
\pgfpathrectangle{\pgfqpoint{0.647939in}{0.492442in}}{\pgfqpoint{3.079299in}{3.079299in}}%
\pgfusepath{clip}%
\pgfsetbuttcap%
\pgfsetroundjoin%
\pgfsetlinewidth{0.301125pt}%
\definecolor{currentstroke}{rgb}{0.500000,0.500000,0.500000}%
\pgfsetstrokecolor{currentstroke}%
\pgfsetstrokeopacity{0.300000}%
\pgfsetdash{}{0pt}%
\pgfpathmoveto{\pgfqpoint{2.248163in}{0.601587in}}%
\pgfpathlineto{\pgfqpoint{2.179849in}{0.605540in}}%
\pgfpathlineto{\pgfqpoint{2.111553in}{0.609788in}}%
\pgfpathlineto{\pgfqpoint{2.043332in}{0.615064in}}%
\pgfpathlineto{\pgfqpoint{1.975293in}{0.622249in}}%
\pgfpathlineto{\pgfqpoint{1.907652in}{0.632410in}}%
\pgfpathlineto{\pgfqpoint{1.840830in}{0.646879in}}%
\pgfusepath{stroke}%
\end{pgfscope}%
\begin{pgfscope}%
\pgfpathrectangle{\pgfqpoint{0.647939in}{0.492442in}}{\pgfqpoint{3.079299in}{3.079299in}}%
\pgfusepath{clip}%
\pgfsetbuttcap%
\pgfsetroundjoin%
\pgfsetlinewidth{0.301125pt}%
\definecolor{currentstroke}{rgb}{0.500000,0.500000,0.500000}%
\pgfsetstrokecolor{currentstroke}%
\pgfsetstrokeopacity{0.300000}%
\pgfsetdash{}{0pt}%
\pgfpathmoveto{\pgfqpoint{3.517286in}{1.542203in}}%
\pgfpathlineto{\pgfqpoint{3.464342in}{1.585539in}}%
\pgfpathlineto{\pgfqpoint{3.412826in}{1.630564in}}%
\pgfpathlineto{\pgfqpoint{3.362736in}{1.677171in}}%
\pgfpathlineto{\pgfqpoint{3.314088in}{1.725280in}}%
\pgfpathlineto{\pgfqpoint{3.266929in}{1.774850in}}%
\pgfpathlineto{\pgfqpoint{3.221344in}{1.825870in}}%
\pgfpathlineto{\pgfqpoint{3.177470in}{1.878365in}}%
\pgfusepath{stroke}%
\end{pgfscope}%
\begin{pgfscope}%
\pgfpathrectangle{\pgfqpoint{0.647939in}{0.492442in}}{\pgfqpoint{3.079299in}{3.079299in}}%
\pgfusepath{clip}%
\pgfsetbuttcap%
\pgfsetroundjoin%
\pgfsetlinewidth{0.301125pt}%
\definecolor{currentstroke}{rgb}{0.500000,0.500000,0.500000}%
\pgfsetstrokecolor{currentstroke}%
\pgfsetstrokeopacity{0.300000}%
\pgfsetdash{}{0pt}%
\pgfpathmoveto{\pgfqpoint{3.517286in}{1.682171in}}%
\pgfpathlineto{\pgfqpoint{3.467692in}{1.729297in}}%
\pgfpathlineto{\pgfqpoint{3.419980in}{1.778328in}}%
\pgfpathlineto{\pgfqpoint{3.374226in}{1.829190in}}%
\pgfpathlineto{\pgfqpoint{3.330548in}{1.881845in}}%
\pgfpathlineto{\pgfqpoint{3.289120in}{1.936284in}}%
\pgfusepath{stroke}%
\end{pgfscope}%
\begin{pgfscope}%
\pgfpathrectangle{\pgfqpoint{0.647939in}{0.492442in}}{\pgfqpoint{3.079299in}{3.079299in}}%
\pgfusepath{clip}%
\pgfsetbuttcap%
\pgfsetroundjoin%
\pgfsetlinewidth{0.301125pt}%
\definecolor{currentstroke}{rgb}{0.500000,0.500000,0.500000}%
\pgfsetstrokecolor{currentstroke}%
\pgfsetstrokeopacity{0.300000}%
\pgfsetdash{}{0pt}%
\pgfpathmoveto{\pgfqpoint{2.656077in}{3.201793in}}%
\pgfpathlineto{\pgfqpoint{2.719502in}{3.227369in}}%
\pgfpathlineto{\pgfqpoint{2.781306in}{3.256657in}}%
\pgfpathlineto{\pgfqpoint{2.841478in}{3.289180in}}%
\pgfpathlineto{\pgfqpoint{2.900121in}{3.324400in}}%
\pgfpathlineto{\pgfqpoint{2.957413in}{3.361789in}}%
\pgfusepath{stroke}%
\end{pgfscope}%
\begin{pgfscope}%
\pgfpathrectangle{\pgfqpoint{0.647939in}{0.492442in}}{\pgfqpoint{3.079299in}{3.079299in}}%
\pgfusepath{clip}%
\pgfsetbuttcap%
\pgfsetroundjoin%
\pgfsetlinewidth{0.301125pt}%
\definecolor{currentstroke}{rgb}{0.500000,0.500000,0.500000}%
\pgfsetstrokecolor{currentstroke}%
\pgfsetstrokeopacity{0.300000}%
\pgfsetdash{}{0pt}%
\pgfpathmoveto{\pgfqpoint{3.447302in}{1.192283in}}%
\pgfpathlineto{\pgfqpoint{3.389198in}{1.228420in}}%
\pgfpathlineto{\pgfqpoint{3.331672in}{1.265475in}}%
\pgfpathlineto{\pgfqpoint{3.274621in}{1.303257in}}%
\pgfpathlineto{\pgfqpoint{3.217933in}{1.341582in}}%
\pgfpathlineto{\pgfqpoint{3.161494in}{1.380271in}}%
\pgfpathlineto{\pgfqpoint{3.105183in}{1.419148in}}%
\pgfpathlineto{\pgfqpoint{3.048874in}{1.458028in}}%
\pgfpathlineto{\pgfqpoint{2.992440in}{1.496724in}}%
\pgfpathlineto{\pgfqpoint{2.935756in}{1.535050in}}%
\pgfpathlineto{\pgfqpoint{2.878696in}{1.572811in}}%
\pgfpathlineto{\pgfqpoint{2.821133in}{1.609803in}}%
\pgfpathlineto{\pgfqpoint{2.762950in}{1.645811in}}%
\pgfpathlineto{\pgfqpoint{2.704045in}{1.680620in}}%
\pgfpathlineto{\pgfqpoint{2.644337in}{1.714028in}}%
\pgfpathlineto{\pgfqpoint{2.583784in}{1.745869in}}%
\pgfpathlineto{\pgfqpoint{2.522394in}{1.776065in}}%
\pgfpathlineto{\pgfqpoint{2.460268in}{1.804719in}}%
\pgfpathlineto{\pgfqpoint{2.397660in}{1.832316in}}%
\pgfpathlineto{\pgfqpoint{2.335169in}{1.860178in}}%
\pgfpathlineto{\pgfqpoint{2.274608in}{1.891801in}}%
\pgfpathlineto{\pgfqpoint{2.274608in}{1.891801in}}%
\pgfpathlineto{\pgfqpoint{2.237173in}{1.920213in}}%
\pgfpathlineto{\pgfqpoint{2.237173in}{1.920213in}}%
\pgfpathlineto{\pgfqpoint{2.218494in}{1.945967in}}%
\pgfpathlineto{\pgfqpoint{2.218494in}{1.945967in}}%
\pgfpathlineto{\pgfqpoint{2.211845in}{1.973543in}}%
\pgfusepath{stroke}%
\end{pgfscope}%
\begin{pgfscope}%
\pgfpathrectangle{\pgfqpoint{0.647939in}{0.492442in}}{\pgfqpoint{3.079299in}{3.079299in}}%
\pgfusepath{clip}%
\pgfsetbuttcap%
\pgfsetroundjoin%
\pgfsetlinewidth{0.301125pt}%
\definecolor{currentstroke}{rgb}{0.500000,0.500000,0.500000}%
\pgfsetstrokecolor{currentstroke}%
\pgfsetstrokeopacity{0.300000}%
\pgfsetdash{}{0pt}%
\pgfpathmoveto{\pgfqpoint{2.195182in}{3.265833in}}%
\pgfpathlineto{\pgfqpoint{2.263472in}{3.270180in}}%
\pgfpathlineto{\pgfqpoint{2.331692in}{3.275477in}}%
\pgfpathlineto{\pgfqpoint{2.399755in}{3.282450in}}%
\pgfpathlineto{\pgfqpoint{2.467525in}{3.291805in}}%
\pgfpathlineto{\pgfqpoint{2.534801in}{3.304172in}}%
\pgfusepath{stroke}%
\end{pgfscope}%
\begin{pgfscope}%
\pgfpathrectangle{\pgfqpoint{0.647939in}{0.492442in}}{\pgfqpoint{3.079299in}{3.079299in}}%
\pgfusepath{clip}%
\pgfsetbuttcap%
\pgfsetroundjoin%
\pgfsetlinewidth{0.301125pt}%
\definecolor{currentstroke}{rgb}{0.500000,0.500000,0.500000}%
\pgfsetstrokecolor{currentstroke}%
\pgfsetstrokeopacity{0.300000}%
\pgfsetdash{}{0pt}%
\pgfpathmoveto{\pgfqpoint{2.458501in}{0.813743in}}%
\pgfpathlineto{\pgfqpoint{2.390494in}{0.821281in}}%
\pgfpathlineto{\pgfqpoint{2.322340in}{0.827376in}}%
\pgfpathlineto{\pgfqpoint{2.254105in}{0.832505in}}%
\pgfpathlineto{\pgfqpoint{2.185841in}{0.837257in}}%
\pgfpathlineto{\pgfqpoint{2.117605in}{0.842362in}}%
\pgfusepath{stroke}%
\end{pgfscope}%
\begin{pgfscope}%
\pgfpathrectangle{\pgfqpoint{0.647939in}{0.492442in}}{\pgfqpoint{3.079299in}{3.079299in}}%
\pgfusepath{clip}%
\pgfsetbuttcap%
\pgfsetroundjoin%
\pgfsetlinewidth{0.301125pt}%
\definecolor{currentstroke}{rgb}{0.500000,0.500000,0.500000}%
\pgfsetstrokecolor{currentstroke}%
\pgfsetstrokeopacity{0.300000}%
\pgfsetdash{}{0pt}%
\pgfpathmoveto{\pgfqpoint{3.377318in}{2.382012in}}%
\pgfpathlineto{\pgfqpoint{3.365879in}{2.449416in}}%
\pgfpathlineto{\pgfqpoint{3.359924in}{2.517524in}}%
\pgfpathlineto{\pgfqpoint{3.359521in}{2.585887in}}%
\pgfpathlineto{\pgfqpoint{3.364587in}{2.654062in}}%
\pgfpathlineto{\pgfqpoint{3.374899in}{2.721648in}}%
\pgfusepath{stroke}%
\end{pgfscope}%
\begin{pgfscope}%
\pgfpathrectangle{\pgfqpoint{0.647939in}{0.492442in}}{\pgfqpoint{3.079299in}{3.079299in}}%
\pgfusepath{clip}%
\pgfsetbuttcap%
\pgfsetroundjoin%
\pgfsetlinewidth{0.301125pt}%
\definecolor{currentstroke}{rgb}{0.500000,0.500000,0.500000}%
\pgfsetstrokecolor{currentstroke}%
\pgfsetstrokeopacity{0.300000}%
\pgfsetdash{}{0pt}%
\pgfpathmoveto{\pgfqpoint{1.635234in}{3.188011in}}%
\pgfpathlineto{\pgfqpoint{1.702457in}{3.200780in}}%
\pgfpathlineto{\pgfqpoint{1.769941in}{3.212081in}}%
\pgfpathlineto{\pgfqpoint{1.837668in}{3.221821in}}%
\pgfpathlineto{\pgfqpoint{1.905602in}{3.229999in}}%
\pgfpathlineto{\pgfqpoint{1.973696in}{3.236716in}}%
\pgfusepath{stroke}%
\end{pgfscope}%
\begin{pgfscope}%
\pgfpathrectangle{\pgfqpoint{0.647939in}{0.492442in}}{\pgfqpoint{3.079299in}{3.079299in}}%
\pgfusepath{clip}%
\pgfsetbuttcap%
\pgfsetroundjoin%
\pgfsetlinewidth{0.301125pt}%
\definecolor{currentstroke}{rgb}{0.500000,0.500000,0.500000}%
\pgfsetstrokecolor{currentstroke}%
\pgfsetstrokeopacity{0.300000}%
\pgfsetdash{}{0pt}%
\pgfpathmoveto{\pgfqpoint{0.997859in}{2.731932in}}%
\pgfpathlineto{\pgfqpoint{1.064360in}{2.748052in}}%
\pgfpathlineto{\pgfqpoint{1.130615in}{2.765153in}}%
\pgfpathlineto{\pgfqpoint{1.196666in}{2.783033in}}%
\pgfpathlineto{\pgfqpoint{1.262566in}{2.801459in}}%
\pgfpathlineto{\pgfqpoint{1.328387in}{2.820171in}}%
\pgfusepath{stroke}%
\end{pgfscope}%
\begin{pgfscope}%
\pgfpathrectangle{\pgfqpoint{0.647939in}{0.492442in}}{\pgfqpoint{3.079299in}{3.079299in}}%
\pgfusepath{clip}%
\pgfsetbuttcap%
\pgfsetroundjoin%
\pgfsetlinewidth{0.301125pt}%
\definecolor{currentstroke}{rgb}{0.500000,0.500000,0.500000}%
\pgfsetstrokecolor{currentstroke}%
\pgfsetstrokeopacity{0.300000}%
\pgfsetdash{}{0pt}%
\pgfpathmoveto{\pgfqpoint{1.154062in}{0.870055in}}%
\pgfpathlineto{\pgfqpoint{1.207812in}{0.912347in}}%
\pgfpathlineto{\pgfqpoint{1.259063in}{0.957628in}}%
\pgfpathlineto{\pgfqpoint{1.307880in}{1.005514in}}%
\pgfpathlineto{\pgfqpoint{1.354402in}{1.055641in}}%
\pgfpathlineto{\pgfqpoint{1.398854in}{1.107616in}}%
\pgfusepath{stroke}%
\end{pgfscope}%
\begin{pgfscope}%
\pgfpathrectangle{\pgfqpoint{0.647939in}{0.492442in}}{\pgfqpoint{3.079299in}{3.079299in}}%
\pgfusepath{clip}%
\pgfsetbuttcap%
\pgfsetroundjoin%
\pgfsetlinewidth{0.301125pt}%
\definecolor{currentstroke}{rgb}{0.500000,0.500000,0.500000}%
\pgfsetstrokecolor{currentstroke}%
\pgfsetstrokeopacity{0.300000}%
\pgfsetdash{}{0pt}%
\pgfpathmoveto{\pgfqpoint{2.110423in}{0.882190in}}%
\pgfpathlineto{\pgfqpoint{2.042344in}{0.889009in}}%
\pgfpathlineto{\pgfqpoint{1.974605in}{0.898520in}}%
\pgfpathlineto{\pgfqpoint{1.907652in}{0.912347in}}%
\pgfpathlineto{\pgfqpoint{1.842464in}{0.932673in}}%
\pgfpathlineto{\pgfqpoint{1.781166in}{0.962349in}}%
\pgfusepath{stroke}%
\end{pgfscope}%
\begin{pgfscope}%
\pgfpathrectangle{\pgfqpoint{0.647939in}{0.492442in}}{\pgfqpoint{3.079299in}{3.079299in}}%
\pgfusepath{clip}%
\pgfsetbuttcap%
\pgfsetroundjoin%
\pgfsetlinewidth{0.301125pt}%
\definecolor{currentstroke}{rgb}{0.500000,0.500000,0.500000}%
\pgfsetstrokecolor{currentstroke}%
\pgfsetstrokeopacity{0.300000}%
\pgfsetdash{}{0pt}%
\pgfpathmoveto{\pgfqpoint{3.472919in}{1.350951in}}%
\pgfpathlineto{\pgfqpoint{3.416866in}{1.390191in}}%
\pgfpathlineto{\pgfqpoint{3.361696in}{1.430666in}}%
\pgfpathlineto{\pgfqpoint{3.307334in}{1.472219in}}%
\pgfpathlineto{\pgfqpoint{3.253707in}{1.514718in}}%
\pgfpathlineto{\pgfqpoint{3.200749in}{1.558048in}}%
\pgfpathlineto{\pgfqpoint{3.148395in}{1.602106in}}%
\pgfpathlineto{\pgfqpoint{3.096592in}{1.646811in}}%
\pgfpathlineto{\pgfqpoint{3.045308in}{1.692110in}}%
\pgfpathlineto{\pgfqpoint{2.994527in}{1.737971in}}%
\pgfpathlineto{\pgfqpoint{2.944265in}{1.784400in}}%
\pgfpathlineto{\pgfqpoint{2.894590in}{1.831454in}}%
\pgfpathlineto{\pgfqpoint{2.845640in}{1.879258in}}%
\pgfpathlineto{\pgfqpoint{2.797672in}{1.928042in}}%
\pgfpathlineto{\pgfqpoint{2.751147in}{1.978195in}}%
\pgfpathlineto{\pgfqpoint{2.706887in}{2.030330in}}%
\pgfpathlineto{\pgfqpoint{2.666376in}{2.085381in}}%
\pgfpathlineto{\pgfqpoint{2.632325in}{2.144539in}}%
\pgfpathlineto{\pgfqpoint{2.609168in}{2.208509in}}%
\pgfpathlineto{\pgfqpoint{2.601470in}{2.274031in}}%
\pgfpathlineto{\pgfqpoint{2.607376in}{2.332516in}}%
\pgfpathlineto{\pgfqpoint{2.623648in}{2.391553in}}%
\pgfpathlineto{\pgfqpoint{2.649824in}{2.454572in}}%
\pgfpathlineto{\pgfqpoint{2.681221in}{2.515258in}}%
\pgfpathlineto{\pgfqpoint{2.715842in}{2.574210in}}%
\pgfusepath{stroke}%
\end{pgfscope}%
\begin{pgfscope}%
\pgfpathrectangle{\pgfqpoint{0.647939in}{0.492442in}}{\pgfqpoint{3.079299in}{3.079299in}}%
\pgfusepath{clip}%
\pgfsetbuttcap%
\pgfsetroundjoin%
\pgfsetlinewidth{0.301125pt}%
\definecolor{currentstroke}{rgb}{0.500000,0.500000,0.500000}%
\pgfsetstrokecolor{currentstroke}%
\pgfsetstrokeopacity{0.300000}%
\pgfsetdash{}{0pt}%
\pgfpathmoveto{\pgfqpoint{3.362541in}{2.116994in}}%
\pgfpathlineto{\pgfqpoint{3.332858in}{2.178605in}}%
\pgfpathlineto{\pgfqpoint{3.307334in}{2.242044in}}%
\pgfpathlineto{\pgfqpoint{3.286422in}{2.307141in}}%
\pgfpathlineto{\pgfqpoint{3.270566in}{2.373643in}}%
\pgfpathlineto{\pgfqpoint{3.260135in}{2.441201in}}%
\pgfpathlineto{\pgfqpoint{3.255363in}{2.509393in}}%
\pgfpathlineto{\pgfqpoint{3.256299in}{2.577751in}}%
\pgfpathlineto{\pgfqpoint{3.262793in}{2.645809in}}%
\pgfusepath{stroke}%
\end{pgfscope}%
\begin{pgfscope}%
\pgfpathrectangle{\pgfqpoint{0.647939in}{0.492442in}}{\pgfqpoint{3.079299in}{3.079299in}}%
\pgfusepath{clip}%
\pgfsetbuttcap%
\pgfsetroundjoin%
\pgfsetlinewidth{0.301125pt}%
\definecolor{currentstroke}{rgb}{0.500000,0.500000,0.500000}%
\pgfsetstrokecolor{currentstroke}%
\pgfsetstrokeopacity{0.300000}%
\pgfsetdash{}{0pt}%
\pgfpathmoveto{\pgfqpoint{2.056603in}{3.123972in}}%
\pgfpathlineto{\pgfqpoint{2.124859in}{3.128824in}}%
\pgfpathlineto{\pgfqpoint{2.193138in}{3.133344in}}%
\pgfpathlineto{\pgfqpoint{2.261395in}{3.138175in}}%
\pgfpathlineto{\pgfqpoint{2.329566in}{3.144067in}}%
\pgfpathlineto{\pgfqpoint{2.397541in}{3.151837in}}%
\pgfusepath{stroke}%
\end{pgfscope}%
\begin{pgfscope}%
\pgfpathrectangle{\pgfqpoint{0.647939in}{0.492442in}}{\pgfqpoint{3.079299in}{3.079299in}}%
\pgfusepath{clip}%
\pgfsetbuttcap%
\pgfsetroundjoin%
\pgfsetlinewidth{0.301125pt}%
\definecolor{currentstroke}{rgb}{0.500000,0.500000,0.500000}%
\pgfsetstrokecolor{currentstroke}%
\pgfsetstrokeopacity{0.300000}%
\pgfsetdash{}{0pt}%
\pgfpathmoveto{\pgfqpoint{3.237350in}{2.032092in}}%
\pgfpathlineto{\pgfqpoint{3.203379in}{2.091455in}}%
\pgfpathlineto{\pgfqpoint{3.172981in}{2.152716in}}%
\pgfpathlineto{\pgfqpoint{3.146734in}{2.215856in}}%
\pgfpathlineto{\pgfqpoint{3.125288in}{2.280770in}}%
\pgfpathlineto{\pgfqpoint{3.109300in}{2.347226in}}%
\pgfpathlineto{\pgfqpoint{3.099322in}{2.414838in}}%
\pgfpathlineto{\pgfqpoint{3.095679in}{2.483087in}}%
\pgfpathlineto{\pgfqpoint{3.098381in}{2.551382in}}%
\pgfpathlineto{\pgfqpoint{3.107119in}{2.619172in}}%
\pgfpathlineto{\pgfqpoint{3.121356in}{2.686033in}}%
\pgfpathlineto{\pgfqpoint{3.140439in}{2.751688in}}%
\pgfpathlineto{\pgfqpoint{3.163716in}{2.815990in}}%
\pgfusepath{stroke}%
\end{pgfscope}%
\begin{pgfscope}%
\pgfpathrectangle{\pgfqpoint{0.647939in}{0.492442in}}{\pgfqpoint{3.079299in}{3.079299in}}%
\pgfusepath{clip}%
\pgfsetbuttcap%
\pgfsetroundjoin%
\pgfsetlinewidth{0.301125pt}%
\definecolor{currentstroke}{rgb}{0.500000,0.500000,0.500000}%
\pgfsetstrokecolor{currentstroke}%
\pgfsetstrokeopacity{0.300000}%
\pgfsetdash{}{0pt}%
\pgfpathmoveto{\pgfqpoint{1.207812in}{2.312028in}}%
\pgfpathlineto{\pgfqpoint{1.272593in}{2.334067in}}%
\pgfpathlineto{\pgfqpoint{1.337182in}{2.356666in}}%
\pgfpathlineto{\pgfqpoint{1.401681in}{2.379521in}}%
\pgfpathlineto{\pgfqpoint{1.466204in}{2.402308in}}%
\pgfpathlineto{\pgfqpoint{1.530870in}{2.424685in}}%
\pgfpathlineto{\pgfqpoint{1.595794in}{2.446297in}}%
\pgfpathlineto{\pgfqpoint{1.661076in}{2.466794in}}%
\pgfpathlineto{\pgfqpoint{1.726794in}{2.485843in}}%
\pgfusepath{stroke}%
\end{pgfscope}%
\begin{pgfscope}%
\pgfpathrectangle{\pgfqpoint{0.647939in}{0.492442in}}{\pgfqpoint{3.079299in}{3.079299in}}%
\pgfusepath{clip}%
\pgfsetbuttcap%
\pgfsetroundjoin%
\pgfsetlinewidth{0.301125pt}%
\definecolor{currentstroke}{rgb}{0.500000,0.500000,0.500000}%
\pgfsetstrokecolor{currentstroke}%
\pgfsetstrokeopacity{0.300000}%
\pgfsetdash{}{0pt}%
\pgfpathmoveto{\pgfqpoint{3.097382in}{1.962108in}}%
\pgfpathlineto{\pgfqpoint{3.058847in}{2.018629in}}%
\pgfpathlineto{\pgfqpoint{3.023249in}{2.077037in}}%
\pgfpathlineto{\pgfqpoint{2.991303in}{2.137504in}}%
\pgfpathlineto{\pgfqpoint{2.963934in}{2.200156in}}%
\pgfpathlineto{\pgfqpoint{2.942245in}{2.264970in}}%
\pgfpathlineto{\pgfqpoint{2.927371in}{2.331652in}}%
\pgfpathlineto{\pgfqpoint{2.920183in}{2.399565in}}%
\pgfpathlineto{\pgfqpoint{2.920969in}{2.467846in}}%
\pgfpathlineto{\pgfqpoint{2.929323in}{2.535635in}}%
\pgfpathlineto{\pgfqpoint{2.944321in}{2.602299in}}%
\pgfpathlineto{\pgfqpoint{2.964837in}{2.667510in}}%
\pgfusepath{stroke}%
\end{pgfscope}%
\begin{pgfscope}%
\pgfpathrectangle{\pgfqpoint{0.647939in}{0.492442in}}{\pgfqpoint{3.079299in}{3.079299in}}%
\pgfusepath{clip}%
\pgfsetbuttcap%
\pgfsetroundjoin%
\pgfsetlinewidth{0.301125pt}%
\definecolor{currentstroke}{rgb}{0.500000,0.500000,0.500000}%
\pgfsetstrokecolor{currentstroke}%
\pgfsetstrokeopacity{0.300000}%
\pgfsetdash{}{0pt}%
\pgfpathmoveto{\pgfqpoint{2.422426in}{2.846011in}}%
\pgfpathlineto{\pgfqpoint{2.488762in}{2.862615in}}%
\pgfpathlineto{\pgfqpoint{2.553664in}{2.884106in}}%
\pgfpathlineto{\pgfqpoint{2.616666in}{2.910640in}}%
\pgfpathlineto{\pgfqpoint{2.677477in}{2.941885in}}%
\pgfpathlineto{\pgfqpoint{2.736033in}{2.977191in}}%
\pgfusepath{stroke}%
\end{pgfscope}%
\begin{pgfscope}%
\pgfpathrectangle{\pgfqpoint{0.647939in}{0.492442in}}{\pgfqpoint{3.079299in}{3.079299in}}%
\pgfusepath{clip}%
\pgfsetbuttcap%
\pgfsetroundjoin%
\pgfsetlinewidth{0.301125pt}%
\definecolor{currentstroke}{rgb}{0.500000,0.500000,0.500000}%
\pgfsetstrokecolor{currentstroke}%
\pgfsetstrokeopacity{0.300000}%
\pgfsetdash{}{0pt}%
\pgfpathmoveto{\pgfqpoint{1.774549in}{2.912348in}}%
\pgfpathlineto{\pgfqpoint{1.841986in}{2.923913in}}%
\pgfpathlineto{\pgfqpoint{1.909701in}{2.933728in}}%
\pgfpathlineto{\pgfqpoint{1.977636in}{2.941885in}}%
\pgfpathlineto{\pgfqpoint{2.045730in}{2.948619in}}%
\pgfpathlineto{\pgfqpoint{2.113918in}{2.954335in}}%
\pgfusepath{stroke}%
\end{pgfscope}%
\begin{pgfscope}%
\pgfpathrectangle{\pgfqpoint{0.647939in}{0.492442in}}{\pgfqpoint{3.079299in}{3.079299in}}%
\pgfusepath{clip}%
\pgfsetbuttcap%
\pgfsetroundjoin%
\pgfsetlinewidth{0.301125pt}%
\definecolor{currentstroke}{rgb}{0.500000,0.500000,0.500000}%
\pgfsetstrokecolor{currentstroke}%
\pgfsetstrokeopacity{0.300000}%
\pgfsetdash{}{0pt}%
\pgfpathmoveto{\pgfqpoint{1.277796in}{1.822139in}}%
\pgfpathlineto{\pgfqpoint{1.339772in}{1.851138in}}%
\pgfpathlineto{\pgfqpoint{1.401340in}{1.880999in}}%
\pgfpathlineto{\pgfqpoint{1.462641in}{1.911403in}}%
\pgfpathlineto{\pgfqpoint{1.523840in}{1.942013in}}%
\pgfpathlineto{\pgfqpoint{1.585117in}{1.972467in}}%
\pgfusepath{stroke}%
\end{pgfscope}%
\begin{pgfscope}%
\pgfpathrectangle{\pgfqpoint{0.647939in}{0.492442in}}{\pgfqpoint{3.079299in}{3.079299in}}%
\pgfusepath{clip}%
\pgfsetbuttcap%
\pgfsetroundjoin%
\pgfsetlinewidth{0.301125pt}%
\definecolor{currentstroke}{rgb}{0.500000,0.500000,0.500000}%
\pgfsetstrokecolor{currentstroke}%
\pgfsetstrokeopacity{0.300000}%
\pgfsetdash{}{0pt}%
\pgfpathmoveto{\pgfqpoint{1.915534in}{2.772260in}}%
\pgfpathlineto{\pgfqpoint{1.983340in}{2.781434in}}%
\pgfpathlineto{\pgfqpoint{2.051335in}{2.789091in}}%
\pgfpathlineto{\pgfqpoint{2.119443in}{2.795696in}}%
\pgfpathlineto{\pgfqpoint{2.187589in}{2.801916in}}%
\pgfpathlineto{\pgfqpoint{2.255686in}{2.808622in}}%
\pgfusepath{stroke}%
\end{pgfscope}%
\begin{pgfscope}%
\pgfpathrectangle{\pgfqpoint{0.647939in}{0.492442in}}{\pgfqpoint{3.079299in}{3.079299in}}%
\pgfusepath{clip}%
\pgfsetbuttcap%
\pgfsetroundjoin%
\pgfsetlinewidth{0.301125pt}%
\definecolor{currentstroke}{rgb}{0.500000,0.500000,0.500000}%
\pgfsetstrokecolor{currentstroke}%
\pgfsetstrokeopacity{0.300000}%
\pgfsetdash{}{0pt}%
\pgfpathmoveto{\pgfqpoint{1.353903in}{2.217462in}}%
\pgfpathlineto{\pgfqpoint{1.417764in}{2.242044in}}%
\pgfpathlineto{\pgfqpoint{1.481632in}{2.266606in}}%
\pgfpathlineto{\pgfqpoint{1.545643in}{2.290793in}}%
\pgfpathlineto{\pgfqpoint{1.609928in}{2.314238in}}%
\pgfpathlineto{\pgfqpoint{1.674606in}{2.336570in}}%
\pgfpathlineto{\pgfqpoint{1.739770in}{2.357433in}}%
\pgfpathlineto{\pgfqpoint{1.805477in}{2.376513in}}%
\pgfpathlineto{\pgfqpoint{1.871733in}{2.393578in}}%
\pgfpathlineto{\pgfqpoint{1.938498in}{2.408528in}}%
\pgfusepath{stroke}%
\end{pgfscope}%
\begin{pgfscope}%
\pgfpathrectangle{\pgfqpoint{0.647939in}{0.492442in}}{\pgfqpoint{3.079299in}{3.079299in}}%
\pgfusepath{clip}%
\pgfsetbuttcap%
\pgfsetroundjoin%
\pgfsetlinewidth{0.301125pt}%
\definecolor{currentstroke}{rgb}{0.500000,0.500000,0.500000}%
\pgfsetstrokecolor{currentstroke}%
\pgfsetstrokeopacity{0.300000}%
\pgfsetdash{}{0pt}%
\pgfpathmoveto{\pgfqpoint{2.887429in}{1.752155in}}%
\pgfpathlineto{\pgfqpoint{2.835100in}{1.796240in}}%
\pgfpathlineto{\pgfqpoint{2.782980in}{1.840569in}}%
\pgfpathlineto{\pgfqpoint{2.731201in}{1.885292in}}%
\pgfpathlineto{\pgfqpoint{2.680047in}{1.930724in}}%
\pgfpathlineto{\pgfqpoint{2.630093in}{1.977454in}}%
\pgfpathlineto{\pgfqpoint{2.582503in}{2.026545in}}%
\pgfpathlineto{\pgfqpoint{2.539863in}{2.079873in}}%
\pgfpathlineto{\pgfqpoint{2.508059in}{2.139845in}}%
\pgfpathlineto{\pgfqpoint{2.508059in}{2.139845in}}%
\pgfpathlineto{\pgfqpoint{2.496754in}{2.187561in}}%
\pgfpathlineto{\pgfqpoint{2.498015in}{2.237178in}}%
\pgfpathlineto{\pgfqpoint{2.509581in}{2.284827in}}%
\pgfpathlineto{\pgfqpoint{2.531087in}{2.337104in}}%
\pgfusepath{stroke}%
\end{pgfscope}%
\begin{pgfscope}%
\pgfpathrectangle{\pgfqpoint{0.647939in}{0.492442in}}{\pgfqpoint{3.079299in}{3.079299in}}%
\pgfusepath{clip}%
\pgfsetbuttcap%
\pgfsetroundjoin%
\pgfsetlinewidth{0.301125pt}%
\definecolor{currentstroke}{rgb}{0.500000,0.500000,0.500000}%
\pgfsetstrokecolor{currentstroke}%
\pgfsetstrokeopacity{0.300000}%
\pgfsetdash{}{0pt}%
\pgfpathmoveto{\pgfqpoint{2.607493in}{2.731932in}}%
\pgfpathlineto{\pgfqpoint{2.663999in}{2.770424in}}%
\pgfpathlineto{\pgfqpoint{2.717992in}{2.812367in}}%
\pgfpathlineto{\pgfqpoint{2.769923in}{2.856863in}}%
\pgfpathlineto{\pgfqpoint{2.820240in}{2.903203in}}%
\pgfpathlineto{\pgfqpoint{2.869334in}{2.950841in}}%
\pgfpathlineto{\pgfqpoint{2.917548in}{2.999382in}}%
\pgfusepath{stroke}%
\end{pgfscope}%
\begin{pgfscope}%
\pgfpathrectangle{\pgfqpoint{0.647939in}{0.492442in}}{\pgfqpoint{3.079299in}{3.079299in}}%
\pgfusepath{clip}%
\pgfsetbuttcap%
\pgfsetroundjoin%
\pgfsetlinewidth{0.301125pt}%
\definecolor{currentstroke}{rgb}{0.500000,0.500000,0.500000}%
\pgfsetstrokecolor{currentstroke}%
\pgfsetstrokeopacity{0.300000}%
\pgfsetdash{}{0pt}%
\pgfpathmoveto{\pgfqpoint{1.627716in}{1.962108in}}%
\pgfpathlineto{\pgfqpoint{1.689094in}{1.992353in}}%
\pgfpathlineto{\pgfqpoint{1.751013in}{2.021470in}}%
\pgfpathlineto{\pgfqpoint{1.813613in}{2.049086in}}%
\pgfpathlineto{\pgfqpoint{1.876981in}{2.074886in}}%
\pgfpathlineto{\pgfqpoint{1.941124in}{2.098687in}}%
\pgfpathlineto{\pgfqpoint{2.005954in}{2.120551in}}%
\pgfusepath{stroke}%
\end{pgfscope}%
\begin{pgfscope}%
\pgfpathrectangle{\pgfqpoint{0.647939in}{0.492442in}}{\pgfqpoint{3.079299in}{3.079299in}}%
\pgfusepath{clip}%
\pgfsetbuttcap%
\pgfsetroundjoin%
\pgfsetlinewidth{0.301125pt}%
\definecolor{currentstroke}{rgb}{0.500000,0.500000,0.500000}%
\pgfsetstrokecolor{currentstroke}%
\pgfsetstrokeopacity{0.300000}%
\pgfsetdash{}{0pt}%
\pgfpathmoveto{\pgfqpoint{2.322355in}{1.518405in}}%
\pgfpathlineto{\pgfqpoint{2.254949in}{1.530181in}}%
\pgfpathlineto{\pgfqpoint{2.187589in}{1.542203in}}%
\pgfpathlineto{\pgfqpoint{2.120789in}{1.556859in}}%
\pgfpathlineto{\pgfqpoint{2.056150in}{1.578582in}}%
\pgfpathlineto{\pgfqpoint{2.056150in}{1.578582in}}%
\pgfpathlineto{\pgfqpoint{2.013244in}{1.603248in}}%
\pgfpathlineto{\pgfqpoint{2.013244in}{1.603248in}}%
\pgfusepath{stroke}%
\end{pgfscope}%
\begin{pgfscope}%
\pgfpathrectangle{\pgfqpoint{0.647939in}{0.492442in}}{\pgfqpoint{3.079299in}{3.079299in}}%
\pgfusepath{clip}%
\pgfsetbuttcap%
\pgfsetroundjoin%
\pgfsetlinewidth{0.301125pt}%
\definecolor{currentstroke}{rgb}{0.500000,0.500000,0.500000}%
\pgfsetstrokecolor{currentstroke}%
\pgfsetstrokeopacity{0.300000}%
\pgfsetdash{}{0pt}%
\pgfpathmoveto{\pgfqpoint{2.665316in}{1.785563in}}%
\pgfpathlineto{\pgfqpoint{2.607493in}{1.822139in}}%
\pgfpathlineto{\pgfqpoint{2.549235in}{1.858023in}}%
\pgfpathlineto{\pgfqpoint{2.490816in}{1.893644in}}%
\pgfpathlineto{\pgfqpoint{2.432912in}{1.930069in}}%
\pgfpathlineto{\pgfqpoint{2.377445in}{1.969923in}}%
\pgfpathlineto{\pgfqpoint{2.377445in}{1.969923in}}%
\pgfpathlineto{\pgfqpoint{2.338940in}{2.008319in}}%
\pgfpathlineto{\pgfqpoint{2.338940in}{2.008319in}}%
\pgfpathlineto{\pgfqpoint{2.321243in}{2.039951in}}%
\pgfpathlineto{\pgfqpoint{2.316515in}{2.077874in}}%
\pgfpathlineto{\pgfqpoint{2.323003in}{2.110323in}}%
\pgfusepath{stroke}%
\end{pgfscope}%
\begin{pgfscope}%
\pgfpathrectangle{\pgfqpoint{0.647939in}{0.492442in}}{\pgfqpoint{3.079299in}{3.079299in}}%
\pgfusepath{clip}%
\pgfsetbuttcap%
\pgfsetroundjoin%
\pgfsetlinewidth{0.301125pt}%
\definecolor{currentstroke}{rgb}{0.500000,0.500000,0.500000}%
\pgfsetstrokecolor{currentstroke}%
\pgfsetstrokeopacity{0.300000}%
\pgfsetdash{}{0pt}%
\pgfpathmoveto{\pgfqpoint{1.987627in}{2.406921in}}%
\pgfpathlineto{\pgfqpoint{2.055016in}{2.418773in}}%
\pgfpathlineto{\pgfqpoint{2.122602in}{2.429469in}}%
\pgfpathlineto{\pgfqpoint{2.190207in}{2.440045in}}%
\pgfpathlineto{\pgfqpoint{2.257573in}{2.451996in}}%
\pgfpathlineto{\pgfqpoint{2.324239in}{2.467236in}}%
\pgfusepath{stroke}%
\end{pgfscope}%
\begin{pgfscope}%
\pgfpathrectangle{\pgfqpoint{0.647939in}{0.492442in}}{\pgfqpoint{3.079299in}{3.079299in}}%
\pgfusepath{clip}%
\pgfsetbuttcap%
\pgfsetroundjoin%
\pgfsetlinewidth{0.301125pt}%
\definecolor{currentstroke}{rgb}{0.500000,0.500000,0.500000}%
\pgfsetstrokecolor{currentstroke}%
\pgfsetstrokeopacity{0.300000}%
\pgfsetdash{}{0pt}%
\pgfpathmoveto{\pgfqpoint{2.461921in}{1.729009in}}%
\pgfpathlineto{\pgfqpoint{2.397541in}{1.752155in}}%
\pgfpathlineto{\pgfqpoint{2.332733in}{1.774096in}}%
\pgfpathlineto{\pgfqpoint{2.268107in}{1.796555in}}%
\pgfpathlineto{\pgfqpoint{2.205519in}{1.823821in}}%
\pgfpathlineto{\pgfqpoint{2.205519in}{1.823821in}}%
\pgfpathlineto{\pgfqpoint{2.169560in}{1.848250in}}%
\pgfpathlineto{\pgfqpoint{2.169560in}{1.848250in}}%
\pgfusepath{stroke}%
\end{pgfscope}%
\begin{pgfscope}%
\pgfpathrectangle{\pgfqpoint{0.647939in}{0.492442in}}{\pgfqpoint{3.079299in}{3.079299in}}%
\pgfusepath{clip}%
\pgfsetbuttcap%
\pgfsetroundjoin%
\pgfsetlinewidth{0.301125pt}%
\definecolor{currentstroke}{rgb}{0.500000,0.500000,0.500000}%
\pgfsetstrokecolor{currentstroke}%
\pgfsetstrokeopacity{0.300000}%
\pgfsetdash{}{0pt}%
\pgfpathmoveto{\pgfqpoint{1.924757in}{2.233168in}}%
\pgfpathlineto{\pgfqpoint{1.990984in}{2.250349in}}%
\pgfpathlineto{\pgfqpoint{2.057655in}{2.265735in}}%
\pgfpathlineto{\pgfqpoint{2.124556in}{2.280105in}}%
\pgfpathlineto{\pgfqpoint{2.191373in}{2.294845in}}%
\pgfpathlineto{\pgfqpoint{2.257573in}{2.312028in}}%
\pgfpathlineto{\pgfqpoint{2.322189in}{2.334220in}}%
\pgfpathlineto{\pgfqpoint{2.383935in}{2.363293in}}%
\pgfpathlineto{\pgfqpoint{2.441819in}{2.399385in}}%
\pgfusepath{stroke}%
\end{pgfscope}%
\begin{pgfscope}%
\pgfpathrectangle{\pgfqpoint{0.647939in}{0.492442in}}{\pgfqpoint{3.079299in}{3.079299in}}%
\pgfusepath{clip}%
\pgfsetroundcap%
\pgfsetroundjoin%
\pgfsetlinewidth{0.301125pt}%
\definecolor{currentstroke}{rgb}{0.500000,0.500000,0.500000}%
\pgfsetstrokecolor{currentstroke}%
\pgfsetstrokeopacity{0.300000}%
\pgfsetdash{}{0pt}%
\pgfpathmoveto{\pgfqpoint{2.115544in}{1.974247in}}%
\pgfusepath{stroke}%
\end{pgfscope}%
\begin{pgfscope}%
\pgfpathrectangle{\pgfqpoint{0.647939in}{0.492442in}}{\pgfqpoint{3.079299in}{3.079299in}}%
\pgfusepath{clip}%
\pgfsetroundcap%
\pgfsetroundjoin%
\definecolor{currentfill}{rgb}{0.500000,0.500000,0.500000}%
\pgfsetfillcolor{currentfill}%
\pgfsetfillopacity{0.300000}%
\pgfsetlinewidth{0.301125pt}%
\definecolor{currentstroke}{rgb}{0.500000,0.500000,0.500000}%
\pgfsetstrokecolor{currentstroke}%
\pgfsetstrokeopacity{0.300000}%
\pgfsetdash{}{0pt}%
\pgfpathmoveto{\pgfqpoint{0.000000in}{0.000000in}}%
\pgfpathlineto{\pgfqpoint{0.000000in}{0.000000in}}%
\pgfpathclose%
\pgfusepath{stroke,fill}%
\end{pgfscope}%
\begin{pgfscope}%
\pgfpathrectangle{\pgfqpoint{0.647939in}{0.492442in}}{\pgfqpoint{3.079299in}{3.079299in}}%
\pgfusepath{clip}%
\pgfsetroundcap%
\pgfsetroundjoin%
\pgfsetlinewidth{0.301125pt}%
\definecolor{currentstroke}{rgb}{0.500000,0.500000,0.500000}%
\pgfsetstrokecolor{currentstroke}%
\pgfsetstrokeopacity{0.300000}%
\pgfsetdash{}{0pt}%
\pgfpathmoveto{\pgfqpoint{1.121833in}{0.620257in}}%
\pgfusepath{stroke}%
\end{pgfscope}%
\begin{pgfscope}%
\pgfpathrectangle{\pgfqpoint{0.647939in}{0.492442in}}{\pgfqpoint{3.079299in}{3.079299in}}%
\pgfusepath{clip}%
\pgfsetroundcap%
\pgfsetroundjoin%
\definecolor{currentfill}{rgb}{0.500000,0.500000,0.500000}%
\pgfsetfillcolor{currentfill}%
\pgfsetfillopacity{0.300000}%
\pgfsetlinewidth{0.301125pt}%
\definecolor{currentstroke}{rgb}{0.500000,0.500000,0.500000}%
\pgfsetstrokecolor{currentstroke}%
\pgfsetstrokeopacity{0.300000}%
\pgfsetdash{}{0pt}%
\pgfpathmoveto{\pgfqpoint{0.000000in}{0.000000in}}%
\pgfpathlineto{\pgfqpoint{0.000000in}{0.000000in}}%
\pgfpathclose%
\pgfusepath{stroke,fill}%
\end{pgfscope}%
\begin{pgfscope}%
\pgfpathrectangle{\pgfqpoint{0.647939in}{0.492442in}}{\pgfqpoint{3.079299in}{3.079299in}}%
\pgfusepath{clip}%
\pgfsetroundcap%
\pgfsetroundjoin%
\pgfsetlinewidth{0.301125pt}%
\definecolor{currentstroke}{rgb}{0.500000,0.500000,0.500000}%
\pgfsetstrokecolor{currentstroke}%
\pgfsetstrokeopacity{0.300000}%
\pgfsetdash{}{0pt}%
\pgfpathmoveto{\pgfqpoint{1.240630in}{0.620625in}}%
\pgfusepath{stroke}%
\end{pgfscope}%
\begin{pgfscope}%
\pgfpathrectangle{\pgfqpoint{0.647939in}{0.492442in}}{\pgfqpoint{3.079299in}{3.079299in}}%
\pgfusepath{clip}%
\pgfsetroundcap%
\pgfsetroundjoin%
\definecolor{currentfill}{rgb}{0.500000,0.500000,0.500000}%
\pgfsetfillcolor{currentfill}%
\pgfsetfillopacity{0.300000}%
\pgfsetlinewidth{0.301125pt}%
\definecolor{currentstroke}{rgb}{0.500000,0.500000,0.500000}%
\pgfsetstrokecolor{currentstroke}%
\pgfsetstrokeopacity{0.300000}%
\pgfsetdash{}{0pt}%
\pgfpathmoveto{\pgfqpoint{0.000000in}{0.000000in}}%
\pgfpathlineto{\pgfqpoint{0.000000in}{0.000000in}}%
\pgfpathclose%
\pgfusepath{stroke,fill}%
\end{pgfscope}%
\begin{pgfscope}%
\pgfpathrectangle{\pgfqpoint{0.647939in}{0.492442in}}{\pgfqpoint{3.079299in}{3.079299in}}%
\pgfusepath{clip}%
\pgfsetroundcap%
\pgfsetroundjoin%
\pgfsetlinewidth{0.301125pt}%
\definecolor{currentstroke}{rgb}{0.500000,0.500000,0.500000}%
\pgfsetstrokecolor{currentstroke}%
\pgfsetstrokeopacity{0.300000}%
\pgfsetdash{}{0pt}%
\pgfpathmoveto{\pgfqpoint{1.449468in}{0.917136in}}%
\pgfusepath{stroke}%
\end{pgfscope}%
\begin{pgfscope}%
\pgfpathrectangle{\pgfqpoint{0.647939in}{0.492442in}}{\pgfqpoint{3.079299in}{3.079299in}}%
\pgfusepath{clip}%
\pgfsetroundcap%
\pgfsetroundjoin%
\definecolor{currentfill}{rgb}{0.500000,0.500000,0.500000}%
\pgfsetfillcolor{currentfill}%
\pgfsetfillopacity{0.300000}%
\pgfsetlinewidth{0.301125pt}%
\definecolor{currentstroke}{rgb}{0.500000,0.500000,0.500000}%
\pgfsetstrokecolor{currentstroke}%
\pgfsetstrokeopacity{0.300000}%
\pgfsetdash{}{0pt}%
\pgfpathmoveto{\pgfqpoint{0.000000in}{0.000000in}}%
\pgfpathlineto{\pgfqpoint{0.000000in}{0.000000in}}%
\pgfpathclose%
\pgfusepath{stroke,fill}%
\end{pgfscope}%
\begin{pgfscope}%
\pgfpathrectangle{\pgfqpoint{0.647939in}{0.492442in}}{\pgfqpoint{3.079299in}{3.079299in}}%
\pgfusepath{clip}%
\pgfsetroundcap%
\pgfsetroundjoin%
\pgfsetlinewidth{0.301125pt}%
\definecolor{currentstroke}{rgb}{0.500000,0.500000,0.500000}%
\pgfsetstrokecolor{currentstroke}%
\pgfsetstrokeopacity{0.300000}%
\pgfsetdash{}{0pt}%
\pgfpathmoveto{\pgfqpoint{1.468984in}{0.687402in}}%
\pgfusepath{stroke}%
\end{pgfscope}%
\begin{pgfscope}%
\pgfpathrectangle{\pgfqpoint{0.647939in}{0.492442in}}{\pgfqpoint{3.079299in}{3.079299in}}%
\pgfusepath{clip}%
\pgfsetroundcap%
\pgfsetroundjoin%
\definecolor{currentfill}{rgb}{0.500000,0.500000,0.500000}%
\pgfsetfillcolor{currentfill}%
\pgfsetfillopacity{0.300000}%
\pgfsetlinewidth{0.301125pt}%
\definecolor{currentstroke}{rgb}{0.500000,0.500000,0.500000}%
\pgfsetstrokecolor{currentstroke}%
\pgfsetstrokeopacity{0.300000}%
\pgfsetdash{}{0pt}%
\pgfpathmoveto{\pgfqpoint{0.000000in}{0.000000in}}%
\pgfpathlineto{\pgfqpoint{0.000000in}{0.000000in}}%
\pgfpathclose%
\pgfusepath{stroke,fill}%
\end{pgfscope}%
\begin{pgfscope}%
\pgfpathrectangle{\pgfqpoint{0.647939in}{0.492442in}}{\pgfqpoint{3.079299in}{3.079299in}}%
\pgfusepath{clip}%
\pgfsetroundcap%
\pgfsetroundjoin%
\pgfsetlinewidth{0.301125pt}%
\definecolor{currentstroke}{rgb}{0.500000,0.500000,0.500000}%
\pgfsetstrokecolor{currentstroke}%
\pgfsetstrokeopacity{0.300000}%
\pgfsetdash{}{0pt}%
\pgfpathmoveto{\pgfqpoint{1.622442in}{0.567007in}}%
\pgfusepath{stroke}%
\end{pgfscope}%
\begin{pgfscope}%
\pgfpathrectangle{\pgfqpoint{0.647939in}{0.492442in}}{\pgfqpoint{3.079299in}{3.079299in}}%
\pgfusepath{clip}%
\pgfsetroundcap%
\pgfsetroundjoin%
\definecolor{currentfill}{rgb}{0.500000,0.500000,0.500000}%
\pgfsetfillcolor{currentfill}%
\pgfsetfillopacity{0.300000}%
\pgfsetlinewidth{0.301125pt}%
\definecolor{currentstroke}{rgb}{0.500000,0.500000,0.500000}%
\pgfsetstrokecolor{currentstroke}%
\pgfsetstrokeopacity{0.300000}%
\pgfsetdash{}{0pt}%
\pgfpathmoveto{\pgfqpoint{0.000000in}{0.000000in}}%
\pgfpathlineto{\pgfqpoint{0.000000in}{0.000000in}}%
\pgfpathclose%
\pgfusepath{stroke,fill}%
\end{pgfscope}%
\begin{pgfscope}%
\pgfpathrectangle{\pgfqpoint{0.647939in}{0.492442in}}{\pgfqpoint{3.079299in}{3.079299in}}%
\pgfusepath{clip}%
\pgfsetroundcap%
\pgfsetroundjoin%
\pgfsetlinewidth{0.301125pt}%
\definecolor{currentstroke}{rgb}{0.500000,0.500000,0.500000}%
\pgfsetstrokecolor{currentstroke}%
\pgfsetstrokeopacity{0.300000}%
\pgfsetdash{}{0pt}%
\pgfpathmoveto{\pgfqpoint{1.884391in}{0.512787in}}%
\pgfusepath{stroke}%
\end{pgfscope}%
\begin{pgfscope}%
\pgfpathrectangle{\pgfqpoint{0.647939in}{0.492442in}}{\pgfqpoint{3.079299in}{3.079299in}}%
\pgfusepath{clip}%
\pgfsetroundcap%
\pgfsetroundjoin%
\definecolor{currentfill}{rgb}{0.500000,0.500000,0.500000}%
\pgfsetfillcolor{currentfill}%
\pgfsetfillopacity{0.300000}%
\pgfsetlinewidth{0.301125pt}%
\definecolor{currentstroke}{rgb}{0.500000,0.500000,0.500000}%
\pgfsetstrokecolor{currentstroke}%
\pgfsetstrokeopacity{0.300000}%
\pgfsetdash{}{0pt}%
\pgfpathmoveto{\pgfqpoint{0.000000in}{0.000000in}}%
\pgfpathlineto{\pgfqpoint{0.000000in}{0.000000in}}%
\pgfpathclose%
\pgfusepath{stroke,fill}%
\end{pgfscope}%
\begin{pgfscope}%
\pgfpathrectangle{\pgfqpoint{0.647939in}{0.492442in}}{\pgfqpoint{3.079299in}{3.079299in}}%
\pgfusepath{clip}%
\pgfsetroundcap%
\pgfsetroundjoin%
\pgfsetlinewidth{0.301125pt}%
\definecolor{currentstroke}{rgb}{0.500000,0.500000,0.500000}%
\pgfsetstrokecolor{currentstroke}%
\pgfsetstrokeopacity{0.300000}%
\pgfsetdash{}{0pt}%
\pgfpathmoveto{\pgfqpoint{2.303433in}{0.505427in}}%
\pgfusepath{stroke}%
\end{pgfscope}%
\begin{pgfscope}%
\pgfpathrectangle{\pgfqpoint{0.647939in}{0.492442in}}{\pgfqpoint{3.079299in}{3.079299in}}%
\pgfusepath{clip}%
\pgfsetroundcap%
\pgfsetroundjoin%
\definecolor{currentfill}{rgb}{0.500000,0.500000,0.500000}%
\pgfsetfillcolor{currentfill}%
\pgfsetfillopacity{0.300000}%
\pgfsetlinewidth{0.301125pt}%
\definecolor{currentstroke}{rgb}{0.500000,0.500000,0.500000}%
\pgfsetstrokecolor{currentstroke}%
\pgfsetstrokeopacity{0.300000}%
\pgfsetdash{}{0pt}%
\pgfpathmoveto{\pgfqpoint{0.000000in}{0.000000in}}%
\pgfpathlineto{\pgfqpoint{0.000000in}{0.000000in}}%
\pgfpathclose%
\pgfusepath{stroke,fill}%
\end{pgfscope}%
\begin{pgfscope}%
\pgfpathrectangle{\pgfqpoint{0.647939in}{0.492442in}}{\pgfqpoint{3.079299in}{3.079299in}}%
\pgfusepath{clip}%
\pgfsetroundcap%
\pgfsetroundjoin%
\pgfsetlinewidth{0.301125pt}%
\definecolor{currentstroke}{rgb}{0.500000,0.500000,0.500000}%
\pgfsetstrokecolor{currentstroke}%
\pgfsetstrokeopacity{0.300000}%
\pgfsetdash{}{0pt}%
\pgfpathmoveto{\pgfqpoint{1.972368in}{0.575390in}}%
\pgfusepath{stroke}%
\end{pgfscope}%
\begin{pgfscope}%
\pgfpathrectangle{\pgfqpoint{0.647939in}{0.492442in}}{\pgfqpoint{3.079299in}{3.079299in}}%
\pgfusepath{clip}%
\pgfsetroundcap%
\pgfsetroundjoin%
\definecolor{currentfill}{rgb}{0.500000,0.500000,0.500000}%
\pgfsetfillcolor{currentfill}%
\pgfsetfillopacity{0.300000}%
\pgfsetlinewidth{0.301125pt}%
\definecolor{currentstroke}{rgb}{0.500000,0.500000,0.500000}%
\pgfsetstrokecolor{currentstroke}%
\pgfsetstrokeopacity{0.300000}%
\pgfsetdash{}{0pt}%
\pgfpathmoveto{\pgfqpoint{0.000000in}{0.000000in}}%
\pgfpathlineto{\pgfqpoint{0.000000in}{0.000000in}}%
\pgfpathclose%
\pgfusepath{stroke,fill}%
\end{pgfscope}%
\begin{pgfscope}%
\pgfpathrectangle{\pgfqpoint{0.647939in}{0.492442in}}{\pgfqpoint{3.079299in}{3.079299in}}%
\pgfusepath{clip}%
\pgfsetroundcap%
\pgfsetroundjoin%
\pgfsetlinewidth{0.301125pt}%
\definecolor{currentstroke}{rgb}{0.500000,0.500000,0.500000}%
\pgfsetstrokecolor{currentstroke}%
\pgfsetstrokeopacity{0.300000}%
\pgfsetdash{}{0pt}%
\pgfpathmoveto{\pgfqpoint{2.799201in}{0.537786in}}%
\pgfusepath{stroke}%
\end{pgfscope}%
\begin{pgfscope}%
\pgfpathrectangle{\pgfqpoint{0.647939in}{0.492442in}}{\pgfqpoint{3.079299in}{3.079299in}}%
\pgfusepath{clip}%
\pgfsetroundcap%
\pgfsetroundjoin%
\definecolor{currentfill}{rgb}{0.500000,0.500000,0.500000}%
\pgfsetfillcolor{currentfill}%
\pgfsetfillopacity{0.300000}%
\pgfsetlinewidth{0.301125pt}%
\definecolor{currentstroke}{rgb}{0.500000,0.500000,0.500000}%
\pgfsetstrokecolor{currentstroke}%
\pgfsetstrokeopacity{0.300000}%
\pgfsetdash{}{0pt}%
\pgfpathmoveto{\pgfqpoint{0.000000in}{0.000000in}}%
\pgfpathlineto{\pgfqpoint{0.000000in}{0.000000in}}%
\pgfpathclose%
\pgfusepath{stroke,fill}%
\end{pgfscope}%
\begin{pgfscope}%
\pgfpathrectangle{\pgfqpoint{0.647939in}{0.492442in}}{\pgfqpoint{3.079299in}{3.079299in}}%
\pgfusepath{clip}%
\pgfsetroundcap%
\pgfsetroundjoin%
\pgfsetlinewidth{0.301125pt}%
\definecolor{currentstroke}{rgb}{0.500000,0.500000,0.500000}%
\pgfsetstrokecolor{currentstroke}%
\pgfsetstrokeopacity{0.300000}%
\pgfsetdash{}{0pt}%
\pgfpathmoveto{\pgfqpoint{2.209117in}{0.699616in}}%
\pgfusepath{stroke}%
\end{pgfscope}%
\begin{pgfscope}%
\pgfpathrectangle{\pgfqpoint{0.647939in}{0.492442in}}{\pgfqpoint{3.079299in}{3.079299in}}%
\pgfusepath{clip}%
\pgfsetroundcap%
\pgfsetroundjoin%
\definecolor{currentfill}{rgb}{0.500000,0.500000,0.500000}%
\pgfsetfillcolor{currentfill}%
\pgfsetfillopacity{0.300000}%
\pgfsetlinewidth{0.301125pt}%
\definecolor{currentstroke}{rgb}{0.500000,0.500000,0.500000}%
\pgfsetstrokecolor{currentstroke}%
\pgfsetstrokeopacity{0.300000}%
\pgfsetdash{}{0pt}%
\pgfpathmoveto{\pgfqpoint{0.000000in}{0.000000in}}%
\pgfpathlineto{\pgfqpoint{0.000000in}{0.000000in}}%
\pgfpathclose%
\pgfusepath{stroke,fill}%
\end{pgfscope}%
\begin{pgfscope}%
\pgfpathrectangle{\pgfqpoint{0.647939in}{0.492442in}}{\pgfqpoint{3.079299in}{3.079299in}}%
\pgfusepath{clip}%
\pgfsetroundcap%
\pgfsetroundjoin%
\pgfsetlinewidth{0.301125pt}%
\definecolor{currentstroke}{rgb}{0.500000,0.500000,0.500000}%
\pgfsetstrokecolor{currentstroke}%
\pgfsetstrokeopacity{0.300000}%
\pgfsetdash{}{0pt}%
\pgfpathmoveto{\pgfqpoint{2.573601in}{0.759068in}}%
\pgfusepath{stroke}%
\end{pgfscope}%
\begin{pgfscope}%
\pgfpathrectangle{\pgfqpoint{0.647939in}{0.492442in}}{\pgfqpoint{3.079299in}{3.079299in}}%
\pgfusepath{clip}%
\pgfsetroundcap%
\pgfsetroundjoin%
\definecolor{currentfill}{rgb}{0.500000,0.500000,0.500000}%
\pgfsetfillcolor{currentfill}%
\pgfsetfillopacity{0.300000}%
\pgfsetlinewidth{0.301125pt}%
\definecolor{currentstroke}{rgb}{0.500000,0.500000,0.500000}%
\pgfsetstrokecolor{currentstroke}%
\pgfsetstrokeopacity{0.300000}%
\pgfsetdash{}{0pt}%
\pgfpathmoveto{\pgfqpoint{0.000000in}{0.000000in}}%
\pgfpathlineto{\pgfqpoint{0.000000in}{0.000000in}}%
\pgfpathclose%
\pgfusepath{stroke,fill}%
\end{pgfscope}%
\begin{pgfscope}%
\pgfpathrectangle{\pgfqpoint{0.647939in}{0.492442in}}{\pgfqpoint{3.079299in}{3.079299in}}%
\pgfusepath{clip}%
\pgfsetroundcap%
\pgfsetroundjoin%
\pgfsetlinewidth{0.301125pt}%
\definecolor{currentstroke}{rgb}{0.500000,0.500000,0.500000}%
\pgfsetstrokecolor{currentstroke}%
\pgfsetstrokeopacity{0.300000}%
\pgfsetdash{}{0pt}%
\pgfpathmoveto{\pgfqpoint{3.175916in}{0.634888in}}%
\pgfusepath{stroke}%
\end{pgfscope}%
\begin{pgfscope}%
\pgfpathrectangle{\pgfqpoint{0.647939in}{0.492442in}}{\pgfqpoint{3.079299in}{3.079299in}}%
\pgfusepath{clip}%
\pgfsetroundcap%
\pgfsetroundjoin%
\definecolor{currentfill}{rgb}{0.500000,0.500000,0.500000}%
\pgfsetfillcolor{currentfill}%
\pgfsetfillopacity{0.300000}%
\pgfsetlinewidth{0.301125pt}%
\definecolor{currentstroke}{rgb}{0.500000,0.500000,0.500000}%
\pgfsetstrokecolor{currentstroke}%
\pgfsetstrokeopacity{0.300000}%
\pgfsetdash{}{0pt}%
\pgfpathmoveto{\pgfqpoint{0.000000in}{0.000000in}}%
\pgfpathlineto{\pgfqpoint{0.000000in}{0.000000in}}%
\pgfpathclose%
\pgfusepath{stroke,fill}%
\end{pgfscope}%
\begin{pgfscope}%
\pgfpathrectangle{\pgfqpoint{0.647939in}{0.492442in}}{\pgfqpoint{3.079299in}{3.079299in}}%
\pgfusepath{clip}%
\pgfsetroundcap%
\pgfsetroundjoin%
\pgfsetlinewidth{0.301125pt}%
\definecolor{currentstroke}{rgb}{0.500000,0.500000,0.500000}%
\pgfsetstrokecolor{currentstroke}%
\pgfsetstrokeopacity{0.300000}%
\pgfsetdash{}{0pt}%
\pgfpathmoveto{\pgfqpoint{2.547910in}{0.926275in}}%
\pgfusepath{stroke}%
\end{pgfscope}%
\begin{pgfscope}%
\pgfpathrectangle{\pgfqpoint{0.647939in}{0.492442in}}{\pgfqpoint{3.079299in}{3.079299in}}%
\pgfusepath{clip}%
\pgfsetroundcap%
\pgfsetroundjoin%
\definecolor{currentfill}{rgb}{0.500000,0.500000,0.500000}%
\pgfsetfillcolor{currentfill}%
\pgfsetfillopacity{0.300000}%
\pgfsetlinewidth{0.301125pt}%
\definecolor{currentstroke}{rgb}{0.500000,0.500000,0.500000}%
\pgfsetstrokecolor{currentstroke}%
\pgfsetstrokeopacity{0.300000}%
\pgfsetdash{}{0pt}%
\pgfpathmoveto{\pgfqpoint{0.000000in}{0.000000in}}%
\pgfpathlineto{\pgfqpoint{0.000000in}{0.000000in}}%
\pgfpathclose%
\pgfusepath{stroke,fill}%
\end{pgfscope}%
\begin{pgfscope}%
\pgfpathrectangle{\pgfqpoint{0.647939in}{0.492442in}}{\pgfqpoint{3.079299in}{3.079299in}}%
\pgfusepath{clip}%
\pgfsetroundcap%
\pgfsetroundjoin%
\pgfsetlinewidth{0.301125pt}%
\definecolor{currentstroke}{rgb}{0.500000,0.500000,0.500000}%
\pgfsetstrokecolor{currentstroke}%
\pgfsetstrokeopacity{0.300000}%
\pgfsetdash{}{0pt}%
\pgfpathmoveto{\pgfqpoint{2.688388in}{0.984721in}}%
\pgfusepath{stroke}%
\end{pgfscope}%
\begin{pgfscope}%
\pgfpathrectangle{\pgfqpoint{0.647939in}{0.492442in}}{\pgfqpoint{3.079299in}{3.079299in}}%
\pgfusepath{clip}%
\pgfsetroundcap%
\pgfsetroundjoin%
\definecolor{currentfill}{rgb}{0.500000,0.500000,0.500000}%
\pgfsetfillcolor{currentfill}%
\pgfsetfillopacity{0.300000}%
\pgfsetlinewidth{0.301125pt}%
\definecolor{currentstroke}{rgb}{0.500000,0.500000,0.500000}%
\pgfsetstrokecolor{currentstroke}%
\pgfsetstrokeopacity{0.300000}%
\pgfsetdash{}{0pt}%
\pgfpathmoveto{\pgfqpoint{0.000000in}{0.000000in}}%
\pgfpathlineto{\pgfqpoint{0.000000in}{0.000000in}}%
\pgfpathclose%
\pgfusepath{stroke,fill}%
\end{pgfscope}%
\begin{pgfscope}%
\pgfpathrectangle{\pgfqpoint{0.647939in}{0.492442in}}{\pgfqpoint{3.079299in}{3.079299in}}%
\pgfusepath{clip}%
\pgfsetroundcap%
\pgfsetroundjoin%
\pgfsetlinewidth{0.301125pt}%
\definecolor{currentstroke}{rgb}{0.500000,0.500000,0.500000}%
\pgfsetstrokecolor{currentstroke}%
\pgfsetstrokeopacity{0.300000}%
\pgfsetdash{}{0pt}%
\pgfpathmoveto{\pgfqpoint{2.825980in}{1.030444in}}%
\pgfusepath{stroke}%
\end{pgfscope}%
\begin{pgfscope}%
\pgfpathrectangle{\pgfqpoint{0.647939in}{0.492442in}}{\pgfqpoint{3.079299in}{3.079299in}}%
\pgfusepath{clip}%
\pgfsetroundcap%
\pgfsetroundjoin%
\definecolor{currentfill}{rgb}{0.500000,0.500000,0.500000}%
\pgfsetfillcolor{currentfill}%
\pgfsetfillopacity{0.300000}%
\pgfsetlinewidth{0.301125pt}%
\definecolor{currentstroke}{rgb}{0.500000,0.500000,0.500000}%
\pgfsetstrokecolor{currentstroke}%
\pgfsetstrokeopacity{0.300000}%
\pgfsetdash{}{0pt}%
\pgfpathmoveto{\pgfqpoint{0.000000in}{0.000000in}}%
\pgfpathlineto{\pgfqpoint{0.000000in}{0.000000in}}%
\pgfpathclose%
\pgfusepath{stroke,fill}%
\end{pgfscope}%
\begin{pgfscope}%
\pgfpathrectangle{\pgfqpoint{0.647939in}{0.492442in}}{\pgfqpoint{3.079299in}{3.079299in}}%
\pgfusepath{clip}%
\pgfsetroundcap%
\pgfsetroundjoin%
\pgfsetlinewidth{0.301125pt}%
\definecolor{currentstroke}{rgb}{0.500000,0.500000,0.500000}%
\pgfsetstrokecolor{currentstroke}%
\pgfsetstrokeopacity{0.300000}%
\pgfsetdash{}{0pt}%
\pgfpathmoveto{\pgfqpoint{2.639359in}{1.183184in}}%
\pgfusepath{stroke}%
\end{pgfscope}%
\begin{pgfscope}%
\pgfpathrectangle{\pgfqpoint{0.647939in}{0.492442in}}{\pgfqpoint{3.079299in}{3.079299in}}%
\pgfusepath{clip}%
\pgfsetroundcap%
\pgfsetroundjoin%
\definecolor{currentfill}{rgb}{0.500000,0.500000,0.500000}%
\pgfsetfillcolor{currentfill}%
\pgfsetfillopacity{0.300000}%
\pgfsetlinewidth{0.301125pt}%
\definecolor{currentstroke}{rgb}{0.500000,0.500000,0.500000}%
\pgfsetstrokecolor{currentstroke}%
\pgfsetstrokeopacity{0.300000}%
\pgfsetdash{}{0pt}%
\pgfpathmoveto{\pgfqpoint{0.000000in}{0.000000in}}%
\pgfpathlineto{\pgfqpoint{0.000000in}{0.000000in}}%
\pgfpathclose%
\pgfusepath{stroke,fill}%
\end{pgfscope}%
\begin{pgfscope}%
\pgfpathrectangle{\pgfqpoint{0.647939in}{0.492442in}}{\pgfqpoint{3.079299in}{3.079299in}}%
\pgfusepath{clip}%
\pgfsetroundcap%
\pgfsetroundjoin%
\pgfsetlinewidth{0.301125pt}%
\definecolor{currentstroke}{rgb}{0.500000,0.500000,0.500000}%
\pgfsetstrokecolor{currentstroke}%
\pgfsetstrokeopacity{0.300000}%
\pgfsetdash{}{0pt}%
\pgfpathmoveto{\pgfqpoint{2.779175in}{1.231668in}}%
\pgfusepath{stroke}%
\end{pgfscope}%
\begin{pgfscope}%
\pgfpathrectangle{\pgfqpoint{0.647939in}{0.492442in}}{\pgfqpoint{3.079299in}{3.079299in}}%
\pgfusepath{clip}%
\pgfsetroundcap%
\pgfsetroundjoin%
\definecolor{currentfill}{rgb}{0.500000,0.500000,0.500000}%
\pgfsetfillcolor{currentfill}%
\pgfsetfillopacity{0.300000}%
\pgfsetlinewidth{0.301125pt}%
\definecolor{currentstroke}{rgb}{0.500000,0.500000,0.500000}%
\pgfsetstrokecolor{currentstroke}%
\pgfsetstrokeopacity{0.300000}%
\pgfsetdash{}{0pt}%
\pgfpathmoveto{\pgfqpoint{0.000000in}{0.000000in}}%
\pgfpathlineto{\pgfqpoint{0.000000in}{0.000000in}}%
\pgfpathclose%
\pgfusepath{stroke,fill}%
\end{pgfscope}%
\begin{pgfscope}%
\pgfpathrectangle{\pgfqpoint{0.647939in}{0.492442in}}{\pgfqpoint{3.079299in}{3.079299in}}%
\pgfusepath{clip}%
\pgfsetroundcap%
\pgfsetroundjoin%
\pgfsetlinewidth{0.301125pt}%
\definecolor{currentstroke}{rgb}{0.500000,0.500000,0.500000}%
\pgfsetstrokecolor{currentstroke}%
\pgfsetstrokeopacity{0.300000}%
\pgfsetdash{}{0pt}%
\pgfpathmoveto{\pgfqpoint{2.852571in}{1.295564in}}%
\pgfusepath{stroke}%
\end{pgfscope}%
\begin{pgfscope}%
\pgfpathrectangle{\pgfqpoint{0.647939in}{0.492442in}}{\pgfqpoint{3.079299in}{3.079299in}}%
\pgfusepath{clip}%
\pgfsetroundcap%
\pgfsetroundjoin%
\definecolor{currentfill}{rgb}{0.500000,0.500000,0.500000}%
\pgfsetfillcolor{currentfill}%
\pgfsetfillopacity{0.300000}%
\pgfsetlinewidth{0.301125pt}%
\definecolor{currentstroke}{rgb}{0.500000,0.500000,0.500000}%
\pgfsetstrokecolor{currentstroke}%
\pgfsetstrokeopacity{0.300000}%
\pgfsetdash{}{0pt}%
\pgfpathmoveto{\pgfqpoint{0.000000in}{0.000000in}}%
\pgfpathlineto{\pgfqpoint{0.000000in}{0.000000in}}%
\pgfpathclose%
\pgfusepath{stroke,fill}%
\end{pgfscope}%
\begin{pgfscope}%
\pgfpathrectangle{\pgfqpoint{0.647939in}{0.492442in}}{\pgfqpoint{3.079299in}{3.079299in}}%
\pgfusepath{clip}%
\pgfsetroundcap%
\pgfsetroundjoin%
\pgfsetlinewidth{0.301125pt}%
\definecolor{currentstroke}{rgb}{0.500000,0.500000,0.500000}%
\pgfsetstrokecolor{currentstroke}%
\pgfsetstrokeopacity{0.300000}%
\pgfsetdash{}{0pt}%
\pgfpathmoveto{\pgfqpoint{2.864401in}{1.387411in}}%
\pgfusepath{stroke}%
\end{pgfscope}%
\begin{pgfscope}%
\pgfpathrectangle{\pgfqpoint{0.647939in}{0.492442in}}{\pgfqpoint{3.079299in}{3.079299in}}%
\pgfusepath{clip}%
\pgfsetroundcap%
\pgfsetroundjoin%
\definecolor{currentfill}{rgb}{0.500000,0.500000,0.500000}%
\pgfsetfillcolor{currentfill}%
\pgfsetfillopacity{0.300000}%
\pgfsetlinewidth{0.301125pt}%
\definecolor{currentstroke}{rgb}{0.500000,0.500000,0.500000}%
\pgfsetstrokecolor{currentstroke}%
\pgfsetstrokeopacity{0.300000}%
\pgfsetdash{}{0pt}%
\pgfpathmoveto{\pgfqpoint{0.000000in}{0.000000in}}%
\pgfpathlineto{\pgfqpoint{0.000000in}{0.000000in}}%
\pgfpathclose%
\pgfusepath{stroke,fill}%
\end{pgfscope}%
\begin{pgfscope}%
\pgfpathrectangle{\pgfqpoint{0.647939in}{0.492442in}}{\pgfqpoint{3.079299in}{3.079299in}}%
\pgfusepath{clip}%
\pgfsetroundcap%
\pgfsetroundjoin%
\pgfsetlinewidth{0.301125pt}%
\definecolor{currentstroke}{rgb}{0.500000,0.500000,0.500000}%
\pgfsetstrokecolor{currentstroke}%
\pgfsetstrokeopacity{0.300000}%
\pgfsetdash{}{0pt}%
\pgfpathmoveto{\pgfqpoint{2.878290in}{1.481321in}}%
\pgfusepath{stroke}%
\end{pgfscope}%
\begin{pgfscope}%
\pgfpathrectangle{\pgfqpoint{0.647939in}{0.492442in}}{\pgfqpoint{3.079299in}{3.079299in}}%
\pgfusepath{clip}%
\pgfsetroundcap%
\pgfsetroundjoin%
\definecolor{currentfill}{rgb}{0.500000,0.500000,0.500000}%
\pgfsetfillcolor{currentfill}%
\pgfsetfillopacity{0.300000}%
\pgfsetlinewidth{0.301125pt}%
\definecolor{currentstroke}{rgb}{0.500000,0.500000,0.500000}%
\pgfsetstrokecolor{currentstroke}%
\pgfsetstrokeopacity{0.300000}%
\pgfsetdash{}{0pt}%
\pgfpathmoveto{\pgfqpoint{0.000000in}{0.000000in}}%
\pgfpathlineto{\pgfqpoint{0.000000in}{0.000000in}}%
\pgfpathclose%
\pgfusepath{stroke,fill}%
\end{pgfscope}%
\begin{pgfscope}%
\pgfpathrectangle{\pgfqpoint{0.647939in}{0.492442in}}{\pgfqpoint{3.079299in}{3.079299in}}%
\pgfusepath{clip}%
\pgfsetroundcap%
\pgfsetroundjoin%
\pgfsetlinewidth{0.301125pt}%
\definecolor{currentstroke}{rgb}{0.500000,0.500000,0.500000}%
\pgfsetstrokecolor{currentstroke}%
\pgfsetstrokeopacity{0.300000}%
\pgfsetdash{}{0pt}%
\pgfpathmoveto{\pgfqpoint{3.022304in}{1.592106in}}%
\pgfusepath{stroke}%
\end{pgfscope}%
\begin{pgfscope}%
\pgfpathrectangle{\pgfqpoint{0.647939in}{0.492442in}}{\pgfqpoint{3.079299in}{3.079299in}}%
\pgfusepath{clip}%
\pgfsetroundcap%
\pgfsetroundjoin%
\definecolor{currentfill}{rgb}{0.500000,0.500000,0.500000}%
\pgfsetfillcolor{currentfill}%
\pgfsetfillopacity{0.300000}%
\pgfsetlinewidth{0.301125pt}%
\definecolor{currentstroke}{rgb}{0.500000,0.500000,0.500000}%
\pgfsetstrokecolor{currentstroke}%
\pgfsetstrokeopacity{0.300000}%
\pgfsetdash{}{0pt}%
\pgfpathmoveto{\pgfqpoint{0.000000in}{0.000000in}}%
\pgfpathlineto{\pgfqpoint{0.000000in}{0.000000in}}%
\pgfpathclose%
\pgfusepath{stroke,fill}%
\end{pgfscope}%
\begin{pgfscope}%
\pgfpathrectangle{\pgfqpoint{0.647939in}{0.492442in}}{\pgfqpoint{3.079299in}{3.079299in}}%
\pgfusepath{clip}%
\pgfsetroundcap%
\pgfsetroundjoin%
\pgfsetlinewidth{0.301125pt}%
\definecolor{currentstroke}{rgb}{0.500000,0.500000,0.500000}%
\pgfsetstrokecolor{currentstroke}%
\pgfsetstrokeopacity{0.300000}%
\pgfsetdash{}{0pt}%
\pgfpathmoveto{\pgfqpoint{3.011918in}{1.830196in}}%
\pgfusepath{stroke}%
\end{pgfscope}%
\begin{pgfscope}%
\pgfpathrectangle{\pgfqpoint{0.647939in}{0.492442in}}{\pgfqpoint{3.079299in}{3.079299in}}%
\pgfusepath{clip}%
\pgfsetroundcap%
\pgfsetroundjoin%
\definecolor{currentfill}{rgb}{0.500000,0.500000,0.500000}%
\pgfsetfillcolor{currentfill}%
\pgfsetfillopacity{0.300000}%
\pgfsetlinewidth{0.301125pt}%
\definecolor{currentstroke}{rgb}{0.500000,0.500000,0.500000}%
\pgfsetstrokecolor{currentstroke}%
\pgfsetstrokeopacity{0.300000}%
\pgfsetdash{}{0pt}%
\pgfpathmoveto{\pgfqpoint{0.000000in}{0.000000in}}%
\pgfpathlineto{\pgfqpoint{0.000000in}{0.000000in}}%
\pgfpathclose%
\pgfusepath{stroke,fill}%
\end{pgfscope}%
\begin{pgfscope}%
\pgfpathrectangle{\pgfqpoint{0.647939in}{0.492442in}}{\pgfqpoint{3.079299in}{3.079299in}}%
\pgfusepath{clip}%
\pgfsetroundcap%
\pgfsetroundjoin%
\pgfsetlinewidth{0.301125pt}%
\definecolor{currentstroke}{rgb}{0.500000,0.500000,0.500000}%
\pgfsetstrokecolor{currentstroke}%
\pgfsetstrokeopacity{0.300000}%
\pgfsetdash{}{0pt}%
\pgfpathmoveto{\pgfqpoint{2.894972in}{2.162335in}}%
\pgfusepath{stroke}%
\end{pgfscope}%
\begin{pgfscope}%
\pgfpathrectangle{\pgfqpoint{0.647939in}{0.492442in}}{\pgfqpoint{3.079299in}{3.079299in}}%
\pgfusepath{clip}%
\pgfsetroundcap%
\pgfsetroundjoin%
\definecolor{currentfill}{rgb}{0.500000,0.500000,0.500000}%
\pgfsetfillcolor{currentfill}%
\pgfsetfillopacity{0.300000}%
\pgfsetlinewidth{0.301125pt}%
\definecolor{currentstroke}{rgb}{0.500000,0.500000,0.500000}%
\pgfsetstrokecolor{currentstroke}%
\pgfsetstrokeopacity{0.300000}%
\pgfsetdash{}{0pt}%
\pgfpathmoveto{\pgfqpoint{0.000000in}{0.000000in}}%
\pgfpathlineto{\pgfqpoint{0.000000in}{0.000000in}}%
\pgfpathclose%
\pgfusepath{stroke,fill}%
\end{pgfscope}%
\begin{pgfscope}%
\pgfpathrectangle{\pgfqpoint{0.647939in}{0.492442in}}{\pgfqpoint{3.079299in}{3.079299in}}%
\pgfusepath{clip}%
\pgfsetroundcap%
\pgfsetroundjoin%
\pgfsetlinewidth{0.301125pt}%
\definecolor{currentstroke}{rgb}{0.500000,0.500000,0.500000}%
\pgfsetstrokecolor{currentstroke}%
\pgfsetstrokeopacity{0.300000}%
\pgfsetdash{}{0pt}%
\pgfpathmoveto{\pgfqpoint{3.030415in}{2.465272in}}%
\pgfusepath{stroke}%
\end{pgfscope}%
\begin{pgfscope}%
\pgfpathrectangle{\pgfqpoint{0.647939in}{0.492442in}}{\pgfqpoint{3.079299in}{3.079299in}}%
\pgfusepath{clip}%
\pgfsetroundcap%
\pgfsetroundjoin%
\definecolor{currentfill}{rgb}{0.500000,0.500000,0.500000}%
\pgfsetfillcolor{currentfill}%
\pgfsetfillopacity{0.300000}%
\pgfsetlinewidth{0.301125pt}%
\definecolor{currentstroke}{rgb}{0.500000,0.500000,0.500000}%
\pgfsetstrokecolor{currentstroke}%
\pgfsetstrokeopacity{0.300000}%
\pgfsetdash{}{0pt}%
\pgfpathmoveto{\pgfqpoint{0.000000in}{0.000000in}}%
\pgfpathlineto{\pgfqpoint{0.000000in}{0.000000in}}%
\pgfpathclose%
\pgfusepath{stroke,fill}%
\end{pgfscope}%
\begin{pgfscope}%
\pgfpathrectangle{\pgfqpoint{0.647939in}{0.492442in}}{\pgfqpoint{3.079299in}{3.079299in}}%
\pgfusepath{clip}%
\pgfsetroundcap%
\pgfsetroundjoin%
\pgfsetlinewidth{0.301125pt}%
\definecolor{currentstroke}{rgb}{0.500000,0.500000,0.500000}%
\pgfsetstrokecolor{currentstroke}%
\pgfsetstrokeopacity{0.300000}%
\pgfsetdash{}{0pt}%
\pgfpathmoveto{\pgfqpoint{3.183741in}{2.544256in}}%
\pgfusepath{stroke}%
\end{pgfscope}%
\begin{pgfscope}%
\pgfpathrectangle{\pgfqpoint{0.647939in}{0.492442in}}{\pgfqpoint{3.079299in}{3.079299in}}%
\pgfusepath{clip}%
\pgfsetroundcap%
\pgfsetroundjoin%
\definecolor{currentfill}{rgb}{0.500000,0.500000,0.500000}%
\pgfsetfillcolor{currentfill}%
\pgfsetfillopacity{0.300000}%
\pgfsetlinewidth{0.301125pt}%
\definecolor{currentstroke}{rgb}{0.500000,0.500000,0.500000}%
\pgfsetstrokecolor{currentstroke}%
\pgfsetstrokeopacity{0.300000}%
\pgfsetdash{}{0pt}%
\pgfpathmoveto{\pgfqpoint{0.000000in}{0.000000in}}%
\pgfpathlineto{\pgfqpoint{0.000000in}{0.000000in}}%
\pgfpathclose%
\pgfusepath{stroke,fill}%
\end{pgfscope}%
\begin{pgfscope}%
\pgfpathrectangle{\pgfqpoint{0.647939in}{0.492442in}}{\pgfqpoint{3.079299in}{3.079299in}}%
\pgfusepath{clip}%
\pgfsetroundcap%
\pgfsetroundjoin%
\pgfsetlinewidth{0.301125pt}%
\definecolor{currentstroke}{rgb}{0.500000,0.500000,0.500000}%
\pgfsetstrokecolor{currentstroke}%
\pgfsetstrokeopacity{0.300000}%
\pgfsetdash{}{0pt}%
\pgfpathmoveto{\pgfqpoint{3.555229in}{1.838847in}}%
\pgfusepath{stroke}%
\end{pgfscope}%
\begin{pgfscope}%
\pgfpathrectangle{\pgfqpoint{0.647939in}{0.492442in}}{\pgfqpoint{3.079299in}{3.079299in}}%
\pgfusepath{clip}%
\pgfsetroundcap%
\pgfsetroundjoin%
\definecolor{currentfill}{rgb}{0.500000,0.500000,0.500000}%
\pgfsetfillcolor{currentfill}%
\pgfsetfillopacity{0.300000}%
\pgfsetlinewidth{0.301125pt}%
\definecolor{currentstroke}{rgb}{0.500000,0.500000,0.500000}%
\pgfsetstrokecolor{currentstroke}%
\pgfsetstrokeopacity{0.300000}%
\pgfsetdash{}{0pt}%
\pgfpathmoveto{\pgfqpoint{0.000000in}{0.000000in}}%
\pgfpathlineto{\pgfqpoint{0.000000in}{0.000000in}}%
\pgfpathclose%
\pgfusepath{stroke,fill}%
\end{pgfscope}%
\begin{pgfscope}%
\pgfpathrectangle{\pgfqpoint{0.647939in}{0.492442in}}{\pgfqpoint{3.079299in}{3.079299in}}%
\pgfusepath{clip}%
\pgfsetroundcap%
\pgfsetroundjoin%
\pgfsetlinewidth{0.301125pt}%
\definecolor{currentstroke}{rgb}{0.500000,0.500000,0.500000}%
\pgfsetstrokecolor{currentstroke}%
\pgfsetstrokeopacity{0.300000}%
\pgfsetdash{}{0pt}%
\pgfpathmoveto{\pgfqpoint{3.312655in}{2.538104in}}%
\pgfusepath{stroke}%
\end{pgfscope}%
\begin{pgfscope}%
\pgfpathrectangle{\pgfqpoint{0.647939in}{0.492442in}}{\pgfqpoint{3.079299in}{3.079299in}}%
\pgfusepath{clip}%
\pgfsetroundcap%
\pgfsetroundjoin%
\definecolor{currentfill}{rgb}{0.500000,0.500000,0.500000}%
\pgfsetfillcolor{currentfill}%
\pgfsetfillopacity{0.300000}%
\pgfsetlinewidth{0.301125pt}%
\definecolor{currentstroke}{rgb}{0.500000,0.500000,0.500000}%
\pgfsetstrokecolor{currentstroke}%
\pgfsetstrokeopacity{0.300000}%
\pgfsetdash{}{0pt}%
\pgfpathmoveto{\pgfqpoint{0.000000in}{0.000000in}}%
\pgfpathlineto{\pgfqpoint{0.000000in}{0.000000in}}%
\pgfpathclose%
\pgfusepath{stroke,fill}%
\end{pgfscope}%
\begin{pgfscope}%
\pgfpathrectangle{\pgfqpoint{0.647939in}{0.492442in}}{\pgfqpoint{3.079299in}{3.079299in}}%
\pgfusepath{clip}%
\pgfsetroundcap%
\pgfsetroundjoin%
\pgfsetlinewidth{0.301125pt}%
\definecolor{currentstroke}{rgb}{0.500000,0.500000,0.500000}%
\pgfsetstrokecolor{currentstroke}%
\pgfsetstrokeopacity{0.300000}%
\pgfsetdash{}{0pt}%
\pgfpathmoveto{\pgfqpoint{3.424724in}{2.587618in}}%
\pgfusepath{stroke}%
\end{pgfscope}%
\begin{pgfscope}%
\pgfpathrectangle{\pgfqpoint{0.647939in}{0.492442in}}{\pgfqpoint{3.079299in}{3.079299in}}%
\pgfusepath{clip}%
\pgfsetroundcap%
\pgfsetroundjoin%
\definecolor{currentfill}{rgb}{0.500000,0.500000,0.500000}%
\pgfsetfillcolor{currentfill}%
\pgfsetfillopacity{0.300000}%
\pgfsetlinewidth{0.301125pt}%
\definecolor{currentstroke}{rgb}{0.500000,0.500000,0.500000}%
\pgfsetstrokecolor{currentstroke}%
\pgfsetstrokeopacity{0.300000}%
\pgfsetdash{}{0pt}%
\pgfpathmoveto{\pgfqpoint{0.000000in}{0.000000in}}%
\pgfpathlineto{\pgfqpoint{0.000000in}{0.000000in}}%
\pgfpathclose%
\pgfusepath{stroke,fill}%
\end{pgfscope}%
\begin{pgfscope}%
\pgfpathrectangle{\pgfqpoint{0.647939in}{0.492442in}}{\pgfqpoint{3.079299in}{3.079299in}}%
\pgfusepath{clip}%
\pgfsetroundcap%
\pgfsetroundjoin%
\pgfsetlinewidth{0.301125pt}%
\definecolor{currentstroke}{rgb}{0.500000,0.500000,0.500000}%
\pgfsetstrokecolor{currentstroke}%
\pgfsetstrokeopacity{0.300000}%
\pgfsetdash{}{0pt}%
\pgfpathmoveto{\pgfqpoint{3.521651in}{2.557853in}}%
\pgfusepath{stroke}%
\end{pgfscope}%
\begin{pgfscope}%
\pgfpathrectangle{\pgfqpoint{0.647939in}{0.492442in}}{\pgfqpoint{3.079299in}{3.079299in}}%
\pgfusepath{clip}%
\pgfsetroundcap%
\pgfsetroundjoin%
\definecolor{currentfill}{rgb}{0.500000,0.500000,0.500000}%
\pgfsetfillcolor{currentfill}%
\pgfsetfillopacity{0.300000}%
\pgfsetlinewidth{0.301125pt}%
\definecolor{currentstroke}{rgb}{0.500000,0.500000,0.500000}%
\pgfsetstrokecolor{currentstroke}%
\pgfsetstrokeopacity{0.300000}%
\pgfsetdash{}{0pt}%
\pgfpathmoveto{\pgfqpoint{0.000000in}{0.000000in}}%
\pgfpathlineto{\pgfqpoint{0.000000in}{0.000000in}}%
\pgfpathclose%
\pgfusepath{stroke,fill}%
\end{pgfscope}%
\begin{pgfscope}%
\pgfpathrectangle{\pgfqpoint{0.647939in}{0.492442in}}{\pgfqpoint{3.079299in}{3.079299in}}%
\pgfusepath{clip}%
\pgfsetroundcap%
\pgfsetroundjoin%
\pgfsetlinewidth{0.301125pt}%
\definecolor{currentstroke}{rgb}{0.500000,0.500000,0.500000}%
\pgfsetstrokecolor{currentstroke}%
\pgfsetstrokeopacity{0.300000}%
\pgfsetdash{}{0pt}%
\pgfpathmoveto{\pgfqpoint{3.602266in}{2.586169in}}%
\pgfusepath{stroke}%
\end{pgfscope}%
\begin{pgfscope}%
\pgfpathrectangle{\pgfqpoint{0.647939in}{0.492442in}}{\pgfqpoint{3.079299in}{3.079299in}}%
\pgfusepath{clip}%
\pgfsetroundcap%
\pgfsetroundjoin%
\definecolor{currentfill}{rgb}{0.500000,0.500000,0.500000}%
\pgfsetfillcolor{currentfill}%
\pgfsetfillopacity{0.300000}%
\pgfsetlinewidth{0.301125pt}%
\definecolor{currentstroke}{rgb}{0.500000,0.500000,0.500000}%
\pgfsetstrokecolor{currentstroke}%
\pgfsetstrokeopacity{0.300000}%
\pgfsetdash{}{0pt}%
\pgfpathmoveto{\pgfqpoint{0.000000in}{0.000000in}}%
\pgfpathlineto{\pgfqpoint{0.000000in}{0.000000in}}%
\pgfpathclose%
\pgfusepath{stroke,fill}%
\end{pgfscope}%
\begin{pgfscope}%
\pgfpathrectangle{\pgfqpoint{0.647939in}{0.492442in}}{\pgfqpoint{3.079299in}{3.079299in}}%
\pgfusepath{clip}%
\pgfsetroundcap%
\pgfsetroundjoin%
\pgfsetlinewidth{0.301125pt}%
\definecolor{currentstroke}{rgb}{0.500000,0.500000,0.500000}%
\pgfsetstrokecolor{currentstroke}%
\pgfsetstrokeopacity{0.300000}%
\pgfsetdash{}{0pt}%
\pgfpathmoveto{\pgfqpoint{3.669474in}{2.536598in}}%
\pgfusepath{stroke}%
\end{pgfscope}%
\begin{pgfscope}%
\pgfpathrectangle{\pgfqpoint{0.647939in}{0.492442in}}{\pgfqpoint{3.079299in}{3.079299in}}%
\pgfusepath{clip}%
\pgfsetroundcap%
\pgfsetroundjoin%
\definecolor{currentfill}{rgb}{0.500000,0.500000,0.500000}%
\pgfsetfillcolor{currentfill}%
\pgfsetfillopacity{0.300000}%
\pgfsetlinewidth{0.301125pt}%
\definecolor{currentstroke}{rgb}{0.500000,0.500000,0.500000}%
\pgfsetstrokecolor{currentstroke}%
\pgfsetstrokeopacity{0.300000}%
\pgfsetdash{}{0pt}%
\pgfpathmoveto{\pgfqpoint{0.000000in}{0.000000in}}%
\pgfpathlineto{\pgfqpoint{0.000000in}{0.000000in}}%
\pgfpathclose%
\pgfusepath{stroke,fill}%
\end{pgfscope}%
\begin{pgfscope}%
\pgfpathrectangle{\pgfqpoint{0.647939in}{0.492442in}}{\pgfqpoint{3.079299in}{3.079299in}}%
\pgfusepath{clip}%
\pgfsetroundcap%
\pgfsetroundjoin%
\pgfsetlinewidth{0.301125pt}%
\definecolor{currentstroke}{rgb}{0.500000,0.500000,0.500000}%
\pgfsetstrokecolor{currentstroke}%
\pgfsetstrokeopacity{0.300000}%
\pgfsetdash{}{0pt}%
\pgfpathmoveto{\pgfqpoint{3.704745in}{2.626535in}}%
\pgfusepath{stroke}%
\end{pgfscope}%
\begin{pgfscope}%
\pgfpathrectangle{\pgfqpoint{0.647939in}{0.492442in}}{\pgfqpoint{3.079299in}{3.079299in}}%
\pgfusepath{clip}%
\pgfsetroundcap%
\pgfsetroundjoin%
\definecolor{currentfill}{rgb}{0.500000,0.500000,0.500000}%
\pgfsetfillcolor{currentfill}%
\pgfsetfillopacity{0.300000}%
\pgfsetlinewidth{0.301125pt}%
\definecolor{currentstroke}{rgb}{0.500000,0.500000,0.500000}%
\pgfsetstrokecolor{currentstroke}%
\pgfsetstrokeopacity{0.300000}%
\pgfsetdash{}{0pt}%
\pgfpathmoveto{\pgfqpoint{0.000000in}{0.000000in}}%
\pgfpathlineto{\pgfqpoint{0.000000in}{0.000000in}}%
\pgfpathclose%
\pgfusepath{stroke,fill}%
\end{pgfscope}%
\begin{pgfscope}%
\pgfpathrectangle{\pgfqpoint{0.647939in}{0.492442in}}{\pgfqpoint{3.079299in}{3.079299in}}%
\pgfusepath{clip}%
\pgfsetroundcap%
\pgfsetroundjoin%
\pgfsetlinewidth{0.301125pt}%
\definecolor{currentstroke}{rgb}{0.500000,0.500000,0.500000}%
\pgfsetstrokecolor{currentstroke}%
\pgfsetstrokeopacity{0.300000}%
\pgfsetdash{}{0pt}%
\pgfpathmoveto{\pgfqpoint{2.185694in}{2.751236in}}%
\pgfusepath{stroke}%
\end{pgfscope}%
\begin{pgfscope}%
\pgfpathrectangle{\pgfqpoint{0.647939in}{0.492442in}}{\pgfqpoint{3.079299in}{3.079299in}}%
\pgfusepath{clip}%
\pgfsetroundcap%
\pgfsetroundjoin%
\definecolor{currentfill}{rgb}{0.500000,0.500000,0.500000}%
\pgfsetfillcolor{currentfill}%
\pgfsetfillopacity{0.300000}%
\pgfsetlinewidth{0.301125pt}%
\definecolor{currentstroke}{rgb}{0.500000,0.500000,0.500000}%
\pgfsetstrokecolor{currentstroke}%
\pgfsetstrokeopacity{0.300000}%
\pgfsetdash{}{0pt}%
\pgfpathmoveto{\pgfqpoint{0.000000in}{0.000000in}}%
\pgfpathlineto{\pgfqpoint{0.000000in}{0.000000in}}%
\pgfpathclose%
\pgfusepath{stroke,fill}%
\end{pgfscope}%
\begin{pgfscope}%
\pgfpathrectangle{\pgfqpoint{0.647939in}{0.492442in}}{\pgfqpoint{3.079299in}{3.079299in}}%
\pgfusepath{clip}%
\pgfsetroundcap%
\pgfsetroundjoin%
\pgfsetlinewidth{0.301125pt}%
\definecolor{currentstroke}{rgb}{0.500000,0.500000,0.500000}%
\pgfsetstrokecolor{currentstroke}%
\pgfsetstrokeopacity{0.300000}%
\pgfsetdash{}{0pt}%
\pgfpathmoveto{\pgfqpoint{2.056328in}{3.016337in}}%
\pgfusepath{stroke}%
\end{pgfscope}%
\begin{pgfscope}%
\pgfpathrectangle{\pgfqpoint{0.647939in}{0.492442in}}{\pgfqpoint{3.079299in}{3.079299in}}%
\pgfusepath{clip}%
\pgfsetroundcap%
\pgfsetroundjoin%
\definecolor{currentfill}{rgb}{0.500000,0.500000,0.500000}%
\pgfsetfillcolor{currentfill}%
\pgfsetfillopacity{0.300000}%
\pgfsetlinewidth{0.301125pt}%
\definecolor{currentstroke}{rgb}{0.500000,0.500000,0.500000}%
\pgfsetstrokecolor{currentstroke}%
\pgfsetstrokeopacity{0.300000}%
\pgfsetdash{}{0pt}%
\pgfpathmoveto{\pgfqpoint{0.000000in}{0.000000in}}%
\pgfpathlineto{\pgfqpoint{0.000000in}{0.000000in}}%
\pgfpathclose%
\pgfusepath{stroke,fill}%
\end{pgfscope}%
\begin{pgfscope}%
\pgfpathrectangle{\pgfqpoint{0.647939in}{0.492442in}}{\pgfqpoint{3.079299in}{3.079299in}}%
\pgfusepath{clip}%
\pgfsetroundcap%
\pgfsetroundjoin%
\pgfsetlinewidth{0.301125pt}%
\definecolor{currentstroke}{rgb}{0.500000,0.500000,0.500000}%
\pgfsetstrokecolor{currentstroke}%
\pgfsetstrokeopacity{0.300000}%
\pgfsetdash{}{0pt}%
\pgfpathmoveto{\pgfqpoint{1.915871in}{3.182057in}}%
\pgfusepath{stroke}%
\end{pgfscope}%
\begin{pgfscope}%
\pgfpathrectangle{\pgfqpoint{0.647939in}{0.492442in}}{\pgfqpoint{3.079299in}{3.079299in}}%
\pgfusepath{clip}%
\pgfsetroundcap%
\pgfsetroundjoin%
\definecolor{currentfill}{rgb}{0.500000,0.500000,0.500000}%
\pgfsetfillcolor{currentfill}%
\pgfsetfillopacity{0.300000}%
\pgfsetlinewidth{0.301125pt}%
\definecolor{currentstroke}{rgb}{0.500000,0.500000,0.500000}%
\pgfsetstrokecolor{currentstroke}%
\pgfsetstrokeopacity{0.300000}%
\pgfsetdash{}{0pt}%
\pgfpathmoveto{\pgfqpoint{0.000000in}{0.000000in}}%
\pgfpathlineto{\pgfqpoint{0.000000in}{0.000000in}}%
\pgfpathclose%
\pgfusepath{stroke,fill}%
\end{pgfscope}%
\begin{pgfscope}%
\pgfpathrectangle{\pgfqpoint{0.647939in}{0.492442in}}{\pgfqpoint{3.079299in}{3.079299in}}%
\pgfusepath{clip}%
\pgfsetroundcap%
\pgfsetroundjoin%
\pgfsetlinewidth{0.301125pt}%
\definecolor{currentstroke}{rgb}{0.500000,0.500000,0.500000}%
\pgfsetstrokecolor{currentstroke}%
\pgfsetstrokeopacity{0.300000}%
\pgfsetdash{}{0pt}%
\pgfpathmoveto{\pgfqpoint{1.870835in}{3.305849in}}%
\pgfusepath{stroke}%
\end{pgfscope}%
\begin{pgfscope}%
\pgfpathrectangle{\pgfqpoint{0.647939in}{0.492442in}}{\pgfqpoint{3.079299in}{3.079299in}}%
\pgfusepath{clip}%
\pgfsetroundcap%
\pgfsetroundjoin%
\definecolor{currentfill}{rgb}{0.500000,0.500000,0.500000}%
\pgfsetfillcolor{currentfill}%
\pgfsetfillopacity{0.300000}%
\pgfsetlinewidth{0.301125pt}%
\definecolor{currentstroke}{rgb}{0.500000,0.500000,0.500000}%
\pgfsetstrokecolor{currentstroke}%
\pgfsetstrokeopacity{0.300000}%
\pgfsetdash{}{0pt}%
\pgfpathmoveto{\pgfqpoint{0.000000in}{0.000000in}}%
\pgfpathlineto{\pgfqpoint{0.000000in}{0.000000in}}%
\pgfpathclose%
\pgfusepath{stroke,fill}%
\end{pgfscope}%
\begin{pgfscope}%
\pgfpathrectangle{\pgfqpoint{0.647939in}{0.492442in}}{\pgfqpoint{3.079299in}{3.079299in}}%
\pgfusepath{clip}%
\pgfsetroundcap%
\pgfsetroundjoin%
\pgfsetlinewidth{0.301125pt}%
\definecolor{currentstroke}{rgb}{0.500000,0.500000,0.500000}%
\pgfsetstrokecolor{currentstroke}%
\pgfsetstrokeopacity{0.300000}%
\pgfsetdash{}{0pt}%
\pgfpathmoveto{\pgfqpoint{1.775370in}{3.386574in}}%
\pgfusepath{stroke}%
\end{pgfscope}%
\begin{pgfscope}%
\pgfpathrectangle{\pgfqpoint{0.647939in}{0.492442in}}{\pgfqpoint{3.079299in}{3.079299in}}%
\pgfusepath{clip}%
\pgfsetroundcap%
\pgfsetroundjoin%
\definecolor{currentfill}{rgb}{0.500000,0.500000,0.500000}%
\pgfsetfillcolor{currentfill}%
\pgfsetfillopacity{0.300000}%
\pgfsetlinewidth{0.301125pt}%
\definecolor{currentstroke}{rgb}{0.500000,0.500000,0.500000}%
\pgfsetstrokecolor{currentstroke}%
\pgfsetstrokeopacity{0.300000}%
\pgfsetdash{}{0pt}%
\pgfpathmoveto{\pgfqpoint{0.000000in}{0.000000in}}%
\pgfpathlineto{\pgfqpoint{0.000000in}{0.000000in}}%
\pgfpathclose%
\pgfusepath{stroke,fill}%
\end{pgfscope}%
\begin{pgfscope}%
\pgfpathrectangle{\pgfqpoint{0.647939in}{0.492442in}}{\pgfqpoint{3.079299in}{3.079299in}}%
\pgfusepath{clip}%
\pgfsetroundcap%
\pgfsetroundjoin%
\pgfsetlinewidth{0.301125pt}%
\definecolor{currentstroke}{rgb}{0.500000,0.500000,0.500000}%
\pgfsetstrokecolor{currentstroke}%
\pgfsetstrokeopacity{0.300000}%
\pgfsetdash{}{0pt}%
\pgfpathmoveto{\pgfqpoint{1.619194in}{3.449798in}}%
\pgfusepath{stroke}%
\end{pgfscope}%
\begin{pgfscope}%
\pgfpathrectangle{\pgfqpoint{0.647939in}{0.492442in}}{\pgfqpoint{3.079299in}{3.079299in}}%
\pgfusepath{clip}%
\pgfsetroundcap%
\pgfsetroundjoin%
\definecolor{currentfill}{rgb}{0.500000,0.500000,0.500000}%
\pgfsetfillcolor{currentfill}%
\pgfsetfillopacity{0.300000}%
\pgfsetlinewidth{0.301125pt}%
\definecolor{currentstroke}{rgb}{0.500000,0.500000,0.500000}%
\pgfsetstrokecolor{currentstroke}%
\pgfsetstrokeopacity{0.300000}%
\pgfsetdash{}{0pt}%
\pgfpathmoveto{\pgfqpoint{0.000000in}{0.000000in}}%
\pgfpathlineto{\pgfqpoint{0.000000in}{0.000000in}}%
\pgfpathclose%
\pgfusepath{stroke,fill}%
\end{pgfscope}%
\begin{pgfscope}%
\pgfpathrectangle{\pgfqpoint{0.647939in}{0.492442in}}{\pgfqpoint{3.079299in}{3.079299in}}%
\pgfusepath{clip}%
\pgfsetroundcap%
\pgfsetroundjoin%
\pgfsetlinewidth{0.301125pt}%
\definecolor{currentstroke}{rgb}{0.500000,0.500000,0.500000}%
\pgfsetstrokecolor{currentstroke}%
\pgfsetstrokeopacity{0.300000}%
\pgfsetdash{}{0pt}%
\pgfpathmoveto{\pgfqpoint{2.222426in}{3.552227in}}%
\pgfusepath{stroke}%
\end{pgfscope}%
\begin{pgfscope}%
\pgfpathrectangle{\pgfqpoint{0.647939in}{0.492442in}}{\pgfqpoint{3.079299in}{3.079299in}}%
\pgfusepath{clip}%
\pgfsetroundcap%
\pgfsetroundjoin%
\definecolor{currentfill}{rgb}{0.500000,0.500000,0.500000}%
\pgfsetfillcolor{currentfill}%
\pgfsetfillopacity{0.300000}%
\pgfsetlinewidth{0.301125pt}%
\definecolor{currentstroke}{rgb}{0.500000,0.500000,0.500000}%
\pgfsetstrokecolor{currentstroke}%
\pgfsetstrokeopacity{0.300000}%
\pgfsetdash{}{0pt}%
\pgfpathmoveto{\pgfqpoint{0.000000in}{0.000000in}}%
\pgfpathlineto{\pgfqpoint{0.000000in}{0.000000in}}%
\pgfpathclose%
\pgfusepath{stroke,fill}%
\end{pgfscope}%
\begin{pgfscope}%
\pgfpathrectangle{\pgfqpoint{0.647939in}{0.492442in}}{\pgfqpoint{3.079299in}{3.079299in}}%
\pgfusepath{clip}%
\pgfsetroundcap%
\pgfsetroundjoin%
\pgfsetlinewidth{0.301125pt}%
\definecolor{currentstroke}{rgb}{0.500000,0.500000,0.500000}%
\pgfsetstrokecolor{currentstroke}%
\pgfsetstrokeopacity{0.300000}%
\pgfsetdash{}{0pt}%
\pgfpathmoveto{\pgfqpoint{1.261577in}{3.458484in}}%
\pgfusepath{stroke}%
\end{pgfscope}%
\begin{pgfscope}%
\pgfpathrectangle{\pgfqpoint{0.647939in}{0.492442in}}{\pgfqpoint{3.079299in}{3.079299in}}%
\pgfusepath{clip}%
\pgfsetroundcap%
\pgfsetroundjoin%
\definecolor{currentfill}{rgb}{0.500000,0.500000,0.500000}%
\pgfsetfillcolor{currentfill}%
\pgfsetfillopacity{0.300000}%
\pgfsetlinewidth{0.301125pt}%
\definecolor{currentstroke}{rgb}{0.500000,0.500000,0.500000}%
\pgfsetstrokecolor{currentstroke}%
\pgfsetstrokeopacity{0.300000}%
\pgfsetdash{}{0pt}%
\pgfpathmoveto{\pgfqpoint{0.000000in}{0.000000in}}%
\pgfpathlineto{\pgfqpoint{0.000000in}{0.000000in}}%
\pgfpathclose%
\pgfusepath{stroke,fill}%
\end{pgfscope}%
\begin{pgfscope}%
\pgfpathrectangle{\pgfqpoint{0.647939in}{0.492442in}}{\pgfqpoint{3.079299in}{3.079299in}}%
\pgfusepath{clip}%
\pgfsetroundcap%
\pgfsetroundjoin%
\pgfsetlinewidth{0.301125pt}%
\definecolor{currentstroke}{rgb}{0.500000,0.500000,0.500000}%
\pgfsetstrokecolor{currentstroke}%
\pgfsetstrokeopacity{0.300000}%
\pgfsetdash{}{0pt}%
\pgfpathmoveto{\pgfqpoint{1.043550in}{3.491116in}}%
\pgfusepath{stroke}%
\end{pgfscope}%
\begin{pgfscope}%
\pgfpathrectangle{\pgfqpoint{0.647939in}{0.492442in}}{\pgfqpoint{3.079299in}{3.079299in}}%
\pgfusepath{clip}%
\pgfsetroundcap%
\pgfsetroundjoin%
\definecolor{currentfill}{rgb}{0.500000,0.500000,0.500000}%
\pgfsetfillcolor{currentfill}%
\pgfsetfillopacity{0.300000}%
\pgfsetlinewidth{0.301125pt}%
\definecolor{currentstroke}{rgb}{0.500000,0.500000,0.500000}%
\pgfsetstrokecolor{currentstroke}%
\pgfsetstrokeopacity{0.300000}%
\pgfsetdash{}{0pt}%
\pgfpathmoveto{\pgfqpoint{0.000000in}{0.000000in}}%
\pgfpathlineto{\pgfqpoint{0.000000in}{0.000000in}}%
\pgfpathclose%
\pgfusepath{stroke,fill}%
\end{pgfscope}%
\begin{pgfscope}%
\pgfpathrectangle{\pgfqpoint{0.647939in}{0.492442in}}{\pgfqpoint{3.079299in}{3.079299in}}%
\pgfusepath{clip}%
\pgfsetroundcap%
\pgfsetroundjoin%
\pgfsetlinewidth{0.301125pt}%
\definecolor{currentstroke}{rgb}{0.500000,0.500000,0.500000}%
\pgfsetstrokecolor{currentstroke}%
\pgfsetstrokeopacity{0.300000}%
\pgfsetdash{}{0pt}%
\pgfpathmoveto{\pgfqpoint{0.825767in}{3.528432in}}%
\pgfusepath{stroke}%
\end{pgfscope}%
\begin{pgfscope}%
\pgfpathrectangle{\pgfqpoint{0.647939in}{0.492442in}}{\pgfqpoint{3.079299in}{3.079299in}}%
\pgfusepath{clip}%
\pgfsetroundcap%
\pgfsetroundjoin%
\definecolor{currentfill}{rgb}{0.500000,0.500000,0.500000}%
\pgfsetfillcolor{currentfill}%
\pgfsetfillopacity{0.300000}%
\pgfsetlinewidth{0.301125pt}%
\definecolor{currentstroke}{rgb}{0.500000,0.500000,0.500000}%
\pgfsetstrokecolor{currentstroke}%
\pgfsetstrokeopacity{0.300000}%
\pgfsetdash{}{0pt}%
\pgfpathmoveto{\pgfqpoint{0.000000in}{0.000000in}}%
\pgfpathlineto{\pgfqpoint{0.000000in}{0.000000in}}%
\pgfpathclose%
\pgfusepath{stroke,fill}%
\end{pgfscope}%
\begin{pgfscope}%
\pgfpathrectangle{\pgfqpoint{0.647939in}{0.492442in}}{\pgfqpoint{3.079299in}{3.079299in}}%
\pgfusepath{clip}%
\pgfsetroundcap%
\pgfsetroundjoin%
\pgfsetlinewidth{0.301125pt}%
\definecolor{currentstroke}{rgb}{0.500000,0.500000,0.500000}%
\pgfsetstrokecolor{currentstroke}%
\pgfsetstrokeopacity{0.300000}%
\pgfsetdash{}{0pt}%
\pgfpathmoveto{\pgfqpoint{1.608568in}{3.019162in}}%
\pgfusepath{stroke}%
\end{pgfscope}%
\begin{pgfscope}%
\pgfpathrectangle{\pgfqpoint{0.647939in}{0.492442in}}{\pgfqpoint{3.079299in}{3.079299in}}%
\pgfusepath{clip}%
\pgfsetroundcap%
\pgfsetroundjoin%
\definecolor{currentfill}{rgb}{0.500000,0.500000,0.500000}%
\pgfsetfillcolor{currentfill}%
\pgfsetfillopacity{0.300000}%
\pgfsetlinewidth{0.301125pt}%
\definecolor{currentstroke}{rgb}{0.500000,0.500000,0.500000}%
\pgfsetstrokecolor{currentstroke}%
\pgfsetstrokeopacity{0.300000}%
\pgfsetdash{}{0pt}%
\pgfpathmoveto{\pgfqpoint{0.000000in}{0.000000in}}%
\pgfpathlineto{\pgfqpoint{0.000000in}{0.000000in}}%
\pgfpathclose%
\pgfusepath{stroke,fill}%
\end{pgfscope}%
\begin{pgfscope}%
\pgfpathrectangle{\pgfqpoint{0.647939in}{0.492442in}}{\pgfqpoint{3.079299in}{3.079299in}}%
\pgfusepath{clip}%
\pgfsetroundcap%
\pgfsetroundjoin%
\pgfsetlinewidth{0.301125pt}%
\definecolor{currentstroke}{rgb}{0.500000,0.500000,0.500000}%
\pgfsetstrokecolor{currentstroke}%
\pgfsetstrokeopacity{0.300000}%
\pgfsetdash{}{0pt}%
\pgfpathmoveto{\pgfqpoint{1.872826in}{2.878935in}}%
\pgfusepath{stroke}%
\end{pgfscope}%
\begin{pgfscope}%
\pgfpathrectangle{\pgfqpoint{0.647939in}{0.492442in}}{\pgfqpoint{3.079299in}{3.079299in}}%
\pgfusepath{clip}%
\pgfsetroundcap%
\pgfsetroundjoin%
\definecolor{currentfill}{rgb}{0.500000,0.500000,0.500000}%
\pgfsetfillcolor{currentfill}%
\pgfsetfillopacity{0.300000}%
\pgfsetlinewidth{0.301125pt}%
\definecolor{currentstroke}{rgb}{0.500000,0.500000,0.500000}%
\pgfsetstrokecolor{currentstroke}%
\pgfsetstrokeopacity{0.300000}%
\pgfsetdash{}{0pt}%
\pgfpathmoveto{\pgfqpoint{0.000000in}{0.000000in}}%
\pgfpathlineto{\pgfqpoint{0.000000in}{0.000000in}}%
\pgfpathclose%
\pgfusepath{stroke,fill}%
\end{pgfscope}%
\begin{pgfscope}%
\pgfpathrectangle{\pgfqpoint{0.647939in}{0.492442in}}{\pgfqpoint{3.079299in}{3.079299in}}%
\pgfusepath{clip}%
\pgfsetroundcap%
\pgfsetroundjoin%
\pgfsetlinewidth{0.301125pt}%
\definecolor{currentstroke}{rgb}{0.500000,0.500000,0.500000}%
\pgfsetstrokecolor{currentstroke}%
\pgfsetstrokeopacity{0.300000}%
\pgfsetdash{}{0pt}%
\pgfpathmoveto{\pgfqpoint{1.143439in}{2.625866in}}%
\pgfusepath{stroke}%
\end{pgfscope}%
\begin{pgfscope}%
\pgfpathrectangle{\pgfqpoint{0.647939in}{0.492442in}}{\pgfqpoint{3.079299in}{3.079299in}}%
\pgfusepath{clip}%
\pgfsetroundcap%
\pgfsetroundjoin%
\definecolor{currentfill}{rgb}{0.500000,0.500000,0.500000}%
\pgfsetfillcolor{currentfill}%
\pgfsetfillopacity{0.300000}%
\pgfsetlinewidth{0.301125pt}%
\definecolor{currentstroke}{rgb}{0.500000,0.500000,0.500000}%
\pgfsetstrokecolor{currentstroke}%
\pgfsetstrokeopacity{0.300000}%
\pgfsetdash{}{0pt}%
\pgfpathmoveto{\pgfqpoint{0.000000in}{0.000000in}}%
\pgfpathlineto{\pgfqpoint{0.000000in}{0.000000in}}%
\pgfpathclose%
\pgfusepath{stroke,fill}%
\end{pgfscope}%
\begin{pgfscope}%
\pgfpathrectangle{\pgfqpoint{0.647939in}{0.492442in}}{\pgfqpoint{3.079299in}{3.079299in}}%
\pgfusepath{clip}%
\pgfsetroundcap%
\pgfsetroundjoin%
\pgfsetlinewidth{0.301125pt}%
\definecolor{currentstroke}{rgb}{0.500000,0.500000,0.500000}%
\pgfsetstrokecolor{currentstroke}%
\pgfsetstrokeopacity{0.300000}%
\pgfsetdash{}{0pt}%
\pgfpathmoveto{\pgfqpoint{1.532798in}{2.547769in}}%
\pgfusepath{stroke}%
\end{pgfscope}%
\begin{pgfscope}%
\pgfpathrectangle{\pgfqpoint{0.647939in}{0.492442in}}{\pgfqpoint{3.079299in}{3.079299in}}%
\pgfusepath{clip}%
\pgfsetroundcap%
\pgfsetroundjoin%
\definecolor{currentfill}{rgb}{0.500000,0.500000,0.500000}%
\pgfsetfillcolor{currentfill}%
\pgfsetfillopacity{0.300000}%
\pgfsetlinewidth{0.301125pt}%
\definecolor{currentstroke}{rgb}{0.500000,0.500000,0.500000}%
\pgfsetstrokecolor{currentstroke}%
\pgfsetstrokeopacity{0.300000}%
\pgfsetdash{}{0pt}%
\pgfpathmoveto{\pgfqpoint{0.000000in}{0.000000in}}%
\pgfpathlineto{\pgfqpoint{0.000000in}{0.000000in}}%
\pgfpathclose%
\pgfusepath{stroke,fill}%
\end{pgfscope}%
\begin{pgfscope}%
\pgfpathrectangle{\pgfqpoint{0.647939in}{0.492442in}}{\pgfqpoint{3.079299in}{3.079299in}}%
\pgfusepath{clip}%
\pgfsetroundcap%
\pgfsetroundjoin%
\pgfsetlinewidth{0.301125pt}%
\definecolor{currentstroke}{rgb}{0.500000,0.500000,0.500000}%
\pgfsetstrokecolor{currentstroke}%
\pgfsetstrokeopacity{0.300000}%
\pgfsetdash{}{0pt}%
\pgfpathmoveto{\pgfqpoint{1.401362in}{2.439917in}}%
\pgfusepath{stroke}%
\end{pgfscope}%
\begin{pgfscope}%
\pgfpathrectangle{\pgfqpoint{0.647939in}{0.492442in}}{\pgfqpoint{3.079299in}{3.079299in}}%
\pgfusepath{clip}%
\pgfsetroundcap%
\pgfsetroundjoin%
\definecolor{currentfill}{rgb}{0.500000,0.500000,0.500000}%
\pgfsetfillcolor{currentfill}%
\pgfsetfillopacity{0.300000}%
\pgfsetlinewidth{0.301125pt}%
\definecolor{currentstroke}{rgb}{0.500000,0.500000,0.500000}%
\pgfsetstrokecolor{currentstroke}%
\pgfsetstrokeopacity{0.300000}%
\pgfsetdash{}{0pt}%
\pgfpathmoveto{\pgfqpoint{0.000000in}{0.000000in}}%
\pgfpathlineto{\pgfqpoint{0.000000in}{0.000000in}}%
\pgfpathclose%
\pgfusepath{stroke,fill}%
\end{pgfscope}%
\begin{pgfscope}%
\pgfpathrectangle{\pgfqpoint{0.647939in}{0.492442in}}{\pgfqpoint{3.079299in}{3.079299in}}%
\pgfusepath{clip}%
\pgfsetroundcap%
\pgfsetroundjoin%
\pgfsetlinewidth{0.301125pt}%
\definecolor{currentstroke}{rgb}{0.500000,0.500000,0.500000}%
\pgfsetstrokecolor{currentstroke}%
\pgfsetstrokeopacity{0.300000}%
\pgfsetdash{}{0pt}%
\pgfpathmoveto{\pgfqpoint{1.656727in}{2.400072in}}%
\pgfusepath{stroke}%
\end{pgfscope}%
\begin{pgfscope}%
\pgfpathrectangle{\pgfqpoint{0.647939in}{0.492442in}}{\pgfqpoint{3.079299in}{3.079299in}}%
\pgfusepath{clip}%
\pgfsetroundcap%
\pgfsetroundjoin%
\definecolor{currentfill}{rgb}{0.500000,0.500000,0.500000}%
\pgfsetfillcolor{currentfill}%
\pgfsetfillopacity{0.300000}%
\pgfsetlinewidth{0.301125pt}%
\definecolor{currentstroke}{rgb}{0.500000,0.500000,0.500000}%
\pgfsetstrokecolor{currentstroke}%
\pgfsetstrokeopacity{0.300000}%
\pgfsetdash{}{0pt}%
\pgfpathmoveto{\pgfqpoint{0.000000in}{0.000000in}}%
\pgfpathlineto{\pgfqpoint{0.000000in}{0.000000in}}%
\pgfpathclose%
\pgfusepath{stroke,fill}%
\end{pgfscope}%
\begin{pgfscope}%
\pgfpathrectangle{\pgfqpoint{0.647939in}{0.492442in}}{\pgfqpoint{3.079299in}{3.079299in}}%
\pgfusepath{clip}%
\pgfsetroundcap%
\pgfsetroundjoin%
\pgfsetlinewidth{0.301125pt}%
\definecolor{currentstroke}{rgb}{0.500000,0.500000,0.500000}%
\pgfsetstrokecolor{currentstroke}%
\pgfsetstrokeopacity{0.300000}%
\pgfsetdash{}{0pt}%
\pgfpathmoveto{\pgfqpoint{0.943310in}{2.091498in}}%
\pgfusepath{stroke}%
\end{pgfscope}%
\begin{pgfscope}%
\pgfpathrectangle{\pgfqpoint{0.647939in}{0.492442in}}{\pgfqpoint{3.079299in}{3.079299in}}%
\pgfusepath{clip}%
\pgfsetroundcap%
\pgfsetroundjoin%
\definecolor{currentfill}{rgb}{0.500000,0.500000,0.500000}%
\pgfsetfillcolor{currentfill}%
\pgfsetfillopacity{0.300000}%
\pgfsetlinewidth{0.301125pt}%
\definecolor{currentstroke}{rgb}{0.500000,0.500000,0.500000}%
\pgfsetstrokecolor{currentstroke}%
\pgfsetstrokeopacity{0.300000}%
\pgfsetdash{}{0pt}%
\pgfpathmoveto{\pgfqpoint{0.000000in}{0.000000in}}%
\pgfpathlineto{\pgfqpoint{0.000000in}{0.000000in}}%
\pgfpathclose%
\pgfusepath{stroke,fill}%
\end{pgfscope}%
\begin{pgfscope}%
\pgfpathrectangle{\pgfqpoint{0.647939in}{0.492442in}}{\pgfqpoint{3.079299in}{3.079299in}}%
\pgfusepath{clip}%
\pgfsetroundcap%
\pgfsetroundjoin%
\pgfsetlinewidth{0.301125pt}%
\definecolor{currentstroke}{rgb}{0.500000,0.500000,0.500000}%
\pgfsetstrokecolor{currentstroke}%
\pgfsetstrokeopacity{0.300000}%
\pgfsetdash{}{0pt}%
\pgfpathmoveto{\pgfqpoint{0.943090in}{2.022537in}}%
\pgfusepath{stroke}%
\end{pgfscope}%
\begin{pgfscope}%
\pgfpathrectangle{\pgfqpoint{0.647939in}{0.492442in}}{\pgfqpoint{3.079299in}{3.079299in}}%
\pgfusepath{clip}%
\pgfsetroundcap%
\pgfsetroundjoin%
\definecolor{currentfill}{rgb}{0.500000,0.500000,0.500000}%
\pgfsetfillcolor{currentfill}%
\pgfsetfillopacity{0.300000}%
\pgfsetlinewidth{0.301125pt}%
\definecolor{currentstroke}{rgb}{0.500000,0.500000,0.500000}%
\pgfsetstrokecolor{currentstroke}%
\pgfsetstrokeopacity{0.300000}%
\pgfsetdash{}{0pt}%
\pgfpathmoveto{\pgfqpoint{0.000000in}{0.000000in}}%
\pgfpathlineto{\pgfqpoint{0.000000in}{0.000000in}}%
\pgfpathclose%
\pgfusepath{stroke,fill}%
\end{pgfscope}%
\begin{pgfscope}%
\pgfpathrectangle{\pgfqpoint{0.647939in}{0.492442in}}{\pgfqpoint{3.079299in}{3.079299in}}%
\pgfusepath{clip}%
\pgfsetroundcap%
\pgfsetroundjoin%
\pgfsetlinewidth{0.301125pt}%
\definecolor{currentstroke}{rgb}{0.500000,0.500000,0.500000}%
\pgfsetstrokecolor{currentstroke}%
\pgfsetstrokeopacity{0.300000}%
\pgfsetdash{}{0pt}%
\pgfpathmoveto{\pgfqpoint{1.583448in}{2.192641in}}%
\pgfusepath{stroke}%
\end{pgfscope}%
\begin{pgfscope}%
\pgfpathrectangle{\pgfqpoint{0.647939in}{0.492442in}}{\pgfqpoint{3.079299in}{3.079299in}}%
\pgfusepath{clip}%
\pgfsetroundcap%
\pgfsetroundjoin%
\definecolor{currentfill}{rgb}{0.500000,0.500000,0.500000}%
\pgfsetfillcolor{currentfill}%
\pgfsetfillopacity{0.300000}%
\pgfsetlinewidth{0.301125pt}%
\definecolor{currentstroke}{rgb}{0.500000,0.500000,0.500000}%
\pgfsetstrokecolor{currentstroke}%
\pgfsetstrokeopacity{0.300000}%
\pgfsetdash{}{0pt}%
\pgfpathmoveto{\pgfqpoint{0.000000in}{0.000000in}}%
\pgfpathlineto{\pgfqpoint{0.000000in}{0.000000in}}%
\pgfpathclose%
\pgfusepath{stroke,fill}%
\end{pgfscope}%
\begin{pgfscope}%
\pgfpathrectangle{\pgfqpoint{0.647939in}{0.492442in}}{\pgfqpoint{3.079299in}{3.079299in}}%
\pgfusepath{clip}%
\pgfsetroundcap%
\pgfsetroundjoin%
\pgfsetlinewidth{0.301125pt}%
\definecolor{currentstroke}{rgb}{0.500000,0.500000,0.500000}%
\pgfsetstrokecolor{currentstroke}%
\pgfsetstrokeopacity{0.300000}%
\pgfsetdash{}{0pt}%
\pgfpathmoveto{\pgfqpoint{1.201973in}{1.971730in}}%
\pgfusepath{stroke}%
\end{pgfscope}%
\begin{pgfscope}%
\pgfpathrectangle{\pgfqpoint{0.647939in}{0.492442in}}{\pgfqpoint{3.079299in}{3.079299in}}%
\pgfusepath{clip}%
\pgfsetroundcap%
\pgfsetroundjoin%
\definecolor{currentfill}{rgb}{0.500000,0.500000,0.500000}%
\pgfsetfillcolor{currentfill}%
\pgfsetfillopacity{0.300000}%
\pgfsetlinewidth{0.301125pt}%
\definecolor{currentstroke}{rgb}{0.500000,0.500000,0.500000}%
\pgfsetstrokecolor{currentstroke}%
\pgfsetstrokeopacity{0.300000}%
\pgfsetdash{}{0pt}%
\pgfpathmoveto{\pgfqpoint{0.000000in}{0.000000in}}%
\pgfpathlineto{\pgfqpoint{0.000000in}{0.000000in}}%
\pgfpathclose%
\pgfusepath{stroke,fill}%
\end{pgfscope}%
\begin{pgfscope}%
\pgfpathrectangle{\pgfqpoint{0.647939in}{0.492442in}}{\pgfqpoint{3.079299in}{3.079299in}}%
\pgfusepath{clip}%
\pgfsetroundcap%
\pgfsetroundjoin%
\pgfsetlinewidth{0.301125pt}%
\definecolor{currentstroke}{rgb}{0.500000,0.500000,0.500000}%
\pgfsetstrokecolor{currentstroke}%
\pgfsetstrokeopacity{0.300000}%
\pgfsetdash{}{0pt}%
\pgfpathmoveto{\pgfqpoint{1.007883in}{1.835536in}}%
\pgfusepath{stroke}%
\end{pgfscope}%
\begin{pgfscope}%
\pgfpathrectangle{\pgfqpoint{0.647939in}{0.492442in}}{\pgfqpoint{3.079299in}{3.079299in}}%
\pgfusepath{clip}%
\pgfsetroundcap%
\pgfsetroundjoin%
\definecolor{currentfill}{rgb}{0.500000,0.500000,0.500000}%
\pgfsetfillcolor{currentfill}%
\pgfsetfillopacity{0.300000}%
\pgfsetlinewidth{0.301125pt}%
\definecolor{currentstroke}{rgb}{0.500000,0.500000,0.500000}%
\pgfsetstrokecolor{currentstroke}%
\pgfsetstrokeopacity{0.300000}%
\pgfsetdash{}{0pt}%
\pgfpathmoveto{\pgfqpoint{0.000000in}{0.000000in}}%
\pgfpathlineto{\pgfqpoint{0.000000in}{0.000000in}}%
\pgfpathclose%
\pgfusepath{stroke,fill}%
\end{pgfscope}%
\begin{pgfscope}%
\pgfpathrectangle{\pgfqpoint{0.647939in}{0.492442in}}{\pgfqpoint{3.079299in}{3.079299in}}%
\pgfusepath{clip}%
\pgfsetroundcap%
\pgfsetroundjoin%
\pgfsetlinewidth{0.301125pt}%
\definecolor{currentstroke}{rgb}{0.500000,0.500000,0.500000}%
\pgfsetstrokecolor{currentstroke}%
\pgfsetstrokeopacity{0.300000}%
\pgfsetdash{}{0pt}%
\pgfpathmoveto{\pgfqpoint{1.449100in}{1.951551in}}%
\pgfusepath{stroke}%
\end{pgfscope}%
\begin{pgfscope}%
\pgfpathrectangle{\pgfqpoint{0.647939in}{0.492442in}}{\pgfqpoint{3.079299in}{3.079299in}}%
\pgfusepath{clip}%
\pgfsetroundcap%
\pgfsetroundjoin%
\definecolor{currentfill}{rgb}{0.500000,0.500000,0.500000}%
\pgfsetfillcolor{currentfill}%
\pgfsetfillopacity{0.300000}%
\pgfsetlinewidth{0.301125pt}%
\definecolor{currentstroke}{rgb}{0.500000,0.500000,0.500000}%
\pgfsetstrokecolor{currentstroke}%
\pgfsetstrokeopacity{0.300000}%
\pgfsetdash{}{0pt}%
\pgfpathmoveto{\pgfqpoint{0.000000in}{0.000000in}}%
\pgfpathlineto{\pgfqpoint{0.000000in}{0.000000in}}%
\pgfpathclose%
\pgfusepath{stroke,fill}%
\end{pgfscope}%
\begin{pgfscope}%
\pgfpathrectangle{\pgfqpoint{0.647939in}{0.492442in}}{\pgfqpoint{3.079299in}{3.079299in}}%
\pgfusepath{clip}%
\pgfsetroundcap%
\pgfsetroundjoin%
\pgfsetlinewidth{0.301125pt}%
\definecolor{currentstroke}{rgb}{0.500000,0.500000,0.500000}%
\pgfsetstrokecolor{currentstroke}%
\pgfsetstrokeopacity{0.300000}%
\pgfsetdash{}{0pt}%
\pgfpathmoveto{\pgfqpoint{0.941798in}{1.678288in}}%
\pgfusepath{stroke}%
\end{pgfscope}%
\begin{pgfscope}%
\pgfpathrectangle{\pgfqpoint{0.647939in}{0.492442in}}{\pgfqpoint{3.079299in}{3.079299in}}%
\pgfusepath{clip}%
\pgfsetroundcap%
\pgfsetroundjoin%
\definecolor{currentfill}{rgb}{0.500000,0.500000,0.500000}%
\pgfsetfillcolor{currentfill}%
\pgfsetfillopacity{0.300000}%
\pgfsetlinewidth{0.301125pt}%
\definecolor{currentstroke}{rgb}{0.500000,0.500000,0.500000}%
\pgfsetstrokecolor{currentstroke}%
\pgfsetstrokeopacity{0.300000}%
\pgfsetdash{}{0pt}%
\pgfpathmoveto{\pgfqpoint{0.000000in}{0.000000in}}%
\pgfpathlineto{\pgfqpoint{0.000000in}{0.000000in}}%
\pgfpathclose%
\pgfusepath{stroke,fill}%
\end{pgfscope}%
\begin{pgfscope}%
\pgfpathrectangle{\pgfqpoint{0.647939in}{0.492442in}}{\pgfqpoint{3.079299in}{3.079299in}}%
\pgfusepath{clip}%
\pgfsetroundcap%
\pgfsetroundjoin%
\pgfsetlinewidth{0.301125pt}%
\definecolor{currentstroke}{rgb}{0.500000,0.500000,0.500000}%
\pgfsetstrokecolor{currentstroke}%
\pgfsetstrokeopacity{0.300000}%
\pgfsetdash{}{0pt}%
\pgfpathmoveto{\pgfqpoint{0.875741in}{1.590688in}}%
\pgfusepath{stroke}%
\end{pgfscope}%
\begin{pgfscope}%
\pgfpathrectangle{\pgfqpoint{0.647939in}{0.492442in}}{\pgfqpoint{3.079299in}{3.079299in}}%
\pgfusepath{clip}%
\pgfsetroundcap%
\pgfsetroundjoin%
\definecolor{currentfill}{rgb}{0.500000,0.500000,0.500000}%
\pgfsetfillcolor{currentfill}%
\pgfsetfillopacity{0.300000}%
\pgfsetlinewidth{0.301125pt}%
\definecolor{currentstroke}{rgb}{0.500000,0.500000,0.500000}%
\pgfsetstrokecolor{currentstroke}%
\pgfsetstrokeopacity{0.300000}%
\pgfsetdash{}{0pt}%
\pgfpathmoveto{\pgfqpoint{0.000000in}{0.000000in}}%
\pgfpathlineto{\pgfqpoint{0.000000in}{0.000000in}}%
\pgfpathclose%
\pgfusepath{stroke,fill}%
\end{pgfscope}%
\begin{pgfscope}%
\pgfpathrectangle{\pgfqpoint{0.647939in}{0.492442in}}{\pgfqpoint{3.079299in}{3.079299in}}%
\pgfusepath{clip}%
\pgfsetroundcap%
\pgfsetroundjoin%
\pgfsetlinewidth{0.301125pt}%
\definecolor{currentstroke}{rgb}{0.500000,0.500000,0.500000}%
\pgfsetstrokecolor{currentstroke}%
\pgfsetstrokeopacity{0.300000}%
\pgfsetdash{}{0pt}%
\pgfpathmoveto{\pgfqpoint{1.439278in}{1.765211in}}%
\pgfusepath{stroke}%
\end{pgfscope}%
\begin{pgfscope}%
\pgfpathrectangle{\pgfqpoint{0.647939in}{0.492442in}}{\pgfqpoint{3.079299in}{3.079299in}}%
\pgfusepath{clip}%
\pgfsetroundcap%
\pgfsetroundjoin%
\definecolor{currentfill}{rgb}{0.500000,0.500000,0.500000}%
\pgfsetfillcolor{currentfill}%
\pgfsetfillopacity{0.300000}%
\pgfsetlinewidth{0.301125pt}%
\definecolor{currentstroke}{rgb}{0.500000,0.500000,0.500000}%
\pgfsetstrokecolor{currentstroke}%
\pgfsetstrokeopacity{0.300000}%
\pgfsetdash{}{0pt}%
\pgfpathmoveto{\pgfqpoint{0.000000in}{0.000000in}}%
\pgfpathlineto{\pgfqpoint{0.000000in}{0.000000in}}%
\pgfpathclose%
\pgfusepath{stroke,fill}%
\end{pgfscope}%
\begin{pgfscope}%
\pgfpathrectangle{\pgfqpoint{0.647939in}{0.492442in}}{\pgfqpoint{3.079299in}{3.079299in}}%
\pgfusepath{clip}%
\pgfsetroundcap%
\pgfsetroundjoin%
\pgfsetlinewidth{0.301125pt}%
\definecolor{currentstroke}{rgb}{0.500000,0.500000,0.500000}%
\pgfsetstrokecolor{currentstroke}%
\pgfsetstrokeopacity{0.300000}%
\pgfsetdash{}{0pt}%
\pgfpathmoveto{\pgfqpoint{0.809165in}{1.435159in}}%
\pgfusepath{stroke}%
\end{pgfscope}%
\begin{pgfscope}%
\pgfpathrectangle{\pgfqpoint{0.647939in}{0.492442in}}{\pgfqpoint{3.079299in}{3.079299in}}%
\pgfusepath{clip}%
\pgfsetroundcap%
\pgfsetroundjoin%
\definecolor{currentfill}{rgb}{0.500000,0.500000,0.500000}%
\pgfsetfillcolor{currentfill}%
\pgfsetfillopacity{0.300000}%
\pgfsetlinewidth{0.301125pt}%
\definecolor{currentstroke}{rgb}{0.500000,0.500000,0.500000}%
\pgfsetstrokecolor{currentstroke}%
\pgfsetstrokeopacity{0.300000}%
\pgfsetdash{}{0pt}%
\pgfpathmoveto{\pgfqpoint{0.000000in}{0.000000in}}%
\pgfpathlineto{\pgfqpoint{0.000000in}{0.000000in}}%
\pgfpathclose%
\pgfusepath{stroke,fill}%
\end{pgfscope}%
\begin{pgfscope}%
\pgfpathrectangle{\pgfqpoint{0.647939in}{0.492442in}}{\pgfqpoint{3.079299in}{3.079299in}}%
\pgfusepath{clip}%
\pgfsetroundcap%
\pgfsetroundjoin%
\pgfsetlinewidth{0.301125pt}%
\definecolor{currentstroke}{rgb}{0.500000,0.500000,0.500000}%
\pgfsetstrokecolor{currentstroke}%
\pgfsetstrokeopacity{0.300000}%
\pgfsetdash{}{0pt}%
\pgfpathmoveto{\pgfqpoint{1.372862in}{1.607067in}}%
\pgfusepath{stroke}%
\end{pgfscope}%
\begin{pgfscope}%
\pgfpathrectangle{\pgfqpoint{0.647939in}{0.492442in}}{\pgfqpoint{3.079299in}{3.079299in}}%
\pgfusepath{clip}%
\pgfsetroundcap%
\pgfsetroundjoin%
\definecolor{currentfill}{rgb}{0.500000,0.500000,0.500000}%
\pgfsetfillcolor{currentfill}%
\pgfsetfillopacity{0.300000}%
\pgfsetlinewidth{0.301125pt}%
\definecolor{currentstroke}{rgb}{0.500000,0.500000,0.500000}%
\pgfsetstrokecolor{currentstroke}%
\pgfsetstrokeopacity{0.300000}%
\pgfsetdash{}{0pt}%
\pgfpathmoveto{\pgfqpoint{0.000000in}{0.000000in}}%
\pgfpathlineto{\pgfqpoint{0.000000in}{0.000000in}}%
\pgfpathclose%
\pgfusepath{stroke,fill}%
\end{pgfscope}%
\begin{pgfscope}%
\pgfpathrectangle{\pgfqpoint{0.647939in}{0.492442in}}{\pgfqpoint{3.079299in}{3.079299in}}%
\pgfusepath{clip}%
\pgfsetroundcap%
\pgfsetroundjoin%
\pgfsetlinewidth{0.301125pt}%
\definecolor{currentstroke}{rgb}{0.500000,0.500000,0.500000}%
\pgfsetstrokecolor{currentstroke}%
\pgfsetstrokeopacity{0.300000}%
\pgfsetdash{}{0pt}%
\pgfpathmoveto{\pgfqpoint{1.130205in}{1.412069in}}%
\pgfusepath{stroke}%
\end{pgfscope}%
\begin{pgfscope}%
\pgfpathrectangle{\pgfqpoint{0.647939in}{0.492442in}}{\pgfqpoint{3.079299in}{3.079299in}}%
\pgfusepath{clip}%
\pgfsetroundcap%
\pgfsetroundjoin%
\definecolor{currentfill}{rgb}{0.500000,0.500000,0.500000}%
\pgfsetfillcolor{currentfill}%
\pgfsetfillopacity{0.300000}%
\pgfsetlinewidth{0.301125pt}%
\definecolor{currentstroke}{rgb}{0.500000,0.500000,0.500000}%
\pgfsetstrokecolor{currentstroke}%
\pgfsetstrokeopacity{0.300000}%
\pgfsetdash{}{0pt}%
\pgfpathmoveto{\pgfqpoint{0.000000in}{0.000000in}}%
\pgfpathlineto{\pgfqpoint{0.000000in}{0.000000in}}%
\pgfpathclose%
\pgfusepath{stroke,fill}%
\end{pgfscope}%
\begin{pgfscope}%
\pgfpathrectangle{\pgfqpoint{0.647939in}{0.492442in}}{\pgfqpoint{3.079299in}{3.079299in}}%
\pgfusepath{clip}%
\pgfsetroundcap%
\pgfsetroundjoin%
\pgfsetlinewidth{0.301125pt}%
\definecolor{currentstroke}{rgb}{0.500000,0.500000,0.500000}%
\pgfsetstrokecolor{currentstroke}%
\pgfsetstrokeopacity{0.300000}%
\pgfsetdash{}{0pt}%
\pgfpathmoveto{\pgfqpoint{1.066910in}{1.316816in}}%
\pgfusepath{stroke}%
\end{pgfscope}%
\begin{pgfscope}%
\pgfpathrectangle{\pgfqpoint{0.647939in}{0.492442in}}{\pgfqpoint{3.079299in}{3.079299in}}%
\pgfusepath{clip}%
\pgfsetroundcap%
\pgfsetroundjoin%
\definecolor{currentfill}{rgb}{0.500000,0.500000,0.500000}%
\pgfsetfillcolor{currentfill}%
\pgfsetfillopacity{0.300000}%
\pgfsetlinewidth{0.301125pt}%
\definecolor{currentstroke}{rgb}{0.500000,0.500000,0.500000}%
\pgfsetstrokecolor{currentstroke}%
\pgfsetstrokeopacity{0.300000}%
\pgfsetdash{}{0pt}%
\pgfpathmoveto{\pgfqpoint{0.000000in}{0.000000in}}%
\pgfpathlineto{\pgfqpoint{0.000000in}{0.000000in}}%
\pgfpathclose%
\pgfusepath{stroke,fill}%
\end{pgfscope}%
\begin{pgfscope}%
\pgfpathrectangle{\pgfqpoint{0.647939in}{0.492442in}}{\pgfqpoint{3.079299in}{3.079299in}}%
\pgfusepath{clip}%
\pgfsetroundcap%
\pgfsetroundjoin%
\pgfsetlinewidth{0.301125pt}%
\definecolor{currentstroke}{rgb}{0.500000,0.500000,0.500000}%
\pgfsetstrokecolor{currentstroke}%
\pgfsetstrokeopacity{0.300000}%
\pgfsetdash{}{0pt}%
\pgfpathmoveto{\pgfqpoint{0.874340in}{1.176660in}}%
\pgfusepath{stroke}%
\end{pgfscope}%
\begin{pgfscope}%
\pgfpathrectangle{\pgfqpoint{0.647939in}{0.492442in}}{\pgfqpoint{3.079299in}{3.079299in}}%
\pgfusepath{clip}%
\pgfsetroundcap%
\pgfsetroundjoin%
\definecolor{currentfill}{rgb}{0.500000,0.500000,0.500000}%
\pgfsetfillcolor{currentfill}%
\pgfsetfillopacity{0.300000}%
\pgfsetlinewidth{0.301125pt}%
\definecolor{currentstroke}{rgb}{0.500000,0.500000,0.500000}%
\pgfsetstrokecolor{currentstroke}%
\pgfsetstrokeopacity{0.300000}%
\pgfsetdash{}{0pt}%
\pgfpathmoveto{\pgfqpoint{0.000000in}{0.000000in}}%
\pgfpathlineto{\pgfqpoint{0.000000in}{0.000000in}}%
\pgfpathclose%
\pgfusepath{stroke,fill}%
\end{pgfscope}%
\begin{pgfscope}%
\pgfpathrectangle{\pgfqpoint{0.647939in}{0.492442in}}{\pgfqpoint{3.079299in}{3.079299in}}%
\pgfusepath{clip}%
\pgfsetroundcap%
\pgfsetroundjoin%
\pgfsetlinewidth{0.301125pt}%
\definecolor{currentstroke}{rgb}{0.500000,0.500000,0.500000}%
\pgfsetstrokecolor{currentstroke}%
\pgfsetstrokeopacity{0.300000}%
\pgfsetdash{}{0pt}%
\pgfpathmoveto{\pgfqpoint{1.356227in}{1.360792in}}%
\pgfusepath{stroke}%
\end{pgfscope}%
\begin{pgfscope}%
\pgfpathrectangle{\pgfqpoint{0.647939in}{0.492442in}}{\pgfqpoint{3.079299in}{3.079299in}}%
\pgfusepath{clip}%
\pgfsetroundcap%
\pgfsetroundjoin%
\definecolor{currentfill}{rgb}{0.500000,0.500000,0.500000}%
\pgfsetfillcolor{currentfill}%
\pgfsetfillopacity{0.300000}%
\pgfsetlinewidth{0.301125pt}%
\definecolor{currentstroke}{rgb}{0.500000,0.500000,0.500000}%
\pgfsetstrokecolor{currentstroke}%
\pgfsetstrokeopacity{0.300000}%
\pgfsetdash{}{0pt}%
\pgfpathmoveto{\pgfqpoint{0.000000in}{0.000000in}}%
\pgfpathlineto{\pgfqpoint{0.000000in}{0.000000in}}%
\pgfpathclose%
\pgfusepath{stroke,fill}%
\end{pgfscope}%
\begin{pgfscope}%
\pgfpathrectangle{\pgfqpoint{0.647939in}{0.492442in}}{\pgfqpoint{3.079299in}{3.079299in}}%
\pgfusepath{clip}%
\pgfsetroundcap%
\pgfsetroundjoin%
\pgfsetlinewidth{0.301125pt}%
\definecolor{currentstroke}{rgb}{0.500000,0.500000,0.500000}%
\pgfsetstrokecolor{currentstroke}%
\pgfsetstrokeopacity{0.300000}%
\pgfsetdash{}{0pt}%
\pgfpathmoveto{\pgfqpoint{1.001636in}{1.087355in}}%
\pgfusepath{stroke}%
\end{pgfscope}%
\begin{pgfscope}%
\pgfpathrectangle{\pgfqpoint{0.647939in}{0.492442in}}{\pgfqpoint{3.079299in}{3.079299in}}%
\pgfusepath{clip}%
\pgfsetroundcap%
\pgfsetroundjoin%
\definecolor{currentfill}{rgb}{0.500000,0.500000,0.500000}%
\pgfsetfillcolor{currentfill}%
\pgfsetfillopacity{0.300000}%
\pgfsetlinewidth{0.301125pt}%
\definecolor{currentstroke}{rgb}{0.500000,0.500000,0.500000}%
\pgfsetstrokecolor{currentstroke}%
\pgfsetstrokeopacity{0.300000}%
\pgfsetdash{}{0pt}%
\pgfpathmoveto{\pgfqpoint{0.000000in}{0.000000in}}%
\pgfpathlineto{\pgfqpoint{0.000000in}{0.000000in}}%
\pgfpathclose%
\pgfusepath{stroke,fill}%
\end{pgfscope}%
\begin{pgfscope}%
\pgfpathrectangle{\pgfqpoint{0.647939in}{0.492442in}}{\pgfqpoint{3.079299in}{3.079299in}}%
\pgfusepath{clip}%
\pgfsetroundcap%
\pgfsetroundjoin%
\pgfsetlinewidth{0.301125pt}%
\definecolor{currentstroke}{rgb}{0.500000,0.500000,0.500000}%
\pgfsetstrokecolor{currentstroke}%
\pgfsetstrokeopacity{0.300000}%
\pgfsetdash{}{0pt}%
\pgfpathmoveto{\pgfqpoint{1.000807in}{1.019855in}}%
\pgfusepath{stroke}%
\end{pgfscope}%
\begin{pgfscope}%
\pgfpathrectangle{\pgfqpoint{0.647939in}{0.492442in}}{\pgfqpoint{3.079299in}{3.079299in}}%
\pgfusepath{clip}%
\pgfsetroundcap%
\pgfsetroundjoin%
\definecolor{currentfill}{rgb}{0.500000,0.500000,0.500000}%
\pgfsetfillcolor{currentfill}%
\pgfsetfillopacity{0.300000}%
\pgfsetlinewidth{0.301125pt}%
\definecolor{currentstroke}{rgb}{0.500000,0.500000,0.500000}%
\pgfsetstrokecolor{currentstroke}%
\pgfsetstrokeopacity{0.300000}%
\pgfsetdash{}{0pt}%
\pgfpathmoveto{\pgfqpoint{0.000000in}{0.000000in}}%
\pgfpathlineto{\pgfqpoint{0.000000in}{0.000000in}}%
\pgfpathclose%
\pgfusepath{stroke,fill}%
\end{pgfscope}%
\begin{pgfscope}%
\pgfpathrectangle{\pgfqpoint{0.647939in}{0.492442in}}{\pgfqpoint{3.079299in}{3.079299in}}%
\pgfusepath{clip}%
\pgfsetroundcap%
\pgfsetroundjoin%
\pgfsetlinewidth{0.301125pt}%
\definecolor{currentstroke}{rgb}{0.500000,0.500000,0.500000}%
\pgfsetstrokecolor{currentstroke}%
\pgfsetstrokeopacity{0.300000}%
\pgfsetdash{}{0pt}%
\pgfpathmoveto{\pgfqpoint{0.937154in}{0.925334in}}%
\pgfusepath{stroke}%
\end{pgfscope}%
\begin{pgfscope}%
\pgfpathrectangle{\pgfqpoint{0.647939in}{0.492442in}}{\pgfqpoint{3.079299in}{3.079299in}}%
\pgfusepath{clip}%
\pgfsetroundcap%
\pgfsetroundjoin%
\definecolor{currentfill}{rgb}{0.500000,0.500000,0.500000}%
\pgfsetfillcolor{currentfill}%
\pgfsetfillopacity{0.300000}%
\pgfsetlinewidth{0.301125pt}%
\definecolor{currentstroke}{rgb}{0.500000,0.500000,0.500000}%
\pgfsetstrokecolor{currentstroke}%
\pgfsetstrokeopacity{0.300000}%
\pgfsetdash{}{0pt}%
\pgfpathmoveto{\pgfqpoint{0.000000in}{0.000000in}}%
\pgfpathlineto{\pgfqpoint{0.000000in}{0.000000in}}%
\pgfpathclose%
\pgfusepath{stroke,fill}%
\end{pgfscope}%
\begin{pgfscope}%
\pgfpathrectangle{\pgfqpoint{0.647939in}{0.492442in}}{\pgfqpoint{3.079299in}{3.079299in}}%
\pgfusepath{clip}%
\pgfsetroundcap%
\pgfsetroundjoin%
\pgfsetlinewidth{0.301125pt}%
\definecolor{currentstroke}{rgb}{0.500000,0.500000,0.500000}%
\pgfsetstrokecolor{currentstroke}%
\pgfsetstrokeopacity{0.300000}%
\pgfsetdash{}{0pt}%
\pgfpathmoveto{\pgfqpoint{1.118646in}{0.951178in}}%
\pgfusepath{stroke}%
\end{pgfscope}%
\begin{pgfscope}%
\pgfpathrectangle{\pgfqpoint{0.647939in}{0.492442in}}{\pgfqpoint{3.079299in}{3.079299in}}%
\pgfusepath{clip}%
\pgfsetroundcap%
\pgfsetroundjoin%
\definecolor{currentfill}{rgb}{0.500000,0.500000,0.500000}%
\pgfsetfillcolor{currentfill}%
\pgfsetfillopacity{0.300000}%
\pgfsetlinewidth{0.301125pt}%
\definecolor{currentstroke}{rgb}{0.500000,0.500000,0.500000}%
\pgfsetstrokecolor{currentstroke}%
\pgfsetstrokeopacity{0.300000}%
\pgfsetdash{}{0pt}%
\pgfpathmoveto{\pgfqpoint{0.000000in}{0.000000in}}%
\pgfpathlineto{\pgfqpoint{0.000000in}{0.000000in}}%
\pgfpathclose%
\pgfusepath{stroke,fill}%
\end{pgfscope}%
\begin{pgfscope}%
\pgfpathrectangle{\pgfqpoint{0.647939in}{0.492442in}}{\pgfqpoint{3.079299in}{3.079299in}}%
\pgfusepath{clip}%
\pgfsetroundcap%
\pgfsetroundjoin%
\pgfsetlinewidth{0.301125pt}%
\definecolor{currentstroke}{rgb}{0.500000,0.500000,0.500000}%
\pgfsetstrokecolor{currentstroke}%
\pgfsetstrokeopacity{0.300000}%
\pgfsetdash{}{0pt}%
\pgfpathmoveto{\pgfqpoint{0.872317in}{0.764152in}}%
\pgfusepath{stroke}%
\end{pgfscope}%
\begin{pgfscope}%
\pgfpathrectangle{\pgfqpoint{0.647939in}{0.492442in}}{\pgfqpoint{3.079299in}{3.079299in}}%
\pgfusepath{clip}%
\pgfsetroundcap%
\pgfsetroundjoin%
\definecolor{currentfill}{rgb}{0.500000,0.500000,0.500000}%
\pgfsetfillcolor{currentfill}%
\pgfsetfillopacity{0.300000}%
\pgfsetlinewidth{0.301125pt}%
\definecolor{currentstroke}{rgb}{0.500000,0.500000,0.500000}%
\pgfsetstrokecolor{currentstroke}%
\pgfsetstrokeopacity{0.300000}%
\pgfsetdash{}{0pt}%
\pgfpathmoveto{\pgfqpoint{0.000000in}{0.000000in}}%
\pgfpathlineto{\pgfqpoint{0.000000in}{0.000000in}}%
\pgfpathclose%
\pgfusepath{stroke,fill}%
\end{pgfscope}%
\begin{pgfscope}%
\pgfpathrectangle{\pgfqpoint{0.647939in}{0.492442in}}{\pgfqpoint{3.079299in}{3.079299in}}%
\pgfusepath{clip}%
\pgfsetroundcap%
\pgfsetroundjoin%
\pgfsetlinewidth{0.301125pt}%
\definecolor{currentstroke}{rgb}{0.500000,0.500000,0.500000}%
\pgfsetstrokecolor{currentstroke}%
\pgfsetstrokeopacity{0.300000}%
\pgfsetdash{}{0pt}%
\pgfpathmoveto{\pgfqpoint{0.871442in}{0.627083in}}%
\pgfusepath{stroke}%
\end{pgfscope}%
\begin{pgfscope}%
\pgfpathrectangle{\pgfqpoint{0.647939in}{0.492442in}}{\pgfqpoint{3.079299in}{3.079299in}}%
\pgfusepath{clip}%
\pgfsetroundcap%
\pgfsetroundjoin%
\definecolor{currentfill}{rgb}{0.500000,0.500000,0.500000}%
\pgfsetfillcolor{currentfill}%
\pgfsetfillopacity{0.300000}%
\pgfsetlinewidth{0.301125pt}%
\definecolor{currentstroke}{rgb}{0.500000,0.500000,0.500000}%
\pgfsetstrokecolor{currentstroke}%
\pgfsetstrokeopacity{0.300000}%
\pgfsetdash{}{0pt}%
\pgfpathmoveto{\pgfqpoint{0.000000in}{0.000000in}}%
\pgfpathlineto{\pgfqpoint{0.000000in}{0.000000in}}%
\pgfpathclose%
\pgfusepath{stroke,fill}%
\end{pgfscope}%
\begin{pgfscope}%
\pgfpathrectangle{\pgfqpoint{0.647939in}{0.492442in}}{\pgfqpoint{3.079299in}{3.079299in}}%
\pgfusepath{clip}%
\pgfsetroundcap%
\pgfsetroundjoin%
\pgfsetlinewidth{0.301125pt}%
\definecolor{currentstroke}{rgb}{0.500000,0.500000,0.500000}%
\pgfsetstrokecolor{currentstroke}%
\pgfsetstrokeopacity{0.300000}%
\pgfsetdash{}{0pt}%
\pgfpathmoveto{\pgfqpoint{1.014531in}{3.065354in}}%
\pgfusepath{stroke}%
\end{pgfscope}%
\begin{pgfscope}%
\pgfpathrectangle{\pgfqpoint{0.647939in}{0.492442in}}{\pgfqpoint{3.079299in}{3.079299in}}%
\pgfusepath{clip}%
\pgfsetroundcap%
\pgfsetroundjoin%
\definecolor{currentfill}{rgb}{0.500000,0.500000,0.500000}%
\pgfsetfillcolor{currentfill}%
\pgfsetfillopacity{0.300000}%
\pgfsetlinewidth{0.301125pt}%
\definecolor{currentstroke}{rgb}{0.500000,0.500000,0.500000}%
\pgfsetstrokecolor{currentstroke}%
\pgfsetstrokeopacity{0.300000}%
\pgfsetdash{}{0pt}%
\pgfpathmoveto{\pgfqpoint{0.000000in}{0.000000in}}%
\pgfpathlineto{\pgfqpoint{0.000000in}{0.000000in}}%
\pgfpathclose%
\pgfusepath{stroke,fill}%
\end{pgfscope}%
\begin{pgfscope}%
\pgfpathrectangle{\pgfqpoint{0.647939in}{0.492442in}}{\pgfqpoint{3.079299in}{3.079299in}}%
\pgfusepath{clip}%
\pgfsetroundcap%
\pgfsetroundjoin%
\pgfsetlinewidth{0.301125pt}%
\definecolor{currentstroke}{rgb}{0.500000,0.500000,0.500000}%
\pgfsetstrokecolor{currentstroke}%
\pgfsetstrokeopacity{0.300000}%
\pgfsetdash{}{0pt}%
\pgfpathmoveto{\pgfqpoint{2.083871in}{0.611929in}}%
\pgfusepath{stroke}%
\end{pgfscope}%
\begin{pgfscope}%
\pgfpathrectangle{\pgfqpoint{0.647939in}{0.492442in}}{\pgfqpoint{3.079299in}{3.079299in}}%
\pgfusepath{clip}%
\pgfsetroundcap%
\pgfsetroundjoin%
\definecolor{currentfill}{rgb}{0.500000,0.500000,0.500000}%
\pgfsetfillcolor{currentfill}%
\pgfsetfillopacity{0.300000}%
\pgfsetlinewidth{0.301125pt}%
\definecolor{currentstroke}{rgb}{0.500000,0.500000,0.500000}%
\pgfsetstrokecolor{currentstroke}%
\pgfsetstrokeopacity{0.300000}%
\pgfsetdash{}{0pt}%
\pgfpathmoveto{\pgfqpoint{0.000000in}{0.000000in}}%
\pgfpathlineto{\pgfqpoint{0.000000in}{0.000000in}}%
\pgfpathclose%
\pgfusepath{stroke,fill}%
\end{pgfscope}%
\begin{pgfscope}%
\pgfpathrectangle{\pgfqpoint{0.647939in}{0.492442in}}{\pgfqpoint{3.079299in}{3.079299in}}%
\pgfusepath{clip}%
\pgfsetroundcap%
\pgfsetroundjoin%
\pgfsetlinewidth{0.301125pt}%
\definecolor{currentstroke}{rgb}{0.500000,0.500000,0.500000}%
\pgfsetstrokecolor{currentstroke}%
\pgfsetstrokeopacity{0.300000}%
\pgfsetdash{}{0pt}%
\pgfpathmoveto{\pgfqpoint{3.342997in}{1.696692in}}%
\pgfusepath{stroke}%
\end{pgfscope}%
\begin{pgfscope}%
\pgfpathrectangle{\pgfqpoint{0.647939in}{0.492442in}}{\pgfqpoint{3.079299in}{3.079299in}}%
\pgfusepath{clip}%
\pgfsetroundcap%
\pgfsetroundjoin%
\definecolor{currentfill}{rgb}{0.500000,0.500000,0.500000}%
\pgfsetfillcolor{currentfill}%
\pgfsetfillopacity{0.300000}%
\pgfsetlinewidth{0.301125pt}%
\definecolor{currentstroke}{rgb}{0.500000,0.500000,0.500000}%
\pgfsetstrokecolor{currentstroke}%
\pgfsetstrokeopacity{0.300000}%
\pgfsetdash{}{0pt}%
\pgfpathmoveto{\pgfqpoint{0.000000in}{0.000000in}}%
\pgfpathlineto{\pgfqpoint{0.000000in}{0.000000in}}%
\pgfpathclose%
\pgfusepath{stroke,fill}%
\end{pgfscope}%
\begin{pgfscope}%
\pgfpathrectangle{\pgfqpoint{0.647939in}{0.492442in}}{\pgfqpoint{3.079299in}{3.079299in}}%
\pgfusepath{clip}%
\pgfsetroundcap%
\pgfsetroundjoin%
\pgfsetlinewidth{0.301125pt}%
\definecolor{currentstroke}{rgb}{0.500000,0.500000,0.500000}%
\pgfsetstrokecolor{currentstroke}%
\pgfsetstrokeopacity{0.300000}%
\pgfsetdash{}{0pt}%
\pgfpathmoveto{\pgfqpoint{3.401415in}{1.798966in}}%
\pgfusepath{stroke}%
\end{pgfscope}%
\begin{pgfscope}%
\pgfpathrectangle{\pgfqpoint{0.647939in}{0.492442in}}{\pgfqpoint{3.079299in}{3.079299in}}%
\pgfusepath{clip}%
\pgfsetroundcap%
\pgfsetroundjoin%
\definecolor{currentfill}{rgb}{0.500000,0.500000,0.500000}%
\pgfsetfillcolor{currentfill}%
\pgfsetfillopacity{0.300000}%
\pgfsetlinewidth{0.301125pt}%
\definecolor{currentstroke}{rgb}{0.500000,0.500000,0.500000}%
\pgfsetstrokecolor{currentstroke}%
\pgfsetstrokeopacity{0.300000}%
\pgfsetdash{}{0pt}%
\pgfpathmoveto{\pgfqpoint{0.000000in}{0.000000in}}%
\pgfpathlineto{\pgfqpoint{0.000000in}{0.000000in}}%
\pgfpathclose%
\pgfusepath{stroke,fill}%
\end{pgfscope}%
\begin{pgfscope}%
\pgfpathrectangle{\pgfqpoint{0.647939in}{0.492442in}}{\pgfqpoint{3.079299in}{3.079299in}}%
\pgfusepath{clip}%
\pgfsetroundcap%
\pgfsetroundjoin%
\pgfsetlinewidth{0.301125pt}%
\definecolor{currentstroke}{rgb}{0.500000,0.500000,0.500000}%
\pgfsetstrokecolor{currentstroke}%
\pgfsetstrokeopacity{0.300000}%
\pgfsetdash{}{0pt}%
\pgfpathmoveto{\pgfqpoint{2.805722in}{3.269854in}}%
\pgfusepath{stroke}%
\end{pgfscope}%
\begin{pgfscope}%
\pgfpathrectangle{\pgfqpoint{0.647939in}{0.492442in}}{\pgfqpoint{3.079299in}{3.079299in}}%
\pgfusepath{clip}%
\pgfsetroundcap%
\pgfsetroundjoin%
\definecolor{currentfill}{rgb}{0.500000,0.500000,0.500000}%
\pgfsetfillcolor{currentfill}%
\pgfsetfillopacity{0.300000}%
\pgfsetlinewidth{0.301125pt}%
\definecolor{currentstroke}{rgb}{0.500000,0.500000,0.500000}%
\pgfsetstrokecolor{currentstroke}%
\pgfsetstrokeopacity{0.300000}%
\pgfsetdash{}{0pt}%
\pgfpathmoveto{\pgfqpoint{0.000000in}{0.000000in}}%
\pgfpathlineto{\pgfqpoint{0.000000in}{0.000000in}}%
\pgfpathclose%
\pgfusepath{stroke,fill}%
\end{pgfscope}%
\begin{pgfscope}%
\pgfpathrectangle{\pgfqpoint{0.647939in}{0.492442in}}{\pgfqpoint{3.079299in}{3.079299in}}%
\pgfusepath{clip}%
\pgfsetroundcap%
\pgfsetroundjoin%
\pgfsetlinewidth{0.301125pt}%
\definecolor{currentstroke}{rgb}{0.500000,0.500000,0.500000}%
\pgfsetstrokecolor{currentstroke}%
\pgfsetstrokeopacity{0.300000}%
\pgfsetdash{}{0pt}%
\pgfpathmoveto{\pgfqpoint{2.855339in}{1.587821in}}%
\pgfusepath{stroke}%
\end{pgfscope}%
\begin{pgfscope}%
\pgfpathrectangle{\pgfqpoint{0.647939in}{0.492442in}}{\pgfqpoint{3.079299in}{3.079299in}}%
\pgfusepath{clip}%
\pgfsetroundcap%
\pgfsetroundjoin%
\definecolor{currentfill}{rgb}{0.500000,0.500000,0.500000}%
\pgfsetfillcolor{currentfill}%
\pgfsetfillopacity{0.300000}%
\pgfsetlinewidth{0.301125pt}%
\definecolor{currentstroke}{rgb}{0.500000,0.500000,0.500000}%
\pgfsetstrokecolor{currentstroke}%
\pgfsetstrokeopacity{0.300000}%
\pgfsetdash{}{0pt}%
\pgfpathmoveto{\pgfqpoint{0.000000in}{0.000000in}}%
\pgfpathlineto{\pgfqpoint{0.000000in}{0.000000in}}%
\pgfpathclose%
\pgfusepath{stroke,fill}%
\end{pgfscope}%
\begin{pgfscope}%
\pgfpathrectangle{\pgfqpoint{0.647939in}{0.492442in}}{\pgfqpoint{3.079299in}{3.079299in}}%
\pgfusepath{clip}%
\pgfsetroundcap%
\pgfsetroundjoin%
\pgfsetlinewidth{0.301125pt}%
\definecolor{currentstroke}{rgb}{0.500000,0.500000,0.500000}%
\pgfsetstrokecolor{currentstroke}%
\pgfsetstrokeopacity{0.300000}%
\pgfsetdash{}{0pt}%
\pgfpathmoveto{\pgfqpoint{2.359309in}{3.278306in}}%
\pgfusepath{stroke}%
\end{pgfscope}%
\begin{pgfscope}%
\pgfpathrectangle{\pgfqpoint{0.647939in}{0.492442in}}{\pgfqpoint{3.079299in}{3.079299in}}%
\pgfusepath{clip}%
\pgfsetroundcap%
\pgfsetroundjoin%
\definecolor{currentfill}{rgb}{0.500000,0.500000,0.500000}%
\pgfsetfillcolor{currentfill}%
\pgfsetfillopacity{0.300000}%
\pgfsetlinewidth{0.301125pt}%
\definecolor{currentstroke}{rgb}{0.500000,0.500000,0.500000}%
\pgfsetstrokecolor{currentstroke}%
\pgfsetstrokeopacity{0.300000}%
\pgfsetdash{}{0pt}%
\pgfpathmoveto{\pgfqpoint{0.000000in}{0.000000in}}%
\pgfpathlineto{\pgfqpoint{0.000000in}{0.000000in}}%
\pgfpathclose%
\pgfusepath{stroke,fill}%
\end{pgfscope}%
\begin{pgfscope}%
\pgfpathrectangle{\pgfqpoint{0.647939in}{0.492442in}}{\pgfqpoint{3.079299in}{3.079299in}}%
\pgfusepath{clip}%
\pgfsetroundcap%
\pgfsetroundjoin%
\pgfsetlinewidth{0.301125pt}%
\definecolor{currentstroke}{rgb}{0.500000,0.500000,0.500000}%
\pgfsetstrokecolor{currentstroke}%
\pgfsetstrokeopacity{0.300000}%
\pgfsetdash{}{0pt}%
\pgfpathmoveto{\pgfqpoint{2.294653in}{0.829457in}}%
\pgfusepath{stroke}%
\end{pgfscope}%
\begin{pgfscope}%
\pgfpathrectangle{\pgfqpoint{0.647939in}{0.492442in}}{\pgfqpoint{3.079299in}{3.079299in}}%
\pgfusepath{clip}%
\pgfsetroundcap%
\pgfsetroundjoin%
\definecolor{currentfill}{rgb}{0.500000,0.500000,0.500000}%
\pgfsetfillcolor{currentfill}%
\pgfsetfillopacity{0.300000}%
\pgfsetlinewidth{0.301125pt}%
\definecolor{currentstroke}{rgb}{0.500000,0.500000,0.500000}%
\pgfsetstrokecolor{currentstroke}%
\pgfsetstrokeopacity{0.300000}%
\pgfsetdash{}{0pt}%
\pgfpathmoveto{\pgfqpoint{0.000000in}{0.000000in}}%
\pgfpathlineto{\pgfqpoint{0.000000in}{0.000000in}}%
\pgfpathclose%
\pgfusepath{stroke,fill}%
\end{pgfscope}%
\begin{pgfscope}%
\pgfpathrectangle{\pgfqpoint{0.647939in}{0.492442in}}{\pgfqpoint{3.079299in}{3.079299in}}%
\pgfusepath{clip}%
\pgfsetroundcap%
\pgfsetroundjoin%
\pgfsetlinewidth{0.301125pt}%
\definecolor{currentstroke}{rgb}{0.500000,0.500000,0.500000}%
\pgfsetstrokecolor{currentstroke}%
\pgfsetstrokeopacity{0.300000}%
\pgfsetdash{}{0pt}%
\pgfpathmoveto{\pgfqpoint{3.359760in}{2.545330in}}%
\pgfusepath{stroke}%
\end{pgfscope}%
\begin{pgfscope}%
\pgfpathrectangle{\pgfqpoint{0.647939in}{0.492442in}}{\pgfqpoint{3.079299in}{3.079299in}}%
\pgfusepath{clip}%
\pgfsetroundcap%
\pgfsetroundjoin%
\definecolor{currentfill}{rgb}{0.500000,0.500000,0.500000}%
\pgfsetfillcolor{currentfill}%
\pgfsetfillopacity{0.300000}%
\pgfsetlinewidth{0.301125pt}%
\definecolor{currentstroke}{rgb}{0.500000,0.500000,0.500000}%
\pgfsetstrokecolor{currentstroke}%
\pgfsetstrokeopacity{0.300000}%
\pgfsetdash{}{0pt}%
\pgfpathmoveto{\pgfqpoint{0.000000in}{0.000000in}}%
\pgfpathlineto{\pgfqpoint{0.000000in}{0.000000in}}%
\pgfpathclose%
\pgfusepath{stroke,fill}%
\end{pgfscope}%
\begin{pgfscope}%
\pgfpathrectangle{\pgfqpoint{0.647939in}{0.492442in}}{\pgfqpoint{3.079299in}{3.079299in}}%
\pgfusepath{clip}%
\pgfsetroundcap%
\pgfsetroundjoin%
\pgfsetlinewidth{0.301125pt}%
\definecolor{currentstroke}{rgb}{0.500000,0.500000,0.500000}%
\pgfsetstrokecolor{currentstroke}%
\pgfsetstrokeopacity{0.300000}%
\pgfsetdash{}{0pt}%
\pgfpathmoveto{\pgfqpoint{1.797422in}{3.216033in}}%
\pgfusepath{stroke}%
\end{pgfscope}%
\begin{pgfscope}%
\pgfpathrectangle{\pgfqpoint{0.647939in}{0.492442in}}{\pgfqpoint{3.079299in}{3.079299in}}%
\pgfusepath{clip}%
\pgfsetroundcap%
\pgfsetroundjoin%
\definecolor{currentfill}{rgb}{0.500000,0.500000,0.500000}%
\pgfsetfillcolor{currentfill}%
\pgfsetfillopacity{0.300000}%
\pgfsetlinewidth{0.301125pt}%
\definecolor{currentstroke}{rgb}{0.500000,0.500000,0.500000}%
\pgfsetstrokecolor{currentstroke}%
\pgfsetstrokeopacity{0.300000}%
\pgfsetdash{}{0pt}%
\pgfpathmoveto{\pgfqpoint{0.000000in}{0.000000in}}%
\pgfpathlineto{\pgfqpoint{0.000000in}{0.000000in}}%
\pgfpathclose%
\pgfusepath{stroke,fill}%
\end{pgfscope}%
\begin{pgfscope}%
\pgfpathrectangle{\pgfqpoint{0.647939in}{0.492442in}}{\pgfqpoint{3.079299in}{3.079299in}}%
\pgfusepath{clip}%
\pgfsetroundcap%
\pgfsetroundjoin%
\pgfsetlinewidth{0.301125pt}%
\definecolor{currentstroke}{rgb}{0.500000,0.500000,0.500000}%
\pgfsetstrokecolor{currentstroke}%
\pgfsetstrokeopacity{0.300000}%
\pgfsetdash{}{0pt}%
\pgfpathmoveto{\pgfqpoint{1.157416in}{2.772408in}}%
\pgfusepath{stroke}%
\end{pgfscope}%
\begin{pgfscope}%
\pgfpathrectangle{\pgfqpoint{0.647939in}{0.492442in}}{\pgfqpoint{3.079299in}{3.079299in}}%
\pgfusepath{clip}%
\pgfsetroundcap%
\pgfsetroundjoin%
\definecolor{currentfill}{rgb}{0.500000,0.500000,0.500000}%
\pgfsetfillcolor{currentfill}%
\pgfsetfillopacity{0.300000}%
\pgfsetlinewidth{0.301125pt}%
\definecolor{currentstroke}{rgb}{0.500000,0.500000,0.500000}%
\pgfsetstrokecolor{currentstroke}%
\pgfsetstrokeopacity{0.300000}%
\pgfsetdash{}{0pt}%
\pgfpathmoveto{\pgfqpoint{0.000000in}{0.000000in}}%
\pgfpathlineto{\pgfqpoint{0.000000in}{0.000000in}}%
\pgfpathclose%
\pgfusepath{stroke,fill}%
\end{pgfscope}%
\begin{pgfscope}%
\pgfpathrectangle{\pgfqpoint{0.647939in}{0.492442in}}{\pgfqpoint{3.079299in}{3.079299in}}%
\pgfusepath{clip}%
\pgfsetroundcap%
\pgfsetroundjoin%
\pgfsetlinewidth{0.301125pt}%
\definecolor{currentstroke}{rgb}{0.500000,0.500000,0.500000}%
\pgfsetstrokecolor{currentstroke}%
\pgfsetstrokeopacity{0.300000}%
\pgfsetdash{}{0pt}%
\pgfpathmoveto{\pgfqpoint{1.278871in}{0.977058in}}%
\pgfusepath{stroke}%
\end{pgfscope}%
\begin{pgfscope}%
\pgfpathrectangle{\pgfqpoint{0.647939in}{0.492442in}}{\pgfqpoint{3.079299in}{3.079299in}}%
\pgfusepath{clip}%
\pgfsetroundcap%
\pgfsetroundjoin%
\definecolor{currentfill}{rgb}{0.500000,0.500000,0.500000}%
\pgfsetfillcolor{currentfill}%
\pgfsetfillopacity{0.300000}%
\pgfsetlinewidth{0.301125pt}%
\definecolor{currentstroke}{rgb}{0.500000,0.500000,0.500000}%
\pgfsetstrokecolor{currentstroke}%
\pgfsetstrokeopacity{0.300000}%
\pgfsetdash{}{0pt}%
\pgfpathmoveto{\pgfqpoint{0.000000in}{0.000000in}}%
\pgfpathlineto{\pgfqpoint{0.000000in}{0.000000in}}%
\pgfpathclose%
\pgfusepath{stroke,fill}%
\end{pgfscope}%
\begin{pgfscope}%
\pgfpathrectangle{\pgfqpoint{0.647939in}{0.492442in}}{\pgfqpoint{3.079299in}{3.079299in}}%
\pgfusepath{clip}%
\pgfsetroundcap%
\pgfsetroundjoin%
\pgfsetlinewidth{0.301125pt}%
\definecolor{currentstroke}{rgb}{0.500000,0.500000,0.500000}%
\pgfsetstrokecolor{currentstroke}%
\pgfsetstrokeopacity{0.300000}%
\pgfsetdash{}{0pt}%
\pgfpathmoveto{\pgfqpoint{1.947373in}{0.904144in}}%
\pgfusepath{stroke}%
\end{pgfscope}%
\begin{pgfscope}%
\pgfpathrectangle{\pgfqpoint{0.647939in}{0.492442in}}{\pgfqpoint{3.079299in}{3.079299in}}%
\pgfusepath{clip}%
\pgfsetroundcap%
\pgfsetroundjoin%
\definecolor{currentfill}{rgb}{0.500000,0.500000,0.500000}%
\pgfsetfillcolor{currentfill}%
\pgfsetfillopacity{0.300000}%
\pgfsetlinewidth{0.301125pt}%
\definecolor{currentstroke}{rgb}{0.500000,0.500000,0.500000}%
\pgfsetstrokecolor{currentstroke}%
\pgfsetstrokeopacity{0.300000}%
\pgfsetdash{}{0pt}%
\pgfpathmoveto{\pgfqpoint{0.000000in}{0.000000in}}%
\pgfpathlineto{\pgfqpoint{0.000000in}{0.000000in}}%
\pgfpathclose%
\pgfusepath{stroke,fill}%
\end{pgfscope}%
\begin{pgfscope}%
\pgfpathrectangle{\pgfqpoint{0.647939in}{0.492442in}}{\pgfqpoint{3.079299in}{3.079299in}}%
\pgfusepath{clip}%
\pgfsetroundcap%
\pgfsetroundjoin%
\pgfsetlinewidth{0.301125pt}%
\definecolor{currentstroke}{rgb}{0.500000,0.500000,0.500000}%
\pgfsetstrokecolor{currentstroke}%
\pgfsetstrokeopacity{0.300000}%
\pgfsetdash{}{0pt}%
\pgfpathmoveto{\pgfqpoint{2.874728in}{1.850851in}}%
\pgfusepath{stroke}%
\end{pgfscope}%
\begin{pgfscope}%
\pgfpathrectangle{\pgfqpoint{0.647939in}{0.492442in}}{\pgfqpoint{3.079299in}{3.079299in}}%
\pgfusepath{clip}%
\pgfsetroundcap%
\pgfsetroundjoin%
\definecolor{currentfill}{rgb}{0.500000,0.500000,0.500000}%
\pgfsetfillcolor{currentfill}%
\pgfsetfillopacity{0.300000}%
\pgfsetlinewidth{0.301125pt}%
\definecolor{currentstroke}{rgb}{0.500000,0.500000,0.500000}%
\pgfsetstrokecolor{currentstroke}%
\pgfsetstrokeopacity{0.300000}%
\pgfsetdash{}{0pt}%
\pgfpathmoveto{\pgfqpoint{0.000000in}{0.000000in}}%
\pgfpathlineto{\pgfqpoint{0.000000in}{0.000000in}}%
\pgfpathclose%
\pgfusepath{stroke,fill}%
\end{pgfscope}%
\begin{pgfscope}%
\pgfpathrectangle{\pgfqpoint{0.647939in}{0.492442in}}{\pgfqpoint{3.079299in}{3.079299in}}%
\pgfusepath{clip}%
\pgfsetroundcap%
\pgfsetroundjoin%
\pgfsetlinewidth{0.301125pt}%
\definecolor{currentstroke}{rgb}{0.500000,0.500000,0.500000}%
\pgfsetstrokecolor{currentstroke}%
\pgfsetstrokeopacity{0.300000}%
\pgfsetdash{}{0pt}%
\pgfpathmoveto{\pgfqpoint{3.279972in}{2.334190in}}%
\pgfusepath{stroke}%
\end{pgfscope}%
\begin{pgfscope}%
\pgfpathrectangle{\pgfqpoint{0.647939in}{0.492442in}}{\pgfqpoint{3.079299in}{3.079299in}}%
\pgfusepath{clip}%
\pgfsetroundcap%
\pgfsetroundjoin%
\definecolor{currentfill}{rgb}{0.500000,0.500000,0.500000}%
\pgfsetfillcolor{currentfill}%
\pgfsetfillopacity{0.300000}%
\pgfsetlinewidth{0.301125pt}%
\definecolor{currentstroke}{rgb}{0.500000,0.500000,0.500000}%
\pgfsetstrokecolor{currentstroke}%
\pgfsetstrokeopacity{0.300000}%
\pgfsetdash{}{0pt}%
\pgfpathmoveto{\pgfqpoint{0.000000in}{0.000000in}}%
\pgfpathlineto{\pgfqpoint{0.000000in}{0.000000in}}%
\pgfpathclose%
\pgfusepath{stroke,fill}%
\end{pgfscope}%
\begin{pgfscope}%
\pgfpathrectangle{\pgfqpoint{0.647939in}{0.492442in}}{\pgfqpoint{3.079299in}{3.079299in}}%
\pgfusepath{clip}%
\pgfsetroundcap%
\pgfsetroundjoin%
\pgfsetlinewidth{0.301125pt}%
\definecolor{currentstroke}{rgb}{0.500000,0.500000,0.500000}%
\pgfsetstrokecolor{currentstroke}%
\pgfsetstrokeopacity{0.300000}%
\pgfsetdash{}{0pt}%
\pgfpathmoveto{\pgfqpoint{2.220834in}{3.135304in}}%
\pgfusepath{stroke}%
\end{pgfscope}%
\begin{pgfscope}%
\pgfpathrectangle{\pgfqpoint{0.647939in}{0.492442in}}{\pgfqpoint{3.079299in}{3.079299in}}%
\pgfusepath{clip}%
\pgfsetroundcap%
\pgfsetroundjoin%
\definecolor{currentfill}{rgb}{0.500000,0.500000,0.500000}%
\pgfsetfillcolor{currentfill}%
\pgfsetfillopacity{0.300000}%
\pgfsetlinewidth{0.301125pt}%
\definecolor{currentstroke}{rgb}{0.500000,0.500000,0.500000}%
\pgfsetstrokecolor{currentstroke}%
\pgfsetstrokeopacity{0.300000}%
\pgfsetdash{}{0pt}%
\pgfpathmoveto{\pgfqpoint{0.000000in}{0.000000in}}%
\pgfpathlineto{\pgfqpoint{0.000000in}{0.000000in}}%
\pgfpathclose%
\pgfusepath{stroke,fill}%
\end{pgfscope}%
\begin{pgfscope}%
\pgfpathrectangle{\pgfqpoint{0.647939in}{0.492442in}}{\pgfqpoint{3.079299in}{3.079299in}}%
\pgfusepath{clip}%
\pgfsetroundcap%
\pgfsetroundjoin%
\pgfsetlinewidth{0.301125pt}%
\definecolor{currentstroke}{rgb}{0.500000,0.500000,0.500000}%
\pgfsetstrokecolor{currentstroke}%
\pgfsetstrokeopacity{0.300000}%
\pgfsetdash{}{0pt}%
\pgfpathmoveto{\pgfqpoint{3.105242in}{2.374727in}}%
\pgfusepath{stroke}%
\end{pgfscope}%
\begin{pgfscope}%
\pgfpathrectangle{\pgfqpoint{0.647939in}{0.492442in}}{\pgfqpoint{3.079299in}{3.079299in}}%
\pgfusepath{clip}%
\pgfsetroundcap%
\pgfsetroundjoin%
\definecolor{currentfill}{rgb}{0.500000,0.500000,0.500000}%
\pgfsetfillcolor{currentfill}%
\pgfsetfillopacity{0.300000}%
\pgfsetlinewidth{0.301125pt}%
\definecolor{currentstroke}{rgb}{0.500000,0.500000,0.500000}%
\pgfsetstrokecolor{currentstroke}%
\pgfsetstrokeopacity{0.300000}%
\pgfsetdash{}{0pt}%
\pgfpathmoveto{\pgfqpoint{0.000000in}{0.000000in}}%
\pgfpathlineto{\pgfqpoint{0.000000in}{0.000000in}}%
\pgfpathclose%
\pgfusepath{stroke,fill}%
\end{pgfscope}%
\begin{pgfscope}%
\pgfpathrectangle{\pgfqpoint{0.647939in}{0.492442in}}{\pgfqpoint{3.079299in}{3.079299in}}%
\pgfusepath{clip}%
\pgfsetroundcap%
\pgfsetroundjoin%
\pgfsetlinewidth{0.301125pt}%
\definecolor{currentstroke}{rgb}{0.500000,0.500000,0.500000}%
\pgfsetstrokecolor{currentstroke}%
\pgfsetstrokeopacity{0.300000}%
\pgfsetdash{}{0pt}%
\pgfpathmoveto{\pgfqpoint{1.427862in}{2.388767in}}%
\pgfusepath{stroke}%
\end{pgfscope}%
\begin{pgfscope}%
\pgfpathrectangle{\pgfqpoint{0.647939in}{0.492442in}}{\pgfqpoint{3.079299in}{3.079299in}}%
\pgfusepath{clip}%
\pgfsetroundcap%
\pgfsetroundjoin%
\definecolor{currentfill}{rgb}{0.500000,0.500000,0.500000}%
\pgfsetfillcolor{currentfill}%
\pgfsetfillopacity{0.300000}%
\pgfsetlinewidth{0.301125pt}%
\definecolor{currentstroke}{rgb}{0.500000,0.500000,0.500000}%
\pgfsetstrokecolor{currentstroke}%
\pgfsetstrokeopacity{0.300000}%
\pgfsetdash{}{0pt}%
\pgfpathmoveto{\pgfqpoint{0.000000in}{0.000000in}}%
\pgfpathlineto{\pgfqpoint{0.000000in}{0.000000in}}%
\pgfpathclose%
\pgfusepath{stroke,fill}%
\end{pgfscope}%
\begin{pgfscope}%
\pgfpathrectangle{\pgfqpoint{0.647939in}{0.492442in}}{\pgfqpoint{3.079299in}{3.079299in}}%
\pgfusepath{clip}%
\pgfsetroundcap%
\pgfsetroundjoin%
\pgfsetlinewidth{0.301125pt}%
\definecolor{currentstroke}{rgb}{0.500000,0.500000,0.500000}%
\pgfsetstrokecolor{currentstroke}%
\pgfsetstrokeopacity{0.300000}%
\pgfsetdash{}{0pt}%
\pgfpathmoveto{\pgfqpoint{2.936195in}{2.292092in}}%
\pgfusepath{stroke}%
\end{pgfscope}%
\begin{pgfscope}%
\pgfpathrectangle{\pgfqpoint{0.647939in}{0.492442in}}{\pgfqpoint{3.079299in}{3.079299in}}%
\pgfusepath{clip}%
\pgfsetroundcap%
\pgfsetroundjoin%
\definecolor{currentfill}{rgb}{0.500000,0.500000,0.500000}%
\pgfsetfillcolor{currentfill}%
\pgfsetfillopacity{0.300000}%
\pgfsetlinewidth{0.301125pt}%
\definecolor{currentstroke}{rgb}{0.500000,0.500000,0.500000}%
\pgfsetstrokecolor{currentstroke}%
\pgfsetstrokeopacity{0.300000}%
\pgfsetdash{}{0pt}%
\pgfpathmoveto{\pgfqpoint{0.000000in}{0.000000in}}%
\pgfpathlineto{\pgfqpoint{0.000000in}{0.000000in}}%
\pgfpathclose%
\pgfusepath{stroke,fill}%
\end{pgfscope}%
\begin{pgfscope}%
\pgfpathrectangle{\pgfqpoint{0.647939in}{0.492442in}}{\pgfqpoint{3.079299in}{3.079299in}}%
\pgfusepath{clip}%
\pgfsetroundcap%
\pgfsetroundjoin%
\pgfsetlinewidth{0.301125pt}%
\definecolor{currentstroke}{rgb}{0.500000,0.500000,0.500000}%
\pgfsetstrokecolor{currentstroke}%
\pgfsetstrokeopacity{0.300000}%
\pgfsetdash{}{0pt}%
\pgfpathmoveto{\pgfqpoint{2.579289in}{2.894899in}}%
\pgfusepath{stroke}%
\end{pgfscope}%
\begin{pgfscope}%
\pgfpathrectangle{\pgfqpoint{0.647939in}{0.492442in}}{\pgfqpoint{3.079299in}{3.079299in}}%
\pgfusepath{clip}%
\pgfsetroundcap%
\pgfsetroundjoin%
\definecolor{currentfill}{rgb}{0.500000,0.500000,0.500000}%
\pgfsetfillcolor{currentfill}%
\pgfsetfillopacity{0.300000}%
\pgfsetlinewidth{0.301125pt}%
\definecolor{currentstroke}{rgb}{0.500000,0.500000,0.500000}%
\pgfsetstrokecolor{currentstroke}%
\pgfsetstrokeopacity{0.300000}%
\pgfsetdash{}{0pt}%
\pgfpathmoveto{\pgfqpoint{0.000000in}{0.000000in}}%
\pgfpathlineto{\pgfqpoint{0.000000in}{0.000000in}}%
\pgfpathclose%
\pgfusepath{stroke,fill}%
\end{pgfscope}%
\begin{pgfscope}%
\pgfpathrectangle{\pgfqpoint{0.647939in}{0.492442in}}{\pgfqpoint{3.079299in}{3.079299in}}%
\pgfusepath{clip}%
\pgfsetroundcap%
\pgfsetroundjoin%
\pgfsetlinewidth{0.301125pt}%
\definecolor{currentstroke}{rgb}{0.500000,0.500000,0.500000}%
\pgfsetstrokecolor{currentstroke}%
\pgfsetstrokeopacity{0.300000}%
\pgfsetdash{}{0pt}%
\pgfpathmoveto{\pgfqpoint{1.937267in}{2.937038in}}%
\pgfusepath{stroke}%
\end{pgfscope}%
\begin{pgfscope}%
\pgfpathrectangle{\pgfqpoint{0.647939in}{0.492442in}}{\pgfqpoint{3.079299in}{3.079299in}}%
\pgfusepath{clip}%
\pgfsetroundcap%
\pgfsetroundjoin%
\definecolor{currentfill}{rgb}{0.500000,0.500000,0.500000}%
\pgfsetfillcolor{currentfill}%
\pgfsetfillopacity{0.300000}%
\pgfsetlinewidth{0.301125pt}%
\definecolor{currentstroke}{rgb}{0.500000,0.500000,0.500000}%
\pgfsetstrokecolor{currentstroke}%
\pgfsetstrokeopacity{0.300000}%
\pgfsetdash{}{0pt}%
\pgfpathmoveto{\pgfqpoint{0.000000in}{0.000000in}}%
\pgfpathlineto{\pgfqpoint{0.000000in}{0.000000in}}%
\pgfpathclose%
\pgfusepath{stroke,fill}%
\end{pgfscope}%
\begin{pgfscope}%
\pgfpathrectangle{\pgfqpoint{0.647939in}{0.492442in}}{\pgfqpoint{3.079299in}{3.079299in}}%
\pgfusepath{clip}%
\pgfsetroundcap%
\pgfsetroundjoin%
\pgfsetlinewidth{0.301125pt}%
\definecolor{currentstroke}{rgb}{0.500000,0.500000,0.500000}%
\pgfsetstrokecolor{currentstroke}%
\pgfsetstrokeopacity{0.300000}%
\pgfsetdash{}{0pt}%
\pgfpathmoveto{\pgfqpoint{1.426213in}{1.893336in}}%
\pgfusepath{stroke}%
\end{pgfscope}%
\begin{pgfscope}%
\pgfpathrectangle{\pgfqpoint{0.647939in}{0.492442in}}{\pgfqpoint{3.079299in}{3.079299in}}%
\pgfusepath{clip}%
\pgfsetroundcap%
\pgfsetroundjoin%
\definecolor{currentfill}{rgb}{0.500000,0.500000,0.500000}%
\pgfsetfillcolor{currentfill}%
\pgfsetfillopacity{0.300000}%
\pgfsetlinewidth{0.301125pt}%
\definecolor{currentstroke}{rgb}{0.500000,0.500000,0.500000}%
\pgfsetstrokecolor{currentstroke}%
\pgfsetstrokeopacity{0.300000}%
\pgfsetdash{}{0pt}%
\pgfpathmoveto{\pgfqpoint{0.000000in}{0.000000in}}%
\pgfpathlineto{\pgfqpoint{0.000000in}{0.000000in}}%
\pgfpathclose%
\pgfusepath{stroke,fill}%
\end{pgfscope}%
\begin{pgfscope}%
\pgfpathrectangle{\pgfqpoint{0.647939in}{0.492442in}}{\pgfqpoint{3.079299in}{3.079299in}}%
\pgfusepath{clip}%
\pgfsetroundcap%
\pgfsetroundjoin%
\pgfsetlinewidth{0.301125pt}%
\definecolor{currentstroke}{rgb}{0.500000,0.500000,0.500000}%
\pgfsetstrokecolor{currentstroke}%
\pgfsetstrokeopacity{0.300000}%
\pgfsetdash{}{0pt}%
\pgfpathmoveto{\pgfqpoint{2.078971in}{2.791771in}}%
\pgfusepath{stroke}%
\end{pgfscope}%
\begin{pgfscope}%
\pgfpathrectangle{\pgfqpoint{0.647939in}{0.492442in}}{\pgfqpoint{3.079299in}{3.079299in}}%
\pgfusepath{clip}%
\pgfsetroundcap%
\pgfsetroundjoin%
\definecolor{currentfill}{rgb}{0.500000,0.500000,0.500000}%
\pgfsetfillcolor{currentfill}%
\pgfsetfillopacity{0.300000}%
\pgfsetlinewidth{0.301125pt}%
\definecolor{currentstroke}{rgb}{0.500000,0.500000,0.500000}%
\pgfsetstrokecolor{currentstroke}%
\pgfsetstrokeopacity{0.300000}%
\pgfsetdash{}{0pt}%
\pgfpathmoveto{\pgfqpoint{0.000000in}{0.000000in}}%
\pgfpathlineto{\pgfqpoint{0.000000in}{0.000000in}}%
\pgfpathclose%
\pgfusepath{stroke,fill}%
\end{pgfscope}%
\begin{pgfscope}%
\pgfpathrectangle{\pgfqpoint{0.647939in}{0.492442in}}{\pgfqpoint{3.079299in}{3.079299in}}%
\pgfusepath{clip}%
\pgfsetroundcap%
\pgfsetroundjoin%
\pgfsetlinewidth{0.301125pt}%
\definecolor{currentstroke}{rgb}{0.500000,0.500000,0.500000}%
\pgfsetstrokecolor{currentstroke}%
\pgfsetstrokeopacity{0.300000}%
\pgfsetdash{}{0pt}%
\pgfpathmoveto{\pgfqpoint{1.636172in}{2.323300in}}%
\pgfusepath{stroke}%
\end{pgfscope}%
\begin{pgfscope}%
\pgfpathrectangle{\pgfqpoint{0.647939in}{0.492442in}}{\pgfqpoint{3.079299in}{3.079299in}}%
\pgfusepath{clip}%
\pgfsetroundcap%
\pgfsetroundjoin%
\definecolor{currentfill}{rgb}{0.500000,0.500000,0.500000}%
\pgfsetfillcolor{currentfill}%
\pgfsetfillopacity{0.300000}%
\pgfsetlinewidth{0.301125pt}%
\definecolor{currentstroke}{rgb}{0.500000,0.500000,0.500000}%
\pgfsetstrokecolor{currentstroke}%
\pgfsetstrokeopacity{0.300000}%
\pgfsetdash{}{0pt}%
\pgfpathmoveto{\pgfqpoint{0.000000in}{0.000000in}}%
\pgfpathlineto{\pgfqpoint{0.000000in}{0.000000in}}%
\pgfpathclose%
\pgfusepath{stroke,fill}%
\end{pgfscope}%
\begin{pgfscope}%
\pgfpathrectangle{\pgfqpoint{0.647939in}{0.492442in}}{\pgfqpoint{3.079299in}{3.079299in}}%
\pgfusepath{clip}%
\pgfsetroundcap%
\pgfsetroundjoin%
\pgfsetlinewidth{0.301125pt}%
\definecolor{currentstroke}{rgb}{0.500000,0.500000,0.500000}%
\pgfsetstrokecolor{currentstroke}%
\pgfsetstrokeopacity{0.300000}%
\pgfsetdash{}{0pt}%
\pgfpathmoveto{\pgfqpoint{2.610736in}{1.997421in}}%
\pgfusepath{stroke}%
\end{pgfscope}%
\begin{pgfscope}%
\pgfpathrectangle{\pgfqpoint{0.647939in}{0.492442in}}{\pgfqpoint{3.079299in}{3.079299in}}%
\pgfusepath{clip}%
\pgfsetroundcap%
\pgfsetroundjoin%
\definecolor{currentfill}{rgb}{0.500000,0.500000,0.500000}%
\pgfsetfillcolor{currentfill}%
\pgfsetfillopacity{0.300000}%
\pgfsetlinewidth{0.301125pt}%
\definecolor{currentstroke}{rgb}{0.500000,0.500000,0.500000}%
\pgfsetstrokecolor{currentstroke}%
\pgfsetstrokeopacity{0.300000}%
\pgfsetdash{}{0pt}%
\pgfpathmoveto{\pgfqpoint{0.000000in}{0.000000in}}%
\pgfpathlineto{\pgfqpoint{0.000000in}{0.000000in}}%
\pgfpathclose%
\pgfusepath{stroke,fill}%
\end{pgfscope}%
\begin{pgfscope}%
\pgfpathrectangle{\pgfqpoint{0.647939in}{0.492442in}}{\pgfqpoint{3.079299in}{3.079299in}}%
\pgfusepath{clip}%
\pgfsetroundcap%
\pgfsetroundjoin%
\pgfsetlinewidth{0.301125pt}%
\definecolor{currentstroke}{rgb}{0.500000,0.500000,0.500000}%
\pgfsetstrokecolor{currentstroke}%
\pgfsetstrokeopacity{0.300000}%
\pgfsetdash{}{0pt}%
\pgfpathmoveto{\pgfqpoint{2.790340in}{2.875666in}}%
\pgfusepath{stroke}%
\end{pgfscope}%
\begin{pgfscope}%
\pgfpathrectangle{\pgfqpoint{0.647939in}{0.492442in}}{\pgfqpoint{3.079299in}{3.079299in}}%
\pgfusepath{clip}%
\pgfsetroundcap%
\pgfsetroundjoin%
\definecolor{currentfill}{rgb}{0.500000,0.500000,0.500000}%
\pgfsetfillcolor{currentfill}%
\pgfsetfillopacity{0.300000}%
\pgfsetlinewidth{0.301125pt}%
\definecolor{currentstroke}{rgb}{0.500000,0.500000,0.500000}%
\pgfsetstrokecolor{currentstroke}%
\pgfsetstrokeopacity{0.300000}%
\pgfsetdash{}{0pt}%
\pgfpathmoveto{\pgfqpoint{0.000000in}{0.000000in}}%
\pgfpathlineto{\pgfqpoint{0.000000in}{0.000000in}}%
\pgfpathclose%
\pgfusepath{stroke,fill}%
\end{pgfscope}%
\begin{pgfscope}%
\pgfpathrectangle{\pgfqpoint{0.647939in}{0.492442in}}{\pgfqpoint{3.079299in}{3.079299in}}%
\pgfusepath{clip}%
\pgfsetroundcap%
\pgfsetroundjoin%
\pgfsetlinewidth{0.301125pt}%
\definecolor{currentstroke}{rgb}{0.500000,0.500000,0.500000}%
\pgfsetstrokecolor{currentstroke}%
\pgfsetstrokeopacity{0.300000}%
\pgfsetdash{}{0pt}%
\pgfpathmoveto{\pgfqpoint{1.776414in}{2.032675in}}%
\pgfusepath{stroke}%
\end{pgfscope}%
\begin{pgfscope}%
\pgfpathrectangle{\pgfqpoint{0.647939in}{0.492442in}}{\pgfqpoint{3.079299in}{3.079299in}}%
\pgfusepath{clip}%
\pgfsetroundcap%
\pgfsetroundjoin%
\definecolor{currentfill}{rgb}{0.500000,0.500000,0.500000}%
\pgfsetfillcolor{currentfill}%
\pgfsetfillopacity{0.300000}%
\pgfsetlinewidth{0.301125pt}%
\definecolor{currentstroke}{rgb}{0.500000,0.500000,0.500000}%
\pgfsetstrokecolor{currentstroke}%
\pgfsetstrokeopacity{0.300000}%
\pgfsetdash{}{0pt}%
\pgfpathmoveto{\pgfqpoint{0.000000in}{0.000000in}}%
\pgfpathlineto{\pgfqpoint{0.000000in}{0.000000in}}%
\pgfpathclose%
\pgfusepath{stroke,fill}%
\end{pgfscope}%
\begin{pgfscope}%
\pgfpathrectangle{\pgfqpoint{0.647939in}{0.492442in}}{\pgfqpoint{3.079299in}{3.079299in}}%
\pgfusepath{clip}%
\pgfsetroundcap%
\pgfsetroundjoin%
\pgfsetlinewidth{0.301125pt}%
\definecolor{currentstroke}{rgb}{0.500000,0.500000,0.500000}%
\pgfsetstrokecolor{currentstroke}%
\pgfsetstrokeopacity{0.300000}%
\pgfsetdash{}{0pt}%
\pgfpathmoveto{\pgfqpoint{2.160484in}{1.548150in}}%
\pgfusepath{stroke}%
\end{pgfscope}%
\begin{pgfscope}%
\pgfpathrectangle{\pgfqpoint{0.647939in}{0.492442in}}{\pgfqpoint{3.079299in}{3.079299in}}%
\pgfusepath{clip}%
\pgfsetroundcap%
\pgfsetroundjoin%
\definecolor{currentfill}{rgb}{0.500000,0.500000,0.500000}%
\pgfsetfillcolor{currentfill}%
\pgfsetfillopacity{0.300000}%
\pgfsetlinewidth{0.301125pt}%
\definecolor{currentstroke}{rgb}{0.500000,0.500000,0.500000}%
\pgfsetstrokecolor{currentstroke}%
\pgfsetstrokeopacity{0.300000}%
\pgfsetdash{}{0pt}%
\pgfpathmoveto{\pgfqpoint{0.000000in}{0.000000in}}%
\pgfpathlineto{\pgfqpoint{0.000000in}{0.000000in}}%
\pgfpathclose%
\pgfusepath{stroke,fill}%
\end{pgfscope}%
\begin{pgfscope}%
\pgfpathrectangle{\pgfqpoint{0.647939in}{0.492442in}}{\pgfqpoint{3.079299in}{3.079299in}}%
\pgfusepath{clip}%
\pgfsetroundcap%
\pgfsetroundjoin%
\pgfsetlinewidth{0.301125pt}%
\definecolor{currentstroke}{rgb}{0.500000,0.500000,0.500000}%
\pgfsetstrokecolor{currentstroke}%
\pgfsetstrokeopacity{0.300000}%
\pgfsetdash{}{0pt}%
\pgfpathmoveto{\pgfqpoint{2.467321in}{1.908424in}}%
\pgfusepath{stroke}%
\end{pgfscope}%
\begin{pgfscope}%
\pgfpathrectangle{\pgfqpoint{0.647939in}{0.492442in}}{\pgfqpoint{3.079299in}{3.079299in}}%
\pgfusepath{clip}%
\pgfsetroundcap%
\pgfsetroundjoin%
\definecolor{currentfill}{rgb}{0.500000,0.500000,0.500000}%
\pgfsetfillcolor{currentfill}%
\pgfsetfillopacity{0.300000}%
\pgfsetlinewidth{0.301125pt}%
\definecolor{currentstroke}{rgb}{0.500000,0.500000,0.500000}%
\pgfsetstrokecolor{currentstroke}%
\pgfsetstrokeopacity{0.300000}%
\pgfsetdash{}{0pt}%
\pgfpathmoveto{\pgfqpoint{0.000000in}{0.000000in}}%
\pgfpathlineto{\pgfqpoint{0.000000in}{0.000000in}}%
\pgfpathclose%
\pgfusepath{stroke,fill}%
\end{pgfscope}%
\begin{pgfscope}%
\pgfpathrectangle{\pgfqpoint{0.647939in}{0.492442in}}{\pgfqpoint{3.079299in}{3.079299in}}%
\pgfusepath{clip}%
\pgfsetroundcap%
\pgfsetroundjoin%
\pgfsetlinewidth{0.301125pt}%
\definecolor{currentstroke}{rgb}{0.500000,0.500000,0.500000}%
\pgfsetstrokecolor{currentstroke}%
\pgfsetstrokeopacity{0.300000}%
\pgfsetdash{}{0pt}%
\pgfpathmoveto{\pgfqpoint{2.150034in}{2.433760in}}%
\pgfusepath{stroke}%
\end{pgfscope}%
\begin{pgfscope}%
\pgfpathrectangle{\pgfqpoint{0.647939in}{0.492442in}}{\pgfqpoint{3.079299in}{3.079299in}}%
\pgfusepath{clip}%
\pgfsetroundcap%
\pgfsetroundjoin%
\definecolor{currentfill}{rgb}{0.500000,0.500000,0.500000}%
\pgfsetfillcolor{currentfill}%
\pgfsetfillopacity{0.300000}%
\pgfsetlinewidth{0.301125pt}%
\definecolor{currentstroke}{rgb}{0.500000,0.500000,0.500000}%
\pgfsetstrokecolor{currentstroke}%
\pgfsetstrokeopacity{0.300000}%
\pgfsetdash{}{0pt}%
\pgfpathmoveto{\pgfqpoint{0.000000in}{0.000000in}}%
\pgfpathlineto{\pgfqpoint{0.000000in}{0.000000in}}%
\pgfpathclose%
\pgfusepath{stroke,fill}%
\end{pgfscope}%
\begin{pgfscope}%
\pgfpathrectangle{\pgfqpoint{0.647939in}{0.492442in}}{\pgfqpoint{3.079299in}{3.079299in}}%
\pgfusepath{clip}%
\pgfsetroundcap%
\pgfsetroundjoin%
\pgfsetlinewidth{0.301125pt}%
\definecolor{currentstroke}{rgb}{0.500000,0.500000,0.500000}%
\pgfsetstrokecolor{currentstroke}%
\pgfsetstrokeopacity{0.300000}%
\pgfsetdash{}{0pt}%
\pgfpathmoveto{\pgfqpoint{2.306510in}{1.783209in}}%
\pgfusepath{stroke}%
\end{pgfscope}%
\begin{pgfscope}%
\pgfpathrectangle{\pgfqpoint{0.647939in}{0.492442in}}{\pgfqpoint{3.079299in}{3.079299in}}%
\pgfusepath{clip}%
\pgfsetroundcap%
\pgfsetroundjoin%
\definecolor{currentfill}{rgb}{0.500000,0.500000,0.500000}%
\pgfsetfillcolor{currentfill}%
\pgfsetfillopacity{0.300000}%
\pgfsetlinewidth{0.301125pt}%
\definecolor{currentstroke}{rgb}{0.500000,0.500000,0.500000}%
\pgfsetstrokecolor{currentstroke}%
\pgfsetstrokeopacity{0.300000}%
\pgfsetdash{}{0pt}%
\pgfpathmoveto{\pgfqpoint{0.000000in}{0.000000in}}%
\pgfpathlineto{\pgfqpoint{0.000000in}{0.000000in}}%
\pgfpathclose%
\pgfusepath{stroke,fill}%
\end{pgfscope}%
\begin{pgfscope}%
\pgfpathrectangle{\pgfqpoint{0.647939in}{0.492442in}}{\pgfqpoint{3.079299in}{3.079299in}}%
\pgfusepath{clip}%
\pgfsetroundcap%
\pgfsetroundjoin%
\pgfsetlinewidth{0.301125pt}%
\definecolor{currentstroke}{rgb}{0.500000,0.500000,0.500000}%
\pgfsetstrokecolor{currentstroke}%
\pgfsetstrokeopacity{0.300000}%
\pgfsetdash{}{0pt}%
\pgfpathmoveto{\pgfqpoint{2.151668in}{2.286086in}}%
\pgfusepath{stroke}%
\end{pgfscope}%
\begin{pgfscope}%
\pgfpathrectangle{\pgfqpoint{0.647939in}{0.492442in}}{\pgfqpoint{3.079299in}{3.079299in}}%
\pgfusepath{clip}%
\pgfsetroundcap%
\pgfsetroundjoin%
\definecolor{currentfill}{rgb}{0.500000,0.500000,0.500000}%
\pgfsetfillcolor{currentfill}%
\pgfsetfillopacity{0.300000}%
\pgfsetlinewidth{0.301125pt}%
\definecolor{currentstroke}{rgb}{0.500000,0.500000,0.500000}%
\pgfsetstrokecolor{currentstroke}%
\pgfsetstrokeopacity{0.300000}%
\pgfsetdash{}{0pt}%
\pgfpathmoveto{\pgfqpoint{0.000000in}{0.000000in}}%
\pgfpathlineto{\pgfqpoint{0.000000in}{0.000000in}}%
\pgfpathclose%
\pgfusepath{stroke,fill}%
\end{pgfscope}%
\begin{pgfscope}%
\pgfsetrectcap%
\pgfsetmiterjoin%
\pgfsetlinewidth{0.803000pt}%
\definecolor{currentstroke}{rgb}{0.000000,0.000000,0.000000}%
\pgfsetstrokecolor{currentstroke}%
\pgfsetdash{}{0pt}%
\pgfpathmoveto{\pgfqpoint{0.647939in}{0.492442in}}%
\pgfpathlineto{\pgfqpoint{0.647939in}{3.571741in}}%
\pgfusepath{stroke}%
\end{pgfscope}%
\begin{pgfscope}%
\pgfsetrectcap%
\pgfsetmiterjoin%
\pgfsetlinewidth{0.803000pt}%
\definecolor{currentstroke}{rgb}{0.000000,0.000000,0.000000}%
\pgfsetstrokecolor{currentstroke}%
\pgfsetdash{}{0pt}%
\pgfpathmoveto{\pgfqpoint{3.727238in}{0.492442in}}%
\pgfpathlineto{\pgfqpoint{3.727238in}{3.571741in}}%
\pgfusepath{stroke}%
\end{pgfscope}%
\begin{pgfscope}%
\pgfsetrectcap%
\pgfsetmiterjoin%
\pgfsetlinewidth{0.803000pt}%
\definecolor{currentstroke}{rgb}{0.000000,0.000000,0.000000}%
\pgfsetstrokecolor{currentstroke}%
\pgfsetdash{}{0pt}%
\pgfpathmoveto{\pgfqpoint{0.647939in}{0.492442in}}%
\pgfpathlineto{\pgfqpoint{3.727238in}{0.492442in}}%
\pgfusepath{stroke}%
\end{pgfscope}%
\begin{pgfscope}%
\pgfsetrectcap%
\pgfsetmiterjoin%
\pgfsetlinewidth{0.803000pt}%
\definecolor{currentstroke}{rgb}{0.000000,0.000000,0.000000}%
\pgfsetstrokecolor{currentstroke}%
\pgfsetdash{}{0pt}%
\pgfpathmoveto{\pgfqpoint{0.647939in}{3.571741in}}%
\pgfpathlineto{\pgfqpoint{3.727238in}{3.571741in}}%
\pgfusepath{stroke}%
\end{pgfscope}%
\begin{pgfscope}%
\pgfsetbuttcap%
\pgfsetmiterjoin%
\definecolor{currentfill}{rgb}{1.000000,1.000000,1.000000}%
\pgfsetfillcolor{currentfill}%
\pgfsetfillopacity{0.800000}%
\pgfsetlinewidth{1.003750pt}%
\definecolor{currentstroke}{rgb}{0.800000,0.800000,0.800000}%
\pgfsetstrokecolor{currentstroke}%
\pgfsetstrokeopacity{0.800000}%
\pgfsetdash{}{0pt}%
\pgfpathmoveto{\pgfqpoint{3.047844in}{0.554942in}}%
\pgfpathlineto{\pgfqpoint{3.639738in}{0.554942in}}%
\pgfpathquadraticcurveto{\pgfqpoint{3.664738in}{0.554942in}}{\pgfqpoint{3.664738in}{0.579942in}}%
\pgfpathlineto{\pgfqpoint{3.664738in}{0.920908in}}%
\pgfpathquadraticcurveto{\pgfqpoint{3.664738in}{0.945908in}}{\pgfqpoint{3.639738in}{0.945908in}}%
\pgfpathlineto{\pgfqpoint{3.047844in}{0.945908in}}%
\pgfpathquadraticcurveto{\pgfqpoint{3.022844in}{0.945908in}}{\pgfqpoint{3.022844in}{0.920908in}}%
\pgfpathlineto{\pgfqpoint{3.022844in}{0.579942in}}%
\pgfpathquadraticcurveto{\pgfqpoint{3.022844in}{0.554942in}}{\pgfqpoint{3.047844in}{0.554942in}}%
\pgfpathlineto{\pgfqpoint{3.047844in}{0.554942in}}%
\pgfpathclose%
\pgfusepath{stroke,fill}%
\end{pgfscope}%
\begin{pgfscope}%
\pgfsetbuttcap%
\pgfsetmiterjoin%
\pgfsetlinewidth{2.007500pt}%
\definecolor{currentstroke}{rgb}{0.000000,0.000000,1.000000}%
\pgfsetstrokecolor{currentstroke}%
\pgfsetdash{}{0pt}%
\pgfpathmoveto{\pgfqpoint{3.072844in}{0.808408in}}%
\pgfpathlineto{\pgfqpoint{3.322844in}{0.808408in}}%
\pgfpathlineto{\pgfqpoint{3.322844in}{0.895908in}}%
\pgfpathlineto{\pgfqpoint{3.072844in}{0.895908in}}%
\pgfpathlineto{\pgfqpoint{3.072844in}{0.808408in}}%
\pgfpathclose%
\pgfusepath{stroke}%
\end{pgfscope}%
\begin{pgfscope}%
\definecolor{textcolor}{rgb}{0.000000,0.000000,0.000000}%
\pgfsetstrokecolor{textcolor}%
\pgfsetfillcolor{textcolor}%
\pgftext[x=3.422844in,y=0.808408in,left,base]{\color{textcolor}{\ifdefined\pdftexversion\else\setmainfont{Times New Roman}\rmfamily\fi\fontsize{9.000000}{10.800000}\selectfont\catcode`\^=\active\def^{\ifmmode\sp\else\^{}\fi}\catcode`\%=\active\def%{\%}$\X_0$}}%
\end{pgfscope}%
\begin{pgfscope}%
\pgfsetbuttcap%
\pgfsetmiterjoin%
\pgfsetlinewidth{2.007500pt}%
\definecolor{currentstroke}{rgb}{1.000000,0.000000,0.000000}%
\pgfsetstrokecolor{currentstroke}%
\pgfsetdash{}{0pt}%
\pgfpathmoveto{\pgfqpoint{3.072844in}{0.631675in}}%
\pgfpathlineto{\pgfqpoint{3.322844in}{0.631675in}}%
\pgfpathlineto{\pgfqpoint{3.322844in}{0.719175in}}%
\pgfpathlineto{\pgfqpoint{3.072844in}{0.719175in}}%
\pgfpathlineto{\pgfqpoint{3.072844in}{0.631675in}}%
\pgfpathclose%
\pgfusepath{stroke}%
\end{pgfscope}%
\begin{pgfscope}%
\definecolor{textcolor}{rgb}{0.000000,0.000000,0.000000}%
\pgfsetstrokecolor{textcolor}%
\pgfsetfillcolor{textcolor}%
\pgftext[x=3.422844in,y=0.631675in,left,base]{\color{textcolor}{\ifdefined\pdftexversion\else\setmainfont{Times New Roman}\rmfamily\fi\fontsize{9.000000}{10.800000}\selectfont\catcode`\^=\active\def^{\ifmmode\sp\else\^{}\fi}\catcode`\%=\active\def%{\%}$\X_U$}}%
\end{pgfscope}%
\end{pgfpicture}%
\makeatother%
\endgroup%

      \caption{Visualization of the stochastic system behavior (based on 10 random transitions from an $100\times 100$ grid of initial states) and safety specification for the \barrII benchmark.}
      \label{fig:model-barrier2}
\end{figure}

\begin{table}[tb]
      \centering
      \begin{tabular}{ccccccc}
            \toprule
            \textbf{Freq.} & \textbf{Lattice} & $\eta$ & $\gamma$ & $c$  & \textbf{Runtime} & \textbf{Safety} \\
                           & \textbf{Size}    &        &          &      & [mm:ss]          & \textbf{Prob.}  \\ % Units
            \midrule
            17             & $700^2$          & 0.11   & 2        & 0.12 & 85:50            & 63.46\%         \\
            16             & $700^2$          & 0.11   & 2        & 0.12 & 57:15            & 63.24\%         \\
            18             & $600^2$          & 0.13   & 2        & 0.12 & 59:23            & 62.17\%         \\
            17             & $600^2$          & 0.13   & 2        & 0.12 & 43:53            & 61.93\%         \\
            16             & $600^2$          & 0.12   & 2        & 0.12 & 33:25            & 61.78\%         \\
            10             & $800^2$          & 0.11   & 2        & 0.13 & 15:27            & 60.57\%         \\
            18             & $500^2$          & 0.14   & 2        & 0.13 & 33:56            & 60.12\%         \\
            17             & $500^2$          & 0.14   & 2        & 0.13 & 24:10            & 59.95\%         \\
            16             & $500^2$          & 0.14   & 2        & 0.13 & 20:44            & 59.90\%         \\
            10             & $700^2$          & 0.11   & 2        & 0.13 & 10:47            & 59.80\%         \\
            9              & $800^2$          & 0.11   & 2        & 0.13 & 10:39            & 59.59\%         \\
            9              & $700^2$          & 0.12   & 2        & 0.13 & 7:14             & 58.89\%         \\
            10             & $600^2$          & 0.12   & 2        & 0.13 & 7:20             & 58.80\%         \\
            9              & $600^2$          & 0.12   & 2        & 0.14 & 5:22             & 57.98\%         \\
            10             & $500^2$          & 0.13   & 2        & 0.14 & 5:05             & 57.53\%         \\
            18             & $400^2$          & 0.17   & 2        & 0.13 & 19:09            & 57.16\%         \\
            15             & $400^2$          & 0.15   & 2        & 0.14 & 10:01            & 56.87\%         \\
            9              & $500^2$          & 0.13   & 2        & 0.14 & 3:40             & 56.82\%         \\
            10             & $400^2$          & 0.15   & 2        & 0.14 & 3:22             & 55.78\%         \\
            9              & $400^2$          & 0.15   & 2        & 0.14 & 2:20             & 55.25\%         \\
            11             & $300^2$          & 0.18   & 2        & 0.15 & 2:25             & 53.25\%         \\
            13             & $300^2$          & 0.18   & 2        & 0.15 & 3:53             & 53.16\%         \\
            10             & $300^2$          & 0.17   & 2        & 0.15 & 1:54             & 53.13\%         \\
            9              & $300^2$          & 0.17   & 2        & 0.15 & 1:25             & 52.86\%         \\
            18             & $300^2$          & 0.20   & 2        & 0.14 & 12:04            & 52.58\%         \\
            20             & $300^2$          & 0.22   & 2        & 0.14 & 16:48            & 51.93\%         \\
            \bottomrule
      \end{tabular}
      \caption{
            \barrII results, sorted by the last column. For each combination of number of frequencies, $M$, and lattice size (i.e., number of lattice points per dimension), we report the values of $c$, $\gamma$, $\lambda$, the runtime, and the achieved lower bound on the safety probability.}
      \label{tab:results-barrier2}
\end{table}


\subsection{Barrier 3}

We consider a black-box system that has the following dynamics:
\begin{equation*}
      \begin{bmatrix}
            {x}_{1, t+1} \\
            {x}_{2, t+1}
      \end{bmatrix}
      = \begin{bmatrix}
            {x}_{1, t} \\
            {x}_{2, t}
      \end{bmatrix} + 0.1 \begin{bmatrix}
            {x}_{2, t} \\
            \frac{1}{3} {x}^3_{1, t} - {x}_{1,t} - {x}_{2,t}
      \end{bmatrix} + w_t,
\end{equation*}
where $w_t\sim\mathcal{N}(\cdotx\vert 0,0.01I_2)$.
The data sampled from this black-box system is used as input for \lucid.
Given
\begin{align*}
       & \X = [ -3, 2.5 ] \times [ -2 , 1 ]                                                 \\
       & \X_0 = [ 1 , 2 ] \times [ -0.7 , 0.3 ] \cup [ -1.8 , -1.4 ] \times [ -0.1 , 0.1 ]  \\
       & \qquad \cup [-1.4, -1.2] \times [-0.5 , 0.1]                                       \\
       & \X_U = [ 0.4 , 0.6 ] \times [ 0.2 , 0.6 ] \cup [ 0.6 , 0.7 ] \times [ 0.2 , 0.4 ],
\end{align*}
we want to ensure that the system, starting in $\X_0$, does not enter the unsafe regions $\X_U$ within $T=5$ time steps.
The system dynamics and safety specification can be visualized in Figure~\ref{fig:model-barrier3}.
The kernel \hp were set to $\sigma_f=1$, $\sigma_l=[2.99, 4.63]$, and $\lambda=10^{-8}$.
The complete configuration for the \barrIII benchmark is shown in Listing~\ref{lst:barrier3}, while the barrier function is plotted in Figure~\ref{fig:CSBarr3}.
A list of experiments using different combinations of frequencies and lattice sizes, showing their impact on performance and final result, is presented in Table~\ref{tab:results-barrier3}.
\lstinputlisting[language=yaml,style={bgnonumbers},caption={Configuration for \barrIII.},captionpos=b,label={lst:barrier3}]{code/barrier3.yaml}

\begin{figure}[ht]
      \centering
      %% Creator: Matplotlib, PGF backend
%%
%% To include the figure in your LaTeX document, write
%%   \input{<filename>.pgf}
%%
%% Make sure the required packages are loaded in your preamble
%%   \usepackage{pgf}
%%
%% Also ensure that all the required font packages are loaded; for instance,
%% the lmodern package is sometimes necessary when using math font.
%%   \usepackage{lmodern}
%%
%% Figures using additional raster images can only be included by \input if
%% they are in the same directory as the main LaTeX file. For loading figures
%% from other directories you can use the `import` package
%%   \usepackage{import}
%%
%% and then include the figures with
%%   \import{<path to file>}{<filename>.pgf}
%%
%% Matplotlib used the following preamble
%%   \def\mathdefault#1{#1}
%%   \everymath=\expandafter{\the\everymath\displaystyle}
%%   \IfFileExists{scrextend.sty}{
%%     \usepackage[fontsize=10.000000pt]{scrextend}
%%   }{
%%     \renewcommand{\normalsize}{\fontsize{10.000000}{12.000000}\selectfont}
%%     \normalsize
%%   }
%%   
%%   \ifdefined\pdftexversion\else  % non-pdftex case.
%%     \usepackage{fontspec}
%%     \setmainfont{DejaVuSerif.ttf}[Path=\detokenize{/home/campus.ncl.ac.uk/c3054737/miniconda3/envs/pylucid/lib/python3.11/site-packages/matplotlib/mpl-data/fonts/ttf/}]
%%     \setsansfont{DejaVuSans.ttf}[Path=\detokenize{/home/campus.ncl.ac.uk/c3054737/miniconda3/envs/pylucid/lib/python3.11/site-packages/matplotlib/mpl-data/fonts/ttf/}]
%%     \setmonofont{DejaVuSansMono.ttf}[Path=\detokenize{/home/campus.ncl.ac.uk/c3054737/miniconda3/envs/pylucid/lib/python3.11/site-packages/matplotlib/mpl-data/fonts/ttf/}]
%%   \fi
%%   \makeatletter\@ifpackageloaded{underscore}{}{\usepackage[strings]{underscore}}\makeatother
%%
\begingroup%
\makeatletter%
\begin{pgfpicture}%
\pgfpathrectangle{\pgfpointorigin}{\pgfqpoint{5.021738in}{2.971823in}}%
\pgfusepath{use as bounding box, clip}%
\begin{pgfscope}%
\pgfsetbuttcap%
\pgfsetmiterjoin%
\definecolor{currentfill}{rgb}{1.000000,1.000000,1.000000}%
\pgfsetfillcolor{currentfill}%
\pgfsetlinewidth{0.000000pt}%
\definecolor{currentstroke}{rgb}{1.000000,1.000000,1.000000}%
\pgfsetstrokecolor{currentstroke}%
\pgfsetdash{}{0pt}%
\pgfpathmoveto{\pgfqpoint{0.000000in}{0.000000in}}%
\pgfpathlineto{\pgfqpoint{5.021738in}{0.000000in}}%
\pgfpathlineto{\pgfqpoint{5.021738in}{2.971823in}}%
\pgfpathlineto{\pgfqpoint{0.000000in}{2.971823in}}%
\pgfpathlineto{\pgfqpoint{0.000000in}{0.000000in}}%
\pgfpathclose%
\pgfusepath{fill}%
\end{pgfscope}%
\begin{pgfscope}%
\pgfsetbuttcap%
\pgfsetmiterjoin%
\definecolor{currentfill}{rgb}{1.000000,1.000000,1.000000}%
\pgfsetfillcolor{currentfill}%
\pgfsetlinewidth{0.000000pt}%
\definecolor{currentstroke}{rgb}{0.000000,0.000000,0.000000}%
\pgfsetstrokecolor{currentstroke}%
\pgfsetstrokeopacity{0.000000}%
\pgfsetdash{}{0pt}%
\pgfpathmoveto{\pgfqpoint{0.647939in}{0.492442in}}%
\pgfpathlineto{\pgfqpoint{4.921738in}{0.492442in}}%
\pgfpathlineto{\pgfqpoint{4.921738in}{2.823605in}}%
\pgfpathlineto{\pgfqpoint{0.647939in}{2.823605in}}%
\pgfpathlineto{\pgfqpoint{0.647939in}{0.492442in}}%
\pgfpathclose%
\pgfusepath{fill}%
\end{pgfscope}%
\begin{pgfscope}%
\pgfsetbuttcap%
\pgfsetroundjoin%
\definecolor{currentfill}{rgb}{0.000000,0.000000,0.000000}%
\pgfsetfillcolor{currentfill}%
\pgfsetlinewidth{0.803000pt}%
\definecolor{currentstroke}{rgb}{0.000000,0.000000,0.000000}%
\pgfsetstrokecolor{currentstroke}%
\pgfsetdash{}{0pt}%
\pgfsys@defobject{currentmarker}{\pgfqpoint{0.000000in}{-0.048611in}}{\pgfqpoint{0.000000in}{0.000000in}}{%
\pgfpathmoveto{\pgfqpoint{0.000000in}{0.000000in}}%
\pgfpathlineto{\pgfqpoint{0.000000in}{-0.048611in}}%
\pgfusepath{stroke,fill}%
}%
\begin{pgfscope}%
\pgfsys@transformshift{0.647939in}{0.492442in}%
\pgfsys@useobject{currentmarker}{}%
\end{pgfscope}%
\end{pgfscope}%
\begin{pgfscope}%
\definecolor{textcolor}{rgb}{0.000000,0.000000,0.000000}%
\pgfsetstrokecolor{textcolor}%
\pgfsetfillcolor{textcolor}%
\pgftext[x=0.647939in,y=0.395220in,,top]{\color{textcolor}{\ifdefined\pdftexversion\else\setmainfont{Times New Roman}\rmfamily\fi\fontsize{10.000000}{12.000000}\selectfont\catcode`\^=\active\def^{\ifmmode\sp\else\^{}\fi}\catcode`\%=\active\def%{\%}\ensuremath{-}3}}%
\end{pgfscope}%
\begin{pgfscope}%
\pgfsetbuttcap%
\pgfsetroundjoin%
\definecolor{currentfill}{rgb}{0.000000,0.000000,0.000000}%
\pgfsetfillcolor{currentfill}%
\pgfsetlinewidth{0.803000pt}%
\definecolor{currentstroke}{rgb}{0.000000,0.000000,0.000000}%
\pgfsetstrokecolor{currentstroke}%
\pgfsetdash{}{0pt}%
\pgfsys@defobject{currentmarker}{\pgfqpoint{0.000000in}{-0.048611in}}{\pgfqpoint{0.000000in}{0.000000in}}{%
\pgfpathmoveto{\pgfqpoint{0.000000in}{0.000000in}}%
\pgfpathlineto{\pgfqpoint{0.000000in}{-0.048611in}}%
\pgfusepath{stroke,fill}%
}%
\begin{pgfscope}%
\pgfsys@transformshift{1.424993in}{0.492442in}%
\pgfsys@useobject{currentmarker}{}%
\end{pgfscope}%
\end{pgfscope}%
\begin{pgfscope}%
\definecolor{textcolor}{rgb}{0.000000,0.000000,0.000000}%
\pgfsetstrokecolor{textcolor}%
\pgfsetfillcolor{textcolor}%
\pgftext[x=1.424993in,y=0.395220in,,top]{\color{textcolor}{\ifdefined\pdftexversion\else\setmainfont{Times New Roman}\rmfamily\fi\fontsize{10.000000}{12.000000}\selectfont\catcode`\^=\active\def^{\ifmmode\sp\else\^{}\fi}\catcode`\%=\active\def%{\%}\ensuremath{-}2}}%
\end{pgfscope}%
\begin{pgfscope}%
\pgfsetbuttcap%
\pgfsetroundjoin%
\definecolor{currentfill}{rgb}{0.000000,0.000000,0.000000}%
\pgfsetfillcolor{currentfill}%
\pgfsetlinewidth{0.803000pt}%
\definecolor{currentstroke}{rgb}{0.000000,0.000000,0.000000}%
\pgfsetstrokecolor{currentstroke}%
\pgfsetdash{}{0pt}%
\pgfsys@defobject{currentmarker}{\pgfqpoint{0.000000in}{-0.048611in}}{\pgfqpoint{0.000000in}{0.000000in}}{%
\pgfpathmoveto{\pgfqpoint{0.000000in}{0.000000in}}%
\pgfpathlineto{\pgfqpoint{0.000000in}{-0.048611in}}%
\pgfusepath{stroke,fill}%
}%
\begin{pgfscope}%
\pgfsys@transformshift{2.202048in}{0.492442in}%
\pgfsys@useobject{currentmarker}{}%
\end{pgfscope}%
\end{pgfscope}%
\begin{pgfscope}%
\definecolor{textcolor}{rgb}{0.000000,0.000000,0.000000}%
\pgfsetstrokecolor{textcolor}%
\pgfsetfillcolor{textcolor}%
\pgftext[x=2.202048in,y=0.395220in,,top]{\color{textcolor}{\ifdefined\pdftexversion\else\setmainfont{Times New Roman}\rmfamily\fi\fontsize{10.000000}{12.000000}\selectfont\catcode`\^=\active\def^{\ifmmode\sp\else\^{}\fi}\catcode`\%=\active\def%{\%}\ensuremath{-}1}}%
\end{pgfscope}%
\begin{pgfscope}%
\pgfsetbuttcap%
\pgfsetroundjoin%
\definecolor{currentfill}{rgb}{0.000000,0.000000,0.000000}%
\pgfsetfillcolor{currentfill}%
\pgfsetlinewidth{0.803000pt}%
\definecolor{currentstroke}{rgb}{0.000000,0.000000,0.000000}%
\pgfsetstrokecolor{currentstroke}%
\pgfsetdash{}{0pt}%
\pgfsys@defobject{currentmarker}{\pgfqpoint{0.000000in}{-0.048611in}}{\pgfqpoint{0.000000in}{0.000000in}}{%
\pgfpathmoveto{\pgfqpoint{0.000000in}{0.000000in}}%
\pgfpathlineto{\pgfqpoint{0.000000in}{-0.048611in}}%
\pgfusepath{stroke,fill}%
}%
\begin{pgfscope}%
\pgfsys@transformshift{2.979102in}{0.492442in}%
\pgfsys@useobject{currentmarker}{}%
\end{pgfscope}%
\end{pgfscope}%
\begin{pgfscope}%
\definecolor{textcolor}{rgb}{0.000000,0.000000,0.000000}%
\pgfsetstrokecolor{textcolor}%
\pgfsetfillcolor{textcolor}%
\pgftext[x=2.979102in,y=0.395220in,,top]{\color{textcolor}{\ifdefined\pdftexversion\else\setmainfont{Times New Roman}\rmfamily\fi\fontsize{10.000000}{12.000000}\selectfont\catcode`\^=\active\def^{\ifmmode\sp\else\^{}\fi}\catcode`\%=\active\def%{\%}0}}%
\end{pgfscope}%
\begin{pgfscope}%
\pgfsetbuttcap%
\pgfsetroundjoin%
\definecolor{currentfill}{rgb}{0.000000,0.000000,0.000000}%
\pgfsetfillcolor{currentfill}%
\pgfsetlinewidth{0.803000pt}%
\definecolor{currentstroke}{rgb}{0.000000,0.000000,0.000000}%
\pgfsetstrokecolor{currentstroke}%
\pgfsetdash{}{0pt}%
\pgfsys@defobject{currentmarker}{\pgfqpoint{0.000000in}{-0.048611in}}{\pgfqpoint{0.000000in}{0.000000in}}{%
\pgfpathmoveto{\pgfqpoint{0.000000in}{0.000000in}}%
\pgfpathlineto{\pgfqpoint{0.000000in}{-0.048611in}}%
\pgfusepath{stroke,fill}%
}%
\begin{pgfscope}%
\pgfsys@transformshift{3.756157in}{0.492442in}%
\pgfsys@useobject{currentmarker}{}%
\end{pgfscope}%
\end{pgfscope}%
\begin{pgfscope}%
\definecolor{textcolor}{rgb}{0.000000,0.000000,0.000000}%
\pgfsetstrokecolor{textcolor}%
\pgfsetfillcolor{textcolor}%
\pgftext[x=3.756157in,y=0.395220in,,top]{\color{textcolor}{\ifdefined\pdftexversion\else\setmainfont{Times New Roman}\rmfamily\fi\fontsize{10.000000}{12.000000}\selectfont\catcode`\^=\active\def^{\ifmmode\sp\else\^{}\fi}\catcode`\%=\active\def%{\%}1}}%
\end{pgfscope}%
\begin{pgfscope}%
\pgfsetbuttcap%
\pgfsetroundjoin%
\definecolor{currentfill}{rgb}{0.000000,0.000000,0.000000}%
\pgfsetfillcolor{currentfill}%
\pgfsetlinewidth{0.803000pt}%
\definecolor{currentstroke}{rgb}{0.000000,0.000000,0.000000}%
\pgfsetstrokecolor{currentstroke}%
\pgfsetdash{}{0pt}%
\pgfsys@defobject{currentmarker}{\pgfqpoint{0.000000in}{-0.048611in}}{\pgfqpoint{0.000000in}{0.000000in}}{%
\pgfpathmoveto{\pgfqpoint{0.000000in}{0.000000in}}%
\pgfpathlineto{\pgfqpoint{0.000000in}{-0.048611in}}%
\pgfusepath{stroke,fill}%
}%
\begin{pgfscope}%
\pgfsys@transformshift{4.533211in}{0.492442in}%
\pgfsys@useobject{currentmarker}{}%
\end{pgfscope}%
\end{pgfscope}%
\begin{pgfscope}%
\definecolor{textcolor}{rgb}{0.000000,0.000000,0.000000}%
\pgfsetstrokecolor{textcolor}%
\pgfsetfillcolor{textcolor}%
\pgftext[x=4.533211in,y=0.395220in,,top]{\color{textcolor}{\ifdefined\pdftexversion\else\setmainfont{Times New Roman}\rmfamily\fi\fontsize{10.000000}{12.000000}\selectfont\catcode`\^=\active\def^{\ifmmode\sp\else\^{}\fi}\catcode`\%=\active\def%{\%}2}}%
\end{pgfscope}%
\begin{pgfscope}%
\definecolor{textcolor}{rgb}{0.000000,0.000000,0.000000}%
\pgfsetstrokecolor{textcolor}%
\pgfsetfillcolor{textcolor}%
\pgftext[x=2.784839in,y=0.213525in,,top]{\color{textcolor}{\ifdefined\pdftexversion\else\setmainfont{Times New Roman}\rmfamily\fi\fontsize{9.000000}{10.800000}\selectfont\catcode`\^=\active\def^{\ifmmode\sp\else\^{}\fi}\catcode`\%=\active\def%{\%}$x_1$}}%
\end{pgfscope}%
\begin{pgfscope}%
\pgfsetbuttcap%
\pgfsetroundjoin%
\definecolor{currentfill}{rgb}{0.000000,0.000000,0.000000}%
\pgfsetfillcolor{currentfill}%
\pgfsetlinewidth{0.803000pt}%
\definecolor{currentstroke}{rgb}{0.000000,0.000000,0.000000}%
\pgfsetstrokecolor{currentstroke}%
\pgfsetdash{}{0pt}%
\pgfsys@defobject{currentmarker}{\pgfqpoint{-0.048611in}{0.000000in}}{\pgfqpoint{-0.000000in}{0.000000in}}{%
\pgfpathmoveto{\pgfqpoint{-0.000000in}{0.000000in}}%
\pgfpathlineto{\pgfqpoint{-0.048611in}{0.000000in}}%
\pgfusepath{stroke,fill}%
}%
\begin{pgfscope}%
\pgfsys@transformshift{0.647939in}{0.492442in}%
\pgfsys@useobject{currentmarker}{}%
\end{pgfscope}%
\end{pgfscope}%
\begin{pgfscope}%
\definecolor{textcolor}{rgb}{0.000000,0.000000,0.000000}%
\pgfsetstrokecolor{textcolor}%
\pgfsetfillcolor{textcolor}%
\pgftext[x=0.269081in, y=0.444224in, left, base]{\color{textcolor}{\ifdefined\pdftexversion\else\setmainfont{Times New Roman}\rmfamily\fi\fontsize{10.000000}{12.000000}\selectfont\catcode`\^=\active\def^{\ifmmode\sp\else\^{}\fi}\catcode`\%=\active\def%{\%}\ensuremath{-}2.0}}%
\end{pgfscope}%
\begin{pgfscope}%
\pgfsetbuttcap%
\pgfsetroundjoin%
\definecolor{currentfill}{rgb}{0.000000,0.000000,0.000000}%
\pgfsetfillcolor{currentfill}%
\pgfsetlinewidth{0.803000pt}%
\definecolor{currentstroke}{rgb}{0.000000,0.000000,0.000000}%
\pgfsetstrokecolor{currentstroke}%
\pgfsetdash{}{0pt}%
\pgfsys@defobject{currentmarker}{\pgfqpoint{-0.048611in}{0.000000in}}{\pgfqpoint{-0.000000in}{0.000000in}}{%
\pgfpathmoveto{\pgfqpoint{-0.000000in}{0.000000in}}%
\pgfpathlineto{\pgfqpoint{-0.048611in}{0.000000in}}%
\pgfusepath{stroke,fill}%
}%
\begin{pgfscope}%
\pgfsys@transformshift{0.647939in}{0.880969in}%
\pgfsys@useobject{currentmarker}{}%
\end{pgfscope}%
\end{pgfscope}%
\begin{pgfscope}%
\definecolor{textcolor}{rgb}{0.000000,0.000000,0.000000}%
\pgfsetstrokecolor{textcolor}%
\pgfsetfillcolor{textcolor}%
\pgftext[x=0.269081in, y=0.832752in, left, base]{\color{textcolor}{\ifdefined\pdftexversion\else\setmainfont{Times New Roman}\rmfamily\fi\fontsize{10.000000}{12.000000}\selectfont\catcode`\^=\active\def^{\ifmmode\sp\else\^{}\fi}\catcode`\%=\active\def%{\%}\ensuremath{-}1.5}}%
\end{pgfscope}%
\begin{pgfscope}%
\pgfsetbuttcap%
\pgfsetroundjoin%
\definecolor{currentfill}{rgb}{0.000000,0.000000,0.000000}%
\pgfsetfillcolor{currentfill}%
\pgfsetlinewidth{0.803000pt}%
\definecolor{currentstroke}{rgb}{0.000000,0.000000,0.000000}%
\pgfsetstrokecolor{currentstroke}%
\pgfsetdash{}{0pt}%
\pgfsys@defobject{currentmarker}{\pgfqpoint{-0.048611in}{0.000000in}}{\pgfqpoint{-0.000000in}{0.000000in}}{%
\pgfpathmoveto{\pgfqpoint{-0.000000in}{0.000000in}}%
\pgfpathlineto{\pgfqpoint{-0.048611in}{0.000000in}}%
\pgfusepath{stroke,fill}%
}%
\begin{pgfscope}%
\pgfsys@transformshift{0.647939in}{1.269497in}%
\pgfsys@useobject{currentmarker}{}%
\end{pgfscope}%
\end{pgfscope}%
\begin{pgfscope}%
\definecolor{textcolor}{rgb}{0.000000,0.000000,0.000000}%
\pgfsetstrokecolor{textcolor}%
\pgfsetfillcolor{textcolor}%
\pgftext[x=0.269081in, y=1.221279in, left, base]{\color{textcolor}{\ifdefined\pdftexversion\else\setmainfont{Times New Roman}\rmfamily\fi\fontsize{10.000000}{12.000000}\selectfont\catcode`\^=\active\def^{\ifmmode\sp\else\^{}\fi}\catcode`\%=\active\def%{\%}\ensuremath{-}1.0}}%
\end{pgfscope}%
\begin{pgfscope}%
\pgfsetbuttcap%
\pgfsetroundjoin%
\definecolor{currentfill}{rgb}{0.000000,0.000000,0.000000}%
\pgfsetfillcolor{currentfill}%
\pgfsetlinewidth{0.803000pt}%
\definecolor{currentstroke}{rgb}{0.000000,0.000000,0.000000}%
\pgfsetstrokecolor{currentstroke}%
\pgfsetdash{}{0pt}%
\pgfsys@defobject{currentmarker}{\pgfqpoint{-0.048611in}{0.000000in}}{\pgfqpoint{-0.000000in}{0.000000in}}{%
\pgfpathmoveto{\pgfqpoint{-0.000000in}{0.000000in}}%
\pgfpathlineto{\pgfqpoint{-0.048611in}{0.000000in}}%
\pgfusepath{stroke,fill}%
}%
\begin{pgfscope}%
\pgfsys@transformshift{0.647939in}{1.658024in}%
\pgfsys@useobject{currentmarker}{}%
\end{pgfscope}%
\end{pgfscope}%
\begin{pgfscope}%
\definecolor{textcolor}{rgb}{0.000000,0.000000,0.000000}%
\pgfsetstrokecolor{textcolor}%
\pgfsetfillcolor{textcolor}%
\pgftext[x=0.269081in, y=1.609806in, left, base]{\color{textcolor}{\ifdefined\pdftexversion\else\setmainfont{Times New Roman}\rmfamily\fi\fontsize{10.000000}{12.000000}\selectfont\catcode`\^=\active\def^{\ifmmode\sp\else\^{}\fi}\catcode`\%=\active\def%{\%}\ensuremath{-}0.5}}%
\end{pgfscope}%
\begin{pgfscope}%
\pgfsetbuttcap%
\pgfsetroundjoin%
\definecolor{currentfill}{rgb}{0.000000,0.000000,0.000000}%
\pgfsetfillcolor{currentfill}%
\pgfsetlinewidth{0.803000pt}%
\definecolor{currentstroke}{rgb}{0.000000,0.000000,0.000000}%
\pgfsetstrokecolor{currentstroke}%
\pgfsetdash{}{0pt}%
\pgfsys@defobject{currentmarker}{\pgfqpoint{-0.048611in}{0.000000in}}{\pgfqpoint{-0.000000in}{0.000000in}}{%
\pgfpathmoveto{\pgfqpoint{-0.000000in}{0.000000in}}%
\pgfpathlineto{\pgfqpoint{-0.048611in}{0.000000in}}%
\pgfusepath{stroke,fill}%
}%
\begin{pgfscope}%
\pgfsys@transformshift{0.647939in}{2.046551in}%
\pgfsys@useobject{currentmarker}{}%
\end{pgfscope}%
\end{pgfscope}%
\begin{pgfscope}%
\definecolor{textcolor}{rgb}{0.000000,0.000000,0.000000}%
\pgfsetstrokecolor{textcolor}%
\pgfsetfillcolor{textcolor}%
\pgftext[x=0.377106in, y=1.998333in, left, base]{\color{textcolor}{\ifdefined\pdftexversion\else\setmainfont{Times New Roman}\rmfamily\fi\fontsize{10.000000}{12.000000}\selectfont\catcode`\^=\active\def^{\ifmmode\sp\else\^{}\fi}\catcode`\%=\active\def%{\%}0.0}}%
\end{pgfscope}%
\begin{pgfscope}%
\pgfsetbuttcap%
\pgfsetroundjoin%
\definecolor{currentfill}{rgb}{0.000000,0.000000,0.000000}%
\pgfsetfillcolor{currentfill}%
\pgfsetlinewidth{0.803000pt}%
\definecolor{currentstroke}{rgb}{0.000000,0.000000,0.000000}%
\pgfsetstrokecolor{currentstroke}%
\pgfsetdash{}{0pt}%
\pgfsys@defobject{currentmarker}{\pgfqpoint{-0.048611in}{0.000000in}}{\pgfqpoint{-0.000000in}{0.000000in}}{%
\pgfpathmoveto{\pgfqpoint{-0.000000in}{0.000000in}}%
\pgfpathlineto{\pgfqpoint{-0.048611in}{0.000000in}}%
\pgfusepath{stroke,fill}%
}%
\begin{pgfscope}%
\pgfsys@transformshift{0.647939in}{2.435078in}%
\pgfsys@useobject{currentmarker}{}%
\end{pgfscope}%
\end{pgfscope}%
\begin{pgfscope}%
\definecolor{textcolor}{rgb}{0.000000,0.000000,0.000000}%
\pgfsetstrokecolor{textcolor}%
\pgfsetfillcolor{textcolor}%
\pgftext[x=0.377106in, y=2.386860in, left, base]{\color{textcolor}{\ifdefined\pdftexversion\else\setmainfont{Times New Roman}\rmfamily\fi\fontsize{10.000000}{12.000000}\selectfont\catcode`\^=\active\def^{\ifmmode\sp\else\^{}\fi}\catcode`\%=\active\def%{\%}0.5}}%
\end{pgfscope}%
\begin{pgfscope}%
\pgfsetbuttcap%
\pgfsetroundjoin%
\definecolor{currentfill}{rgb}{0.000000,0.000000,0.000000}%
\pgfsetfillcolor{currentfill}%
\pgfsetlinewidth{0.803000pt}%
\definecolor{currentstroke}{rgb}{0.000000,0.000000,0.000000}%
\pgfsetstrokecolor{currentstroke}%
\pgfsetdash{}{0pt}%
\pgfsys@defobject{currentmarker}{\pgfqpoint{-0.048611in}{0.000000in}}{\pgfqpoint{-0.000000in}{0.000000in}}{%
\pgfpathmoveto{\pgfqpoint{-0.000000in}{0.000000in}}%
\pgfpathlineto{\pgfqpoint{-0.048611in}{0.000000in}}%
\pgfusepath{stroke,fill}%
}%
\begin{pgfscope}%
\pgfsys@transformshift{0.647939in}{2.823605in}%
\pgfsys@useobject{currentmarker}{}%
\end{pgfscope}%
\end{pgfscope}%
\begin{pgfscope}%
\definecolor{textcolor}{rgb}{0.000000,0.000000,0.000000}%
\pgfsetstrokecolor{textcolor}%
\pgfsetfillcolor{textcolor}%
\pgftext[x=0.377106in, y=2.775388in, left, base]{\color{textcolor}{\ifdefined\pdftexversion\else\setmainfont{Times New Roman}\rmfamily\fi\fontsize{10.000000}{12.000000}\selectfont\catcode`\^=\active\def^{\ifmmode\sp\else\^{}\fi}\catcode`\%=\active\def%{\%}1.0}}%
\end{pgfscope}%
\begin{pgfscope}%
\definecolor{textcolor}{rgb}{0.000000,0.000000,0.000000}%
\pgfsetstrokecolor{textcolor}%
\pgfsetfillcolor{textcolor}%
\pgftext[x=0.213525in,y=1.658024in,,bottom,rotate=90.000000]{\color{textcolor}{\ifdefined\pdftexversion\else\setmainfont{Times New Roman}\rmfamily\fi\fontsize{9.000000}{10.800000}\selectfont\catcode`\^=\active\def^{\ifmmode\sp\else\^{}\fi}\catcode`\%=\active\def%{\%}$x_2$}}%
\end{pgfscope}%
\begin{pgfscope}%
\pgfpathrectangle{\pgfqpoint{0.647939in}{0.492442in}}{\pgfqpoint{4.273799in}{2.331163in}}%
\pgfusepath{clip}%
\pgfsetbuttcap%
\pgfsetroundjoin%
\pgfsetlinewidth{0.803000pt}%
\definecolor{currentstroke}{rgb}{0.501961,0.501961,0.501961}%
\pgfsetstrokecolor{currentstroke}%
\pgfsetdash{}{0pt}%
\pgfpathmoveto{\pgfqpoint{2.262612in}{0.492442in}}%
\pgfpathlineto{\pgfqpoint{2.235563in}{0.528690in}}%
\pgfpathlineto{\pgfqpoint{2.199898in}{0.576702in}}%
\pgfpathlineto{\pgfqpoint{2.164491in}{0.624771in}}%
\pgfpathlineto{\pgfqpoint{2.129298in}{0.672887in}}%
\pgfpathlineto{\pgfqpoint{2.094279in}{0.721040in}}%
\pgfpathlineto{\pgfqpoint{2.059388in}{0.769221in}}%
\pgfpathlineto{\pgfqpoint{2.024576in}{0.817419in}}%
\pgfpathlineto{\pgfqpoint{1.989781in}{0.865621in}}%
\pgfpathlineto{\pgfqpoint{1.954925in}{0.913809in}}%
\pgfpathlineto{\pgfqpoint{1.919917in}{0.961965in}}%
\pgfpathlineto{\pgfqpoint{1.884653in}{1.010065in}}%
\pgfpathlineto{\pgfqpoint{1.849004in}{1.058079in}}%
\pgfpathlineto{\pgfqpoint{1.812801in}{1.105971in}}%
\pgfpathlineto{\pgfqpoint{1.775802in}{1.153680in}}%
\pgfpathlineto{\pgfqpoint{1.737671in}{1.201122in}}%
\pgfpathlineto{\pgfqpoint{1.697937in}{1.248166in}}%
\pgfpathlineto{\pgfqpoint{1.655864in}{1.294592in}}%
\pgfpathlineto{\pgfqpoint{1.610218in}{1.339981in}}%
\pgfpathlineto{\pgfqpoint{1.558680in}{1.383392in}}%
\pgfpathlineto{\pgfqpoint{1.514356in}{1.412749in}}%
\pgfpathlineto{\pgfqpoint{1.474424in}{1.431827in}}%
\pgfpathlineto{\pgfqpoint{1.434121in}{1.443241in}}%
\pgfpathlineto{\pgfqpoint{1.382835in}{1.445679in}}%
\pgfpathlineto{\pgfqpoint{1.334520in}{1.435539in}}%
\pgfpathlineto{\pgfqpoint{1.334520in}{1.435539in}}%
\pgfpathlineto{\pgfqpoint{1.278198in}{1.409025in}}%
\pgfpathlineto{\pgfqpoint{1.278198in}{1.409025in}}%
\pgfpathlineto{\pgfqpoint{1.221257in}{1.367953in}}%
\pgfpathlineto{\pgfqpoint{1.172970in}{1.323472in}}%
\pgfpathlineto{\pgfqpoint{1.129941in}{1.277347in}}%
\pgfpathlineto{\pgfqpoint{1.090455in}{1.230274in}}%
\pgfpathlineto{\pgfqpoint{1.053528in}{1.182584in}}%
\pgfpathlineto{\pgfqpoint{1.018546in}{1.134453in}}%
\pgfpathlineto{\pgfqpoint{0.985097in}{1.085990in}}%
\pgfpathlineto{\pgfqpoint{0.952902in}{1.037264in}}%
\pgfpathlineto{\pgfqpoint{0.921781in}{0.988330in}}%
\pgfpathlineto{\pgfqpoint{0.891590in}{0.939227in}}%
\pgfpathlineto{\pgfqpoint{0.862193in}{0.889975in}}%
\pgfpathlineto{\pgfqpoint{0.833509in}{0.840596in}}%
\pgfpathlineto{\pgfqpoint{0.805475in}{0.791110in}}%
\pgfpathlineto{\pgfqpoint{0.778013in}{0.741524in}}%
\pgfpathlineto{\pgfqpoint{0.751083in}{0.691851in}}%
\pgfpathlineto{\pgfqpoint{0.724649in}{0.642101in}}%
\pgfpathlineto{\pgfqpoint{0.698665in}{0.592278in}}%
\pgfpathlineto{\pgfqpoint{0.673106in}{0.542391in}}%
\pgfpathlineto{\pgfqpoint{0.647939in}{0.492442in}}%
\pgfpathlineto{\pgfqpoint{0.647939in}{0.492442in}}%
\pgfusepath{stroke}%
\end{pgfscope}%
\begin{pgfscope}%
\pgfpathrectangle{\pgfqpoint{0.647939in}{0.492442in}}{\pgfqpoint{4.273799in}{2.331163in}}%
\pgfusepath{clip}%
\pgfsetbuttcap%
\pgfsetroundjoin%
\pgfsetlinewidth{0.803000pt}%
\definecolor{currentstroke}{rgb}{0.501961,0.501961,0.501961}%
\pgfsetstrokecolor{currentstroke}%
\pgfsetdash{}{0pt}%
\pgfpathmoveto{\pgfqpoint{1.810609in}{0.492442in}}%
\pgfpathlineto{\pgfqpoint{1.808656in}{0.494753in}}%
\pgfpathlineto{\pgfqpoint{1.768487in}{0.541694in}}%
\pgfpathlineto{\pgfqpoint{1.727341in}{0.588382in}}%
\pgfpathlineto{\pgfqpoint{1.684917in}{0.634728in}}%
\pgfpathlineto{\pgfqpoint{1.640806in}{0.680601in}}%
\pgfpathlineto{\pgfqpoint{1.594431in}{0.725799in}}%
\pgfpathlineto{\pgfqpoint{1.544929in}{0.769987in}}%
\pgfpathlineto{\pgfqpoint{1.490945in}{0.812548in}}%
\pgfpathlineto{\pgfqpoint{1.431529in}{0.851483in}}%
\pgfpathlineto{\pgfqpoint{1.378365in}{0.878357in}}%
\pgfpathlineto{\pgfqpoint{1.328298in}{0.895989in}}%
\pgfpathlineto{\pgfqpoint{1.276190in}{0.905732in}}%
\pgfpathlineto{\pgfqpoint{1.214512in}{0.905260in}}%
\pgfpathlineto{\pgfqpoint{1.156963in}{0.892774in}}%
\pgfpathlineto{\pgfqpoint{1.156963in}{0.892774in}}%
\pgfpathlineto{\pgfqpoint{1.084374in}{0.860220in}}%
\pgfpathlineto{\pgfqpoint{1.024693in}{0.820103in}}%
\pgfpathlineto{\pgfqpoint{0.973555in}{0.776571in}}%
\pgfpathlineto{\pgfqpoint{0.928068in}{0.731166in}}%
\pgfpathlineto{\pgfqpoint{0.886598in}{0.684608in}}%
\pgfpathlineto{\pgfqpoint{0.848134in}{0.637273in}}%
\pgfpathlineto{\pgfqpoint{0.812010in}{0.589383in}}%
\pgfpathlineto{\pgfqpoint{0.777760in}{0.541074in}}%
\pgfpathlineto{\pgfqpoint{0.745071in}{0.492442in}}%
\pgfpathlineto{\pgfqpoint{0.745071in}{0.492442in}}%
\pgfusepath{stroke}%
\end{pgfscope}%
\begin{pgfscope}%
\pgfpathrectangle{\pgfqpoint{0.647939in}{0.492442in}}{\pgfqpoint{4.273799in}{2.331163in}}%
\pgfusepath{clip}%
\pgfsetbuttcap%
\pgfsetroundjoin%
\pgfsetlinewidth{0.803000pt}%
\definecolor{currentstroke}{rgb}{0.501961,0.501961,0.501961}%
\pgfsetstrokecolor{currentstroke}%
\pgfsetdash{}{0pt}%
\pgfpathmoveto{\pgfqpoint{1.577512in}{0.492442in}}%
\pgfpathlineto{\pgfqpoint{1.570843in}{0.498531in}}%
\pgfpathlineto{\pgfqpoint{1.520472in}{0.542427in}}%
\pgfpathlineto{\pgfqpoint{1.465984in}{0.584816in}}%
\pgfpathlineto{\pgfqpoint{1.405531in}{0.624650in}}%
\pgfpathlineto{\pgfqpoint{1.347860in}{0.654653in}}%
\pgfpathlineto{\pgfqpoint{1.294343in}{0.674741in}}%
\pgfpathlineto{\pgfqpoint{1.240648in}{0.686699in}}%
\pgfpathlineto{\pgfqpoint{1.180588in}{0.689666in}}%
\pgfpathlineto{\pgfqpoint{1.121927in}{0.681554in}}%
\pgfpathlineto{\pgfqpoint{1.121927in}{0.681554in}}%
\pgfpathlineto{\pgfqpoint{1.058941in}{0.660384in}}%
\pgfpathlineto{\pgfqpoint{1.058941in}{0.660384in}}%
\pgfpathlineto{\pgfqpoint{0.992163in}{0.623956in}}%
\pgfpathlineto{\pgfqpoint{0.935830in}{0.582391in}}%
\pgfpathlineto{\pgfqpoint{0.886555in}{0.538195in}}%
\pgfpathlineto{\pgfqpoint{0.842203in}{0.492442in}}%
\pgfpathlineto{\pgfqpoint{0.842203in}{0.492442in}}%
\pgfusepath{stroke}%
\end{pgfscope}%
\begin{pgfscope}%
\pgfpathrectangle{\pgfqpoint{0.647939in}{0.492442in}}{\pgfqpoint{4.273799in}{2.331163in}}%
\pgfusepath{clip}%
\pgfsetbuttcap%
\pgfsetroundjoin%
\pgfsetlinewidth{0.803000pt}%
\definecolor{currentstroke}{rgb}{0.501961,0.501961,0.501961}%
\pgfsetstrokecolor{currentstroke}%
\pgfsetdash{}{0pt}%
\pgfpathmoveto{\pgfqpoint{1.412058in}{0.492442in}}%
\pgfpathlineto{\pgfqpoint{1.371106in}{0.514992in}}%
\pgfpathlineto{\pgfqpoint{1.313134in}{0.542763in}}%
\pgfpathlineto{\pgfqpoint{1.258561in}{0.561102in}}%
\pgfpathlineto{\pgfqpoint{1.202658in}{0.571386in}}%
\pgfpathlineto{\pgfqpoint{1.138994in}{0.571915in}}%
\pgfpathlineto{\pgfqpoint{1.078557in}{0.560991in}}%
\pgfpathlineto{\pgfqpoint{1.078557in}{0.560991in}}%
\pgfpathlineto{\pgfqpoint{1.002447in}{0.530905in}}%
\pgfpathlineto{\pgfqpoint{0.939334in}{0.492442in}}%
\pgfpathlineto{\pgfqpoint{0.939334in}{0.492442in}}%
\pgfusepath{stroke}%
\end{pgfscope}%
\begin{pgfscope}%
\pgfpathrectangle{\pgfqpoint{0.647939in}{0.492442in}}{\pgfqpoint{4.273799in}{2.331163in}}%
\pgfusepath{clip}%
\pgfsetbuttcap%
\pgfsetroundjoin%
\pgfsetlinewidth{0.803000pt}%
\definecolor{currentstroke}{rgb}{0.501961,0.501961,0.501961}%
\pgfsetstrokecolor{currentstroke}%
\pgfsetdash{}{0pt}%
\pgfpathmoveto{\pgfqpoint{1.716389in}{0.492442in}}%
\pgfpathlineto{\pgfqpoint{1.716389in}{0.492442in}}%
\pgfpathlineto{\pgfqpoint{1.673377in}{0.538625in}}%
\pgfpathlineto{\pgfqpoint{1.628660in}{0.584320in}}%
\pgfpathlineto{\pgfqpoint{1.581672in}{0.629326in}}%
\pgfpathlineto{\pgfqpoint{1.531583in}{0.673314in}}%
\pgfpathlineto{\pgfqpoint{1.477115in}{0.715702in}}%
\pgfpathlineto{\pgfqpoint{1.416169in}{0.755319in}}%
\pgfpathlineto{\pgfqpoint{1.345280in}{0.789363in}}%
\pgfpathlineto{\pgfqpoint{1.345280in}{0.789363in}}%
\pgfpathlineto{\pgfqpoint{1.284237in}{0.807072in}}%
\pgfpathlineto{\pgfqpoint{1.284237in}{0.807072in}}%
\pgfpathlineto{\pgfqpoint{1.228193in}{0.812972in}}%
\pgfpathlineto{\pgfqpoint{1.171386in}{0.808331in}}%
\pgfpathlineto{\pgfqpoint{1.121822in}{0.795413in}}%
\pgfpathlineto{\pgfqpoint{1.073575in}{0.774850in}}%
\pgfpathlineto{\pgfqpoint{1.023172in}{0.745052in}}%
\pgfpathlineto{\pgfqpoint{0.968433in}{0.703376in}}%
\pgfpathlineto{\pgfqpoint{0.920068in}{0.658890in}}%
\pgfusepath{stroke}%
\end{pgfscope}%
\begin{pgfscope}%
\pgfpathrectangle{\pgfqpoint{0.647939in}{0.492442in}}{\pgfqpoint{4.273799in}{2.331163in}}%
\pgfusepath{clip}%
\pgfsetbuttcap%
\pgfsetroundjoin%
\pgfsetlinewidth{0.803000pt}%
\definecolor{currentstroke}{rgb}{0.501961,0.501961,0.501961}%
\pgfsetstrokecolor{currentstroke}%
\pgfsetdash{}{0pt}%
\pgfpathmoveto{\pgfqpoint{1.910652in}{0.492442in}}%
\pgfpathlineto{\pgfqpoint{1.910652in}{0.492442in}}%
\pgfpathlineto{\pgfqpoint{1.872558in}{0.539895in}}%
\pgfpathlineto{\pgfqpoint{1.833974in}{0.587230in}}%
\pgfpathlineto{\pgfqpoint{1.794739in}{0.634405in}}%
\pgfpathlineto{\pgfqpoint{1.754648in}{0.681365in}}%
\pgfpathlineto{\pgfqpoint{1.713431in}{0.728032in}}%
\pgfpathlineto{\pgfqpoint{1.670722in}{0.774296in}}%
\pgfpathlineto{\pgfqpoint{1.626002in}{0.819987in}}%
\pgfpathlineto{\pgfqpoint{1.578511in}{0.864829in}}%
\pgfpathlineto{\pgfqpoint{1.527057in}{0.908328in}}%
\pgfpathlineto{\pgfqpoint{1.469654in}{0.949510in}}%
\pgfpathlineto{\pgfqpoint{1.402815in}{0.985972in}}%
\pgfpathlineto{\pgfqpoint{1.402815in}{0.985972in}}%
\pgfpathlineto{\pgfqpoint{1.342902in}{1.006691in}}%
\pgfpathlineto{\pgfqpoint{1.342902in}{1.006691in}}%
\pgfpathlineto{\pgfqpoint{1.289056in}{1.014723in}}%
\pgfpathlineto{\pgfqpoint{1.233243in}{1.011786in}}%
\pgfpathlineto{\pgfqpoint{1.186210in}{1.000342in}}%
\pgfpathlineto{\pgfqpoint{1.140488in}{0.981375in}}%
\pgfpathlineto{\pgfqpoint{1.092383in}{0.953275in}}%
\pgfpathlineto{\pgfqpoint{1.039858in}{0.913482in}}%
\pgfusepath{stroke}%
\end{pgfscope}%
\begin{pgfscope}%
\pgfpathrectangle{\pgfqpoint{0.647939in}{0.492442in}}{\pgfqpoint{4.273799in}{2.331163in}}%
\pgfusepath{clip}%
\pgfsetbuttcap%
\pgfsetroundjoin%
\pgfsetlinewidth{0.803000pt}%
\definecolor{currentstroke}{rgb}{0.501961,0.501961,0.501961}%
\pgfsetstrokecolor{currentstroke}%
\pgfsetdash{}{0pt}%
\pgfpathmoveto{\pgfqpoint{2.007784in}{0.492442in}}%
\pgfpathlineto{\pgfqpoint{2.007784in}{0.492442in}}%
\pgfpathlineto{\pgfqpoint{1.970946in}{0.540190in}}%
\pgfpathlineto{\pgfqpoint{1.933927in}{0.587896in}}%
\pgfpathlineto{\pgfqpoint{1.896632in}{0.635538in}}%
\pgfpathlineto{\pgfqpoint{1.858942in}{0.683087in}}%
\pgfpathlineto{\pgfqpoint{1.820704in}{0.730506in}}%
\pgfpathlineto{\pgfqpoint{1.781730in}{0.777745in}}%
\pgfpathlineto{\pgfqpoint{1.741778in}{0.824739in}}%
\pgfpathlineto{\pgfqpoint{1.700519in}{0.871395in}}%
\pgfpathlineto{\pgfqpoint{1.657497in}{0.917570in}}%
\pgfpathlineto{\pgfqpoint{1.612037in}{0.963038in}}%
\pgfpathlineto{\pgfqpoint{1.563093in}{1.007404in}}%
\pgfpathlineto{\pgfqpoint{1.508918in}{1.049897in}}%
\pgfpathlineto{\pgfqpoint{1.446350in}{1.088630in}}%
\pgfpathlineto{\pgfqpoint{1.446350in}{1.088630in}}%
\pgfpathlineto{\pgfqpoint{1.385101in}{1.114133in}}%
\pgfpathlineto{\pgfqpoint{1.385101in}{1.114133in}}%
\pgfpathlineto{\pgfqpoint{1.332084in}{1.125202in}}%
\pgfpathlineto{\pgfqpoint{1.274761in}{1.124750in}}%
\pgfpathlineto{\pgfqpoint{1.228289in}{1.114865in}}%
\pgfpathlineto{\pgfqpoint{1.184299in}{1.097688in}}%
\pgfpathlineto{\pgfqpoint{1.138197in}{1.071751in}}%
\pgfpathlineto{\pgfqpoint{1.087769in}{1.034552in}}%
\pgfpathlineto{\pgfqpoint{1.038826in}{0.990252in}}%
\pgfpathlineto{\pgfqpoint{0.995037in}{0.944316in}}%
\pgfusepath{stroke}%
\end{pgfscope}%
\begin{pgfscope}%
\pgfpathrectangle{\pgfqpoint{0.647939in}{0.492442in}}{\pgfqpoint{4.273799in}{2.331163in}}%
\pgfusepath{clip}%
\pgfsetbuttcap%
\pgfsetroundjoin%
\pgfsetlinewidth{0.803000pt}%
\definecolor{currentstroke}{rgb}{0.501961,0.501961,0.501961}%
\pgfsetstrokecolor{currentstroke}%
\pgfsetdash{}{0pt}%
\pgfpathmoveto{\pgfqpoint{2.104916in}{0.492442in}}%
\pgfpathlineto{\pgfqpoint{2.104916in}{0.492442in}}%
\pgfpathlineto{\pgfqpoint{2.068793in}{0.540352in}}%
\pgfpathlineto{\pgfqpoint{2.032706in}{0.588270in}}%
\pgfpathlineto{\pgfqpoint{1.996587in}{0.636182in}}%
\pgfpathlineto{\pgfqpoint{1.960361in}{0.684068in}}%
\pgfpathlineto{\pgfqpoint{1.923944in}{0.731912in}}%
\pgfpathlineto{\pgfqpoint{1.887233in}{0.779689in}}%
\pgfpathlineto{\pgfqpoint{1.850103in}{0.827368in}}%
\pgfpathlineto{\pgfqpoint{1.812394in}{0.874913in}}%
\pgfpathlineto{\pgfqpoint{1.773890in}{0.922267in}}%
\pgfpathlineto{\pgfqpoint{1.734304in}{0.969354in}}%
\pgfpathlineto{\pgfqpoint{1.693243in}{1.016060in}}%
\pgfpathlineto{\pgfqpoint{1.650141in}{1.062212in}}%
\pgfpathlineto{\pgfqpoint{1.604125in}{1.107510in}}%
\pgfpathlineto{\pgfqpoint{1.553748in}{1.151389in}}%
\pgfpathlineto{\pgfqpoint{1.496378in}{1.192536in}}%
\pgfpathlineto{\pgfqpoint{1.426980in}{1.227262in}}%
\pgfpathlineto{\pgfqpoint{1.426980in}{1.227262in}}%
\pgfpathlineto{\pgfqpoint{1.375447in}{1.241049in}}%
\pgfpathlineto{\pgfqpoint{1.375447in}{1.241049in}}%
\pgfpathlineto{\pgfqpoint{1.327290in}{1.243801in}}%
\pgfpathlineto{\pgfqpoint{1.280481in}{1.236882in}}%
\pgfpathlineto{\pgfqpoint{1.238428in}{1.222713in}}%
\pgfpathlineto{\pgfqpoint{1.195106in}{1.200476in}}%
\pgfpathlineto{\pgfqpoint{1.147925in}{1.167909in}}%
\pgfpathlineto{\pgfqpoint{1.097477in}{1.124204in}}%
\pgfpathlineto{\pgfqpoint{1.052813in}{1.078578in}}%
\pgfpathlineto{\pgfqpoint{1.012099in}{1.031833in}}%
\pgfusepath{stroke}%
\end{pgfscope}%
\begin{pgfscope}%
\pgfpathrectangle{\pgfqpoint{0.647939in}{0.492442in}}{\pgfqpoint{4.273799in}{2.331163in}}%
\pgfusepath{clip}%
\pgfsetbuttcap%
\pgfsetroundjoin%
\pgfsetlinewidth{0.803000pt}%
\definecolor{currentstroke}{rgb}{0.501961,0.501961,0.501961}%
\pgfsetstrokecolor{currentstroke}%
\pgfsetdash{}{0pt}%
\pgfpathmoveto{\pgfqpoint{2.396312in}{0.492442in}}%
\pgfpathlineto{\pgfqpoint{2.396312in}{0.492442in}}%
\pgfpathlineto{\pgfqpoint{2.359901in}{0.540287in}}%
\pgfpathlineto{\pgfqpoint{2.323937in}{0.588233in}}%
\pgfpathlineto{\pgfqpoint{2.288395in}{0.636272in}}%
\pgfpathlineto{\pgfqpoint{2.253246in}{0.684397in}}%
\pgfpathlineto{\pgfqpoint{2.218457in}{0.732600in}}%
\pgfpathlineto{\pgfqpoint{2.184000in}{0.780874in}}%
\pgfpathlineto{\pgfqpoint{2.149847in}{0.829212in}}%
\pgfpathlineto{\pgfqpoint{2.115967in}{0.877607in}}%
\pgfpathlineto{\pgfqpoint{2.082317in}{0.926050in}}%
\pgfpathlineto{\pgfqpoint{2.048848in}{0.974530in}}%
\pgfpathlineto{\pgfqpoint{2.015518in}{1.023039in}}%
\pgfpathlineto{\pgfqpoint{1.982275in}{1.071565in}}%
\pgfpathlineto{\pgfqpoint{1.949052in}{1.120096in}}%
\pgfpathlineto{\pgfqpoint{1.915760in}{1.168612in}}%
\pgfpathlineto{\pgfqpoint{1.882297in}{1.217093in}}%
\pgfpathlineto{\pgfqpoint{1.848541in}{1.265513in}}%
\pgfpathlineto{\pgfqpoint{1.814325in}{1.313837in}}%
\pgfpathlineto{\pgfqpoint{1.779400in}{1.362009in}}%
\pgfpathlineto{\pgfqpoint{1.743403in}{1.409942in}}%
\pgfpathlineto{\pgfqpoint{1.705800in}{1.457502in}}%
\pgfpathlineto{\pgfqpoint{1.665690in}{1.504435in}}%
\pgfpathlineto{\pgfqpoint{1.621373in}{1.550192in}}%
\pgfpathlineto{\pgfqpoint{1.569098in}{1.593209in}}%
\pgfpathlineto{\pgfqpoint{1.569098in}{1.593209in}}%
\pgfpathlineto{\pgfqpoint{1.521009in}{1.619591in}}%
\pgfpathlineto{\pgfqpoint{1.521009in}{1.619591in}}%
\pgfpathlineto{\pgfqpoint{1.481284in}{1.630245in}}%
\pgfpathlineto{\pgfqpoint{1.481284in}{1.630245in}}%
\pgfpathlineto{\pgfqpoint{1.443662in}{1.630444in}}%
\pgfpathlineto{\pgfqpoint{1.408726in}{1.622270in}}%
\pgfpathlineto{\pgfqpoint{1.375176in}{1.607384in}}%
\pgfpathlineto{\pgfqpoint{1.337908in}{1.583614in}}%
\pgfpathlineto{\pgfqpoint{1.294613in}{1.547778in}}%
\pgfpathlineto{\pgfqpoint{1.249214in}{1.502354in}}%
\pgfpathlineto{\pgfqpoint{1.208178in}{1.455673in}}%
\pgfpathlineto{\pgfqpoint{1.170024in}{1.408251in}}%
\pgfusepath{stroke}%
\end{pgfscope}%
\begin{pgfscope}%
\pgfpathrectangle{\pgfqpoint{0.647939in}{0.492442in}}{\pgfqpoint{4.273799in}{2.331163in}}%
\pgfusepath{clip}%
\pgfsetbuttcap%
\pgfsetroundjoin%
\pgfsetlinewidth{0.803000pt}%
\definecolor{currentstroke}{rgb}{0.501961,0.501961,0.501961}%
\pgfsetstrokecolor{currentstroke}%
\pgfsetdash{}{0pt}%
\pgfpathmoveto{\pgfqpoint{2.493443in}{0.492442in}}%
\pgfpathlineto{\pgfqpoint{2.493443in}{0.492442in}}%
\pgfpathlineto{\pgfqpoint{2.456287in}{0.540116in}}%
\pgfpathlineto{\pgfqpoint{2.419682in}{0.587917in}}%
\pgfpathlineto{\pgfqpoint{2.383603in}{0.635837in}}%
\pgfpathlineto{\pgfqpoint{2.348026in}{0.683868in}}%
\pgfpathlineto{\pgfqpoint{2.312929in}{0.732005in}}%
\pgfpathlineto{\pgfqpoint{2.278292in}{0.780240in}}%
\pgfpathlineto{\pgfqpoint{2.244094in}{0.828569in}}%
\pgfpathlineto{\pgfqpoint{2.210307in}{0.876983in}}%
\pgfpathlineto{\pgfqpoint{2.176901in}{0.925476in}}%
\pgfpathlineto{\pgfqpoint{2.143851in}{0.974042in}}%
\pgfpathlineto{\pgfqpoint{2.111136in}{1.022675in}}%
\pgfpathlineto{\pgfqpoint{2.078723in}{1.071368in}}%
\pgfpathlineto{\pgfqpoint{2.046571in}{1.120113in}}%
\pgfpathlineto{\pgfqpoint{2.014645in}{1.168901in}}%
\pgfpathlineto{\pgfqpoint{1.982909in}{1.217727in}}%
\pgfpathlineto{\pgfqpoint{1.951308in}{1.266579in}}%
\pgfpathlineto{\pgfqpoint{1.919771in}{1.315442in}}%
\pgfpathlineto{\pgfqpoint{1.888225in}{1.364304in}}%
\pgfpathlineto{\pgfqpoint{1.856582in}{1.413147in}}%
\pgfpathlineto{\pgfqpoint{1.824694in}{1.461943in}}%
\pgfpathlineto{\pgfqpoint{1.792357in}{1.510648in}}%
\pgfpathlineto{\pgfqpoint{1.759295in}{1.559206in}}%
\pgfpathlineto{\pgfqpoint{1.725067in}{1.607523in}}%
\pgfpathlineto{\pgfqpoint{1.688853in}{1.655402in}}%
\pgfpathlineto{\pgfqpoint{1.648999in}{1.702382in}}%
\pgfpathlineto{\pgfqpoint{1.601327in}{1.746990in}}%
\pgfpathlineto{\pgfqpoint{1.601327in}{1.746990in}}%
\pgfpathlineto{\pgfqpoint{1.562374in}{1.770725in}}%
\pgfpathlineto{\pgfqpoint{1.562374in}{1.770725in}}%
\pgfpathlineto{\pgfqpoint{1.529981in}{1.779802in}}%
\pgfpathlineto{\pgfqpoint{1.529981in}{1.779802in}}%
\pgfpathlineto{\pgfqpoint{1.498322in}{1.778852in}}%
\pgfpathlineto{\pgfqpoint{1.469396in}{1.770213in}}%
\pgfpathlineto{\pgfqpoint{1.440485in}{1.755050in}}%
\pgfpathlineto{\pgfqpoint{1.407176in}{1.730773in}}%
\pgfpathlineto{\pgfqpoint{1.364873in}{1.691896in}}%
\pgfpathlineto{\pgfqpoint{1.322068in}{1.645836in}}%
\pgfusepath{stroke}%
\end{pgfscope}%
\begin{pgfscope}%
\pgfpathrectangle{\pgfqpoint{0.647939in}{0.492442in}}{\pgfqpoint{4.273799in}{2.331163in}}%
\pgfusepath{clip}%
\pgfsetbuttcap%
\pgfsetroundjoin%
\pgfsetlinewidth{0.803000pt}%
\definecolor{currentstroke}{rgb}{0.501961,0.501961,0.501961}%
\pgfsetstrokecolor{currentstroke}%
\pgfsetdash{}{0pt}%
\pgfpathmoveto{\pgfqpoint{2.687707in}{0.492442in}}%
\pgfpathlineto{\pgfqpoint{2.609435in}{0.586841in}}%
\pgfpathlineto{\pgfqpoint{2.534070in}{0.681942in}}%
\pgfpathlineto{\pgfqpoint{2.461467in}{0.777680in}}%
\pgfpathlineto{\pgfqpoint{2.391477in}{0.873996in}}%
\pgfpathlineto{\pgfqpoint{2.323968in}{0.970837in}}%
\pgfpathlineto{\pgfqpoint{2.258841in}{1.068163in}}%
\pgfpathlineto{\pgfqpoint{2.195998in}{1.165934in}}%
\pgfpathlineto{\pgfqpoint{2.135397in}{1.264126in}}%
\pgfpathlineto{\pgfqpoint{2.077026in}{1.362718in}}%
\pgfpathlineto{\pgfqpoint{2.020945in}{1.461705in}}%
\pgfpathlineto{\pgfqpoint{1.967333in}{1.561098in}}%
\pgfpathlineto{\pgfqpoint{1.916573in}{1.660934in}}%
\pgfpathlineto{\pgfqpoint{1.869447in}{1.761296in}}%
\pgfpathlineto{\pgfqpoint{1.847733in}{1.811724in}}%
\pgfpathlineto{\pgfqpoint{1.827733in}{1.862360in}}%
\pgfpathlineto{\pgfqpoint{1.810100in}{1.913254in}}%
\pgfpathlineto{\pgfqpoint{1.795923in}{1.964459in}}%
\pgfpathlineto{\pgfqpoint{1.787087in}{2.016000in}}%
\pgfpathlineto{\pgfqpoint{1.786736in}{2.067717in}}%
\pgfpathlineto{\pgfqpoint{1.798839in}{2.118908in}}%
\pgfpathlineto{\pgfqpoint{1.825246in}{2.168404in}}%
\pgfpathlineto{\pgfqpoint{1.863944in}{2.215477in}}%
\pgfpathlineto{\pgfqpoint{1.911775in}{2.260049in}}%
\pgfpathlineto{\pgfqpoint{1.966587in}{2.302206in}}%
\pgfpathlineto{\pgfqpoint{2.027317in}{2.341863in}}%
\pgfpathlineto{\pgfqpoint{2.093896in}{2.378624in}}%
\pgfpathlineto{\pgfqpoint{2.166656in}{2.411703in}}%
\pgfpathlineto{\pgfqpoint{2.245985in}{2.439837in}}%
\pgfpathlineto{\pgfqpoint{2.331988in}{2.461156in}}%
\pgfpathlineto{\pgfqpoint{2.423553in}{2.473278in}}%
\pgfpathlineto{\pgfqpoint{2.511063in}{2.474499in}}%
\pgfpathlineto{\pgfqpoint{2.591972in}{2.466269in}}%
\pgfpathlineto{\pgfqpoint{2.668271in}{2.449739in}}%
\pgfpathlineto{\pgfqpoint{2.741419in}{2.425208in}}%
\pgfpathlineto{\pgfqpoint{2.812763in}{2.392177in}}%
\pgfpathlineto{\pgfqpoint{2.876451in}{2.353963in}}%
\pgfpathlineto{\pgfqpoint{2.932289in}{2.312195in}}%
\pgfpathlineto{\pgfqpoint{2.980790in}{2.267745in}}%
\pgfpathlineto{\pgfqpoint{3.022172in}{2.221184in}}%
\pgfpathlineto{\pgfqpoint{3.056229in}{2.172891in}}%
\pgfpathlineto{\pgfqpoint{3.082069in}{2.123136in}}%
\pgfpathlineto{\pgfqpoint{3.097371in}{2.072146in}}%
\pgfpathlineto{\pgfqpoint{3.095369in}{2.020765in}}%
\pgfpathlineto{\pgfqpoint{3.095369in}{2.020765in}}%
\pgfpathlineto{\pgfqpoint{3.081679in}{1.996880in}}%
\pgfpathlineto{\pgfqpoint{3.081679in}{1.996880in}}%
\pgfpathlineto{\pgfqpoint{3.060936in}{1.984563in}}%
\pgfpathlineto{\pgfqpoint{3.060936in}{1.984563in}}%
\pgfpathlineto{\pgfqpoint{3.037901in}{1.982713in}}%
\pgfpathlineto{\pgfqpoint{3.015806in}{1.988127in}}%
\pgfpathlineto{\pgfqpoint{2.993780in}{2.000018in}}%
\pgfpathlineto{\pgfqpoint{2.973530in}{2.019047in}}%
\pgfpathlineto{\pgfqpoint{2.961976in}{2.046572in}}%
\pgfpathlineto{\pgfqpoint{2.961976in}{2.046572in}}%
\pgfpathlineto{\pgfqpoint{2.966279in}{2.054479in}}%
\pgfpathlineto{\pgfqpoint{2.973423in}{2.054612in}}%
\pgfpathlineto{\pgfqpoint{2.982182in}{2.046299in}}%
\pgfpathlineto{\pgfqpoint{2.982182in}{2.046299in}}%
\pgfpathlineto{\pgfqpoint{2.978209in}{2.044857in}}%
\pgfpathlineto{\pgfqpoint{2.977935in}{2.048599in}}%
\pgfpathlineto{\pgfqpoint{2.981546in}{2.041276in}}%
\pgfpathlineto{\pgfqpoint{2.976635in}{2.051344in}}%
\pgfpathlineto{\pgfqpoint{2.981332in}{2.042148in}}%
\pgfusepath{stroke}%
\end{pgfscope}%
\begin{pgfscope}%
\pgfpathrectangle{\pgfqpoint{0.647939in}{0.492442in}}{\pgfqpoint{4.273799in}{2.331163in}}%
\pgfusepath{clip}%
\pgfsetbuttcap%
\pgfsetroundjoin%
\pgfsetlinewidth{0.803000pt}%
\definecolor{currentstroke}{rgb}{0.501961,0.501961,0.501961}%
\pgfsetstrokecolor{currentstroke}%
\pgfsetdash{}{0pt}%
\pgfpathmoveto{\pgfqpoint{2.784839in}{0.492442in}}%
\pgfpathlineto{\pgfqpoint{2.784839in}{0.492442in}}%
\pgfpathlineto{\pgfqpoint{2.743753in}{0.539146in}}%
\pgfpathlineto{\pgfqpoint{2.703508in}{0.586068in}}%
\pgfpathlineto{\pgfqpoint{2.664087in}{0.633197in}}%
\pgfpathlineto{\pgfqpoint{2.625473in}{0.680525in}}%
\pgfpathlineto{\pgfqpoint{2.587648in}{0.728042in}}%
\pgfpathlineto{\pgfqpoint{2.550594in}{0.775739in}}%
\pgfpathlineto{\pgfqpoint{2.514293in}{0.823609in}}%
\pgfpathlineto{\pgfqpoint{2.478727in}{0.871642in}}%
\pgfusepath{stroke}%
\end{pgfscope}%
\begin{pgfscope}%
\pgfpathrectangle{\pgfqpoint{0.647939in}{0.492442in}}{\pgfqpoint{4.273799in}{2.331163in}}%
\pgfusepath{clip}%
\pgfsetbuttcap%
\pgfsetroundjoin%
\pgfsetlinewidth{0.803000pt}%
\definecolor{currentstroke}{rgb}{0.501961,0.501961,0.501961}%
\pgfsetstrokecolor{currentstroke}%
\pgfsetdash{}{0pt}%
\pgfpathmoveto{\pgfqpoint{2.881971in}{0.492442in}}%
\pgfpathlineto{\pgfqpoint{2.881971in}{0.492442in}}%
\pgfpathlineto{\pgfqpoint{2.839060in}{0.538655in}}%
\pgfpathlineto{\pgfqpoint{2.797083in}{0.585123in}}%
\pgfpathlineto{\pgfqpoint{2.756025in}{0.631834in}}%
\pgfpathlineto{\pgfqpoint{2.715871in}{0.678779in}}%
\pgfpathlineto{\pgfqpoint{2.676604in}{0.725947in}}%
\pgfusepath{stroke}%
\end{pgfscope}%
\begin{pgfscope}%
\pgfpathrectangle{\pgfqpoint{0.647939in}{0.492442in}}{\pgfqpoint{4.273799in}{2.331163in}}%
\pgfusepath{clip}%
\pgfsetbuttcap%
\pgfsetroundjoin%
\pgfsetlinewidth{0.803000pt}%
\definecolor{currentstroke}{rgb}{0.501961,0.501961,0.501961}%
\pgfsetstrokecolor{currentstroke}%
\pgfsetdash{}{0pt}%
\pgfpathmoveto{\pgfqpoint{3.076234in}{0.492442in}}%
\pgfpathlineto{\pgfqpoint{3.076234in}{0.492442in}}%
\pgfpathlineto{\pgfqpoint{3.029037in}{0.537394in}}%
\pgfpathlineto{\pgfqpoint{2.982928in}{0.582681in}}%
\pgfpathlineto{\pgfqpoint{2.937910in}{0.628294in}}%
\pgfpathlineto{\pgfqpoint{2.893977in}{0.674220in}}%
\pgfpathlineto{\pgfqpoint{2.851121in}{0.720449in}}%
\pgfpathlineto{\pgfqpoint{2.809329in}{0.766966in}}%
\pgfpathlineto{\pgfqpoint{2.768586in}{0.813759in}}%
\pgfpathlineto{\pgfqpoint{2.728876in}{0.860816in}}%
\pgfpathlineto{\pgfqpoint{2.690185in}{0.908124in}}%
\pgfpathlineto{\pgfqpoint{2.652497in}{0.955673in}}%
\pgfpathlineto{\pgfqpoint{2.615800in}{1.003451in}}%
\pgfpathlineto{\pgfqpoint{2.580084in}{1.051451in}}%
\pgfpathlineto{\pgfqpoint{2.545344in}{1.099664in}}%
\pgfpathlineto{\pgfqpoint{2.511573in}{1.148081in}}%
\pgfpathlineto{\pgfqpoint{2.478768in}{1.196695in}}%
\pgfpathlineto{\pgfqpoint{2.446935in}{1.245501in}}%
\pgfpathlineto{\pgfqpoint{2.416091in}{1.294495in}}%
\pgfpathlineto{\pgfqpoint{2.386249in}{1.343674in}}%
\pgfpathlineto{\pgfqpoint{2.357433in}{1.393035in}}%
\pgfpathlineto{\pgfqpoint{2.329685in}{1.442577in}}%
\pgfpathlineto{\pgfqpoint{2.303053in}{1.492301in}}%
\pgfpathlineto{\pgfqpoint{2.277597in}{1.542208in}}%
\pgfpathlineto{\pgfqpoint{2.253406in}{1.592301in}}%
\pgfpathlineto{\pgfqpoint{2.230581in}{1.642584in}}%
\pgfpathlineto{\pgfqpoint{2.209263in}{1.693064in}}%
\pgfpathlineto{\pgfqpoint{2.189623in}{1.743745in}}%
\pgfpathlineto{\pgfqpoint{2.171884in}{1.794634in}}%
\pgfpathlineto{\pgfqpoint{2.156338in}{1.845734in}}%
\pgfpathlineto{\pgfqpoint{2.143349in}{1.897044in}}%
\pgfpathlineto{\pgfqpoint{2.133391in}{1.948553in}}%
\pgfpathlineto{\pgfqpoint{2.127081in}{2.000230in}}%
\pgfpathlineto{\pgfqpoint{2.125206in}{2.052004in}}%
\pgfpathlineto{\pgfqpoint{2.128755in}{2.103741in}}%
\pgfpathlineto{\pgfqpoint{2.138943in}{2.155199in}}%
\pgfpathlineto{\pgfqpoint{2.157180in}{2.205968in}}%
\pgfpathlineto{\pgfqpoint{2.185026in}{2.255393in}}%
\pgfpathlineto{\pgfqpoint{2.224062in}{2.302465in}}%
\pgfpathlineto{\pgfqpoint{2.275689in}{2.345686in}}%
\pgfpathlineto{\pgfqpoint{2.340867in}{2.382877in}}%
\pgfpathlineto{\pgfqpoint{2.416197in}{2.409985in}}%
\pgfpathlineto{\pgfqpoint{2.492189in}{2.424341in}}%
\pgfusepath{stroke}%
\end{pgfscope}%
\begin{pgfscope}%
\pgfpathrectangle{\pgfqpoint{0.647939in}{0.492442in}}{\pgfqpoint{4.273799in}{2.331163in}}%
\pgfusepath{clip}%
\pgfsetbuttcap%
\pgfsetroundjoin%
\pgfsetlinewidth{0.803000pt}%
\definecolor{currentstroke}{rgb}{0.501961,0.501961,0.501961}%
\pgfsetstrokecolor{currentstroke}%
\pgfsetdash{}{0pt}%
\pgfpathmoveto{\pgfqpoint{3.270498in}{0.492442in}}%
\pgfpathlineto{\pgfqpoint{3.270498in}{0.492442in}}%
\pgfpathlineto{\pgfqpoint{3.218519in}{0.535796in}}%
\pgfpathlineto{\pgfqpoint{3.167671in}{0.579546in}}%
\pgfpathlineto{\pgfqpoint{3.117997in}{0.623696in}}%
\pgfpathlineto{\pgfqpoint{3.069529in}{0.668244in}}%
\pgfpathlineto{\pgfqpoint{3.022286in}{0.713181in}}%
\pgfpathlineto{\pgfqpoint{2.976276in}{0.758498in}}%
\pgfpathlineto{\pgfqpoint{2.931498in}{0.804181in}}%
\pgfpathlineto{\pgfqpoint{2.887947in}{0.850214in}}%
\pgfpathlineto{\pgfqpoint{2.845613in}{0.896585in}}%
\pgfpathlineto{\pgfqpoint{2.804481in}{0.943276in}}%
\pgfpathlineto{\pgfqpoint{2.764541in}{0.990275in}}%
\pgfpathlineto{\pgfqpoint{2.725776in}{1.037565in}}%
\pgfpathlineto{\pgfqpoint{2.688174in}{1.085134in}}%
\pgfpathlineto{\pgfqpoint{2.651726in}{1.132969in}}%
\pgfpathlineto{\pgfqpoint{2.616428in}{1.181060in}}%
\pgfpathlineto{\pgfqpoint{2.582279in}{1.229397in}}%
\pgfpathlineto{\pgfqpoint{2.549287in}{1.277973in}}%
\pgfpathlineto{\pgfqpoint{2.517459in}{1.326780in}}%
\pgfpathlineto{\pgfqpoint{2.486812in}{1.375810in}}%
\pgfpathlineto{\pgfqpoint{2.457382in}{1.425063in}}%
\pgfpathlineto{\pgfqpoint{2.429209in}{1.474533in}}%
\pgfpathlineto{\pgfqpoint{2.402343in}{1.524219in}}%
\pgfpathlineto{\pgfqpoint{2.376863in}{1.574121in}}%
\pgfpathlineto{\pgfqpoint{2.352859in}{1.624241in}}%
\pgfpathlineto{\pgfqpoint{2.330450in}{1.674580in}}%
\pgfpathlineto{\pgfqpoint{2.309791in}{1.725140in}}%
\pgfpathlineto{\pgfqpoint{2.291073in}{1.775924in}}%
\pgfpathlineto{\pgfqpoint{2.274546in}{1.826932in}}%
\pgfpathlineto{\pgfqpoint{2.260520in}{1.878161in}}%
\pgfpathlineto{\pgfqpoint{2.249408in}{1.929602in}}%
\pgfpathlineto{\pgfqpoint{2.241731in}{1.981225in}}%
\pgfpathlineto{\pgfqpoint{2.238159in}{2.032975in}}%
\pgfpathlineto{\pgfqpoint{2.239564in}{2.084747in}}%
\pgfpathlineto{\pgfqpoint{2.247067in}{2.136351in}}%
\pgfpathlineto{\pgfqpoint{2.262118in}{2.187445in}}%
\pgfpathlineto{\pgfqpoint{2.286535in}{2.237414in}}%
\pgfpathlineto{\pgfqpoint{2.322547in}{2.285178in}}%
\pgfpathlineto{\pgfqpoint{2.372744in}{2.328832in}}%
\pgfpathlineto{\pgfqpoint{2.439242in}{2.364945in}}%
\pgfusepath{stroke}%
\end{pgfscope}%
\begin{pgfscope}%
\pgfpathrectangle{\pgfqpoint{0.647939in}{0.492442in}}{\pgfqpoint{4.273799in}{2.331163in}}%
\pgfusepath{clip}%
\pgfsetbuttcap%
\pgfsetroundjoin%
\pgfsetlinewidth{0.803000pt}%
\definecolor{currentstroke}{rgb}{0.501961,0.501961,0.501961}%
\pgfsetstrokecolor{currentstroke}%
\pgfsetdash{}{0pt}%
\pgfpathmoveto{\pgfqpoint{3.464761in}{0.492442in}}%
\pgfpathlineto{\pgfqpoint{3.464761in}{0.492442in}}%
\pgfpathlineto{\pgfqpoint{3.408254in}{0.534077in}}%
\pgfpathlineto{\pgfqpoint{3.352662in}{0.576076in}}%
\pgfpathlineto{\pgfqpoint{3.298124in}{0.618484in}}%
\pgfpathlineto{\pgfqpoint{3.244743in}{0.661327in}}%
\pgfpathlineto{\pgfqpoint{3.192604in}{0.704624in}}%
\pgfpathlineto{\pgfqpoint{3.141773in}{0.748380in}}%
\pgfpathlineto{\pgfqpoint{3.092292in}{0.792593in}}%
\pgfpathlineto{\pgfqpoint{3.044186in}{0.837256in}}%
\pgfpathlineto{\pgfqpoint{2.997467in}{0.882354in}}%
\pgfpathlineto{\pgfqpoint{2.952137in}{0.927872in}}%
\pgfpathlineto{\pgfqpoint{2.908192in}{0.973793in}}%
\pgfpathlineto{\pgfqpoint{2.865622in}{1.020098in}}%
\pgfpathlineto{\pgfqpoint{2.824417in}{1.066769in}}%
\pgfpathlineto{\pgfqpoint{2.784562in}{1.113787in}}%
\pgfpathlineto{\pgfqpoint{2.746049in}{1.161137in}}%
\pgfpathlineto{\pgfqpoint{2.708870in}{1.208804in}}%
\pgfpathlineto{\pgfqpoint{2.673022in}{1.256773in}}%
\pgfusepath{stroke}%
\end{pgfscope}%
\begin{pgfscope}%
\pgfpathrectangle{\pgfqpoint{0.647939in}{0.492442in}}{\pgfqpoint{4.273799in}{2.331163in}}%
\pgfusepath{clip}%
\pgfsetbuttcap%
\pgfsetroundjoin%
\pgfsetlinewidth{0.803000pt}%
\definecolor{currentstroke}{rgb}{0.501961,0.501961,0.501961}%
\pgfsetstrokecolor{currentstroke}%
\pgfsetdash{}{0pt}%
\pgfpathmoveto{\pgfqpoint{3.659025in}{0.492442in}}%
\pgfpathlineto{\pgfqpoint{3.659025in}{0.492442in}}%
\pgfpathlineto{\pgfqpoint{3.599386in}{0.532758in}}%
\pgfpathlineto{\pgfqpoint{3.540042in}{0.573203in}}%
\pgfpathlineto{\pgfqpoint{3.481252in}{0.613887in}}%
\pgfpathlineto{\pgfqpoint{3.423252in}{0.654906in}}%
\pgfpathlineto{\pgfqpoint{3.366232in}{0.696331in}}%
\pgfpathlineto{\pgfqpoint{3.310362in}{0.738220in}}%
\pgfpathlineto{\pgfqpoint{3.255781in}{0.780610in}}%
\pgfpathlineto{\pgfqpoint{3.202587in}{0.823521in}}%
\pgfpathlineto{\pgfqpoint{3.150853in}{0.866960in}}%
\pgfpathlineto{\pgfqpoint{3.100630in}{0.910924in}}%
\pgfpathlineto{\pgfqpoint{3.051954in}{0.955402in}}%
\pgfpathlineto{\pgfqpoint{3.004835in}{1.000375in}}%
\pgfpathlineto{\pgfqpoint{2.959277in}{1.045825in}}%
\pgfpathlineto{\pgfqpoint{2.915274in}{1.091728in}}%
\pgfpathlineto{\pgfqpoint{2.872818in}{1.138063in}}%
\pgfusepath{stroke}%
\end{pgfscope}%
\begin{pgfscope}%
\pgfpathrectangle{\pgfqpoint{0.647939in}{0.492442in}}{\pgfqpoint{4.273799in}{2.331163in}}%
\pgfusepath{clip}%
\pgfsetbuttcap%
\pgfsetroundjoin%
\pgfsetlinewidth{0.803000pt}%
\definecolor{currentstroke}{rgb}{0.501961,0.501961,0.501961}%
\pgfsetstrokecolor{currentstroke}%
\pgfsetdash{}{0pt}%
\pgfpathmoveto{\pgfqpoint{3.853289in}{0.492442in}}%
\pgfpathlineto{\pgfqpoint{3.853289in}{0.492442in}}%
\pgfpathlineto{\pgfqpoint{3.793173in}{0.532546in}}%
\pgfpathlineto{\pgfqpoint{3.732360in}{0.572336in}}%
\pgfpathlineto{\pgfqpoint{3.671162in}{0.611951in}}%
\pgfpathlineto{\pgfqpoint{3.609910in}{0.651540in}}%
\pgfpathlineto{\pgfqpoint{3.548934in}{0.691256in}}%
\pgfpathlineto{\pgfqpoint{3.488532in}{0.731231in}}%
\pgfpathlineto{\pgfqpoint{3.428985in}{0.771585in}}%
\pgfpathlineto{\pgfqpoint{3.370528in}{0.812409in}}%
\pgfpathlineto{\pgfqpoint{3.313362in}{0.853773in}}%
\pgfpathlineto{\pgfqpoint{3.257651in}{0.895721in}}%
\pgfpathlineto{\pgfqpoint{3.203508in}{0.938276in}}%
\pgfpathlineto{\pgfqpoint{3.151015in}{0.981443in}}%
\pgfpathlineto{\pgfqpoint{3.100231in}{1.025214in}}%
\pgfpathlineto{\pgfqpoint{3.051191in}{1.069571in}}%
\pgfpathlineto{\pgfqpoint{3.003905in}{1.114491in}}%
\pgfpathlineto{\pgfqpoint{2.958375in}{1.159947in}}%
\pgfpathlineto{\pgfqpoint{2.914594in}{1.205912in}}%
\pgfpathlineto{\pgfqpoint{2.872553in}{1.252359in}}%
\pgfpathlineto{\pgfqpoint{2.832243in}{1.299259in}}%
\pgfpathlineto{\pgfqpoint{2.793658in}{1.346590in}}%
\pgfpathlineto{\pgfqpoint{2.756801in}{1.394329in}}%
\pgfpathlineto{\pgfqpoint{2.721684in}{1.442458in}}%
\pgfpathlineto{\pgfqpoint{2.688327in}{1.490958in}}%
\pgfpathlineto{\pgfqpoint{2.656767in}{1.539814in}}%
\pgfpathlineto{\pgfqpoint{2.627067in}{1.589015in}}%
\pgfpathlineto{\pgfqpoint{2.599308in}{1.638552in}}%
\pgfpathlineto{\pgfqpoint{2.573601in}{1.688417in}}%
\pgfpathlineto{\pgfqpoint{2.550099in}{1.738605in}}%
\pgfpathlineto{\pgfqpoint{2.528999in}{1.789108in}}%
\pgfpathlineto{\pgfqpoint{2.510563in}{1.839920in}}%
\pgfpathlineto{\pgfqpoint{2.495141in}{1.891026in}}%
\pgfpathlineto{\pgfqpoint{2.483197in}{1.942406in}}%
\pgfpathlineto{\pgfqpoint{2.475363in}{1.994016in}}%
\pgfpathlineto{\pgfqpoint{2.472516in}{2.045773in}}%
\pgfpathlineto{\pgfqpoint{2.475906in}{2.097506in}}%
\pgfpathlineto{\pgfqpoint{2.487393in}{2.148856in}}%
\pgfpathlineto{\pgfqpoint{2.509856in}{2.199040in}}%
\pgfpathlineto{\pgfqpoint{2.547993in}{2.246099in}}%
\pgfpathlineto{\pgfqpoint{2.547993in}{2.246099in}}%
\pgfpathlineto{\pgfqpoint{2.590183in}{2.276245in}}%
\pgfpathlineto{\pgfqpoint{2.590183in}{2.276245in}}%
\pgfpathlineto{\pgfqpoint{2.636614in}{2.294983in}}%
\pgfpathlineto{\pgfqpoint{2.692158in}{2.303989in}}%
\pgfpathlineto{\pgfqpoint{2.743847in}{2.302368in}}%
\pgfpathlineto{\pgfqpoint{2.793606in}{2.292844in}}%
\pgfpathlineto{\pgfqpoint{2.843822in}{2.275583in}}%
\pgfpathlineto{\pgfqpoint{2.894681in}{2.249859in}}%
\pgfusepath{stroke}%
\end{pgfscope}%
\begin{pgfscope}%
\pgfpathrectangle{\pgfqpoint{0.647939in}{0.492442in}}{\pgfqpoint{4.273799in}{2.331163in}}%
\pgfusepath{clip}%
\pgfsetbuttcap%
\pgfsetroundjoin%
\pgfsetlinewidth{0.803000pt}%
\definecolor{currentstroke}{rgb}{0.501961,0.501961,0.501961}%
\pgfsetstrokecolor{currentstroke}%
\pgfsetdash{}{0pt}%
\pgfpathmoveto{\pgfqpoint{4.047552in}{0.492442in}}%
\pgfpathlineto{\pgfqpoint{4.047552in}{0.492442in}}%
\pgfpathlineto{\pgfqpoint{3.990596in}{0.533890in}}%
\pgfpathlineto{\pgfqpoint{3.931874in}{0.574600in}}%
\pgfpathlineto{\pgfqpoint{3.871597in}{0.614629in}}%
\pgfpathlineto{\pgfqpoint{3.810053in}{0.654082in}}%
\pgfpathlineto{\pgfqpoint{3.747564in}{0.693091in}}%
\pgfpathlineto{\pgfqpoint{3.684510in}{0.731830in}}%
\pgfpathlineto{\pgfqpoint{3.621282in}{0.770485in}}%
\pgfpathlineto{\pgfqpoint{3.558280in}{0.809247in}}%
\pgfpathlineto{\pgfqpoint{3.495874in}{0.848296in}}%
\pgfpathlineto{\pgfqpoint{3.434409in}{0.887784in}}%
\pgfpathlineto{\pgfqpoint{3.374184in}{0.927834in}}%
\pgfpathlineto{\pgfqpoint{3.315432in}{0.968530in}}%
\pgfpathlineto{\pgfqpoint{3.258350in}{1.009925in}}%
\pgfpathlineto{\pgfqpoint{3.203073in}{1.052041in}}%
\pgfpathlineto{\pgfqpoint{3.149692in}{1.094881in}}%
\pgfpathlineto{\pgfqpoint{3.098271in}{1.138428in}}%
\pgfpathlineto{\pgfqpoint{3.048844in}{1.182654in}}%
\pgfpathlineto{\pgfqpoint{3.001418in}{1.227528in}}%
\pgfpathlineto{\pgfqpoint{2.955992in}{1.273014in}}%
\pgfpathlineto{\pgfqpoint{2.912557in}{1.319075in}}%
\pgfpathlineto{\pgfqpoint{2.871104in}{1.365676in}}%
\pgfusepath{stroke}%
\end{pgfscope}%
\begin{pgfscope}%
\pgfpathrectangle{\pgfqpoint{0.647939in}{0.492442in}}{\pgfqpoint{4.273799in}{2.331163in}}%
\pgfusepath{clip}%
\pgfsetbuttcap%
\pgfsetroundjoin%
\pgfsetlinewidth{0.803000pt}%
\definecolor{currentstroke}{rgb}{0.501961,0.501961,0.501961}%
\pgfsetstrokecolor{currentstroke}%
\pgfsetdash{}{0pt}%
\pgfpathmoveto{\pgfqpoint{4.241816in}{0.492442in}}%
\pgfpathlineto{\pgfqpoint{4.241816in}{0.492442in}}%
\pgfpathlineto{\pgfqpoint{4.191622in}{0.536412in}}%
\pgfpathlineto{\pgfqpoint{4.139153in}{0.579584in}}%
\pgfpathlineto{\pgfqpoint{4.084372in}{0.621893in}}%
\pgfpathlineto{\pgfqpoint{4.027282in}{0.663283in}}%
\pgfpathlineto{\pgfqpoint{3.967949in}{0.703724in}}%
\pgfpathlineto{\pgfqpoint{3.906567in}{0.743248in}}%
\pgfpathlineto{\pgfqpoint{3.843410in}{0.781932in}}%
\pgfpathlineto{\pgfqpoint{3.778849in}{0.819923in}}%
\pgfpathlineto{\pgfqpoint{3.713335in}{0.857426in}}%
\pgfpathlineto{\pgfqpoint{3.647359in}{0.894689in}}%
\pgfpathlineto{\pgfqpoint{3.581435in}{0.931979in}}%
\pgfpathlineto{\pgfqpoint{3.516053in}{0.969549in}}%
\pgfpathlineto{\pgfqpoint{3.451667in}{1.007626in}}%
\pgfpathlineto{\pgfqpoint{3.388666in}{1.046383in}}%
\pgfpathlineto{\pgfqpoint{3.327374in}{1.085946in}}%
\pgfpathlineto{\pgfqpoint{3.268034in}{1.126382in}}%
\pgfpathlineto{\pgfqpoint{3.210832in}{1.167723in}}%
\pgfpathlineto{\pgfqpoint{3.155875in}{1.209961in}}%
\pgfpathlineto{\pgfqpoint{3.103231in}{1.253068in}}%
\pgfpathlineto{\pgfqpoint{3.052937in}{1.297001in}}%
\pgfpathlineto{\pgfqpoint{3.004999in}{1.341709in}}%
\pgfpathlineto{\pgfqpoint{2.959409in}{1.387143in}}%
\pgfpathlineto{\pgfqpoint{2.916155in}{1.433251in}}%
\pgfpathlineto{\pgfqpoint{2.875229in}{1.479989in}}%
\pgfpathlineto{\pgfqpoint{2.836636in}{1.527313in}}%
\pgfpathlineto{\pgfqpoint{2.800396in}{1.575190in}}%
\pgfpathlineto{\pgfqpoint{2.766554in}{1.623587in}}%
\pgfpathlineto{\pgfqpoint{2.735184in}{1.672475in}}%
\pgfpathlineto{\pgfqpoint{2.706401in}{1.721834in}}%
\pgfpathlineto{\pgfqpoint{2.680378in}{1.771647in}}%
\pgfpathlineto{\pgfqpoint{2.657353in}{1.821895in}}%
\pgfpathlineto{\pgfqpoint{2.637663in}{1.872561in}}%
\pgfpathlineto{\pgfqpoint{2.621784in}{1.923621in}}%
\pgfpathlineto{\pgfqpoint{2.610397in}{1.975031in}}%
\pgfpathlineto{\pgfqpoint{2.604520in}{2.026706in}}%
\pgfpathlineto{\pgfqpoint{2.605729in}{2.078451in}}%
\pgfpathlineto{\pgfqpoint{2.616671in}{2.129794in}}%
\pgfpathlineto{\pgfqpoint{2.642227in}{2.179394in}}%
\pgfpathlineto{\pgfqpoint{2.642227in}{2.179394in}}%
\pgfpathlineto{\pgfqpoint{2.675023in}{2.212061in}}%
\pgfpathlineto{\pgfqpoint{2.675023in}{2.212061in}}%
\pgfpathlineto{\pgfqpoint{2.712764in}{2.232172in}}%
\pgfpathlineto{\pgfqpoint{2.712764in}{2.232172in}}%
\pgfpathlineto{\pgfqpoint{2.753663in}{2.241728in}}%
\pgfpathlineto{\pgfqpoint{2.798240in}{2.241880in}}%
\pgfusepath{stroke}%
\end{pgfscope}%
\begin{pgfscope}%
\pgfpathrectangle{\pgfqpoint{0.647939in}{0.492442in}}{\pgfqpoint{4.273799in}{2.331163in}}%
\pgfusepath{clip}%
\pgfsetbuttcap%
\pgfsetroundjoin%
\pgfsetlinewidth{0.803000pt}%
\definecolor{currentstroke}{rgb}{0.501961,0.501961,0.501961}%
\pgfsetstrokecolor{currentstroke}%
\pgfsetdash{}{0pt}%
\pgfpathmoveto{\pgfqpoint{4.338948in}{0.492442in}}%
\pgfpathlineto{\pgfqpoint{4.338948in}{0.492442in}}%
\pgfpathlineto{\pgfqpoint{4.293101in}{0.537801in}}%
\pgfpathlineto{\pgfqpoint{4.245074in}{0.582484in}}%
\pgfpathlineto{\pgfqpoint{4.194705in}{0.626392in}}%
\pgfpathlineto{\pgfqpoint{4.141852in}{0.669423in}}%
\pgfpathlineto{\pgfqpoint{4.086443in}{0.711486in}}%
\pgfpathlineto{\pgfqpoint{4.028461in}{0.752502in}}%
\pgfpathlineto{\pgfqpoint{3.967952in}{0.792420in}}%
\pgfpathlineto{\pgfqpoint{3.905121in}{0.831257in}}%
\pgfpathlineto{\pgfqpoint{3.840271in}{0.869097in}}%
\pgfpathlineto{\pgfqpoint{3.773839in}{0.906115in}}%
\pgfusepath{stroke}%
\end{pgfscope}%
\begin{pgfscope}%
\pgfpathrectangle{\pgfqpoint{0.647939in}{0.492442in}}{\pgfqpoint{4.273799in}{2.331163in}}%
\pgfusepath{clip}%
\pgfsetbuttcap%
\pgfsetroundjoin%
\pgfsetlinewidth{0.803000pt}%
\definecolor{currentstroke}{rgb}{0.501961,0.501961,0.501961}%
\pgfsetstrokecolor{currentstroke}%
\pgfsetdash{}{0pt}%
\pgfpathmoveto{\pgfqpoint{4.533211in}{0.492442in}}%
\pgfpathlineto{\pgfqpoint{4.533211in}{0.492442in}}%
\pgfpathlineto{\pgfqpoint{4.496553in}{0.540228in}}%
\pgfpathlineto{\pgfqpoint{4.458385in}{0.587660in}}%
\pgfpathlineto{\pgfqpoint{4.418527in}{0.634674in}}%
\pgfpathlineto{\pgfqpoint{4.376775in}{0.681195in}}%
\pgfpathlineto{\pgfqpoint{4.332900in}{0.727129in}}%
\pgfpathlineto{\pgfqpoint{4.286645in}{0.772362in}}%
\pgfpathlineto{\pgfqpoint{4.237727in}{0.816752in}}%
\pgfpathlineto{\pgfqpoint{4.185847in}{0.860130in}}%
\pgfpathlineto{\pgfqpoint{4.130757in}{0.902311in}}%
\pgfpathlineto{\pgfqpoint{4.072248in}{0.943093in}}%
\pgfpathlineto{\pgfqpoint{4.010181in}{0.982279in}}%
\pgfpathlineto{\pgfqpoint{3.944645in}{1.019749in}}%
\pgfpathlineto{\pgfqpoint{3.875945in}{1.055495in}}%
\pgfpathlineto{\pgfqpoint{3.804620in}{1.089686in}}%
\pgfpathlineto{\pgfqpoint{3.731443in}{1.122700in}}%
\pgfpathlineto{\pgfqpoint{3.657306in}{1.155075in}}%
\pgfpathlineto{\pgfqpoint{3.583111in}{1.187410in}}%
\pgfpathlineto{\pgfqpoint{3.509765in}{1.220309in}}%
\pgfpathlineto{\pgfqpoint{3.438075in}{1.254266in}}%
\pgfpathlineto{\pgfqpoint{3.368659in}{1.289597in}}%
\pgfpathlineto{\pgfqpoint{3.302032in}{1.326485in}}%
\pgfpathlineto{\pgfqpoint{3.238531in}{1.364977in}}%
\pgfpathlineto{\pgfqpoint{3.178352in}{1.405025in}}%
\pgfpathlineto{\pgfqpoint{3.121598in}{1.446531in}}%
\pgfpathlineto{\pgfqpoint{3.068270in}{1.489374in}}%
\pgfpathlineto{\pgfqpoint{3.018347in}{1.533424in}}%
\pgfpathlineto{\pgfqpoint{2.971798in}{1.578560in}}%
\pgfpathlineto{\pgfqpoint{2.928593in}{1.624675in}}%
\pgfpathlineto{\pgfqpoint{2.888730in}{1.671679in}}%
\pgfpathlineto{\pgfqpoint{2.852254in}{1.719497in}}%
\pgfpathlineto{\pgfqpoint{2.819273in}{1.768061in}}%
\pgfpathlineto{\pgfqpoint{2.789983in}{1.817322in}}%
\pgfpathlineto{\pgfqpoint{2.764715in}{1.867239in}}%
\pgfpathlineto{\pgfqpoint{2.743979in}{1.917770in}}%
\pgfpathlineto{\pgfqpoint{2.728592in}{1.968857in}}%
\pgfpathlineto{\pgfqpoint{2.719906in}{2.020397in}}%
\pgfpathlineto{\pgfqpoint{2.720349in}{2.072111in}}%
\pgfpathlineto{\pgfqpoint{2.734932in}{2.123032in}}%
\pgfpathlineto{\pgfqpoint{2.734932in}{2.123032in}}%
\pgfpathlineto{\pgfqpoint{2.759082in}{2.156003in}}%
\pgfpathlineto{\pgfqpoint{2.759082in}{2.156003in}}%
\pgfpathlineto{\pgfqpoint{2.788761in}{2.175423in}}%
\pgfpathlineto{\pgfqpoint{2.788761in}{2.175423in}}%
\pgfpathlineto{\pgfqpoint{2.822384in}{2.184180in}}%
\pgfpathlineto{\pgfqpoint{2.859537in}{2.183369in}}%
\pgfpathlineto{\pgfqpoint{2.893462in}{2.175085in}}%
\pgfpathlineto{\pgfqpoint{2.927764in}{2.159681in}}%
\pgfpathlineto{\pgfqpoint{2.961882in}{2.136355in}}%
\pgfpathlineto{\pgfqpoint{2.993713in}{2.103655in}}%
\pgfpathlineto{\pgfqpoint{3.015212in}{2.063164in}}%
\pgfpathlineto{\pgfqpoint{3.015212in}{2.063164in}}%
\pgfusepath{stroke}%
\end{pgfscope}%
\begin{pgfscope}%
\pgfpathrectangle{\pgfqpoint{0.647939in}{0.492442in}}{\pgfqpoint{4.273799in}{2.331163in}}%
\pgfusepath{clip}%
\pgfsetbuttcap%
\pgfsetroundjoin%
\pgfsetlinewidth{0.803000pt}%
\definecolor{currentstroke}{rgb}{0.501961,0.501961,0.501961}%
\pgfsetstrokecolor{currentstroke}%
\pgfsetdash{}{0pt}%
\pgfpathmoveto{\pgfqpoint{4.630343in}{0.492442in}}%
\pgfpathlineto{\pgfqpoint{4.630343in}{0.492442in}}%
\pgfpathlineto{\pgfqpoint{4.598071in}{0.541160in}}%
\pgfpathlineto{\pgfqpoint{4.564681in}{0.589652in}}%
\pgfpathlineto{\pgfqpoint{4.530036in}{0.637883in}}%
\pgfpathlineto{\pgfqpoint{4.493988in}{0.685807in}}%
\pgfpathlineto{\pgfqpoint{4.456370in}{0.733369in}}%
\pgfpathlineto{\pgfqpoint{4.416971in}{0.780497in}}%
\pgfpathlineto{\pgfqpoint{4.375542in}{0.827103in}}%
\pgfpathlineto{\pgfqpoint{4.331785in}{0.873068in}}%
\pgfpathlineto{\pgfqpoint{4.285354in}{0.918244in}}%
\pgfpathlineto{\pgfqpoint{4.235846in}{0.962435in}}%
\pgfpathlineto{\pgfqpoint{4.182821in}{1.005393in}}%
\pgfpathlineto{\pgfqpoint{4.125869in}{1.046820in}}%
\pgfpathlineto{\pgfqpoint{4.064604in}{1.086363in}}%
\pgfpathlineto{\pgfqpoint{3.998803in}{1.123673in}}%
\pgfpathlineto{\pgfqpoint{3.928569in}{1.158495in}}%
\pgfpathlineto{\pgfqpoint{3.854415in}{1.190821in}}%
\pgfpathlineto{\pgfqpoint{3.777284in}{1.221022in}}%
\pgfpathlineto{\pgfqpoint{3.698326in}{1.249800in}}%
\pgfpathlineto{\pgfqpoint{3.618790in}{1.278102in}}%
\pgfpathlineto{\pgfqpoint{3.539878in}{1.306912in}}%
\pgfpathlineto{\pgfqpoint{3.462694in}{1.337067in}}%
\pgfpathlineto{\pgfqpoint{3.388179in}{1.369133in}}%
\pgfpathlineto{\pgfqpoint{3.317005in}{1.403372in}}%
\pgfpathlineto{\pgfqpoint{3.249627in}{1.439826in}}%
\pgfpathlineto{\pgfqpoint{3.186317in}{1.478388in}}%
\pgfpathlineto{\pgfqpoint{3.127160in}{1.518867in}}%
\pgfpathlineto{\pgfqpoint{3.072142in}{1.561052in}}%
\pgfusepath{stroke}%
\end{pgfscope}%
\begin{pgfscope}%
\pgfpathrectangle{\pgfqpoint{0.647939in}{0.492442in}}{\pgfqpoint{4.273799in}{2.331163in}}%
\pgfusepath{clip}%
\pgfsetbuttcap%
\pgfsetroundjoin%
\pgfsetlinewidth{0.803000pt}%
\definecolor{currentstroke}{rgb}{0.501961,0.501961,0.501961}%
\pgfsetstrokecolor{currentstroke}%
\pgfsetdash{}{0pt}%
\pgfpathmoveto{\pgfqpoint{4.727475in}{0.492442in}}%
\pgfpathlineto{\pgfqpoint{4.727475in}{0.492442in}}%
\pgfpathlineto{\pgfqpoint{4.699209in}{0.541896in}}%
\pgfpathlineto{\pgfqpoint{4.670177in}{0.591218in}}%
\pgfpathlineto{\pgfqpoint{4.640309in}{0.640391in}}%
\pgfpathlineto{\pgfqpoint{4.609509in}{0.689391in}}%
\pgfpathlineto{\pgfqpoint{4.577660in}{0.738193in}}%
\pgfpathlineto{\pgfqpoint{4.544640in}{0.786761in}}%
\pgfpathlineto{\pgfqpoint{4.510302in}{0.835055in}}%
\pgfpathlineto{\pgfqpoint{4.474447in}{0.883018in}}%
\pgfpathlineto{\pgfqpoint{4.436839in}{0.930579in}}%
\pgfpathlineto{\pgfqpoint{4.397184in}{0.977643in}}%
\pgfpathlineto{\pgfqpoint{4.355123in}{1.024080in}}%
\pgfpathlineto{\pgfqpoint{4.310212in}{1.069712in}}%
\pgfpathlineto{\pgfqpoint{4.261899in}{1.114290in}}%
\pgfpathlineto{\pgfqpoint{4.209507in}{1.157462in}}%
\pgfpathlineto{\pgfqpoint{4.152228in}{1.198731in}}%
\pgfpathlineto{\pgfqpoint{4.089234in}{1.237426in}}%
\pgfpathlineto{\pgfqpoint{4.019950in}{1.272750in}}%
\pgfpathlineto{\pgfqpoint{3.944362in}{1.303979in}}%
\pgfpathlineto{\pgfqpoint{3.863384in}{1.330910in}}%
\pgfpathlineto{\pgfqpoint{3.778642in}{1.354210in}}%
\pgfpathlineto{\pgfqpoint{3.692011in}{1.375403in}}%
\pgfpathlineto{\pgfqpoint{3.605207in}{1.396380in}}%
\pgfpathlineto{\pgfqpoint{3.519733in}{1.418872in}}%
\pgfpathlineto{\pgfqpoint{3.436929in}{1.444108in}}%
\pgfpathlineto{\pgfqpoint{3.357925in}{1.472712in}}%
\pgfpathlineto{\pgfqpoint{3.283519in}{1.504797in}}%
\pgfpathlineto{\pgfqpoint{3.214238in}{1.540130in}}%
\pgfpathlineto{\pgfqpoint{3.150235in}{1.578316in}}%
\pgfpathlineto{\pgfqpoint{3.091444in}{1.618936in}}%
\pgfpathlineto{\pgfqpoint{3.037741in}{1.661599in}}%
\pgfpathlineto{\pgfqpoint{2.988952in}{1.705995in}}%
\pgfpathlineto{\pgfqpoint{2.944974in}{1.751870in}}%
\pgfpathlineto{\pgfqpoint{2.905807in}{1.799026in}}%
\pgfpathlineto{\pgfqpoint{2.871595in}{1.847312in}}%
\pgfpathlineto{\pgfqpoint{2.842702in}{1.896621in}}%
\pgfusepath{stroke}%
\end{pgfscope}%
\begin{pgfscope}%
\pgfpathrectangle{\pgfqpoint{0.647939in}{0.492442in}}{\pgfqpoint{4.273799in}{2.331163in}}%
\pgfusepath{clip}%
\pgfsetbuttcap%
\pgfsetroundjoin%
\pgfsetlinewidth{0.803000pt}%
\definecolor{currentstroke}{rgb}{0.501961,0.501961,0.501961}%
\pgfsetstrokecolor{currentstroke}%
\pgfsetdash{}{0pt}%
\pgfpathmoveto{\pgfqpoint{4.824607in}{0.492442in}}%
\pgfpathlineto{\pgfqpoint{4.824607in}{0.492442in}}%
\pgfpathlineto{\pgfqpoint{4.799916in}{0.542463in}}%
\pgfpathlineto{\pgfqpoint{4.774753in}{0.592414in}}%
\pgfpathlineto{\pgfqpoint{4.749071in}{0.642286in}}%
\pgfpathlineto{\pgfqpoint{4.722829in}{0.692072in}}%
\pgfpathlineto{\pgfqpoint{4.695978in}{0.741759in}}%
\pgfpathlineto{\pgfqpoint{4.668446in}{0.791337in}}%
\pgfpathlineto{\pgfqpoint{4.640166in}{0.840788in}}%
\pgfpathlineto{\pgfqpoint{4.611048in}{0.890093in}}%
\pgfpathlineto{\pgfqpoint{4.580975in}{0.939229in}}%
\pgfpathlineto{\pgfqpoint{4.549826in}{0.988163in}}%
\pgfpathlineto{\pgfqpoint{4.517438in}{1.036855in}}%
\pgfpathlineto{\pgfqpoint{4.483590in}{1.085251in}}%
\pgfpathlineto{\pgfqpoint{4.448007in}{1.133276in}}%
\pgfpathlineto{\pgfqpoint{4.410348in}{1.180825in}}%
\pgfpathlineto{\pgfqpoint{4.370148in}{1.227745in}}%
\pgfpathlineto{\pgfqpoint{4.326752in}{1.273802in}}%
\pgfpathlineto{\pgfqpoint{4.279239in}{1.318623in}}%
\pgfpathlineto{\pgfqpoint{4.226306in}{1.361584in}}%
\pgfpathlineto{\pgfqpoint{4.166290in}{1.401611in}}%
\pgfpathlineto{\pgfqpoint{4.097131in}{1.436870in}}%
\pgfpathlineto{\pgfqpoint{4.017719in}{1.464776in}}%
\pgfpathlineto{\pgfqpoint{3.929681in}{1.483412in}}%
\pgfpathlineto{\pgfqpoint{3.838668in}{1.493752in}}%
\pgfpathlineto{\pgfqpoint{3.744448in}{1.499912in}}%
\pgfpathlineto{\pgfqpoint{3.650118in}{1.505735in}}%
\pgfpathlineto{\pgfqpoint{3.556673in}{1.514508in}}%
\pgfpathlineto{\pgfqpoint{3.465484in}{1.528420in}}%
\pgfpathlineto{\pgfqpoint{3.378272in}{1.548479in}}%
\pgfpathlineto{\pgfqpoint{3.296661in}{1.574611in}}%
\pgfpathlineto{\pgfqpoint{3.221588in}{1.606068in}}%
\pgfpathlineto{\pgfqpoint{3.153285in}{1.641892in}}%
\pgfusepath{stroke}%
\end{pgfscope}%
\begin{pgfscope}%
\pgfpathrectangle{\pgfqpoint{0.647939in}{0.492442in}}{\pgfqpoint{4.273799in}{2.331163in}}%
\pgfusepath{clip}%
\pgfsetbuttcap%
\pgfsetroundjoin%
\pgfsetlinewidth{0.803000pt}%
\definecolor{currentstroke}{rgb}{0.501961,0.501961,0.501961}%
\pgfsetstrokecolor{currentstroke}%
\pgfsetdash{}{0pt}%
\pgfpathmoveto{\pgfqpoint{4.921738in}{0.492442in}}%
\pgfpathlineto{\pgfqpoint{4.921738in}{0.492442in}}%
\pgfpathlineto{\pgfqpoint{4.900182in}{0.542893in}}%
\pgfpathlineto{\pgfqpoint{4.878366in}{0.593311in}}%
\pgfpathlineto{\pgfqpoint{4.856278in}{0.643693in}}%
\pgfpathlineto{\pgfqpoint{4.833900in}{0.694037in}}%
\pgfpathlineto{\pgfqpoint{4.811215in}{0.744340in}}%
\pgfpathlineto{\pgfqpoint{4.788201in}{0.794599in}}%
\pgfpathlineto{\pgfqpoint{4.764834in}{0.844808in}}%
\pgfpathlineto{\pgfqpoint{4.741088in}{0.894965in}}%
\pgfpathlineto{\pgfqpoint{4.716927in}{0.945063in}}%
\pgfpathlineto{\pgfqpoint{4.692317in}{0.995094in}}%
\pgfpathlineto{\pgfqpoint{4.667205in}{1.045052in}}%
\pgfpathlineto{\pgfqpoint{4.641543in}{1.094927in}}%
\pgfpathlineto{\pgfqpoint{4.615259in}{1.144704in}}%
\pgfpathlineto{\pgfqpoint{4.588270in}{1.194369in}}%
\pgfpathlineto{\pgfqpoint{4.560475in}{1.243900in}}%
\pgfpathlineto{\pgfqpoint{4.531722in}{1.293269in}}%
\pgfpathlineto{\pgfqpoint{4.501833in}{1.342436in}}%
\pgfpathlineto{\pgfqpoint{4.470572in}{1.391343in}}%
\pgfpathlineto{\pgfqpoint{4.437561in}{1.439905in}}%
\pgfpathlineto{\pgfqpoint{4.402242in}{1.487983in}}%
\pgfpathlineto{\pgfqpoint{4.363776in}{1.535329in}}%
\pgfpathlineto{\pgfqpoint{4.320718in}{1.581461in}}%
\pgfpathlineto{\pgfqpoint{4.270304in}{1.625267in}}%
\pgfpathlineto{\pgfqpoint{4.207379in}{1.663526in}}%
\pgfpathlineto{\pgfqpoint{4.207379in}{1.663526in}}%
\pgfpathlineto{\pgfqpoint{4.155757in}{1.681545in}}%
\pgfpathlineto{\pgfqpoint{4.094019in}{1.689067in}}%
\pgfpathlineto{\pgfqpoint{4.039625in}{1.686256in}}%
\pgfpathlineto{\pgfqpoint{3.981097in}{1.676626in}}%
\pgfpathlineto{\pgfqpoint{3.906873in}{1.659270in}}%
\pgfpathlineto{\pgfqpoint{3.820883in}{1.637507in}}%
\pgfpathlineto{\pgfqpoint{3.733288in}{1.617773in}}%
\pgfpathlineto{\pgfqpoint{3.642588in}{1.602937in}}%
\pgfpathlineto{\pgfqpoint{3.549106in}{1.595611in}}%
\pgfpathlineto{\pgfqpoint{3.454887in}{1.597835in}}%
\pgfpathlineto{\pgfqpoint{3.368563in}{1.609322in}}%
\pgfpathlineto{\pgfqpoint{3.288874in}{1.628853in}}%
\pgfusepath{stroke}%
\end{pgfscope}%
\begin{pgfscope}%
\pgfpathrectangle{\pgfqpoint{0.647939in}{0.492442in}}{\pgfqpoint{4.273799in}{2.331163in}}%
\pgfusepath{clip}%
\pgfsetbuttcap%
\pgfsetroundjoin%
\pgfsetlinewidth{0.803000pt}%
\definecolor{currentstroke}{rgb}{0.501961,0.501961,0.501961}%
\pgfsetstrokecolor{currentstroke}%
\pgfsetdash{}{0pt}%
\pgfpathmoveto{\pgfqpoint{4.921738in}{0.704366in}}%
\pgfpathlineto{\pgfqpoint{4.921738in}{0.704366in}}%
\pgfpathlineto{\pgfqpoint{4.901945in}{0.755031in}}%
\pgfpathlineto{\pgfqpoint{4.882018in}{0.805681in}}%
\pgfpathlineto{\pgfqpoint{4.861957in}{0.856315in}}%
\pgfpathlineto{\pgfqpoint{4.841759in}{0.906932in}}%
\pgfpathlineto{\pgfqpoint{4.821421in}{0.957533in}}%
\pgfpathlineto{\pgfqpoint{4.800944in}{1.008117in}}%
\pgfpathlineto{\pgfqpoint{4.780325in}{1.058684in}}%
\pgfpathlineto{\pgfqpoint{4.759564in}{1.109233in}}%
\pgfpathlineto{\pgfqpoint{4.738658in}{1.159765in}}%
\pgfpathlineto{\pgfqpoint{4.717609in}{1.210278in}}%
\pgfpathlineto{\pgfqpoint{4.696415in}{1.260773in}}%
\pgfpathlineto{\pgfqpoint{4.675079in}{1.311251in}}%
\pgfpathlineto{\pgfqpoint{4.653600in}{1.361710in}}%
\pgfpathlineto{\pgfqpoint{4.631984in}{1.412151in}}%
\pgfpathlineto{\pgfqpoint{4.610235in}{1.462576in}}%
\pgfpathlineto{\pgfqpoint{4.588361in}{1.512983in}}%
\pgfpathlineto{\pgfqpoint{4.566374in}{1.563375in}}%
\pgfpathlineto{\pgfqpoint{4.544290in}{1.613755in}}%
\pgfpathlineto{\pgfqpoint{4.522139in}{1.664124in}}%
\pgfpathlineto{\pgfqpoint{4.499960in}{1.714489in}}%
\pgfpathlineto{\pgfqpoint{4.477821in}{1.764858in}}%
\pgfpathlineto{\pgfqpoint{4.455835in}{1.815244in}}%
\pgfpathlineto{\pgfqpoint{4.434213in}{1.865672in}}%
\pgfpathlineto{\pgfqpoint{4.413374in}{1.916192in}}%
\pgfpathlineto{\pgfqpoint{4.394330in}{1.966904in}}%
\pgfpathlineto{\pgfqpoint{4.380207in}{2.018046in}}%
\pgfpathlineto{\pgfqpoint{4.380207in}{2.018046in}}%
\pgfpathlineto{\pgfqpoint{4.379812in}{2.061378in}}%
\pgfpathlineto{\pgfqpoint{4.379812in}{2.061378in}}%
\pgfpathlineto{\pgfqpoint{4.393342in}{2.096201in}}%
\pgfpathlineto{\pgfqpoint{4.416189in}{2.131076in}}%
\pgfpathlineto{\pgfqpoint{4.452619in}{2.177548in}}%
\pgfpathlineto{\pgfqpoint{4.491053in}{2.224752in}}%
\pgfpathlineto{\pgfqpoint{4.529288in}{2.272077in}}%
\pgfpathlineto{\pgfqpoint{4.566769in}{2.319548in}}%
\pgfpathlineto{\pgfqpoint{4.603410in}{2.367212in}}%
\pgfpathlineto{\pgfqpoint{4.639227in}{2.415086in}}%
\pgfpathlineto{\pgfqpoint{4.674260in}{2.463164in}}%
\pgfpathlineto{\pgfqpoint{4.708557in}{2.511437in}}%
\pgfpathlineto{\pgfqpoint{4.742120in}{2.559876in}}%
\pgfpathlineto{\pgfqpoint{4.774948in}{2.608454in}}%
\pgfpathlineto{\pgfqpoint{4.807102in}{2.657166in}}%
\pgfpathlineto{\pgfqpoint{4.838648in}{2.706011in}}%
\pgfpathlineto{\pgfqpoint{4.869603in}{2.754974in}}%
\pgfpathlineto{\pgfqpoint{4.899972in}{2.804039in}}%
\pgfpathlineto{\pgfqpoint{4.911966in}{2.823605in}}%
\pgfusepath{stroke}%
\end{pgfscope}%
\begin{pgfscope}%
\pgfpathrectangle{\pgfqpoint{0.647939in}{0.492442in}}{\pgfqpoint{4.273799in}{2.331163in}}%
\pgfusepath{clip}%
\pgfsetbuttcap%
\pgfsetroundjoin%
\pgfsetlinewidth{0.803000pt}%
\definecolor{currentstroke}{rgb}{0.501961,0.501961,0.501961}%
\pgfsetstrokecolor{currentstroke}%
\pgfsetdash{}{0pt}%
\pgfpathmoveto{\pgfqpoint{4.921738in}{0.969271in}}%
\pgfpathlineto{\pgfqpoint{4.921738in}{0.969271in}}%
\pgfpathlineto{\pgfqpoint{4.904522in}{1.020216in}}%
\pgfpathlineto{\pgfqpoint{4.887373in}{1.071167in}}%
\pgfpathlineto{\pgfqpoint{4.870306in}{1.122126in}}%
\pgfpathlineto{\pgfqpoint{4.853347in}{1.173096in}}%
\pgfpathlineto{\pgfqpoint{4.836520in}{1.224080in}}%
\pgfpathlineto{\pgfqpoint{4.819856in}{1.275079in}}%
\pgfpathlineto{\pgfqpoint{4.803397in}{1.326097in}}%
\pgfpathlineto{\pgfqpoint{4.787179in}{1.377139in}}%
\pgfpathlineto{\pgfqpoint{4.771258in}{1.428209in}}%
\pgfpathlineto{\pgfqpoint{4.755695in}{1.479311in}}%
\pgfpathlineto{\pgfqpoint{4.740559in}{1.530451in}}%
\pgfpathlineto{\pgfqpoint{4.725944in}{1.581636in}}%
\pgfpathlineto{\pgfqpoint{4.711952in}{1.632872in}}%
\pgfpathlineto{\pgfqpoint{4.698710in}{1.684169in}}%
\pgfpathlineto{\pgfqpoint{4.686379in}{1.735532in}}%
\pgfpathlineto{\pgfqpoint{4.675147in}{1.786969in}}%
\pgfpathlineto{\pgfqpoint{4.665234in}{1.838488in}}%
\pgfpathlineto{\pgfqpoint{4.656915in}{1.890089in}}%
\pgfpathlineto{\pgfqpoint{4.650509in}{1.941771in}}%
\pgfpathlineto{\pgfqpoint{4.646372in}{1.993520in}}%
\pgfpathlineto{\pgfqpoint{4.644883in}{2.045311in}}%
\pgfpathlineto{\pgfqpoint{4.646412in}{2.097099in}}%
\pgfpathlineto{\pgfqpoint{4.651265in}{2.148825in}}%
\pgfpathlineto{\pgfqpoint{4.659623in}{2.200414in}}%
\pgfpathlineto{\pgfqpoint{4.671481in}{2.251799in}}%
\pgfpathlineto{\pgfqpoint{4.686624in}{2.302924in}}%
\pgfpathlineto{\pgfqpoint{4.704702in}{2.353767in}}%
\pgfpathlineto{\pgfqpoint{4.725242in}{2.404332in}}%
\pgfpathlineto{\pgfqpoint{4.747753in}{2.454646in}}%
\pgfusepath{stroke}%
\end{pgfscope}%
\begin{pgfscope}%
\pgfpathrectangle{\pgfqpoint{0.647939in}{0.492442in}}{\pgfqpoint{4.273799in}{2.331163in}}%
\pgfusepath{clip}%
\pgfsetbuttcap%
\pgfsetroundjoin%
\pgfsetlinewidth{0.803000pt}%
\definecolor{currentstroke}{rgb}{0.501961,0.501961,0.501961}%
\pgfsetstrokecolor{currentstroke}%
\pgfsetdash{}{0pt}%
\pgfpathmoveto{\pgfqpoint{4.921738in}{1.234176in}}%
\pgfpathlineto{\pgfqpoint{4.921738in}{1.234176in}}%
\pgfpathlineto{\pgfqpoint{4.907624in}{1.285404in}}%
\pgfpathlineto{\pgfqpoint{4.893830in}{1.336657in}}%
\pgfpathlineto{\pgfqpoint{4.880397in}{1.387939in}}%
\pgfpathlineto{\pgfqpoint{4.867378in}{1.439253in}}%
\pgfpathlineto{\pgfqpoint{4.854837in}{1.490602in}}%
\pgfpathlineto{\pgfqpoint{4.842837in}{1.541990in}}%
\pgfpathlineto{\pgfqpoint{4.831455in}{1.593419in}}%
\pgfpathlineto{\pgfqpoint{4.820779in}{1.644894in}}%
\pgfpathlineto{\pgfqpoint{4.810910in}{1.696415in}}%
\pgfpathlineto{\pgfqpoint{4.801956in}{1.747987in}}%
\pgfpathlineto{\pgfqpoint{4.794039in}{1.799609in}}%
\pgfpathlineto{\pgfqpoint{4.787297in}{1.851280in}}%
\pgfpathlineto{\pgfqpoint{4.781879in}{1.902998in}}%
\pgfpathlineto{\pgfqpoint{4.777943in}{1.954755in}}%
\pgfpathlineto{\pgfqpoint{4.775648in}{2.006541in}}%
\pgfpathlineto{\pgfqpoint{4.775151in}{2.058341in}}%
\pgfpathlineto{\pgfqpoint{4.776598in}{2.110135in}}%
\pgfpathlineto{\pgfqpoint{4.780108in}{2.161900in}}%
\pgfpathlineto{\pgfqpoint{4.785765in}{2.213607in}}%
\pgfpathlineto{\pgfqpoint{4.793608in}{2.265229in}}%
\pgfpathlineto{\pgfqpoint{4.803608in}{2.316740in}}%
\pgfpathlineto{\pgfqpoint{4.815685in}{2.368117in}}%
\pgfpathlineto{\pgfqpoint{4.829716in}{2.419346in}}%
\pgfpathlineto{\pgfqpoint{4.845530in}{2.470421in}}%
\pgfpathlineto{\pgfqpoint{4.862924in}{2.521342in}}%
\pgfpathlineto{\pgfqpoint{4.881699in}{2.572118in}}%
\pgfpathlineto{\pgfqpoint{4.901639in}{2.622760in}}%
\pgfpathlineto{\pgfqpoint{4.921738in}{2.672488in}}%
\pgfusepath{stroke}%
\end{pgfscope}%
\begin{pgfscope}%
\pgfpathrectangle{\pgfqpoint{0.647939in}{0.492442in}}{\pgfqpoint{4.273799in}{2.331163in}}%
\pgfusepath{clip}%
\pgfsetbuttcap%
\pgfsetroundjoin%
\pgfsetlinewidth{0.803000pt}%
\definecolor{currentstroke}{rgb}{0.501961,0.501961,0.501961}%
\pgfsetstrokecolor{currentstroke}%
\pgfsetdash{}{0pt}%
\pgfpathmoveto{\pgfqpoint{4.921738in}{1.499081in}}%
\pgfpathlineto{\pgfqpoint{4.921738in}{1.499081in}}%
\pgfpathlineto{\pgfqpoint{4.911408in}{1.550577in}}%
\pgfpathlineto{\pgfqpoint{4.901708in}{1.602108in}}%
\pgfpathlineto{\pgfqpoint{4.892708in}{1.653678in}}%
\pgfpathlineto{\pgfqpoint{4.884484in}{1.705286in}}%
\pgfpathlineto{\pgfqpoint{4.877120in}{1.756933in}}%
\pgfusepath{stroke}%
\end{pgfscope}%
\begin{pgfscope}%
\pgfpathrectangle{\pgfqpoint{0.647939in}{0.492442in}}{\pgfqpoint{4.273799in}{2.331163in}}%
\pgfusepath{clip}%
\pgfsetbuttcap%
\pgfsetroundjoin%
\pgfsetlinewidth{0.803000pt}%
\definecolor{currentstroke}{rgb}{0.501961,0.501961,0.501961}%
\pgfsetstrokecolor{currentstroke}%
\pgfsetdash{}{0pt}%
\pgfpathmoveto{\pgfqpoint{4.921738in}{1.816967in}}%
\pgfpathlineto{\pgfqpoint{4.921738in}{1.816967in}}%
\pgfpathlineto{\pgfqpoint{4.917155in}{1.868709in}}%
\pgfpathlineto{\pgfqpoint{4.913624in}{1.920476in}}%
\pgfpathlineto{\pgfqpoint{4.911227in}{1.972262in}}%
\pgfpathlineto{\pgfqpoint{4.910039in}{2.024060in}}%
\pgfpathlineto{\pgfqpoint{4.910134in}{2.075862in}}%
\pgfpathlineto{\pgfqpoint{4.911578in}{2.127658in}}%
\pgfpathlineto{\pgfqpoint{4.914425in}{2.179437in}}%
\pgfpathlineto{\pgfqpoint{4.918717in}{2.231186in}}%
\pgfpathlineto{\pgfqpoint{4.921738in}{2.262355in}}%
\pgfusepath{stroke}%
\end{pgfscope}%
\begin{pgfscope}%
\pgfpathrectangle{\pgfqpoint{0.647939in}{0.492442in}}{\pgfqpoint{4.273799in}{2.331163in}}%
\pgfusepath{clip}%
\pgfsetbuttcap%
\pgfsetroundjoin%
\pgfsetlinewidth{0.803000pt}%
\definecolor{currentstroke}{rgb}{0.501961,0.501961,0.501961}%
\pgfsetstrokecolor{currentstroke}%
\pgfsetdash{}{0pt}%
\pgfpathmoveto{\pgfqpoint{4.338948in}{2.823605in}}%
\pgfpathlineto{\pgfqpoint{4.338948in}{2.823605in}}%
\pgfpathlineto{\pgfqpoint{4.388616in}{2.779519in}}%
\pgfpathlineto{\pgfqpoint{4.446510in}{2.738644in}}%
\pgfpathlineto{\pgfqpoint{4.446510in}{2.738644in}}%
\pgfpathlineto{\pgfqpoint{4.502229in}{2.712052in}}%
\pgfpathlineto{\pgfqpoint{4.502229in}{2.712052in}}%
\pgfpathlineto{\pgfqpoint{4.549614in}{2.700655in}}%
\pgfpathlineto{\pgfqpoint{4.549614in}{2.700655in}}%
\pgfpathlineto{\pgfqpoint{4.594162in}{2.699939in}}%
\pgfpathlineto{\pgfqpoint{4.636348in}{2.708219in}}%
\pgfpathlineto{\pgfqpoint{4.675539in}{2.723482in}}%
\pgfpathlineto{\pgfqpoint{4.717136in}{2.747245in}}%
\pgfpathlineto{\pgfqpoint{4.763485in}{2.782167in}}%
\pgfpathlineto{\pgfqpoint{4.812834in}{2.823605in}}%
\pgfusepath{stroke}%
\end{pgfscope}%
\begin{pgfscope}%
\pgfpathrectangle{\pgfqpoint{0.647939in}{0.492442in}}{\pgfqpoint{4.273799in}{2.331163in}}%
\pgfusepath{clip}%
\pgfsetbuttcap%
\pgfsetroundjoin%
\pgfsetlinewidth{0.803000pt}%
\definecolor{currentstroke}{rgb}{0.501961,0.501961,0.501961}%
\pgfsetstrokecolor{currentstroke}%
\pgfsetdash{}{0pt}%
\pgfpathmoveto{\pgfqpoint{4.241816in}{2.823605in}}%
\pgfpathlineto{\pgfqpoint{4.241816in}{2.823605in}}%
\pgfpathlineto{\pgfqpoint{4.282441in}{2.776791in}}%
\pgfpathlineto{\pgfqpoint{4.326092in}{2.730806in}}%
\pgfpathlineto{\pgfqpoint{4.374638in}{2.686350in}}%
\pgfpathlineto{\pgfqpoint{4.431935in}{2.645313in}}%
\pgfpathlineto{\pgfqpoint{4.431935in}{2.645313in}}%
\pgfpathlineto{\pgfqpoint{4.483640in}{2.621018in}}%
\pgfpathlineto{\pgfqpoint{4.483640in}{2.621018in}}%
\pgfpathlineto{\pgfqpoint{4.528239in}{2.611070in}}%
\pgfpathlineto{\pgfqpoint{4.577411in}{2.612674in}}%
\pgfpathlineto{\pgfqpoint{4.615972in}{2.622855in}}%
\pgfpathlineto{\pgfqpoint{4.653227in}{2.639887in}}%
\pgfpathlineto{\pgfqpoint{4.694304in}{2.666248in}}%
\pgfpathlineto{\pgfqpoint{4.741386in}{2.705168in}}%
\pgfusepath{stroke}%
\end{pgfscope}%
\begin{pgfscope}%
\pgfpathrectangle{\pgfqpoint{0.647939in}{0.492442in}}{\pgfqpoint{4.273799in}{2.331163in}}%
\pgfusepath{clip}%
\pgfsetbuttcap%
\pgfsetroundjoin%
\pgfsetlinewidth{0.803000pt}%
\definecolor{currentstroke}{rgb}{0.501961,0.501961,0.501961}%
\pgfsetstrokecolor{currentstroke}%
\pgfsetdash{}{0pt}%
\pgfpathmoveto{\pgfqpoint{4.144684in}{2.823605in}}%
\pgfpathlineto{\pgfqpoint{4.144684in}{2.823605in}}%
\pgfpathlineto{\pgfqpoint{4.180147in}{2.775550in}}%
\pgfpathlineto{\pgfqpoint{4.216713in}{2.727745in}}%
\pgfpathlineto{\pgfqpoint{4.254912in}{2.680326in}}%
\pgfpathlineto{\pgfqpoint{4.295613in}{2.633543in}}%
\pgfpathlineto{\pgfqpoint{4.340415in}{2.587921in}}%
\pgfpathlineto{\pgfqpoint{4.392700in}{2.544864in}}%
\pgfpathlineto{\pgfqpoint{4.392700in}{2.544864in}}%
\pgfpathlineto{\pgfqpoint{4.444316in}{2.515831in}}%
\pgfpathlineto{\pgfqpoint{4.444316in}{2.515831in}}%
\pgfpathlineto{\pgfqpoint{4.485998in}{2.503792in}}%
\pgfpathlineto{\pgfqpoint{4.485998in}{2.503792in}}%
\pgfpathlineto{\pgfqpoint{4.525073in}{2.502488in}}%
\pgfpathlineto{\pgfqpoint{4.561873in}{2.509963in}}%
\pgfpathlineto{\pgfqpoint{4.596004in}{2.524015in}}%
\pgfpathlineto{\pgfqpoint{4.633512in}{2.546659in}}%
\pgfpathlineto{\pgfqpoint{4.676678in}{2.580882in}}%
\pgfusepath{stroke}%
\end{pgfscope}%
\begin{pgfscope}%
\pgfpathrectangle{\pgfqpoint{0.647939in}{0.492442in}}{\pgfqpoint{4.273799in}{2.331163in}}%
\pgfusepath{clip}%
\pgfsetbuttcap%
\pgfsetroundjoin%
\pgfsetlinewidth{0.803000pt}%
\definecolor{currentstroke}{rgb}{0.501961,0.501961,0.501961}%
\pgfsetstrokecolor{currentstroke}%
\pgfsetdash{}{0pt}%
\pgfpathmoveto{\pgfqpoint{4.047552in}{2.823605in}}%
\pgfpathlineto{\pgfqpoint{4.047552in}{2.823605in}}%
\pgfpathlineto{\pgfqpoint{4.079942in}{2.774908in}}%
\pgfpathlineto{\pgfqpoint{4.112559in}{2.726257in}}%
\pgfpathlineto{\pgfqpoint{4.145558in}{2.677682in}}%
\pgfpathlineto{\pgfqpoint{4.179184in}{2.629236in}}%
\pgfpathlineto{\pgfqpoint{4.213807in}{2.581003in}}%
\pgfpathlineto{\pgfqpoint{4.249992in}{2.533119in}}%
\pgfpathlineto{\pgfqpoint{4.288745in}{2.485851in}}%
\pgfpathlineto{\pgfqpoint{4.332134in}{2.439848in}}%
\pgfpathlineto{\pgfqpoint{4.385409in}{2.397392in}}%
\pgfpathlineto{\pgfqpoint{4.385409in}{2.397392in}}%
\pgfpathlineto{\pgfqpoint{4.423572in}{2.379031in}}%
\pgfpathlineto{\pgfqpoint{4.423572in}{2.379031in}}%
\pgfpathlineto{\pgfqpoint{4.459185in}{2.372429in}}%
\pgfpathlineto{\pgfqpoint{4.496553in}{2.376637in}}%
\pgfpathlineto{\pgfqpoint{4.526855in}{2.387630in}}%
\pgfpathlineto{\pgfqpoint{4.558835in}{2.405882in}}%
\pgfusepath{stroke}%
\end{pgfscope}%
\begin{pgfscope}%
\pgfpathrectangle{\pgfqpoint{0.647939in}{0.492442in}}{\pgfqpoint{4.273799in}{2.331163in}}%
\pgfusepath{clip}%
\pgfsetbuttcap%
\pgfsetroundjoin%
\pgfsetlinewidth{0.803000pt}%
\definecolor{currentstroke}{rgb}{0.501961,0.501961,0.501961}%
\pgfsetstrokecolor{currentstroke}%
\pgfsetdash{}{0pt}%
\pgfpathmoveto{\pgfqpoint{3.950420in}{2.823605in}}%
\pgfpathlineto{\pgfqpoint{3.950420in}{2.823605in}}%
\pgfpathlineto{\pgfqpoint{3.980974in}{2.774556in}}%
\pgfpathlineto{\pgfqpoint{4.011292in}{2.725463in}}%
\pgfpathlineto{\pgfqpoint{4.041428in}{2.676337in}}%
\pgfpathlineto{\pgfqpoint{4.071438in}{2.627189in}}%
\pgfpathlineto{\pgfqpoint{4.101377in}{2.578027in}}%
\pgfpathlineto{\pgfqpoint{4.131350in}{2.528872in}}%
\pgfpathlineto{\pgfqpoint{4.161506in}{2.479753in}}%
\pgfpathlineto{\pgfqpoint{4.192032in}{2.430702in}}%
\pgfpathlineto{\pgfqpoint{4.223269in}{2.381786in}}%
\pgfpathlineto{\pgfqpoint{4.255873in}{2.333147in}}%
\pgfpathlineto{\pgfqpoint{4.291220in}{2.285120in}}%
\pgfpathlineto{\pgfqpoint{4.333347in}{2.238968in}}%
\pgfpathlineto{\pgfqpoint{4.333347in}{2.238968in}}%
\pgfpathlineto{\pgfqpoint{4.364221in}{2.216894in}}%
\pgfpathlineto{\pgfqpoint{4.364221in}{2.216894in}}%
\pgfusepath{stroke}%
\end{pgfscope}%
\begin{pgfscope}%
\pgfpathrectangle{\pgfqpoint{0.647939in}{0.492442in}}{\pgfqpoint{4.273799in}{2.331163in}}%
\pgfusepath{clip}%
\pgfsetbuttcap%
\pgfsetroundjoin%
\pgfsetlinewidth{0.803000pt}%
\definecolor{currentstroke}{rgb}{0.501961,0.501961,0.501961}%
\pgfsetstrokecolor{currentstroke}%
\pgfsetdash{}{0pt}%
\pgfpathmoveto{\pgfqpoint{3.853289in}{2.823605in}}%
\pgfpathlineto{\pgfqpoint{3.853289in}{2.823605in}}%
\pgfpathlineto{\pgfqpoint{3.882848in}{2.774375in}}%
\pgfpathlineto{\pgfqpoint{3.911906in}{2.725056in}}%
\pgfpathlineto{\pgfqpoint{3.940461in}{2.675650in}}%
\pgfpathlineto{\pgfqpoint{3.968496in}{2.626155in}}%
\pgfpathlineto{\pgfqpoint{3.996001in}{2.576572in}}%
\pgfpathlineto{\pgfqpoint{4.022955in}{2.526900in}}%
\pgfpathlineto{\pgfqpoint{4.049312in}{2.477132in}}%
\pgfpathlineto{\pgfqpoint{4.075025in}{2.427265in}}%
\pgfpathlineto{\pgfqpoint{4.099993in}{2.377285in}}%
\pgfpathlineto{\pgfqpoint{4.124091in}{2.327179in}}%
\pgfpathlineto{\pgfqpoint{4.147104in}{2.276923in}}%
\pgfpathlineto{\pgfqpoint{4.168666in}{2.226478in}}%
\pgfpathlineto{\pgfqpoint{4.188131in}{2.175784in}}%
\pgfpathlineto{\pgfqpoint{4.204248in}{2.124752in}}%
\pgfpathlineto{\pgfqpoint{4.214462in}{2.073301in}}%
\pgfpathlineto{\pgfqpoint{4.213560in}{2.021663in}}%
\pgfpathlineto{\pgfqpoint{4.195338in}{1.971320in}}%
\pgfpathlineto{\pgfqpoint{4.164395in}{1.927114in}}%
\pgfpathlineto{\pgfqpoint{4.120938in}{1.882260in}}%
\pgfpathlineto{\pgfqpoint{4.069852in}{1.838822in}}%
\pgfpathlineto{\pgfqpoint{4.013519in}{1.797297in}}%
\pgfpathlineto{\pgfqpoint{3.952246in}{1.757890in}}%
\pgfpathlineto{\pgfqpoint{3.885871in}{1.720991in}}%
\pgfpathlineto{\pgfqpoint{3.813930in}{1.687372in}}%
\pgfpathlineto{\pgfqpoint{3.735794in}{1.658240in}}%
\pgfpathlineto{\pgfqpoint{3.651189in}{1.635288in}}%
\pgfusepath{stroke}%
\end{pgfscope}%
\begin{pgfscope}%
\pgfpathrectangle{\pgfqpoint{0.647939in}{0.492442in}}{\pgfqpoint{4.273799in}{2.331163in}}%
\pgfusepath{clip}%
\pgfsetbuttcap%
\pgfsetroundjoin%
\pgfsetlinewidth{0.803000pt}%
\definecolor{currentstroke}{rgb}{0.501961,0.501961,0.501961}%
\pgfsetstrokecolor{currentstroke}%
\pgfsetdash{}{0pt}%
\pgfpathmoveto{\pgfqpoint{3.756157in}{2.823605in}}%
\pgfpathlineto{\pgfqpoint{3.756157in}{2.823605in}}%
\pgfpathlineto{\pgfqpoint{3.785362in}{2.774312in}}%
\pgfpathlineto{\pgfqpoint{3.813881in}{2.724900in}}%
\pgfpathlineto{\pgfqpoint{3.841681in}{2.675366in}}%
\pgfpathlineto{\pgfqpoint{3.868734in}{2.625709in}}%
\pgfpathlineto{\pgfqpoint{3.895002in}{2.575927in}}%
\pgfpathlineto{\pgfqpoint{3.920415in}{2.526013in}}%
\pgfpathlineto{\pgfqpoint{3.944897in}{2.475962in}}%
\pgfpathlineto{\pgfqpoint{3.968330in}{2.425761in}}%
\pgfpathlineto{\pgfqpoint{3.990567in}{2.375399in}}%
\pgfpathlineto{\pgfqpoint{4.011395in}{2.324859in}}%
\pgfpathlineto{\pgfqpoint{4.030518in}{2.274120in}}%
\pgfpathlineto{\pgfqpoint{4.047521in}{2.223159in}}%
\pgfpathlineto{\pgfqpoint{4.061788in}{2.171952in}}%
\pgfpathlineto{\pgfqpoint{4.072446in}{2.120490in}}%
\pgfpathlineto{\pgfqpoint{4.078235in}{2.068807in}}%
\pgfpathlineto{\pgfqpoint{4.077427in}{2.017052in}}%
\pgfpathlineto{\pgfqpoint{4.067976in}{1.965592in}}%
\pgfpathlineto{\pgfqpoint{4.048118in}{1.915064in}}%
\pgfpathlineto{\pgfqpoint{4.017128in}{1.866263in}}%
\pgfpathlineto{\pgfqpoint{3.975620in}{1.819848in}}%
\pgfusepath{stroke}%
\end{pgfscope}%
\begin{pgfscope}%
\pgfpathrectangle{\pgfqpoint{0.647939in}{0.492442in}}{\pgfqpoint{4.273799in}{2.331163in}}%
\pgfusepath{clip}%
\pgfsetbuttcap%
\pgfsetroundjoin%
\pgfsetlinewidth{0.803000pt}%
\definecolor{currentstroke}{rgb}{0.501961,0.501961,0.501961}%
\pgfsetstrokecolor{currentstroke}%
\pgfsetdash{}{0pt}%
\pgfpathmoveto{\pgfqpoint{3.659025in}{2.823605in}}%
\pgfpathlineto{\pgfqpoint{3.659025in}{2.823605in}}%
\pgfpathlineto{\pgfqpoint{3.688403in}{2.774343in}}%
\pgfpathlineto{\pgfqpoint{3.716941in}{2.724934in}}%
\pgfpathlineto{\pgfqpoint{3.744608in}{2.675378in}}%
\pgfpathlineto{\pgfqpoint{3.771370in}{2.625675in}}%
\pgfpathlineto{\pgfqpoint{3.797164in}{2.575819in}}%
\pgfpathlineto{\pgfqpoint{3.821925in}{2.525808in}}%
\pgfpathlineto{\pgfqpoint{3.845557in}{2.475635in}}%
\pgfpathlineto{\pgfqpoint{3.867938in}{2.425292in}}%
\pgfpathlineto{\pgfqpoint{3.888909in}{2.374769in}}%
\pgfpathlineto{\pgfqpoint{3.908259in}{2.324055in}}%
\pgfpathlineto{\pgfqpoint{3.925708in}{2.273137in}}%
\pgfpathlineto{\pgfqpoint{3.940885in}{2.222005in}}%
\pgfpathlineto{\pgfqpoint{3.953299in}{2.170654in}}%
\pgfpathlineto{\pgfqpoint{3.962281in}{2.119094in}}%
\pgfpathlineto{\pgfqpoint{3.966951in}{2.067371in}}%
\pgfpathlineto{\pgfqpoint{3.966189in}{2.015599in}}%
\pgfpathlineto{\pgfqpoint{3.958636in}{1.964010in}}%
\pgfpathlineto{\pgfqpoint{3.942822in}{1.913006in}}%
\pgfpathlineto{\pgfqpoint{3.917459in}{1.863187in}}%
\pgfpathlineto{\pgfqpoint{3.881696in}{1.815328in}}%
\pgfpathlineto{\pgfqpoint{3.835217in}{1.770323in}}%
\pgfpathlineto{\pgfqpoint{3.777999in}{1.729221in}}%
\pgfpathlineto{\pgfqpoint{3.710217in}{1.693266in}}%
\pgfusepath{stroke}%
\end{pgfscope}%
\begin{pgfscope}%
\pgfpathrectangle{\pgfqpoint{0.647939in}{0.492442in}}{\pgfqpoint{4.273799in}{2.331163in}}%
\pgfusepath{clip}%
\pgfsetbuttcap%
\pgfsetroundjoin%
\pgfsetlinewidth{0.803000pt}%
\definecolor{currentstroke}{rgb}{0.501961,0.501961,0.501961}%
\pgfsetstrokecolor{currentstroke}%
\pgfsetdash{}{0pt}%
\pgfpathmoveto{\pgfqpoint{3.561893in}{2.823605in}}%
\pgfpathlineto{\pgfqpoint{3.561893in}{2.823605in}}%
\pgfpathlineto{\pgfqpoint{3.591908in}{2.774458in}}%
\pgfpathlineto{\pgfqpoint{3.620950in}{2.725137in}}%
\pgfpathlineto{\pgfqpoint{3.648996in}{2.675644in}}%
\pgfpathlineto{\pgfqpoint{3.675998in}{2.625979in}}%
\pgfpathlineto{\pgfqpoint{3.701900in}{2.576141in}}%
\pgfpathlineto{\pgfqpoint{3.726633in}{2.526126in}}%
\pgfpathlineto{\pgfqpoint{3.750096in}{2.475930in}}%
\pgfpathlineto{\pgfqpoint{3.772171in}{2.425547in}}%
\pgfpathlineto{\pgfqpoint{3.792703in}{2.374970in}}%
\pgfpathlineto{\pgfqpoint{3.811490in}{2.324193in}}%
\pgfpathlineto{\pgfqpoint{3.828276in}{2.273209in}}%
\pgfpathlineto{\pgfqpoint{3.842721in}{2.222014in}}%
\pgfpathlineto{\pgfqpoint{3.854395in}{2.170611in}}%
\pgfpathlineto{\pgfqpoint{3.862744in}{2.119019in}}%
\pgfpathlineto{\pgfqpoint{3.867036in}{2.067284in}}%
\pgfpathlineto{\pgfqpoint{3.866352in}{2.015506in}}%
\pgfpathlineto{\pgfqpoint{3.859556in}{1.963875in}}%
\pgfpathlineto{\pgfqpoint{3.845323in}{1.912716in}}%
\pgfpathlineto{\pgfqpoint{3.822202in}{1.862558in}}%
\pgfpathlineto{\pgfqpoint{3.788774in}{1.814192in}}%
\pgfpathlineto{\pgfqpoint{3.743807in}{1.768746in}}%
\pgfpathlineto{\pgfqpoint{3.686347in}{1.727813in}}%
\pgfpathlineto{\pgfqpoint{3.615972in}{1.693565in}}%
\pgfpathlineto{\pgfqpoint{3.537659in}{1.669947in}}%
\pgfpathlineto{\pgfqpoint{3.459868in}{1.658631in}}%
\pgfpathlineto{\pgfqpoint{3.385745in}{1.657961in}}%
\pgfpathlineto{\pgfqpoint{3.314917in}{1.666202in}}%
\pgfpathlineto{\pgfqpoint{3.246064in}{1.682621in}}%
\pgfpathlineto{\pgfqpoint{3.178673in}{1.707184in}}%
\pgfpathlineto{\pgfqpoint{3.112238in}{1.740470in}}%
\pgfpathlineto{\pgfqpoint{3.051506in}{1.780081in}}%
\pgfpathlineto{\pgfqpoint{2.999113in}{1.823123in}}%
\pgfpathlineto{\pgfqpoint{2.954551in}{1.868742in}}%
\pgfusepath{stroke}%
\end{pgfscope}%
\begin{pgfscope}%
\pgfpathrectangle{\pgfqpoint{0.647939in}{0.492442in}}{\pgfqpoint{4.273799in}{2.331163in}}%
\pgfusepath{clip}%
\pgfsetbuttcap%
\pgfsetroundjoin%
\pgfsetlinewidth{0.803000pt}%
\definecolor{currentstroke}{rgb}{0.501961,0.501961,0.501961}%
\pgfsetstrokecolor{currentstroke}%
\pgfsetdash{}{0pt}%
\pgfpathmoveto{\pgfqpoint{3.464761in}{2.823605in}}%
\pgfpathlineto{\pgfqpoint{3.464761in}{2.823605in}}%
\pgfpathlineto{\pgfqpoint{3.495862in}{2.774659in}}%
\pgfpathlineto{\pgfqpoint{3.525868in}{2.725510in}}%
\pgfpathlineto{\pgfqpoint{3.554752in}{2.676162in}}%
\pgfpathlineto{\pgfqpoint{3.582464in}{2.626614in}}%
\pgfpathlineto{\pgfqpoint{3.608955in}{2.576868in}}%
\pgfpathlineto{\pgfqpoint{3.634155in}{2.526923in}}%
\pgfpathlineto{\pgfqpoint{3.657969in}{2.476776in}}%
\pgfpathlineto{\pgfqpoint{3.680281in}{2.426424in}}%
\pgfpathlineto{\pgfqpoint{3.700940in}{2.375864in}}%
\pgfpathlineto{\pgfqpoint{3.719755in}{2.325090in}}%
\pgfpathlineto{\pgfqpoint{3.736484in}{2.274101in}}%
\pgfpathlineto{\pgfqpoint{3.750813in}{2.222896in}}%
\pgfpathlineto{\pgfqpoint{3.762343in}{2.171483in}}%
\pgfpathlineto{\pgfqpoint{3.770563in}{2.119885in}}%
\pgfpathlineto{\pgfqpoint{3.774813in}{2.068147in}}%
\pgfpathlineto{\pgfqpoint{3.774245in}{2.016367in}}%
\pgfpathlineto{\pgfqpoint{3.767790in}{1.964720in}}%
\pgfpathlineto{\pgfqpoint{3.754126in}{1.913512in}}%
\pgfpathlineto{\pgfqpoint{3.731635in}{1.863269in}}%
\pgfpathlineto{\pgfqpoint{3.698467in}{1.814862in}}%
\pgfpathlineto{\pgfqpoint{3.652628in}{1.769726in}}%
\pgfpathlineto{\pgfqpoint{3.592272in}{1.730203in}}%
\pgfpathlineto{\pgfqpoint{3.520855in}{1.701079in}}%
\pgfpathlineto{\pgfqpoint{3.448177in}{1.685358in}}%
\pgfusepath{stroke}%
\end{pgfscope}%
\begin{pgfscope}%
\pgfpathrectangle{\pgfqpoint{0.647939in}{0.492442in}}{\pgfqpoint{4.273799in}{2.331163in}}%
\pgfusepath{clip}%
\pgfsetbuttcap%
\pgfsetroundjoin%
\pgfsetlinewidth{0.803000pt}%
\definecolor{currentstroke}{rgb}{0.501961,0.501961,0.501961}%
\pgfsetstrokecolor{currentstroke}%
\pgfsetdash{}{0pt}%
\pgfpathmoveto{\pgfqpoint{3.270498in}{2.823605in}}%
\pgfpathlineto{\pgfqpoint{3.270498in}{2.823605in}}%
\pgfpathlineto{\pgfqpoint{3.305217in}{2.775390in}}%
\pgfpathlineto{\pgfqpoint{3.338550in}{2.726884in}}%
\pgfpathlineto{\pgfqpoint{3.370474in}{2.678097in}}%
\pgfpathlineto{\pgfqpoint{3.400960in}{2.629037in}}%
\pgfpathlineto{\pgfqpoint{3.429967in}{2.579711in}}%
\pgfpathlineto{\pgfqpoint{3.457428in}{2.530123in}}%
\pgfpathlineto{\pgfqpoint{3.483263in}{2.480276in}}%
\pgfpathlineto{\pgfqpoint{3.507367in}{2.430172in}}%
\pgfpathlineto{\pgfqpoint{3.529600in}{2.379811in}}%
\pgfpathlineto{\pgfqpoint{3.549784in}{2.329195in}}%
\pgfpathlineto{\pgfqpoint{3.567692in}{2.278325in}}%
\pgfpathlineto{\pgfqpoint{3.583025in}{2.227207in}}%
\pgfpathlineto{\pgfqpoint{3.595407in}{2.175855in}}%
\pgfpathlineto{\pgfqpoint{3.604346in}{2.124293in}}%
\pgfpathlineto{\pgfqpoint{3.609187in}{2.072573in}}%
\pgfpathlineto{\pgfqpoint{3.609060in}{2.020792in}}%
\pgfpathlineto{\pgfqpoint{3.602790in}{1.969140in}}%
\pgfpathlineto{\pgfqpoint{3.588771in}{1.917971in}}%
\pgfpathlineto{\pgfqpoint{3.564755in}{1.867973in}}%
\pgfpathlineto{\pgfqpoint{3.527597in}{1.820545in}}%
\pgfpathlineto{\pgfqpoint{3.473209in}{1.778691in}}%
\pgfpathlineto{\pgfqpoint{3.473209in}{1.778691in}}%
\pgfpathlineto{\pgfqpoint{3.420976in}{1.754836in}}%
\pgfpathlineto{\pgfqpoint{3.357793in}{1.740243in}}%
\pgfpathlineto{\pgfqpoint{3.297301in}{1.737465in}}%
\pgfpathlineto{\pgfqpoint{3.239719in}{1.743596in}}%
\pgfpathlineto{\pgfqpoint{3.182896in}{1.757614in}}%
\pgfpathlineto{\pgfqpoint{3.126242in}{1.779594in}}%
\pgfpathlineto{\pgfqpoint{3.069150in}{1.810549in}}%
\pgfusepath{stroke}%
\end{pgfscope}%
\begin{pgfscope}%
\pgfpathrectangle{\pgfqpoint{0.647939in}{0.492442in}}{\pgfqpoint{4.273799in}{2.331163in}}%
\pgfusepath{clip}%
\pgfsetbuttcap%
\pgfsetroundjoin%
\pgfsetlinewidth{0.803000pt}%
\definecolor{currentstroke}{rgb}{0.501961,0.501961,0.501961}%
\pgfsetstrokecolor{currentstroke}%
\pgfsetdash{}{0pt}%
\pgfpathmoveto{\pgfqpoint{3.173366in}{2.823605in}}%
\pgfpathlineto{\pgfqpoint{3.173366in}{2.823605in}}%
\pgfpathlineto{\pgfqpoint{3.210694in}{2.775975in}}%
\pgfpathlineto{\pgfqpoint{3.246454in}{2.727987in}}%
\pgfpathlineto{\pgfqpoint{3.280632in}{2.679657in}}%
\pgfpathlineto{\pgfqpoint{3.313208in}{2.630999in}}%
\pgfpathlineto{\pgfqpoint{3.344149in}{2.582025in}}%
\pgfpathlineto{\pgfqpoint{3.373398in}{2.532742in}}%
\pgfpathlineto{\pgfqpoint{3.400879in}{2.483158in}}%
\pgfusepath{stroke}%
\end{pgfscope}%
\begin{pgfscope}%
\pgfpathrectangle{\pgfqpoint{0.647939in}{0.492442in}}{\pgfqpoint{4.273799in}{2.331163in}}%
\pgfusepath{clip}%
\pgfsetbuttcap%
\pgfsetroundjoin%
\pgfsetlinewidth{0.803000pt}%
\definecolor{currentstroke}{rgb}{0.501961,0.501961,0.501961}%
\pgfsetstrokecolor{currentstroke}%
\pgfsetdash{}{0pt}%
\pgfpathmoveto{\pgfqpoint{2.979102in}{2.823605in}}%
\pgfpathlineto{\pgfqpoint{2.979102in}{2.823605in}}%
\pgfpathlineto{\pgfqpoint{3.023535in}{2.777826in}}%
\pgfpathlineto{\pgfqpoint{3.065929in}{2.731474in}}%
\pgfpathlineto{\pgfqpoint{3.106291in}{2.684586in}}%
\pgfpathlineto{\pgfqpoint{3.144621in}{2.637194in}}%
\pgfpathlineto{\pgfqpoint{3.180909in}{2.589326in}}%
\pgfpathlineto{\pgfqpoint{3.215128in}{2.541008in}}%
\pgfpathlineto{\pgfqpoint{3.247226in}{2.492259in}}%
\pgfpathlineto{\pgfqpoint{3.277124in}{2.443094in}}%
\pgfpathlineto{\pgfqpoint{3.304719in}{2.393532in}}%
\pgfpathlineto{\pgfqpoint{3.329855in}{2.343582in}}%
\pgfpathlineto{\pgfqpoint{3.352320in}{2.293256in}}%
\pgfpathlineto{\pgfqpoint{3.371830in}{2.242567in}}%
\pgfpathlineto{\pgfqpoint{3.387993in}{2.191530in}}%
\pgfpathlineto{\pgfqpoint{3.400274in}{2.140176in}}%
\pgfpathlineto{\pgfqpoint{3.407906in}{2.088560in}}%
\pgfpathlineto{\pgfqpoint{3.409785in}{2.036802in}}%
\pgfpathlineto{\pgfqpoint{3.404224in}{1.985149in}}%
\pgfpathlineto{\pgfqpoint{3.388458in}{1.934188in}}%
\pgfpathlineto{\pgfqpoint{3.357651in}{1.885482in}}%
\pgfpathlineto{\pgfqpoint{3.357651in}{1.885482in}}%
\pgfpathlineto{\pgfqpoint{3.320907in}{1.853682in}}%
\pgfpathlineto{\pgfqpoint{3.320907in}{1.853682in}}%
\pgfpathlineto{\pgfqpoint{3.279404in}{1.833872in}}%
\pgfpathlineto{\pgfqpoint{3.227006in}{1.823914in}}%
\pgfpathlineto{\pgfqpoint{3.179729in}{1.825291in}}%
\pgfpathlineto{\pgfqpoint{3.134554in}{1.834432in}}%
\pgfpathlineto{\pgfqpoint{3.088873in}{1.851131in}}%
\pgfpathlineto{\pgfqpoint{3.042461in}{1.876235in}}%
\pgfpathlineto{\pgfqpoint{2.996671in}{1.910753in}}%
\pgfusepath{stroke}%
\end{pgfscope}%
\begin{pgfscope}%
\pgfpathrectangle{\pgfqpoint{0.647939in}{0.492442in}}{\pgfqpoint{4.273799in}{2.331163in}}%
\pgfusepath{clip}%
\pgfsetbuttcap%
\pgfsetroundjoin%
\pgfsetlinewidth{0.803000pt}%
\definecolor{currentstroke}{rgb}{0.501961,0.501961,0.501961}%
\pgfsetstrokecolor{currentstroke}%
\pgfsetdash{}{0pt}%
\pgfpathmoveto{\pgfqpoint{2.784839in}{2.823605in}}%
\pgfpathlineto{\pgfqpoint{2.784839in}{2.823605in}}%
\pgfpathlineto{\pgfqpoint{2.839078in}{2.781098in}}%
\pgfpathlineto{\pgfqpoint{2.890693in}{2.737624in}}%
\pgfpathlineto{\pgfqpoint{2.939671in}{2.693251in}}%
\pgfpathlineto{\pgfqpoint{2.986020in}{2.648046in}}%
\pgfpathlineto{\pgfqpoint{3.029755in}{2.602073in}}%
\pgfpathlineto{\pgfqpoint{3.070887in}{2.555390in}}%
\pgfpathlineto{\pgfqpoint{3.109406in}{2.508046in}}%
\pgfpathlineto{\pgfqpoint{3.145280in}{2.460089in}}%
\pgfpathlineto{\pgfqpoint{3.178447in}{2.411556in}}%
\pgfpathlineto{\pgfqpoint{3.208795in}{2.362480in}}%
\pgfpathlineto{\pgfqpoint{3.236149in}{2.312881in}}%
\pgfpathlineto{\pgfqpoint{3.260262in}{2.262786in}}%
\pgfpathlineto{\pgfqpoint{3.280766in}{2.212218in}}%
\pgfpathlineto{\pgfqpoint{3.297131in}{2.161206in}}%
\pgfpathlineto{\pgfqpoint{3.308573in}{2.109805in}}%
\pgfpathlineto{\pgfqpoint{3.313870in}{2.058123in}}%
\pgfpathlineto{\pgfqpoint{3.311018in}{2.006419in}}%
\pgfpathlineto{\pgfqpoint{3.296367in}{1.955402in}}%
\pgfpathlineto{\pgfqpoint{3.296367in}{1.955402in}}%
\pgfpathlineto{\pgfqpoint{3.269367in}{1.914278in}}%
\pgfpathlineto{\pgfqpoint{3.269367in}{1.914278in}}%
\pgfpathlineto{\pgfqpoint{3.236401in}{1.888496in}}%
\pgfpathlineto{\pgfqpoint{3.236401in}{1.888496in}}%
\pgfpathlineto{\pgfqpoint{3.199495in}{1.874349in}}%
\pgfpathlineto{\pgfqpoint{3.154758in}{1.870157in}}%
\pgfusepath{stroke}%
\end{pgfscope}%
\begin{pgfscope}%
\pgfpathrectangle{\pgfqpoint{0.647939in}{0.492442in}}{\pgfqpoint{4.273799in}{2.331163in}}%
\pgfusepath{clip}%
\pgfsetbuttcap%
\pgfsetroundjoin%
\pgfsetlinewidth{0.803000pt}%
\definecolor{currentstroke}{rgb}{0.501961,0.501961,0.501961}%
\pgfsetstrokecolor{currentstroke}%
\pgfsetdash{}{0pt}%
\pgfpathmoveto{\pgfqpoint{2.590575in}{2.823605in}}%
\pgfpathlineto{\pgfqpoint{2.590575in}{2.823605in}}%
\pgfpathlineto{\pgfqpoint{2.656459in}{2.786319in}}%
\pgfpathlineto{\pgfqpoint{2.719352in}{2.747528in}}%
\pgfpathlineto{\pgfqpoint{2.779066in}{2.707271in}}%
\pgfpathlineto{\pgfqpoint{2.835496in}{2.665631in}}%
\pgfpathlineto{\pgfqpoint{2.888634in}{2.622717in}}%
\pgfpathlineto{\pgfqpoint{2.938493in}{2.578644in}}%
\pgfpathlineto{\pgfqpoint{2.985100in}{2.533523in}}%
\pgfusepath{stroke}%
\end{pgfscope}%
\begin{pgfscope}%
\pgfpathrectangle{\pgfqpoint{0.647939in}{0.492442in}}{\pgfqpoint{4.273799in}{2.331163in}}%
\pgfusepath{clip}%
\pgfsetbuttcap%
\pgfsetroundjoin%
\pgfsetlinewidth{0.803000pt}%
\definecolor{currentstroke}{rgb}{0.501961,0.501961,0.501961}%
\pgfsetstrokecolor{currentstroke}%
\pgfsetdash{}{0pt}%
\pgfpathmoveto{\pgfqpoint{2.493443in}{2.823605in}}%
\pgfpathlineto{\pgfqpoint{2.493443in}{2.823605in}}%
\pgfpathlineto{\pgfqpoint{2.564920in}{2.789526in}}%
\pgfpathlineto{\pgfqpoint{2.633610in}{2.753784in}}%
\pgfpathlineto{\pgfqpoint{2.699084in}{2.716292in}}%
\pgfpathlineto{\pgfqpoint{2.761083in}{2.677083in}}%
\pgfpathlineto{\pgfqpoint{2.819490in}{2.636264in}}%
\pgfusepath{stroke}%
\end{pgfscope}%
\begin{pgfscope}%
\pgfpathrectangle{\pgfqpoint{0.647939in}{0.492442in}}{\pgfqpoint{4.273799in}{2.331163in}}%
\pgfusepath{clip}%
\pgfsetbuttcap%
\pgfsetroundjoin%
\pgfsetlinewidth{0.803000pt}%
\definecolor{currentstroke}{rgb}{0.501961,0.501961,0.501961}%
\pgfsetstrokecolor{currentstroke}%
\pgfsetdash{}{0pt}%
\pgfpathmoveto{\pgfqpoint{2.299180in}{2.823605in}}%
\pgfpathlineto{\pgfqpoint{2.299180in}{2.823605in}}%
\pgfpathlineto{\pgfqpoint{2.378862in}{2.795426in}}%
\pgfpathlineto{\pgfqpoint{2.457331in}{2.766265in}}%
\pgfpathlineto{\pgfqpoint{2.533539in}{2.735397in}}%
\pgfpathlineto{\pgfqpoint{2.606669in}{2.702406in}}%
\pgfpathlineto{\pgfqpoint{2.676169in}{2.667153in}}%
\pgfusepath{stroke}%
\end{pgfscope}%
\begin{pgfscope}%
\pgfpathrectangle{\pgfqpoint{0.647939in}{0.492442in}}{\pgfqpoint{4.273799in}{2.331163in}}%
\pgfusepath{clip}%
\pgfsetbuttcap%
\pgfsetroundjoin%
\pgfsetlinewidth{0.803000pt}%
\definecolor{currentstroke}{rgb}{0.501961,0.501961,0.501961}%
\pgfsetstrokecolor{currentstroke}%
\pgfsetdash{}{0pt}%
\pgfpathmoveto{\pgfqpoint{2.007784in}{2.823605in}}%
\pgfpathlineto{\pgfqpoint{2.007784in}{2.823605in}}%
\pgfpathlineto{\pgfqpoint{2.086570in}{2.794783in}}%
\pgfpathlineto{\pgfqpoint{2.169210in}{2.769326in}}%
\pgfpathlineto{\pgfqpoint{2.254007in}{2.746028in}}%
\pgfpathlineto{\pgfqpoint{2.339305in}{2.723271in}}%
\pgfpathlineto{\pgfqpoint{2.423618in}{2.699475in}}%
\pgfpathlineto{\pgfqpoint{2.505637in}{2.673435in}}%
\pgfpathlineto{\pgfqpoint{2.584239in}{2.644450in}}%
\pgfpathlineto{\pgfqpoint{2.658547in}{2.612298in}}%
\pgfpathlineto{\pgfqpoint{2.728075in}{2.577111in}}%
\pgfpathlineto{\pgfqpoint{2.792625in}{2.539193in}}%
\pgfpathlineto{\pgfqpoint{2.852180in}{2.498904in}}%
\pgfpathlineto{\pgfqpoint{2.906849in}{2.456594in}}%
\pgfpathlineto{\pgfqpoint{2.956755in}{2.412570in}}%
\pgfpathlineto{\pgfqpoint{3.002018in}{2.367073in}}%
\pgfpathlineto{\pgfqpoint{3.042674in}{2.320296in}}%
\pgfpathlineto{\pgfqpoint{3.078638in}{2.272387in}}%
\pgfpathlineto{\pgfqpoint{3.109654in}{2.223456in}}%
\pgfpathlineto{\pgfqpoint{3.135208in}{2.173598in}}%
\pgfpathlineto{\pgfqpoint{3.154349in}{2.122906in}}%
\pgfpathlineto{\pgfqpoint{3.165271in}{2.071519in}}%
\pgfpathlineto{\pgfqpoint{3.164119in}{2.019894in}}%
\pgfpathlineto{\pgfqpoint{3.164119in}{2.019894in}}%
\pgfpathlineto{\pgfqpoint{3.149069in}{1.981024in}}%
\pgfpathlineto{\pgfqpoint{3.149069in}{1.981024in}}%
\pgfpathlineto{\pgfqpoint{3.126678in}{1.958785in}}%
\pgfpathlineto{\pgfqpoint{3.126678in}{1.958785in}}%
\pgfpathlineto{\pgfqpoint{3.099260in}{1.947995in}}%
\pgfpathlineto{\pgfqpoint{3.065709in}{1.947568in}}%
\pgfusepath{stroke}%
\end{pgfscope}%
\begin{pgfscope}%
\pgfpathrectangle{\pgfqpoint{0.647939in}{0.492442in}}{\pgfqpoint{4.273799in}{2.331163in}}%
\pgfusepath{clip}%
\pgfsetbuttcap%
\pgfsetroundjoin%
\pgfsetlinewidth{0.803000pt}%
\definecolor{currentstroke}{rgb}{0.501961,0.501961,0.501961}%
\pgfsetstrokecolor{currentstroke}%
\pgfsetdash{}{0pt}%
\pgfpathmoveto{\pgfqpoint{1.813521in}{2.823605in}}%
\pgfpathlineto{\pgfqpoint{1.813521in}{2.823605in}}%
\pgfpathlineto{\pgfqpoint{1.879398in}{2.786398in}}%
\pgfpathlineto{\pgfqpoint{1.952615in}{2.753579in}}%
\pgfpathlineto{\pgfqpoint{2.032875in}{2.726102in}}%
\pgfpathlineto{\pgfqpoint{2.118524in}{2.703910in}}%
\pgfpathlineto{\pgfqpoint{2.207278in}{2.685554in}}%
\pgfpathlineto{\pgfqpoint{2.297063in}{2.668706in}}%
\pgfpathlineto{\pgfqpoint{2.386238in}{2.650984in}}%
\pgfpathlineto{\pgfqpoint{2.473353in}{2.630520in}}%
\pgfpathlineto{\pgfqpoint{2.557028in}{2.606216in}}%
\pgfpathlineto{\pgfqpoint{2.636119in}{2.577716in}}%
\pgfusepath{stroke}%
\end{pgfscope}%
\begin{pgfscope}%
\pgfpathrectangle{\pgfqpoint{0.647939in}{0.492442in}}{\pgfqpoint{4.273799in}{2.331163in}}%
\pgfusepath{clip}%
\pgfsetbuttcap%
\pgfsetroundjoin%
\pgfsetlinewidth{0.803000pt}%
\definecolor{currentstroke}{rgb}{0.501961,0.501961,0.501961}%
\pgfsetstrokecolor{currentstroke}%
\pgfsetdash{}{0pt}%
\pgfpathmoveto{\pgfqpoint{1.716389in}{2.823605in}}%
\pgfpathlineto{\pgfqpoint{1.716389in}{2.823605in}}%
\pgfpathlineto{\pgfqpoint{1.772619in}{2.781917in}}%
\pgfpathlineto{\pgfqpoint{1.835821in}{2.743367in}}%
\pgfpathlineto{\pgfqpoint{1.907284in}{2.709460in}}%
\pgfpathlineto{\pgfqpoint{1.987265in}{2.681859in}}%
\pgfpathlineto{\pgfqpoint{2.074121in}{2.661265in}}%
\pgfusepath{stroke}%
\end{pgfscope}%
\begin{pgfscope}%
\pgfpathrectangle{\pgfqpoint{0.647939in}{0.492442in}}{\pgfqpoint{4.273799in}{2.331163in}}%
\pgfusepath{clip}%
\pgfsetbuttcap%
\pgfsetroundjoin%
\pgfsetlinewidth{0.803000pt}%
\definecolor{currentstroke}{rgb}{0.501961,0.501961,0.501961}%
\pgfsetstrokecolor{currentstroke}%
\pgfsetdash{}{0pt}%
\pgfpathmoveto{\pgfqpoint{1.522125in}{2.823605in}}%
\pgfpathlineto{\pgfqpoint{1.522125in}{2.823605in}}%
\pgfpathlineto{\pgfqpoint{1.559555in}{2.776006in}}%
\pgfpathlineto{\pgfqpoint{1.599941in}{2.729133in}}%
\pgfpathlineto{\pgfqpoint{1.644242in}{2.683333in}}%
\pgfpathlineto{\pgfqpoint{1.693916in}{2.639236in}}%
\pgfpathlineto{\pgfqpoint{1.751229in}{2.598083in}}%
\pgfpathlineto{\pgfqpoint{1.819559in}{2.562472in}}%
\pgfpathlineto{\pgfqpoint{1.819559in}{2.562472in}}%
\pgfpathlineto{\pgfqpoint{1.886757in}{2.540326in}}%
\pgfpathlineto{\pgfqpoint{1.962707in}{2.527858in}}%
\pgfpathlineto{\pgfqpoint{2.037217in}{2.524818in}}%
\pgfpathlineto{\pgfqpoint{2.122019in}{2.527691in}}%
\pgfpathlineto{\pgfqpoint{2.216311in}{2.533406in}}%
\pgfpathlineto{\pgfqpoint{2.310872in}{2.537343in}}%
\pgfpathlineto{\pgfqpoint{2.405537in}{2.536204in}}%
\pgfpathlineto{\pgfqpoint{2.498803in}{2.527825in}}%
\pgfpathlineto{\pgfqpoint{2.588351in}{2.511347in}}%
\pgfpathlineto{\pgfqpoint{2.671941in}{2.487221in}}%
\pgfpathlineto{\pgfqpoint{2.748233in}{2.456706in}}%
\pgfusepath{stroke}%
\end{pgfscope}%
\begin{pgfscope}%
\pgfpathrectangle{\pgfqpoint{0.647939in}{0.492442in}}{\pgfqpoint{4.273799in}{2.331163in}}%
\pgfusepath{clip}%
\pgfsetbuttcap%
\pgfsetroundjoin%
\pgfsetlinewidth{0.803000pt}%
\definecolor{currentstroke}{rgb}{0.501961,0.501961,0.501961}%
\pgfsetstrokecolor{currentstroke}%
\pgfsetdash{}{0pt}%
\pgfpathmoveto{\pgfqpoint{1.424993in}{2.823605in}}%
\pgfpathlineto{\pgfqpoint{1.424993in}{2.823605in}}%
\pgfpathlineto{\pgfqpoint{1.455132in}{2.774484in}}%
\pgfpathlineto{\pgfqpoint{1.486659in}{2.725627in}}%
\pgfpathlineto{\pgfqpoint{1.519946in}{2.677121in}}%
\pgfpathlineto{\pgfqpoint{1.555535in}{2.629102in}}%
\pgfpathlineto{\pgfqpoint{1.594228in}{2.581809in}}%
\pgfpathlineto{\pgfqpoint{1.637369in}{2.535697in}}%
\pgfpathlineto{\pgfqpoint{1.687456in}{2.491768in}}%
\pgfpathlineto{\pgfqpoint{1.748979in}{2.452719in}}%
\pgfpathlineto{\pgfqpoint{1.748979in}{2.452719in}}%
\pgfpathlineto{\pgfqpoint{1.802938in}{2.431822in}}%
\pgfpathlineto{\pgfqpoint{1.802938in}{2.431822in}}%
\pgfpathlineto{\pgfqpoint{1.857450in}{2.422149in}}%
\pgfpathlineto{\pgfqpoint{1.914947in}{2.421763in}}%
\pgfpathlineto{\pgfqpoint{1.973775in}{2.428597in}}%
\pgfpathlineto{\pgfqpoint{2.044899in}{2.442450in}}%
\pgfpathlineto{\pgfqpoint{2.132514in}{2.462233in}}%
\pgfpathlineto{\pgfqpoint{2.221093in}{2.480632in}}%
\pgfpathlineto{\pgfqpoint{2.312272in}{2.494519in}}%
\pgfpathlineto{\pgfqpoint{2.405963in}{2.501206in}}%
\pgfusepath{stroke}%
\end{pgfscope}%
\begin{pgfscope}%
\pgfpathrectangle{\pgfqpoint{0.647939in}{0.492442in}}{\pgfqpoint{4.273799in}{2.331163in}}%
\pgfusepath{clip}%
\pgfsetbuttcap%
\pgfsetroundjoin%
\pgfsetlinewidth{0.803000pt}%
\definecolor{currentstroke}{rgb}{0.501961,0.501961,0.501961}%
\pgfsetstrokecolor{currentstroke}%
\pgfsetdash{}{0pt}%
\pgfpathmoveto{\pgfqpoint{1.327862in}{2.823605in}}%
\pgfpathlineto{\pgfqpoint{1.327862in}{2.823605in}}%
\pgfpathlineto{\pgfqpoint{1.352231in}{2.773539in}}%
\pgfpathlineto{\pgfqpoint{1.377096in}{2.723546in}}%
\pgfpathlineto{\pgfqpoint{1.402524in}{2.673637in}}%
\pgfpathlineto{\pgfqpoint{1.428614in}{2.623831in}}%
\pgfpathlineto{\pgfqpoint{1.455501in}{2.574152in}}%
\pgfpathlineto{\pgfqpoint{1.483365in}{2.524638in}}%
\pgfpathlineto{\pgfqpoint{1.512526in}{2.475344in}}%
\pgfpathlineto{\pgfqpoint{1.543447in}{2.426376in}}%
\pgfpathlineto{\pgfqpoint{1.576950in}{2.377938in}}%
\pgfpathlineto{\pgfqpoint{1.614814in}{2.330513in}}%
\pgfpathlineto{\pgfqpoint{1.661572in}{2.285751in}}%
\pgfpathlineto{\pgfqpoint{1.661572in}{2.285751in}}%
\pgfpathlineto{\pgfqpoint{1.699734in}{2.262847in}}%
\pgfpathlineto{\pgfqpoint{1.699734in}{2.262847in}}%
\pgfpathlineto{\pgfqpoint{1.735233in}{2.253310in}}%
\pgfpathlineto{\pgfqpoint{1.775342in}{2.254640in}}%
\pgfpathlineto{\pgfqpoint{1.809228in}{2.263000in}}%
\pgfpathlineto{\pgfqpoint{1.849023in}{2.278344in}}%
\pgfpathlineto{\pgfqpoint{1.904168in}{2.304667in}}%
\pgfusepath{stroke}%
\end{pgfscope}%
\begin{pgfscope}%
\pgfpathrectangle{\pgfqpoint{0.647939in}{0.492442in}}{\pgfqpoint{4.273799in}{2.331163in}}%
\pgfusepath{clip}%
\pgfsetbuttcap%
\pgfsetroundjoin%
\pgfsetlinewidth{0.803000pt}%
\definecolor{currentstroke}{rgb}{0.501961,0.501961,0.501961}%
\pgfsetstrokecolor{currentstroke}%
\pgfsetdash{}{0pt}%
\pgfpathmoveto{\pgfqpoint{1.230730in}{2.823605in}}%
\pgfpathlineto{\pgfqpoint{1.230730in}{2.823605in}}%
\pgfpathlineto{\pgfqpoint{1.250617in}{2.772952in}}%
\pgfpathlineto{\pgfqpoint{1.270506in}{2.722298in}}%
\pgfpathlineto{\pgfqpoint{1.290369in}{2.671642in}}%
\pgfpathlineto{\pgfqpoint{1.310174in}{2.620979in}}%
\pgfpathlineto{\pgfqpoint{1.329874in}{2.570305in}}%
\pgfpathlineto{\pgfqpoint{1.349417in}{2.519613in}}%
\pgfpathlineto{\pgfqpoint{1.368721in}{2.468894in}}%
\pgfpathlineto{\pgfqpoint{1.387686in}{2.418137in}}%
\pgfpathlineto{\pgfqpoint{1.406162in}{2.367327in}}%
\pgfpathlineto{\pgfqpoint{1.423949in}{2.316444in}}%
\pgfpathlineto{\pgfqpoint{1.440736in}{2.265461in}}%
\pgfpathlineto{\pgfqpoint{1.456053in}{2.214344in}}%
\pgfpathlineto{\pgfqpoint{1.469207in}{2.163048in}}%
\pgfpathlineto{\pgfqpoint{1.479075in}{2.111539in}}%
\pgfpathlineto{\pgfqpoint{1.483954in}{2.059836in}}%
\pgfpathlineto{\pgfqpoint{1.481664in}{2.008110in}}%
\pgfpathlineto{\pgfqpoint{1.470448in}{1.956769in}}%
\pgfpathlineto{\pgfqpoint{1.450600in}{1.906208in}}%
\pgfpathlineto{\pgfqpoint{1.424371in}{1.856498in}}%
\pgfpathlineto{\pgfqpoint{1.394326in}{1.807419in}}%
\pgfpathlineto{\pgfqpoint{1.362356in}{1.758732in}}%
\pgfusepath{stroke}%
\end{pgfscope}%
\begin{pgfscope}%
\pgfpathrectangle{\pgfqpoint{0.647939in}{0.492442in}}{\pgfqpoint{4.273799in}{2.331163in}}%
\pgfusepath{clip}%
\pgfsetbuttcap%
\pgfsetroundjoin%
\pgfsetlinewidth{0.803000pt}%
\definecolor{currentstroke}{rgb}{0.501961,0.501961,0.501961}%
\pgfsetstrokecolor{currentstroke}%
\pgfsetdash{}{0pt}%
\pgfpathmoveto{\pgfqpoint{1.133598in}{2.823605in}}%
\pgfpathlineto{\pgfqpoint{1.133598in}{2.823605in}}%
\pgfpathlineto{\pgfqpoint{1.149988in}{2.772580in}}%
\pgfpathlineto{\pgfqpoint{1.166137in}{2.721532in}}%
\pgfpathlineto{\pgfqpoint{1.181992in}{2.670456in}}%
\pgfpathlineto{\pgfqpoint{1.197488in}{2.619348in}}%
\pgfpathlineto{\pgfqpoint{1.212560in}{2.568202in}}%
\pgfpathlineto{\pgfqpoint{1.227119in}{2.517012in}}%
\pgfpathlineto{\pgfqpoint{1.241059in}{2.465771in}}%
\pgfpathlineto{\pgfqpoint{1.254261in}{2.414472in}}%
\pgfpathlineto{\pgfqpoint{1.266571in}{2.363107in}}%
\pgfpathlineto{\pgfqpoint{1.277802in}{2.311670in}}%
\pgfpathlineto{\pgfqpoint{1.287741in}{2.260153in}}%
\pgfpathlineto{\pgfqpoint{1.296129in}{2.208554in}}%
\pgfpathlineto{\pgfqpoint{1.302660in}{2.156877in}}%
\pgfpathlineto{\pgfqpoint{1.306989in}{2.105132in}}%
\pgfpathlineto{\pgfqpoint{1.308749in}{2.053344in}}%
\pgfpathlineto{\pgfqpoint{1.307583in}{2.001552in}}%
\pgfpathlineto{\pgfqpoint{1.303184in}{1.949813in}}%
\pgfpathlineto{\pgfqpoint{1.295358in}{1.898197in}}%
\pgfpathlineto{\pgfqpoint{1.284088in}{1.846772in}}%
\pgfpathlineto{\pgfqpoint{1.269553in}{1.795595in}}%
\pgfpathlineto{\pgfqpoint{1.252061in}{1.744691in}}%
\pgfpathlineto{\pgfqpoint{1.232065in}{1.694061in}}%
\pgfpathlineto{\pgfqpoint{1.210036in}{1.643682in}}%
\pgfpathlineto{\pgfqpoint{1.186430in}{1.593516in}}%
\pgfpathlineto{\pgfqpoint{1.161649in}{1.543520in}}%
\pgfpathlineto{\pgfqpoint{1.136022in}{1.493651in}}%
\pgfpathlineto{\pgfqpoint{1.109827in}{1.443872in}}%
\pgfpathlineto{\pgfqpoint{1.083255in}{1.394148in}}%
\pgfpathlineto{\pgfqpoint{1.056481in}{1.344457in}}%
\pgfpathlineto{\pgfqpoint{1.029628in}{1.294778in}}%
\pgfpathlineto{\pgfqpoint{1.002780in}{1.245096in}}%
\pgfpathlineto{\pgfqpoint{0.976019in}{1.195401in}}%
\pgfpathlineto{\pgfqpoint{0.949381in}{1.145684in}}%
\pgfusepath{stroke}%
\end{pgfscope}%
\begin{pgfscope}%
\pgfpathrectangle{\pgfqpoint{0.647939in}{0.492442in}}{\pgfqpoint{4.273799in}{2.331163in}}%
\pgfusepath{clip}%
\pgfsetbuttcap%
\pgfsetroundjoin%
\pgfsetlinewidth{0.803000pt}%
\definecolor{currentstroke}{rgb}{0.501961,0.501961,0.501961}%
\pgfsetstrokecolor{currentstroke}%
\pgfsetdash{}{0pt}%
\pgfpathmoveto{\pgfqpoint{1.036466in}{2.823605in}}%
\pgfpathlineto{\pgfqpoint{1.036466in}{2.823605in}}%
\pgfpathlineto{\pgfqpoint{1.050120in}{2.772340in}}%
\pgfpathlineto{\pgfqpoint{1.063414in}{2.721048in}}%
\pgfpathlineto{\pgfqpoint{1.076306in}{2.669724in}}%
\pgfpathlineto{\pgfqpoint{1.088738in}{2.618367in}}%
\pgfpathlineto{\pgfqpoint{1.100642in}{2.566973in}}%
\pgfpathlineto{\pgfqpoint{1.111950in}{2.515539in}}%
\pgfpathlineto{\pgfqpoint{1.122583in}{2.464062in}}%
\pgfpathlineto{\pgfqpoint{1.132447in}{2.412539in}}%
\pgfpathlineto{\pgfqpoint{1.141436in}{2.360969in}}%
\pgfpathlineto{\pgfqpoint{1.149438in}{2.309351in}}%
\pgfpathlineto{\pgfqpoint{1.156330in}{2.257686in}}%
\pgfpathlineto{\pgfqpoint{1.161975in}{2.205975in}}%
\pgfpathlineto{\pgfqpoint{1.166230in}{2.154225in}}%
\pgfpathlineto{\pgfqpoint{1.168944in}{2.102445in}}%
\pgfpathlineto{\pgfqpoint{1.169969in}{2.050647in}}%
\pgfpathlineto{\pgfqpoint{1.169166in}{1.998848in}}%
\pgfpathlineto{\pgfqpoint{1.166416in}{1.947069in}}%
\pgfpathlineto{\pgfqpoint{1.161630in}{1.895335in}}%
\pgfpathlineto{\pgfqpoint{1.154760in}{1.843672in}}%
\pgfpathlineto{\pgfqpoint{1.145808in}{1.792104in}}%
\pgfpathlineto{\pgfqpoint{1.134820in}{1.740653in}}%
\pgfpathlineto{\pgfqpoint{1.121889in}{1.689336in}}%
\pgfpathlineto{\pgfqpoint{1.107166in}{1.638164in}}%
\pgfpathlineto{\pgfqpoint{1.090825in}{1.587139in}}%
\pgfpathlineto{\pgfqpoint{1.073046in}{1.536257in}}%
\pgfpathlineto{\pgfqpoint{1.054030in}{1.485509in}}%
\pgfpathlineto{\pgfqpoint{1.033962in}{1.434880in}}%
\pgfusepath{stroke}%
\end{pgfscope}%
\begin{pgfscope}%
\pgfpathrectangle{\pgfqpoint{0.647939in}{0.492442in}}{\pgfqpoint{4.273799in}{2.331163in}}%
\pgfusepath{clip}%
\pgfsetbuttcap%
\pgfsetroundjoin%
\pgfsetlinewidth{0.803000pt}%
\definecolor{currentstroke}{rgb}{0.501961,0.501961,0.501961}%
\pgfsetstrokecolor{currentstroke}%
\pgfsetdash{}{0pt}%
\pgfpathmoveto{\pgfqpoint{0.939334in}{2.823605in}}%
\pgfpathlineto{\pgfqpoint{0.939334in}{2.823605in}}%
\pgfpathlineto{\pgfqpoint{0.950819in}{2.772182in}}%
\pgfpathlineto{\pgfqpoint{0.961909in}{2.720733in}}%
\pgfpathlineto{\pgfqpoint{0.972559in}{2.669257in}}%
\pgfpathlineto{\pgfqpoint{0.982728in}{2.617752in}}%
\pgfpathlineto{\pgfqpoint{0.992367in}{2.566216in}}%
\pgfpathlineto{\pgfqpoint{1.001422in}{2.514649in}}%
\pgfpathlineto{\pgfqpoint{1.009831in}{2.463050in}}%
\pgfpathlineto{\pgfqpoint{1.017532in}{2.411417in}}%
\pgfpathlineto{\pgfqpoint{1.024459in}{2.359753in}}%
\pgfpathlineto{\pgfqpoint{1.030542in}{2.308056in}}%
\pgfpathlineto{\pgfqpoint{1.035705in}{2.256330in}}%
\pgfpathlineto{\pgfqpoint{1.039870in}{2.204577in}}%
\pgfpathlineto{\pgfqpoint{1.042957in}{2.152802in}}%
\pgfpathlineto{\pgfqpoint{1.044888in}{2.101010in}}%
\pgfpathlineto{\pgfqpoint{1.045588in}{2.049209in}}%
\pgfpathlineto{\pgfqpoint{1.044988in}{1.997408in}}%
\pgfpathlineto{\pgfqpoint{1.043026in}{1.945617in}}%
\pgfpathlineto{\pgfqpoint{1.039654in}{1.893848in}}%
\pgfpathlineto{\pgfqpoint{1.034837in}{1.842113in}}%
\pgfpathlineto{\pgfqpoint{1.028561in}{1.790424in}}%
\pgfpathlineto{\pgfqpoint{1.020832in}{1.738795in}}%
\pgfpathlineto{\pgfqpoint{1.011674in}{1.687235in}}%
\pgfpathlineto{\pgfqpoint{1.001128in}{1.635755in}}%
\pgfpathlineto{\pgfqpoint{0.989255in}{1.584360in}}%
\pgfpathlineto{\pgfqpoint{0.976139in}{1.533055in}}%
\pgfpathlineto{\pgfqpoint{0.961870in}{1.481843in}}%
\pgfpathlineto{\pgfqpoint{0.946538in}{1.430721in}}%
\pgfpathlineto{\pgfqpoint{0.930254in}{1.379687in}}%
\pgfpathlineto{\pgfqpoint{0.913120in}{1.328737in}}%
\pgfpathlineto{\pgfqpoint{0.895235in}{1.277864in}}%
\pgfpathlineto{\pgfqpoint{0.876699in}{1.227060in}}%
\pgfpathlineto{\pgfqpoint{0.857599in}{1.176318in}}%
\pgfpathlineto{\pgfqpoint{0.838022in}{1.125630in}}%
\pgfpathlineto{\pgfqpoint{0.818042in}{1.074988in}}%
\pgfpathlineto{\pgfqpoint{0.797731in}{1.024386in}}%
\pgfpathlineto{\pgfqpoint{0.777149in}{0.973816in}}%
\pgfpathlineto{\pgfqpoint{0.756352in}{0.923271in}}%
\pgfpathlineto{\pgfqpoint{0.735389in}{0.872748in}}%
\pgfpathlineto{\pgfqpoint{0.714301in}{0.822240in}}%
\pgfpathlineto{\pgfqpoint{0.693126in}{0.771742in}}%
\pgfpathlineto{\pgfqpoint{0.671895in}{0.721252in}}%
\pgfpathlineto{\pgfqpoint{0.650636in}{0.670765in}}%
\pgfpathlineto{\pgfqpoint{0.647939in}{0.664363in}}%
\pgfusepath{stroke}%
\end{pgfscope}%
\begin{pgfscope}%
\pgfpathrectangle{\pgfqpoint{0.647939in}{0.492442in}}{\pgfqpoint{4.273799in}{2.331163in}}%
\pgfusepath{clip}%
\pgfsetbuttcap%
\pgfsetroundjoin%
\pgfsetlinewidth{0.803000pt}%
\definecolor{currentstroke}{rgb}{0.501961,0.501961,0.501961}%
\pgfsetstrokecolor{currentstroke}%
\pgfsetdash{}{0pt}%
\pgfpathmoveto{\pgfqpoint{0.842203in}{2.823605in}}%
\pgfpathlineto{\pgfqpoint{0.842203in}{2.823605in}}%
\pgfpathlineto{\pgfqpoint{0.851951in}{2.772076in}}%
\pgfpathlineto{\pgfqpoint{0.861303in}{2.720524in}}%
\pgfpathlineto{\pgfqpoint{0.870225in}{2.668950in}}%
\pgfpathlineto{\pgfqpoint{0.878681in}{2.617353in}}%
\pgfpathlineto{\pgfqpoint{0.886634in}{2.565731in}}%
\pgfpathlineto{\pgfqpoint{0.894045in}{2.514086in}}%
\pgfpathlineto{\pgfqpoint{0.900874in}{2.462417in}}%
\pgfpathlineto{\pgfqpoint{0.907078in}{2.410725in}}%
\pgfpathlineto{\pgfqpoint{0.912611in}{2.359009in}}%
\pgfpathlineto{\pgfqpoint{0.917424in}{2.307273in}}%
\pgfpathlineto{\pgfqpoint{0.921472in}{2.255517in}}%
\pgfpathlineto{\pgfqpoint{0.924707in}{2.203744in}}%
\pgfpathlineto{\pgfqpoint{0.927082in}{2.151957in}}%
\pgfpathlineto{\pgfqpoint{0.928551in}{2.100160in}}%
\pgfpathlineto{\pgfqpoint{0.929071in}{2.048358in}}%
\pgfpathlineto{\pgfqpoint{0.928601in}{1.996556in}}%
\pgfpathlineto{\pgfqpoint{0.927106in}{1.944760in}}%
\pgfpathlineto{\pgfqpoint{0.924557in}{1.892976in}}%
\pgfpathlineto{\pgfqpoint{0.920931in}{1.841211in}}%
\pgfpathlineto{\pgfqpoint{0.916213in}{1.789473in}}%
\pgfpathlineto{\pgfqpoint{0.910399in}{1.737767in}}%
\pgfpathlineto{\pgfqpoint{0.903493in}{1.686102in}}%
\pgfpathlineto{\pgfqpoint{0.895512in}{1.634483in}}%
\pgfpathlineto{\pgfqpoint{0.886481in}{1.582916in}}%
\pgfpathlineto{\pgfqpoint{0.876431in}{1.531404in}}%
\pgfpathlineto{\pgfqpoint{0.865402in}{1.479953in}}%
\pgfpathlineto{\pgfqpoint{0.853444in}{1.428562in}}%
\pgfpathlineto{\pgfqpoint{0.840615in}{1.377236in}}%
\pgfpathlineto{\pgfqpoint{0.826970in}{1.325971in}}%
\pgfpathlineto{\pgfqpoint{0.812569in}{1.274768in}}%
\pgfusepath{stroke}%
\end{pgfscope}%
\begin{pgfscope}%
\pgfpathrectangle{\pgfqpoint{0.647939in}{0.492442in}}{\pgfqpoint{4.273799in}{2.331163in}}%
\pgfusepath{clip}%
\pgfsetbuttcap%
\pgfsetroundjoin%
\pgfsetlinewidth{0.803000pt}%
\definecolor{currentstroke}{rgb}{0.501961,0.501961,0.501961}%
\pgfsetstrokecolor{currentstroke}%
\pgfsetdash{}{0pt}%
\pgfpathmoveto{\pgfqpoint{0.745071in}{2.823605in}}%
\pgfpathlineto{\pgfqpoint{0.745071in}{2.823605in}}%
\pgfpathlineto{\pgfqpoint{0.753418in}{2.772003in}}%
\pgfpathlineto{\pgfqpoint{0.761383in}{2.720382in}}%
\pgfpathlineto{\pgfqpoint{0.768942in}{2.668743in}}%
\pgfpathlineto{\pgfqpoint{0.776069in}{2.617085in}}%
\pgfpathlineto{\pgfqpoint{0.782736in}{2.565410in}}%
\pgfpathlineto{\pgfqpoint{0.788912in}{2.513716in}}%
\pgfpathlineto{\pgfqpoint{0.794569in}{2.462005in}}%
\pgfpathlineto{\pgfqpoint{0.799676in}{2.410277in}}%
\pgfpathlineto{\pgfqpoint{0.804204in}{2.358532in}}%
\pgfpathlineto{\pgfqpoint{0.808121in}{2.306773in}}%
\pgfpathlineto{\pgfqpoint{0.811396in}{2.255001in}}%
\pgfpathlineto{\pgfqpoint{0.813997in}{2.203217in}}%
\pgfpathlineto{\pgfqpoint{0.815895in}{2.151424in}}%
\pgfpathlineto{\pgfqpoint{0.817061in}{2.099625in}}%
\pgfpathlineto{\pgfqpoint{0.817467in}{2.047822in}}%
\pgfpathlineto{\pgfqpoint{0.817087in}{1.996019in}}%
\pgfpathlineto{\pgfqpoint{0.815901in}{1.944220in}}%
\pgfpathlineto{\pgfqpoint{0.813887in}{1.892429in}}%
\pgfpathlineto{\pgfqpoint{0.811033in}{1.840649in}}%
\pgfpathlineto{\pgfqpoint{0.807326in}{1.788886in}}%
\pgfpathlineto{\pgfqpoint{0.802761in}{1.737143in}}%
\pgfpathlineto{\pgfqpoint{0.797338in}{1.685424in}}%
\pgfpathlineto{\pgfqpoint{0.791059in}{1.633735in}}%
\pgfpathlineto{\pgfqpoint{0.783935in}{1.582078in}}%
\pgfpathlineto{\pgfqpoint{0.775979in}{1.530457in}}%
\pgfpathlineto{\pgfqpoint{0.767213in}{1.478876in}}%
\pgfpathlineto{\pgfqpoint{0.757664in}{1.427336in}}%
\pgfpathlineto{\pgfqpoint{0.747359in}{1.375839in}}%
\pgfpathlineto{\pgfqpoint{0.736328in}{1.324387in}}%
\pgfpathlineto{\pgfqpoint{0.724608in}{1.272980in}}%
\pgfpathlineto{\pgfqpoint{0.712240in}{1.221619in}}%
\pgfpathlineto{\pgfqpoint{0.699260in}{1.170303in}}%
\pgfpathlineto{\pgfqpoint{0.685708in}{1.119030in}}%
\pgfpathlineto{\pgfqpoint{0.671628in}{1.067800in}}%
\pgfpathlineto{\pgfqpoint{0.657060in}{1.016611in}}%
\pgfpathlineto{\pgfqpoint{0.647939in}{0.985076in}}%
\pgfusepath{stroke}%
\end{pgfscope}%
\begin{pgfscope}%
\pgfpathrectangle{\pgfqpoint{0.647939in}{0.492442in}}{\pgfqpoint{4.273799in}{2.331163in}}%
\pgfusepath{clip}%
\pgfsetbuttcap%
\pgfsetroundjoin%
\pgfsetlinewidth{0.803000pt}%
\definecolor{currentstroke}{rgb}{0.501961,0.501961,0.501961}%
\pgfsetstrokecolor{currentstroke}%
\pgfsetdash{}{0pt}%
\pgfpathmoveto{\pgfqpoint{0.647939in}{2.823605in}}%
\pgfpathlineto{\pgfqpoint{0.647939in}{2.823605in}}%
\pgfpathlineto{\pgfqpoint{0.655144in}{2.771951in}}%
\pgfpathlineto{\pgfqpoint{0.661990in}{2.720283in}}%
\pgfpathlineto{\pgfqpoint{0.668460in}{2.668600in}}%
\pgfpathlineto{\pgfqpoint{0.674534in}{2.616902in}}%
\pgfpathlineto{\pgfqpoint{0.680191in}{2.565191in}}%
\pgfpathlineto{\pgfqpoint{0.685410in}{2.513466in}}%
\pgfpathlineto{\pgfqpoint{0.690172in}{2.461727in}}%
\pgfpathlineto{\pgfqpoint{0.694453in}{2.409977in}}%
\pgfpathlineto{\pgfqpoint{0.698231in}{2.358214in}}%
\pgfpathlineto{\pgfqpoint{0.701485in}{2.306441in}}%
\pgfpathlineto{\pgfqpoint{0.704194in}{2.254659in}}%
\pgfpathlineto{\pgfqpoint{0.706337in}{2.202869in}}%
\pgfpathlineto{\pgfqpoint{0.707894in}{2.151072in}}%
\pgfpathlineto{\pgfqpoint{0.708845in}{2.099272in}}%
\pgfpathlineto{\pgfqpoint{0.709173in}{2.047469in}}%
\pgfpathlineto{\pgfqpoint{0.708860in}{1.995666in}}%
\pgfpathlineto{\pgfqpoint{0.707892in}{1.943865in}}%
\pgfpathlineto{\pgfqpoint{0.706255in}{1.892070in}}%
\pgfpathlineto{\pgfqpoint{0.703940in}{1.840282in}}%
\pgfpathlineto{\pgfqpoint{0.700938in}{1.788505in}}%
\pgfpathlineto{\pgfqpoint{0.697243in}{1.736741in}}%
\pgfpathlineto{\pgfqpoint{0.692853in}{1.684993in}}%
\pgfpathlineto{\pgfqpoint{0.687768in}{1.633264in}}%
\pgfpathlineto{\pgfqpoint{0.681993in}{1.581557in}}%
\pgfpathlineto{\pgfqpoint{0.675536in}{1.529873in}}%
\pgfpathlineto{\pgfqpoint{0.668405in}{1.478217in}}%
\pgfpathlineto{\pgfqpoint{0.660614in}{1.426588in}}%
\pgfpathlineto{\pgfqpoint{0.652177in}{1.374990in}}%
\pgfpathlineto{\pgfqpoint{0.647939in}{1.350013in}}%
\pgfusepath{stroke}%
\end{pgfscope}%
\begin{pgfscope}%
\pgfpathrectangle{\pgfqpoint{0.647939in}{0.492442in}}{\pgfqpoint{4.273799in}{2.331163in}}%
\pgfusepath{clip}%
\pgfsetbuttcap%
\pgfsetroundjoin%
\pgfsetlinewidth{0.803000pt}%
\definecolor{currentstroke}{rgb}{0.501961,0.501961,0.501961}%
\pgfsetstrokecolor{currentstroke}%
\pgfsetdash{}{0pt}%
\pgfpathmoveto{\pgfqpoint{0.647939in}{2.346777in}}%
\pgfpathlineto{\pgfqpoint{0.647939in}{2.346777in}}%
\pgfpathlineto{\pgfqpoint{0.650822in}{2.294997in}}%
\pgfpathlineto{\pgfqpoint{0.653195in}{2.243210in}}%
\pgfpathlineto{\pgfqpoint{0.655042in}{2.191416in}}%
\pgfpathlineto{\pgfqpoint{0.656346in}{2.139618in}}%
\pgfpathlineto{\pgfqpoint{0.657091in}{2.087816in}}%
\pgfpathlineto{\pgfqpoint{0.657262in}{2.036013in}}%
\pgfpathlineto{\pgfqpoint{0.656846in}{1.984210in}}%
\pgfpathlineto{\pgfqpoint{0.655830in}{1.932409in}}%
\pgfpathlineto{\pgfqpoint{0.654204in}{1.880614in}}%
\pgfpathlineto{\pgfqpoint{0.651960in}{1.828825in}}%
\pgfpathlineto{\pgfqpoint{0.649090in}{1.777045in}}%
\pgfpathlineto{\pgfqpoint{0.647939in}{1.758323in}}%
\pgfusepath{stroke}%
\end{pgfscope}%
\begin{pgfscope}%
\pgfpathrectangle{\pgfqpoint{0.647939in}{0.492442in}}{\pgfqpoint{4.273799in}{2.331163in}}%
\pgfusepath{clip}%
\pgfsetbuttcap%
\pgfsetroundjoin%
\pgfsetlinewidth{0.803000pt}%
\definecolor{currentstroke}{rgb}{0.501961,0.501961,0.501961}%
\pgfsetstrokecolor{currentstroke}%
\pgfsetdash{}{0pt}%
\pgfpathmoveto{\pgfqpoint{0.729905in}{1.118427in}}%
\pgfpathlineto{\pgfqpoint{0.714370in}{1.067322in}}%
\pgfpathlineto{\pgfqpoint{0.698360in}{1.016262in}}%
\pgfpathlineto{\pgfqpoint{0.681921in}{0.965242in}}%
\pgfpathlineto{\pgfqpoint{0.665101in}{0.914259in}}%
\pgfpathlineto{\pgfqpoint{0.647939in}{0.863309in}}%
\pgfpathlineto{\pgfqpoint{0.647939in}{0.863309in}}%
\pgfusepath{stroke}%
\end{pgfscope}%
\begin{pgfscope}%
\pgfpathrectangle{\pgfqpoint{0.647939in}{0.492442in}}{\pgfqpoint{4.273799in}{2.331163in}}%
\pgfusepath{clip}%
\pgfsetbuttcap%
\pgfsetroundjoin%
\pgfsetlinewidth{0.803000pt}%
\definecolor{currentstroke}{rgb}{0.501961,0.501961,0.501961}%
\pgfsetstrokecolor{currentstroke}%
\pgfsetdash{}{0pt}%
\pgfpathmoveto{\pgfqpoint{1.498149in}{0.492442in}}%
\pgfpathlineto{\pgfqpoint{1.482700in}{0.504341in}}%
\pgfpathlineto{\pgfqpoint{1.424993in}{0.545423in}}%
\pgfpathlineto{\pgfqpoint{1.360060in}{0.583057in}}%
\pgfpathlineto{\pgfqpoint{1.284365in}{0.613735in}}%
\pgfpathlineto{\pgfqpoint{1.284365in}{0.613735in}}%
\pgfpathlineto{\pgfqpoint{1.223152in}{0.627044in}}%
\pgfpathlineto{\pgfqpoint{1.156370in}{0.628793in}}%
\pgfusepath{stroke}%
\end{pgfscope}%
\begin{pgfscope}%
\pgfpathrectangle{\pgfqpoint{0.647939in}{0.492442in}}{\pgfqpoint{4.273799in}{2.331163in}}%
\pgfusepath{clip}%
\pgfsetbuttcap%
\pgfsetroundjoin%
\pgfsetlinewidth{0.803000pt}%
\definecolor{currentstroke}{rgb}{0.501961,0.501961,0.501961}%
\pgfsetstrokecolor{currentstroke}%
\pgfsetdash{}{0pt}%
\pgfpathmoveto{\pgfqpoint{1.572586in}{0.554542in}}%
\pgfpathlineto{\pgfqpoint{1.522125in}{0.598404in}}%
\pgfpathlineto{\pgfqpoint{1.467380in}{0.640698in}}%
\pgfpathlineto{\pgfqpoint{1.406347in}{0.680274in}}%
\pgfpathlineto{\pgfqpoint{1.335751in}{0.714583in}}%
\pgfpathlineto{\pgfqpoint{1.335751in}{0.714583in}}%
\pgfpathlineto{\pgfqpoint{1.272944in}{0.733502in}}%
\pgfpathlineto{\pgfqpoint{1.272944in}{0.733502in}}%
\pgfpathlineto{\pgfqpoint{1.215483in}{0.740343in}}%
\pgfpathlineto{\pgfqpoint{1.156753in}{0.736413in}}%
\pgfusepath{stroke}%
\end{pgfscope}%
\begin{pgfscope}%
\pgfpathrectangle{\pgfqpoint{0.647939in}{0.492442in}}{\pgfqpoint{4.273799in}{2.331163in}}%
\pgfusepath{clip}%
\pgfsetbuttcap%
\pgfsetroundjoin%
\pgfsetlinewidth{0.803000pt}%
\definecolor{currentstroke}{rgb}{0.501961,0.501961,0.501961}%
\pgfsetstrokecolor{currentstroke}%
\pgfsetdash{}{0pt}%
\pgfpathmoveto{\pgfqpoint{1.602104in}{2.823605in}}%
\pgfpathlineto{\pgfqpoint{1.618716in}{2.806327in}}%
\pgfpathlineto{\pgfqpoint{1.664951in}{2.761119in}}%
\pgfpathlineto{\pgfqpoint{1.716389in}{2.717643in}}%
\pgfpathlineto{\pgfqpoint{1.774865in}{2.676926in}}%
\pgfpathlineto{\pgfqpoint{1.842503in}{2.640813in}}%
\pgfpathlineto{\pgfqpoint{1.920900in}{2.612152in}}%
\pgfpathlineto{\pgfqpoint{2.006744in}{2.593752in}}%
\pgfpathlineto{\pgfqpoint{2.093313in}{2.584546in}}%
\pgfusepath{stroke}%
\end{pgfscope}%
\begin{pgfscope}%
\pgfpathrectangle{\pgfqpoint{0.647939in}{0.492442in}}{\pgfqpoint{4.273799in}{2.331163in}}%
\pgfusepath{clip}%
\pgfsetbuttcap%
\pgfsetroundjoin%
\pgfsetlinewidth{0.803000pt}%
\definecolor{currentstroke}{rgb}{0.501961,0.501961,0.501961}%
\pgfsetstrokecolor{currentstroke}%
\pgfsetdash{}{0pt}%
\pgfpathmoveto{\pgfqpoint{3.950420in}{0.651385in}}%
\pgfpathlineto{\pgfqpoint{3.889444in}{0.691096in}}%
\pgfpathlineto{\pgfqpoint{3.826929in}{0.730091in}}%
\pgfpathlineto{\pgfqpoint{3.763242in}{0.768519in}}%
\pgfpathlineto{\pgfqpoint{3.698794in}{0.806569in}}%
\pgfpathlineto{\pgfqpoint{3.634038in}{0.844463in}}%
\pgfpathlineto{\pgfqpoint{3.569429in}{0.882430in}}%
\pgfpathlineto{\pgfqpoint{3.505405in}{0.920691in}}%
\pgfpathlineto{\pgfqpoint{3.442366in}{0.959432in}}%
\pgfusepath{stroke}%
\end{pgfscope}%
\begin{pgfscope}%
\pgfpathrectangle{\pgfqpoint{0.647939in}{0.492442in}}{\pgfqpoint{4.273799in}{2.331163in}}%
\pgfusepath{clip}%
\pgfsetbuttcap%
\pgfsetroundjoin%
\pgfsetlinewidth{0.803000pt}%
\definecolor{currentstroke}{rgb}{0.501961,0.501961,0.501961}%
\pgfsetstrokecolor{currentstroke}%
\pgfsetdash{}{0pt}%
\pgfpathmoveto{\pgfqpoint{4.648150in}{1.554161in}}%
\pgfpathlineto{\pgfqpoint{4.630343in}{1.605043in}}%
\pgfpathlineto{\pgfqpoint{4.612995in}{1.655972in}}%
\pgfpathlineto{\pgfqpoint{4.596260in}{1.706962in}}%
\pgfpathlineto{\pgfqpoint{4.580334in}{1.758029in}}%
\pgfpathlineto{\pgfqpoint{4.565487in}{1.809192in}}%
\pgfpathlineto{\pgfqpoint{4.552100in}{1.860473in}}%
\pgfpathlineto{\pgfqpoint{4.540681in}{1.911894in}}%
\pgfpathlineto{\pgfqpoint{4.531930in}{1.963468in}}%
\pgfpathlineto{\pgfqpoint{4.526789in}{2.015181in}}%
\pgfpathlineto{\pgfqpoint{4.526383in}{2.066961in}}%
\pgfpathlineto{\pgfqpoint{4.531794in}{2.118648in}}%
\pgfpathlineto{\pgfqpoint{4.543601in}{2.170006in}}%
\pgfusepath{stroke}%
\end{pgfscope}%
\begin{pgfscope}%
\pgfpathrectangle{\pgfqpoint{0.647939in}{0.492442in}}{\pgfqpoint{4.273799in}{2.331163in}}%
\pgfusepath{clip}%
\pgfsetbuttcap%
\pgfsetroundjoin%
\pgfsetlinewidth{0.803000pt}%
\definecolor{currentstroke}{rgb}{0.501961,0.501961,0.501961}%
\pgfsetstrokecolor{currentstroke}%
\pgfsetdash{}{0pt}%
\pgfpathmoveto{\pgfqpoint{3.783584in}{0.626147in}}%
\pgfpathlineto{\pgfqpoint{3.721493in}{0.665346in}}%
\pgfpathlineto{\pgfqpoint{3.659025in}{0.704366in}}%
\pgfpathlineto{\pgfqpoint{3.596556in}{0.743386in}}%
\pgfpathlineto{\pgfqpoint{3.534446in}{0.782575in}}%
\pgfpathlineto{\pgfqpoint{3.473039in}{0.822090in}}%
\pgfpathlineto{\pgfqpoint{3.412633in}{0.862062in}}%
\pgfusepath{stroke}%
\end{pgfscope}%
\begin{pgfscope}%
\pgfpathrectangle{\pgfqpoint{0.647939in}{0.492442in}}{\pgfqpoint{4.273799in}{2.331163in}}%
\pgfusepath{clip}%
\pgfsetbuttcap%
\pgfsetroundjoin%
\pgfsetlinewidth{0.803000pt}%
\definecolor{currentstroke}{rgb}{0.501961,0.501961,0.501961}%
\pgfsetstrokecolor{currentstroke}%
\pgfsetdash{}{0pt}%
\pgfpathmoveto{\pgfqpoint{4.241816in}{0.704366in}}%
\pgfpathlineto{\pgfqpoint{4.190768in}{0.748038in}}%
\pgfpathlineto{\pgfqpoint{4.136901in}{0.790692in}}%
\pgfpathlineto{\pgfqpoint{4.080094in}{0.832191in}}%
\pgfpathlineto{\pgfqpoint{4.020269in}{0.872407in}}%
\pgfpathlineto{\pgfqpoint{3.957487in}{0.911261in}}%
\pgfpathlineto{\pgfqpoint{3.891985in}{0.948758in}}%
\pgfpathlineto{\pgfqpoint{3.824164in}{0.985012in}}%
\pgfpathlineto{\pgfqpoint{3.754579in}{1.020263in}}%
\pgfpathlineto{\pgfqpoint{3.683922in}{1.054876in}}%
\pgfpathlineto{\pgfqpoint{3.612944in}{1.089294in}}%
\pgfpathlineto{\pgfqpoint{3.542357in}{1.123949in}}%
\pgfpathlineto{\pgfqpoint{3.472861in}{1.159247in}}%
\pgfpathlineto{\pgfqpoint{3.405051in}{1.195502in}}%
\pgfpathlineto{\pgfqpoint{3.339399in}{1.232916in}}%
\pgfpathlineto{\pgfqpoint{3.276267in}{1.271596in}}%
\pgfpathlineto{\pgfqpoint{3.215898in}{1.311566in}}%
\pgfpathlineto{\pgfqpoint{3.158438in}{1.352791in}}%
\pgfpathlineto{\pgfqpoint{3.103953in}{1.395203in}}%
\pgfusepath{stroke}%
\end{pgfscope}%
\begin{pgfscope}%
\pgfpathrectangle{\pgfqpoint{0.647939in}{0.492442in}}{\pgfqpoint{4.273799in}{2.331163in}}%
\pgfusepath{clip}%
\pgfsetbuttcap%
\pgfsetroundjoin%
\pgfsetlinewidth{0.803000pt}%
\definecolor{currentstroke}{rgb}{0.501961,0.501961,0.501961}%
\pgfsetstrokecolor{currentstroke}%
\pgfsetdash{}{0pt}%
\pgfpathmoveto{\pgfqpoint{4.563865in}{1.079186in}}%
\pgfpathlineto{\pgfqpoint{4.533211in}{1.128214in}}%
\pgfpathlineto{\pgfqpoint{4.501270in}{1.176996in}}%
\pgfpathlineto{\pgfqpoint{4.467805in}{1.225468in}}%
\pgfpathlineto{\pgfqpoint{4.432469in}{1.273541in}}%
\pgfpathlineto{\pgfqpoint{4.394783in}{1.321077in}}%
\pgfpathlineto{\pgfqpoint{4.354070in}{1.367861in}}%
\pgfpathlineto{\pgfqpoint{4.309336in}{1.413527in}}%
\pgfpathlineto{\pgfqpoint{4.259058in}{1.457406in}}%
\pgfpathlineto{\pgfqpoint{4.200859in}{1.498171in}}%
\pgfpathlineto{\pgfqpoint{4.131375in}{1.533051in}}%
\pgfpathlineto{\pgfqpoint{4.131375in}{1.533051in}}%
\pgfpathlineto{\pgfqpoint{4.064176in}{1.553988in}}%
\pgfpathlineto{\pgfqpoint{3.988854in}{1.565373in}}%
\pgfpathlineto{\pgfqpoint{3.913793in}{1.567866in}}%
\pgfpathlineto{\pgfqpoint{3.826959in}{1.564624in}}%
\pgfpathlineto{\pgfqpoint{3.732652in}{1.558978in}}%
\pgfpathlineto{\pgfqpoint{3.638041in}{1.555439in}}%
\pgfpathlineto{\pgfqpoint{3.543403in}{1.557210in}}%
\pgfusepath{stroke}%
\end{pgfscope}%
\begin{pgfscope}%
\pgfpathrectangle{\pgfqpoint{0.647939in}{0.492442in}}{\pgfqpoint{4.273799in}{2.331163in}}%
\pgfusepath{clip}%
\pgfsetbuttcap%
\pgfsetroundjoin%
\pgfsetlinewidth{0.803000pt}%
\definecolor{currentstroke}{rgb}{0.501961,0.501961,0.501961}%
\pgfsetstrokecolor{currentstroke}%
\pgfsetdash{}{0pt}%
\pgfpathmoveto{\pgfqpoint{4.559503in}{1.396324in}}%
\pgfpathlineto{\pgfqpoint{4.533211in}{1.446100in}}%
\pgfpathlineto{\pgfqpoint{4.506101in}{1.495743in}}%
\pgfpathlineto{\pgfqpoint{4.477991in}{1.545217in}}%
\pgfpathlineto{\pgfqpoint{4.448564in}{1.594464in}}%
\pgfpathlineto{\pgfqpoint{4.417370in}{1.643380in}}%
\pgfpathlineto{\pgfqpoint{4.383622in}{1.691769in}}%
\pgfpathlineto{\pgfqpoint{4.345690in}{1.739190in}}%
\pgfpathlineto{\pgfqpoint{4.299698in}{1.784272in}}%
\pgfpathlineto{\pgfqpoint{4.299698in}{1.784272in}}%
\pgfpathlineto{\pgfqpoint{4.257148in}{1.811483in}}%
\pgfpathlineto{\pgfqpoint{4.257148in}{1.811483in}}%
\pgfpathlineto{\pgfqpoint{4.220181in}{1.823007in}}%
\pgfpathlineto{\pgfqpoint{4.220181in}{1.823007in}}%
\pgfpathlineto{\pgfqpoint{4.182643in}{1.824187in}}%
\pgfpathlineto{\pgfqpoint{4.146459in}{1.817593in}}%
\pgfpathlineto{\pgfqpoint{4.107252in}{1.804613in}}%
\pgfpathlineto{\pgfqpoint{4.055261in}{1.782148in}}%
\pgfusepath{stroke}%
\end{pgfscope}%
\begin{pgfscope}%
\pgfpathrectangle{\pgfqpoint{0.647939in}{0.492442in}}{\pgfqpoint{4.273799in}{2.331163in}}%
\pgfusepath{clip}%
\pgfsetbuttcap%
\pgfsetroundjoin%
\pgfsetlinewidth{0.803000pt}%
\definecolor{currentstroke}{rgb}{0.501961,0.501961,0.501961}%
\pgfsetstrokecolor{currentstroke}%
\pgfsetdash{}{0pt}%
\pgfpathmoveto{\pgfqpoint{4.256457in}{2.613173in}}%
\pgfpathlineto{\pgfqpoint{4.296775in}{2.566298in}}%
\pgfpathlineto{\pgfqpoint{4.341711in}{2.520743in}}%
\pgfpathlineto{\pgfqpoint{4.382433in}{2.487315in}}%
\pgfpathlineto{\pgfqpoint{4.417469in}{2.465866in}}%
\pgfpathlineto{\pgfqpoint{4.450055in}{2.452854in}}%
\pgfpathlineto{\pgfqpoint{4.491249in}{2.446824in}}%
\pgfpathlineto{\pgfqpoint{4.533211in}{2.452738in}}%
\pgfpathlineto{\pgfqpoint{4.533211in}{2.452738in}}%
\pgfpathlineto{\pgfqpoint{4.533211in}{2.452738in}}%
\pgfpathlineto{\pgfqpoint{4.578180in}{2.471698in}}%
\pgfusepath{stroke}%
\end{pgfscope}%
\begin{pgfscope}%
\pgfpathrectangle{\pgfqpoint{0.647939in}{0.492442in}}{\pgfqpoint{4.273799in}{2.331163in}}%
\pgfusepath{clip}%
\pgfsetbuttcap%
\pgfsetroundjoin%
\pgfsetlinewidth{0.803000pt}%
\definecolor{currentstroke}{rgb}{0.501961,0.501961,0.501961}%
\pgfsetstrokecolor{currentstroke}%
\pgfsetdash{}{0pt}%
\pgfpathmoveto{\pgfqpoint{2.687707in}{0.757347in}}%
\pgfpathlineto{\pgfqpoint{2.649268in}{0.804717in}}%
\pgfpathlineto{\pgfqpoint{2.611711in}{0.852297in}}%
\pgfpathlineto{\pgfqpoint{2.575023in}{0.900078in}}%
\pgfpathlineto{\pgfqpoint{2.539191in}{0.948052in}}%
\pgfpathlineto{\pgfqpoint{2.504204in}{0.996212in}}%
\pgfpathlineto{\pgfqpoint{2.470052in}{1.044549in}}%
\pgfpathlineto{\pgfqpoint{2.436722in}{1.093057in}}%
\pgfpathlineto{\pgfqpoint{2.404215in}{1.141731in}}%
\pgfpathlineto{\pgfqpoint{2.372535in}{1.190567in}}%
\pgfpathlineto{\pgfqpoint{2.341686in}{1.239561in}}%
\pgfpathlineto{\pgfqpoint{2.311672in}{1.288709in}}%
\pgfpathlineto{\pgfqpoint{2.282517in}{1.338011in}}%
\pgfpathlineto{\pgfqpoint{2.254247in}{1.387466in}}%
\pgfpathlineto{\pgfqpoint{2.226891in}{1.437073in}}%
\pgfpathlineto{\pgfqpoint{2.200502in}{1.486836in}}%
\pgfpathlineto{\pgfqpoint{2.175139in}{1.536757in}}%
\pgfpathlineto{\pgfqpoint{2.150878in}{1.586841in}}%
\pgfpathlineto{\pgfqpoint{2.127829in}{1.637094in}}%
\pgfpathlineto{\pgfqpoint{2.106125in}{1.687525in}}%
\pgfpathlineto{\pgfqpoint{2.085944in}{1.738143in}}%
\pgfpathlineto{\pgfqpoint{2.067518in}{1.788959in}}%
\pgfpathlineto{\pgfqpoint{2.051150in}{1.839984in}}%
\pgfpathlineto{\pgfqpoint{2.037242in}{1.891222in}}%
\pgfpathlineto{\pgfqpoint{2.026331in}{1.942673in}}%
\pgfpathlineto{\pgfqpoint{2.019115in}{1.994313in}}%
\pgfpathlineto{\pgfqpoint{2.016505in}{2.046077in}}%
\pgfpathlineto{\pgfqpoint{2.019651in}{2.097820in}}%
\pgfpathlineto{\pgfqpoint{2.029892in}{2.149268in}}%
\pgfpathlineto{\pgfqpoint{2.048655in}{2.199968in}}%
\pgfpathlineto{\pgfqpoint{2.077176in}{2.249264in}}%
\pgfpathlineto{\pgfqpoint{2.116321in}{2.296304in}}%
\pgfpathlineto{\pgfqpoint{2.166478in}{2.340079in}}%
\pgfusepath{stroke}%
\end{pgfscope}%
\begin{pgfscope}%
\pgfpathrectangle{\pgfqpoint{0.647939in}{0.492442in}}{\pgfqpoint{4.273799in}{2.331163in}}%
\pgfusepath{clip}%
\pgfsetbuttcap%
\pgfsetroundjoin%
\pgfsetlinewidth{0.803000pt}%
\definecolor{currentstroke}{rgb}{0.501961,0.501961,0.501961}%
\pgfsetstrokecolor{currentstroke}%
\pgfsetdash{}{0pt}%
\pgfpathmoveto{\pgfqpoint{4.338948in}{0.916290in}}%
\pgfpathlineto{\pgfqpoint{4.292950in}{0.961596in}}%
\pgfpathlineto{\pgfqpoint{4.243788in}{1.005898in}}%
\pgfpathlineto{\pgfqpoint{4.190952in}{1.048920in}}%
\pgfpathlineto{\pgfqpoint{4.133921in}{1.090312in}}%
\pgfpathlineto{\pgfqpoint{4.072249in}{1.129660in}}%
\pgfpathlineto{\pgfqpoint{4.005610in}{1.166511in}}%
\pgfusepath{stroke}%
\end{pgfscope}%
\begin{pgfscope}%
\pgfpathrectangle{\pgfqpoint{0.647939in}{0.492442in}}{\pgfqpoint{4.273799in}{2.331163in}}%
\pgfusepath{clip}%
\pgfsetbuttcap%
\pgfsetroundjoin%
\pgfsetlinewidth{0.803000pt}%
\definecolor{currentstroke}{rgb}{0.501961,0.501961,0.501961}%
\pgfsetstrokecolor{currentstroke}%
\pgfsetdash{}{0pt}%
\pgfpathmoveto{\pgfqpoint{4.338948in}{1.128214in}}%
\pgfpathlineto{\pgfqpoint{4.292815in}{1.173470in}}%
\pgfpathlineto{\pgfqpoint{4.242600in}{1.217403in}}%
\pgfpathlineto{\pgfqpoint{4.187294in}{1.259468in}}%
\pgfpathlineto{\pgfqpoint{4.125788in}{1.298857in}}%
\pgfpathlineto{\pgfqpoint{4.057022in}{1.334442in}}%
\pgfpathlineto{\pgfqpoint{3.980557in}{1.364955in}}%
\pgfpathlineto{\pgfqpoint{3.897346in}{1.389661in}}%
\pgfpathlineto{\pgfqpoint{3.809493in}{1.409155in}}%
\pgfpathlineto{\pgfqpoint{3.719378in}{1.425438in}}%
\pgfpathlineto{\pgfqpoint{3.628898in}{1.441126in}}%
\pgfpathlineto{\pgfqpoint{3.539570in}{1.458615in}}%
\pgfpathlineto{\pgfqpoint{3.452859in}{1.479588in}}%
\pgfusepath{stroke}%
\end{pgfscope}%
\begin{pgfscope}%
\pgfpathrectangle{\pgfqpoint{0.647939in}{0.492442in}}{\pgfqpoint{4.273799in}{2.331163in}}%
\pgfusepath{clip}%
\pgfsetbuttcap%
\pgfsetroundjoin%
\pgfsetlinewidth{0.803000pt}%
\definecolor{currentstroke}{rgb}{0.501961,0.501961,0.501961}%
\pgfsetstrokecolor{currentstroke}%
\pgfsetdash{}{0pt}%
\pgfpathmoveto{\pgfqpoint{1.230730in}{2.240815in}}%
\pgfpathlineto{\pgfqpoint{1.237120in}{2.189131in}}%
\pgfpathlineto{\pgfqpoint{1.241801in}{2.137393in}}%
\pgfpathlineto{\pgfqpoint{1.244539in}{2.085614in}}%
\pgfpathlineto{\pgfqpoint{1.245100in}{2.033816in}}%
\pgfpathlineto{\pgfqpoint{1.243265in}{1.982028in}}%
\pgfpathlineto{\pgfqpoint{1.238862in}{1.930286in}}%
\pgfpathlineto{\pgfqpoint{1.231785in}{1.878633in}}%
\pgfusepath{stroke}%
\end{pgfscope}%
\begin{pgfscope}%
\pgfpathrectangle{\pgfqpoint{0.647939in}{0.492442in}}{\pgfqpoint{4.273799in}{2.331163in}}%
\pgfusepath{clip}%
\pgfsetbuttcap%
\pgfsetroundjoin%
\pgfsetlinewidth{0.803000pt}%
\definecolor{currentstroke}{rgb}{0.501961,0.501961,0.501961}%
\pgfsetstrokecolor{currentstroke}%
\pgfsetdash{}{0pt}%
\pgfpathmoveto{\pgfqpoint{1.715422in}{1.087795in}}%
\pgfpathlineto{\pgfqpoint{1.673840in}{1.134361in}}%
\pgfpathlineto{\pgfqpoint{1.629659in}{1.180198in}}%
\pgfpathlineto{\pgfqpoint{1.581566in}{1.224822in}}%
\pgfpathlineto{\pgfqpoint{1.528093in}{1.266524in}}%
\pgfpathlineto{\pgfqpoint{1.481453in}{1.294905in}}%
\pgfpathlineto{\pgfqpoint{1.438794in}{1.313407in}}%
\pgfpathlineto{\pgfqpoint{1.395078in}{1.324259in}}%
\pgfpathlineto{\pgfqpoint{1.340431in}{1.325602in}}%
\pgfpathlineto{\pgfqpoint{1.289352in}{1.314441in}}%
\pgfpathlineto{\pgfqpoint{1.289352in}{1.314441in}}%
\pgfpathlineto{\pgfqpoint{1.230730in}{1.287157in}}%
\pgfpathlineto{\pgfqpoint{1.230730in}{1.287157in}}%
\pgfusepath{stroke}%
\end{pgfscope}%
\begin{pgfscope}%
\pgfpathrectangle{\pgfqpoint{0.647939in}{0.492442in}}{\pgfqpoint{4.273799in}{2.331163in}}%
\pgfusepath{clip}%
\pgfsetbuttcap%
\pgfsetroundjoin%
\pgfsetlinewidth{0.803000pt}%
\definecolor{currentstroke}{rgb}{0.501961,0.501961,0.501961}%
\pgfsetstrokecolor{currentstroke}%
\pgfsetdash{}{0pt}%
\pgfpathmoveto{\pgfqpoint{2.367996in}{0.766741in}}%
\pgfpathlineto{\pgfqpoint{2.333328in}{0.814970in}}%
\pgfpathlineto{\pgfqpoint{2.299180in}{0.863309in}}%
\pgfpathlineto{\pgfqpoint{2.265528in}{0.911751in}}%
\pgfpathlineto{\pgfqpoint{2.232358in}{0.960293in}}%
\pgfpathlineto{\pgfqpoint{2.199654in}{1.008928in}}%
\pgfpathlineto{\pgfqpoint{2.167395in}{1.057651in}}%
\pgfpathlineto{\pgfqpoint{2.135556in}{1.106457in}}%
\pgfpathlineto{\pgfqpoint{2.104122in}{1.155341in}}%
\pgfpathlineto{\pgfqpoint{2.073079in}{1.204299in}}%
\pgfusepath{stroke}%
\end{pgfscope}%
\begin{pgfscope}%
\pgfpathrectangle{\pgfqpoint{0.647939in}{0.492442in}}{\pgfqpoint{4.273799in}{2.331163in}}%
\pgfusepath{clip}%
\pgfsetbuttcap%
\pgfsetroundjoin%
\pgfsetlinewidth{0.803000pt}%
\definecolor{currentstroke}{rgb}{0.501961,0.501961,0.501961}%
\pgfsetstrokecolor{currentstroke}%
\pgfsetdash{}{0pt}%
\pgfpathmoveto{\pgfqpoint{3.950420in}{0.863309in}}%
\pgfpathlineto{\pgfqpoint{3.885794in}{0.901258in}}%
\pgfpathlineto{\pgfqpoint{3.819080in}{0.938121in}}%
\pgfpathlineto{\pgfqpoint{3.750806in}{0.974127in}}%
\pgfpathlineto{\pgfqpoint{3.681579in}{1.009591in}}%
\pgfpathlineto{\pgfqpoint{3.612075in}{1.044893in}}%
\pgfpathlineto{\pgfqpoint{3.542939in}{1.080408in}}%
\pgfpathlineto{\pgfqpoint{3.474803in}{1.116489in}}%
\pgfpathlineto{\pgfqpoint{3.408205in}{1.153410in}}%
\pgfusepath{stroke}%
\end{pgfscope}%
\begin{pgfscope}%
\pgfpathrectangle{\pgfqpoint{0.647939in}{0.492442in}}{\pgfqpoint{4.273799in}{2.331163in}}%
\pgfusepath{clip}%
\pgfsetbuttcap%
\pgfsetroundjoin%
\pgfsetlinewidth{0.803000pt}%
\definecolor{currentstroke}{rgb}{0.501961,0.501961,0.501961}%
\pgfsetstrokecolor{currentstroke}%
\pgfsetdash{}{0pt}%
\pgfpathmoveto{\pgfqpoint{4.341667in}{1.464303in}}%
\pgfpathlineto{\pgfqpoint{4.295444in}{1.509482in}}%
\pgfpathlineto{\pgfqpoint{4.241816in}{1.552062in}}%
\pgfpathlineto{\pgfqpoint{4.176644in}{1.589342in}}%
\pgfpathlineto{\pgfqpoint{4.176644in}{1.589342in}}%
\pgfpathlineto{\pgfqpoint{4.116995in}{1.610350in}}%
\pgfpathlineto{\pgfqpoint{4.047304in}{1.621278in}}%
\pgfpathlineto{\pgfqpoint{3.982669in}{1.621727in}}%
\pgfpathlineto{\pgfqpoint{3.912252in}{1.615481in}}%
\pgfpathlineto{\pgfqpoint{3.820754in}{1.602685in}}%
\pgfusepath{stroke}%
\end{pgfscope}%
\begin{pgfscope}%
\pgfpathrectangle{\pgfqpoint{0.647939in}{0.492442in}}{\pgfqpoint{4.273799in}{2.331163in}}%
\pgfusepath{clip}%
\pgfsetbuttcap%
\pgfsetroundjoin%
\pgfsetlinewidth{0.803000pt}%
\definecolor{currentstroke}{rgb}{0.501961,0.501961,0.501961}%
\pgfsetstrokecolor{currentstroke}%
\pgfsetdash{}{0pt}%
\pgfpathmoveto{\pgfqpoint{3.367630in}{2.452738in}}%
\pgfpathlineto{\pgfqpoint{3.393076in}{2.402833in}}%
\pgfpathlineto{\pgfqpoint{3.416333in}{2.352612in}}%
\pgfpathlineto{\pgfqpoint{3.437194in}{2.302079in}}%
\pgfpathlineto{\pgfqpoint{3.455395in}{2.251242in}}%
\pgfpathlineto{\pgfqpoint{3.470584in}{2.200115in}}%
\pgfpathlineto{\pgfqpoint{3.482299in}{2.148719in}}%
\pgfpathlineto{\pgfqpoint{3.489899in}{2.097098in}}%
\pgfpathlineto{\pgfqpoint{3.492511in}{2.045337in}}%
\pgfpathlineto{\pgfqpoint{3.488887in}{1.993610in}}%
\pgfpathlineto{\pgfqpoint{3.477178in}{1.942276in}}%
\pgfpathlineto{\pgfqpoint{3.454527in}{1.892118in}}%
\pgfpathlineto{\pgfqpoint{3.416350in}{1.845060in}}%
\pgfpathlineto{\pgfqpoint{3.416350in}{1.845060in}}%
\pgfpathlineto{\pgfqpoint{3.373825in}{1.814573in}}%
\pgfpathlineto{\pgfqpoint{3.373825in}{1.814573in}}%
\pgfpathlineto{\pgfqpoint{3.327034in}{1.795535in}}%
\pgfusepath{stroke}%
\end{pgfscope}%
\begin{pgfscope}%
\pgfpathrectangle{\pgfqpoint{0.647939in}{0.492442in}}{\pgfqpoint{4.273799in}{2.331163in}}%
\pgfusepath{clip}%
\pgfsetbuttcap%
\pgfsetroundjoin%
\pgfsetlinewidth{0.803000pt}%
\definecolor{currentstroke}{rgb}{0.501961,0.501961,0.501961}%
\pgfsetstrokecolor{currentstroke}%
\pgfsetdash{}{0pt}%
\pgfpathmoveto{\pgfqpoint{1.830503in}{1.172147in}}%
\pgfpathlineto{\pgfqpoint{1.794678in}{1.220121in}}%
\pgfpathlineto{\pgfqpoint{1.757919in}{1.267883in}}%
\pgfpathlineto{\pgfqpoint{1.719813in}{1.315328in}}%
\pgfpathlineto{\pgfqpoint{1.679725in}{1.362280in}}%
\pgfpathlineto{\pgfqpoint{1.636580in}{1.408408in}}%
\pgfpathlineto{\pgfqpoint{1.588367in}{1.452963in}}%
\pgfpathlineto{\pgfqpoint{1.544424in}{1.485622in}}%
\pgfpathlineto{\pgfqpoint{1.505924in}{1.506870in}}%
\pgfpathlineto{\pgfqpoint{1.468949in}{1.519994in}}%
\pgfpathlineto{\pgfqpoint{1.423962in}{1.525552in}}%
\pgfpathlineto{\pgfqpoint{1.378965in}{1.519525in}}%
\pgfpathlineto{\pgfqpoint{1.378965in}{1.519525in}}%
\pgfpathlineto{\pgfqpoint{1.327862in}{1.499081in}}%
\pgfpathlineto{\pgfqpoint{1.327862in}{1.499081in}}%
\pgfusepath{stroke}%
\end{pgfscope}%
\begin{pgfscope}%
\pgfpathrectangle{\pgfqpoint{0.647939in}{0.492442in}}{\pgfqpoint{4.273799in}{2.331163in}}%
\pgfusepath{clip}%
\pgfsetbuttcap%
\pgfsetroundjoin%
\pgfsetlinewidth{0.803000pt}%
\definecolor{currentstroke}{rgb}{0.501961,0.501961,0.501961}%
\pgfsetstrokecolor{currentstroke}%
\pgfsetdash{}{0pt}%
\pgfpathmoveto{\pgfqpoint{4.205422in}{1.300422in}}%
\pgfpathlineto{\pgfqpoint{4.144684in}{1.340138in}}%
\pgfpathlineto{\pgfqpoint{4.075977in}{1.375729in}}%
\pgfpathlineto{\pgfqpoint{3.998653in}{1.405488in}}%
\pgfpathlineto{\pgfqpoint{3.913695in}{1.428225in}}%
\pgfpathlineto{\pgfqpoint{3.823801in}{1.444584in}}%
\pgfusepath{stroke}%
\end{pgfscope}%
\begin{pgfscope}%
\pgfpathrectangle{\pgfqpoint{0.647939in}{0.492442in}}{\pgfqpoint{4.273799in}{2.331163in}}%
\pgfusepath{clip}%
\pgfsetbuttcap%
\pgfsetroundjoin%
\pgfsetlinewidth{0.803000pt}%
\definecolor{currentstroke}{rgb}{0.501961,0.501961,0.501961}%
\pgfsetstrokecolor{currentstroke}%
\pgfsetdash{}{0pt}%
\pgfpathmoveto{\pgfqpoint{4.108550in}{0.982577in}}%
\pgfpathlineto{\pgfqpoint{4.047552in}{1.022252in}}%
\pgfpathlineto{\pgfqpoint{3.982600in}{1.060013in}}%
\pgfpathlineto{\pgfqpoint{3.913872in}{1.095732in}}%
\pgfpathlineto{\pgfqpoint{3.841833in}{1.129464in}}%
\pgfpathlineto{\pgfqpoint{3.767275in}{1.161539in}}%
\pgfusepath{stroke}%
\end{pgfscope}%
\begin{pgfscope}%
\pgfpathrectangle{\pgfqpoint{0.647939in}{0.492442in}}{\pgfqpoint{4.273799in}{2.331163in}}%
\pgfusepath{clip}%
\pgfsetbuttcap%
\pgfsetroundjoin%
\pgfsetlinewidth{0.803000pt}%
\definecolor{currentstroke}{rgb}{0.501961,0.501961,0.501961}%
\pgfsetstrokecolor{currentstroke}%
\pgfsetdash{}{0pt}%
\pgfpathmoveto{\pgfqpoint{2.693792in}{1.296012in}}%
\pgfpathlineto{\pgfqpoint{2.659225in}{1.344260in}}%
\pgfpathlineto{\pgfqpoint{2.626100in}{1.392808in}}%
\pgfpathlineto{\pgfqpoint{2.594441in}{1.441647in}}%
\pgfpathlineto{\pgfqpoint{2.564281in}{1.490766in}}%
\pgfpathlineto{\pgfqpoint{2.535671in}{1.540161in}}%
\pgfpathlineto{\pgfqpoint{2.508687in}{1.589827in}}%
\pgfpathlineto{\pgfqpoint{2.483417in}{1.639760in}}%
\pgfpathlineto{\pgfqpoint{2.459987in}{1.689959in}}%
\pgfpathlineto{\pgfqpoint{2.438550in}{1.740422in}}%
\pgfpathlineto{\pgfqpoint{2.419312in}{1.791147in}}%
\pgfpathlineto{\pgfqpoint{2.402534in}{1.842131in}}%
\pgfpathlineto{\pgfqpoint{2.388550in}{1.893362in}}%
\pgfpathlineto{\pgfqpoint{2.377807in}{1.944822in}}%
\pgfpathlineto{\pgfqpoint{2.370885in}{1.996474in}}%
\pgfpathlineto{\pgfqpoint{2.368552in}{2.048243in}}%
\pgfpathlineto{\pgfqpoint{2.371856in}{2.099983in}}%
\pgfpathlineto{\pgfqpoint{2.382244in}{2.151421in}}%
\pgfpathlineto{\pgfqpoint{2.401755in}{2.202022in}}%
\pgfpathlineto{\pgfqpoint{2.431768in}{2.248820in}}%
\pgfpathlineto{\pgfqpoint{2.468782in}{2.285975in}}%
\pgfpathlineto{\pgfqpoint{2.511985in}{2.314398in}}%
\pgfpathlineto{\pgfqpoint{2.562354in}{2.334804in}}%
\pgfpathlineto{\pgfqpoint{2.622532in}{2.346439in}}%
\pgfpathlineto{\pgfqpoint{2.687707in}{2.346777in}}%
\pgfpathlineto{\pgfqpoint{2.687707in}{2.346777in}}%
\pgfpathlineto{\pgfqpoint{2.687707in}{2.346777in}}%
\pgfpathlineto{\pgfqpoint{2.749166in}{2.336919in}}%
\pgfusepath{stroke}%
\end{pgfscope}%
\begin{pgfscope}%
\pgfpathrectangle{\pgfqpoint{0.647939in}{0.492442in}}{\pgfqpoint{4.273799in}{2.331163in}}%
\pgfusepath{clip}%
\pgfsetbuttcap%
\pgfsetroundjoin%
\pgfsetlinewidth{0.803000pt}%
\definecolor{currentstroke}{rgb}{0.501961,0.501961,0.501961}%
\pgfsetstrokecolor{currentstroke}%
\pgfsetdash{}{0pt}%
\pgfpathmoveto{\pgfqpoint{1.444961in}{2.496348in}}%
\pgfpathlineto{\pgfqpoint{1.469968in}{2.446378in}}%
\pgfpathlineto{\pgfqpoint{1.495627in}{2.396508in}}%
\pgfpathlineto{\pgfqpoint{1.522125in}{2.346777in}}%
\pgfpathlineto{\pgfqpoint{1.549863in}{2.297248in}}%
\pgfpathlineto{\pgfqpoint{1.579592in}{2.248088in}}%
\pgfpathlineto{\pgfqpoint{1.613386in}{2.199821in}}%
\pgfpathlineto{\pgfqpoint{1.660444in}{2.156729in}}%
\pgfpathlineto{\pgfqpoint{1.660444in}{2.156729in}}%
\pgfpathlineto{\pgfqpoint{1.686609in}{2.150001in}}%
\pgfpathlineto{\pgfqpoint{1.712484in}{2.154692in}}%
\pgfpathlineto{\pgfqpoint{1.733458in}{2.163888in}}%
\pgfpathlineto{\pgfqpoint{1.764049in}{2.182241in}}%
\pgfusepath{stroke}%
\end{pgfscope}%
\begin{pgfscope}%
\pgfpathrectangle{\pgfqpoint{0.647939in}{0.492442in}}{\pgfqpoint{4.273799in}{2.331163in}}%
\pgfusepath{clip}%
\pgfsetbuttcap%
\pgfsetroundjoin%
\pgfsetlinewidth{0.803000pt}%
\definecolor{currentstroke}{rgb}{0.501961,0.501961,0.501961}%
\pgfsetstrokecolor{currentstroke}%
\pgfsetdash{}{0pt}%
\pgfpathmoveto{\pgfqpoint{1.836127in}{1.629940in}}%
\pgfpathlineto{\pgfqpoint{1.808135in}{1.679440in}}%
\pgfpathlineto{\pgfqpoint{1.780252in}{1.728957in}}%
\pgfpathlineto{\pgfqpoint{1.752350in}{1.778467in}}%
\pgfpathlineto{\pgfqpoint{1.724246in}{1.827942in}}%
\pgfpathlineto{\pgfqpoint{1.695498in}{1.877298in}}%
\pgfpathlineto{\pgfqpoint{1.664798in}{1.926218in}}%
\pgfpathlineto{\pgfqpoint{1.642812in}{1.956008in}}%
\pgfpathlineto{\pgfqpoint{1.627156in}{1.971858in}}%
\pgfpathlineto{\pgfqpoint{1.612965in}{1.980136in}}%
\pgfpathlineto{\pgfqpoint{1.593558in}{1.980500in}}%
\pgfpathlineto{\pgfqpoint{1.593558in}{1.980500in}}%
\pgfpathlineto{\pgfqpoint{1.567106in}{1.965345in}}%
\pgfpathlineto{\pgfqpoint{1.567106in}{1.965345in}}%
\pgfpathlineto{\pgfqpoint{1.522125in}{1.922929in}}%
\pgfpathlineto{\pgfqpoint{1.480433in}{1.877189in}}%
\pgfusepath{stroke}%
\end{pgfscope}%
\begin{pgfscope}%
\pgfpathrectangle{\pgfqpoint{0.647939in}{0.492442in}}{\pgfqpoint{4.273799in}{2.331163in}}%
\pgfusepath{clip}%
\pgfsetbuttcap%
\pgfsetroundjoin%
\pgfsetlinewidth{0.803000pt}%
\definecolor{currentstroke}{rgb}{0.501961,0.501961,0.501961}%
\pgfsetstrokecolor{currentstroke}%
\pgfsetdash{}{0pt}%
\pgfpathmoveto{\pgfqpoint{3.641193in}{2.344674in}}%
\pgfpathlineto{\pgfqpoint{3.659025in}{2.293796in}}%
\pgfpathlineto{\pgfqpoint{3.674512in}{2.242691in}}%
\pgfpathlineto{\pgfqpoint{3.687305in}{2.191366in}}%
\pgfpathlineto{\pgfqpoint{3.696950in}{2.139840in}}%
\pgfpathlineto{\pgfqpoint{3.702860in}{2.088149in}}%
\pgfpathlineto{\pgfqpoint{3.704276in}{2.036372in}}%
\pgfpathlineto{\pgfqpoint{3.700213in}{1.984647in}}%
\pgfpathlineto{\pgfqpoint{3.689390in}{1.933229in}}%
\pgfpathlineto{\pgfqpoint{3.670164in}{1.882568in}}%
\pgfusepath{stroke}%
\end{pgfscope}%
\begin{pgfscope}%
\pgfpathrectangle{\pgfqpoint{0.647939in}{0.492442in}}{\pgfqpoint{4.273799in}{2.331163in}}%
\pgfusepath{clip}%
\pgfsetbuttcap%
\pgfsetroundjoin%
\pgfsetlinewidth{0.803000pt}%
\definecolor{currentstroke}{rgb}{0.501961,0.501961,0.501961}%
\pgfsetstrokecolor{currentstroke}%
\pgfsetdash{}{0pt}%
\pgfpathmoveto{\pgfqpoint{1.919572in}{1.829047in}}%
\pgfpathlineto{\pgfqpoint{1.902948in}{1.880045in}}%
\pgfpathlineto{\pgfqpoint{1.889203in}{1.931294in}}%
\pgfpathlineto{\pgfqpoint{1.879283in}{1.982797in}}%
\pgfpathlineto{\pgfqpoint{1.874570in}{2.034504in}}%
\pgfpathlineto{\pgfqpoint{1.876905in}{2.086236in}}%
\pgfpathlineto{\pgfqpoint{1.888377in}{2.137576in}}%
\pgfpathlineto{\pgfqpoint{1.910652in}{2.187834in}}%
\pgfusepath{stroke}%
\end{pgfscope}%
\begin{pgfscope}%
\pgfpathrectangle{\pgfqpoint{0.647939in}{0.492442in}}{\pgfqpoint{4.273799in}{2.331163in}}%
\pgfusepath{clip}%
\pgfsetbuttcap%
\pgfsetroundjoin%
\pgfsetlinewidth{0.803000pt}%
\definecolor{currentstroke}{rgb}{0.501961,0.501961,0.501961}%
\pgfsetstrokecolor{currentstroke}%
\pgfsetdash{}{0pt}%
\pgfpathmoveto{\pgfqpoint{2.000670in}{1.298651in}}%
\pgfpathlineto{\pgfqpoint{1.970427in}{1.347757in}}%
\pgfpathlineto{\pgfqpoint{1.940436in}{1.396910in}}%
\pgfpathlineto{\pgfqpoint{1.910652in}{1.446100in}}%
\pgfpathlineto{\pgfqpoint{1.881017in}{1.495316in}}%
\pgfpathlineto{\pgfqpoint{1.851476in}{1.544548in}}%
\pgfusepath{stroke}%
\end{pgfscope}%
\begin{pgfscope}%
\pgfpathrectangle{\pgfqpoint{0.647939in}{0.492442in}}{\pgfqpoint{4.273799in}{2.331163in}}%
\pgfusepath{clip}%
\pgfsetbuttcap%
\pgfsetroundjoin%
\pgfsetlinewidth{0.803000pt}%
\definecolor{currentstroke}{rgb}{0.501961,0.501961,0.501961}%
\pgfsetstrokecolor{currentstroke}%
\pgfsetdash{}{0pt}%
\pgfpathmoveto{\pgfqpoint{3.885477in}{1.231810in}}%
\pgfpathlineto{\pgfqpoint{3.806803in}{1.260782in}}%
\pgfpathlineto{\pgfqpoint{3.725703in}{1.287719in}}%
\pgfpathlineto{\pgfqpoint{3.643607in}{1.313755in}}%
\pgfpathlineto{\pgfqpoint{3.561893in}{1.340138in}}%
\pgfpathlineto{\pgfqpoint{3.481827in}{1.367953in}}%
\pgfusepath{stroke}%
\end{pgfscope}%
\begin{pgfscope}%
\pgfpathrectangle{\pgfqpoint{0.647939in}{0.492442in}}{\pgfqpoint{4.273799in}{2.331163in}}%
\pgfusepath{clip}%
\pgfsetbuttcap%
\pgfsetroundjoin%
\pgfsetlinewidth{0.803000pt}%
\definecolor{currentstroke}{rgb}{0.501961,0.501961,0.501961}%
\pgfsetstrokecolor{currentstroke}%
\pgfsetdash{}{0pt}%
\pgfpathmoveto{\pgfqpoint{2.241028in}{1.262639in}}%
\pgfpathlineto{\pgfqpoint{2.211715in}{1.311913in}}%
\pgfpathlineto{\pgfqpoint{2.183143in}{1.361316in}}%
\pgfpathlineto{\pgfqpoint{2.155344in}{1.410851in}}%
\pgfpathlineto{\pgfqpoint{2.128349in}{1.460517in}}%
\pgfpathlineto{\pgfqpoint{2.102202in}{1.510318in}}%
\pgfpathlineto{\pgfqpoint{2.076974in}{1.560260in}}%
\pgfpathlineto{\pgfqpoint{2.052745in}{1.610348in}}%
\pgfpathlineto{\pgfqpoint{2.029633in}{1.660592in}}%
\pgfpathlineto{\pgfqpoint{2.007784in}{1.711005in}}%
\pgfpathlineto{\pgfqpoint{1.987405in}{1.761599in}}%
\pgfusepath{stroke}%
\end{pgfscope}%
\begin{pgfscope}%
\pgfpathrectangle{\pgfqpoint{0.647939in}{0.492442in}}{\pgfqpoint{4.273799in}{2.331163in}}%
\pgfusepath{clip}%
\pgfsetroundcap%
\pgfsetroundjoin%
\pgfsetlinewidth{0.803000pt}%
\definecolor{currentstroke}{rgb}{0.501961,0.501961,0.501961}%
\pgfsetstrokecolor{currentstroke}%
\pgfsetdash{}{0pt}%
\pgfpathmoveto{\pgfqpoint{1.434121in}{1.443241in}}%
\pgfpathquadraticcurveto{\pgfqpoint{1.421299in}{1.443850in}}{\pgfqpoint{1.420887in}{1.443870in}}%
\pgfusepath{stroke}%
\end{pgfscope}%
\begin{pgfscope}%
\pgfpathrectangle{\pgfqpoint{0.647939in}{0.492442in}}{\pgfqpoint{4.273799in}{2.331163in}}%
\pgfusepath{clip}%
\pgfsetroundcap%
\pgfsetroundjoin%
\definecolor{currentfill}{rgb}{0.501961,0.501961,0.501961}%
\pgfsetfillcolor{currentfill}%
\pgfsetlinewidth{0.803000pt}%
\definecolor{currentstroke}{rgb}{0.501961,0.501961,0.501961}%
\pgfsetstrokecolor{currentstroke}%
\pgfsetdash{}{0pt}%
\pgfpathmoveto{\pgfqpoint{1.485895in}{1.407408in}}%
\pgfpathlineto{\pgfqpoint{1.420887in}{1.443870in}}%
\pgfpathlineto{\pgfqpoint{1.489061in}{1.473999in}}%
\pgfpathlineto{\pgfqpoint{1.485895in}{1.407408in}}%
\pgfpathclose%
\pgfusepath{stroke,fill}%
\end{pgfscope}%
\begin{pgfscope}%
\pgfpathrectangle{\pgfqpoint{0.647939in}{0.492442in}}{\pgfqpoint{4.273799in}{2.331163in}}%
\pgfusepath{clip}%
\pgfsetroundcap%
\pgfsetroundjoin%
\pgfsetlinewidth{0.803000pt}%
\definecolor{currentstroke}{rgb}{0.501961,0.501961,0.501961}%
\pgfsetstrokecolor{currentstroke}%
\pgfsetdash{}{0pt}%
\pgfpathmoveto{\pgfqpoint{1.248393in}{0.905520in}}%
\pgfpathquadraticcurveto{\pgfqpoint{1.246872in}{0.905508in}}{\pgfqpoint{1.257773in}{0.905591in}}%
\pgfusepath{stroke}%
\end{pgfscope}%
\begin{pgfscope}%
\pgfpathrectangle{\pgfqpoint{0.647939in}{0.492442in}}{\pgfqpoint{4.273799in}{2.331163in}}%
\pgfusepath{clip}%
\pgfsetroundcap%
\pgfsetroundjoin%
\definecolor{currentfill}{rgb}{0.501961,0.501961,0.501961}%
\pgfsetfillcolor{currentfill}%
\pgfsetlinewidth{0.803000pt}%
\definecolor{currentstroke}{rgb}{0.501961,0.501961,0.501961}%
\pgfsetstrokecolor{currentstroke}%
\pgfsetdash{}{0pt}%
\pgfpathmoveto{\pgfqpoint{1.324693in}{0.872769in}}%
\pgfpathlineto{\pgfqpoint{1.257773in}{0.905591in}}%
\pgfpathlineto{\pgfqpoint{1.324183in}{0.939434in}}%
\pgfpathlineto{\pgfqpoint{1.324693in}{0.872769in}}%
\pgfpathclose%
\pgfusepath{stroke,fill}%
\end{pgfscope}%
\begin{pgfscope}%
\pgfpathrectangle{\pgfqpoint{0.647939in}{0.492442in}}{\pgfqpoint{4.273799in}{2.331163in}}%
\pgfusepath{clip}%
\pgfsetroundcap%
\pgfsetroundjoin%
\pgfsetlinewidth{0.803000pt}%
\definecolor{currentstroke}{rgb}{0.501961,0.501961,0.501961}%
\pgfsetstrokecolor{currentstroke}%
\pgfsetdash{}{0pt}%
\pgfpathmoveto{\pgfqpoint{1.212876in}{0.688071in}}%
\pgfpathquadraticcurveto{\pgfqpoint{1.211747in}{0.688127in}}{\pgfqpoint{1.223026in}{0.687570in}}%
\pgfusepath{stroke}%
\end{pgfscope}%
\begin{pgfscope}%
\pgfpathrectangle{\pgfqpoint{0.647939in}{0.492442in}}{\pgfqpoint{4.273799in}{2.331163in}}%
\pgfusepath{clip}%
\pgfsetroundcap%
\pgfsetroundjoin%
\definecolor{currentfill}{rgb}{0.501961,0.501961,0.501961}%
\pgfsetfillcolor{currentfill}%
\pgfsetlinewidth{0.803000pt}%
\definecolor{currentstroke}{rgb}{0.501961,0.501961,0.501961}%
\pgfsetstrokecolor{currentstroke}%
\pgfsetdash{}{0pt}%
\pgfpathmoveto{\pgfqpoint{1.287966in}{0.650987in}}%
\pgfpathlineto{\pgfqpoint{1.223026in}{0.687570in}}%
\pgfpathlineto{\pgfqpoint{1.291256in}{0.717572in}}%
\pgfpathlineto{\pgfqpoint{1.287966in}{0.650987in}}%
\pgfpathclose%
\pgfusepath{stroke,fill}%
\end{pgfscope}%
\begin{pgfscope}%
\pgfpathrectangle{\pgfqpoint{0.647939in}{0.492442in}}{\pgfqpoint{4.273799in}{2.331163in}}%
\pgfusepath{clip}%
\pgfsetroundcap%
\pgfsetroundjoin%
\pgfsetlinewidth{0.803000pt}%
\definecolor{currentstroke}{rgb}{0.501961,0.501961,0.501961}%
\pgfsetstrokecolor{currentstroke}%
\pgfsetdash{}{0pt}%
\pgfpathmoveto{\pgfqpoint{1.174898in}{0.571616in}}%
\pgfpathquadraticcurveto{\pgfqpoint{1.172862in}{0.571633in}}{\pgfqpoint{1.183248in}{0.571547in}}%
\pgfusepath{stroke}%
\end{pgfscope}%
\begin{pgfscope}%
\pgfpathrectangle{\pgfqpoint{0.647939in}{0.492442in}}{\pgfqpoint{4.273799in}{2.331163in}}%
\pgfusepath{clip}%
\pgfsetroundcap%
\pgfsetroundjoin%
\definecolor{currentfill}{rgb}{0.501961,0.501961,0.501961}%
\pgfsetfillcolor{currentfill}%
\pgfsetlinewidth{0.803000pt}%
\definecolor{currentstroke}{rgb}{0.501961,0.501961,0.501961}%
\pgfsetstrokecolor{currentstroke}%
\pgfsetdash{}{0pt}%
\pgfpathmoveto{\pgfqpoint{1.249635in}{0.537660in}}%
\pgfpathlineto{\pgfqpoint{1.183248in}{0.571547in}}%
\pgfpathlineto{\pgfqpoint{1.250190in}{0.604325in}}%
\pgfpathlineto{\pgfqpoint{1.249635in}{0.537660in}}%
\pgfpathclose%
\pgfusepath{stroke,fill}%
\end{pgfscope}%
\begin{pgfscope}%
\pgfpathrectangle{\pgfqpoint{0.647939in}{0.492442in}}{\pgfqpoint{4.273799in}{2.331163in}}%
\pgfusepath{clip}%
\pgfsetroundcap%
\pgfsetroundjoin%
\pgfsetlinewidth{0.803000pt}%
\definecolor{currentstroke}{rgb}{0.501961,0.501961,0.501961}%
\pgfsetstrokecolor{currentstroke}%
\pgfsetdash{}{0pt}%
\pgfpathmoveto{\pgfqpoint{1.391143in}{0.767338in}}%
\pgfpathquadraticcurveto{\pgfqpoint{1.385934in}{0.769839in}}{\pgfqpoint{1.391923in}{0.766963in}}%
\pgfusepath{stroke}%
\end{pgfscope}%
\begin{pgfscope}%
\pgfpathrectangle{\pgfqpoint{0.647939in}{0.492442in}}{\pgfqpoint{4.273799in}{2.331163in}}%
\pgfusepath{clip}%
\pgfsetroundcap%
\pgfsetroundjoin%
\definecolor{currentfill}{rgb}{0.501961,0.501961,0.501961}%
\pgfsetfillcolor{currentfill}%
\pgfsetlinewidth{0.803000pt}%
\definecolor{currentstroke}{rgb}{0.501961,0.501961,0.501961}%
\pgfsetstrokecolor{currentstroke}%
\pgfsetdash{}{0pt}%
\pgfpathmoveto{\pgfqpoint{1.437589in}{0.708055in}}%
\pgfpathlineto{\pgfqpoint{1.391923in}{0.766963in}}%
\pgfpathlineto{\pgfqpoint{1.466449in}{0.768152in}}%
\pgfpathlineto{\pgfqpoint{1.437589in}{0.708055in}}%
\pgfpathclose%
\pgfusepath{stroke,fill}%
\end{pgfscope}%
\begin{pgfscope}%
\pgfpathrectangle{\pgfqpoint{0.647939in}{0.492442in}}{\pgfqpoint{4.273799in}{2.331163in}}%
\pgfusepath{clip}%
\pgfsetroundcap%
\pgfsetroundjoin%
\pgfsetlinewidth{0.803000pt}%
\definecolor{currentstroke}{rgb}{0.501961,0.501961,0.501961}%
\pgfsetstrokecolor{currentstroke}%
\pgfsetdash{}{0pt}%
\pgfpathmoveto{\pgfqpoint{1.557281in}{0.882777in}}%
\pgfpathquadraticcurveto{\pgfqpoint{1.555033in}{0.884678in}}{\pgfqpoint{1.562271in}{0.878558in}}%
\pgfusepath{stroke}%
\end{pgfscope}%
\begin{pgfscope}%
\pgfpathrectangle{\pgfqpoint{0.647939in}{0.492442in}}{\pgfqpoint{4.273799in}{2.331163in}}%
\pgfusepath{clip}%
\pgfsetroundcap%
\pgfsetroundjoin%
\definecolor{currentfill}{rgb}{0.501961,0.501961,0.501961}%
\pgfsetfillcolor{currentfill}%
\pgfsetlinewidth{0.803000pt}%
\definecolor{currentstroke}{rgb}{0.501961,0.501961,0.501961}%
\pgfsetstrokecolor{currentstroke}%
\pgfsetdash{}{0pt}%
\pgfpathmoveto{\pgfqpoint{1.591661in}{0.810062in}}%
\pgfpathlineto{\pgfqpoint{1.562271in}{0.878558in}}%
\pgfpathlineto{\pgfqpoint{1.634702in}{0.860973in}}%
\pgfpathlineto{\pgfqpoint{1.591661in}{0.810062in}}%
\pgfpathclose%
\pgfusepath{stroke,fill}%
\end{pgfscope}%
\begin{pgfscope}%
\pgfpathrectangle{\pgfqpoint{0.647939in}{0.492442in}}{\pgfqpoint{4.273799in}{2.331163in}}%
\pgfusepath{clip}%
\pgfsetroundcap%
\pgfsetroundjoin%
\pgfsetlinewidth{0.803000pt}%
\definecolor{currentstroke}{rgb}{0.501961,0.501961,0.501961}%
\pgfsetstrokecolor{currentstroke}%
\pgfsetdash{}{0pt}%
\pgfpathmoveto{\pgfqpoint{1.591460in}{0.981690in}}%
\pgfpathquadraticcurveto{\pgfqpoint{1.589513in}{0.983455in}}{\pgfqpoint{1.596769in}{0.976878in}}%
\pgfusepath{stroke}%
\end{pgfscope}%
\begin{pgfscope}%
\pgfpathrectangle{\pgfqpoint{0.647939in}{0.492442in}}{\pgfqpoint{4.273799in}{2.331163in}}%
\pgfusepath{clip}%
\pgfsetroundcap%
\pgfsetroundjoin%
\definecolor{currentfill}{rgb}{0.501961,0.501961,0.501961}%
\pgfsetfillcolor{currentfill}%
\pgfsetlinewidth{0.803000pt}%
\definecolor{currentstroke}{rgb}{0.501961,0.501961,0.501961}%
\pgfsetstrokecolor{currentstroke}%
\pgfsetdash{}{0pt}%
\pgfpathmoveto{\pgfqpoint{1.623776in}{0.907407in}}%
\pgfpathlineto{\pgfqpoint{1.596769in}{0.976878in}}%
\pgfpathlineto{\pgfqpoint{1.668550in}{0.956801in}}%
\pgfpathlineto{\pgfqpoint{1.623776in}{0.907407in}}%
\pgfpathclose%
\pgfusepath{stroke,fill}%
\end{pgfscope}%
\begin{pgfscope}%
\pgfpathrectangle{\pgfqpoint{0.647939in}{0.492442in}}{\pgfqpoint{4.273799in}{2.331163in}}%
\pgfusepath{clip}%
\pgfsetroundcap%
\pgfsetroundjoin%
\pgfsetlinewidth{0.803000pt}%
\definecolor{currentstroke}{rgb}{0.501961,0.501961,0.501961}%
\pgfsetstrokecolor{currentstroke}%
\pgfsetdash{}{0pt}%
\pgfpathmoveto{\pgfqpoint{1.630346in}{1.081698in}}%
\pgfpathquadraticcurveto{\pgfqpoint{1.628740in}{1.083280in}}{\pgfqpoint{1.635986in}{1.076147in}}%
\pgfusepath{stroke}%
\end{pgfscope}%
\begin{pgfscope}%
\pgfpathrectangle{\pgfqpoint{0.647939in}{0.492442in}}{\pgfqpoint{4.273799in}{2.331163in}}%
\pgfusepath{clip}%
\pgfsetroundcap%
\pgfsetroundjoin%
\definecolor{currentfill}{rgb}{0.501961,0.501961,0.501961}%
\pgfsetfillcolor{currentfill}%
\pgfsetlinewidth{0.803000pt}%
\definecolor{currentstroke}{rgb}{0.501961,0.501961,0.501961}%
\pgfsetstrokecolor{currentstroke}%
\pgfsetdash{}{0pt}%
\pgfpathmoveto{\pgfqpoint{1.660112in}{1.005624in}}%
\pgfpathlineto{\pgfqpoint{1.635986in}{1.076147in}}%
\pgfpathlineto{\pgfqpoint{1.706880in}{1.053134in}}%
\pgfpathlineto{\pgfqpoint{1.660112in}{1.005624in}}%
\pgfpathclose%
\pgfusepath{stroke,fill}%
\end{pgfscope}%
\begin{pgfscope}%
\pgfpathrectangle{\pgfqpoint{0.647939in}{0.492442in}}{\pgfqpoint{4.273799in}{2.331163in}}%
\pgfusepath{clip}%
\pgfsetroundcap%
\pgfsetroundjoin%
\pgfsetlinewidth{0.803000pt}%
\definecolor{currentstroke}{rgb}{0.501961,0.501961,0.501961}%
\pgfsetstrokecolor{currentstroke}%
\pgfsetdash{}{0pt}%
\pgfpathmoveto{\pgfqpoint{1.866424in}{1.239860in}}%
\pgfpathquadraticcurveto{\pgfqpoint{1.865922in}{1.240581in}}{\pgfqpoint{1.872523in}{1.231112in}}%
\pgfusepath{stroke}%
\end{pgfscope}%
\begin{pgfscope}%
\pgfpathrectangle{\pgfqpoint{0.647939in}{0.492442in}}{\pgfqpoint{4.273799in}{2.331163in}}%
\pgfusepath{clip}%
\pgfsetroundcap%
\pgfsetroundjoin%
\definecolor{currentfill}{rgb}{0.501961,0.501961,0.501961}%
\pgfsetfillcolor{currentfill}%
\pgfsetlinewidth{0.803000pt}%
\definecolor{currentstroke}{rgb}{0.501961,0.501961,0.501961}%
\pgfsetstrokecolor{currentstroke}%
\pgfsetdash{}{0pt}%
\pgfpathmoveto{\pgfqpoint{1.883305in}{1.157360in}}%
\pgfpathlineto{\pgfqpoint{1.872523in}{1.231112in}}%
\pgfpathlineto{\pgfqpoint{1.937994in}{1.195487in}}%
\pgfpathlineto{\pgfqpoint{1.883305in}{1.157360in}}%
\pgfpathclose%
\pgfusepath{stroke,fill}%
\end{pgfscope}%
\begin{pgfscope}%
\pgfpathrectangle{\pgfqpoint{0.647939in}{0.492442in}}{\pgfqpoint{4.273799in}{2.331163in}}%
\pgfusepath{clip}%
\pgfsetroundcap%
\pgfsetroundjoin%
\pgfsetlinewidth{0.803000pt}%
\definecolor{currentstroke}{rgb}{0.501961,0.501961,0.501961}%
\pgfsetstrokecolor{currentstroke}%
\pgfsetdash{}{0pt}%
\pgfpathmoveto{\pgfqpoint{1.967834in}{1.241032in}}%
\pgfpathquadraticcurveto{\pgfqpoint{1.967471in}{1.241592in}}{\pgfqpoint{1.973856in}{1.231722in}}%
\pgfusepath{stroke}%
\end{pgfscope}%
\begin{pgfscope}%
\pgfpathrectangle{\pgfqpoint{0.647939in}{0.492442in}}{\pgfqpoint{4.273799in}{2.331163in}}%
\pgfusepath{clip}%
\pgfsetroundcap%
\pgfsetroundjoin%
\definecolor{currentfill}{rgb}{0.501961,0.501961,0.501961}%
\pgfsetfillcolor{currentfill}%
\pgfsetlinewidth{0.803000pt}%
\definecolor{currentstroke}{rgb}{0.501961,0.501961,0.501961}%
\pgfsetstrokecolor{currentstroke}%
\pgfsetdash{}{0pt}%
\pgfpathmoveto{\pgfqpoint{1.982077in}{1.157641in}}%
\pgfpathlineto{\pgfqpoint{1.973856in}{1.231722in}}%
\pgfpathlineto{\pgfqpoint{2.038053in}{1.193851in}}%
\pgfpathlineto{\pgfqpoint{1.982077in}{1.157641in}}%
\pgfpathclose%
\pgfusepath{stroke,fill}%
\end{pgfscope}%
\begin{pgfscope}%
\pgfpathrectangle{\pgfqpoint{0.647939in}{0.492442in}}{\pgfqpoint{4.273799in}{2.331163in}}%
\pgfusepath{clip}%
\pgfsetroundcap%
\pgfsetroundjoin%
\pgfsetlinewidth{0.803000pt}%
\definecolor{currentstroke}{rgb}{0.501961,0.501961,0.501961}%
\pgfsetstrokecolor{currentstroke}%
\pgfsetdash{}{0pt}%
\pgfpathmoveto{\pgfqpoint{2.192802in}{2.420976in}}%
\pgfpathquadraticcurveto{\pgfqpoint{2.199561in}{2.423373in}}{\pgfqpoint{2.194612in}{2.421618in}}%
\pgfusepath{stroke}%
\end{pgfscope}%
\begin{pgfscope}%
\pgfpathrectangle{\pgfqpoint{0.647939in}{0.492442in}}{\pgfqpoint{4.273799in}{2.331163in}}%
\pgfusepath{clip}%
\pgfsetroundcap%
\pgfsetroundjoin%
\definecolor{currentfill}{rgb}{0.501961,0.501961,0.501961}%
\pgfsetfillcolor{currentfill}%
\pgfsetlinewidth{0.803000pt}%
\definecolor{currentstroke}{rgb}{0.501961,0.501961,0.501961}%
\pgfsetstrokecolor{currentstroke}%
\pgfsetdash{}{0pt}%
\pgfpathmoveto{\pgfqpoint{2.120639in}{2.430751in}}%
\pgfpathlineto{\pgfqpoint{2.194612in}{2.421618in}}%
\pgfpathlineto{\pgfqpoint{2.142922in}{2.367918in}}%
\pgfpathlineto{\pgfqpoint{2.120639in}{2.430751in}}%
\pgfpathclose%
\pgfusepath{stroke,fill}%
\end{pgfscope}%
\begin{pgfscope}%
\pgfpathrectangle{\pgfqpoint{0.647939in}{0.492442in}}{\pgfqpoint{4.273799in}{2.331163in}}%
\pgfusepath{clip}%
\pgfsetroundcap%
\pgfsetroundjoin%
\pgfsetlinewidth{0.803000pt}%
\definecolor{currentstroke}{rgb}{0.501961,0.501961,0.501961}%
\pgfsetstrokecolor{currentstroke}%
\pgfsetdash{}{0pt}%
\pgfpathmoveto{\pgfqpoint{2.646534in}{0.654712in}}%
\pgfpathquadraticcurveto{\pgfqpoint{2.645657in}{0.655786in}}{\pgfqpoint{2.652633in}{0.647235in}}%
\pgfusepath{stroke}%
\end{pgfscope}%
\begin{pgfscope}%
\pgfpathrectangle{\pgfqpoint{0.647939in}{0.492442in}}{\pgfqpoint{4.273799in}{2.331163in}}%
\pgfusepath{clip}%
\pgfsetroundcap%
\pgfsetroundjoin%
\definecolor{currentfill}{rgb}{0.501961,0.501961,0.501961}%
\pgfsetfillcolor{currentfill}%
\pgfsetlinewidth{0.803000pt}%
\definecolor{currentstroke}{rgb}{0.501961,0.501961,0.501961}%
\pgfsetstrokecolor{currentstroke}%
\pgfsetdash{}{0pt}%
\pgfpathmoveto{\pgfqpoint{2.668951in}{0.574508in}}%
\pgfpathlineto{\pgfqpoint{2.652633in}{0.647235in}}%
\pgfpathlineto{\pgfqpoint{2.720606in}{0.616653in}}%
\pgfpathlineto{\pgfqpoint{2.668951in}{0.574508in}}%
\pgfpathclose%
\pgfusepath{stroke,fill}%
\end{pgfscope}%
\begin{pgfscope}%
\pgfpathrectangle{\pgfqpoint{0.647939in}{0.492442in}}{\pgfqpoint{4.273799in}{2.331163in}}%
\pgfusepath{clip}%
\pgfsetroundcap%
\pgfsetroundjoin%
\pgfsetlinewidth{0.803000pt}%
\definecolor{currentstroke}{rgb}{0.501961,0.501961,0.501961}%
\pgfsetstrokecolor{currentstroke}%
\pgfsetdash{}{0pt}%
\pgfpathmoveto{\pgfqpoint{2.778739in}{0.605992in}}%
\pgfpathquadraticcurveto{\pgfqpoint{2.777647in}{0.607235in}}{\pgfqpoint{2.784755in}{0.599148in}}%
\pgfusepath{stroke}%
\end{pgfscope}%
\begin{pgfscope}%
\pgfpathrectangle{\pgfqpoint{0.647939in}{0.492442in}}{\pgfqpoint{4.273799in}{2.331163in}}%
\pgfusepath{clip}%
\pgfsetroundcap%
\pgfsetroundjoin%
\definecolor{currentfill}{rgb}{0.501961,0.501961,0.501961}%
\pgfsetfillcolor{currentfill}%
\pgfsetlinewidth{0.803000pt}%
\definecolor{currentstroke}{rgb}{0.501961,0.501961,0.501961}%
\pgfsetstrokecolor{currentstroke}%
\pgfsetdash{}{0pt}%
\pgfpathmoveto{\pgfqpoint{2.803731in}{0.527068in}}%
\pgfpathlineto{\pgfqpoint{2.784755in}{0.599148in}}%
\pgfpathlineto{\pgfqpoint{2.853805in}{0.571081in}}%
\pgfpathlineto{\pgfqpoint{2.803731in}{0.527068in}}%
\pgfpathclose%
\pgfusepath{stroke,fill}%
\end{pgfscope}%
\begin{pgfscope}%
\pgfpathrectangle{\pgfqpoint{0.647939in}{0.492442in}}{\pgfqpoint{4.273799in}{2.331163in}}%
\pgfusepath{clip}%
\pgfsetroundcap%
\pgfsetroundjoin%
\pgfsetlinewidth{0.803000pt}%
\definecolor{currentstroke}{rgb}{0.501961,0.501961,0.501961}%
\pgfsetstrokecolor{currentstroke}%
\pgfsetdash{}{0pt}%
\pgfpathmoveto{\pgfqpoint{2.316564in}{1.467075in}}%
\pgfpathquadraticcurveto{\pgfqpoint{2.316467in}{1.467257in}}{\pgfqpoint{2.322235in}{1.456488in}}%
\pgfusepath{stroke}%
\end{pgfscope}%
\begin{pgfscope}%
\pgfpathrectangle{\pgfqpoint{0.647939in}{0.492442in}}{\pgfqpoint{4.273799in}{2.331163in}}%
\pgfusepath{clip}%
\pgfsetroundcap%
\pgfsetroundjoin%
\definecolor{currentfill}{rgb}{0.501961,0.501961,0.501961}%
\pgfsetfillcolor{currentfill}%
\pgfsetlinewidth{0.803000pt}%
\definecolor{currentstroke}{rgb}{0.501961,0.501961,0.501961}%
\pgfsetstrokecolor{currentstroke}%
\pgfsetdash{}{0pt}%
\pgfpathmoveto{\pgfqpoint{2.324326in}{1.381982in}}%
\pgfpathlineto{\pgfqpoint{2.322235in}{1.456488in}}%
\pgfpathlineto{\pgfqpoint{2.383094in}{1.413458in}}%
\pgfpathlineto{\pgfqpoint{2.324326in}{1.381982in}}%
\pgfpathclose%
\pgfusepath{stroke,fill}%
\end{pgfscope}%
\begin{pgfscope}%
\pgfpathrectangle{\pgfqpoint{0.647939in}{0.492442in}}{\pgfqpoint{4.273799in}{2.331163in}}%
\pgfusepath{clip}%
\pgfsetroundcap%
\pgfsetroundjoin%
\pgfsetlinewidth{0.803000pt}%
\definecolor{currentstroke}{rgb}{0.501961,0.501961,0.501961}%
\pgfsetstrokecolor{currentstroke}%
\pgfsetdash{}{0pt}%
\pgfpathmoveto{\pgfqpoint{2.502749in}{1.350314in}}%
\pgfpathquadraticcurveto{\pgfqpoint{2.502442in}{1.350804in}}{\pgfqpoint{2.508720in}{1.340761in}}%
\pgfusepath{stroke}%
\end{pgfscope}%
\begin{pgfscope}%
\pgfpathrectangle{\pgfqpoint{0.647939in}{0.492442in}}{\pgfqpoint{4.273799in}{2.331163in}}%
\pgfusepath{clip}%
\pgfsetroundcap%
\pgfsetroundjoin%
\definecolor{currentfill}{rgb}{0.501961,0.501961,0.501961}%
\pgfsetfillcolor{currentfill}%
\pgfsetlinewidth{0.803000pt}%
\definecolor{currentstroke}{rgb}{0.501961,0.501961,0.501961}%
\pgfsetstrokecolor{currentstroke}%
\pgfsetdash{}{0pt}%
\pgfpathmoveto{\pgfqpoint{2.515789in}{1.266561in}}%
\pgfpathlineto{\pgfqpoint{2.508720in}{1.340761in}}%
\pgfpathlineto{\pgfqpoint{2.572321in}{1.301897in}}%
\pgfpathlineto{\pgfqpoint{2.515789in}{1.266561in}}%
\pgfpathclose%
\pgfusepath{stroke,fill}%
\end{pgfscope}%
\begin{pgfscope}%
\pgfpathrectangle{\pgfqpoint{0.647939in}{0.492442in}}{\pgfqpoint{4.273799in}{2.331163in}}%
\pgfusepath{clip}%
\pgfsetroundcap%
\pgfsetroundjoin%
\pgfsetlinewidth{0.803000pt}%
\definecolor{currentstroke}{rgb}{0.501961,0.501961,0.501961}%
\pgfsetstrokecolor{currentstroke}%
\pgfsetdash{}{0pt}%
\pgfpathmoveto{\pgfqpoint{3.024180in}{0.856568in}}%
\pgfpathquadraticcurveto{\pgfqpoint{3.022503in}{0.858186in}}{\pgfqpoint{3.029764in}{0.851177in}}%
\pgfusepath{stroke}%
\end{pgfscope}%
\begin{pgfscope}%
\pgfpathrectangle{\pgfqpoint{0.647939in}{0.492442in}}{\pgfqpoint{4.273799in}{2.331163in}}%
\pgfusepath{clip}%
\pgfsetroundcap%
\pgfsetroundjoin%
\definecolor{currentfill}{rgb}{0.501961,0.501961,0.501961}%
\pgfsetfillcolor{currentfill}%
\pgfsetlinewidth{0.803000pt}%
\definecolor{currentstroke}{rgb}{0.501961,0.501961,0.501961}%
\pgfsetstrokecolor{currentstroke}%
\pgfsetdash{}{0pt}%
\pgfpathmoveto{\pgfqpoint{3.054579in}{0.780894in}}%
\pgfpathlineto{\pgfqpoint{3.029764in}{0.851177in}}%
\pgfpathlineto{\pgfqpoint{3.100880in}{0.828859in}}%
\pgfpathlineto{\pgfqpoint{3.054579in}{0.780894in}}%
\pgfpathclose%
\pgfusepath{stroke,fill}%
\end{pgfscope}%
\begin{pgfscope}%
\pgfpathrectangle{\pgfqpoint{0.647939in}{0.492442in}}{\pgfqpoint{4.273799in}{2.331163in}}%
\pgfusepath{clip}%
\pgfsetroundcap%
\pgfsetroundjoin%
\pgfsetlinewidth{0.803000pt}%
\definecolor{currentstroke}{rgb}{0.501961,0.501961,0.501961}%
\pgfsetstrokecolor{currentstroke}%
\pgfsetdash{}{0pt}%
\pgfpathmoveto{\pgfqpoint{3.234145in}{0.798063in}}%
\pgfpathquadraticcurveto{\pgfqpoint{3.231665in}{0.800064in}}{\pgfqpoint{3.238853in}{0.794265in}}%
\pgfusepath{stroke}%
\end{pgfscope}%
\begin{pgfscope}%
\pgfpathrectangle{\pgfqpoint{0.647939in}{0.492442in}}{\pgfqpoint{4.273799in}{2.331163in}}%
\pgfusepath{clip}%
\pgfsetroundcap%
\pgfsetroundjoin%
\definecolor{currentfill}{rgb}{0.501961,0.501961,0.501961}%
\pgfsetfillcolor{currentfill}%
\pgfsetlinewidth{0.803000pt}%
\definecolor{currentstroke}{rgb}{0.501961,0.501961,0.501961}%
\pgfsetstrokecolor{currentstroke}%
\pgfsetdash{}{0pt}%
\pgfpathmoveto{\pgfqpoint{3.269813in}{0.726464in}}%
\pgfpathlineto{\pgfqpoint{3.238853in}{0.794265in}}%
\pgfpathlineto{\pgfqpoint{3.311670in}{0.778352in}}%
\pgfpathlineto{\pgfqpoint{3.269813in}{0.726464in}}%
\pgfpathclose%
\pgfusepath{stroke,fill}%
\end{pgfscope}%
\begin{pgfscope}%
\pgfpathrectangle{\pgfqpoint{0.647939in}{0.492442in}}{\pgfqpoint{4.273799in}{2.331163in}}%
\pgfusepath{clip}%
\pgfsetroundcap%
\pgfsetroundjoin%
\pgfsetlinewidth{0.803000pt}%
\definecolor{currentstroke}{rgb}{0.501961,0.501961,0.501961}%
\pgfsetstrokecolor{currentstroke}%
\pgfsetdash{}{0pt}%
\pgfpathmoveto{\pgfqpoint{2.776687in}{1.368572in}}%
\pgfpathquadraticcurveto{\pgfqpoint{2.775958in}{1.369516in}}{\pgfqpoint{2.782821in}{1.360627in}}%
\pgfusepath{stroke}%
\end{pgfscope}%
\begin{pgfscope}%
\pgfpathrectangle{\pgfqpoint{0.647939in}{0.492442in}}{\pgfqpoint{4.273799in}{2.331163in}}%
\pgfusepath{clip}%
\pgfsetroundcap%
\pgfsetroundjoin%
\definecolor{currentfill}{rgb}{0.501961,0.501961,0.501961}%
\pgfsetfillcolor{currentfill}%
\pgfsetlinewidth{0.803000pt}%
\definecolor{currentstroke}{rgb}{0.501961,0.501961,0.501961}%
\pgfsetstrokecolor{currentstroke}%
\pgfsetdash{}{0pt}%
\pgfpathmoveto{\pgfqpoint{2.797177in}{1.287487in}}%
\pgfpathlineto{\pgfqpoint{2.782821in}{1.360627in}}%
\pgfpathlineto{\pgfqpoint{2.849946in}{1.328227in}}%
\pgfpathlineto{\pgfqpoint{2.797177in}{1.287487in}}%
\pgfpathclose%
\pgfusepath{stroke,fill}%
\end{pgfscope}%
\begin{pgfscope}%
\pgfpathrectangle{\pgfqpoint{0.647939in}{0.492442in}}{\pgfqpoint{4.273799in}{2.331163in}}%
\pgfusepath{clip}%
\pgfsetroundcap%
\pgfsetroundjoin%
\pgfsetlinewidth{0.803000pt}%
\definecolor{currentstroke}{rgb}{0.501961,0.501961,0.501961}%
\pgfsetstrokecolor{currentstroke}%
\pgfsetdash{}{0pt}%
\pgfpathmoveto{\pgfqpoint{3.411266in}{0.903174in}}%
\pgfpathquadraticcurveto{\pgfqpoint{3.407781in}{0.905492in}}{\pgfqpoint{3.414641in}{0.900930in}}%
\pgfusepath{stroke}%
\end{pgfscope}%
\begin{pgfscope}%
\pgfpathrectangle{\pgfqpoint{0.647939in}{0.492442in}}{\pgfqpoint{4.273799in}{2.331163in}}%
\pgfusepath{clip}%
\pgfsetroundcap%
\pgfsetroundjoin%
\definecolor{currentfill}{rgb}{0.501961,0.501961,0.501961}%
\pgfsetfillcolor{currentfill}%
\pgfsetlinewidth{0.803000pt}%
\definecolor{currentstroke}{rgb}{0.501961,0.501961,0.501961}%
\pgfsetstrokecolor{currentstroke}%
\pgfsetdash{}{0pt}%
\pgfpathmoveto{\pgfqpoint{3.451695in}{0.836258in}}%
\pgfpathlineto{\pgfqpoint{3.414641in}{0.900930in}}%
\pgfpathlineto{\pgfqpoint{3.488611in}{0.891771in}}%
\pgfpathlineto{\pgfqpoint{3.451695in}{0.836258in}}%
\pgfpathclose%
\pgfusepath{stroke,fill}%
\end{pgfscope}%
\begin{pgfscope}%
\pgfpathrectangle{\pgfqpoint{0.647939in}{0.492442in}}{\pgfqpoint{4.273799in}{2.331163in}}%
\pgfusepath{clip}%
\pgfsetroundcap%
\pgfsetroundjoin%
\pgfsetlinewidth{0.803000pt}%
\definecolor{currentstroke}{rgb}{0.501961,0.501961,0.501961}%
\pgfsetstrokecolor{currentstroke}%
\pgfsetdash{}{0pt}%
\pgfpathmoveto{\pgfqpoint{3.134359in}{1.227578in}}%
\pgfpathquadraticcurveto{\pgfqpoint{3.131956in}{1.229546in}}{\pgfqpoint{3.139164in}{1.223644in}}%
\pgfusepath{stroke}%
\end{pgfscope}%
\begin{pgfscope}%
\pgfpathrectangle{\pgfqpoint{0.647939in}{0.492442in}}{\pgfqpoint{4.273799in}{2.331163in}}%
\pgfusepath{clip}%
\pgfsetroundcap%
\pgfsetroundjoin%
\definecolor{currentfill}{rgb}{0.501961,0.501961,0.501961}%
\pgfsetfillcolor{currentfill}%
\pgfsetlinewidth{0.803000pt}%
\definecolor{currentstroke}{rgb}{0.501961,0.501961,0.501961}%
\pgfsetstrokecolor{currentstroke}%
\pgfsetdash{}{0pt}%
\pgfpathmoveto{\pgfqpoint{3.169626in}{1.155618in}}%
\pgfpathlineto{\pgfqpoint{3.139164in}{1.223644in}}%
\pgfpathlineto{\pgfqpoint{3.211863in}{1.207198in}}%
\pgfpathlineto{\pgfqpoint{3.169626in}{1.155618in}}%
\pgfpathclose%
\pgfusepath{stroke,fill}%
\end{pgfscope}%
\begin{pgfscope}%
\pgfpathrectangle{\pgfqpoint{0.647939in}{0.492442in}}{\pgfqpoint{4.273799in}{2.331163in}}%
\pgfusepath{clip}%
\pgfsetroundcap%
\pgfsetroundjoin%
\pgfsetlinewidth{0.803000pt}%
\definecolor{currentstroke}{rgb}{0.501961,0.501961,0.501961}%
\pgfsetstrokecolor{currentstroke}%
\pgfsetdash{}{0pt}%
\pgfpathmoveto{\pgfqpoint{4.063765in}{0.727528in}}%
\pgfpathquadraticcurveto{\pgfqpoint{4.060609in}{0.729761in}}{\pgfqpoint{4.067594in}{0.724820in}}%
\pgfusepath{stroke}%
\end{pgfscope}%
\begin{pgfscope}%
\pgfpathrectangle{\pgfqpoint{0.647939in}{0.492442in}}{\pgfqpoint{4.273799in}{2.331163in}}%
\pgfusepath{clip}%
\pgfsetroundcap%
\pgfsetroundjoin%
\definecolor{currentfill}{rgb}{0.501961,0.501961,0.501961}%
\pgfsetfillcolor{currentfill}%
\pgfsetlinewidth{0.803000pt}%
\definecolor{currentstroke}{rgb}{0.501961,0.501961,0.501961}%
\pgfsetstrokecolor{currentstroke}%
\pgfsetdash{}{0pt}%
\pgfpathmoveto{\pgfqpoint{4.102769in}{0.659107in}}%
\pgfpathlineto{\pgfqpoint{4.067594in}{0.724820in}}%
\pgfpathlineto{\pgfqpoint{4.141269in}{0.713532in}}%
\pgfpathlineto{\pgfqpoint{4.102769in}{0.659107in}}%
\pgfpathclose%
\pgfusepath{stroke,fill}%
\end{pgfscope}%
\begin{pgfscope}%
\pgfpathrectangle{\pgfqpoint{0.647939in}{0.492442in}}{\pgfqpoint{4.273799in}{2.331163in}}%
\pgfusepath{clip}%
\pgfsetroundcap%
\pgfsetroundjoin%
\pgfsetlinewidth{0.803000pt}%
\definecolor{currentstroke}{rgb}{0.501961,0.501961,0.501961}%
\pgfsetstrokecolor{currentstroke}%
\pgfsetdash{}{0pt}%
\pgfpathmoveto{\pgfqpoint{3.344357in}{1.303052in}}%
\pgfpathquadraticcurveto{\pgfqpoint{3.339851in}{1.305546in}}{\pgfqpoint{3.346213in}{1.302024in}}%
\pgfusepath{stroke}%
\end{pgfscope}%
\begin{pgfscope}%
\pgfpathrectangle{\pgfqpoint{0.647939in}{0.492442in}}{\pgfqpoint{4.273799in}{2.331163in}}%
\pgfusepath{clip}%
\pgfsetroundcap%
\pgfsetroundjoin%
\definecolor{currentfill}{rgb}{0.501961,0.501961,0.501961}%
\pgfsetfillcolor{currentfill}%
\pgfsetlinewidth{0.803000pt}%
\definecolor{currentstroke}{rgb}{0.501961,0.501961,0.501961}%
\pgfsetstrokecolor{currentstroke}%
\pgfsetdash{}{0pt}%
\pgfpathmoveto{\pgfqpoint{3.388392in}{1.240571in}}%
\pgfpathlineto{\pgfqpoint{3.346213in}{1.302024in}}%
\pgfpathlineto{\pgfqpoint{3.420683in}{1.298895in}}%
\pgfpathlineto{\pgfqpoint{3.388392in}{1.240571in}}%
\pgfpathclose%
\pgfusepath{stroke,fill}%
\end{pgfscope}%
\begin{pgfscope}%
\pgfpathrectangle{\pgfqpoint{0.647939in}{0.492442in}}{\pgfqpoint{4.273799in}{2.331163in}}%
\pgfusepath{clip}%
\pgfsetroundcap%
\pgfsetroundjoin%
\pgfsetlinewidth{0.803000pt}%
\definecolor{currentstroke}{rgb}{0.501961,0.501961,0.501961}%
\pgfsetstrokecolor{currentstroke}%
\pgfsetdash{}{0pt}%
\pgfpathmoveto{\pgfqpoint{3.973940in}{1.136000in}}%
\pgfpathquadraticcurveto{\pgfqpoint{3.968813in}{1.138542in}}{\pgfqpoint{3.974816in}{1.135566in}}%
\pgfusepath{stroke}%
\end{pgfscope}%
\begin{pgfscope}%
\pgfpathrectangle{\pgfqpoint{0.647939in}{0.492442in}}{\pgfqpoint{4.273799in}{2.331163in}}%
\pgfusepath{clip}%
\pgfsetroundcap%
\pgfsetroundjoin%
\definecolor{currentfill}{rgb}{0.501961,0.501961,0.501961}%
\pgfsetfillcolor{currentfill}%
\pgfsetlinewidth{0.803000pt}%
\definecolor{currentstroke}{rgb}{0.501961,0.501961,0.501961}%
\pgfsetstrokecolor{currentstroke}%
\pgfsetdash{}{0pt}%
\pgfpathmoveto{\pgfqpoint{4.019738in}{1.076089in}}%
\pgfpathlineto{\pgfqpoint{3.974816in}{1.135566in}}%
\pgfpathlineto{\pgfqpoint{4.049351in}{1.135817in}}%
\pgfpathlineto{\pgfqpoint{4.019738in}{1.076089in}}%
\pgfpathclose%
\pgfusepath{stroke,fill}%
\end{pgfscope}%
\begin{pgfscope}%
\pgfpathrectangle{\pgfqpoint{0.647939in}{0.492442in}}{\pgfqpoint{4.273799in}{2.331163in}}%
\pgfusepath{clip}%
\pgfsetroundcap%
\pgfsetroundjoin%
\pgfsetlinewidth{0.803000pt}%
\definecolor{currentstroke}{rgb}{0.501961,0.501961,0.501961}%
\pgfsetstrokecolor{currentstroke}%
\pgfsetdash{}{0pt}%
\pgfpathmoveto{\pgfqpoint{3.917989in}{1.312750in}}%
\pgfpathquadraticcurveto{\pgfqpoint{3.910931in}{1.315097in}}{\pgfqpoint{3.915661in}{1.313524in}}%
\pgfusepath{stroke}%
\end{pgfscope}%
\begin{pgfscope}%
\pgfpathrectangle{\pgfqpoint{0.647939in}{0.492442in}}{\pgfqpoint{4.273799in}{2.331163in}}%
\pgfusepath{clip}%
\pgfsetroundcap%
\pgfsetroundjoin%
\definecolor{currentfill}{rgb}{0.501961,0.501961,0.501961}%
\pgfsetfillcolor{currentfill}%
\pgfsetlinewidth{0.803000pt}%
\definecolor{currentstroke}{rgb}{0.501961,0.501961,0.501961}%
\pgfsetstrokecolor{currentstroke}%
\pgfsetdash{}{0pt}%
\pgfpathmoveto{\pgfqpoint{3.968402in}{1.260856in}}%
\pgfpathlineto{\pgfqpoint{3.915661in}{1.313524in}}%
\pgfpathlineto{\pgfqpoint{3.989440in}{1.324116in}}%
\pgfpathlineto{\pgfqpoint{3.968402in}{1.260856in}}%
\pgfpathclose%
\pgfusepath{stroke,fill}%
\end{pgfscope}%
\begin{pgfscope}%
\pgfpathrectangle{\pgfqpoint{0.647939in}{0.492442in}}{\pgfqpoint{4.273799in}{2.331163in}}%
\pgfusepath{clip}%
\pgfsetroundcap%
\pgfsetroundjoin%
\pgfsetlinewidth{0.803000pt}%
\definecolor{currentstroke}{rgb}{0.501961,0.501961,0.501961}%
\pgfsetstrokecolor{currentstroke}%
\pgfsetdash{}{0pt}%
\pgfpathmoveto{\pgfqpoint{4.203185in}{1.377005in}}%
\pgfpathquadraticcurveto{\pgfqpoint{4.199741in}{1.379301in}}{\pgfqpoint{4.206633in}{1.374705in}}%
\pgfusepath{stroke}%
\end{pgfscope}%
\begin{pgfscope}%
\pgfpathrectangle{\pgfqpoint{0.647939in}{0.492442in}}{\pgfqpoint{4.273799in}{2.331163in}}%
\pgfusepath{clip}%
\pgfsetroundcap%
\pgfsetroundjoin%
\definecolor{currentfill}{rgb}{0.501961,0.501961,0.501961}%
\pgfsetfillcolor{currentfill}%
\pgfsetlinewidth{0.803000pt}%
\definecolor{currentstroke}{rgb}{0.501961,0.501961,0.501961}%
\pgfsetstrokecolor{currentstroke}%
\pgfsetdash{}{0pt}%
\pgfpathmoveto{\pgfqpoint{4.243601in}{1.309983in}}%
\pgfpathlineto{\pgfqpoint{4.206633in}{1.374705in}}%
\pgfpathlineto{\pgfqpoint{4.280591in}{1.365446in}}%
\pgfpathlineto{\pgfqpoint{4.243601in}{1.309983in}}%
\pgfpathclose%
\pgfusepath{stroke,fill}%
\end{pgfscope}%
\begin{pgfscope}%
\pgfpathrectangle{\pgfqpoint{0.647939in}{0.492442in}}{\pgfqpoint{4.273799in}{2.331163in}}%
\pgfusepath{clip}%
\pgfsetroundcap%
\pgfsetroundjoin%
\pgfsetlinewidth{0.803000pt}%
\definecolor{currentstroke}{rgb}{0.501961,0.501961,0.501961}%
\pgfsetstrokecolor{currentstroke}%
\pgfsetdash{}{0pt}%
\pgfpathmoveto{\pgfqpoint{4.384718in}{1.509552in}}%
\pgfpathquadraticcurveto{\pgfqpoint{4.383863in}{1.510604in}}{\pgfqpoint{4.390842in}{1.502014in}}%
\pgfusepath{stroke}%
\end{pgfscope}%
\begin{pgfscope}%
\pgfpathrectangle{\pgfqpoint{0.647939in}{0.492442in}}{\pgfqpoint{4.273799in}{2.331163in}}%
\pgfusepath{clip}%
\pgfsetroundcap%
\pgfsetroundjoin%
\definecolor{currentfill}{rgb}{0.501961,0.501961,0.501961}%
\pgfsetfillcolor{currentfill}%
\pgfsetlinewidth{0.803000pt}%
\definecolor{currentstroke}{rgb}{0.501961,0.501961,0.501961}%
\pgfsetstrokecolor{currentstroke}%
\pgfsetdash{}{0pt}%
\pgfpathmoveto{\pgfqpoint{4.407008in}{1.429253in}}%
\pgfpathlineto{\pgfqpoint{4.390842in}{1.502014in}}%
\pgfpathlineto{\pgfqpoint{4.458751in}{1.471290in}}%
\pgfpathlineto{\pgfqpoint{4.407008in}{1.429253in}}%
\pgfpathclose%
\pgfusepath{stroke,fill}%
\end{pgfscope}%
\begin{pgfscope}%
\pgfpathrectangle{\pgfqpoint{0.647939in}{0.492442in}}{\pgfqpoint{4.273799in}{2.331163in}}%
\pgfusepath{clip}%
\pgfsetroundcap%
\pgfsetroundjoin%
\pgfsetlinewidth{0.803000pt}%
\definecolor{currentstroke}{rgb}{0.501961,0.501961,0.501961}%
\pgfsetstrokecolor{currentstroke}%
\pgfsetdash{}{0pt}%
\pgfpathmoveto{\pgfqpoint{4.477821in}{1.764858in}}%
\pgfpathquadraticcurveto{\pgfqpoint{4.472325in}{1.777454in}}{\pgfqpoint{4.471797in}{1.778665in}}%
\pgfusepath{stroke}%
\end{pgfscope}%
\begin{pgfscope}%
\pgfpathrectangle{\pgfqpoint{0.647939in}{0.492442in}}{\pgfqpoint{4.273799in}{2.331163in}}%
\pgfusepath{clip}%
\pgfsetroundcap%
\pgfsetroundjoin%
\definecolor{currentfill}{rgb}{0.501961,0.501961,0.501961}%
\pgfsetfillcolor{currentfill}%
\pgfsetlinewidth{0.803000pt}%
\definecolor{currentstroke}{rgb}{0.501961,0.501961,0.501961}%
\pgfsetstrokecolor{currentstroke}%
\pgfsetdash{}{0pt}%
\pgfpathmoveto{\pgfqpoint{4.467907in}{1.704231in}}%
\pgfpathlineto{\pgfqpoint{4.471797in}{1.778665in}}%
\pgfpathlineto{\pgfqpoint{4.529010in}{1.730893in}}%
\pgfpathlineto{\pgfqpoint{4.467907in}{1.704231in}}%
\pgfpathclose%
\pgfusepath{stroke,fill}%
\end{pgfscope}%
\begin{pgfscope}%
\pgfpathrectangle{\pgfqpoint{0.647939in}{0.492442in}}{\pgfqpoint{4.273799in}{2.331163in}}%
\pgfusepath{clip}%
\pgfsetroundcap%
\pgfsetroundjoin%
\pgfsetlinewidth{0.803000pt}%
\definecolor{currentstroke}{rgb}{0.501961,0.501961,0.501961}%
\pgfsetstrokecolor{currentstroke}%
\pgfsetdash{}{0pt}%
\pgfpathmoveto{\pgfqpoint{4.698710in}{1.684169in}}%
\pgfpathquadraticcurveto{\pgfqpoint{4.695627in}{1.697009in}}{\pgfqpoint{4.695445in}{1.697771in}}%
\pgfusepath{stroke}%
\end{pgfscope}%
\begin{pgfscope}%
\pgfpathrectangle{\pgfqpoint{0.647939in}{0.492442in}}{\pgfqpoint{4.273799in}{2.331163in}}%
\pgfusepath{clip}%
\pgfsetroundcap%
\pgfsetroundjoin%
\definecolor{currentfill}{rgb}{0.501961,0.501961,0.501961}%
\pgfsetfillcolor{currentfill}%
\pgfsetlinewidth{0.803000pt}%
\definecolor{currentstroke}{rgb}{0.501961,0.501961,0.501961}%
\pgfsetstrokecolor{currentstroke}%
\pgfsetdash{}{0pt}%
\pgfpathmoveto{\pgfqpoint{4.678595in}{1.625165in}}%
\pgfpathlineto{\pgfqpoint{4.695445in}{1.697771in}}%
\pgfpathlineto{\pgfqpoint{4.743419in}{1.640727in}}%
\pgfpathlineto{\pgfqpoint{4.678595in}{1.625165in}}%
\pgfpathclose%
\pgfusepath{stroke,fill}%
\end{pgfscope}%
\begin{pgfscope}%
\pgfpathrectangle{\pgfqpoint{0.647939in}{0.492442in}}{\pgfqpoint{4.273799in}{2.331163in}}%
\pgfusepath{clip}%
\pgfsetroundcap%
\pgfsetroundjoin%
\pgfsetlinewidth{0.803000pt}%
\definecolor{currentstroke}{rgb}{0.501961,0.501961,0.501961}%
\pgfsetstrokecolor{currentstroke}%
\pgfsetdash{}{0pt}%
\pgfpathmoveto{\pgfqpoint{4.777943in}{1.954755in}}%
\pgfpathquadraticcurveto{\pgfqpoint{4.777369in}{1.967702in}}{\pgfqpoint{4.777346in}{1.968238in}}%
\pgfusepath{stroke}%
\end{pgfscope}%
\begin{pgfscope}%
\pgfpathrectangle{\pgfqpoint{0.647939in}{0.492442in}}{\pgfqpoint{4.273799in}{2.331163in}}%
\pgfusepath{clip}%
\pgfsetroundcap%
\pgfsetroundjoin%
\definecolor{currentfill}{rgb}{0.501961,0.501961,0.501961}%
\pgfsetfillcolor{currentfill}%
\pgfsetlinewidth{0.803000pt}%
\definecolor{currentstroke}{rgb}{0.501961,0.501961,0.501961}%
\pgfsetstrokecolor{currentstroke}%
\pgfsetdash{}{0pt}%
\pgfpathmoveto{\pgfqpoint{4.746997in}{1.900161in}}%
\pgfpathlineto{\pgfqpoint{4.777346in}{1.968238in}}%
\pgfpathlineto{\pgfqpoint{4.813598in}{1.903112in}}%
\pgfpathlineto{\pgfqpoint{4.746997in}{1.900161in}}%
\pgfpathclose%
\pgfusepath{stroke,fill}%
\end{pgfscope}%
\begin{pgfscope}%
\pgfpathrectangle{\pgfqpoint{0.647939in}{0.492442in}}{\pgfqpoint{4.273799in}{2.331163in}}%
\pgfusepath{clip}%
\pgfsetroundcap%
\pgfsetroundjoin%
\pgfsetlinewidth{0.803000pt}%
\definecolor{currentstroke}{rgb}{0.501961,0.501961,0.501961}%
\pgfsetstrokecolor{currentstroke}%
\pgfsetdash{}{0pt}%
\pgfpathmoveto{\pgfqpoint{4.901708in}{1.602108in}}%
\pgfpathquadraticcurveto{\pgfqpoint{4.899458in}{1.615001in}}{\pgfqpoint{4.899343in}{1.615656in}}%
\pgfusepath{stroke}%
\end{pgfscope}%
\begin{pgfscope}%
\pgfpathrectangle{\pgfqpoint{0.647939in}{0.492442in}}{\pgfqpoint{4.273799in}{2.331163in}}%
\pgfusepath{clip}%
\pgfsetroundcap%
\pgfsetroundjoin%
\definecolor{currentfill}{rgb}{0.501961,0.501961,0.501961}%
\pgfsetfillcolor{currentfill}%
\pgfsetlinewidth{0.803000pt}%
\definecolor{currentstroke}{rgb}{0.501961,0.501961,0.501961}%
\pgfsetstrokecolor{currentstroke}%
\pgfsetdash{}{0pt}%
\pgfpathmoveto{\pgfqpoint{4.877968in}{1.544251in}}%
\pgfpathlineto{\pgfqpoint{4.899343in}{1.615656in}}%
\pgfpathlineto{\pgfqpoint{4.943642in}{1.555712in}}%
\pgfpathlineto{\pgfqpoint{4.877968in}{1.544251in}}%
\pgfpathclose%
\pgfusepath{stroke,fill}%
\end{pgfscope}%
\begin{pgfscope}%
\pgfpathrectangle{\pgfqpoint{0.647939in}{0.492442in}}{\pgfqpoint{4.273799in}{2.331163in}}%
\pgfusepath{clip}%
\pgfsetroundcap%
\pgfsetroundjoin%
\pgfsetlinewidth{0.803000pt}%
\definecolor{currentstroke}{rgb}{0.501961,0.501961,0.501961}%
\pgfsetstrokecolor{currentstroke}%
\pgfsetdash{}{0pt}%
\pgfpathmoveto{\pgfqpoint{4.910039in}{2.024060in}}%
\pgfpathquadraticcurveto{\pgfqpoint{4.910063in}{2.037011in}}{\pgfqpoint{4.910064in}{2.037539in}}%
\pgfusepath{stroke}%
\end{pgfscope}%
\begin{pgfscope}%
\pgfpathrectangle{\pgfqpoint{0.647939in}{0.492442in}}{\pgfqpoint{4.273799in}{2.331163in}}%
\pgfusepath{clip}%
\pgfsetroundcap%
\pgfsetroundjoin%
\definecolor{currentfill}{rgb}{0.501961,0.501961,0.501961}%
\pgfsetfillcolor{currentfill}%
\pgfsetlinewidth{0.803000pt}%
\definecolor{currentstroke}{rgb}{0.501961,0.501961,0.501961}%
\pgfsetstrokecolor{currentstroke}%
\pgfsetdash{}{0pt}%
\pgfpathmoveto{\pgfqpoint{4.876608in}{1.970934in}}%
\pgfpathlineto{\pgfqpoint{4.910064in}{2.037539in}}%
\pgfpathlineto{\pgfqpoint{4.943274in}{1.970811in}}%
\pgfpathlineto{\pgfqpoint{4.876608in}{1.970934in}}%
\pgfpathclose%
\pgfusepath{stroke,fill}%
\end{pgfscope}%
\begin{pgfscope}%
\pgfpathrectangle{\pgfqpoint{0.647939in}{0.492442in}}{\pgfqpoint{4.273799in}{2.331163in}}%
\pgfusepath{clip}%
\pgfsetroundcap%
\pgfsetroundjoin%
\pgfsetlinewidth{0.803000pt}%
\definecolor{currentstroke}{rgb}{0.501961,0.501961,0.501961}%
\pgfsetstrokecolor{currentstroke}%
\pgfsetdash{}{0pt}%
\pgfpathmoveto{\pgfqpoint{4.549614in}{2.700655in}}%
\pgfpathquadraticcurveto{\pgfqpoint{4.560751in}{2.700476in}}{\pgfqpoint{4.559467in}{2.700496in}}%
\pgfusepath{stroke}%
\end{pgfscope}%
\begin{pgfscope}%
\pgfpathrectangle{\pgfqpoint{0.647939in}{0.492442in}}{\pgfqpoint{4.273799in}{2.331163in}}%
\pgfusepath{clip}%
\pgfsetroundcap%
\pgfsetroundjoin%
\definecolor{currentfill}{rgb}{0.501961,0.501961,0.501961}%
\pgfsetfillcolor{currentfill}%
\pgfsetlinewidth{0.803000pt}%
\definecolor{currentstroke}{rgb}{0.501961,0.501961,0.501961}%
\pgfsetstrokecolor{currentstroke}%
\pgfsetdash{}{0pt}%
\pgfpathmoveto{\pgfqpoint{4.493344in}{2.734896in}}%
\pgfpathlineto{\pgfqpoint{4.559467in}{2.700496in}}%
\pgfpathlineto{\pgfqpoint{4.492273in}{2.668238in}}%
\pgfpathlineto{\pgfqpoint{4.493344in}{2.734896in}}%
\pgfpathclose%
\pgfusepath{stroke,fill}%
\end{pgfscope}%
\begin{pgfscope}%
\pgfpathrectangle{\pgfqpoint{0.647939in}{0.492442in}}{\pgfqpoint{4.273799in}{2.331163in}}%
\pgfusepath{clip}%
\pgfsetroundcap%
\pgfsetroundjoin%
\pgfsetlinewidth{0.803000pt}%
\definecolor{currentstroke}{rgb}{0.501961,0.501961,0.501961}%
\pgfsetstrokecolor{currentstroke}%
\pgfsetdash{}{0pt}%
\pgfpathmoveto{\pgfqpoint{4.457055in}{2.633510in}}%
\pgfpathquadraticcurveto{\pgfqpoint{4.457422in}{2.633338in}}{\pgfqpoint{4.446544in}{2.638449in}}%
\pgfusepath{stroke}%
\end{pgfscope}%
\begin{pgfscope}%
\pgfpathrectangle{\pgfqpoint{0.647939in}{0.492442in}}{\pgfqpoint{4.273799in}{2.331163in}}%
\pgfusepath{clip}%
\pgfsetroundcap%
\pgfsetroundjoin%
\definecolor{currentfill}{rgb}{0.501961,0.501961,0.501961}%
\pgfsetfillcolor{currentfill}%
\pgfsetlinewidth{0.803000pt}%
\definecolor{currentstroke}{rgb}{0.501961,0.501961,0.501961}%
\pgfsetstrokecolor{currentstroke}%
\pgfsetdash{}{0pt}%
\pgfpathmoveto{\pgfqpoint{4.400382in}{2.696969in}}%
\pgfpathlineto{\pgfqpoint{4.446544in}{2.638449in}}%
\pgfpathlineto{\pgfqpoint{4.372031in}{2.636631in}}%
\pgfpathlineto{\pgfqpoint{4.400382in}{2.696969in}}%
\pgfpathclose%
\pgfusepath{stroke,fill}%
\end{pgfscope}%
\begin{pgfscope}%
\pgfpathrectangle{\pgfqpoint{0.647939in}{0.492442in}}{\pgfqpoint{4.273799in}{2.331163in}}%
\pgfusepath{clip}%
\pgfsetroundcap%
\pgfsetroundjoin%
\pgfsetlinewidth{0.803000pt}%
\definecolor{currentstroke}{rgb}{0.501961,0.501961,0.501961}%
\pgfsetstrokecolor{currentstroke}%
\pgfsetdash{}{0pt}%
\pgfpathmoveto{\pgfqpoint{4.361834in}{2.570282in}}%
\pgfpathquadraticcurveto{\pgfqpoint{4.364196in}{2.568337in}}{\pgfqpoint{4.356968in}{2.574290in}}%
\pgfusepath{stroke}%
\end{pgfscope}%
\begin{pgfscope}%
\pgfpathrectangle{\pgfqpoint{0.647939in}{0.492442in}}{\pgfqpoint{4.273799in}{2.331163in}}%
\pgfusepath{clip}%
\pgfsetroundcap%
\pgfsetroundjoin%
\definecolor{currentfill}{rgb}{0.501961,0.501961,0.501961}%
\pgfsetfillcolor{currentfill}%
\pgfsetlinewidth{0.803000pt}%
\definecolor{currentstroke}{rgb}{0.501961,0.501961,0.501961}%
\pgfsetstrokecolor{currentstroke}%
\pgfsetdash{}{0pt}%
\pgfpathmoveto{\pgfqpoint{4.326695in}{2.642401in}}%
\pgfpathlineto{\pgfqpoint{4.356968in}{2.574290in}}%
\pgfpathlineto{\pgfqpoint{4.284315in}{2.590938in}}%
\pgfpathlineto{\pgfqpoint{4.326695in}{2.642401in}}%
\pgfpathclose%
\pgfusepath{stroke,fill}%
\end{pgfscope}%
\begin{pgfscope}%
\pgfpathrectangle{\pgfqpoint{0.647939in}{0.492442in}}{\pgfqpoint{4.273799in}{2.331163in}}%
\pgfusepath{clip}%
\pgfsetroundcap%
\pgfsetroundjoin%
\pgfsetlinewidth{0.803000pt}%
\definecolor{currentstroke}{rgb}{0.501961,0.501961,0.501961}%
\pgfsetstrokecolor{currentstroke}%
\pgfsetdash{}{0pt}%
\pgfpathmoveto{\pgfqpoint{4.267609in}{2.511632in}}%
\pgfpathquadraticcurveto{\pgfqpoint{4.268489in}{2.510559in}}{\pgfqpoint{4.261493in}{2.519092in}}%
\pgfusepath{stroke}%
\end{pgfscope}%
\begin{pgfscope}%
\pgfpathrectangle{\pgfqpoint{0.647939in}{0.492442in}}{\pgfqpoint{4.273799in}{2.331163in}}%
\pgfusepath{clip}%
\pgfsetroundcap%
\pgfsetroundjoin%
\definecolor{currentfill}{rgb}{0.501961,0.501961,0.501961}%
\pgfsetfillcolor{currentfill}%
\pgfsetlinewidth{0.803000pt}%
\definecolor{currentstroke}{rgb}{0.501961,0.501961,0.501961}%
\pgfsetstrokecolor{currentstroke}%
\pgfsetdash{}{0pt}%
\pgfpathmoveto{\pgfqpoint{4.245002in}{2.591781in}}%
\pgfpathlineto{\pgfqpoint{4.261493in}{2.519092in}}%
\pgfpathlineto{\pgfqpoint{4.193448in}{2.549513in}}%
\pgfpathlineto{\pgfqpoint{4.245002in}{2.591781in}}%
\pgfpathclose%
\pgfusepath{stroke,fill}%
\end{pgfscope}%
\begin{pgfscope}%
\pgfpathrectangle{\pgfqpoint{0.647939in}{0.492442in}}{\pgfqpoint{4.273799in}{2.331163in}}%
\pgfusepath{clip}%
\pgfsetroundcap%
\pgfsetroundjoin%
\pgfsetlinewidth{0.803000pt}%
\definecolor{currentstroke}{rgb}{0.501961,0.501961,0.501961}%
\pgfsetstrokecolor{currentstroke}%
\pgfsetdash{}{0pt}%
\pgfpathmoveto{\pgfqpoint{4.145883in}{2.505200in}}%
\pgfpathquadraticcurveto{\pgfqpoint{4.146156in}{2.504756in}}{\pgfqpoint{4.139929in}{2.514899in}}%
\pgfusepath{stroke}%
\end{pgfscope}%
\begin{pgfscope}%
\pgfpathrectangle{\pgfqpoint{0.647939in}{0.492442in}}{\pgfqpoint{4.273799in}{2.331163in}}%
\pgfusepath{clip}%
\pgfsetroundcap%
\pgfsetroundjoin%
\definecolor{currentfill}{rgb}{0.501961,0.501961,0.501961}%
\pgfsetfillcolor{currentfill}%
\pgfsetlinewidth{0.803000pt}%
\definecolor{currentstroke}{rgb}{0.501961,0.501961,0.501961}%
\pgfsetstrokecolor{currentstroke}%
\pgfsetdash{}{0pt}%
\pgfpathmoveto{\pgfqpoint{4.133455in}{2.589153in}}%
\pgfpathlineto{\pgfqpoint{4.139929in}{2.514899in}}%
\pgfpathlineto{\pgfqpoint{4.076641in}{2.554273in}}%
\pgfpathlineto{\pgfqpoint{4.133455in}{2.589153in}}%
\pgfpathclose%
\pgfusepath{stroke,fill}%
\end{pgfscope}%
\begin{pgfscope}%
\pgfpathrectangle{\pgfqpoint{0.647939in}{0.492442in}}{\pgfqpoint{4.273799in}{2.331163in}}%
\pgfusepath{clip}%
\pgfsetroundcap%
\pgfsetroundjoin%
\pgfsetlinewidth{0.803000pt}%
\definecolor{currentstroke}{rgb}{0.501961,0.501961,0.501961}%
\pgfsetstrokecolor{currentstroke}%
\pgfsetdash{}{0pt}%
\pgfpathmoveto{\pgfqpoint{4.204248in}{2.124752in}}%
\pgfpathquadraticcurveto{\pgfqpoint{4.206802in}{2.111890in}}{\pgfqpoint{4.206936in}{2.111212in}}%
\pgfusepath{stroke}%
\end{pgfscope}%
\begin{pgfscope}%
\pgfpathrectangle{\pgfqpoint{0.647939in}{0.492442in}}{\pgfqpoint{4.273799in}{2.331163in}}%
\pgfusepath{clip}%
\pgfsetroundcap%
\pgfsetroundjoin%
\definecolor{currentfill}{rgb}{0.501961,0.501961,0.501961}%
\pgfsetfillcolor{currentfill}%
\pgfsetlinewidth{0.803000pt}%
\definecolor{currentstroke}{rgb}{0.501961,0.501961,0.501961}%
\pgfsetstrokecolor{currentstroke}%
\pgfsetdash{}{0pt}%
\pgfpathmoveto{\pgfqpoint{4.226651in}{2.183093in}}%
\pgfpathlineto{\pgfqpoint{4.206936in}{2.111212in}}%
\pgfpathlineto{\pgfqpoint{4.161261in}{2.170112in}}%
\pgfpathlineto{\pgfqpoint{4.226651in}{2.183093in}}%
\pgfpathclose%
\pgfusepath{stroke,fill}%
\end{pgfscope}%
\begin{pgfscope}%
\pgfpathrectangle{\pgfqpoint{0.647939in}{0.492442in}}{\pgfqpoint{4.273799in}{2.331163in}}%
\pgfusepath{clip}%
\pgfsetroundcap%
\pgfsetroundjoin%
\pgfsetlinewidth{0.803000pt}%
\definecolor{currentstroke}{rgb}{0.501961,0.501961,0.501961}%
\pgfsetstrokecolor{currentstroke}%
\pgfsetdash{}{0pt}%
\pgfpathmoveto{\pgfqpoint{3.990567in}{2.375399in}}%
\pgfpathquadraticcurveto{\pgfqpoint{3.995774in}{2.362764in}}{\pgfqpoint{3.996248in}{2.361614in}}%
\pgfusepath{stroke}%
\end{pgfscope}%
\begin{pgfscope}%
\pgfpathrectangle{\pgfqpoint{0.647939in}{0.492442in}}{\pgfqpoint{4.273799in}{2.331163in}}%
\pgfusepath{clip}%
\pgfsetroundcap%
\pgfsetroundjoin%
\definecolor{currentfill}{rgb}{0.501961,0.501961,0.501961}%
\pgfsetfillcolor{currentfill}%
\pgfsetlinewidth{0.803000pt}%
\definecolor{currentstroke}{rgb}{0.501961,0.501961,0.501961}%
\pgfsetstrokecolor{currentstroke}%
\pgfsetdash{}{0pt}%
\pgfpathmoveto{\pgfqpoint{4.001666in}{2.435953in}}%
\pgfpathlineto{\pgfqpoint{3.996248in}{2.361614in}}%
\pgfpathlineto{\pgfqpoint{3.940028in}{2.410552in}}%
\pgfpathlineto{\pgfqpoint{4.001666in}{2.435953in}}%
\pgfpathclose%
\pgfusepath{stroke,fill}%
\end{pgfscope}%
\begin{pgfscope}%
\pgfpathrectangle{\pgfqpoint{0.647939in}{0.492442in}}{\pgfqpoint{4.273799in}{2.331163in}}%
\pgfusepath{clip}%
\pgfsetroundcap%
\pgfsetroundjoin%
\pgfsetlinewidth{0.803000pt}%
\definecolor{currentstroke}{rgb}{0.501961,0.501961,0.501961}%
\pgfsetstrokecolor{currentstroke}%
\pgfsetdash{}{0pt}%
\pgfpathmoveto{\pgfqpoint{3.925708in}{2.273137in}}%
\pgfpathquadraticcurveto{\pgfqpoint{3.929502in}{2.260354in}}{\pgfqpoint{3.929762in}{2.259480in}}%
\pgfusepath{stroke}%
\end{pgfscope}%
\begin{pgfscope}%
\pgfpathrectangle{\pgfqpoint{0.647939in}{0.492442in}}{\pgfqpoint{4.273799in}{2.331163in}}%
\pgfusepath{clip}%
\pgfsetroundcap%
\pgfsetroundjoin%
\definecolor{currentfill}{rgb}{0.501961,0.501961,0.501961}%
\pgfsetfillcolor{currentfill}%
\pgfsetlinewidth{0.803000pt}%
\definecolor{currentstroke}{rgb}{0.501961,0.501961,0.501961}%
\pgfsetstrokecolor{currentstroke}%
\pgfsetdash{}{0pt}%
\pgfpathmoveto{\pgfqpoint{3.942747in}{2.332876in}}%
\pgfpathlineto{\pgfqpoint{3.929762in}{2.259480in}}%
\pgfpathlineto{\pgfqpoint{3.878836in}{2.313906in}}%
\pgfpathlineto{\pgfqpoint{3.942747in}{2.332876in}}%
\pgfpathclose%
\pgfusepath{stroke,fill}%
\end{pgfscope}%
\begin{pgfscope}%
\pgfpathrectangle{\pgfqpoint{0.647939in}{0.492442in}}{\pgfqpoint{4.273799in}{2.331163in}}%
\pgfusepath{clip}%
\pgfsetroundcap%
\pgfsetroundjoin%
\pgfsetlinewidth{0.803000pt}%
\definecolor{currentstroke}{rgb}{0.501961,0.501961,0.501961}%
\pgfsetstrokecolor{currentstroke}%
\pgfsetdash{}{0pt}%
\pgfpathmoveto{\pgfqpoint{3.845323in}{1.912716in}}%
\pgfpathquadraticcurveto{\pgfqpoint{3.839543in}{1.900177in}}{\pgfqpoint{3.838963in}{1.898919in}}%
\pgfusepath{stroke}%
\end{pgfscope}%
\begin{pgfscope}%
\pgfpathrectangle{\pgfqpoint{0.647939in}{0.492442in}}{\pgfqpoint{4.273799in}{2.331163in}}%
\pgfusepath{clip}%
\pgfsetroundcap%
\pgfsetroundjoin%
\definecolor{currentfill}{rgb}{0.501961,0.501961,0.501961}%
\pgfsetfillcolor{currentfill}%
\pgfsetlinewidth{0.803000pt}%
\definecolor{currentstroke}{rgb}{0.501961,0.501961,0.501961}%
\pgfsetstrokecolor{currentstroke}%
\pgfsetdash{}{0pt}%
\pgfpathmoveto{\pgfqpoint{3.897143in}{1.945509in}}%
\pgfpathlineto{\pgfqpoint{3.838963in}{1.898919in}}%
\pgfpathlineto{\pgfqpoint{3.836599in}{1.973417in}}%
\pgfpathlineto{\pgfqpoint{3.897143in}{1.945509in}}%
\pgfpathclose%
\pgfusepath{stroke,fill}%
\end{pgfscope}%
\begin{pgfscope}%
\pgfpathrectangle{\pgfqpoint{0.647939in}{0.492442in}}{\pgfqpoint{4.273799in}{2.331163in}}%
\pgfusepath{clip}%
\pgfsetroundcap%
\pgfsetroundjoin%
\pgfsetlinewidth{0.803000pt}%
\definecolor{currentstroke}{rgb}{0.501961,0.501961,0.501961}%
\pgfsetstrokecolor{currentstroke}%
\pgfsetdash{}{0pt}%
\pgfpathmoveto{\pgfqpoint{3.750813in}{2.222896in}}%
\pgfpathquadraticcurveto{\pgfqpoint{3.753695in}{2.210043in}}{\pgfqpoint{3.753860in}{2.209311in}}%
\pgfusepath{stroke}%
\end{pgfscope}%
\begin{pgfscope}%
\pgfpathrectangle{\pgfqpoint{0.647939in}{0.492442in}}{\pgfqpoint{4.273799in}{2.331163in}}%
\pgfusepath{clip}%
\pgfsetroundcap%
\pgfsetroundjoin%
\definecolor{currentfill}{rgb}{0.501961,0.501961,0.501961}%
\pgfsetfillcolor{currentfill}%
\pgfsetlinewidth{0.803000pt}%
\definecolor{currentstroke}{rgb}{0.501961,0.501961,0.501961}%
\pgfsetstrokecolor{currentstroke}%
\pgfsetdash{}{0pt}%
\pgfpathmoveto{\pgfqpoint{3.771797in}{2.281656in}}%
\pgfpathlineto{\pgfqpoint{3.753860in}{2.209311in}}%
\pgfpathlineto{\pgfqpoint{3.706746in}{2.267068in}}%
\pgfpathlineto{\pgfqpoint{3.771797in}{2.281656in}}%
\pgfpathclose%
\pgfusepath{stroke,fill}%
\end{pgfscope}%
\begin{pgfscope}%
\pgfpathrectangle{\pgfqpoint{0.647939in}{0.492442in}}{\pgfqpoint{4.273799in}{2.331163in}}%
\pgfusepath{clip}%
\pgfsetroundcap%
\pgfsetroundjoin%
\pgfsetlinewidth{0.803000pt}%
\definecolor{currentstroke}{rgb}{0.501961,0.501961,0.501961}%
\pgfsetstrokecolor{currentstroke}%
\pgfsetdash{}{0pt}%
\pgfpathmoveto{\pgfqpoint{3.604346in}{2.124293in}}%
\pgfpathquadraticcurveto{\pgfqpoint{3.605556in}{2.111363in}}{\pgfqpoint{3.605609in}{2.110802in}}%
\pgfusepath{stroke}%
\end{pgfscope}%
\begin{pgfscope}%
\pgfpathrectangle{\pgfqpoint{0.647939in}{0.492442in}}{\pgfqpoint{4.273799in}{2.331163in}}%
\pgfusepath{clip}%
\pgfsetroundcap%
\pgfsetroundjoin%
\definecolor{currentfill}{rgb}{0.501961,0.501961,0.501961}%
\pgfsetfillcolor{currentfill}%
\pgfsetlinewidth{0.803000pt}%
\definecolor{currentstroke}{rgb}{0.501961,0.501961,0.501961}%
\pgfsetstrokecolor{currentstroke}%
\pgfsetdash{}{0pt}%
\pgfpathmoveto{\pgfqpoint{3.632585in}{2.180284in}}%
\pgfpathlineto{\pgfqpoint{3.605609in}{2.110802in}}%
\pgfpathlineto{\pgfqpoint{3.566208in}{2.174072in}}%
\pgfpathlineto{\pgfqpoint{3.632585in}{2.180284in}}%
\pgfpathclose%
\pgfusepath{stroke,fill}%
\end{pgfscope}%
\begin{pgfscope}%
\pgfpathrectangle{\pgfqpoint{0.647939in}{0.492442in}}{\pgfqpoint{4.273799in}{2.331163in}}%
\pgfusepath{clip}%
\pgfsetroundcap%
\pgfsetroundjoin%
\pgfsetlinewidth{0.803000pt}%
\definecolor{currentstroke}{rgb}{0.501961,0.501961,0.501961}%
\pgfsetstrokecolor{currentstroke}%
\pgfsetdash{}{0pt}%
\pgfpathmoveto{\pgfqpoint{3.296077in}{2.656587in}}%
\pgfpathquadraticcurveto{\pgfqpoint{3.296498in}{2.655957in}}{\pgfqpoint{3.290009in}{2.665650in}}%
\pgfusepath{stroke}%
\end{pgfscope}%
\begin{pgfscope}%
\pgfpathrectangle{\pgfqpoint{0.647939in}{0.492442in}}{\pgfqpoint{4.273799in}{2.331163in}}%
\pgfusepath{clip}%
\pgfsetroundcap%
\pgfsetroundjoin%
\definecolor{currentfill}{rgb}{0.501961,0.501961,0.501961}%
\pgfsetfillcolor{currentfill}%
\pgfsetlinewidth{0.803000pt}%
\definecolor{currentstroke}{rgb}{0.501961,0.501961,0.501961}%
\pgfsetstrokecolor{currentstroke}%
\pgfsetdash{}{0pt}%
\pgfpathmoveto{\pgfqpoint{3.280619in}{2.739592in}}%
\pgfpathlineto{\pgfqpoint{3.290009in}{2.665650in}}%
\pgfpathlineto{\pgfqpoint{3.225221in}{2.702504in}}%
\pgfpathlineto{\pgfqpoint{3.280619in}{2.739592in}}%
\pgfpathclose%
\pgfusepath{stroke,fill}%
\end{pgfscope}%
\begin{pgfscope}%
\pgfpathrectangle{\pgfqpoint{0.647939in}{0.492442in}}{\pgfqpoint{4.273799in}{2.331163in}}%
\pgfusepath{clip}%
\pgfsetroundcap%
\pgfsetroundjoin%
\pgfsetlinewidth{0.803000pt}%
\definecolor{currentstroke}{rgb}{0.501961,0.501961,0.501961}%
\pgfsetstrokecolor{currentstroke}%
\pgfsetdash{}{0pt}%
\pgfpathmoveto{\pgfqpoint{3.371830in}{2.242567in}}%
\pgfpathquadraticcurveto{\pgfqpoint{3.375871in}{2.229808in}}{\pgfqpoint{3.376161in}{2.228891in}}%
\pgfusepath{stroke}%
\end{pgfscope}%
\begin{pgfscope}%
\pgfpathrectangle{\pgfqpoint{0.647939in}{0.492442in}}{\pgfqpoint{4.273799in}{2.331163in}}%
\pgfusepath{clip}%
\pgfsetroundcap%
\pgfsetroundjoin%
\definecolor{currentfill}{rgb}{0.501961,0.501961,0.501961}%
\pgfsetfillcolor{currentfill}%
\pgfsetlinewidth{0.803000pt}%
\definecolor{currentstroke}{rgb}{0.501961,0.501961,0.501961}%
\pgfsetstrokecolor{currentstroke}%
\pgfsetdash{}{0pt}%
\pgfpathmoveto{\pgfqpoint{3.387810in}{2.302511in}}%
\pgfpathlineto{\pgfqpoint{3.376161in}{2.228891in}}%
\pgfpathlineto{\pgfqpoint{3.324255in}{2.282382in}}%
\pgfpathlineto{\pgfqpoint{3.387810in}{2.302511in}}%
\pgfpathclose%
\pgfusepath{stroke,fill}%
\end{pgfscope}%
\begin{pgfscope}%
\pgfpathrectangle{\pgfqpoint{0.647939in}{0.492442in}}{\pgfqpoint{4.273799in}{2.331163in}}%
\pgfusepath{clip}%
\pgfsetroundcap%
\pgfsetroundjoin%
\pgfsetlinewidth{0.803000pt}%
\definecolor{currentstroke}{rgb}{0.501961,0.501961,0.501961}%
\pgfsetstrokecolor{currentstroke}%
\pgfsetdash{}{0pt}%
\pgfpathmoveto{\pgfqpoint{3.193043in}{2.387953in}}%
\pgfpathquadraticcurveto{\pgfqpoint{3.193332in}{2.387485in}}{\pgfqpoint{3.187088in}{2.397584in}}%
\pgfusepath{stroke}%
\end{pgfscope}%
\begin{pgfscope}%
\pgfpathrectangle{\pgfqpoint{0.647939in}{0.492442in}}{\pgfqpoint{4.273799in}{2.331163in}}%
\pgfusepath{clip}%
\pgfsetroundcap%
\pgfsetroundjoin%
\definecolor{currentfill}{rgb}{0.501961,0.501961,0.501961}%
\pgfsetfillcolor{currentfill}%
\pgfsetlinewidth{0.803000pt}%
\definecolor{currentstroke}{rgb}{0.501961,0.501961,0.501961}%
\pgfsetstrokecolor{currentstroke}%
\pgfsetdash{}{0pt}%
\pgfpathmoveto{\pgfqpoint{3.180375in}{2.471816in}}%
\pgfpathlineto{\pgfqpoint{3.187088in}{2.397584in}}%
\pgfpathlineto{\pgfqpoint{3.123674in}{2.436753in}}%
\pgfpathlineto{\pgfqpoint{3.180375in}{2.471816in}}%
\pgfpathclose%
\pgfusepath{stroke,fill}%
\end{pgfscope}%
\begin{pgfscope}%
\pgfpathrectangle{\pgfqpoint{0.647939in}{0.492442in}}{\pgfqpoint{4.273799in}{2.331163in}}%
\pgfusepath{clip}%
\pgfsetroundcap%
\pgfsetroundjoin%
\pgfsetlinewidth{0.803000pt}%
\definecolor{currentstroke}{rgb}{0.501961,0.501961,0.501961}%
\pgfsetstrokecolor{currentstroke}%
\pgfsetdash{}{0pt}%
\pgfpathmoveto{\pgfqpoint{2.801412in}{2.690782in}}%
\pgfpathquadraticcurveto{\pgfqpoint{2.804347in}{2.688616in}}{\pgfqpoint{2.797285in}{2.693827in}}%
\pgfusepath{stroke}%
\end{pgfscope}%
\begin{pgfscope}%
\pgfpathrectangle{\pgfqpoint{0.647939in}{0.492442in}}{\pgfqpoint{4.273799in}{2.331163in}}%
\pgfusepath{clip}%
\pgfsetroundcap%
\pgfsetroundjoin%
\definecolor{currentfill}{rgb}{0.501961,0.501961,0.501961}%
\pgfsetfillcolor{currentfill}%
\pgfsetlinewidth{0.803000pt}%
\definecolor{currentstroke}{rgb}{0.501961,0.501961,0.501961}%
\pgfsetstrokecolor{currentstroke}%
\pgfsetdash{}{0pt}%
\pgfpathmoveto{\pgfqpoint{2.763434in}{2.760232in}}%
\pgfpathlineto{\pgfqpoint{2.797285in}{2.693827in}}%
\pgfpathlineto{\pgfqpoint{2.723850in}{2.706588in}}%
\pgfpathlineto{\pgfqpoint{2.763434in}{2.760232in}}%
\pgfpathclose%
\pgfusepath{stroke,fill}%
\end{pgfscope}%
\begin{pgfscope}%
\pgfpathrectangle{\pgfqpoint{0.647939in}{0.492442in}}{\pgfqpoint{4.273799in}{2.331163in}}%
\pgfusepath{clip}%
\pgfsetroundcap%
\pgfsetroundjoin%
\pgfsetlinewidth{0.803000pt}%
\definecolor{currentstroke}{rgb}{0.501961,0.501961,0.501961}%
\pgfsetstrokecolor{currentstroke}%
\pgfsetdash{}{0pt}%
\pgfpathmoveto{\pgfqpoint{2.657747in}{2.739962in}}%
\pgfpathquadraticcurveto{\pgfqpoint{2.662047in}{2.737500in}}{\pgfqpoint{2.655567in}{2.741211in}}%
\pgfusepath{stroke}%
\end{pgfscope}%
\begin{pgfscope}%
\pgfpathrectangle{\pgfqpoint{0.647939in}{0.492442in}}{\pgfqpoint{4.273799in}{2.331163in}}%
\pgfusepath{clip}%
\pgfsetroundcap%
\pgfsetroundjoin%
\definecolor{currentfill}{rgb}{0.501961,0.501961,0.501961}%
\pgfsetfillcolor{currentfill}%
\pgfsetlinewidth{0.803000pt}%
\definecolor{currentstroke}{rgb}{0.501961,0.501961,0.501961}%
\pgfsetstrokecolor{currentstroke}%
\pgfsetdash{}{0pt}%
\pgfpathmoveto{\pgfqpoint{2.614278in}{2.803266in}}%
\pgfpathlineto{\pgfqpoint{2.655567in}{2.741211in}}%
\pgfpathlineto{\pgfqpoint{2.581150in}{2.745412in}}%
\pgfpathlineto{\pgfqpoint{2.614278in}{2.803266in}}%
\pgfpathclose%
\pgfusepath{stroke,fill}%
\end{pgfscope}%
\begin{pgfscope}%
\pgfpathrectangle{\pgfqpoint{0.647939in}{0.492442in}}{\pgfqpoint{4.273799in}{2.331163in}}%
\pgfusepath{clip}%
\pgfsetroundcap%
\pgfsetroundjoin%
\pgfsetlinewidth{0.803000pt}%
\definecolor{currentstroke}{rgb}{0.501961,0.501961,0.501961}%
\pgfsetstrokecolor{currentstroke}%
\pgfsetdash{}{0pt}%
\pgfpathmoveto{\pgfqpoint{2.483044in}{2.755850in}}%
\pgfpathquadraticcurveto{\pgfqpoint{2.489239in}{2.753340in}}{\pgfqpoint{2.483921in}{2.755495in}}%
\pgfusepath{stroke}%
\end{pgfscope}%
\begin{pgfscope}%
\pgfpathrectangle{\pgfqpoint{0.647939in}{0.492442in}}{\pgfqpoint{4.273799in}{2.331163in}}%
\pgfusepath{clip}%
\pgfsetroundcap%
\pgfsetroundjoin%
\definecolor{currentfill}{rgb}{0.501961,0.501961,0.501961}%
\pgfsetfillcolor{currentfill}%
\pgfsetlinewidth{0.803000pt}%
\definecolor{currentstroke}{rgb}{0.501961,0.501961,0.501961}%
\pgfsetstrokecolor{currentstroke}%
\pgfsetdash{}{0pt}%
\pgfpathmoveto{\pgfqpoint{2.434645in}{2.811418in}}%
\pgfpathlineto{\pgfqpoint{2.483921in}{2.755495in}}%
\pgfpathlineto{\pgfqpoint{2.409617in}{2.749628in}}%
\pgfpathlineto{\pgfqpoint{2.434645in}{2.811418in}}%
\pgfpathclose%
\pgfusepath{stroke,fill}%
\end{pgfscope}%
\begin{pgfscope}%
\pgfpathrectangle{\pgfqpoint{0.647939in}{0.492442in}}{\pgfqpoint{4.273799in}{2.331163in}}%
\pgfusepath{clip}%
\pgfsetroundcap%
\pgfsetroundjoin%
\pgfsetlinewidth{0.803000pt}%
\definecolor{currentstroke}{rgb}{0.501961,0.501961,0.501961}%
\pgfsetstrokecolor{currentstroke}%
\pgfsetdash{}{0pt}%
\pgfpathmoveto{\pgfqpoint{2.751998in}{2.563058in}}%
\pgfpathquadraticcurveto{\pgfqpoint{2.756174in}{2.560605in}}{\pgfqpoint{2.749639in}{2.564444in}}%
\pgfusepath{stroke}%
\end{pgfscope}%
\begin{pgfscope}%
\pgfpathrectangle{\pgfqpoint{0.647939in}{0.492442in}}{\pgfqpoint{4.273799in}{2.331163in}}%
\pgfusepath{clip}%
\pgfsetroundcap%
\pgfsetroundjoin%
\definecolor{currentfill}{rgb}{0.501961,0.501961,0.501961}%
\pgfsetfillcolor{currentfill}%
\pgfsetlinewidth{0.803000pt}%
\definecolor{currentstroke}{rgb}{0.501961,0.501961,0.501961}%
\pgfsetstrokecolor{currentstroke}%
\pgfsetdash{}{0pt}%
\pgfpathmoveto{\pgfqpoint{2.709040in}{2.626952in}}%
\pgfpathlineto{\pgfqpoint{2.749639in}{2.564444in}}%
\pgfpathlineto{\pgfqpoint{2.675273in}{2.569469in}}%
\pgfpathlineto{\pgfqpoint{2.709040in}{2.626952in}}%
\pgfpathclose%
\pgfusepath{stroke,fill}%
\end{pgfscope}%
\begin{pgfscope}%
\pgfpathrectangle{\pgfqpoint{0.647939in}{0.492442in}}{\pgfqpoint{4.273799in}{2.331163in}}%
\pgfusepath{clip}%
\pgfsetroundcap%
\pgfsetroundjoin%
\pgfsetlinewidth{0.803000pt}%
\definecolor{currentstroke}{rgb}{0.501961,0.501961,0.501961}%
\pgfsetstrokecolor{currentstroke}%
\pgfsetdash{}{0pt}%
\pgfpathmoveto{\pgfqpoint{2.234591in}{2.680429in}}%
\pgfpathquadraticcurveto{\pgfqpoint{2.243381in}{2.678779in}}{\pgfqpoint{2.239961in}{2.679421in}}%
\pgfusepath{stroke}%
\end{pgfscope}%
\begin{pgfscope}%
\pgfpathrectangle{\pgfqpoint{0.647939in}{0.492442in}}{\pgfqpoint{4.273799in}{2.331163in}}%
\pgfusepath{clip}%
\pgfsetroundcap%
\pgfsetroundjoin%
\definecolor{currentfill}{rgb}{0.501961,0.501961,0.501961}%
\pgfsetfillcolor{currentfill}%
\pgfsetlinewidth{0.803000pt}%
\definecolor{currentstroke}{rgb}{0.501961,0.501961,0.501961}%
\pgfsetstrokecolor{currentstroke}%
\pgfsetdash{}{0pt}%
\pgfpathmoveto{\pgfqpoint{2.180585in}{2.724478in}}%
\pgfpathlineto{\pgfqpoint{2.239961in}{2.679421in}}%
\pgfpathlineto{\pgfqpoint{2.168290in}{2.658955in}}%
\pgfpathlineto{\pgfqpoint{2.180585in}{2.724478in}}%
\pgfpathclose%
\pgfusepath{stroke,fill}%
\end{pgfscope}%
\begin{pgfscope}%
\pgfpathrectangle{\pgfqpoint{0.647939in}{0.492442in}}{\pgfqpoint{4.273799in}{2.331163in}}%
\pgfusepath{clip}%
\pgfsetroundcap%
\pgfsetroundjoin%
\pgfsetlinewidth{0.803000pt}%
\definecolor{currentstroke}{rgb}{0.501961,0.501961,0.501961}%
\pgfsetstrokecolor{currentstroke}%
\pgfsetdash{}{0pt}%
\pgfpathmoveto{\pgfqpoint{1.860910in}{2.731463in}}%
\pgfpathquadraticcurveto{\pgfqpoint{1.866231in}{2.728938in}}{\pgfqpoint{1.860329in}{2.731738in}}%
\pgfusepath{stroke}%
\end{pgfscope}%
\begin{pgfscope}%
\pgfpathrectangle{\pgfqpoint{0.647939in}{0.492442in}}{\pgfqpoint{4.273799in}{2.331163in}}%
\pgfusepath{clip}%
\pgfsetroundcap%
\pgfsetroundjoin%
\definecolor{currentfill}{rgb}{0.501961,0.501961,0.501961}%
\pgfsetfillcolor{currentfill}%
\pgfsetlinewidth{0.803000pt}%
\definecolor{currentstroke}{rgb}{0.501961,0.501961,0.501961}%
\pgfsetstrokecolor{currentstroke}%
\pgfsetdash{}{0pt}%
\pgfpathmoveto{\pgfqpoint{1.814387in}{2.790431in}}%
\pgfpathlineto{\pgfqpoint{1.860329in}{2.731738in}}%
\pgfpathlineto{\pgfqpoint{1.785809in}{2.730200in}}%
\pgfpathlineto{\pgfqpoint{1.814387in}{2.790431in}}%
\pgfpathclose%
\pgfusepath{stroke,fill}%
\end{pgfscope}%
\begin{pgfscope}%
\pgfpathrectangle{\pgfqpoint{0.647939in}{0.492442in}}{\pgfqpoint{4.273799in}{2.331163in}}%
\pgfusepath{clip}%
\pgfsetroundcap%
\pgfsetroundjoin%
\pgfsetlinewidth{0.803000pt}%
\definecolor{currentstroke}{rgb}{0.501961,0.501961,0.501961}%
\pgfsetstrokecolor{currentstroke}%
\pgfsetdash{}{0pt}%
\pgfpathmoveto{\pgfqpoint{2.065001in}{2.525759in}}%
\pgfpathquadraticcurveto{\pgfqpoint{2.072310in}{2.526007in}}{\pgfqpoint{2.067203in}{2.525834in}}%
\pgfusepath{stroke}%
\end{pgfscope}%
\begin{pgfscope}%
\pgfpathrectangle{\pgfqpoint{0.647939in}{0.492442in}}{\pgfqpoint{4.273799in}{2.331163in}}%
\pgfusepath{clip}%
\pgfsetroundcap%
\pgfsetroundjoin%
\definecolor{currentfill}{rgb}{0.501961,0.501961,0.501961}%
\pgfsetfillcolor{currentfill}%
\pgfsetlinewidth{0.803000pt}%
\definecolor{currentstroke}{rgb}{0.501961,0.501961,0.501961}%
\pgfsetstrokecolor{currentstroke}%
\pgfsetdash{}{0pt}%
\pgfpathmoveto{\pgfqpoint{1.999445in}{2.556890in}}%
\pgfpathlineto{\pgfqpoint{2.067203in}{2.525834in}}%
\pgfpathlineto{\pgfqpoint{2.001703in}{2.490262in}}%
\pgfpathlineto{\pgfqpoint{1.999445in}{2.556890in}}%
\pgfpathclose%
\pgfusepath{stroke,fill}%
\end{pgfscope}%
\begin{pgfscope}%
\pgfpathrectangle{\pgfqpoint{0.647939in}{0.492442in}}{\pgfqpoint{4.273799in}{2.331163in}}%
\pgfusepath{clip}%
\pgfsetroundcap%
\pgfsetroundjoin%
\pgfsetlinewidth{0.803000pt}%
\definecolor{currentstroke}{rgb}{0.501961,0.501961,0.501961}%
\pgfsetstrokecolor{currentstroke}%
\pgfsetdash{}{0pt}%
\pgfpathmoveto{\pgfqpoint{1.802938in}{2.431822in}}%
\pgfpathquadraticcurveto{\pgfqpoint{1.816566in}{2.429404in}}{\pgfqpoint{1.817962in}{2.429156in}}%
\pgfusepath{stroke}%
\end{pgfscope}%
\begin{pgfscope}%
\pgfpathrectangle{\pgfqpoint{0.647939in}{0.492442in}}{\pgfqpoint{4.273799in}{2.331163in}}%
\pgfusepath{clip}%
\pgfsetroundcap%
\pgfsetroundjoin%
\definecolor{currentfill}{rgb}{0.501961,0.501961,0.501961}%
\pgfsetfillcolor{currentfill}%
\pgfsetlinewidth{0.803000pt}%
\definecolor{currentstroke}{rgb}{0.501961,0.501961,0.501961}%
\pgfsetstrokecolor{currentstroke}%
\pgfsetdash{}{0pt}%
\pgfpathmoveto{\pgfqpoint{1.758145in}{2.473624in}}%
\pgfpathlineto{\pgfqpoint{1.817962in}{2.429156in}}%
\pgfpathlineto{\pgfqpoint{1.746497in}{2.407983in}}%
\pgfpathlineto{\pgfqpoint{1.758145in}{2.473624in}}%
\pgfpathclose%
\pgfusepath{stroke,fill}%
\end{pgfscope}%
\begin{pgfscope}%
\pgfpathrectangle{\pgfqpoint{0.647939in}{0.492442in}}{\pgfqpoint{4.273799in}{2.331163in}}%
\pgfusepath{clip}%
\pgfsetroundcap%
\pgfsetroundjoin%
\pgfsetlinewidth{0.803000pt}%
\definecolor{currentstroke}{rgb}{0.501961,0.501961,0.501961}%
\pgfsetstrokecolor{currentstroke}%
\pgfsetdash{}{0pt}%
\pgfpathmoveto{\pgfqpoint{1.527367in}{2.451840in}}%
\pgfpathquadraticcurveto{\pgfqpoint{1.527677in}{2.451350in}}{\pgfqpoint{1.521354in}{2.461364in}}%
\pgfusepath{stroke}%
\end{pgfscope}%
\begin{pgfscope}%
\pgfpathrectangle{\pgfqpoint{0.647939in}{0.492442in}}{\pgfqpoint{4.273799in}{2.331163in}}%
\pgfusepath{clip}%
\pgfsetroundcap%
\pgfsetroundjoin%
\definecolor{currentfill}{rgb}{0.501961,0.501961,0.501961}%
\pgfsetfillcolor{currentfill}%
\pgfsetlinewidth{0.803000pt}%
\definecolor{currentstroke}{rgb}{0.501961,0.501961,0.501961}%
\pgfsetstrokecolor{currentstroke}%
\pgfsetdash{}{0pt}%
\pgfpathmoveto{\pgfqpoint{1.513945in}{2.535530in}}%
\pgfpathlineto{\pgfqpoint{1.521354in}{2.461364in}}%
\pgfpathlineto{\pgfqpoint{1.457575in}{2.499936in}}%
\pgfpathlineto{\pgfqpoint{1.513945in}{2.535530in}}%
\pgfpathclose%
\pgfusepath{stroke,fill}%
\end{pgfscope}%
\begin{pgfscope}%
\pgfpathrectangle{\pgfqpoint{0.647939in}{0.492442in}}{\pgfqpoint{4.273799in}{2.331163in}}%
\pgfusepath{clip}%
\pgfsetroundcap%
\pgfsetroundjoin%
\pgfsetlinewidth{0.803000pt}%
\definecolor{currentstroke}{rgb}{0.501961,0.501961,0.501961}%
\pgfsetstrokecolor{currentstroke}%
\pgfsetdash{}{0pt}%
\pgfpathmoveto{\pgfqpoint{1.423949in}{2.316444in}}%
\pgfpathquadraticcurveto{\pgfqpoint{1.428146in}{2.303698in}}{\pgfqpoint{1.428457in}{2.302752in}}%
\pgfusepath{stroke}%
\end{pgfscope}%
\begin{pgfscope}%
\pgfpathrectangle{\pgfqpoint{0.647939in}{0.492442in}}{\pgfqpoint{4.273799in}{2.331163in}}%
\pgfusepath{clip}%
\pgfsetroundcap%
\pgfsetroundjoin%
\definecolor{currentfill}{rgb}{0.501961,0.501961,0.501961}%
\pgfsetfillcolor{currentfill}%
\pgfsetlinewidth{0.803000pt}%
\definecolor{currentstroke}{rgb}{0.501961,0.501961,0.501961}%
\pgfsetstrokecolor{currentstroke}%
\pgfsetdash{}{0pt}%
\pgfpathmoveto{\pgfqpoint{1.439269in}{2.376499in}}%
\pgfpathlineto{\pgfqpoint{1.428457in}{2.302752in}}%
\pgfpathlineto{\pgfqpoint{1.375946in}{2.355649in}}%
\pgfpathlineto{\pgfqpoint{1.439269in}{2.376499in}}%
\pgfpathclose%
\pgfusepath{stroke,fill}%
\end{pgfscope}%
\begin{pgfscope}%
\pgfpathrectangle{\pgfqpoint{0.647939in}{0.492442in}}{\pgfqpoint{4.273799in}{2.331163in}}%
\pgfusepath{clip}%
\pgfsetroundcap%
\pgfsetroundjoin%
\pgfsetlinewidth{0.803000pt}%
\definecolor{currentstroke}{rgb}{0.501961,0.501961,0.501961}%
\pgfsetstrokecolor{currentstroke}%
\pgfsetdash{}{0pt}%
\pgfpathmoveto{\pgfqpoint{1.307583in}{2.001552in}}%
\pgfpathquadraticcurveto{\pgfqpoint{1.306483in}{1.988617in}}{\pgfqpoint{1.306436in}{1.988061in}}%
\pgfusepath{stroke}%
\end{pgfscope}%
\begin{pgfscope}%
\pgfpathrectangle{\pgfqpoint{0.647939in}{0.492442in}}{\pgfqpoint{4.273799in}{2.331163in}}%
\pgfusepath{clip}%
\pgfsetroundcap%
\pgfsetroundjoin%
\definecolor{currentfill}{rgb}{0.501961,0.501961,0.501961}%
\pgfsetfillcolor{currentfill}%
\pgfsetlinewidth{0.803000pt}%
\definecolor{currentstroke}{rgb}{0.501961,0.501961,0.501961}%
\pgfsetstrokecolor{currentstroke}%
\pgfsetdash{}{0pt}%
\pgfpathmoveto{\pgfqpoint{1.345298in}{2.051664in}}%
\pgfpathlineto{\pgfqpoint{1.306436in}{1.988061in}}%
\pgfpathlineto{\pgfqpoint{1.278871in}{2.057312in}}%
\pgfpathlineto{\pgfqpoint{1.345298in}{2.051664in}}%
\pgfpathclose%
\pgfusepath{stroke,fill}%
\end{pgfscope}%
\begin{pgfscope}%
\pgfpathrectangle{\pgfqpoint{0.647939in}{0.492442in}}{\pgfqpoint{4.273799in}{2.331163in}}%
\pgfusepath{clip}%
\pgfsetroundcap%
\pgfsetroundjoin%
\pgfsetlinewidth{0.803000pt}%
\definecolor{currentstroke}{rgb}{0.501961,0.501961,0.501961}%
\pgfsetstrokecolor{currentstroke}%
\pgfsetdash{}{0pt}%
\pgfpathmoveto{\pgfqpoint{1.166230in}{2.154225in}}%
\pgfpathquadraticcurveto{\pgfqpoint{1.166909in}{2.141280in}}{\pgfqpoint{1.166937in}{2.140741in}}%
\pgfusepath{stroke}%
\end{pgfscope}%
\begin{pgfscope}%
\pgfpathrectangle{\pgfqpoint{0.647939in}{0.492442in}}{\pgfqpoint{4.273799in}{2.331163in}}%
\pgfusepath{clip}%
\pgfsetroundcap%
\pgfsetroundjoin%
\definecolor{currentfill}{rgb}{0.501961,0.501961,0.501961}%
\pgfsetfillcolor{currentfill}%
\pgfsetlinewidth{0.803000pt}%
\definecolor{currentstroke}{rgb}{0.501961,0.501961,0.501961}%
\pgfsetstrokecolor{currentstroke}%
\pgfsetdash{}{0pt}%
\pgfpathmoveto{\pgfqpoint{1.196735in}{2.209061in}}%
\pgfpathlineto{\pgfqpoint{1.166937in}{2.140741in}}%
\pgfpathlineto{\pgfqpoint{1.130160in}{2.205571in}}%
\pgfpathlineto{\pgfqpoint{1.196735in}{2.209061in}}%
\pgfpathclose%
\pgfusepath{stroke,fill}%
\end{pgfscope}%
\begin{pgfscope}%
\pgfpathrectangle{\pgfqpoint{0.647939in}{0.492442in}}{\pgfqpoint{4.273799in}{2.331163in}}%
\pgfusepath{clip}%
\pgfsetroundcap%
\pgfsetroundjoin%
\pgfsetlinewidth{0.803000pt}%
\definecolor{currentstroke}{rgb}{0.501961,0.501961,0.501961}%
\pgfsetstrokecolor{currentstroke}%
\pgfsetdash{}{0pt}%
\pgfpathmoveto{\pgfqpoint{1.020832in}{1.738795in}}%
\pgfpathquadraticcurveto{\pgfqpoint{1.018543in}{1.725905in}}{\pgfqpoint{1.018426in}{1.725246in}}%
\pgfusepath{stroke}%
\end{pgfscope}%
\begin{pgfscope}%
\pgfpathrectangle{\pgfqpoint{0.647939in}{0.492442in}}{\pgfqpoint{4.273799in}{2.331163in}}%
\pgfusepath{clip}%
\pgfsetroundcap%
\pgfsetroundjoin%
\definecolor{currentfill}{rgb}{0.501961,0.501961,0.501961}%
\pgfsetfillcolor{currentfill}%
\pgfsetlinewidth{0.803000pt}%
\definecolor{currentstroke}{rgb}{0.501961,0.501961,0.501961}%
\pgfsetstrokecolor{currentstroke}%
\pgfsetdash{}{0pt}%
\pgfpathmoveto{\pgfqpoint{1.062905in}{1.785056in}}%
\pgfpathlineto{\pgfqpoint{1.018426in}{1.725246in}}%
\pgfpathlineto{\pgfqpoint{0.997265in}{1.796715in}}%
\pgfpathlineto{\pgfqpoint{1.062905in}{1.785056in}}%
\pgfpathclose%
\pgfusepath{stroke,fill}%
\end{pgfscope}%
\begin{pgfscope}%
\pgfpathrectangle{\pgfqpoint{0.647939in}{0.492442in}}{\pgfqpoint{4.273799in}{2.331163in}}%
\pgfusepath{clip}%
\pgfsetroundcap%
\pgfsetroundjoin%
\pgfsetlinewidth{0.803000pt}%
\definecolor{currentstroke}{rgb}{0.501961,0.501961,0.501961}%
\pgfsetstrokecolor{currentstroke}%
\pgfsetdash{}{0pt}%
\pgfpathmoveto{\pgfqpoint{0.929071in}{2.048358in}}%
\pgfpathquadraticcurveto{\pgfqpoint{0.928953in}{2.035408in}}{\pgfqpoint{0.928949in}{2.034879in}}%
\pgfusepath{stroke}%
\end{pgfscope}%
\begin{pgfscope}%
\pgfpathrectangle{\pgfqpoint{0.647939in}{0.492442in}}{\pgfqpoint{4.273799in}{2.331163in}}%
\pgfusepath{clip}%
\pgfsetroundcap%
\pgfsetroundjoin%
\definecolor{currentfill}{rgb}{0.501961,0.501961,0.501961}%
\pgfsetfillcolor{currentfill}%
\pgfsetlinewidth{0.803000pt}%
\definecolor{currentstroke}{rgb}{0.501961,0.501961,0.501961}%
\pgfsetstrokecolor{currentstroke}%
\pgfsetdash{}{0pt}%
\pgfpathmoveto{\pgfqpoint{0.962885in}{2.101241in}}%
\pgfpathlineto{\pgfqpoint{0.928949in}{2.034879in}}%
\pgfpathlineto{\pgfqpoint{0.896222in}{2.101846in}}%
\pgfpathlineto{\pgfqpoint{0.962885in}{2.101241in}}%
\pgfpathclose%
\pgfusepath{stroke,fill}%
\end{pgfscope}%
\begin{pgfscope}%
\pgfpathrectangle{\pgfqpoint{0.647939in}{0.492442in}}{\pgfqpoint{4.273799in}{2.331163in}}%
\pgfusepath{clip}%
\pgfsetroundcap%
\pgfsetroundjoin%
\pgfsetlinewidth{0.803000pt}%
\definecolor{currentstroke}{rgb}{0.501961,0.501961,0.501961}%
\pgfsetstrokecolor{currentstroke}%
\pgfsetdash{}{0pt}%
\pgfpathmoveto{\pgfqpoint{0.815901in}{1.944220in}}%
\pgfpathquadraticcurveto{\pgfqpoint{0.815397in}{1.931272in}}{\pgfqpoint{0.815376in}{1.930738in}}%
\pgfusepath{stroke}%
\end{pgfscope}%
\begin{pgfscope}%
\pgfpathrectangle{\pgfqpoint{0.647939in}{0.492442in}}{\pgfqpoint{4.273799in}{2.331163in}}%
\pgfusepath{clip}%
\pgfsetroundcap%
\pgfsetroundjoin%
\definecolor{currentfill}{rgb}{0.501961,0.501961,0.501961}%
\pgfsetfillcolor{currentfill}%
\pgfsetlinewidth{0.803000pt}%
\definecolor{currentstroke}{rgb}{0.501961,0.501961,0.501961}%
\pgfsetstrokecolor{currentstroke}%
\pgfsetdash{}{0pt}%
\pgfpathmoveto{\pgfqpoint{0.851274in}{1.996059in}}%
\pgfpathlineto{\pgfqpoint{0.815376in}{1.930738in}}%
\pgfpathlineto{\pgfqpoint{0.784658in}{1.998649in}}%
\pgfpathlineto{\pgfqpoint{0.851274in}{1.996059in}}%
\pgfpathclose%
\pgfusepath{stroke,fill}%
\end{pgfscope}%
\begin{pgfscope}%
\pgfpathrectangle{\pgfqpoint{0.647939in}{0.492442in}}{\pgfqpoint{4.273799in}{2.331163in}}%
\pgfusepath{clip}%
\pgfsetroundcap%
\pgfsetroundjoin%
\pgfsetlinewidth{0.803000pt}%
\definecolor{currentstroke}{rgb}{0.501961,0.501961,0.501961}%
\pgfsetstrokecolor{currentstroke}%
\pgfsetdash{}{0pt}%
\pgfpathmoveto{\pgfqpoint{0.708845in}{2.099272in}}%
\pgfpathquadraticcurveto{\pgfqpoint{0.708927in}{2.086321in}}{\pgfqpoint{0.708930in}{2.085793in}}%
\pgfusepath{stroke}%
\end{pgfscope}%
\begin{pgfscope}%
\pgfpathrectangle{\pgfqpoint{0.647939in}{0.492442in}}{\pgfqpoint{4.273799in}{2.331163in}}%
\pgfusepath{clip}%
\pgfsetroundcap%
\pgfsetroundjoin%
\definecolor{currentfill}{rgb}{0.501961,0.501961,0.501961}%
\pgfsetfillcolor{currentfill}%
\pgfsetlinewidth{0.803000pt}%
\definecolor{currentstroke}{rgb}{0.501961,0.501961,0.501961}%
\pgfsetstrokecolor{currentstroke}%
\pgfsetdash{}{0pt}%
\pgfpathmoveto{\pgfqpoint{0.741842in}{2.152669in}}%
\pgfpathlineto{\pgfqpoint{0.708930in}{2.085793in}}%
\pgfpathlineto{\pgfqpoint{0.675176in}{2.152247in}}%
\pgfpathlineto{\pgfqpoint{0.741842in}{2.152669in}}%
\pgfpathclose%
\pgfusepath{stroke,fill}%
\end{pgfscope}%
\begin{pgfscope}%
\pgfpathrectangle{\pgfqpoint{0.647939in}{0.492442in}}{\pgfqpoint{4.273799in}{2.331163in}}%
\pgfusepath{clip}%
\pgfsetroundcap%
\pgfsetroundjoin%
\pgfsetlinewidth{0.803000pt}%
\definecolor{currentstroke}{rgb}{0.501961,0.501961,0.501961}%
\pgfsetstrokecolor{currentstroke}%
\pgfsetdash{}{0pt}%
\pgfpathmoveto{\pgfqpoint{0.657091in}{2.087816in}}%
\pgfpathquadraticcurveto{\pgfqpoint{0.657134in}{2.074865in}}{\pgfqpoint{0.657135in}{2.074337in}}%
\pgfusepath{stroke}%
\end{pgfscope}%
\begin{pgfscope}%
\pgfpathrectangle{\pgfqpoint{0.647939in}{0.492442in}}{\pgfqpoint{4.273799in}{2.331163in}}%
\pgfusepath{clip}%
\pgfsetroundcap%
\pgfsetroundjoin%
\definecolor{currentfill}{rgb}{0.501961,0.501961,0.501961}%
\pgfsetfillcolor{currentfill}%
\pgfsetlinewidth{0.803000pt}%
\definecolor{currentstroke}{rgb}{0.501961,0.501961,0.501961}%
\pgfsetstrokecolor{currentstroke}%
\pgfsetdash{}{0pt}%
\pgfpathmoveto{\pgfqpoint{0.690248in}{2.141113in}}%
\pgfpathlineto{\pgfqpoint{0.657135in}{2.074337in}}%
\pgfpathlineto{\pgfqpoint{0.623582in}{2.140893in}}%
\pgfpathlineto{\pgfqpoint{0.690248in}{2.141113in}}%
\pgfpathclose%
\pgfusepath{stroke,fill}%
\end{pgfscope}%
\begin{pgfscope}%
\pgfpathrectangle{\pgfqpoint{0.647939in}{0.492442in}}{\pgfqpoint{4.273799in}{2.331163in}}%
\pgfusepath{clip}%
\pgfsetroundcap%
\pgfsetroundjoin%
\pgfsetlinewidth{0.803000pt}%
\definecolor{currentstroke}{rgb}{0.501961,0.501961,0.501961}%
\pgfsetstrokecolor{currentstroke}%
\pgfsetdash{}{0pt}%
\pgfpathmoveto{\pgfqpoint{0.698360in}{1.016262in}}%
\pgfpathquadraticcurveto{\pgfqpoint{0.694250in}{1.003507in}}{\pgfqpoint{0.693950in}{1.002576in}}%
\pgfusepath{stroke}%
\end{pgfscope}%
\begin{pgfscope}%
\pgfpathrectangle{\pgfqpoint{0.647939in}{0.492442in}}{\pgfqpoint{4.273799in}{2.331163in}}%
\pgfusepath{clip}%
\pgfsetroundcap%
\pgfsetroundjoin%
\definecolor{currentfill}{rgb}{0.501961,0.501961,0.501961}%
\pgfsetfillcolor{currentfill}%
\pgfsetlinewidth{0.803000pt}%
\definecolor{currentstroke}{rgb}{0.501961,0.501961,0.501961}%
\pgfsetstrokecolor{currentstroke}%
\pgfsetdash{}{0pt}%
\pgfpathmoveto{\pgfqpoint{0.746122in}{1.055807in}}%
\pgfpathlineto{\pgfqpoint{0.693950in}{1.002576in}}%
\pgfpathlineto{\pgfqpoint{0.682668in}{1.076253in}}%
\pgfpathlineto{\pgfqpoint{0.746122in}{1.055807in}}%
\pgfpathclose%
\pgfusepath{stroke,fill}%
\end{pgfscope}%
\begin{pgfscope}%
\pgfpathrectangle{\pgfqpoint{0.647939in}{0.492442in}}{\pgfqpoint{4.273799in}{2.331163in}}%
\pgfusepath{clip}%
\pgfsetroundcap%
\pgfsetroundjoin%
\pgfsetlinewidth{0.803000pt}%
\definecolor{currentstroke}{rgb}{0.501961,0.501961,0.501961}%
\pgfsetstrokecolor{currentstroke}%
\pgfsetdash{}{0pt}%
\pgfpathmoveto{\pgfqpoint{1.334299in}{0.593498in}}%
\pgfpathquadraticcurveto{\pgfqpoint{1.328256in}{0.595947in}}{\pgfqpoint{1.333726in}{0.593730in}}%
\pgfusepath{stroke}%
\end{pgfscope}%
\begin{pgfscope}%
\pgfpathrectangle{\pgfqpoint{0.647939in}{0.492442in}}{\pgfqpoint{4.273799in}{2.331163in}}%
\pgfusepath{clip}%
\pgfsetroundcap%
\pgfsetroundjoin%
\definecolor{currentfill}{rgb}{0.501961,0.501961,0.501961}%
\pgfsetfillcolor{currentfill}%
\pgfsetlinewidth{0.803000pt}%
\definecolor{currentstroke}{rgb}{0.501961,0.501961,0.501961}%
\pgfsetstrokecolor{currentstroke}%
\pgfsetdash{}{0pt}%
\pgfpathmoveto{\pgfqpoint{1.382991in}{0.537797in}}%
\pgfpathlineto{\pgfqpoint{1.333726in}{0.593730in}}%
\pgfpathlineto{\pgfqpoint{1.408031in}{0.599582in}}%
\pgfpathlineto{\pgfqpoint{1.382991in}{0.537797in}}%
\pgfpathclose%
\pgfusepath{stroke,fill}%
\end{pgfscope}%
\begin{pgfscope}%
\pgfpathrectangle{\pgfqpoint{0.647939in}{0.492442in}}{\pgfqpoint{4.273799in}{2.331163in}}%
\pgfusepath{clip}%
\pgfsetroundcap%
\pgfsetroundjoin%
\pgfsetlinewidth{0.803000pt}%
\definecolor{currentstroke}{rgb}{0.501961,0.501961,0.501961}%
\pgfsetstrokecolor{currentstroke}%
\pgfsetdash{}{0pt}%
\pgfpathmoveto{\pgfqpoint{1.381356in}{0.692420in}}%
\pgfpathquadraticcurveto{\pgfqpoint{1.376203in}{0.694924in}}{\pgfqpoint{1.382222in}{0.691999in}}%
\pgfusepath{stroke}%
\end{pgfscope}%
\begin{pgfscope}%
\pgfpathrectangle{\pgfqpoint{0.647939in}{0.492442in}}{\pgfqpoint{4.273799in}{2.331163in}}%
\pgfusepath{clip}%
\pgfsetroundcap%
\pgfsetroundjoin%
\definecolor{currentfill}{rgb}{0.501961,0.501961,0.501961}%
\pgfsetfillcolor{currentfill}%
\pgfsetlinewidth{0.803000pt}%
\definecolor{currentstroke}{rgb}{0.501961,0.501961,0.501961}%
\pgfsetstrokecolor{currentstroke}%
\pgfsetdash{}{0pt}%
\pgfpathmoveto{\pgfqpoint{1.427613in}{0.632878in}}%
\pgfpathlineto{\pgfqpoint{1.382222in}{0.691999in}}%
\pgfpathlineto{\pgfqpoint{1.456753in}{0.692839in}}%
\pgfpathlineto{\pgfqpoint{1.427613in}{0.632878in}}%
\pgfpathclose%
\pgfusepath{stroke,fill}%
\end{pgfscope}%
\begin{pgfscope}%
\pgfpathrectangle{\pgfqpoint{0.647939in}{0.492442in}}{\pgfqpoint{4.273799in}{2.331163in}}%
\pgfusepath{clip}%
\pgfsetroundcap%
\pgfsetroundjoin%
\pgfsetlinewidth{0.803000pt}%
\definecolor{currentstroke}{rgb}{0.501961,0.501961,0.501961}%
\pgfsetstrokecolor{currentstroke}%
\pgfsetdash{}{0pt}%
\pgfpathmoveto{\pgfqpoint{1.799337in}{2.663860in}}%
\pgfpathquadraticcurveto{\pgfqpoint{1.804011in}{2.661364in}}{\pgfqpoint{1.797726in}{2.664720in}}%
\pgfusepath{stroke}%
\end{pgfscope}%
\begin{pgfscope}%
\pgfpathrectangle{\pgfqpoint{0.647939in}{0.492442in}}{\pgfqpoint{4.273799in}{2.331163in}}%
\pgfusepath{clip}%
\pgfsetroundcap%
\pgfsetroundjoin%
\definecolor{currentfill}{rgb}{0.501961,0.501961,0.501961}%
\pgfsetfillcolor{currentfill}%
\pgfsetlinewidth{0.803000pt}%
\definecolor{currentstroke}{rgb}{0.501961,0.501961,0.501961}%
\pgfsetstrokecolor{currentstroke}%
\pgfsetdash{}{0pt}%
\pgfpathmoveto{\pgfqpoint{1.754616in}{2.725524in}}%
\pgfpathlineto{\pgfqpoint{1.797726in}{2.664720in}}%
\pgfpathlineto{\pgfqpoint{1.723217in}{2.666714in}}%
\pgfpathlineto{\pgfqpoint{1.754616in}{2.725524in}}%
\pgfpathclose%
\pgfusepath{stroke,fill}%
\end{pgfscope}%
\begin{pgfscope}%
\pgfpathrectangle{\pgfqpoint{0.647939in}{0.492442in}}{\pgfqpoint{4.273799in}{2.331163in}}%
\pgfusepath{clip}%
\pgfsetroundcap%
\pgfsetroundjoin%
\pgfsetlinewidth{0.803000pt}%
\definecolor{currentstroke}{rgb}{0.501961,0.501961,0.501961}%
\pgfsetstrokecolor{currentstroke}%
\pgfsetdash{}{0pt}%
\pgfpathmoveto{\pgfqpoint{3.674795in}{0.820612in}}%
\pgfpathquadraticcurveto{\pgfqpoint{3.670606in}{0.823064in}}{\pgfqpoint{3.677138in}{0.819241in}}%
\pgfusepath{stroke}%
\end{pgfscope}%
\begin{pgfscope}%
\pgfpathrectangle{\pgfqpoint{0.647939in}{0.492442in}}{\pgfqpoint{4.273799in}{2.331163in}}%
\pgfusepath{clip}%
\pgfsetroundcap%
\pgfsetroundjoin%
\definecolor{currentfill}{rgb}{0.501961,0.501961,0.501961}%
\pgfsetfillcolor{currentfill}%
\pgfsetlinewidth{0.803000pt}%
\definecolor{currentstroke}{rgb}{0.501961,0.501961,0.501961}%
\pgfsetstrokecolor{currentstroke}%
\pgfsetdash{}{0pt}%
\pgfpathmoveto{\pgfqpoint{3.717842in}{0.756801in}}%
\pgfpathlineto{\pgfqpoint{3.677138in}{0.819241in}}%
\pgfpathlineto{\pgfqpoint{3.751512in}{0.814341in}}%
\pgfpathlineto{\pgfqpoint{3.717842in}{0.756801in}}%
\pgfpathclose%
\pgfusepath{stroke,fill}%
\end{pgfscope}%
\begin{pgfscope}%
\pgfpathrectangle{\pgfqpoint{0.647939in}{0.492442in}}{\pgfqpoint{4.273799in}{2.331163in}}%
\pgfusepath{clip}%
\pgfsetroundcap%
\pgfsetroundjoin%
\pgfsetlinewidth{0.803000pt}%
\definecolor{currentstroke}{rgb}{0.501961,0.501961,0.501961}%
\pgfsetstrokecolor{currentstroke}%
\pgfsetdash{}{0pt}%
\pgfpathmoveto{\pgfqpoint{4.565487in}{1.809192in}}%
\pgfpathquadraticcurveto{\pgfqpoint{4.562140in}{1.822012in}}{\pgfqpoint{4.561931in}{1.822813in}}%
\pgfusepath{stroke}%
\end{pgfscope}%
\begin{pgfscope}%
\pgfpathrectangle{\pgfqpoint{0.647939in}{0.492442in}}{\pgfqpoint{4.273799in}{2.331163in}}%
\pgfusepath{clip}%
\pgfsetroundcap%
\pgfsetroundjoin%
\definecolor{currentfill}{rgb}{0.501961,0.501961,0.501961}%
\pgfsetfillcolor{currentfill}%
\pgfsetlinewidth{0.803000pt}%
\definecolor{currentstroke}{rgb}{0.501961,0.501961,0.501961}%
\pgfsetstrokecolor{currentstroke}%
\pgfsetdash{}{0pt}%
\pgfpathmoveto{\pgfqpoint{4.546517in}{1.749888in}}%
\pgfpathlineto{\pgfqpoint{4.561931in}{1.822813in}}%
\pgfpathlineto{\pgfqpoint{4.611022in}{1.766727in}}%
\pgfpathlineto{\pgfqpoint{4.546517in}{1.749888in}}%
\pgfpathclose%
\pgfusepath{stroke,fill}%
\end{pgfscope}%
\begin{pgfscope}%
\pgfpathrectangle{\pgfqpoint{0.647939in}{0.492442in}}{\pgfqpoint{4.273799in}{2.331163in}}%
\pgfusepath{clip}%
\pgfsetroundcap%
\pgfsetroundjoin%
\pgfsetlinewidth{0.803000pt}%
\definecolor{currentstroke}{rgb}{0.501961,0.501961,0.501961}%
\pgfsetstrokecolor{currentstroke}%
\pgfsetdash{}{0pt}%
\pgfpathmoveto{\pgfqpoint{3.635447in}{0.719094in}}%
\pgfpathquadraticcurveto{\pgfqpoint{3.631619in}{0.721485in}}{\pgfqpoint{3.638327in}{0.717295in}}%
\pgfusepath{stroke}%
\end{pgfscope}%
\begin{pgfscope}%
\pgfpathrectangle{\pgfqpoint{0.647939in}{0.492442in}}{\pgfqpoint{4.273799in}{2.331163in}}%
\pgfusepath{clip}%
\pgfsetroundcap%
\pgfsetroundjoin%
\definecolor{currentfill}{rgb}{0.501961,0.501961,0.501961}%
\pgfsetfillcolor{currentfill}%
\pgfsetlinewidth{0.803000pt}%
\definecolor{currentstroke}{rgb}{0.501961,0.501961,0.501961}%
\pgfsetstrokecolor{currentstroke}%
\pgfsetdash{}{0pt}%
\pgfpathmoveto{\pgfqpoint{3.677210in}{0.653705in}}%
\pgfpathlineto{\pgfqpoint{3.638327in}{0.717295in}}%
\pgfpathlineto{\pgfqpoint{3.712528in}{0.710248in}}%
\pgfpathlineto{\pgfqpoint{3.677210in}{0.653705in}}%
\pgfpathclose%
\pgfusepath{stroke,fill}%
\end{pgfscope}%
\begin{pgfscope}%
\pgfpathrectangle{\pgfqpoint{0.647939in}{0.492442in}}{\pgfqpoint{4.273799in}{2.331163in}}%
\pgfusepath{clip}%
\pgfsetroundcap%
\pgfsetroundjoin%
\pgfsetlinewidth{0.803000pt}%
\definecolor{currentstroke}{rgb}{0.501961,0.501961,0.501961}%
\pgfsetstrokecolor{currentstroke}%
\pgfsetdash{}{0pt}%
\pgfpathmoveto{\pgfqpoint{3.658934in}{1.066993in}}%
\pgfpathquadraticcurveto{\pgfqpoint{3.653684in}{1.069539in}}{\pgfqpoint{3.659611in}{1.066665in}}%
\pgfusepath{stroke}%
\end{pgfscope}%
\begin{pgfscope}%
\pgfpathrectangle{\pgfqpoint{0.647939in}{0.492442in}}{\pgfqpoint{4.273799in}{2.331163in}}%
\pgfusepath{clip}%
\pgfsetroundcap%
\pgfsetroundjoin%
\definecolor{currentfill}{rgb}{0.501961,0.501961,0.501961}%
\pgfsetfillcolor{currentfill}%
\pgfsetlinewidth{0.803000pt}%
\definecolor{currentstroke}{rgb}{0.501961,0.501961,0.501961}%
\pgfsetstrokecolor{currentstroke}%
\pgfsetdash{}{0pt}%
\pgfpathmoveto{\pgfqpoint{3.705053in}{1.007584in}}%
\pgfpathlineto{\pgfqpoint{3.659611in}{1.066665in}}%
\pgfpathlineto{\pgfqpoint{3.734141in}{1.067570in}}%
\pgfpathlineto{\pgfqpoint{3.705053in}{1.007584in}}%
\pgfpathclose%
\pgfusepath{stroke,fill}%
\end{pgfscope}%
\begin{pgfscope}%
\pgfpathrectangle{\pgfqpoint{0.647939in}{0.492442in}}{\pgfqpoint{4.273799in}{2.331163in}}%
\pgfusepath{clip}%
\pgfsetroundcap%
\pgfsetroundjoin%
\pgfsetlinewidth{0.803000pt}%
\definecolor{currentstroke}{rgb}{0.501961,0.501961,0.501961}%
\pgfsetstrokecolor{currentstroke}%
\pgfsetdash{}{0pt}%
\pgfpathmoveto{\pgfqpoint{4.176058in}{1.510621in}}%
\pgfpathquadraticcurveto{\pgfqpoint{4.171088in}{1.513116in}}{\pgfqpoint{4.177219in}{1.510038in}}%
\pgfusepath{stroke}%
\end{pgfscope}%
\begin{pgfscope}%
\pgfpathrectangle{\pgfqpoint{0.647939in}{0.492442in}}{\pgfqpoint{4.273799in}{2.331163in}}%
\pgfusepath{clip}%
\pgfsetroundcap%
\pgfsetroundjoin%
\definecolor{currentfill}{rgb}{0.501961,0.501961,0.501961}%
\pgfsetfillcolor{currentfill}%
\pgfsetlinewidth{0.803000pt}%
\definecolor{currentstroke}{rgb}{0.501961,0.501961,0.501961}%
\pgfsetstrokecolor{currentstroke}%
\pgfsetdash{}{0pt}%
\pgfpathmoveto{\pgfqpoint{4.221846in}{1.450339in}}%
\pgfpathlineto{\pgfqpoint{4.177219in}{1.510038in}}%
\pgfpathlineto{\pgfqpoint{4.251755in}{1.509920in}}%
\pgfpathlineto{\pgfqpoint{4.221846in}{1.450339in}}%
\pgfpathclose%
\pgfusepath{stroke,fill}%
\end{pgfscope}%
\begin{pgfscope}%
\pgfpathrectangle{\pgfqpoint{0.647939in}{0.492442in}}{\pgfqpoint{4.273799in}{2.331163in}}%
\pgfusepath{clip}%
\pgfsetroundcap%
\pgfsetroundjoin%
\pgfsetlinewidth{0.803000pt}%
\definecolor{currentstroke}{rgb}{0.501961,0.501961,0.501961}%
\pgfsetstrokecolor{currentstroke}%
\pgfsetdash{}{0pt}%
\pgfpathmoveto{\pgfqpoint{4.366267in}{1.713465in}}%
\pgfpathquadraticcurveto{\pgfqpoint{4.365462in}{1.714472in}}{\pgfqpoint{4.372416in}{1.705779in}}%
\pgfusepath{stroke}%
\end{pgfscope}%
\begin{pgfscope}%
\pgfpathrectangle{\pgfqpoint{0.647939in}{0.492442in}}{\pgfqpoint{4.273799in}{2.331163in}}%
\pgfusepath{clip}%
\pgfsetroundcap%
\pgfsetroundjoin%
\definecolor{currentfill}{rgb}{0.501961,0.501961,0.501961}%
\pgfsetfillcolor{currentfill}%
\pgfsetlinewidth{0.803000pt}%
\definecolor{currentstroke}{rgb}{0.501961,0.501961,0.501961}%
\pgfsetstrokecolor{currentstroke}%
\pgfsetdash{}{0pt}%
\pgfpathmoveto{\pgfqpoint{4.388029in}{1.632897in}}%
\pgfpathlineto{\pgfqpoint{4.372416in}{1.705779in}}%
\pgfpathlineto{\pgfqpoint{4.440089in}{1.674540in}}%
\pgfpathlineto{\pgfqpoint{4.388029in}{1.632897in}}%
\pgfpathclose%
\pgfusepath{stroke,fill}%
\end{pgfscope}%
\begin{pgfscope}%
\pgfpathrectangle{\pgfqpoint{0.647939in}{0.492442in}}{\pgfqpoint{4.273799in}{2.331163in}}%
\pgfusepath{clip}%
\pgfsetroundcap%
\pgfsetroundjoin%
\pgfsetlinewidth{0.803000pt}%
\definecolor{currentstroke}{rgb}{0.501961,0.501961,0.501961}%
\pgfsetstrokecolor{currentstroke}%
\pgfsetdash{}{0pt}%
\pgfpathmoveto{\pgfqpoint{4.382433in}{2.487315in}}%
\pgfpathquadraticcurveto{\pgfqpoint{4.391192in}{2.481953in}}{\pgfqpoint{4.389356in}{2.483077in}}%
\pgfusepath{stroke}%
\end{pgfscope}%
\begin{pgfscope}%
\pgfpathrectangle{\pgfqpoint{0.647939in}{0.492442in}}{\pgfqpoint{4.273799in}{2.331163in}}%
\pgfusepath{clip}%
\pgfsetroundcap%
\pgfsetroundjoin%
\definecolor{currentfill}{rgb}{0.501961,0.501961,0.501961}%
\pgfsetfillcolor{currentfill}%
\pgfsetlinewidth{0.803000pt}%
\definecolor{currentstroke}{rgb}{0.501961,0.501961,0.501961}%
\pgfsetstrokecolor{currentstroke}%
\pgfsetdash{}{0pt}%
\pgfpathmoveto{\pgfqpoint{4.349902in}{2.546314in}}%
\pgfpathlineto{\pgfqpoint{4.389356in}{2.483077in}}%
\pgfpathlineto{\pgfqpoint{4.315094in}{2.489457in}}%
\pgfpathlineto{\pgfqpoint{4.349902in}{2.546314in}}%
\pgfpathclose%
\pgfusepath{stroke,fill}%
\end{pgfscope}%
\begin{pgfscope}%
\pgfpathrectangle{\pgfqpoint{0.647939in}{0.492442in}}{\pgfqpoint{4.273799in}{2.331163in}}%
\pgfusepath{clip}%
\pgfsetroundcap%
\pgfsetroundjoin%
\pgfsetlinewidth{0.803000pt}%
\definecolor{currentstroke}{rgb}{0.501961,0.501961,0.501961}%
\pgfsetstrokecolor{currentstroke}%
\pgfsetdash{}{0pt}%
\pgfpathmoveto{\pgfqpoint{2.187932in}{1.511577in}}%
\pgfpathquadraticcurveto{\pgfqpoint{2.187876in}{1.511687in}}{\pgfqpoint{2.193447in}{1.500721in}}%
\pgfusepath{stroke}%
\end{pgfscope}%
\begin{pgfscope}%
\pgfpathrectangle{\pgfqpoint{0.647939in}{0.492442in}}{\pgfqpoint{4.273799in}{2.331163in}}%
\pgfusepath{clip}%
\pgfsetroundcap%
\pgfsetroundjoin%
\definecolor{currentfill}{rgb}{0.501961,0.501961,0.501961}%
\pgfsetfillcolor{currentfill}%
\pgfsetlinewidth{0.803000pt}%
\definecolor{currentstroke}{rgb}{0.501961,0.501961,0.501961}%
\pgfsetstrokecolor{currentstroke}%
\pgfsetdash{}{0pt}%
\pgfpathmoveto{\pgfqpoint{2.193926in}{1.426187in}}%
\pgfpathlineto{\pgfqpoint{2.193447in}{1.500721in}}%
\pgfpathlineto{\pgfqpoint{2.253362in}{1.456384in}}%
\pgfpathlineto{\pgfqpoint{2.193926in}{1.426187in}}%
\pgfpathclose%
\pgfusepath{stroke,fill}%
\end{pgfscope}%
\begin{pgfscope}%
\pgfpathrectangle{\pgfqpoint{0.647939in}{0.492442in}}{\pgfqpoint{4.273799in}{2.331163in}}%
\pgfusepath{clip}%
\pgfsetroundcap%
\pgfsetroundjoin%
\pgfsetlinewidth{0.803000pt}%
\definecolor{currentstroke}{rgb}{0.501961,0.501961,0.501961}%
\pgfsetstrokecolor{currentstroke}%
\pgfsetdash{}{0pt}%
\pgfpathmoveto{\pgfqpoint{4.168480in}{1.065230in}}%
\pgfpathquadraticcurveto{\pgfqpoint{4.165458in}{1.067423in}}{\pgfqpoint{4.172490in}{1.062319in}}%
\pgfusepath{stroke}%
\end{pgfscope}%
\begin{pgfscope}%
\pgfpathrectangle{\pgfqpoint{0.647939in}{0.492442in}}{\pgfqpoint{4.273799in}{2.331163in}}%
\pgfusepath{clip}%
\pgfsetroundcap%
\pgfsetroundjoin%
\definecolor{currentfill}{rgb}{0.501961,0.501961,0.501961}%
\pgfsetfillcolor{currentfill}%
\pgfsetlinewidth{0.803000pt}%
\definecolor{currentstroke}{rgb}{0.501961,0.501961,0.501961}%
\pgfsetstrokecolor{currentstroke}%
\pgfsetdash{}{0pt}%
\pgfpathmoveto{\pgfqpoint{4.206865in}{0.996183in}}%
\pgfpathlineto{\pgfqpoint{4.172490in}{1.062319in}}%
\pgfpathlineto{\pgfqpoint{4.246024in}{1.050137in}}%
\pgfpathlineto{\pgfqpoint{4.206865in}{0.996183in}}%
\pgfpathclose%
\pgfusepath{stroke,fill}%
\end{pgfscope}%
\begin{pgfscope}%
\pgfpathrectangle{\pgfqpoint{0.647939in}{0.492442in}}{\pgfqpoint{4.273799in}{2.331163in}}%
\pgfusepath{clip}%
\pgfsetroundcap%
\pgfsetroundjoin%
\pgfsetlinewidth{0.803000pt}%
\definecolor{currentstroke}{rgb}{0.501961,0.501961,0.501961}%
\pgfsetstrokecolor{currentstroke}%
\pgfsetdash{}{0pt}%
\pgfpathmoveto{\pgfqpoint{3.953944in}{1.372856in}}%
\pgfpathquadraticcurveto{\pgfqpoint{3.946448in}{1.375082in}}{\pgfqpoint{3.950861in}{1.373772in}}%
\pgfusepath{stroke}%
\end{pgfscope}%
\begin{pgfscope}%
\pgfpathrectangle{\pgfqpoint{0.647939in}{0.492442in}}{\pgfqpoint{4.273799in}{2.331163in}}%
\pgfusepath{clip}%
\pgfsetroundcap%
\pgfsetroundjoin%
\definecolor{currentfill}{rgb}{0.501961,0.501961,0.501961}%
\pgfsetfillcolor{currentfill}%
\pgfsetlinewidth{0.803000pt}%
\definecolor{currentstroke}{rgb}{0.501961,0.501961,0.501961}%
\pgfsetstrokecolor{currentstroke}%
\pgfsetdash{}{0pt}%
\pgfpathmoveto{\pgfqpoint{4.005282in}{1.322842in}}%
\pgfpathlineto{\pgfqpoint{3.950861in}{1.373772in}}%
\pgfpathlineto{\pgfqpoint{4.024257in}{1.386751in}}%
\pgfpathlineto{\pgfqpoint{4.005282in}{1.322842in}}%
\pgfpathclose%
\pgfusepath{stroke,fill}%
\end{pgfscope}%
\begin{pgfscope}%
\pgfpathrectangle{\pgfqpoint{0.647939in}{0.492442in}}{\pgfqpoint{4.273799in}{2.331163in}}%
\pgfusepath{clip}%
\pgfsetroundcap%
\pgfsetroundjoin%
\pgfsetlinewidth{0.803000pt}%
\definecolor{currentstroke}{rgb}{0.501961,0.501961,0.501961}%
\pgfsetstrokecolor{currentstroke}%
\pgfsetdash{}{0pt}%
\pgfpathmoveto{\pgfqpoint{1.244539in}{2.085614in}}%
\pgfpathquadraticcurveto{\pgfqpoint{1.244679in}{2.072665in}}{\pgfqpoint{1.244685in}{2.072137in}}%
\pgfusepath{stroke}%
\end{pgfscope}%
\begin{pgfscope}%
\pgfpathrectangle{\pgfqpoint{0.647939in}{0.492442in}}{\pgfqpoint{4.273799in}{2.331163in}}%
\pgfusepath{clip}%
\pgfsetroundcap%
\pgfsetroundjoin%
\definecolor{currentfill}{rgb}{0.501961,0.501961,0.501961}%
\pgfsetfillcolor{currentfill}%
\pgfsetlinewidth{0.803000pt}%
\definecolor{currentstroke}{rgb}{0.501961,0.501961,0.501961}%
\pgfsetstrokecolor{currentstroke}%
\pgfsetdash{}{0pt}%
\pgfpathmoveto{\pgfqpoint{1.277295in}{2.139160in}}%
\pgfpathlineto{\pgfqpoint{1.244685in}{2.072137in}}%
\pgfpathlineto{\pgfqpoint{1.210633in}{2.138439in}}%
\pgfpathlineto{\pgfqpoint{1.277295in}{2.139160in}}%
\pgfpathclose%
\pgfusepath{stroke,fill}%
\end{pgfscope}%
\begin{pgfscope}%
\pgfpathrectangle{\pgfqpoint{0.647939in}{0.492442in}}{\pgfqpoint{4.273799in}{2.331163in}}%
\pgfusepath{clip}%
\pgfsetroundcap%
\pgfsetroundjoin%
\pgfsetlinewidth{0.803000pt}%
\definecolor{currentstroke}{rgb}{0.501961,0.501961,0.501961}%
\pgfsetstrokecolor{currentstroke}%
\pgfsetdash{}{0pt}%
\pgfpathmoveto{\pgfqpoint{1.528093in}{1.266524in}}%
\pgfpathquadraticcurveto{\pgfqpoint{1.516433in}{1.273619in}}{\pgfqpoint{1.515385in}{1.274257in}}%
\pgfusepath{stroke}%
\end{pgfscope}%
\begin{pgfscope}%
\pgfpathrectangle{\pgfqpoint{0.647939in}{0.492442in}}{\pgfqpoint{4.273799in}{2.331163in}}%
\pgfusepath{clip}%
\pgfsetroundcap%
\pgfsetroundjoin%
\definecolor{currentfill}{rgb}{0.501961,0.501961,0.501961}%
\pgfsetfillcolor{currentfill}%
\pgfsetlinewidth{0.803000pt}%
\definecolor{currentstroke}{rgb}{0.501961,0.501961,0.501961}%
\pgfsetstrokecolor{currentstroke}%
\pgfsetdash{}{0pt}%
\pgfpathmoveto{\pgfqpoint{1.555009in}{1.211126in}}%
\pgfpathlineto{\pgfqpoint{1.515385in}{1.274257in}}%
\pgfpathlineto{\pgfqpoint{1.589664in}{1.268077in}}%
\pgfpathlineto{\pgfqpoint{1.555009in}{1.211126in}}%
\pgfpathclose%
\pgfusepath{stroke,fill}%
\end{pgfscope}%
\begin{pgfscope}%
\pgfpathrectangle{\pgfqpoint{0.647939in}{0.492442in}}{\pgfqpoint{4.273799in}{2.331163in}}%
\pgfusepath{clip}%
\pgfsetroundcap%
\pgfsetroundjoin%
\pgfsetlinewidth{0.803000pt}%
\definecolor{currentstroke}{rgb}{0.501961,0.501961,0.501961}%
\pgfsetstrokecolor{currentstroke}%
\pgfsetdash{}{0pt}%
\pgfpathmoveto{\pgfqpoint{2.216852in}{0.983352in}}%
\pgfpathquadraticcurveto{\pgfqpoint{2.216429in}{0.983981in}}{\pgfqpoint{2.222938in}{0.974301in}}%
\pgfusepath{stroke}%
\end{pgfscope}%
\begin{pgfscope}%
\pgfpathrectangle{\pgfqpoint{0.647939in}{0.492442in}}{\pgfqpoint{4.273799in}{2.331163in}}%
\pgfusepath{clip}%
\pgfsetroundcap%
\pgfsetroundjoin%
\definecolor{currentfill}{rgb}{0.501961,0.501961,0.501961}%
\pgfsetfillcolor{currentfill}%
\pgfsetlinewidth{0.803000pt}%
\definecolor{currentstroke}{rgb}{0.501961,0.501961,0.501961}%
\pgfsetstrokecolor{currentstroke}%
\pgfsetdash{}{0pt}%
\pgfpathmoveto{\pgfqpoint{2.232477in}{0.900379in}}%
\pgfpathlineto{\pgfqpoint{2.222938in}{0.974301in}}%
\pgfpathlineto{\pgfqpoint{2.287799in}{0.937579in}}%
\pgfpathlineto{\pgfqpoint{2.232477in}{0.900379in}}%
\pgfpathclose%
\pgfusepath{stroke,fill}%
\end{pgfscope}%
\begin{pgfscope}%
\pgfpathrectangle{\pgfqpoint{0.647939in}{0.492442in}}{\pgfqpoint{4.273799in}{2.331163in}}%
\pgfusepath{clip}%
\pgfsetroundcap%
\pgfsetroundjoin%
\pgfsetlinewidth{0.803000pt}%
\definecolor{currentstroke}{rgb}{0.501961,0.501961,0.501961}%
\pgfsetstrokecolor{currentstroke}%
\pgfsetdash{}{0pt}%
\pgfpathmoveto{\pgfqpoint{3.656839in}{1.022157in}}%
\pgfpathquadraticcurveto{\pgfqpoint{3.651833in}{1.024700in}}{\pgfqpoint{3.657903in}{1.021616in}}%
\pgfusepath{stroke}%
\end{pgfscope}%
\begin{pgfscope}%
\pgfpathrectangle{\pgfqpoint{0.647939in}{0.492442in}}{\pgfqpoint{4.273799in}{2.331163in}}%
\pgfusepath{clip}%
\pgfsetroundcap%
\pgfsetroundjoin%
\definecolor{currentfill}{rgb}{0.501961,0.501961,0.501961}%
\pgfsetfillcolor{currentfill}%
\pgfsetlinewidth{0.803000pt}%
\definecolor{currentstroke}{rgb}{0.501961,0.501961,0.501961}%
\pgfsetstrokecolor{currentstroke}%
\pgfsetdash{}{0pt}%
\pgfpathmoveto{\pgfqpoint{3.702247in}{0.961707in}}%
\pgfpathlineto{\pgfqpoint{3.657903in}{1.021616in}}%
\pgfpathlineto{\pgfqpoint{3.732437in}{1.021146in}}%
\pgfpathlineto{\pgfqpoint{3.702247in}{0.961707in}}%
\pgfpathclose%
\pgfusepath{stroke,fill}%
\end{pgfscope}%
\begin{pgfscope}%
\pgfpathrectangle{\pgfqpoint{0.647939in}{0.492442in}}{\pgfqpoint{4.273799in}{2.331163in}}%
\pgfusepath{clip}%
\pgfsetroundcap%
\pgfsetroundjoin%
\pgfsetlinewidth{0.803000pt}%
\definecolor{currentstroke}{rgb}{0.501961,0.501961,0.501961}%
\pgfsetstrokecolor{currentstroke}%
\pgfsetdash{}{0pt}%
\pgfpathmoveto{\pgfqpoint{4.089534in}{1.614656in}}%
\pgfpathquadraticcurveto{\pgfqpoint{4.085842in}{1.615235in}}{\pgfqpoint{4.094422in}{1.613890in}}%
\pgfusepath{stroke}%
\end{pgfscope}%
\begin{pgfscope}%
\pgfpathrectangle{\pgfqpoint{0.647939in}{0.492442in}}{\pgfqpoint{4.273799in}{2.331163in}}%
\pgfusepath{clip}%
\pgfsetroundcap%
\pgfsetroundjoin%
\definecolor{currentfill}{rgb}{0.501961,0.501961,0.501961}%
\pgfsetfillcolor{currentfill}%
\pgfsetlinewidth{0.803000pt}%
\definecolor{currentstroke}{rgb}{0.501961,0.501961,0.501961}%
\pgfsetstrokecolor{currentstroke}%
\pgfsetdash{}{0pt}%
\pgfpathmoveto{\pgfqpoint{4.155121in}{1.570632in}}%
\pgfpathlineto{\pgfqpoint{4.094422in}{1.613890in}}%
\pgfpathlineto{\pgfqpoint{4.165448in}{1.636493in}}%
\pgfpathlineto{\pgfqpoint{4.155121in}{1.570632in}}%
\pgfpathclose%
\pgfusepath{stroke,fill}%
\end{pgfscope}%
\begin{pgfscope}%
\pgfpathrectangle{\pgfqpoint{0.647939in}{0.492442in}}{\pgfqpoint{4.273799in}{2.331163in}}%
\pgfusepath{clip}%
\pgfsetroundcap%
\pgfsetroundjoin%
\pgfsetlinewidth{0.803000pt}%
\definecolor{currentstroke}{rgb}{0.501961,0.501961,0.501961}%
\pgfsetstrokecolor{currentstroke}%
\pgfsetdash{}{0pt}%
\pgfpathmoveto{\pgfqpoint{3.482299in}{2.148719in}}%
\pgfpathquadraticcurveto{\pgfqpoint{3.484199in}{2.135814in}}{\pgfqpoint{3.484289in}{2.135199in}}%
\pgfusepath{stroke}%
\end{pgfscope}%
\begin{pgfscope}%
\pgfpathrectangle{\pgfqpoint{0.647939in}{0.492442in}}{\pgfqpoint{4.273799in}{2.331163in}}%
\pgfusepath{clip}%
\pgfsetroundcap%
\pgfsetroundjoin%
\definecolor{currentfill}{rgb}{0.501961,0.501961,0.501961}%
\pgfsetfillcolor{currentfill}%
\pgfsetlinewidth{0.803000pt}%
\definecolor{currentstroke}{rgb}{0.501961,0.501961,0.501961}%
\pgfsetstrokecolor{currentstroke}%
\pgfsetdash{}{0pt}%
\pgfpathmoveto{\pgfqpoint{3.507555in}{2.206010in}}%
\pgfpathlineto{\pgfqpoint{3.484289in}{2.135199in}}%
\pgfpathlineto{\pgfqpoint{3.441600in}{2.196298in}}%
\pgfpathlineto{\pgfqpoint{3.507555in}{2.206010in}}%
\pgfpathclose%
\pgfusepath{stroke,fill}%
\end{pgfscope}%
\begin{pgfscope}%
\pgfpathrectangle{\pgfqpoint{0.647939in}{0.492442in}}{\pgfqpoint{4.273799in}{2.331163in}}%
\pgfusepath{clip}%
\pgfsetroundcap%
\pgfsetroundjoin%
\pgfsetlinewidth{0.803000pt}%
\definecolor{currentstroke}{rgb}{0.501961,0.501961,0.501961}%
\pgfsetstrokecolor{currentstroke}%
\pgfsetdash{}{0pt}%
\pgfpathmoveto{\pgfqpoint{1.616169in}{1.427270in}}%
\pgfpathquadraticcurveto{\pgfqpoint{1.614321in}{1.428978in}}{\pgfqpoint{1.621597in}{1.422254in}}%
\pgfusepath{stroke}%
\end{pgfscope}%
\begin{pgfscope}%
\pgfpathrectangle{\pgfqpoint{0.647939in}{0.492442in}}{\pgfqpoint{4.273799in}{2.331163in}}%
\pgfusepath{clip}%
\pgfsetroundcap%
\pgfsetroundjoin%
\definecolor{currentfill}{rgb}{0.501961,0.501961,0.501961}%
\pgfsetfillcolor{currentfill}%
\pgfsetlinewidth{0.803000pt}%
\definecolor{currentstroke}{rgb}{0.501961,0.501961,0.501961}%
\pgfsetstrokecolor{currentstroke}%
\pgfsetdash{}{0pt}%
\pgfpathmoveto{\pgfqpoint{1.647935in}{1.352527in}}%
\pgfpathlineto{\pgfqpoint{1.621597in}{1.422254in}}%
\pgfpathlineto{\pgfqpoint{1.693181in}{1.401489in}}%
\pgfpathlineto{\pgfqpoint{1.647935in}{1.352527in}}%
\pgfpathclose%
\pgfusepath{stroke,fill}%
\end{pgfscope}%
\begin{pgfscope}%
\pgfpathrectangle{\pgfqpoint{0.647939in}{0.492442in}}{\pgfqpoint{4.273799in}{2.331163in}}%
\pgfusepath{clip}%
\pgfsetroundcap%
\pgfsetroundjoin%
\pgfsetlinewidth{0.803000pt}%
\definecolor{currentstroke}{rgb}{0.501961,0.501961,0.501961}%
\pgfsetstrokecolor{currentstroke}%
\pgfsetdash{}{0pt}%
\pgfpathmoveto{\pgfqpoint{4.050039in}{1.385712in}}%
\pgfpathquadraticcurveto{\pgfqpoint{4.043677in}{1.388160in}}{\pgfqpoint{4.048909in}{1.386147in}}%
\pgfusepath{stroke}%
\end{pgfscope}%
\begin{pgfscope}%
\pgfpathrectangle{\pgfqpoint{0.647939in}{0.492442in}}{\pgfqpoint{4.273799in}{2.331163in}}%
\pgfusepath{clip}%
\pgfsetroundcap%
\pgfsetroundjoin%
\definecolor{currentfill}{rgb}{0.501961,0.501961,0.501961}%
\pgfsetfillcolor{currentfill}%
\pgfsetlinewidth{0.803000pt}%
\definecolor{currentstroke}{rgb}{0.501961,0.501961,0.501961}%
\pgfsetstrokecolor{currentstroke}%
\pgfsetdash{}{0pt}%
\pgfpathmoveto{\pgfqpoint{4.099154in}{1.331093in}}%
\pgfpathlineto{\pgfqpoint{4.048909in}{1.386147in}}%
\pgfpathlineto{\pgfqpoint{4.123099in}{1.393310in}}%
\pgfpathlineto{\pgfqpoint{4.099154in}{1.331093in}}%
\pgfpathclose%
\pgfusepath{stroke,fill}%
\end{pgfscope}%
\begin{pgfscope}%
\pgfpathrectangle{\pgfqpoint{0.647939in}{0.492442in}}{\pgfqpoint{4.273799in}{2.331163in}}%
\pgfusepath{clip}%
\pgfsetroundcap%
\pgfsetroundjoin%
\pgfsetlinewidth{0.803000pt}%
\definecolor{currentstroke}{rgb}{0.501961,0.501961,0.501961}%
\pgfsetstrokecolor{currentstroke}%
\pgfsetdash{}{0pt}%
\pgfpathmoveto{\pgfqpoint{3.957934in}{1.072832in}}%
\pgfpathquadraticcurveto{\pgfqpoint{3.953085in}{1.075352in}}{\pgfqpoint{3.959259in}{1.072144in}}%
\pgfusepath{stroke}%
\end{pgfscope}%
\begin{pgfscope}%
\pgfpathrectangle{\pgfqpoint{0.647939in}{0.492442in}}{\pgfqpoint{4.273799in}{2.331163in}}%
\pgfusepath{clip}%
\pgfsetroundcap%
\pgfsetroundjoin%
\definecolor{currentfill}{rgb}{0.501961,0.501961,0.501961}%
\pgfsetfillcolor{currentfill}%
\pgfsetlinewidth{0.803000pt}%
\definecolor{currentstroke}{rgb}{0.501961,0.501961,0.501961}%
\pgfsetstrokecolor{currentstroke}%
\pgfsetdash{}{0pt}%
\pgfpathmoveto{\pgfqpoint{4.003041in}{1.011823in}}%
\pgfpathlineto{\pgfqpoint{3.959259in}{1.072144in}}%
\pgfpathlineto{\pgfqpoint{4.033785in}{1.070977in}}%
\pgfpathlineto{\pgfqpoint{4.003041in}{1.011823in}}%
\pgfpathclose%
\pgfusepath{stroke,fill}%
\end{pgfscope}%
\begin{pgfscope}%
\pgfpathrectangle{\pgfqpoint{0.647939in}{0.492442in}}{\pgfqpoint{4.273799in}{2.331163in}}%
\pgfusepath{clip}%
\pgfsetroundcap%
\pgfsetroundjoin%
\pgfsetlinewidth{0.803000pt}%
\definecolor{currentstroke}{rgb}{0.501961,0.501961,0.501961}%
\pgfsetstrokecolor{currentstroke}%
\pgfsetdash{}{0pt}%
\pgfpathmoveto{\pgfqpoint{2.388550in}{1.893362in}}%
\pgfpathquadraticcurveto{\pgfqpoint{2.385864in}{1.906227in}}{\pgfqpoint{2.385717in}{1.906931in}}%
\pgfusepath{stroke}%
\end{pgfscope}%
\begin{pgfscope}%
\pgfpathrectangle{\pgfqpoint{0.647939in}{0.492442in}}{\pgfqpoint{4.273799in}{2.331163in}}%
\pgfusepath{clip}%
\pgfsetroundcap%
\pgfsetroundjoin%
\definecolor{currentfill}{rgb}{0.501961,0.501961,0.501961}%
\pgfsetfillcolor{currentfill}%
\pgfsetlinewidth{0.803000pt}%
\definecolor{currentstroke}{rgb}{0.501961,0.501961,0.501961}%
\pgfsetstrokecolor{currentstroke}%
\pgfsetdash{}{0pt}%
\pgfpathmoveto{\pgfqpoint{2.366711in}{1.834860in}}%
\pgfpathlineto{\pgfqpoint{2.385717in}{1.906931in}}%
\pgfpathlineto{\pgfqpoint{2.431970in}{1.848483in}}%
\pgfpathlineto{\pgfqpoint{2.366711in}{1.834860in}}%
\pgfpathclose%
\pgfusepath{stroke,fill}%
\end{pgfscope}%
\begin{pgfscope}%
\pgfpathrectangle{\pgfqpoint{0.647939in}{0.492442in}}{\pgfqpoint{4.273799in}{2.331163in}}%
\pgfusepath{clip}%
\pgfsetroundcap%
\pgfsetroundjoin%
\pgfsetlinewidth{0.803000pt}%
\definecolor{currentstroke}{rgb}{0.501961,0.501961,0.501961}%
\pgfsetstrokecolor{currentstroke}%
\pgfsetdash{}{0pt}%
\pgfpathmoveto{\pgfqpoint{1.564248in}{2.273460in}}%
\pgfpathquadraticcurveto{\pgfqpoint{1.564488in}{2.273064in}}{\pgfqpoint{1.558299in}{2.283298in}}%
\pgfusepath{stroke}%
\end{pgfscope}%
\begin{pgfscope}%
\pgfpathrectangle{\pgfqpoint{0.647939in}{0.492442in}}{\pgfqpoint{4.273799in}{2.331163in}}%
\pgfusepath{clip}%
\pgfsetroundcap%
\pgfsetroundjoin%
\definecolor{currentfill}{rgb}{0.501961,0.501961,0.501961}%
\pgfsetfillcolor{currentfill}%
\pgfsetlinewidth{0.803000pt}%
\definecolor{currentstroke}{rgb}{0.501961,0.501961,0.501961}%
\pgfsetstrokecolor{currentstroke}%
\pgfsetdash{}{0pt}%
\pgfpathmoveto{\pgfqpoint{1.552323in}{2.357594in}}%
\pgfpathlineto{\pgfqpoint{1.558299in}{2.283298in}}%
\pgfpathlineto{\pgfqpoint{1.495277in}{2.323095in}}%
\pgfpathlineto{\pgfqpoint{1.552323in}{2.357594in}}%
\pgfpathclose%
\pgfusepath{stroke,fill}%
\end{pgfscope}%
\begin{pgfscope}%
\pgfpathrectangle{\pgfqpoint{0.647939in}{0.492442in}}{\pgfqpoint{4.273799in}{2.331163in}}%
\pgfusepath{clip}%
\pgfsetroundcap%
\pgfsetroundjoin%
\pgfsetlinewidth{0.803000pt}%
\definecolor{currentstroke}{rgb}{0.501961,0.501961,0.501961}%
\pgfsetstrokecolor{currentstroke}%
\pgfsetdash{}{0pt}%
\pgfpathmoveto{\pgfqpoint{1.680732in}{1.900826in}}%
\pgfpathquadraticcurveto{\pgfqpoint{1.680440in}{1.901292in}}{\pgfqpoint{1.686751in}{1.891236in}}%
\pgfusepath{stroke}%
\end{pgfscope}%
\begin{pgfscope}%
\pgfpathrectangle{\pgfqpoint{0.647939in}{0.492442in}}{\pgfqpoint{4.273799in}{2.331163in}}%
\pgfusepath{clip}%
\pgfsetroundcap%
\pgfsetroundjoin%
\definecolor{currentfill}{rgb}{0.501961,0.501961,0.501961}%
\pgfsetfillcolor{currentfill}%
\pgfsetlinewidth{0.803000pt}%
\definecolor{currentstroke}{rgb}{0.501961,0.501961,0.501961}%
\pgfsetstrokecolor{currentstroke}%
\pgfsetdash{}{0pt}%
\pgfpathmoveto{\pgfqpoint{1.693954in}{1.817049in}}%
\pgfpathlineto{\pgfqpoint{1.686751in}{1.891236in}}%
\pgfpathlineto{\pgfqpoint{1.750422in}{1.852486in}}%
\pgfpathlineto{\pgfqpoint{1.693954in}{1.817049in}}%
\pgfpathclose%
\pgfusepath{stroke,fill}%
\end{pgfscope}%
\begin{pgfscope}%
\pgfpathrectangle{\pgfqpoint{0.647939in}{0.492442in}}{\pgfqpoint{4.273799in}{2.331163in}}%
\pgfusepath{clip}%
\pgfsetroundcap%
\pgfsetroundjoin%
\pgfsetlinewidth{0.803000pt}%
\definecolor{currentstroke}{rgb}{0.501961,0.501961,0.501961}%
\pgfsetstrokecolor{currentstroke}%
\pgfsetdash{}{0pt}%
\pgfpathmoveto{\pgfqpoint{3.696950in}{2.139840in}}%
\pgfpathquadraticcurveto{\pgfqpoint{3.698428in}{2.126917in}}{\pgfqpoint{3.698494in}{2.126337in}}%
\pgfusepath{stroke}%
\end{pgfscope}%
\begin{pgfscope}%
\pgfpathrectangle{\pgfqpoint{0.647939in}{0.492442in}}{\pgfqpoint{4.273799in}{2.331163in}}%
\pgfusepath{clip}%
\pgfsetroundcap%
\pgfsetroundjoin%
\definecolor{currentfill}{rgb}{0.501961,0.501961,0.501961}%
\pgfsetfillcolor{currentfill}%
\pgfsetlinewidth{0.803000pt}%
\definecolor{currentstroke}{rgb}{0.501961,0.501961,0.501961}%
\pgfsetstrokecolor{currentstroke}%
\pgfsetdash{}{0pt}%
\pgfpathmoveto{\pgfqpoint{3.724039in}{2.196358in}}%
\pgfpathlineto{\pgfqpoint{3.698494in}{2.126337in}}%
\pgfpathlineto{\pgfqpoint{3.657804in}{2.188785in}}%
\pgfpathlineto{\pgfqpoint{3.724039in}{2.196358in}}%
\pgfpathclose%
\pgfusepath{stroke,fill}%
\end{pgfscope}%
\begin{pgfscope}%
\pgfpathrectangle{\pgfqpoint{0.647939in}{0.492442in}}{\pgfqpoint{4.273799in}{2.331163in}}%
\pgfusepath{clip}%
\pgfsetroundcap%
\pgfsetroundjoin%
\pgfsetlinewidth{0.803000pt}%
\definecolor{currentstroke}{rgb}{0.501961,0.501961,0.501961}%
\pgfsetstrokecolor{currentstroke}%
\pgfsetdash{}{0pt}%
\pgfpathmoveto{\pgfqpoint{1.879283in}{1.982797in}}%
\pgfpathquadraticcurveto{\pgfqpoint{1.878104in}{1.995724in}}{\pgfqpoint{1.878054in}{1.996279in}}%
\pgfusepath{stroke}%
\end{pgfscope}%
\begin{pgfscope}%
\pgfpathrectangle{\pgfqpoint{0.647939in}{0.492442in}}{\pgfqpoint{4.273799in}{2.331163in}}%
\pgfusepath{clip}%
\pgfsetroundcap%
\pgfsetroundjoin%
\definecolor{currentfill}{rgb}{0.501961,0.501961,0.501961}%
\pgfsetfillcolor{currentfill}%
\pgfsetlinewidth{0.803000pt}%
\definecolor{currentstroke}{rgb}{0.501961,0.501961,0.501961}%
\pgfsetstrokecolor{currentstroke}%
\pgfsetdash{}{0pt}%
\pgfpathmoveto{\pgfqpoint{1.850909in}{1.926862in}}%
\pgfpathlineto{\pgfqpoint{1.878054in}{1.996279in}}%
\pgfpathlineto{\pgfqpoint{1.917301in}{1.932913in}}%
\pgfpathlineto{\pgfqpoint{1.850909in}{1.926862in}}%
\pgfpathclose%
\pgfusepath{stroke,fill}%
\end{pgfscope}%
\begin{pgfscope}%
\pgfpathrectangle{\pgfqpoint{0.647939in}{0.492442in}}{\pgfqpoint{4.273799in}{2.331163in}}%
\pgfusepath{clip}%
\pgfsetroundcap%
\pgfsetroundjoin%
\pgfsetlinewidth{0.803000pt}%
\definecolor{currentstroke}{rgb}{0.501961,0.501961,0.501961}%
\pgfsetstrokecolor{currentstroke}%
\pgfsetdash{}{0pt}%
\pgfpathmoveto{\pgfqpoint{1.926053in}{1.420664in}}%
\pgfpathquadraticcurveto{\pgfqpoint{1.925799in}{1.421085in}}{\pgfqpoint{1.931978in}{1.410878in}}%
\pgfusepath{stroke}%
\end{pgfscope}%
\begin{pgfscope}%
\pgfpathrectangle{\pgfqpoint{0.647939in}{0.492442in}}{\pgfqpoint{4.273799in}{2.331163in}}%
\pgfusepath{clip}%
\pgfsetroundcap%
\pgfsetroundjoin%
\definecolor{currentfill}{rgb}{0.501961,0.501961,0.501961}%
\pgfsetfillcolor{currentfill}%
\pgfsetlinewidth{0.803000pt}%
\definecolor{currentstroke}{rgb}{0.501961,0.501961,0.501961}%
\pgfsetstrokecolor{currentstroke}%
\pgfsetdash{}{0pt}%
\pgfpathmoveto{\pgfqpoint{1.937994in}{1.336586in}}%
\pgfpathlineto{\pgfqpoint{1.931978in}{1.410878in}}%
\pgfpathlineto{\pgfqpoint{1.995021in}{1.371115in}}%
\pgfpathlineto{\pgfqpoint{1.937994in}{1.336586in}}%
\pgfpathclose%
\pgfusepath{stroke,fill}%
\end{pgfscope}%
\begin{pgfscope}%
\pgfpathrectangle{\pgfqpoint{0.647939in}{0.492442in}}{\pgfqpoint{4.273799in}{2.331163in}}%
\pgfusepath{clip}%
\pgfsetroundcap%
\pgfsetroundjoin%
\pgfsetlinewidth{0.803000pt}%
\definecolor{currentstroke}{rgb}{0.501961,0.501961,0.501961}%
\pgfsetstrokecolor{currentstroke}%
\pgfsetdash{}{0pt}%
\pgfpathmoveto{\pgfqpoint{3.699206in}{1.296122in}}%
\pgfpathquadraticcurveto{\pgfqpoint{3.691931in}{1.298430in}}{\pgfqpoint{3.696497in}{1.296982in}}%
\pgfusepath{stroke}%
\end{pgfscope}%
\begin{pgfscope}%
\pgfpathrectangle{\pgfqpoint{0.647939in}{0.492442in}}{\pgfqpoint{4.273799in}{2.331163in}}%
\pgfusepath{clip}%
\pgfsetroundcap%
\pgfsetroundjoin%
\definecolor{currentfill}{rgb}{0.501961,0.501961,0.501961}%
\pgfsetfillcolor{currentfill}%
\pgfsetlinewidth{0.803000pt}%
\definecolor{currentstroke}{rgb}{0.501961,0.501961,0.501961}%
\pgfsetstrokecolor{currentstroke}%
\pgfsetdash{}{0pt}%
\pgfpathmoveto{\pgfqpoint{3.749967in}{1.245054in}}%
\pgfpathlineto{\pgfqpoint{3.696497in}{1.296982in}}%
\pgfpathlineto{\pgfqpoint{3.770121in}{1.308602in}}%
\pgfpathlineto{\pgfqpoint{3.749967in}{1.245054in}}%
\pgfpathclose%
\pgfusepath{stroke,fill}%
\end{pgfscope}%
\begin{pgfscope}%
\pgfpathrectangle{\pgfqpoint{0.647939in}{0.492442in}}{\pgfqpoint{4.273799in}{2.331163in}}%
\pgfusepath{clip}%
\pgfsetroundcap%
\pgfsetroundjoin%
\pgfsetlinewidth{0.803000pt}%
\definecolor{currentstroke}{rgb}{0.501961,0.501961,0.501961}%
\pgfsetstrokecolor{currentstroke}%
\pgfsetdash{}{0pt}%
\pgfpathmoveto{\pgfqpoint{2.115442in}{1.485101in}}%
\pgfpathquadraticcurveto{\pgfqpoint{2.115359in}{1.485259in}}{\pgfqpoint{2.121050in}{1.474419in}}%
\pgfusepath{stroke}%
\end{pgfscope}%
\begin{pgfscope}%
\pgfpathrectangle{\pgfqpoint{0.647939in}{0.492442in}}{\pgfqpoint{4.273799in}{2.331163in}}%
\pgfusepath{clip}%
\pgfsetroundcap%
\pgfsetroundjoin%
\definecolor{currentfill}{rgb}{0.501961,0.501961,0.501961}%
\pgfsetfillcolor{currentfill}%
\pgfsetlinewidth{0.803000pt}%
\definecolor{currentstroke}{rgb}{0.501961,0.501961,0.501961}%
\pgfsetstrokecolor{currentstroke}%
\pgfsetdash{}{0pt}%
\pgfpathmoveto{\pgfqpoint{2.122527in}{1.399898in}}%
\pgfpathlineto{\pgfqpoint{2.121050in}{1.474419in}}%
\pgfpathlineto{\pgfqpoint{2.181553in}{1.430888in}}%
\pgfpathlineto{\pgfqpoint{2.122527in}{1.399898in}}%
\pgfpathclose%
\pgfusepath{stroke,fill}%
\end{pgfscope}%
\begin{pgfscope}%
\pgfpathrectangle{\pgfqpoint{0.647939in}{0.492442in}}{\pgfqpoint{4.273799in}{2.331163in}}%
\pgfusepath{clip}%
\pgfsetbuttcap%
\pgfsetroundjoin%
\pgfsetlinewidth{0.301125pt}%
\definecolor{currentstroke}{rgb}{0.500000,0.500000,0.500000}%
\pgfsetstrokecolor{currentstroke}%
\pgfsetstrokeopacity{0.300000}%
\pgfsetdash{}{0pt}%
\pgfpathmoveto{\pgfqpoint{2.322688in}{0.492442in}}%
\pgfpathlineto{\pgfqpoint{2.318069in}{0.498044in}}%
\pgfpathlineto{\pgfqpoint{2.279269in}{0.545327in}}%
\pgfpathlineto{\pgfqpoint{2.240769in}{0.592683in}}%
\pgfpathlineto{\pgfqpoint{2.202524in}{0.640101in}}%
\pgfpathlineto{\pgfqpoint{2.164479in}{0.687566in}}%
\pgfpathlineto{\pgfqpoint{2.126574in}{0.735065in}}%
\pgfpathlineto{\pgfqpoint{2.088744in}{0.782582in}}%
\pgfpathlineto{\pgfqpoint{2.050917in}{0.830099in}}%
\pgfpathlineto{\pgfqpoint{2.013008in}{0.877597in}}%
\pgfpathlineto{\pgfqpoint{1.974917in}{0.925051in}}%
\pgfpathlineto{\pgfqpoint{1.936519in}{0.972432in}}%
\pgfpathlineto{\pgfqpoint{1.897661in}{1.019700in}}%
\pgfpathlineto{\pgfqpoint{1.858141in}{1.066805in}}%
\pgfpathlineto{\pgfqpoint{1.817687in}{1.113672in}}%
\pgfpathlineto{\pgfqpoint{1.775930in}{1.160197in}}%
\pgfpathlineto{\pgfqpoint{1.732348in}{1.206217in}}%
\pgfpathlineto{\pgfqpoint{1.686143in}{1.251463in}}%
\pgfpathlineto{\pgfqpoint{1.636002in}{1.295421in}}%
\pgfpathlineto{\pgfqpoint{1.580263in}{1.336530in}}%
\pgfpathlineto{\pgfqpoint{1.531415in}{1.364622in}}%
\pgfpathlineto{\pgfqpoint{1.486498in}{1.382961in}}%
\pgfpathlineto{\pgfqpoint{1.440279in}{1.393582in}}%
\pgfpathlineto{\pgfqpoint{1.383236in}{1.394437in}}%
\pgfpathlineto{\pgfqpoint{1.330002in}{1.382834in}}%
\pgfpathlineto{\pgfqpoint{1.330002in}{1.382834in}}%
\pgfpathlineto{\pgfqpoint{1.269083in}{1.355121in}}%
\pgfpathlineto{\pgfqpoint{1.269083in}{1.355121in}}%
\pgfpathlineto{\pgfqpoint{1.210037in}{1.314818in}}%
\pgfpathlineto{\pgfqpoint{1.159638in}{1.270985in}}%
\pgfpathlineto{\pgfqpoint{1.114788in}{1.225378in}}%
\pgfpathlineto{\pgfqpoint{1.073722in}{1.178702in}}%
\pgfpathlineto{\pgfqpoint{1.035420in}{1.131314in}}%
\pgfpathlineto{\pgfqpoint{0.999263in}{1.083423in}}%
\pgfpathlineto{\pgfqpoint{0.964834in}{1.035160in}}%
\pgfpathlineto{\pgfqpoint{0.931820in}{0.986604in}}%
\pgfpathlineto{\pgfqpoint{0.899985in}{0.937810in}}%
\pgfpathlineto{\pgfqpoint{0.869161in}{0.888818in}}%
\pgfpathlineto{\pgfqpoint{0.839230in}{0.839663in}}%
\pgfpathlineto{\pgfqpoint{0.810090in}{0.790368in}}%
\pgfpathlineto{\pgfqpoint{0.781644in}{0.740948in}}%
\pgfpathlineto{\pgfqpoint{0.753836in}{0.691420in}}%
\pgfpathlineto{\pgfqpoint{0.726608in}{0.641798in}}%
\pgfpathlineto{\pgfqpoint{0.699907in}{0.592088in}}%
\pgfpathlineto{\pgfqpoint{0.673699in}{0.542301in}}%
\pgfpathlineto{\pgfqpoint{0.647939in}{0.492442in}}%
\pgfpathlineto{\pgfqpoint{0.647939in}{0.492442in}}%
\pgfusepath{stroke}%
\end{pgfscope}%
\begin{pgfscope}%
\pgfpathrectangle{\pgfqpoint{0.647939in}{0.492442in}}{\pgfqpoint{4.273799in}{2.331163in}}%
\pgfusepath{clip}%
\pgfsetbuttcap%
\pgfsetroundjoin%
\pgfsetlinewidth{0.301125pt}%
\definecolor{currentstroke}{rgb}{0.500000,0.500000,0.500000}%
\pgfsetstrokecolor{currentstroke}%
\pgfsetstrokeopacity{0.300000}%
\pgfsetdash{}{0pt}%
\pgfpathmoveto{\pgfqpoint{1.867741in}{0.492442in}}%
\pgfpathlineto{\pgfqpoint{1.847379in}{0.515025in}}%
\pgfpathlineto{\pgfqpoint{1.805016in}{0.561388in}}%
\pgfpathlineto{\pgfqpoint{1.761607in}{0.607462in}}%
\pgfpathlineto{\pgfqpoint{1.716826in}{0.653141in}}%
\pgfpathlineto{\pgfqpoint{1.670232in}{0.698273in}}%
\pgfpathlineto{\pgfqpoint{1.621202in}{0.742624in}}%
\pgfpathlineto{\pgfqpoint{1.568816in}{0.785803in}}%
\pgfpathlineto{\pgfqpoint{1.511647in}{0.827113in}}%
\pgfpathlineto{\pgfqpoint{1.450714in}{0.863328in}}%
\pgfpathlineto{\pgfqpoint{1.395393in}{0.888301in}}%
\pgfpathlineto{\pgfqpoint{1.342671in}{0.904333in}}%
\pgfpathlineto{\pgfqpoint{1.286276in}{0.912370in}}%
\pgfpathlineto{\pgfqpoint{1.219315in}{0.908904in}}%
\pgfpathlineto{\pgfqpoint{1.158552in}{0.893232in}}%
\pgfpathlineto{\pgfqpoint{1.158552in}{0.893232in}}%
\pgfpathlineto{\pgfqpoint{1.087420in}{0.859557in}}%
\pgfpathlineto{\pgfqpoint{1.028048in}{0.819296in}}%
\pgfpathlineto{\pgfqpoint{0.976645in}{0.775838in}}%
\pgfpathlineto{\pgfqpoint{0.930693in}{0.730567in}}%
\pgfpathlineto{\pgfqpoint{0.888686in}{0.684149in}}%
\pgfpathlineto{\pgfqpoint{0.849672in}{0.636947in}}%
\pgfpathlineto{\pgfqpoint{0.813009in}{0.589176in}}%
\pgfpathlineto{\pgfqpoint{0.778246in}{0.540976in}}%
\pgfpathlineto{\pgfqpoint{0.745071in}{0.492442in}}%
\pgfpathlineto{\pgfqpoint{0.745071in}{0.492442in}}%
\pgfusepath{stroke}%
\end{pgfscope}%
\begin{pgfscope}%
\pgfpathrectangle{\pgfqpoint{0.647939in}{0.492442in}}{\pgfqpoint{4.273799in}{2.331163in}}%
\pgfusepath{clip}%
\pgfsetbuttcap%
\pgfsetroundjoin%
\pgfsetlinewidth{0.301125pt}%
\definecolor{currentstroke}{rgb}{0.500000,0.500000,0.500000}%
\pgfsetstrokecolor{currentstroke}%
\pgfsetstrokeopacity{0.300000}%
\pgfsetdash{}{0pt}%
\pgfpathmoveto{\pgfqpoint{1.626687in}{0.492442in}}%
\pgfpathlineto{\pgfqpoint{1.602135in}{0.513638in}}%
\pgfpathlineto{\pgfqpoint{1.549700in}{0.556810in}}%
\pgfpathlineto{\pgfqpoint{1.493075in}{0.598350in}}%
\pgfpathlineto{\pgfqpoint{1.430451in}{0.637179in}}%
\pgfpathlineto{\pgfqpoint{1.370674in}{0.666338in}}%
\pgfpathlineto{\pgfqpoint{1.314693in}{0.685836in}}%
\pgfpathlineto{\pgfqpoint{1.258359in}{0.697140in}}%
\pgfpathlineto{\pgfqpoint{1.195242in}{0.699185in}}%
\pgfpathlineto{\pgfqpoint{1.134559in}{0.690085in}}%
\pgfpathlineto{\pgfqpoint{1.134559in}{0.690085in}}%
\pgfpathlineto{\pgfqpoint{1.056302in}{0.661790in}}%
\pgfpathlineto{\pgfqpoint{0.991290in}{0.624333in}}%
\pgfpathlineto{\pgfqpoint{0.935691in}{0.582455in}}%
\pgfpathlineto{\pgfqpoint{0.886615in}{0.538183in}}%
\pgfpathlineto{\pgfqpoint{0.842203in}{0.492442in}}%
\pgfpathlineto{\pgfqpoint{0.842203in}{0.492442in}}%
\pgfusepath{stroke}%
\end{pgfscope}%
\begin{pgfscope}%
\pgfpathrectangle{\pgfqpoint{0.647939in}{0.492442in}}{\pgfqpoint{4.273799in}{2.331163in}}%
\pgfusepath{clip}%
\pgfsetbuttcap%
\pgfsetroundjoin%
\pgfsetlinewidth{0.301125pt}%
\definecolor{currentstroke}{rgb}{0.500000,0.500000,0.500000}%
\pgfsetstrokecolor{currentstroke}%
\pgfsetstrokeopacity{0.300000}%
\pgfsetdash{}{0pt}%
\pgfpathmoveto{\pgfqpoint{1.452175in}{0.492442in}}%
\pgfpathlineto{\pgfqpoint{1.447841in}{0.495160in}}%
\pgfpathlineto{\pgfqpoint{1.383044in}{0.531458in}}%
\pgfpathlineto{\pgfqpoint{1.323764in}{0.556695in}}%
\pgfpathlineto{\pgfqpoint{1.267030in}{0.572956in}}%
\pgfpathlineto{\pgfqpoint{1.207380in}{0.581061in}}%
\pgfpathlineto{\pgfqpoint{1.138479in}{0.578176in}}%
\pgfpathlineto{\pgfqpoint{1.075053in}{0.563482in}}%
\pgfpathlineto{\pgfqpoint{1.075053in}{0.563482in}}%
\pgfpathlineto{\pgfqpoint{1.001308in}{0.531501in}}%
\pgfpathlineto{\pgfqpoint{0.939334in}{0.492442in}}%
\pgfpathlineto{\pgfqpoint{0.939334in}{0.492442in}}%
\pgfusepath{stroke}%
\end{pgfscope}%
\begin{pgfscope}%
\pgfpathrectangle{\pgfqpoint{0.647939in}{0.492442in}}{\pgfqpoint{4.273799in}{2.331163in}}%
\pgfusepath{clip}%
\pgfsetbuttcap%
\pgfsetroundjoin%
\pgfsetlinewidth{0.301125pt}%
\definecolor{currentstroke}{rgb}{0.500000,0.500000,0.500000}%
\pgfsetstrokecolor{currentstroke}%
\pgfsetstrokeopacity{0.300000}%
\pgfsetdash{}{0pt}%
\pgfpathmoveto{\pgfqpoint{1.716389in}{0.492442in}}%
\pgfpathlineto{\pgfqpoint{1.716389in}{0.492442in}}%
\pgfpathlineto{\pgfqpoint{1.669897in}{0.537608in}}%
\pgfpathlineto{\pgfqpoint{1.621251in}{0.582089in}}%
\pgfpathlineto{\pgfqpoint{1.569711in}{0.625579in}}%
\pgfpathlineto{\pgfqpoint{1.514175in}{0.667566in}}%
\pgfpathlineto{\pgfqpoint{1.452902in}{0.707054in}}%
\pgfpathlineto{\pgfqpoint{1.383121in}{0.741922in}}%
\pgfpathlineto{\pgfqpoint{1.383121in}{0.741922in}}%
\pgfpathlineto{\pgfqpoint{1.315764in}{0.764038in}}%
\pgfpathlineto{\pgfqpoint{1.315764in}{0.764038in}}%
\pgfpathlineto{\pgfqpoint{1.255293in}{0.773241in}}%
\pgfpathlineto{\pgfqpoint{1.192118in}{0.771310in}}%
\pgfpathlineto{\pgfqpoint{1.138796in}{0.760230in}}%
\pgfpathlineto{\pgfqpoint{1.087961in}{0.741488in}}%
\pgfpathlineto{\pgfqpoint{1.035736in}{0.713932in}}%
\pgfpathlineto{\pgfqpoint{0.980134in}{0.675504in}}%
\pgfpathlineto{\pgfqpoint{0.928854in}{0.632017in}}%
\pgfusepath{stroke}%
\end{pgfscope}%
\begin{pgfscope}%
\pgfpathrectangle{\pgfqpoint{0.647939in}{0.492442in}}{\pgfqpoint{4.273799in}{2.331163in}}%
\pgfusepath{clip}%
\pgfsetbuttcap%
\pgfsetroundjoin%
\pgfsetlinewidth{0.301125pt}%
\definecolor{currentstroke}{rgb}{0.500000,0.500000,0.500000}%
\pgfsetstrokecolor{currentstroke}%
\pgfsetstrokeopacity{0.300000}%
\pgfsetdash{}{0pt}%
\pgfpathmoveto{\pgfqpoint{2.007784in}{0.492442in}}%
\pgfpathlineto{\pgfqpoint{2.007784in}{0.492442in}}%
\pgfpathlineto{\pgfqpoint{1.968059in}{0.539496in}}%
\pgfpathlineto{\pgfqpoint{1.927993in}{0.586464in}}%
\pgfpathlineto{\pgfqpoint{1.887451in}{0.633309in}}%
\pgfpathlineto{\pgfqpoint{1.846262in}{0.679987in}}%
\pgfpathlineto{\pgfqpoint{1.804212in}{0.726434in}}%
\pgfpathlineto{\pgfqpoint{1.761019in}{0.772567in}}%
\pgfpathlineto{\pgfqpoint{1.716302in}{0.818264in}}%
\pgfpathlineto{\pgfqpoint{1.669533in}{0.863341in}}%
\pgfpathlineto{\pgfqpoint{1.619937in}{0.907499in}}%
\pgfpathlineto{\pgfqpoint{1.566324in}{0.950216in}}%
\pgfpathlineto{\pgfqpoint{1.506747in}{0.990475in}}%
\pgfpathlineto{\pgfqpoint{1.437915in}{1.025816in}}%
\pgfpathlineto{\pgfqpoint{1.437915in}{1.025816in}}%
\pgfpathlineto{\pgfqpoint{1.375739in}{1.046056in}}%
\pgfpathlineto{\pgfqpoint{1.375739in}{1.046056in}}%
\pgfpathlineto{\pgfqpoint{1.319812in}{1.053764in}}%
\pgfpathlineto{\pgfqpoint{1.262018in}{1.050642in}}%
\pgfpathlineto{\pgfqpoint{1.212421in}{1.038872in}}%
\pgfpathlineto{\pgfqpoint{1.164673in}{1.019562in}}%
\pgfpathlineto{\pgfqpoint{1.114941in}{0.991267in}}%
\pgfpathlineto{\pgfqpoint{1.061143in}{0.951591in}}%
\pgfpathlineto{\pgfqpoint{1.011840in}{0.907432in}}%
\pgfpathlineto{\pgfqpoint{0.967461in}{0.861698in}}%
\pgfpathlineto{\pgfqpoint{0.926671in}{0.814960in}}%
\pgfpathlineto{\pgfqpoint{0.888617in}{0.767527in}}%
\pgfusepath{stroke}%
\end{pgfscope}%
\begin{pgfscope}%
\pgfpathrectangle{\pgfqpoint{0.647939in}{0.492442in}}{\pgfqpoint{4.273799in}{2.331163in}}%
\pgfusepath{clip}%
\pgfsetbuttcap%
\pgfsetroundjoin%
\pgfsetlinewidth{0.301125pt}%
\definecolor{currentstroke}{rgb}{0.500000,0.500000,0.500000}%
\pgfsetstrokecolor{currentstroke}%
\pgfsetstrokeopacity{0.300000}%
\pgfsetdash{}{0pt}%
\pgfpathmoveto{\pgfqpoint{2.104916in}{0.492442in}}%
\pgfpathlineto{\pgfqpoint{2.104916in}{0.492442in}}%
\pgfpathlineto{\pgfqpoint{2.065977in}{0.539691in}}%
\pgfpathlineto{\pgfqpoint{2.026959in}{0.586921in}}%
\pgfpathlineto{\pgfqpoint{1.987770in}{0.634109in}}%
\pgfpathlineto{\pgfqpoint{1.948298in}{0.681226in}}%
\pgfpathlineto{\pgfqpoint{1.908411in}{0.728240in}}%
\pgfpathlineto{\pgfqpoint{1.867949in}{0.775106in}}%
\pgfpathlineto{\pgfqpoint{1.826707in}{0.821768in}}%
\pgfpathlineto{\pgfqpoint{1.784420in}{0.868151in}}%
\pgfpathlineto{\pgfqpoint{1.740729in}{0.914142in}}%
\pgfpathlineto{\pgfqpoint{1.695135in}{0.959577in}}%
\pgfpathlineto{\pgfqpoint{1.646905in}{1.004189in}}%
\pgfpathlineto{\pgfqpoint{1.594908in}{1.047513in}}%
\pgfpathlineto{\pgfqpoint{1.537241in}{1.088584in}}%
\pgfpathlineto{\pgfqpoint{1.470613in}{1.125214in}}%
\pgfpathlineto{\pgfqpoint{1.470613in}{1.125214in}}%
\pgfpathlineto{\pgfqpoint{1.409128in}{1.147205in}}%
\pgfpathlineto{\pgfqpoint{1.409128in}{1.147205in}}%
\pgfpathlineto{\pgfqpoint{1.354238in}{1.156160in}}%
\pgfpathlineto{\pgfqpoint{1.296322in}{1.154012in}}%
\pgfpathlineto{\pgfqpoint{1.247783in}{1.142921in}}%
\pgfpathlineto{\pgfqpoint{1.201801in}{1.124577in}}%
\pgfpathlineto{\pgfqpoint{1.153456in}{1.097261in}}%
\pgfpathlineto{\pgfqpoint{1.100811in}{1.058570in}}%
\pgfpathlineto{\pgfqpoint{1.051496in}{1.014421in}}%
\pgfusepath{stroke}%
\end{pgfscope}%
\begin{pgfscope}%
\pgfpathrectangle{\pgfqpoint{0.647939in}{0.492442in}}{\pgfqpoint{4.273799in}{2.331163in}}%
\pgfusepath{clip}%
\pgfsetbuttcap%
\pgfsetroundjoin%
\pgfsetlinewidth{0.301125pt}%
\definecolor{currentstroke}{rgb}{0.500000,0.500000,0.500000}%
\pgfsetstrokecolor{currentstroke}%
\pgfsetstrokeopacity{0.300000}%
\pgfsetdash{}{0pt}%
\pgfpathmoveto{\pgfqpoint{2.202048in}{0.492442in}}%
\pgfpathlineto{\pgfqpoint{2.202048in}{0.492442in}}%
\pgfpathlineto{\pgfqpoint{2.163412in}{0.539765in}}%
\pgfpathlineto{\pgfqpoint{2.124884in}{0.587115in}}%
\pgfpathlineto{\pgfqpoint{2.086401in}{0.634475in}}%
\pgfpathlineto{\pgfqpoint{2.047890in}{0.681829in}}%
\pgfpathlineto{\pgfqpoint{2.009266in}{0.729155in}}%
\pgfpathlineto{\pgfqpoint{1.970429in}{0.776429in}}%
\pgfpathlineto{\pgfqpoint{1.931259in}{0.823621in}}%
\pgfpathlineto{\pgfqpoint{1.891606in}{0.870692in}}%
\pgfpathlineto{\pgfqpoint{1.851280in}{0.917593in}}%
\pgfpathlineto{\pgfqpoint{1.810034in}{0.964255in}}%
\pgfpathlineto{\pgfqpoint{1.767535in}{1.010579in}}%
\pgfpathlineto{\pgfqpoint{1.723313in}{1.056420in}}%
\pgfpathlineto{\pgfqpoint{1.676681in}{1.101537in}}%
\pgfpathlineto{\pgfqpoint{1.626572in}{1.145513in}}%
\pgfpathlineto{\pgfqpoint{1.571214in}{1.187526in}}%
\pgfpathlineto{\pgfqpoint{1.507439in}{1.225709in}}%
\pgfpathlineto{\pgfqpoint{1.507439in}{1.225709in}}%
\pgfpathlineto{\pgfqpoint{1.446003in}{1.250369in}}%
\pgfpathlineto{\pgfqpoint{1.446003in}{1.250369in}}%
\pgfpathlineto{\pgfqpoint{1.392309in}{1.261010in}}%
\pgfpathlineto{\pgfqpoint{1.334287in}{1.260236in}}%
\pgfpathlineto{\pgfqpoint{1.286989in}{1.250088in}}%
\pgfpathlineto{\pgfqpoint{1.242515in}{1.232769in}}%
\pgfpathlineto{\pgfqpoint{1.195715in}{1.206646in}}%
\pgfpathlineto{\pgfqpoint{1.144555in}{1.169311in}}%
\pgfpathlineto{\pgfqpoint{1.095149in}{1.125206in}}%
\pgfusepath{stroke}%
\end{pgfscope}%
\begin{pgfscope}%
\pgfpathrectangle{\pgfqpoint{0.647939in}{0.492442in}}{\pgfqpoint{4.273799in}{2.331163in}}%
\pgfusepath{clip}%
\pgfsetbuttcap%
\pgfsetroundjoin%
\pgfsetlinewidth{0.301125pt}%
\definecolor{currentstroke}{rgb}{0.500000,0.500000,0.500000}%
\pgfsetstrokecolor{currentstroke}%
\pgfsetstrokeopacity{0.300000}%
\pgfsetdash{}{0pt}%
\pgfpathmoveto{\pgfqpoint{2.493443in}{0.492442in}}%
\pgfpathlineto{\pgfqpoint{2.493443in}{0.492442in}}%
\pgfpathlineto{\pgfqpoint{2.453402in}{0.539416in}}%
\pgfpathlineto{\pgfqpoint{2.413889in}{0.586523in}}%
\pgfpathlineto{\pgfqpoint{2.374872in}{0.633753in}}%
\pgfpathlineto{\pgfqpoint{2.336310in}{0.681094in}}%
\pgfpathlineto{\pgfqpoint{2.298164in}{0.728535in}}%
\pgfpathlineto{\pgfqpoint{2.260393in}{0.776066in}}%
\pgfpathlineto{\pgfqpoint{2.222954in}{0.823675in}}%
\pgfpathlineto{\pgfqpoint{2.185803in}{0.871350in}}%
\pgfpathlineto{\pgfqpoint{2.148890in}{0.919081in}}%
\pgfpathlineto{\pgfqpoint{2.112158in}{0.966853in}}%
\pgfpathlineto{\pgfqpoint{2.075543in}{1.014652in}}%
\pgfpathlineto{\pgfqpoint{2.038961in}{1.062458in}}%
\pgfpathlineto{\pgfqpoint{2.002318in}{1.110251in}}%
\pgfpathlineto{\pgfqpoint{1.965505in}{1.158004in}}%
\pgfpathlineto{\pgfqpoint{1.928388in}{1.205687in}}%
\pgfpathlineto{\pgfqpoint{1.890788in}{1.253257in}}%
\pgfpathlineto{\pgfqpoint{1.852471in}{1.300655in}}%
\pgfpathlineto{\pgfqpoint{1.813105in}{1.347798in}}%
\pgfpathlineto{\pgfqpoint{1.772198in}{1.394545in}}%
\pgfpathlineto{\pgfqpoint{1.728985in}{1.440667in}}%
\pgfpathlineto{\pgfqpoint{1.682189in}{1.485725in}}%
\pgfpathlineto{\pgfqpoint{1.629377in}{1.528686in}}%
\pgfpathlineto{\pgfqpoint{1.565435in}{1.566463in}}%
\pgfpathlineto{\pgfqpoint{1.565435in}{1.566463in}}%
\pgfpathlineto{\pgfqpoint{1.517656in}{1.582751in}}%
\pgfpathlineto{\pgfqpoint{1.517656in}{1.582751in}}%
\pgfpathlineto{\pgfqpoint{1.473216in}{1.587685in}}%
\pgfpathlineto{\pgfqpoint{1.428394in}{1.582439in}}%
\pgfpathlineto{\pgfqpoint{1.389955in}{1.570047in}}%
\pgfpathlineto{\pgfqpoint{1.350402in}{1.550071in}}%
\pgfpathlineto{\pgfqpoint{1.306201in}{1.519858in}}%
\pgfpathlineto{\pgfqpoint{1.255532in}{1.476222in}}%
\pgfpathlineto{\pgfqpoint{1.210590in}{1.430638in}}%
\pgfpathlineto{\pgfqpoint{1.169451in}{1.383970in}}%
\pgfpathlineto{\pgfqpoint{1.131049in}{1.336617in}}%
\pgfusepath{stroke}%
\end{pgfscope}%
\begin{pgfscope}%
\pgfpathrectangle{\pgfqpoint{0.647939in}{0.492442in}}{\pgfqpoint{4.273799in}{2.331163in}}%
\pgfusepath{clip}%
\pgfsetbuttcap%
\pgfsetroundjoin%
\pgfsetlinewidth{0.301125pt}%
\definecolor{currentstroke}{rgb}{0.500000,0.500000,0.500000}%
\pgfsetstrokecolor{currentstroke}%
\pgfsetstrokeopacity{0.300000}%
\pgfsetdash{}{0pt}%
\pgfpathmoveto{\pgfqpoint{2.590575in}{0.492442in}}%
\pgfpathlineto{\pgfqpoint{2.590575in}{0.492442in}}%
\pgfpathlineto{\pgfqpoint{2.549413in}{0.539127in}}%
\pgfpathlineto{\pgfqpoint{2.508894in}{0.585979in}}%
\pgfpathlineto{\pgfqpoint{2.468985in}{0.632986in}}%
\pgfpathlineto{\pgfqpoint{2.429654in}{0.680139in}}%
\pgfpathlineto{\pgfqpoint{2.390867in}{0.727425in}}%
\pgfpathlineto{\pgfqpoint{2.352590in}{0.774835in}}%
\pgfpathlineto{\pgfqpoint{2.314788in}{0.822358in}}%
\pgfpathlineto{\pgfqpoint{2.277424in}{0.869983in}}%
\pgfpathlineto{\pgfqpoint{2.240459in}{0.917702in}}%
\pgfpathlineto{\pgfqpoint{2.203853in}{0.965503in}}%
\pgfpathlineto{\pgfqpoint{2.167552in}{1.013373in}}%
\pgfpathlineto{\pgfqpoint{2.131504in}{1.061300in}}%
\pgfpathlineto{\pgfqpoint{2.095650in}{1.109270in}}%
\pgfpathlineto{\pgfqpoint{2.059929in}{1.157270in}}%
\pgfusepath{stroke}%
\end{pgfscope}%
\begin{pgfscope}%
\pgfpathrectangle{\pgfqpoint{0.647939in}{0.492442in}}{\pgfqpoint{4.273799in}{2.331163in}}%
\pgfusepath{clip}%
\pgfsetbuttcap%
\pgfsetroundjoin%
\pgfsetlinewidth{0.301125pt}%
\definecolor{currentstroke}{rgb}{0.500000,0.500000,0.500000}%
\pgfsetstrokecolor{currentstroke}%
\pgfsetstrokeopacity{0.300000}%
\pgfsetdash{}{0pt}%
\pgfpathmoveto{\pgfqpoint{2.687707in}{0.492442in}}%
\pgfpathlineto{\pgfqpoint{2.687707in}{0.492442in}}%
\pgfpathlineto{\pgfqpoint{2.645118in}{0.538744in}}%
\pgfpathlineto{\pgfqpoint{2.603286in}{0.585251in}}%
\pgfpathlineto{\pgfqpoint{2.562181in}{0.631951in}}%
\pgfpathlineto{\pgfqpoint{2.521772in}{0.678831in}}%
\pgfpathlineto{\pgfqpoint{2.482029in}{0.725879in}}%
\pgfusepath{stroke}%
\end{pgfscope}%
\begin{pgfscope}%
\pgfpathrectangle{\pgfqpoint{0.647939in}{0.492442in}}{\pgfqpoint{4.273799in}{2.331163in}}%
\pgfusepath{clip}%
\pgfsetbuttcap%
\pgfsetroundjoin%
\pgfsetlinewidth{0.301125pt}%
\definecolor{currentstroke}{rgb}{0.500000,0.500000,0.500000}%
\pgfsetstrokecolor{currentstroke}%
\pgfsetstrokeopacity{0.300000}%
\pgfsetdash{}{0pt}%
\pgfpathmoveto{\pgfqpoint{2.881971in}{0.492442in}}%
\pgfpathlineto{\pgfqpoint{2.881971in}{0.492442in}}%
\pgfpathlineto{\pgfqpoint{2.835699in}{0.537680in}}%
\pgfpathlineto{\pgfqpoint{2.790404in}{0.583210in}}%
\pgfpathlineto{\pgfqpoint{2.746063in}{0.629020in}}%
\pgfpathlineto{\pgfqpoint{2.702648in}{0.675092in}}%
\pgfpathlineto{\pgfqpoint{2.660131in}{0.721414in}}%
\pgfpathlineto{\pgfqpoint{2.618482in}{0.767969in}}%
\pgfpathlineto{\pgfqpoint{2.577671in}{0.814745in}}%
\pgfpathlineto{\pgfqpoint{2.537669in}{0.861728in}}%
\pgfpathlineto{\pgfqpoint{2.498446in}{0.908907in}}%
\pgfpathlineto{\pgfqpoint{2.459974in}{0.956269in}}%
\pgfpathlineto{\pgfqpoint{2.422221in}{1.003804in}}%
\pgfpathlineto{\pgfqpoint{2.385156in}{1.051499in}}%
\pgfpathlineto{\pgfqpoint{2.348751in}{1.099345in}}%
\pgfpathlineto{\pgfqpoint{2.312980in}{1.147334in}}%
\pgfpathlineto{\pgfqpoint{2.277820in}{1.195456in}}%
\pgfpathlineto{\pgfqpoint{2.243249in}{1.243705in}}%
\pgfpathlineto{\pgfqpoint{2.209239in}{1.292073in}}%
\pgfpathlineto{\pgfqpoint{2.175758in}{1.340551in}}%
\pgfpathlineto{\pgfqpoint{2.142785in}{1.389132in}}%
\pgfpathlineto{\pgfqpoint{2.110302in}{1.437811in}}%
\pgfpathlineto{\pgfqpoint{2.078287in}{1.486582in}}%
\pgfpathlineto{\pgfqpoint{2.046704in}{1.535437in}}%
\pgfpathlineto{\pgfqpoint{2.015526in}{1.584369in}}%
\pgfpathlineto{\pgfqpoint{1.984740in}{1.633375in}}%
\pgfpathlineto{\pgfqpoint{1.954303in}{1.682446in}}%
\pgfpathlineto{\pgfqpoint{1.924160in}{1.731569in}}%
\pgfpathlineto{\pgfqpoint{1.894275in}{1.780739in}}%
\pgfpathlineto{\pgfqpoint{1.864571in}{1.829941in}}%
\pgfpathlineto{\pgfqpoint{1.834903in}{1.879146in}}%
\pgfpathlineto{\pgfqpoint{1.805051in}{1.928312in}}%
\pgfpathlineto{\pgfqpoint{1.774514in}{1.977350in}}%
\pgfpathlineto{\pgfqpoint{1.741597in}{2.025843in}}%
\pgfpathlineto{\pgfqpoint{1.741597in}{2.025843in}}%
\pgfpathlineto{\pgfqpoint{1.713478in}{2.057388in}}%
\pgfpathlineto{\pgfqpoint{1.713478in}{2.057388in}}%
\pgfpathlineto{\pgfqpoint{1.695406in}{2.067399in}}%
\pgfpathlineto{\pgfqpoint{1.695406in}{2.067399in}}%
\pgfpathlineto{\pgfqpoint{1.677747in}{2.066652in}}%
\pgfpathlineto{\pgfqpoint{1.662766in}{2.060348in}}%
\pgfpathlineto{\pgfqpoint{1.644920in}{2.048309in}}%
\pgfpathlineto{\pgfqpoint{1.617448in}{2.025004in}}%
\pgfpathlineto{\pgfqpoint{1.573360in}{1.982134in}}%
\pgfpathlineto{\pgfqpoint{1.529845in}{1.936701in}}%
\pgfpathlineto{\pgfqpoint{1.487932in}{1.890563in}}%
\pgfpathlineto{\pgfqpoint{1.447422in}{1.843932in}}%
\pgfpathlineto{\pgfqpoint{1.408167in}{1.796911in}}%
\pgfpathlineto{\pgfqpoint{1.370045in}{1.749562in}}%
\pgfpathlineto{\pgfqpoint{1.332951in}{1.701930in}}%
\pgfpathlineto{\pgfqpoint{1.296805in}{1.654049in}}%
\pgfpathlineto{\pgfqpoint{1.261585in}{1.605964in}}%
\pgfpathlineto{\pgfqpoint{1.227244in}{1.557705in}}%
\pgfusepath{stroke}%
\end{pgfscope}%
\begin{pgfscope}%
\pgfpathrectangle{\pgfqpoint{0.647939in}{0.492442in}}{\pgfqpoint{4.273799in}{2.331163in}}%
\pgfusepath{clip}%
\pgfsetbuttcap%
\pgfsetroundjoin%
\pgfsetlinewidth{0.301125pt}%
\definecolor{currentstroke}{rgb}{0.500000,0.500000,0.500000}%
\pgfsetstrokecolor{currentstroke}%
\pgfsetstrokeopacity{0.300000}%
\pgfsetdash{}{0pt}%
\pgfpathmoveto{\pgfqpoint{3.076234in}{0.492442in}}%
\pgfpathlineto{\pgfqpoint{3.076234in}{0.492442in}}%
\pgfpathlineto{\pgfqpoint{3.025341in}{0.536178in}}%
\pgfpathlineto{\pgfqpoint{2.975608in}{0.580309in}}%
\pgfpathlineto{\pgfqpoint{2.927031in}{0.624821in}}%
\pgfpathlineto{\pgfqpoint{2.879596in}{0.669698in}}%
\pgfpathlineto{\pgfqpoint{2.833284in}{0.714923in}}%
\pgfpathlineto{\pgfqpoint{2.788071in}{0.760477in}}%
\pgfpathlineto{\pgfqpoint{2.743929in}{0.806343in}}%
\pgfpathlineto{\pgfqpoint{2.700828in}{0.852503in}}%
\pgfpathlineto{\pgfqpoint{2.658740in}{0.898940in}}%
\pgfpathlineto{\pgfqpoint{2.617632in}{0.945638in}}%
\pgfpathlineto{\pgfqpoint{2.577477in}{0.992582in}}%
\pgfpathlineto{\pgfqpoint{2.538245in}{1.039758in}}%
\pgfpathlineto{\pgfqpoint{2.499910in}{1.087154in}}%
\pgfpathlineto{\pgfqpoint{2.462448in}{1.134756in}}%
\pgfpathlineto{\pgfqpoint{2.425833in}{1.182554in}}%
\pgfpathlineto{\pgfqpoint{2.390040in}{1.230537in}}%
\pgfpathlineto{\pgfqpoint{2.355056in}{1.278698in}}%
\pgfpathlineto{\pgfqpoint{2.320869in}{1.327028in}}%
\pgfpathlineto{\pgfqpoint{2.287475in}{1.375523in}}%
\pgfpathlineto{\pgfqpoint{2.254867in}{1.424177in}}%
\pgfpathlineto{\pgfqpoint{2.223039in}{1.472984in}}%
\pgfpathlineto{\pgfqpoint{2.192007in}{1.521944in}}%
\pgfpathlineto{\pgfqpoint{2.161799in}{1.571057in}}%
\pgfpathlineto{\pgfqpoint{2.132439in}{1.620322in}}%
\pgfpathlineto{\pgfqpoint{2.103974in}{1.669743in}}%
\pgfpathlineto{\pgfqpoint{2.076491in}{1.719330in}}%
\pgfpathlineto{\pgfqpoint{2.050088in}{1.769090in}}%
\pgfpathlineto{\pgfqpoint{2.024925in}{1.819040in}}%
\pgfpathlineto{\pgfqpoint{2.001239in}{1.869203in}}%
\pgfpathlineto{\pgfqpoint{1.979378in}{1.919611in}}%
\pgfpathlineto{\pgfqpoint{1.959897in}{1.970305in}}%
\pgfpathlineto{\pgfqpoint{1.943688in}{2.021336in}}%
\pgfpathlineto{\pgfqpoint{1.932220in}{2.072736in}}%
\pgfpathlineto{\pgfqpoint{1.927926in}{2.124423in}}%
\pgfpathlineto{\pgfqpoint{1.934539in}{2.175946in}}%
\pgfpathlineto{\pgfqpoint{1.956059in}{2.226103in}}%
\pgfpathlineto{\pgfqpoint{1.993890in}{2.273216in}}%
\pgfpathlineto{\pgfqpoint{2.044272in}{2.314921in}}%
\pgfpathlineto{\pgfqpoint{2.107729in}{2.353028in}}%
\pgfpathlineto{\pgfqpoint{2.181141in}{2.385409in}}%
\pgfpathlineto{\pgfqpoint{2.263458in}{2.410358in}}%
\pgfpathlineto{\pgfqpoint{2.349524in}{2.425092in}}%
\pgfpathlineto{\pgfqpoint{2.431020in}{2.428684in}}%
\pgfpathlineto{\pgfqpoint{2.506340in}{2.422752in}}%
\pgfpathlineto{\pgfqpoint{2.576637in}{2.408557in}}%
\pgfpathlineto{\pgfqpoint{2.642542in}{2.386778in}}%
\pgfpathlineto{\pgfqpoint{2.705828in}{2.356905in}}%
\pgfpathlineto{\pgfqpoint{2.765431in}{2.318728in}}%
\pgfpathlineto{\pgfqpoint{2.816492in}{2.275269in}}%
\pgfpathlineto{\pgfqpoint{2.857199in}{2.228662in}}%
\pgfpathlineto{\pgfqpoint{2.886442in}{2.179557in}}%
\pgfpathlineto{\pgfqpoint{2.899594in}{2.128651in}}%
\pgfpathlineto{\pgfqpoint{2.899594in}{2.128651in}}%
\pgfpathlineto{\pgfqpoint{2.894324in}{2.100889in}}%
\pgfpathlineto{\pgfqpoint{2.894324in}{2.100889in}}%
\pgfpathlineto{\pgfqpoint{2.880711in}{2.087107in}}%
\pgfpathlineto{\pgfqpoint{2.880711in}{2.087107in}}%
\pgfpathlineto{\pgfqpoint{2.862999in}{2.083552in}}%
\pgfpathlineto{\pgfqpoint{2.843840in}{2.088135in}}%
\pgfpathlineto{\pgfqpoint{2.826966in}{2.098413in}}%
\pgfpathlineto{\pgfqpoint{2.813696in}{2.114205in}}%
\pgfpathlineto{\pgfqpoint{2.813696in}{2.114205in}}%
\pgfpathlineto{\pgfqpoint{2.812965in}{2.129006in}}%
\pgfpathlineto{\pgfqpoint{2.821093in}{2.127357in}}%
\pgfpathlineto{\pgfqpoint{2.822788in}{2.124576in}}%
\pgfpathlineto{\pgfqpoint{2.818861in}{2.124468in}}%
\pgfpathlineto{\pgfqpoint{2.818861in}{2.124468in}}%
\pgfpathlineto{\pgfqpoint{2.823418in}{2.124264in}}%
\pgfpathlineto{\pgfqpoint{2.820186in}{2.123958in}}%
\pgfpathlineto{\pgfqpoint{2.823212in}{2.124291in}}%
\pgfpathlineto{\pgfqpoint{2.818253in}{2.125026in}}%
\pgfpathlineto{\pgfqpoint{2.818253in}{2.125026in}}%
\pgfpathlineto{\pgfqpoint{2.823148in}{2.124351in}}%
\pgfpathlineto{\pgfqpoint{2.820843in}{2.123745in}}%
\pgfpathlineto{\pgfqpoint{2.822615in}{2.124661in}}%
\pgfpathlineto{\pgfqpoint{2.819123in}{2.124281in}}%
\pgfpathlineto{\pgfqpoint{2.819123in}{2.124281in}}%
\pgfpathlineto{\pgfqpoint{2.823581in}{2.124214in}}%
\pgfpathlineto{\pgfqpoint{2.819740in}{2.124136in}}%
\pgfpathlineto{\pgfqpoint{2.823529in}{2.124107in}}%
\pgfpathlineto{\pgfqpoint{2.818470in}{2.124978in}}%
\pgfpathlineto{\pgfqpoint{2.818470in}{2.124978in}}%
\pgfpathlineto{\pgfqpoint{2.823309in}{2.124161in}}%
\pgfpathlineto{\pgfqpoint{2.820532in}{2.123912in}}%
\pgfpathlineto{\pgfqpoint{2.822903in}{2.124377in}}%
\pgfpathlineto{\pgfqpoint{2.818750in}{2.124779in}}%
\pgfpathlineto{\pgfqpoint{2.818750in}{2.124779in}}%
\pgfpathlineto{\pgfqpoint{2.823470in}{2.124044in}}%
\pgfpathlineto{\pgfqpoint{2.820151in}{2.124103in}}%
\pgfpathlineto{\pgfqpoint{2.823212in}{2.124123in}}%
\pgfpathlineto{\pgfqpoint{2.818327in}{2.125246in}}%
\pgfpathlineto{\pgfqpoint{2.818327in}{2.125246in}}%
\pgfpathlineto{\pgfqpoint{2.823326in}{2.124060in}}%
\pgfpathlineto{\pgfqpoint{2.820577in}{2.123940in}}%
\pgfpathlineto{\pgfqpoint{2.822828in}{2.124357in}}%
\pgfpathlineto{\pgfqpoint{2.818887in}{2.124771in}}%
\pgfpathlineto{\pgfqpoint{2.818887in}{2.124771in}}%
\pgfpathlineto{\pgfqpoint{2.823588in}{2.123895in}}%
\pgfpathlineto{\pgfqpoint{2.819861in}{2.124315in}}%
\pgfpathlineto{\pgfqpoint{2.823421in}{2.123893in}}%
\pgfpathlineto{\pgfqpoint{2.818157in}{2.125571in}}%
\pgfpathlineto{\pgfqpoint{2.818157in}{2.125571in}}%
\pgfpathlineto{\pgfqpoint{2.823337in}{2.123965in}}%
\pgfpathlineto{\pgfqpoint{2.820634in}{2.123955in}}%
\pgfpathlineto{\pgfqpoint{2.822745in}{2.124355in}}%
\pgfpathlineto{\pgfqpoint{2.819033in}{2.124738in}}%
\pgfpathlineto{\pgfqpoint{2.819033in}{2.124738in}}%
\pgfpathlineto{\pgfqpoint{2.823700in}{2.123761in}}%
\pgfpathlineto{\pgfqpoint{2.819541in}{2.124561in}}%
\pgfpathlineto{\pgfqpoint{2.823642in}{2.123660in}}%
\pgfpathlineto{\pgfqpoint{2.818631in}{2.125363in}}%
\pgfpathlineto{\pgfqpoint{2.824783in}{2.122530in}}%
\pgfpathlineto{\pgfqpoint{2.819040in}{2.125130in}}%
\pgfpathlineto{\pgfqpoint{2.822969in}{2.123975in}}%
\pgfpathlineto{\pgfqpoint{2.820008in}{2.124505in}}%
\pgfusepath{stroke}%
\end{pgfscope}%
\begin{pgfscope}%
\pgfpathrectangle{\pgfqpoint{0.647939in}{0.492442in}}{\pgfqpoint{4.273799in}{2.331163in}}%
\pgfusepath{clip}%
\pgfsetbuttcap%
\pgfsetroundjoin%
\pgfsetlinewidth{0.301125pt}%
\definecolor{currentstroke}{rgb}{0.500000,0.500000,0.500000}%
\pgfsetstrokecolor{currentstroke}%
\pgfsetstrokeopacity{0.300000}%
\pgfsetdash{}{0pt}%
\pgfpathmoveto{\pgfqpoint{3.270498in}{0.492442in}}%
\pgfpathlineto{\pgfqpoint{3.270498in}{0.492442in}}%
\pgfpathlineto{\pgfqpoint{3.214485in}{0.534273in}}%
\pgfpathlineto{\pgfqpoint{3.159685in}{0.576580in}}%
\pgfpathlineto{\pgfqpoint{3.106145in}{0.619365in}}%
\pgfpathlineto{\pgfqpoint{3.053898in}{0.662622in}}%
\pgfpathlineto{\pgfqpoint{3.002956in}{0.706339in}}%
\pgfpathlineto{\pgfqpoint{2.953317in}{0.750500in}}%
\pgfpathlineto{\pgfqpoint{2.904968in}{0.795085in}}%
\pgfpathlineto{\pgfqpoint{2.857890in}{0.840073in}}%
\pgfpathlineto{\pgfqpoint{2.812058in}{0.885443in}}%
\pgfpathlineto{\pgfqpoint{2.767443in}{0.931173in}}%
\pgfpathlineto{\pgfqpoint{2.724014in}{0.977241in}}%
\pgfpathlineto{\pgfqpoint{2.681738in}{1.023628in}}%
\pgfpathlineto{\pgfqpoint{2.640585in}{1.070314in}}%
\pgfpathlineto{\pgfqpoint{2.600525in}{1.117282in}}%
\pgfpathlineto{\pgfqpoint{2.561528in}{1.164515in}}%
\pgfpathlineto{\pgfqpoint{2.523573in}{1.212001in}}%
\pgfpathlineto{\pgfqpoint{2.486639in}{1.259725in}}%
\pgfpathlineto{\pgfqpoint{2.450713in}{1.307678in}}%
\pgfpathlineto{\pgfqpoint{2.415786in}{1.355850in}}%
\pgfpathlineto{\pgfqpoint{2.381855in}{1.404234in}}%
\pgfpathlineto{\pgfqpoint{2.348915in}{1.452821in}}%
\pgfpathlineto{\pgfqpoint{2.316982in}{1.501607in}}%
\pgfpathlineto{\pgfqpoint{2.286089in}{1.550593in}}%
\pgfpathlineto{\pgfqpoint{2.256272in}{1.599776in}}%
\pgfpathlineto{\pgfqpoint{2.227581in}{1.649158in}}%
\pgfpathlineto{\pgfqpoint{2.200107in}{1.698745in}}%
\pgfpathlineto{\pgfqpoint{2.173963in}{1.748546in}}%
\pgfpathlineto{\pgfqpoint{2.149300in}{1.798569in}}%
\pgfpathlineto{\pgfqpoint{2.126345in}{1.848834in}}%
\pgfpathlineto{\pgfqpoint{2.105405in}{1.899358in}}%
\pgfpathlineto{\pgfqpoint{2.086919in}{1.950165in}}%
\pgfpathlineto{\pgfqpoint{2.071525in}{2.001274in}}%
\pgfpathlineto{\pgfqpoint{2.060146in}{2.052687in}}%
\pgfpathlineto{\pgfqpoint{2.054126in}{2.104354in}}%
\pgfpathlineto{\pgfqpoint{2.055418in}{2.156091in}}%
\pgfpathlineto{\pgfqpoint{2.066624in}{2.207422in}}%
\pgfpathlineto{\pgfqpoint{2.090664in}{2.257358in}}%
\pgfpathlineto{\pgfqpoint{2.129824in}{2.304270in}}%
\pgfpathlineto{\pgfqpoint{2.184836in}{2.346055in}}%
\pgfusepath{stroke}%
\end{pgfscope}%
\begin{pgfscope}%
\pgfpathrectangle{\pgfqpoint{0.647939in}{0.492442in}}{\pgfqpoint{4.273799in}{2.331163in}}%
\pgfusepath{clip}%
\pgfsetbuttcap%
\pgfsetroundjoin%
\pgfsetlinewidth{0.301125pt}%
\definecolor{currentstroke}{rgb}{0.500000,0.500000,0.500000}%
\pgfsetstrokecolor{currentstroke}%
\pgfsetstrokeopacity{0.300000}%
\pgfsetdash{}{0pt}%
\pgfpathmoveto{\pgfqpoint{3.464761in}{0.492442in}}%
\pgfpathlineto{\pgfqpoint{3.464761in}{0.492442in}}%
\pgfpathlineto{\pgfqpoint{3.403945in}{0.532229in}}%
\pgfpathlineto{\pgfqpoint{3.344107in}{0.572454in}}%
\pgfpathlineto{\pgfqpoint{3.285403in}{0.613173in}}%
\pgfpathlineto{\pgfqpoint{3.227955in}{0.654421in}}%
\pgfpathlineto{\pgfqpoint{3.171855in}{0.696217in}}%
\pgfpathlineto{\pgfqpoint{3.117162in}{0.738563in}}%
\pgfpathlineto{\pgfqpoint{3.063910in}{0.781453in}}%
\pgfpathlineto{\pgfqpoint{3.012114in}{0.824871in}}%
\pgfpathlineto{\pgfqpoint{2.961774in}{0.868794in}}%
\pgfpathlineto{\pgfqpoint{2.912876in}{0.913200in}}%
\pgfpathlineto{\pgfqpoint{2.865397in}{0.958061in}}%
\pgfpathlineto{\pgfqpoint{2.819306in}{1.003352in}}%
\pgfpathlineto{\pgfqpoint{2.774573in}{1.049046in}}%
\pgfusepath{stroke}%
\end{pgfscope}%
\begin{pgfscope}%
\pgfpathrectangle{\pgfqpoint{0.647939in}{0.492442in}}{\pgfqpoint{4.273799in}{2.331163in}}%
\pgfusepath{clip}%
\pgfsetbuttcap%
\pgfsetroundjoin%
\pgfsetlinewidth{0.301125pt}%
\definecolor{currentstroke}{rgb}{0.500000,0.500000,0.500000}%
\pgfsetstrokecolor{currentstroke}%
\pgfsetstrokeopacity{0.300000}%
\pgfsetdash{}{0pt}%
\pgfpathmoveto{\pgfqpoint{3.659025in}{0.492442in}}%
\pgfpathlineto{\pgfqpoint{3.659025in}{0.492442in}}%
\pgfpathlineto{\pgfqpoint{3.594902in}{0.530655in}}%
\pgfpathlineto{\pgfqpoint{3.531084in}{0.569019in}}%
\pgfpathlineto{\pgfqpoint{3.467870in}{0.607679in}}%
\pgfpathlineto{\pgfqpoint{3.405521in}{0.646753in}}%
\pgfpathlineto{\pgfqpoint{3.344262in}{0.686335in}}%
\pgfpathlineto{\pgfqpoint{3.284272in}{0.726492in}}%
\pgfpathlineto{\pgfqpoint{3.225692in}{0.767262in}}%
\pgfpathlineto{\pgfqpoint{3.168623in}{0.808664in}}%
\pgfpathlineto{\pgfqpoint{3.113125in}{0.850697in}}%
\pgfpathlineto{\pgfqpoint{3.059234in}{0.893348in}}%
\pgfpathlineto{\pgfqpoint{3.006964in}{0.936595in}}%
\pgfpathlineto{\pgfqpoint{2.956308in}{0.980409in}}%
\pgfpathlineto{\pgfqpoint{2.907245in}{1.024759in}}%
\pgfpathlineto{\pgfqpoint{2.859747in}{1.069614in}}%
\pgfpathlineto{\pgfqpoint{2.813780in}{1.114942in}}%
\pgfpathlineto{\pgfqpoint{2.769307in}{1.160711in}}%
\pgfpathlineto{\pgfqpoint{2.726293in}{1.206894in}}%
\pgfpathlineto{\pgfqpoint{2.684703in}{1.253464in}}%
\pgfpathlineto{\pgfqpoint{2.644508in}{1.300396in}}%
\pgfpathlineto{\pgfqpoint{2.605683in}{1.347671in}}%
\pgfpathlineto{\pgfqpoint{2.568212in}{1.395269in}}%
\pgfpathlineto{\pgfqpoint{2.532086in}{1.443176in}}%
\pgfpathlineto{\pgfqpoint{2.497312in}{1.491380in}}%
\pgfpathlineto{\pgfqpoint{2.463908in}{1.539872in}}%
\pgfpathlineto{\pgfqpoint{2.431897in}{1.588642in}}%
\pgfpathlineto{\pgfqpoint{2.401327in}{1.637686in}}%
\pgfpathlineto{\pgfqpoint{2.372281in}{1.687005in}}%
\pgfpathlineto{\pgfqpoint{2.344862in}{1.736600in}}%
\pgfpathlineto{\pgfqpoint{2.319208in}{1.786475in}}%
\pgfpathlineto{\pgfqpoint{2.295526in}{1.836639in}}%
\pgfpathlineto{\pgfqpoint{2.274082in}{1.887100in}}%
\pgfpathlineto{\pgfqpoint{2.255250in}{1.937867in}}%
\pgfpathlineto{\pgfqpoint{2.239548in}{1.988948in}}%
\pgfpathlineto{\pgfqpoint{2.227695in}{2.040330in}}%
\pgfpathlineto{\pgfqpoint{2.220711in}{2.091970in}}%
\pgfpathlineto{\pgfqpoint{2.220062in}{2.143732in}}%
\pgfpathlineto{\pgfqpoint{2.227836in}{2.195283in}}%
\pgfpathlineto{\pgfqpoint{2.246921in}{2.245872in}}%
\pgfpathlineto{\pgfqpoint{2.281090in}{2.293877in}}%
\pgfpathlineto{\pgfqpoint{2.281090in}{2.293877in}}%
\pgfpathlineto{\pgfqpoint{2.324291in}{2.330056in}}%
\pgfpathlineto{\pgfqpoint{2.383332in}{2.359320in}}%
\pgfusepath{stroke}%
\end{pgfscope}%
\begin{pgfscope}%
\pgfpathrectangle{\pgfqpoint{0.647939in}{0.492442in}}{\pgfqpoint{4.273799in}{2.331163in}}%
\pgfusepath{clip}%
\pgfsetbuttcap%
\pgfsetroundjoin%
\pgfsetlinewidth{0.301125pt}%
\definecolor{currentstroke}{rgb}{0.500000,0.500000,0.500000}%
\pgfsetstrokecolor{currentstroke}%
\pgfsetstrokeopacity{0.300000}%
\pgfsetdash{}{0pt}%
\pgfpathmoveto{\pgfqpoint{3.756157in}{0.492442in}}%
\pgfpathlineto{\pgfqpoint{3.756157in}{0.492442in}}%
\pgfpathlineto{\pgfqpoint{3.691342in}{0.530306in}}%
\pgfpathlineto{\pgfqpoint{3.626324in}{0.568067in}}%
\pgfpathlineto{\pgfqpoint{3.561459in}{0.605906in}}%
\pgfpathlineto{\pgfqpoint{3.497084in}{0.643991in}}%
\pgfpathlineto{\pgfqpoint{3.433499in}{0.682469in}}%
\pgfusepath{stroke}%
\end{pgfscope}%
\begin{pgfscope}%
\pgfpathrectangle{\pgfqpoint{0.647939in}{0.492442in}}{\pgfqpoint{4.273799in}{2.331163in}}%
\pgfusepath{clip}%
\pgfsetbuttcap%
\pgfsetroundjoin%
\pgfsetlinewidth{0.301125pt}%
\definecolor{currentstroke}{rgb}{0.500000,0.500000,0.500000}%
\pgfsetstrokecolor{currentstroke}%
\pgfsetstrokeopacity{0.300000}%
\pgfsetdash{}{0pt}%
\pgfpathmoveto{\pgfqpoint{3.853289in}{0.492442in}}%
\pgfpathlineto{\pgfqpoint{3.853289in}{0.492442in}}%
\pgfpathlineto{\pgfqpoint{3.788636in}{0.530387in}}%
\pgfpathlineto{\pgfqpoint{3.723183in}{0.567923in}}%
\pgfpathlineto{\pgfqpoint{3.657318in}{0.605244in}}%
\pgfpathlineto{\pgfqpoint{3.591430in}{0.642553in}}%
\pgfpathlineto{\pgfqpoint{3.525895in}{0.680046in}}%
\pgfusepath{stroke}%
\end{pgfscope}%
\begin{pgfscope}%
\pgfpathrectangle{\pgfqpoint{0.647939in}{0.492442in}}{\pgfqpoint{4.273799in}{2.331163in}}%
\pgfusepath{clip}%
\pgfsetbuttcap%
\pgfsetroundjoin%
\pgfsetlinewidth{0.301125pt}%
\definecolor{currentstroke}{rgb}{0.500000,0.500000,0.500000}%
\pgfsetstrokecolor{currentstroke}%
\pgfsetstrokeopacity{0.300000}%
\pgfsetdash{}{0pt}%
\pgfpathmoveto{\pgfqpoint{3.950420in}{0.492442in}}%
\pgfpathlineto{\pgfqpoint{3.950420in}{0.492442in}}%
\pgfpathlineto{\pgfqpoint{3.886880in}{0.530939in}}%
\pgfpathlineto{\pgfqpoint{3.821905in}{0.568718in}}%
\pgfpathlineto{\pgfqpoint{3.755865in}{0.605946in}}%
\pgfpathlineto{\pgfqpoint{3.689170in}{0.642827in}}%
\pgfpathlineto{\pgfqpoint{3.622250in}{0.679585in}}%
\pgfusepath{stroke}%
\end{pgfscope}%
\begin{pgfscope}%
\pgfpathrectangle{\pgfqpoint{0.647939in}{0.492442in}}{\pgfqpoint{4.273799in}{2.331163in}}%
\pgfusepath{clip}%
\pgfsetbuttcap%
\pgfsetroundjoin%
\pgfsetlinewidth{0.301125pt}%
\definecolor{currentstroke}{rgb}{0.500000,0.500000,0.500000}%
\pgfsetstrokecolor{currentstroke}%
\pgfsetstrokeopacity{0.300000}%
\pgfsetdash{}{0pt}%
\pgfpathmoveto{\pgfqpoint{4.047552in}{0.492442in}}%
\pgfpathlineto{\pgfqpoint{4.047552in}{0.492442in}}%
\pgfpathlineto{\pgfqpoint{3.986155in}{0.531956in}}%
\pgfpathlineto{\pgfqpoint{3.922727in}{0.570506in}}%
\pgfpathlineto{\pgfqpoint{3.857562in}{0.608186in}}%
\pgfpathlineto{\pgfqpoint{3.791027in}{0.645149in}}%
\pgfpathlineto{\pgfqpoint{3.723544in}{0.681600in}}%
\pgfusepath{stroke}%
\end{pgfscope}%
\begin{pgfscope}%
\pgfpathrectangle{\pgfqpoint{0.647939in}{0.492442in}}{\pgfqpoint{4.273799in}{2.331163in}}%
\pgfusepath{clip}%
\pgfsetbuttcap%
\pgfsetroundjoin%
\pgfsetlinewidth{0.301125pt}%
\definecolor{currentstroke}{rgb}{0.500000,0.500000,0.500000}%
\pgfsetstrokecolor{currentstroke}%
\pgfsetstrokeopacity{0.300000}%
\pgfsetdash{}{0pt}%
\pgfpathmoveto{\pgfqpoint{4.144684in}{0.492442in}}%
\pgfpathlineto{\pgfqpoint{4.144684in}{0.492442in}}%
\pgfpathlineto{\pgfqpoint{4.086415in}{0.533340in}}%
\pgfpathlineto{\pgfqpoint{4.025679in}{0.573153in}}%
\pgfpathlineto{\pgfqpoint{3.962599in}{0.611870in}}%
\pgfpathlineto{\pgfqpoint{3.897435in}{0.649548in}}%
\pgfpathlineto{\pgfqpoint{3.830532in}{0.686311in}}%
\pgfusepath{stroke}%
\end{pgfscope}%
\begin{pgfscope}%
\pgfpathrectangle{\pgfqpoint{0.647939in}{0.492442in}}{\pgfqpoint{4.273799in}{2.331163in}}%
\pgfusepath{clip}%
\pgfsetbuttcap%
\pgfsetroundjoin%
\pgfsetlinewidth{0.301125pt}%
\definecolor{currentstroke}{rgb}{0.500000,0.500000,0.500000}%
\pgfsetstrokecolor{currentstroke}%
\pgfsetstrokeopacity{0.300000}%
\pgfsetdash{}{0pt}%
\pgfpathmoveto{\pgfqpoint{4.338948in}{0.492442in}}%
\pgfpathlineto{\pgfqpoint{4.338948in}{0.492442in}}%
\pgfpathlineto{\pgfqpoint{4.289375in}{0.536617in}}%
\pgfpathlineto{\pgfqpoint{4.237144in}{0.579870in}}%
\pgfpathlineto{\pgfqpoint{4.182071in}{0.622061in}}%
\pgfpathlineto{\pgfqpoint{4.124049in}{0.663057in}}%
\pgfpathlineto{\pgfqpoint{4.063021in}{0.702732in}}%
\pgfpathlineto{\pgfqpoint{3.999053in}{0.741006in}}%
\pgfpathlineto{\pgfqpoint{3.932364in}{0.777874in}}%
\pgfpathlineto{\pgfqpoint{3.863316in}{0.813432in}}%
\pgfpathlineto{\pgfqpoint{3.792416in}{0.847893in}}%
\pgfpathlineto{\pgfqpoint{3.720254in}{0.881570in}}%
\pgfpathlineto{\pgfqpoint{3.647494in}{0.914862in}}%
\pgfpathlineto{\pgfqpoint{3.574778in}{0.948183in}}%
\pgfpathlineto{\pgfqpoint{3.502711in}{0.981918in}}%
\pgfpathlineto{\pgfqpoint{3.431848in}{1.016398in}}%
\pgfpathlineto{\pgfqpoint{3.362637in}{1.051859in}}%
\pgfpathlineto{\pgfqpoint{3.295430in}{1.088446in}}%
\pgfpathlineto{\pgfqpoint{3.230488in}{1.126228in}}%
\pgfpathlineto{\pgfqpoint{3.167968in}{1.165206in}}%
\pgfpathlineto{\pgfqpoint{3.107954in}{1.205340in}}%
\pgfpathlineto{\pgfqpoint{3.050472in}{1.246564in}}%
\pgfpathlineto{\pgfqpoint{2.995507in}{1.288795in}}%
\pgfpathlineto{\pgfqpoint{2.943007in}{1.331951in}}%
\pgfpathlineto{\pgfqpoint{2.892909in}{1.375952in}}%
\pgfpathlineto{\pgfqpoint{2.845147in}{1.420721in}}%
\pgfpathlineto{\pgfqpoint{2.799659in}{1.466189in}}%
\pgfpathlineto{\pgfqpoint{2.756390in}{1.512296in}}%
\pgfpathlineto{\pgfqpoint{2.715302in}{1.558995in}}%
\pgfpathlineto{\pgfqpoint{2.676379in}{1.606242in}}%
\pgfpathlineto{\pgfqpoint{2.639625in}{1.654004in}}%
\pgfpathlineto{\pgfqpoint{2.605072in}{1.702251in}}%
\pgfpathlineto{\pgfqpoint{2.572795in}{1.750965in}}%
\pgfpathlineto{\pgfqpoint{2.542925in}{1.800132in}}%
\pgfpathlineto{\pgfqpoint{2.515650in}{1.849746in}}%
\pgfpathlineto{\pgfqpoint{2.491237in}{1.899800in}}%
\pgfpathlineto{\pgfqpoint{2.470091in}{1.950292in}}%
\pgfpathlineto{\pgfqpoint{2.452784in}{2.001214in}}%
\pgfpathlineto{\pgfqpoint{2.440164in}{2.052536in}}%
\pgfpathlineto{\pgfqpoint{2.433527in}{2.104173in}}%
\pgfpathlineto{\pgfqpoint{2.434934in}{2.155896in}}%
\pgfpathlineto{\pgfqpoint{2.447895in}{2.207055in}}%
\pgfpathlineto{\pgfqpoint{2.478769in}{2.255607in}}%
\pgfpathlineto{\pgfqpoint{2.478769in}{2.255607in}}%
\pgfpathlineto{\pgfqpoint{2.513562in}{2.283373in}}%
\pgfpathlineto{\pgfqpoint{2.513562in}{2.283373in}}%
\pgfpathlineto{\pgfqpoint{2.553783in}{2.300275in}}%
\pgfpathlineto{\pgfqpoint{2.602675in}{2.307543in}}%
\pgfpathlineto{\pgfqpoint{2.648012in}{2.304885in}}%
\pgfpathlineto{\pgfqpoint{2.691825in}{2.294805in}}%
\pgfpathlineto{\pgfqpoint{2.735861in}{2.277236in}}%
\pgfpathlineto{\pgfqpoint{2.779676in}{2.251351in}}%
\pgfpathlineto{\pgfqpoint{2.820746in}{2.216646in}}%
\pgfusepath{stroke}%
\end{pgfscope}%
\begin{pgfscope}%
\pgfpathrectangle{\pgfqpoint{0.647939in}{0.492442in}}{\pgfqpoint{4.273799in}{2.331163in}}%
\pgfusepath{clip}%
\pgfsetbuttcap%
\pgfsetroundjoin%
\pgfsetlinewidth{0.301125pt}%
\definecolor{currentstroke}{rgb}{0.500000,0.500000,0.500000}%
\pgfsetstrokecolor{currentstroke}%
\pgfsetstrokeopacity{0.300000}%
\pgfsetdash{}{0pt}%
\pgfpathmoveto{\pgfqpoint{4.436079in}{0.492442in}}%
\pgfpathlineto{\pgfqpoint{4.436079in}{0.492442in}}%
\pgfpathlineto{\pgfqpoint{4.391494in}{0.538174in}}%
\pgfpathlineto{\pgfqpoint{4.344555in}{0.583197in}}%
\pgfpathlineto{\pgfqpoint{4.295014in}{0.627381in}}%
\pgfpathlineto{\pgfqpoint{4.242613in}{0.670569in}}%
\pgfpathlineto{\pgfqpoint{4.187098in}{0.712584in}}%
\pgfpathlineto{\pgfqpoint{4.128301in}{0.753248in}}%
\pgfpathlineto{\pgfqpoint{4.066144in}{0.792393in}}%
\pgfpathlineto{\pgfqpoint{4.000657in}{0.829888in}}%
\pgfpathlineto{\pgfqpoint{3.932072in}{0.865701in}}%
\pgfpathlineto{\pgfqpoint{3.860820in}{0.899937in}}%
\pgfpathlineto{\pgfqpoint{3.787510in}{0.932860in}}%
\pgfpathlineto{\pgfqpoint{3.712833in}{0.964863in}}%
\pgfpathlineto{\pgfqpoint{3.637571in}{0.996457in}}%
\pgfpathlineto{\pgfqpoint{3.562492in}{1.028178in}}%
\pgfpathlineto{\pgfqpoint{3.488274in}{1.060491in}}%
\pgfusepath{stroke}%
\end{pgfscope}%
\begin{pgfscope}%
\pgfpathrectangle{\pgfqpoint{0.647939in}{0.492442in}}{\pgfqpoint{4.273799in}{2.331163in}}%
\pgfusepath{clip}%
\pgfsetbuttcap%
\pgfsetroundjoin%
\pgfsetlinewidth{0.301125pt}%
\definecolor{currentstroke}{rgb}{0.500000,0.500000,0.500000}%
\pgfsetstrokecolor{currentstroke}%
\pgfsetstrokeopacity{0.300000}%
\pgfsetdash{}{0pt}%
\pgfpathmoveto{\pgfqpoint{4.630343in}{0.492442in}}%
\pgfpathlineto{\pgfqpoint{4.630343in}{0.492442in}}%
\pgfpathlineto{\pgfqpoint{4.595556in}{0.540641in}}%
\pgfpathlineto{\pgfqpoint{4.559331in}{0.588522in}}%
\pgfpathlineto{\pgfqpoint{4.521481in}{0.636028in}}%
\pgfpathlineto{\pgfqpoint{4.481784in}{0.683082in}}%
\pgfpathlineto{\pgfqpoint{4.439986in}{0.729591in}}%
\pgfpathlineto{\pgfqpoint{4.395786in}{0.775433in}}%
\pgfpathlineto{\pgfqpoint{4.348833in}{0.820449in}}%
\pgfpathlineto{\pgfqpoint{4.298725in}{0.864435in}}%
\pgfpathlineto{\pgfqpoint{4.245009in}{0.907129in}}%
\pgfpathlineto{\pgfqpoint{4.187196in}{0.948198in}}%
\pgfpathlineto{\pgfqpoint{4.124900in}{0.987260in}}%
\pgfpathlineto{\pgfqpoint{4.057880in}{1.023912in}}%
\pgfpathlineto{\pgfqpoint{3.986184in}{1.057830in}}%
\pgfpathlineto{\pgfqpoint{3.910301in}{1.088924in}}%
\pgfpathlineto{\pgfqpoint{3.831051in}{1.117430in}}%
\pgfpathlineto{\pgfqpoint{3.749530in}{1.143986in}}%
\pgfpathlineto{\pgfqpoint{3.666906in}{1.169523in}}%
\pgfpathlineto{\pgfqpoint{3.584294in}{1.195071in}}%
\pgfpathlineto{\pgfqpoint{3.502710in}{1.221573in}}%
\pgfpathlineto{\pgfqpoint{3.423044in}{1.249737in}}%
\pgfpathlineto{\pgfqpoint{3.346020in}{1.279988in}}%
\pgfpathlineto{\pgfqpoint{3.272159in}{1.312494in}}%
\pgfpathlineto{\pgfqpoint{3.201793in}{1.347237in}}%
\pgfpathlineto{\pgfqpoint{3.135108in}{1.384076in}}%
\pgfpathlineto{\pgfqpoint{3.072122in}{1.422809in}}%
\pgfpathlineto{\pgfqpoint{3.012765in}{1.463216in}}%
\pgfpathlineto{\pgfqpoint{2.956920in}{1.505088in}}%
\pgfpathlineto{\pgfqpoint{2.904445in}{1.548241in}}%
\pgfpathlineto{\pgfqpoint{2.855203in}{1.592518in}}%
\pgfpathlineto{\pgfqpoint{2.809082in}{1.637790in}}%
\pgfpathlineto{\pgfqpoint{2.766002in}{1.683946in}}%
\pgfpathlineto{\pgfqpoint{2.725929in}{1.730901in}}%
\pgfpathlineto{\pgfqpoint{2.688879in}{1.778587in}}%
\pgfpathlineto{\pgfqpoint{2.654940in}{1.826956in}}%
\pgfpathlineto{\pgfqpoint{2.624298in}{1.875975in}}%
\pgfpathlineto{\pgfqpoint{2.597264in}{1.925621in}}%
\pgfpathlineto{\pgfqpoint{2.574319in}{1.975872in}}%
\pgfpathlineto{\pgfqpoint{2.556239in}{2.026703in}}%
\pgfpathlineto{\pgfqpoint{2.544284in}{2.078053in}}%
\pgfpathlineto{\pgfqpoint{2.540625in}{2.129736in}}%
\pgfpathlineto{\pgfqpoint{2.549427in}{2.181117in}}%
\pgfpathlineto{\pgfqpoint{2.549427in}{2.181117in}}%
\pgfpathlineto{\pgfqpoint{2.570844in}{2.220124in}}%
\pgfpathlineto{\pgfqpoint{2.570844in}{2.220124in}}%
\pgfpathlineto{\pgfqpoint{2.599551in}{2.244701in}}%
\pgfpathlineto{\pgfqpoint{2.599551in}{2.244701in}}%
\pgfpathlineto{\pgfqpoint{2.633411in}{2.258187in}}%
\pgfpathlineto{\pgfqpoint{2.675132in}{2.261760in}}%
\pgfpathlineto{\pgfqpoint{2.711597in}{2.256504in}}%
\pgfusepath{stroke}%
\end{pgfscope}%
\begin{pgfscope}%
\pgfpathrectangle{\pgfqpoint{0.647939in}{0.492442in}}{\pgfqpoint{4.273799in}{2.331163in}}%
\pgfusepath{clip}%
\pgfsetbuttcap%
\pgfsetroundjoin%
\pgfsetlinewidth{0.301125pt}%
\definecolor{currentstroke}{rgb}{0.500000,0.500000,0.500000}%
\pgfsetstrokecolor{currentstroke}%
\pgfsetstrokeopacity{0.300000}%
\pgfsetdash{}{0pt}%
\pgfpathmoveto{\pgfqpoint{4.727475in}{0.492442in}}%
\pgfpathlineto{\pgfqpoint{4.727475in}{0.492442in}}%
\pgfpathlineto{\pgfqpoint{4.697069in}{0.541516in}}%
\pgfpathlineto{\pgfqpoint{4.665650in}{0.590400in}}%
\pgfpathlineto{\pgfqpoint{4.633105in}{0.639065in}}%
\pgfpathlineto{\pgfqpoint{4.599301in}{0.687470in}}%
\pgfpathlineto{\pgfqpoint{4.564060in}{0.735570in}}%
\pgfpathlineto{\pgfqpoint{4.527172in}{0.783301in}}%
\pgfpathlineto{\pgfqpoint{4.488385in}{0.830582in}}%
\pgfpathlineto{\pgfqpoint{4.447399in}{0.877307in}}%
\pgfpathlineto{\pgfqpoint{4.403839in}{0.923332in}}%
\pgfpathlineto{\pgfqpoint{4.357245in}{0.968461in}}%
\pgfpathlineto{\pgfqpoint{4.307074in}{1.012425in}}%
\pgfpathlineto{\pgfqpoint{4.252674in}{1.054850in}}%
\pgfpathlineto{\pgfqpoint{4.193295in}{1.095225in}}%
\pgfpathlineto{\pgfqpoint{4.128243in}{1.132892in}}%
\pgfpathlineto{\pgfqpoint{4.057099in}{1.167105in}}%
\pgfpathlineto{\pgfqpoint{3.980012in}{1.197243in}}%
\pgfpathlineto{\pgfqpoint{3.897920in}{1.223168in}}%
\pgfpathlineto{\pgfqpoint{3.812271in}{1.245474in}}%
\pgfpathlineto{\pgfqpoint{3.724659in}{1.265445in}}%
\pgfpathlineto{\pgfqpoint{3.636507in}{1.284705in}}%
\pgfpathlineto{\pgfqpoint{3.549024in}{1.304823in}}%
\pgfpathlineto{\pgfqpoint{3.463293in}{1.327032in}}%
\pgfpathlineto{\pgfqpoint{3.380297in}{1.352108in}}%
\pgfpathlineto{\pgfqpoint{3.300841in}{1.380369in}}%
\pgfpathlineto{\pgfqpoint{3.225469in}{1.411783in}}%
\pgfpathlineto{\pgfqpoint{3.154445in}{1.446104in}}%
\pgfpathlineto{\pgfqpoint{3.087860in}{1.482981in}}%
\pgfusepath{stroke}%
\end{pgfscope}%
\begin{pgfscope}%
\pgfpathrectangle{\pgfqpoint{0.647939in}{0.492442in}}{\pgfqpoint{4.273799in}{2.331163in}}%
\pgfusepath{clip}%
\pgfsetbuttcap%
\pgfsetroundjoin%
\pgfsetlinewidth{0.301125pt}%
\definecolor{currentstroke}{rgb}{0.500000,0.500000,0.500000}%
\pgfsetstrokecolor{currentstroke}%
\pgfsetstrokeopacity{0.300000}%
\pgfsetdash{}{0pt}%
\pgfpathmoveto{\pgfqpoint{4.824607in}{0.492442in}}%
\pgfpathlineto{\pgfqpoint{4.824607in}{0.492442in}}%
\pgfpathlineto{\pgfqpoint{4.798103in}{0.542186in}}%
\pgfpathlineto{\pgfqpoint{4.770946in}{0.591825in}}%
\pgfpathlineto{\pgfqpoint{4.743067in}{0.641344in}}%
\pgfpathlineto{\pgfqpoint{4.714388in}{0.690727in}}%
\pgfpathlineto{\pgfqpoint{4.684822in}{0.739953in}}%
\pgfpathlineto{\pgfqpoint{4.654257in}{0.788997in}}%
\pgfpathlineto{\pgfqpoint{4.622553in}{0.837826in}}%
\pgfpathlineto{\pgfqpoint{4.589559in}{0.886399in}}%
\pgfpathlineto{\pgfqpoint{4.555080in}{0.934661in}}%
\pgfpathlineto{\pgfqpoint{4.518858in}{0.982540in}}%
\pgfpathlineto{\pgfqpoint{4.480570in}{1.029938in}}%
\pgfpathlineto{\pgfqpoint{4.439801in}{1.076714in}}%
\pgfpathlineto{\pgfqpoint{4.396015in}{1.122667in}}%
\pgfpathlineto{\pgfqpoint{4.348510in}{1.167497in}}%
\pgfpathlineto{\pgfqpoint{4.296362in}{1.210744in}}%
\pgfpathlineto{\pgfqpoint{4.238366in}{1.251686in}}%
\pgfpathlineto{\pgfqpoint{4.173145in}{1.289209in}}%
\pgfpathlineto{\pgfqpoint{4.099589in}{1.321743in}}%
\pgfpathlineto{\pgfqpoint{4.017716in}{1.347632in}}%
\pgfpathlineto{\pgfqpoint{3.929379in}{1.366212in}}%
\pgfpathlineto{\pgfqpoint{3.837360in}{1.378708in}}%
\pgfpathlineto{\pgfqpoint{3.743891in}{1.387770in}}%
\pgfpathlineto{\pgfqpoint{3.650231in}{1.396273in}}%
\pgfpathlineto{\pgfqpoint{3.557275in}{1.406664in}}%
\pgfpathlineto{\pgfqpoint{3.466018in}{1.420718in}}%
\pgfpathlineto{\pgfqpoint{3.377674in}{1.439441in}}%
\pgfpathlineto{\pgfqpoint{3.293420in}{1.463102in}}%
\pgfpathlineto{\pgfqpoint{3.214117in}{1.491401in}}%
\pgfpathlineto{\pgfqpoint{3.140158in}{1.523742in}}%
\pgfpathlineto{\pgfqpoint{3.071551in}{1.559466in}}%
\pgfpathlineto{\pgfqpoint{3.008131in}{1.597950in}}%
\pgfpathlineto{\pgfqpoint{2.949598in}{1.638684in}}%
\pgfpathlineto{\pgfqpoint{2.895640in}{1.681274in}}%
\pgfpathlineto{\pgfqpoint{2.845998in}{1.725408in}}%
\pgfpathlineto{\pgfqpoint{2.800489in}{1.770851in}}%
\pgfpathlineto{\pgfqpoint{2.759022in}{1.817433in}}%
\pgfpathlineto{\pgfqpoint{2.721624in}{1.865032in}}%
\pgfpathlineto{\pgfqpoint{2.688464in}{1.913553in}}%
\pgfpathlineto{\pgfqpoint{2.659921in}{1.962933in}}%
\pgfusepath{stroke}%
\end{pgfscope}%
\begin{pgfscope}%
\pgfpathrectangle{\pgfqpoint{0.647939in}{0.492442in}}{\pgfqpoint{4.273799in}{2.331163in}}%
\pgfusepath{clip}%
\pgfsetbuttcap%
\pgfsetroundjoin%
\pgfsetlinewidth{0.301125pt}%
\definecolor{currentstroke}{rgb}{0.500000,0.500000,0.500000}%
\pgfsetstrokecolor{currentstroke}%
\pgfsetstrokeopacity{0.300000}%
\pgfsetdash{}{0pt}%
\pgfpathmoveto{\pgfqpoint{4.921738in}{0.492442in}}%
\pgfpathlineto{\pgfqpoint{4.921738in}{0.492442in}}%
\pgfpathlineto{\pgfqpoint{4.898649in}{0.542691in}}%
\pgfpathlineto{\pgfqpoint{4.875172in}{0.592885in}}%
\pgfpathlineto{\pgfqpoint{4.851275in}{0.643021in}}%
\pgfpathlineto{\pgfqpoint{4.826925in}{0.693091in}}%
\pgfpathlineto{\pgfqpoint{4.802080in}{0.743089in}}%
\pgfpathlineto{\pgfqpoint{4.776696in}{0.793006in}}%
\pgfpathlineto{\pgfqpoint{4.750711in}{0.842831in}}%
\pgfpathlineto{\pgfqpoint{4.724064in}{0.892552in}}%
\pgfpathlineto{\pgfqpoint{4.696680in}{0.942153in}}%
\pgfpathlineto{\pgfqpoint{4.668457in}{0.991613in}}%
\pgfpathlineto{\pgfqpoint{4.639282in}{1.040908in}}%
\pgfpathlineto{\pgfqpoint{4.609008in}{1.090004in}}%
\pgfpathlineto{\pgfqpoint{4.577434in}{1.138858in}}%
\pgfpathlineto{\pgfqpoint{4.544326in}{1.187406in}}%
\pgfpathlineto{\pgfqpoint{4.509370in}{1.235562in}}%
\pgfpathlineto{\pgfqpoint{4.472112in}{1.283196in}}%
\pgfpathlineto{\pgfqpoint{4.431915in}{1.330108in}}%
\pgfpathlineto{\pgfqpoint{4.387858in}{1.375964in}}%
\pgfpathlineto{\pgfqpoint{4.338552in}{1.420170in}}%
\pgfpathlineto{\pgfqpoint{4.281850in}{1.461585in}}%
\pgfpathlineto{\pgfqpoint{4.214687in}{1.497871in}}%
\pgfpathlineto{\pgfqpoint{4.134475in}{1.524619in}}%
\pgfpathlineto{\pgfqpoint{4.056933in}{1.536354in}}%
\pgfpathlineto{\pgfqpoint{3.982889in}{1.538095in}}%
\pgfpathlineto{\pgfqpoint{3.900214in}{1.533147in}}%
\pgfpathlineto{\pgfqpoint{3.806994in}{1.523490in}}%
\pgfpathlineto{\pgfqpoint{3.713834in}{1.513600in}}%
\pgfpathlineto{\pgfqpoint{3.619863in}{1.506612in}}%
\pgfpathlineto{\pgfqpoint{3.525194in}{1.504952in}}%
\pgfpathlineto{\pgfqpoint{3.431097in}{1.510290in}}%
\pgfpathlineto{\pgfqpoint{3.339650in}{1.523333in}}%
\pgfpathlineto{\pgfqpoint{3.252838in}{1.543777in}}%
\pgfpathlineto{\pgfqpoint{3.171998in}{1.570601in}}%
\pgfpathlineto{\pgfqpoint{3.097593in}{1.602564in}}%
\pgfpathlineto{\pgfqpoint{3.029491in}{1.638539in}}%
\pgfusepath{stroke}%
\end{pgfscope}%
\begin{pgfscope}%
\pgfpathrectangle{\pgfqpoint{0.647939in}{0.492442in}}{\pgfqpoint{4.273799in}{2.331163in}}%
\pgfusepath{clip}%
\pgfsetbuttcap%
\pgfsetroundjoin%
\pgfsetlinewidth{0.301125pt}%
\definecolor{currentstroke}{rgb}{0.500000,0.500000,0.500000}%
\pgfsetstrokecolor{currentstroke}%
\pgfsetstrokeopacity{0.300000}%
\pgfsetdash{}{0pt}%
\pgfpathmoveto{\pgfqpoint{4.921738in}{0.704366in}}%
\pgfpathlineto{\pgfqpoint{4.921738in}{0.704366in}}%
\pgfpathlineto{\pgfqpoint{4.900337in}{0.754836in}}%
\pgfpathlineto{\pgfqpoint{4.878665in}{0.805272in}}%
\pgfpathlineto{\pgfqpoint{4.856706in}{0.855671in}}%
\pgfpathlineto{\pgfqpoint{4.834438in}{0.906030in}}%
\pgfpathlineto{\pgfqpoint{4.811839in}{0.956344in}}%
\pgfpathlineto{\pgfqpoint{4.788882in}{1.006610in}}%
\pgfpathlineto{\pgfqpoint{4.765532in}{1.056822in}}%
\pgfpathlineto{\pgfqpoint{4.741753in}{1.106974in}}%
\pgfpathlineto{\pgfqpoint{4.717496in}{1.157057in}}%
\pgfpathlineto{\pgfqpoint{4.692706in}{1.207061in}}%
\pgfpathlineto{\pgfqpoint{4.667307in}{1.256975in}}%
\pgfpathlineto{\pgfqpoint{4.641213in}{1.306782in}}%
\pgfpathlineto{\pgfqpoint{4.614304in}{1.356458in}}%
\pgfpathlineto{\pgfqpoint{4.586424in}{1.405975in}}%
\pgfpathlineto{\pgfqpoint{4.557368in}{1.455286in}}%
\pgfpathlineto{\pgfqpoint{4.526822in}{1.504328in}}%
\pgfpathlineto{\pgfqpoint{4.494325in}{1.552994in}}%
\pgfpathlineto{\pgfqpoint{4.459176in}{1.601098in}}%
\pgfpathlineto{\pgfqpoint{4.420153in}{1.648277in}}%
\pgfpathlineto{\pgfqpoint{4.374885in}{1.693696in}}%
\pgfpathlineto{\pgfqpoint{4.318169in}{1.734848in}}%
\pgfpathlineto{\pgfqpoint{4.318169in}{1.734848in}}%
\pgfpathlineto{\pgfqpoint{4.270890in}{1.755211in}}%
\pgfpathlineto{\pgfqpoint{4.270890in}{1.755211in}}%
\pgfpathlineto{\pgfqpoint{4.224664in}{1.763624in}}%
\pgfpathlineto{\pgfqpoint{4.176012in}{1.762174in}}%
\pgfpathlineto{\pgfqpoint{4.129833in}{1.753390in}}%
\pgfpathlineto{\pgfqpoint{4.076294in}{1.737047in}}%
\pgfpathlineto{\pgfqpoint{4.006236in}{1.710188in}}%
\pgfpathlineto{\pgfqpoint{3.930685in}{1.678903in}}%
\pgfpathlineto{\pgfqpoint{3.854173in}{1.648286in}}%
\pgfpathlineto{\pgfqpoint{3.774802in}{1.619945in}}%
\pgfpathlineto{\pgfqpoint{3.691380in}{1.595401in}}%
\pgfpathlineto{\pgfqpoint{3.603334in}{1.576419in}}%
\pgfpathlineto{\pgfqpoint{3.511181in}{1.564996in}}%
\pgfpathlineto{\pgfqpoint{3.416920in}{1.562895in}}%
\pgfpathlineto{\pgfqpoint{3.326745in}{1.570461in}}%
\pgfpathlineto{\pgfqpoint{3.242038in}{1.586604in}}%
\pgfpathlineto{\pgfqpoint{3.160439in}{1.610971in}}%
\pgfusepath{stroke}%
\end{pgfscope}%
\begin{pgfscope}%
\pgfpathrectangle{\pgfqpoint{0.647939in}{0.492442in}}{\pgfqpoint{4.273799in}{2.331163in}}%
\pgfusepath{clip}%
\pgfsetbuttcap%
\pgfsetroundjoin%
\pgfsetlinewidth{0.301125pt}%
\definecolor{currentstroke}{rgb}{0.500000,0.500000,0.500000}%
\pgfsetstrokecolor{currentstroke}%
\pgfsetstrokeopacity{0.300000}%
\pgfsetdash{}{0pt}%
\pgfpathmoveto{\pgfqpoint{4.921738in}{0.916290in}}%
\pgfpathlineto{\pgfqpoint{4.921738in}{0.916290in}}%
\pgfpathlineto{\pgfqpoint{4.902282in}{0.966994in}}%
\pgfpathlineto{\pgfqpoint{4.882709in}{1.017685in}}%
\pgfpathlineto{\pgfqpoint{4.863017in}{1.068361in}}%
\pgfpathlineto{\pgfqpoint{4.843207in}{1.119024in}}%
\pgfpathlineto{\pgfqpoint{4.823279in}{1.169673in}}%
\pgfpathlineto{\pgfqpoint{4.803233in}{1.220308in}}%
\pgfpathlineto{\pgfqpoint{4.783071in}{1.270929in}}%
\pgfpathlineto{\pgfqpoint{4.762794in}{1.321537in}}%
\pgfpathlineto{\pgfqpoint{4.742405in}{1.372131in}}%
\pgfpathlineto{\pgfqpoint{4.721909in}{1.422711in}}%
\pgfpathlineto{\pgfqpoint{4.701311in}{1.473280in}}%
\pgfpathlineto{\pgfqpoint{4.680622in}{1.523837in}}%
\pgfpathlineto{\pgfqpoint{4.659851in}{1.574384in}}%
\pgfpathlineto{\pgfqpoint{4.639017in}{1.624923in}}%
\pgfpathlineto{\pgfqpoint{4.618143in}{1.675457in}}%
\pgfpathlineto{\pgfqpoint{4.597266in}{1.725989in}}%
\pgfpathlineto{\pgfqpoint{4.576436in}{1.776528in}}%
\pgfpathlineto{\pgfqpoint{4.555738in}{1.827080in}}%
\pgfpathlineto{\pgfqpoint{4.535303in}{1.877664in}}%
\pgfpathlineto{\pgfqpoint{4.515358in}{1.928302in}}%
\pgfpathlineto{\pgfqpoint{4.496333in}{1.979041in}}%
\pgfpathlineto{\pgfqpoint{4.479108in}{2.029961in}}%
\pgfpathlineto{\pgfqpoint{4.465851in}{2.081203in}}%
\pgfpathlineto{\pgfqpoint{4.462311in}{2.132793in}}%
\pgfpathlineto{\pgfqpoint{4.477042in}{2.183184in}}%
\pgfpathlineto{\pgfqpoint{4.500530in}{2.226214in}}%
\pgfpathlineto{\pgfqpoint{4.533219in}{2.274588in}}%
\pgfpathlineto{\pgfqpoint{4.567904in}{2.322682in}}%
\pgfpathlineto{\pgfqpoint{4.602829in}{2.370666in}}%
\pgfpathlineto{\pgfqpoint{4.637498in}{2.418750in}}%
\pgfpathlineto{\pgfqpoint{4.671710in}{2.466989in}}%
\pgfpathlineto{\pgfqpoint{4.705388in}{2.515396in}}%
\pgfpathlineto{\pgfqpoint{4.738429in}{2.563938in}}%
\pgfpathlineto{\pgfqpoint{4.770807in}{2.612592in}}%
\pgfpathlineto{\pgfqpoint{4.802588in}{2.661373in}}%
\pgfpathlineto{\pgfqpoint{4.833817in}{2.710280in}}%
\pgfpathlineto{\pgfqpoint{4.864478in}{2.759293in}}%
\pgfpathlineto{\pgfqpoint{4.894583in}{2.808400in}}%
\pgfpathlineto{\pgfqpoint{4.903820in}{2.823605in}}%
\pgfusepath{stroke}%
\end{pgfscope}%
\begin{pgfscope}%
\pgfpathrectangle{\pgfqpoint{0.647939in}{0.492442in}}{\pgfqpoint{4.273799in}{2.331163in}}%
\pgfusepath{clip}%
\pgfsetbuttcap%
\pgfsetroundjoin%
\pgfsetlinewidth{0.301125pt}%
\definecolor{currentstroke}{rgb}{0.500000,0.500000,0.500000}%
\pgfsetstrokecolor{currentstroke}%
\pgfsetstrokeopacity{0.300000}%
\pgfsetdash{}{0pt}%
\pgfpathmoveto{\pgfqpoint{4.921738in}{1.181195in}}%
\pgfpathlineto{\pgfqpoint{4.921738in}{1.181195in}}%
\pgfpathlineto{\pgfqpoint{4.905167in}{1.232203in}}%
\pgfpathlineto{\pgfqpoint{4.888726in}{1.283224in}}%
\pgfpathlineto{\pgfqpoint{4.872438in}{1.334259in}}%
\pgfpathlineto{\pgfqpoint{4.856340in}{1.385312in}}%
\pgfpathlineto{\pgfqpoint{4.840470in}{1.436386in}}%
\pgfpathlineto{\pgfqpoint{4.824876in}{1.487485in}}%
\pgfpathlineto{\pgfqpoint{4.809616in}{1.538614in}}%
\pgfpathlineto{\pgfqpoint{4.794753in}{1.589779in}}%
\pgfpathlineto{\pgfqpoint{4.780369in}{1.640983in}}%
\pgfpathlineto{\pgfqpoint{4.766565in}{1.692235in}}%
\pgfpathlineto{\pgfqpoint{4.753453in}{1.743541in}}%
\pgfpathlineto{\pgfqpoint{4.741173in}{1.794908in}}%
\pgfpathlineto{\pgfqpoint{4.729902in}{1.846344in}}%
\pgfpathlineto{\pgfqpoint{4.719844in}{1.897853in}}%
\pgfpathlineto{\pgfqpoint{4.711238in}{1.949441in}}%
\pgfpathlineto{\pgfqpoint{4.704372in}{2.001106in}}%
\pgfpathlineto{\pgfqpoint{4.699575in}{2.052839in}}%
\pgfpathlineto{\pgfqpoint{4.697200in}{2.104621in}}%
\pgfpathlineto{\pgfqpoint{4.697598in}{2.156416in}}%
\pgfpathlineto{\pgfqpoint{4.701076in}{2.208176in}}%
\pgfpathlineto{\pgfqpoint{4.707845in}{2.259839in}}%
\pgfpathlineto{\pgfqpoint{4.717947in}{2.311338in}}%
\pgfpathlineto{\pgfqpoint{4.731250in}{2.362615in}}%
\pgfpathlineto{\pgfqpoint{4.747477in}{2.413643in}}%
\pgfpathlineto{\pgfqpoint{4.766217in}{2.464415in}}%
\pgfpathlineto{\pgfqpoint{4.787017in}{2.514947in}}%
\pgfpathlineto{\pgfqpoint{4.809441in}{2.565273in}}%
\pgfpathlineto{\pgfqpoint{4.833103in}{2.615431in}}%
\pgfusepath{stroke}%
\end{pgfscope}%
\begin{pgfscope}%
\pgfpathrectangle{\pgfqpoint{0.647939in}{0.492442in}}{\pgfqpoint{4.273799in}{2.331163in}}%
\pgfusepath{clip}%
\pgfsetbuttcap%
\pgfsetroundjoin%
\pgfsetlinewidth{0.301125pt}%
\definecolor{currentstroke}{rgb}{0.500000,0.500000,0.500000}%
\pgfsetstrokecolor{currentstroke}%
\pgfsetstrokeopacity{0.300000}%
\pgfsetdash{}{0pt}%
\pgfpathmoveto{\pgfqpoint{4.921738in}{1.499081in}}%
\pgfpathlineto{\pgfqpoint{4.921738in}{1.499081in}}%
\pgfpathlineto{\pgfqpoint{4.909507in}{1.550452in}}%
\pgfpathlineto{\pgfqpoint{4.897793in}{1.601860in}}%
\pgfpathlineto{\pgfqpoint{4.886663in}{1.653305in}}%
\pgfpathlineto{\pgfqpoint{4.876193in}{1.704792in}}%
\pgfpathlineto{\pgfqpoint{4.866478in}{1.756323in}}%
\pgfpathlineto{\pgfqpoint{4.857619in}{1.807900in}}%
\pgfpathlineto{\pgfqpoint{4.849726in}{1.859523in}}%
\pgfpathlineto{\pgfqpoint{4.842920in}{1.911192in}}%
\pgfpathlineto{\pgfqpoint{4.837332in}{1.962904in}}%
\pgfpathlineto{\pgfqpoint{4.833106in}{2.014655in}}%
\pgfpathlineto{\pgfqpoint{4.830388in}{2.066435in}}%
\pgfpathlineto{\pgfqpoint{4.829320in}{2.118233in}}%
\pgfpathlineto{\pgfqpoint{4.830036in}{2.170033in}}%
\pgfpathlineto{\pgfqpoint{4.832652in}{2.221813in}}%
\pgfpathlineto{\pgfqpoint{4.837253in}{2.273552in}}%
\pgfpathlineto{\pgfqpoint{4.843884in}{2.325226in}}%
\pgfpathlineto{\pgfqpoint{4.852540in}{2.376809in}}%
\pgfpathlineto{\pgfqpoint{4.863172in}{2.428282in}}%
\pgfpathlineto{\pgfqpoint{4.875690in}{2.479629in}}%
\pgfpathlineto{\pgfqpoint{4.889954in}{2.530840in}}%
\pgfpathlineto{\pgfqpoint{4.905790in}{2.581912in}}%
\pgfpathlineto{\pgfqpoint{4.921738in}{2.631046in}}%
\pgfusepath{stroke}%
\end{pgfscope}%
\begin{pgfscope}%
\pgfpathrectangle{\pgfqpoint{0.647939in}{0.492442in}}{\pgfqpoint{4.273799in}{2.331163in}}%
\pgfusepath{clip}%
\pgfsetbuttcap%
\pgfsetroundjoin%
\pgfsetlinewidth{0.301125pt}%
\definecolor{currentstroke}{rgb}{0.500000,0.500000,0.500000}%
\pgfsetstrokecolor{currentstroke}%
\pgfsetstrokeopacity{0.300000}%
\pgfsetdash{}{0pt}%
\pgfpathmoveto{\pgfqpoint{4.921738in}{1.763986in}}%
\pgfpathlineto{\pgfqpoint{4.921738in}{1.763986in}}%
\pgfpathlineto{\pgfqpoint{4.914116in}{1.815622in}}%
\pgfpathlineto{\pgfqpoint{4.907406in}{1.867295in}}%
\pgfpathlineto{\pgfqpoint{4.901701in}{1.919003in}}%
\pgfpathlineto{\pgfqpoint{4.897101in}{1.970745in}}%
\pgfpathlineto{\pgfqpoint{4.893708in}{2.022514in}}%
\pgfpathlineto{\pgfqpoint{4.891624in}{2.074304in}}%
\pgfpathlineto{\pgfqpoint{4.890950in}{2.126104in}}%
\pgfpathlineto{\pgfqpoint{4.891776in}{2.177904in}}%
\pgfpathlineto{\pgfqpoint{4.894180in}{2.229689in}}%
\pgfpathlineto{\pgfqpoint{4.898222in}{2.281443in}}%
\pgfpathlineto{\pgfqpoint{4.903936in}{2.333150in}}%
\pgfpathlineto{\pgfqpoint{4.911332in}{2.384793in}}%
\pgfpathlineto{\pgfqpoint{4.920389in}{2.436358in}}%
\pgfpathlineto{\pgfqpoint{4.921738in}{2.443408in}}%
\pgfusepath{stroke}%
\end{pgfscope}%
\begin{pgfscope}%
\pgfpathrectangle{\pgfqpoint{0.647939in}{0.492442in}}{\pgfqpoint{4.273799in}{2.331163in}}%
\pgfusepath{clip}%
\pgfsetbuttcap%
\pgfsetroundjoin%
\pgfsetlinewidth{0.301125pt}%
\definecolor{currentstroke}{rgb}{0.500000,0.500000,0.500000}%
\pgfsetstrokecolor{currentstroke}%
\pgfsetstrokeopacity{0.300000}%
\pgfsetdash{}{0pt}%
\pgfpathmoveto{\pgfqpoint{4.436079in}{2.823605in}}%
\pgfpathlineto{\pgfqpoint{4.436079in}{2.823605in}}%
\pgfpathlineto{\pgfqpoint{4.491934in}{2.781887in}}%
\pgfpathlineto{\pgfqpoint{4.491934in}{2.781887in}}%
\pgfpathlineto{\pgfqpoint{4.544622in}{2.755583in}}%
\pgfpathlineto{\pgfqpoint{4.544622in}{2.755583in}}%
\pgfpathlineto{\pgfqpoint{4.589386in}{2.744546in}}%
\pgfpathlineto{\pgfqpoint{4.589386in}{2.744546in}}%
\pgfpathlineto{\pgfqpoint{4.631537in}{2.744202in}}%
\pgfpathlineto{\pgfqpoint{4.671088in}{2.752685in}}%
\pgfpathlineto{\pgfqpoint{4.708108in}{2.768047in}}%
\pgfpathlineto{\pgfqpoint{4.747866in}{2.792046in}}%
\pgfpathlineto{\pgfqpoint{4.793384in}{2.823605in}}%
\pgfusepath{stroke}%
\end{pgfscope}%
\begin{pgfscope}%
\pgfpathrectangle{\pgfqpoint{0.647939in}{0.492442in}}{\pgfqpoint{4.273799in}{2.331163in}}%
\pgfusepath{clip}%
\pgfsetbuttcap%
\pgfsetroundjoin%
\pgfsetlinewidth{0.301125pt}%
\definecolor{currentstroke}{rgb}{0.500000,0.500000,0.500000}%
\pgfsetstrokecolor{currentstroke}%
\pgfsetstrokeopacity{0.300000}%
\pgfsetdash{}{0pt}%
\pgfpathmoveto{\pgfqpoint{4.338948in}{2.823605in}}%
\pgfpathlineto{\pgfqpoint{4.338948in}{2.823605in}}%
\pgfpathlineto{\pgfqpoint{4.380557in}{2.777068in}}%
\pgfpathlineto{\pgfqpoint{4.427127in}{2.732001in}}%
\pgfpathlineto{\pgfqpoint{4.482900in}{2.690363in}}%
\pgfpathlineto{\pgfqpoint{4.482900in}{2.690363in}}%
\pgfpathlineto{\pgfqpoint{4.530669in}{2.667510in}}%
\pgfpathlineto{\pgfqpoint{4.530669in}{2.667510in}}%
\pgfpathlineto{\pgfqpoint{4.572450in}{2.658501in}}%
\pgfpathlineto{\pgfqpoint{4.618075in}{2.660959in}}%
\pgfpathlineto{\pgfqpoint{4.653565in}{2.671383in}}%
\pgfpathlineto{\pgfqpoint{4.688675in}{2.688707in}}%
\pgfpathlineto{\pgfqpoint{4.728261in}{2.715856in}}%
\pgfusepath{stroke}%
\end{pgfscope}%
\begin{pgfscope}%
\pgfpathrectangle{\pgfqpoint{0.647939in}{0.492442in}}{\pgfqpoint{4.273799in}{2.331163in}}%
\pgfusepath{clip}%
\pgfsetbuttcap%
\pgfsetroundjoin%
\pgfsetlinewidth{0.301125pt}%
\definecolor{currentstroke}{rgb}{0.500000,0.500000,0.500000}%
\pgfsetstrokecolor{currentstroke}%
\pgfsetstrokeopacity{0.300000}%
\pgfsetdash{}{0pt}%
\pgfpathmoveto{\pgfqpoint{4.241816in}{2.823605in}}%
\pgfpathlineto{\pgfqpoint{4.241816in}{2.823605in}}%
\pgfpathlineto{\pgfqpoint{4.276091in}{2.775296in}}%
\pgfpathlineto{\pgfqpoint{4.311878in}{2.727317in}}%
\pgfpathlineto{\pgfqpoint{4.350020in}{2.679891in}}%
\pgfpathlineto{\pgfqpoint{4.392060in}{2.633485in}}%
\pgfpathlineto{\pgfqpoint{4.441394in}{2.589380in}}%
\pgfpathlineto{\pgfqpoint{4.441394in}{2.589380in}}%
\pgfpathlineto{\pgfqpoint{4.490196in}{2.559673in}}%
\pgfpathlineto{\pgfqpoint{4.490196in}{2.559673in}}%
\pgfpathlineto{\pgfqpoint{4.528881in}{2.547708in}}%
\pgfpathlineto{\pgfqpoint{4.528881in}{2.547708in}}%
\pgfpathlineto{\pgfqpoint{4.565070in}{2.546507in}}%
\pgfpathlineto{\pgfqpoint{4.598824in}{2.553878in}}%
\pgfpathlineto{\pgfqpoint{4.630566in}{2.567781in}}%
\pgfpathlineto{\pgfqpoint{4.665989in}{2.590453in}}%
\pgfusepath{stroke}%
\end{pgfscope}%
\begin{pgfscope}%
\pgfpathrectangle{\pgfqpoint{0.647939in}{0.492442in}}{\pgfqpoint{4.273799in}{2.331163in}}%
\pgfusepath{clip}%
\pgfsetbuttcap%
\pgfsetroundjoin%
\pgfsetlinewidth{0.301125pt}%
\definecolor{currentstroke}{rgb}{0.500000,0.500000,0.500000}%
\pgfsetstrokecolor{currentstroke}%
\pgfsetstrokeopacity{0.300000}%
\pgfsetdash{}{0pt}%
\pgfpathmoveto{\pgfqpoint{4.144684in}{2.823605in}}%
\pgfpathlineto{\pgfqpoint{4.144684in}{2.823605in}}%
\pgfpathlineto{\pgfqpoint{4.174785in}{2.774473in}}%
\pgfpathlineto{\pgfqpoint{4.205176in}{2.725396in}}%
\pgfpathlineto{\pgfqpoint{4.236023in}{2.676404in}}%
\pgfpathlineto{\pgfqpoint{4.267601in}{2.627551in}}%
\pgfpathlineto{\pgfqpoint{4.300360in}{2.578935in}}%
\pgfpathlineto{\pgfqpoint{4.335069in}{2.530737in}}%
\pgfpathlineto{\pgfqpoint{4.373263in}{2.483360in}}%
\pgfpathlineto{\pgfqpoint{4.418836in}{2.438140in}}%
\pgfpathlineto{\pgfqpoint{4.418836in}{2.438140in}}%
\pgfpathlineto{\pgfqpoint{4.458085in}{2.412555in}}%
\pgfpathlineto{\pgfqpoint{4.458085in}{2.412555in}}%
\pgfpathlineto{\pgfqpoint{4.490916in}{2.402733in}}%
\pgfpathlineto{\pgfqpoint{4.490916in}{2.402733in}}%
\pgfpathlineto{\pgfqpoint{4.520976in}{2.403570in}}%
\pgfpathlineto{\pgfqpoint{4.547378in}{2.411525in}}%
\pgfpathlineto{\pgfqpoint{4.575238in}{2.426355in}}%
\pgfusepath{stroke}%
\end{pgfscope}%
\begin{pgfscope}%
\pgfpathrectangle{\pgfqpoint{0.647939in}{0.492442in}}{\pgfqpoint{4.273799in}{2.331163in}}%
\pgfusepath{clip}%
\pgfsetbuttcap%
\pgfsetroundjoin%
\pgfsetlinewidth{0.301125pt}%
\definecolor{currentstroke}{rgb}{0.500000,0.500000,0.500000}%
\pgfsetstrokecolor{currentstroke}%
\pgfsetstrokeopacity{0.300000}%
\pgfsetdash{}{0pt}%
\pgfpathmoveto{\pgfqpoint{4.047552in}{2.823605in}}%
\pgfpathlineto{\pgfqpoint{4.047552in}{2.823605in}}%
\pgfpathlineto{\pgfqpoint{4.075151in}{2.774038in}}%
\pgfpathlineto{\pgfqpoint{4.102469in}{2.724424in}}%
\pgfpathlineto{\pgfqpoint{4.129529in}{2.674768in}}%
\pgfpathlineto{\pgfqpoint{4.156358in}{2.625076in}}%
\pgfpathlineto{\pgfqpoint{4.182965in}{2.575349in}}%
\pgfpathlineto{\pgfqpoint{4.209390in}{2.525593in}}%
\pgfpathlineto{\pgfqpoint{4.235670in}{2.475816in}}%
\pgfpathlineto{\pgfqpoint{4.261866in}{2.426026in}}%
\pgfpathlineto{\pgfqpoint{4.288091in}{2.376246in}}%
\pgfpathlineto{\pgfqpoint{4.314516in}{2.326497in}}%
\pgfpathlineto{\pgfqpoint{4.341592in}{2.276874in}}%
\pgfpathlineto{\pgfqpoint{4.370683in}{2.227709in}}%
\pgfusepath{stroke}%
\end{pgfscope}%
\begin{pgfscope}%
\pgfpathrectangle{\pgfqpoint{0.647939in}{0.492442in}}{\pgfqpoint{4.273799in}{2.331163in}}%
\pgfusepath{clip}%
\pgfsetbuttcap%
\pgfsetroundjoin%
\pgfsetlinewidth{0.301125pt}%
\definecolor{currentstroke}{rgb}{0.500000,0.500000,0.500000}%
\pgfsetstrokecolor{currentstroke}%
\pgfsetstrokeopacity{0.300000}%
\pgfsetdash{}{0pt}%
\pgfpathmoveto{\pgfqpoint{3.950420in}{2.823605in}}%
\pgfpathlineto{\pgfqpoint{3.950420in}{2.823605in}}%
\pgfpathlineto{\pgfqpoint{3.976520in}{2.773797in}}%
\pgfpathlineto{\pgfqpoint{4.002051in}{2.723900in}}%
\pgfpathlineto{\pgfqpoint{4.026984in}{2.673915in}}%
\pgfpathlineto{\pgfqpoint{4.051264in}{2.623833in}}%
\pgfpathlineto{\pgfqpoint{4.074832in}{2.573651in}}%
\pgfpathlineto{\pgfqpoint{4.097594in}{2.523359in}}%
\pgfpathlineto{\pgfqpoint{4.119430in}{2.472945in}}%
\pgfpathlineto{\pgfqpoint{4.140168in}{2.422393in}}%
\pgfpathlineto{\pgfqpoint{4.159560in}{2.371684in}}%
\pgfpathlineto{\pgfqpoint{4.177251in}{2.320792in}}%
\pgfpathlineto{\pgfqpoint{4.192697in}{2.269686in}}%
\pgfpathlineto{\pgfqpoint{4.205091in}{2.218337in}}%
\pgfpathlineto{\pgfqpoint{4.213212in}{2.166745in}}%
\pgfpathlineto{\pgfqpoint{4.215287in}{2.114992in}}%
\pgfpathlineto{\pgfqpoint{4.209143in}{2.063373in}}%
\pgfpathlineto{\pgfqpoint{4.192999in}{2.012450in}}%
\pgfpathlineto{\pgfqpoint{4.166585in}{1.962842in}}%
\pgfpathlineto{\pgfqpoint{4.131181in}{1.914916in}}%
\pgfpathlineto{\pgfqpoint{4.088530in}{1.868755in}}%
\pgfpathlineto{\pgfqpoint{4.039952in}{1.824360in}}%
\pgfpathlineto{\pgfqpoint{3.986179in}{1.781771in}}%
\pgfpathlineto{\pgfqpoint{3.927468in}{1.741127in}}%
\pgfpathlineto{\pgfqpoint{3.863648in}{1.702853in}}%
\pgfpathlineto{\pgfqpoint{3.794369in}{1.667549in}}%
\pgfpathlineto{\pgfqpoint{3.719108in}{1.636142in}}%
\pgfusepath{stroke}%
\end{pgfscope}%
\begin{pgfscope}%
\pgfpathrectangle{\pgfqpoint{0.647939in}{0.492442in}}{\pgfqpoint{4.273799in}{2.331163in}}%
\pgfusepath{clip}%
\pgfsetbuttcap%
\pgfsetroundjoin%
\pgfsetlinewidth{0.301125pt}%
\definecolor{currentstroke}{rgb}{0.500000,0.500000,0.500000}%
\pgfsetstrokecolor{currentstroke}%
\pgfsetstrokeopacity{0.300000}%
\pgfsetdash{}{0pt}%
\pgfpathmoveto{\pgfqpoint{3.853289in}{2.823605in}}%
\pgfpathlineto{\pgfqpoint{3.853289in}{2.823605in}}%
\pgfpathlineto{\pgfqpoint{3.878582in}{2.773673in}}%
\pgfpathlineto{\pgfqpoint{3.903132in}{2.723631in}}%
\pgfpathlineto{\pgfqpoint{3.926881in}{2.673474in}}%
\pgfpathlineto{\pgfqpoint{3.949766in}{2.623197in}}%
\pgfpathlineto{\pgfqpoint{3.971699in}{2.572795in}}%
\pgfpathlineto{\pgfqpoint{3.992573in}{2.522260in}}%
\pgfpathlineto{\pgfqpoint{4.012245in}{2.471581in}}%
\pgfpathlineto{\pgfqpoint{4.030529in}{2.420748in}}%
\pgfpathlineto{\pgfqpoint{4.047185in}{2.369751in}}%
\pgfpathlineto{\pgfqpoint{4.061887in}{2.318576in}}%
\pgfpathlineto{\pgfqpoint{4.074219in}{2.267217in}}%
\pgfpathlineto{\pgfqpoint{4.083631in}{2.215678in}}%
\pgfpathlineto{\pgfqpoint{4.089404in}{2.163982in}}%
\pgfpathlineto{\pgfqpoint{4.090644in}{2.112203in}}%
\pgfpathlineto{\pgfqpoint{4.086315in}{2.060485in}}%
\pgfpathlineto{\pgfqpoint{4.075345in}{2.009077in}}%
\pgfpathlineto{\pgfqpoint{4.056868in}{1.958324in}}%
\pgfpathlineto{\pgfqpoint{4.030382in}{1.908648in}}%
\pgfpathlineto{\pgfqpoint{3.995870in}{1.860462in}}%
\pgfpathlineto{\pgfqpoint{3.953668in}{1.814134in}}%
\pgfusepath{stroke}%
\end{pgfscope}%
\begin{pgfscope}%
\pgfpathrectangle{\pgfqpoint{0.647939in}{0.492442in}}{\pgfqpoint{4.273799in}{2.331163in}}%
\pgfusepath{clip}%
\pgfsetbuttcap%
\pgfsetroundjoin%
\pgfsetlinewidth{0.301125pt}%
\definecolor{currentstroke}{rgb}{0.500000,0.500000,0.500000}%
\pgfsetstrokecolor{currentstroke}%
\pgfsetstrokeopacity{0.300000}%
\pgfsetdash{}{0pt}%
\pgfpathmoveto{\pgfqpoint{3.756157in}{2.823605in}}%
\pgfpathlineto{\pgfqpoint{3.756157in}{2.823605in}}%
\pgfpathlineto{\pgfqpoint{3.781166in}{2.773631in}}%
\pgfpathlineto{\pgfqpoint{3.805305in}{2.723529in}}%
\pgfpathlineto{\pgfqpoint{3.828514in}{2.673297in}}%
\pgfpathlineto{\pgfqpoint{3.850725in}{2.622931in}}%
\pgfpathlineto{\pgfqpoint{3.871846in}{2.572426in}}%
\pgfpathlineto{\pgfqpoint{3.891767in}{2.521776in}}%
\pgfpathlineto{\pgfqpoint{3.910351in}{2.470976in}}%
\pgfpathlineto{\pgfqpoint{3.927421in}{2.420018in}}%
\pgfpathlineto{\pgfqpoint{3.942762in}{2.368897in}}%
\pgfpathlineto{\pgfqpoint{3.956105in}{2.317612in}}%
\pgfpathlineto{\pgfqpoint{3.967099in}{2.266162in}}%
\pgfpathlineto{\pgfqpoint{3.975318in}{2.214560in}}%
\pgfpathlineto{\pgfqpoint{3.980234in}{2.162836in}}%
\pgfpathlineto{\pgfqpoint{3.981210in}{2.111050in}}%
\pgfpathlineto{\pgfqpoint{3.977502in}{2.059308in}}%
\pgfpathlineto{\pgfqpoint{3.968296in}{2.007779in}}%
\pgfpathlineto{\pgfqpoint{3.952747in}{1.956716in}}%
\pgfpathlineto{\pgfqpoint{3.930111in}{1.906454in}}%
\pgfpathlineto{\pgfqpoint{3.899794in}{1.857422in}}%
\pgfpathlineto{\pgfqpoint{3.861373in}{1.810123in}}%
\pgfpathlineto{\pgfqpoint{3.814585in}{1.765131in}}%
\pgfpathlineto{\pgfqpoint{3.759220in}{1.723167in}}%
\pgfpathlineto{\pgfqpoint{3.694997in}{1.685212in}}%
\pgfpathlineto{\pgfqpoint{3.621606in}{1.652644in}}%
\pgfpathlineto{\pgfqpoint{3.539191in}{1.627465in}}%
\pgfpathlineto{\pgfqpoint{3.449889in}{1.612213in}}%
\pgfpathlineto{\pgfqpoint{3.364896in}{1.608500in}}%
\pgfusepath{stroke}%
\end{pgfscope}%
\begin{pgfscope}%
\pgfpathrectangle{\pgfqpoint{0.647939in}{0.492442in}}{\pgfqpoint{4.273799in}{2.331163in}}%
\pgfusepath{clip}%
\pgfsetbuttcap%
\pgfsetroundjoin%
\pgfsetlinewidth{0.301125pt}%
\definecolor{currentstroke}{rgb}{0.500000,0.500000,0.500000}%
\pgfsetstrokecolor{currentstroke}%
\pgfsetstrokeopacity{0.300000}%
\pgfsetdash{}{0pt}%
\pgfpathmoveto{\pgfqpoint{3.659025in}{2.823605in}}%
\pgfpathlineto{\pgfqpoint{3.659025in}{2.823605in}}%
\pgfpathlineto{\pgfqpoint{3.684186in}{2.773654in}}%
\pgfpathlineto{\pgfqpoint{3.708369in}{2.723558in}}%
\pgfpathlineto{\pgfqpoint{3.731520in}{2.673318in}}%
\pgfpathlineto{\pgfqpoint{3.753567in}{2.622931in}}%
\pgfpathlineto{\pgfqpoint{3.774425in}{2.572394in}}%
\pgfpathlineto{\pgfqpoint{3.793987in}{2.521703in}}%
\pgfpathlineto{\pgfqpoint{3.812122in}{2.470854in}}%
\pgfpathlineto{\pgfqpoint{3.828670in}{2.419845in}}%
\pgfpathlineto{\pgfqpoint{3.843428in}{2.368674in}}%
\pgfpathlineto{\pgfqpoint{3.856155in}{2.317341in}}%
\pgfpathlineto{\pgfqpoint{3.866556in}{2.265855in}}%
\pgfpathlineto{\pgfqpoint{3.874260in}{2.214229in}}%
\pgfpathlineto{\pgfqpoint{3.878821in}{2.162494in}}%
\pgfpathlineto{\pgfqpoint{3.879704in}{2.110705in}}%
\pgfpathlineto{\pgfqpoint{3.876279in}{2.058954in}}%
\pgfpathlineto{\pgfqpoint{3.867816in}{2.007382in}}%
\pgfpathlineto{\pgfqpoint{3.853512in}{1.956203in}}%
\pgfpathlineto{\pgfqpoint{3.832524in}{1.905722in}}%
\pgfpathlineto{\pgfqpoint{3.803984in}{1.856371in}}%
\pgfpathlineto{\pgfqpoint{3.767055in}{1.808727in}}%
\pgfpathlineto{\pgfqpoint{3.720910in}{1.763563in}}%
\pgfusepath{stroke}%
\end{pgfscope}%
\begin{pgfscope}%
\pgfpathrectangle{\pgfqpoint{0.647939in}{0.492442in}}{\pgfqpoint{4.273799in}{2.331163in}}%
\pgfusepath{clip}%
\pgfsetbuttcap%
\pgfsetroundjoin%
\pgfsetlinewidth{0.301125pt}%
\definecolor{currentstroke}{rgb}{0.500000,0.500000,0.500000}%
\pgfsetstrokecolor{currentstroke}%
\pgfsetstrokeopacity{0.300000}%
\pgfsetdash{}{0pt}%
\pgfpathmoveto{\pgfqpoint{3.561893in}{2.823605in}}%
\pgfpathlineto{\pgfqpoint{3.561893in}{2.823605in}}%
\pgfpathlineto{\pgfqpoint{3.587590in}{2.773735in}}%
\pgfpathlineto{\pgfqpoint{3.612215in}{2.723704in}}%
\pgfpathlineto{\pgfqpoint{3.635715in}{2.673513in}}%
\pgfpathlineto{\pgfqpoint{3.658020in}{2.623159in}}%
\pgfpathlineto{\pgfqpoint{3.679048in}{2.572643in}}%
\pgfpathlineto{\pgfqpoint{3.698695in}{2.521962in}}%
\pgfpathlineto{\pgfqpoint{3.716836in}{2.471114in}}%
\pgfpathlineto{\pgfqpoint{3.733321in}{2.420099in}}%
\pgfpathlineto{\pgfqpoint{3.747964in}{2.368918in}}%
\pgfpathlineto{\pgfqpoint{3.760536in}{2.317575in}}%
\pgfpathlineto{\pgfqpoint{3.770765in}{2.266078in}}%
\pgfpathlineto{\pgfqpoint{3.778315in}{2.214445in}}%
\pgfpathlineto{\pgfqpoint{3.782776in}{2.162707in}}%
\pgfpathlineto{\pgfqpoint{3.783653in}{2.110917in}}%
\pgfpathlineto{\pgfqpoint{3.780353in}{2.059162in}}%
\pgfpathlineto{\pgfqpoint{3.772170in}{2.007575in}}%
\pgfpathlineto{\pgfqpoint{3.758284in}{1.956361in}}%
\pgfpathlineto{\pgfqpoint{3.737737in}{1.905827in}}%
\pgfusepath{stroke}%
\end{pgfscope}%
\begin{pgfscope}%
\pgfpathrectangle{\pgfqpoint{0.647939in}{0.492442in}}{\pgfqpoint{4.273799in}{2.331163in}}%
\pgfusepath{clip}%
\pgfsetbuttcap%
\pgfsetroundjoin%
\pgfsetlinewidth{0.301125pt}%
\definecolor{currentstroke}{rgb}{0.500000,0.500000,0.500000}%
\pgfsetstrokecolor{currentstroke}%
\pgfsetstrokeopacity{0.300000}%
\pgfsetdash{}{0pt}%
\pgfpathmoveto{\pgfqpoint{3.464761in}{2.823605in}}%
\pgfpathlineto{\pgfqpoint{3.464761in}{2.823605in}}%
\pgfpathlineto{\pgfqpoint{3.491371in}{2.773878in}}%
\pgfpathlineto{\pgfqpoint{3.516813in}{2.723969in}}%
\pgfpathlineto{\pgfqpoint{3.541033in}{2.673880in}}%
\pgfpathlineto{\pgfqpoint{3.563967in}{2.623611in}}%
\pgfpathlineto{\pgfqpoint{3.585536in}{2.573163in}}%
\pgfpathlineto{\pgfqpoint{3.605639in}{2.522535in}}%
\pgfpathlineto{\pgfqpoint{3.624158in}{2.471728in}}%
\pgfpathlineto{\pgfqpoint{3.640946in}{2.420743in}}%
\pgfpathlineto{\pgfqpoint{3.655828in}{2.369582in}}%
\pgfpathlineto{\pgfqpoint{3.668580in}{2.318252in}}%
\pgfpathlineto{\pgfqpoint{3.678940in}{2.266763in}}%
\pgfpathlineto{\pgfqpoint{3.686587in}{2.215135in}}%
\pgfpathlineto{\pgfqpoint{3.691121in}{2.163399in}}%
\pgfpathlineto{\pgfqpoint{3.692056in}{2.111609in}}%
\pgfpathlineto{\pgfqpoint{3.688793in}{2.059854in}}%
\pgfpathlineto{\pgfqpoint{3.680601in}{2.008268in}}%
\pgfpathlineto{\pgfqpoint{3.666588in}{1.957066in}}%
\pgfpathlineto{\pgfqpoint{3.645630in}{1.906587in}}%
\pgfpathlineto{\pgfqpoint{3.616367in}{1.857376in}}%
\pgfpathlineto{\pgfqpoint{3.577126in}{1.810322in}}%
\pgfpathlineto{\pgfqpoint{3.525864in}{1.766932in}}%
\pgfpathlineto{\pgfqpoint{3.460564in}{1.729829in}}%
\pgfpathlineto{\pgfqpoint{3.385861in}{1.704316in}}%
\pgfpathlineto{\pgfqpoint{3.312689in}{1.692659in}}%
\pgfpathlineto{\pgfqpoint{3.242985in}{1.691962in}}%
\pgfpathlineto{\pgfqpoint{3.175877in}{1.700095in}}%
\pgfpathlineto{\pgfqpoint{3.109580in}{1.716321in}}%
\pgfpathlineto{\pgfqpoint{3.042568in}{1.741053in}}%
\pgfpathlineto{\pgfqpoint{2.975029in}{1.774897in}}%
\pgfpathlineto{\pgfqpoint{2.913604in}{1.814229in}}%
\pgfpathlineto{\pgfqpoint{2.859616in}{1.856722in}}%
\pgfpathlineto{\pgfqpoint{2.812418in}{1.901595in}}%
\pgfpathlineto{\pgfqpoint{2.771786in}{1.948353in}}%
\pgfpathlineto{\pgfqpoint{2.737968in}{1.996690in}}%
\pgfpathlineto{\pgfqpoint{2.711944in}{2.046430in}}%
\pgfpathlineto{\pgfqpoint{2.696195in}{2.097387in}}%
\pgfpathlineto{\pgfqpoint{2.697697in}{2.148711in}}%
\pgfpathlineto{\pgfqpoint{2.697697in}{2.148711in}}%
\pgfpathlineto{\pgfqpoint{2.711284in}{2.173033in}}%
\pgfpathlineto{\pgfqpoint{2.711284in}{2.173033in}}%
\pgfpathlineto{\pgfqpoint{2.732198in}{2.185988in}}%
\pgfpathlineto{\pgfqpoint{2.732198in}{2.185988in}}%
\pgfpathlineto{\pgfqpoint{2.756121in}{2.188621in}}%
\pgfpathlineto{\pgfqpoint{2.779931in}{2.183582in}}%
\pgfpathlineto{\pgfqpoint{2.801952in}{2.172785in}}%
\pgfusepath{stroke}%
\end{pgfscope}%
\begin{pgfscope}%
\pgfpathrectangle{\pgfqpoint{0.647939in}{0.492442in}}{\pgfqpoint{4.273799in}{2.331163in}}%
\pgfusepath{clip}%
\pgfsetbuttcap%
\pgfsetroundjoin%
\pgfsetlinewidth{0.301125pt}%
\definecolor{currentstroke}{rgb}{0.500000,0.500000,0.500000}%
\pgfsetstrokecolor{currentstroke}%
\pgfsetstrokeopacity{0.300000}%
\pgfsetdash{}{0pt}%
\pgfpathmoveto{\pgfqpoint{3.367630in}{2.823605in}}%
\pgfpathlineto{\pgfqpoint{3.367630in}{2.823605in}}%
\pgfpathlineto{\pgfqpoint{3.395539in}{2.774091in}}%
\pgfpathlineto{\pgfqpoint{3.422177in}{2.724368in}}%
\pgfpathlineto{\pgfqpoint{3.447491in}{2.674440in}}%
\pgfpathlineto{\pgfqpoint{3.471420in}{2.624310in}}%
\pgfpathlineto{\pgfqpoint{3.493886in}{2.573979in}}%
\pgfusepath{stroke}%
\end{pgfscope}%
\begin{pgfscope}%
\pgfpathrectangle{\pgfqpoint{0.647939in}{0.492442in}}{\pgfqpoint{4.273799in}{2.331163in}}%
\pgfusepath{clip}%
\pgfsetbuttcap%
\pgfsetroundjoin%
\pgfsetlinewidth{0.301125pt}%
\definecolor{currentstroke}{rgb}{0.500000,0.500000,0.500000}%
\pgfsetstrokecolor{currentstroke}%
\pgfsetstrokeopacity{0.300000}%
\pgfsetdash{}{0pt}%
\pgfpathmoveto{\pgfqpoint{3.270498in}{2.823605in}}%
\pgfpathlineto{\pgfqpoint{3.270498in}{2.823605in}}%
\pgfpathlineto{\pgfqpoint{3.300126in}{2.774389in}}%
\pgfpathlineto{\pgfqpoint{3.328360in}{2.724930in}}%
\pgfpathlineto{\pgfqpoint{3.355155in}{2.675233in}}%
\pgfpathlineto{\pgfqpoint{3.380453in}{2.625303in}}%
\pgfpathlineto{\pgfqpoint{3.404179in}{2.575145in}}%
\pgfpathlineto{\pgfqpoint{3.426242in}{2.524761in}}%
\pgfpathlineto{\pgfqpoint{3.446528in}{2.474156in}}%
\pgfpathlineto{\pgfqpoint{3.464896in}{2.423334in}}%
\pgfpathlineto{\pgfqpoint{3.481172in}{2.372301in}}%
\pgfpathlineto{\pgfqpoint{3.495137in}{2.321065in}}%
\pgfpathlineto{\pgfqpoint{3.506523in}{2.269642in}}%
\pgfpathlineto{\pgfqpoint{3.514998in}{2.218053in}}%
\pgfpathlineto{\pgfqpoint{3.520134in}{2.166336in}}%
\pgfpathlineto{\pgfqpoint{3.521390in}{2.114550in}}%
\pgfpathlineto{\pgfqpoint{3.518066in}{2.062799in}}%
\pgfpathlineto{\pgfqpoint{3.509238in}{2.011251in}}%
\pgfpathlineto{\pgfqpoint{3.493671in}{1.960197in}}%
\pgfpathlineto{\pgfqpoint{3.469625in}{1.910163in}}%
\pgfpathlineto{\pgfqpoint{3.434646in}{1.862153in}}%
\pgfpathlineto{\pgfqpoint{3.385306in}{1.818243in}}%
\pgfpathlineto{\pgfqpoint{3.385306in}{1.818243in}}%
\pgfpathlineto{\pgfqpoint{3.332563in}{1.788612in}}%
\pgfpathlineto{\pgfqpoint{3.266125in}{1.768028in}}%
\pgfpathlineto{\pgfqpoint{3.202441in}{1.761226in}}%
\pgfpathlineto{\pgfqpoint{3.142543in}{1.764407in}}%
\pgfusepath{stroke}%
\end{pgfscope}%
\begin{pgfscope}%
\pgfpathrectangle{\pgfqpoint{0.647939in}{0.492442in}}{\pgfqpoint{4.273799in}{2.331163in}}%
\pgfusepath{clip}%
\pgfsetbuttcap%
\pgfsetroundjoin%
\pgfsetlinewidth{0.301125pt}%
\definecolor{currentstroke}{rgb}{0.500000,0.500000,0.500000}%
\pgfsetstrokecolor{currentstroke}%
\pgfsetstrokeopacity{0.300000}%
\pgfsetdash{}{0pt}%
\pgfpathmoveto{\pgfqpoint{3.173366in}{2.823605in}}%
\pgfpathlineto{\pgfqpoint{3.173366in}{2.823605in}}%
\pgfpathlineto{\pgfqpoint{3.205171in}{2.774796in}}%
\pgfpathlineto{\pgfqpoint{3.235441in}{2.725696in}}%
\pgfpathlineto{\pgfqpoint{3.264138in}{2.676317in}}%
\pgfpathlineto{\pgfqpoint{3.291208in}{2.626665in}}%
\pgfpathlineto{\pgfqpoint{3.316579in}{2.576747in}}%
\pgfpathlineto{\pgfqpoint{3.340164in}{2.526570in}}%
\pgfpathlineto{\pgfqpoint{3.361850in}{2.476139in}}%
\pgfpathlineto{\pgfqpoint{3.381498in}{2.425460in}}%
\pgfpathlineto{\pgfqpoint{3.398933in}{2.374541in}}%
\pgfpathlineto{\pgfqpoint{3.413931in}{2.323393in}}%
\pgfusepath{stroke}%
\end{pgfscope}%
\begin{pgfscope}%
\pgfpathrectangle{\pgfqpoint{0.647939in}{0.492442in}}{\pgfqpoint{4.273799in}{2.331163in}}%
\pgfusepath{clip}%
\pgfsetbuttcap%
\pgfsetroundjoin%
\pgfsetlinewidth{0.301125pt}%
\definecolor{currentstroke}{rgb}{0.500000,0.500000,0.500000}%
\pgfsetstrokecolor{currentstroke}%
\pgfsetstrokeopacity{0.300000}%
\pgfsetdash{}{0pt}%
\pgfpathmoveto{\pgfqpoint{2.979102in}{2.823605in}}%
\pgfpathlineto{\pgfqpoint{2.979102in}{2.823605in}}%
\pgfpathlineto{\pgfqpoint{3.016856in}{2.776074in}}%
\pgfpathlineto{\pgfqpoint{3.052724in}{2.728112in}}%
\pgfpathlineto{\pgfqpoint{3.086675in}{2.679737in}}%
\pgfpathlineto{\pgfqpoint{3.118664in}{2.630965in}}%
\pgfpathlineto{\pgfqpoint{3.148629in}{2.581812in}}%
\pgfpathlineto{\pgfqpoint{3.176493in}{2.532294in}}%
\pgfpathlineto{\pgfqpoint{3.202153in}{2.482422in}}%
\pgfpathlineto{\pgfqpoint{3.225470in}{2.432210in}}%
\pgfpathlineto{\pgfqpoint{3.246267in}{2.381670in}}%
\pgfpathlineto{\pgfqpoint{3.264311in}{2.330818in}}%
\pgfpathlineto{\pgfqpoint{3.279292in}{2.279671in}}%
\pgfpathlineto{\pgfqpoint{3.290805in}{2.228261in}}%
\pgfpathlineto{\pgfqpoint{3.298298in}{2.176636in}}%
\pgfpathlineto{\pgfqpoint{3.300997in}{2.124877in}}%
\pgfpathlineto{\pgfqpoint{3.297787in}{2.073138in}}%
\pgfpathlineto{\pgfqpoint{3.286967in}{2.021730in}}%
\pgfpathlineto{\pgfqpoint{3.265767in}{1.971360in}}%
\pgfpathlineto{\pgfqpoint{3.229330in}{1.923861in}}%
\pgfpathlineto{\pgfqpoint{3.229330in}{1.923861in}}%
\pgfpathlineto{\pgfqpoint{3.189085in}{1.894007in}}%
\pgfpathlineto{\pgfqpoint{3.189085in}{1.894007in}}%
\pgfpathlineto{\pgfqpoint{3.144937in}{1.876024in}}%
\pgfpathlineto{\pgfqpoint{3.091662in}{1.868082in}}%
\pgfpathlineto{\pgfqpoint{3.043147in}{1.870708in}}%
\pgfpathlineto{\pgfqpoint{2.996052in}{1.881016in}}%
\pgfpathlineto{\pgfqpoint{2.947778in}{1.899112in}}%
\pgfpathlineto{\pgfqpoint{2.897997in}{1.926020in}}%
\pgfpathlineto{\pgfqpoint{2.848011in}{1.962806in}}%
\pgfpathlineto{\pgfqpoint{2.802588in}{2.008100in}}%
\pgfusepath{stroke}%
\end{pgfscope}%
\begin{pgfscope}%
\pgfpathrectangle{\pgfqpoint{0.647939in}{0.492442in}}{\pgfqpoint{4.273799in}{2.331163in}}%
\pgfusepath{clip}%
\pgfsetbuttcap%
\pgfsetroundjoin%
\pgfsetlinewidth{0.301125pt}%
\definecolor{currentstroke}{rgb}{0.500000,0.500000,0.500000}%
\pgfsetstrokecolor{currentstroke}%
\pgfsetstrokeopacity{0.300000}%
\pgfsetdash{}{0pt}%
\pgfpathmoveto{\pgfqpoint{2.784839in}{2.823605in}}%
\pgfpathlineto{\pgfqpoint{2.784839in}{2.823605in}}%
\pgfpathlineto{\pgfqpoint{2.830938in}{2.778324in}}%
\pgfpathlineto{\pgfqpoint{2.874728in}{2.732367in}}%
\pgfpathlineto{\pgfqpoint{2.916159in}{2.685763in}}%
\pgfpathlineto{\pgfqpoint{2.955182in}{2.638544in}}%
\pgfpathlineto{\pgfqpoint{2.991745in}{2.590743in}}%
\pgfpathlineto{\pgfqpoint{3.025781in}{2.542389in}}%
\pgfpathlineto{\pgfqpoint{3.057201in}{2.493510in}}%
\pgfpathlineto{\pgfqpoint{3.085881in}{2.444134in}}%
\pgfpathlineto{\pgfqpoint{3.111653in}{2.394284in}}%
\pgfpathlineto{\pgfqpoint{3.134283in}{2.343983in}}%
\pgfpathlineto{\pgfqpoint{3.153449in}{2.293258in}}%
\pgfpathlineto{\pgfqpoint{3.168696in}{2.242142in}}%
\pgfpathlineto{\pgfqpoint{3.179375in}{2.190688in}}%
\pgfpathlineto{\pgfqpoint{3.184508in}{2.138990in}}%
\pgfpathlineto{\pgfqpoint{3.182552in}{2.087250in}}%
\pgfpathlineto{\pgfqpoint{3.170845in}{2.035958in}}%
\pgfpathlineto{\pgfqpoint{3.144229in}{1.986547in}}%
\pgfpathlineto{\pgfqpoint{3.144229in}{1.986547in}}%
\pgfpathlineto{\pgfqpoint{3.111667in}{1.955674in}}%
\pgfpathlineto{\pgfqpoint{3.111667in}{1.955674in}}%
\pgfpathlineto{\pgfqpoint{3.074775in}{1.937404in}}%
\pgfpathlineto{\pgfqpoint{3.074775in}{1.937404in}}%
\pgfpathlineto{\pgfqpoint{3.034838in}{1.929543in}}%
\pgfusepath{stroke}%
\end{pgfscope}%
\begin{pgfscope}%
\pgfpathrectangle{\pgfqpoint{0.647939in}{0.492442in}}{\pgfqpoint{4.273799in}{2.331163in}}%
\pgfusepath{clip}%
\pgfsetbuttcap%
\pgfsetroundjoin%
\pgfsetlinewidth{0.301125pt}%
\definecolor{currentstroke}{rgb}{0.500000,0.500000,0.500000}%
\pgfsetstrokecolor{currentstroke}%
\pgfsetstrokeopacity{0.300000}%
\pgfsetdash{}{0pt}%
\pgfpathmoveto{\pgfqpoint{2.590575in}{2.823605in}}%
\pgfpathlineto{\pgfqpoint{2.590575in}{2.823605in}}%
\pgfpathlineto{\pgfqpoint{2.647013in}{2.781954in}}%
\pgfpathlineto{\pgfqpoint{2.700889in}{2.739310in}}%
\pgfpathlineto{\pgfqpoint{2.752026in}{2.695675in}}%
\pgfpathlineto{\pgfqpoint{2.800291in}{2.651075in}}%
\pgfpathlineto{\pgfqpoint{2.845584in}{2.605555in}}%
\pgfpathlineto{\pgfqpoint{2.887824in}{2.559170in}}%
\pgfpathlineto{\pgfqpoint{2.926927in}{2.511975in}}%
\pgfpathlineto{\pgfqpoint{2.962797in}{2.464024in}}%
\pgfpathlineto{\pgfqpoint{2.995300in}{2.415366in}}%
\pgfpathlineto{\pgfqpoint{3.024236in}{2.366041in}}%
\pgfpathlineto{\pgfqpoint{3.049313in}{2.316093in}}%
\pgfpathlineto{\pgfqpoint{3.070091in}{2.265565in}}%
\pgfpathlineto{\pgfqpoint{3.085907in}{2.214511in}}%
\pgfpathlineto{\pgfqpoint{3.095702in}{2.163019in}}%
\pgfpathlineto{\pgfqpoint{3.097671in}{2.111293in}}%
\pgfpathlineto{\pgfqpoint{3.088355in}{2.059903in}}%
\pgfpathlineto{\pgfqpoint{3.059830in}{2.011104in}}%
\pgfpathlineto{\pgfqpoint{3.059830in}{2.011104in}}%
\pgfusepath{stroke}%
\end{pgfscope}%
\begin{pgfscope}%
\pgfpathrectangle{\pgfqpoint{0.647939in}{0.492442in}}{\pgfqpoint{4.273799in}{2.331163in}}%
\pgfusepath{clip}%
\pgfsetbuttcap%
\pgfsetroundjoin%
\pgfsetlinewidth{0.301125pt}%
\definecolor{currentstroke}{rgb}{0.500000,0.500000,0.500000}%
\pgfsetstrokecolor{currentstroke}%
\pgfsetstrokeopacity{0.300000}%
\pgfsetdash{}{0pt}%
\pgfpathmoveto{\pgfqpoint{2.396312in}{2.823605in}}%
\pgfpathlineto{\pgfqpoint{2.396312in}{2.823605in}}%
\pgfpathlineto{\pgfqpoint{2.462684in}{2.786561in}}%
\pgfpathlineto{\pgfqpoint{2.527201in}{2.748563in}}%
\pgfpathlineto{\pgfqpoint{2.589249in}{2.709363in}}%
\pgfpathlineto{\pgfqpoint{2.648347in}{2.668836in}}%
\pgfpathlineto{\pgfqpoint{2.704143in}{2.626946in}}%
\pgfpathlineto{\pgfqpoint{2.756425in}{2.583727in}}%
\pgfpathlineto{\pgfqpoint{2.805044in}{2.539248in}}%
\pgfusepath{stroke}%
\end{pgfscope}%
\begin{pgfscope}%
\pgfpathrectangle{\pgfqpoint{0.647939in}{0.492442in}}{\pgfqpoint{4.273799in}{2.331163in}}%
\pgfusepath{clip}%
\pgfsetbuttcap%
\pgfsetroundjoin%
\pgfsetlinewidth{0.301125pt}%
\definecolor{currentstroke}{rgb}{0.500000,0.500000,0.500000}%
\pgfsetstrokecolor{currentstroke}%
\pgfsetstrokeopacity{0.300000}%
\pgfsetdash{}{0pt}%
\pgfpathmoveto{\pgfqpoint{2.299180in}{2.823605in}}%
\pgfpathlineto{\pgfqpoint{2.299180in}{2.823605in}}%
\pgfpathlineto{\pgfqpoint{2.369076in}{2.788536in}}%
\pgfpathlineto{\pgfqpoint{2.438069in}{2.752949in}}%
\pgfpathlineto{\pgfqpoint{2.505270in}{2.716361in}}%
\pgfpathlineto{\pgfqpoint{2.569904in}{2.678432in}}%
\pgfpathlineto{\pgfqpoint{2.631375in}{2.638974in}}%
\pgfusepath{stroke}%
\end{pgfscope}%
\begin{pgfscope}%
\pgfpathrectangle{\pgfqpoint{0.647939in}{0.492442in}}{\pgfqpoint{4.273799in}{2.331163in}}%
\pgfusepath{clip}%
\pgfsetbuttcap%
\pgfsetroundjoin%
\pgfsetlinewidth{0.301125pt}%
\definecolor{currentstroke}{rgb}{0.500000,0.500000,0.500000}%
\pgfsetstrokecolor{currentstroke}%
\pgfsetstrokeopacity{0.300000}%
\pgfsetdash{}{0pt}%
\pgfpathmoveto{\pgfqpoint{2.104916in}{2.823605in}}%
\pgfpathlineto{\pgfqpoint{2.104916in}{2.823605in}}%
\pgfpathlineto{\pgfqpoint{2.176326in}{2.789476in}}%
\pgfpathlineto{\pgfqpoint{2.249982in}{2.756785in}}%
\pgfpathlineto{\pgfqpoint{2.324610in}{2.724750in}}%
\pgfpathlineto{\pgfqpoint{2.398827in}{2.692443in}}%
\pgfpathlineto{\pgfqpoint{2.471360in}{2.659036in}}%
\pgfpathlineto{\pgfqpoint{2.541127in}{2.623930in}}%
\pgfpathlineto{\pgfqpoint{2.607220in}{2.586789in}}%
\pgfpathlineto{\pgfqpoint{2.669049in}{2.547527in}}%
\pgfpathlineto{\pgfqpoint{2.726236in}{2.506224in}}%
\pgfpathlineto{\pgfqpoint{2.778560in}{2.463047in}}%
\pgfpathlineto{\pgfqpoint{2.825888in}{2.418190in}}%
\pgfpathlineto{\pgfqpoint{2.868102in}{2.371837in}}%
\pgfpathlineto{\pgfqpoint{2.904995in}{2.324150in}}%
\pgfpathlineto{\pgfqpoint{2.936179in}{2.275262in}}%
\pgfpathlineto{\pgfqpoint{2.960921in}{2.225297in}}%
\pgfpathlineto{\pgfqpoint{2.977826in}{2.174397in}}%
\pgfpathlineto{\pgfqpoint{2.983939in}{2.122844in}}%
\pgfpathlineto{\pgfqpoint{2.971250in}{2.071985in}}%
\pgfpathlineto{\pgfqpoint{2.971250in}{2.071985in}}%
\pgfpathlineto{\pgfqpoint{2.951819in}{2.050165in}}%
\pgfpathlineto{\pgfqpoint{2.951819in}{2.050165in}}%
\pgfpathlineto{\pgfqpoint{2.926851in}{2.039573in}}%
\pgfpathlineto{\pgfqpoint{2.896079in}{2.039354in}}%
\pgfpathlineto{\pgfqpoint{2.870164in}{2.046597in}}%
\pgfusepath{stroke}%
\end{pgfscope}%
\begin{pgfscope}%
\pgfpathrectangle{\pgfqpoint{0.647939in}{0.492442in}}{\pgfqpoint{4.273799in}{2.331163in}}%
\pgfusepath{clip}%
\pgfsetbuttcap%
\pgfsetroundjoin%
\pgfsetlinewidth{0.301125pt}%
\definecolor{currentstroke}{rgb}{0.500000,0.500000,0.500000}%
\pgfsetstrokecolor{currentstroke}%
\pgfsetstrokeopacity{0.300000}%
\pgfsetdash{}{0pt}%
\pgfpathmoveto{\pgfqpoint{1.910652in}{2.823605in}}%
\pgfpathlineto{\pgfqpoint{1.910652in}{2.823605in}}%
\pgfpathlineto{\pgfqpoint{1.973912in}{2.785027in}}%
\pgfpathlineto{\pgfqpoint{2.042666in}{2.749366in}}%
\pgfpathlineto{\pgfqpoint{2.116570in}{2.716906in}}%
\pgfpathlineto{\pgfqpoint{2.194485in}{2.687335in}}%
\pgfpathlineto{\pgfqpoint{2.274691in}{2.659608in}}%
\pgfpathlineto{\pgfqpoint{2.355310in}{2.632241in}}%
\pgfpathlineto{\pgfqpoint{2.434529in}{2.603730in}}%
\pgfpathlineto{\pgfqpoint{2.510771in}{2.572940in}}%
\pgfpathlineto{\pgfqpoint{2.582769in}{2.539256in}}%
\pgfusepath{stroke}%
\end{pgfscope}%
\begin{pgfscope}%
\pgfpathrectangle{\pgfqpoint{0.647939in}{0.492442in}}{\pgfqpoint{4.273799in}{2.331163in}}%
\pgfusepath{clip}%
\pgfsetbuttcap%
\pgfsetroundjoin%
\pgfsetlinewidth{0.301125pt}%
\definecolor{currentstroke}{rgb}{0.500000,0.500000,0.500000}%
\pgfsetstrokecolor{currentstroke}%
\pgfsetstrokeopacity{0.300000}%
\pgfsetdash{}{0pt}%
\pgfpathmoveto{\pgfqpoint{1.716389in}{2.823605in}}%
\pgfpathlineto{\pgfqpoint{1.716389in}{2.823605in}}%
\pgfpathlineto{\pgfqpoint{1.763605in}{2.778685in}}%
\pgfpathlineto{\pgfqpoint{1.815552in}{2.735355in}}%
\pgfpathlineto{\pgfqpoint{1.873476in}{2.694376in}}%
\pgfpathlineto{\pgfqpoint{1.938828in}{2.656934in}}%
\pgfpathlineto{\pgfqpoint{2.012653in}{2.624588in}}%
\pgfpathlineto{\pgfqpoint{2.094348in}{2.598508in}}%
\pgfpathlineto{\pgfqpoint{2.181467in}{2.578148in}}%
\pgfpathlineto{\pgfqpoint{2.270940in}{2.560879in}}%
\pgfpathlineto{\pgfqpoint{2.360203in}{2.543303in}}%
\pgfpathlineto{\pgfqpoint{2.447053in}{2.522603in}}%
\pgfpathlineto{\pgfqpoint{2.529434in}{2.497169in}}%
\pgfpathlineto{\pgfqpoint{2.605683in}{2.466604in}}%
\pgfpathlineto{\pgfqpoint{2.674828in}{2.431314in}}%
\pgfusepath{stroke}%
\end{pgfscope}%
\begin{pgfscope}%
\pgfpathrectangle{\pgfqpoint{0.647939in}{0.492442in}}{\pgfqpoint{4.273799in}{2.331163in}}%
\pgfusepath{clip}%
\pgfsetbuttcap%
\pgfsetroundjoin%
\pgfsetlinewidth{0.301125pt}%
\definecolor{currentstroke}{rgb}{0.500000,0.500000,0.500000}%
\pgfsetstrokecolor{currentstroke}%
\pgfsetstrokeopacity{0.300000}%
\pgfsetdash{}{0pt}%
\pgfpathmoveto{\pgfqpoint{1.619257in}{2.823605in}}%
\pgfpathlineto{\pgfqpoint{1.619257in}{2.823605in}}%
\pgfpathlineto{\pgfqpoint{1.658139in}{2.776358in}}%
\pgfpathlineto{\pgfqpoint{1.700015in}{2.729886in}}%
\pgfpathlineto{\pgfqpoint{1.745821in}{2.684544in}}%
\pgfpathlineto{\pgfqpoint{1.796951in}{2.640959in}}%
\pgfpathlineto{\pgfqpoint{1.855511in}{2.600306in}}%
\pgfpathlineto{\pgfqpoint{1.924116in}{2.564790in}}%
\pgfpathlineto{\pgfqpoint{2.004271in}{2.537769in}}%
\pgfpathlineto{\pgfqpoint{2.086352in}{2.522187in}}%
\pgfpathlineto{\pgfqpoint{2.172525in}{2.513535in}}%
\pgfpathlineto{\pgfqpoint{2.266623in}{2.506966in}}%
\pgfpathlineto{\pgfqpoint{2.360029in}{2.498326in}}%
\pgfusepath{stroke}%
\end{pgfscope}%
\begin{pgfscope}%
\pgfpathrectangle{\pgfqpoint{0.647939in}{0.492442in}}{\pgfqpoint{4.273799in}{2.331163in}}%
\pgfusepath{clip}%
\pgfsetbuttcap%
\pgfsetroundjoin%
\pgfsetlinewidth{0.301125pt}%
\definecolor{currentstroke}{rgb}{0.500000,0.500000,0.500000}%
\pgfsetstrokecolor{currentstroke}%
\pgfsetstrokeopacity{0.300000}%
\pgfsetdash{}{0pt}%
\pgfpathmoveto{\pgfqpoint{1.522125in}{2.823605in}}%
\pgfpathlineto{\pgfqpoint{1.522125in}{2.823605in}}%
\pgfpathlineto{\pgfqpoint{1.553759in}{2.774764in}}%
\pgfpathlineto{\pgfqpoint{1.586915in}{2.726226in}}%
\pgfpathlineto{\pgfqpoint{1.621970in}{2.678094in}}%
\pgfpathlineto{\pgfqpoint{1.659526in}{2.630535in}}%
\pgfpathlineto{\pgfqpoint{1.700525in}{2.583842in}}%
\pgfpathlineto{\pgfqpoint{1.746543in}{2.538595in}}%
\pgfpathlineto{\pgfqpoint{1.800430in}{2.496116in}}%
\pgfpathlineto{\pgfqpoint{1.867376in}{2.459937in}}%
\pgfpathlineto{\pgfqpoint{1.867376in}{2.459937in}}%
\pgfpathlineto{\pgfqpoint{1.923613in}{2.442693in}}%
\pgfpathlineto{\pgfqpoint{1.988057in}{2.435119in}}%
\pgfpathlineto{\pgfqpoint{2.049613in}{2.436026in}}%
\pgfpathlineto{\pgfqpoint{2.123625in}{2.442627in}}%
\pgfpathlineto{\pgfqpoint{2.216673in}{2.452519in}}%
\pgfpathlineto{\pgfqpoint{2.310564in}{2.458690in}}%
\pgfusepath{stroke}%
\end{pgfscope}%
\begin{pgfscope}%
\pgfpathrectangle{\pgfqpoint{0.647939in}{0.492442in}}{\pgfqpoint{4.273799in}{2.331163in}}%
\pgfusepath{clip}%
\pgfsetbuttcap%
\pgfsetroundjoin%
\pgfsetlinewidth{0.301125pt}%
\definecolor{currentstroke}{rgb}{0.500000,0.500000,0.500000}%
\pgfsetstrokecolor{currentstroke}%
\pgfsetstrokeopacity{0.300000}%
\pgfsetdash{}{0pt}%
\pgfpathmoveto{\pgfqpoint{1.424993in}{2.823605in}}%
\pgfpathlineto{\pgfqpoint{1.424993in}{2.823605in}}%
\pgfpathlineto{\pgfqpoint{1.450659in}{2.773732in}}%
\pgfpathlineto{\pgfqpoint{1.476865in}{2.723943in}}%
\pgfpathlineto{\pgfqpoint{1.503708in}{2.674256in}}%
\pgfpathlineto{\pgfqpoint{1.531294in}{2.624692in}}%
\pgfpathlineto{\pgfqpoint{1.559794in}{2.575283in}}%
\pgfpathlineto{\pgfqpoint{1.589460in}{2.526080in}}%
\pgfpathlineto{\pgfqpoint{1.620665in}{2.477168in}}%
\pgfpathlineto{\pgfqpoint{1.654108in}{2.428711in}}%
\pgfpathlineto{\pgfqpoint{1.691216in}{2.381071in}}%
\pgfpathlineto{\pgfqpoint{1.735461in}{2.335371in}}%
\pgfpathlineto{\pgfqpoint{1.735461in}{2.335371in}}%
\pgfpathlineto{\pgfqpoint{1.778235in}{2.305865in}}%
\pgfpathlineto{\pgfqpoint{1.778235in}{2.305865in}}%
\pgfpathlineto{\pgfqpoint{1.813238in}{2.293768in}}%
\pgfpathlineto{\pgfqpoint{1.813238in}{2.293768in}}%
\pgfpathlineto{\pgfqpoint{1.848484in}{2.291796in}}%
\pgfpathlineto{\pgfqpoint{1.882981in}{2.297155in}}%
\pgfpathlineto{\pgfqpoint{1.921729in}{2.308713in}}%
\pgfpathlineto{\pgfqpoint{1.974834in}{2.329197in}}%
\pgfpathlineto{\pgfqpoint{2.050784in}{2.359984in}}%
\pgfpathlineto{\pgfqpoint{2.129305in}{2.388704in}}%
\pgfusepath{stroke}%
\end{pgfscope}%
\begin{pgfscope}%
\pgfpathrectangle{\pgfqpoint{0.647939in}{0.492442in}}{\pgfqpoint{4.273799in}{2.331163in}}%
\pgfusepath{clip}%
\pgfsetbuttcap%
\pgfsetroundjoin%
\pgfsetlinewidth{0.301125pt}%
\definecolor{currentstroke}{rgb}{0.500000,0.500000,0.500000}%
\pgfsetstrokecolor{currentstroke}%
\pgfsetstrokeopacity{0.300000}%
\pgfsetdash{}{0pt}%
\pgfpathmoveto{\pgfqpoint{1.327862in}{2.823605in}}%
\pgfpathlineto{\pgfqpoint{1.327862in}{2.823605in}}%
\pgfpathlineto{\pgfqpoint{1.348778in}{2.773076in}}%
\pgfpathlineto{\pgfqpoint{1.369688in}{2.722545in}}%
\pgfpathlineto{\pgfqpoint{1.390558in}{2.672010in}}%
\pgfpathlineto{\pgfqpoint{1.411339in}{2.621465in}}%
\pgfpathlineto{\pgfqpoint{1.431969in}{2.570901in}}%
\pgfpathlineto{\pgfqpoint{1.452361in}{2.520309in}}%
\pgfpathlineto{\pgfqpoint{1.472398in}{2.469675in}}%
\pgfpathlineto{\pgfqpoint{1.491905in}{2.418980in}}%
\pgfpathlineto{\pgfqpoint{1.510638in}{2.368198in}}%
\pgfpathlineto{\pgfqpoint{1.528209in}{2.317295in}}%
\pgfpathlineto{\pgfqpoint{1.544006in}{2.266223in}}%
\pgfpathlineto{\pgfqpoint{1.557068in}{2.214925in}}%
\pgfpathlineto{\pgfqpoint{1.565760in}{2.163362in}}%
\pgfpathlineto{\pgfqpoint{1.567545in}{2.111625in}}%
\pgfpathlineto{\pgfqpoint{1.559474in}{2.060138in}}%
\pgfpathlineto{\pgfqpoint{1.540498in}{2.009547in}}%
\pgfpathlineto{\pgfqpoint{1.513029in}{1.960115in}}%
\pgfusepath{stroke}%
\end{pgfscope}%
\begin{pgfscope}%
\pgfpathrectangle{\pgfqpoint{0.647939in}{0.492442in}}{\pgfqpoint{4.273799in}{2.331163in}}%
\pgfusepath{clip}%
\pgfsetbuttcap%
\pgfsetroundjoin%
\pgfsetlinewidth{0.301125pt}%
\definecolor{currentstroke}{rgb}{0.500000,0.500000,0.500000}%
\pgfsetstrokecolor{currentstroke}%
\pgfsetstrokeopacity{0.300000}%
\pgfsetdash{}{0pt}%
\pgfpathmoveto{\pgfqpoint{1.230730in}{2.823605in}}%
\pgfpathlineto{\pgfqpoint{1.230730in}{2.823605in}}%
\pgfpathlineto{\pgfqpoint{1.247914in}{2.772658in}}%
\pgfpathlineto{\pgfqpoint{1.264810in}{2.721682in}}%
\pgfpathlineto{\pgfqpoint{1.281348in}{2.670671in}}%
\pgfpathlineto{\pgfqpoint{1.297454in}{2.619619in}}%
\pgfpathlineto{\pgfqpoint{1.313025in}{2.568518in}}%
\pgfpathlineto{\pgfqpoint{1.327938in}{2.517359in}}%
\pgfpathlineto{\pgfqpoint{1.342052in}{2.466132in}}%
\pgfpathlineto{\pgfqpoint{1.355177in}{2.414828in}}%
\pgfpathlineto{\pgfqpoint{1.367083in}{2.363436in}}%
\pgfpathlineto{\pgfqpoint{1.377498in}{2.311948in}}%
\pgfpathlineto{\pgfqpoint{1.386076in}{2.260360in}}%
\pgfpathlineto{\pgfqpoint{1.392406in}{2.208677in}}%
\pgfpathlineto{\pgfqpoint{1.396016in}{2.156918in}}%
\pgfpathlineto{\pgfqpoint{1.396407in}{2.105125in}}%
\pgfpathlineto{\pgfqpoint{1.393104in}{2.053367in}}%
\pgfpathlineto{\pgfqpoint{1.385749in}{2.001734in}}%
\pgfpathlineto{\pgfqpoint{1.374214in}{1.950331in}}%
\pgfpathlineto{\pgfqpoint{1.358678in}{1.899246in}}%
\pgfpathlineto{\pgfqpoint{1.339552in}{1.848525in}}%
\pgfpathlineto{\pgfqpoint{1.317449in}{1.798164in}}%
\pgfusepath{stroke}%
\end{pgfscope}%
\begin{pgfscope}%
\pgfpathrectangle{\pgfqpoint{0.647939in}{0.492442in}}{\pgfqpoint{4.273799in}{2.331163in}}%
\pgfusepath{clip}%
\pgfsetbuttcap%
\pgfsetroundjoin%
\pgfsetlinewidth{0.301125pt}%
\definecolor{currentstroke}{rgb}{0.500000,0.500000,0.500000}%
\pgfsetstrokecolor{currentstroke}%
\pgfsetstrokeopacity{0.300000}%
\pgfsetdash{}{0pt}%
\pgfpathmoveto{\pgfqpoint{1.133598in}{2.823605in}}%
\pgfpathlineto{\pgfqpoint{1.133598in}{2.823605in}}%
\pgfpathlineto{\pgfqpoint{1.147843in}{2.772389in}}%
\pgfpathlineto{\pgfqpoint{1.161672in}{2.721138in}}%
\pgfpathlineto{\pgfqpoint{1.175020in}{2.669849in}}%
\pgfpathlineto{\pgfqpoint{1.187810in}{2.618519in}}%
\pgfpathlineto{\pgfqpoint{1.199958in}{2.567142in}}%
\pgfpathlineto{\pgfqpoint{1.211371in}{2.515715in}}%
\pgfpathlineto{\pgfqpoint{1.221931in}{2.464233in}}%
\pgfpathlineto{\pgfqpoint{1.231509in}{2.412696in}}%
\pgfpathlineto{\pgfqpoint{1.239963in}{2.361099in}}%
\pgfpathlineto{\pgfqpoint{1.247133in}{2.309445in}}%
\pgfpathlineto{\pgfqpoint{1.252839in}{2.257737in}}%
\pgfpathlineto{\pgfqpoint{1.256890in}{2.205983in}}%
\pgfpathlineto{\pgfqpoint{1.259084in}{2.154197in}}%
\pgfpathlineto{\pgfqpoint{1.259228in}{2.102397in}}%
\pgfpathlineto{\pgfqpoint{1.257141in}{2.050611in}}%
\pgfpathlineto{\pgfqpoint{1.252679in}{1.998869in}}%
\pgfpathlineto{\pgfqpoint{1.245749in}{1.947209in}}%
\pgfpathlineto{\pgfqpoint{1.236333in}{1.895667in}}%
\pgfpathlineto{\pgfqpoint{1.224484in}{1.844276in}}%
\pgfpathlineto{\pgfqpoint{1.210320in}{1.793059in}}%
\pgfpathlineto{\pgfqpoint{1.194040in}{1.742029in}}%
\pgfpathlineto{\pgfqpoint{1.175888in}{1.691188in}}%
\pgfpathlineto{\pgfqpoint{1.156113in}{1.640528in}}%
\pgfpathlineto{\pgfqpoint{1.134984in}{1.590031in}}%
\pgfpathlineto{\pgfqpoint{1.112746in}{1.539676in}}%
\pgfpathlineto{\pgfqpoint{1.089625in}{1.489440in}}%
\pgfpathlineto{\pgfqpoint{1.065823in}{1.439298in}}%
\pgfpathlineto{\pgfqpoint{1.041503in}{1.389228in}}%
\pgfpathlineto{\pgfqpoint{1.016816in}{1.339212in}}%
\pgfpathlineto{\pgfqpoint{0.991873in}{1.289233in}}%
\pgfpathlineto{\pgfqpoint{0.966775in}{1.239276in}}%
\pgfpathlineto{\pgfqpoint{0.941607in}{1.189332in}}%
\pgfpathlineto{\pgfqpoint{0.916423in}{1.139388in}}%
\pgfpathlineto{\pgfqpoint{0.891279in}{1.089440in}}%
\pgfpathlineto{\pgfqpoint{0.866207in}{1.039479in}}%
\pgfpathlineto{\pgfqpoint{0.841246in}{0.989501in}}%
\pgfpathlineto{\pgfqpoint{0.816419in}{0.939503in}}%
\pgfpathlineto{\pgfqpoint{0.791746in}{0.889482in}}%
\pgfpathlineto{\pgfqpoint{0.767243in}{0.839437in}}%
\pgfpathlineto{\pgfqpoint{0.742918in}{0.789365in}}%
\pgfusepath{stroke}%
\end{pgfscope}%
\begin{pgfscope}%
\pgfpathrectangle{\pgfqpoint{0.647939in}{0.492442in}}{\pgfqpoint{4.273799in}{2.331163in}}%
\pgfusepath{clip}%
\pgfsetbuttcap%
\pgfsetroundjoin%
\pgfsetlinewidth{0.301125pt}%
\definecolor{currentstroke}{rgb}{0.500000,0.500000,0.500000}%
\pgfsetstrokecolor{currentstroke}%
\pgfsetstrokeopacity{0.300000}%
\pgfsetdash{}{0pt}%
\pgfpathmoveto{\pgfqpoint{1.036466in}{2.823605in}}%
\pgfpathlineto{\pgfqpoint{1.036466in}{2.823605in}}%
\pgfpathlineto{\pgfqpoint{1.048390in}{2.772212in}}%
\pgfpathlineto{\pgfqpoint{1.059849in}{2.720788in}}%
\pgfpathlineto{\pgfqpoint{1.070792in}{2.669330in}}%
\pgfpathlineto{\pgfqpoint{1.081164in}{2.617837in}}%
\pgfpathlineto{\pgfqpoint{1.090897in}{2.566306in}}%
\pgfpathlineto{\pgfqpoint{1.099915in}{2.514738in}}%
\pgfpathlineto{\pgfqpoint{1.108143in}{2.463130in}}%
\pgfpathlineto{\pgfqpoint{1.115499in}{2.411483in}}%
\pgfpathlineto{\pgfqpoint{1.121890in}{2.359797in}}%
\pgfpathlineto{\pgfqpoint{1.127220in}{2.308077in}}%
\pgfpathlineto{\pgfqpoint{1.131385in}{2.256324in}}%
\pgfpathlineto{\pgfqpoint{1.134283in}{2.204546in}}%
\pgfpathlineto{\pgfqpoint{1.135813in}{2.152751in}}%
\pgfpathlineto{\pgfqpoint{1.135877in}{2.100949in}}%
\pgfpathlineto{\pgfqpoint{1.134387in}{2.049154in}}%
\pgfpathlineto{\pgfqpoint{1.131267in}{1.997380in}}%
\pgfpathlineto{\pgfqpoint{1.126463in}{1.945646in}}%
\pgfpathlineto{\pgfqpoint{1.119943in}{1.893967in}}%
\pgfpathlineto{\pgfqpoint{1.111709in}{1.842362in}}%
\pgfpathlineto{\pgfqpoint{1.101790in}{1.790845in}}%
\pgfpathlineto{\pgfqpoint{1.090242in}{1.739429in}}%
\pgfpathlineto{\pgfqpoint{1.077143in}{1.688124in}}%
\pgfpathlineto{\pgfqpoint{1.062607in}{1.636934in}}%
\pgfpathlineto{\pgfqpoint{1.046760in}{1.585861in}}%
\pgfpathlineto{\pgfqpoint{1.029726in}{1.534901in}}%
\pgfpathlineto{\pgfqpoint{1.011649in}{1.484049in}}%
\pgfusepath{stroke}%
\end{pgfscope}%
\begin{pgfscope}%
\pgfpathrectangle{\pgfqpoint{0.647939in}{0.492442in}}{\pgfqpoint{4.273799in}{2.331163in}}%
\pgfusepath{clip}%
\pgfsetbuttcap%
\pgfsetroundjoin%
\pgfsetlinewidth{0.301125pt}%
\definecolor{currentstroke}{rgb}{0.500000,0.500000,0.500000}%
\pgfsetstrokecolor{currentstroke}%
\pgfsetstrokeopacity{0.300000}%
\pgfsetdash{}{0pt}%
\pgfpathmoveto{\pgfqpoint{0.939334in}{2.823605in}}%
\pgfpathlineto{\pgfqpoint{0.939334in}{2.823605in}}%
\pgfpathlineto{\pgfqpoint{0.949403in}{2.772094in}}%
\pgfpathlineto{\pgfqpoint{0.959010in}{2.720556in}}%
\pgfpathlineto{\pgfqpoint{0.968110in}{2.668992in}}%
\pgfpathlineto{\pgfqpoint{0.976661in}{2.617399in}}%
\pgfpathlineto{\pgfqpoint{0.984616in}{2.565778in}}%
\pgfpathlineto{\pgfqpoint{0.991923in}{2.514128in}}%
\pgfpathlineto{\pgfqpoint{0.998529in}{2.462451in}}%
\pgfpathlineto{\pgfqpoint{1.004374in}{2.410746in}}%
\pgfpathlineto{\pgfqpoint{1.009401in}{2.359016in}}%
\pgfpathlineto{\pgfqpoint{1.013548in}{2.307262in}}%
\pgfpathlineto{\pgfqpoint{1.016756in}{2.255489in}}%
\pgfpathlineto{\pgfqpoint{1.018963in}{2.203700in}}%
\pgfpathlineto{\pgfqpoint{1.020110in}{2.151901in}}%
\pgfpathlineto{\pgfqpoint{1.020142in}{2.100099in}}%
\pgfpathlineto{\pgfqpoint{1.019008in}{2.048300in}}%
\pgfpathlineto{\pgfqpoint{1.016665in}{1.996513in}}%
\pgfpathlineto{\pgfqpoint{1.013079in}{1.944748in}}%
\pgfpathlineto{\pgfqpoint{1.008228in}{1.893014in}}%
\pgfpathlineto{\pgfqpoint{1.002099in}{1.841320in}}%
\pgfpathlineto{\pgfqpoint{0.994696in}{1.789675in}}%
\pgfpathlineto{\pgfqpoint{0.986038in}{1.738089in}}%
\pgfpathlineto{\pgfqpoint{0.976158in}{1.686568in}}%
\pgfpathlineto{\pgfqpoint{0.965102in}{1.635119in}}%
\pgfpathlineto{\pgfqpoint{0.952919in}{1.583746in}}%
\pgfpathlineto{\pgfqpoint{0.939680in}{1.532450in}}%
\pgfpathlineto{\pgfqpoint{0.925460in}{1.481232in}}%
\pgfpathlineto{\pgfqpoint{0.910334in}{1.430092in}}%
\pgfpathlineto{\pgfqpoint{0.894382in}{1.379027in}}%
\pgfpathlineto{\pgfqpoint{0.877690in}{1.328032in}}%
\pgfpathlineto{\pgfqpoint{0.860332in}{1.277103in}}%
\pgfpathlineto{\pgfqpoint{0.842390in}{1.226235in}}%
\pgfpathlineto{\pgfqpoint{0.823934in}{1.175421in}}%
\pgfpathlineto{\pgfqpoint{0.805034in}{1.124656in}}%
\pgfpathlineto{\pgfqpoint{0.785755in}{1.073933in}}%
\pgfpathlineto{\pgfqpoint{0.766153in}{1.023248in}}%
\pgfpathlineto{\pgfqpoint{0.746282in}{0.972593in}}%
\pgfpathlineto{\pgfqpoint{0.726189in}{0.921965in}}%
\pgfusepath{stroke}%
\end{pgfscope}%
\begin{pgfscope}%
\pgfpathrectangle{\pgfqpoint{0.647939in}{0.492442in}}{\pgfqpoint{4.273799in}{2.331163in}}%
\pgfusepath{clip}%
\pgfsetbuttcap%
\pgfsetroundjoin%
\pgfsetlinewidth{0.301125pt}%
\definecolor{currentstroke}{rgb}{0.500000,0.500000,0.500000}%
\pgfsetstrokecolor{currentstroke}%
\pgfsetstrokeopacity{0.300000}%
\pgfsetdash{}{0pt}%
\pgfpathmoveto{\pgfqpoint{0.842203in}{2.823605in}}%
\pgfpathlineto{\pgfqpoint{0.842203in}{2.823605in}}%
\pgfpathlineto{\pgfqpoint{0.850776in}{2.772014in}}%
\pgfpathlineto{\pgfqpoint{0.858910in}{2.720401in}}%
\pgfpathlineto{\pgfqpoint{0.866571in}{2.668766in}}%
\pgfpathlineto{\pgfqpoint{0.873726in}{2.617110in}}%
\pgfpathlineto{\pgfqpoint{0.880337in}{2.565433in}}%
\pgfpathlineto{\pgfqpoint{0.886369in}{2.513734in}}%
\pgfpathlineto{\pgfqpoint{0.891785in}{2.462015in}}%
\pgfpathlineto{\pgfqpoint{0.896547in}{2.410277in}}%
\pgfpathlineto{\pgfqpoint{0.900615in}{2.358521in}}%
\pgfpathlineto{\pgfqpoint{0.903949in}{2.306750in}}%
\pgfpathlineto{\pgfqpoint{0.906510in}{2.254966in}}%
\pgfpathlineto{\pgfqpoint{0.908259in}{2.203172in}}%
\pgfpathlineto{\pgfqpoint{0.909159in}{2.151371in}}%
\pgfpathlineto{\pgfqpoint{0.909176in}{2.099568in}}%
\pgfpathlineto{\pgfqpoint{0.908278in}{2.047767in}}%
\pgfpathlineto{\pgfqpoint{0.906439in}{1.995974in}}%
\pgfpathlineto{\pgfqpoint{0.903636in}{1.944194in}}%
\pgfpathlineto{\pgfqpoint{0.899853in}{1.892432in}}%
\pgfpathlineto{\pgfqpoint{0.895081in}{1.840695in}}%
\pgfpathlineto{\pgfqpoint{0.889317in}{1.788988in}}%
\pgfpathlineto{\pgfqpoint{0.882565in}{1.737316in}}%
\pgfpathlineto{\pgfqpoint{0.874837in}{1.685686in}}%
\pgfpathlineto{\pgfqpoint{0.866150in}{1.634100in}}%
\pgfpathlineto{\pgfqpoint{0.856531in}{1.582564in}}%
\pgfpathlineto{\pgfqpoint{0.846018in}{1.531080in}}%
\pgfpathlineto{\pgfqpoint{0.834648in}{1.479651in}}%
\pgfpathlineto{\pgfqpoint{0.822459in}{1.428277in}}%
\pgfpathlineto{\pgfqpoint{0.809503in}{1.376959in}}%
\pgfusepath{stroke}%
\end{pgfscope}%
\begin{pgfscope}%
\pgfpathrectangle{\pgfqpoint{0.647939in}{0.492442in}}{\pgfqpoint{4.273799in}{2.331163in}}%
\pgfusepath{clip}%
\pgfsetbuttcap%
\pgfsetroundjoin%
\pgfsetlinewidth{0.301125pt}%
\definecolor{currentstroke}{rgb}{0.500000,0.500000,0.500000}%
\pgfsetstrokecolor{currentstroke}%
\pgfsetstrokeopacity{0.300000}%
\pgfsetdash{}{0pt}%
\pgfpathmoveto{\pgfqpoint{0.745071in}{2.823605in}}%
\pgfpathlineto{\pgfqpoint{0.745071in}{2.823605in}}%
\pgfpathlineto{\pgfqpoint{0.752432in}{2.771958in}}%
\pgfpathlineto{\pgfqpoint{0.759382in}{2.720293in}}%
\pgfpathlineto{\pgfqpoint{0.765898in}{2.668612in}}%
\pgfpathlineto{\pgfqpoint{0.771954in}{2.616914in}}%
\pgfpathlineto{\pgfqpoint{0.777526in}{2.565200in}}%
\pgfpathlineto{\pgfqpoint{0.782585in}{2.513470in}}%
\pgfpathlineto{\pgfqpoint{0.787105in}{2.461726in}}%
\pgfpathlineto{\pgfqpoint{0.791059in}{2.409967in}}%
\pgfpathlineto{\pgfqpoint{0.794420in}{2.358196in}}%
\pgfpathlineto{\pgfqpoint{0.797161in}{2.306415in}}%
\pgfpathlineto{\pgfqpoint{0.799256in}{2.254624in}}%
\pgfpathlineto{\pgfqpoint{0.800680in}{2.202827in}}%
\pgfpathlineto{\pgfqpoint{0.801407in}{2.151025in}}%
\pgfpathlineto{\pgfqpoint{0.801417in}{2.099222in}}%
\pgfpathlineto{\pgfqpoint{0.800688in}{2.047420in}}%
\pgfpathlineto{\pgfqpoint{0.799202in}{1.995623in}}%
\pgfpathlineto{\pgfqpoint{0.796944in}{1.943835in}}%
\pgfpathlineto{\pgfqpoint{0.793902in}{1.892058in}}%
\pgfpathlineto{\pgfqpoint{0.790068in}{1.840298in}}%
\pgfpathlineto{\pgfqpoint{0.785438in}{1.788556in}}%
\pgfpathlineto{\pgfqpoint{0.780011in}{1.736838in}}%
\pgfpathlineto{\pgfqpoint{0.773791in}{1.685146in}}%
\pgfpathlineto{\pgfqpoint{0.766788in}{1.633484in}}%
\pgfpathlineto{\pgfqpoint{0.759017in}{1.581855in}}%
\pgfpathlineto{\pgfqpoint{0.750492in}{1.530261in}}%
\pgfpathlineto{\pgfqpoint{0.741234in}{1.478705in}}%
\pgfpathlineto{\pgfqpoint{0.731266in}{1.427188in}}%
\pgfpathlineto{\pgfqpoint{0.720618in}{1.375712in}}%
\pgfpathlineto{\pgfqpoint{0.709321in}{1.324277in}}%
\pgfpathlineto{\pgfqpoint{0.697406in}{1.272884in}}%
\pgfpathlineto{\pgfqpoint{0.684904in}{1.221532in}}%
\pgfpathlineto{\pgfqpoint{0.671854in}{1.170221in}}%
\pgfpathlineto{\pgfqpoint{0.658291in}{1.118949in}}%
\pgfpathlineto{\pgfqpoint{0.647939in}{1.080528in}}%
\pgfusepath{stroke}%
\end{pgfscope}%
\begin{pgfscope}%
\pgfpathrectangle{\pgfqpoint{0.647939in}{0.492442in}}{\pgfqpoint{4.273799in}{2.331163in}}%
\pgfusepath{clip}%
\pgfsetbuttcap%
\pgfsetroundjoin%
\pgfsetlinewidth{0.301125pt}%
\definecolor{currentstroke}{rgb}{0.500000,0.500000,0.500000}%
\pgfsetstrokecolor{currentstroke}%
\pgfsetstrokeopacity{0.300000}%
\pgfsetdash{}{0pt}%
\pgfpathmoveto{\pgfqpoint{0.647939in}{2.823605in}}%
\pgfpathlineto{\pgfqpoint{0.647939in}{2.823605in}}%
\pgfpathlineto{\pgfqpoint{0.654307in}{2.771918in}}%
\pgfpathlineto{\pgfqpoint{0.660296in}{2.720218in}}%
\pgfpathlineto{\pgfqpoint{0.665891in}{2.668505in}}%
\pgfpathlineto{\pgfqpoint{0.671071in}{2.616778in}}%
\pgfpathlineto{\pgfqpoint{0.675818in}{2.565040in}}%
\pgfpathlineto{\pgfqpoint{0.680113in}{2.513289in}}%
\pgfpathlineto{\pgfqpoint{0.683937in}{2.461528in}}%
\pgfpathlineto{\pgfqpoint{0.687271in}{2.409756in}}%
\pgfpathlineto{\pgfqpoint{0.690095in}{2.357976in}}%
\pgfpathlineto{\pgfqpoint{0.692390in}{2.306188in}}%
\pgfpathlineto{\pgfqpoint{0.694138in}{2.254393in}}%
\pgfpathlineto{\pgfqpoint{0.695321in}{2.202594in}}%
\pgfpathlineto{\pgfqpoint{0.695923in}{2.150791in}}%
\pgfpathlineto{\pgfqpoint{0.695929in}{2.098988in}}%
\pgfpathlineto{\pgfqpoint{0.695323in}{2.047186in}}%
\pgfpathlineto{\pgfqpoint{0.694095in}{1.995387in}}%
\pgfpathlineto{\pgfqpoint{0.692233in}{1.943594in}}%
\pgfpathlineto{\pgfqpoint{0.689728in}{1.891808in}}%
\pgfpathlineto{\pgfqpoint{0.686575in}{1.840034in}}%
\pgfpathlineto{\pgfqpoint{0.682769in}{1.788272in}}%
\pgfpathlineto{\pgfqpoint{0.678309in}{1.736526in}}%
\pgfpathlineto{\pgfqpoint{0.673196in}{1.684798in}}%
\pgfpathlineto{\pgfqpoint{0.667434in}{1.633090in}}%
\pgfpathlineto{\pgfqpoint{0.661029in}{1.581405in}}%
\pgfpathlineto{\pgfqpoint{0.653989in}{1.529744in}}%
\pgfpathlineto{\pgfqpoint{0.647939in}{1.487253in}}%
\pgfusepath{stroke}%
\end{pgfscope}%
\begin{pgfscope}%
\pgfpathrectangle{\pgfqpoint{0.647939in}{0.492442in}}{\pgfqpoint{4.273799in}{2.331163in}}%
\pgfusepath{clip}%
\pgfsetbuttcap%
\pgfsetroundjoin%
\pgfsetlinewidth{0.301125pt}%
\definecolor{currentstroke}{rgb}{0.500000,0.500000,0.500000}%
\pgfsetstrokecolor{currentstroke}%
\pgfsetstrokeopacity{0.300000}%
\pgfsetdash{}{0pt}%
\pgfpathmoveto{\pgfqpoint{0.745071in}{2.346777in}}%
\pgfpathlineto{\pgfqpoint{0.747456in}{2.294990in}}%
\pgfpathlineto{\pgfqpoint{0.749240in}{2.243195in}}%
\pgfpathlineto{\pgfqpoint{0.750403in}{2.191396in}}%
\pgfpathlineto{\pgfqpoint{0.750925in}{2.139593in}}%
\pgfpathlineto{\pgfqpoint{0.750787in}{2.087790in}}%
\pgfpathlineto{\pgfqpoint{0.749972in}{2.035989in}}%
\pgfpathlineto{\pgfqpoint{0.748467in}{1.984192in}}%
\pgfpathlineto{\pgfqpoint{0.746258in}{1.932403in}}%
\pgfpathlineto{\pgfqpoint{0.743336in}{1.880624in}}%
\pgfusepath{stroke}%
\end{pgfscope}%
\begin{pgfscope}%
\pgfpathrectangle{\pgfqpoint{0.647939in}{0.492442in}}{\pgfqpoint{4.273799in}{2.331163in}}%
\pgfusepath{clip}%
\pgfsetbuttcap%
\pgfsetroundjoin%
\pgfsetlinewidth{0.301125pt}%
\definecolor{currentstroke}{rgb}{0.500000,0.500000,0.500000}%
\pgfsetstrokecolor{currentstroke}%
\pgfsetstrokeopacity{0.300000}%
\pgfsetdash{}{0pt}%
\pgfpathmoveto{\pgfqpoint{4.436079in}{0.598404in}}%
\pgfpathlineto{\pgfqpoint{4.391557in}{0.644153in}}%
\pgfpathlineto{\pgfqpoint{4.344509in}{0.689140in}}%
\pgfpathlineto{\pgfqpoint{4.294626in}{0.733207in}}%
\pgfpathlineto{\pgfqpoint{4.241579in}{0.776156in}}%
\pgfpathlineto{\pgfqpoint{4.185033in}{0.817756in}}%
\pgfpathlineto{\pgfqpoint{4.124756in}{0.857762in}}%
\pgfpathlineto{\pgfqpoint{4.060601in}{0.895927in}}%
\pgfpathlineto{\pgfqpoint{3.992609in}{0.932063in}}%
\pgfpathlineto{\pgfqpoint{3.921095in}{0.966120in}}%
\pgfpathlineto{\pgfqpoint{3.846607in}{0.998234in}}%
\pgfpathlineto{\pgfqpoint{3.769937in}{1.028793in}}%
\pgfpathlineto{\pgfqpoint{3.691985in}{1.058382in}}%
\pgfpathlineto{\pgfqpoint{3.613657in}{1.087674in}}%
\pgfpathlineto{\pgfqpoint{3.535815in}{1.117343in}}%
\pgfpathlineto{\pgfqpoint{3.459238in}{1.147969in}}%
\pgfpathlineto{\pgfqpoint{3.384606in}{1.179981in}}%
\pgfpathlineto{\pgfqpoint{3.312435in}{1.213620in}}%
\pgfpathlineto{\pgfqpoint{3.243073in}{1.248976in}}%
\pgfpathlineto{\pgfqpoint{3.176761in}{1.286030in}}%
\pgfpathlineto{\pgfqpoint{3.113597in}{1.324687in}}%
\pgfpathlineto{\pgfqpoint{3.053593in}{1.364817in}}%
\pgfpathlineto{\pgfqpoint{2.996701in}{1.406272in}}%
\pgfusepath{stroke}%
\end{pgfscope}%
\begin{pgfscope}%
\pgfpathrectangle{\pgfqpoint{0.647939in}{0.492442in}}{\pgfqpoint{4.273799in}{2.331163in}}%
\pgfusepath{clip}%
\pgfsetbuttcap%
\pgfsetroundjoin%
\pgfsetlinewidth{0.301125pt}%
\definecolor{currentstroke}{rgb}{0.500000,0.500000,0.500000}%
\pgfsetstrokecolor{currentstroke}%
\pgfsetstrokeopacity{0.300000}%
\pgfsetdash{}{0pt}%
\pgfpathmoveto{\pgfqpoint{4.047552in}{0.651385in}}%
\pgfpathlineto{\pgfqpoint{3.983919in}{0.689827in}}%
\pgfpathlineto{\pgfqpoint{3.917835in}{0.727022in}}%
\pgfpathlineto{\pgfqpoint{3.849662in}{0.763081in}}%
\pgfpathlineto{\pgfqpoint{3.779878in}{0.798215in}}%
\pgfpathlineto{\pgfqpoint{3.709029in}{0.832712in}}%
\pgfpathlineto{\pgfqpoint{3.637706in}{0.866918in}}%
\pgfpathlineto{\pgfqpoint{3.566505in}{0.901199in}}%
\pgfpathlineto{\pgfqpoint{3.495961in}{0.935879in}}%
\pgfusepath{stroke}%
\end{pgfscope}%
\begin{pgfscope}%
\pgfpathrectangle{\pgfqpoint{0.647939in}{0.492442in}}{\pgfqpoint{4.273799in}{2.331163in}}%
\pgfusepath{clip}%
\pgfsetbuttcap%
\pgfsetroundjoin%
\pgfsetlinewidth{0.301125pt}%
\definecolor{currentstroke}{rgb}{0.500000,0.500000,0.500000}%
\pgfsetstrokecolor{currentstroke}%
\pgfsetstrokeopacity{0.300000}%
\pgfsetdash{}{0pt}%
\pgfpathmoveto{\pgfqpoint{4.646637in}{1.765935in}}%
\pgfpathlineto{\pgfqpoint{4.630343in}{1.816967in}}%
\pgfpathlineto{\pgfqpoint{4.614949in}{1.868081in}}%
\pgfpathlineto{\pgfqpoint{4.600790in}{1.919301in}}%
\pgfpathlineto{\pgfqpoint{4.588335in}{1.970650in}}%
\pgfpathlineto{\pgfqpoint{4.578230in}{2.022152in}}%
\pgfpathlineto{\pgfqpoint{4.571395in}{2.073810in}}%
\pgfpathlineto{\pgfqpoint{4.569013in}{2.125575in}}%
\pgfpathlineto{\pgfqpoint{4.572342in}{2.177311in}}%
\pgfpathlineto{\pgfqpoint{4.582233in}{2.228787in}}%
\pgfpathlineto{\pgfqpoint{4.598507in}{2.279767in}}%
\pgfusepath{stroke}%
\end{pgfscope}%
\begin{pgfscope}%
\pgfpathrectangle{\pgfqpoint{0.647939in}{0.492442in}}{\pgfqpoint{4.273799in}{2.331163in}}%
\pgfusepath{clip}%
\pgfsetbuttcap%
\pgfsetroundjoin%
\pgfsetlinewidth{0.301125pt}%
\definecolor{currentstroke}{rgb}{0.500000,0.500000,0.500000}%
\pgfsetstrokecolor{currentstroke}%
\pgfsetstrokeopacity{0.300000}%
\pgfsetdash{}{0pt}%
\pgfpathmoveto{\pgfqpoint{4.565481in}{1.291422in}}%
\pgfpathlineto{\pgfqpoint{4.533211in}{1.340138in}}%
\pgfpathlineto{\pgfqpoint{4.498972in}{1.388450in}}%
\pgfpathlineto{\pgfqpoint{4.462203in}{1.436194in}}%
\pgfpathlineto{\pgfqpoint{4.422012in}{1.483096in}}%
\pgfpathlineto{\pgfqpoint{4.376959in}{1.528634in}}%
\pgfpathlineto{\pgfqpoint{4.324549in}{1.571694in}}%
\pgfpathlineto{\pgfqpoint{4.260427in}{1.609493in}}%
\pgfpathlineto{\pgfqpoint{4.260427in}{1.609493in}}%
\pgfpathlineto{\pgfqpoint{4.203098in}{1.630064in}}%
\pgfpathlineto{\pgfqpoint{4.135246in}{1.640173in}}%
\pgfpathlineto{\pgfqpoint{4.074414in}{1.639058in}}%
\pgfpathlineto{\pgfqpoint{4.010898in}{1.630661in}}%
\pgfpathlineto{\pgfqpoint{3.934151in}{1.614671in}}%
\pgfpathlineto{\pgfqpoint{3.847422in}{1.593684in}}%
\pgfpathlineto{\pgfqpoint{3.760183in}{1.573311in}}%
\pgfpathlineto{\pgfqpoint{3.670769in}{1.556054in}}%
\pgfusepath{stroke}%
\end{pgfscope}%
\begin{pgfscope}%
\pgfpathrectangle{\pgfqpoint{0.647939in}{0.492442in}}{\pgfqpoint{4.273799in}{2.331163in}}%
\pgfusepath{clip}%
\pgfsetbuttcap%
\pgfsetroundjoin%
\pgfsetlinewidth{0.301125pt}%
\definecolor{currentstroke}{rgb}{0.500000,0.500000,0.500000}%
\pgfsetstrokecolor{currentstroke}%
\pgfsetstrokeopacity{0.300000}%
\pgfsetdash{}{0pt}%
\pgfpathmoveto{\pgfqpoint{4.533211in}{1.605043in}}%
\pgfpathlineto{\pgfqpoint{4.503313in}{1.654202in}}%
\pgfpathlineto{\pgfqpoint{4.471210in}{1.702930in}}%
\pgfpathlineto{\pgfqpoint{4.435692in}{1.750929in}}%
\pgfpathlineto{\pgfqpoint{4.394203in}{1.797423in}}%
\pgfpathlineto{\pgfqpoint{4.340090in}{1.839341in}}%
\pgfpathlineto{\pgfqpoint{4.340090in}{1.839341in}}%
\pgfpathlineto{\pgfqpoint{4.303045in}{1.854588in}}%
\pgfpathlineto{\pgfqpoint{4.303045in}{1.854588in}}%
\pgfpathlineto{\pgfqpoint{4.265405in}{1.859164in}}%
\pgfpathlineto{\pgfqpoint{4.227390in}{1.854501in}}%
\pgfpathlineto{\pgfqpoint{4.190757in}{1.843406in}}%
\pgfpathlineto{\pgfqpoint{4.146103in}{1.824097in}}%
\pgfusepath{stroke}%
\end{pgfscope}%
\begin{pgfscope}%
\pgfpathrectangle{\pgfqpoint{0.647939in}{0.492442in}}{\pgfqpoint{4.273799in}{2.331163in}}%
\pgfusepath{clip}%
\pgfsetbuttcap%
\pgfsetroundjoin%
\pgfsetlinewidth{0.301125pt}%
\definecolor{currentstroke}{rgb}{0.500000,0.500000,0.500000}%
\pgfsetstrokecolor{currentstroke}%
\pgfsetstrokeopacity{0.300000}%
\pgfsetdash{}{0pt}%
\pgfpathmoveto{\pgfqpoint{1.825907in}{2.823605in}}%
\pgfpathlineto{\pgfqpoint{1.856319in}{2.801544in}}%
\pgfpathlineto{\pgfqpoint{1.916170in}{2.761389in}}%
\pgfpathlineto{\pgfqpoint{1.982204in}{2.724257in}}%
\pgfpathlineto{\pgfqpoint{2.054658in}{2.690893in}}%
\pgfpathlineto{\pgfqpoint{2.132844in}{2.661603in}}%
\pgfpathlineto{\pgfqpoint{2.215093in}{2.635754in}}%
\pgfpathlineto{\pgfqpoint{2.299180in}{2.611681in}}%
\pgfusepath{stroke}%
\end{pgfscope}%
\begin{pgfscope}%
\pgfpathrectangle{\pgfqpoint{0.647939in}{0.492442in}}{\pgfqpoint{4.273799in}{2.331163in}}%
\pgfusepath{clip}%
\pgfsetbuttcap%
\pgfsetroundjoin%
\pgfsetlinewidth{0.301125pt}%
\definecolor{currentstroke}{rgb}{0.500000,0.500000,0.500000}%
\pgfsetstrokecolor{currentstroke}%
\pgfsetstrokeopacity{0.300000}%
\pgfsetdash{}{0pt}%
\pgfpathmoveto{\pgfqpoint{1.629223in}{0.671899in}}%
\pgfpathlineto{\pgfqpoint{1.577775in}{0.715419in}}%
\pgfpathlineto{\pgfqpoint{1.522125in}{0.757347in}}%
\pgfpathlineto{\pgfqpoint{1.460396in}{0.796616in}}%
\pgfpathlineto{\pgfqpoint{1.389513in}{0.830736in}}%
\pgfpathlineto{\pgfqpoint{1.389513in}{0.830736in}}%
\pgfpathlineto{\pgfqpoint{1.324964in}{0.850308in}}%
\pgfpathlineto{\pgfqpoint{1.324964in}{0.850308in}}%
\pgfpathlineto{\pgfqpoint{1.266358in}{0.857621in}}%
\pgfpathlineto{\pgfqpoint{1.206130in}{0.854135in}}%
\pgfusepath{stroke}%
\end{pgfscope}%
\begin{pgfscope}%
\pgfpathrectangle{\pgfqpoint{0.647939in}{0.492442in}}{\pgfqpoint{4.273799in}{2.331163in}}%
\pgfusepath{clip}%
\pgfsetbuttcap%
\pgfsetroundjoin%
\pgfsetlinewidth{0.301125pt}%
\definecolor{currentstroke}{rgb}{0.500000,0.500000,0.500000}%
\pgfsetstrokecolor{currentstroke}%
\pgfsetstrokeopacity{0.300000}%
\pgfsetdash{}{0pt}%
\pgfpathmoveto{\pgfqpoint{2.493443in}{0.757347in}}%
\pgfpathlineto{\pgfqpoint{2.454337in}{0.804555in}}%
\pgfpathlineto{\pgfqpoint{2.415864in}{0.851917in}}%
\pgfpathlineto{\pgfqpoint{2.377991in}{0.899423in}}%
\pgfpathlineto{\pgfqpoint{2.340683in}{0.947062in}}%
\pgfpathlineto{\pgfqpoint{2.303904in}{0.994823in}}%
\pgfpathlineto{\pgfqpoint{2.267623in}{1.042698in}}%
\pgfpathlineto{\pgfqpoint{2.231804in}{1.090676in}}%
\pgfpathlineto{\pgfqpoint{2.196415in}{1.138748in}}%
\pgfpathlineto{\pgfqpoint{2.161415in}{1.186906in}}%
\pgfpathlineto{\pgfqpoint{2.126756in}{1.235136in}}%
\pgfpathlineto{\pgfqpoint{2.092383in}{1.283428in}}%
\pgfpathlineto{\pgfqpoint{2.058247in}{1.331770in}}%
\pgfpathlineto{\pgfqpoint{2.024287in}{1.380148in}}%
\pgfpathlineto{\pgfqpoint{1.990425in}{1.428547in}}%
\pgfpathlineto{\pgfqpoint{1.956549in}{1.476942in}}%
\pgfpathlineto{\pgfqpoint{1.922525in}{1.525306in}}%
\pgfpathlineto{\pgfqpoint{1.888180in}{1.573602in}}%
\pgfpathlineto{\pgfqpoint{1.853267in}{1.621776in}}%
\pgfpathlineto{\pgfqpoint{1.817380in}{1.669736in}}%
\pgfpathlineto{\pgfqpoint{1.779833in}{1.717310in}}%
\pgfpathlineto{\pgfqpoint{1.739349in}{1.764141in}}%
\pgfpathlineto{\pgfqpoint{1.693074in}{1.809266in}}%
\pgfpathlineto{\pgfqpoint{1.632467in}{1.848452in}}%
\pgfpathlineto{\pgfqpoint{1.632467in}{1.848452in}}%
\pgfpathlineto{\pgfqpoint{1.597385in}{1.858138in}}%
\pgfpathlineto{\pgfqpoint{1.597385in}{1.858138in}}%
\pgfusepath{stroke}%
\end{pgfscope}%
\begin{pgfscope}%
\pgfpathrectangle{\pgfqpoint{0.647939in}{0.492442in}}{\pgfqpoint{4.273799in}{2.331163in}}%
\pgfusepath{clip}%
\pgfsetbuttcap%
\pgfsetroundjoin%
\pgfsetlinewidth{0.301125pt}%
\definecolor{currentstroke}{rgb}{0.500000,0.500000,0.500000}%
\pgfsetstrokecolor{currentstroke}%
\pgfsetstrokeopacity{0.300000}%
\pgfsetdash{}{0pt}%
\pgfpathmoveto{\pgfqpoint{3.367630in}{0.757347in}}%
\pgfpathlineto{\pgfqpoint{3.305957in}{0.796737in}}%
\pgfpathlineto{\pgfqpoint{3.245788in}{0.836811in}}%
\pgfpathlineto{\pgfqpoint{3.187238in}{0.877593in}}%
\pgfpathlineto{\pgfqpoint{3.130394in}{0.919086in}}%
\pgfpathlineto{\pgfqpoint{3.075300in}{0.961275in}}%
\pgfpathlineto{\pgfqpoint{3.021970in}{1.004134in}}%
\pgfpathlineto{\pgfqpoint{2.970400in}{1.047631in}}%
\pgfusepath{stroke}%
\end{pgfscope}%
\begin{pgfscope}%
\pgfpathrectangle{\pgfqpoint{0.647939in}{0.492442in}}{\pgfqpoint{4.273799in}{2.331163in}}%
\pgfusepath{clip}%
\pgfsetbuttcap%
\pgfsetroundjoin%
\pgfsetlinewidth{0.301125pt}%
\definecolor{currentstroke}{rgb}{0.500000,0.500000,0.500000}%
\pgfsetstrokecolor{currentstroke}%
\pgfsetstrokeopacity{0.300000}%
\pgfsetdash{}{0pt}%
\pgfpathmoveto{\pgfqpoint{3.561893in}{0.757347in}}%
\pgfpathlineto{\pgfqpoint{3.494799in}{0.794009in}}%
\pgfpathlineto{\pgfqpoint{3.428655in}{0.831179in}}%
\pgfpathlineto{\pgfqpoint{3.363785in}{0.869009in}}%
\pgfpathlineto{\pgfqpoint{3.300445in}{0.907601in}}%
\pgfpathlineto{\pgfqpoint{3.238828in}{0.947014in}}%
\pgfpathlineto{\pgfqpoint{3.179073in}{0.987271in}}%
\pgfpathlineto{\pgfqpoint{3.121267in}{1.028363in}}%
\pgfpathlineto{\pgfqpoint{3.065444in}{1.070265in}}%
\pgfpathlineto{\pgfqpoint{3.011613in}{1.112937in}}%
\pgfpathlineto{\pgfqpoint{2.959763in}{1.156331in}}%
\pgfpathlineto{\pgfqpoint{2.909858in}{1.200397in}}%
\pgfpathlineto{\pgfqpoint{2.861855in}{1.245090in}}%
\pgfpathlineto{\pgfqpoint{2.815709in}{1.290361in}}%
\pgfpathlineto{\pgfqpoint{2.771376in}{1.336169in}}%
\pgfpathlineto{\pgfqpoint{2.728814in}{1.382474in}}%
\pgfpathlineto{\pgfqpoint{2.687990in}{1.429243in}}%
\pgfpathlineto{\pgfqpoint{2.648880in}{1.476446in}}%
\pgfpathlineto{\pgfqpoint{2.611473in}{1.524058in}}%
\pgfpathlineto{\pgfqpoint{2.575776in}{1.572059in}}%
\pgfpathlineto{\pgfqpoint{2.541819in}{1.620434in}}%
\pgfpathlineto{\pgfqpoint{2.509649in}{1.669172in}}%
\pgfpathlineto{\pgfqpoint{2.479339in}{1.718263in}}%
\pgfpathlineto{\pgfqpoint{2.451006in}{1.767703in}}%
\pgfpathlineto{\pgfqpoint{2.424821in}{1.817494in}}%
\pgfpathlineto{\pgfqpoint{2.401010in}{1.867637in}}%
\pgfpathlineto{\pgfqpoint{2.379896in}{1.918137in}}%
\pgfpathlineto{\pgfqpoint{2.361925in}{1.968994in}}%
\pgfpathlineto{\pgfqpoint{2.347722in}{2.020201in}}%
\pgfpathlineto{\pgfqpoint{2.338198in}{2.071722in}}%
\pgfpathlineto{\pgfqpoint{2.334674in}{2.123452in}}%
\pgfpathlineto{\pgfqpoint{2.339142in}{2.175126in}}%
\pgfpathlineto{\pgfqpoint{2.354704in}{2.226077in}}%
\pgfpathlineto{\pgfqpoint{2.386158in}{2.274604in}}%
\pgfpathlineto{\pgfqpoint{2.386158in}{2.274604in}}%
\pgfpathlineto{\pgfqpoint{2.424554in}{2.307443in}}%
\pgfpathlineto{\pgfqpoint{2.424554in}{2.307443in}}%
\pgfpathlineto{\pgfqpoint{2.469158in}{2.329443in}}%
\pgfpathlineto{\pgfqpoint{2.524800in}{2.342343in}}%
\pgfpathlineto{\pgfqpoint{2.577918in}{2.344043in}}%
\pgfpathlineto{\pgfqpoint{2.629227in}{2.337370in}}%
\pgfusepath{stroke}%
\end{pgfscope}%
\begin{pgfscope}%
\pgfpathrectangle{\pgfqpoint{0.647939in}{0.492442in}}{\pgfqpoint{4.273799in}{2.331163in}}%
\pgfusepath{clip}%
\pgfsetbuttcap%
\pgfsetroundjoin%
\pgfsetlinewidth{0.301125pt}%
\definecolor{currentstroke}{rgb}{0.500000,0.500000,0.500000}%
\pgfsetstrokecolor{currentstroke}%
\pgfsetstrokeopacity{0.300000}%
\pgfsetdash{}{0pt}%
\pgfpathmoveto{\pgfqpoint{3.659025in}{0.757347in}}%
\pgfpathlineto{\pgfqpoint{3.590134in}{0.793005in}}%
\pgfpathlineto{\pgfqpoint{3.521671in}{0.828906in}}%
\pgfpathlineto{\pgfqpoint{3.454060in}{0.865283in}}%
\pgfpathlineto{\pgfqpoint{3.387681in}{0.902326in}}%
\pgfpathlineto{\pgfqpoint{3.322841in}{0.940170in}}%
\pgfusepath{stroke}%
\end{pgfscope}%
\begin{pgfscope}%
\pgfpathrectangle{\pgfqpoint{0.647939in}{0.492442in}}{\pgfqpoint{4.273799in}{2.331163in}}%
\pgfusepath{clip}%
\pgfsetbuttcap%
\pgfsetroundjoin%
\pgfsetlinewidth{0.301125pt}%
\definecolor{currentstroke}{rgb}{0.500000,0.500000,0.500000}%
\pgfsetstrokecolor{currentstroke}%
\pgfsetstrokeopacity{0.300000}%
\pgfsetdash{}{0pt}%
\pgfpathmoveto{\pgfqpoint{4.436079in}{1.181195in}}%
\pgfpathlineto{\pgfqpoint{4.392160in}{1.227103in}}%
\pgfpathlineto{\pgfqpoint{4.344009in}{1.271715in}}%
\pgfpathlineto{\pgfqpoint{4.290356in}{1.314387in}}%
\pgfpathlineto{\pgfqpoint{4.229446in}{1.354012in}}%
\pgfpathlineto{\pgfqpoint{4.159400in}{1.388736in}}%
\pgfpathlineto{\pgfqpoint{4.079241in}{1.415945in}}%
\pgfpathlineto{\pgfqpoint{3.990974in}{1.433499in}}%
\pgfpathlineto{\pgfqpoint{3.901029in}{1.442016in}}%
\pgfpathlineto{\pgfqpoint{3.806368in}{1.445162in}}%
\pgfpathlineto{\pgfqpoint{3.711467in}{1.446298in}}%
\pgfpathlineto{\pgfqpoint{3.616685in}{1.448714in}}%
\pgfpathlineto{\pgfqpoint{3.522563in}{1.454875in}}%
\pgfusepath{stroke}%
\end{pgfscope}%
\begin{pgfscope}%
\pgfpathrectangle{\pgfqpoint{0.647939in}{0.492442in}}{\pgfqpoint{4.273799in}{2.331163in}}%
\pgfusepath{clip}%
\pgfsetbuttcap%
\pgfsetroundjoin%
\pgfsetlinewidth{0.301125pt}%
\definecolor{currentstroke}{rgb}{0.500000,0.500000,0.500000}%
\pgfsetstrokecolor{currentstroke}%
\pgfsetstrokeopacity{0.300000}%
\pgfsetdash{}{0pt}%
\pgfpathmoveto{\pgfqpoint{4.436079in}{1.869948in}}%
\pgfpathlineto{\pgfqpoint{4.396748in}{1.916953in}}%
\pgfpathlineto{\pgfqpoint{4.396748in}{1.916953in}}%
\pgfpathlineto{\pgfqpoint{4.361327in}{1.944643in}}%
\pgfpathlineto{\pgfqpoint{4.361327in}{1.944643in}}%
\pgfpathlineto{\pgfqpoint{4.333850in}{1.954708in}}%
\pgfpathlineto{\pgfqpoint{4.333850in}{1.954708in}}%
\pgfpathlineto{\pgfqpoint{4.305356in}{1.954960in}}%
\pgfpathlineto{\pgfqpoint{4.278537in}{1.947671in}}%
\pgfpathlineto{\pgfqpoint{4.250034in}{1.934362in}}%
\pgfpathlineto{\pgfqpoint{4.210839in}{1.910527in}}%
\pgfpathlineto{\pgfqpoint{4.152504in}{1.870026in}}%
\pgfusepath{stroke}%
\end{pgfscope}%
\begin{pgfscope}%
\pgfpathrectangle{\pgfqpoint{0.647939in}{0.492442in}}{\pgfqpoint{4.273799in}{2.331163in}}%
\pgfusepath{clip}%
\pgfsetbuttcap%
\pgfsetroundjoin%
\pgfsetlinewidth{0.301125pt}%
\definecolor{currentstroke}{rgb}{0.500000,0.500000,0.500000}%
\pgfsetstrokecolor{currentstroke}%
\pgfsetstrokeopacity{0.300000}%
\pgfsetdash{}{0pt}%
\pgfpathmoveto{\pgfqpoint{1.374664in}{0.806926in}}%
\pgfpathlineto{\pgfqpoint{1.320829in}{0.822690in}}%
\pgfpathlineto{\pgfqpoint{1.263227in}{0.830354in}}%
\pgfpathlineto{\pgfqpoint{1.194969in}{0.826367in}}%
\pgfpathlineto{\pgfqpoint{1.133598in}{0.810328in}}%
\pgfpathlineto{\pgfqpoint{1.133598in}{0.810328in}}%
\pgfpathlineto{\pgfqpoint{1.062550in}{0.776896in}}%
\pgfpathlineto{\pgfqpoint{1.010979in}{0.742906in}}%
\pgfusepath{stroke}%
\end{pgfscope}%
\begin{pgfscope}%
\pgfpathrectangle{\pgfqpoint{0.647939in}{0.492442in}}{\pgfqpoint{4.273799in}{2.331163in}}%
\pgfusepath{clip}%
\pgfsetbuttcap%
\pgfsetroundjoin%
\pgfsetlinewidth{0.301125pt}%
\definecolor{currentstroke}{rgb}{0.500000,0.500000,0.500000}%
\pgfsetstrokecolor{currentstroke}%
\pgfsetstrokeopacity{0.300000}%
\pgfsetdash{}{0pt}%
\pgfpathmoveto{\pgfqpoint{1.619257in}{0.810328in}}%
\pgfpathlineto{\pgfqpoint{1.566307in}{0.853299in}}%
\pgfpathlineto{\pgfqpoint{1.508107in}{0.894176in}}%
\pgfpathlineto{\pgfqpoint{1.441952in}{0.931116in}}%
\pgfpathlineto{\pgfqpoint{1.363847in}{0.959607in}}%
\pgfpathlineto{\pgfqpoint{1.363847in}{0.959607in}}%
\pgfpathlineto{\pgfqpoint{1.306047in}{0.969035in}}%
\pgfpathlineto{\pgfqpoint{1.245030in}{0.967342in}}%
\pgfpathlineto{\pgfqpoint{1.193783in}{0.956466in}}%
\pgfusepath{stroke}%
\end{pgfscope}%
\begin{pgfscope}%
\pgfpathrectangle{\pgfqpoint{0.647939in}{0.492442in}}{\pgfqpoint{4.273799in}{2.331163in}}%
\pgfusepath{clip}%
\pgfsetbuttcap%
\pgfsetroundjoin%
\pgfsetlinewidth{0.301125pt}%
\definecolor{currentstroke}{rgb}{0.500000,0.500000,0.500000}%
\pgfsetstrokecolor{currentstroke}%
\pgfsetstrokeopacity{0.300000}%
\pgfsetdash{}{0pt}%
\pgfpathmoveto{\pgfqpoint{4.338948in}{0.916290in}}%
\pgfpathlineto{\pgfqpoint{4.287704in}{0.959879in}}%
\pgfpathlineto{\pgfqpoint{4.232320in}{1.001930in}}%
\pgfpathlineto{\pgfqpoint{4.172217in}{1.041996in}}%
\pgfpathlineto{\pgfqpoint{4.106913in}{1.079545in}}%
\pgfpathlineto{\pgfqpoint{4.036155in}{1.114018in}}%
\pgfpathlineto{\pgfqpoint{3.960198in}{1.145021in}}%
\pgfpathlineto{\pgfqpoint{3.879828in}{1.172542in}}%
\pgfpathlineto{\pgfqpoint{3.796294in}{1.197145in}}%
\pgfpathlineto{\pgfqpoint{3.710989in}{1.219904in}}%
\pgfpathlineto{\pgfqpoint{3.625220in}{1.242144in}}%
\pgfusepath{stroke}%
\end{pgfscope}%
\begin{pgfscope}%
\pgfpathrectangle{\pgfqpoint{0.647939in}{0.492442in}}{\pgfqpoint{4.273799in}{2.331163in}}%
\pgfusepath{clip}%
\pgfsetbuttcap%
\pgfsetroundjoin%
\pgfsetlinewidth{0.301125pt}%
\definecolor{currentstroke}{rgb}{0.500000,0.500000,0.500000}%
\pgfsetstrokecolor{currentstroke}%
\pgfsetstrokeopacity{0.300000}%
\pgfsetdash{}{0pt}%
\pgfpathmoveto{\pgfqpoint{4.338948in}{1.075233in}}%
\pgfpathlineto{\pgfqpoint{4.286664in}{1.118439in}}%
\pgfpathlineto{\pgfqpoint{4.229190in}{1.159629in}}%
\pgfpathlineto{\pgfqpoint{4.165592in}{1.198016in}}%
\pgfpathlineto{\pgfqpoint{4.095067in}{1.232572in}}%
\pgfpathlineto{\pgfqpoint{4.017452in}{1.262232in}}%
\pgfpathlineto{\pgfqpoint{3.933684in}{1.286424in}}%
\pgfpathlineto{\pgfqpoint{3.845623in}{1.305644in}}%
\pgfpathlineto{\pgfqpoint{3.755241in}{1.321468in}}%
\pgfpathlineto{\pgfqpoint{3.664127in}{1.336057in}}%
\pgfpathlineto{\pgfqpoint{3.573505in}{1.351484in}}%
\pgfpathlineto{\pgfqpoint{3.484466in}{1.369367in}}%
\pgfpathlineto{\pgfqpoint{3.398089in}{1.390732in}}%
\pgfusepath{stroke}%
\end{pgfscope}%
\begin{pgfscope}%
\pgfpathrectangle{\pgfqpoint{0.647939in}{0.492442in}}{\pgfqpoint{4.273799in}{2.331163in}}%
\pgfusepath{clip}%
\pgfsetbuttcap%
\pgfsetroundjoin%
\pgfsetlinewidth{0.301125pt}%
\definecolor{currentstroke}{rgb}{0.500000,0.500000,0.500000}%
\pgfsetstrokecolor{currentstroke}%
\pgfsetstrokeopacity{0.300000}%
\pgfsetdash{}{0pt}%
\pgfpathmoveto{\pgfqpoint{4.328431in}{2.213240in}}%
\pgfpathlineto{\pgfqpoint{4.344462in}{2.163168in}}%
\pgfpathlineto{\pgfqpoint{4.349684in}{2.122094in}}%
\pgfpathlineto{\pgfqpoint{4.338948in}{2.081872in}}%
\pgfpathlineto{\pgfqpoint{4.338948in}{2.081872in}}%
\pgfpathlineto{\pgfqpoint{4.302880in}{2.035310in}}%
\pgfpathlineto{\pgfqpoint{4.266252in}{1.995009in}}%
\pgfpathlineto{\pgfqpoint{4.220382in}{1.949896in}}%
\pgfusepath{stroke}%
\end{pgfscope}%
\begin{pgfscope}%
\pgfpathrectangle{\pgfqpoint{0.647939in}{0.492442in}}{\pgfqpoint{4.273799in}{2.331163in}}%
\pgfusepath{clip}%
\pgfsetbuttcap%
\pgfsetroundjoin%
\pgfsetlinewidth{0.301125pt}%
\definecolor{currentstroke}{rgb}{0.500000,0.500000,0.500000}%
\pgfsetstrokecolor{currentstroke}%
\pgfsetstrokeopacity{0.300000}%
\pgfsetdash{}{0pt}%
\pgfpathmoveto{\pgfqpoint{1.522125in}{2.505719in}}%
\pgfpathlineto{\pgfqpoint{1.546471in}{2.455653in}}%
\pgfpathlineto{\pgfqpoint{1.571022in}{2.405620in}}%
\pgfpathlineto{\pgfqpoint{1.595778in}{2.355615in}}%
\pgfpathlineto{\pgfqpoint{1.620687in}{2.305639in}}%
\pgfpathlineto{\pgfqpoint{1.645656in}{2.255682in}}%
\pgfpathlineto{\pgfqpoint{1.670417in}{2.205722in}}%
\pgfpathlineto{\pgfqpoint{1.693420in}{2.155896in}}%
\pgfpathlineto{\pgfqpoint{1.693420in}{2.155896in}}%
\pgfpathlineto{\pgfqpoint{1.700033in}{2.137523in}}%
\pgfpathlineto{\pgfqpoint{1.700033in}{2.137523in}}%
\pgfpathlineto{\pgfqpoint{1.701661in}{2.124275in}}%
\pgfpathlineto{\pgfqpoint{1.697673in}{2.112847in}}%
\pgfusepath{stroke}%
\end{pgfscope}%
\begin{pgfscope}%
\pgfpathrectangle{\pgfqpoint{0.647939in}{0.492442in}}{\pgfqpoint{4.273799in}{2.331163in}}%
\pgfusepath{clip}%
\pgfsetbuttcap%
\pgfsetroundjoin%
\pgfsetlinewidth{0.301125pt}%
\definecolor{currentstroke}{rgb}{0.500000,0.500000,0.500000}%
\pgfsetstrokecolor{currentstroke}%
\pgfsetstrokeopacity{0.300000}%
\pgfsetdash{}{0pt}%
\pgfpathmoveto{\pgfqpoint{1.734720in}{1.314179in}}%
\pgfpathlineto{\pgfqpoint{1.688586in}{1.359441in}}%
\pgfpathlineto{\pgfqpoint{1.637921in}{1.403206in}}%
\pgfpathlineto{\pgfqpoint{1.586832in}{1.439560in}}%
\pgfpathlineto{\pgfqpoint{1.541770in}{1.463957in}}%
\pgfpathlineto{\pgfqpoint{1.499402in}{1.479451in}}%
\pgfpathlineto{\pgfqpoint{1.452932in}{1.487392in}}%
\pgfpathlineto{\pgfqpoint{1.404701in}{1.485200in}}%
\pgfpathlineto{\pgfqpoint{1.404701in}{1.485200in}}%
\pgfpathlineto{\pgfqpoint{1.352892in}{1.470994in}}%
\pgfpathlineto{\pgfqpoint{1.352892in}{1.470994in}}%
\pgfpathlineto{\pgfqpoint{1.285595in}{1.435153in}}%
\pgfpathlineto{\pgfqpoint{1.230730in}{1.393119in}}%
\pgfusepath{stroke}%
\end{pgfscope}%
\begin{pgfscope}%
\pgfpathrectangle{\pgfqpoint{0.647939in}{0.492442in}}{\pgfqpoint{4.273799in}{2.331163in}}%
\pgfusepath{clip}%
\pgfsetbuttcap%
\pgfsetroundjoin%
\pgfsetlinewidth{0.301125pt}%
\definecolor{currentstroke}{rgb}{0.500000,0.500000,0.500000}%
\pgfsetstrokecolor{currentstroke}%
\pgfsetstrokeopacity{0.300000}%
\pgfsetdash{}{0pt}%
\pgfpathmoveto{\pgfqpoint{4.302141in}{1.353278in}}%
\pgfpathlineto{\pgfqpoint{4.241816in}{1.393119in}}%
\pgfpathlineto{\pgfqpoint{4.171681in}{1.427729in}}%
\pgfpathlineto{\pgfqpoint{4.090462in}{1.453859in}}%
\pgfpathlineto{\pgfqpoint{4.006374in}{1.468289in}}%
\pgfpathlineto{\pgfqpoint{3.921602in}{1.473690in}}%
\pgfusepath{stroke}%
\end{pgfscope}%
\begin{pgfscope}%
\pgfpathrectangle{\pgfqpoint{0.647939in}{0.492442in}}{\pgfqpoint{4.273799in}{2.331163in}}%
\pgfusepath{clip}%
\pgfsetbuttcap%
\pgfsetroundjoin%
\pgfsetlinewidth{0.301125pt}%
\definecolor{currentstroke}{rgb}{0.500000,0.500000,0.500000}%
\pgfsetstrokecolor{currentstroke}%
\pgfsetstrokeopacity{0.300000}%
\pgfsetdash{}{0pt}%
\pgfpathmoveto{\pgfqpoint{1.637320in}{2.546804in}}%
\pgfpathlineto{\pgfqpoint{1.674574in}{2.499179in}}%
\pgfpathlineto{\pgfqpoint{1.716389in}{2.452738in}}%
\pgfpathlineto{\pgfqpoint{1.766237in}{2.408825in}}%
\pgfpathlineto{\pgfqpoint{1.766237in}{2.408825in}}%
\pgfpathlineto{\pgfqpoint{1.817109in}{2.378667in}}%
\pgfpathlineto{\pgfqpoint{1.817109in}{2.378667in}}%
\pgfpathlineto{\pgfqpoint{1.862132in}{2.364450in}}%
\pgfpathlineto{\pgfqpoint{1.914712in}{2.360646in}}%
\pgfusepath{stroke}%
\end{pgfscope}%
\begin{pgfscope}%
\pgfpathrectangle{\pgfqpoint{0.647939in}{0.492442in}}{\pgfqpoint{4.273799in}{2.331163in}}%
\pgfusepath{clip}%
\pgfsetbuttcap%
\pgfsetroundjoin%
\pgfsetlinewidth{0.301125pt}%
\definecolor{currentstroke}{rgb}{0.500000,0.500000,0.500000}%
\pgfsetstrokecolor{currentstroke}%
\pgfsetstrokeopacity{0.300000}%
\pgfsetdash{}{0pt}%
\pgfpathmoveto{\pgfqpoint{1.327862in}{2.293796in}}%
\pgfpathlineto{\pgfqpoint{1.334454in}{2.242120in}}%
\pgfpathlineto{\pgfqpoint{1.338952in}{2.190379in}}%
\pgfpathlineto{\pgfqpoint{1.341025in}{2.138593in}}%
\pgfpathlineto{\pgfqpoint{1.340339in}{2.086798in}}%
\pgfpathlineto{\pgfqpoint{1.336602in}{2.035043in}}%
\pgfusepath{stroke}%
\end{pgfscope}%
\begin{pgfscope}%
\pgfpathrectangle{\pgfqpoint{0.647939in}{0.492442in}}{\pgfqpoint{4.273799in}{2.331163in}}%
\pgfusepath{clip}%
\pgfsetbuttcap%
\pgfsetroundjoin%
\pgfsetlinewidth{0.301125pt}%
\definecolor{currentstroke}{rgb}{0.500000,0.500000,0.500000}%
\pgfsetstrokecolor{currentstroke}%
\pgfsetstrokeopacity{0.300000}%
\pgfsetdash{}{0pt}%
\pgfpathmoveto{\pgfqpoint{2.006976in}{1.230586in}}%
\pgfpathlineto{\pgfqpoint{1.971221in}{1.278578in}}%
\pgfpathlineto{\pgfqpoint{1.935241in}{1.326519in}}%
\pgfpathlineto{\pgfqpoint{1.898869in}{1.374373in}}%
\pgfpathlineto{\pgfqpoint{1.861862in}{1.422081in}}%
\pgfpathlineto{\pgfqpoint{1.823870in}{1.469555in}}%
\pgfpathlineto{\pgfqpoint{1.784386in}{1.516662in}}%
\pgfpathlineto{\pgfqpoint{1.742587in}{1.563160in}}%
\pgfpathlineto{\pgfqpoint{1.696988in}{1.608555in}}%
\pgfpathlineto{\pgfqpoint{1.650144in}{1.647566in}}%
\pgfpathlineto{\pgfqpoint{1.610219in}{1.673233in}}%
\pgfpathlineto{\pgfqpoint{1.573878in}{1.689429in}}%
\pgfpathlineto{\pgfqpoint{1.534301in}{1.698446in}}%
\pgfpathlineto{\pgfqpoint{1.491803in}{1.697731in}}%
\pgfpathlineto{\pgfqpoint{1.491803in}{1.697731in}}%
\pgfpathlineto{\pgfqpoint{1.445446in}{1.684911in}}%
\pgfpathlineto{\pgfqpoint{1.445446in}{1.684911in}}%
\pgfpathlineto{\pgfqpoint{1.380462in}{1.647986in}}%
\pgfpathlineto{\pgfqpoint{1.327862in}{1.605043in}}%
\pgfusepath{stroke}%
\end{pgfscope}%
\begin{pgfscope}%
\pgfpathrectangle{\pgfqpoint{0.647939in}{0.492442in}}{\pgfqpoint{4.273799in}{2.331163in}}%
\pgfusepath{clip}%
\pgfsetbuttcap%
\pgfsetroundjoin%
\pgfsetlinewidth{0.301125pt}%
\definecolor{currentstroke}{rgb}{0.500000,0.500000,0.500000}%
\pgfsetstrokecolor{currentstroke}%
\pgfsetstrokeopacity{0.300000}%
\pgfsetdash{}{0pt}%
\pgfpathmoveto{\pgfqpoint{4.260612in}{0.834328in}}%
\pgfpathlineto{\pgfqpoint{4.204666in}{0.876160in}}%
\pgfpathlineto{\pgfqpoint{4.144684in}{0.916290in}}%
\pgfpathlineto{\pgfqpoint{4.080422in}{0.954397in}}%
\pgfpathlineto{\pgfqpoint{4.011854in}{0.990196in}}%
\pgfpathlineto{\pgfqpoint{3.939241in}{1.023543in}}%
\pgfpathlineto{\pgfqpoint{3.863198in}{1.054542in}}%
\pgfpathlineto{\pgfqpoint{3.784586in}{1.083590in}}%
\pgfpathlineto{\pgfqpoint{3.704418in}{1.111357in}}%
\pgfpathlineto{\pgfqpoint{3.623733in}{1.138678in}}%
\pgfusepath{stroke}%
\end{pgfscope}%
\begin{pgfscope}%
\pgfpathrectangle{\pgfqpoint{0.647939in}{0.492442in}}{\pgfqpoint{4.273799in}{2.331163in}}%
\pgfusepath{clip}%
\pgfsetbuttcap%
\pgfsetroundjoin%
\pgfsetlinewidth{0.301125pt}%
\definecolor{currentstroke}{rgb}{0.500000,0.500000,0.500000}%
\pgfsetstrokecolor{currentstroke}%
\pgfsetstrokeopacity{0.300000}%
\pgfsetdash{}{0pt}%
\pgfpathmoveto{\pgfqpoint{3.401226in}{0.947121in}}%
\pgfpathlineto{\pgfqpoint{3.334930in}{0.984205in}}%
\pgfpathlineto{\pgfqpoint{3.270498in}{1.022252in}}%
\pgfpathlineto{\pgfqpoint{3.208125in}{1.061305in}}%
\pgfpathlineto{\pgfqpoint{3.147927in}{1.101360in}}%
\pgfpathlineto{\pgfqpoint{3.089968in}{1.142387in}}%
\pgfusepath{stroke}%
\end{pgfscope}%
\begin{pgfscope}%
\pgfpathrectangle{\pgfqpoint{0.647939in}{0.492442in}}{\pgfqpoint{4.273799in}{2.331163in}}%
\pgfusepath{clip}%
\pgfsetbuttcap%
\pgfsetroundjoin%
\pgfsetlinewidth{0.301125pt}%
\definecolor{currentstroke}{rgb}{0.500000,0.500000,0.500000}%
\pgfsetstrokecolor{currentstroke}%
\pgfsetstrokeopacity{0.300000}%
\pgfsetdash{}{0pt}%
\pgfpathmoveto{\pgfqpoint{3.367630in}{2.293796in}}%
\pgfpathlineto{\pgfqpoint{3.378786in}{2.242360in}}%
\pgfpathlineto{\pgfqpoint{3.386460in}{2.190737in}}%
\pgfpathlineto{\pgfqpoint{3.390068in}{2.138986in}}%
\pgfpathlineto{\pgfqpoint{3.388820in}{2.087209in}}%
\pgfpathlineto{\pgfqpoint{3.381611in}{2.035593in}}%
\pgfpathlineto{\pgfqpoint{3.366832in}{1.984485in}}%
\pgfpathlineto{\pgfqpoint{3.342000in}{1.934606in}}%
\pgfpathlineto{\pgfqpoint{3.303136in}{1.887631in}}%
\pgfpathlineto{\pgfqpoint{3.303136in}{1.887631in}}%
\pgfpathlineto{\pgfqpoint{3.258559in}{1.855029in}}%
\pgfpathlineto{\pgfqpoint{3.258559in}{1.855029in}}%
\pgfpathlineto{\pgfqpoint{3.210247in}{1.834502in}}%
\pgfpathlineto{\pgfqpoint{3.151754in}{1.823783in}}%
\pgfpathlineto{\pgfqpoint{3.097482in}{1.824408in}}%
\pgfpathlineto{\pgfqpoint{3.045828in}{1.833144in}}%
\pgfusepath{stroke}%
\end{pgfscope}%
\begin{pgfscope}%
\pgfpathrectangle{\pgfqpoint{0.647939in}{0.492442in}}{\pgfqpoint{4.273799in}{2.331163in}}%
\pgfusepath{clip}%
\pgfsetbuttcap%
\pgfsetroundjoin%
\pgfsetlinewidth{0.301125pt}%
\definecolor{currentstroke}{rgb}{0.500000,0.500000,0.500000}%
\pgfsetstrokecolor{currentstroke}%
\pgfsetstrokeopacity{0.300000}%
\pgfsetdash{}{0pt}%
\pgfpathmoveto{\pgfqpoint{1.754821in}{1.650740in}}%
\pgfpathlineto{\pgfqpoint{1.710779in}{1.696581in}}%
\pgfpathlineto{\pgfqpoint{1.670455in}{1.731811in}}%
\pgfpathlineto{\pgfqpoint{1.619257in}{1.763986in}}%
\pgfpathlineto{\pgfqpoint{1.619257in}{1.763986in}}%
\pgfpathlineto{\pgfqpoint{1.619257in}{1.763986in}}%
\pgfpathlineto{\pgfqpoint{1.580382in}{1.777707in}}%
\pgfpathlineto{\pgfqpoint{1.580382in}{1.777707in}}%
\pgfpathlineto{\pgfqpoint{1.542686in}{1.780978in}}%
\pgfpathlineto{\pgfqpoint{1.505779in}{1.774754in}}%
\pgfpathlineto{\pgfqpoint{1.472851in}{1.762010in}}%
\pgfpathlineto{\pgfqpoint{1.437209in}{1.741345in}}%
\pgfusepath{stroke}%
\end{pgfscope}%
\begin{pgfscope}%
\pgfpathrectangle{\pgfqpoint{0.647939in}{0.492442in}}{\pgfqpoint{4.273799in}{2.331163in}}%
\pgfusepath{clip}%
\pgfsetbuttcap%
\pgfsetroundjoin%
\pgfsetlinewidth{0.301125pt}%
\definecolor{currentstroke}{rgb}{0.500000,0.500000,0.500000}%
\pgfsetstrokecolor{currentstroke}%
\pgfsetstrokeopacity{0.300000}%
\pgfsetdash{}{0pt}%
\pgfpathmoveto{\pgfqpoint{1.619257in}{1.234176in}}%
\pgfpathlineto{\pgfqpoint{1.561657in}{1.275245in}}%
\pgfpathlineto{\pgfqpoint{1.493405in}{1.310786in}}%
\pgfpathlineto{\pgfqpoint{1.493405in}{1.310786in}}%
\pgfpathlineto{\pgfqpoint{1.438833in}{1.327428in}}%
\pgfpathlineto{\pgfqpoint{1.438833in}{1.327428in}}%
\pgfpathlineto{\pgfqpoint{1.388304in}{1.332582in}}%
\pgfpathlineto{\pgfqpoint{1.337319in}{1.327378in}}%
\pgfpathlineto{\pgfqpoint{1.293218in}{1.314497in}}%
\pgfpathlineto{\pgfqpoint{1.249221in}{1.294045in}}%
\pgfpathlineto{\pgfqpoint{1.201798in}{1.263889in}}%
\pgfusepath{stroke}%
\end{pgfscope}%
\begin{pgfscope}%
\pgfpathrectangle{\pgfqpoint{0.647939in}{0.492442in}}{\pgfqpoint{4.273799in}{2.331163in}}%
\pgfusepath{clip}%
\pgfsetbuttcap%
\pgfsetroundjoin%
\pgfsetlinewidth{0.301125pt}%
\definecolor{currentstroke}{rgb}{0.500000,0.500000,0.500000}%
\pgfsetstrokecolor{currentstroke}%
\pgfsetstrokeopacity{0.300000}%
\pgfsetdash{}{0pt}%
\pgfpathmoveto{\pgfqpoint{3.410280in}{1.111114in}}%
\pgfpathlineto{\pgfqpoint{3.339171in}{1.145435in}}%
\pgfpathlineto{\pgfqpoint{3.270498in}{1.181195in}}%
\pgfpathlineto{\pgfqpoint{3.204506in}{1.218425in}}%
\pgfpathlineto{\pgfqpoint{3.141337in}{1.257084in}}%
\pgfpathlineto{\pgfqpoint{3.081041in}{1.297085in}}%
\pgfusepath{stroke}%
\end{pgfscope}%
\begin{pgfscope}%
\pgfpathrectangle{\pgfqpoint{0.647939in}{0.492442in}}{\pgfqpoint{4.273799in}{2.331163in}}%
\pgfusepath{clip}%
\pgfsetbuttcap%
\pgfsetroundjoin%
\pgfsetlinewidth{0.301125pt}%
\definecolor{currentstroke}{rgb}{0.500000,0.500000,0.500000}%
\pgfsetstrokecolor{currentstroke}%
\pgfsetstrokeopacity{0.300000}%
\pgfsetdash{}{0pt}%
\pgfpathmoveto{\pgfqpoint{3.705811in}{1.808946in}}%
\pgfpathlineto{\pgfqpoint{3.659025in}{1.763986in}}%
\pgfpathlineto{\pgfqpoint{3.601175in}{1.723125in}}%
\pgfpathlineto{\pgfqpoint{3.531337in}{1.688463in}}%
\pgfpathlineto{\pgfqpoint{3.450323in}{1.663206in}}%
\pgfpathlineto{\pgfqpoint{3.370564in}{1.651056in}}%
\pgfpathlineto{\pgfqpoint{3.294662in}{1.649888in}}%
\pgfpathlineto{\pgfqpoint{3.221937in}{1.657748in}}%
\pgfpathlineto{\pgfqpoint{3.150461in}{1.673882in}}%
\pgfusepath{stroke}%
\end{pgfscope}%
\begin{pgfscope}%
\pgfpathrectangle{\pgfqpoint{0.647939in}{0.492442in}}{\pgfqpoint{4.273799in}{2.331163in}}%
\pgfusepath{clip}%
\pgfsetbuttcap%
\pgfsetroundjoin%
\pgfsetlinewidth{0.301125pt}%
\definecolor{currentstroke}{rgb}{0.500000,0.500000,0.500000}%
\pgfsetstrokecolor{currentstroke}%
\pgfsetstrokeopacity{0.300000}%
\pgfsetdash{}{0pt}%
\pgfpathmoveto{\pgfqpoint{1.910652in}{1.869948in}}%
\pgfpathlineto{\pgfqpoint{1.885597in}{1.919911in}}%
\pgfpathlineto{\pgfqpoint{1.862169in}{1.970106in}}%
\pgfpathlineto{\pgfqpoint{1.841228in}{2.020620in}}%
\pgfpathlineto{\pgfqpoint{1.824740in}{2.071604in}}%
\pgfpathlineto{\pgfqpoint{1.817784in}{2.123120in}}%
\pgfpathlineto{\pgfqpoint{1.817784in}{2.123120in}}%
\pgfpathlineto{\pgfqpoint{1.825881in}{2.164343in}}%
\pgfpathlineto{\pgfqpoint{1.852087in}{2.204798in}}%
\pgfusepath{stroke}%
\end{pgfscope}%
\begin{pgfscope}%
\pgfpathrectangle{\pgfqpoint{0.647939in}{0.492442in}}{\pgfqpoint{4.273799in}{2.331163in}}%
\pgfusepath{clip}%
\pgfsetbuttcap%
\pgfsetroundjoin%
\pgfsetlinewidth{0.301125pt}%
\definecolor{currentstroke}{rgb}{0.500000,0.500000,0.500000}%
\pgfsetstrokecolor{currentstroke}%
\pgfsetstrokeopacity{0.300000}%
\pgfsetdash{}{0pt}%
\pgfpathmoveto{\pgfqpoint{3.594274in}{2.024858in}}%
\pgfpathlineto{\pgfqpoint{3.581634in}{1.973548in}}%
\pgfpathlineto{\pgfqpoint{3.561893in}{1.922929in}}%
\pgfpathlineto{\pgfqpoint{3.533430in}{1.873597in}}%
\pgfpathlineto{\pgfqpoint{3.494041in}{1.826625in}}%
\pgfpathlineto{\pgfqpoint{3.440917in}{1.784028in}}%
\pgfpathlineto{\pgfqpoint{3.440917in}{1.784028in}}%
\pgfusepath{stroke}%
\end{pgfscope}%
\begin{pgfscope}%
\pgfpathrectangle{\pgfqpoint{0.647939in}{0.492442in}}{\pgfqpoint{4.273799in}{2.331163in}}%
\pgfusepath{clip}%
\pgfsetroundcap%
\pgfsetroundjoin%
\pgfsetlinewidth{0.301125pt}%
\definecolor{currentstroke}{rgb}{0.500000,0.500000,0.500000}%
\pgfsetstrokecolor{currentstroke}%
\pgfsetstrokeopacity{0.300000}%
\pgfsetdash{}{0pt}%
\pgfpathmoveto{\pgfqpoint{1.486498in}{1.382961in}}%
\pgfusepath{stroke}%
\end{pgfscope}%
\begin{pgfscope}%
\pgfpathrectangle{\pgfqpoint{0.647939in}{0.492442in}}{\pgfqpoint{4.273799in}{2.331163in}}%
\pgfusepath{clip}%
\pgfsetroundcap%
\pgfsetroundjoin%
\definecolor{currentfill}{rgb}{0.500000,0.500000,0.500000}%
\pgfsetfillcolor{currentfill}%
\pgfsetfillopacity{0.300000}%
\pgfsetlinewidth{0.301125pt}%
\definecolor{currentstroke}{rgb}{0.500000,0.500000,0.500000}%
\pgfsetstrokecolor{currentstroke}%
\pgfsetstrokeopacity{0.300000}%
\pgfsetdash{}{0pt}%
\pgfpathmoveto{\pgfqpoint{0.000000in}{0.000000in}}%
\pgfpathlineto{\pgfqpoint{0.000000in}{0.000000in}}%
\pgfpathclose%
\pgfusepath{stroke,fill}%
\end{pgfscope}%
\begin{pgfscope}%
\pgfpathrectangle{\pgfqpoint{0.647939in}{0.492442in}}{\pgfqpoint{4.273799in}{2.331163in}}%
\pgfusepath{clip}%
\pgfsetroundcap%
\pgfsetroundjoin%
\pgfsetlinewidth{0.301125pt}%
\definecolor{currentstroke}{rgb}{0.500000,0.500000,0.500000}%
\pgfsetstrokecolor{currentstroke}%
\pgfsetstrokeopacity{0.300000}%
\pgfsetdash{}{0pt}%
\pgfpathmoveto{\pgfqpoint{1.315162in}{0.908253in}}%
\pgfusepath{stroke}%
\end{pgfscope}%
\begin{pgfscope}%
\pgfpathrectangle{\pgfqpoint{0.647939in}{0.492442in}}{\pgfqpoint{4.273799in}{2.331163in}}%
\pgfusepath{clip}%
\pgfsetroundcap%
\pgfsetroundjoin%
\definecolor{currentfill}{rgb}{0.500000,0.500000,0.500000}%
\pgfsetfillcolor{currentfill}%
\pgfsetfillopacity{0.300000}%
\pgfsetlinewidth{0.301125pt}%
\definecolor{currentstroke}{rgb}{0.500000,0.500000,0.500000}%
\pgfsetstrokecolor{currentstroke}%
\pgfsetstrokeopacity{0.300000}%
\pgfsetdash{}{0pt}%
\pgfpathmoveto{\pgfqpoint{0.000000in}{0.000000in}}%
\pgfpathlineto{\pgfqpoint{0.000000in}{0.000000in}}%
\pgfpathclose%
\pgfusepath{stroke,fill}%
\end{pgfscope}%
\begin{pgfscope}%
\pgfpathrectangle{\pgfqpoint{0.647939in}{0.492442in}}{\pgfqpoint{4.273799in}{2.331163in}}%
\pgfusepath{clip}%
\pgfsetroundcap%
\pgfsetroundjoin%
\pgfsetlinewidth{0.301125pt}%
\definecolor{currentstroke}{rgb}{0.500000,0.500000,0.500000}%
\pgfsetstrokecolor{currentstroke}%
\pgfsetstrokeopacity{0.300000}%
\pgfsetdash{}{0pt}%
\pgfpathmoveto{\pgfqpoint{1.230591in}{0.698040in}}%
\pgfusepath{stroke}%
\end{pgfscope}%
\begin{pgfscope}%
\pgfpathrectangle{\pgfqpoint{0.647939in}{0.492442in}}{\pgfqpoint{4.273799in}{2.331163in}}%
\pgfusepath{clip}%
\pgfsetroundcap%
\pgfsetroundjoin%
\definecolor{currentfill}{rgb}{0.500000,0.500000,0.500000}%
\pgfsetfillcolor{currentfill}%
\pgfsetfillopacity{0.300000}%
\pgfsetlinewidth{0.301125pt}%
\definecolor{currentstroke}{rgb}{0.500000,0.500000,0.500000}%
\pgfsetstrokecolor{currentstroke}%
\pgfsetstrokeopacity{0.300000}%
\pgfsetdash{}{0pt}%
\pgfpathmoveto{\pgfqpoint{0.000000in}{0.000000in}}%
\pgfpathlineto{\pgfqpoint{0.000000in}{0.000000in}}%
\pgfpathclose%
\pgfusepath{stroke,fill}%
\end{pgfscope}%
\begin{pgfscope}%
\pgfpathrectangle{\pgfqpoint{0.647939in}{0.492442in}}{\pgfqpoint{4.273799in}{2.331163in}}%
\pgfusepath{clip}%
\pgfsetroundcap%
\pgfsetroundjoin%
\pgfsetlinewidth{0.301125pt}%
\definecolor{currentstroke}{rgb}{0.500000,0.500000,0.500000}%
\pgfsetstrokecolor{currentstroke}%
\pgfsetstrokeopacity{0.300000}%
\pgfsetdash{}{0pt}%
\pgfpathmoveto{\pgfqpoint{1.179624in}{0.579899in}}%
\pgfusepath{stroke}%
\end{pgfscope}%
\begin{pgfscope}%
\pgfpathrectangle{\pgfqpoint{0.647939in}{0.492442in}}{\pgfqpoint{4.273799in}{2.331163in}}%
\pgfusepath{clip}%
\pgfsetroundcap%
\pgfsetroundjoin%
\definecolor{currentfill}{rgb}{0.500000,0.500000,0.500000}%
\pgfsetfillcolor{currentfill}%
\pgfsetfillopacity{0.300000}%
\pgfsetlinewidth{0.301125pt}%
\definecolor{currentstroke}{rgb}{0.500000,0.500000,0.500000}%
\pgfsetstrokecolor{currentstroke}%
\pgfsetstrokeopacity{0.300000}%
\pgfsetdash{}{0pt}%
\pgfpathmoveto{\pgfqpoint{0.000000in}{0.000000in}}%
\pgfpathlineto{\pgfqpoint{0.000000in}{0.000000in}}%
\pgfpathclose%
\pgfusepath{stroke,fill}%
\end{pgfscope}%
\begin{pgfscope}%
\pgfpathrectangle{\pgfqpoint{0.647939in}{0.492442in}}{\pgfqpoint{4.273799in}{2.331163in}}%
\pgfusepath{clip}%
\pgfsetroundcap%
\pgfsetroundjoin%
\pgfsetlinewidth{0.301125pt}%
\definecolor{currentstroke}{rgb}{0.500000,0.500000,0.500000}%
\pgfsetstrokecolor{currentstroke}%
\pgfsetstrokeopacity{0.300000}%
\pgfsetdash{}{0pt}%
\pgfpathmoveto{\pgfqpoint{1.356711in}{0.750593in}}%
\pgfusepath{stroke}%
\end{pgfscope}%
\begin{pgfscope}%
\pgfpathrectangle{\pgfqpoint{0.647939in}{0.492442in}}{\pgfqpoint{4.273799in}{2.331163in}}%
\pgfusepath{clip}%
\pgfsetroundcap%
\pgfsetroundjoin%
\definecolor{currentfill}{rgb}{0.500000,0.500000,0.500000}%
\pgfsetfillcolor{currentfill}%
\pgfsetfillopacity{0.300000}%
\pgfsetlinewidth{0.301125pt}%
\definecolor{currentstroke}{rgb}{0.500000,0.500000,0.500000}%
\pgfsetstrokecolor{currentstroke}%
\pgfsetstrokeopacity{0.300000}%
\pgfsetdash{}{0pt}%
\pgfpathmoveto{\pgfqpoint{0.000000in}{0.000000in}}%
\pgfpathlineto{\pgfqpoint{0.000000in}{0.000000in}}%
\pgfpathclose%
\pgfusepath{stroke,fill}%
\end{pgfscope}%
\begin{pgfscope}%
\pgfpathrectangle{\pgfqpoint{0.647939in}{0.492442in}}{\pgfqpoint{4.273799in}{2.331163in}}%
\pgfusepath{clip}%
\pgfsetroundcap%
\pgfsetroundjoin%
\pgfsetlinewidth{0.301125pt}%
\definecolor{currentstroke}{rgb}{0.500000,0.500000,0.500000}%
\pgfsetstrokecolor{currentstroke}%
\pgfsetstrokeopacity{0.300000}%
\pgfsetdash{}{0pt}%
\pgfpathmoveto{\pgfqpoint{1.482044in}{1.003158in}}%
\pgfusepath{stroke}%
\end{pgfscope}%
\begin{pgfscope}%
\pgfpathrectangle{\pgfqpoint{0.647939in}{0.492442in}}{\pgfqpoint{4.273799in}{2.331163in}}%
\pgfusepath{clip}%
\pgfsetroundcap%
\pgfsetroundjoin%
\definecolor{currentfill}{rgb}{0.500000,0.500000,0.500000}%
\pgfsetfillcolor{currentfill}%
\pgfsetfillopacity{0.300000}%
\pgfsetlinewidth{0.301125pt}%
\definecolor{currentstroke}{rgb}{0.500000,0.500000,0.500000}%
\pgfsetstrokecolor{currentstroke}%
\pgfsetstrokeopacity{0.300000}%
\pgfsetdash{}{0pt}%
\pgfpathmoveto{\pgfqpoint{0.000000in}{0.000000in}}%
\pgfpathlineto{\pgfqpoint{0.000000in}{0.000000in}}%
\pgfpathclose%
\pgfusepath{stroke,fill}%
\end{pgfscope}%
\begin{pgfscope}%
\pgfpathrectangle{\pgfqpoint{0.647939in}{0.492442in}}{\pgfqpoint{4.273799in}{2.331163in}}%
\pgfusepath{clip}%
\pgfsetroundcap%
\pgfsetroundjoin%
\pgfsetlinewidth{0.301125pt}%
\definecolor{currentstroke}{rgb}{0.500000,0.500000,0.500000}%
\pgfsetstrokecolor{currentstroke}%
\pgfsetstrokeopacity{0.300000}%
\pgfsetdash{}{0pt}%
\pgfpathmoveto{\pgfqpoint{1.674764in}{0.978419in}}%
\pgfusepath{stroke}%
\end{pgfscope}%
\begin{pgfscope}%
\pgfpathrectangle{\pgfqpoint{0.647939in}{0.492442in}}{\pgfqpoint{4.273799in}{2.331163in}}%
\pgfusepath{clip}%
\pgfsetroundcap%
\pgfsetroundjoin%
\definecolor{currentfill}{rgb}{0.500000,0.500000,0.500000}%
\pgfsetfillcolor{currentfill}%
\pgfsetfillopacity{0.300000}%
\pgfsetlinewidth{0.301125pt}%
\definecolor{currentstroke}{rgb}{0.500000,0.500000,0.500000}%
\pgfsetstrokecolor{currentstroke}%
\pgfsetstrokeopacity{0.300000}%
\pgfsetdash{}{0pt}%
\pgfpathmoveto{\pgfqpoint{0.000000in}{0.000000in}}%
\pgfpathlineto{\pgfqpoint{0.000000in}{0.000000in}}%
\pgfpathclose%
\pgfusepath{stroke,fill}%
\end{pgfscope}%
\begin{pgfscope}%
\pgfpathrectangle{\pgfqpoint{0.647939in}{0.492442in}}{\pgfqpoint{4.273799in}{2.331163in}}%
\pgfusepath{clip}%
\pgfsetroundcap%
\pgfsetroundjoin%
\pgfsetlinewidth{0.301125pt}%
\definecolor{currentstroke}{rgb}{0.500000,0.500000,0.500000}%
\pgfsetstrokecolor{currentstroke}%
\pgfsetstrokeopacity{0.300000}%
\pgfsetdash{}{0pt}%
\pgfpathmoveto{\pgfqpoint{1.748252in}{1.030567in}}%
\pgfusepath{stroke}%
\end{pgfscope}%
\begin{pgfscope}%
\pgfpathrectangle{\pgfqpoint{0.647939in}{0.492442in}}{\pgfqpoint{4.273799in}{2.331163in}}%
\pgfusepath{clip}%
\pgfsetroundcap%
\pgfsetroundjoin%
\definecolor{currentfill}{rgb}{0.500000,0.500000,0.500000}%
\pgfsetfillcolor{currentfill}%
\pgfsetfillopacity{0.300000}%
\pgfsetlinewidth{0.301125pt}%
\definecolor{currentstroke}{rgb}{0.500000,0.500000,0.500000}%
\pgfsetstrokecolor{currentstroke}%
\pgfsetstrokeopacity{0.300000}%
\pgfsetdash{}{0pt}%
\pgfpathmoveto{\pgfqpoint{0.000000in}{0.000000in}}%
\pgfpathlineto{\pgfqpoint{0.000000in}{0.000000in}}%
\pgfpathclose%
\pgfusepath{stroke,fill}%
\end{pgfscope}%
\begin{pgfscope}%
\pgfpathrectangle{\pgfqpoint{0.647939in}{0.492442in}}{\pgfqpoint{4.273799in}{2.331163in}}%
\pgfusepath{clip}%
\pgfsetroundcap%
\pgfsetroundjoin%
\pgfsetlinewidth{0.301125pt}%
\definecolor{currentstroke}{rgb}{0.500000,0.500000,0.500000}%
\pgfsetstrokecolor{currentstroke}%
\pgfsetstrokeopacity{0.300000}%
\pgfsetdash{}{0pt}%
\pgfpathmoveto{\pgfqpoint{1.873332in}{1.274850in}}%
\pgfusepath{stroke}%
\end{pgfscope}%
\begin{pgfscope}%
\pgfpathrectangle{\pgfqpoint{0.647939in}{0.492442in}}{\pgfqpoint{4.273799in}{2.331163in}}%
\pgfusepath{clip}%
\pgfsetroundcap%
\pgfsetroundjoin%
\definecolor{currentfill}{rgb}{0.500000,0.500000,0.500000}%
\pgfsetfillcolor{currentfill}%
\pgfsetfillopacity{0.300000}%
\pgfsetlinewidth{0.301125pt}%
\definecolor{currentstroke}{rgb}{0.500000,0.500000,0.500000}%
\pgfsetstrokecolor{currentstroke}%
\pgfsetstrokeopacity{0.300000}%
\pgfsetdash{}{0pt}%
\pgfpathmoveto{\pgfqpoint{0.000000in}{0.000000in}}%
\pgfpathlineto{\pgfqpoint{0.000000in}{0.000000in}}%
\pgfpathclose%
\pgfusepath{stroke,fill}%
\end{pgfscope}%
\begin{pgfscope}%
\pgfpathrectangle{\pgfqpoint{0.647939in}{0.492442in}}{\pgfqpoint{4.273799in}{2.331163in}}%
\pgfusepath{clip}%
\pgfsetroundcap%
\pgfsetroundjoin%
\pgfsetlinewidth{0.301125pt}%
\definecolor{currentstroke}{rgb}{0.500000,0.500000,0.500000}%
\pgfsetstrokecolor{currentstroke}%
\pgfsetstrokeopacity{0.300000}%
\pgfsetdash{}{0pt}%
\pgfpathmoveto{\pgfqpoint{2.335295in}{0.796577in}}%
\pgfusepath{stroke}%
\end{pgfscope}%
\begin{pgfscope}%
\pgfpathrectangle{\pgfqpoint{0.647939in}{0.492442in}}{\pgfqpoint{4.273799in}{2.331163in}}%
\pgfusepath{clip}%
\pgfsetroundcap%
\pgfsetroundjoin%
\definecolor{currentfill}{rgb}{0.500000,0.500000,0.500000}%
\pgfsetfillcolor{currentfill}%
\pgfsetfillopacity{0.300000}%
\pgfsetlinewidth{0.301125pt}%
\definecolor{currentstroke}{rgb}{0.500000,0.500000,0.500000}%
\pgfsetstrokecolor{currentstroke}%
\pgfsetstrokeopacity{0.300000}%
\pgfsetdash{}{0pt}%
\pgfpathmoveto{\pgfqpoint{0.000000in}{0.000000in}}%
\pgfpathlineto{\pgfqpoint{0.000000in}{0.000000in}}%
\pgfpathclose%
\pgfusepath{stroke,fill}%
\end{pgfscope}%
\begin{pgfscope}%
\pgfpathrectangle{\pgfqpoint{0.647939in}{0.492442in}}{\pgfqpoint{4.273799in}{2.331163in}}%
\pgfusepath{clip}%
\pgfsetroundcap%
\pgfsetroundjoin%
\pgfsetlinewidth{0.301125pt}%
\definecolor{currentstroke}{rgb}{0.500000,0.500000,0.500000}%
\pgfsetstrokecolor{currentstroke}%
\pgfsetstrokeopacity{0.300000}%
\pgfsetdash{}{0pt}%
\pgfpathmoveto{\pgfqpoint{2.584921in}{0.606115in}}%
\pgfusepath{stroke}%
\end{pgfscope}%
\begin{pgfscope}%
\pgfpathrectangle{\pgfqpoint{0.647939in}{0.492442in}}{\pgfqpoint{4.273799in}{2.331163in}}%
\pgfusepath{clip}%
\pgfsetroundcap%
\pgfsetroundjoin%
\definecolor{currentfill}{rgb}{0.500000,0.500000,0.500000}%
\pgfsetfillcolor{currentfill}%
\pgfsetfillopacity{0.300000}%
\pgfsetlinewidth{0.301125pt}%
\definecolor{currentstroke}{rgb}{0.500000,0.500000,0.500000}%
\pgfsetstrokecolor{currentstroke}%
\pgfsetstrokeopacity{0.300000}%
\pgfsetdash{}{0pt}%
\pgfpathmoveto{\pgfqpoint{0.000000in}{0.000000in}}%
\pgfpathlineto{\pgfqpoint{0.000000in}{0.000000in}}%
\pgfpathclose%
\pgfusepath{stroke,fill}%
\end{pgfscope}%
\begin{pgfscope}%
\pgfpathrectangle{\pgfqpoint{0.647939in}{0.492442in}}{\pgfqpoint{4.273799in}{2.331163in}}%
\pgfusepath{clip}%
\pgfsetroundcap%
\pgfsetroundjoin%
\pgfsetlinewidth{0.301125pt}%
\definecolor{currentstroke}{rgb}{0.500000,0.500000,0.500000}%
\pgfsetstrokecolor{currentstroke}%
\pgfsetstrokeopacity{0.300000}%
\pgfsetdash{}{0pt}%
\pgfpathmoveto{\pgfqpoint{2.031769in}{1.558876in}}%
\pgfusepath{stroke}%
\end{pgfscope}%
\begin{pgfscope}%
\pgfpathrectangle{\pgfqpoint{0.647939in}{0.492442in}}{\pgfqpoint{4.273799in}{2.331163in}}%
\pgfusepath{clip}%
\pgfsetroundcap%
\pgfsetroundjoin%
\definecolor{currentfill}{rgb}{0.500000,0.500000,0.500000}%
\pgfsetfillcolor{currentfill}%
\pgfsetfillopacity{0.300000}%
\pgfsetlinewidth{0.301125pt}%
\definecolor{currentstroke}{rgb}{0.500000,0.500000,0.500000}%
\pgfsetstrokecolor{currentstroke}%
\pgfsetstrokeopacity{0.300000}%
\pgfsetdash{}{0pt}%
\pgfpathmoveto{\pgfqpoint{0.000000in}{0.000000in}}%
\pgfpathlineto{\pgfqpoint{0.000000in}{0.000000in}}%
\pgfpathclose%
\pgfusepath{stroke,fill}%
\end{pgfscope}%
\begin{pgfscope}%
\pgfpathrectangle{\pgfqpoint{0.647939in}{0.492442in}}{\pgfqpoint{4.273799in}{2.331163in}}%
\pgfusepath{clip}%
\pgfsetroundcap%
\pgfsetroundjoin%
\pgfsetlinewidth{0.301125pt}%
\definecolor{currentstroke}{rgb}{0.500000,0.500000,0.500000}%
\pgfsetstrokecolor{currentstroke}%
\pgfsetstrokeopacity{0.300000}%
\pgfsetdash{}{0pt}%
\pgfpathmoveto{\pgfqpoint{2.001239in}{1.869203in}}%
\pgfusepath{stroke}%
\end{pgfscope}%
\begin{pgfscope}%
\pgfpathrectangle{\pgfqpoint{0.647939in}{0.492442in}}{\pgfqpoint{4.273799in}{2.331163in}}%
\pgfusepath{clip}%
\pgfsetroundcap%
\pgfsetroundjoin%
\definecolor{currentfill}{rgb}{0.500000,0.500000,0.500000}%
\pgfsetfillcolor{currentfill}%
\pgfsetfillopacity{0.300000}%
\pgfsetlinewidth{0.301125pt}%
\definecolor{currentstroke}{rgb}{0.500000,0.500000,0.500000}%
\pgfsetstrokecolor{currentstroke}%
\pgfsetstrokeopacity{0.300000}%
\pgfsetdash{}{0pt}%
\pgfpathmoveto{\pgfqpoint{0.000000in}{0.000000in}}%
\pgfpathlineto{\pgfqpoint{0.000000in}{0.000000in}}%
\pgfpathclose%
\pgfusepath{stroke,fill}%
\end{pgfscope}%
\begin{pgfscope}%
\pgfpathrectangle{\pgfqpoint{0.647939in}{0.492442in}}{\pgfqpoint{4.273799in}{2.331163in}}%
\pgfusepath{clip}%
\pgfsetroundcap%
\pgfsetroundjoin%
\pgfsetlinewidth{0.301125pt}%
\definecolor{currentstroke}{rgb}{0.500000,0.500000,0.500000}%
\pgfsetstrokecolor{currentstroke}%
\pgfsetstrokeopacity{0.300000}%
\pgfsetdash{}{0pt}%
\pgfpathmoveto{\pgfqpoint{2.434392in}{1.330188in}}%
\pgfusepath{stroke}%
\end{pgfscope}%
\begin{pgfscope}%
\pgfpathrectangle{\pgfqpoint{0.647939in}{0.492442in}}{\pgfqpoint{4.273799in}{2.331163in}}%
\pgfusepath{clip}%
\pgfsetroundcap%
\pgfsetroundjoin%
\definecolor{currentfill}{rgb}{0.500000,0.500000,0.500000}%
\pgfsetfillcolor{currentfill}%
\pgfsetfillopacity{0.300000}%
\pgfsetlinewidth{0.301125pt}%
\definecolor{currentstroke}{rgb}{0.500000,0.500000,0.500000}%
\pgfsetstrokecolor{currentstroke}%
\pgfsetstrokeopacity{0.300000}%
\pgfsetdash{}{0pt}%
\pgfpathmoveto{\pgfqpoint{0.000000in}{0.000000in}}%
\pgfpathlineto{\pgfqpoint{0.000000in}{0.000000in}}%
\pgfpathclose%
\pgfusepath{stroke,fill}%
\end{pgfscope}%
\begin{pgfscope}%
\pgfpathrectangle{\pgfqpoint{0.647939in}{0.492442in}}{\pgfqpoint{4.273799in}{2.331163in}}%
\pgfusepath{clip}%
\pgfsetroundcap%
\pgfsetroundjoin%
\pgfsetlinewidth{0.301125pt}%
\definecolor{currentstroke}{rgb}{0.500000,0.500000,0.500000}%
\pgfsetstrokecolor{currentstroke}%
\pgfsetstrokeopacity{0.300000}%
\pgfsetdash{}{0pt}%
\pgfpathmoveto{\pgfqpoint{3.095554in}{0.755966in}}%
\pgfusepath{stroke}%
\end{pgfscope}%
\begin{pgfscope}%
\pgfpathrectangle{\pgfqpoint{0.647939in}{0.492442in}}{\pgfqpoint{4.273799in}{2.331163in}}%
\pgfusepath{clip}%
\pgfsetroundcap%
\pgfsetroundjoin%
\definecolor{currentfill}{rgb}{0.500000,0.500000,0.500000}%
\pgfsetfillcolor{currentfill}%
\pgfsetfillopacity{0.300000}%
\pgfsetlinewidth{0.301125pt}%
\definecolor{currentstroke}{rgb}{0.500000,0.500000,0.500000}%
\pgfsetstrokecolor{currentstroke}%
\pgfsetstrokeopacity{0.300000}%
\pgfsetdash{}{0pt}%
\pgfpathmoveto{\pgfqpoint{0.000000in}{0.000000in}}%
\pgfpathlineto{\pgfqpoint{0.000000in}{0.000000in}}%
\pgfpathclose%
\pgfusepath{stroke,fill}%
\end{pgfscope}%
\begin{pgfscope}%
\pgfpathrectangle{\pgfqpoint{0.647939in}{0.492442in}}{\pgfqpoint{4.273799in}{2.331163in}}%
\pgfusepath{clip}%
\pgfsetroundcap%
\pgfsetroundjoin%
\pgfsetlinewidth{0.301125pt}%
\definecolor{currentstroke}{rgb}{0.500000,0.500000,0.500000}%
\pgfsetstrokecolor{currentstroke}%
\pgfsetstrokeopacity{0.300000}%
\pgfsetdash{}{0pt}%
\pgfpathmoveto{\pgfqpoint{2.666627in}{1.274570in}}%
\pgfusepath{stroke}%
\end{pgfscope}%
\begin{pgfscope}%
\pgfpathrectangle{\pgfqpoint{0.647939in}{0.492442in}}{\pgfqpoint{4.273799in}{2.331163in}}%
\pgfusepath{clip}%
\pgfsetroundcap%
\pgfsetroundjoin%
\definecolor{currentfill}{rgb}{0.500000,0.500000,0.500000}%
\pgfsetfillcolor{currentfill}%
\pgfsetfillopacity{0.300000}%
\pgfsetlinewidth{0.301125pt}%
\definecolor{currentstroke}{rgb}{0.500000,0.500000,0.500000}%
\pgfsetstrokecolor{currentstroke}%
\pgfsetstrokeopacity{0.300000}%
\pgfsetdash{}{0pt}%
\pgfpathmoveto{\pgfqpoint{0.000000in}{0.000000in}}%
\pgfpathlineto{\pgfqpoint{0.000000in}{0.000000in}}%
\pgfpathclose%
\pgfusepath{stroke,fill}%
\end{pgfscope}%
\begin{pgfscope}%
\pgfpathrectangle{\pgfqpoint{0.647939in}{0.492442in}}{\pgfqpoint{4.273799in}{2.331163in}}%
\pgfusepath{clip}%
\pgfsetroundcap%
\pgfsetroundjoin%
\pgfsetlinewidth{0.301125pt}%
\definecolor{currentstroke}{rgb}{0.500000,0.500000,0.500000}%
\pgfsetstrokecolor{currentstroke}%
\pgfsetstrokeopacity{0.300000}%
\pgfsetdash{}{0pt}%
\pgfpathmoveto{\pgfqpoint{3.602348in}{0.582053in}}%
\pgfusepath{stroke}%
\end{pgfscope}%
\begin{pgfscope}%
\pgfpathrectangle{\pgfqpoint{0.647939in}{0.492442in}}{\pgfqpoint{4.273799in}{2.331163in}}%
\pgfusepath{clip}%
\pgfsetroundcap%
\pgfsetroundjoin%
\definecolor{currentfill}{rgb}{0.500000,0.500000,0.500000}%
\pgfsetfillcolor{currentfill}%
\pgfsetfillopacity{0.300000}%
\pgfsetlinewidth{0.301125pt}%
\definecolor{currentstroke}{rgb}{0.500000,0.500000,0.500000}%
\pgfsetstrokecolor{currentstroke}%
\pgfsetstrokeopacity{0.300000}%
\pgfsetdash{}{0pt}%
\pgfpathmoveto{\pgfqpoint{0.000000in}{0.000000in}}%
\pgfpathlineto{\pgfqpoint{0.000000in}{0.000000in}}%
\pgfpathclose%
\pgfusepath{stroke,fill}%
\end{pgfscope}%
\begin{pgfscope}%
\pgfpathrectangle{\pgfqpoint{0.647939in}{0.492442in}}{\pgfqpoint{4.273799in}{2.331163in}}%
\pgfusepath{clip}%
\pgfsetroundcap%
\pgfsetroundjoin%
\pgfsetlinewidth{0.301125pt}%
\definecolor{currentstroke}{rgb}{0.500000,0.500000,0.500000}%
\pgfsetstrokecolor{currentstroke}%
\pgfsetstrokeopacity{0.300000}%
\pgfsetdash{}{0pt}%
\pgfpathmoveto{\pgfqpoint{3.699031in}{0.581609in}}%
\pgfusepath{stroke}%
\end{pgfscope}%
\begin{pgfscope}%
\pgfpathrectangle{\pgfqpoint{0.647939in}{0.492442in}}{\pgfqpoint{4.273799in}{2.331163in}}%
\pgfusepath{clip}%
\pgfsetroundcap%
\pgfsetroundjoin%
\definecolor{currentfill}{rgb}{0.500000,0.500000,0.500000}%
\pgfsetfillcolor{currentfill}%
\pgfsetfillopacity{0.300000}%
\pgfsetlinewidth{0.301125pt}%
\definecolor{currentstroke}{rgb}{0.500000,0.500000,0.500000}%
\pgfsetstrokecolor{currentstroke}%
\pgfsetstrokeopacity{0.300000}%
\pgfsetdash{}{0pt}%
\pgfpathmoveto{\pgfqpoint{0.000000in}{0.000000in}}%
\pgfpathlineto{\pgfqpoint{0.000000in}{0.000000in}}%
\pgfpathclose%
\pgfusepath{stroke,fill}%
\end{pgfscope}%
\begin{pgfscope}%
\pgfpathrectangle{\pgfqpoint{0.647939in}{0.492442in}}{\pgfqpoint{4.273799in}{2.331163in}}%
\pgfusepath{clip}%
\pgfsetroundcap%
\pgfsetroundjoin%
\pgfsetlinewidth{0.301125pt}%
\definecolor{currentstroke}{rgb}{0.500000,0.500000,0.500000}%
\pgfsetstrokecolor{currentstroke}%
\pgfsetstrokeopacity{0.300000}%
\pgfsetdash{}{0pt}%
\pgfpathmoveto{\pgfqpoint{3.797688in}{0.582370in}}%
\pgfusepath{stroke}%
\end{pgfscope}%
\begin{pgfscope}%
\pgfpathrectangle{\pgfqpoint{0.647939in}{0.492442in}}{\pgfqpoint{4.273799in}{2.331163in}}%
\pgfusepath{clip}%
\pgfsetroundcap%
\pgfsetroundjoin%
\definecolor{currentfill}{rgb}{0.500000,0.500000,0.500000}%
\pgfsetfillcolor{currentfill}%
\pgfsetfillopacity{0.300000}%
\pgfsetlinewidth{0.301125pt}%
\definecolor{currentstroke}{rgb}{0.500000,0.500000,0.500000}%
\pgfsetstrokecolor{currentstroke}%
\pgfsetstrokeopacity{0.300000}%
\pgfsetdash{}{0pt}%
\pgfpathmoveto{\pgfqpoint{0.000000in}{0.000000in}}%
\pgfpathlineto{\pgfqpoint{0.000000in}{0.000000in}}%
\pgfpathclose%
\pgfusepath{stroke,fill}%
\end{pgfscope}%
\begin{pgfscope}%
\pgfpathrectangle{\pgfqpoint{0.647939in}{0.492442in}}{\pgfqpoint{4.273799in}{2.331163in}}%
\pgfusepath{clip}%
\pgfsetroundcap%
\pgfsetroundjoin%
\pgfsetlinewidth{0.301125pt}%
\definecolor{currentstroke}{rgb}{0.500000,0.500000,0.500000}%
\pgfsetstrokecolor{currentstroke}%
\pgfsetstrokeopacity{0.300000}%
\pgfsetdash{}{0pt}%
\pgfpathmoveto{\pgfqpoint{3.898703in}{0.584397in}}%
\pgfusepath{stroke}%
\end{pgfscope}%
\begin{pgfscope}%
\pgfpathrectangle{\pgfqpoint{0.647939in}{0.492442in}}{\pgfqpoint{4.273799in}{2.331163in}}%
\pgfusepath{clip}%
\pgfsetroundcap%
\pgfsetroundjoin%
\definecolor{currentfill}{rgb}{0.500000,0.500000,0.500000}%
\pgfsetfillcolor{currentfill}%
\pgfsetfillopacity{0.300000}%
\pgfsetlinewidth{0.301125pt}%
\definecolor{currentstroke}{rgb}{0.500000,0.500000,0.500000}%
\pgfsetstrokecolor{currentstroke}%
\pgfsetstrokeopacity{0.300000}%
\pgfsetdash{}{0pt}%
\pgfpathmoveto{\pgfqpoint{0.000000in}{0.000000in}}%
\pgfpathlineto{\pgfqpoint{0.000000in}{0.000000in}}%
\pgfpathclose%
\pgfusepath{stroke,fill}%
\end{pgfscope}%
\begin{pgfscope}%
\pgfpathrectangle{\pgfqpoint{0.647939in}{0.492442in}}{\pgfqpoint{4.273799in}{2.331163in}}%
\pgfusepath{clip}%
\pgfsetroundcap%
\pgfsetroundjoin%
\pgfsetlinewidth{0.301125pt}%
\definecolor{currentstroke}{rgb}{0.500000,0.500000,0.500000}%
\pgfsetstrokecolor{currentstroke}%
\pgfsetstrokeopacity{0.300000}%
\pgfsetdash{}{0pt}%
\pgfpathmoveto{\pgfqpoint{4.001993in}{0.587691in}}%
\pgfusepath{stroke}%
\end{pgfscope}%
\begin{pgfscope}%
\pgfpathrectangle{\pgfqpoint{0.647939in}{0.492442in}}{\pgfqpoint{4.273799in}{2.331163in}}%
\pgfusepath{clip}%
\pgfsetroundcap%
\pgfsetroundjoin%
\definecolor{currentfill}{rgb}{0.500000,0.500000,0.500000}%
\pgfsetfillcolor{currentfill}%
\pgfsetfillopacity{0.300000}%
\pgfsetlinewidth{0.301125pt}%
\definecolor{currentstroke}{rgb}{0.500000,0.500000,0.500000}%
\pgfsetstrokecolor{currentstroke}%
\pgfsetstrokeopacity{0.300000}%
\pgfsetdash{}{0pt}%
\pgfpathmoveto{\pgfqpoint{0.000000in}{0.000000in}}%
\pgfpathlineto{\pgfqpoint{0.000000in}{0.000000in}}%
\pgfpathclose%
\pgfusepath{stroke,fill}%
\end{pgfscope}%
\begin{pgfscope}%
\pgfpathrectangle{\pgfqpoint{0.647939in}{0.492442in}}{\pgfqpoint{4.273799in}{2.331163in}}%
\pgfusepath{clip}%
\pgfsetroundcap%
\pgfsetroundjoin%
\pgfsetlinewidth{0.301125pt}%
\definecolor{currentstroke}{rgb}{0.500000,0.500000,0.500000}%
\pgfsetstrokecolor{currentstroke}%
\pgfsetstrokeopacity{0.300000}%
\pgfsetdash{}{0pt}%
\pgfpathmoveto{\pgfqpoint{3.028438in}{1.263494in}}%
\pgfusepath{stroke}%
\end{pgfscope}%
\begin{pgfscope}%
\pgfpathrectangle{\pgfqpoint{0.647939in}{0.492442in}}{\pgfqpoint{4.273799in}{2.331163in}}%
\pgfusepath{clip}%
\pgfsetroundcap%
\pgfsetroundjoin%
\definecolor{currentfill}{rgb}{0.500000,0.500000,0.500000}%
\pgfsetfillcolor{currentfill}%
\pgfsetfillopacity{0.300000}%
\pgfsetlinewidth{0.301125pt}%
\definecolor{currentstroke}{rgb}{0.500000,0.500000,0.500000}%
\pgfsetstrokecolor{currentstroke}%
\pgfsetstrokeopacity{0.300000}%
\pgfsetdash{}{0pt}%
\pgfpathmoveto{\pgfqpoint{0.000000in}{0.000000in}}%
\pgfpathlineto{\pgfqpoint{0.000000in}{0.000000in}}%
\pgfpathclose%
\pgfusepath{stroke,fill}%
\end{pgfscope}%
\begin{pgfscope}%
\pgfpathrectangle{\pgfqpoint{0.647939in}{0.492442in}}{\pgfqpoint{4.273799in}{2.331163in}}%
\pgfusepath{clip}%
\pgfsetroundcap%
\pgfsetroundjoin%
\pgfsetlinewidth{0.301125pt}%
\definecolor{currentstroke}{rgb}{0.500000,0.500000,0.500000}%
\pgfsetstrokecolor{currentstroke}%
\pgfsetstrokeopacity{0.300000}%
\pgfsetdash{}{0pt}%
\pgfpathmoveto{\pgfqpoint{3.976043in}{0.842741in}}%
\pgfusepath{stroke}%
\end{pgfscope}%
\begin{pgfscope}%
\pgfpathrectangle{\pgfqpoint{0.647939in}{0.492442in}}{\pgfqpoint{4.273799in}{2.331163in}}%
\pgfusepath{clip}%
\pgfsetroundcap%
\pgfsetroundjoin%
\definecolor{currentfill}{rgb}{0.500000,0.500000,0.500000}%
\pgfsetfillcolor{currentfill}%
\pgfsetfillopacity{0.300000}%
\pgfsetlinewidth{0.301125pt}%
\definecolor{currentstroke}{rgb}{0.500000,0.500000,0.500000}%
\pgfsetstrokecolor{currentstroke}%
\pgfsetstrokeopacity{0.300000}%
\pgfsetdash{}{0pt}%
\pgfpathmoveto{\pgfqpoint{0.000000in}{0.000000in}}%
\pgfpathlineto{\pgfqpoint{0.000000in}{0.000000in}}%
\pgfpathclose%
\pgfusepath{stroke,fill}%
\end{pgfscope}%
\begin{pgfscope}%
\pgfpathrectangle{\pgfqpoint{0.647939in}{0.492442in}}{\pgfqpoint{4.273799in}{2.331163in}}%
\pgfusepath{clip}%
\pgfsetroundcap%
\pgfsetroundjoin%
\pgfsetlinewidth{0.301125pt}%
\definecolor{currentstroke}{rgb}{0.500000,0.500000,0.500000}%
\pgfsetstrokecolor{currentstroke}%
\pgfsetstrokeopacity{0.300000}%
\pgfsetdash{}{0pt}%
\pgfpathmoveto{\pgfqpoint{3.397206in}{1.259885in}}%
\pgfusepath{stroke}%
\end{pgfscope}%
\begin{pgfscope}%
\pgfpathrectangle{\pgfqpoint{0.647939in}{0.492442in}}{\pgfqpoint{4.273799in}{2.331163in}}%
\pgfusepath{clip}%
\pgfsetroundcap%
\pgfsetroundjoin%
\definecolor{currentfill}{rgb}{0.500000,0.500000,0.500000}%
\pgfsetfillcolor{currentfill}%
\pgfsetfillopacity{0.300000}%
\pgfsetlinewidth{0.301125pt}%
\definecolor{currentstroke}{rgb}{0.500000,0.500000,0.500000}%
\pgfsetstrokecolor{currentstroke}%
\pgfsetstrokeopacity{0.300000}%
\pgfsetdash{}{0pt}%
\pgfpathmoveto{\pgfqpoint{0.000000in}{0.000000in}}%
\pgfpathlineto{\pgfqpoint{0.000000in}{0.000000in}}%
\pgfpathclose%
\pgfusepath{stroke,fill}%
\end{pgfscope}%
\begin{pgfscope}%
\pgfpathrectangle{\pgfqpoint{0.647939in}{0.492442in}}{\pgfqpoint{4.273799in}{2.331163in}}%
\pgfusepath{clip}%
\pgfsetroundcap%
\pgfsetroundjoin%
\pgfsetlinewidth{0.301125pt}%
\definecolor{currentstroke}{rgb}{0.500000,0.500000,0.500000}%
\pgfsetstrokecolor{currentstroke}%
\pgfsetstrokeopacity{0.300000}%
\pgfsetdash{}{0pt}%
\pgfpathmoveto{\pgfqpoint{4.031241in}{1.177215in}}%
\pgfusepath{stroke}%
\end{pgfscope}%
\begin{pgfscope}%
\pgfpathrectangle{\pgfqpoint{0.647939in}{0.492442in}}{\pgfqpoint{4.273799in}{2.331163in}}%
\pgfusepath{clip}%
\pgfsetroundcap%
\pgfsetroundjoin%
\definecolor{currentfill}{rgb}{0.500000,0.500000,0.500000}%
\pgfsetfillcolor{currentfill}%
\pgfsetfillopacity{0.300000}%
\pgfsetlinewidth{0.301125pt}%
\definecolor{currentstroke}{rgb}{0.500000,0.500000,0.500000}%
\pgfsetstrokecolor{currentstroke}%
\pgfsetstrokeopacity{0.300000}%
\pgfsetdash{}{0pt}%
\pgfpathmoveto{\pgfqpoint{0.000000in}{0.000000in}}%
\pgfpathlineto{\pgfqpoint{0.000000in}{0.000000in}}%
\pgfpathclose%
\pgfusepath{stroke,fill}%
\end{pgfscope}%
\begin{pgfscope}%
\pgfpathrectangle{\pgfqpoint{0.647939in}{0.492442in}}{\pgfqpoint{4.273799in}{2.331163in}}%
\pgfusepath{clip}%
\pgfsetroundcap%
\pgfsetroundjoin%
\pgfsetlinewidth{0.301125pt}%
\definecolor{currentstroke}{rgb}{0.500000,0.500000,0.500000}%
\pgfsetstrokecolor{currentstroke}%
\pgfsetstrokeopacity{0.300000}%
\pgfsetdash{}{0pt}%
\pgfpathmoveto{\pgfqpoint{3.901836in}{1.369952in}}%
\pgfusepath{stroke}%
\end{pgfscope}%
\begin{pgfscope}%
\pgfpathrectangle{\pgfqpoint{0.647939in}{0.492442in}}{\pgfqpoint{4.273799in}{2.331163in}}%
\pgfusepath{clip}%
\pgfsetroundcap%
\pgfsetroundjoin%
\definecolor{currentfill}{rgb}{0.500000,0.500000,0.500000}%
\pgfsetfillcolor{currentfill}%
\pgfsetfillopacity{0.300000}%
\pgfsetlinewidth{0.301125pt}%
\definecolor{currentstroke}{rgb}{0.500000,0.500000,0.500000}%
\pgfsetstrokecolor{currentstroke}%
\pgfsetstrokeopacity{0.300000}%
\pgfsetdash{}{0pt}%
\pgfpathmoveto{\pgfqpoint{0.000000in}{0.000000in}}%
\pgfpathlineto{\pgfqpoint{0.000000in}{0.000000in}}%
\pgfpathclose%
\pgfusepath{stroke,fill}%
\end{pgfscope}%
\begin{pgfscope}%
\pgfpathrectangle{\pgfqpoint{0.647939in}{0.492442in}}{\pgfqpoint{4.273799in}{2.331163in}}%
\pgfusepath{clip}%
\pgfsetroundcap%
\pgfsetroundjoin%
\pgfsetlinewidth{0.301125pt}%
\definecolor{currentstroke}{rgb}{0.500000,0.500000,0.500000}%
\pgfsetstrokecolor{currentstroke}%
\pgfsetstrokeopacity{0.300000}%
\pgfsetdash{}{0pt}%
\pgfpathmoveto{\pgfqpoint{4.257418in}{1.474785in}}%
\pgfusepath{stroke}%
\end{pgfscope}%
\begin{pgfscope}%
\pgfpathrectangle{\pgfqpoint{0.647939in}{0.492442in}}{\pgfqpoint{4.273799in}{2.331163in}}%
\pgfusepath{clip}%
\pgfsetroundcap%
\pgfsetroundjoin%
\definecolor{currentfill}{rgb}{0.500000,0.500000,0.500000}%
\pgfsetfillcolor{currentfill}%
\pgfsetfillopacity{0.300000}%
\pgfsetlinewidth{0.301125pt}%
\definecolor{currentstroke}{rgb}{0.500000,0.500000,0.500000}%
\pgfsetstrokecolor{currentstroke}%
\pgfsetstrokeopacity{0.300000}%
\pgfsetdash{}{0pt}%
\pgfpathmoveto{\pgfqpoint{0.000000in}{0.000000in}}%
\pgfpathlineto{\pgfqpoint{0.000000in}{0.000000in}}%
\pgfpathclose%
\pgfusepath{stroke,fill}%
\end{pgfscope}%
\begin{pgfscope}%
\pgfpathrectangle{\pgfqpoint{0.647939in}{0.492442in}}{\pgfqpoint{4.273799in}{2.331163in}}%
\pgfusepath{clip}%
\pgfsetroundcap%
\pgfsetroundjoin%
\pgfsetlinewidth{0.301125pt}%
\definecolor{currentstroke}{rgb}{0.500000,0.500000,0.500000}%
\pgfsetstrokecolor{currentstroke}%
\pgfsetstrokeopacity{0.300000}%
\pgfsetdash{}{0pt}%
\pgfpathmoveto{\pgfqpoint{4.352426in}{1.709992in}}%
\pgfusepath{stroke}%
\end{pgfscope}%
\begin{pgfscope}%
\pgfpathrectangle{\pgfqpoint{0.647939in}{0.492442in}}{\pgfqpoint{4.273799in}{2.331163in}}%
\pgfusepath{clip}%
\pgfsetroundcap%
\pgfsetroundjoin%
\definecolor{currentfill}{rgb}{0.500000,0.500000,0.500000}%
\pgfsetfillcolor{currentfill}%
\pgfsetfillopacity{0.300000}%
\pgfsetlinewidth{0.301125pt}%
\definecolor{currentstroke}{rgb}{0.500000,0.500000,0.500000}%
\pgfsetstrokecolor{currentstroke}%
\pgfsetstrokeopacity{0.300000}%
\pgfsetdash{}{0pt}%
\pgfpathmoveto{\pgfqpoint{0.000000in}{0.000000in}}%
\pgfpathlineto{\pgfqpoint{0.000000in}{0.000000in}}%
\pgfpathclose%
\pgfusepath{stroke,fill}%
\end{pgfscope}%
\begin{pgfscope}%
\pgfpathrectangle{\pgfqpoint{0.647939in}{0.492442in}}{\pgfqpoint{4.273799in}{2.331163in}}%
\pgfusepath{clip}%
\pgfsetroundcap%
\pgfsetroundjoin%
\pgfsetlinewidth{0.301125pt}%
\definecolor{currentstroke}{rgb}{0.500000,0.500000,0.500000}%
\pgfsetstrokecolor{currentstroke}%
\pgfsetstrokeopacity{0.300000}%
\pgfsetdash{}{0pt}%
\pgfpathmoveto{\pgfqpoint{4.535303in}{1.877664in}}%
\pgfusepath{stroke}%
\end{pgfscope}%
\begin{pgfscope}%
\pgfpathrectangle{\pgfqpoint{0.647939in}{0.492442in}}{\pgfqpoint{4.273799in}{2.331163in}}%
\pgfusepath{clip}%
\pgfsetroundcap%
\pgfsetroundjoin%
\definecolor{currentfill}{rgb}{0.500000,0.500000,0.500000}%
\pgfsetfillcolor{currentfill}%
\pgfsetfillopacity{0.300000}%
\pgfsetlinewidth{0.301125pt}%
\definecolor{currentstroke}{rgb}{0.500000,0.500000,0.500000}%
\pgfsetstrokecolor{currentstroke}%
\pgfsetstrokeopacity{0.300000}%
\pgfsetdash{}{0pt}%
\pgfpathmoveto{\pgfqpoint{0.000000in}{0.000000in}}%
\pgfpathlineto{\pgfqpoint{0.000000in}{0.000000in}}%
\pgfpathclose%
\pgfusepath{stroke,fill}%
\end{pgfscope}%
\begin{pgfscope}%
\pgfpathrectangle{\pgfqpoint{0.647939in}{0.492442in}}{\pgfqpoint{4.273799in}{2.331163in}}%
\pgfusepath{clip}%
\pgfsetroundcap%
\pgfsetroundjoin%
\pgfsetlinewidth{0.301125pt}%
\definecolor{currentstroke}{rgb}{0.500000,0.500000,0.500000}%
\pgfsetstrokecolor{currentstroke}%
\pgfsetstrokeopacity{0.300000}%
\pgfsetdash{}{0pt}%
\pgfpathmoveto{\pgfqpoint{4.729902in}{1.846344in}}%
\pgfusepath{stroke}%
\end{pgfscope}%
\begin{pgfscope}%
\pgfpathrectangle{\pgfqpoint{0.647939in}{0.492442in}}{\pgfqpoint{4.273799in}{2.331163in}}%
\pgfusepath{clip}%
\pgfsetroundcap%
\pgfsetroundjoin%
\definecolor{currentfill}{rgb}{0.500000,0.500000,0.500000}%
\pgfsetfillcolor{currentfill}%
\pgfsetfillopacity{0.300000}%
\pgfsetlinewidth{0.301125pt}%
\definecolor{currentstroke}{rgb}{0.500000,0.500000,0.500000}%
\pgfsetstrokecolor{currentstroke}%
\pgfsetstrokeopacity{0.300000}%
\pgfsetdash{}{0pt}%
\pgfpathmoveto{\pgfqpoint{0.000000in}{0.000000in}}%
\pgfpathlineto{\pgfqpoint{0.000000in}{0.000000in}}%
\pgfpathclose%
\pgfusepath{stroke,fill}%
\end{pgfscope}%
\begin{pgfscope}%
\pgfpathrectangle{\pgfqpoint{0.647939in}{0.492442in}}{\pgfqpoint{4.273799in}{2.331163in}}%
\pgfusepath{clip}%
\pgfsetroundcap%
\pgfsetroundjoin%
\pgfsetlinewidth{0.301125pt}%
\definecolor{currentstroke}{rgb}{0.500000,0.500000,0.500000}%
\pgfsetstrokecolor{currentstroke}%
\pgfsetstrokeopacity{0.300000}%
\pgfsetdash{}{0pt}%
\pgfpathmoveto{\pgfqpoint{4.833106in}{2.014655in}}%
\pgfusepath{stroke}%
\end{pgfscope}%
\begin{pgfscope}%
\pgfpathrectangle{\pgfqpoint{0.647939in}{0.492442in}}{\pgfqpoint{4.273799in}{2.331163in}}%
\pgfusepath{clip}%
\pgfsetroundcap%
\pgfsetroundjoin%
\definecolor{currentfill}{rgb}{0.500000,0.500000,0.500000}%
\pgfsetfillcolor{currentfill}%
\pgfsetfillopacity{0.300000}%
\pgfsetlinewidth{0.301125pt}%
\definecolor{currentstroke}{rgb}{0.500000,0.500000,0.500000}%
\pgfsetstrokecolor{currentstroke}%
\pgfsetstrokeopacity{0.300000}%
\pgfsetdash{}{0pt}%
\pgfpathmoveto{\pgfqpoint{0.000000in}{0.000000in}}%
\pgfpathlineto{\pgfqpoint{0.000000in}{0.000000in}}%
\pgfpathclose%
\pgfusepath{stroke,fill}%
\end{pgfscope}%
\begin{pgfscope}%
\pgfpathrectangle{\pgfqpoint{0.647939in}{0.492442in}}{\pgfqpoint{4.273799in}{2.331163in}}%
\pgfusepath{clip}%
\pgfsetroundcap%
\pgfsetroundjoin%
\pgfsetlinewidth{0.301125pt}%
\definecolor{currentstroke}{rgb}{0.500000,0.500000,0.500000}%
\pgfsetstrokecolor{currentstroke}%
\pgfsetstrokeopacity{0.300000}%
\pgfsetdash{}{0pt}%
\pgfpathmoveto{\pgfqpoint{4.891624in}{2.074304in}}%
\pgfusepath{stroke}%
\end{pgfscope}%
\begin{pgfscope}%
\pgfpathrectangle{\pgfqpoint{0.647939in}{0.492442in}}{\pgfqpoint{4.273799in}{2.331163in}}%
\pgfusepath{clip}%
\pgfsetroundcap%
\pgfsetroundjoin%
\definecolor{currentfill}{rgb}{0.500000,0.500000,0.500000}%
\pgfsetfillcolor{currentfill}%
\pgfsetfillopacity{0.300000}%
\pgfsetlinewidth{0.301125pt}%
\definecolor{currentstroke}{rgb}{0.500000,0.500000,0.500000}%
\pgfsetstrokecolor{currentstroke}%
\pgfsetstrokeopacity{0.300000}%
\pgfsetdash{}{0pt}%
\pgfpathmoveto{\pgfqpoint{0.000000in}{0.000000in}}%
\pgfpathlineto{\pgfqpoint{0.000000in}{0.000000in}}%
\pgfpathclose%
\pgfusepath{stroke,fill}%
\end{pgfscope}%
\begin{pgfscope}%
\pgfpathrectangle{\pgfqpoint{0.647939in}{0.492442in}}{\pgfqpoint{4.273799in}{2.331163in}}%
\pgfusepath{clip}%
\pgfsetroundcap%
\pgfsetroundjoin%
\pgfsetlinewidth{0.301125pt}%
\definecolor{currentstroke}{rgb}{0.500000,0.500000,0.500000}%
\pgfsetstrokecolor{currentstroke}%
\pgfsetstrokeopacity{0.300000}%
\pgfsetdash{}{0pt}%
\pgfpathmoveto{\pgfqpoint{4.589386in}{2.744546in}}%
\pgfusepath{stroke}%
\end{pgfscope}%
\begin{pgfscope}%
\pgfpathrectangle{\pgfqpoint{0.647939in}{0.492442in}}{\pgfqpoint{4.273799in}{2.331163in}}%
\pgfusepath{clip}%
\pgfsetroundcap%
\pgfsetroundjoin%
\definecolor{currentfill}{rgb}{0.500000,0.500000,0.500000}%
\pgfsetfillcolor{currentfill}%
\pgfsetfillopacity{0.300000}%
\pgfsetlinewidth{0.301125pt}%
\definecolor{currentstroke}{rgb}{0.500000,0.500000,0.500000}%
\pgfsetstrokecolor{currentstroke}%
\pgfsetstrokeopacity{0.300000}%
\pgfsetdash{}{0pt}%
\pgfpathmoveto{\pgfqpoint{0.000000in}{0.000000in}}%
\pgfpathlineto{\pgfqpoint{0.000000in}{0.000000in}}%
\pgfpathclose%
\pgfusepath{stroke,fill}%
\end{pgfscope}%
\begin{pgfscope}%
\pgfpathrectangle{\pgfqpoint{0.647939in}{0.492442in}}{\pgfqpoint{4.273799in}{2.331163in}}%
\pgfusepath{clip}%
\pgfsetroundcap%
\pgfsetroundjoin%
\pgfsetlinewidth{0.301125pt}%
\definecolor{currentstroke}{rgb}{0.500000,0.500000,0.500000}%
\pgfsetstrokecolor{currentstroke}%
\pgfsetstrokeopacity{0.300000}%
\pgfsetdash{}{0pt}%
\pgfpathmoveto{\pgfqpoint{4.482900in}{2.690363in}}%
\pgfusepath{stroke}%
\end{pgfscope}%
\begin{pgfscope}%
\pgfpathrectangle{\pgfqpoint{0.647939in}{0.492442in}}{\pgfqpoint{4.273799in}{2.331163in}}%
\pgfusepath{clip}%
\pgfsetroundcap%
\pgfsetroundjoin%
\definecolor{currentfill}{rgb}{0.500000,0.500000,0.500000}%
\pgfsetfillcolor{currentfill}%
\pgfsetfillopacity{0.300000}%
\pgfsetlinewidth{0.301125pt}%
\definecolor{currentstroke}{rgb}{0.500000,0.500000,0.500000}%
\pgfsetstrokecolor{currentstroke}%
\pgfsetstrokeopacity{0.300000}%
\pgfsetdash{}{0pt}%
\pgfpathmoveto{\pgfqpoint{0.000000in}{0.000000in}}%
\pgfpathlineto{\pgfqpoint{0.000000in}{0.000000in}}%
\pgfpathclose%
\pgfusepath{stroke,fill}%
\end{pgfscope}%
\begin{pgfscope}%
\pgfpathrectangle{\pgfqpoint{0.647939in}{0.492442in}}{\pgfqpoint{4.273799in}{2.331163in}}%
\pgfusepath{clip}%
\pgfsetroundcap%
\pgfsetroundjoin%
\pgfsetlinewidth{0.301125pt}%
\definecolor{currentstroke}{rgb}{0.500000,0.500000,0.500000}%
\pgfsetstrokecolor{currentstroke}%
\pgfsetstrokeopacity{0.300000}%
\pgfsetdash{}{0pt}%
\pgfpathmoveto{\pgfqpoint{4.412753in}{2.614986in}}%
\pgfusepath{stroke}%
\end{pgfscope}%
\begin{pgfscope}%
\pgfpathrectangle{\pgfqpoint{0.647939in}{0.492442in}}{\pgfqpoint{4.273799in}{2.331163in}}%
\pgfusepath{clip}%
\pgfsetroundcap%
\pgfsetroundjoin%
\definecolor{currentfill}{rgb}{0.500000,0.500000,0.500000}%
\pgfsetfillcolor{currentfill}%
\pgfsetfillopacity{0.300000}%
\pgfsetlinewidth{0.301125pt}%
\definecolor{currentstroke}{rgb}{0.500000,0.500000,0.500000}%
\pgfsetstrokecolor{currentstroke}%
\pgfsetstrokeopacity{0.300000}%
\pgfsetdash{}{0pt}%
\pgfpathmoveto{\pgfqpoint{0.000000in}{0.000000in}}%
\pgfpathlineto{\pgfqpoint{0.000000in}{0.000000in}}%
\pgfpathclose%
\pgfusepath{stroke,fill}%
\end{pgfscope}%
\begin{pgfscope}%
\pgfpathrectangle{\pgfqpoint{0.647939in}{0.492442in}}{\pgfqpoint{4.273799in}{2.331163in}}%
\pgfusepath{clip}%
\pgfsetroundcap%
\pgfsetroundjoin%
\pgfsetlinewidth{0.301125pt}%
\definecolor{currentstroke}{rgb}{0.500000,0.500000,0.500000}%
\pgfsetstrokecolor{currentstroke}%
\pgfsetstrokeopacity{0.300000}%
\pgfsetdash{}{0pt}%
\pgfpathmoveto{\pgfqpoint{4.316579in}{2.556413in}}%
\pgfusepath{stroke}%
\end{pgfscope}%
\begin{pgfscope}%
\pgfpathrectangle{\pgfqpoint{0.647939in}{0.492442in}}{\pgfqpoint{4.273799in}{2.331163in}}%
\pgfusepath{clip}%
\pgfsetroundcap%
\pgfsetroundjoin%
\definecolor{currentfill}{rgb}{0.500000,0.500000,0.500000}%
\pgfsetfillcolor{currentfill}%
\pgfsetfillopacity{0.300000}%
\pgfsetlinewidth{0.301125pt}%
\definecolor{currentstroke}{rgb}{0.500000,0.500000,0.500000}%
\pgfsetstrokecolor{currentstroke}%
\pgfsetstrokeopacity{0.300000}%
\pgfsetdash{}{0pt}%
\pgfpathmoveto{\pgfqpoint{0.000000in}{0.000000in}}%
\pgfpathlineto{\pgfqpoint{0.000000in}{0.000000in}}%
\pgfpathclose%
\pgfusepath{stroke,fill}%
\end{pgfscope}%
\begin{pgfscope}%
\pgfpathrectangle{\pgfqpoint{0.647939in}{0.492442in}}{\pgfqpoint{4.273799in}{2.331163in}}%
\pgfusepath{clip}%
\pgfsetroundcap%
\pgfsetroundjoin%
\pgfsetlinewidth{0.301125pt}%
\definecolor{currentstroke}{rgb}{0.500000,0.500000,0.500000}%
\pgfsetstrokecolor{currentstroke}%
\pgfsetstrokeopacity{0.300000}%
\pgfsetdash{}{0pt}%
\pgfpathmoveto{\pgfqpoint{4.195984in}{2.550835in}}%
\pgfusepath{stroke}%
\end{pgfscope}%
\begin{pgfscope}%
\pgfpathrectangle{\pgfqpoint{0.647939in}{0.492442in}}{\pgfqpoint{4.273799in}{2.331163in}}%
\pgfusepath{clip}%
\pgfsetroundcap%
\pgfsetroundjoin%
\definecolor{currentfill}{rgb}{0.500000,0.500000,0.500000}%
\pgfsetfillcolor{currentfill}%
\pgfsetfillopacity{0.300000}%
\pgfsetlinewidth{0.301125pt}%
\definecolor{currentstroke}{rgb}{0.500000,0.500000,0.500000}%
\pgfsetstrokecolor{currentstroke}%
\pgfsetstrokeopacity{0.300000}%
\pgfsetdash{}{0pt}%
\pgfpathmoveto{\pgfqpoint{0.000000in}{0.000000in}}%
\pgfpathlineto{\pgfqpoint{0.000000in}{0.000000in}}%
\pgfpathclose%
\pgfusepath{stroke,fill}%
\end{pgfscope}%
\begin{pgfscope}%
\pgfpathrectangle{\pgfqpoint{0.647939in}{0.492442in}}{\pgfqpoint{4.273799in}{2.331163in}}%
\pgfusepath{clip}%
\pgfsetroundcap%
\pgfsetroundjoin%
\pgfsetlinewidth{0.301125pt}%
\definecolor{currentstroke}{rgb}{0.500000,0.500000,0.500000}%
\pgfsetstrokecolor{currentstroke}%
\pgfsetstrokeopacity{0.300000}%
\pgfsetdash{}{0pt}%
\pgfpathmoveto{\pgfqpoint{4.213212in}{2.166745in}}%
\pgfusepath{stroke}%
\end{pgfscope}%
\begin{pgfscope}%
\pgfpathrectangle{\pgfqpoint{0.647939in}{0.492442in}}{\pgfqpoint{4.273799in}{2.331163in}}%
\pgfusepath{clip}%
\pgfsetroundcap%
\pgfsetroundjoin%
\definecolor{currentfill}{rgb}{0.500000,0.500000,0.500000}%
\pgfsetfillcolor{currentfill}%
\pgfsetfillopacity{0.300000}%
\pgfsetlinewidth{0.301125pt}%
\definecolor{currentstroke}{rgb}{0.500000,0.500000,0.500000}%
\pgfsetstrokecolor{currentstroke}%
\pgfsetstrokeopacity{0.300000}%
\pgfsetdash{}{0pt}%
\pgfpathmoveto{\pgfqpoint{0.000000in}{0.000000in}}%
\pgfpathlineto{\pgfqpoint{0.000000in}{0.000000in}}%
\pgfpathclose%
\pgfusepath{stroke,fill}%
\end{pgfscope}%
\begin{pgfscope}%
\pgfpathrectangle{\pgfqpoint{0.647939in}{0.492442in}}{\pgfqpoint{4.273799in}{2.331163in}}%
\pgfusepath{clip}%
\pgfsetroundcap%
\pgfsetroundjoin%
\pgfsetlinewidth{0.301125pt}%
\definecolor{currentstroke}{rgb}{0.500000,0.500000,0.500000}%
\pgfsetstrokecolor{currentstroke}%
\pgfsetstrokeopacity{0.300000}%
\pgfsetdash{}{0pt}%
\pgfpathmoveto{\pgfqpoint{4.047185in}{2.369751in}}%
\pgfusepath{stroke}%
\end{pgfscope}%
\begin{pgfscope}%
\pgfpathrectangle{\pgfqpoint{0.647939in}{0.492442in}}{\pgfqpoint{4.273799in}{2.331163in}}%
\pgfusepath{clip}%
\pgfsetroundcap%
\pgfsetroundjoin%
\definecolor{currentfill}{rgb}{0.500000,0.500000,0.500000}%
\pgfsetfillcolor{currentfill}%
\pgfsetfillopacity{0.300000}%
\pgfsetlinewidth{0.301125pt}%
\definecolor{currentstroke}{rgb}{0.500000,0.500000,0.500000}%
\pgfsetstrokecolor{currentstroke}%
\pgfsetstrokeopacity{0.300000}%
\pgfsetdash{}{0pt}%
\pgfpathmoveto{\pgfqpoint{0.000000in}{0.000000in}}%
\pgfpathlineto{\pgfqpoint{0.000000in}{0.000000in}}%
\pgfpathclose%
\pgfusepath{stroke,fill}%
\end{pgfscope}%
\begin{pgfscope}%
\pgfpathrectangle{\pgfqpoint{0.647939in}{0.492442in}}{\pgfqpoint{4.273799in}{2.331163in}}%
\pgfusepath{clip}%
\pgfsetroundcap%
\pgfsetroundjoin%
\pgfsetlinewidth{0.301125pt}%
\definecolor{currentstroke}{rgb}{0.500000,0.500000,0.500000}%
\pgfsetstrokecolor{currentstroke}%
\pgfsetstrokeopacity{0.300000}%
\pgfsetdash{}{0pt}%
\pgfpathmoveto{\pgfqpoint{3.977502in}{2.059308in}}%
\pgfusepath{stroke}%
\end{pgfscope}%
\begin{pgfscope}%
\pgfpathrectangle{\pgfqpoint{0.647939in}{0.492442in}}{\pgfqpoint{4.273799in}{2.331163in}}%
\pgfusepath{clip}%
\pgfsetroundcap%
\pgfsetroundjoin%
\definecolor{currentfill}{rgb}{0.500000,0.500000,0.500000}%
\pgfsetfillcolor{currentfill}%
\pgfsetfillopacity{0.300000}%
\pgfsetlinewidth{0.301125pt}%
\definecolor{currentstroke}{rgb}{0.500000,0.500000,0.500000}%
\pgfsetstrokecolor{currentstroke}%
\pgfsetstrokeopacity{0.300000}%
\pgfsetdash{}{0pt}%
\pgfpathmoveto{\pgfqpoint{0.000000in}{0.000000in}}%
\pgfpathlineto{\pgfqpoint{0.000000in}{0.000000in}}%
\pgfpathclose%
\pgfusepath{stroke,fill}%
\end{pgfscope}%
\begin{pgfscope}%
\pgfpathrectangle{\pgfqpoint{0.647939in}{0.492442in}}{\pgfqpoint{4.273799in}{2.331163in}}%
\pgfusepath{clip}%
\pgfsetroundcap%
\pgfsetroundjoin%
\pgfsetlinewidth{0.301125pt}%
\definecolor{currentstroke}{rgb}{0.500000,0.500000,0.500000}%
\pgfsetstrokecolor{currentstroke}%
\pgfsetstrokeopacity{0.300000}%
\pgfsetdash{}{0pt}%
\pgfpathmoveto{\pgfqpoint{3.856155in}{2.317341in}}%
\pgfusepath{stroke}%
\end{pgfscope}%
\begin{pgfscope}%
\pgfpathrectangle{\pgfqpoint{0.647939in}{0.492442in}}{\pgfqpoint{4.273799in}{2.331163in}}%
\pgfusepath{clip}%
\pgfsetroundcap%
\pgfsetroundjoin%
\definecolor{currentfill}{rgb}{0.500000,0.500000,0.500000}%
\pgfsetfillcolor{currentfill}%
\pgfsetfillopacity{0.300000}%
\pgfsetlinewidth{0.301125pt}%
\definecolor{currentstroke}{rgb}{0.500000,0.500000,0.500000}%
\pgfsetstrokecolor{currentstroke}%
\pgfsetstrokeopacity{0.300000}%
\pgfsetdash{}{0pt}%
\pgfpathmoveto{\pgfqpoint{0.000000in}{0.000000in}}%
\pgfpathlineto{\pgfqpoint{0.000000in}{0.000000in}}%
\pgfpathclose%
\pgfusepath{stroke,fill}%
\end{pgfscope}%
\begin{pgfscope}%
\pgfpathrectangle{\pgfqpoint{0.647939in}{0.492442in}}{\pgfqpoint{4.273799in}{2.331163in}}%
\pgfusepath{clip}%
\pgfsetroundcap%
\pgfsetroundjoin%
\pgfsetlinewidth{0.301125pt}%
\definecolor{currentstroke}{rgb}{0.500000,0.500000,0.500000}%
\pgfsetstrokecolor{currentstroke}%
\pgfsetstrokeopacity{0.300000}%
\pgfsetdash{}{0pt}%
\pgfpathmoveto{\pgfqpoint{3.733321in}{2.420099in}}%
\pgfusepath{stroke}%
\end{pgfscope}%
\begin{pgfscope}%
\pgfpathrectangle{\pgfqpoint{0.647939in}{0.492442in}}{\pgfqpoint{4.273799in}{2.331163in}}%
\pgfusepath{clip}%
\pgfsetroundcap%
\pgfsetroundjoin%
\definecolor{currentfill}{rgb}{0.500000,0.500000,0.500000}%
\pgfsetfillcolor{currentfill}%
\pgfsetfillopacity{0.300000}%
\pgfsetlinewidth{0.301125pt}%
\definecolor{currentstroke}{rgb}{0.500000,0.500000,0.500000}%
\pgfsetstrokecolor{currentstroke}%
\pgfsetstrokeopacity{0.300000}%
\pgfsetdash{}{0pt}%
\pgfpathmoveto{\pgfqpoint{0.000000in}{0.000000in}}%
\pgfpathlineto{\pgfqpoint{0.000000in}{0.000000in}}%
\pgfpathclose%
\pgfusepath{stroke,fill}%
\end{pgfscope}%
\begin{pgfscope}%
\pgfpathrectangle{\pgfqpoint{0.647939in}{0.492442in}}{\pgfqpoint{4.273799in}{2.331163in}}%
\pgfusepath{clip}%
\pgfsetroundcap%
\pgfsetroundjoin%
\pgfsetlinewidth{0.301125pt}%
\definecolor{currentstroke}{rgb}{0.500000,0.500000,0.500000}%
\pgfsetstrokecolor{currentstroke}%
\pgfsetstrokeopacity{0.300000}%
\pgfsetdash{}{0pt}%
\pgfpathmoveto{\pgfqpoint{3.501727in}{1.753218in}}%
\pgfusepath{stroke}%
\end{pgfscope}%
\begin{pgfscope}%
\pgfpathrectangle{\pgfqpoint{0.647939in}{0.492442in}}{\pgfqpoint{4.273799in}{2.331163in}}%
\pgfusepath{clip}%
\pgfsetroundcap%
\pgfsetroundjoin%
\definecolor{currentfill}{rgb}{0.500000,0.500000,0.500000}%
\pgfsetfillcolor{currentfill}%
\pgfsetfillopacity{0.300000}%
\pgfsetlinewidth{0.301125pt}%
\definecolor{currentstroke}{rgb}{0.500000,0.500000,0.500000}%
\pgfsetstrokecolor{currentstroke}%
\pgfsetstrokeopacity{0.300000}%
\pgfsetdash{}{0pt}%
\pgfpathmoveto{\pgfqpoint{0.000000in}{0.000000in}}%
\pgfpathlineto{\pgfqpoint{0.000000in}{0.000000in}}%
\pgfpathclose%
\pgfusepath{stroke,fill}%
\end{pgfscope}%
\begin{pgfscope}%
\pgfpathrectangle{\pgfqpoint{0.647939in}{0.492442in}}{\pgfqpoint{4.273799in}{2.331163in}}%
\pgfusepath{clip}%
\pgfsetroundcap%
\pgfsetroundjoin%
\pgfsetlinewidth{0.301125pt}%
\definecolor{currentstroke}{rgb}{0.500000,0.500000,0.500000}%
\pgfsetstrokecolor{currentstroke}%
\pgfsetstrokeopacity{0.300000}%
\pgfsetdash{}{0pt}%
\pgfpathmoveto{\pgfqpoint{3.434747in}{2.699575in}}%
\pgfusepath{stroke}%
\end{pgfscope}%
\begin{pgfscope}%
\pgfpathrectangle{\pgfqpoint{0.647939in}{0.492442in}}{\pgfqpoint{4.273799in}{2.331163in}}%
\pgfusepath{clip}%
\pgfsetroundcap%
\pgfsetroundjoin%
\definecolor{currentfill}{rgb}{0.500000,0.500000,0.500000}%
\pgfsetfillcolor{currentfill}%
\pgfsetfillopacity{0.300000}%
\pgfsetlinewidth{0.301125pt}%
\definecolor{currentstroke}{rgb}{0.500000,0.500000,0.500000}%
\pgfsetstrokecolor{currentstroke}%
\pgfsetstrokeopacity{0.300000}%
\pgfsetdash{}{0pt}%
\pgfpathmoveto{\pgfqpoint{0.000000in}{0.000000in}}%
\pgfpathlineto{\pgfqpoint{0.000000in}{0.000000in}}%
\pgfpathclose%
\pgfusepath{stroke,fill}%
\end{pgfscope}%
\begin{pgfscope}%
\pgfpathrectangle{\pgfqpoint{0.647939in}{0.492442in}}{\pgfqpoint{4.273799in}{2.331163in}}%
\pgfusepath{clip}%
\pgfsetroundcap%
\pgfsetroundjoin%
\pgfsetlinewidth{0.301125pt}%
\definecolor{currentstroke}{rgb}{0.500000,0.500000,0.500000}%
\pgfsetstrokecolor{currentstroke}%
\pgfsetstrokeopacity{0.300000}%
\pgfsetdash{}{0pt}%
\pgfpathmoveto{\pgfqpoint{3.514998in}{2.218053in}}%
\pgfusepath{stroke}%
\end{pgfscope}%
\begin{pgfscope}%
\pgfpathrectangle{\pgfqpoint{0.647939in}{0.492442in}}{\pgfqpoint{4.273799in}{2.331163in}}%
\pgfusepath{clip}%
\pgfsetroundcap%
\pgfsetroundjoin%
\definecolor{currentfill}{rgb}{0.500000,0.500000,0.500000}%
\pgfsetfillcolor{currentfill}%
\pgfsetfillopacity{0.300000}%
\pgfsetlinewidth{0.301125pt}%
\definecolor{currentstroke}{rgb}{0.500000,0.500000,0.500000}%
\pgfsetstrokecolor{currentstroke}%
\pgfsetstrokeopacity{0.300000}%
\pgfsetdash{}{0pt}%
\pgfpathmoveto{\pgfqpoint{0.000000in}{0.000000in}}%
\pgfpathlineto{\pgfqpoint{0.000000in}{0.000000in}}%
\pgfpathclose%
\pgfusepath{stroke,fill}%
\end{pgfscope}%
\begin{pgfscope}%
\pgfpathrectangle{\pgfqpoint{0.647939in}{0.492442in}}{\pgfqpoint{4.273799in}{2.331163in}}%
\pgfusepath{clip}%
\pgfsetroundcap%
\pgfsetroundjoin%
\pgfsetlinewidth{0.301125pt}%
\definecolor{currentstroke}{rgb}{0.500000,0.500000,0.500000}%
\pgfsetstrokecolor{currentstroke}%
\pgfsetstrokeopacity{0.300000}%
\pgfsetdash{}{0pt}%
\pgfpathmoveto{\pgfqpoint{3.303782in}{2.601926in}}%
\pgfusepath{stroke}%
\end{pgfscope}%
\begin{pgfscope}%
\pgfpathrectangle{\pgfqpoint{0.647939in}{0.492442in}}{\pgfqpoint{4.273799in}{2.331163in}}%
\pgfusepath{clip}%
\pgfsetroundcap%
\pgfsetroundjoin%
\definecolor{currentfill}{rgb}{0.500000,0.500000,0.500000}%
\pgfsetfillcolor{currentfill}%
\pgfsetfillopacity{0.300000}%
\pgfsetlinewidth{0.301125pt}%
\definecolor{currentstroke}{rgb}{0.500000,0.500000,0.500000}%
\pgfsetstrokecolor{currentstroke}%
\pgfsetstrokeopacity{0.300000}%
\pgfsetdash{}{0pt}%
\pgfpathmoveto{\pgfqpoint{0.000000in}{0.000000in}}%
\pgfpathlineto{\pgfqpoint{0.000000in}{0.000000in}}%
\pgfpathclose%
\pgfusepath{stroke,fill}%
\end{pgfscope}%
\begin{pgfscope}%
\pgfpathrectangle{\pgfqpoint{0.647939in}{0.492442in}}{\pgfqpoint{4.273799in}{2.331163in}}%
\pgfusepath{clip}%
\pgfsetroundcap%
\pgfsetroundjoin%
\pgfsetlinewidth{0.301125pt}%
\definecolor{currentstroke}{rgb}{0.500000,0.500000,0.500000}%
\pgfsetstrokecolor{currentstroke}%
\pgfsetstrokeopacity{0.300000}%
\pgfsetdash{}{0pt}%
\pgfpathmoveto{\pgfqpoint{3.298298in}{2.176636in}}%
\pgfusepath{stroke}%
\end{pgfscope}%
\begin{pgfscope}%
\pgfpathrectangle{\pgfqpoint{0.647939in}{0.492442in}}{\pgfqpoint{4.273799in}{2.331163in}}%
\pgfusepath{clip}%
\pgfsetroundcap%
\pgfsetroundjoin%
\definecolor{currentfill}{rgb}{0.500000,0.500000,0.500000}%
\pgfsetfillcolor{currentfill}%
\pgfsetfillopacity{0.300000}%
\pgfsetlinewidth{0.301125pt}%
\definecolor{currentstroke}{rgb}{0.500000,0.500000,0.500000}%
\pgfsetstrokecolor{currentstroke}%
\pgfsetstrokeopacity{0.300000}%
\pgfsetdash{}{0pt}%
\pgfpathmoveto{\pgfqpoint{0.000000in}{0.000000in}}%
\pgfpathlineto{\pgfqpoint{0.000000in}{0.000000in}}%
\pgfpathclose%
\pgfusepath{stroke,fill}%
\end{pgfscope}%
\begin{pgfscope}%
\pgfpathrectangle{\pgfqpoint{0.647939in}{0.492442in}}{\pgfqpoint{4.273799in}{2.331163in}}%
\pgfusepath{clip}%
\pgfsetroundcap%
\pgfsetroundjoin%
\pgfsetlinewidth{0.301125pt}%
\definecolor{currentstroke}{rgb}{0.500000,0.500000,0.500000}%
\pgfsetstrokecolor{currentstroke}%
\pgfsetstrokeopacity{0.300000}%
\pgfsetdash{}{0pt}%
\pgfpathmoveto{\pgfqpoint{3.111653in}{2.394284in}}%
\pgfusepath{stroke}%
\end{pgfscope}%
\begin{pgfscope}%
\pgfpathrectangle{\pgfqpoint{0.647939in}{0.492442in}}{\pgfqpoint{4.273799in}{2.331163in}}%
\pgfusepath{clip}%
\pgfsetroundcap%
\pgfsetroundjoin%
\definecolor{currentfill}{rgb}{0.500000,0.500000,0.500000}%
\pgfsetfillcolor{currentfill}%
\pgfsetfillopacity{0.300000}%
\pgfsetlinewidth{0.301125pt}%
\definecolor{currentstroke}{rgb}{0.500000,0.500000,0.500000}%
\pgfsetstrokecolor{currentstroke}%
\pgfsetstrokeopacity{0.300000}%
\pgfsetdash{}{0pt}%
\pgfpathmoveto{\pgfqpoint{0.000000in}{0.000000in}}%
\pgfpathlineto{\pgfqpoint{0.000000in}{0.000000in}}%
\pgfpathclose%
\pgfusepath{stroke,fill}%
\end{pgfscope}%
\begin{pgfscope}%
\pgfpathrectangle{\pgfqpoint{0.647939in}{0.492442in}}{\pgfqpoint{4.273799in}{2.331163in}}%
\pgfusepath{clip}%
\pgfsetroundcap%
\pgfsetroundjoin%
\pgfsetlinewidth{0.301125pt}%
\definecolor{currentstroke}{rgb}{0.500000,0.500000,0.500000}%
\pgfsetstrokecolor{currentstroke}%
\pgfsetstrokeopacity{0.300000}%
\pgfsetdash{}{0pt}%
\pgfpathmoveto{\pgfqpoint{2.943584in}{2.489709in}}%
\pgfusepath{stroke}%
\end{pgfscope}%
\begin{pgfscope}%
\pgfpathrectangle{\pgfqpoint{0.647939in}{0.492442in}}{\pgfqpoint{4.273799in}{2.331163in}}%
\pgfusepath{clip}%
\pgfsetroundcap%
\pgfsetroundjoin%
\definecolor{currentfill}{rgb}{0.500000,0.500000,0.500000}%
\pgfsetfillcolor{currentfill}%
\pgfsetfillopacity{0.300000}%
\pgfsetlinewidth{0.301125pt}%
\definecolor{currentstroke}{rgb}{0.500000,0.500000,0.500000}%
\pgfsetstrokecolor{currentstroke}%
\pgfsetstrokeopacity{0.300000}%
\pgfsetdash{}{0pt}%
\pgfpathmoveto{\pgfqpoint{0.000000in}{0.000000in}}%
\pgfpathlineto{\pgfqpoint{0.000000in}{0.000000in}}%
\pgfpathclose%
\pgfusepath{stroke,fill}%
\end{pgfscope}%
\begin{pgfscope}%
\pgfpathrectangle{\pgfqpoint{0.647939in}{0.492442in}}{\pgfqpoint{4.273799in}{2.331163in}}%
\pgfusepath{clip}%
\pgfsetroundcap%
\pgfsetroundjoin%
\pgfsetlinewidth{0.301125pt}%
\definecolor{currentstroke}{rgb}{0.500000,0.500000,0.500000}%
\pgfsetstrokecolor{currentstroke}%
\pgfsetstrokeopacity{0.300000}%
\pgfsetdash{}{0pt}%
\pgfpathmoveto{\pgfqpoint{2.612132in}{2.693671in}}%
\pgfusepath{stroke}%
\end{pgfscope}%
\begin{pgfscope}%
\pgfpathrectangle{\pgfqpoint{0.647939in}{0.492442in}}{\pgfqpoint{4.273799in}{2.331163in}}%
\pgfusepath{clip}%
\pgfsetroundcap%
\pgfsetroundjoin%
\definecolor{currentfill}{rgb}{0.500000,0.500000,0.500000}%
\pgfsetfillcolor{currentfill}%
\pgfsetfillopacity{0.300000}%
\pgfsetlinewidth{0.301125pt}%
\definecolor{currentstroke}{rgb}{0.500000,0.500000,0.500000}%
\pgfsetstrokecolor{currentstroke}%
\pgfsetstrokeopacity{0.300000}%
\pgfsetdash{}{0pt}%
\pgfpathmoveto{\pgfqpoint{0.000000in}{0.000000in}}%
\pgfpathlineto{\pgfqpoint{0.000000in}{0.000000in}}%
\pgfpathclose%
\pgfusepath{stroke,fill}%
\end{pgfscope}%
\begin{pgfscope}%
\pgfpathrectangle{\pgfqpoint{0.647939in}{0.492442in}}{\pgfqpoint{4.273799in}{2.331163in}}%
\pgfusepath{clip}%
\pgfsetroundcap%
\pgfsetroundjoin%
\pgfsetlinewidth{0.301125pt}%
\definecolor{currentstroke}{rgb}{0.500000,0.500000,0.500000}%
\pgfsetstrokecolor{currentstroke}%
\pgfsetstrokeopacity{0.300000}%
\pgfsetdash{}{0pt}%
\pgfpathmoveto{\pgfqpoint{2.462449in}{2.739675in}}%
\pgfusepath{stroke}%
\end{pgfscope}%
\begin{pgfscope}%
\pgfpathrectangle{\pgfqpoint{0.647939in}{0.492442in}}{\pgfqpoint{4.273799in}{2.331163in}}%
\pgfusepath{clip}%
\pgfsetroundcap%
\pgfsetroundjoin%
\definecolor{currentfill}{rgb}{0.500000,0.500000,0.500000}%
\pgfsetfillcolor{currentfill}%
\pgfsetfillopacity{0.300000}%
\pgfsetlinewidth{0.301125pt}%
\definecolor{currentstroke}{rgb}{0.500000,0.500000,0.500000}%
\pgfsetstrokecolor{currentstroke}%
\pgfsetstrokeopacity{0.300000}%
\pgfsetdash{}{0pt}%
\pgfpathmoveto{\pgfqpoint{0.000000in}{0.000000in}}%
\pgfpathlineto{\pgfqpoint{0.000000in}{0.000000in}}%
\pgfpathclose%
\pgfusepath{stroke,fill}%
\end{pgfscope}%
\begin{pgfscope}%
\pgfpathrectangle{\pgfqpoint{0.647939in}{0.492442in}}{\pgfqpoint{4.273799in}{2.331163in}}%
\pgfusepath{clip}%
\pgfsetroundcap%
\pgfsetroundjoin%
\pgfsetlinewidth{0.301125pt}%
\definecolor{currentstroke}{rgb}{0.500000,0.500000,0.500000}%
\pgfsetstrokecolor{currentstroke}%
\pgfsetstrokeopacity{0.300000}%
\pgfsetdash{}{0pt}%
\pgfpathmoveto{\pgfqpoint{2.691583in}{2.531252in}}%
\pgfusepath{stroke}%
\end{pgfscope}%
\begin{pgfscope}%
\pgfpathrectangle{\pgfqpoint{0.647939in}{0.492442in}}{\pgfqpoint{4.273799in}{2.331163in}}%
\pgfusepath{clip}%
\pgfsetroundcap%
\pgfsetroundjoin%
\definecolor{currentfill}{rgb}{0.500000,0.500000,0.500000}%
\pgfsetfillcolor{currentfill}%
\pgfsetfillopacity{0.300000}%
\pgfsetlinewidth{0.301125pt}%
\definecolor{currentstroke}{rgb}{0.500000,0.500000,0.500000}%
\pgfsetstrokecolor{currentstroke}%
\pgfsetstrokeopacity{0.300000}%
\pgfsetdash{}{0pt}%
\pgfpathmoveto{\pgfqpoint{0.000000in}{0.000000in}}%
\pgfpathlineto{\pgfqpoint{0.000000in}{0.000000in}}%
\pgfpathclose%
\pgfusepath{stroke,fill}%
\end{pgfscope}%
\begin{pgfscope}%
\pgfpathrectangle{\pgfqpoint{0.647939in}{0.492442in}}{\pgfqpoint{4.273799in}{2.331163in}}%
\pgfusepath{clip}%
\pgfsetroundcap%
\pgfsetroundjoin%
\pgfsetlinewidth{0.301125pt}%
\definecolor{currentstroke}{rgb}{0.500000,0.500000,0.500000}%
\pgfsetstrokecolor{currentstroke}%
\pgfsetstrokeopacity{0.300000}%
\pgfsetdash{}{0pt}%
\pgfpathmoveto{\pgfqpoint{2.220763in}{2.678250in}}%
\pgfusepath{stroke}%
\end{pgfscope}%
\begin{pgfscope}%
\pgfpathrectangle{\pgfqpoint{0.647939in}{0.492442in}}{\pgfqpoint{4.273799in}{2.331163in}}%
\pgfusepath{clip}%
\pgfsetroundcap%
\pgfsetroundjoin%
\definecolor{currentfill}{rgb}{0.500000,0.500000,0.500000}%
\pgfsetfillcolor{currentfill}%
\pgfsetfillopacity{0.300000}%
\pgfsetlinewidth{0.301125pt}%
\definecolor{currentstroke}{rgb}{0.500000,0.500000,0.500000}%
\pgfsetstrokecolor{currentstroke}%
\pgfsetstrokeopacity{0.300000}%
\pgfsetdash{}{0pt}%
\pgfpathmoveto{\pgfqpoint{0.000000in}{0.000000in}}%
\pgfpathlineto{\pgfqpoint{0.000000in}{0.000000in}}%
\pgfpathclose%
\pgfusepath{stroke,fill}%
\end{pgfscope}%
\begin{pgfscope}%
\pgfpathrectangle{\pgfqpoint{0.647939in}{0.492442in}}{\pgfqpoint{4.273799in}{2.331163in}}%
\pgfusepath{clip}%
\pgfsetroundcap%
\pgfsetroundjoin%
\pgfsetlinewidth{0.301125pt}%
\definecolor{currentstroke}{rgb}{0.500000,0.500000,0.500000}%
\pgfsetstrokecolor{currentstroke}%
\pgfsetstrokeopacity{0.300000}%
\pgfsetdash{}{0pt}%
\pgfpathmoveto{\pgfqpoint{2.121360in}{2.592195in}}%
\pgfusepath{stroke}%
\end{pgfscope}%
\begin{pgfscope}%
\pgfpathrectangle{\pgfqpoint{0.647939in}{0.492442in}}{\pgfqpoint{4.273799in}{2.331163in}}%
\pgfusepath{clip}%
\pgfsetroundcap%
\pgfsetroundjoin%
\definecolor{currentfill}{rgb}{0.500000,0.500000,0.500000}%
\pgfsetfillcolor{currentfill}%
\pgfsetfillopacity{0.300000}%
\pgfsetlinewidth{0.301125pt}%
\definecolor{currentstroke}{rgb}{0.500000,0.500000,0.500000}%
\pgfsetstrokecolor{currentstroke}%
\pgfsetstrokeopacity{0.300000}%
\pgfsetdash{}{0pt}%
\pgfpathmoveto{\pgfqpoint{0.000000in}{0.000000in}}%
\pgfpathlineto{\pgfqpoint{0.000000in}{0.000000in}}%
\pgfpathclose%
\pgfusepath{stroke,fill}%
\end{pgfscope}%
\begin{pgfscope}%
\pgfpathrectangle{\pgfqpoint{0.647939in}{0.492442in}}{\pgfqpoint{4.273799in}{2.331163in}}%
\pgfusepath{clip}%
\pgfsetroundcap%
\pgfsetroundjoin%
\pgfsetlinewidth{0.301125pt}%
\definecolor{currentstroke}{rgb}{0.500000,0.500000,0.500000}%
\pgfsetstrokecolor{currentstroke}%
\pgfsetstrokeopacity{0.300000}%
\pgfsetdash{}{0pt}%
\pgfpathmoveto{\pgfqpoint{1.950456in}{2.555910in}}%
\pgfusepath{stroke}%
\end{pgfscope}%
\begin{pgfscope}%
\pgfpathrectangle{\pgfqpoint{0.647939in}{0.492442in}}{\pgfqpoint{4.273799in}{2.331163in}}%
\pgfusepath{clip}%
\pgfsetroundcap%
\pgfsetroundjoin%
\definecolor{currentfill}{rgb}{0.500000,0.500000,0.500000}%
\pgfsetfillcolor{currentfill}%
\pgfsetfillopacity{0.300000}%
\pgfsetlinewidth{0.301125pt}%
\definecolor{currentstroke}{rgb}{0.500000,0.500000,0.500000}%
\pgfsetstrokecolor{currentstroke}%
\pgfsetstrokeopacity{0.300000}%
\pgfsetdash{}{0pt}%
\pgfpathmoveto{\pgfqpoint{0.000000in}{0.000000in}}%
\pgfpathlineto{\pgfqpoint{0.000000in}{0.000000in}}%
\pgfpathclose%
\pgfusepath{stroke,fill}%
\end{pgfscope}%
\begin{pgfscope}%
\pgfpathrectangle{\pgfqpoint{0.647939in}{0.492442in}}{\pgfqpoint{4.273799in}{2.331163in}}%
\pgfusepath{clip}%
\pgfsetroundcap%
\pgfsetroundjoin%
\pgfsetlinewidth{0.301125pt}%
\definecolor{currentstroke}{rgb}{0.500000,0.500000,0.500000}%
\pgfsetstrokecolor{currentstroke}%
\pgfsetstrokeopacity{0.300000}%
\pgfsetdash{}{0pt}%
\pgfpathmoveto{\pgfqpoint{1.824848in}{2.482920in}}%
\pgfusepath{stroke}%
\end{pgfscope}%
\begin{pgfscope}%
\pgfpathrectangle{\pgfqpoint{0.647939in}{0.492442in}}{\pgfqpoint{4.273799in}{2.331163in}}%
\pgfusepath{clip}%
\pgfsetroundcap%
\pgfsetroundjoin%
\definecolor{currentfill}{rgb}{0.500000,0.500000,0.500000}%
\pgfsetfillcolor{currentfill}%
\pgfsetfillopacity{0.300000}%
\pgfsetlinewidth{0.301125pt}%
\definecolor{currentstroke}{rgb}{0.500000,0.500000,0.500000}%
\pgfsetstrokecolor{currentstroke}%
\pgfsetstrokeopacity{0.300000}%
\pgfsetdash{}{0pt}%
\pgfpathmoveto{\pgfqpoint{0.000000in}{0.000000in}}%
\pgfpathlineto{\pgfqpoint{0.000000in}{0.000000in}}%
\pgfpathclose%
\pgfusepath{stroke,fill}%
\end{pgfscope}%
\begin{pgfscope}%
\pgfpathrectangle{\pgfqpoint{0.647939in}{0.492442in}}{\pgfqpoint{4.273799in}{2.331163in}}%
\pgfusepath{clip}%
\pgfsetroundcap%
\pgfsetroundjoin%
\pgfsetlinewidth{0.301125pt}%
\definecolor{currentstroke}{rgb}{0.500000,0.500000,0.500000}%
\pgfsetstrokecolor{currentstroke}%
\pgfsetstrokeopacity{0.300000}%
\pgfsetdash{}{0pt}%
\pgfpathmoveto{\pgfqpoint{1.671194in}{2.406775in}}%
\pgfusepath{stroke}%
\end{pgfscope}%
\begin{pgfscope}%
\pgfpathrectangle{\pgfqpoint{0.647939in}{0.492442in}}{\pgfqpoint{4.273799in}{2.331163in}}%
\pgfusepath{clip}%
\pgfsetroundcap%
\pgfsetroundjoin%
\definecolor{currentfill}{rgb}{0.500000,0.500000,0.500000}%
\pgfsetfillcolor{currentfill}%
\pgfsetfillopacity{0.300000}%
\pgfsetlinewidth{0.301125pt}%
\definecolor{currentstroke}{rgb}{0.500000,0.500000,0.500000}%
\pgfsetstrokecolor{currentstroke}%
\pgfsetstrokeopacity{0.300000}%
\pgfsetdash{}{0pt}%
\pgfpathmoveto{\pgfqpoint{0.000000in}{0.000000in}}%
\pgfpathlineto{\pgfqpoint{0.000000in}{0.000000in}}%
\pgfpathclose%
\pgfusepath{stroke,fill}%
\end{pgfscope}%
\begin{pgfscope}%
\pgfpathrectangle{\pgfqpoint{0.647939in}{0.492442in}}{\pgfqpoint{4.273799in}{2.331163in}}%
\pgfusepath{clip}%
\pgfsetroundcap%
\pgfsetroundjoin%
\pgfsetlinewidth{0.301125pt}%
\definecolor{currentstroke}{rgb}{0.500000,0.500000,0.500000}%
\pgfsetstrokecolor{currentstroke}%
\pgfsetstrokeopacity{0.300000}%
\pgfsetdash{}{0pt}%
\pgfpathmoveto{\pgfqpoint{1.491905in}{2.418980in}}%
\pgfusepath{stroke}%
\end{pgfscope}%
\begin{pgfscope}%
\pgfpathrectangle{\pgfqpoint{0.647939in}{0.492442in}}{\pgfqpoint{4.273799in}{2.331163in}}%
\pgfusepath{clip}%
\pgfsetroundcap%
\pgfsetroundjoin%
\definecolor{currentfill}{rgb}{0.500000,0.500000,0.500000}%
\pgfsetfillcolor{currentfill}%
\pgfsetfillopacity{0.300000}%
\pgfsetlinewidth{0.301125pt}%
\definecolor{currentstroke}{rgb}{0.500000,0.500000,0.500000}%
\pgfsetstrokecolor{currentstroke}%
\pgfsetstrokeopacity{0.300000}%
\pgfsetdash{}{0pt}%
\pgfpathmoveto{\pgfqpoint{0.000000in}{0.000000in}}%
\pgfpathlineto{\pgfqpoint{0.000000in}{0.000000in}}%
\pgfpathclose%
\pgfusepath{stroke,fill}%
\end{pgfscope}%
\begin{pgfscope}%
\pgfpathrectangle{\pgfqpoint{0.647939in}{0.492442in}}{\pgfqpoint{4.273799in}{2.331163in}}%
\pgfusepath{clip}%
\pgfsetroundcap%
\pgfsetroundjoin%
\pgfsetlinewidth{0.301125pt}%
\definecolor{currentstroke}{rgb}{0.500000,0.500000,0.500000}%
\pgfsetstrokecolor{currentstroke}%
\pgfsetstrokeopacity{0.300000}%
\pgfsetdash{}{0pt}%
\pgfpathmoveto{\pgfqpoint{1.367083in}{2.363436in}}%
\pgfusepath{stroke}%
\end{pgfscope}%
\begin{pgfscope}%
\pgfpathrectangle{\pgfqpoint{0.647939in}{0.492442in}}{\pgfqpoint{4.273799in}{2.331163in}}%
\pgfusepath{clip}%
\pgfsetroundcap%
\pgfsetroundjoin%
\definecolor{currentfill}{rgb}{0.500000,0.500000,0.500000}%
\pgfsetfillcolor{currentfill}%
\pgfsetfillopacity{0.300000}%
\pgfsetlinewidth{0.301125pt}%
\definecolor{currentstroke}{rgb}{0.500000,0.500000,0.500000}%
\pgfsetstrokecolor{currentstroke}%
\pgfsetstrokeopacity{0.300000}%
\pgfsetdash{}{0pt}%
\pgfpathmoveto{\pgfqpoint{0.000000in}{0.000000in}}%
\pgfpathlineto{\pgfqpoint{0.000000in}{0.000000in}}%
\pgfpathclose%
\pgfusepath{stroke,fill}%
\end{pgfscope}%
\begin{pgfscope}%
\pgfpathrectangle{\pgfqpoint{0.647939in}{0.492442in}}{\pgfqpoint{4.273799in}{2.331163in}}%
\pgfusepath{clip}%
\pgfsetroundcap%
\pgfsetroundjoin%
\pgfsetlinewidth{0.301125pt}%
\definecolor{currentstroke}{rgb}{0.500000,0.500000,0.500000}%
\pgfsetstrokecolor{currentstroke}%
\pgfsetstrokeopacity{0.300000}%
\pgfsetdash{}{0pt}%
\pgfpathmoveto{\pgfqpoint{1.210320in}{1.793059in}}%
\pgfusepath{stroke}%
\end{pgfscope}%
\begin{pgfscope}%
\pgfpathrectangle{\pgfqpoint{0.647939in}{0.492442in}}{\pgfqpoint{4.273799in}{2.331163in}}%
\pgfusepath{clip}%
\pgfsetroundcap%
\pgfsetroundjoin%
\definecolor{currentfill}{rgb}{0.500000,0.500000,0.500000}%
\pgfsetfillcolor{currentfill}%
\pgfsetfillopacity{0.300000}%
\pgfsetlinewidth{0.301125pt}%
\definecolor{currentstroke}{rgb}{0.500000,0.500000,0.500000}%
\pgfsetstrokecolor{currentstroke}%
\pgfsetstrokeopacity{0.300000}%
\pgfsetdash{}{0pt}%
\pgfpathmoveto{\pgfqpoint{0.000000in}{0.000000in}}%
\pgfpathlineto{\pgfqpoint{0.000000in}{0.000000in}}%
\pgfpathclose%
\pgfusepath{stroke,fill}%
\end{pgfscope}%
\begin{pgfscope}%
\pgfpathrectangle{\pgfqpoint{0.647939in}{0.492442in}}{\pgfqpoint{4.273799in}{2.331163in}}%
\pgfusepath{clip}%
\pgfsetroundcap%
\pgfsetroundjoin%
\pgfsetlinewidth{0.301125pt}%
\definecolor{currentstroke}{rgb}{0.500000,0.500000,0.500000}%
\pgfsetstrokecolor{currentstroke}%
\pgfsetstrokeopacity{0.300000}%
\pgfsetdash{}{0pt}%
\pgfpathmoveto{\pgfqpoint{1.135813in}{2.152751in}}%
\pgfusepath{stroke}%
\end{pgfscope}%
\begin{pgfscope}%
\pgfpathrectangle{\pgfqpoint{0.647939in}{0.492442in}}{\pgfqpoint{4.273799in}{2.331163in}}%
\pgfusepath{clip}%
\pgfsetroundcap%
\pgfsetroundjoin%
\definecolor{currentfill}{rgb}{0.500000,0.500000,0.500000}%
\pgfsetfillcolor{currentfill}%
\pgfsetfillopacity{0.300000}%
\pgfsetlinewidth{0.301125pt}%
\definecolor{currentstroke}{rgb}{0.500000,0.500000,0.500000}%
\pgfsetstrokecolor{currentstroke}%
\pgfsetstrokeopacity{0.300000}%
\pgfsetdash{}{0pt}%
\pgfpathmoveto{\pgfqpoint{0.000000in}{0.000000in}}%
\pgfpathlineto{\pgfqpoint{0.000000in}{0.000000in}}%
\pgfpathclose%
\pgfusepath{stroke,fill}%
\end{pgfscope}%
\begin{pgfscope}%
\pgfpathrectangle{\pgfqpoint{0.647939in}{0.492442in}}{\pgfqpoint{4.273799in}{2.331163in}}%
\pgfusepath{clip}%
\pgfsetroundcap%
\pgfsetroundjoin%
\pgfsetlinewidth{0.301125pt}%
\definecolor{currentstroke}{rgb}{0.500000,0.500000,0.500000}%
\pgfsetstrokecolor{currentstroke}%
\pgfsetstrokeopacity{0.300000}%
\pgfsetdash{}{0pt}%
\pgfpathmoveto{\pgfqpoint{1.008228in}{1.893014in}}%
\pgfusepath{stroke}%
\end{pgfscope}%
\begin{pgfscope}%
\pgfpathrectangle{\pgfqpoint{0.647939in}{0.492442in}}{\pgfqpoint{4.273799in}{2.331163in}}%
\pgfusepath{clip}%
\pgfsetroundcap%
\pgfsetroundjoin%
\definecolor{currentfill}{rgb}{0.500000,0.500000,0.500000}%
\pgfsetfillcolor{currentfill}%
\pgfsetfillopacity{0.300000}%
\pgfsetlinewidth{0.301125pt}%
\definecolor{currentstroke}{rgb}{0.500000,0.500000,0.500000}%
\pgfsetstrokecolor{currentstroke}%
\pgfsetstrokeopacity{0.300000}%
\pgfsetdash{}{0pt}%
\pgfpathmoveto{\pgfqpoint{0.000000in}{0.000000in}}%
\pgfpathlineto{\pgfqpoint{0.000000in}{0.000000in}}%
\pgfpathclose%
\pgfusepath{stroke,fill}%
\end{pgfscope}%
\begin{pgfscope}%
\pgfpathrectangle{\pgfqpoint{0.647939in}{0.492442in}}{\pgfqpoint{4.273799in}{2.331163in}}%
\pgfusepath{clip}%
\pgfsetroundcap%
\pgfsetroundjoin%
\pgfsetlinewidth{0.301125pt}%
\definecolor{currentstroke}{rgb}{0.500000,0.500000,0.500000}%
\pgfsetstrokecolor{currentstroke}%
\pgfsetstrokeopacity{0.300000}%
\pgfsetdash{}{0pt}%
\pgfpathmoveto{\pgfqpoint{0.909176in}{2.099568in}}%
\pgfusepath{stroke}%
\end{pgfscope}%
\begin{pgfscope}%
\pgfpathrectangle{\pgfqpoint{0.647939in}{0.492442in}}{\pgfqpoint{4.273799in}{2.331163in}}%
\pgfusepath{clip}%
\pgfsetroundcap%
\pgfsetroundjoin%
\definecolor{currentfill}{rgb}{0.500000,0.500000,0.500000}%
\pgfsetfillcolor{currentfill}%
\pgfsetfillopacity{0.300000}%
\pgfsetlinewidth{0.301125pt}%
\definecolor{currentstroke}{rgb}{0.500000,0.500000,0.500000}%
\pgfsetstrokecolor{currentstroke}%
\pgfsetstrokeopacity{0.300000}%
\pgfsetdash{}{0pt}%
\pgfpathmoveto{\pgfqpoint{0.000000in}{0.000000in}}%
\pgfpathlineto{\pgfqpoint{0.000000in}{0.000000in}}%
\pgfpathclose%
\pgfusepath{stroke,fill}%
\end{pgfscope}%
\begin{pgfscope}%
\pgfpathrectangle{\pgfqpoint{0.647939in}{0.492442in}}{\pgfqpoint{4.273799in}{2.331163in}}%
\pgfusepath{clip}%
\pgfsetroundcap%
\pgfsetroundjoin%
\pgfsetlinewidth{0.301125pt}%
\definecolor{currentstroke}{rgb}{0.500000,0.500000,0.500000}%
\pgfsetstrokecolor{currentstroke}%
\pgfsetstrokeopacity{0.300000}%
\pgfsetdash{}{0pt}%
\pgfpathmoveto{\pgfqpoint{0.799202in}{1.995623in}}%
\pgfusepath{stroke}%
\end{pgfscope}%
\begin{pgfscope}%
\pgfpathrectangle{\pgfqpoint{0.647939in}{0.492442in}}{\pgfqpoint{4.273799in}{2.331163in}}%
\pgfusepath{clip}%
\pgfsetroundcap%
\pgfsetroundjoin%
\definecolor{currentfill}{rgb}{0.500000,0.500000,0.500000}%
\pgfsetfillcolor{currentfill}%
\pgfsetfillopacity{0.300000}%
\pgfsetlinewidth{0.301125pt}%
\definecolor{currentstroke}{rgb}{0.500000,0.500000,0.500000}%
\pgfsetstrokecolor{currentstroke}%
\pgfsetstrokeopacity{0.300000}%
\pgfsetdash{}{0pt}%
\pgfpathmoveto{\pgfqpoint{0.000000in}{0.000000in}}%
\pgfpathlineto{\pgfqpoint{0.000000in}{0.000000in}}%
\pgfpathclose%
\pgfusepath{stroke,fill}%
\end{pgfscope}%
\begin{pgfscope}%
\pgfpathrectangle{\pgfqpoint{0.647939in}{0.492442in}}{\pgfqpoint{4.273799in}{2.331163in}}%
\pgfusepath{clip}%
\pgfsetroundcap%
\pgfsetroundjoin%
\pgfsetlinewidth{0.301125pt}%
\definecolor{currentstroke}{rgb}{0.500000,0.500000,0.500000}%
\pgfsetstrokecolor{currentstroke}%
\pgfsetstrokeopacity{0.300000}%
\pgfsetdash{}{0pt}%
\pgfpathmoveto{\pgfqpoint{0.695321in}{2.202594in}}%
\pgfusepath{stroke}%
\end{pgfscope}%
\begin{pgfscope}%
\pgfpathrectangle{\pgfqpoint{0.647939in}{0.492442in}}{\pgfqpoint{4.273799in}{2.331163in}}%
\pgfusepath{clip}%
\pgfsetroundcap%
\pgfsetroundjoin%
\definecolor{currentfill}{rgb}{0.500000,0.500000,0.500000}%
\pgfsetfillcolor{currentfill}%
\pgfsetfillopacity{0.300000}%
\pgfsetlinewidth{0.301125pt}%
\definecolor{currentstroke}{rgb}{0.500000,0.500000,0.500000}%
\pgfsetstrokecolor{currentstroke}%
\pgfsetstrokeopacity{0.300000}%
\pgfsetdash{}{0pt}%
\pgfpathmoveto{\pgfqpoint{0.000000in}{0.000000in}}%
\pgfpathlineto{\pgfqpoint{0.000000in}{0.000000in}}%
\pgfpathclose%
\pgfusepath{stroke,fill}%
\end{pgfscope}%
\begin{pgfscope}%
\pgfpathrectangle{\pgfqpoint{0.647939in}{0.492442in}}{\pgfqpoint{4.273799in}{2.331163in}}%
\pgfusepath{clip}%
\pgfsetroundcap%
\pgfsetroundjoin%
\pgfsetlinewidth{0.301125pt}%
\definecolor{currentstroke}{rgb}{0.500000,0.500000,0.500000}%
\pgfsetstrokecolor{currentstroke}%
\pgfsetstrokeopacity{0.300000}%
\pgfsetdash{}{0pt}%
\pgfpathmoveto{\pgfqpoint{0.750925in}{2.139593in}}%
\pgfusepath{stroke}%
\end{pgfscope}%
\begin{pgfscope}%
\pgfpathrectangle{\pgfqpoint{0.647939in}{0.492442in}}{\pgfqpoint{4.273799in}{2.331163in}}%
\pgfusepath{clip}%
\pgfsetroundcap%
\pgfsetroundjoin%
\definecolor{currentfill}{rgb}{0.500000,0.500000,0.500000}%
\pgfsetfillcolor{currentfill}%
\pgfsetfillopacity{0.300000}%
\pgfsetlinewidth{0.301125pt}%
\definecolor{currentstroke}{rgb}{0.500000,0.500000,0.500000}%
\pgfsetstrokecolor{currentstroke}%
\pgfsetstrokeopacity{0.300000}%
\pgfsetdash{}{0pt}%
\pgfpathmoveto{\pgfqpoint{0.000000in}{0.000000in}}%
\pgfpathlineto{\pgfqpoint{0.000000in}{0.000000in}}%
\pgfpathclose%
\pgfusepath{stroke,fill}%
\end{pgfscope}%
\begin{pgfscope}%
\pgfpathrectangle{\pgfqpoint{0.647939in}{0.492442in}}{\pgfqpoint{4.273799in}{2.331163in}}%
\pgfusepath{clip}%
\pgfsetroundcap%
\pgfsetroundjoin%
\pgfsetlinewidth{0.301125pt}%
\definecolor{currentstroke}{rgb}{0.500000,0.500000,0.500000}%
\pgfsetstrokecolor{currentstroke}%
\pgfsetstrokeopacity{0.300000}%
\pgfsetdash{}{0pt}%
\pgfpathmoveto{\pgfqpoint{3.743941in}{1.038661in}}%
\pgfusepath{stroke}%
\end{pgfscope}%
\begin{pgfscope}%
\pgfpathrectangle{\pgfqpoint{0.647939in}{0.492442in}}{\pgfqpoint{4.273799in}{2.331163in}}%
\pgfusepath{clip}%
\pgfsetroundcap%
\pgfsetroundjoin%
\definecolor{currentfill}{rgb}{0.500000,0.500000,0.500000}%
\pgfsetfillcolor{currentfill}%
\pgfsetfillopacity{0.300000}%
\pgfsetlinewidth{0.301125pt}%
\definecolor{currentstroke}{rgb}{0.500000,0.500000,0.500000}%
\pgfsetstrokecolor{currentstroke}%
\pgfsetstrokeopacity{0.300000}%
\pgfsetdash{}{0pt}%
\pgfpathmoveto{\pgfqpoint{0.000000in}{0.000000in}}%
\pgfpathlineto{\pgfqpoint{0.000000in}{0.000000in}}%
\pgfpathclose%
\pgfusepath{stroke,fill}%
\end{pgfscope}%
\begin{pgfscope}%
\pgfpathrectangle{\pgfqpoint{0.647939in}{0.492442in}}{\pgfqpoint{4.273799in}{2.331163in}}%
\pgfusepath{clip}%
\pgfsetroundcap%
\pgfsetroundjoin%
\pgfsetlinewidth{0.301125pt}%
\definecolor{currentstroke}{rgb}{0.500000,0.500000,0.500000}%
\pgfsetstrokecolor{currentstroke}%
\pgfsetstrokeopacity{0.300000}%
\pgfsetdash{}{0pt}%
\pgfpathmoveto{\pgfqpoint{3.754936in}{0.810360in}}%
\pgfusepath{stroke}%
\end{pgfscope}%
\begin{pgfscope}%
\pgfpathrectangle{\pgfqpoint{0.647939in}{0.492442in}}{\pgfqpoint{4.273799in}{2.331163in}}%
\pgfusepath{clip}%
\pgfsetroundcap%
\pgfsetroundjoin%
\definecolor{currentfill}{rgb}{0.500000,0.500000,0.500000}%
\pgfsetfillcolor{currentfill}%
\pgfsetfillopacity{0.300000}%
\pgfsetlinewidth{0.301125pt}%
\definecolor{currentstroke}{rgb}{0.500000,0.500000,0.500000}%
\pgfsetstrokecolor{currentstroke}%
\pgfsetstrokeopacity{0.300000}%
\pgfsetdash{}{0pt}%
\pgfpathmoveto{\pgfqpoint{0.000000in}{0.000000in}}%
\pgfpathlineto{\pgfqpoint{0.000000in}{0.000000in}}%
\pgfpathclose%
\pgfusepath{stroke,fill}%
\end{pgfscope}%
\begin{pgfscope}%
\pgfpathrectangle{\pgfqpoint{0.647939in}{0.492442in}}{\pgfqpoint{4.273799in}{2.331163in}}%
\pgfusepath{clip}%
\pgfsetroundcap%
\pgfsetroundjoin%
\pgfsetlinewidth{0.301125pt}%
\definecolor{currentstroke}{rgb}{0.500000,0.500000,0.500000}%
\pgfsetstrokecolor{currentstroke}%
\pgfsetstrokeopacity{0.300000}%
\pgfsetdash{}{0pt}%
\pgfpathmoveto{\pgfqpoint{4.588335in}{1.970650in}}%
\pgfusepath{stroke}%
\end{pgfscope}%
\begin{pgfscope}%
\pgfpathrectangle{\pgfqpoint{0.647939in}{0.492442in}}{\pgfqpoint{4.273799in}{2.331163in}}%
\pgfusepath{clip}%
\pgfsetroundcap%
\pgfsetroundjoin%
\definecolor{currentfill}{rgb}{0.500000,0.500000,0.500000}%
\pgfsetfillcolor{currentfill}%
\pgfsetfillopacity{0.300000}%
\pgfsetlinewidth{0.301125pt}%
\definecolor{currentstroke}{rgb}{0.500000,0.500000,0.500000}%
\pgfsetstrokecolor{currentstroke}%
\pgfsetstrokeopacity{0.300000}%
\pgfsetdash{}{0pt}%
\pgfpathmoveto{\pgfqpoint{0.000000in}{0.000000in}}%
\pgfpathlineto{\pgfqpoint{0.000000in}{0.000000in}}%
\pgfpathclose%
\pgfusepath{stroke,fill}%
\end{pgfscope}%
\begin{pgfscope}%
\pgfpathrectangle{\pgfqpoint{0.647939in}{0.492442in}}{\pgfqpoint{4.273799in}{2.331163in}}%
\pgfusepath{clip}%
\pgfsetroundcap%
\pgfsetroundjoin%
\pgfsetlinewidth{0.301125pt}%
\definecolor{currentstroke}{rgb}{0.500000,0.500000,0.500000}%
\pgfsetstrokecolor{currentstroke}%
\pgfsetstrokeopacity{0.300000}%
\pgfsetdash{}{0pt}%
\pgfpathmoveto{\pgfqpoint{4.175633in}{1.634156in}}%
\pgfusepath{stroke}%
\end{pgfscope}%
\begin{pgfscope}%
\pgfpathrectangle{\pgfqpoint{0.647939in}{0.492442in}}{\pgfqpoint{4.273799in}{2.331163in}}%
\pgfusepath{clip}%
\pgfsetroundcap%
\pgfsetroundjoin%
\definecolor{currentfill}{rgb}{0.500000,0.500000,0.500000}%
\pgfsetfillcolor{currentfill}%
\pgfsetfillopacity{0.300000}%
\pgfsetlinewidth{0.301125pt}%
\definecolor{currentstroke}{rgb}{0.500000,0.500000,0.500000}%
\pgfsetstrokecolor{currentstroke}%
\pgfsetstrokeopacity{0.300000}%
\pgfsetdash{}{0pt}%
\pgfpathmoveto{\pgfqpoint{0.000000in}{0.000000in}}%
\pgfpathlineto{\pgfqpoint{0.000000in}{0.000000in}}%
\pgfpathclose%
\pgfusepath{stroke,fill}%
\end{pgfscope}%
\begin{pgfscope}%
\pgfpathrectangle{\pgfqpoint{0.647939in}{0.492442in}}{\pgfqpoint{4.273799in}{2.331163in}}%
\pgfusepath{clip}%
\pgfsetroundcap%
\pgfsetroundjoin%
\pgfsetlinewidth{0.301125pt}%
\definecolor{currentstroke}{rgb}{0.500000,0.500000,0.500000}%
\pgfsetstrokecolor{currentstroke}%
\pgfsetstrokeopacity{0.300000}%
\pgfsetdash{}{0pt}%
\pgfpathmoveto{\pgfqpoint{4.372246in}{1.814432in}}%
\pgfusepath{stroke}%
\end{pgfscope}%
\begin{pgfscope}%
\pgfpathrectangle{\pgfqpoint{0.647939in}{0.492442in}}{\pgfqpoint{4.273799in}{2.331163in}}%
\pgfusepath{clip}%
\pgfsetroundcap%
\pgfsetroundjoin%
\definecolor{currentfill}{rgb}{0.500000,0.500000,0.500000}%
\pgfsetfillcolor{currentfill}%
\pgfsetfillopacity{0.300000}%
\pgfsetlinewidth{0.301125pt}%
\definecolor{currentstroke}{rgb}{0.500000,0.500000,0.500000}%
\pgfsetstrokecolor{currentstroke}%
\pgfsetstrokeopacity{0.300000}%
\pgfsetdash{}{0pt}%
\pgfpathmoveto{\pgfqpoint{0.000000in}{0.000000in}}%
\pgfpathlineto{\pgfqpoint{0.000000in}{0.000000in}}%
\pgfpathclose%
\pgfusepath{stroke,fill}%
\end{pgfscope}%
\begin{pgfscope}%
\pgfpathrectangle{\pgfqpoint{0.647939in}{0.492442in}}{\pgfqpoint{4.273799in}{2.331163in}}%
\pgfusepath{clip}%
\pgfsetroundcap%
\pgfsetroundjoin%
\pgfsetlinewidth{0.301125pt}%
\definecolor{currentstroke}{rgb}{0.500000,0.500000,0.500000}%
\pgfsetstrokecolor{currentstroke}%
\pgfsetstrokeopacity{0.300000}%
\pgfsetdash{}{0pt}%
\pgfpathmoveto{\pgfqpoint{2.007428in}{2.712642in}}%
\pgfusepath{stroke}%
\end{pgfscope}%
\begin{pgfscope}%
\pgfpathrectangle{\pgfqpoint{0.647939in}{0.492442in}}{\pgfqpoint{4.273799in}{2.331163in}}%
\pgfusepath{clip}%
\pgfsetroundcap%
\pgfsetroundjoin%
\definecolor{currentfill}{rgb}{0.500000,0.500000,0.500000}%
\pgfsetfillcolor{currentfill}%
\pgfsetfillopacity{0.300000}%
\pgfsetlinewidth{0.301125pt}%
\definecolor{currentstroke}{rgb}{0.500000,0.500000,0.500000}%
\pgfsetstrokecolor{currentstroke}%
\pgfsetstrokeopacity{0.300000}%
\pgfsetdash{}{0pt}%
\pgfpathmoveto{\pgfqpoint{0.000000in}{0.000000in}}%
\pgfpathlineto{\pgfqpoint{0.000000in}{0.000000in}}%
\pgfpathclose%
\pgfusepath{stroke,fill}%
\end{pgfscope}%
\begin{pgfscope}%
\pgfpathrectangle{\pgfqpoint{0.647939in}{0.492442in}}{\pgfqpoint{4.273799in}{2.331163in}}%
\pgfusepath{clip}%
\pgfsetroundcap%
\pgfsetroundjoin%
\pgfsetlinewidth{0.301125pt}%
\definecolor{currentstroke}{rgb}{0.500000,0.500000,0.500000}%
\pgfsetstrokecolor{currentstroke}%
\pgfsetstrokeopacity{0.300000}%
\pgfsetdash{}{0pt}%
\pgfpathmoveto{\pgfqpoint{1.435372in}{0.808661in}}%
\pgfusepath{stroke}%
\end{pgfscope}%
\begin{pgfscope}%
\pgfpathrectangle{\pgfqpoint{0.647939in}{0.492442in}}{\pgfqpoint{4.273799in}{2.331163in}}%
\pgfusepath{clip}%
\pgfsetroundcap%
\pgfsetroundjoin%
\definecolor{currentfill}{rgb}{0.500000,0.500000,0.500000}%
\pgfsetfillcolor{currentfill}%
\pgfsetfillopacity{0.300000}%
\pgfsetlinewidth{0.301125pt}%
\definecolor{currentstroke}{rgb}{0.500000,0.500000,0.500000}%
\pgfsetstrokecolor{currentstroke}%
\pgfsetstrokeopacity{0.300000}%
\pgfsetdash{}{0pt}%
\pgfpathmoveto{\pgfqpoint{0.000000in}{0.000000in}}%
\pgfpathlineto{\pgfqpoint{0.000000in}{0.000000in}}%
\pgfpathclose%
\pgfusepath{stroke,fill}%
\end{pgfscope}%
\begin{pgfscope}%
\pgfpathrectangle{\pgfqpoint{0.647939in}{0.492442in}}{\pgfqpoint{4.273799in}{2.331163in}}%
\pgfusepath{clip}%
\pgfsetroundcap%
\pgfsetroundjoin%
\pgfsetlinewidth{0.301125pt}%
\definecolor{currentstroke}{rgb}{0.500000,0.500000,0.500000}%
\pgfsetstrokecolor{currentstroke}%
\pgfsetstrokeopacity{0.300000}%
\pgfsetdash{}{0pt}%
\pgfpathmoveto{\pgfqpoint{2.076365in}{1.306112in}}%
\pgfusepath{stroke}%
\end{pgfscope}%
\begin{pgfscope}%
\pgfpathrectangle{\pgfqpoint{0.647939in}{0.492442in}}{\pgfqpoint{4.273799in}{2.331163in}}%
\pgfusepath{clip}%
\pgfsetroundcap%
\pgfsetroundjoin%
\definecolor{currentfill}{rgb}{0.500000,0.500000,0.500000}%
\pgfsetfillcolor{currentfill}%
\pgfsetfillopacity{0.300000}%
\pgfsetlinewidth{0.301125pt}%
\definecolor{currentstroke}{rgb}{0.500000,0.500000,0.500000}%
\pgfsetstrokecolor{currentstroke}%
\pgfsetstrokeopacity{0.300000}%
\pgfsetdash{}{0pt}%
\pgfpathmoveto{\pgfqpoint{0.000000in}{0.000000in}}%
\pgfpathlineto{\pgfqpoint{0.000000in}{0.000000in}}%
\pgfpathclose%
\pgfusepath{stroke,fill}%
\end{pgfscope}%
\begin{pgfscope}%
\pgfpathrectangle{\pgfqpoint{0.647939in}{0.492442in}}{\pgfqpoint{4.273799in}{2.331163in}}%
\pgfusepath{clip}%
\pgfsetroundcap%
\pgfsetroundjoin%
\pgfsetlinewidth{0.301125pt}%
\definecolor{currentstroke}{rgb}{0.500000,0.500000,0.500000}%
\pgfsetstrokecolor{currentstroke}%
\pgfsetstrokeopacity{0.300000}%
\pgfsetdash{}{0pt}%
\pgfpathmoveto{\pgfqpoint{3.164784in}{0.893983in}}%
\pgfusepath{stroke}%
\end{pgfscope}%
\begin{pgfscope}%
\pgfpathrectangle{\pgfqpoint{0.647939in}{0.492442in}}{\pgfqpoint{4.273799in}{2.331163in}}%
\pgfusepath{clip}%
\pgfsetroundcap%
\pgfsetroundjoin%
\definecolor{currentfill}{rgb}{0.500000,0.500000,0.500000}%
\pgfsetfillcolor{currentfill}%
\pgfsetfillopacity{0.300000}%
\pgfsetlinewidth{0.301125pt}%
\definecolor{currentstroke}{rgb}{0.500000,0.500000,0.500000}%
\pgfsetstrokecolor{currentstroke}%
\pgfsetstrokeopacity{0.300000}%
\pgfsetdash{}{0pt}%
\pgfpathmoveto{\pgfqpoint{0.000000in}{0.000000in}}%
\pgfpathlineto{\pgfqpoint{0.000000in}{0.000000in}}%
\pgfpathclose%
\pgfusepath{stroke,fill}%
\end{pgfscope}%
\begin{pgfscope}%
\pgfpathrectangle{\pgfqpoint{0.647939in}{0.492442in}}{\pgfqpoint{4.273799in}{2.331163in}}%
\pgfusepath{clip}%
\pgfsetroundcap%
\pgfsetroundjoin%
\pgfsetlinewidth{0.301125pt}%
\definecolor{currentstroke}{rgb}{0.500000,0.500000,0.500000}%
\pgfsetstrokecolor{currentstroke}%
\pgfsetstrokeopacity{0.300000}%
\pgfsetdash{}{0pt}%
\pgfpathmoveto{\pgfqpoint{2.670249in}{1.450655in}}%
\pgfusepath{stroke}%
\end{pgfscope}%
\begin{pgfscope}%
\pgfpathrectangle{\pgfqpoint{0.647939in}{0.492442in}}{\pgfqpoint{4.273799in}{2.331163in}}%
\pgfusepath{clip}%
\pgfsetroundcap%
\pgfsetroundjoin%
\definecolor{currentfill}{rgb}{0.500000,0.500000,0.500000}%
\pgfsetfillcolor{currentfill}%
\pgfsetfillopacity{0.300000}%
\pgfsetlinewidth{0.301125pt}%
\definecolor{currentstroke}{rgb}{0.500000,0.500000,0.500000}%
\pgfsetstrokecolor{currentstroke}%
\pgfsetstrokeopacity{0.300000}%
\pgfsetdash{}{0pt}%
\pgfpathmoveto{\pgfqpoint{0.000000in}{0.000000in}}%
\pgfpathlineto{\pgfqpoint{0.000000in}{0.000000in}}%
\pgfpathclose%
\pgfusepath{stroke,fill}%
\end{pgfscope}%
\begin{pgfscope}%
\pgfpathrectangle{\pgfqpoint{0.647939in}{0.492442in}}{\pgfqpoint{4.273799in}{2.331163in}}%
\pgfusepath{clip}%
\pgfsetroundcap%
\pgfsetroundjoin%
\pgfsetlinewidth{0.301125pt}%
\definecolor{currentstroke}{rgb}{0.500000,0.500000,0.500000}%
\pgfsetstrokecolor{currentstroke}%
\pgfsetstrokeopacity{0.300000}%
\pgfsetdash{}{0pt}%
\pgfpathmoveto{\pgfqpoint{3.497208in}{0.842068in}}%
\pgfusepath{stroke}%
\end{pgfscope}%
\begin{pgfscope}%
\pgfpathrectangle{\pgfqpoint{0.647939in}{0.492442in}}{\pgfqpoint{4.273799in}{2.331163in}}%
\pgfusepath{clip}%
\pgfsetroundcap%
\pgfsetroundjoin%
\definecolor{currentfill}{rgb}{0.500000,0.500000,0.500000}%
\pgfsetfillcolor{currentfill}%
\pgfsetfillopacity{0.300000}%
\pgfsetlinewidth{0.301125pt}%
\definecolor{currentstroke}{rgb}{0.500000,0.500000,0.500000}%
\pgfsetstrokecolor{currentstroke}%
\pgfsetstrokeopacity{0.300000}%
\pgfsetdash{}{0pt}%
\pgfpathmoveto{\pgfqpoint{0.000000in}{0.000000in}}%
\pgfpathlineto{\pgfqpoint{0.000000in}{0.000000in}}%
\pgfpathclose%
\pgfusepath{stroke,fill}%
\end{pgfscope}%
\begin{pgfscope}%
\pgfpathrectangle{\pgfqpoint{0.647939in}{0.492442in}}{\pgfqpoint{4.273799in}{2.331163in}}%
\pgfusepath{clip}%
\pgfsetroundcap%
\pgfsetroundjoin%
\pgfsetlinewidth{0.301125pt}%
\definecolor{currentstroke}{rgb}{0.500000,0.500000,0.500000}%
\pgfsetstrokecolor{currentstroke}%
\pgfsetstrokeopacity{0.300000}%
\pgfsetdash{}{0pt}%
\pgfpathmoveto{\pgfqpoint{4.051959in}{1.421371in}}%
\pgfusepath{stroke}%
\end{pgfscope}%
\begin{pgfscope}%
\pgfpathrectangle{\pgfqpoint{0.647939in}{0.492442in}}{\pgfqpoint{4.273799in}{2.331163in}}%
\pgfusepath{clip}%
\pgfsetroundcap%
\pgfsetroundjoin%
\definecolor{currentfill}{rgb}{0.500000,0.500000,0.500000}%
\pgfsetfillcolor{currentfill}%
\pgfsetfillopacity{0.300000}%
\pgfsetlinewidth{0.301125pt}%
\definecolor{currentstroke}{rgb}{0.500000,0.500000,0.500000}%
\pgfsetstrokecolor{currentstroke}%
\pgfsetstrokeopacity{0.300000}%
\pgfsetdash{}{0pt}%
\pgfpathmoveto{\pgfqpoint{0.000000in}{0.000000in}}%
\pgfpathlineto{\pgfqpoint{0.000000in}{0.000000in}}%
\pgfpathclose%
\pgfusepath{stroke,fill}%
\end{pgfscope}%
\begin{pgfscope}%
\pgfpathrectangle{\pgfqpoint{0.647939in}{0.492442in}}{\pgfqpoint{4.273799in}{2.331163in}}%
\pgfusepath{clip}%
\pgfsetroundcap%
\pgfsetroundjoin%
\pgfsetlinewidth{0.301125pt}%
\definecolor{currentstroke}{rgb}{0.500000,0.500000,0.500000}%
\pgfsetstrokecolor{currentstroke}%
\pgfsetstrokeopacity{0.300000}%
\pgfsetdash{}{0pt}%
\pgfpathmoveto{\pgfqpoint{4.305356in}{1.954960in}}%
\pgfusepath{stroke}%
\end{pgfscope}%
\begin{pgfscope}%
\pgfpathrectangle{\pgfqpoint{0.647939in}{0.492442in}}{\pgfqpoint{4.273799in}{2.331163in}}%
\pgfusepath{clip}%
\pgfsetroundcap%
\pgfsetroundjoin%
\definecolor{currentfill}{rgb}{0.500000,0.500000,0.500000}%
\pgfsetfillcolor{currentfill}%
\pgfsetfillopacity{0.300000}%
\pgfsetlinewidth{0.301125pt}%
\definecolor{currentstroke}{rgb}{0.500000,0.500000,0.500000}%
\pgfsetstrokecolor{currentstroke}%
\pgfsetstrokeopacity{0.300000}%
\pgfsetdash{}{0pt}%
\pgfpathmoveto{\pgfqpoint{0.000000in}{0.000000in}}%
\pgfpathlineto{\pgfqpoint{0.000000in}{0.000000in}}%
\pgfpathclose%
\pgfusepath{stroke,fill}%
\end{pgfscope}%
\begin{pgfscope}%
\pgfpathrectangle{\pgfqpoint{0.647939in}{0.492442in}}{\pgfqpoint{4.273799in}{2.331163in}}%
\pgfusepath{clip}%
\pgfsetroundcap%
\pgfsetroundjoin%
\pgfsetlinewidth{0.301125pt}%
\definecolor{currentstroke}{rgb}{0.500000,0.500000,0.500000}%
\pgfsetstrokecolor{currentstroke}%
\pgfsetstrokeopacity{0.300000}%
\pgfsetdash{}{0pt}%
\pgfpathmoveto{\pgfqpoint{1.168089in}{0.819342in}}%
\pgfusepath{stroke}%
\end{pgfscope}%
\begin{pgfscope}%
\pgfpathrectangle{\pgfqpoint{0.647939in}{0.492442in}}{\pgfqpoint{4.273799in}{2.331163in}}%
\pgfusepath{clip}%
\pgfsetroundcap%
\pgfsetroundjoin%
\definecolor{currentfill}{rgb}{0.500000,0.500000,0.500000}%
\pgfsetfillcolor{currentfill}%
\pgfsetfillopacity{0.300000}%
\pgfsetlinewidth{0.301125pt}%
\definecolor{currentstroke}{rgb}{0.500000,0.500000,0.500000}%
\pgfsetstrokecolor{currentstroke}%
\pgfsetstrokeopacity{0.300000}%
\pgfsetdash{}{0pt}%
\pgfpathmoveto{\pgfqpoint{0.000000in}{0.000000in}}%
\pgfpathlineto{\pgfqpoint{0.000000in}{0.000000in}}%
\pgfpathclose%
\pgfusepath{stroke,fill}%
\end{pgfscope}%
\begin{pgfscope}%
\pgfpathrectangle{\pgfqpoint{0.647939in}{0.492442in}}{\pgfqpoint{4.273799in}{2.331163in}}%
\pgfusepath{clip}%
\pgfsetroundcap%
\pgfsetroundjoin%
\pgfsetlinewidth{0.301125pt}%
\definecolor{currentstroke}{rgb}{0.500000,0.500000,0.500000}%
\pgfsetstrokecolor{currentstroke}%
\pgfsetstrokeopacity{0.300000}%
\pgfsetdash{}{0pt}%
\pgfpathmoveto{\pgfqpoint{1.415828in}{0.940645in}}%
\pgfusepath{stroke}%
\end{pgfscope}%
\begin{pgfscope}%
\pgfpathrectangle{\pgfqpoint{0.647939in}{0.492442in}}{\pgfqpoint{4.273799in}{2.331163in}}%
\pgfusepath{clip}%
\pgfsetroundcap%
\pgfsetroundjoin%
\definecolor{currentfill}{rgb}{0.500000,0.500000,0.500000}%
\pgfsetfillcolor{currentfill}%
\pgfsetfillopacity{0.300000}%
\pgfsetlinewidth{0.301125pt}%
\definecolor{currentstroke}{rgb}{0.500000,0.500000,0.500000}%
\pgfsetstrokecolor{currentstroke}%
\pgfsetstrokeopacity{0.300000}%
\pgfsetdash{}{0pt}%
\pgfpathmoveto{\pgfqpoint{0.000000in}{0.000000in}}%
\pgfpathlineto{\pgfqpoint{0.000000in}{0.000000in}}%
\pgfpathclose%
\pgfusepath{stroke,fill}%
\end{pgfscope}%
\begin{pgfscope}%
\pgfpathrectangle{\pgfqpoint{0.647939in}{0.492442in}}{\pgfqpoint{4.273799in}{2.331163in}}%
\pgfusepath{clip}%
\pgfsetroundcap%
\pgfsetroundjoin%
\pgfsetlinewidth{0.301125pt}%
\definecolor{currentstroke}{rgb}{0.500000,0.500000,0.500000}%
\pgfsetstrokecolor{currentstroke}%
\pgfsetstrokeopacity{0.300000}%
\pgfsetdash{}{0pt}%
\pgfpathmoveto{\pgfqpoint{4.010453in}{1.124508in}}%
\pgfusepath{stroke}%
\end{pgfscope}%
\begin{pgfscope}%
\pgfpathrectangle{\pgfqpoint{0.647939in}{0.492442in}}{\pgfqpoint{4.273799in}{2.331163in}}%
\pgfusepath{clip}%
\pgfsetroundcap%
\pgfsetroundjoin%
\definecolor{currentfill}{rgb}{0.500000,0.500000,0.500000}%
\pgfsetfillcolor{currentfill}%
\pgfsetfillopacity{0.300000}%
\pgfsetlinewidth{0.301125pt}%
\definecolor{currentstroke}{rgb}{0.500000,0.500000,0.500000}%
\pgfsetstrokecolor{currentstroke}%
\pgfsetstrokeopacity{0.300000}%
\pgfsetdash{}{0pt}%
\pgfpathmoveto{\pgfqpoint{0.000000in}{0.000000in}}%
\pgfpathlineto{\pgfqpoint{0.000000in}{0.000000in}}%
\pgfpathclose%
\pgfusepath{stroke,fill}%
\end{pgfscope}%
\begin{pgfscope}%
\pgfpathrectangle{\pgfqpoint{0.647939in}{0.492442in}}{\pgfqpoint{4.273799in}{2.331163in}}%
\pgfusepath{clip}%
\pgfsetroundcap%
\pgfsetroundjoin%
\pgfsetlinewidth{0.301125pt}%
\definecolor{currentstroke}{rgb}{0.500000,0.500000,0.500000}%
\pgfsetstrokecolor{currentstroke}%
\pgfsetstrokeopacity{0.300000}%
\pgfsetdash{}{0pt}%
\pgfpathmoveto{\pgfqpoint{3.906552in}{1.292346in}}%
\pgfusepath{stroke}%
\end{pgfscope}%
\begin{pgfscope}%
\pgfpathrectangle{\pgfqpoint{0.647939in}{0.492442in}}{\pgfqpoint{4.273799in}{2.331163in}}%
\pgfusepath{clip}%
\pgfsetroundcap%
\pgfsetroundjoin%
\definecolor{currentfill}{rgb}{0.500000,0.500000,0.500000}%
\pgfsetfillcolor{currentfill}%
\pgfsetfillopacity{0.300000}%
\pgfsetlinewidth{0.301125pt}%
\definecolor{currentstroke}{rgb}{0.500000,0.500000,0.500000}%
\pgfsetstrokecolor{currentstroke}%
\pgfsetstrokeopacity{0.300000}%
\pgfsetdash{}{0pt}%
\pgfpathmoveto{\pgfqpoint{0.000000in}{0.000000in}}%
\pgfpathlineto{\pgfqpoint{0.000000in}{0.000000in}}%
\pgfpathclose%
\pgfusepath{stroke,fill}%
\end{pgfscope}%
\begin{pgfscope}%
\pgfpathrectangle{\pgfqpoint{0.647939in}{0.492442in}}{\pgfqpoint{4.273799in}{2.331163in}}%
\pgfusepath{clip}%
\pgfsetroundcap%
\pgfsetroundjoin%
\pgfsetlinewidth{0.301125pt}%
\definecolor{currentstroke}{rgb}{0.500000,0.500000,0.500000}%
\pgfsetstrokecolor{currentstroke}%
\pgfsetstrokeopacity{0.300000}%
\pgfsetdash{}{0pt}%
\pgfpathmoveto{\pgfqpoint{4.321953in}{2.059932in}}%
\pgfusepath{stroke}%
\end{pgfscope}%
\begin{pgfscope}%
\pgfpathrectangle{\pgfqpoint{0.647939in}{0.492442in}}{\pgfqpoint{4.273799in}{2.331163in}}%
\pgfusepath{clip}%
\pgfsetroundcap%
\pgfsetroundjoin%
\definecolor{currentfill}{rgb}{0.500000,0.500000,0.500000}%
\pgfsetfillcolor{currentfill}%
\pgfsetfillopacity{0.300000}%
\pgfsetlinewidth{0.301125pt}%
\definecolor{currentstroke}{rgb}{0.500000,0.500000,0.500000}%
\pgfsetstrokecolor{currentstroke}%
\pgfsetstrokeopacity{0.300000}%
\pgfsetdash{}{0pt}%
\pgfpathmoveto{\pgfqpoint{0.000000in}{0.000000in}}%
\pgfpathlineto{\pgfqpoint{0.000000in}{0.000000in}}%
\pgfpathclose%
\pgfusepath{stroke,fill}%
\end{pgfscope}%
\begin{pgfscope}%
\pgfpathrectangle{\pgfqpoint{0.647939in}{0.492442in}}{\pgfqpoint{4.273799in}{2.331163in}}%
\pgfusepath{clip}%
\pgfsetroundcap%
\pgfsetroundjoin%
\pgfsetlinewidth{0.301125pt}%
\definecolor{currentstroke}{rgb}{0.500000,0.500000,0.500000}%
\pgfsetstrokecolor{currentstroke}%
\pgfsetstrokeopacity{0.300000}%
\pgfsetdash{}{0pt}%
\pgfpathmoveto{\pgfqpoint{1.608171in}{2.330749in}}%
\pgfusepath{stroke}%
\end{pgfscope}%
\begin{pgfscope}%
\pgfpathrectangle{\pgfqpoint{0.647939in}{0.492442in}}{\pgfqpoint{4.273799in}{2.331163in}}%
\pgfusepath{clip}%
\pgfsetroundcap%
\pgfsetroundjoin%
\definecolor{currentfill}{rgb}{0.500000,0.500000,0.500000}%
\pgfsetfillcolor{currentfill}%
\pgfsetfillopacity{0.300000}%
\pgfsetlinewidth{0.301125pt}%
\definecolor{currentstroke}{rgb}{0.500000,0.500000,0.500000}%
\pgfsetstrokecolor{currentstroke}%
\pgfsetstrokeopacity{0.300000}%
\pgfsetdash{}{0pt}%
\pgfpathmoveto{\pgfqpoint{0.000000in}{0.000000in}}%
\pgfpathlineto{\pgfqpoint{0.000000in}{0.000000in}}%
\pgfpathclose%
\pgfusepath{stroke,fill}%
\end{pgfscope}%
\begin{pgfscope}%
\pgfpathrectangle{\pgfqpoint{0.647939in}{0.492442in}}{\pgfqpoint{4.273799in}{2.331163in}}%
\pgfusepath{clip}%
\pgfsetroundcap%
\pgfsetroundjoin%
\pgfsetlinewidth{0.301125pt}%
\definecolor{currentstroke}{rgb}{0.500000,0.500000,0.500000}%
\pgfsetstrokecolor{currentstroke}%
\pgfsetstrokeopacity{0.300000}%
\pgfsetdash{}{0pt}%
\pgfpathmoveto{\pgfqpoint{1.499402in}{1.479451in}}%
\pgfusepath{stroke}%
\end{pgfscope}%
\begin{pgfscope}%
\pgfpathrectangle{\pgfqpoint{0.647939in}{0.492442in}}{\pgfqpoint{4.273799in}{2.331163in}}%
\pgfusepath{clip}%
\pgfsetroundcap%
\pgfsetroundjoin%
\definecolor{currentfill}{rgb}{0.500000,0.500000,0.500000}%
\pgfsetfillcolor{currentfill}%
\pgfsetfillopacity{0.300000}%
\pgfsetlinewidth{0.301125pt}%
\definecolor{currentstroke}{rgb}{0.500000,0.500000,0.500000}%
\pgfsetstrokecolor{currentstroke}%
\pgfsetstrokeopacity{0.300000}%
\pgfsetdash{}{0pt}%
\pgfpathmoveto{\pgfqpoint{0.000000in}{0.000000in}}%
\pgfpathlineto{\pgfqpoint{0.000000in}{0.000000in}}%
\pgfpathclose%
\pgfusepath{stroke,fill}%
\end{pgfscope}%
\begin{pgfscope}%
\pgfpathrectangle{\pgfqpoint{0.647939in}{0.492442in}}{\pgfqpoint{4.273799in}{2.331163in}}%
\pgfusepath{clip}%
\pgfsetroundcap%
\pgfsetroundjoin%
\pgfsetlinewidth{0.301125pt}%
\definecolor{currentstroke}{rgb}{0.500000,0.500000,0.500000}%
\pgfsetstrokecolor{currentstroke}%
\pgfsetstrokeopacity{0.300000}%
\pgfsetdash{}{0pt}%
\pgfpathmoveto{\pgfqpoint{4.145229in}{1.436239in}}%
\pgfusepath{stroke}%
\end{pgfscope}%
\begin{pgfscope}%
\pgfpathrectangle{\pgfqpoint{0.647939in}{0.492442in}}{\pgfqpoint{4.273799in}{2.331163in}}%
\pgfusepath{clip}%
\pgfsetroundcap%
\pgfsetroundjoin%
\definecolor{currentfill}{rgb}{0.500000,0.500000,0.500000}%
\pgfsetfillcolor{currentfill}%
\pgfsetfillopacity{0.300000}%
\pgfsetlinewidth{0.301125pt}%
\definecolor{currentstroke}{rgb}{0.500000,0.500000,0.500000}%
\pgfsetstrokecolor{currentstroke}%
\pgfsetstrokeopacity{0.300000}%
\pgfsetdash{}{0pt}%
\pgfpathmoveto{\pgfqpoint{0.000000in}{0.000000in}}%
\pgfpathlineto{\pgfqpoint{0.000000in}{0.000000in}}%
\pgfpathclose%
\pgfusepath{stroke,fill}%
\end{pgfscope}%
\begin{pgfscope}%
\pgfpathrectangle{\pgfqpoint{0.647939in}{0.492442in}}{\pgfqpoint{4.273799in}{2.331163in}}%
\pgfusepath{clip}%
\pgfsetroundcap%
\pgfsetroundjoin%
\pgfsetlinewidth{0.301125pt}%
\definecolor{currentstroke}{rgb}{0.500000,0.500000,0.500000}%
\pgfsetstrokecolor{currentstroke}%
\pgfsetstrokeopacity{0.300000}%
\pgfsetdash{}{0pt}%
\pgfpathmoveto{\pgfqpoint{1.737248in}{2.434363in}}%
\pgfusepath{stroke}%
\end{pgfscope}%
\begin{pgfscope}%
\pgfpathrectangle{\pgfqpoint{0.647939in}{0.492442in}}{\pgfqpoint{4.273799in}{2.331163in}}%
\pgfusepath{clip}%
\pgfsetroundcap%
\pgfsetroundjoin%
\definecolor{currentfill}{rgb}{0.500000,0.500000,0.500000}%
\pgfsetfillcolor{currentfill}%
\pgfsetfillopacity{0.300000}%
\pgfsetlinewidth{0.301125pt}%
\definecolor{currentstroke}{rgb}{0.500000,0.500000,0.500000}%
\pgfsetstrokecolor{currentstroke}%
\pgfsetstrokeopacity{0.300000}%
\pgfsetdash{}{0pt}%
\pgfpathmoveto{\pgfqpoint{0.000000in}{0.000000in}}%
\pgfpathlineto{\pgfqpoint{0.000000in}{0.000000in}}%
\pgfpathclose%
\pgfusepath{stroke,fill}%
\end{pgfscope}%
\begin{pgfscope}%
\pgfpathrectangle{\pgfqpoint{0.647939in}{0.492442in}}{\pgfqpoint{4.273799in}{2.331163in}}%
\pgfusepath{clip}%
\pgfsetroundcap%
\pgfsetroundjoin%
\pgfsetlinewidth{0.301125pt}%
\definecolor{currentstroke}{rgb}{0.500000,0.500000,0.500000}%
\pgfsetstrokecolor{currentstroke}%
\pgfsetstrokeopacity{0.300000}%
\pgfsetdash{}{0pt}%
\pgfpathmoveto{\pgfqpoint{1.338952in}{2.190379in}}%
\pgfusepath{stroke}%
\end{pgfscope}%
\begin{pgfscope}%
\pgfpathrectangle{\pgfqpoint{0.647939in}{0.492442in}}{\pgfqpoint{4.273799in}{2.331163in}}%
\pgfusepath{clip}%
\pgfsetroundcap%
\pgfsetroundjoin%
\definecolor{currentfill}{rgb}{0.500000,0.500000,0.500000}%
\pgfsetfillcolor{currentfill}%
\pgfsetfillopacity{0.300000}%
\pgfsetlinewidth{0.301125pt}%
\definecolor{currentstroke}{rgb}{0.500000,0.500000,0.500000}%
\pgfsetstrokecolor{currentstroke}%
\pgfsetstrokeopacity{0.300000}%
\pgfsetdash{}{0pt}%
\pgfpathmoveto{\pgfqpoint{0.000000in}{0.000000in}}%
\pgfpathlineto{\pgfqpoint{0.000000in}{0.000000in}}%
\pgfpathclose%
\pgfusepath{stroke,fill}%
\end{pgfscope}%
\begin{pgfscope}%
\pgfpathrectangle{\pgfqpoint{0.647939in}{0.492442in}}{\pgfqpoint{4.273799in}{2.331163in}}%
\pgfusepath{clip}%
\pgfsetroundcap%
\pgfsetroundjoin%
\pgfsetlinewidth{0.301125pt}%
\definecolor{currentstroke}{rgb}{0.500000,0.500000,0.500000}%
\pgfsetstrokecolor{currentstroke}%
\pgfsetstrokeopacity{0.300000}%
\pgfsetdash{}{0pt}%
\pgfpathmoveto{\pgfqpoint{1.722882in}{1.582777in}}%
\pgfusepath{stroke}%
\end{pgfscope}%
\begin{pgfscope}%
\pgfpathrectangle{\pgfqpoint{0.647939in}{0.492442in}}{\pgfqpoint{4.273799in}{2.331163in}}%
\pgfusepath{clip}%
\pgfsetroundcap%
\pgfsetroundjoin%
\definecolor{currentfill}{rgb}{0.500000,0.500000,0.500000}%
\pgfsetfillcolor{currentfill}%
\pgfsetfillopacity{0.300000}%
\pgfsetlinewidth{0.301125pt}%
\definecolor{currentstroke}{rgb}{0.500000,0.500000,0.500000}%
\pgfsetstrokecolor{currentstroke}%
\pgfsetstrokeopacity{0.300000}%
\pgfsetdash{}{0pt}%
\pgfpathmoveto{\pgfqpoint{0.000000in}{0.000000in}}%
\pgfpathlineto{\pgfqpoint{0.000000in}{0.000000in}}%
\pgfpathclose%
\pgfusepath{stroke,fill}%
\end{pgfscope}%
\begin{pgfscope}%
\pgfpathrectangle{\pgfqpoint{0.647939in}{0.492442in}}{\pgfqpoint{4.273799in}{2.331163in}}%
\pgfusepath{clip}%
\pgfsetroundcap%
\pgfsetroundjoin%
\pgfsetlinewidth{0.301125pt}%
\definecolor{currentstroke}{rgb}{0.500000,0.500000,0.500000}%
\pgfsetstrokecolor{currentstroke}%
\pgfsetstrokeopacity{0.300000}%
\pgfsetdash{}{0pt}%
\pgfpathmoveto{\pgfqpoint{3.986645in}{1.001773in}}%
\pgfusepath{stroke}%
\end{pgfscope}%
\begin{pgfscope}%
\pgfpathrectangle{\pgfqpoint{0.647939in}{0.492442in}}{\pgfqpoint{4.273799in}{2.331163in}}%
\pgfusepath{clip}%
\pgfsetroundcap%
\pgfsetroundjoin%
\definecolor{currentfill}{rgb}{0.500000,0.500000,0.500000}%
\pgfsetfillcolor{currentfill}%
\pgfsetfillopacity{0.300000}%
\pgfsetlinewidth{0.301125pt}%
\definecolor{currentstroke}{rgb}{0.500000,0.500000,0.500000}%
\pgfsetstrokecolor{currentstroke}%
\pgfsetstrokeopacity{0.300000}%
\pgfsetdash{}{0pt}%
\pgfpathmoveto{\pgfqpoint{0.000000in}{0.000000in}}%
\pgfpathlineto{\pgfqpoint{0.000000in}{0.000000in}}%
\pgfpathclose%
\pgfusepath{stroke,fill}%
\end{pgfscope}%
\begin{pgfscope}%
\pgfpathrectangle{\pgfqpoint{0.647939in}{0.492442in}}{\pgfqpoint{4.273799in}{2.331163in}}%
\pgfusepath{clip}%
\pgfsetroundcap%
\pgfsetroundjoin%
\pgfsetlinewidth{0.301125pt}%
\definecolor{currentstroke}{rgb}{0.500000,0.500000,0.500000}%
\pgfsetstrokecolor{currentstroke}%
\pgfsetstrokeopacity{0.300000}%
\pgfsetdash{}{0pt}%
\pgfpathmoveto{\pgfqpoint{3.246956in}{1.036992in}}%
\pgfusepath{stroke}%
\end{pgfscope}%
\begin{pgfscope}%
\pgfpathrectangle{\pgfqpoint{0.647939in}{0.492442in}}{\pgfqpoint{4.273799in}{2.331163in}}%
\pgfusepath{clip}%
\pgfsetroundcap%
\pgfsetroundjoin%
\definecolor{currentfill}{rgb}{0.500000,0.500000,0.500000}%
\pgfsetfillcolor{currentfill}%
\pgfsetfillopacity{0.300000}%
\pgfsetlinewidth{0.301125pt}%
\definecolor{currentstroke}{rgb}{0.500000,0.500000,0.500000}%
\pgfsetstrokecolor{currentstroke}%
\pgfsetstrokeopacity{0.300000}%
\pgfsetdash{}{0pt}%
\pgfpathmoveto{\pgfqpoint{0.000000in}{0.000000in}}%
\pgfpathlineto{\pgfqpoint{0.000000in}{0.000000in}}%
\pgfpathclose%
\pgfusepath{stroke,fill}%
\end{pgfscope}%
\begin{pgfscope}%
\pgfpathrectangle{\pgfqpoint{0.647939in}{0.492442in}}{\pgfqpoint{4.273799in}{2.331163in}}%
\pgfusepath{clip}%
\pgfsetroundcap%
\pgfsetroundjoin%
\pgfsetlinewidth{0.301125pt}%
\definecolor{currentstroke}{rgb}{0.500000,0.500000,0.500000}%
\pgfsetstrokecolor{currentstroke}%
\pgfsetstrokeopacity{0.300000}%
\pgfsetdash{}{0pt}%
\pgfpathmoveto{\pgfqpoint{3.354452in}{1.959619in}}%
\pgfusepath{stroke}%
\end{pgfscope}%
\begin{pgfscope}%
\pgfpathrectangle{\pgfqpoint{0.647939in}{0.492442in}}{\pgfqpoint{4.273799in}{2.331163in}}%
\pgfusepath{clip}%
\pgfsetroundcap%
\pgfsetroundjoin%
\definecolor{currentfill}{rgb}{0.500000,0.500000,0.500000}%
\pgfsetfillcolor{currentfill}%
\pgfsetfillopacity{0.300000}%
\pgfsetlinewidth{0.301125pt}%
\definecolor{currentstroke}{rgb}{0.500000,0.500000,0.500000}%
\pgfsetstrokecolor{currentstroke}%
\pgfsetstrokeopacity{0.300000}%
\pgfsetdash{}{0pt}%
\pgfpathmoveto{\pgfqpoint{0.000000in}{0.000000in}}%
\pgfpathlineto{\pgfqpoint{0.000000in}{0.000000in}}%
\pgfpathclose%
\pgfusepath{stroke,fill}%
\end{pgfscope}%
\begin{pgfscope}%
\pgfpathrectangle{\pgfqpoint{0.647939in}{0.492442in}}{\pgfqpoint{4.273799in}{2.331163in}}%
\pgfusepath{clip}%
\pgfsetroundcap%
\pgfsetroundjoin%
\pgfsetlinewidth{0.301125pt}%
\definecolor{currentstroke}{rgb}{0.500000,0.500000,0.500000}%
\pgfsetstrokecolor{currentstroke}%
\pgfsetstrokeopacity{0.300000}%
\pgfsetdash{}{0pt}%
\pgfpathmoveto{\pgfqpoint{1.619257in}{1.763986in}}%
\pgfusepath{stroke}%
\end{pgfscope}%
\begin{pgfscope}%
\pgfpathrectangle{\pgfqpoint{0.647939in}{0.492442in}}{\pgfqpoint{4.273799in}{2.331163in}}%
\pgfusepath{clip}%
\pgfsetroundcap%
\pgfsetroundjoin%
\definecolor{currentfill}{rgb}{0.500000,0.500000,0.500000}%
\pgfsetfillcolor{currentfill}%
\pgfsetfillopacity{0.300000}%
\pgfsetlinewidth{0.301125pt}%
\definecolor{currentstroke}{rgb}{0.500000,0.500000,0.500000}%
\pgfsetstrokecolor{currentstroke}%
\pgfsetstrokeopacity{0.300000}%
\pgfsetdash{}{0pt}%
\pgfpathmoveto{\pgfqpoint{0.000000in}{0.000000in}}%
\pgfpathlineto{\pgfqpoint{0.000000in}{0.000000in}}%
\pgfpathclose%
\pgfusepath{stroke,fill}%
\end{pgfscope}%
\begin{pgfscope}%
\pgfpathrectangle{\pgfqpoint{0.647939in}{0.492442in}}{\pgfqpoint{4.273799in}{2.331163in}}%
\pgfusepath{clip}%
\pgfsetroundcap%
\pgfsetroundjoin%
\pgfsetlinewidth{0.301125pt}%
\definecolor{currentstroke}{rgb}{0.500000,0.500000,0.500000}%
\pgfsetstrokecolor{currentstroke}%
\pgfsetstrokeopacity{0.300000}%
\pgfsetdash{}{0pt}%
\pgfpathmoveto{\pgfqpoint{1.438833in}{1.327428in}}%
\pgfusepath{stroke}%
\end{pgfscope}%
\begin{pgfscope}%
\pgfpathrectangle{\pgfqpoint{0.647939in}{0.492442in}}{\pgfqpoint{4.273799in}{2.331163in}}%
\pgfusepath{clip}%
\pgfsetroundcap%
\pgfsetroundjoin%
\definecolor{currentfill}{rgb}{0.500000,0.500000,0.500000}%
\pgfsetfillcolor{currentfill}%
\pgfsetfillopacity{0.300000}%
\pgfsetlinewidth{0.301125pt}%
\definecolor{currentstroke}{rgb}{0.500000,0.500000,0.500000}%
\pgfsetstrokecolor{currentstroke}%
\pgfsetstrokeopacity{0.300000}%
\pgfsetdash{}{0pt}%
\pgfpathmoveto{\pgfqpoint{0.000000in}{0.000000in}}%
\pgfpathlineto{\pgfqpoint{0.000000in}{0.000000in}}%
\pgfpathclose%
\pgfusepath{stroke,fill}%
\end{pgfscope}%
\begin{pgfscope}%
\pgfpathrectangle{\pgfqpoint{0.647939in}{0.492442in}}{\pgfqpoint{4.273799in}{2.331163in}}%
\pgfusepath{clip}%
\pgfsetroundcap%
\pgfsetroundjoin%
\pgfsetlinewidth{0.301125pt}%
\definecolor{currentstroke}{rgb}{0.500000,0.500000,0.500000}%
\pgfsetstrokecolor{currentstroke}%
\pgfsetstrokeopacity{0.300000}%
\pgfsetdash{}{0pt}%
\pgfpathmoveto{\pgfqpoint{3.246299in}{1.194847in}}%
\pgfusepath{stroke}%
\end{pgfscope}%
\begin{pgfscope}%
\pgfpathrectangle{\pgfqpoint{0.647939in}{0.492442in}}{\pgfqpoint{4.273799in}{2.331163in}}%
\pgfusepath{clip}%
\pgfsetroundcap%
\pgfsetroundjoin%
\definecolor{currentfill}{rgb}{0.500000,0.500000,0.500000}%
\pgfsetfillcolor{currentfill}%
\pgfsetfillopacity{0.300000}%
\pgfsetlinewidth{0.301125pt}%
\definecolor{currentstroke}{rgb}{0.500000,0.500000,0.500000}%
\pgfsetstrokecolor{currentstroke}%
\pgfsetstrokeopacity{0.300000}%
\pgfsetdash{}{0pt}%
\pgfpathmoveto{\pgfqpoint{0.000000in}{0.000000in}}%
\pgfpathlineto{\pgfqpoint{0.000000in}{0.000000in}}%
\pgfpathclose%
\pgfusepath{stroke,fill}%
\end{pgfscope}%
\begin{pgfscope}%
\pgfpathrectangle{\pgfqpoint{0.647939in}{0.492442in}}{\pgfqpoint{4.273799in}{2.331163in}}%
\pgfusepath{clip}%
\pgfsetroundcap%
\pgfsetroundjoin%
\pgfsetlinewidth{0.301125pt}%
\definecolor{currentstroke}{rgb}{0.500000,0.500000,0.500000}%
\pgfsetstrokecolor{currentstroke}%
\pgfsetstrokeopacity{0.300000}%
\pgfsetdash{}{0pt}%
\pgfpathmoveto{\pgfqpoint{3.422867in}{1.659023in}}%
\pgfusepath{stroke}%
\end{pgfscope}%
\begin{pgfscope}%
\pgfpathrectangle{\pgfqpoint{0.647939in}{0.492442in}}{\pgfqpoint{4.273799in}{2.331163in}}%
\pgfusepath{clip}%
\pgfsetroundcap%
\pgfsetroundjoin%
\definecolor{currentfill}{rgb}{0.500000,0.500000,0.500000}%
\pgfsetfillcolor{currentfill}%
\pgfsetfillopacity{0.300000}%
\pgfsetlinewidth{0.301125pt}%
\definecolor{currentstroke}{rgb}{0.500000,0.500000,0.500000}%
\pgfsetstrokecolor{currentstroke}%
\pgfsetstrokeopacity{0.300000}%
\pgfsetdash{}{0pt}%
\pgfpathmoveto{\pgfqpoint{0.000000in}{0.000000in}}%
\pgfpathlineto{\pgfqpoint{0.000000in}{0.000000in}}%
\pgfpathclose%
\pgfusepath{stroke,fill}%
\end{pgfscope}%
\begin{pgfscope}%
\pgfpathrectangle{\pgfqpoint{0.647939in}{0.492442in}}{\pgfqpoint{4.273799in}{2.331163in}}%
\pgfusepath{clip}%
\pgfsetroundcap%
\pgfsetroundjoin%
\pgfsetlinewidth{0.301125pt}%
\definecolor{currentstroke}{rgb}{0.500000,0.500000,0.500000}%
\pgfsetstrokecolor{currentstroke}%
\pgfsetstrokeopacity{0.300000}%
\pgfsetdash{}{0pt}%
\pgfpathmoveto{\pgfqpoint{1.841228in}{2.020620in}}%
\pgfusepath{stroke}%
\end{pgfscope}%
\begin{pgfscope}%
\pgfpathrectangle{\pgfqpoint{0.647939in}{0.492442in}}{\pgfqpoint{4.273799in}{2.331163in}}%
\pgfusepath{clip}%
\pgfsetroundcap%
\pgfsetroundjoin%
\definecolor{currentfill}{rgb}{0.500000,0.500000,0.500000}%
\pgfsetfillcolor{currentfill}%
\pgfsetfillopacity{0.300000}%
\pgfsetlinewidth{0.301125pt}%
\definecolor{currentstroke}{rgb}{0.500000,0.500000,0.500000}%
\pgfsetstrokecolor{currentstroke}%
\pgfsetstrokeopacity{0.300000}%
\pgfsetdash{}{0pt}%
\pgfpathmoveto{\pgfqpoint{0.000000in}{0.000000in}}%
\pgfpathlineto{\pgfqpoint{0.000000in}{0.000000in}}%
\pgfpathclose%
\pgfusepath{stroke,fill}%
\end{pgfscope}%
\begin{pgfscope}%
\pgfpathrectangle{\pgfqpoint{0.647939in}{0.492442in}}{\pgfqpoint{4.273799in}{2.331163in}}%
\pgfusepath{clip}%
\pgfsetroundcap%
\pgfsetroundjoin%
\pgfsetlinewidth{0.301125pt}%
\definecolor{currentstroke}{rgb}{0.500000,0.500000,0.500000}%
\pgfsetstrokecolor{currentstroke}%
\pgfsetstrokeopacity{0.300000}%
\pgfsetdash{}{0pt}%
\pgfpathmoveto{\pgfqpoint{3.548009in}{1.898865in}}%
\pgfusepath{stroke}%
\end{pgfscope}%
\begin{pgfscope}%
\pgfpathrectangle{\pgfqpoint{0.647939in}{0.492442in}}{\pgfqpoint{4.273799in}{2.331163in}}%
\pgfusepath{clip}%
\pgfsetroundcap%
\pgfsetroundjoin%
\definecolor{currentfill}{rgb}{0.500000,0.500000,0.500000}%
\pgfsetfillcolor{currentfill}%
\pgfsetfillopacity{0.300000}%
\pgfsetlinewidth{0.301125pt}%
\definecolor{currentstroke}{rgb}{0.500000,0.500000,0.500000}%
\pgfsetstrokecolor{currentstroke}%
\pgfsetstrokeopacity{0.300000}%
\pgfsetdash{}{0pt}%
\pgfpathmoveto{\pgfqpoint{0.000000in}{0.000000in}}%
\pgfpathlineto{\pgfqpoint{0.000000in}{0.000000in}}%
\pgfpathclose%
\pgfusepath{stroke,fill}%
\end{pgfscope}%
\begin{pgfscope}%
\pgfpathrectangle{\pgfqpoint{0.647939in}{0.492442in}}{\pgfqpoint{4.273799in}{2.331163in}}%
\pgfusepath{clip}%
\pgfsetbuttcap%
\pgfsetroundjoin%
\pgfsetlinewidth{0.301125pt}%
\definecolor{currentstroke}{rgb}{0.500000,0.500000,0.500000}%
\pgfsetstrokecolor{currentstroke}%
\pgfsetstrokeopacity{0.300000}%
\pgfsetdash{}{0pt}%
\pgfpathmoveto{\pgfqpoint{2.308163in}{0.492442in}}%
\pgfpathlineto{\pgfqpoint{2.298723in}{0.504151in}}%
\pgfpathlineto{\pgfqpoint{2.260645in}{0.551609in}}%
\pgfpathlineto{\pgfqpoint{2.222849in}{0.599133in}}%
\pgfpathlineto{\pgfqpoint{2.185294in}{0.646715in}}%
\pgfpathlineto{\pgfqpoint{2.147931in}{0.694341in}}%
\pgfpathlineto{\pgfqpoint{2.110707in}{0.742000in}}%
\pgfpathlineto{\pgfqpoint{2.073561in}{0.789677in}}%
\pgfpathlineto{\pgfqpoint{2.036417in}{0.837354in}}%
\pgfpathlineto{\pgfqpoint{1.999187in}{0.885011in}}%
\pgfpathlineto{\pgfqpoint{1.961771in}{0.932625in}}%
\pgfpathlineto{\pgfqpoint{1.924051in}{0.980167in}}%
\pgfpathlineto{\pgfqpoint{1.885875in}{1.027600in}}%
\pgfpathlineto{\pgfqpoint{1.847047in}{1.074875in}}%
\pgfpathlineto{\pgfqpoint{1.807310in}{1.121924in}}%
\pgfpathlineto{\pgfqpoint{1.766301in}{1.168646in}}%
\pgfpathlineto{\pgfqpoint{1.723499in}{1.214884in}}%
\pgfpathlineto{\pgfqpoint{1.678101in}{1.260371in}}%
\pgfpathlineto{\pgfqpoint{1.628788in}{1.304601in}}%
\pgfpathlineto{\pgfqpoint{1.574081in}{1.345863in}}%
\pgfpathlineto{\pgfqpoint{1.526302in}{1.373974in}}%
\pgfpathlineto{\pgfqpoint{1.482492in}{1.392290in}}%
\pgfpathlineto{\pgfqpoint{1.437373in}{1.402942in}}%
\pgfpathlineto{\pgfqpoint{1.380918in}{1.403815in}}%
\pgfpathlineto{\pgfqpoint{1.328707in}{1.392137in}}%
\pgfpathlineto{\pgfqpoint{1.328707in}{1.392137in}}%
\pgfpathlineto{\pgfqpoint{1.258717in}{1.358107in}}%
\pgfpathlineto{\pgfqpoint{1.201951in}{1.316748in}}%
\pgfpathlineto{\pgfqpoint{1.153213in}{1.272393in}}%
\pgfpathlineto{\pgfqpoint{1.109546in}{1.226458in}}%
\pgfpathlineto{\pgfqpoint{1.069381in}{1.179558in}}%
\pgfpathlineto{\pgfqpoint{1.031798in}{1.132006in}}%
\pgfpathlineto{\pgfqpoint{0.996216in}{1.083988in}}%
\pgfpathlineto{\pgfqpoint{0.962260in}{1.035621in}}%
\pgfpathlineto{\pgfqpoint{0.929654in}{0.986983in}}%
\pgfpathlineto{\pgfqpoint{0.898173in}{0.938121in}}%
\pgfpathlineto{\pgfqpoint{0.867655in}{0.889072in}}%
\pgfpathlineto{\pgfqpoint{0.837992in}{0.839868in}}%
\pgfpathlineto{\pgfqpoint{0.809090in}{0.790530in}}%
\pgfpathlineto{\pgfqpoint{0.780857in}{0.741074in}}%
\pgfpathlineto{\pgfqpoint{0.753239in}{0.691515in}}%
\pgfpathlineto{\pgfqpoint{0.726183in}{0.641864in}}%
\pgfpathlineto{\pgfqpoint{0.699638in}{0.592130in}}%
\pgfpathlineto{\pgfqpoint{0.673570in}{0.542321in}}%
\pgfpathlineto{\pgfqpoint{0.647939in}{0.492442in}}%
\pgfpathlineto{\pgfqpoint{0.647939in}{0.492442in}}%
\pgfusepath{stroke}%
\end{pgfscope}%
\begin{pgfscope}%
\pgfpathrectangle{\pgfqpoint{0.647939in}{0.492442in}}{\pgfqpoint{4.273799in}{2.331163in}}%
\pgfusepath{clip}%
\pgfsetbuttcap%
\pgfsetroundjoin%
\pgfsetlinewidth{0.301125pt}%
\definecolor{currentstroke}{rgb}{0.500000,0.500000,0.500000}%
\pgfsetstrokecolor{currentstroke}%
\pgfsetstrokeopacity{0.300000}%
\pgfsetdash{}{0pt}%
\pgfpathmoveto{\pgfqpoint{1.854722in}{0.492442in}}%
\pgfpathlineto{\pgfqpoint{1.838754in}{0.510411in}}%
\pgfpathlineto{\pgfqpoint{1.796884in}{0.556906in}}%
\pgfpathlineto{\pgfqpoint{1.753982in}{0.603120in}}%
\pgfpathlineto{\pgfqpoint{1.709728in}{0.648951in}}%
\pgfpathlineto{\pgfqpoint{1.663688in}{0.694253in}}%
\pgfpathlineto{\pgfqpoint{1.615252in}{0.738797in}}%
\pgfpathlineto{\pgfqpoint{1.563514in}{0.782211in}}%
\pgfpathlineto{\pgfqpoint{1.507059in}{0.823809in}}%
\pgfpathlineto{\pgfqpoint{1.446462in}{0.860631in}}%
\pgfpathlineto{\pgfqpoint{1.391603in}{0.886033in}}%
\pgfpathlineto{\pgfqpoint{1.339513in}{0.902421in}}%
\pgfpathlineto{\pgfqpoint{1.284066in}{0.910863in}}%
\pgfpathlineto{\pgfqpoint{1.218283in}{0.908082in}}%
\pgfpathlineto{\pgfqpoint{1.158246in}{0.893126in}}%
\pgfpathlineto{\pgfqpoint{1.158246in}{0.893126in}}%
\pgfpathlineto{\pgfqpoint{1.086792in}{0.859689in}}%
\pgfpathlineto{\pgfqpoint{1.027346in}{0.819465in}}%
\pgfpathlineto{\pgfqpoint{0.975992in}{0.775994in}}%
\pgfpathlineto{\pgfqpoint{0.930134in}{0.730695in}}%
\pgfpathlineto{\pgfqpoint{0.888240in}{0.684247in}}%
\pgfpathlineto{\pgfqpoint{0.849342in}{0.637017in}}%
\pgfpathlineto{\pgfqpoint{0.812794in}{0.589221in}}%
\pgfpathlineto{\pgfqpoint{0.778141in}{0.540997in}}%
\pgfpathlineto{\pgfqpoint{0.745071in}{0.492442in}}%
\pgfpathlineto{\pgfqpoint{0.745071in}{0.492442in}}%
\pgfusepath{stroke}%
\end{pgfscope}%
\begin{pgfscope}%
\pgfpathrectangle{\pgfqpoint{0.647939in}{0.492442in}}{\pgfqpoint{4.273799in}{2.331163in}}%
\pgfusepath{clip}%
\pgfsetbuttcap%
\pgfsetroundjoin%
\pgfsetlinewidth{0.301125pt}%
\definecolor{currentstroke}{rgb}{0.500000,0.500000,0.500000}%
\pgfsetstrokecolor{currentstroke}%
\pgfsetstrokeopacity{0.300000}%
\pgfsetdash{}{0pt}%
\pgfpathmoveto{\pgfqpoint{1.615606in}{0.492442in}}%
\pgfpathlineto{\pgfqpoint{1.589540in}{0.515018in}}%
\pgfpathlineto{\pgfqpoint{1.537167in}{0.558207in}}%
\pgfpathlineto{\pgfqpoint{1.480453in}{0.599705in}}%
\pgfpathlineto{\pgfqpoint{1.417494in}{0.638353in}}%
\pgfpathlineto{\pgfqpoint{1.358753in}{0.666448in}}%
\pgfpathlineto{\pgfqpoint{1.303576in}{0.685036in}}%
\pgfpathlineto{\pgfqpoint{1.247332in}{0.695531in}}%
\pgfpathlineto{\pgfqpoint{1.183560in}{0.696380in}}%
\pgfpathlineto{\pgfqpoint{1.122730in}{0.685817in}}%
\pgfpathlineto{\pgfqpoint{1.122730in}{0.685817in}}%
\pgfpathlineto{\pgfqpoint{1.056855in}{0.661502in}}%
\pgfpathlineto{\pgfqpoint{1.056855in}{0.661502in}}%
\pgfpathlineto{\pgfqpoint{0.991474in}{0.624254in}}%
\pgfpathlineto{\pgfqpoint{0.935720in}{0.582441in}}%
\pgfpathlineto{\pgfqpoint{0.886602in}{0.538185in}}%
\pgfpathlineto{\pgfqpoint{0.842203in}{0.492442in}}%
\pgfpathlineto{\pgfqpoint{0.842203in}{0.492442in}}%
\pgfusepath{stroke}%
\end{pgfscope}%
\begin{pgfscope}%
\pgfpathrectangle{\pgfqpoint{0.647939in}{0.492442in}}{\pgfqpoint{4.273799in}{2.331163in}}%
\pgfusepath{clip}%
\pgfsetbuttcap%
\pgfsetroundjoin%
\pgfsetlinewidth{0.301125pt}%
\definecolor{currentstroke}{rgb}{0.500000,0.500000,0.500000}%
\pgfsetstrokecolor{currentstroke}%
\pgfsetstrokeopacity{0.300000}%
\pgfsetdash{}{0pt}%
\pgfpathmoveto{\pgfqpoint{1.380275in}{0.527899in}}%
\pgfpathlineto{\pgfqpoint{1.321311in}{0.553670in}}%
\pgfpathlineto{\pgfqpoint{1.265085in}{0.570370in}}%
\pgfpathlineto{\pgfqpoint{1.206297in}{0.578956in}}%
\pgfpathlineto{\pgfqpoint{1.138549in}{0.576839in}}%
\pgfpathlineto{\pgfqpoint{1.075794in}{0.562973in}}%
\pgfpathlineto{\pgfqpoint{1.075794in}{0.562973in}}%
\pgfpathlineto{\pgfqpoint{1.001549in}{0.531377in}}%
\pgfpathlineto{\pgfqpoint{0.939334in}{0.492442in}}%
\pgfpathlineto{\pgfqpoint{0.939334in}{0.492442in}}%
\pgfusepath{stroke}%
\end{pgfscope}%
\begin{pgfscope}%
\pgfpathrectangle{\pgfqpoint{0.647939in}{0.492442in}}{\pgfqpoint{4.273799in}{2.331163in}}%
\pgfusepath{clip}%
\pgfsetbuttcap%
\pgfsetroundjoin%
\pgfsetlinewidth{0.301125pt}%
\definecolor{currentstroke}{rgb}{0.500000,0.500000,0.500000}%
\pgfsetstrokecolor{currentstroke}%
\pgfsetstrokeopacity{0.300000}%
\pgfsetdash{}{0pt}%
\pgfpathmoveto{\pgfqpoint{1.716389in}{0.492442in}}%
\pgfpathlineto{\pgfqpoint{1.716389in}{0.492442in}}%
\pgfpathlineto{\pgfqpoint{1.670680in}{0.537846in}}%
\pgfpathlineto{\pgfqpoint{1.622921in}{0.582612in}}%
\pgfpathlineto{\pgfqpoint{1.572411in}{0.626463in}}%
\pgfpathlineto{\pgfqpoint{1.518110in}{0.668929in}}%
\pgfpathlineto{\pgfqpoint{1.458387in}{0.709132in}}%
\pgfpathlineto{\pgfqpoint{1.390605in}{0.745176in}}%
\pgfpathlineto{\pgfqpoint{1.311163in}{0.772542in}}%
\pgfpathlineto{\pgfqpoint{1.311163in}{0.772542in}}%
\pgfpathlineto{\pgfqpoint{1.251568in}{0.781498in}}%
\pgfpathlineto{\pgfqpoint{1.189278in}{0.779361in}}%
\pgfpathlineto{\pgfqpoint{1.136777in}{0.768187in}}%
\pgfpathlineto{\pgfqpoint{1.086471in}{0.749338in}}%
\pgfpathlineto{\pgfqpoint{1.034707in}{0.721645in}}%
\pgfusepath{stroke}%
\end{pgfscope}%
\begin{pgfscope}%
\pgfpathrectangle{\pgfqpoint{0.647939in}{0.492442in}}{\pgfqpoint{4.273799in}{2.331163in}}%
\pgfusepath{clip}%
\pgfsetbuttcap%
\pgfsetroundjoin%
\pgfsetlinewidth{0.301125pt}%
\definecolor{currentstroke}{rgb}{0.500000,0.500000,0.500000}%
\pgfsetstrokecolor{currentstroke}%
\pgfsetstrokeopacity{0.300000}%
\pgfsetdash{}{0pt}%
\pgfpathmoveto{\pgfqpoint{1.910652in}{0.492442in}}%
\pgfpathlineto{\pgfqpoint{1.910652in}{0.492442in}}%
\pgfpathlineto{\pgfqpoint{1.870227in}{0.539318in}}%
\pgfpathlineto{\pgfqpoint{1.829146in}{0.586023in}}%
\pgfpathlineto{\pgfqpoint{1.787201in}{0.632499in}}%
\pgfpathlineto{\pgfqpoint{1.744123in}{0.678663in}}%
\pgfpathlineto{\pgfqpoint{1.699551in}{0.724402in}}%
\pgfusepath{stroke}%
\end{pgfscope}%
\begin{pgfscope}%
\pgfpathrectangle{\pgfqpoint{0.647939in}{0.492442in}}{\pgfqpoint{4.273799in}{2.331163in}}%
\pgfusepath{clip}%
\pgfsetbuttcap%
\pgfsetroundjoin%
\pgfsetlinewidth{0.301125pt}%
\definecolor{currentstroke}{rgb}{0.500000,0.500000,0.500000}%
\pgfsetstrokecolor{currentstroke}%
\pgfsetstrokeopacity{0.300000}%
\pgfsetdash{}{0pt}%
\pgfpathmoveto{\pgfqpoint{2.007784in}{0.492442in}}%
\pgfpathlineto{\pgfqpoint{2.007784in}{0.492442in}}%
\pgfpathlineto{\pgfqpoint{1.968710in}{0.539658in}}%
\pgfpathlineto{\pgfqpoint{1.929332in}{0.586798in}}%
\pgfpathlineto{\pgfqpoint{1.889523in}{0.633831in}}%
\pgfpathlineto{\pgfqpoint{1.849128in}{0.680714in}}%
\pgfpathlineto{\pgfqpoint{1.807947in}{0.727393in}}%
\pgfpathlineto{\pgfqpoint{1.765723in}{0.773794in}}%
\pgfpathlineto{\pgfqpoint{1.722111in}{0.819808in}}%
\pgfpathlineto{\pgfqpoint{1.676629in}{0.865279in}}%
\pgfpathlineto{\pgfqpoint{1.628584in}{0.909949in}}%
\pgfpathlineto{\pgfqpoint{1.576914in}{0.953380in}}%
\pgfpathlineto{\pgfqpoint{1.519905in}{0.994734in}}%
\pgfpathlineto{\pgfqpoint{1.454621in}{1.032124in}}%
\pgfpathlineto{\pgfqpoint{1.454621in}{1.032124in}}%
\pgfpathlineto{\pgfqpoint{1.390112in}{1.056999in}}%
\pgfpathlineto{\pgfqpoint{1.390112in}{1.056999in}}%
\pgfpathlineto{\pgfqpoint{1.333727in}{1.067853in}}%
\pgfpathlineto{\pgfqpoint{1.273350in}{1.067275in}}%
\pgfpathlineto{\pgfqpoint{1.223727in}{1.057205in}}%
\pgfpathlineto{\pgfqpoint{1.176817in}{1.039691in}}%
\pgfpathlineto{\pgfqpoint{1.128203in}{1.013485in}}%
\pgfpathlineto{\pgfqpoint{1.075689in}{0.976276in}}%
\pgfpathlineto{\pgfqpoint{1.025377in}{0.932441in}}%
\pgfpathlineto{\pgfqpoint{0.980386in}{0.886885in}}%
\pgfpathlineto{\pgfqpoint{0.939189in}{0.840250in}}%
\pgfpathlineto{\pgfqpoint{0.900844in}{0.792881in}}%
\pgfpathlineto{\pgfqpoint{0.864724in}{0.744980in}}%
\pgfusepath{stroke}%
\end{pgfscope}%
\begin{pgfscope}%
\pgfpathrectangle{\pgfqpoint{0.647939in}{0.492442in}}{\pgfqpoint{4.273799in}{2.331163in}}%
\pgfusepath{clip}%
\pgfsetbuttcap%
\pgfsetroundjoin%
\pgfsetlinewidth{0.301125pt}%
\definecolor{currentstroke}{rgb}{0.500000,0.500000,0.500000}%
\pgfsetstrokecolor{currentstroke}%
\pgfsetstrokeopacity{0.300000}%
\pgfsetdash{}{0pt}%
\pgfpathmoveto{\pgfqpoint{2.104916in}{0.492442in}}%
\pgfpathlineto{\pgfqpoint{2.104916in}{0.492442in}}%
\pgfpathlineto{\pgfqpoint{2.066612in}{0.539846in}}%
\pgfpathlineto{\pgfqpoint{2.028256in}{0.587236in}}%
\pgfpathlineto{\pgfqpoint{1.989761in}{0.634594in}}%
\pgfpathlineto{\pgfqpoint{1.951023in}{0.681892in}}%
\pgfpathlineto{\pgfqpoint{1.911922in}{0.729101in}}%
\pgfpathlineto{\pgfqpoint{1.872312in}{0.776183in}}%
\pgfpathlineto{\pgfqpoint{1.832007in}{0.823089in}}%
\pgfpathlineto{\pgfqpoint{1.790768in}{0.869752in}}%
\pgfpathlineto{\pgfqpoint{1.748275in}{0.916076in}}%
\pgfpathlineto{\pgfqpoint{1.704082in}{0.961922in}}%
\pgfpathlineto{\pgfqpoint{1.657547in}{1.007066in}}%
\pgfpathlineto{\pgfqpoint{1.607686in}{1.051129in}}%
\pgfpathlineto{\pgfqpoint{1.552898in}{1.093393in}}%
\pgfpathlineto{\pgfqpoint{1.490365in}{1.132186in}}%
\pgfpathlineto{\pgfqpoint{1.415275in}{1.163092in}}%
\pgfpathlineto{\pgfqpoint{1.415275in}{1.163092in}}%
\pgfpathlineto{\pgfqpoint{1.361134in}{1.173503in}}%
\pgfpathlineto{\pgfqpoint{1.302788in}{1.172642in}}%
\pgfpathlineto{\pgfqpoint{1.254503in}{1.162378in}}%
\pgfpathlineto{\pgfqpoint{1.209110in}{1.144824in}}%
\pgfpathlineto{\pgfqpoint{1.161818in}{1.118579in}}%
\pgfpathlineto{\pgfqpoint{1.110441in}{1.081268in}}%
\pgfpathlineto{\pgfqpoint{1.060901in}{1.037198in}}%
\pgfusepath{stroke}%
\end{pgfscope}%
\begin{pgfscope}%
\pgfpathrectangle{\pgfqpoint{0.647939in}{0.492442in}}{\pgfqpoint{4.273799in}{2.331163in}}%
\pgfusepath{clip}%
\pgfsetbuttcap%
\pgfsetroundjoin%
\pgfsetlinewidth{0.301125pt}%
\definecolor{currentstroke}{rgb}{0.500000,0.500000,0.500000}%
\pgfsetstrokecolor{currentstroke}%
\pgfsetstrokeopacity{0.300000}%
\pgfsetdash{}{0pt}%
\pgfpathmoveto{\pgfqpoint{2.202048in}{0.492442in}}%
\pgfpathlineto{\pgfqpoint{2.202048in}{0.492442in}}%
\pgfpathlineto{\pgfqpoint{2.164039in}{0.539916in}}%
\pgfpathlineto{\pgfqpoint{2.126158in}{0.587421in}}%
\pgfpathlineto{\pgfqpoint{2.088345in}{0.634941in}}%
\pgfpathlineto{\pgfqpoint{2.050536in}{0.682463in}}%
\pgfpathlineto{\pgfqpoint{2.012653in}{0.729967in}}%
\pgfpathlineto{\pgfqpoint{1.974605in}{0.777431in}}%
\pgfpathlineto{\pgfqpoint{1.936284in}{0.824830in}}%
\pgfpathlineto{\pgfqpoint{1.897557in}{0.872131in}}%
\pgfpathlineto{\pgfqpoint{1.858253in}{0.919290in}}%
\pgfpathlineto{\pgfqpoint{1.818147in}{0.966247in}}%
\pgfpathlineto{\pgfqpoint{1.776946in}{1.012920in}}%
\pgfpathlineto{\pgfqpoint{1.734248in}{1.059188in}}%
\pgfpathlineto{\pgfqpoint{1.689474in}{1.104864in}}%
\pgfpathlineto{\pgfqpoint{1.641738in}{1.149633in}}%
\pgfpathlineto{\pgfqpoint{1.589581in}{1.192884in}}%
\pgfpathlineto{\pgfqpoint{1.530406in}{1.233245in}}%
\pgfpathlineto{\pgfqpoint{1.459452in}{1.267111in}}%
\pgfpathlineto{\pgfqpoint{1.459452in}{1.267111in}}%
\pgfpathlineto{\pgfqpoint{1.406134in}{1.280701in}}%
\pgfpathlineto{\pgfqpoint{1.406134in}{1.280701in}}%
\pgfpathlineto{\pgfqpoint{1.356182in}{1.283363in}}%
\pgfpathlineto{\pgfqpoint{1.307406in}{1.276313in}}%
\pgfpathlineto{\pgfqpoint{1.263511in}{1.261889in}}%
\pgfpathlineto{\pgfqpoint{1.218656in}{1.239429in}}%
\pgfpathlineto{\pgfqpoint{1.169973in}{1.206666in}}%
\pgfpathlineto{\pgfqpoint{1.118562in}{1.163243in}}%
\pgfusepath{stroke}%
\end{pgfscope}%
\begin{pgfscope}%
\pgfpathrectangle{\pgfqpoint{0.647939in}{0.492442in}}{\pgfqpoint{4.273799in}{2.331163in}}%
\pgfusepath{clip}%
\pgfsetbuttcap%
\pgfsetroundjoin%
\pgfsetlinewidth{0.301125pt}%
\definecolor{currentstroke}{rgb}{0.500000,0.500000,0.500000}%
\pgfsetstrokecolor{currentstroke}%
\pgfsetstrokeopacity{0.300000}%
\pgfsetdash{}{0pt}%
\pgfpathmoveto{\pgfqpoint{2.396312in}{0.492442in}}%
\pgfpathlineto{\pgfqpoint{2.396312in}{0.492442in}}%
\pgfpathlineto{\pgfqpoint{2.357714in}{0.539775in}}%
\pgfpathlineto{\pgfqpoint{2.319531in}{0.587207in}}%
\pgfpathlineto{\pgfqpoint{2.281727in}{0.634729in}}%
\pgfpathlineto{\pgfqpoint{2.244263in}{0.682332in}}%
\pgfpathlineto{\pgfqpoint{2.207097in}{0.730004in}}%
\pgfpathlineto{\pgfqpoint{2.170178in}{0.777734in}}%
\pgfusepath{stroke}%
\end{pgfscope}%
\begin{pgfscope}%
\pgfpathrectangle{\pgfqpoint{0.647939in}{0.492442in}}{\pgfqpoint{4.273799in}{2.331163in}}%
\pgfusepath{clip}%
\pgfsetbuttcap%
\pgfsetroundjoin%
\pgfsetlinewidth{0.301125pt}%
\definecolor{currentstroke}{rgb}{0.500000,0.500000,0.500000}%
\pgfsetstrokecolor{currentstroke}%
\pgfsetstrokeopacity{0.300000}%
\pgfsetdash{}{0pt}%
\pgfpathmoveto{\pgfqpoint{2.493443in}{0.492442in}}%
\pgfpathlineto{\pgfqpoint{2.493443in}{0.492442in}}%
\pgfpathlineto{\pgfqpoint{2.454052in}{0.539579in}}%
\pgfpathlineto{\pgfqpoint{2.415194in}{0.586848in}}%
\pgfpathlineto{\pgfqpoint{2.376838in}{0.634239in}}%
\pgfpathlineto{\pgfqpoint{2.338949in}{0.681741in}}%
\pgfpathlineto{\pgfqpoint{2.301489in}{0.729345in}}%
\pgfpathlineto{\pgfqpoint{2.264424in}{0.777041in}}%
\pgfpathlineto{\pgfqpoint{2.227717in}{0.824818in}}%
\pgfpathlineto{\pgfqpoint{2.191327in}{0.872668in}}%
\pgfpathlineto{\pgfqpoint{2.155213in}{0.920581in}}%
\pgfpathlineto{\pgfqpoint{2.119324in}{0.968543in}}%
\pgfpathlineto{\pgfqpoint{2.083597in}{1.016541in}}%
\pgfpathlineto{\pgfqpoint{2.047966in}{1.064561in}}%
\pgfpathlineto{\pgfqpoint{2.012358in}{1.112586in}}%
\pgfpathlineto{\pgfqpoint{1.976685in}{1.160596in}}%
\pgfpathlineto{\pgfqpoint{1.940841in}{1.208568in}}%
\pgfpathlineto{\pgfqpoint{1.904676in}{1.256468in}}%
\pgfpathlineto{\pgfqpoint{1.867995in}{1.304251in}}%
\pgfpathlineto{\pgfqpoint{1.830540in}{1.351853in}}%
\pgfpathlineto{\pgfqpoint{1.791948in}{1.399182in}}%
\pgfpathlineto{\pgfqpoint{1.751672in}{1.446086in}}%
\pgfpathlineto{\pgfqpoint{1.708816in}{1.492295in}}%
\pgfpathlineto{\pgfqpoint{1.661785in}{1.537245in}}%
\pgfpathlineto{\pgfqpoint{1.607324in}{1.579504in}}%
\pgfpathlineto{\pgfqpoint{1.607324in}{1.579504in}}%
\pgfpathlineto{\pgfqpoint{1.552694in}{1.608745in}}%
\pgfpathlineto{\pgfqpoint{1.552694in}{1.608745in}}%
\pgfpathlineto{\pgfqpoint{1.508165in}{1.621222in}}%
\pgfpathlineto{\pgfqpoint{1.508165in}{1.621222in}}%
\pgfpathlineto{\pgfqpoint{1.466307in}{1.622846in}}%
\pgfpathlineto{\pgfqpoint{1.426390in}{1.615342in}}%
\pgfpathlineto{\pgfqpoint{1.390101in}{1.601231in}}%
\pgfpathlineto{\pgfqpoint{1.351083in}{1.578868in}}%
\pgfpathlineto{\pgfqpoint{1.306365in}{1.545126in}}%
\pgfpathlineto{\pgfqpoint{1.258291in}{1.500600in}}%
\pgfpathlineto{\pgfqpoint{1.215153in}{1.454510in}}%
\pgfpathlineto{\pgfqpoint{1.175337in}{1.407523in}}%
\pgfpathlineto{\pgfqpoint{1.137926in}{1.359948in}}%
\pgfusepath{stroke}%
\end{pgfscope}%
\begin{pgfscope}%
\pgfpathrectangle{\pgfqpoint{0.647939in}{0.492442in}}{\pgfqpoint{4.273799in}{2.331163in}}%
\pgfusepath{clip}%
\pgfsetbuttcap%
\pgfsetroundjoin%
\pgfsetlinewidth{0.301125pt}%
\definecolor{currentstroke}{rgb}{0.500000,0.500000,0.500000}%
\pgfsetstrokecolor{currentstroke}%
\pgfsetstrokeopacity{0.300000}%
\pgfsetdash{}{0pt}%
\pgfpathmoveto{\pgfqpoint{2.590575in}{0.492442in}}%
\pgfpathlineto{\pgfqpoint{2.590575in}{0.492442in}}%
\pgfpathlineto{\pgfqpoint{2.550082in}{0.539301in}}%
\pgfpathlineto{\pgfqpoint{2.510232in}{0.586323in}}%
\pgfpathlineto{\pgfqpoint{2.470996in}{0.633499in}}%
\pgfpathlineto{\pgfqpoint{2.432344in}{0.680818in}}%
\pgfpathlineto{\pgfqpoint{2.394247in}{0.728270in}}%
\pgfpathlineto{\pgfqpoint{2.356673in}{0.775847in}}%
\pgfpathlineto{\pgfqpoint{2.319592in}{0.823539in}}%
\pgfpathlineto{\pgfqpoint{2.282973in}{0.871337in}}%
\pgfusepath{stroke}%
\end{pgfscope}%
\begin{pgfscope}%
\pgfpathrectangle{\pgfqpoint{0.647939in}{0.492442in}}{\pgfqpoint{4.273799in}{2.331163in}}%
\pgfusepath{clip}%
\pgfsetbuttcap%
\pgfsetroundjoin%
\pgfsetlinewidth{0.301125pt}%
\definecolor{currentstroke}{rgb}{0.500000,0.500000,0.500000}%
\pgfsetstrokecolor{currentstroke}%
\pgfsetstrokeopacity{0.300000}%
\pgfsetdash{}{0pt}%
\pgfpathmoveto{\pgfqpoint{2.687707in}{0.492442in}}%
\pgfpathlineto{\pgfqpoint{2.687707in}{0.492442in}}%
\pgfpathlineto{\pgfqpoint{2.645811in}{0.538932in}}%
\pgfpathlineto{\pgfqpoint{2.604670in}{0.585622in}}%
\pgfpathlineto{\pgfqpoint{2.564254in}{0.632500in}}%
\pgfpathlineto{\pgfqpoint{2.524538in}{0.679556in}}%
\pgfpathlineto{\pgfqpoint{2.485492in}{0.726778in}}%
\pgfpathlineto{\pgfqpoint{2.447088in}{0.774157in}}%
\pgfpathlineto{\pgfqpoint{2.409299in}{0.821683in}}%
\pgfpathlineto{\pgfqpoint{2.372097in}{0.869346in}}%
\pgfpathlineto{\pgfqpoint{2.335451in}{0.917138in}}%
\pgfpathlineto{\pgfqpoint{2.299329in}{0.965048in}}%
\pgfpathlineto{\pgfqpoint{2.263697in}{1.013068in}}%
\pgfpathlineto{\pgfqpoint{2.228527in}{1.061188in}}%
\pgfpathlineto{\pgfqpoint{2.193790in}{1.109402in}}%
\pgfpathlineto{\pgfqpoint{2.159454in}{1.157701in}}%
\pgfpathlineto{\pgfqpoint{2.125476in}{1.206076in}}%
\pgfpathlineto{\pgfqpoint{2.091808in}{1.254515in}}%
\pgfpathlineto{\pgfqpoint{2.058408in}{1.303010in}}%
\pgfpathlineto{\pgfqpoint{2.025232in}{1.351549in}}%
\pgfpathlineto{\pgfqpoint{1.992212in}{1.400121in}}%
\pgfpathlineto{\pgfqpoint{1.959261in}{1.448706in}}%
\pgfpathlineto{\pgfqpoint{1.926278in}{1.497285in}}%
\pgfpathlineto{\pgfqpoint{1.893140in}{1.545831in}}%
\pgfpathlineto{\pgfqpoint{1.859663in}{1.594309in}}%
\pgfpathlineto{\pgfqpoint{1.825556in}{1.642654in}}%
\pgfpathlineto{\pgfqpoint{1.790365in}{1.690762in}}%
\pgfpathlineto{\pgfqpoint{1.753318in}{1.738443in}}%
\pgfpathlineto{\pgfqpoint{1.712851in}{1.785258in}}%
\pgfpathlineto{\pgfqpoint{1.665003in}{1.829771in}}%
\pgfpathlineto{\pgfqpoint{1.665003in}{1.829771in}}%
\pgfpathlineto{\pgfqpoint{1.624909in}{1.854461in}}%
\pgfpathlineto{\pgfqpoint{1.624909in}{1.854461in}}%
\pgfpathlineto{\pgfqpoint{1.591122in}{1.864209in}}%
\pgfpathlineto{\pgfqpoint{1.591122in}{1.864209in}}%
\pgfpathlineto{\pgfqpoint{1.559163in}{1.863484in}}%
\pgfpathlineto{\pgfqpoint{1.530391in}{1.855250in}}%
\pgfpathlineto{\pgfqpoint{1.501753in}{1.840809in}}%
\pgfpathlineto{\pgfqpoint{1.466805in}{1.816374in}}%
\pgfpathlineto{\pgfqpoint{1.423855in}{1.778511in}}%
\pgfpathlineto{\pgfqpoint{1.379499in}{1.732788in}}%
\pgfpathlineto{\pgfqpoint{1.338529in}{1.686092in}}%
\pgfpathlineto{\pgfqpoint{1.299839in}{1.638806in}}%
\pgfpathlineto{\pgfqpoint{1.262885in}{1.591128in}}%
\pgfpathlineto{\pgfqpoint{1.227298in}{1.543147in}}%
\pgfusepath{stroke}%
\end{pgfscope}%
\begin{pgfscope}%
\pgfpathrectangle{\pgfqpoint{0.647939in}{0.492442in}}{\pgfqpoint{4.273799in}{2.331163in}}%
\pgfusepath{clip}%
\pgfsetbuttcap%
\pgfsetroundjoin%
\pgfsetlinewidth{0.301125pt}%
\definecolor{currentstroke}{rgb}{0.500000,0.500000,0.500000}%
\pgfsetstrokecolor{currentstroke}%
\pgfsetstrokeopacity{0.300000}%
\pgfsetdash{}{0pt}%
\pgfpathmoveto{\pgfqpoint{2.784839in}{0.492442in}}%
\pgfpathlineto{\pgfqpoint{2.784839in}{0.492442in}}%
\pgfpathlineto{\pgfqpoint{2.741263in}{0.538470in}}%
\pgfpathlineto{\pgfqpoint{2.698550in}{0.584737in}}%
\pgfpathlineto{\pgfqpoint{2.656674in}{0.631232in}}%
\pgfpathlineto{\pgfqpoint{2.615607in}{0.677941in}}%
\pgfpathlineto{\pgfqpoint{2.575325in}{0.724853in}}%
\pgfusepath{stroke}%
\end{pgfscope}%
\begin{pgfscope}%
\pgfpathrectangle{\pgfqpoint{0.647939in}{0.492442in}}{\pgfqpoint{4.273799in}{2.331163in}}%
\pgfusepath{clip}%
\pgfsetbuttcap%
\pgfsetroundjoin%
\pgfsetlinewidth{0.301125pt}%
\definecolor{currentstroke}{rgb}{0.500000,0.500000,0.500000}%
\pgfsetstrokecolor{currentstroke}%
\pgfsetstrokeopacity{0.300000}%
\pgfsetdash{}{0pt}%
\pgfpathmoveto{\pgfqpoint{2.979102in}{0.492442in}}%
\pgfpathlineto{\pgfqpoint{2.979102in}{0.492442in}}%
\pgfpathlineto{\pgfqpoint{2.931408in}{0.537238in}}%
\pgfpathlineto{\pgfqpoint{2.884779in}{0.582366in}}%
\pgfpathlineto{\pgfqpoint{2.839197in}{0.627812in}}%
\pgfpathlineto{\pgfqpoint{2.794644in}{0.673560in}}%
\pgfpathlineto{\pgfqpoint{2.751096in}{0.719595in}}%
\pgfpathlineto{\pgfqpoint{2.708528in}{0.765902in}}%
\pgfpathlineto{\pgfqpoint{2.666914in}{0.812467in}}%
\pgfpathlineto{\pgfqpoint{2.626228in}{0.859275in}}%
\pgfpathlineto{\pgfqpoint{2.586444in}{0.906313in}}%
\pgfpathlineto{\pgfqpoint{2.547537in}{0.953570in}}%
\pgfpathlineto{\pgfqpoint{2.509480in}{1.001032in}}%
\pgfpathlineto{\pgfqpoint{2.472247in}{1.048688in}}%
\pgfpathlineto{\pgfqpoint{2.435816in}{1.096528in}}%
\pgfpathlineto{\pgfqpoint{2.400170in}{1.144543in}}%
\pgfpathlineto{\pgfqpoint{2.365293in}{1.192727in}}%
\pgfpathlineto{\pgfqpoint{2.331176in}{1.241072in}}%
\pgfpathlineto{\pgfqpoint{2.297803in}{1.289571in}}%
\pgfpathlineto{\pgfqpoint{2.265160in}{1.338218in}}%
\pgfpathlineto{\pgfqpoint{2.233251in}{1.387010in}}%
\pgfpathlineto{\pgfqpoint{2.202088in}{1.435945in}}%
\pgfpathlineto{\pgfqpoint{2.171673in}{1.485020in}}%
\pgfpathlineto{\pgfqpoint{2.142018in}{1.534233in}}%
\pgfpathlineto{\pgfqpoint{2.113166in}{1.583587in}}%
\pgfpathlineto{\pgfqpoint{2.085161in}{1.633087in}}%
\pgfpathlineto{\pgfqpoint{2.058059in}{1.682735in}}%
\pgfpathlineto{\pgfqpoint{2.031973in}{1.732545in}}%
\pgfpathlineto{\pgfqpoint{2.007032in}{1.782528in}}%
\pgfpathlineto{\pgfqpoint{1.983439in}{1.832706in}}%
\pgfpathlineto{\pgfqpoint{1.961498in}{1.883105in}}%
\pgfpathlineto{\pgfqpoint{1.941669in}{1.933761in}}%
\pgfpathlineto{\pgfqpoint{1.924678in}{1.984718in}}%
\pgfpathlineto{\pgfqpoint{1.911687in}{2.036014in}}%
\pgfpathlineto{\pgfqpoint{1.904629in}{2.087631in}}%
\pgfpathlineto{\pgfqpoint{1.906478in}{2.139336in}}%
\pgfpathlineto{\pgfqpoint{1.921005in}{2.190341in}}%
\pgfpathlineto{\pgfqpoint{1.950597in}{2.239269in}}%
\pgfpathlineto{\pgfqpoint{1.994481in}{2.284891in}}%
\pgfpathlineto{\pgfqpoint{2.050283in}{2.326542in}}%
\pgfpathlineto{\pgfqpoint{2.115799in}{2.363703in}}%
\pgfpathlineto{\pgfqpoint{2.190174in}{2.395505in}}%
\pgfpathlineto{\pgfqpoint{2.272977in}{2.420222in}}%
\pgfpathlineto{\pgfqpoint{2.361568in}{2.435285in}}%
\pgfpathlineto{\pgfqpoint{2.444878in}{2.438981in}}%
\pgfpathlineto{\pgfqpoint{2.521749in}{2.433040in}}%
\pgfpathlineto{\pgfqpoint{2.593588in}{2.418817in}}%
\pgfpathlineto{\pgfqpoint{2.661879in}{2.396713in}}%
\pgfpathlineto{\pgfqpoint{2.726873in}{2.366731in}}%
\pgfpathlineto{\pgfqpoint{2.789102in}{2.328065in}}%
\pgfpathlineto{\pgfqpoint{2.841822in}{2.285191in}}%
\pgfpathlineto{\pgfqpoint{2.885159in}{2.239246in}}%
\pgfpathlineto{\pgfqpoint{2.918614in}{2.190919in}}%
\pgfpathlineto{\pgfqpoint{2.940104in}{2.140654in}}%
\pgfpathlineto{\pgfqpoint{2.940104in}{2.140654in}}%
\pgfpathlineto{\pgfqpoint{2.943982in}{2.097438in}}%
\pgfpathlineto{\pgfqpoint{2.943982in}{2.097438in}}%
\pgfpathlineto{\pgfqpoint{2.934452in}{2.073766in}}%
\pgfpathlineto{\pgfqpoint{2.934452in}{2.073766in}}%
\pgfpathlineto{\pgfqpoint{2.918380in}{2.062722in}}%
\pgfpathlineto{\pgfqpoint{2.918380in}{2.062722in}}%
\pgfpathlineto{\pgfqpoint{2.898644in}{2.061217in}}%
\pgfpathlineto{\pgfqpoint{2.879087in}{2.067212in}}%
\pgfpathlineto{\pgfqpoint{2.861810in}{2.078622in}}%
\pgfpathlineto{\pgfqpoint{2.847103in}{2.097714in}}%
\pgfpathlineto{\pgfqpoint{2.847103in}{2.097714in}}%
\pgfpathlineto{\pgfqpoint{2.846765in}{2.111330in}}%
\pgfpathlineto{\pgfqpoint{2.846765in}{2.111330in}}%
\pgfpathlineto{\pgfqpoint{2.846765in}{2.111330in}}%
\pgfpathlineto{\pgfqpoint{2.854948in}{2.111983in}}%
\pgfpathlineto{\pgfqpoint{2.857778in}{2.109493in}}%
\pgfusepath{stroke}%
\end{pgfscope}%
\begin{pgfscope}%
\pgfpathrectangle{\pgfqpoint{0.647939in}{0.492442in}}{\pgfqpoint{4.273799in}{2.331163in}}%
\pgfusepath{clip}%
\pgfsetbuttcap%
\pgfsetroundjoin%
\pgfsetlinewidth{0.301125pt}%
\definecolor{currentstroke}{rgb}{0.500000,0.500000,0.500000}%
\pgfsetstrokecolor{currentstroke}%
\pgfsetstrokeopacity{0.300000}%
\pgfsetdash{}{0pt}%
\pgfpathmoveto{\pgfqpoint{3.173366in}{0.492442in}}%
\pgfpathlineto{\pgfqpoint{3.173366in}{0.492442in}}%
\pgfpathlineto{\pgfqpoint{3.120800in}{0.535584in}}%
\pgfpathlineto{\pgfqpoint{3.069426in}{0.579152in}}%
\pgfpathlineto{\pgfqpoint{3.019261in}{0.623137in}}%
\pgfpathlineto{\pgfqpoint{2.970311in}{0.667527in}}%
\pgfpathlineto{\pgfqpoint{2.922569in}{0.712307in}}%
\pgfpathlineto{\pgfqpoint{2.876022in}{0.757460in}}%
\pgfpathlineto{\pgfqpoint{2.830650in}{0.802967in}}%
\pgfpathlineto{\pgfqpoint{2.786431in}{0.848810in}}%
\pgfpathlineto{\pgfqpoint{2.743338in}{0.894972in}}%
\pgfpathlineto{\pgfqpoint{2.701345in}{0.941434in}}%
\pgfpathlineto{\pgfqpoint{2.660425in}{0.988181in}}%
\pgfpathlineto{\pgfqpoint{2.620552in}{1.035196in}}%
\pgfpathlineto{\pgfqpoint{2.581703in}{1.082466in}}%
\pgfpathlineto{\pgfqpoint{2.543856in}{1.129978in}}%
\pgfpathlineto{\pgfqpoint{2.506994in}{1.177719in}}%
\pgfpathlineto{\pgfqpoint{2.471095in}{1.225678in}}%
\pgfpathlineto{\pgfqpoint{2.436147in}{1.273846in}}%
\pgfpathlineto{\pgfqpoint{2.402146in}{1.322215in}}%
\pgfpathlineto{\pgfqpoint{2.369096in}{1.370780in}}%
\pgfpathlineto{\pgfqpoint{2.337005in}{1.419536in}}%
\pgfpathlineto{\pgfqpoint{2.305882in}{1.468478in}}%
\pgfpathlineto{\pgfqpoint{2.275752in}{1.517604in}}%
\pgfpathlineto{\pgfqpoint{2.246667in}{1.566918in}}%
\pgfpathlineto{\pgfqpoint{2.218674in}{1.616419in}}%
\pgfpathlineto{\pgfqpoint{2.191849in}{1.666112in}}%
\pgfpathlineto{\pgfqpoint{2.166308in}{1.716005in}}%
\pgfpathlineto{\pgfqpoint{2.142191in}{1.766109in}}%
\pgfpathlineto{\pgfqpoint{2.119704in}{1.816437in}}%
\pgfpathlineto{\pgfqpoint{2.099120in}{1.867005in}}%
\pgfpathlineto{\pgfqpoint{2.080820in}{1.917832in}}%
\pgfpathlineto{\pgfqpoint{2.065345in}{1.968934in}}%
\pgfpathlineto{\pgfqpoint{2.053455in}{2.020317in}}%
\pgfpathlineto{\pgfqpoint{2.046236in}{2.071949in}}%
\pgfpathlineto{\pgfqpoint{2.045214in}{2.123709in}}%
\pgfpathlineto{\pgfqpoint{2.052424in}{2.175286in}}%
\pgfpathlineto{\pgfqpoint{2.070262in}{2.226032in}}%
\pgfpathlineto{\pgfqpoint{2.100980in}{2.274834in}}%
\pgfpathlineto{\pgfqpoint{2.145942in}{2.320144in}}%
\pgfpathlineto{\pgfqpoint{2.205198in}{2.360149in}}%
\pgfusepath{stroke}%
\end{pgfscope}%
\begin{pgfscope}%
\pgfpathrectangle{\pgfqpoint{0.647939in}{0.492442in}}{\pgfqpoint{4.273799in}{2.331163in}}%
\pgfusepath{clip}%
\pgfsetbuttcap%
\pgfsetroundjoin%
\pgfsetlinewidth{0.301125pt}%
\definecolor{currentstroke}{rgb}{0.500000,0.500000,0.500000}%
\pgfsetstrokecolor{currentstroke}%
\pgfsetstrokeopacity{0.300000}%
\pgfsetdash{}{0pt}%
\pgfpathmoveto{\pgfqpoint{3.367630in}{0.492442in}}%
\pgfpathlineto{\pgfqpoint{3.367630in}{0.492442in}}%
\pgfpathlineto{\pgfqpoint{3.310037in}{0.533631in}}%
\pgfpathlineto{\pgfqpoint{3.253573in}{0.575282in}}%
\pgfpathlineto{\pgfqpoint{3.198324in}{0.617415in}}%
\pgfpathlineto{\pgfqpoint{3.144360in}{0.660040in}}%
\pgfpathlineto{\pgfqpoint{3.091727in}{0.703157in}}%
\pgfpathlineto{\pgfqpoint{3.040450in}{0.746757in}}%
\pgfpathlineto{\pgfqpoint{2.990535in}{0.790825in}}%
\pgfpathlineto{\pgfqpoint{2.941978in}{0.835343in}}%
\pgfpathlineto{\pgfqpoint{2.894767in}{0.880290in}}%
\pgfpathlineto{\pgfqpoint{2.848881in}{0.925644in}}%
\pgfpathlineto{\pgfqpoint{2.804295in}{0.971382in}}%
\pgfpathlineto{\pgfqpoint{2.760983in}{1.017483in}}%
\pgfpathlineto{\pgfqpoint{2.718916in}{1.063926in}}%
\pgfpathlineto{\pgfqpoint{2.678069in}{1.110692in}}%
\pgfpathlineto{\pgfqpoint{2.638414in}{1.157762in}}%
\pgfpathlineto{\pgfqpoint{2.599928in}{1.205120in}}%
\pgfpathlineto{\pgfqpoint{2.562594in}{1.252752in}}%
\pgfpathlineto{\pgfqpoint{2.526398in}{1.300644in}}%
\pgfpathlineto{\pgfqpoint{2.491335in}{1.348786in}}%
\pgfpathlineto{\pgfqpoint{2.457410in}{1.397170in}}%
\pgfpathlineto{\pgfqpoint{2.424632in}{1.445789in}}%
\pgfpathlineto{\pgfqpoint{2.393013in}{1.494636in}}%
\pgfpathlineto{\pgfqpoint{2.362589in}{1.543708in}}%
\pgfpathlineto{\pgfqpoint{2.333417in}{1.593006in}}%
\pgfpathlineto{\pgfqpoint{2.305558in}{1.642529in}}%
\pgfpathlineto{\pgfqpoint{2.279105in}{1.692281in}}%
\pgfpathlineto{\pgfqpoint{2.254193in}{1.742268in}}%
\pgfpathlineto{\pgfqpoint{2.230987in}{1.792499in}}%
\pgfpathlineto{\pgfqpoint{2.209722in}{1.842984in}}%
\pgfpathlineto{\pgfqpoint{2.190708in}{1.893733in}}%
\pgfpathlineto{\pgfqpoint{2.174365in}{1.944756in}}%
\pgfpathlineto{\pgfqpoint{2.161278in}{1.996055in}}%
\pgfpathlineto{\pgfqpoint{2.152242in}{2.047608in}}%
\pgfpathlineto{\pgfqpoint{2.148356in}{2.099341in}}%
\pgfpathlineto{\pgfqpoint{2.151133in}{2.151074in}}%
\pgfpathlineto{\pgfqpoint{2.162569in}{2.202415in}}%
\pgfpathlineto{\pgfqpoint{2.185176in}{2.252582in}}%
\pgfpathlineto{\pgfqpoint{2.221658in}{2.300157in}}%
\pgfpathlineto{\pgfqpoint{2.274339in}{2.342779in}}%
\pgfusepath{stroke}%
\end{pgfscope}%
\begin{pgfscope}%
\pgfpathrectangle{\pgfqpoint{0.647939in}{0.492442in}}{\pgfqpoint{4.273799in}{2.331163in}}%
\pgfusepath{clip}%
\pgfsetbuttcap%
\pgfsetroundjoin%
\pgfsetlinewidth{0.301125pt}%
\definecolor{currentstroke}{rgb}{0.500000,0.500000,0.500000}%
\pgfsetstrokecolor{currentstroke}%
\pgfsetstrokeopacity{0.300000}%
\pgfsetdash{}{0pt}%
\pgfpathmoveto{\pgfqpoint{3.561893in}{0.492442in}}%
\pgfpathlineto{\pgfqpoint{3.561893in}{0.492442in}}%
\pgfpathlineto{\pgfqpoint{3.500131in}{0.531795in}}%
\pgfpathlineto{\pgfqpoint{3.439060in}{0.571466in}}%
\pgfpathlineto{\pgfqpoint{3.378904in}{0.611549in}}%
\pgfpathlineto{\pgfqpoint{3.319839in}{0.652113in}}%
\pgfpathlineto{\pgfqpoint{3.262018in}{0.693205in}}%
\pgfpathlineto{\pgfqpoint{3.205552in}{0.734853in}}%
\pgfpathlineto{\pgfqpoint{3.150517in}{0.777069in}}%
\pgfpathlineto{\pgfqpoint{3.096967in}{0.819848in}}%
\pgfpathlineto{\pgfqpoint{3.044932in}{0.863179in}}%
\pgfpathlineto{\pgfqpoint{2.994418in}{0.907042in}}%
\pgfpathlineto{\pgfqpoint{2.945419in}{0.951414in}}%
\pgfpathlineto{\pgfqpoint{2.897917in}{0.996269in}}%
\pgfpathlineto{\pgfqpoint{2.851889in}{1.041579in}}%
\pgfpathlineto{\pgfqpoint{2.807309in}{1.087319in}}%
\pgfpathlineto{\pgfqpoint{2.764147in}{1.133462in}}%
\pgfpathlineto{\pgfqpoint{2.722374in}{1.179983in}}%
\pgfpathlineto{\pgfqpoint{2.681961in}{1.226861in}}%
\pgfpathlineto{\pgfqpoint{2.642887in}{1.274074in}}%
\pgfpathlineto{\pgfqpoint{2.605133in}{1.321606in}}%
\pgfpathlineto{\pgfqpoint{2.568689in}{1.369442in}}%
\pgfusepath{stroke}%
\end{pgfscope}%
\begin{pgfscope}%
\pgfpathrectangle{\pgfqpoint{0.647939in}{0.492442in}}{\pgfqpoint{4.273799in}{2.331163in}}%
\pgfusepath{clip}%
\pgfsetbuttcap%
\pgfsetroundjoin%
\pgfsetlinewidth{0.301125pt}%
\definecolor{currentstroke}{rgb}{0.500000,0.500000,0.500000}%
\pgfsetstrokecolor{currentstroke}%
\pgfsetstrokeopacity{0.300000}%
\pgfsetdash{}{0pt}%
\pgfpathmoveto{\pgfqpoint{3.756157in}{0.492442in}}%
\pgfpathlineto{\pgfqpoint{3.756157in}{0.492442in}}%
\pgfpathlineto{\pgfqpoint{3.692342in}{0.530809in}}%
\pgfpathlineto{\pgfqpoint{3.628333in}{0.569079in}}%
\pgfpathlineto{\pgfqpoint{3.564470in}{0.607421in}}%
\pgfpathlineto{\pgfqpoint{3.501081in}{0.645996in}}%
\pgfpathlineto{\pgfqpoint{3.438459in}{0.684940in}}%
\pgfpathlineto{\pgfqpoint{3.376863in}{0.724367in}}%
\pgfpathlineto{\pgfqpoint{3.316500in}{0.764356in}}%
\pgfpathlineto{\pgfqpoint{3.257540in}{0.804963in}}%
\pgfpathlineto{\pgfqpoint{3.200105in}{0.846214in}}%
\pgfpathlineto{\pgfqpoint{3.144280in}{0.888118in}}%
\pgfpathlineto{\pgfqpoint{3.090119in}{0.930666in}}%
\pgfpathlineto{\pgfqpoint{3.037647in}{0.973838in}}%
\pgfpathlineto{\pgfqpoint{2.986865in}{1.017609in}}%
\pgfpathlineto{\pgfqpoint{2.937761in}{1.061946in}}%
\pgfpathlineto{\pgfqpoint{2.890313in}{1.106818in}}%
\pgfpathlineto{\pgfqpoint{2.844493in}{1.152190in}}%
\pgfpathlineto{\pgfqpoint{2.800268in}{1.198031in}}%
\pgfpathlineto{\pgfqpoint{2.757607in}{1.244311in}}%
\pgfusepath{stroke}%
\end{pgfscope}%
\begin{pgfscope}%
\pgfpathrectangle{\pgfqpoint{0.647939in}{0.492442in}}{\pgfqpoint{4.273799in}{2.331163in}}%
\pgfusepath{clip}%
\pgfsetbuttcap%
\pgfsetroundjoin%
\pgfsetlinewidth{0.301125pt}%
\definecolor{currentstroke}{rgb}{0.500000,0.500000,0.500000}%
\pgfsetstrokecolor{currentstroke}%
\pgfsetstrokeopacity{0.300000}%
\pgfsetdash{}{0pt}%
\pgfpathmoveto{\pgfqpoint{3.853289in}{0.492442in}}%
\pgfpathlineto{\pgfqpoint{3.853289in}{0.492442in}}%
\pgfpathlineto{\pgfqpoint{3.789639in}{0.530889in}}%
\pgfpathlineto{\pgfqpoint{3.725210in}{0.568948in}}%
\pgfpathlineto{\pgfqpoint{3.660370in}{0.606800in}}%
\pgfpathlineto{\pgfqpoint{3.595498in}{0.644634in}}%
\pgfpathlineto{\pgfqpoint{3.530957in}{0.682637in}}%
\pgfusepath{stroke}%
\end{pgfscope}%
\begin{pgfscope}%
\pgfpathrectangle{\pgfqpoint{0.647939in}{0.492442in}}{\pgfqpoint{4.273799in}{2.331163in}}%
\pgfusepath{clip}%
\pgfsetbuttcap%
\pgfsetroundjoin%
\pgfsetlinewidth{0.301125pt}%
\definecolor{currentstroke}{rgb}{0.500000,0.500000,0.500000}%
\pgfsetstrokecolor{currentstroke}%
\pgfsetstrokeopacity{0.300000}%
\pgfsetdash{}{0pt}%
\pgfpathmoveto{\pgfqpoint{3.950420in}{0.492442in}}%
\pgfpathlineto{\pgfqpoint{3.950420in}{0.492442in}}%
\pgfpathlineto{\pgfqpoint{3.887882in}{0.531424in}}%
\pgfpathlineto{\pgfqpoint{3.823943in}{0.569727in}}%
\pgfpathlineto{\pgfqpoint{3.758958in}{0.607503in}}%
\pgfpathlineto{\pgfqpoint{3.693317in}{0.644941in}}%
\pgfpathlineto{\pgfqpoint{3.627428in}{0.682250in}}%
\pgfusepath{stroke}%
\end{pgfscope}%
\begin{pgfscope}%
\pgfpathrectangle{\pgfqpoint{0.647939in}{0.492442in}}{\pgfqpoint{4.273799in}{2.331163in}}%
\pgfusepath{clip}%
\pgfsetbuttcap%
\pgfsetroundjoin%
\pgfsetlinewidth{0.301125pt}%
\definecolor{currentstroke}{rgb}{0.500000,0.500000,0.500000}%
\pgfsetstrokecolor{currentstroke}%
\pgfsetstrokeopacity{0.300000}%
\pgfsetdash{}{0pt}%
\pgfpathmoveto{\pgfqpoint{4.047552in}{0.492442in}}%
\pgfpathlineto{\pgfqpoint{4.047552in}{0.492442in}}%
\pgfpathlineto{\pgfqpoint{3.987142in}{0.532407in}}%
\pgfpathlineto{\pgfqpoint{3.924754in}{0.571459in}}%
\pgfpathlineto{\pgfqpoint{3.860665in}{0.609685in}}%
\pgfpathlineto{\pgfqpoint{3.795219in}{0.647222in}}%
\pgfpathlineto{\pgfqpoint{3.728822in}{0.684261in}}%
\pgfusepath{stroke}%
\end{pgfscope}%
\begin{pgfscope}%
\pgfpathrectangle{\pgfqpoint{0.647939in}{0.492442in}}{\pgfqpoint{4.273799in}{2.331163in}}%
\pgfusepath{clip}%
\pgfsetbuttcap%
\pgfsetroundjoin%
\pgfsetlinewidth{0.301125pt}%
\definecolor{currentstroke}{rgb}{0.500000,0.500000,0.500000}%
\pgfsetstrokecolor{currentstroke}%
\pgfsetstrokeopacity{0.300000}%
\pgfsetdash{}{0pt}%
\pgfpathmoveto{\pgfqpoint{4.144684in}{0.492442in}}%
\pgfpathlineto{\pgfqpoint{4.144684in}{0.492442in}}%
\pgfpathlineto{\pgfqpoint{4.087372in}{0.533741in}}%
\pgfpathlineto{\pgfqpoint{4.027670in}{0.574018in}}%
\pgfpathlineto{\pgfqpoint{3.965685in}{0.613258in}}%
\pgfpathlineto{\pgfqpoint{3.901653in}{0.651510in}}%
\pgfpathlineto{\pgfqpoint{3.835897in}{0.688883in}}%
\pgfusepath{stroke}%
\end{pgfscope}%
\begin{pgfscope}%
\pgfpathrectangle{\pgfqpoint{0.647939in}{0.492442in}}{\pgfqpoint{4.273799in}{2.331163in}}%
\pgfusepath{clip}%
\pgfsetbuttcap%
\pgfsetroundjoin%
\pgfsetlinewidth{0.301125pt}%
\definecolor{currentstroke}{rgb}{0.500000,0.500000,0.500000}%
\pgfsetstrokecolor{currentstroke}%
\pgfsetstrokeopacity{0.300000}%
\pgfsetdash{}{0pt}%
\pgfpathmoveto{\pgfqpoint{4.338948in}{0.492442in}}%
\pgfpathlineto{\pgfqpoint{4.338948in}{0.492442in}}%
\pgfpathlineto{\pgfqpoint{4.290213in}{0.536894in}}%
\pgfpathlineto{\pgfqpoint{4.238928in}{0.580483in}}%
\pgfpathlineto{\pgfqpoint{4.184907in}{0.623078in}}%
\pgfpathlineto{\pgfqpoint{4.128038in}{0.664555in}}%
\pgfpathlineto{\pgfqpoint{4.068265in}{0.704795in}}%
\pgfpathlineto{\pgfqpoint{4.005614in}{0.743712in}}%
\pgfpathlineto{\pgfqpoint{3.940273in}{0.781292in}}%
\pgfpathlineto{\pgfqpoint{3.872561in}{0.817606in}}%
\pgfpathlineto{\pgfqpoint{3.802933in}{0.852829in}}%
\pgfpathlineto{\pgfqpoint{3.731962in}{0.887250in}}%
\pgfpathlineto{\pgfqpoint{3.660282in}{0.921234in}}%
\pgfpathlineto{\pgfqpoint{3.588522in}{0.955166in}}%
\pgfpathlineto{\pgfqpoint{3.517310in}{0.989437in}}%
\pgfpathlineto{\pgfqpoint{3.447195in}{1.024370in}}%
\pgfpathlineto{\pgfqpoint{3.378642in}{1.060211in}}%
\pgfpathlineto{\pgfqpoint{3.312032in}{1.097121in}}%
\pgfpathlineto{\pgfqpoint{3.247635in}{1.135180in}}%
\pgfpathlineto{\pgfqpoint{3.185634in}{1.174405in}}%
\pgfpathlineto{\pgfqpoint{3.126133in}{1.214764in}}%
\pgfpathlineto{\pgfqpoint{3.069169in}{1.256201in}}%
\pgfpathlineto{\pgfqpoint{3.014739in}{1.298640in}}%
\pgfpathlineto{\pgfqpoint{2.962810in}{1.342001in}}%
\pgfpathlineto{\pgfqpoint{2.913329in}{1.386205in}}%
\pgfpathlineto{\pgfqpoint{2.866239in}{1.431181in}}%
\pgfpathlineto{\pgfqpoint{2.821486in}{1.476862in}}%
\pgfpathlineto{\pgfqpoint{2.779028in}{1.523191in}}%
\pgfpathlineto{\pgfqpoint{2.738835in}{1.570118in}}%
\pgfpathlineto{\pgfqpoint{2.700903in}{1.617602in}}%
\pgfpathlineto{\pgfqpoint{2.665255in}{1.665610in}}%
\pgfpathlineto{\pgfqpoint{2.631949in}{1.714117in}}%
\pgfpathlineto{\pgfqpoint{2.601081in}{1.763102in}}%
\pgfpathlineto{\pgfqpoint{2.572801in}{1.812548in}}%
\pgfpathlineto{\pgfqpoint{2.547345in}{1.862448in}}%
\pgfpathlineto{\pgfqpoint{2.525040in}{1.912792in}}%
\pgfpathlineto{\pgfqpoint{2.506358in}{1.963569in}}%
\pgfpathlineto{\pgfqpoint{2.491992in}{2.014757in}}%
\pgfpathlineto{\pgfqpoint{2.482967in}{2.066296in}}%
\pgfpathlineto{\pgfqpoint{2.480895in}{2.118039in}}%
\pgfpathlineto{\pgfqpoint{2.488442in}{2.169575in}}%
\pgfpathlineto{\pgfqpoint{2.510359in}{2.219700in}}%
\pgfpathlineto{\pgfqpoint{2.510359in}{2.219700in}}%
\pgfpathlineto{\pgfqpoint{2.541313in}{2.254374in}}%
\pgfpathlineto{\pgfqpoint{2.541313in}{2.254374in}}%
\pgfpathlineto{\pgfqpoint{2.578321in}{2.276663in}}%
\pgfpathlineto{\pgfqpoint{2.578321in}{2.276663in}}%
\pgfpathlineto{\pgfqpoint{2.619307in}{2.288426in}}%
\pgfpathlineto{\pgfqpoint{2.665461in}{2.290739in}}%
\pgfpathlineto{\pgfqpoint{2.708671in}{2.284626in}}%
\pgfpathlineto{\pgfqpoint{2.751476in}{2.271244in}}%
\pgfpathlineto{\pgfqpoint{2.794620in}{2.250019in}}%
\pgfusepath{stroke}%
\end{pgfscope}%
\begin{pgfscope}%
\pgfpathrectangle{\pgfqpoint{0.647939in}{0.492442in}}{\pgfqpoint{4.273799in}{2.331163in}}%
\pgfusepath{clip}%
\pgfsetbuttcap%
\pgfsetroundjoin%
\pgfsetlinewidth{0.301125pt}%
\definecolor{currentstroke}{rgb}{0.500000,0.500000,0.500000}%
\pgfsetstrokecolor{currentstroke}%
\pgfsetstrokeopacity{0.300000}%
\pgfsetdash{}{0pt}%
\pgfpathmoveto{\pgfqpoint{4.436079in}{0.492442in}}%
\pgfpathlineto{\pgfqpoint{4.436079in}{0.492442in}}%
\pgfpathlineto{\pgfqpoint{4.392246in}{0.538390in}}%
\pgfpathlineto{\pgfqpoint{4.346162in}{0.583677in}}%
\pgfpathlineto{\pgfqpoint{4.297592in}{0.628181in}}%
\pgfpathlineto{\pgfqpoint{4.246290in}{0.671761in}}%
\pgfpathlineto{\pgfqpoint{4.192011in}{0.714254in}}%
\pgfpathlineto{\pgfqpoint{4.134563in}{0.755489in}}%
\pgfpathlineto{\pgfqpoint{4.073853in}{0.795305in}}%
\pgfpathlineto{\pgfqpoint{4.009869in}{0.833566in}}%
\pgfpathlineto{\pgfqpoint{3.942798in}{0.870223in}}%
\pgfpathlineto{\pgfqpoint{3.873011in}{0.905344in}}%
\pgfpathlineto{\pgfqpoint{3.801035in}{0.939132in}}%
\pgfpathlineto{\pgfqpoint{3.727562in}{0.971953in}}%
\pgfpathlineto{\pgfqpoint{3.653351in}{1.004278in}}%
\pgfpathlineto{\pgfqpoint{3.579136in}{1.036600in}}%
\pgfpathlineto{\pgfqpoint{3.505630in}{1.069397in}}%
\pgfpathlineto{\pgfqpoint{3.433481in}{1.103070in}}%
\pgfpathlineto{\pgfqpoint{3.363197in}{1.137893in}}%
\pgfpathlineto{\pgfqpoint{3.295182in}{1.174028in}}%
\pgfpathlineto{\pgfqpoint{3.229729in}{1.211542in}}%
\pgfusepath{stroke}%
\end{pgfscope}%
\begin{pgfscope}%
\pgfpathrectangle{\pgfqpoint{0.647939in}{0.492442in}}{\pgfqpoint{4.273799in}{2.331163in}}%
\pgfusepath{clip}%
\pgfsetbuttcap%
\pgfsetroundjoin%
\pgfsetlinewidth{0.301125pt}%
\definecolor{currentstroke}{rgb}{0.500000,0.500000,0.500000}%
\pgfsetstrokecolor{currentstroke}%
\pgfsetstrokeopacity{0.300000}%
\pgfsetdash{}{0pt}%
\pgfpathmoveto{\pgfqpoint{4.533211in}{0.492442in}}%
\pgfpathlineto{\pgfqpoint{4.533211in}{0.492442in}}%
\pgfpathlineto{\pgfqpoint{4.494288in}{0.539692in}}%
\pgfpathlineto{\pgfqpoint{4.453545in}{0.586481in}}%
\pgfpathlineto{\pgfqpoint{4.410754in}{0.632721in}}%
\pgfpathlineto{\pgfqpoint{4.365655in}{0.678302in}}%
\pgfpathlineto{\pgfqpoint{4.317955in}{0.723088in}}%
\pgfpathlineto{\pgfqpoint{4.267347in}{0.766912in}}%
\pgfpathlineto{\pgfqpoint{4.213533in}{0.809578in}}%
\pgfpathlineto{\pgfqpoint{4.156206in}{0.850854in}}%
\pgfpathlineto{\pgfqpoint{4.095114in}{0.890489in}}%
\pgfpathlineto{\pgfqpoint{4.030190in}{0.928267in}}%
\pgfpathlineto{\pgfqpoint{3.961552in}{0.964039in}}%
\pgfpathlineto{\pgfqpoint{3.889574in}{0.997807in}}%
\pgfpathlineto{\pgfqpoint{3.814903in}{1.029800in}}%
\pgfpathlineto{\pgfqpoint{3.738374in}{1.060469in}}%
\pgfpathlineto{\pgfqpoint{3.660889in}{1.090420in}}%
\pgfpathlineto{\pgfqpoint{3.583352in}{1.120331in}}%
\pgfusepath{stroke}%
\end{pgfscope}%
\begin{pgfscope}%
\pgfpathrectangle{\pgfqpoint{0.647939in}{0.492442in}}{\pgfqpoint{4.273799in}{2.331163in}}%
\pgfusepath{clip}%
\pgfsetbuttcap%
\pgfsetroundjoin%
\pgfsetlinewidth{0.301125pt}%
\definecolor{currentstroke}{rgb}{0.500000,0.500000,0.500000}%
\pgfsetstrokecolor{currentstroke}%
\pgfsetstrokeopacity{0.300000}%
\pgfsetdash{}{0pt}%
\pgfpathmoveto{\pgfqpoint{4.630343in}{0.492442in}}%
\pgfpathlineto{\pgfqpoint{4.630343in}{0.492442in}}%
\pgfpathlineto{\pgfqpoint{4.596124in}{0.540762in}}%
\pgfpathlineto{\pgfqpoint{4.560541in}{0.588787in}}%
\pgfpathlineto{\pgfqpoint{4.523419in}{0.636464in}}%
\pgfpathlineto{\pgfqpoint{4.484554in}{0.683725in}}%
\pgfpathlineto{\pgfqpoint{4.443706in}{0.730487in}}%
\pgfpathlineto{\pgfqpoint{4.400596in}{0.776639in}}%
\pgfpathlineto{\pgfqpoint{4.354897in}{0.822042in}}%
\pgfpathlineto{\pgfqpoint{4.306235in}{0.866513in}}%
\pgfpathlineto{\pgfqpoint{4.254182in}{0.909819in}}%
\pgfpathlineto{\pgfqpoint{4.198271in}{0.951663in}}%
\pgfpathlineto{\pgfqpoint{4.138046in}{0.991683in}}%
\pgfpathlineto{\pgfqpoint{4.073218in}{1.029493in}}%
\pgfpathlineto{\pgfqpoint{4.003704in}{1.064733in}}%
\pgfpathlineto{\pgfqpoint{3.929784in}{1.097203in}}%
\pgfpathlineto{\pgfqpoint{3.852175in}{1.127021in}}%
\pgfpathlineto{\pgfqpoint{3.771934in}{1.154710in}}%
\pgfpathlineto{\pgfqpoint{3.690220in}{1.181101in}}%
\pgfpathlineto{\pgfqpoint{3.608191in}{1.207200in}}%
\pgfpathlineto{\pgfqpoint{3.526915in}{1.233975in}}%
\pgfpathlineto{\pgfqpoint{3.447330in}{1.262206in}}%
\pgfpathlineto{\pgfqpoint{3.370220in}{1.292407in}}%
\pgfpathlineto{\pgfqpoint{3.296223in}{1.324829in}}%
\pgfpathlineto{\pgfqpoint{3.225749in}{1.359498in}}%
\pgfpathlineto{\pgfqpoint{3.158987in}{1.396296in}}%
\pgfpathlineto{\pgfqpoint{3.096013in}{1.435031in}}%
\pgfpathlineto{\pgfqpoint{3.036781in}{1.475488in}}%
\pgfpathlineto{\pgfqpoint{2.981181in}{1.517459in}}%
\pgfpathlineto{\pgfqpoint{2.929087in}{1.560753in}}%
\pgfpathlineto{\pgfqpoint{2.880376in}{1.605203in}}%
\pgfpathlineto{\pgfqpoint{2.834945in}{1.650675in}}%
\pgfpathlineto{\pgfqpoint{2.792726in}{1.697060in}}%
\pgfpathlineto{\pgfqpoint{2.753703in}{1.744273in}}%
\pgfpathlineto{\pgfqpoint{2.717928in}{1.792247in}}%
\pgfpathlineto{\pgfqpoint{2.685529in}{1.840930in}}%
\pgfpathlineto{\pgfqpoint{2.656741in}{1.890279in}}%
\pgfpathlineto{\pgfqpoint{2.631967in}{1.940270in}}%
\pgfpathlineto{\pgfqpoint{2.611841in}{1.990873in}}%
\pgfpathlineto{\pgfqpoint{2.597383in}{2.042038in}}%
\pgfpathlineto{\pgfqpoint{2.590331in}{2.093643in}}%
\pgfpathlineto{\pgfqpoint{2.593915in}{2.145280in}}%
\pgfpathlineto{\pgfqpoint{2.615122in}{2.195293in}}%
\pgfpathlineto{\pgfqpoint{2.615122in}{2.195293in}}%
\pgfpathlineto{\pgfqpoint{2.641875in}{2.221709in}}%
\pgfpathlineto{\pgfqpoint{2.641875in}{2.221709in}}%
\pgfpathlineto{\pgfqpoint{2.674003in}{2.236628in}}%
\pgfpathlineto{\pgfqpoint{2.674003in}{2.236628in}}%
\pgfpathlineto{\pgfqpoint{2.709021in}{2.241623in}}%
\pgfusepath{stroke}%
\end{pgfscope}%
\begin{pgfscope}%
\pgfpathrectangle{\pgfqpoint{0.647939in}{0.492442in}}{\pgfqpoint{4.273799in}{2.331163in}}%
\pgfusepath{clip}%
\pgfsetbuttcap%
\pgfsetroundjoin%
\pgfsetlinewidth{0.301125pt}%
\definecolor{currentstroke}{rgb}{0.500000,0.500000,0.500000}%
\pgfsetstrokecolor{currentstroke}%
\pgfsetstrokeopacity{0.300000}%
\pgfsetdash{}{0pt}%
\pgfpathmoveto{\pgfqpoint{4.727475in}{0.492442in}}%
\pgfpathlineto{\pgfqpoint{4.727475in}{0.492442in}}%
\pgfpathlineto{\pgfqpoint{4.697552in}{0.541605in}}%
\pgfpathlineto{\pgfqpoint{4.666673in}{0.590591in}}%
\pgfpathlineto{\pgfqpoint{4.634737in}{0.639375in}}%
\pgfpathlineto{\pgfqpoint{4.601619in}{0.687922in}}%
\pgfpathlineto{\pgfqpoint{4.567155in}{0.736190in}}%
\pgfpathlineto{\pgfqpoint{4.531152in}{0.784123in}}%
\pgfpathlineto{\pgfqpoint{4.493391in}{0.831651in}}%
\pgfpathlineto{\pgfqpoint{4.453601in}{0.878682in}}%
\pgfpathlineto{\pgfqpoint{4.411442in}{0.925091in}}%
\pgfpathlineto{\pgfqpoint{4.366488in}{0.970710in}}%
\pgfpathlineto{\pgfqpoint{4.318212in}{1.015304in}}%
\pgfpathlineto{\pgfqpoint{4.265997in}{1.058551in}}%
\pgfpathlineto{\pgfqpoint{4.209162in}{1.100009in}}%
\pgfpathlineto{\pgfqpoint{4.146935in}{1.139077in}}%
\pgfpathlineto{\pgfqpoint{4.078720in}{1.175032in}}%
\pgfpathlineto{\pgfqpoint{4.004394in}{1.207155in}}%
\pgfpathlineto{\pgfqpoint{3.924493in}{1.235027in}}%
\pgfpathlineto{\pgfqpoint{3.840329in}{1.258925in}}%
\pgfpathlineto{\pgfqpoint{3.753556in}{1.279929in}}%
\pgfpathlineto{\pgfqpoint{3.665738in}{1.299633in}}%
\pgfpathlineto{\pgfqpoint{3.578207in}{1.319706in}}%
\pgfpathlineto{\pgfqpoint{3.492138in}{1.341546in}}%
\pgfpathlineto{\pgfqpoint{3.408594in}{1.366099in}}%
\pgfpathlineto{\pgfqpoint{3.328476in}{1.393829in}}%
\pgfpathlineto{\pgfqpoint{3.252475in}{1.424794in}}%
\pgfpathlineto{\pgfqpoint{3.180951in}{1.458780in}}%
\pgfusepath{stroke}%
\end{pgfscope}%
\begin{pgfscope}%
\pgfpathrectangle{\pgfqpoint{0.647939in}{0.492442in}}{\pgfqpoint{4.273799in}{2.331163in}}%
\pgfusepath{clip}%
\pgfsetbuttcap%
\pgfsetroundjoin%
\pgfsetlinewidth{0.301125pt}%
\definecolor{currentstroke}{rgb}{0.500000,0.500000,0.500000}%
\pgfsetstrokecolor{currentstroke}%
\pgfsetstrokeopacity{0.300000}%
\pgfsetdash{}{0pt}%
\pgfpathmoveto{\pgfqpoint{4.824607in}{0.492442in}}%
\pgfpathlineto{\pgfqpoint{4.824607in}{0.492442in}}%
\pgfpathlineto{\pgfqpoint{4.798512in}{0.542250in}}%
\pgfpathlineto{\pgfqpoint{4.771806in}{0.591962in}}%
\pgfpathlineto{\pgfqpoint{4.744426in}{0.641564in}}%
\pgfpathlineto{\pgfqpoint{4.716301in}{0.691042in}}%
\pgfpathlineto{\pgfqpoint{4.687357in}{0.740378in}}%
\pgfpathlineto{\pgfqpoint{4.657487in}{0.789550in}}%
\pgfpathlineto{\pgfqpoint{4.626574in}{0.838530in}}%
\pgfpathlineto{\pgfqpoint{4.594483in}{0.887282in}}%
\pgfpathlineto{\pgfqpoint{4.561036in}{0.935760in}}%
\pgfpathlineto{\pgfqpoint{4.526003in}{0.983904in}}%
\pgfpathlineto{\pgfqpoint{4.489096in}{1.031631in}}%
\pgfpathlineto{\pgfqpoint{4.449966in}{1.078826in}}%
\pgfpathlineto{\pgfqpoint{4.408146in}{1.125325in}}%
\pgfpathlineto{\pgfqpoint{4.363008in}{1.170886in}}%
\pgfpathlineto{\pgfqpoint{4.313707in}{1.215137in}}%
\pgfpathlineto{\pgfqpoint{4.259183in}{1.257498in}}%
\pgfpathlineto{\pgfqpoint{4.198080in}{1.297046in}}%
\pgfpathlineto{\pgfqpoint{4.128939in}{1.332368in}}%
\pgfpathlineto{\pgfqpoint{4.051000in}{1.361641in}}%
\pgfpathlineto{\pgfqpoint{3.965247in}{1.383449in}}%
\pgfpathlineto{\pgfqpoint{3.874426in}{1.398135in}}%
\pgfpathlineto{\pgfqpoint{3.781270in}{1.407986in}}%
\pgfpathlineto{\pgfqpoint{3.687483in}{1.416067in}}%
\pgfpathlineto{\pgfqpoint{3.594068in}{1.425212in}}%
\pgfpathlineto{\pgfqpoint{3.501974in}{1.437596in}}%
\pgfpathlineto{\pgfqpoint{3.412435in}{1.454589in}}%
\pgfpathlineto{\pgfqpoint{3.326774in}{1.476722in}}%
\pgfpathlineto{\pgfqpoint{3.246066in}{1.503830in}}%
\pgfpathlineto{\pgfqpoint{3.170902in}{1.535325in}}%
\pgfpathlineto{\pgfqpoint{3.101385in}{1.570490in}}%
\pgfpathlineto{\pgfqpoint{3.037344in}{1.608656in}}%
\pgfpathlineto{\pgfqpoint{2.978503in}{1.649254in}}%
\pgfpathlineto{\pgfqpoint{2.924544in}{1.691841in}}%
\pgfpathlineto{\pgfqpoint{2.875214in}{1.736064in}}%
\pgfusepath{stroke}%
\end{pgfscope}%
\begin{pgfscope}%
\pgfpathrectangle{\pgfqpoint{0.647939in}{0.492442in}}{\pgfqpoint{4.273799in}{2.331163in}}%
\pgfusepath{clip}%
\pgfsetbuttcap%
\pgfsetroundjoin%
\pgfsetlinewidth{0.301125pt}%
\definecolor{currentstroke}{rgb}{0.500000,0.500000,0.500000}%
\pgfsetstrokecolor{currentstroke}%
\pgfsetstrokeopacity{0.300000}%
\pgfsetdash{}{0pt}%
\pgfpathmoveto{\pgfqpoint{4.921738in}{0.492442in}}%
\pgfpathlineto{\pgfqpoint{4.921738in}{0.492442in}}%
\pgfpathlineto{\pgfqpoint{4.898994in}{0.542737in}}%
\pgfpathlineto{\pgfqpoint{4.875892in}{0.592984in}}%
\pgfpathlineto{\pgfqpoint{4.852405in}{0.643177in}}%
\pgfpathlineto{\pgfqpoint{4.828503in}{0.693312in}}%
\pgfpathlineto{\pgfqpoint{4.804151in}{0.743382in}}%
\pgfpathlineto{\pgfqpoint{4.779310in}{0.793380in}}%
\pgfpathlineto{\pgfqpoint{4.753927in}{0.843298in}}%
\pgfpathlineto{\pgfqpoint{4.727952in}{0.893124in}}%
\pgfpathlineto{\pgfqpoint{4.701316in}{0.942846in}}%
\pgfpathlineto{\pgfqpoint{4.673936in}{0.992448in}}%
\pgfpathlineto{\pgfqpoint{4.645722in}{1.041909in}}%
\pgfpathlineto{\pgfqpoint{4.616538in}{1.091203in}}%
\pgfpathlineto{\pgfqpoint{4.586234in}{1.140295in}}%
\pgfpathlineto{\pgfqpoint{4.554620in}{1.189137in}}%
\pgfpathlineto{\pgfqpoint{4.521424in}{1.237665in}}%
\pgfpathlineto{\pgfqpoint{4.486274in}{1.285781in}}%
\pgfpathlineto{\pgfqpoint{4.448670in}{1.333341in}}%
\pgfpathlineto{\pgfqpoint{4.407892in}{1.380109in}}%
\pgfpathlineto{\pgfqpoint{4.362841in}{1.425681in}}%
\pgfpathlineto{\pgfqpoint{4.311770in}{1.469295in}}%
\pgfpathlineto{\pgfqpoint{4.252055in}{1.509374in}}%
\pgfpathlineto{\pgfqpoint{4.179932in}{1.542513in}}%
\pgfpathlineto{\pgfqpoint{4.179932in}{1.542513in}}%
\pgfpathlineto{\pgfqpoint{4.114590in}{1.559731in}}%
\pgfpathlineto{\pgfqpoint{4.042677in}{1.567263in}}%
\pgfpathlineto{\pgfqpoint{3.971154in}{1.566150in}}%
\pgfpathlineto{\pgfqpoint{3.889479in}{1.558522in}}%
\pgfpathlineto{\pgfqpoint{3.797128in}{1.546613in}}%
\pgfpathlineto{\pgfqpoint{3.704541in}{1.535282in}}%
\pgfpathlineto{\pgfqpoint{3.610813in}{1.527586in}}%
\pgfpathlineto{\pgfqpoint{3.516213in}{1.525883in}}%
\pgfpathlineto{\pgfqpoint{3.422296in}{1.531797in}}%
\pgfpathlineto{\pgfqpoint{3.331392in}{1.545894in}}%
\pgfpathlineto{\pgfqpoint{3.245645in}{1.567646in}}%
\pgfpathlineto{\pgfqpoint{3.166361in}{1.595829in}}%
\pgfusepath{stroke}%
\end{pgfscope}%
\begin{pgfscope}%
\pgfpathrectangle{\pgfqpoint{0.647939in}{0.492442in}}{\pgfqpoint{4.273799in}{2.331163in}}%
\pgfusepath{clip}%
\pgfsetbuttcap%
\pgfsetroundjoin%
\pgfsetlinewidth{0.301125pt}%
\definecolor{currentstroke}{rgb}{0.500000,0.500000,0.500000}%
\pgfsetstrokecolor{currentstroke}%
\pgfsetstrokeopacity{0.300000}%
\pgfsetdash{}{0pt}%
\pgfpathmoveto{\pgfqpoint{4.921738in}{0.704366in}}%
\pgfpathlineto{\pgfqpoint{4.921738in}{0.704366in}}%
\pgfpathlineto{\pgfqpoint{4.900699in}{0.754882in}}%
\pgfpathlineto{\pgfqpoint{4.879422in}{0.805368in}}%
\pgfpathlineto{\pgfqpoint{4.857894in}{0.855822in}}%
\pgfpathlineto{\pgfqpoint{4.836098in}{0.906242in}}%
\pgfpathlineto{\pgfqpoint{4.814015in}{0.956625in}}%
\pgfpathlineto{\pgfqpoint{4.791626in}{1.006967in}}%
\pgfpathlineto{\pgfqpoint{4.768906in}{1.057265in}}%
\pgfpathlineto{\pgfqpoint{4.745826in}{1.107514in}}%
\pgfpathlineto{\pgfqpoint{4.722352in}{1.157709in}}%
\pgfpathlineto{\pgfqpoint{4.698444in}{1.207841in}}%
\pgfpathlineto{\pgfqpoint{4.674047in}{1.257904in}}%
\pgfpathlineto{\pgfqpoint{4.649101in}{1.307885in}}%
\pgfpathlineto{\pgfqpoint{4.623516in}{1.357770in}}%
\pgfpathlineto{\pgfqpoint{4.597191in}{1.407540in}}%
\pgfpathlineto{\pgfqpoint{4.569981in}{1.457166in}}%
\pgfpathlineto{\pgfqpoint{4.541680in}{1.506610in}}%
\pgfpathlineto{\pgfqpoint{4.512011in}{1.555810in}}%
\pgfpathlineto{\pgfqpoint{4.480521in}{1.604668in}}%
\pgfpathlineto{\pgfqpoint{4.446451in}{1.653008in}}%
\pgfpathlineto{\pgfqpoint{4.408494in}{1.700462in}}%
\pgfpathlineto{\pgfqpoint{4.363927in}{1.746101in}}%
\pgfpathlineto{\pgfqpoint{4.305871in}{1.786579in}}%
\pgfpathlineto{\pgfqpoint{4.305871in}{1.786579in}}%
\pgfpathlineto{\pgfqpoint{4.264282in}{1.801783in}}%
\pgfpathlineto{\pgfqpoint{4.264282in}{1.801783in}}%
\pgfpathlineto{\pgfqpoint{4.222428in}{1.806189in}}%
\pgfpathlineto{\pgfqpoint{4.180469in}{1.801662in}}%
\pgfpathlineto{\pgfqpoint{4.138357in}{1.790481in}}%
\pgfpathlineto{\pgfqpoint{4.086815in}{1.770968in}}%
\pgfpathlineto{\pgfqpoint{4.015479in}{1.738777in}}%
\pgfpathlineto{\pgfqpoint{3.943570in}{1.705012in}}%
\pgfpathlineto{\pgfqpoint{3.870055in}{1.672312in}}%
\pgfpathlineto{\pgfqpoint{3.793262in}{1.641950in}}%
\pgfpathlineto{\pgfqpoint{3.712030in}{1.615333in}}%
\pgfpathlineto{\pgfqpoint{3.625709in}{1.594168in}}%
\pgfpathlineto{\pgfqpoint{3.534564in}{1.580496in}}%
\pgfusepath{stroke}%
\end{pgfscope}%
\begin{pgfscope}%
\pgfpathrectangle{\pgfqpoint{0.647939in}{0.492442in}}{\pgfqpoint{4.273799in}{2.331163in}}%
\pgfusepath{clip}%
\pgfsetbuttcap%
\pgfsetroundjoin%
\pgfsetlinewidth{0.301125pt}%
\definecolor{currentstroke}{rgb}{0.500000,0.500000,0.500000}%
\pgfsetstrokecolor{currentstroke}%
\pgfsetstrokeopacity{0.300000}%
\pgfsetdash{}{0pt}%
\pgfpathmoveto{\pgfqpoint{4.921738in}{0.916290in}}%
\pgfpathlineto{\pgfqpoint{4.921738in}{0.916290in}}%
\pgfpathlineto{\pgfqpoint{4.902663in}{0.967037in}}%
\pgfpathlineto{\pgfqpoint{4.883504in}{1.017775in}}%
\pgfpathlineto{\pgfqpoint{4.864263in}{1.068503in}}%
\pgfpathlineto{\pgfqpoint{4.844945in}{1.119223in}}%
\pgfpathlineto{\pgfqpoint{4.825554in}{1.169934in}}%
\pgfpathlineto{\pgfqpoint{4.806100in}{1.220637in}}%
\pgfpathlineto{\pgfqpoint{4.786587in}{1.271334in}}%
\pgfpathlineto{\pgfqpoint{4.767029in}{1.322026in}}%
\pgfpathlineto{\pgfqpoint{4.747437in}{1.372714in}}%
\pgfpathlineto{\pgfqpoint{4.727830in}{1.423400in}}%
\pgfpathlineto{\pgfqpoint{4.708231in}{1.474087in}}%
\pgfpathlineto{\pgfqpoint{4.688668in}{1.524778in}}%
\pgfpathlineto{\pgfqpoint{4.669182in}{1.575477in}}%
\pgfpathlineto{\pgfqpoint{4.649818in}{1.626190in}}%
\pgfpathlineto{\pgfqpoint{4.630647in}{1.676923in}}%
\pgfpathlineto{\pgfqpoint{4.611755in}{1.727688in}}%
\pgfpathlineto{\pgfqpoint{4.593277in}{1.778497in}}%
\pgfpathlineto{\pgfqpoint{4.575387in}{1.829369in}}%
\pgfpathlineto{\pgfqpoint{4.558366in}{1.880329in}}%
\pgfpathlineto{\pgfqpoint{4.542643in}{1.931410in}}%
\pgfpathlineto{\pgfqpoint{4.528868in}{1.982656in}}%
\pgfpathlineto{\pgfqpoint{4.518131in}{2.034111in}}%
\pgfpathlineto{\pgfqpoint{4.512146in}{2.085780in}}%
\pgfpathlineto{\pgfqpoint{4.513272in}{2.137521in}}%
\pgfpathlineto{\pgfqpoint{4.523547in}{2.188915in}}%
\pgfpathlineto{\pgfqpoint{4.542772in}{2.239541in}}%
\pgfpathlineto{\pgfqpoint{4.568404in}{2.289303in}}%
\pgfpathlineto{\pgfqpoint{4.597770in}{2.338453in}}%
\pgfpathlineto{\pgfqpoint{4.629030in}{2.387278in}}%
\pgfpathlineto{\pgfqpoint{4.661139in}{2.435985in}}%
\pgfpathlineto{\pgfqpoint{4.693435in}{2.484659in}}%
\pgfpathlineto{\pgfqpoint{4.725570in}{2.533348in}}%
\pgfpathlineto{\pgfqpoint{4.757426in}{2.582109in}}%
\pgfpathlineto{\pgfqpoint{4.788919in}{2.630961in}}%
\pgfpathlineto{\pgfqpoint{4.819961in}{2.679897in}}%
\pgfpathlineto{\pgfqpoint{4.850527in}{2.728917in}}%
\pgfpathlineto{\pgfqpoint{4.880648in}{2.778032in}}%
\pgfpathlineto{\pgfqpoint{4.908386in}{2.823605in}}%
\pgfusepath{stroke}%
\end{pgfscope}%
\begin{pgfscope}%
\pgfpathrectangle{\pgfqpoint{0.647939in}{0.492442in}}{\pgfqpoint{4.273799in}{2.331163in}}%
\pgfusepath{clip}%
\pgfsetbuttcap%
\pgfsetroundjoin%
\pgfsetlinewidth{0.301125pt}%
\definecolor{currentstroke}{rgb}{0.500000,0.500000,0.500000}%
\pgfsetstrokecolor{currentstroke}%
\pgfsetstrokeopacity{0.300000}%
\pgfsetdash{}{0pt}%
\pgfpathmoveto{\pgfqpoint{4.921738in}{1.181195in}}%
\pgfpathlineto{\pgfqpoint{4.921738in}{1.181195in}}%
\pgfpathlineto{\pgfqpoint{4.905572in}{1.232242in}}%
\pgfpathlineto{\pgfqpoint{4.889568in}{1.283304in}}%
\pgfpathlineto{\pgfqpoint{4.873752in}{1.334383in}}%
\pgfpathlineto{\pgfqpoint{4.858166in}{1.385483in}}%
\pgfpathlineto{\pgfqpoint{4.842851in}{1.436608in}}%
\pgfpathlineto{\pgfqpoint{4.827858in}{1.487761in}}%
\pgfpathlineto{\pgfqpoint{4.813250in}{1.538947in}}%
\pgfpathlineto{\pgfqpoint{4.799097in}{1.590171in}}%
\pgfpathlineto{\pgfqpoint{4.785481in}{1.641438in}}%
\pgfpathlineto{\pgfqpoint{4.772507in}{1.692754in}}%
\pgfpathlineto{\pgfqpoint{4.760295in}{1.744126in}}%
\pgfpathlineto{\pgfqpoint{4.748981in}{1.795560in}}%
\pgfpathlineto{\pgfqpoint{4.738737in}{1.847059in}}%
\pgfpathlineto{\pgfqpoint{4.729768in}{1.898629in}}%
\pgfpathlineto{\pgfqpoint{4.722299in}{1.950269in}}%
\pgfpathlineto{\pgfqpoint{4.716590in}{2.001976in}}%
\pgfpathlineto{\pgfqpoint{4.712928in}{2.053737in}}%
\pgfpathlineto{\pgfqpoint{4.711610in}{2.105530in}}%
\pgfpathlineto{\pgfqpoint{4.712920in}{2.157322in}}%
\pgfpathlineto{\pgfqpoint{4.717092in}{2.209068in}}%
\pgfpathlineto{\pgfqpoint{4.724267in}{2.260715in}}%
\pgfpathlineto{\pgfqpoint{4.734464in}{2.312209in}}%
\pgfpathlineto{\pgfqpoint{4.747574in}{2.363506in}}%
\pgfpathlineto{\pgfqpoint{4.763343in}{2.414579in}}%
\pgfpathlineto{\pgfqpoint{4.781431in}{2.465421in}}%
\pgfpathlineto{\pgfqpoint{4.801476in}{2.516047in}}%
\pgfpathlineto{\pgfqpoint{4.823096in}{2.566479in}}%
\pgfusepath{stroke}%
\end{pgfscope}%
\begin{pgfscope}%
\pgfpathrectangle{\pgfqpoint{0.647939in}{0.492442in}}{\pgfqpoint{4.273799in}{2.331163in}}%
\pgfusepath{clip}%
\pgfsetbuttcap%
\pgfsetroundjoin%
\pgfsetlinewidth{0.301125pt}%
\definecolor{currentstroke}{rgb}{0.500000,0.500000,0.500000}%
\pgfsetstrokecolor{currentstroke}%
\pgfsetstrokeopacity{0.300000}%
\pgfsetdash{}{0pt}%
\pgfpathmoveto{\pgfqpoint{4.921738in}{1.446100in}}%
\pgfpathlineto{\pgfqpoint{4.921738in}{1.446100in}}%
\pgfpathlineto{\pgfqpoint{4.909132in}{1.497444in}}%
\pgfpathlineto{\pgfqpoint{4.897000in}{1.548823in}}%
\pgfpathlineto{\pgfqpoint{4.885403in}{1.600238in}}%
\pgfpathlineto{\pgfqpoint{4.874413in}{1.651693in}}%
\pgfpathlineto{\pgfqpoint{4.864117in}{1.703190in}}%
\pgfpathlineto{\pgfqpoint{4.854609in}{1.754732in}}%
\pgfpathlineto{\pgfqpoint{4.845990in}{1.806321in}}%
\pgfpathlineto{\pgfqpoint{4.838374in}{1.857956in}}%
\pgfpathlineto{\pgfqpoint{4.831890in}{1.909638in}}%
\pgfpathlineto{\pgfqpoint{4.826675in}{1.961362in}}%
\pgfpathlineto{\pgfqpoint{4.822874in}{2.013122in}}%
\pgfpathlineto{\pgfqpoint{4.820631in}{2.064909in}}%
\pgfpathlineto{\pgfqpoint{4.820089in}{2.116709in}}%
\pgfpathlineto{\pgfqpoint{4.821379in}{2.168505in}}%
\pgfpathlineto{\pgfqpoint{4.824608in}{2.220275in}}%
\pgfpathlineto{\pgfqpoint{4.829854in}{2.271996in}}%
\pgfpathlineto{\pgfqpoint{4.837152in}{2.323642in}}%
\pgfpathlineto{\pgfqpoint{4.846481in}{2.375191in}}%
\pgfusepath{stroke}%
\end{pgfscope}%
\begin{pgfscope}%
\pgfpathrectangle{\pgfqpoint{0.647939in}{0.492442in}}{\pgfqpoint{4.273799in}{2.331163in}}%
\pgfusepath{clip}%
\pgfsetbuttcap%
\pgfsetroundjoin%
\pgfsetlinewidth{0.301125pt}%
\definecolor{currentstroke}{rgb}{0.500000,0.500000,0.500000}%
\pgfsetstrokecolor{currentstroke}%
\pgfsetstrokeopacity{0.300000}%
\pgfsetdash{}{0pt}%
\pgfpathmoveto{\pgfqpoint{4.921738in}{1.711005in}}%
\pgfpathlineto{\pgfqpoint{4.921738in}{1.711005in}}%
\pgfpathlineto{\pgfqpoint{4.913554in}{1.762615in}}%
\pgfpathlineto{\pgfqpoint{4.906219in}{1.814263in}}%
\pgfpathlineto{\pgfqpoint{4.899824in}{1.865948in}}%
\pgfpathlineto{\pgfqpoint{4.894463in}{1.917668in}}%
\pgfpathlineto{\pgfqpoint{4.890235in}{1.969419in}}%
\pgfpathlineto{\pgfqpoint{4.887243in}{2.021195in}}%
\pgfpathlineto{\pgfqpoint{4.885588in}{2.072989in}}%
\pgfpathlineto{\pgfqpoint{4.885367in}{2.124791in}}%
\pgfpathlineto{\pgfqpoint{4.886668in}{2.176588in}}%
\pgfpathlineto{\pgfqpoint{4.889562in}{2.228365in}}%
\pgfpathlineto{\pgfqpoint{4.894101in}{2.280107in}}%
\pgfpathlineto{\pgfqpoint{4.900314in}{2.331797in}}%
\pgfpathlineto{\pgfqpoint{4.908199in}{2.383418in}}%
\pgfpathlineto{\pgfqpoint{4.917730in}{2.434957in}}%
\pgfpathlineto{\pgfqpoint{4.921738in}{2.454941in}}%
\pgfusepath{stroke}%
\end{pgfscope}%
\begin{pgfscope}%
\pgfpathrectangle{\pgfqpoint{0.647939in}{0.492442in}}{\pgfqpoint{4.273799in}{2.331163in}}%
\pgfusepath{clip}%
\pgfsetbuttcap%
\pgfsetroundjoin%
\pgfsetlinewidth{0.301125pt}%
\definecolor{currentstroke}{rgb}{0.500000,0.500000,0.500000}%
\pgfsetstrokecolor{currentstroke}%
\pgfsetstrokeopacity{0.300000}%
\pgfsetdash{}{0pt}%
\pgfpathmoveto{\pgfqpoint{4.351532in}{2.823605in}}%
\pgfpathlineto{\pgfqpoint{4.364123in}{2.810319in}}%
\pgfpathlineto{\pgfqpoint{4.410412in}{2.765138in}}%
\pgfpathlineto{\pgfqpoint{4.456228in}{2.728439in}}%
\pgfpathlineto{\pgfqpoint{4.496028in}{2.704203in}}%
\pgfpathlineto{\pgfqpoint{4.532885in}{2.688922in}}%
\pgfpathlineto{\pgfqpoint{4.573769in}{2.680759in}}%
\pgfpathlineto{\pgfqpoint{4.617516in}{2.682567in}}%
\pgfpathlineto{\pgfqpoint{4.617516in}{2.682567in}}%
\pgfpathlineto{\pgfqpoint{4.665270in}{2.696883in}}%
\pgfpathlineto{\pgfqpoint{4.665270in}{2.696883in}}%
\pgfpathlineto{\pgfqpoint{4.728885in}{2.734567in}}%
\pgfpathlineto{\pgfqpoint{4.780120in}{2.777936in}}%
\pgfpathlineto{\pgfqpoint{4.824607in}{2.823605in}}%
\pgfpathlineto{\pgfqpoint{4.824607in}{2.823605in}}%
\pgfusepath{stroke}%
\end{pgfscope}%
\begin{pgfscope}%
\pgfpathrectangle{\pgfqpoint{0.647939in}{0.492442in}}{\pgfqpoint{4.273799in}{2.331163in}}%
\pgfusepath{clip}%
\pgfsetbuttcap%
\pgfsetroundjoin%
\pgfsetlinewidth{0.301125pt}%
\definecolor{currentstroke}{rgb}{0.500000,0.500000,0.500000}%
\pgfsetstrokecolor{currentstroke}%
\pgfsetstrokeopacity{0.300000}%
\pgfsetdash{}{0pt}%
\pgfpathmoveto{\pgfqpoint{4.436079in}{2.823605in}}%
\pgfpathlineto{\pgfqpoint{4.436079in}{2.823605in}}%
\pgfpathlineto{\pgfqpoint{4.494319in}{2.782897in}}%
\pgfpathlineto{\pgfqpoint{4.494319in}{2.782897in}}%
\pgfpathlineto{\pgfqpoint{4.546054in}{2.759432in}}%
\pgfpathlineto{\pgfqpoint{4.546054in}{2.759432in}}%
\pgfpathlineto{\pgfqpoint{4.591307in}{2.749924in}}%
\pgfpathlineto{\pgfqpoint{4.640203in}{2.751850in}}%
\pgfpathlineto{\pgfqpoint{4.679194in}{2.762236in}}%
\pgfpathlineto{\pgfqpoint{4.717403in}{2.779778in}}%
\pgfpathlineto{\pgfqpoint{4.759032in}{2.806646in}}%
\pgfpathlineto{\pgfqpoint{4.782137in}{2.823605in}}%
\pgfusepath{stroke}%
\end{pgfscope}%
\begin{pgfscope}%
\pgfpathrectangle{\pgfqpoint{0.647939in}{0.492442in}}{\pgfqpoint{4.273799in}{2.331163in}}%
\pgfusepath{clip}%
\pgfsetbuttcap%
\pgfsetroundjoin%
\pgfsetlinewidth{0.301125pt}%
\definecolor{currentstroke}{rgb}{0.500000,0.500000,0.500000}%
\pgfsetstrokecolor{currentstroke}%
\pgfsetstrokeopacity{0.300000}%
\pgfsetdash{}{0pt}%
\pgfpathmoveto{\pgfqpoint{4.241816in}{2.823605in}}%
\pgfpathlineto{\pgfqpoint{4.241816in}{2.823605in}}%
\pgfpathlineto{\pgfqpoint{4.277452in}{2.775591in}}%
\pgfpathlineto{\pgfqpoint{4.314896in}{2.727990in}}%
\pgfpathlineto{\pgfqpoint{4.355173in}{2.681094in}}%
\pgfpathlineto{\pgfqpoint{4.400236in}{2.635546in}}%
\pgfpathlineto{\pgfqpoint{4.454541in}{2.593259in}}%
\pgfpathlineto{\pgfqpoint{4.454541in}{2.593259in}}%
\pgfpathlineto{\pgfqpoint{4.499548in}{2.571034in}}%
\pgfpathlineto{\pgfqpoint{4.499548in}{2.571034in}}%
\pgfpathlineto{\pgfqpoint{4.539420in}{2.562420in}}%
\pgfpathlineto{\pgfqpoint{4.582443in}{2.565350in}}%
\pgfpathlineto{\pgfqpoint{4.616017in}{2.575907in}}%
\pgfpathlineto{\pgfqpoint{4.650006in}{2.593537in}}%
\pgfpathlineto{\pgfqpoint{4.688390in}{2.620975in}}%
\pgfpathlineto{\pgfqpoint{4.733886in}{2.662296in}}%
\pgfusepath{stroke}%
\end{pgfscope}%
\begin{pgfscope}%
\pgfpathrectangle{\pgfqpoint{0.647939in}{0.492442in}}{\pgfqpoint{4.273799in}{2.331163in}}%
\pgfusepath{clip}%
\pgfsetbuttcap%
\pgfsetroundjoin%
\pgfsetlinewidth{0.301125pt}%
\definecolor{currentstroke}{rgb}{0.500000,0.500000,0.500000}%
\pgfsetstrokecolor{currentstroke}%
\pgfsetstrokeopacity{0.300000}%
\pgfsetdash{}{0pt}%
\pgfpathmoveto{\pgfqpoint{4.144684in}{2.823605in}}%
\pgfpathlineto{\pgfqpoint{4.144684in}{2.823605in}}%
\pgfpathlineto{\pgfqpoint{4.175936in}{2.774688in}}%
\pgfpathlineto{\pgfqpoint{4.207639in}{2.725858in}}%
\pgfpathlineto{\pgfqpoint{4.240028in}{2.677164in}}%
\pgfpathlineto{\pgfqpoint{4.273462in}{2.628681in}}%
\pgfpathlineto{\pgfqpoint{4.308572in}{2.580555in}}%
\pgfpathlineto{\pgfqpoint{4.346504in}{2.533093in}}%
\pgfpathlineto{\pgfqpoint{4.389691in}{2.487057in}}%
\pgfpathlineto{\pgfqpoint{4.444929in}{2.445490in}}%
\pgfpathlineto{\pgfqpoint{4.444929in}{2.445490in}}%
\pgfpathlineto{\pgfqpoint{4.479527in}{2.431800in}}%
\pgfpathlineto{\pgfqpoint{4.479527in}{2.431800in}}%
\pgfpathlineto{\pgfqpoint{4.512573in}{2.428784in}}%
\pgfpathlineto{\pgfqpoint{4.544484in}{2.434926in}}%
\pgfpathlineto{\pgfqpoint{4.573576in}{2.447518in}}%
\pgfusepath{stroke}%
\end{pgfscope}%
\begin{pgfscope}%
\pgfpathrectangle{\pgfqpoint{0.647939in}{0.492442in}}{\pgfqpoint{4.273799in}{2.331163in}}%
\pgfusepath{clip}%
\pgfsetbuttcap%
\pgfsetroundjoin%
\pgfsetlinewidth{0.301125pt}%
\definecolor{currentstroke}{rgb}{0.500000,0.500000,0.500000}%
\pgfsetstrokecolor{currentstroke}%
\pgfsetstrokeopacity{0.300000}%
\pgfsetdash{}{0pt}%
\pgfpathmoveto{\pgfqpoint{4.047552in}{2.823605in}}%
\pgfpathlineto{\pgfqpoint{4.047552in}{2.823605in}}%
\pgfpathlineto{\pgfqpoint{4.076181in}{2.774212in}}%
\pgfpathlineto{\pgfqpoint{4.104630in}{2.724788in}}%
\pgfpathlineto{\pgfqpoint{4.132947in}{2.675341in}}%
\pgfpathlineto{\pgfqpoint{4.161196in}{2.625884in}}%
\pgfpathlineto{\pgfqpoint{4.189438in}{2.576425in}}%
\pgfpathlineto{\pgfqpoint{4.217788in}{2.526986in}}%
\pgfpathlineto{\pgfqpoint{4.246425in}{2.477599in}}%
\pgfpathlineto{\pgfqpoint{4.275598in}{2.428305in}}%
\pgfpathlineto{\pgfqpoint{4.305825in}{2.379207in}}%
\pgfpathlineto{\pgfqpoint{4.338236in}{2.330559in}}%
\pgfpathlineto{\pgfqpoint{4.376043in}{2.283259in}}%
\pgfpathlineto{\pgfqpoint{4.376043in}{2.283259in}}%
\pgfpathlineto{\pgfqpoint{4.406937in}{2.257135in}}%
\pgfpathlineto{\pgfqpoint{4.406937in}{2.257135in}}%
\pgfpathlineto{\pgfqpoint{4.429847in}{2.248271in}}%
\pgfpathlineto{\pgfqpoint{4.429847in}{2.248271in}}%
\pgfpathlineto{\pgfqpoint{4.453394in}{2.249298in}}%
\pgfpathlineto{\pgfqpoint{4.473992in}{2.257307in}}%
\pgfusepath{stroke}%
\end{pgfscope}%
\begin{pgfscope}%
\pgfpathrectangle{\pgfqpoint{0.647939in}{0.492442in}}{\pgfqpoint{4.273799in}{2.331163in}}%
\pgfusepath{clip}%
\pgfsetbuttcap%
\pgfsetroundjoin%
\pgfsetlinewidth{0.301125pt}%
\definecolor{currentstroke}{rgb}{0.500000,0.500000,0.500000}%
\pgfsetstrokecolor{currentstroke}%
\pgfsetstrokeopacity{0.300000}%
\pgfsetdash{}{0pt}%
\pgfpathmoveto{\pgfqpoint{3.950420in}{2.823605in}}%
\pgfpathlineto{\pgfqpoint{3.950420in}{2.823605in}}%
\pgfpathlineto{\pgfqpoint{3.977478in}{2.773949in}}%
\pgfpathlineto{\pgfqpoint{4.004035in}{2.724212in}}%
\pgfpathlineto{\pgfqpoint{4.030076in}{2.674395in}}%
\pgfpathlineto{\pgfqpoint{4.055562in}{2.624492in}}%
\pgfpathlineto{\pgfqpoint{4.080456in}{2.574501in}}%
\pgfpathlineto{\pgfqpoint{4.104696in}{2.524414in}}%
\pgfpathlineto{\pgfqpoint{4.128194in}{2.474223in}}%
\pgfpathlineto{\pgfqpoint{4.150826in}{2.423913in}}%
\pgfpathlineto{\pgfqpoint{4.172400in}{2.373467in}}%
\pgfpathlineto{\pgfqpoint{4.192625in}{2.322856in}}%
\pgfpathlineto{\pgfqpoint{4.211040in}{2.272043in}}%
\pgfpathlineto{\pgfqpoint{4.226876in}{2.220976in}}%
\pgfpathlineto{\pgfqpoint{4.238821in}{2.169605in}}%
\pgfpathlineto{\pgfqpoint{4.244636in}{2.117942in}}%
\pgfpathlineto{\pgfqpoint{4.240927in}{2.066287in}}%
\pgfpathlineto{\pgfqpoint{4.224608in}{2.015453in}}%
\pgfpathlineto{\pgfqpoint{4.195736in}{1.966366in}}%
\pgfpathlineto{\pgfqpoint{4.157244in}{1.919244in}}%
\pgfpathlineto{\pgfqpoint{4.111894in}{1.873921in}}%
\pgfpathlineto{\pgfqpoint{4.061304in}{1.830213in}}%
\pgfusepath{stroke}%
\end{pgfscope}%
\begin{pgfscope}%
\pgfpathrectangle{\pgfqpoint{0.647939in}{0.492442in}}{\pgfqpoint{4.273799in}{2.331163in}}%
\pgfusepath{clip}%
\pgfsetbuttcap%
\pgfsetroundjoin%
\pgfsetlinewidth{0.301125pt}%
\definecolor{currentstroke}{rgb}{0.500000,0.500000,0.500000}%
\pgfsetstrokecolor{currentstroke}%
\pgfsetstrokeopacity{0.300000}%
\pgfsetdash{}{0pt}%
\pgfpathmoveto{\pgfqpoint{3.853289in}{2.823605in}}%
\pgfpathlineto{\pgfqpoint{3.853289in}{2.823605in}}%
\pgfpathlineto{\pgfqpoint{3.879501in}{2.773814in}}%
\pgfpathlineto{\pgfqpoint{3.905019in}{2.723916in}}%
\pgfpathlineto{\pgfqpoint{3.929794in}{2.673907in}}%
\pgfpathlineto{\pgfqpoint{3.953772in}{2.623782in}}%
\pgfpathlineto{\pgfqpoint{3.976881in}{2.573537in}}%
\pgfpathlineto{\pgfqpoint{3.999023in}{2.523162in}}%
\pgfpathlineto{\pgfqpoint{4.020071in}{2.472648in}}%
\pgfpathlineto{\pgfqpoint{4.039854in}{2.421982in}}%
\pgfpathlineto{\pgfqpoint{4.058143in}{2.371151in}}%
\pgfpathlineto{\pgfqpoint{4.074633in}{2.320138in}}%
\pgfpathlineto{\pgfqpoint{4.088917in}{2.268930in}}%
\pgfpathlineto{\pgfqpoint{4.100411in}{2.217516in}}%
\pgfpathlineto{\pgfqpoint{4.108351in}{2.165906in}}%
\pgfpathlineto{\pgfqpoint{4.111727in}{2.114155in}}%
\pgfpathlineto{\pgfqpoint{4.109285in}{2.062404in}}%
\pgfpathlineto{\pgfqpoint{4.099685in}{2.010922in}}%
\pgfpathlineto{\pgfqpoint{4.081789in}{1.960125in}}%
\pgfpathlineto{\pgfqpoint{4.055095in}{1.910500in}}%
\pgfpathlineto{\pgfqpoint{4.019781in}{1.862506in}}%
\pgfpathlineto{\pgfqpoint{3.976484in}{1.816499in}}%
\pgfpathlineto{\pgfqpoint{3.925893in}{1.772761in}}%
\pgfpathlineto{\pgfqpoint{3.868357in}{1.731674in}}%
\pgfpathlineto{\pgfqpoint{3.803947in}{1.693738in}}%
\pgfusepath{stroke}%
\end{pgfscope}%
\begin{pgfscope}%
\pgfpathrectangle{\pgfqpoint{0.647939in}{0.492442in}}{\pgfqpoint{4.273799in}{2.331163in}}%
\pgfusepath{clip}%
\pgfsetbuttcap%
\pgfsetroundjoin%
\pgfsetlinewidth{0.301125pt}%
\definecolor{currentstroke}{rgb}{0.500000,0.500000,0.500000}%
\pgfsetstrokecolor{currentstroke}%
\pgfsetstrokeopacity{0.300000}%
\pgfsetdash{}{0pt}%
\pgfpathmoveto{\pgfqpoint{3.756157in}{2.823605in}}%
\pgfpathlineto{\pgfqpoint{3.756157in}{2.823605in}}%
\pgfpathlineto{\pgfqpoint{3.782070in}{2.773768in}}%
\pgfpathlineto{\pgfqpoint{3.807151in}{2.723804in}}%
\pgfpathlineto{\pgfqpoint{3.831345in}{2.673710in}}%
\pgfpathlineto{\pgfqpoint{3.854590in}{2.623483in}}%
\pgfpathlineto{\pgfqpoint{3.876803in}{2.573118in}}%
\pgfpathlineto{\pgfqpoint{3.897881in}{2.522607in}}%
\pgfpathlineto{\pgfqpoint{3.917696in}{2.471945in}}%
\pgfpathlineto{\pgfqpoint{3.936081in}{2.421123in}}%
\pgfpathlineto{\pgfqpoint{3.952826in}{2.370134in}}%
\pgfpathlineto{\pgfqpoint{3.967659in}{2.318970in}}%
\pgfpathlineto{\pgfqpoint{3.980236in}{2.267628in}}%
\pgfpathlineto{\pgfqpoint{3.990121in}{2.216113in}}%
\pgfpathlineto{\pgfqpoint{3.996755in}{2.164446in}}%
\pgfpathlineto{\pgfqpoint{3.999444in}{2.112678in}}%
\pgfpathlineto{\pgfqpoint{3.997357in}{2.060910in}}%
\pgfpathlineto{\pgfqpoint{3.989559in}{2.009315in}}%
\pgfpathlineto{\pgfqpoint{3.975098in}{1.958162in}}%
\pgfpathlineto{\pgfqpoint{3.953119in}{1.907823in}}%
\pgfpathlineto{\pgfqpoint{3.923013in}{1.858758in}}%
\pgfpathlineto{\pgfqpoint{3.884443in}{1.811498in}}%
\pgfpathlineto{\pgfqpoint{3.837289in}{1.766625in}}%
\pgfusepath{stroke}%
\end{pgfscope}%
\begin{pgfscope}%
\pgfpathrectangle{\pgfqpoint{0.647939in}{0.492442in}}{\pgfqpoint{4.273799in}{2.331163in}}%
\pgfusepath{clip}%
\pgfsetbuttcap%
\pgfsetroundjoin%
\pgfsetlinewidth{0.301125pt}%
\definecolor{currentstroke}{rgb}{0.500000,0.500000,0.500000}%
\pgfsetstrokecolor{currentstroke}%
\pgfsetstrokeopacity{0.300000}%
\pgfsetdash{}{0pt}%
\pgfpathmoveto{\pgfqpoint{3.659025in}{2.823605in}}%
\pgfpathlineto{\pgfqpoint{3.659025in}{2.823605in}}%
\pgfpathlineto{\pgfqpoint{3.685095in}{2.773792in}}%
\pgfpathlineto{\pgfqpoint{3.710215in}{2.723834in}}%
\pgfpathlineto{\pgfqpoint{3.734336in}{2.673730in}}%
\pgfpathlineto{\pgfqpoint{3.757392in}{2.623478in}}%
\pgfpathlineto{\pgfqpoint{3.779302in}{2.573073in}}%
\pgfpathlineto{\pgfqpoint{3.799965in}{2.522512in}}%
\pgfpathlineto{\pgfqpoint{3.819257in}{2.471790in}}%
\pgfpathlineto{\pgfqpoint{3.837023in}{2.420903in}}%
\pgfpathlineto{\pgfqpoint{3.853067in}{2.369846in}}%
\pgfpathlineto{\pgfqpoint{3.867151in}{2.318619in}}%
\pgfpathlineto{\pgfqpoint{3.878970in}{2.267224in}}%
\pgfpathlineto{\pgfqpoint{3.888151in}{2.215670in}}%
\pgfpathlineto{\pgfqpoint{3.894235in}{2.163981in}}%
\pgfpathlineto{\pgfqpoint{3.896658in}{2.112206in}}%
\pgfpathlineto{\pgfqpoint{3.894743in}{2.060432in}}%
\pgfpathlineto{\pgfqpoint{3.887703in}{2.008798in}}%
\pgfpathlineto{\pgfqpoint{3.874662in}{1.957522in}}%
\pgfpathlineto{\pgfqpoint{3.854678in}{1.906925in}}%
\pgfpathlineto{\pgfqpoint{3.826851in}{1.857457in}}%
\pgfpathlineto{\pgfqpoint{3.790343in}{1.809719in}}%
\pgfpathlineto{\pgfqpoint{3.744376in}{1.764502in}}%
\pgfpathlineto{\pgfqpoint{3.688229in}{1.722900in}}%
\pgfpathlineto{\pgfqpoint{3.621377in}{1.686421in}}%
\pgfpathlineto{\pgfqpoint{3.543610in}{1.657281in}}%
\pgfpathlineto{\pgfqpoint{3.459538in}{1.638803in}}%
\pgfpathlineto{\pgfqpoint{3.378386in}{1.632230in}}%
\pgfpathlineto{\pgfqpoint{3.301319in}{1.635625in}}%
\pgfpathlineto{\pgfqpoint{3.227100in}{1.647620in}}%
\pgfpathlineto{\pgfqpoint{3.154173in}{1.667872in}}%
\pgfpathlineto{\pgfqpoint{3.081781in}{1.696681in}}%
\pgfpathlineto{\pgfqpoint{3.013521in}{1.732433in}}%
\pgfpathlineto{\pgfqpoint{2.952795in}{1.772111in}}%
\pgfpathlineto{\pgfqpoint{2.898914in}{1.814670in}}%
\pgfpathlineto{\pgfqpoint{2.851361in}{1.859450in}}%
\pgfpathlineto{\pgfqpoint{2.809919in}{1.906017in}}%
\pgfpathlineto{\pgfqpoint{2.774713in}{1.954077in}}%
\pgfpathlineto{\pgfqpoint{2.746358in}{2.003452in}}%
\pgfpathlineto{\pgfqpoint{2.726363in}{2.054006in}}%
\pgfpathlineto{\pgfqpoint{2.718312in}{2.105429in}}%
\pgfpathlineto{\pgfqpoint{2.718312in}{2.105429in}}%
\pgfpathlineto{\pgfqpoint{2.726046in}{2.144581in}}%
\pgfpathlineto{\pgfqpoint{2.726046in}{2.144581in}}%
\pgfpathlineto{\pgfqpoint{2.743014in}{2.166485in}}%
\pgfpathlineto{\pgfqpoint{2.743014in}{2.166485in}}%
\pgfpathlineto{\pgfqpoint{2.765969in}{2.177081in}}%
\pgfpathlineto{\pgfqpoint{2.794661in}{2.177351in}}%
\pgfpathlineto{\pgfqpoint{2.818007in}{2.170558in}}%
\pgfusepath{stroke}%
\end{pgfscope}%
\begin{pgfscope}%
\pgfpathrectangle{\pgfqpoint{0.647939in}{0.492442in}}{\pgfqpoint{4.273799in}{2.331163in}}%
\pgfusepath{clip}%
\pgfsetbuttcap%
\pgfsetroundjoin%
\pgfsetlinewidth{0.301125pt}%
\definecolor{currentstroke}{rgb}{0.500000,0.500000,0.500000}%
\pgfsetstrokecolor{currentstroke}%
\pgfsetstrokeopacity{0.300000}%
\pgfsetdash{}{0pt}%
\pgfpathmoveto{\pgfqpoint{3.561893in}{2.823605in}}%
\pgfpathlineto{\pgfqpoint{3.561893in}{2.823605in}}%
\pgfpathlineto{\pgfqpoint{3.588521in}{2.773880in}}%
\pgfpathlineto{\pgfqpoint{3.614097in}{2.723992in}}%
\pgfpathlineto{\pgfqpoint{3.638576in}{2.673940in}}%
\pgfpathlineto{\pgfqpoint{3.661889in}{2.623722in}}%
\pgfpathlineto{\pgfqpoint{3.683960in}{2.573338in}}%
\pgfpathlineto{\pgfqpoint{3.704692in}{2.522786in}}%
\pgfpathlineto{\pgfqpoint{3.723963in}{2.472062in}}%
\pgfpathlineto{\pgfqpoint{3.741629in}{2.421165in}}%
\pgfpathlineto{\pgfqpoint{3.757503in}{2.370093in}}%
\pgfpathlineto{\pgfqpoint{3.771365in}{2.318848in}}%
\pgfpathlineto{\pgfqpoint{3.782943in}{2.267436in}}%
\pgfpathlineto{\pgfqpoint{3.791893in}{2.215869in}}%
\pgfpathlineto{\pgfqpoint{3.797797in}{2.164174in}}%
\pgfpathlineto{\pgfqpoint{3.800150in}{2.112398in}}%
\pgfpathlineto{\pgfqpoint{3.798333in}{2.060621in}}%
\pgfpathlineto{\pgfqpoint{3.791605in}{2.008972in}}%
\pgfpathlineto{\pgfqpoint{3.779086in}{1.957654in}}%
\pgfpathlineto{\pgfqpoint{3.759776in}{1.906978in}}%
\pgfpathlineto{\pgfqpoint{3.732541in}{1.857414in}}%
\pgfpathlineto{\pgfqpoint{3.696131in}{1.809665in}}%
\pgfpathlineto{\pgfqpoint{3.649207in}{1.764776in}}%
\pgfpathlineto{\pgfqpoint{3.590429in}{1.724356in}}%
\pgfpathlineto{\pgfqpoint{3.518851in}{1.690876in}}%
\pgfusepath{stroke}%
\end{pgfscope}%
\begin{pgfscope}%
\pgfpathrectangle{\pgfqpoint{0.647939in}{0.492442in}}{\pgfqpoint{4.273799in}{2.331163in}}%
\pgfusepath{clip}%
\pgfsetbuttcap%
\pgfsetroundjoin%
\pgfsetlinewidth{0.301125pt}%
\definecolor{currentstroke}{rgb}{0.500000,0.500000,0.500000}%
\pgfsetstrokecolor{currentstroke}%
\pgfsetstrokeopacity{0.300000}%
\pgfsetdash{}{0pt}%
\pgfpathmoveto{\pgfqpoint{3.464761in}{2.823605in}}%
\pgfpathlineto{\pgfqpoint{3.464761in}{2.823605in}}%
\pgfpathlineto{\pgfqpoint{3.492339in}{2.774035in}}%
\pgfpathlineto{\pgfqpoint{3.518765in}{2.724278in}}%
\pgfpathlineto{\pgfqpoint{3.543991in}{2.674337in}}%
\pgfpathlineto{\pgfqpoint{3.567954in}{2.624211in}}%
\pgfpathlineto{\pgfqpoint{3.590581in}{2.573901in}}%
\pgfusepath{stroke}%
\end{pgfscope}%
\begin{pgfscope}%
\pgfpathrectangle{\pgfqpoint{0.647939in}{0.492442in}}{\pgfqpoint{4.273799in}{2.331163in}}%
\pgfusepath{clip}%
\pgfsetbuttcap%
\pgfsetroundjoin%
\pgfsetlinewidth{0.301125pt}%
\definecolor{currentstroke}{rgb}{0.500000,0.500000,0.500000}%
\pgfsetstrokecolor{currentstroke}%
\pgfsetstrokeopacity{0.300000}%
\pgfsetdash{}{0pt}%
\pgfpathmoveto{\pgfqpoint{3.367630in}{2.823605in}}%
\pgfpathlineto{\pgfqpoint{3.367630in}{2.823605in}}%
\pgfpathlineto{\pgfqpoint{3.396563in}{2.774266in}}%
\pgfpathlineto{\pgfqpoint{3.424235in}{2.724711in}}%
\pgfpathlineto{\pgfqpoint{3.450595in}{2.674945in}}%
\pgfpathlineto{\pgfqpoint{3.475590in}{2.624970in}}%
\pgfpathlineto{\pgfqpoint{3.499146in}{2.574787in}}%
\pgfpathlineto{\pgfqpoint{3.521174in}{2.524398in}}%
\pgfpathlineto{\pgfqpoint{3.541563in}{2.473805in}}%
\pgfpathlineto{\pgfqpoint{3.560174in}{2.423008in}}%
\pgfpathlineto{\pgfqpoint{3.576841in}{2.372012in}}%
\pgfpathlineto{\pgfqpoint{3.591350in}{2.320821in}}%
\pgfpathlineto{\pgfqpoint{3.603445in}{2.269445in}}%
\pgfpathlineto{\pgfqpoint{3.612805in}{2.217901in}}%
\pgfpathlineto{\pgfqpoint{3.619026in}{2.166218in}}%
\pgfpathlineto{\pgfqpoint{3.621606in}{2.114445in}}%
\pgfpathlineto{\pgfqpoint{3.619912in}{2.062667in}}%
\pgfpathlineto{\pgfqpoint{3.613133in}{2.011022in}}%
\pgfpathlineto{\pgfqpoint{3.600228in}{1.959739in}}%
\pgfpathlineto{\pgfqpoint{3.579853in}{1.909200in}}%
\pgfpathlineto{\pgfqpoint{3.550204in}{1.860081in}}%
\pgfpathlineto{\pgfqpoint{3.508920in}{1.813604in}}%
\pgfpathlineto{\pgfqpoint{3.453111in}{1.772086in}}%
\pgfpathlineto{\pgfqpoint{3.453111in}{1.772086in}}%
\pgfpathlineto{\pgfqpoint{3.395708in}{1.744878in}}%
\pgfpathlineto{\pgfqpoint{3.326629in}{1.726869in}}%
\pgfpathlineto{\pgfqpoint{3.259911in}{1.721352in}}%
\pgfpathlineto{\pgfqpoint{3.196647in}{1.725439in}}%
\pgfpathlineto{\pgfqpoint{3.134452in}{1.737706in}}%
\pgfpathlineto{\pgfqpoint{3.072254in}{1.758025in}}%
\pgfusepath{stroke}%
\end{pgfscope}%
\begin{pgfscope}%
\pgfpathrectangle{\pgfqpoint{0.647939in}{0.492442in}}{\pgfqpoint{4.273799in}{2.331163in}}%
\pgfusepath{clip}%
\pgfsetbuttcap%
\pgfsetroundjoin%
\pgfsetlinewidth{0.301125pt}%
\definecolor{currentstroke}{rgb}{0.500000,0.500000,0.500000}%
\pgfsetstrokecolor{currentstroke}%
\pgfsetstrokeopacity{0.300000}%
\pgfsetdash{}{0pt}%
\pgfpathmoveto{\pgfqpoint{3.270498in}{2.823605in}}%
\pgfpathlineto{\pgfqpoint{3.270498in}{2.823605in}}%
\pgfpathlineto{\pgfqpoint{3.301224in}{2.774590in}}%
\pgfpathlineto{\pgfqpoint{3.330559in}{2.725322in}}%
\pgfpathlineto{\pgfqpoint{3.358461in}{2.675806in}}%
\pgfpathlineto{\pgfqpoint{3.384881in}{2.626050in}}%
\pgfpathlineto{\pgfqpoint{3.409747in}{2.576056in}}%
\pgfpathlineto{\pgfqpoint{3.432975in}{2.525828in}}%
\pgfpathlineto{\pgfqpoint{3.454457in}{2.475370in}}%
\pgfpathlineto{\pgfqpoint{3.474057in}{2.424685in}}%
\pgfpathlineto{\pgfqpoint{3.491611in}{2.373778in}}%
\pgfpathlineto{\pgfqpoint{3.506906in}{2.322655in}}%
\pgfusepath{stroke}%
\end{pgfscope}%
\begin{pgfscope}%
\pgfpathrectangle{\pgfqpoint{0.647939in}{0.492442in}}{\pgfqpoint{4.273799in}{2.331163in}}%
\pgfusepath{clip}%
\pgfsetbuttcap%
\pgfsetroundjoin%
\pgfsetlinewidth{0.301125pt}%
\definecolor{currentstroke}{rgb}{0.500000,0.500000,0.500000}%
\pgfsetstrokecolor{currentstroke}%
\pgfsetstrokeopacity{0.300000}%
\pgfsetdash{}{0pt}%
\pgfpathmoveto{\pgfqpoint{3.076234in}{2.823605in}}%
\pgfpathlineto{\pgfqpoint{3.076234in}{2.823605in}}%
\pgfpathlineto{\pgfqpoint{3.112034in}{2.775627in}}%
\pgfpathlineto{\pgfqpoint{3.146124in}{2.727278in}}%
\pgfpathlineto{\pgfqpoint{3.178473in}{2.678576in}}%
\pgfpathlineto{\pgfqpoint{3.209042in}{2.629534in}}%
\pgfpathlineto{\pgfqpoint{3.237768in}{2.580160in}}%
\pgfpathlineto{\pgfqpoint{3.264579in}{2.530468in}}%
\pgfpathlineto{\pgfqpoint{3.289377in}{2.480466in}}%
\pgfpathlineto{\pgfqpoint{3.312031in}{2.430162in}}%
\pgfpathlineto{\pgfqpoint{3.332374in}{2.379566in}}%
\pgfpathlineto{\pgfqpoint{3.350194in}{2.328689in}}%
\pgfpathlineto{\pgfqpoint{3.365210in}{2.277544in}}%
\pgfpathlineto{\pgfqpoint{3.377065in}{2.226155in}}%
\pgfpathlineto{\pgfqpoint{3.385283in}{2.174559in}}%
\pgfpathlineto{\pgfqpoint{3.389223in}{2.122819in}}%
\pgfpathlineto{\pgfqpoint{3.387996in}{2.071044in}}%
\pgfpathlineto{\pgfqpoint{3.380332in}{2.019450in}}%
\pgfpathlineto{\pgfqpoint{3.364329in}{1.968465in}}%
\pgfpathlineto{\pgfqpoint{3.336941in}{1.919034in}}%
\pgfpathlineto{\pgfqpoint{3.293176in}{1.873532in}}%
\pgfpathlineto{\pgfqpoint{3.293176in}{1.873532in}}%
\pgfpathlineto{\pgfqpoint{3.248693in}{1.846771in}}%
\pgfpathlineto{\pgfqpoint{3.248693in}{1.846771in}}%
\pgfpathlineto{\pgfqpoint{3.200746in}{1.831161in}}%
\pgfpathlineto{\pgfqpoint{3.145587in}{1.825404in}}%
\pgfpathlineto{\pgfqpoint{3.094484in}{1.829356in}}%
\pgfpathlineto{\pgfqpoint{3.044370in}{1.840970in}}%
\pgfpathlineto{\pgfqpoint{2.993074in}{1.860536in}}%
\pgfpathlineto{\pgfqpoint{2.940403in}{1.889081in}}%
\pgfpathlineto{\pgfqpoint{2.887813in}{1.927519in}}%
\pgfpathlineto{\pgfqpoint{2.841511in}{1.972554in}}%
\pgfusepath{stroke}%
\end{pgfscope}%
\begin{pgfscope}%
\pgfpathrectangle{\pgfqpoint{0.647939in}{0.492442in}}{\pgfqpoint{4.273799in}{2.331163in}}%
\pgfusepath{clip}%
\pgfsetbuttcap%
\pgfsetroundjoin%
\pgfsetlinewidth{0.301125pt}%
\definecolor{currentstroke}{rgb}{0.500000,0.500000,0.500000}%
\pgfsetstrokecolor{currentstroke}%
\pgfsetstrokeopacity{0.300000}%
\pgfsetdash{}{0pt}%
\pgfpathmoveto{\pgfqpoint{2.979102in}{2.823605in}}%
\pgfpathlineto{\pgfqpoint{2.979102in}{2.823605in}}%
\pgfpathlineto{\pgfqpoint{3.018301in}{2.776424in}}%
\pgfpathlineto{\pgfqpoint{3.055587in}{2.728784in}}%
\pgfpathlineto{\pgfqpoint{3.090935in}{2.680708in}}%
\pgfpathlineto{\pgfqpoint{3.124311in}{2.632214in}}%
\pgfpathlineto{\pgfqpoint{3.155663in}{2.583320in}}%
\pgfpathlineto{\pgfqpoint{3.184923in}{2.534041in}}%
\pgfpathlineto{\pgfqpoint{3.212000in}{2.484393in}}%
\pgfusepath{stroke}%
\end{pgfscope}%
\begin{pgfscope}%
\pgfpathrectangle{\pgfqpoint{0.647939in}{0.492442in}}{\pgfqpoint{4.273799in}{2.331163in}}%
\pgfusepath{clip}%
\pgfsetbuttcap%
\pgfsetroundjoin%
\pgfsetlinewidth{0.301125pt}%
\definecolor{currentstroke}{rgb}{0.500000,0.500000,0.500000}%
\pgfsetstrokecolor{currentstroke}%
\pgfsetstrokeopacity{0.300000}%
\pgfsetdash{}{0pt}%
\pgfpathmoveto{\pgfqpoint{2.784839in}{2.823605in}}%
\pgfpathlineto{\pgfqpoint{2.784839in}{2.823605in}}%
\pgfpathlineto{\pgfqpoint{2.832717in}{2.778878in}}%
\pgfpathlineto{\pgfqpoint{2.878226in}{2.733421in}}%
\pgfpathlineto{\pgfqpoint{2.921324in}{2.687269in}}%
\pgfpathlineto{\pgfqpoint{2.961973in}{2.640461in}}%
\pgfpathlineto{\pgfqpoint{3.000136in}{2.593034in}}%
\pgfpathlineto{\pgfqpoint{3.035761in}{2.545023in}}%
\pgfpathlineto{\pgfqpoint{3.068776in}{2.496459in}}%
\pgfpathlineto{\pgfqpoint{3.099079in}{2.447371in}}%
\pgfpathlineto{\pgfqpoint{3.126525in}{2.397788in}}%
\pgfpathlineto{\pgfqpoint{3.150908in}{2.347730in}}%
\pgfpathlineto{\pgfqpoint{3.171943in}{2.297225in}}%
\pgfpathlineto{\pgfqpoint{3.189228in}{2.246302in}}%
\pgfpathlineto{\pgfqpoint{3.202187in}{2.195002in}}%
\pgfpathlineto{\pgfqpoint{3.209969in}{2.143400in}}%
\pgfpathlineto{\pgfqpoint{3.211252in}{2.091642in}}%
\pgfpathlineto{\pgfqpoint{3.203822in}{2.040082in}}%
\pgfpathlineto{\pgfqpoint{3.183557in}{1.989692in}}%
\pgfpathlineto{\pgfqpoint{3.183557in}{1.989692in}}%
\pgfpathlineto{\pgfqpoint{3.153202in}{1.952702in}}%
\pgfpathlineto{\pgfqpoint{3.153202in}{1.952702in}}%
\pgfpathlineto{\pgfqpoint{3.118060in}{1.930120in}}%
\pgfpathlineto{\pgfqpoint{3.118060in}{1.930120in}}%
\pgfpathlineto{\pgfqpoint{3.079594in}{1.918672in}}%
\pgfpathlineto{\pgfqpoint{3.035933in}{1.916952in}}%
\pgfpathlineto{\pgfqpoint{2.995862in}{1.923583in}}%
\pgfpathlineto{\pgfqpoint{2.955653in}{1.937400in}}%
\pgfusepath{stroke}%
\end{pgfscope}%
\begin{pgfscope}%
\pgfpathrectangle{\pgfqpoint{0.647939in}{0.492442in}}{\pgfqpoint{4.273799in}{2.331163in}}%
\pgfusepath{clip}%
\pgfsetbuttcap%
\pgfsetroundjoin%
\pgfsetlinewidth{0.301125pt}%
\definecolor{currentstroke}{rgb}{0.500000,0.500000,0.500000}%
\pgfsetstrokecolor{currentstroke}%
\pgfsetstrokeopacity{0.300000}%
\pgfsetdash{}{0pt}%
\pgfpathmoveto{\pgfqpoint{2.590575in}{2.823605in}}%
\pgfpathlineto{\pgfqpoint{2.590575in}{2.823605in}}%
\pgfpathlineto{\pgfqpoint{2.649127in}{2.782833in}}%
\pgfpathlineto{\pgfqpoint{2.705030in}{2.740974in}}%
\pgfpathlineto{\pgfqpoint{2.758109in}{2.698034in}}%
\pgfpathlineto{\pgfqpoint{2.808235in}{2.654049in}}%
\pgfpathlineto{\pgfqpoint{2.855322in}{2.609075in}}%
\pgfpathlineto{\pgfqpoint{2.899308in}{2.563177in}}%
\pgfpathlineto{\pgfqpoint{2.940141in}{2.516420in}}%
\pgfpathlineto{\pgfqpoint{2.977753in}{2.468864in}}%
\pgfpathlineto{\pgfqpoint{3.012040in}{2.420567in}}%
\pgfpathlineto{\pgfqpoint{3.042842in}{2.371581in}}%
\pgfpathlineto{\pgfqpoint{3.069922in}{2.321945in}}%
\pgfpathlineto{\pgfqpoint{3.092917in}{2.271702in}}%
\pgfpathlineto{\pgfqpoint{3.111275in}{2.220899in}}%
\pgfpathlineto{\pgfqpoint{3.124137in}{2.169605in}}%
\pgfusepath{stroke}%
\end{pgfscope}%
\begin{pgfscope}%
\pgfpathrectangle{\pgfqpoint{0.647939in}{0.492442in}}{\pgfqpoint{4.273799in}{2.331163in}}%
\pgfusepath{clip}%
\pgfsetbuttcap%
\pgfsetroundjoin%
\pgfsetlinewidth{0.301125pt}%
\definecolor{currentstroke}{rgb}{0.500000,0.500000,0.500000}%
\pgfsetstrokecolor{currentstroke}%
\pgfsetstrokeopacity{0.300000}%
\pgfsetdash{}{0pt}%
\pgfpathmoveto{\pgfqpoint{2.396312in}{2.823605in}}%
\pgfpathlineto{\pgfqpoint{2.396312in}{2.823605in}}%
\pgfpathlineto{\pgfqpoint{2.464978in}{2.787829in}}%
\pgfpathlineto{\pgfqpoint{2.531714in}{2.750991in}}%
\pgfpathlineto{\pgfqpoint{2.595871in}{2.712822in}}%
\pgfpathlineto{\pgfqpoint{2.656969in}{2.673193in}}%
\pgfpathlineto{\pgfqpoint{2.714676in}{2.632079in}}%
\pgfpathlineto{\pgfqpoint{2.768782in}{2.589532in}}%
\pgfpathlineto{\pgfqpoint{2.819157in}{2.545645in}}%
\pgfpathlineto{\pgfqpoint{2.865724in}{2.500526in}}%
\pgfpathlineto{\pgfqpoint{2.908424in}{2.454282in}}%
\pgfusepath{stroke}%
\end{pgfscope}%
\begin{pgfscope}%
\pgfpathrectangle{\pgfqpoint{0.647939in}{0.492442in}}{\pgfqpoint{4.273799in}{2.331163in}}%
\pgfusepath{clip}%
\pgfsetbuttcap%
\pgfsetroundjoin%
\pgfsetlinewidth{0.301125pt}%
\definecolor{currentstroke}{rgb}{0.500000,0.500000,0.500000}%
\pgfsetstrokecolor{currentstroke}%
\pgfsetstrokeopacity{0.300000}%
\pgfsetdash{}{0pt}%
\pgfpathmoveto{\pgfqpoint{2.202048in}{2.823605in}}%
\pgfpathlineto{\pgfqpoint{2.202048in}{2.823605in}}%
\pgfpathlineto{\pgfqpoint{2.276115in}{2.791186in}}%
\pgfpathlineto{\pgfqpoint{2.350602in}{2.759056in}}%
\pgfpathlineto{\pgfqpoint{2.424335in}{2.726420in}}%
\pgfpathlineto{\pgfqpoint{2.496203in}{2.692583in}}%
\pgfpathlineto{\pgfqpoint{2.565245in}{2.657057in}}%
\pgfpathlineto{\pgfqpoint{2.630782in}{2.619608in}}%
\pgfpathlineto{\pgfqpoint{2.692309in}{2.580195in}}%
\pgfusepath{stroke}%
\end{pgfscope}%
\begin{pgfscope}%
\pgfpathrectangle{\pgfqpoint{0.647939in}{0.492442in}}{\pgfqpoint{4.273799in}{2.331163in}}%
\pgfusepath{clip}%
\pgfsetbuttcap%
\pgfsetroundjoin%
\pgfsetlinewidth{0.301125pt}%
\definecolor{currentstroke}{rgb}{0.500000,0.500000,0.500000}%
\pgfsetstrokecolor{currentstroke}%
\pgfsetstrokeopacity{0.300000}%
\pgfsetdash{}{0pt}%
\pgfpathmoveto{\pgfqpoint{2.007784in}{2.823605in}}%
\pgfpathlineto{\pgfqpoint{2.007784in}{2.823605in}}%
\pgfpathlineto{\pgfqpoint{2.078822in}{2.789283in}}%
\pgfpathlineto{\pgfqpoint{2.153990in}{2.757667in}}%
\pgfpathlineto{\pgfqpoint{2.232014in}{2.728154in}}%
\pgfpathlineto{\pgfqpoint{2.311326in}{2.699668in}}%
\pgfpathlineto{\pgfqpoint{2.390328in}{2.670945in}}%
\pgfpathlineto{\pgfqpoint{2.467548in}{2.640840in}}%
\pgfpathlineto{\pgfqpoint{2.541718in}{2.608548in}}%
\pgfpathlineto{\pgfqpoint{2.611795in}{2.573671in}}%
\pgfusepath{stroke}%
\end{pgfscope}%
\begin{pgfscope}%
\pgfpathrectangle{\pgfqpoint{0.647939in}{0.492442in}}{\pgfqpoint{4.273799in}{2.331163in}}%
\pgfusepath{clip}%
\pgfsetbuttcap%
\pgfsetroundjoin%
\pgfsetlinewidth{0.301125pt}%
\definecolor{currentstroke}{rgb}{0.500000,0.500000,0.500000}%
\pgfsetstrokecolor{currentstroke}%
\pgfsetstrokeopacity{0.300000}%
\pgfsetdash{}{0pt}%
\pgfpathmoveto{\pgfqpoint{1.813521in}{2.823605in}}%
\pgfpathlineto{\pgfqpoint{1.813521in}{2.823605in}}%
\pgfpathlineto{\pgfqpoint{1.871557in}{2.782652in}}%
\pgfpathlineto{\pgfqpoint{1.935774in}{2.744580in}}%
\pgfpathlineto{\pgfqpoint{2.006804in}{2.710325in}}%
\pgfpathlineto{\pgfqpoint{2.084339in}{2.680569in}}%
\pgfpathlineto{\pgfqpoint{2.166937in}{2.655122in}}%
\pgfpathlineto{\pgfqpoint{2.252389in}{2.632564in}}%
\pgfpathlineto{\pgfqpoint{2.338463in}{2.610692in}}%
\pgfpathlineto{\pgfqpoint{2.423164in}{2.587332in}}%
\pgfpathlineto{\pgfqpoint{2.504738in}{2.560926in}}%
\pgfpathlineto{\pgfqpoint{2.581685in}{2.530713in}}%
\pgfpathlineto{\pgfqpoint{2.652911in}{2.496623in}}%
\pgfpathlineto{\pgfqpoint{2.717955in}{2.459005in}}%
\pgfpathlineto{\pgfqpoint{2.776596in}{2.418362in}}%
\pgfpathlineto{\pgfqpoint{2.828820in}{2.375195in}}%
\pgfpathlineto{\pgfqpoint{2.874673in}{2.329918in}}%
\pgfusepath{stroke}%
\end{pgfscope}%
\begin{pgfscope}%
\pgfpathrectangle{\pgfqpoint{0.647939in}{0.492442in}}{\pgfqpoint{4.273799in}{2.331163in}}%
\pgfusepath{clip}%
\pgfsetbuttcap%
\pgfsetroundjoin%
\pgfsetlinewidth{0.301125pt}%
\definecolor{currentstroke}{rgb}{0.500000,0.500000,0.500000}%
\pgfsetstrokecolor{currentstroke}%
\pgfsetstrokeopacity{0.300000}%
\pgfsetdash{}{0pt}%
\pgfpathmoveto{\pgfqpoint{1.716389in}{2.823605in}}%
\pgfpathlineto{\pgfqpoint{1.716389in}{2.823605in}}%
\pgfpathlineto{\pgfqpoint{1.765562in}{2.779320in}}%
\pgfpathlineto{\pgfqpoint{1.819979in}{2.736912in}}%
\pgfpathlineto{\pgfqpoint{1.880960in}{2.697301in}}%
\pgfpathlineto{\pgfqpoint{1.949930in}{2.661866in}}%
\pgfpathlineto{\pgfqpoint{2.027392in}{2.632180in}}%
\pgfpathlineto{\pgfqpoint{2.112006in}{2.608974in}}%
\pgfpathlineto{\pgfqpoint{2.200918in}{2.590929in}}%
\pgfpathlineto{\pgfqpoint{2.291272in}{2.575044in}}%
\pgfpathlineto{\pgfqpoint{2.380866in}{2.558025in}}%
\pgfusepath{stroke}%
\end{pgfscope}%
\begin{pgfscope}%
\pgfpathrectangle{\pgfqpoint{0.647939in}{0.492442in}}{\pgfqpoint{4.273799in}{2.331163in}}%
\pgfusepath{clip}%
\pgfsetbuttcap%
\pgfsetroundjoin%
\pgfsetlinewidth{0.301125pt}%
\definecolor{currentstroke}{rgb}{0.500000,0.500000,0.500000}%
\pgfsetstrokecolor{currentstroke}%
\pgfsetstrokeopacity{0.300000}%
\pgfsetdash{}{0pt}%
\pgfpathmoveto{\pgfqpoint{1.619257in}{2.823605in}}%
\pgfpathlineto{\pgfqpoint{1.619257in}{2.823605in}}%
\pgfpathlineto{\pgfqpoint{1.659724in}{2.776759in}}%
\pgfpathlineto{\pgfqpoint{1.703609in}{2.730848in}}%
\pgfpathlineto{\pgfqpoint{1.752013in}{2.686328in}}%
\pgfpathlineto{\pgfqpoint{1.806569in}{2.644014in}}%
\pgfpathlineto{\pgfqpoint{1.869541in}{2.605414in}}%
\pgfpathlineto{\pgfqpoint{1.943202in}{2.573146in}}%
\pgfpathlineto{\pgfqpoint{2.027703in}{2.550323in}}%
\pgfpathlineto{\pgfqpoint{2.111753in}{2.537963in}}%
\pgfpathlineto{\pgfqpoint{2.205540in}{2.530387in}}%
\pgfpathlineto{\pgfqpoint{2.299623in}{2.523786in}}%
\pgfpathlineto{\pgfqpoint{2.392611in}{2.513887in}}%
\pgfpathlineto{\pgfqpoint{2.482575in}{2.498041in}}%
\pgfpathlineto{\pgfqpoint{2.567285in}{2.475254in}}%
\pgfusepath{stroke}%
\end{pgfscope}%
\begin{pgfscope}%
\pgfpathrectangle{\pgfqpoint{0.647939in}{0.492442in}}{\pgfqpoint{4.273799in}{2.331163in}}%
\pgfusepath{clip}%
\pgfsetbuttcap%
\pgfsetroundjoin%
\pgfsetlinewidth{0.301125pt}%
\definecolor{currentstroke}{rgb}{0.500000,0.500000,0.500000}%
\pgfsetstrokecolor{currentstroke}%
\pgfsetstrokeopacity{0.300000}%
\pgfsetdash{}{0pt}%
\pgfpathmoveto{\pgfqpoint{1.522125in}{2.823605in}}%
\pgfpathlineto{\pgfqpoint{1.522125in}{2.823605in}}%
\pgfpathlineto{\pgfqpoint{1.555003in}{2.775010in}}%
\pgfpathlineto{\pgfqpoint{1.589686in}{2.726791in}}%
\pgfpathlineto{\pgfqpoint{1.626657in}{2.679090in}}%
\pgfpathlineto{\pgfqpoint{1.666680in}{2.632138in}}%
\pgfpathlineto{\pgfqpoint{1.710970in}{2.586363in}}%
\pgfpathlineto{\pgfqpoint{1.761589in}{2.542634in}}%
\pgfpathlineto{\pgfqpoint{1.822173in}{2.502988in}}%
\pgfpathlineto{\pgfqpoint{1.822173in}{2.502988in}}%
\pgfpathlineto{\pgfqpoint{1.882884in}{2.476740in}}%
\pgfpathlineto{\pgfqpoint{1.882884in}{2.476740in}}%
\pgfpathlineto{\pgfqpoint{1.943372in}{2.462530in}}%
\pgfpathlineto{\pgfqpoint{2.008886in}{2.457444in}}%
\pgfpathlineto{\pgfqpoint{2.077362in}{2.459452in}}%
\pgfpathlineto{\pgfqpoint{2.162954in}{2.466678in}}%
\pgfpathlineto{\pgfqpoint{2.256720in}{2.474343in}}%
\pgfusepath{stroke}%
\end{pgfscope}%
\begin{pgfscope}%
\pgfpathrectangle{\pgfqpoint{0.647939in}{0.492442in}}{\pgfqpoint{4.273799in}{2.331163in}}%
\pgfusepath{clip}%
\pgfsetbuttcap%
\pgfsetroundjoin%
\pgfsetlinewidth{0.301125pt}%
\definecolor{currentstroke}{rgb}{0.500000,0.500000,0.500000}%
\pgfsetstrokecolor{currentstroke}%
\pgfsetstrokeopacity{0.300000}%
\pgfsetdash{}{0pt}%
\pgfpathmoveto{\pgfqpoint{1.424993in}{2.823605in}}%
\pgfpathlineto{\pgfqpoint{1.424993in}{2.823605in}}%
\pgfpathlineto{\pgfqpoint{1.451619in}{2.773882in}}%
\pgfpathlineto{\pgfqpoint{1.478951in}{2.724275in}}%
\pgfpathlineto{\pgfqpoint{1.507141in}{2.674811in}}%
\pgfpathlineto{\pgfqpoint{1.536361in}{2.625527in}}%
\pgfpathlineto{\pgfqpoint{1.566874in}{2.576480in}}%
\pgfpathlineto{\pgfqpoint{1.599127in}{2.527765in}}%
\pgfpathlineto{\pgfqpoint{1.633795in}{2.479551in}}%
\pgfpathlineto{\pgfqpoint{1.672119in}{2.432197in}}%
\pgfpathlineto{\pgfqpoint{1.716813in}{2.386620in}}%
\pgfpathlineto{\pgfqpoint{1.775030in}{2.346272in}}%
\pgfpathlineto{\pgfqpoint{1.775030in}{2.346272in}}%
\pgfpathlineto{\pgfqpoint{1.815764in}{2.331700in}}%
\pgfpathlineto{\pgfqpoint{1.815764in}{2.331700in}}%
\pgfpathlineto{\pgfqpoint{1.857417in}{2.327544in}}%
\pgfpathlineto{\pgfqpoint{1.899355in}{2.331730in}}%
\pgfpathlineto{\pgfqpoint{1.943536in}{2.342116in}}%
\pgfusepath{stroke}%
\end{pgfscope}%
\begin{pgfscope}%
\pgfpathrectangle{\pgfqpoint{0.647939in}{0.492442in}}{\pgfqpoint{4.273799in}{2.331163in}}%
\pgfusepath{clip}%
\pgfsetbuttcap%
\pgfsetroundjoin%
\pgfsetlinewidth{0.301125pt}%
\definecolor{currentstroke}{rgb}{0.500000,0.500000,0.500000}%
\pgfsetstrokecolor{currentstroke}%
\pgfsetstrokeopacity{0.300000}%
\pgfsetdash{}{0pt}%
\pgfpathmoveto{\pgfqpoint{1.327862in}{2.823605in}}%
\pgfpathlineto{\pgfqpoint{1.327862in}{2.823605in}}%
\pgfpathlineto{\pgfqpoint{1.349522in}{2.773169in}}%
\pgfpathlineto{\pgfqpoint{1.371275in}{2.722745in}}%
\pgfpathlineto{\pgfqpoint{1.393105in}{2.672331in}}%
\pgfpathlineto{\pgfqpoint{1.414987in}{2.621925in}}%
\pgfpathlineto{\pgfqpoint{1.436889in}{2.571521in}}%
\pgfpathlineto{\pgfqpoint{1.458766in}{2.521114in}}%
\pgfpathlineto{\pgfqpoint{1.480549in}{2.470696in}}%
\pgfpathlineto{\pgfqpoint{1.502139in}{2.420254in}}%
\pgfpathlineto{\pgfqpoint{1.523383in}{2.369770in}}%
\pgfpathlineto{\pgfqpoint{1.544029in}{2.319214in}}%
\pgfpathlineto{\pgfqpoint{1.563648in}{2.268540in}}%
\pgfpathlineto{\pgfqpoint{1.581425in}{2.217670in}}%
\pgfpathlineto{\pgfqpoint{1.595711in}{2.166481in}}%
\pgfpathlineto{\pgfqpoint{1.602769in}{2.114895in}}%
\pgfpathlineto{\pgfqpoint{1.595635in}{2.063590in}}%
\pgfpathlineto{\pgfqpoint{1.572012in}{2.014093in}}%
\pgfpathlineto{\pgfqpoint{1.540030in}{1.967443in}}%
\pgfpathlineto{\pgfqpoint{1.503061in}{1.919956in}}%
\pgfpathlineto{\pgfqpoint{1.464903in}{1.872634in}}%
\pgfusepath{stroke}%
\end{pgfscope}%
\begin{pgfscope}%
\pgfpathrectangle{\pgfqpoint{0.647939in}{0.492442in}}{\pgfqpoint{4.273799in}{2.331163in}}%
\pgfusepath{clip}%
\pgfsetbuttcap%
\pgfsetroundjoin%
\pgfsetlinewidth{0.301125pt}%
\definecolor{currentstroke}{rgb}{0.500000,0.500000,0.500000}%
\pgfsetstrokecolor{currentstroke}%
\pgfsetstrokeopacity{0.300000}%
\pgfsetdash{}{0pt}%
\pgfpathmoveto{\pgfqpoint{1.230730in}{2.823605in}}%
\pgfpathlineto{\pgfqpoint{1.230730in}{2.823605in}}%
\pgfpathlineto{\pgfqpoint{1.248499in}{2.772718in}}%
\pgfpathlineto{\pgfqpoint{1.266037in}{2.721806in}}%
\pgfpathlineto{\pgfqpoint{1.283283in}{2.670866in}}%
\pgfpathlineto{\pgfqpoint{1.300168in}{2.619889in}}%
\pgfpathlineto{\pgfqpoint{1.316598in}{2.568868in}}%
\pgfpathlineto{\pgfqpoint{1.332462in}{2.517794in}}%
\pgfpathlineto{\pgfqpoint{1.347621in}{2.466656in}}%
\pgfpathlineto{\pgfqpoint{1.361894in}{2.415444in}}%
\pgfpathlineto{\pgfqpoint{1.375060in}{2.364143in}}%
\pgfpathlineto{\pgfqpoint{1.386839in}{2.312743in}}%
\pgfpathlineto{\pgfqpoint{1.396866in}{2.261233in}}%
\pgfpathlineto{\pgfqpoint{1.404694in}{2.209611in}}%
\pgfpathlineto{\pgfqpoint{1.409789in}{2.157890in}}%
\pgfpathlineto{\pgfqpoint{1.411543in}{2.106107in}}%
\pgfpathlineto{\pgfqpoint{1.409340in}{2.054333in}}%
\pgfpathlineto{\pgfqpoint{1.402683in}{2.002677in}}%
\pgfpathlineto{\pgfqpoint{1.391345in}{1.951268in}}%
\pgfpathlineto{\pgfqpoint{1.375481in}{1.900215in}}%
\pgfpathlineto{\pgfqpoint{1.355650in}{1.849582in}}%
\pgfpathlineto{\pgfqpoint{1.332588in}{1.799355in}}%
\pgfpathlineto{\pgfqpoint{1.307103in}{1.749477in}}%
\pgfusepath{stroke}%
\end{pgfscope}%
\begin{pgfscope}%
\pgfpathrectangle{\pgfqpoint{0.647939in}{0.492442in}}{\pgfqpoint{4.273799in}{2.331163in}}%
\pgfusepath{clip}%
\pgfsetbuttcap%
\pgfsetroundjoin%
\pgfsetlinewidth{0.301125pt}%
\definecolor{currentstroke}{rgb}{0.500000,0.500000,0.500000}%
\pgfsetstrokecolor{currentstroke}%
\pgfsetstrokeopacity{0.300000}%
\pgfsetdash{}{0pt}%
\pgfpathmoveto{\pgfqpoint{1.133598in}{2.823605in}}%
\pgfpathlineto{\pgfqpoint{1.133598in}{2.823605in}}%
\pgfpathlineto{\pgfqpoint{1.148309in}{2.772428in}}%
\pgfpathlineto{\pgfqpoint{1.162639in}{2.721218in}}%
\pgfpathlineto{\pgfqpoint{1.176525in}{2.669972in}}%
\pgfpathlineto{\pgfqpoint{1.189892in}{2.618685in}}%
\pgfpathlineto{\pgfqpoint{1.202659in}{2.567353in}}%
\pgfpathlineto{\pgfqpoint{1.214731in}{2.515971in}}%
\pgfpathlineto{\pgfqpoint{1.225992in}{2.464534in}}%
\pgfpathlineto{\pgfqpoint{1.236312in}{2.413039in}}%
\pgfpathlineto{\pgfqpoint{1.245546in}{2.361482in}}%
\pgfpathlineto{\pgfqpoint{1.253528in}{2.309864in}}%
\pgfpathlineto{\pgfqpoint{1.260064in}{2.258185in}}%
\pgfpathlineto{\pgfqpoint{1.264947in}{2.206452in}}%
\pgfpathlineto{\pgfqpoint{1.267957in}{2.154678in}}%
\pgfpathlineto{\pgfqpoint{1.268875in}{2.102881in}}%
\pgfpathlineto{\pgfqpoint{1.267489in}{2.051088in}}%
\pgfpathlineto{\pgfqpoint{1.263623in}{1.999333in}}%
\pgfpathlineto{\pgfqpoint{1.257153in}{1.947655in}}%
\pgfpathlineto{\pgfqpoint{1.248034in}{1.896098in}}%
\pgfpathlineto{\pgfqpoint{1.236318in}{1.844698in}}%
\pgfpathlineto{\pgfqpoint{1.222140in}{1.793485in}}%
\pgfpathlineto{\pgfqpoint{1.205699in}{1.742471in}}%
\pgfpathlineto{\pgfqpoint{1.187271in}{1.691659in}}%
\pgfpathlineto{\pgfqpoint{1.167140in}{1.641040in}}%
\pgfpathlineto{\pgfqpoint{1.145600in}{1.590595in}}%
\pgfpathlineto{\pgfqpoint{1.122923in}{1.540297in}}%
\pgfpathlineto{\pgfqpoint{1.099356in}{1.490121in}}%
\pgfpathlineto{\pgfqpoint{1.075117in}{1.440041in}}%
\pgfpathlineto{\pgfqpoint{1.050380in}{1.390033in}}%
\pgfpathlineto{\pgfqpoint{1.025300in}{1.340077in}}%
\pgfpathlineto{\pgfqpoint{0.999988in}{1.290153in}}%
\pgfpathlineto{\pgfqpoint{0.974550in}{1.240250in}}%
\pgfpathlineto{\pgfqpoint{0.949056in}{1.190353in}}%
\pgfpathlineto{\pgfqpoint{0.923575in}{1.140454in}}%
\pgfpathlineto{\pgfqpoint{0.898160in}{1.090547in}}%
\pgfpathlineto{\pgfqpoint{0.872842in}{1.040624in}}%
\pgfpathlineto{\pgfqpoint{0.847656in}{0.990681in}}%
\pgfpathlineto{\pgfqpoint{0.822617in}{0.940714in}}%
\pgfpathlineto{\pgfqpoint{0.797750in}{0.890722in}}%
\pgfpathlineto{\pgfqpoint{0.773068in}{0.840702in}}%
\pgfpathlineto{\pgfqpoint{0.748578in}{0.790654in}}%
\pgfusepath{stroke}%
\end{pgfscope}%
\begin{pgfscope}%
\pgfpathrectangle{\pgfqpoint{0.647939in}{0.492442in}}{\pgfqpoint{4.273799in}{2.331163in}}%
\pgfusepath{clip}%
\pgfsetbuttcap%
\pgfsetroundjoin%
\pgfsetlinewidth{0.301125pt}%
\definecolor{currentstroke}{rgb}{0.500000,0.500000,0.500000}%
\pgfsetstrokecolor{currentstroke}%
\pgfsetstrokeopacity{0.300000}%
\pgfsetdash{}{0pt}%
\pgfpathmoveto{\pgfqpoint{1.036466in}{2.823605in}}%
\pgfpathlineto{\pgfqpoint{1.036466in}{2.823605in}}%
\pgfpathlineto{\pgfqpoint{1.048767in}{2.772238in}}%
\pgfpathlineto{\pgfqpoint{1.060624in}{2.720841in}}%
\pgfpathlineto{\pgfqpoint{1.071989in}{2.669410in}}%
\pgfpathlineto{\pgfqpoint{1.082803in}{2.617944in}}%
\pgfpathlineto{\pgfqpoint{1.093000in}{2.566441in}}%
\pgfpathlineto{\pgfqpoint{1.102505in}{2.514898in}}%
\pgfpathlineto{\pgfqpoint{1.111241in}{2.463315in}}%
\pgfpathlineto{\pgfqpoint{1.119124in}{2.411691in}}%
\pgfpathlineto{\pgfqpoint{1.126057in}{2.360027in}}%
\pgfpathlineto{\pgfqpoint{1.131938in}{2.308324in}}%
\pgfpathlineto{\pgfqpoint{1.136661in}{2.256585in}}%
\pgfpathlineto{\pgfqpoint{1.140117in}{2.204818in}}%
\pgfpathlineto{\pgfqpoint{1.142195in}{2.153028in}}%
\pgfpathlineto{\pgfqpoint{1.142787in}{2.101227in}}%
\pgfpathlineto{\pgfqpoint{1.141794in}{2.049429in}}%
\pgfpathlineto{\pgfqpoint{1.139131in}{1.997648in}}%
\pgfpathlineto{\pgfqpoint{1.134731in}{1.945903in}}%
\pgfpathlineto{\pgfqpoint{1.128554in}{1.894212in}}%
\pgfpathlineto{\pgfqpoint{1.120589in}{1.842594in}}%
\pgfpathlineto{\pgfqpoint{1.110863in}{1.791066in}}%
\pgfpathlineto{\pgfqpoint{1.099436in}{1.739643in}}%
\pgfpathlineto{\pgfqpoint{1.086391in}{1.688334in}}%
\pgfpathlineto{\pgfqpoint{1.071838in}{1.637146in}}%
\pgfpathlineto{\pgfqpoint{1.055915in}{1.586080in}}%
\pgfpathlineto{\pgfqpoint{1.038758in}{1.535133in}}%
\pgfpathlineto{\pgfqpoint{1.020516in}{1.484298in}}%
\pgfusepath{stroke}%
\end{pgfscope}%
\begin{pgfscope}%
\pgfpathrectangle{\pgfqpoint{0.647939in}{0.492442in}}{\pgfqpoint{4.273799in}{2.331163in}}%
\pgfusepath{clip}%
\pgfsetbuttcap%
\pgfsetroundjoin%
\pgfsetlinewidth{0.301125pt}%
\definecolor{currentstroke}{rgb}{0.500000,0.500000,0.500000}%
\pgfsetstrokecolor{currentstroke}%
\pgfsetstrokeopacity{0.300000}%
\pgfsetdash{}{0pt}%
\pgfpathmoveto{\pgfqpoint{0.939334in}{2.823605in}}%
\pgfpathlineto{\pgfqpoint{0.939334in}{2.823605in}}%
\pgfpathlineto{\pgfqpoint{0.949713in}{2.772112in}}%
\pgfpathlineto{\pgfqpoint{0.959642in}{2.720593in}}%
\pgfpathlineto{\pgfqpoint{0.969080in}{2.669046in}}%
\pgfpathlineto{\pgfqpoint{0.977981in}{2.617471in}}%
\pgfpathlineto{\pgfqpoint{0.986298in}{2.565867in}}%
\pgfpathlineto{\pgfqpoint{0.993981in}{2.514234in}}%
\pgfpathlineto{\pgfqpoint{1.000973in}{2.462571in}}%
\pgfpathlineto{\pgfqpoint{1.007213in}{2.410880in}}%
\pgfpathlineto{\pgfqpoint{1.012642in}{2.359162in}}%
\pgfpathlineto{\pgfqpoint{1.017198in}{2.307419in}}%
\pgfpathlineto{\pgfqpoint{1.020815in}{2.255654in}}%
\pgfpathlineto{\pgfqpoint{1.023431in}{2.203871in}}%
\pgfpathlineto{\pgfqpoint{1.024981in}{2.152075in}}%
\pgfpathlineto{\pgfqpoint{1.025405in}{2.100273in}}%
\pgfpathlineto{\pgfqpoint{1.024648in}{2.048472in}}%
\pgfpathlineto{\pgfqpoint{1.022662in}{1.996681in}}%
\pgfpathlineto{\pgfqpoint{1.019408in}{1.944909in}}%
\pgfpathlineto{\pgfqpoint{1.014859in}{1.893167in}}%
\pgfpathlineto{\pgfqpoint{1.008999in}{1.841464in}}%
\pgfpathlineto{\pgfqpoint{1.001829in}{1.789810in}}%
\pgfpathlineto{\pgfqpoint{0.993361in}{1.738214in}}%
\pgfpathlineto{\pgfqpoint{0.983629in}{1.686685in}}%
\pgfpathlineto{\pgfqpoint{0.972681in}{1.635229in}}%
\pgfpathlineto{\pgfqpoint{0.960572in}{1.583851in}}%
\pgfpathlineto{\pgfqpoint{0.947365in}{1.532552in}}%
\pgfpathlineto{\pgfqpoint{0.933143in}{1.481335in}}%
\pgfpathlineto{\pgfqpoint{0.917988in}{1.430198in}}%
\pgfpathlineto{\pgfqpoint{0.901980in}{1.379137in}}%
\pgfpathlineto{\pgfqpoint{0.885208in}{1.328151in}}%
\pgfpathlineto{\pgfqpoint{0.867754in}{1.277232in}}%
\pgfpathlineto{\pgfqpoint{0.849700in}{1.226375in}}%
\pgfpathlineto{\pgfqpoint{0.831124in}{1.175575in}}%
\pgfpathlineto{\pgfqpoint{0.812095in}{1.124824in}}%
\pgfpathlineto{\pgfqpoint{0.792681in}{1.074117in}}%
\pgfpathlineto{\pgfqpoint{0.772940in}{1.023448in}}%
\pgfpathlineto{\pgfqpoint{0.752929in}{0.972810in}}%
\pgfpathlineto{\pgfqpoint{0.732695in}{0.922197in}}%
\pgfusepath{stroke}%
\end{pgfscope}%
\begin{pgfscope}%
\pgfpathrectangle{\pgfqpoint{0.647939in}{0.492442in}}{\pgfqpoint{4.273799in}{2.331163in}}%
\pgfusepath{clip}%
\pgfsetbuttcap%
\pgfsetroundjoin%
\pgfsetlinewidth{0.301125pt}%
\definecolor{currentstroke}{rgb}{0.500000,0.500000,0.500000}%
\pgfsetstrokecolor{currentstroke}%
\pgfsetstrokeopacity{0.300000}%
\pgfsetdash{}{0pt}%
\pgfpathmoveto{\pgfqpoint{0.842203in}{2.823605in}}%
\pgfpathlineto{\pgfqpoint{0.842203in}{2.823605in}}%
\pgfpathlineto{\pgfqpoint{0.851034in}{2.772027in}}%
\pgfpathlineto{\pgfqpoint{0.859434in}{2.720426in}}%
\pgfpathlineto{\pgfqpoint{0.867370in}{2.668804in}}%
\pgfpathlineto{\pgfqpoint{0.874807in}{2.617160in}}%
\pgfpathlineto{\pgfqpoint{0.881709in}{2.565494in}}%
\pgfpathlineto{\pgfqpoint{0.888040in}{2.513806in}}%
\pgfpathlineto{\pgfqpoint{0.893761in}{2.462097in}}%
\pgfpathlineto{\pgfqpoint{0.898832in}{2.410367in}}%
\pgfpathlineto{\pgfqpoint{0.903214in}{2.358619in}}%
\pgfpathlineto{\pgfqpoint{0.906864in}{2.306855in}}%
\pgfpathlineto{\pgfqpoint{0.909742in}{2.255075in}}%
\pgfpathlineto{\pgfqpoint{0.911807in}{2.203284in}}%
\pgfpathlineto{\pgfqpoint{0.913019in}{2.151486in}}%
\pgfpathlineto{\pgfqpoint{0.913342in}{2.099683in}}%
\pgfpathlineto{\pgfqpoint{0.912743in}{2.047881in}}%
\pgfpathlineto{\pgfqpoint{0.911191in}{1.996085in}}%
\pgfpathlineto{\pgfqpoint{0.908661in}{1.944301in}}%
\pgfpathlineto{\pgfqpoint{0.905136in}{1.892534in}}%
\pgfpathlineto{\pgfqpoint{0.900601in}{1.840790in}}%
\pgfpathlineto{\pgfqpoint{0.895055in}{1.789076in}}%
\pgfpathlineto{\pgfqpoint{0.888499in}{1.737397in}}%
\pgfpathlineto{\pgfqpoint{0.880943in}{1.685759in}}%
\pgfpathlineto{\pgfqpoint{0.872406in}{1.634166in}}%
\pgfpathlineto{\pgfqpoint{0.862911in}{1.582623in}}%
\pgfpathlineto{\pgfqpoint{0.852495in}{1.531133in}}%
\pgfpathlineto{\pgfqpoint{0.841199in}{1.479699in}}%
\pgfpathlineto{\pgfqpoint{0.829065in}{1.428321in}}%
\pgfpathlineto{\pgfqpoint{0.816140in}{1.377001in}}%
\pgfusepath{stroke}%
\end{pgfscope}%
\begin{pgfscope}%
\pgfpathrectangle{\pgfqpoint{0.647939in}{0.492442in}}{\pgfqpoint{4.273799in}{2.331163in}}%
\pgfusepath{clip}%
\pgfsetbuttcap%
\pgfsetroundjoin%
\pgfsetlinewidth{0.301125pt}%
\definecolor{currentstroke}{rgb}{0.500000,0.500000,0.500000}%
\pgfsetstrokecolor{currentstroke}%
\pgfsetstrokeopacity{0.300000}%
\pgfsetdash{}{0pt}%
\pgfpathmoveto{\pgfqpoint{0.745071in}{2.823605in}}%
\pgfpathlineto{\pgfqpoint{0.745071in}{2.823605in}}%
\pgfpathlineto{\pgfqpoint{0.752648in}{2.771967in}}%
\pgfpathlineto{\pgfqpoint{0.759821in}{2.720312in}}%
\pgfpathlineto{\pgfqpoint{0.766565in}{2.668639in}}%
\pgfpathlineto{\pgfqpoint{0.772855in}{2.616950in}}%
\pgfpathlineto{\pgfqpoint{0.778665in}{2.565243in}}%
\pgfpathlineto{\pgfqpoint{0.783967in}{2.513521in}}%
\pgfpathlineto{\pgfqpoint{0.788734in}{2.461783in}}%
\pgfpathlineto{\pgfqpoint{0.792937in}{2.410030in}}%
\pgfpathlineto{\pgfqpoint{0.796550in}{2.358264in}}%
\pgfpathlineto{\pgfqpoint{0.799545in}{2.306487in}}%
\pgfpathlineto{\pgfqpoint{0.801893in}{2.254699in}}%
\pgfpathlineto{\pgfqpoint{0.803569in}{2.202904in}}%
\pgfpathlineto{\pgfqpoint{0.804547in}{2.151104in}}%
\pgfpathlineto{\pgfqpoint{0.804804in}{2.099301in}}%
\pgfpathlineto{\pgfqpoint{0.804316in}{2.047498in}}%
\pgfpathlineto{\pgfqpoint{0.803066in}{1.995700in}}%
\pgfpathlineto{\pgfqpoint{0.801035in}{1.943908in}}%
\pgfpathlineto{\pgfqpoint{0.798212in}{1.892128in}}%
\pgfpathlineto{\pgfqpoint{0.794585in}{1.840363in}}%
\pgfpathlineto{\pgfqpoint{0.790151in}{1.788617in}}%
\pgfpathlineto{\pgfqpoint{0.784906in}{1.736893in}}%
\pgfpathlineto{\pgfqpoint{0.778855in}{1.685195in}}%
\pgfpathlineto{\pgfqpoint{0.772004in}{1.633527in}}%
\pgfpathlineto{\pgfqpoint{0.764369in}{1.581892in}}%
\pgfpathlineto{\pgfqpoint{0.755966in}{1.530292in}}%
\pgfpathlineto{\pgfqpoint{0.746814in}{1.478730in}}%
\pgfpathlineto{\pgfqpoint{0.736936in}{1.427209in}}%
\pgfpathlineto{\pgfqpoint{0.726361in}{1.375728in}}%
\pgfpathlineto{\pgfqpoint{0.715123in}{1.324289in}}%
\pgfpathlineto{\pgfqpoint{0.703253in}{1.272893in}}%
\pgfpathlineto{\pgfqpoint{0.690781in}{1.221539in}}%
\pgfpathlineto{\pgfqpoint{0.677747in}{1.170226in}}%
\pgfpathlineto{\pgfqpoint{0.664190in}{1.118954in}}%
\pgfpathlineto{\pgfqpoint{0.650143in}{1.067721in}}%
\pgfpathlineto{\pgfqpoint{0.647939in}{1.059820in}}%
\pgfusepath{stroke}%
\end{pgfscope}%
\begin{pgfscope}%
\pgfpathrectangle{\pgfqpoint{0.647939in}{0.492442in}}{\pgfqpoint{4.273799in}{2.331163in}}%
\pgfusepath{clip}%
\pgfsetbuttcap%
\pgfsetroundjoin%
\pgfsetlinewidth{0.301125pt}%
\definecolor{currentstroke}{rgb}{0.500000,0.500000,0.500000}%
\pgfsetstrokecolor{currentstroke}%
\pgfsetstrokeopacity{0.300000}%
\pgfsetdash{}{0pt}%
\pgfpathmoveto{\pgfqpoint{0.647939in}{2.823605in}}%
\pgfpathlineto{\pgfqpoint{0.647939in}{2.823605in}}%
\pgfpathlineto{\pgfqpoint{0.654491in}{2.771925in}}%
\pgfpathlineto{\pgfqpoint{0.660668in}{2.720232in}}%
\pgfpathlineto{\pgfqpoint{0.666455in}{2.668524in}}%
\pgfpathlineto{\pgfqpoint{0.671830in}{2.616804in}}%
\pgfpathlineto{\pgfqpoint{0.676776in}{2.565071in}}%
\pgfpathlineto{\pgfqpoint{0.681273in}{2.513326in}}%
\pgfpathlineto{\pgfqpoint{0.685302in}{2.461569in}}%
\pgfpathlineto{\pgfqpoint{0.688841in}{2.409801in}}%
\pgfpathlineto{\pgfqpoint{0.691873in}{2.358024in}}%
\pgfpathlineto{\pgfqpoint{0.694376in}{2.306239in}}%
\pgfpathlineto{\pgfqpoint{0.696332in}{2.254447in}}%
\pgfpathlineto{\pgfqpoint{0.697722in}{2.202649in}}%
\pgfpathlineto{\pgfqpoint{0.698530in}{2.150847in}}%
\pgfpathlineto{\pgfqpoint{0.698740in}{2.099044in}}%
\pgfpathlineto{\pgfqpoint{0.698335in}{2.047241in}}%
\pgfpathlineto{\pgfqpoint{0.697303in}{1.995441in}}%
\pgfpathlineto{\pgfqpoint{0.695633in}{1.943646in}}%
\pgfpathlineto{\pgfqpoint{0.693314in}{1.891858in}}%
\pgfpathlineto{\pgfqpoint{0.690340in}{1.840080in}}%
\pgfpathlineto{\pgfqpoint{0.686705in}{1.788315in}}%
\pgfpathlineto{\pgfqpoint{0.682409in}{1.736565in}}%
\pgfpathlineto{\pgfqpoint{0.677450in}{1.684832in}}%
\pgfpathlineto{\pgfqpoint{0.671833in}{1.633120in}}%
\pgfpathlineto{\pgfqpoint{0.665562in}{1.581430in}}%
\pgfpathlineto{\pgfqpoint{0.658647in}{1.529764in}}%
\pgfpathlineto{\pgfqpoint{0.651097in}{1.478125in}}%
\pgfpathlineto{\pgfqpoint{0.647939in}{1.457385in}}%
\pgfusepath{stroke}%
\end{pgfscope}%
\begin{pgfscope}%
\pgfpathrectangle{\pgfqpoint{0.647939in}{0.492442in}}{\pgfqpoint{4.273799in}{2.331163in}}%
\pgfusepath{clip}%
\pgfsetbuttcap%
\pgfsetroundjoin%
\pgfsetlinewidth{0.301125pt}%
\definecolor{currentstroke}{rgb}{0.500000,0.500000,0.500000}%
\pgfsetstrokecolor{currentstroke}%
\pgfsetstrokeopacity{0.300000}%
\pgfsetdash{}{0pt}%
\pgfpathmoveto{\pgfqpoint{4.047552in}{0.651385in}}%
\pgfpathlineto{\pgfqpoint{3.985037in}{0.690370in}}%
\pgfpathlineto{\pgfqpoint{3.920133in}{0.728179in}}%
\pgfpathlineto{\pgfqpoint{3.853176in}{0.764910in}}%
\pgfpathlineto{\pgfqpoint{3.784612in}{0.800752in}}%
\pgfpathlineto{\pgfqpoint{3.714955in}{0.835963in}}%
\pgfpathlineto{\pgfqpoint{3.644782in}{0.870870in}}%
\pgfpathlineto{\pgfqpoint{3.574672in}{0.905813in}}%
\pgfpathlineto{\pgfqpoint{3.505155in}{0.941105in}}%
\pgfpathlineto{\pgfqpoint{3.436727in}{0.977022in}}%
\pgfpathlineto{\pgfqpoint{3.369791in}{1.013762in}}%
\pgfusepath{stroke}%
\end{pgfscope}%
\begin{pgfscope}%
\pgfpathrectangle{\pgfqpoint{0.647939in}{0.492442in}}{\pgfqpoint{4.273799in}{2.331163in}}%
\pgfusepath{clip}%
\pgfsetbuttcap%
\pgfsetroundjoin%
\pgfsetlinewidth{0.301125pt}%
\definecolor{currentstroke}{rgb}{0.500000,0.500000,0.500000}%
\pgfsetstrokecolor{currentstroke}%
\pgfsetstrokeopacity{0.300000}%
\pgfsetdash{}{0pt}%
\pgfpathmoveto{\pgfqpoint{4.643783in}{1.818670in}}%
\pgfpathlineto{\pgfqpoint{4.630343in}{1.869948in}}%
\pgfpathlineto{\pgfqpoint{4.618360in}{1.921333in}}%
\pgfpathlineto{\pgfqpoint{4.608289in}{1.972840in}}%
\pgfpathlineto{\pgfqpoint{4.600720in}{2.024471in}}%
\pgfpathlineto{\pgfqpoint{4.596372in}{2.076209in}}%
\pgfpathlineto{\pgfqpoint{4.596061in}{2.127996in}}%
\pgfpathlineto{\pgfqpoint{4.600547in}{2.179717in}}%
\pgfpathlineto{\pgfqpoint{4.610276in}{2.231216in}}%
\pgfpathlineto{\pgfqpoint{4.625156in}{2.282352in}}%
\pgfusepath{stroke}%
\end{pgfscope}%
\begin{pgfscope}%
\pgfpathrectangle{\pgfqpoint{0.647939in}{0.492442in}}{\pgfqpoint{4.273799in}{2.331163in}}%
\pgfusepath{clip}%
\pgfsetbuttcap%
\pgfsetroundjoin%
\pgfsetlinewidth{0.301125pt}%
\definecolor{currentstroke}{rgb}{0.500000,0.500000,0.500000}%
\pgfsetstrokecolor{currentstroke}%
\pgfsetstrokeopacity{0.300000}%
\pgfsetdash{}{0pt}%
\pgfpathmoveto{\pgfqpoint{4.533211in}{1.075233in}}%
\pgfpathlineto{\pgfqpoint{4.497457in}{1.123220in}}%
\pgfpathlineto{\pgfqpoint{4.459520in}{1.170699in}}%
\pgfpathlineto{\pgfqpoint{4.418878in}{1.217501in}}%
\pgfpathlineto{\pgfqpoint{4.374817in}{1.263364in}}%
\pgfpathlineto{\pgfqpoint{4.326329in}{1.307863in}}%
\pgfpathlineto{\pgfqpoint{4.271963in}{1.350276in}}%
\pgfpathlineto{\pgfqpoint{4.209879in}{1.389330in}}%
\pgfpathlineto{\pgfqpoint{4.137976in}{1.422850in}}%
\pgfpathlineto{\pgfqpoint{4.055487in}{1.447848in}}%
\pgfpathlineto{\pgfqpoint{3.969729in}{1.461966in}}%
\pgfpathlineto{\pgfqpoint{3.880763in}{1.468049in}}%
\pgfpathlineto{\pgfqpoint{3.785934in}{1.469524in}}%
\pgfpathlineto{\pgfqpoint{3.691021in}{1.469868in}}%
\pgfusepath{stroke}%
\end{pgfscope}%
\begin{pgfscope}%
\pgfpathrectangle{\pgfqpoint{0.647939in}{0.492442in}}{\pgfqpoint{4.273799in}{2.331163in}}%
\pgfusepath{clip}%
\pgfsetbuttcap%
\pgfsetroundjoin%
\pgfsetlinewidth{0.301125pt}%
\definecolor{currentstroke}{rgb}{0.500000,0.500000,0.500000}%
\pgfsetstrokecolor{currentstroke}%
\pgfsetstrokeopacity{0.300000}%
\pgfsetdash{}{0pt}%
\pgfpathmoveto{\pgfqpoint{4.533211in}{1.340138in}}%
\pgfpathlineto{\pgfqpoint{4.500167in}{1.388698in}}%
\pgfpathlineto{\pgfqpoint{4.464906in}{1.436782in}}%
\pgfpathlineto{\pgfqpoint{4.426678in}{1.484175in}}%
\pgfpathlineto{\pgfqpoint{4.384285in}{1.530481in}}%
\pgfpathlineto{\pgfqpoint{4.335685in}{1.574880in}}%
\pgfpathlineto{\pgfqpoint{4.277127in}{1.615379in}}%
\pgfpathlineto{\pgfqpoint{4.277127in}{1.615379in}}%
\pgfpathlineto{\pgfqpoint{4.219146in}{1.641512in}}%
\pgfpathlineto{\pgfqpoint{4.219146in}{1.641512in}}%
\pgfpathlineto{\pgfqpoint{4.163458in}{1.654678in}}%
\pgfpathlineto{\pgfqpoint{4.102366in}{1.657823in}}%
\pgfpathlineto{\pgfqpoint{4.043909in}{1.652630in}}%
\pgfpathlineto{\pgfqpoint{3.977041in}{1.640277in}}%
\pgfpathlineto{\pgfqpoint{3.890541in}{1.619315in}}%
\pgfusepath{stroke}%
\end{pgfscope}%
\begin{pgfscope}%
\pgfpathrectangle{\pgfqpoint{0.647939in}{0.492442in}}{\pgfqpoint{4.273799in}{2.331163in}}%
\pgfusepath{clip}%
\pgfsetbuttcap%
\pgfsetroundjoin%
\pgfsetlinewidth{0.301125pt}%
\definecolor{currentstroke}{rgb}{0.500000,0.500000,0.500000}%
\pgfsetstrokecolor{currentstroke}%
\pgfsetstrokeopacity{0.300000}%
\pgfsetdash{}{0pt}%
\pgfpathmoveto{\pgfqpoint{4.559136in}{1.608193in}}%
\pgfpathlineto{\pgfqpoint{4.533211in}{1.658024in}}%
\pgfpathlineto{\pgfqpoint{4.506268in}{1.707691in}}%
\pgfpathlineto{\pgfqpoint{4.477936in}{1.757116in}}%
\pgfpathlineto{\pgfqpoint{4.447459in}{1.806162in}}%
\pgfpathlineto{\pgfqpoint{4.413327in}{1.854454in}}%
\pgfpathlineto{\pgfqpoint{4.371320in}{1.900626in}}%
\pgfpathlineto{\pgfqpoint{4.371320in}{1.900626in}}%
\pgfpathlineto{\pgfqpoint{4.339094in}{1.923051in}}%
\pgfpathlineto{\pgfqpoint{4.339094in}{1.923051in}}%
\pgfpathlineto{\pgfqpoint{4.308966in}{1.932286in}}%
\pgfpathlineto{\pgfqpoint{4.308966in}{1.932286in}}%
\pgfpathlineto{\pgfqpoint{4.279476in}{1.931108in}}%
\pgfpathlineto{\pgfqpoint{4.252307in}{1.923257in}}%
\pgfpathlineto{\pgfqpoint{4.221058in}{1.908648in}}%
\pgfusepath{stroke}%
\end{pgfscope}%
\begin{pgfscope}%
\pgfpathrectangle{\pgfqpoint{0.647939in}{0.492442in}}{\pgfqpoint{4.273799in}{2.331163in}}%
\pgfusepath{clip}%
\pgfsetbuttcap%
\pgfsetroundjoin%
\pgfsetlinewidth{0.301125pt}%
\definecolor{currentstroke}{rgb}{0.500000,0.500000,0.500000}%
\pgfsetstrokecolor{currentstroke}%
\pgfsetstrokeopacity{0.300000}%
\pgfsetdash{}{0pt}%
\pgfpathmoveto{\pgfqpoint{2.590575in}{0.757347in}}%
\pgfpathlineto{\pgfqpoint{2.550958in}{0.804428in}}%
\pgfpathlineto{\pgfqpoint{2.512105in}{0.851697in}}%
\pgfpathlineto{\pgfqpoint{2.473988in}{0.899145in}}%
\pgfpathlineto{\pgfqpoint{2.436583in}{0.946761in}}%
\pgfpathlineto{\pgfqpoint{2.399863in}{0.994535in}}%
\pgfpathlineto{\pgfqpoint{2.363807in}{1.042460in}}%
\pgfpathlineto{\pgfqpoint{2.328393in}{1.090527in}}%
\pgfpathlineto{\pgfqpoint{2.293597in}{1.138728in}}%
\pgfpathlineto{\pgfqpoint{2.259393in}{1.187055in}}%
\pgfpathlineto{\pgfqpoint{2.225758in}{1.235501in}}%
\pgfpathlineto{\pgfqpoint{2.192679in}{1.284061in}}%
\pgfpathlineto{\pgfqpoint{2.160142in}{1.332729in}}%
\pgfpathlineto{\pgfqpoint{2.128122in}{1.381500in}}%
\pgfpathlineto{\pgfqpoint{2.096598in}{1.430366in}}%
\pgfpathlineto{\pgfqpoint{2.065564in}{1.479325in}}%
\pgfpathlineto{\pgfqpoint{2.035009in}{1.528375in}}%
\pgfpathlineto{\pgfqpoint{2.004913in}{1.577508in}}%
\pgfpathlineto{\pgfqpoint{1.975268in}{1.626722in}}%
\pgfpathlineto{\pgfqpoint{1.946080in}{1.676018in}}%
\pgfpathlineto{\pgfqpoint{1.917334in}{1.725390in}}%
\pgfpathlineto{\pgfqpoint{1.889031in}{1.774838in}}%
\pgfpathlineto{\pgfqpoint{1.861205in}{1.824365in}}%
\pgfpathlineto{\pgfqpoint{1.833873in}{1.873972in}}%
\pgfpathlineto{\pgfqpoint{1.807113in}{1.923668in}}%
\pgfpathlineto{\pgfqpoint{1.781111in}{1.973480in}}%
\pgfpathlineto{\pgfqpoint{1.756331in}{2.023450in}}%
\pgfpathlineto{\pgfqpoint{1.734902in}{2.073836in}}%
\pgfpathlineto{\pgfqpoint{1.734902in}{2.073836in}}%
\pgfpathlineto{\pgfqpoint{1.728524in}{2.099144in}}%
\pgfpathlineto{\pgfqpoint{1.728524in}{2.099144in}}%
\pgfpathlineto{\pgfqpoint{1.730219in}{2.117761in}}%
\pgfpathlineto{\pgfqpoint{1.730219in}{2.117761in}}%
\pgfpathlineto{\pgfqpoint{1.744137in}{2.138228in}}%
\pgfpathlineto{\pgfqpoint{1.762109in}{2.157417in}}%
\pgfpathlineto{\pgfqpoint{1.796260in}{2.187052in}}%
\pgfpathlineto{\pgfqpoint{1.850084in}{2.228533in}}%
\pgfpathlineto{\pgfqpoint{1.908013in}{2.269027in}}%
\pgfusepath{stroke}%
\end{pgfscope}%
\begin{pgfscope}%
\pgfpathrectangle{\pgfqpoint{0.647939in}{0.492442in}}{\pgfqpoint{4.273799in}{2.331163in}}%
\pgfusepath{clip}%
\pgfsetbuttcap%
\pgfsetroundjoin%
\pgfsetlinewidth{0.301125pt}%
\definecolor{currentstroke}{rgb}{0.500000,0.500000,0.500000}%
\pgfsetstrokecolor{currentstroke}%
\pgfsetstrokeopacity{0.300000}%
\pgfsetdash{}{0pt}%
\pgfpathmoveto{\pgfqpoint{3.464761in}{0.757347in}}%
\pgfpathlineto{\pgfqpoint{3.401330in}{0.795898in}}%
\pgfpathlineto{\pgfqpoint{3.339175in}{0.835061in}}%
\pgfpathlineto{\pgfqpoint{3.278503in}{0.874910in}}%
\pgfpathlineto{\pgfqpoint{3.219471in}{0.915483in}}%
\pgfpathlineto{\pgfqpoint{3.162189in}{0.956796in}}%
\pgfpathlineto{\pgfqpoint{3.106718in}{0.998838in}}%
\pgfpathlineto{\pgfqpoint{3.053093in}{1.041587in}}%
\pgfusepath{stroke}%
\end{pgfscope}%
\begin{pgfscope}%
\pgfpathrectangle{\pgfqpoint{0.647939in}{0.492442in}}{\pgfqpoint{4.273799in}{2.331163in}}%
\pgfusepath{clip}%
\pgfsetbuttcap%
\pgfsetroundjoin%
\pgfsetlinewidth{0.301125pt}%
\definecolor{currentstroke}{rgb}{0.500000,0.500000,0.500000}%
\pgfsetstrokecolor{currentstroke}%
\pgfsetstrokeopacity{0.300000}%
\pgfsetdash{}{0pt}%
\pgfpathmoveto{\pgfqpoint{3.561893in}{0.757347in}}%
\pgfpathlineto{\pgfqpoint{3.495999in}{0.794651in}}%
\pgfpathlineto{\pgfqpoint{3.431033in}{0.832435in}}%
\pgfpathlineto{\pgfqpoint{3.367310in}{0.870841in}}%
\pgfpathlineto{\pgfqpoint{3.305080in}{0.909968in}}%
\pgfpathlineto{\pgfqpoint{3.244540in}{0.949874in}}%
\pgfusepath{stroke}%
\end{pgfscope}%
\begin{pgfscope}%
\pgfpathrectangle{\pgfqpoint{0.647939in}{0.492442in}}{\pgfqpoint{4.273799in}{2.331163in}}%
\pgfusepath{clip}%
\pgfsetbuttcap%
\pgfsetroundjoin%
\pgfsetlinewidth{0.301125pt}%
\definecolor{currentstroke}{rgb}{0.500000,0.500000,0.500000}%
\pgfsetstrokecolor{currentstroke}%
\pgfsetstrokeopacity{0.300000}%
\pgfsetdash{}{0pt}%
\pgfpathmoveto{\pgfqpoint{3.659025in}{0.757347in}}%
\pgfpathlineto{\pgfqpoint{3.591348in}{0.793690in}}%
\pgfpathlineto{\pgfqpoint{3.524084in}{0.830260in}}%
\pgfpathlineto{\pgfqpoint{3.457649in}{0.867276in}}%
\pgfpathlineto{\pgfqpoint{3.392411in}{0.904917in}}%
\pgfpathlineto{\pgfqpoint{3.328669in}{0.943312in}}%
\pgfusepath{stroke}%
\end{pgfscope}%
\begin{pgfscope}%
\pgfpathrectangle{\pgfqpoint{0.647939in}{0.492442in}}{\pgfqpoint{4.273799in}{2.331163in}}%
\pgfusepath{clip}%
\pgfsetbuttcap%
\pgfsetroundjoin%
\pgfsetlinewidth{0.301125pt}%
\definecolor{currentstroke}{rgb}{0.500000,0.500000,0.500000}%
\pgfsetstrokecolor{currentstroke}%
\pgfsetstrokeopacity{0.300000}%
\pgfsetdash{}{0pt}%
\pgfpathmoveto{\pgfqpoint{4.436079in}{0.969271in}}%
\pgfpathlineto{\pgfqpoint{4.392936in}{1.015407in}}%
\pgfpathlineto{\pgfqpoint{4.346613in}{1.060609in}}%
\pgfpathlineto{\pgfqpoint{4.296453in}{1.104566in}}%
\pgfpathlineto{\pgfqpoint{4.241635in}{1.146821in}}%
\pgfpathlineto{\pgfqpoint{4.181203in}{1.186718in}}%
\pgfpathlineto{\pgfqpoint{4.114311in}{1.223383in}}%
\pgfpathlineto{\pgfqpoint{4.040462in}{1.255790in}}%
\pgfpathlineto{\pgfqpoint{3.960024in}{1.283123in}}%
\pgfpathlineto{\pgfqpoint{3.874422in}{1.305368in}}%
\pgfpathlineto{\pgfqpoint{3.785632in}{1.323640in}}%
\pgfpathlineto{\pgfqpoint{3.695449in}{1.339858in}}%
\pgfusepath{stroke}%
\end{pgfscope}%
\begin{pgfscope}%
\pgfpathrectangle{\pgfqpoint{0.647939in}{0.492442in}}{\pgfqpoint{4.273799in}{2.331163in}}%
\pgfusepath{clip}%
\pgfsetbuttcap%
\pgfsetroundjoin%
\pgfsetlinewidth{0.301125pt}%
\definecolor{currentstroke}{rgb}{0.500000,0.500000,0.500000}%
\pgfsetstrokecolor{currentstroke}%
\pgfsetstrokeopacity{0.300000}%
\pgfsetdash{}{0pt}%
\pgfpathmoveto{\pgfqpoint{4.477438in}{1.927851in}}%
\pgfpathlineto{\pgfqpoint{4.456137in}{1.978309in}}%
\pgfpathlineto{\pgfqpoint{4.436079in}{2.028891in}}%
\pgfpathlineto{\pgfqpoint{4.420262in}{2.079841in}}%
\pgfpathlineto{\pgfqpoint{4.420262in}{2.079841in}}%
\pgfpathlineto{\pgfqpoint{4.418183in}{2.114018in}}%
\pgfpathlineto{\pgfqpoint{4.418183in}{2.114018in}}%
\pgfpathlineto{\pgfqpoint{4.428344in}{2.142305in}}%
\pgfpathlineto{\pgfqpoint{4.446646in}{2.171246in}}%
\pgfpathlineto{\pgfqpoint{4.477599in}{2.211945in}}%
\pgfusepath{stroke}%
\end{pgfscope}%
\begin{pgfscope}%
\pgfpathrectangle{\pgfqpoint{0.647939in}{0.492442in}}{\pgfqpoint{4.273799in}{2.331163in}}%
\pgfusepath{clip}%
\pgfsetbuttcap%
\pgfsetroundjoin%
\pgfsetlinewidth{0.301125pt}%
\definecolor{currentstroke}{rgb}{0.500000,0.500000,0.500000}%
\pgfsetstrokecolor{currentstroke}%
\pgfsetstrokeopacity{0.300000}%
\pgfsetdash{}{0pt}%
\pgfpathmoveto{\pgfqpoint{1.619257in}{0.810328in}}%
\pgfpathlineto{\pgfqpoint{1.567433in}{0.853709in}}%
\pgfpathlineto{\pgfqpoint{1.510617in}{0.895164in}}%
\pgfpathlineto{\pgfqpoint{1.446230in}{0.933055in}}%
\pgfpathlineto{\pgfqpoint{1.370194in}{0.963343in}}%
\pgfpathlineto{\pgfqpoint{1.370194in}{0.963343in}}%
\pgfpathlineto{\pgfqpoint{1.312419in}{0.974709in}}%
\pgfpathlineto{\pgfqpoint{1.249962in}{0.974705in}}%
\pgfpathlineto{\pgfqpoint{1.198273in}{0.964810in}}%
\pgfpathlineto{\pgfqpoint{1.150589in}{0.947622in}}%
\pgfusepath{stroke}%
\end{pgfscope}%
\begin{pgfscope}%
\pgfpathrectangle{\pgfqpoint{0.647939in}{0.492442in}}{\pgfqpoint{4.273799in}{2.331163in}}%
\pgfusepath{clip}%
\pgfsetbuttcap%
\pgfsetroundjoin%
\pgfsetlinewidth{0.301125pt}%
\definecolor{currentstroke}{rgb}{0.500000,0.500000,0.500000}%
\pgfsetstrokecolor{currentstroke}%
\pgfsetstrokeopacity{0.300000}%
\pgfsetdash{}{0pt}%
\pgfpathmoveto{\pgfqpoint{4.144684in}{0.810328in}}%
\pgfpathlineto{\pgfqpoint{4.083626in}{0.849985in}}%
\pgfpathlineto{\pgfqpoint{4.019007in}{0.887921in}}%
\pgfpathlineto{\pgfqpoint{3.950969in}{0.924039in}}%
\pgfpathlineto{\pgfqpoint{3.879901in}{0.958382in}}%
\pgfpathlineto{\pgfqpoint{3.806395in}{0.991171in}}%
\pgfpathlineto{\pgfqpoint{3.731224in}{1.022826in}}%
\pgfusepath{stroke}%
\end{pgfscope}%
\begin{pgfscope}%
\pgfpathrectangle{\pgfqpoint{0.647939in}{0.492442in}}{\pgfqpoint{4.273799in}{2.331163in}}%
\pgfusepath{clip}%
\pgfsetbuttcap%
\pgfsetroundjoin%
\pgfsetlinewidth{0.301125pt}%
\definecolor{currentstroke}{rgb}{0.500000,0.500000,0.500000}%
\pgfsetstrokecolor{currentstroke}%
\pgfsetstrokeopacity{0.300000}%
\pgfsetdash{}{0pt}%
\pgfpathmoveto{\pgfqpoint{4.338948in}{1.499081in}}%
\pgfpathlineto{\pgfqpoint{4.283109in}{1.540808in}}%
\pgfpathlineto{\pgfqpoint{4.215095in}{1.576456in}}%
\pgfpathlineto{\pgfqpoint{4.215095in}{1.576456in}}%
\pgfpathlineto{\pgfqpoint{4.154204in}{1.595322in}}%
\pgfpathlineto{\pgfqpoint{4.084985in}{1.604090in}}%
\pgfpathlineto{\pgfqpoint{4.019465in}{1.603038in}}%
\pgfpathlineto{\pgfqpoint{3.948068in}{1.595077in}}%
\pgfusepath{stroke}%
\end{pgfscope}%
\begin{pgfscope}%
\pgfpathrectangle{\pgfqpoint{0.647939in}{0.492442in}}{\pgfqpoint{4.273799in}{2.331163in}}%
\pgfusepath{clip}%
\pgfsetbuttcap%
\pgfsetroundjoin%
\pgfsetlinewidth{0.301125pt}%
\definecolor{currentstroke}{rgb}{0.500000,0.500000,0.500000}%
\pgfsetstrokecolor{currentstroke}%
\pgfsetstrokeopacity{0.300000}%
\pgfsetdash{}{0pt}%
\pgfpathmoveto{\pgfqpoint{4.293530in}{2.256499in}}%
\pgfpathlineto{\pgfqpoint{4.315466in}{2.206115in}}%
\pgfpathlineto{\pgfqpoint{4.334429in}{2.155427in}}%
\pgfpathlineto{\pgfqpoint{4.342993in}{2.118732in}}%
\pgfpathlineto{\pgfqpoint{4.338948in}{2.081872in}}%
\pgfpathlineto{\pgfqpoint{4.338948in}{2.081872in}}%
\pgfpathlineto{\pgfqpoint{4.338948in}{2.081872in}}%
\pgfpathlineto{\pgfqpoint{4.318005in}{2.050410in}}%
\pgfpathlineto{\pgfqpoint{4.290754in}{2.018542in}}%
\pgfusepath{stroke}%
\end{pgfscope}%
\begin{pgfscope}%
\pgfpathrectangle{\pgfqpoint{0.647939in}{0.492442in}}{\pgfqpoint{4.273799in}{2.331163in}}%
\pgfusepath{clip}%
\pgfsetbuttcap%
\pgfsetroundjoin%
\pgfsetlinewidth{0.301125pt}%
\definecolor{currentstroke}{rgb}{0.500000,0.500000,0.500000}%
\pgfsetstrokecolor{currentstroke}%
\pgfsetstrokeopacity{0.300000}%
\pgfsetdash{}{0pt}%
\pgfpathmoveto{\pgfqpoint{1.716389in}{2.505719in}}%
\pgfpathlineto{\pgfqpoint{1.769496in}{2.462942in}}%
\pgfpathlineto{\pgfqpoint{1.837043in}{2.427326in}}%
\pgfpathlineto{\pgfqpoint{1.837043in}{2.427326in}}%
\pgfpathlineto{\pgfqpoint{1.888798in}{2.413331in}}%
\pgfpathlineto{\pgfqpoint{1.947476in}{2.409075in}}%
\pgfpathlineto{\pgfqpoint{2.003258in}{2.412850in}}%
\pgfpathlineto{\pgfqpoint{2.069827in}{2.423026in}}%
\pgfusepath{stroke}%
\end{pgfscope}%
\begin{pgfscope}%
\pgfpathrectangle{\pgfqpoint{0.647939in}{0.492442in}}{\pgfqpoint{4.273799in}{2.331163in}}%
\pgfusepath{clip}%
\pgfsetbuttcap%
\pgfsetroundjoin%
\pgfsetlinewidth{0.301125pt}%
\definecolor{currentstroke}{rgb}{0.500000,0.500000,0.500000}%
\pgfsetstrokecolor{currentstroke}%
\pgfsetstrokeopacity{0.300000}%
\pgfsetdash{}{0pt}%
\pgfpathmoveto{\pgfqpoint{1.810870in}{1.211398in}}%
\pgfpathlineto{\pgfqpoint{1.770456in}{1.258275in}}%
\pgfpathlineto{\pgfqpoint{1.728185in}{1.304658in}}%
\pgfpathlineto{\pgfqpoint{1.683168in}{1.350259in}}%
\pgfpathlineto{\pgfqpoint{1.633879in}{1.394493in}}%
\pgfpathlineto{\pgfqpoint{1.582741in}{1.432581in}}%
\pgfpathlineto{\pgfqpoint{1.537946in}{1.458196in}}%
\pgfpathlineto{\pgfqpoint{1.496375in}{1.474577in}}%
\pgfpathlineto{\pgfqpoint{1.451738in}{1.483493in}}%
\pgfpathlineto{\pgfqpoint{1.404527in}{1.482698in}}%
\pgfpathlineto{\pgfqpoint{1.404527in}{1.482698in}}%
\pgfpathlineto{\pgfqpoint{1.354138in}{1.470202in}}%
\pgfpathlineto{\pgfqpoint{1.354138in}{1.470202in}}%
\pgfpathlineto{\pgfqpoint{1.285895in}{1.435017in}}%
\pgfpathlineto{\pgfqpoint{1.230730in}{1.393119in}}%
\pgfusepath{stroke}%
\end{pgfscope}%
\begin{pgfscope}%
\pgfpathrectangle{\pgfqpoint{0.647939in}{0.492442in}}{\pgfqpoint{4.273799in}{2.331163in}}%
\pgfusepath{clip}%
\pgfsetbuttcap%
\pgfsetroundjoin%
\pgfsetlinewidth{0.301125pt}%
\definecolor{currentstroke}{rgb}{0.500000,0.500000,0.500000}%
\pgfsetstrokecolor{currentstroke}%
\pgfsetstrokeopacity{0.300000}%
\pgfsetdash{}{0pt}%
\pgfpathmoveto{\pgfqpoint{2.104916in}{0.863309in}}%
\pgfpathlineto{\pgfqpoint{2.068414in}{0.911133in}}%
\pgfpathlineto{\pgfqpoint{2.031938in}{0.958964in}}%
\pgfpathlineto{\pgfqpoint{1.995398in}{1.006780in}}%
\pgfpathlineto{\pgfqpoint{1.958687in}{1.054557in}}%
\pgfpathlineto{\pgfqpoint{1.921682in}{1.102266in}}%
\pgfpathlineto{\pgfqpoint{1.884224in}{1.149869in}}%
\pgfusepath{stroke}%
\end{pgfscope}%
\begin{pgfscope}%
\pgfpathrectangle{\pgfqpoint{0.647939in}{0.492442in}}{\pgfqpoint{4.273799in}{2.331163in}}%
\pgfusepath{clip}%
\pgfsetbuttcap%
\pgfsetroundjoin%
\pgfsetlinewidth{0.301125pt}%
\definecolor{currentstroke}{rgb}{0.500000,0.500000,0.500000}%
\pgfsetstrokecolor{currentstroke}%
\pgfsetstrokeopacity{0.300000}%
\pgfsetdash{}{0pt}%
\pgfpathmoveto{\pgfqpoint{4.241816in}{1.022252in}}%
\pgfpathlineto{\pgfqpoint{4.183566in}{1.063128in}}%
\pgfpathlineto{\pgfqpoint{4.120168in}{1.101640in}}%
\pgfpathlineto{\pgfqpoint{4.051227in}{1.137190in}}%
\pgfpathlineto{\pgfqpoint{3.976791in}{1.169270in}}%
\pgfpathlineto{\pgfqpoint{3.897534in}{1.197724in}}%
\pgfusepath{stroke}%
\end{pgfscope}%
\begin{pgfscope}%
\pgfpathrectangle{\pgfqpoint{0.647939in}{0.492442in}}{\pgfqpoint{4.273799in}{2.331163in}}%
\pgfusepath{clip}%
\pgfsetbuttcap%
\pgfsetroundjoin%
\pgfsetlinewidth{0.301125pt}%
\definecolor{currentstroke}{rgb}{0.500000,0.500000,0.500000}%
\pgfsetstrokecolor{currentstroke}%
\pgfsetstrokeopacity{0.300000}%
\pgfsetdash{}{0pt}%
\pgfpathmoveto{\pgfqpoint{4.353045in}{1.362555in}}%
\pgfpathlineto{\pgfqpoint{4.301347in}{1.405928in}}%
\pgfpathlineto{\pgfqpoint{4.241816in}{1.446100in}}%
\pgfpathlineto{\pgfqpoint{4.171609in}{1.480594in}}%
\pgfpathlineto{\pgfqpoint{4.089150in}{1.505265in}}%
\pgfpathlineto{\pgfqpoint{4.009907in}{1.516175in}}%
\pgfpathlineto{\pgfqpoint{3.930739in}{1.518336in}}%
\pgfpathlineto{\pgfqpoint{3.838490in}{1.514608in}}%
\pgfpathlineto{\pgfqpoint{3.744234in}{1.508496in}}%
\pgfpathlineto{\pgfqpoint{3.649732in}{1.503813in}}%
\pgfpathlineto{\pgfqpoint{3.554958in}{1.503359in}}%
\pgfpathlineto{\pgfqpoint{3.460872in}{1.509118in}}%
\pgfusepath{stroke}%
\end{pgfscope}%
\begin{pgfscope}%
\pgfpathrectangle{\pgfqpoint{0.647939in}{0.492442in}}{\pgfqpoint{4.273799in}{2.331163in}}%
\pgfusepath{clip}%
\pgfsetbuttcap%
\pgfsetroundjoin%
\pgfsetlinewidth{0.301125pt}%
\definecolor{currentstroke}{rgb}{0.500000,0.500000,0.500000}%
\pgfsetstrokecolor{currentstroke}%
\pgfsetstrokeopacity{0.300000}%
\pgfsetdash{}{0pt}%
\pgfpathmoveto{\pgfqpoint{4.459876in}{1.527624in}}%
\pgfpathlineto{\pgfqpoint{4.422073in}{1.575109in}}%
\pgfpathlineto{\pgfqpoint{4.379500in}{1.621336in}}%
\pgfpathlineto{\pgfqpoint{4.334066in}{1.661338in}}%
\pgfpathlineto{\pgfqpoint{4.293583in}{1.688396in}}%
\pgfpathlineto{\pgfqpoint{4.241816in}{1.711005in}}%
\pgfpathlineto{\pgfqpoint{4.241816in}{1.711005in}}%
\pgfpathlineto{\pgfqpoint{4.241816in}{1.711005in}}%
\pgfpathlineto{\pgfqpoint{4.192659in}{1.720903in}}%
\pgfpathlineto{\pgfqpoint{4.139428in}{1.720936in}}%
\pgfpathlineto{\pgfqpoint{4.088776in}{1.713301in}}%
\pgfpathlineto{\pgfqpoint{4.030332in}{1.698301in}}%
\pgfusepath{stroke}%
\end{pgfscope}%
\begin{pgfscope}%
\pgfpathrectangle{\pgfqpoint{0.647939in}{0.492442in}}{\pgfqpoint{4.273799in}{2.331163in}}%
\pgfusepath{clip}%
\pgfsetbuttcap%
\pgfsetroundjoin%
\pgfsetlinewidth{0.301125pt}%
\definecolor{currentstroke}{rgb}{0.500000,0.500000,0.500000}%
\pgfsetstrokecolor{currentstroke}%
\pgfsetstrokeopacity{0.300000}%
\pgfsetdash{}{0pt}%
\pgfpathmoveto{\pgfqpoint{1.327862in}{2.293796in}}%
\pgfpathlineto{\pgfqpoint{1.335371in}{2.242158in}}%
\pgfpathlineto{\pgfqpoint{1.340835in}{2.190444in}}%
\pgfpathlineto{\pgfqpoint{1.343893in}{2.138672in}}%
\pgfpathlineto{\pgfqpoint{1.344178in}{2.086876in}}%
\pgfpathlineto{\pgfqpoint{1.341347in}{2.035105in}}%
\pgfpathlineto{\pgfqpoint{1.335136in}{1.983423in}}%
\pgfusepath{stroke}%
\end{pgfscope}%
\begin{pgfscope}%
\pgfpathrectangle{\pgfqpoint{0.647939in}{0.492442in}}{\pgfqpoint{4.273799in}{2.331163in}}%
\pgfusepath{clip}%
\pgfsetbuttcap%
\pgfsetroundjoin%
\pgfsetlinewidth{0.301125pt}%
\definecolor{currentstroke}{rgb}{0.500000,0.500000,0.500000}%
\pgfsetstrokecolor{currentstroke}%
\pgfsetstrokeopacity{0.300000}%
\pgfsetdash{}{0pt}%
\pgfpathmoveto{\pgfqpoint{4.203440in}{0.875621in}}%
\pgfpathlineto{\pgfqpoint{4.144684in}{0.916290in}}%
\pgfpathlineto{\pgfqpoint{4.081809in}{0.955083in}}%
\pgfpathlineto{\pgfqpoint{4.014753in}{0.991727in}}%
\pgfpathlineto{\pgfqpoint{3.943701in}{1.026064in}}%
\pgfpathlineto{\pgfqpoint{3.869198in}{1.058158in}}%
\pgfusepath{stroke}%
\end{pgfscope}%
\begin{pgfscope}%
\pgfpathrectangle{\pgfqpoint{0.647939in}{0.492442in}}{\pgfqpoint{4.273799in}{2.331163in}}%
\pgfusepath{clip}%
\pgfsetbuttcap%
\pgfsetroundjoin%
\pgfsetlinewidth{0.301125pt}%
\definecolor{currentstroke}{rgb}{0.500000,0.500000,0.500000}%
\pgfsetstrokecolor{currentstroke}%
\pgfsetstrokeopacity{0.300000}%
\pgfsetdash{}{0pt}%
\pgfpathmoveto{\pgfqpoint{1.558843in}{2.497945in}}%
\pgfpathlineto{\pgfqpoint{1.588184in}{2.448685in}}%
\pgfpathlineto{\pgfqpoint{1.619257in}{2.399758in}}%
\pgfpathlineto{\pgfqpoint{1.653106in}{2.351401in}}%
\pgfpathlineto{\pgfqpoint{1.692314in}{2.304320in}}%
\pgfpathlineto{\pgfqpoint{1.745658in}{2.262473in}}%
\pgfpathlineto{\pgfqpoint{1.745658in}{2.262473in}}%
\pgfusepath{stroke}%
\end{pgfscope}%
\begin{pgfscope}%
\pgfpathrectangle{\pgfqpoint{0.647939in}{0.492442in}}{\pgfqpoint{4.273799in}{2.331163in}}%
\pgfusepath{clip}%
\pgfsetbuttcap%
\pgfsetroundjoin%
\pgfsetlinewidth{0.301125pt}%
\definecolor{currentstroke}{rgb}{0.500000,0.500000,0.500000}%
\pgfsetstrokecolor{currentstroke}%
\pgfsetstrokeopacity{0.300000}%
\pgfsetdash{}{0pt}%
\pgfpathmoveto{\pgfqpoint{3.250072in}{2.397361in}}%
\pgfpathlineto{\pgfqpoint{3.270498in}{2.346777in}}%
\pgfpathlineto{\pgfqpoint{3.288026in}{2.295871in}}%
\pgfpathlineto{\pgfqpoint{3.302301in}{2.244667in}}%
\pgfpathlineto{\pgfqpoint{3.312848in}{2.193198in}}%
\pgfpathlineto{\pgfqpoint{3.319008in}{2.141521in}}%
\pgfusepath{stroke}%
\end{pgfscope}%
\begin{pgfscope}%
\pgfpathrectangle{\pgfqpoint{0.647939in}{0.492442in}}{\pgfqpoint{4.273799in}{2.331163in}}%
\pgfusepath{clip}%
\pgfsetbuttcap%
\pgfsetroundjoin%
\pgfsetlinewidth{0.301125pt}%
\definecolor{currentstroke}{rgb}{0.500000,0.500000,0.500000}%
\pgfsetstrokecolor{currentstroke}%
\pgfsetstrokeopacity{0.300000}%
\pgfsetdash{}{0pt}%
\pgfpathmoveto{\pgfqpoint{2.774678in}{1.293259in}}%
\pgfpathlineto{\pgfqpoint{2.733227in}{1.339864in}}%
\pgfpathlineto{\pgfqpoint{2.693414in}{1.386892in}}%
\pgfpathlineto{\pgfqpoint{2.655226in}{1.434319in}}%
\pgfpathlineto{\pgfqpoint{2.618658in}{1.482125in}}%
\pgfpathlineto{\pgfqpoint{2.583720in}{1.530293in}}%
\pgfpathlineto{\pgfqpoint{2.550440in}{1.578809in}}%
\pgfpathlineto{\pgfqpoint{2.518859in}{1.627662in}}%
\pgfpathlineto{\pgfqpoint{2.489045in}{1.676843in}}%
\pgfpathlineto{\pgfqpoint{2.461110in}{1.726351in}}%
\pgfpathlineto{\pgfqpoint{2.435191in}{1.776184in}}%
\pgfpathlineto{\pgfqpoint{2.411480in}{1.826342in}}%
\pgfpathlineto{\pgfqpoint{2.390247in}{1.876828in}}%
\pgfpathlineto{\pgfqpoint{2.371855in}{1.927644in}}%
\pgfpathlineto{\pgfqpoint{2.356797in}{1.978782in}}%
\pgfpathlineto{\pgfqpoint{2.345761in}{2.030218in}}%
\pgfpathlineto{\pgfqpoint{2.339724in}{2.081890in}}%
\pgfpathlineto{\pgfqpoint{2.340089in}{2.133649in}}%
\pgfpathlineto{\pgfqpoint{2.348934in}{2.185151in}}%
\pgfpathlineto{\pgfqpoint{2.368672in}{2.234319in}}%
\pgfpathlineto{\pgfqpoint{2.396975in}{2.273850in}}%
\pgfpathlineto{\pgfqpoint{2.432702in}{2.304716in}}%
\pgfpathlineto{\pgfqpoint{2.476137in}{2.327754in}}%
\pgfpathlineto{\pgfqpoint{2.529300in}{2.342607in}}%
\pgfpathlineto{\pgfqpoint{2.590575in}{2.346777in}}%
\pgfpathlineto{\pgfqpoint{2.590575in}{2.346777in}}%
\pgfpathlineto{\pgfqpoint{2.590575in}{2.346777in}}%
\pgfusepath{stroke}%
\end{pgfscope}%
\begin{pgfscope}%
\pgfpathrectangle{\pgfqpoint{0.647939in}{0.492442in}}{\pgfqpoint{4.273799in}{2.331163in}}%
\pgfusepath{clip}%
\pgfsetbuttcap%
\pgfsetroundjoin%
\pgfsetlinewidth{0.301125pt}%
\definecolor{currentstroke}{rgb}{0.500000,0.500000,0.500000}%
\pgfsetstrokecolor{currentstroke}%
\pgfsetstrokeopacity{0.300000}%
\pgfsetdash{}{0pt}%
\pgfpathmoveto{\pgfqpoint{1.938324in}{1.370780in}}%
\pgfpathlineto{\pgfqpoint{1.903772in}{1.419033in}}%
\pgfpathlineto{\pgfqpoint{1.868793in}{1.467194in}}%
\pgfpathlineto{\pgfqpoint{1.833119in}{1.515201in}}%
\pgfpathlineto{\pgfqpoint{1.796361in}{1.562960in}}%
\pgfpathlineto{\pgfqpoint{1.757885in}{1.610310in}}%
\pgfpathlineto{\pgfqpoint{1.716563in}{1.656923in}}%
\pgfpathlineto{\pgfqpoint{1.670060in}{1.701986in}}%
\pgfpathlineto{\pgfqpoint{1.631582in}{1.731553in}}%
\pgfpathlineto{\pgfqpoint{1.598317in}{1.750169in}}%
\pgfpathlineto{\pgfqpoint{1.564429in}{1.761463in}}%
\pgfpathlineto{\pgfqpoint{1.522125in}{1.763986in}}%
\pgfpathlineto{\pgfqpoint{1.522125in}{1.763986in}}%
\pgfpathlineto{\pgfqpoint{1.522125in}{1.763986in}}%
\pgfpathlineto{\pgfqpoint{1.480455in}{1.754176in}}%
\pgfusepath{stroke}%
\end{pgfscope}%
\begin{pgfscope}%
\pgfpathrectangle{\pgfqpoint{0.647939in}{0.492442in}}{\pgfqpoint{4.273799in}{2.331163in}}%
\pgfusepath{clip}%
\pgfsetbuttcap%
\pgfsetroundjoin%
\pgfsetlinewidth{0.301125pt}%
\definecolor{currentstroke}{rgb}{0.500000,0.500000,0.500000}%
\pgfsetstrokecolor{currentstroke}%
\pgfsetstrokeopacity{0.300000}%
\pgfsetdash{}{0pt}%
\pgfpathmoveto{\pgfqpoint{3.173366in}{1.022252in}}%
\pgfpathlineto{\pgfqpoint{3.116709in}{1.063818in}}%
\pgfpathlineto{\pgfqpoint{3.062068in}{1.106181in}}%
\pgfpathlineto{\pgfqpoint{3.009452in}{1.149300in}}%
\pgfpathlineto{\pgfqpoint{2.958847in}{1.193129in}}%
\pgfpathlineto{\pgfqpoint{2.910221in}{1.237619in}}%
\pgfpathlineto{\pgfqpoint{2.863538in}{1.282725in}}%
\pgfpathlineto{\pgfqpoint{2.818757in}{1.328402in}}%
\pgfpathlineto{\pgfqpoint{2.775843in}{1.374609in}}%
\pgfusepath{stroke}%
\end{pgfscope}%
\begin{pgfscope}%
\pgfpathrectangle{\pgfqpoint{0.647939in}{0.492442in}}{\pgfqpoint{4.273799in}{2.331163in}}%
\pgfusepath{clip}%
\pgfsetbuttcap%
\pgfsetroundjoin%
\pgfsetlinewidth{0.301125pt}%
\definecolor{currentstroke}{rgb}{0.500000,0.500000,0.500000}%
\pgfsetstrokecolor{currentstroke}%
\pgfsetstrokeopacity{0.300000}%
\pgfsetdash{}{0pt}%
\pgfpathmoveto{\pgfqpoint{3.464761in}{2.293796in}}%
\pgfpathlineto{\pgfqpoint{3.476412in}{2.242391in}}%
\pgfpathlineto{\pgfqpoint{3.484888in}{2.190803in}}%
\pgfpathlineto{\pgfqpoint{3.489686in}{2.139078in}}%
\pgfpathlineto{\pgfqpoint{3.490150in}{2.087293in}}%
\pgfpathlineto{\pgfqpoint{3.485401in}{2.035582in}}%
\pgfpathlineto{\pgfqpoint{3.474230in}{1.984183in}}%
\pgfpathlineto{\pgfqpoint{3.454934in}{1.933532in}}%
\pgfpathlineto{\pgfqpoint{3.424997in}{1.884504in}}%
\pgfpathlineto{\pgfqpoint{3.380682in}{1.839001in}}%
\pgfpathlineto{\pgfqpoint{3.380682in}{1.839001in}}%
\pgfpathlineto{\pgfqpoint{3.331860in}{1.807864in}}%
\pgfpathlineto{\pgfqpoint{3.331860in}{1.807864in}}%
\pgfpathlineto{\pgfqpoint{3.279380in}{1.788168in}}%
\pgfusepath{stroke}%
\end{pgfscope}%
\begin{pgfscope}%
\pgfpathrectangle{\pgfqpoint{0.647939in}{0.492442in}}{\pgfqpoint{4.273799in}{2.331163in}}%
\pgfusepath{clip}%
\pgfsetbuttcap%
\pgfsetroundjoin%
\pgfsetlinewidth{0.301125pt}%
\definecolor{currentstroke}{rgb}{0.500000,0.500000,0.500000}%
\pgfsetstrokecolor{currentstroke}%
\pgfsetstrokeopacity{0.300000}%
\pgfsetdash{}{0pt}%
\pgfpathmoveto{\pgfqpoint{2.909506in}{2.390093in}}%
\pgfpathlineto{\pgfqpoint{2.946786in}{2.342480in}}%
\pgfpathlineto{\pgfqpoint{2.979102in}{2.293796in}}%
\pgfpathlineto{\pgfqpoint{3.005993in}{2.244147in}}%
\pgfpathlineto{\pgfqpoint{3.026619in}{2.193627in}}%
\pgfpathlineto{\pgfqpoint{3.039437in}{2.142364in}}%
\pgfpathlineto{\pgfqpoint{3.041233in}{2.090710in}}%
\pgfpathlineto{\pgfqpoint{3.041233in}{2.090710in}}%
\pgfpathlineto{\pgfqpoint{3.028336in}{2.047607in}}%
\pgfpathlineto{\pgfqpoint{3.028336in}{2.047607in}}%
\pgfpathlineto{\pgfqpoint{3.007318in}{2.023025in}}%
\pgfpathlineto{\pgfqpoint{3.007318in}{2.023025in}}%
\pgfpathlineto{\pgfqpoint{2.981367in}{2.010579in}}%
\pgfpathlineto{\pgfqpoint{2.981367in}{2.010579in}}%
\pgfpathlineto{\pgfqpoint{2.952694in}{2.007967in}}%
\pgfpathlineto{\pgfqpoint{2.924916in}{2.012989in}}%
\pgfpathlineto{\pgfqpoint{2.896602in}{2.024852in}}%
\pgfpathlineto{\pgfqpoint{2.868473in}{2.044080in}}%
\pgfusepath{stroke}%
\end{pgfscope}%
\begin{pgfscope}%
\pgfpathrectangle{\pgfqpoint{0.647939in}{0.492442in}}{\pgfqpoint{4.273799in}{2.331163in}}%
\pgfusepath{clip}%
\pgfsetbuttcap%
\pgfsetroundjoin%
\pgfsetlinewidth{0.301125pt}%
\definecolor{currentstroke}{rgb}{0.500000,0.500000,0.500000}%
\pgfsetstrokecolor{currentstroke}%
\pgfsetstrokeopacity{0.300000}%
\pgfsetdash{}{0pt}%
\pgfpathmoveto{\pgfqpoint{1.669067in}{1.190113in}}%
\pgfpathlineto{\pgfqpoint{1.619257in}{1.234176in}}%
\pgfpathlineto{\pgfqpoint{1.563494in}{1.276007in}}%
\pgfpathlineto{\pgfqpoint{1.497695in}{1.312989in}}%
\pgfpathlineto{\pgfqpoint{1.497695in}{1.312989in}}%
\pgfpathlineto{\pgfqpoint{1.442653in}{1.331983in}}%
\pgfpathlineto{\pgfqpoint{1.442653in}{1.331983in}}%
\pgfpathlineto{\pgfqpoint{1.392574in}{1.338806in}}%
\pgfpathlineto{\pgfqpoint{1.340768in}{1.334891in}}%
\pgfpathlineto{\pgfqpoint{1.296959in}{1.322932in}}%
\pgfusepath{stroke}%
\end{pgfscope}%
\begin{pgfscope}%
\pgfpathrectangle{\pgfqpoint{0.647939in}{0.492442in}}{\pgfqpoint{4.273799in}{2.331163in}}%
\pgfusepath{clip}%
\pgfsetbuttcap%
\pgfsetroundjoin%
\pgfsetlinewidth{0.301125pt}%
\definecolor{currentstroke}{rgb}{0.500000,0.500000,0.500000}%
\pgfsetstrokecolor{currentstroke}%
\pgfsetstrokeopacity{0.300000}%
\pgfsetdash{}{0pt}%
\pgfpathmoveto{\pgfqpoint{3.441852in}{1.201881in}}%
\pgfpathlineto{\pgfqpoint{3.367630in}{1.234176in}}%
\pgfpathlineto{\pgfqpoint{3.296156in}{1.268256in}}%
\pgfpathlineto{\pgfqpoint{3.227785in}{1.304172in}}%
\pgfpathlineto{\pgfqpoint{3.162697in}{1.341862in}}%
\pgfpathlineto{\pgfqpoint{3.100971in}{1.381201in}}%
\pgfusepath{stroke}%
\end{pgfscope}%
\begin{pgfscope}%
\pgfpathrectangle{\pgfqpoint{0.647939in}{0.492442in}}{\pgfqpoint{4.273799in}{2.331163in}}%
\pgfusepath{clip}%
\pgfsetbuttcap%
\pgfsetroundjoin%
\pgfsetlinewidth{0.301125pt}%
\definecolor{currentstroke}{rgb}{0.500000,0.500000,0.500000}%
\pgfsetstrokecolor{currentstroke}%
\pgfsetstrokeopacity{0.300000}%
\pgfsetdash{}{0pt}%
\pgfpathmoveto{\pgfqpoint{3.270498in}{2.081872in}}%
\pgfpathlineto{\pgfqpoint{3.262561in}{2.030318in}}%
\pgfpathlineto{\pgfqpoint{3.243667in}{1.979694in}}%
\pgfpathlineto{\pgfqpoint{3.207935in}{1.932121in}}%
\pgfpathlineto{\pgfqpoint{3.207935in}{1.932121in}}%
\pgfpathlineto{\pgfqpoint{3.170459in}{1.905440in}}%
\pgfpathlineto{\pgfqpoint{3.170459in}{1.905440in}}%
\pgfpathlineto{\pgfqpoint{3.129263in}{1.890304in}}%
\pgfpathlineto{\pgfqpoint{3.080460in}{1.885143in}}%
\pgfpathlineto{\pgfqpoint{3.035910in}{1.889610in}}%
\pgfusepath{stroke}%
\end{pgfscope}%
\begin{pgfscope}%
\pgfpathrectangle{\pgfqpoint{0.647939in}{0.492442in}}{\pgfqpoint{4.273799in}{2.331163in}}%
\pgfusepath{clip}%
\pgfsetbuttcap%
\pgfsetroundjoin%
\pgfsetlinewidth{0.301125pt}%
\definecolor{currentstroke}{rgb}{0.500000,0.500000,0.500000}%
\pgfsetstrokecolor{currentstroke}%
\pgfsetstrokeopacity{0.300000}%
\pgfsetdash{}{0pt}%
\pgfpathmoveto{\pgfqpoint{2.519319in}{1.462018in}}%
\pgfpathlineto{\pgfqpoint{2.486410in}{1.510610in}}%
\pgfpathlineto{\pgfqpoint{2.454902in}{1.559477in}}%
\pgfpathlineto{\pgfqpoint{2.424848in}{1.608616in}}%
\pgfpathlineto{\pgfqpoint{2.396312in}{1.658024in}}%
\pgfpathlineto{\pgfqpoint{2.369396in}{1.707701in}}%
\pgfusepath{stroke}%
\end{pgfscope}%
\begin{pgfscope}%
\pgfpathrectangle{\pgfqpoint{0.647939in}{0.492442in}}{\pgfqpoint{4.273799in}{2.331163in}}%
\pgfusepath{clip}%
\pgfsetroundcap%
\pgfsetroundjoin%
\pgfsetlinewidth{0.301125pt}%
\definecolor{currentstroke}{rgb}{0.500000,0.500000,0.500000}%
\pgfsetstrokecolor{currentstroke}%
\pgfsetstrokeopacity{0.300000}%
\pgfsetdash{}{0pt}%
\pgfpathmoveto{\pgfqpoint{1.482492in}{1.392290in}}%
\pgfusepath{stroke}%
\end{pgfscope}%
\begin{pgfscope}%
\pgfpathrectangle{\pgfqpoint{0.647939in}{0.492442in}}{\pgfqpoint{4.273799in}{2.331163in}}%
\pgfusepath{clip}%
\pgfsetroundcap%
\pgfsetroundjoin%
\definecolor{currentfill}{rgb}{0.500000,0.500000,0.500000}%
\pgfsetfillcolor{currentfill}%
\pgfsetfillopacity{0.300000}%
\pgfsetlinewidth{0.301125pt}%
\definecolor{currentstroke}{rgb}{0.500000,0.500000,0.500000}%
\pgfsetstrokecolor{currentstroke}%
\pgfsetstrokeopacity{0.300000}%
\pgfsetdash{}{0pt}%
\pgfpathmoveto{\pgfqpoint{0.000000in}{0.000000in}}%
\pgfpathlineto{\pgfqpoint{0.000000in}{0.000000in}}%
\pgfpathclose%
\pgfusepath{stroke,fill}%
\end{pgfscope}%
\begin{pgfscope}%
\pgfpathrectangle{\pgfqpoint{0.647939in}{0.492442in}}{\pgfqpoint{4.273799in}{2.331163in}}%
\pgfusepath{clip}%
\pgfsetroundcap%
\pgfsetroundjoin%
\pgfsetlinewidth{0.301125pt}%
\definecolor{currentstroke}{rgb}{0.500000,0.500000,0.500000}%
\pgfsetstrokecolor{currentstroke}%
\pgfsetstrokeopacity{0.300000}%
\pgfsetdash{}{0pt}%
\pgfpathmoveto{\pgfqpoint{1.312033in}{0.906605in}}%
\pgfusepath{stroke}%
\end{pgfscope}%
\begin{pgfscope}%
\pgfpathrectangle{\pgfqpoint{0.647939in}{0.492442in}}{\pgfqpoint{4.273799in}{2.331163in}}%
\pgfusepath{clip}%
\pgfsetroundcap%
\pgfsetroundjoin%
\definecolor{currentfill}{rgb}{0.500000,0.500000,0.500000}%
\pgfsetfillcolor{currentfill}%
\pgfsetfillopacity{0.300000}%
\pgfsetlinewidth{0.301125pt}%
\definecolor{currentstroke}{rgb}{0.500000,0.500000,0.500000}%
\pgfsetstrokecolor{currentstroke}%
\pgfsetstrokeopacity{0.300000}%
\pgfsetdash{}{0pt}%
\pgfpathmoveto{\pgfqpoint{0.000000in}{0.000000in}}%
\pgfpathlineto{\pgfqpoint{0.000000in}{0.000000in}}%
\pgfpathclose%
\pgfusepath{stroke,fill}%
\end{pgfscope}%
\begin{pgfscope}%
\pgfpathrectangle{\pgfqpoint{0.647939in}{0.492442in}}{\pgfqpoint{4.273799in}{2.331163in}}%
\pgfusepath{clip}%
\pgfsetroundcap%
\pgfsetroundjoin%
\pgfsetlinewidth{0.301125pt}%
\definecolor{currentstroke}{rgb}{0.500000,0.500000,0.500000}%
\pgfsetstrokecolor{currentstroke}%
\pgfsetstrokeopacity{0.300000}%
\pgfsetdash{}{0pt}%
\pgfpathmoveto{\pgfqpoint{1.219587in}{0.695901in}}%
\pgfusepath{stroke}%
\end{pgfscope}%
\begin{pgfscope}%
\pgfpathrectangle{\pgfqpoint{0.647939in}{0.492442in}}{\pgfqpoint{4.273799in}{2.331163in}}%
\pgfusepath{clip}%
\pgfsetroundcap%
\pgfsetroundjoin%
\definecolor{currentfill}{rgb}{0.500000,0.500000,0.500000}%
\pgfsetfillcolor{currentfill}%
\pgfsetfillopacity{0.300000}%
\pgfsetlinewidth{0.301125pt}%
\definecolor{currentstroke}{rgb}{0.500000,0.500000,0.500000}%
\pgfsetstrokecolor{currentstroke}%
\pgfsetstrokeopacity{0.300000}%
\pgfsetdash{}{0pt}%
\pgfpathmoveto{\pgfqpoint{0.000000in}{0.000000in}}%
\pgfpathlineto{\pgfqpoint{0.000000in}{0.000000in}}%
\pgfpathclose%
\pgfusepath{stroke,fill}%
\end{pgfscope}%
\begin{pgfscope}%
\pgfpathrectangle{\pgfqpoint{0.647939in}{0.492442in}}{\pgfqpoint{4.273799in}{2.331163in}}%
\pgfusepath{clip}%
\pgfsetroundcap%
\pgfsetroundjoin%
\pgfsetlinewidth{0.301125pt}%
\definecolor{currentstroke}{rgb}{0.500000,0.500000,0.500000}%
\pgfsetstrokecolor{currentstroke}%
\pgfsetstrokeopacity{0.300000}%
\pgfsetdash{}{0pt}%
\pgfpathmoveto{\pgfqpoint{1.178543in}{0.578089in}}%
\pgfusepath{stroke}%
\end{pgfscope}%
\begin{pgfscope}%
\pgfpathrectangle{\pgfqpoint{0.647939in}{0.492442in}}{\pgfqpoint{4.273799in}{2.331163in}}%
\pgfusepath{clip}%
\pgfsetroundcap%
\pgfsetroundjoin%
\definecolor{currentfill}{rgb}{0.500000,0.500000,0.500000}%
\pgfsetfillcolor{currentfill}%
\pgfsetfillopacity{0.300000}%
\pgfsetlinewidth{0.301125pt}%
\definecolor{currentstroke}{rgb}{0.500000,0.500000,0.500000}%
\pgfsetstrokecolor{currentstroke}%
\pgfsetstrokeopacity{0.300000}%
\pgfsetdash{}{0pt}%
\pgfpathmoveto{\pgfqpoint{0.000000in}{0.000000in}}%
\pgfpathlineto{\pgfqpoint{0.000000in}{0.000000in}}%
\pgfpathclose%
\pgfusepath{stroke,fill}%
\end{pgfscope}%
\begin{pgfscope}%
\pgfpathrectangle{\pgfqpoint{0.647939in}{0.492442in}}{\pgfqpoint{4.273799in}{2.331163in}}%
\pgfusepath{clip}%
\pgfsetroundcap%
\pgfsetroundjoin%
\pgfsetlinewidth{0.301125pt}%
\definecolor{currentstroke}{rgb}{0.500000,0.500000,0.500000}%
\pgfsetstrokecolor{currentstroke}%
\pgfsetstrokeopacity{0.300000}%
\pgfsetdash{}{0pt}%
\pgfpathmoveto{\pgfqpoint{1.433862in}{0.722173in}}%
\pgfusepath{stroke}%
\end{pgfscope}%
\begin{pgfscope}%
\pgfpathrectangle{\pgfqpoint{0.647939in}{0.492442in}}{\pgfqpoint{4.273799in}{2.331163in}}%
\pgfusepath{clip}%
\pgfsetroundcap%
\pgfsetroundjoin%
\definecolor{currentfill}{rgb}{0.500000,0.500000,0.500000}%
\pgfsetfillcolor{currentfill}%
\pgfsetfillopacity{0.300000}%
\pgfsetlinewidth{0.301125pt}%
\definecolor{currentstroke}{rgb}{0.500000,0.500000,0.500000}%
\pgfsetstrokecolor{currentstroke}%
\pgfsetstrokeopacity{0.300000}%
\pgfsetdash{}{0pt}%
\pgfpathmoveto{\pgfqpoint{0.000000in}{0.000000in}}%
\pgfpathlineto{\pgfqpoint{0.000000in}{0.000000in}}%
\pgfpathclose%
\pgfusepath{stroke,fill}%
\end{pgfscope}%
\begin{pgfscope}%
\pgfpathrectangle{\pgfqpoint{0.647939in}{0.492442in}}{\pgfqpoint{4.273799in}{2.331163in}}%
\pgfusepath{clip}%
\pgfsetroundcap%
\pgfsetroundjoin%
\pgfsetlinewidth{0.301125pt}%
\definecolor{currentstroke}{rgb}{0.500000,0.500000,0.500000}%
\pgfsetstrokecolor{currentstroke}%
\pgfsetstrokeopacity{0.300000}%
\pgfsetdash{}{0pt}%
\pgfpathmoveto{\pgfqpoint{1.810529in}{0.606651in}}%
\pgfusepath{stroke}%
\end{pgfscope}%
\begin{pgfscope}%
\pgfpathrectangle{\pgfqpoint{0.647939in}{0.492442in}}{\pgfqpoint{4.273799in}{2.331163in}}%
\pgfusepath{clip}%
\pgfsetroundcap%
\pgfsetroundjoin%
\definecolor{currentfill}{rgb}{0.500000,0.500000,0.500000}%
\pgfsetfillcolor{currentfill}%
\pgfsetfillopacity{0.300000}%
\pgfsetlinewidth{0.301125pt}%
\definecolor{currentstroke}{rgb}{0.500000,0.500000,0.500000}%
\pgfsetstrokecolor{currentstroke}%
\pgfsetstrokeopacity{0.300000}%
\pgfsetdash{}{0pt}%
\pgfpathmoveto{\pgfqpoint{0.000000in}{0.000000in}}%
\pgfpathlineto{\pgfqpoint{0.000000in}{0.000000in}}%
\pgfpathclose%
\pgfusepath{stroke,fill}%
\end{pgfscope}%
\begin{pgfscope}%
\pgfpathrectangle{\pgfqpoint{0.647939in}{0.492442in}}{\pgfqpoint{4.273799in}{2.331163in}}%
\pgfusepath{clip}%
\pgfsetroundcap%
\pgfsetroundjoin%
\pgfsetlinewidth{0.301125pt}%
\definecolor{currentstroke}{rgb}{0.500000,0.500000,0.500000}%
\pgfsetstrokecolor{currentstroke}%
\pgfsetstrokeopacity{0.300000}%
\pgfsetdash{}{0pt}%
\pgfpathmoveto{\pgfqpoint{1.495774in}{1.008554in}}%
\pgfusepath{stroke}%
\end{pgfscope}%
\begin{pgfscope}%
\pgfpathrectangle{\pgfqpoint{0.647939in}{0.492442in}}{\pgfqpoint{4.273799in}{2.331163in}}%
\pgfusepath{clip}%
\pgfsetroundcap%
\pgfsetroundjoin%
\definecolor{currentfill}{rgb}{0.500000,0.500000,0.500000}%
\pgfsetfillcolor{currentfill}%
\pgfsetfillopacity{0.300000}%
\pgfsetlinewidth{0.301125pt}%
\definecolor{currentstroke}{rgb}{0.500000,0.500000,0.500000}%
\pgfsetstrokecolor{currentstroke}%
\pgfsetstrokeopacity{0.300000}%
\pgfsetdash{}{0pt}%
\pgfpathmoveto{\pgfqpoint{0.000000in}{0.000000in}}%
\pgfpathlineto{\pgfqpoint{0.000000in}{0.000000in}}%
\pgfpathclose%
\pgfusepath{stroke,fill}%
\end{pgfscope}%
\begin{pgfscope}%
\pgfpathrectangle{\pgfqpoint{0.647939in}{0.492442in}}{\pgfqpoint{4.273799in}{2.331163in}}%
\pgfusepath{clip}%
\pgfsetroundcap%
\pgfsetroundjoin%
\pgfsetlinewidth{0.301125pt}%
\definecolor{currentstroke}{rgb}{0.500000,0.500000,0.500000}%
\pgfsetstrokecolor{currentstroke}%
\pgfsetstrokeopacity{0.300000}%
\pgfsetdash{}{0pt}%
\pgfpathmoveto{\pgfqpoint{1.636731in}{1.025462in}}%
\pgfusepath{stroke}%
\end{pgfscope}%
\begin{pgfscope}%
\pgfpathrectangle{\pgfqpoint{0.647939in}{0.492442in}}{\pgfqpoint{4.273799in}{2.331163in}}%
\pgfusepath{clip}%
\pgfsetroundcap%
\pgfsetroundjoin%
\definecolor{currentfill}{rgb}{0.500000,0.500000,0.500000}%
\pgfsetfillcolor{currentfill}%
\pgfsetfillopacity{0.300000}%
\pgfsetlinewidth{0.301125pt}%
\definecolor{currentstroke}{rgb}{0.500000,0.500000,0.500000}%
\pgfsetstrokecolor{currentstroke}%
\pgfsetstrokeopacity{0.300000}%
\pgfsetdash{}{0pt}%
\pgfpathmoveto{\pgfqpoint{0.000000in}{0.000000in}}%
\pgfpathlineto{\pgfqpoint{0.000000in}{0.000000in}}%
\pgfpathclose%
\pgfusepath{stroke,fill}%
\end{pgfscope}%
\begin{pgfscope}%
\pgfpathrectangle{\pgfqpoint{0.647939in}{0.492442in}}{\pgfqpoint{4.273799in}{2.331163in}}%
\pgfusepath{clip}%
\pgfsetroundcap%
\pgfsetroundjoin%
\pgfsetlinewidth{0.301125pt}%
\definecolor{currentstroke}{rgb}{0.500000,0.500000,0.500000}%
\pgfsetstrokecolor{currentstroke}%
\pgfsetstrokeopacity{0.300000}%
\pgfsetdash{}{0pt}%
\pgfpathmoveto{\pgfqpoint{1.758120in}{1.033320in}}%
\pgfusepath{stroke}%
\end{pgfscope}%
\begin{pgfscope}%
\pgfpathrectangle{\pgfqpoint{0.647939in}{0.492442in}}{\pgfqpoint{4.273799in}{2.331163in}}%
\pgfusepath{clip}%
\pgfsetroundcap%
\pgfsetroundjoin%
\definecolor{currentfill}{rgb}{0.500000,0.500000,0.500000}%
\pgfsetfillcolor{currentfill}%
\pgfsetfillopacity{0.300000}%
\pgfsetlinewidth{0.301125pt}%
\definecolor{currentstroke}{rgb}{0.500000,0.500000,0.500000}%
\pgfsetstrokecolor{currentstroke}%
\pgfsetstrokeopacity{0.300000}%
\pgfsetdash{}{0pt}%
\pgfpathmoveto{\pgfqpoint{0.000000in}{0.000000in}}%
\pgfpathlineto{\pgfqpoint{0.000000in}{0.000000in}}%
\pgfpathclose%
\pgfusepath{stroke,fill}%
\end{pgfscope}%
\begin{pgfscope}%
\pgfpathrectangle{\pgfqpoint{0.647939in}{0.492442in}}{\pgfqpoint{4.273799in}{2.331163in}}%
\pgfusepath{clip}%
\pgfsetroundcap%
\pgfsetroundjoin%
\pgfsetlinewidth{0.301125pt}%
\definecolor{currentstroke}{rgb}{0.500000,0.500000,0.500000}%
\pgfsetstrokecolor{currentstroke}%
\pgfsetstrokeopacity{0.300000}%
\pgfsetdash{}{0pt}%
\pgfpathmoveto{\pgfqpoint{2.302235in}{0.608949in}}%
\pgfusepath{stroke}%
\end{pgfscope}%
\begin{pgfscope}%
\pgfpathrectangle{\pgfqpoint{0.647939in}{0.492442in}}{\pgfqpoint{4.273799in}{2.331163in}}%
\pgfusepath{clip}%
\pgfsetroundcap%
\pgfsetroundjoin%
\definecolor{currentfill}{rgb}{0.500000,0.500000,0.500000}%
\pgfsetfillcolor{currentfill}%
\pgfsetfillopacity{0.300000}%
\pgfsetlinewidth{0.301125pt}%
\definecolor{currentstroke}{rgb}{0.500000,0.500000,0.500000}%
\pgfsetstrokecolor{currentstroke}%
\pgfsetstrokeopacity{0.300000}%
\pgfsetdash{}{0pt}%
\pgfpathmoveto{\pgfqpoint{0.000000in}{0.000000in}}%
\pgfpathlineto{\pgfqpoint{0.000000in}{0.000000in}}%
\pgfpathclose%
\pgfusepath{stroke,fill}%
\end{pgfscope}%
\begin{pgfscope}%
\pgfpathrectangle{\pgfqpoint{0.647939in}{0.492442in}}{\pgfqpoint{4.273799in}{2.331163in}}%
\pgfusepath{clip}%
\pgfsetroundcap%
\pgfsetroundjoin%
\pgfsetlinewidth{0.301125pt}%
\definecolor{currentstroke}{rgb}{0.500000,0.500000,0.500000}%
\pgfsetstrokecolor{currentstroke}%
\pgfsetstrokeopacity{0.300000}%
\pgfsetdash{}{0pt}%
\pgfpathmoveto{\pgfqpoint{1.887751in}{1.278516in}}%
\pgfusepath{stroke}%
\end{pgfscope}%
\begin{pgfscope}%
\pgfpathrectangle{\pgfqpoint{0.647939in}{0.492442in}}{\pgfqpoint{4.273799in}{2.331163in}}%
\pgfusepath{clip}%
\pgfsetroundcap%
\pgfsetroundjoin%
\definecolor{currentfill}{rgb}{0.500000,0.500000,0.500000}%
\pgfsetfillcolor{currentfill}%
\pgfsetfillopacity{0.300000}%
\pgfsetlinewidth{0.301125pt}%
\definecolor{currentstroke}{rgb}{0.500000,0.500000,0.500000}%
\pgfsetstrokecolor{currentstroke}%
\pgfsetstrokeopacity{0.300000}%
\pgfsetdash{}{0pt}%
\pgfpathmoveto{\pgfqpoint{0.000000in}{0.000000in}}%
\pgfpathlineto{\pgfqpoint{0.000000in}{0.000000in}}%
\pgfpathclose%
\pgfusepath{stroke,fill}%
\end{pgfscope}%
\begin{pgfscope}%
\pgfpathrectangle{\pgfqpoint{0.647939in}{0.492442in}}{\pgfqpoint{4.273799in}{2.331163in}}%
\pgfusepath{clip}%
\pgfsetroundcap%
\pgfsetroundjoin%
\pgfsetlinewidth{0.301125pt}%
\definecolor{currentstroke}{rgb}{0.500000,0.500000,0.500000}%
\pgfsetstrokecolor{currentstroke}%
\pgfsetstrokeopacity{0.300000}%
\pgfsetdash{}{0pt}%
\pgfpathmoveto{\pgfqpoint{2.453425in}{0.655009in}}%
\pgfusepath{stroke}%
\end{pgfscope}%
\begin{pgfscope}%
\pgfpathrectangle{\pgfqpoint{0.647939in}{0.492442in}}{\pgfqpoint{4.273799in}{2.331163in}}%
\pgfusepath{clip}%
\pgfsetroundcap%
\pgfsetroundjoin%
\definecolor{currentfill}{rgb}{0.500000,0.500000,0.500000}%
\pgfsetfillcolor{currentfill}%
\pgfsetfillopacity{0.300000}%
\pgfsetlinewidth{0.301125pt}%
\definecolor{currentstroke}{rgb}{0.500000,0.500000,0.500000}%
\pgfsetstrokecolor{currentstroke}%
\pgfsetstrokeopacity{0.300000}%
\pgfsetdash{}{0pt}%
\pgfpathmoveto{\pgfqpoint{0.000000in}{0.000000in}}%
\pgfpathlineto{\pgfqpoint{0.000000in}{0.000000in}}%
\pgfpathclose%
\pgfusepath{stroke,fill}%
\end{pgfscope}%
\begin{pgfscope}%
\pgfpathrectangle{\pgfqpoint{0.647939in}{0.492442in}}{\pgfqpoint{4.273799in}{2.331163in}}%
\pgfusepath{clip}%
\pgfsetroundcap%
\pgfsetroundjoin%
\pgfsetlinewidth{0.301125pt}%
\definecolor{currentstroke}{rgb}{0.500000,0.500000,0.500000}%
\pgfsetstrokecolor{currentstroke}%
\pgfsetstrokeopacity{0.300000}%
\pgfsetdash{}{0pt}%
\pgfpathmoveto{\pgfqpoint{2.009609in}{1.374531in}}%
\pgfusepath{stroke}%
\end{pgfscope}%
\begin{pgfscope}%
\pgfpathrectangle{\pgfqpoint{0.647939in}{0.492442in}}{\pgfqpoint{4.273799in}{2.331163in}}%
\pgfusepath{clip}%
\pgfsetroundcap%
\pgfsetroundjoin%
\definecolor{currentfill}{rgb}{0.500000,0.500000,0.500000}%
\pgfsetfillcolor{currentfill}%
\pgfsetfillopacity{0.300000}%
\pgfsetlinewidth{0.301125pt}%
\definecolor{currentstroke}{rgb}{0.500000,0.500000,0.500000}%
\pgfsetstrokecolor{currentstroke}%
\pgfsetstrokeopacity{0.300000}%
\pgfsetdash{}{0pt}%
\pgfpathmoveto{\pgfqpoint{0.000000in}{0.000000in}}%
\pgfpathlineto{\pgfqpoint{0.000000in}{0.000000in}}%
\pgfpathclose%
\pgfusepath{stroke,fill}%
\end{pgfscope}%
\begin{pgfscope}%
\pgfpathrectangle{\pgfqpoint{0.647939in}{0.492442in}}{\pgfqpoint{4.273799in}{2.331163in}}%
\pgfusepath{clip}%
\pgfsetroundcap%
\pgfsetroundjoin%
\pgfsetlinewidth{0.301125pt}%
\definecolor{currentstroke}{rgb}{0.500000,0.500000,0.500000}%
\pgfsetstrokecolor{currentstroke}%
\pgfsetstrokeopacity{0.300000}%
\pgfsetdash{}{0pt}%
\pgfpathmoveto{\pgfqpoint{2.679963in}{0.605374in}}%
\pgfusepath{stroke}%
\end{pgfscope}%
\begin{pgfscope}%
\pgfpathrectangle{\pgfqpoint{0.647939in}{0.492442in}}{\pgfqpoint{4.273799in}{2.331163in}}%
\pgfusepath{clip}%
\pgfsetroundcap%
\pgfsetroundjoin%
\definecolor{currentfill}{rgb}{0.500000,0.500000,0.500000}%
\pgfsetfillcolor{currentfill}%
\pgfsetfillopacity{0.300000}%
\pgfsetlinewidth{0.301125pt}%
\definecolor{currentstroke}{rgb}{0.500000,0.500000,0.500000}%
\pgfsetstrokecolor{currentstroke}%
\pgfsetstrokeopacity{0.300000}%
\pgfsetdash{}{0pt}%
\pgfpathmoveto{\pgfqpoint{0.000000in}{0.000000in}}%
\pgfpathlineto{\pgfqpoint{0.000000in}{0.000000in}}%
\pgfpathclose%
\pgfusepath{stroke,fill}%
\end{pgfscope}%
\begin{pgfscope}%
\pgfpathrectangle{\pgfqpoint{0.647939in}{0.492442in}}{\pgfqpoint{4.273799in}{2.331163in}}%
\pgfusepath{clip}%
\pgfsetroundcap%
\pgfsetroundjoin%
\pgfsetlinewidth{0.301125pt}%
\definecolor{currentstroke}{rgb}{0.500000,0.500000,0.500000}%
\pgfsetstrokecolor{currentstroke}%
\pgfsetstrokeopacity{0.300000}%
\pgfsetdash{}{0pt}%
\pgfpathmoveto{\pgfqpoint{1.961498in}{1.883105in}}%
\pgfusepath{stroke}%
\end{pgfscope}%
\begin{pgfscope}%
\pgfpathrectangle{\pgfqpoint{0.647939in}{0.492442in}}{\pgfqpoint{4.273799in}{2.331163in}}%
\pgfusepath{clip}%
\pgfsetroundcap%
\pgfsetroundjoin%
\definecolor{currentfill}{rgb}{0.500000,0.500000,0.500000}%
\pgfsetfillcolor{currentfill}%
\pgfsetfillopacity{0.300000}%
\pgfsetlinewidth{0.301125pt}%
\definecolor{currentstroke}{rgb}{0.500000,0.500000,0.500000}%
\pgfsetstrokecolor{currentstroke}%
\pgfsetstrokeopacity{0.300000}%
\pgfsetdash{}{0pt}%
\pgfpathmoveto{\pgfqpoint{0.000000in}{0.000000in}}%
\pgfpathlineto{\pgfqpoint{0.000000in}{0.000000in}}%
\pgfpathclose%
\pgfusepath{stroke,fill}%
\end{pgfscope}%
\begin{pgfscope}%
\pgfpathrectangle{\pgfqpoint{0.647939in}{0.492442in}}{\pgfqpoint{4.273799in}{2.331163in}}%
\pgfusepath{clip}%
\pgfsetroundcap%
\pgfsetroundjoin%
\pgfsetlinewidth{0.301125pt}%
\definecolor{currentstroke}{rgb}{0.500000,0.500000,0.500000}%
\pgfsetstrokecolor{currentstroke}%
\pgfsetstrokeopacity{0.300000}%
\pgfsetdash{}{0pt}%
\pgfpathmoveto{\pgfqpoint{2.386509in}{1.345193in}}%
\pgfusepath{stroke}%
\end{pgfscope}%
\begin{pgfscope}%
\pgfpathrectangle{\pgfqpoint{0.647939in}{0.492442in}}{\pgfqpoint{4.273799in}{2.331163in}}%
\pgfusepath{clip}%
\pgfsetroundcap%
\pgfsetroundjoin%
\definecolor{currentfill}{rgb}{0.500000,0.500000,0.500000}%
\pgfsetfillcolor{currentfill}%
\pgfsetfillopacity{0.300000}%
\pgfsetlinewidth{0.301125pt}%
\definecolor{currentstroke}{rgb}{0.500000,0.500000,0.500000}%
\pgfsetstrokecolor{currentstroke}%
\pgfsetstrokeopacity{0.300000}%
\pgfsetdash{}{0pt}%
\pgfpathmoveto{\pgfqpoint{0.000000in}{0.000000in}}%
\pgfpathlineto{\pgfqpoint{0.000000in}{0.000000in}}%
\pgfpathclose%
\pgfusepath{stroke,fill}%
\end{pgfscope}%
\begin{pgfscope}%
\pgfpathrectangle{\pgfqpoint{0.647939in}{0.492442in}}{\pgfqpoint{4.273799in}{2.331163in}}%
\pgfusepath{clip}%
\pgfsetroundcap%
\pgfsetroundjoin%
\pgfsetlinewidth{0.301125pt}%
\definecolor{currentstroke}{rgb}{0.500000,0.500000,0.500000}%
\pgfsetstrokecolor{currentstroke}%
\pgfsetstrokeopacity{0.300000}%
\pgfsetdash{}{0pt}%
\pgfpathmoveto{\pgfqpoint{2.510048in}{1.323093in}}%
\pgfusepath{stroke}%
\end{pgfscope}%
\begin{pgfscope}%
\pgfpathrectangle{\pgfqpoint{0.647939in}{0.492442in}}{\pgfqpoint{4.273799in}{2.331163in}}%
\pgfusepath{clip}%
\pgfsetroundcap%
\pgfsetroundjoin%
\definecolor{currentfill}{rgb}{0.500000,0.500000,0.500000}%
\pgfsetfillcolor{currentfill}%
\pgfsetfillopacity{0.300000}%
\pgfsetlinewidth{0.301125pt}%
\definecolor{currentstroke}{rgb}{0.500000,0.500000,0.500000}%
\pgfsetstrokecolor{currentstroke}%
\pgfsetstrokeopacity{0.300000}%
\pgfsetdash{}{0pt}%
\pgfpathmoveto{\pgfqpoint{0.000000in}{0.000000in}}%
\pgfpathlineto{\pgfqpoint{0.000000in}{0.000000in}}%
\pgfpathclose%
\pgfusepath{stroke,fill}%
\end{pgfscope}%
\begin{pgfscope}%
\pgfpathrectangle{\pgfqpoint{0.647939in}{0.492442in}}{\pgfqpoint{4.273799in}{2.331163in}}%
\pgfusepath{clip}%
\pgfsetroundcap%
\pgfsetroundjoin%
\pgfsetlinewidth{0.301125pt}%
\definecolor{currentstroke}{rgb}{0.500000,0.500000,0.500000}%
\pgfsetstrokecolor{currentstroke}%
\pgfsetstrokeopacity{0.300000}%
\pgfsetdash{}{0pt}%
\pgfpathmoveto{\pgfqpoint{3.023942in}{0.881405in}}%
\pgfusepath{stroke}%
\end{pgfscope}%
\begin{pgfscope}%
\pgfpathrectangle{\pgfqpoint{0.647939in}{0.492442in}}{\pgfqpoint{4.273799in}{2.331163in}}%
\pgfusepath{clip}%
\pgfsetroundcap%
\pgfsetroundjoin%
\definecolor{currentfill}{rgb}{0.500000,0.500000,0.500000}%
\pgfsetfillcolor{currentfill}%
\pgfsetfillopacity{0.300000}%
\pgfsetlinewidth{0.301125pt}%
\definecolor{currentstroke}{rgb}{0.500000,0.500000,0.500000}%
\pgfsetstrokecolor{currentstroke}%
\pgfsetstrokeopacity{0.300000}%
\pgfsetdash{}{0pt}%
\pgfpathmoveto{\pgfqpoint{0.000000in}{0.000000in}}%
\pgfpathlineto{\pgfqpoint{0.000000in}{0.000000in}}%
\pgfpathclose%
\pgfusepath{stroke,fill}%
\end{pgfscope}%
\begin{pgfscope}%
\pgfpathrectangle{\pgfqpoint{0.647939in}{0.492442in}}{\pgfqpoint{4.273799in}{2.331163in}}%
\pgfusepath{clip}%
\pgfsetroundcap%
\pgfsetroundjoin%
\pgfsetlinewidth{0.301125pt}%
\definecolor{currentstroke}{rgb}{0.500000,0.500000,0.500000}%
\pgfsetstrokecolor{currentstroke}%
\pgfsetstrokeopacity{0.300000}%
\pgfsetdash{}{0pt}%
\pgfpathmoveto{\pgfqpoint{3.234964in}{0.821177in}}%
\pgfusepath{stroke}%
\end{pgfscope}%
\begin{pgfscope}%
\pgfpathrectangle{\pgfqpoint{0.647939in}{0.492442in}}{\pgfqpoint{4.273799in}{2.331163in}}%
\pgfusepath{clip}%
\pgfsetroundcap%
\pgfsetroundjoin%
\definecolor{currentfill}{rgb}{0.500000,0.500000,0.500000}%
\pgfsetfillcolor{currentfill}%
\pgfsetfillopacity{0.300000}%
\pgfsetlinewidth{0.301125pt}%
\definecolor{currentstroke}{rgb}{0.500000,0.500000,0.500000}%
\pgfsetstrokecolor{currentstroke}%
\pgfsetstrokeopacity{0.300000}%
\pgfsetdash{}{0pt}%
\pgfpathmoveto{\pgfqpoint{0.000000in}{0.000000in}}%
\pgfpathlineto{\pgfqpoint{0.000000in}{0.000000in}}%
\pgfpathclose%
\pgfusepath{stroke,fill}%
\end{pgfscope}%
\begin{pgfscope}%
\pgfpathrectangle{\pgfqpoint{0.647939in}{0.492442in}}{\pgfqpoint{4.273799in}{2.331163in}}%
\pgfusepath{clip}%
\pgfsetroundcap%
\pgfsetroundjoin%
\pgfsetlinewidth{0.301125pt}%
\definecolor{currentstroke}{rgb}{0.500000,0.500000,0.500000}%
\pgfsetstrokecolor{currentstroke}%
\pgfsetstrokeopacity{0.300000}%
\pgfsetdash{}{0pt}%
\pgfpathmoveto{\pgfqpoint{3.701243in}{0.582939in}}%
\pgfusepath{stroke}%
\end{pgfscope}%
\begin{pgfscope}%
\pgfpathrectangle{\pgfqpoint{0.647939in}{0.492442in}}{\pgfqpoint{4.273799in}{2.331163in}}%
\pgfusepath{clip}%
\pgfsetroundcap%
\pgfsetroundjoin%
\definecolor{currentfill}{rgb}{0.500000,0.500000,0.500000}%
\pgfsetfillcolor{currentfill}%
\pgfsetfillopacity{0.300000}%
\pgfsetlinewidth{0.301125pt}%
\definecolor{currentstroke}{rgb}{0.500000,0.500000,0.500000}%
\pgfsetstrokecolor{currentstroke}%
\pgfsetstrokeopacity{0.300000}%
\pgfsetdash{}{0pt}%
\pgfpathmoveto{\pgfqpoint{0.000000in}{0.000000in}}%
\pgfpathlineto{\pgfqpoint{0.000000in}{0.000000in}}%
\pgfpathclose%
\pgfusepath{stroke,fill}%
\end{pgfscope}%
\begin{pgfscope}%
\pgfpathrectangle{\pgfqpoint{0.647939in}{0.492442in}}{\pgfqpoint{4.273799in}{2.331163in}}%
\pgfusepath{clip}%
\pgfsetroundcap%
\pgfsetroundjoin%
\pgfsetlinewidth{0.301125pt}%
\definecolor{currentstroke}{rgb}{0.500000,0.500000,0.500000}%
\pgfsetstrokecolor{currentstroke}%
\pgfsetstrokeopacity{0.300000}%
\pgfsetdash{}{0pt}%
\pgfpathmoveto{\pgfqpoint{3.799923in}{0.583690in}}%
\pgfusepath{stroke}%
\end{pgfscope}%
\begin{pgfscope}%
\pgfpathrectangle{\pgfqpoint{0.647939in}{0.492442in}}{\pgfqpoint{4.273799in}{2.331163in}}%
\pgfusepath{clip}%
\pgfsetroundcap%
\pgfsetroundjoin%
\definecolor{currentfill}{rgb}{0.500000,0.500000,0.500000}%
\pgfsetfillcolor{currentfill}%
\pgfsetfillopacity{0.300000}%
\pgfsetlinewidth{0.301125pt}%
\definecolor{currentstroke}{rgb}{0.500000,0.500000,0.500000}%
\pgfsetstrokecolor{currentstroke}%
\pgfsetstrokeopacity{0.300000}%
\pgfsetdash{}{0pt}%
\pgfpathmoveto{\pgfqpoint{0.000000in}{0.000000in}}%
\pgfpathlineto{\pgfqpoint{0.000000in}{0.000000in}}%
\pgfpathclose%
\pgfusepath{stroke,fill}%
\end{pgfscope}%
\begin{pgfscope}%
\pgfpathrectangle{\pgfqpoint{0.647939in}{0.492442in}}{\pgfqpoint{4.273799in}{2.331163in}}%
\pgfusepath{clip}%
\pgfsetroundcap%
\pgfsetroundjoin%
\pgfsetlinewidth{0.301125pt}%
\definecolor{currentstroke}{rgb}{0.500000,0.500000,0.500000}%
\pgfsetstrokecolor{currentstroke}%
\pgfsetstrokeopacity{0.300000}%
\pgfsetdash{}{0pt}%
\pgfpathmoveto{\pgfqpoint{3.900877in}{0.585700in}}%
\pgfusepath{stroke}%
\end{pgfscope}%
\begin{pgfscope}%
\pgfpathrectangle{\pgfqpoint{0.647939in}{0.492442in}}{\pgfqpoint{4.273799in}{2.331163in}}%
\pgfusepath{clip}%
\pgfsetroundcap%
\pgfsetroundjoin%
\definecolor{currentfill}{rgb}{0.500000,0.500000,0.500000}%
\pgfsetfillcolor{currentfill}%
\pgfsetfillopacity{0.300000}%
\pgfsetlinewidth{0.301125pt}%
\definecolor{currentstroke}{rgb}{0.500000,0.500000,0.500000}%
\pgfsetstrokecolor{currentstroke}%
\pgfsetstrokeopacity{0.300000}%
\pgfsetdash{}{0pt}%
\pgfpathmoveto{\pgfqpoint{0.000000in}{0.000000in}}%
\pgfpathlineto{\pgfqpoint{0.000000in}{0.000000in}}%
\pgfpathclose%
\pgfusepath{stroke,fill}%
\end{pgfscope}%
\begin{pgfscope}%
\pgfpathrectangle{\pgfqpoint{0.647939in}{0.492442in}}{\pgfqpoint{4.273799in}{2.331163in}}%
\pgfusepath{clip}%
\pgfsetroundcap%
\pgfsetroundjoin%
\pgfsetlinewidth{0.301125pt}%
\definecolor{currentstroke}{rgb}{0.500000,0.500000,0.500000}%
\pgfsetstrokecolor{currentstroke}%
\pgfsetstrokeopacity{0.300000}%
\pgfsetdash{}{0pt}%
\pgfpathmoveto{\pgfqpoint{4.004214in}{0.588867in}}%
\pgfusepath{stroke}%
\end{pgfscope}%
\begin{pgfscope}%
\pgfpathrectangle{\pgfqpoint{0.647939in}{0.492442in}}{\pgfqpoint{4.273799in}{2.331163in}}%
\pgfusepath{clip}%
\pgfsetroundcap%
\pgfsetroundjoin%
\definecolor{currentfill}{rgb}{0.500000,0.500000,0.500000}%
\pgfsetfillcolor{currentfill}%
\pgfsetfillopacity{0.300000}%
\pgfsetlinewidth{0.301125pt}%
\definecolor{currentstroke}{rgb}{0.500000,0.500000,0.500000}%
\pgfsetstrokecolor{currentstroke}%
\pgfsetstrokeopacity{0.300000}%
\pgfsetdash{}{0pt}%
\pgfpathmoveto{\pgfqpoint{0.000000in}{0.000000in}}%
\pgfpathlineto{\pgfqpoint{0.000000in}{0.000000in}}%
\pgfpathclose%
\pgfusepath{stroke,fill}%
\end{pgfscope}%
\begin{pgfscope}%
\pgfpathrectangle{\pgfqpoint{0.647939in}{0.492442in}}{\pgfqpoint{4.273799in}{2.331163in}}%
\pgfusepath{clip}%
\pgfsetroundcap%
\pgfsetroundjoin%
\pgfsetlinewidth{0.301125pt}%
\definecolor{currentstroke}{rgb}{0.500000,0.500000,0.500000}%
\pgfsetstrokecolor{currentstroke}%
\pgfsetstrokeopacity{0.300000}%
\pgfsetdash{}{0pt}%
\pgfpathmoveto{\pgfqpoint{3.047243in}{1.273296in}}%
\pgfusepath{stroke}%
\end{pgfscope}%
\begin{pgfscope}%
\pgfpathrectangle{\pgfqpoint{0.647939in}{0.492442in}}{\pgfqpoint{4.273799in}{2.331163in}}%
\pgfusepath{clip}%
\pgfsetroundcap%
\pgfsetroundjoin%
\definecolor{currentfill}{rgb}{0.500000,0.500000,0.500000}%
\pgfsetfillcolor{currentfill}%
\pgfsetfillopacity{0.300000}%
\pgfsetlinewidth{0.301125pt}%
\definecolor{currentstroke}{rgb}{0.500000,0.500000,0.500000}%
\pgfsetstrokecolor{currentstroke}%
\pgfsetstrokeopacity{0.300000}%
\pgfsetdash{}{0pt}%
\pgfpathmoveto{\pgfqpoint{0.000000in}{0.000000in}}%
\pgfpathlineto{\pgfqpoint{0.000000in}{0.000000in}}%
\pgfpathclose%
\pgfusepath{stroke,fill}%
\end{pgfscope}%
\begin{pgfscope}%
\pgfpathrectangle{\pgfqpoint{0.647939in}{0.492442in}}{\pgfqpoint{4.273799in}{2.331163in}}%
\pgfusepath{clip}%
\pgfsetroundcap%
\pgfsetroundjoin%
\pgfsetlinewidth{0.301125pt}%
\definecolor{currentstroke}{rgb}{0.500000,0.500000,0.500000}%
\pgfsetstrokecolor{currentstroke}%
\pgfsetstrokeopacity{0.300000}%
\pgfsetdash{}{0pt}%
\pgfpathmoveto{\pgfqpoint{3.847883in}{0.917140in}}%
\pgfusepath{stroke}%
\end{pgfscope}%
\begin{pgfscope}%
\pgfpathrectangle{\pgfqpoint{0.647939in}{0.492442in}}{\pgfqpoint{4.273799in}{2.331163in}}%
\pgfusepath{clip}%
\pgfsetroundcap%
\pgfsetroundjoin%
\definecolor{currentfill}{rgb}{0.500000,0.500000,0.500000}%
\pgfsetfillcolor{currentfill}%
\pgfsetfillopacity{0.300000}%
\pgfsetlinewidth{0.301125pt}%
\definecolor{currentstroke}{rgb}{0.500000,0.500000,0.500000}%
\pgfsetstrokecolor{currentstroke}%
\pgfsetstrokeopacity{0.300000}%
\pgfsetdash{}{0pt}%
\pgfpathmoveto{\pgfqpoint{0.000000in}{0.000000in}}%
\pgfpathlineto{\pgfqpoint{0.000000in}{0.000000in}}%
\pgfpathclose%
\pgfusepath{stroke,fill}%
\end{pgfscope}%
\begin{pgfscope}%
\pgfpathrectangle{\pgfqpoint{0.647939in}{0.492442in}}{\pgfqpoint{4.273799in}{2.331163in}}%
\pgfusepath{clip}%
\pgfsetroundcap%
\pgfsetroundjoin%
\pgfsetlinewidth{0.301125pt}%
\definecolor{currentstroke}{rgb}{0.500000,0.500000,0.500000}%
\pgfsetstrokecolor{currentstroke}%
\pgfsetstrokeopacity{0.300000}%
\pgfsetdash{}{0pt}%
\pgfpathmoveto{\pgfqpoint{4.132909in}{0.865969in}}%
\pgfusepath{stroke}%
\end{pgfscope}%
\begin{pgfscope}%
\pgfpathrectangle{\pgfqpoint{0.647939in}{0.492442in}}{\pgfqpoint{4.273799in}{2.331163in}}%
\pgfusepath{clip}%
\pgfsetroundcap%
\pgfsetroundjoin%
\definecolor{currentfill}{rgb}{0.500000,0.500000,0.500000}%
\pgfsetfillcolor{currentfill}%
\pgfsetfillopacity{0.300000}%
\pgfsetlinewidth{0.301125pt}%
\definecolor{currentstroke}{rgb}{0.500000,0.500000,0.500000}%
\pgfsetstrokecolor{currentstroke}%
\pgfsetstrokeopacity{0.300000}%
\pgfsetdash{}{0pt}%
\pgfpathmoveto{\pgfqpoint{0.000000in}{0.000000in}}%
\pgfpathlineto{\pgfqpoint{0.000000in}{0.000000in}}%
\pgfpathclose%
\pgfusepath{stroke,fill}%
\end{pgfscope}%
\begin{pgfscope}%
\pgfpathrectangle{\pgfqpoint{0.647939in}{0.492442in}}{\pgfqpoint{4.273799in}{2.331163in}}%
\pgfusepath{clip}%
\pgfsetroundcap%
\pgfsetroundjoin%
\pgfsetlinewidth{0.301125pt}%
\definecolor{currentstroke}{rgb}{0.500000,0.500000,0.500000}%
\pgfsetstrokecolor{currentstroke}%
\pgfsetstrokeopacity{0.300000}%
\pgfsetdash{}{0pt}%
\pgfpathmoveto{\pgfqpoint{3.421464in}{1.272337in}}%
\pgfusepath{stroke}%
\end{pgfscope}%
\begin{pgfscope}%
\pgfpathrectangle{\pgfqpoint{0.647939in}{0.492442in}}{\pgfqpoint{4.273799in}{2.331163in}}%
\pgfusepath{clip}%
\pgfsetroundcap%
\pgfsetroundjoin%
\definecolor{currentfill}{rgb}{0.500000,0.500000,0.500000}%
\pgfsetfillcolor{currentfill}%
\pgfsetfillopacity{0.300000}%
\pgfsetlinewidth{0.301125pt}%
\definecolor{currentstroke}{rgb}{0.500000,0.500000,0.500000}%
\pgfsetstrokecolor{currentstroke}%
\pgfsetstrokeopacity{0.300000}%
\pgfsetdash{}{0pt}%
\pgfpathmoveto{\pgfqpoint{0.000000in}{0.000000in}}%
\pgfpathlineto{\pgfqpoint{0.000000in}{0.000000in}}%
\pgfpathclose%
\pgfusepath{stroke,fill}%
\end{pgfscope}%
\begin{pgfscope}%
\pgfpathrectangle{\pgfqpoint{0.647939in}{0.492442in}}{\pgfqpoint{4.273799in}{2.331163in}}%
\pgfusepath{clip}%
\pgfsetroundcap%
\pgfsetroundjoin%
\pgfsetlinewidth{0.301125pt}%
\definecolor{currentstroke}{rgb}{0.500000,0.500000,0.500000}%
\pgfsetstrokecolor{currentstroke}%
\pgfsetstrokeopacity{0.300000}%
\pgfsetdash{}{0pt}%
\pgfpathmoveto{\pgfqpoint{4.122387in}{1.152016in}}%
\pgfusepath{stroke}%
\end{pgfscope}%
\begin{pgfscope}%
\pgfpathrectangle{\pgfqpoint{0.647939in}{0.492442in}}{\pgfqpoint{4.273799in}{2.331163in}}%
\pgfusepath{clip}%
\pgfsetroundcap%
\pgfsetroundjoin%
\definecolor{currentfill}{rgb}{0.500000,0.500000,0.500000}%
\pgfsetfillcolor{currentfill}%
\pgfsetfillopacity{0.300000}%
\pgfsetlinewidth{0.301125pt}%
\definecolor{currentstroke}{rgb}{0.500000,0.500000,0.500000}%
\pgfsetstrokecolor{currentstroke}%
\pgfsetstrokeopacity{0.300000}%
\pgfsetdash{}{0pt}%
\pgfpathmoveto{\pgfqpoint{0.000000in}{0.000000in}}%
\pgfpathlineto{\pgfqpoint{0.000000in}{0.000000in}}%
\pgfpathclose%
\pgfusepath{stroke,fill}%
\end{pgfscope}%
\begin{pgfscope}%
\pgfpathrectangle{\pgfqpoint{0.647939in}{0.492442in}}{\pgfqpoint{4.273799in}{2.331163in}}%
\pgfusepath{clip}%
\pgfsetroundcap%
\pgfsetroundjoin%
\pgfsetlinewidth{0.301125pt}%
\definecolor{currentstroke}{rgb}{0.500000,0.500000,0.500000}%
\pgfsetstrokecolor{currentstroke}%
\pgfsetstrokeopacity{0.300000}%
\pgfsetdash{}{0pt}%
\pgfpathmoveto{\pgfqpoint{4.024076in}{1.368488in}}%
\pgfusepath{stroke}%
\end{pgfscope}%
\begin{pgfscope}%
\pgfpathrectangle{\pgfqpoint{0.647939in}{0.492442in}}{\pgfqpoint{4.273799in}{2.331163in}}%
\pgfusepath{clip}%
\pgfsetroundcap%
\pgfsetroundjoin%
\definecolor{currentfill}{rgb}{0.500000,0.500000,0.500000}%
\pgfsetfillcolor{currentfill}%
\pgfsetfillopacity{0.300000}%
\pgfsetlinewidth{0.301125pt}%
\definecolor{currentstroke}{rgb}{0.500000,0.500000,0.500000}%
\pgfsetstrokecolor{currentstroke}%
\pgfsetstrokeopacity{0.300000}%
\pgfsetdash{}{0pt}%
\pgfpathmoveto{\pgfqpoint{0.000000in}{0.000000in}}%
\pgfpathlineto{\pgfqpoint{0.000000in}{0.000000in}}%
\pgfpathclose%
\pgfusepath{stroke,fill}%
\end{pgfscope}%
\begin{pgfscope}%
\pgfpathrectangle{\pgfqpoint{0.647939in}{0.492442in}}{\pgfqpoint{4.273799in}{2.331163in}}%
\pgfusepath{clip}%
\pgfsetroundcap%
\pgfsetroundjoin%
\pgfsetlinewidth{0.301125pt}%
\definecolor{currentstroke}{rgb}{0.500000,0.500000,0.500000}%
\pgfsetstrokecolor{currentstroke}%
\pgfsetstrokeopacity{0.300000}%
\pgfsetdash{}{0pt}%
\pgfpathmoveto{\pgfqpoint{4.288707in}{1.484775in}}%
\pgfusepath{stroke}%
\end{pgfscope}%
\begin{pgfscope}%
\pgfpathrectangle{\pgfqpoint{0.647939in}{0.492442in}}{\pgfqpoint{4.273799in}{2.331163in}}%
\pgfusepath{clip}%
\pgfsetroundcap%
\pgfsetroundjoin%
\definecolor{currentfill}{rgb}{0.500000,0.500000,0.500000}%
\pgfsetfillcolor{currentfill}%
\pgfsetfillopacity{0.300000}%
\pgfsetlinewidth{0.301125pt}%
\definecolor{currentstroke}{rgb}{0.500000,0.500000,0.500000}%
\pgfsetstrokecolor{currentstroke}%
\pgfsetstrokeopacity{0.300000}%
\pgfsetdash{}{0pt}%
\pgfpathmoveto{\pgfqpoint{0.000000in}{0.000000in}}%
\pgfpathlineto{\pgfqpoint{0.000000in}{0.000000in}}%
\pgfpathclose%
\pgfusepath{stroke,fill}%
\end{pgfscope}%
\begin{pgfscope}%
\pgfpathrectangle{\pgfqpoint{0.647939in}{0.492442in}}{\pgfqpoint{4.273799in}{2.331163in}}%
\pgfusepath{clip}%
\pgfsetroundcap%
\pgfsetroundjoin%
\pgfsetlinewidth{0.301125pt}%
\definecolor{currentstroke}{rgb}{0.500000,0.500000,0.500000}%
\pgfsetstrokecolor{currentstroke}%
\pgfsetstrokeopacity{0.300000}%
\pgfsetdash{}{0pt}%
\pgfpathmoveto{\pgfqpoint{4.464534in}{1.627351in}}%
\pgfusepath{stroke}%
\end{pgfscope}%
\begin{pgfscope}%
\pgfpathrectangle{\pgfqpoint{0.647939in}{0.492442in}}{\pgfqpoint{4.273799in}{2.331163in}}%
\pgfusepath{clip}%
\pgfsetroundcap%
\pgfsetroundjoin%
\definecolor{currentfill}{rgb}{0.500000,0.500000,0.500000}%
\pgfsetfillcolor{currentfill}%
\pgfsetfillopacity{0.300000}%
\pgfsetlinewidth{0.301125pt}%
\definecolor{currentstroke}{rgb}{0.500000,0.500000,0.500000}%
\pgfsetstrokecolor{currentstroke}%
\pgfsetstrokeopacity{0.300000}%
\pgfsetdash{}{0pt}%
\pgfpathmoveto{\pgfqpoint{0.000000in}{0.000000in}}%
\pgfpathlineto{\pgfqpoint{0.000000in}{0.000000in}}%
\pgfpathclose%
\pgfusepath{stroke,fill}%
\end{pgfscope}%
\begin{pgfscope}%
\pgfpathrectangle{\pgfqpoint{0.647939in}{0.492442in}}{\pgfqpoint{4.273799in}{2.331163in}}%
\pgfusepath{clip}%
\pgfsetroundcap%
\pgfsetroundjoin%
\pgfsetlinewidth{0.301125pt}%
\definecolor{currentstroke}{rgb}{0.500000,0.500000,0.500000}%
\pgfsetstrokecolor{currentstroke}%
\pgfsetstrokeopacity{0.300000}%
\pgfsetdash{}{0pt}%
\pgfpathmoveto{\pgfqpoint{4.558366in}{1.880329in}}%
\pgfusepath{stroke}%
\end{pgfscope}%
\begin{pgfscope}%
\pgfpathrectangle{\pgfqpoint{0.647939in}{0.492442in}}{\pgfqpoint{4.273799in}{2.331163in}}%
\pgfusepath{clip}%
\pgfsetroundcap%
\pgfsetroundjoin%
\definecolor{currentfill}{rgb}{0.500000,0.500000,0.500000}%
\pgfsetfillcolor{currentfill}%
\pgfsetfillopacity{0.300000}%
\pgfsetlinewidth{0.301125pt}%
\definecolor{currentstroke}{rgb}{0.500000,0.500000,0.500000}%
\pgfsetstrokecolor{currentstroke}%
\pgfsetstrokeopacity{0.300000}%
\pgfsetdash{}{0pt}%
\pgfpathmoveto{\pgfqpoint{0.000000in}{0.000000in}}%
\pgfpathlineto{\pgfqpoint{0.000000in}{0.000000in}}%
\pgfpathclose%
\pgfusepath{stroke,fill}%
\end{pgfscope}%
\begin{pgfscope}%
\pgfpathrectangle{\pgfqpoint{0.647939in}{0.492442in}}{\pgfqpoint{4.273799in}{2.331163in}}%
\pgfusepath{clip}%
\pgfsetroundcap%
\pgfsetroundjoin%
\pgfsetlinewidth{0.301125pt}%
\definecolor{currentstroke}{rgb}{0.500000,0.500000,0.500000}%
\pgfsetstrokecolor{currentstroke}%
\pgfsetstrokeopacity{0.300000}%
\pgfsetdash{}{0pt}%
\pgfpathmoveto{\pgfqpoint{4.738737in}{1.847059in}}%
\pgfusepath{stroke}%
\end{pgfscope}%
\begin{pgfscope}%
\pgfpathrectangle{\pgfqpoint{0.647939in}{0.492442in}}{\pgfqpoint{4.273799in}{2.331163in}}%
\pgfusepath{clip}%
\pgfsetroundcap%
\pgfsetroundjoin%
\definecolor{currentfill}{rgb}{0.500000,0.500000,0.500000}%
\pgfsetfillcolor{currentfill}%
\pgfsetfillopacity{0.300000}%
\pgfsetlinewidth{0.301125pt}%
\definecolor{currentstroke}{rgb}{0.500000,0.500000,0.500000}%
\pgfsetstrokecolor{currentstroke}%
\pgfsetstrokeopacity{0.300000}%
\pgfsetdash{}{0pt}%
\pgfpathmoveto{\pgfqpoint{0.000000in}{0.000000in}}%
\pgfpathlineto{\pgfqpoint{0.000000in}{0.000000in}}%
\pgfpathclose%
\pgfusepath{stroke,fill}%
\end{pgfscope}%
\begin{pgfscope}%
\pgfpathrectangle{\pgfqpoint{0.647939in}{0.492442in}}{\pgfqpoint{4.273799in}{2.331163in}}%
\pgfusepath{clip}%
\pgfsetroundcap%
\pgfsetroundjoin%
\pgfsetlinewidth{0.301125pt}%
\definecolor{currentstroke}{rgb}{0.500000,0.500000,0.500000}%
\pgfsetstrokecolor{currentstroke}%
\pgfsetstrokeopacity{0.300000}%
\pgfsetdash{}{0pt}%
\pgfpathmoveto{\pgfqpoint{4.838374in}{1.857956in}}%
\pgfusepath{stroke}%
\end{pgfscope}%
\begin{pgfscope}%
\pgfpathrectangle{\pgfqpoint{0.647939in}{0.492442in}}{\pgfqpoint{4.273799in}{2.331163in}}%
\pgfusepath{clip}%
\pgfsetroundcap%
\pgfsetroundjoin%
\definecolor{currentfill}{rgb}{0.500000,0.500000,0.500000}%
\pgfsetfillcolor{currentfill}%
\pgfsetfillopacity{0.300000}%
\pgfsetlinewidth{0.301125pt}%
\definecolor{currentstroke}{rgb}{0.500000,0.500000,0.500000}%
\pgfsetstrokecolor{currentstroke}%
\pgfsetstrokeopacity{0.300000}%
\pgfsetdash{}{0pt}%
\pgfpathmoveto{\pgfqpoint{0.000000in}{0.000000in}}%
\pgfpathlineto{\pgfqpoint{0.000000in}{0.000000in}}%
\pgfpathclose%
\pgfusepath{stroke,fill}%
\end{pgfscope}%
\begin{pgfscope}%
\pgfpathrectangle{\pgfqpoint{0.647939in}{0.492442in}}{\pgfqpoint{4.273799in}{2.331163in}}%
\pgfusepath{clip}%
\pgfsetroundcap%
\pgfsetroundjoin%
\pgfsetlinewidth{0.301125pt}%
\definecolor{currentstroke}{rgb}{0.500000,0.500000,0.500000}%
\pgfsetstrokecolor{currentstroke}%
\pgfsetstrokeopacity{0.300000}%
\pgfsetdash{}{0pt}%
\pgfpathmoveto{\pgfqpoint{4.885588in}{2.072989in}}%
\pgfusepath{stroke}%
\end{pgfscope}%
\begin{pgfscope}%
\pgfpathrectangle{\pgfqpoint{0.647939in}{0.492442in}}{\pgfqpoint{4.273799in}{2.331163in}}%
\pgfusepath{clip}%
\pgfsetroundcap%
\pgfsetroundjoin%
\definecolor{currentfill}{rgb}{0.500000,0.500000,0.500000}%
\pgfsetfillcolor{currentfill}%
\pgfsetfillopacity{0.300000}%
\pgfsetlinewidth{0.301125pt}%
\definecolor{currentstroke}{rgb}{0.500000,0.500000,0.500000}%
\pgfsetstrokecolor{currentstroke}%
\pgfsetstrokeopacity{0.300000}%
\pgfsetdash{}{0pt}%
\pgfpathmoveto{\pgfqpoint{0.000000in}{0.000000in}}%
\pgfpathlineto{\pgfqpoint{0.000000in}{0.000000in}}%
\pgfpathclose%
\pgfusepath{stroke,fill}%
\end{pgfscope}%
\begin{pgfscope}%
\pgfpathrectangle{\pgfqpoint{0.647939in}{0.492442in}}{\pgfqpoint{4.273799in}{2.331163in}}%
\pgfusepath{clip}%
\pgfsetroundcap%
\pgfsetroundjoin%
\pgfsetlinewidth{0.301125pt}%
\definecolor{currentstroke}{rgb}{0.500000,0.500000,0.500000}%
\pgfsetstrokecolor{currentstroke}%
\pgfsetstrokeopacity{0.300000}%
\pgfsetdash{}{0pt}%
\pgfpathmoveto{\pgfqpoint{4.573769in}{2.680759in}}%
\pgfusepath{stroke}%
\end{pgfscope}%
\begin{pgfscope}%
\pgfpathrectangle{\pgfqpoint{0.647939in}{0.492442in}}{\pgfqpoint{4.273799in}{2.331163in}}%
\pgfusepath{clip}%
\pgfsetroundcap%
\pgfsetroundjoin%
\definecolor{currentfill}{rgb}{0.500000,0.500000,0.500000}%
\pgfsetfillcolor{currentfill}%
\pgfsetfillopacity{0.300000}%
\pgfsetlinewidth{0.301125pt}%
\definecolor{currentstroke}{rgb}{0.500000,0.500000,0.500000}%
\pgfsetstrokecolor{currentstroke}%
\pgfsetstrokeopacity{0.300000}%
\pgfsetdash{}{0pt}%
\pgfpathmoveto{\pgfqpoint{0.000000in}{0.000000in}}%
\pgfpathlineto{\pgfqpoint{0.000000in}{0.000000in}}%
\pgfpathclose%
\pgfusepath{stroke,fill}%
\end{pgfscope}%
\begin{pgfscope}%
\pgfpathrectangle{\pgfqpoint{0.647939in}{0.492442in}}{\pgfqpoint{4.273799in}{2.331163in}}%
\pgfusepath{clip}%
\pgfsetroundcap%
\pgfsetroundjoin%
\pgfsetlinewidth{0.301125pt}%
\definecolor{currentstroke}{rgb}{0.500000,0.500000,0.500000}%
\pgfsetstrokecolor{currentstroke}%
\pgfsetstrokeopacity{0.300000}%
\pgfsetdash{}{0pt}%
\pgfpathmoveto{\pgfqpoint{4.591307in}{2.749924in}}%
\pgfusepath{stroke}%
\end{pgfscope}%
\begin{pgfscope}%
\pgfpathrectangle{\pgfqpoint{0.647939in}{0.492442in}}{\pgfqpoint{4.273799in}{2.331163in}}%
\pgfusepath{clip}%
\pgfsetroundcap%
\pgfsetroundjoin%
\definecolor{currentfill}{rgb}{0.500000,0.500000,0.500000}%
\pgfsetfillcolor{currentfill}%
\pgfsetfillopacity{0.300000}%
\pgfsetlinewidth{0.301125pt}%
\definecolor{currentstroke}{rgb}{0.500000,0.500000,0.500000}%
\pgfsetstrokecolor{currentstroke}%
\pgfsetstrokeopacity{0.300000}%
\pgfsetdash{}{0pt}%
\pgfpathmoveto{\pgfqpoint{0.000000in}{0.000000in}}%
\pgfpathlineto{\pgfqpoint{0.000000in}{0.000000in}}%
\pgfpathclose%
\pgfusepath{stroke,fill}%
\end{pgfscope}%
\begin{pgfscope}%
\pgfpathrectangle{\pgfqpoint{0.647939in}{0.492442in}}{\pgfqpoint{4.273799in}{2.331163in}}%
\pgfusepath{clip}%
\pgfsetroundcap%
\pgfsetroundjoin%
\pgfsetlinewidth{0.301125pt}%
\definecolor{currentstroke}{rgb}{0.500000,0.500000,0.500000}%
\pgfsetstrokecolor{currentstroke}%
\pgfsetstrokeopacity{0.300000}%
\pgfsetdash{}{0pt}%
\pgfpathmoveto{\pgfqpoint{4.454541in}{2.593259in}}%
\pgfusepath{stroke}%
\end{pgfscope}%
\begin{pgfscope}%
\pgfpathrectangle{\pgfqpoint{0.647939in}{0.492442in}}{\pgfqpoint{4.273799in}{2.331163in}}%
\pgfusepath{clip}%
\pgfsetroundcap%
\pgfsetroundjoin%
\definecolor{currentfill}{rgb}{0.500000,0.500000,0.500000}%
\pgfsetfillcolor{currentfill}%
\pgfsetfillopacity{0.300000}%
\pgfsetlinewidth{0.301125pt}%
\definecolor{currentstroke}{rgb}{0.500000,0.500000,0.500000}%
\pgfsetstrokecolor{currentstroke}%
\pgfsetstrokeopacity{0.300000}%
\pgfsetdash{}{0pt}%
\pgfpathmoveto{\pgfqpoint{0.000000in}{0.000000in}}%
\pgfpathlineto{\pgfqpoint{0.000000in}{0.000000in}}%
\pgfpathclose%
\pgfusepath{stroke,fill}%
\end{pgfscope}%
\begin{pgfscope}%
\pgfpathrectangle{\pgfqpoint{0.647939in}{0.492442in}}{\pgfqpoint{4.273799in}{2.331163in}}%
\pgfusepath{clip}%
\pgfsetroundcap%
\pgfsetroundjoin%
\pgfsetlinewidth{0.301125pt}%
\definecolor{currentstroke}{rgb}{0.500000,0.500000,0.500000}%
\pgfsetstrokecolor{currentstroke}%
\pgfsetstrokeopacity{0.300000}%
\pgfsetdash{}{0pt}%
\pgfpathmoveto{\pgfqpoint{4.325927in}{2.558840in}}%
\pgfusepath{stroke}%
\end{pgfscope}%
\begin{pgfscope}%
\pgfpathrectangle{\pgfqpoint{0.647939in}{0.492442in}}{\pgfqpoint{4.273799in}{2.331163in}}%
\pgfusepath{clip}%
\pgfsetroundcap%
\pgfsetroundjoin%
\definecolor{currentfill}{rgb}{0.500000,0.500000,0.500000}%
\pgfsetfillcolor{currentfill}%
\pgfsetfillopacity{0.300000}%
\pgfsetlinewidth{0.301125pt}%
\definecolor{currentstroke}{rgb}{0.500000,0.500000,0.500000}%
\pgfsetstrokecolor{currentstroke}%
\pgfsetstrokeopacity{0.300000}%
\pgfsetdash{}{0pt}%
\pgfpathmoveto{\pgfqpoint{0.000000in}{0.000000in}}%
\pgfpathlineto{\pgfqpoint{0.000000in}{0.000000in}}%
\pgfpathclose%
\pgfusepath{stroke,fill}%
\end{pgfscope}%
\begin{pgfscope}%
\pgfpathrectangle{\pgfqpoint{0.647939in}{0.492442in}}{\pgfqpoint{4.273799in}{2.331163in}}%
\pgfusepath{clip}%
\pgfsetroundcap%
\pgfsetroundjoin%
\pgfsetlinewidth{0.301125pt}%
\definecolor{currentstroke}{rgb}{0.500000,0.500000,0.500000}%
\pgfsetstrokecolor{currentstroke}%
\pgfsetstrokeopacity{0.300000}%
\pgfsetdash{}{0pt}%
\pgfpathmoveto{\pgfqpoint{4.231729in}{2.502944in}}%
\pgfusepath{stroke}%
\end{pgfscope}%
\begin{pgfscope}%
\pgfpathrectangle{\pgfqpoint{0.647939in}{0.492442in}}{\pgfqpoint{4.273799in}{2.331163in}}%
\pgfusepath{clip}%
\pgfsetroundcap%
\pgfsetroundjoin%
\definecolor{currentfill}{rgb}{0.500000,0.500000,0.500000}%
\pgfsetfillcolor{currentfill}%
\pgfsetfillopacity{0.300000}%
\pgfsetlinewidth{0.301125pt}%
\definecolor{currentstroke}{rgb}{0.500000,0.500000,0.500000}%
\pgfsetstrokecolor{currentstroke}%
\pgfsetstrokeopacity{0.300000}%
\pgfsetdash{}{0pt}%
\pgfpathmoveto{\pgfqpoint{0.000000in}{0.000000in}}%
\pgfpathlineto{\pgfqpoint{0.000000in}{0.000000in}}%
\pgfpathclose%
\pgfusepath{stroke,fill}%
\end{pgfscope}%
\begin{pgfscope}%
\pgfpathrectangle{\pgfqpoint{0.647939in}{0.492442in}}{\pgfqpoint{4.273799in}{2.331163in}}%
\pgfusepath{clip}%
\pgfsetroundcap%
\pgfsetroundjoin%
\pgfsetlinewidth{0.301125pt}%
\definecolor{currentstroke}{rgb}{0.500000,0.500000,0.500000}%
\pgfsetstrokecolor{currentstroke}%
\pgfsetstrokeopacity{0.300000}%
\pgfsetdash{}{0pt}%
\pgfpathmoveto{\pgfqpoint{4.192625in}{2.322856in}}%
\pgfusepath{stroke}%
\end{pgfscope}%
\begin{pgfscope}%
\pgfpathrectangle{\pgfqpoint{0.647939in}{0.492442in}}{\pgfqpoint{4.273799in}{2.331163in}}%
\pgfusepath{clip}%
\pgfsetroundcap%
\pgfsetroundjoin%
\definecolor{currentfill}{rgb}{0.500000,0.500000,0.500000}%
\pgfsetfillcolor{currentfill}%
\pgfsetfillopacity{0.300000}%
\pgfsetlinewidth{0.301125pt}%
\definecolor{currentstroke}{rgb}{0.500000,0.500000,0.500000}%
\pgfsetstrokecolor{currentstroke}%
\pgfsetstrokeopacity{0.300000}%
\pgfsetdash{}{0pt}%
\pgfpathmoveto{\pgfqpoint{0.000000in}{0.000000in}}%
\pgfpathlineto{\pgfqpoint{0.000000in}{0.000000in}}%
\pgfpathclose%
\pgfusepath{stroke,fill}%
\end{pgfscope}%
\begin{pgfscope}%
\pgfpathrectangle{\pgfqpoint{0.647939in}{0.492442in}}{\pgfqpoint{4.273799in}{2.331163in}}%
\pgfusepath{clip}%
\pgfsetroundcap%
\pgfsetroundjoin%
\pgfsetlinewidth{0.301125pt}%
\definecolor{currentstroke}{rgb}{0.500000,0.500000,0.500000}%
\pgfsetstrokecolor{currentstroke}%
\pgfsetstrokeopacity{0.300000}%
\pgfsetdash{}{0pt}%
\pgfpathmoveto{\pgfqpoint{4.088917in}{2.268930in}}%
\pgfusepath{stroke}%
\end{pgfscope}%
\begin{pgfscope}%
\pgfpathrectangle{\pgfqpoint{0.647939in}{0.492442in}}{\pgfqpoint{4.273799in}{2.331163in}}%
\pgfusepath{clip}%
\pgfsetroundcap%
\pgfsetroundjoin%
\definecolor{currentfill}{rgb}{0.500000,0.500000,0.500000}%
\pgfsetfillcolor{currentfill}%
\pgfsetfillopacity{0.300000}%
\pgfsetlinewidth{0.301125pt}%
\definecolor{currentstroke}{rgb}{0.500000,0.500000,0.500000}%
\pgfsetstrokecolor{currentstroke}%
\pgfsetstrokeopacity{0.300000}%
\pgfsetdash{}{0pt}%
\pgfpathmoveto{\pgfqpoint{0.000000in}{0.000000in}}%
\pgfpathlineto{\pgfqpoint{0.000000in}{0.000000in}}%
\pgfpathclose%
\pgfusepath{stroke,fill}%
\end{pgfscope}%
\begin{pgfscope}%
\pgfpathrectangle{\pgfqpoint{0.647939in}{0.492442in}}{\pgfqpoint{4.273799in}{2.331163in}}%
\pgfusepath{clip}%
\pgfsetroundcap%
\pgfsetroundjoin%
\pgfsetlinewidth{0.301125pt}%
\definecolor{currentstroke}{rgb}{0.500000,0.500000,0.500000}%
\pgfsetstrokecolor{currentstroke}%
\pgfsetstrokeopacity{0.300000}%
\pgfsetdash{}{0pt}%
\pgfpathmoveto{\pgfqpoint{3.967659in}{2.318970in}}%
\pgfusepath{stroke}%
\end{pgfscope}%
\begin{pgfscope}%
\pgfpathrectangle{\pgfqpoint{0.647939in}{0.492442in}}{\pgfqpoint{4.273799in}{2.331163in}}%
\pgfusepath{clip}%
\pgfsetroundcap%
\pgfsetroundjoin%
\definecolor{currentfill}{rgb}{0.500000,0.500000,0.500000}%
\pgfsetfillcolor{currentfill}%
\pgfsetfillopacity{0.300000}%
\pgfsetlinewidth{0.301125pt}%
\definecolor{currentstroke}{rgb}{0.500000,0.500000,0.500000}%
\pgfsetstrokecolor{currentstroke}%
\pgfsetstrokeopacity{0.300000}%
\pgfsetdash{}{0pt}%
\pgfpathmoveto{\pgfqpoint{0.000000in}{0.000000in}}%
\pgfpathlineto{\pgfqpoint{0.000000in}{0.000000in}}%
\pgfpathclose%
\pgfusepath{stroke,fill}%
\end{pgfscope}%
\begin{pgfscope}%
\pgfpathrectangle{\pgfqpoint{0.647939in}{0.492442in}}{\pgfqpoint{4.273799in}{2.331163in}}%
\pgfusepath{clip}%
\pgfsetroundcap%
\pgfsetroundjoin%
\pgfsetlinewidth{0.301125pt}%
\definecolor{currentstroke}{rgb}{0.500000,0.500000,0.500000}%
\pgfsetstrokecolor{currentstroke}%
\pgfsetstrokeopacity{0.300000}%
\pgfsetdash{}{0pt}%
\pgfpathmoveto{\pgfqpoint{3.663845in}{1.709594in}}%
\pgfusepath{stroke}%
\end{pgfscope}%
\begin{pgfscope}%
\pgfpathrectangle{\pgfqpoint{0.647939in}{0.492442in}}{\pgfqpoint{4.273799in}{2.331163in}}%
\pgfusepath{clip}%
\pgfsetroundcap%
\pgfsetroundjoin%
\definecolor{currentfill}{rgb}{0.500000,0.500000,0.500000}%
\pgfsetfillcolor{currentfill}%
\pgfsetfillopacity{0.300000}%
\pgfsetlinewidth{0.301125pt}%
\definecolor{currentstroke}{rgb}{0.500000,0.500000,0.500000}%
\pgfsetstrokecolor{currentstroke}%
\pgfsetstrokeopacity{0.300000}%
\pgfsetdash{}{0pt}%
\pgfpathmoveto{\pgfqpoint{0.000000in}{0.000000in}}%
\pgfpathlineto{\pgfqpoint{0.000000in}{0.000000in}}%
\pgfpathclose%
\pgfusepath{stroke,fill}%
\end{pgfscope}%
\begin{pgfscope}%
\pgfpathrectangle{\pgfqpoint{0.647939in}{0.492442in}}{\pgfqpoint{4.273799in}{2.331163in}}%
\pgfusepath{clip}%
\pgfsetroundcap%
\pgfsetroundjoin%
\pgfsetlinewidth{0.301125pt}%
\definecolor{currentstroke}{rgb}{0.500000,0.500000,0.500000}%
\pgfsetstrokecolor{currentstroke}%
\pgfsetstrokeopacity{0.300000}%
\pgfsetdash{}{0pt}%
\pgfpathmoveto{\pgfqpoint{3.782943in}{2.267436in}}%
\pgfusepath{stroke}%
\end{pgfscope}%
\begin{pgfscope}%
\pgfpathrectangle{\pgfqpoint{0.647939in}{0.492442in}}{\pgfqpoint{4.273799in}{2.331163in}}%
\pgfusepath{clip}%
\pgfsetroundcap%
\pgfsetroundjoin%
\definecolor{currentfill}{rgb}{0.500000,0.500000,0.500000}%
\pgfsetfillcolor{currentfill}%
\pgfsetfillopacity{0.300000}%
\pgfsetlinewidth{0.301125pt}%
\definecolor{currentstroke}{rgb}{0.500000,0.500000,0.500000}%
\pgfsetstrokecolor{currentstroke}%
\pgfsetstrokeopacity{0.300000}%
\pgfsetdash{}{0pt}%
\pgfpathmoveto{\pgfqpoint{0.000000in}{0.000000in}}%
\pgfpathlineto{\pgfqpoint{0.000000in}{0.000000in}}%
\pgfpathclose%
\pgfusepath{stroke,fill}%
\end{pgfscope}%
\begin{pgfscope}%
\pgfpathrectangle{\pgfqpoint{0.647939in}{0.492442in}}{\pgfqpoint{4.273799in}{2.331163in}}%
\pgfusepath{clip}%
\pgfsetroundcap%
\pgfsetroundjoin%
\pgfsetlinewidth{0.301125pt}%
\definecolor{currentstroke}{rgb}{0.500000,0.500000,0.500000}%
\pgfsetstrokecolor{currentstroke}%
\pgfsetstrokeopacity{0.300000}%
\pgfsetdash{}{0pt}%
\pgfpathmoveto{\pgfqpoint{3.531292in}{2.699478in}}%
\pgfusepath{stroke}%
\end{pgfscope}%
\begin{pgfscope}%
\pgfpathrectangle{\pgfqpoint{0.647939in}{0.492442in}}{\pgfqpoint{4.273799in}{2.331163in}}%
\pgfusepath{clip}%
\pgfsetroundcap%
\pgfsetroundjoin%
\definecolor{currentfill}{rgb}{0.500000,0.500000,0.500000}%
\pgfsetfillcolor{currentfill}%
\pgfsetfillopacity{0.300000}%
\pgfsetlinewidth{0.301125pt}%
\definecolor{currentstroke}{rgb}{0.500000,0.500000,0.500000}%
\pgfsetstrokecolor{currentstroke}%
\pgfsetstrokeopacity{0.300000}%
\pgfsetdash{}{0pt}%
\pgfpathmoveto{\pgfqpoint{0.000000in}{0.000000in}}%
\pgfpathlineto{\pgfqpoint{0.000000in}{0.000000in}}%
\pgfpathclose%
\pgfusepath{stroke,fill}%
\end{pgfscope}%
\begin{pgfscope}%
\pgfpathrectangle{\pgfqpoint{0.647939in}{0.492442in}}{\pgfqpoint{4.273799in}{2.331163in}}%
\pgfusepath{clip}%
\pgfsetroundcap%
\pgfsetroundjoin%
\pgfsetlinewidth{0.301125pt}%
\definecolor{currentstroke}{rgb}{0.500000,0.500000,0.500000}%
\pgfsetstrokecolor{currentstroke}%
\pgfsetstrokeopacity{0.300000}%
\pgfsetdash{}{0pt}%
\pgfpathmoveto{\pgfqpoint{3.621606in}{2.114445in}}%
\pgfusepath{stroke}%
\end{pgfscope}%
\begin{pgfscope}%
\pgfpathrectangle{\pgfqpoint{0.647939in}{0.492442in}}{\pgfqpoint{4.273799in}{2.331163in}}%
\pgfusepath{clip}%
\pgfsetroundcap%
\pgfsetroundjoin%
\definecolor{currentfill}{rgb}{0.500000,0.500000,0.500000}%
\pgfsetfillcolor{currentfill}%
\pgfsetfillopacity{0.300000}%
\pgfsetlinewidth{0.301125pt}%
\definecolor{currentstroke}{rgb}{0.500000,0.500000,0.500000}%
\pgfsetstrokecolor{currentstroke}%
\pgfsetstrokeopacity{0.300000}%
\pgfsetdash{}{0pt}%
\pgfpathmoveto{\pgfqpoint{0.000000in}{0.000000in}}%
\pgfpathlineto{\pgfqpoint{0.000000in}{0.000000in}}%
\pgfpathclose%
\pgfusepath{stroke,fill}%
\end{pgfscope}%
\begin{pgfscope}%
\pgfpathrectangle{\pgfqpoint{0.647939in}{0.492442in}}{\pgfqpoint{4.273799in}{2.331163in}}%
\pgfusepath{clip}%
\pgfsetroundcap%
\pgfsetroundjoin%
\pgfsetlinewidth{0.301125pt}%
\definecolor{currentstroke}{rgb}{0.500000,0.500000,0.500000}%
\pgfsetstrokecolor{currentstroke}%
\pgfsetstrokeopacity{0.300000}%
\pgfsetdash{}{0pt}%
\pgfpathmoveto{\pgfqpoint{3.397254in}{2.601175in}}%
\pgfusepath{stroke}%
\end{pgfscope}%
\begin{pgfscope}%
\pgfpathrectangle{\pgfqpoint{0.647939in}{0.492442in}}{\pgfqpoint{4.273799in}{2.331163in}}%
\pgfusepath{clip}%
\pgfsetroundcap%
\pgfsetroundjoin%
\definecolor{currentfill}{rgb}{0.500000,0.500000,0.500000}%
\pgfsetfillcolor{currentfill}%
\pgfsetfillopacity{0.300000}%
\pgfsetlinewidth{0.301125pt}%
\definecolor{currentstroke}{rgb}{0.500000,0.500000,0.500000}%
\pgfsetstrokecolor{currentstroke}%
\pgfsetstrokeopacity{0.300000}%
\pgfsetdash{}{0pt}%
\pgfpathmoveto{\pgfqpoint{0.000000in}{0.000000in}}%
\pgfpathlineto{\pgfqpoint{0.000000in}{0.000000in}}%
\pgfpathclose%
\pgfusepath{stroke,fill}%
\end{pgfscope}%
\begin{pgfscope}%
\pgfpathrectangle{\pgfqpoint{0.647939in}{0.492442in}}{\pgfqpoint{4.273799in}{2.331163in}}%
\pgfusepath{clip}%
\pgfsetroundcap%
\pgfsetroundjoin%
\pgfsetlinewidth{0.301125pt}%
\definecolor{currentstroke}{rgb}{0.500000,0.500000,0.500000}%
\pgfsetstrokecolor{currentstroke}%
\pgfsetstrokeopacity{0.300000}%
\pgfsetdash{}{0pt}%
\pgfpathmoveto{\pgfqpoint{3.389223in}{2.122819in}}%
\pgfusepath{stroke}%
\end{pgfscope}%
\begin{pgfscope}%
\pgfpathrectangle{\pgfqpoint{0.647939in}{0.492442in}}{\pgfqpoint{4.273799in}{2.331163in}}%
\pgfusepath{clip}%
\pgfsetroundcap%
\pgfsetroundjoin%
\definecolor{currentfill}{rgb}{0.500000,0.500000,0.500000}%
\pgfsetfillcolor{currentfill}%
\pgfsetfillopacity{0.300000}%
\pgfsetlinewidth{0.301125pt}%
\definecolor{currentstroke}{rgb}{0.500000,0.500000,0.500000}%
\pgfsetstrokecolor{currentstroke}%
\pgfsetstrokeopacity{0.300000}%
\pgfsetdash{}{0pt}%
\pgfpathmoveto{\pgfqpoint{0.000000in}{0.000000in}}%
\pgfpathlineto{\pgfqpoint{0.000000in}{0.000000in}}%
\pgfpathclose%
\pgfusepath{stroke,fill}%
\end{pgfscope}%
\begin{pgfscope}%
\pgfpathrectangle{\pgfqpoint{0.647939in}{0.492442in}}{\pgfqpoint{4.273799in}{2.331163in}}%
\pgfusepath{clip}%
\pgfsetroundcap%
\pgfsetroundjoin%
\pgfsetlinewidth{0.301125pt}%
\definecolor{currentstroke}{rgb}{0.500000,0.500000,0.500000}%
\pgfsetstrokecolor{currentstroke}%
\pgfsetstrokeopacity{0.300000}%
\pgfsetdash{}{0pt}%
\pgfpathmoveto{\pgfqpoint{3.106694in}{2.657811in}}%
\pgfusepath{stroke}%
\end{pgfscope}%
\begin{pgfscope}%
\pgfpathrectangle{\pgfqpoint{0.647939in}{0.492442in}}{\pgfqpoint{4.273799in}{2.331163in}}%
\pgfusepath{clip}%
\pgfsetroundcap%
\pgfsetroundjoin%
\definecolor{currentfill}{rgb}{0.500000,0.500000,0.500000}%
\pgfsetfillcolor{currentfill}%
\pgfsetfillopacity{0.300000}%
\pgfsetlinewidth{0.301125pt}%
\definecolor{currentstroke}{rgb}{0.500000,0.500000,0.500000}%
\pgfsetstrokecolor{currentstroke}%
\pgfsetstrokeopacity{0.300000}%
\pgfsetdash{}{0pt}%
\pgfpathmoveto{\pgfqpoint{0.000000in}{0.000000in}}%
\pgfpathlineto{\pgfqpoint{0.000000in}{0.000000in}}%
\pgfpathclose%
\pgfusepath{stroke,fill}%
\end{pgfscope}%
\begin{pgfscope}%
\pgfpathrectangle{\pgfqpoint{0.647939in}{0.492442in}}{\pgfqpoint{4.273799in}{2.331163in}}%
\pgfusepath{clip}%
\pgfsetroundcap%
\pgfsetroundjoin%
\pgfsetlinewidth{0.301125pt}%
\definecolor{currentstroke}{rgb}{0.500000,0.500000,0.500000}%
\pgfsetstrokecolor{currentstroke}%
\pgfsetstrokeopacity{0.300000}%
\pgfsetdash{}{0pt}%
\pgfpathmoveto{\pgfqpoint{3.150908in}{2.347730in}}%
\pgfusepath{stroke}%
\end{pgfscope}%
\begin{pgfscope}%
\pgfpathrectangle{\pgfqpoint{0.647939in}{0.492442in}}{\pgfqpoint{4.273799in}{2.331163in}}%
\pgfusepath{clip}%
\pgfsetroundcap%
\pgfsetroundjoin%
\definecolor{currentfill}{rgb}{0.500000,0.500000,0.500000}%
\pgfsetfillcolor{currentfill}%
\pgfsetfillopacity{0.300000}%
\pgfsetlinewidth{0.301125pt}%
\definecolor{currentstroke}{rgb}{0.500000,0.500000,0.500000}%
\pgfsetstrokecolor{currentstroke}%
\pgfsetstrokeopacity{0.300000}%
\pgfsetdash{}{0pt}%
\pgfpathmoveto{\pgfqpoint{0.000000in}{0.000000in}}%
\pgfpathlineto{\pgfqpoint{0.000000in}{0.000000in}}%
\pgfpathclose%
\pgfusepath{stroke,fill}%
\end{pgfscope}%
\begin{pgfscope}%
\pgfpathrectangle{\pgfqpoint{0.647939in}{0.492442in}}{\pgfqpoint{4.273799in}{2.331163in}}%
\pgfusepath{clip}%
\pgfsetroundcap%
\pgfsetroundjoin%
\pgfsetlinewidth{0.301125pt}%
\definecolor{currentstroke}{rgb}{0.500000,0.500000,0.500000}%
\pgfsetstrokecolor{currentstroke}%
\pgfsetstrokeopacity{0.300000}%
\pgfsetdash{}{0pt}%
\pgfpathmoveto{\pgfqpoint{2.917591in}{2.542242in}}%
\pgfusepath{stroke}%
\end{pgfscope}%
\begin{pgfscope}%
\pgfpathrectangle{\pgfqpoint{0.647939in}{0.492442in}}{\pgfqpoint{4.273799in}{2.331163in}}%
\pgfusepath{clip}%
\pgfsetroundcap%
\pgfsetroundjoin%
\definecolor{currentfill}{rgb}{0.500000,0.500000,0.500000}%
\pgfsetfillcolor{currentfill}%
\pgfsetfillopacity{0.300000}%
\pgfsetlinewidth{0.301125pt}%
\definecolor{currentstroke}{rgb}{0.500000,0.500000,0.500000}%
\pgfsetstrokecolor{currentstroke}%
\pgfsetstrokeopacity{0.300000}%
\pgfsetdash{}{0pt}%
\pgfpathmoveto{\pgfqpoint{0.000000in}{0.000000in}}%
\pgfpathlineto{\pgfqpoint{0.000000in}{0.000000in}}%
\pgfpathclose%
\pgfusepath{stroke,fill}%
\end{pgfscope}%
\begin{pgfscope}%
\pgfpathrectangle{\pgfqpoint{0.647939in}{0.492442in}}{\pgfqpoint{4.273799in}{2.331163in}}%
\pgfusepath{clip}%
\pgfsetroundcap%
\pgfsetroundjoin%
\pgfsetlinewidth{0.301125pt}%
\definecolor{currentstroke}{rgb}{0.500000,0.500000,0.500000}%
\pgfsetstrokecolor{currentstroke}%
\pgfsetstrokeopacity{0.300000}%
\pgfsetdash{}{0pt}%
\pgfpathmoveto{\pgfqpoint{2.679595in}{2.657072in}}%
\pgfusepath{stroke}%
\end{pgfscope}%
\begin{pgfscope}%
\pgfpathrectangle{\pgfqpoint{0.647939in}{0.492442in}}{\pgfqpoint{4.273799in}{2.331163in}}%
\pgfusepath{clip}%
\pgfsetroundcap%
\pgfsetroundjoin%
\definecolor{currentfill}{rgb}{0.500000,0.500000,0.500000}%
\pgfsetfillcolor{currentfill}%
\pgfsetfillopacity{0.300000}%
\pgfsetlinewidth{0.301125pt}%
\definecolor{currentstroke}{rgb}{0.500000,0.500000,0.500000}%
\pgfsetstrokecolor{currentstroke}%
\pgfsetstrokeopacity{0.300000}%
\pgfsetdash{}{0pt}%
\pgfpathmoveto{\pgfqpoint{0.000000in}{0.000000in}}%
\pgfpathlineto{\pgfqpoint{0.000000in}{0.000000in}}%
\pgfpathclose%
\pgfusepath{stroke,fill}%
\end{pgfscope}%
\begin{pgfscope}%
\pgfpathrectangle{\pgfqpoint{0.647939in}{0.492442in}}{\pgfqpoint{4.273799in}{2.331163in}}%
\pgfusepath{clip}%
\pgfsetroundcap%
\pgfsetroundjoin%
\pgfsetlinewidth{0.301125pt}%
\definecolor{currentstroke}{rgb}{0.500000,0.500000,0.500000}%
\pgfsetstrokecolor{currentstroke}%
\pgfsetstrokeopacity{0.300000}%
\pgfsetdash{}{0pt}%
\pgfpathmoveto{\pgfqpoint{2.449496in}{2.714574in}}%
\pgfusepath{stroke}%
\end{pgfscope}%
\begin{pgfscope}%
\pgfpathrectangle{\pgfqpoint{0.647939in}{0.492442in}}{\pgfqpoint{4.273799in}{2.331163in}}%
\pgfusepath{clip}%
\pgfsetroundcap%
\pgfsetroundjoin%
\definecolor{currentfill}{rgb}{0.500000,0.500000,0.500000}%
\pgfsetfillcolor{currentfill}%
\pgfsetfillopacity{0.300000}%
\pgfsetlinewidth{0.301125pt}%
\definecolor{currentstroke}{rgb}{0.500000,0.500000,0.500000}%
\pgfsetstrokecolor{currentstroke}%
\pgfsetstrokeopacity{0.300000}%
\pgfsetdash{}{0pt}%
\pgfpathmoveto{\pgfqpoint{0.000000in}{0.000000in}}%
\pgfpathlineto{\pgfqpoint{0.000000in}{0.000000in}}%
\pgfpathclose%
\pgfusepath{stroke,fill}%
\end{pgfscope}%
\begin{pgfscope}%
\pgfpathrectangle{\pgfqpoint{0.647939in}{0.492442in}}{\pgfqpoint{4.273799in}{2.331163in}}%
\pgfusepath{clip}%
\pgfsetroundcap%
\pgfsetroundjoin%
\pgfsetlinewidth{0.301125pt}%
\definecolor{currentstroke}{rgb}{0.500000,0.500000,0.500000}%
\pgfsetstrokecolor{currentstroke}%
\pgfsetstrokeopacity{0.300000}%
\pgfsetdash{}{0pt}%
\pgfpathmoveto{\pgfqpoint{2.258154in}{2.718765in}}%
\pgfusepath{stroke}%
\end{pgfscope}%
\begin{pgfscope}%
\pgfpathrectangle{\pgfqpoint{0.647939in}{0.492442in}}{\pgfqpoint{4.273799in}{2.331163in}}%
\pgfusepath{clip}%
\pgfsetroundcap%
\pgfsetroundjoin%
\definecolor{currentfill}{rgb}{0.500000,0.500000,0.500000}%
\pgfsetfillcolor{currentfill}%
\pgfsetfillopacity{0.300000}%
\pgfsetlinewidth{0.301125pt}%
\definecolor{currentstroke}{rgb}{0.500000,0.500000,0.500000}%
\pgfsetstrokecolor{currentstroke}%
\pgfsetstrokeopacity{0.300000}%
\pgfsetdash{}{0pt}%
\pgfpathmoveto{\pgfqpoint{0.000000in}{0.000000in}}%
\pgfpathlineto{\pgfqpoint{0.000000in}{0.000000in}}%
\pgfpathclose%
\pgfusepath{stroke,fill}%
\end{pgfscope}%
\begin{pgfscope}%
\pgfpathrectangle{\pgfqpoint{0.647939in}{0.492442in}}{\pgfqpoint{4.273799in}{2.331163in}}%
\pgfusepath{clip}%
\pgfsetroundcap%
\pgfsetroundjoin%
\pgfsetlinewidth{0.301125pt}%
\definecolor{currentstroke}{rgb}{0.500000,0.500000,0.500000}%
\pgfsetstrokecolor{currentstroke}%
\pgfsetstrokeopacity{0.300000}%
\pgfsetdash{}{0pt}%
\pgfpathmoveto{\pgfqpoint{2.365221in}{2.603312in}}%
\pgfusepath{stroke}%
\end{pgfscope}%
\begin{pgfscope}%
\pgfpathrectangle{\pgfqpoint{0.647939in}{0.492442in}}{\pgfqpoint{4.273799in}{2.331163in}}%
\pgfusepath{clip}%
\pgfsetroundcap%
\pgfsetroundjoin%
\definecolor{currentfill}{rgb}{0.500000,0.500000,0.500000}%
\pgfsetfillcolor{currentfill}%
\pgfsetfillopacity{0.300000}%
\pgfsetlinewidth{0.301125pt}%
\definecolor{currentstroke}{rgb}{0.500000,0.500000,0.500000}%
\pgfsetstrokecolor{currentstroke}%
\pgfsetstrokeopacity{0.300000}%
\pgfsetdash{}{0pt}%
\pgfpathmoveto{\pgfqpoint{0.000000in}{0.000000in}}%
\pgfpathlineto{\pgfqpoint{0.000000in}{0.000000in}}%
\pgfpathclose%
\pgfusepath{stroke,fill}%
\end{pgfscope}%
\begin{pgfscope}%
\pgfpathrectangle{\pgfqpoint{0.647939in}{0.492442in}}{\pgfqpoint{4.273799in}{2.331163in}}%
\pgfusepath{clip}%
\pgfsetroundcap%
\pgfsetroundjoin%
\pgfsetlinewidth{0.301125pt}%
\definecolor{currentstroke}{rgb}{0.500000,0.500000,0.500000}%
\pgfsetstrokecolor{currentstroke}%
\pgfsetstrokeopacity{0.300000}%
\pgfsetdash{}{0pt}%
\pgfpathmoveto{\pgfqpoint{1.975839in}{2.651937in}}%
\pgfusepath{stroke}%
\end{pgfscope}%
\begin{pgfscope}%
\pgfpathrectangle{\pgfqpoint{0.647939in}{0.492442in}}{\pgfqpoint{4.273799in}{2.331163in}}%
\pgfusepath{clip}%
\pgfsetroundcap%
\pgfsetroundjoin%
\definecolor{currentfill}{rgb}{0.500000,0.500000,0.500000}%
\pgfsetfillcolor{currentfill}%
\pgfsetfillopacity{0.300000}%
\pgfsetlinewidth{0.301125pt}%
\definecolor{currentstroke}{rgb}{0.500000,0.500000,0.500000}%
\pgfsetstrokecolor{currentstroke}%
\pgfsetstrokeopacity{0.300000}%
\pgfsetdash{}{0pt}%
\pgfpathmoveto{\pgfqpoint{0.000000in}{0.000000in}}%
\pgfpathlineto{\pgfqpoint{0.000000in}{0.000000in}}%
\pgfpathclose%
\pgfusepath{stroke,fill}%
\end{pgfscope}%
\begin{pgfscope}%
\pgfpathrectangle{\pgfqpoint{0.647939in}{0.492442in}}{\pgfqpoint{4.273799in}{2.331163in}}%
\pgfusepath{clip}%
\pgfsetroundcap%
\pgfsetroundjoin%
\pgfsetlinewidth{0.301125pt}%
\definecolor{currentstroke}{rgb}{0.500000,0.500000,0.500000}%
\pgfsetstrokecolor{currentstroke}%
\pgfsetstrokeopacity{0.300000}%
\pgfsetdash{}{0pt}%
\pgfpathmoveto{\pgfqpoint{2.055158in}{2.546286in}}%
\pgfusepath{stroke}%
\end{pgfscope}%
\begin{pgfscope}%
\pgfpathrectangle{\pgfqpoint{0.647939in}{0.492442in}}{\pgfqpoint{4.273799in}{2.331163in}}%
\pgfusepath{clip}%
\pgfsetroundcap%
\pgfsetroundjoin%
\definecolor{currentfill}{rgb}{0.500000,0.500000,0.500000}%
\pgfsetfillcolor{currentfill}%
\pgfsetfillopacity{0.300000}%
\pgfsetlinewidth{0.301125pt}%
\definecolor{currentstroke}{rgb}{0.500000,0.500000,0.500000}%
\pgfsetstrokecolor{currentstroke}%
\pgfsetstrokeopacity{0.300000}%
\pgfsetdash{}{0pt}%
\pgfpathmoveto{\pgfqpoint{0.000000in}{0.000000in}}%
\pgfpathlineto{\pgfqpoint{0.000000in}{0.000000in}}%
\pgfpathclose%
\pgfusepath{stroke,fill}%
\end{pgfscope}%
\begin{pgfscope}%
\pgfpathrectangle{\pgfqpoint{0.647939in}{0.492442in}}{\pgfqpoint{4.273799in}{2.331163in}}%
\pgfusepath{clip}%
\pgfsetroundcap%
\pgfsetroundjoin%
\pgfsetlinewidth{0.301125pt}%
\definecolor{currentstroke}{rgb}{0.500000,0.500000,0.500000}%
\pgfsetstrokecolor{currentstroke}%
\pgfsetstrokeopacity{0.300000}%
\pgfsetdash{}{0pt}%
\pgfpathmoveto{\pgfqpoint{1.784811in}{2.527437in}}%
\pgfusepath{stroke}%
\end{pgfscope}%
\begin{pgfscope}%
\pgfpathrectangle{\pgfqpoint{0.647939in}{0.492442in}}{\pgfqpoint{4.273799in}{2.331163in}}%
\pgfusepath{clip}%
\pgfsetroundcap%
\pgfsetroundjoin%
\definecolor{currentfill}{rgb}{0.500000,0.500000,0.500000}%
\pgfsetfillcolor{currentfill}%
\pgfsetfillopacity{0.300000}%
\pgfsetlinewidth{0.301125pt}%
\definecolor{currentstroke}{rgb}{0.500000,0.500000,0.500000}%
\pgfsetstrokecolor{currentstroke}%
\pgfsetstrokeopacity{0.300000}%
\pgfsetdash{}{0pt}%
\pgfpathmoveto{\pgfqpoint{0.000000in}{0.000000in}}%
\pgfpathlineto{\pgfqpoint{0.000000in}{0.000000in}}%
\pgfpathclose%
\pgfusepath{stroke,fill}%
\end{pgfscope}%
\begin{pgfscope}%
\pgfpathrectangle{\pgfqpoint{0.647939in}{0.492442in}}{\pgfqpoint{4.273799in}{2.331163in}}%
\pgfusepath{clip}%
\pgfsetroundcap%
\pgfsetroundjoin%
\pgfsetlinewidth{0.301125pt}%
\definecolor{currentstroke}{rgb}{0.500000,0.500000,0.500000}%
\pgfsetstrokecolor{currentstroke}%
\pgfsetstrokeopacity{0.300000}%
\pgfsetdash{}{0pt}%
\pgfpathmoveto{\pgfqpoint{1.615327in}{2.505235in}}%
\pgfusepath{stroke}%
\end{pgfscope}%
\begin{pgfscope}%
\pgfpathrectangle{\pgfqpoint{0.647939in}{0.492442in}}{\pgfqpoint{4.273799in}{2.331163in}}%
\pgfusepath{clip}%
\pgfsetroundcap%
\pgfsetroundjoin%
\definecolor{currentfill}{rgb}{0.500000,0.500000,0.500000}%
\pgfsetfillcolor{currentfill}%
\pgfsetfillopacity{0.300000}%
\pgfsetlinewidth{0.301125pt}%
\definecolor{currentstroke}{rgb}{0.500000,0.500000,0.500000}%
\pgfsetstrokecolor{currentstroke}%
\pgfsetstrokeopacity{0.300000}%
\pgfsetdash{}{0pt}%
\pgfpathmoveto{\pgfqpoint{0.000000in}{0.000000in}}%
\pgfpathlineto{\pgfqpoint{0.000000in}{0.000000in}}%
\pgfpathclose%
\pgfusepath{stroke,fill}%
\end{pgfscope}%
\begin{pgfscope}%
\pgfpathrectangle{\pgfqpoint{0.647939in}{0.492442in}}{\pgfqpoint{4.273799in}{2.331163in}}%
\pgfusepath{clip}%
\pgfsetroundcap%
\pgfsetroundjoin%
\pgfsetlinewidth{0.301125pt}%
\definecolor{currentstroke}{rgb}{0.500000,0.500000,0.500000}%
\pgfsetstrokecolor{currentstroke}%
\pgfsetstrokeopacity{0.300000}%
\pgfsetdash{}{0pt}%
\pgfpathmoveto{\pgfqpoint{1.523383in}{2.369770in}}%
\pgfusepath{stroke}%
\end{pgfscope}%
\begin{pgfscope}%
\pgfpathrectangle{\pgfqpoint{0.647939in}{0.492442in}}{\pgfqpoint{4.273799in}{2.331163in}}%
\pgfusepath{clip}%
\pgfsetroundcap%
\pgfsetroundjoin%
\definecolor{currentfill}{rgb}{0.500000,0.500000,0.500000}%
\pgfsetfillcolor{currentfill}%
\pgfsetfillopacity{0.300000}%
\pgfsetlinewidth{0.301125pt}%
\definecolor{currentstroke}{rgb}{0.500000,0.500000,0.500000}%
\pgfsetstrokecolor{currentstroke}%
\pgfsetstrokeopacity{0.300000}%
\pgfsetdash{}{0pt}%
\pgfpathmoveto{\pgfqpoint{0.000000in}{0.000000in}}%
\pgfpathlineto{\pgfqpoint{0.000000in}{0.000000in}}%
\pgfpathclose%
\pgfusepath{stroke,fill}%
\end{pgfscope}%
\begin{pgfscope}%
\pgfpathrectangle{\pgfqpoint{0.647939in}{0.492442in}}{\pgfqpoint{4.273799in}{2.331163in}}%
\pgfusepath{clip}%
\pgfsetroundcap%
\pgfsetroundjoin%
\pgfsetlinewidth{0.301125pt}%
\definecolor{currentstroke}{rgb}{0.500000,0.500000,0.500000}%
\pgfsetstrokecolor{currentstroke}%
\pgfsetstrokeopacity{0.300000}%
\pgfsetdash{}{0pt}%
\pgfpathmoveto{\pgfqpoint{1.386839in}{2.312743in}}%
\pgfusepath{stroke}%
\end{pgfscope}%
\begin{pgfscope}%
\pgfpathrectangle{\pgfqpoint{0.647939in}{0.492442in}}{\pgfqpoint{4.273799in}{2.331163in}}%
\pgfusepath{clip}%
\pgfsetroundcap%
\pgfsetroundjoin%
\definecolor{currentfill}{rgb}{0.500000,0.500000,0.500000}%
\pgfsetfillcolor{currentfill}%
\pgfsetfillopacity{0.300000}%
\pgfsetlinewidth{0.301125pt}%
\definecolor{currentstroke}{rgb}{0.500000,0.500000,0.500000}%
\pgfsetstrokecolor{currentstroke}%
\pgfsetstrokeopacity{0.300000}%
\pgfsetdash{}{0pt}%
\pgfpathmoveto{\pgfqpoint{0.000000in}{0.000000in}}%
\pgfpathlineto{\pgfqpoint{0.000000in}{0.000000in}}%
\pgfpathclose%
\pgfusepath{stroke,fill}%
\end{pgfscope}%
\begin{pgfscope}%
\pgfpathrectangle{\pgfqpoint{0.647939in}{0.492442in}}{\pgfqpoint{4.273799in}{2.331163in}}%
\pgfusepath{clip}%
\pgfsetroundcap%
\pgfsetroundjoin%
\pgfsetlinewidth{0.301125pt}%
\definecolor{currentstroke}{rgb}{0.500000,0.500000,0.500000}%
\pgfsetstrokecolor{currentstroke}%
\pgfsetstrokeopacity{0.300000}%
\pgfsetdash{}{0pt}%
\pgfpathmoveto{\pgfqpoint{1.222140in}{1.793485in}}%
\pgfusepath{stroke}%
\end{pgfscope}%
\begin{pgfscope}%
\pgfpathrectangle{\pgfqpoint{0.647939in}{0.492442in}}{\pgfqpoint{4.273799in}{2.331163in}}%
\pgfusepath{clip}%
\pgfsetroundcap%
\pgfsetroundjoin%
\definecolor{currentfill}{rgb}{0.500000,0.500000,0.500000}%
\pgfsetfillcolor{currentfill}%
\pgfsetfillopacity{0.300000}%
\pgfsetlinewidth{0.301125pt}%
\definecolor{currentstroke}{rgb}{0.500000,0.500000,0.500000}%
\pgfsetstrokecolor{currentstroke}%
\pgfsetstrokeopacity{0.300000}%
\pgfsetdash{}{0pt}%
\pgfpathmoveto{\pgfqpoint{0.000000in}{0.000000in}}%
\pgfpathlineto{\pgfqpoint{0.000000in}{0.000000in}}%
\pgfpathclose%
\pgfusepath{stroke,fill}%
\end{pgfscope}%
\begin{pgfscope}%
\pgfpathrectangle{\pgfqpoint{0.647939in}{0.492442in}}{\pgfqpoint{4.273799in}{2.331163in}}%
\pgfusepath{clip}%
\pgfsetroundcap%
\pgfsetroundjoin%
\pgfsetlinewidth{0.301125pt}%
\definecolor{currentstroke}{rgb}{0.500000,0.500000,0.500000}%
\pgfsetstrokecolor{currentstroke}%
\pgfsetstrokeopacity{0.300000}%
\pgfsetdash{}{0pt}%
\pgfpathmoveto{\pgfqpoint{1.142195in}{2.153028in}}%
\pgfusepath{stroke}%
\end{pgfscope}%
\begin{pgfscope}%
\pgfpathrectangle{\pgfqpoint{0.647939in}{0.492442in}}{\pgfqpoint{4.273799in}{2.331163in}}%
\pgfusepath{clip}%
\pgfsetroundcap%
\pgfsetroundjoin%
\definecolor{currentfill}{rgb}{0.500000,0.500000,0.500000}%
\pgfsetfillcolor{currentfill}%
\pgfsetfillopacity{0.300000}%
\pgfsetlinewidth{0.301125pt}%
\definecolor{currentstroke}{rgb}{0.500000,0.500000,0.500000}%
\pgfsetstrokecolor{currentstroke}%
\pgfsetstrokeopacity{0.300000}%
\pgfsetdash{}{0pt}%
\pgfpathmoveto{\pgfqpoint{0.000000in}{0.000000in}}%
\pgfpathlineto{\pgfqpoint{0.000000in}{0.000000in}}%
\pgfpathclose%
\pgfusepath{stroke,fill}%
\end{pgfscope}%
\begin{pgfscope}%
\pgfpathrectangle{\pgfqpoint{0.647939in}{0.492442in}}{\pgfqpoint{4.273799in}{2.331163in}}%
\pgfusepath{clip}%
\pgfsetroundcap%
\pgfsetroundjoin%
\pgfsetlinewidth{0.301125pt}%
\definecolor{currentstroke}{rgb}{0.500000,0.500000,0.500000}%
\pgfsetstrokecolor{currentstroke}%
\pgfsetstrokeopacity{0.300000}%
\pgfsetdash{}{0pt}%
\pgfpathmoveto{\pgfqpoint{1.014859in}{1.893167in}}%
\pgfusepath{stroke}%
\end{pgfscope}%
\begin{pgfscope}%
\pgfpathrectangle{\pgfqpoint{0.647939in}{0.492442in}}{\pgfqpoint{4.273799in}{2.331163in}}%
\pgfusepath{clip}%
\pgfsetroundcap%
\pgfsetroundjoin%
\definecolor{currentfill}{rgb}{0.500000,0.500000,0.500000}%
\pgfsetfillcolor{currentfill}%
\pgfsetfillopacity{0.300000}%
\pgfsetlinewidth{0.301125pt}%
\definecolor{currentstroke}{rgb}{0.500000,0.500000,0.500000}%
\pgfsetstrokecolor{currentstroke}%
\pgfsetstrokeopacity{0.300000}%
\pgfsetdash{}{0pt}%
\pgfpathmoveto{\pgfqpoint{0.000000in}{0.000000in}}%
\pgfpathlineto{\pgfqpoint{0.000000in}{0.000000in}}%
\pgfpathclose%
\pgfusepath{stroke,fill}%
\end{pgfscope}%
\begin{pgfscope}%
\pgfpathrectangle{\pgfqpoint{0.647939in}{0.492442in}}{\pgfqpoint{4.273799in}{2.331163in}}%
\pgfusepath{clip}%
\pgfsetroundcap%
\pgfsetroundjoin%
\pgfsetlinewidth{0.301125pt}%
\definecolor{currentstroke}{rgb}{0.500000,0.500000,0.500000}%
\pgfsetstrokecolor{currentstroke}%
\pgfsetstrokeopacity{0.300000}%
\pgfsetdash{}{0pt}%
\pgfpathmoveto{\pgfqpoint{0.913342in}{2.099683in}}%
\pgfusepath{stroke}%
\end{pgfscope}%
\begin{pgfscope}%
\pgfpathrectangle{\pgfqpoint{0.647939in}{0.492442in}}{\pgfqpoint{4.273799in}{2.331163in}}%
\pgfusepath{clip}%
\pgfsetroundcap%
\pgfsetroundjoin%
\definecolor{currentfill}{rgb}{0.500000,0.500000,0.500000}%
\pgfsetfillcolor{currentfill}%
\pgfsetfillopacity{0.300000}%
\pgfsetlinewidth{0.301125pt}%
\definecolor{currentstroke}{rgb}{0.500000,0.500000,0.500000}%
\pgfsetstrokecolor{currentstroke}%
\pgfsetstrokeopacity{0.300000}%
\pgfsetdash{}{0pt}%
\pgfpathmoveto{\pgfqpoint{0.000000in}{0.000000in}}%
\pgfpathlineto{\pgfqpoint{0.000000in}{0.000000in}}%
\pgfpathclose%
\pgfusepath{stroke,fill}%
\end{pgfscope}%
\begin{pgfscope}%
\pgfpathrectangle{\pgfqpoint{0.647939in}{0.492442in}}{\pgfqpoint{4.273799in}{2.331163in}}%
\pgfusepath{clip}%
\pgfsetroundcap%
\pgfsetroundjoin%
\pgfsetlinewidth{0.301125pt}%
\definecolor{currentstroke}{rgb}{0.500000,0.500000,0.500000}%
\pgfsetstrokecolor{currentstroke}%
\pgfsetstrokeopacity{0.300000}%
\pgfsetdash{}{0pt}%
\pgfpathmoveto{\pgfqpoint{0.801035in}{1.943908in}}%
\pgfusepath{stroke}%
\end{pgfscope}%
\begin{pgfscope}%
\pgfpathrectangle{\pgfqpoint{0.647939in}{0.492442in}}{\pgfqpoint{4.273799in}{2.331163in}}%
\pgfusepath{clip}%
\pgfsetroundcap%
\pgfsetroundjoin%
\definecolor{currentfill}{rgb}{0.500000,0.500000,0.500000}%
\pgfsetfillcolor{currentfill}%
\pgfsetfillopacity{0.300000}%
\pgfsetlinewidth{0.301125pt}%
\definecolor{currentstroke}{rgb}{0.500000,0.500000,0.500000}%
\pgfsetstrokecolor{currentstroke}%
\pgfsetstrokeopacity{0.300000}%
\pgfsetdash{}{0pt}%
\pgfpathmoveto{\pgfqpoint{0.000000in}{0.000000in}}%
\pgfpathlineto{\pgfqpoint{0.000000in}{0.000000in}}%
\pgfpathclose%
\pgfusepath{stroke,fill}%
\end{pgfscope}%
\begin{pgfscope}%
\pgfpathrectangle{\pgfqpoint{0.647939in}{0.492442in}}{\pgfqpoint{4.273799in}{2.331163in}}%
\pgfusepath{clip}%
\pgfsetroundcap%
\pgfsetroundjoin%
\pgfsetlinewidth{0.301125pt}%
\definecolor{currentstroke}{rgb}{0.500000,0.500000,0.500000}%
\pgfsetstrokecolor{currentstroke}%
\pgfsetstrokeopacity{0.300000}%
\pgfsetdash{}{0pt}%
\pgfpathmoveto{\pgfqpoint{0.698530in}{2.150847in}}%
\pgfusepath{stroke}%
\end{pgfscope}%
\begin{pgfscope}%
\pgfpathrectangle{\pgfqpoint{0.647939in}{0.492442in}}{\pgfqpoint{4.273799in}{2.331163in}}%
\pgfusepath{clip}%
\pgfsetroundcap%
\pgfsetroundjoin%
\definecolor{currentfill}{rgb}{0.500000,0.500000,0.500000}%
\pgfsetfillcolor{currentfill}%
\pgfsetfillopacity{0.300000}%
\pgfsetlinewidth{0.301125pt}%
\definecolor{currentstroke}{rgb}{0.500000,0.500000,0.500000}%
\pgfsetstrokecolor{currentstroke}%
\pgfsetstrokeopacity{0.300000}%
\pgfsetdash{}{0pt}%
\pgfpathmoveto{\pgfqpoint{0.000000in}{0.000000in}}%
\pgfpathlineto{\pgfqpoint{0.000000in}{0.000000in}}%
\pgfpathclose%
\pgfusepath{stroke,fill}%
\end{pgfscope}%
\begin{pgfscope}%
\pgfpathrectangle{\pgfqpoint{0.647939in}{0.492442in}}{\pgfqpoint{4.273799in}{2.331163in}}%
\pgfusepath{clip}%
\pgfsetroundcap%
\pgfsetroundjoin%
\pgfsetlinewidth{0.301125pt}%
\definecolor{currentstroke}{rgb}{0.500000,0.500000,0.500000}%
\pgfsetstrokecolor{currentstroke}%
\pgfsetstrokeopacity{0.300000}%
\pgfsetdash{}{0pt}%
\pgfpathmoveto{\pgfqpoint{3.690114in}{0.848321in}}%
\pgfusepath{stroke}%
\end{pgfscope}%
\begin{pgfscope}%
\pgfpathrectangle{\pgfqpoint{0.647939in}{0.492442in}}{\pgfqpoint{4.273799in}{2.331163in}}%
\pgfusepath{clip}%
\pgfsetroundcap%
\pgfsetroundjoin%
\definecolor{currentfill}{rgb}{0.500000,0.500000,0.500000}%
\pgfsetfillcolor{currentfill}%
\pgfsetfillopacity{0.300000}%
\pgfsetlinewidth{0.301125pt}%
\definecolor{currentstroke}{rgb}{0.500000,0.500000,0.500000}%
\pgfsetstrokecolor{currentstroke}%
\pgfsetstrokeopacity{0.300000}%
\pgfsetdash{}{0pt}%
\pgfpathmoveto{\pgfqpoint{0.000000in}{0.000000in}}%
\pgfpathlineto{\pgfqpoint{0.000000in}{0.000000in}}%
\pgfpathclose%
\pgfusepath{stroke,fill}%
\end{pgfscope}%
\begin{pgfscope}%
\pgfpathrectangle{\pgfqpoint{0.647939in}{0.492442in}}{\pgfqpoint{4.273799in}{2.331163in}}%
\pgfusepath{clip}%
\pgfsetroundcap%
\pgfsetroundjoin%
\pgfsetlinewidth{0.301125pt}%
\definecolor{currentstroke}{rgb}{0.500000,0.500000,0.500000}%
\pgfsetstrokecolor{currentstroke}%
\pgfsetstrokeopacity{0.300000}%
\pgfsetdash{}{0pt}%
\pgfpathmoveto{\pgfqpoint{4.600720in}{2.024471in}}%
\pgfusepath{stroke}%
\end{pgfscope}%
\begin{pgfscope}%
\pgfpathrectangle{\pgfqpoint{0.647939in}{0.492442in}}{\pgfqpoint{4.273799in}{2.331163in}}%
\pgfusepath{clip}%
\pgfsetroundcap%
\pgfsetroundjoin%
\definecolor{currentfill}{rgb}{0.500000,0.500000,0.500000}%
\pgfsetfillcolor{currentfill}%
\pgfsetfillopacity{0.300000}%
\pgfsetlinewidth{0.301125pt}%
\definecolor{currentstroke}{rgb}{0.500000,0.500000,0.500000}%
\pgfsetstrokecolor{currentstroke}%
\pgfsetstrokeopacity{0.300000}%
\pgfsetdash{}{0pt}%
\pgfpathmoveto{\pgfqpoint{0.000000in}{0.000000in}}%
\pgfpathlineto{\pgfqpoint{0.000000in}{0.000000in}}%
\pgfpathclose%
\pgfusepath{stroke,fill}%
\end{pgfscope}%
\begin{pgfscope}%
\pgfpathrectangle{\pgfqpoint{0.647939in}{0.492442in}}{\pgfqpoint{4.273799in}{2.331163in}}%
\pgfusepath{clip}%
\pgfsetroundcap%
\pgfsetroundjoin%
\pgfsetlinewidth{0.301125pt}%
\definecolor{currentstroke}{rgb}{0.500000,0.500000,0.500000}%
\pgfsetstrokecolor{currentstroke}%
\pgfsetstrokeopacity{0.300000}%
\pgfsetdash{}{0pt}%
\pgfpathmoveto{\pgfqpoint{4.184706in}{1.401065in}}%
\pgfusepath{stroke}%
\end{pgfscope}%
\begin{pgfscope}%
\pgfpathrectangle{\pgfqpoint{0.647939in}{0.492442in}}{\pgfqpoint{4.273799in}{2.331163in}}%
\pgfusepath{clip}%
\pgfsetroundcap%
\pgfsetroundjoin%
\definecolor{currentfill}{rgb}{0.500000,0.500000,0.500000}%
\pgfsetfillcolor{currentfill}%
\pgfsetfillopacity{0.300000}%
\pgfsetlinewidth{0.301125pt}%
\definecolor{currentstroke}{rgb}{0.500000,0.500000,0.500000}%
\pgfsetstrokecolor{currentstroke}%
\pgfsetstrokeopacity{0.300000}%
\pgfsetdash{}{0pt}%
\pgfpathmoveto{\pgfqpoint{0.000000in}{0.000000in}}%
\pgfpathlineto{\pgfqpoint{0.000000in}{0.000000in}}%
\pgfpathclose%
\pgfusepath{stroke,fill}%
\end{pgfscope}%
\begin{pgfscope}%
\pgfpathrectangle{\pgfqpoint{0.647939in}{0.492442in}}{\pgfqpoint{4.273799in}{2.331163in}}%
\pgfusepath{clip}%
\pgfsetroundcap%
\pgfsetroundjoin%
\pgfsetlinewidth{0.301125pt}%
\definecolor{currentstroke}{rgb}{0.500000,0.500000,0.500000}%
\pgfsetstrokecolor{currentstroke}%
\pgfsetstrokeopacity{0.300000}%
\pgfsetdash{}{0pt}%
\pgfpathmoveto{\pgfqpoint{4.251789in}{1.626800in}}%
\pgfusepath{stroke}%
\end{pgfscope}%
\begin{pgfscope}%
\pgfpathrectangle{\pgfqpoint{0.647939in}{0.492442in}}{\pgfqpoint{4.273799in}{2.331163in}}%
\pgfusepath{clip}%
\pgfsetroundcap%
\pgfsetroundjoin%
\definecolor{currentfill}{rgb}{0.500000,0.500000,0.500000}%
\pgfsetfillcolor{currentfill}%
\pgfsetfillopacity{0.300000}%
\pgfsetlinewidth{0.301125pt}%
\definecolor{currentstroke}{rgb}{0.500000,0.500000,0.500000}%
\pgfsetstrokecolor{currentstroke}%
\pgfsetstrokeopacity{0.300000}%
\pgfsetdash{}{0pt}%
\pgfpathmoveto{\pgfqpoint{0.000000in}{0.000000in}}%
\pgfpathlineto{\pgfqpoint{0.000000in}{0.000000in}}%
\pgfpathclose%
\pgfusepath{stroke,fill}%
\end{pgfscope}%
\begin{pgfscope}%
\pgfpathrectangle{\pgfqpoint{0.647939in}{0.492442in}}{\pgfqpoint{4.273799in}{2.331163in}}%
\pgfusepath{clip}%
\pgfsetroundcap%
\pgfsetroundjoin%
\pgfsetlinewidth{0.301125pt}%
\definecolor{currentstroke}{rgb}{0.500000,0.500000,0.500000}%
\pgfsetstrokecolor{currentstroke}%
\pgfsetstrokeopacity{0.300000}%
\pgfsetdash{}{0pt}%
\pgfpathmoveto{\pgfqpoint{4.431443in}{1.828823in}}%
\pgfusepath{stroke}%
\end{pgfscope}%
\begin{pgfscope}%
\pgfpathrectangle{\pgfqpoint{0.647939in}{0.492442in}}{\pgfqpoint{4.273799in}{2.331163in}}%
\pgfusepath{clip}%
\pgfsetroundcap%
\pgfsetroundjoin%
\definecolor{currentfill}{rgb}{0.500000,0.500000,0.500000}%
\pgfsetfillcolor{currentfill}%
\pgfsetfillopacity{0.300000}%
\pgfsetlinewidth{0.301125pt}%
\definecolor{currentstroke}{rgb}{0.500000,0.500000,0.500000}%
\pgfsetstrokecolor{currentstroke}%
\pgfsetstrokeopacity{0.300000}%
\pgfsetdash{}{0pt}%
\pgfpathmoveto{\pgfqpoint{0.000000in}{0.000000in}}%
\pgfpathlineto{\pgfqpoint{0.000000in}{0.000000in}}%
\pgfpathclose%
\pgfusepath{stroke,fill}%
\end{pgfscope}%
\begin{pgfscope}%
\pgfpathrectangle{\pgfqpoint{0.647939in}{0.492442in}}{\pgfqpoint{4.273799in}{2.331163in}}%
\pgfusepath{clip}%
\pgfsetroundcap%
\pgfsetroundjoin%
\pgfsetlinewidth{0.301125pt}%
\definecolor{currentstroke}{rgb}{0.500000,0.500000,0.500000}%
\pgfsetstrokecolor{currentstroke}%
\pgfsetstrokeopacity{0.300000}%
\pgfsetdash{}{0pt}%
\pgfpathmoveto{\pgfqpoint{2.050868in}{1.502916in}}%
\pgfusepath{stroke}%
\end{pgfscope}%
\begin{pgfscope}%
\pgfpathrectangle{\pgfqpoint{0.647939in}{0.492442in}}{\pgfqpoint{4.273799in}{2.331163in}}%
\pgfusepath{clip}%
\pgfsetroundcap%
\pgfsetroundjoin%
\definecolor{currentfill}{rgb}{0.500000,0.500000,0.500000}%
\pgfsetfillcolor{currentfill}%
\pgfsetfillopacity{0.300000}%
\pgfsetlinewidth{0.301125pt}%
\definecolor{currentstroke}{rgb}{0.500000,0.500000,0.500000}%
\pgfsetstrokecolor{currentstroke}%
\pgfsetstrokeopacity{0.300000}%
\pgfsetdash{}{0pt}%
\pgfpathmoveto{\pgfqpoint{0.000000in}{0.000000in}}%
\pgfpathlineto{\pgfqpoint{0.000000in}{0.000000in}}%
\pgfpathclose%
\pgfusepath{stroke,fill}%
\end{pgfscope}%
\begin{pgfscope}%
\pgfpathrectangle{\pgfqpoint{0.647939in}{0.492442in}}{\pgfqpoint{4.273799in}{2.331163in}}%
\pgfusepath{clip}%
\pgfsetroundcap%
\pgfsetroundjoin%
\pgfsetlinewidth{0.301125pt}%
\definecolor{currentstroke}{rgb}{0.500000,0.500000,0.500000}%
\pgfsetstrokecolor{currentstroke}%
\pgfsetstrokeopacity{0.300000}%
\pgfsetdash{}{0pt}%
\pgfpathmoveto{\pgfqpoint{3.255588in}{0.890660in}}%
\pgfusepath{stroke}%
\end{pgfscope}%
\begin{pgfscope}%
\pgfpathrectangle{\pgfqpoint{0.647939in}{0.492442in}}{\pgfqpoint{4.273799in}{2.331163in}}%
\pgfusepath{clip}%
\pgfsetroundcap%
\pgfsetroundjoin%
\definecolor{currentfill}{rgb}{0.500000,0.500000,0.500000}%
\pgfsetfillcolor{currentfill}%
\pgfsetfillopacity{0.300000}%
\pgfsetlinewidth{0.301125pt}%
\definecolor{currentstroke}{rgb}{0.500000,0.500000,0.500000}%
\pgfsetstrokecolor{currentstroke}%
\pgfsetstrokeopacity{0.300000}%
\pgfsetdash{}{0pt}%
\pgfpathmoveto{\pgfqpoint{0.000000in}{0.000000in}}%
\pgfpathlineto{\pgfqpoint{0.000000in}{0.000000in}}%
\pgfpathclose%
\pgfusepath{stroke,fill}%
\end{pgfscope}%
\begin{pgfscope}%
\pgfpathrectangle{\pgfqpoint{0.647939in}{0.492442in}}{\pgfqpoint{4.273799in}{2.331163in}}%
\pgfusepath{clip}%
\pgfsetroundcap%
\pgfsetroundjoin%
\pgfsetlinewidth{0.301125pt}%
\definecolor{currentstroke}{rgb}{0.500000,0.500000,0.500000}%
\pgfsetstrokecolor{currentstroke}%
\pgfsetstrokeopacity{0.300000}%
\pgfsetdash{}{0pt}%
\pgfpathmoveto{\pgfqpoint{3.407231in}{0.846781in}}%
\pgfusepath{stroke}%
\end{pgfscope}%
\begin{pgfscope}%
\pgfpathrectangle{\pgfqpoint{0.647939in}{0.492442in}}{\pgfqpoint{4.273799in}{2.331163in}}%
\pgfusepath{clip}%
\pgfsetroundcap%
\pgfsetroundjoin%
\definecolor{currentfill}{rgb}{0.500000,0.500000,0.500000}%
\pgfsetfillcolor{currentfill}%
\pgfsetfillopacity{0.300000}%
\pgfsetlinewidth{0.301125pt}%
\definecolor{currentstroke}{rgb}{0.500000,0.500000,0.500000}%
\pgfsetstrokecolor{currentstroke}%
\pgfsetstrokeopacity{0.300000}%
\pgfsetdash{}{0pt}%
\pgfpathmoveto{\pgfqpoint{0.000000in}{0.000000in}}%
\pgfpathlineto{\pgfqpoint{0.000000in}{0.000000in}}%
\pgfpathclose%
\pgfusepath{stroke,fill}%
\end{pgfscope}%
\begin{pgfscope}%
\pgfpathrectangle{\pgfqpoint{0.647939in}{0.492442in}}{\pgfqpoint{4.273799in}{2.331163in}}%
\pgfusepath{clip}%
\pgfsetroundcap%
\pgfsetroundjoin%
\pgfsetlinewidth{0.301125pt}%
\definecolor{currentstroke}{rgb}{0.500000,0.500000,0.500000}%
\pgfsetstrokecolor{currentstroke}%
\pgfsetstrokeopacity{0.300000}%
\pgfsetdash{}{0pt}%
\pgfpathmoveto{\pgfqpoint{3.499788in}{0.843798in}}%
\pgfusepath{stroke}%
\end{pgfscope}%
\begin{pgfscope}%
\pgfpathrectangle{\pgfqpoint{0.647939in}{0.492442in}}{\pgfqpoint{4.273799in}{2.331163in}}%
\pgfusepath{clip}%
\pgfsetroundcap%
\pgfsetroundjoin%
\definecolor{currentfill}{rgb}{0.500000,0.500000,0.500000}%
\pgfsetfillcolor{currentfill}%
\pgfsetfillopacity{0.300000}%
\pgfsetlinewidth{0.301125pt}%
\definecolor{currentstroke}{rgb}{0.500000,0.500000,0.500000}%
\pgfsetstrokecolor{currentstroke}%
\pgfsetstrokeopacity{0.300000}%
\pgfsetdash{}{0pt}%
\pgfpathmoveto{\pgfqpoint{0.000000in}{0.000000in}}%
\pgfpathlineto{\pgfqpoint{0.000000in}{0.000000in}}%
\pgfpathclose%
\pgfusepath{stroke,fill}%
\end{pgfscope}%
\begin{pgfscope}%
\pgfpathrectangle{\pgfqpoint{0.647939in}{0.492442in}}{\pgfqpoint{4.273799in}{2.331163in}}%
\pgfusepath{clip}%
\pgfsetroundcap%
\pgfsetroundjoin%
\pgfsetlinewidth{0.301125pt}%
\definecolor{currentstroke}{rgb}{0.500000,0.500000,0.500000}%
\pgfsetstrokecolor{currentstroke}%
\pgfsetstrokeopacity{0.300000}%
\pgfsetdash{}{0pt}%
\pgfpathmoveto{\pgfqpoint{4.088890in}{1.234539in}}%
\pgfusepath{stroke}%
\end{pgfscope}%
\begin{pgfscope}%
\pgfpathrectangle{\pgfqpoint{0.647939in}{0.492442in}}{\pgfqpoint{4.273799in}{2.331163in}}%
\pgfusepath{clip}%
\pgfsetroundcap%
\pgfsetroundjoin%
\definecolor{currentfill}{rgb}{0.500000,0.500000,0.500000}%
\pgfsetfillcolor{currentfill}%
\pgfsetfillopacity{0.300000}%
\pgfsetlinewidth{0.301125pt}%
\definecolor{currentstroke}{rgb}{0.500000,0.500000,0.500000}%
\pgfsetstrokecolor{currentstroke}%
\pgfsetstrokeopacity{0.300000}%
\pgfsetdash{}{0pt}%
\pgfpathmoveto{\pgfqpoint{0.000000in}{0.000000in}}%
\pgfpathlineto{\pgfqpoint{0.000000in}{0.000000in}}%
\pgfpathclose%
\pgfusepath{stroke,fill}%
\end{pgfscope}%
\begin{pgfscope}%
\pgfpathrectangle{\pgfqpoint{0.647939in}{0.492442in}}{\pgfqpoint{4.273799in}{2.331163in}}%
\pgfusepath{clip}%
\pgfsetroundcap%
\pgfsetroundjoin%
\pgfsetlinewidth{0.301125pt}%
\definecolor{currentstroke}{rgb}{0.500000,0.500000,0.500000}%
\pgfsetstrokecolor{currentstroke}%
\pgfsetstrokeopacity{0.300000}%
\pgfsetdash{}{0pt}%
\pgfpathmoveto{\pgfqpoint{4.436079in}{2.028891in}}%
\pgfusepath{stroke}%
\end{pgfscope}%
\begin{pgfscope}%
\pgfpathrectangle{\pgfqpoint{0.647939in}{0.492442in}}{\pgfqpoint{4.273799in}{2.331163in}}%
\pgfusepath{clip}%
\pgfsetroundcap%
\pgfsetroundjoin%
\definecolor{currentfill}{rgb}{0.500000,0.500000,0.500000}%
\pgfsetfillcolor{currentfill}%
\pgfsetfillopacity{0.300000}%
\pgfsetlinewidth{0.301125pt}%
\definecolor{currentstroke}{rgb}{0.500000,0.500000,0.500000}%
\pgfsetstrokecolor{currentstroke}%
\pgfsetstrokeopacity{0.300000}%
\pgfsetdash{}{0pt}%
\pgfpathmoveto{\pgfqpoint{0.000000in}{0.000000in}}%
\pgfpathlineto{\pgfqpoint{0.000000in}{0.000000in}}%
\pgfpathclose%
\pgfusepath{stroke,fill}%
\end{pgfscope}%
\begin{pgfscope}%
\pgfpathrectangle{\pgfqpoint{0.647939in}{0.492442in}}{\pgfqpoint{4.273799in}{2.331163in}}%
\pgfusepath{clip}%
\pgfsetroundcap%
\pgfsetroundjoin%
\pgfsetlinewidth{0.301125pt}%
\definecolor{currentstroke}{rgb}{0.500000,0.500000,0.500000}%
\pgfsetstrokecolor{currentstroke}%
\pgfsetstrokeopacity{0.300000}%
\pgfsetdash{}{0pt}%
\pgfpathmoveto{\pgfqpoint{1.420427in}{0.943333in}}%
\pgfusepath{stroke}%
\end{pgfscope}%
\begin{pgfscope}%
\pgfpathrectangle{\pgfqpoint{0.647939in}{0.492442in}}{\pgfqpoint{4.273799in}{2.331163in}}%
\pgfusepath{clip}%
\pgfsetroundcap%
\pgfsetroundjoin%
\definecolor{currentfill}{rgb}{0.500000,0.500000,0.500000}%
\pgfsetfillcolor{currentfill}%
\pgfsetfillopacity{0.300000}%
\pgfsetlinewidth{0.301125pt}%
\definecolor{currentstroke}{rgb}{0.500000,0.500000,0.500000}%
\pgfsetstrokecolor{currentstroke}%
\pgfsetstrokeopacity{0.300000}%
\pgfsetdash{}{0pt}%
\pgfpathmoveto{\pgfqpoint{0.000000in}{0.000000in}}%
\pgfpathlineto{\pgfqpoint{0.000000in}{0.000000in}}%
\pgfpathclose%
\pgfusepath{stroke,fill}%
\end{pgfscope}%
\begin{pgfscope}%
\pgfpathrectangle{\pgfqpoint{0.647939in}{0.492442in}}{\pgfqpoint{4.273799in}{2.331163in}}%
\pgfusepath{clip}%
\pgfsetroundcap%
\pgfsetroundjoin%
\pgfsetlinewidth{0.301125pt}%
\definecolor{currentstroke}{rgb}{0.500000,0.500000,0.500000}%
\pgfsetstrokecolor{currentstroke}%
\pgfsetstrokeopacity{0.300000}%
\pgfsetdash{}{0pt}%
\pgfpathmoveto{\pgfqpoint{3.925950in}{0.936130in}}%
\pgfusepath{stroke}%
\end{pgfscope}%
\begin{pgfscope}%
\pgfpathrectangle{\pgfqpoint{0.647939in}{0.492442in}}{\pgfqpoint{4.273799in}{2.331163in}}%
\pgfusepath{clip}%
\pgfsetroundcap%
\pgfsetroundjoin%
\definecolor{currentfill}{rgb}{0.500000,0.500000,0.500000}%
\pgfsetfillcolor{currentfill}%
\pgfsetfillopacity{0.300000}%
\pgfsetlinewidth{0.301125pt}%
\definecolor{currentstroke}{rgb}{0.500000,0.500000,0.500000}%
\pgfsetstrokecolor{currentstroke}%
\pgfsetstrokeopacity{0.300000}%
\pgfsetdash{}{0pt}%
\pgfpathmoveto{\pgfqpoint{0.000000in}{0.000000in}}%
\pgfpathlineto{\pgfqpoint{0.000000in}{0.000000in}}%
\pgfpathclose%
\pgfusepath{stroke,fill}%
\end{pgfscope}%
\begin{pgfscope}%
\pgfpathrectangle{\pgfqpoint{0.647939in}{0.492442in}}{\pgfqpoint{4.273799in}{2.331163in}}%
\pgfusepath{clip}%
\pgfsetroundcap%
\pgfsetroundjoin%
\pgfsetlinewidth{0.301125pt}%
\definecolor{currentstroke}{rgb}{0.500000,0.500000,0.500000}%
\pgfsetstrokecolor{currentstroke}%
\pgfsetstrokeopacity{0.300000}%
\pgfsetdash{}{0pt}%
\pgfpathmoveto{\pgfqpoint{4.188544in}{1.584682in}}%
\pgfusepath{stroke}%
\end{pgfscope}%
\begin{pgfscope}%
\pgfpathrectangle{\pgfqpoint{0.647939in}{0.492442in}}{\pgfqpoint{4.273799in}{2.331163in}}%
\pgfusepath{clip}%
\pgfsetroundcap%
\pgfsetroundjoin%
\definecolor{currentfill}{rgb}{0.500000,0.500000,0.500000}%
\pgfsetfillcolor{currentfill}%
\pgfsetfillopacity{0.300000}%
\pgfsetlinewidth{0.301125pt}%
\definecolor{currentstroke}{rgb}{0.500000,0.500000,0.500000}%
\pgfsetstrokecolor{currentstroke}%
\pgfsetstrokeopacity{0.300000}%
\pgfsetdash{}{0pt}%
\pgfpathmoveto{\pgfqpoint{0.000000in}{0.000000in}}%
\pgfpathlineto{\pgfqpoint{0.000000in}{0.000000in}}%
\pgfpathclose%
\pgfusepath{stroke,fill}%
\end{pgfscope}%
\begin{pgfscope}%
\pgfpathrectangle{\pgfqpoint{0.647939in}{0.492442in}}{\pgfqpoint{4.273799in}{2.331163in}}%
\pgfusepath{clip}%
\pgfsetroundcap%
\pgfsetroundjoin%
\pgfsetlinewidth{0.301125pt}%
\definecolor{currentstroke}{rgb}{0.500000,0.500000,0.500000}%
\pgfsetstrokecolor{currentstroke}%
\pgfsetstrokeopacity{0.300000}%
\pgfsetdash{}{0pt}%
\pgfpathmoveto{\pgfqpoint{4.334429in}{2.155427in}}%
\pgfusepath{stroke}%
\end{pgfscope}%
\begin{pgfscope}%
\pgfpathrectangle{\pgfqpoint{0.647939in}{0.492442in}}{\pgfqpoint{4.273799in}{2.331163in}}%
\pgfusepath{clip}%
\pgfsetroundcap%
\pgfsetroundjoin%
\definecolor{currentfill}{rgb}{0.500000,0.500000,0.500000}%
\pgfsetfillcolor{currentfill}%
\pgfsetfillopacity{0.300000}%
\pgfsetlinewidth{0.301125pt}%
\definecolor{currentstroke}{rgb}{0.500000,0.500000,0.500000}%
\pgfsetstrokecolor{currentstroke}%
\pgfsetstrokeopacity{0.300000}%
\pgfsetdash{}{0pt}%
\pgfpathmoveto{\pgfqpoint{0.000000in}{0.000000in}}%
\pgfpathlineto{\pgfqpoint{0.000000in}{0.000000in}}%
\pgfpathclose%
\pgfusepath{stroke,fill}%
\end{pgfscope}%
\begin{pgfscope}%
\pgfpathrectangle{\pgfqpoint{0.647939in}{0.492442in}}{\pgfqpoint{4.273799in}{2.331163in}}%
\pgfusepath{clip}%
\pgfsetroundcap%
\pgfsetroundjoin%
\pgfsetlinewidth{0.301125pt}%
\definecolor{currentstroke}{rgb}{0.500000,0.500000,0.500000}%
\pgfsetstrokecolor{currentstroke}%
\pgfsetstrokeopacity{0.300000}%
\pgfsetdash{}{0pt}%
\pgfpathmoveto{\pgfqpoint{1.837043in}{2.427326in}}%
\pgfusepath{stroke}%
\end{pgfscope}%
\begin{pgfscope}%
\pgfpathrectangle{\pgfqpoint{0.647939in}{0.492442in}}{\pgfqpoint{4.273799in}{2.331163in}}%
\pgfusepath{clip}%
\pgfsetroundcap%
\pgfsetroundjoin%
\definecolor{currentfill}{rgb}{0.500000,0.500000,0.500000}%
\pgfsetfillcolor{currentfill}%
\pgfsetfillopacity{0.300000}%
\pgfsetlinewidth{0.301125pt}%
\definecolor{currentstroke}{rgb}{0.500000,0.500000,0.500000}%
\pgfsetstrokecolor{currentstroke}%
\pgfsetstrokeopacity{0.300000}%
\pgfsetdash{}{0pt}%
\pgfpathmoveto{\pgfqpoint{0.000000in}{0.000000in}}%
\pgfpathlineto{\pgfqpoint{0.000000in}{0.000000in}}%
\pgfpathclose%
\pgfusepath{stroke,fill}%
\end{pgfscope}%
\begin{pgfscope}%
\pgfpathrectangle{\pgfqpoint{0.647939in}{0.492442in}}{\pgfqpoint{4.273799in}{2.331163in}}%
\pgfusepath{clip}%
\pgfsetroundcap%
\pgfsetroundjoin%
\pgfsetlinewidth{0.301125pt}%
\definecolor{currentstroke}{rgb}{0.500000,0.500000,0.500000}%
\pgfsetstrokecolor{currentstroke}%
\pgfsetstrokeopacity{0.300000}%
\pgfsetdash{}{0pt}%
\pgfpathmoveto{\pgfqpoint{1.582741in}{1.432581in}}%
\pgfusepath{stroke}%
\end{pgfscope}%
\begin{pgfscope}%
\pgfpathrectangle{\pgfqpoint{0.647939in}{0.492442in}}{\pgfqpoint{4.273799in}{2.331163in}}%
\pgfusepath{clip}%
\pgfsetroundcap%
\pgfsetroundjoin%
\definecolor{currentfill}{rgb}{0.500000,0.500000,0.500000}%
\pgfsetfillcolor{currentfill}%
\pgfsetfillopacity{0.300000}%
\pgfsetlinewidth{0.301125pt}%
\definecolor{currentstroke}{rgb}{0.500000,0.500000,0.500000}%
\pgfsetstrokecolor{currentstroke}%
\pgfsetstrokeopacity{0.300000}%
\pgfsetdash{}{0pt}%
\pgfpathmoveto{\pgfqpoint{0.000000in}{0.000000in}}%
\pgfpathlineto{\pgfqpoint{0.000000in}{0.000000in}}%
\pgfpathclose%
\pgfusepath{stroke,fill}%
\end{pgfscope}%
\begin{pgfscope}%
\pgfpathrectangle{\pgfqpoint{0.647939in}{0.492442in}}{\pgfqpoint{4.273799in}{2.331163in}}%
\pgfusepath{clip}%
\pgfsetroundcap%
\pgfsetroundjoin%
\pgfsetlinewidth{0.301125pt}%
\definecolor{currentstroke}{rgb}{0.500000,0.500000,0.500000}%
\pgfsetstrokecolor{currentstroke}%
\pgfsetstrokeopacity{0.300000}%
\pgfsetdash{}{0pt}%
\pgfpathmoveto{\pgfqpoint{1.978459in}{1.028825in}}%
\pgfusepath{stroke}%
\end{pgfscope}%
\begin{pgfscope}%
\pgfpathrectangle{\pgfqpoint{0.647939in}{0.492442in}}{\pgfqpoint{4.273799in}{2.331163in}}%
\pgfusepath{clip}%
\pgfsetroundcap%
\pgfsetroundjoin%
\definecolor{currentfill}{rgb}{0.500000,0.500000,0.500000}%
\pgfsetfillcolor{currentfill}%
\pgfsetfillopacity{0.300000}%
\pgfsetlinewidth{0.301125pt}%
\definecolor{currentstroke}{rgb}{0.500000,0.500000,0.500000}%
\pgfsetstrokecolor{currentstroke}%
\pgfsetstrokeopacity{0.300000}%
\pgfsetdash{}{0pt}%
\pgfpathmoveto{\pgfqpoint{0.000000in}{0.000000in}}%
\pgfpathlineto{\pgfqpoint{0.000000in}{0.000000in}}%
\pgfpathclose%
\pgfusepath{stroke,fill}%
\end{pgfscope}%
\begin{pgfscope}%
\pgfpathrectangle{\pgfqpoint{0.647939in}{0.492442in}}{\pgfqpoint{4.273799in}{2.331163in}}%
\pgfusepath{clip}%
\pgfsetroundcap%
\pgfsetroundjoin%
\pgfsetlinewidth{0.301125pt}%
\definecolor{currentstroke}{rgb}{0.500000,0.500000,0.500000}%
\pgfsetstrokecolor{currentstroke}%
\pgfsetstrokeopacity{0.300000}%
\pgfsetdash{}{0pt}%
\pgfpathmoveto{\pgfqpoint{4.095494in}{1.114364in}}%
\pgfusepath{stroke}%
\end{pgfscope}%
\begin{pgfscope}%
\pgfpathrectangle{\pgfqpoint{0.647939in}{0.492442in}}{\pgfqpoint{4.273799in}{2.331163in}}%
\pgfusepath{clip}%
\pgfsetroundcap%
\pgfsetroundjoin%
\definecolor{currentfill}{rgb}{0.500000,0.500000,0.500000}%
\pgfsetfillcolor{currentfill}%
\pgfsetfillopacity{0.300000}%
\pgfsetlinewidth{0.301125pt}%
\definecolor{currentstroke}{rgb}{0.500000,0.500000,0.500000}%
\pgfsetstrokecolor{currentstroke}%
\pgfsetstrokeopacity{0.300000}%
\pgfsetdash{}{0pt}%
\pgfpathmoveto{\pgfqpoint{0.000000in}{0.000000in}}%
\pgfpathlineto{\pgfqpoint{0.000000in}{0.000000in}}%
\pgfpathclose%
\pgfusepath{stroke,fill}%
\end{pgfscope}%
\begin{pgfscope}%
\pgfpathrectangle{\pgfqpoint{0.647939in}{0.492442in}}{\pgfqpoint{4.273799in}{2.331163in}}%
\pgfusepath{clip}%
\pgfsetroundcap%
\pgfsetroundjoin%
\pgfsetlinewidth{0.301125pt}%
\definecolor{currentstroke}{rgb}{0.500000,0.500000,0.500000}%
\pgfsetstrokecolor{currentstroke}%
\pgfsetstrokeopacity{0.300000}%
\pgfsetdash{}{0pt}%
\pgfpathmoveto{\pgfqpoint{3.902947in}{1.517213in}}%
\pgfusepath{stroke}%
\end{pgfscope}%
\begin{pgfscope}%
\pgfpathrectangle{\pgfqpoint{0.647939in}{0.492442in}}{\pgfqpoint{4.273799in}{2.331163in}}%
\pgfusepath{clip}%
\pgfsetroundcap%
\pgfsetroundjoin%
\definecolor{currentfill}{rgb}{0.500000,0.500000,0.500000}%
\pgfsetfillcolor{currentfill}%
\pgfsetfillopacity{0.300000}%
\pgfsetlinewidth{0.301125pt}%
\definecolor{currentstroke}{rgb}{0.500000,0.500000,0.500000}%
\pgfsetstrokecolor{currentstroke}%
\pgfsetstrokeopacity{0.300000}%
\pgfsetdash{}{0pt}%
\pgfpathmoveto{\pgfqpoint{0.000000in}{0.000000in}}%
\pgfpathlineto{\pgfqpoint{0.000000in}{0.000000in}}%
\pgfpathclose%
\pgfusepath{stroke,fill}%
\end{pgfscope}%
\begin{pgfscope}%
\pgfpathrectangle{\pgfqpoint{0.647939in}{0.492442in}}{\pgfqpoint{4.273799in}{2.331163in}}%
\pgfusepath{clip}%
\pgfsetroundcap%
\pgfsetroundjoin%
\pgfsetlinewidth{0.301125pt}%
\definecolor{currentstroke}{rgb}{0.500000,0.500000,0.500000}%
\pgfsetstrokecolor{currentstroke}%
\pgfsetstrokeopacity{0.300000}%
\pgfsetdash{}{0pt}%
\pgfpathmoveto{\pgfqpoint{4.268129in}{1.699513in}}%
\pgfusepath{stroke}%
\end{pgfscope}%
\begin{pgfscope}%
\pgfpathrectangle{\pgfqpoint{0.647939in}{0.492442in}}{\pgfqpoint{4.273799in}{2.331163in}}%
\pgfusepath{clip}%
\pgfsetroundcap%
\pgfsetroundjoin%
\definecolor{currentfill}{rgb}{0.500000,0.500000,0.500000}%
\pgfsetfillcolor{currentfill}%
\pgfsetfillopacity{0.300000}%
\pgfsetlinewidth{0.301125pt}%
\definecolor{currentstroke}{rgb}{0.500000,0.500000,0.500000}%
\pgfsetstrokecolor{currentstroke}%
\pgfsetstrokeopacity{0.300000}%
\pgfsetdash{}{0pt}%
\pgfpathmoveto{\pgfqpoint{0.000000in}{0.000000in}}%
\pgfpathlineto{\pgfqpoint{0.000000in}{0.000000in}}%
\pgfpathclose%
\pgfusepath{stroke,fill}%
\end{pgfscope}%
\begin{pgfscope}%
\pgfpathrectangle{\pgfqpoint{0.647939in}{0.492442in}}{\pgfqpoint{4.273799in}{2.331163in}}%
\pgfusepath{clip}%
\pgfsetroundcap%
\pgfsetroundjoin%
\pgfsetlinewidth{0.301125pt}%
\definecolor{currentstroke}{rgb}{0.500000,0.500000,0.500000}%
\pgfsetstrokecolor{currentstroke}%
\pgfsetstrokeopacity{0.300000}%
\pgfsetdash{}{0pt}%
\pgfpathmoveto{\pgfqpoint{1.340835in}{2.190444in}}%
\pgfusepath{stroke}%
\end{pgfscope}%
\begin{pgfscope}%
\pgfpathrectangle{\pgfqpoint{0.647939in}{0.492442in}}{\pgfqpoint{4.273799in}{2.331163in}}%
\pgfusepath{clip}%
\pgfsetroundcap%
\pgfsetroundjoin%
\definecolor{currentfill}{rgb}{0.500000,0.500000,0.500000}%
\pgfsetfillcolor{currentfill}%
\pgfsetfillopacity{0.300000}%
\pgfsetlinewidth{0.301125pt}%
\definecolor{currentstroke}{rgb}{0.500000,0.500000,0.500000}%
\pgfsetstrokecolor{currentstroke}%
\pgfsetstrokeopacity{0.300000}%
\pgfsetdash{}{0pt}%
\pgfpathmoveto{\pgfqpoint{0.000000in}{0.000000in}}%
\pgfpathlineto{\pgfqpoint{0.000000in}{0.000000in}}%
\pgfpathclose%
\pgfusepath{stroke,fill}%
\end{pgfscope}%
\begin{pgfscope}%
\pgfpathrectangle{\pgfqpoint{0.647939in}{0.492442in}}{\pgfqpoint{4.273799in}{2.331163in}}%
\pgfusepath{clip}%
\pgfsetroundcap%
\pgfsetroundjoin%
\pgfsetlinewidth{0.301125pt}%
\definecolor{currentstroke}{rgb}{0.500000,0.500000,0.500000}%
\pgfsetstrokecolor{currentstroke}%
\pgfsetstrokeopacity{0.300000}%
\pgfsetdash{}{0pt}%
\pgfpathmoveto{\pgfqpoint{4.057416in}{0.968413in}}%
\pgfusepath{stroke}%
\end{pgfscope}%
\begin{pgfscope}%
\pgfpathrectangle{\pgfqpoint{0.647939in}{0.492442in}}{\pgfqpoint{4.273799in}{2.331163in}}%
\pgfusepath{clip}%
\pgfsetroundcap%
\pgfsetroundjoin%
\definecolor{currentfill}{rgb}{0.500000,0.500000,0.500000}%
\pgfsetfillcolor{currentfill}%
\pgfsetfillopacity{0.300000}%
\pgfsetlinewidth{0.301125pt}%
\definecolor{currentstroke}{rgb}{0.500000,0.500000,0.500000}%
\pgfsetstrokecolor{currentstroke}%
\pgfsetstrokeopacity{0.300000}%
\pgfsetdash{}{0pt}%
\pgfpathmoveto{\pgfqpoint{0.000000in}{0.000000in}}%
\pgfpathlineto{\pgfqpoint{0.000000in}{0.000000in}}%
\pgfpathclose%
\pgfusepath{stroke,fill}%
\end{pgfscope}%
\begin{pgfscope}%
\pgfpathrectangle{\pgfqpoint{0.647939in}{0.492442in}}{\pgfqpoint{4.273799in}{2.331163in}}%
\pgfusepath{clip}%
\pgfsetroundcap%
\pgfsetroundjoin%
\pgfsetlinewidth{0.301125pt}%
\definecolor{currentstroke}{rgb}{0.500000,0.500000,0.500000}%
\pgfsetstrokecolor{currentstroke}%
\pgfsetstrokeopacity{0.300000}%
\pgfsetdash{}{0pt}%
\pgfpathmoveto{\pgfqpoint{1.635173in}{2.377019in}}%
\pgfusepath{stroke}%
\end{pgfscope}%
\begin{pgfscope}%
\pgfpathrectangle{\pgfqpoint{0.647939in}{0.492442in}}{\pgfqpoint{4.273799in}{2.331163in}}%
\pgfusepath{clip}%
\pgfsetroundcap%
\pgfsetroundjoin%
\definecolor{currentfill}{rgb}{0.500000,0.500000,0.500000}%
\pgfsetfillcolor{currentfill}%
\pgfsetfillopacity{0.300000}%
\pgfsetlinewidth{0.301125pt}%
\definecolor{currentstroke}{rgb}{0.500000,0.500000,0.500000}%
\pgfsetstrokecolor{currentstroke}%
\pgfsetstrokeopacity{0.300000}%
\pgfsetdash{}{0pt}%
\pgfpathmoveto{\pgfqpoint{0.000000in}{0.000000in}}%
\pgfpathlineto{\pgfqpoint{0.000000in}{0.000000in}}%
\pgfpathclose%
\pgfusepath{stroke,fill}%
\end{pgfscope}%
\begin{pgfscope}%
\pgfpathrectangle{\pgfqpoint{0.647939in}{0.492442in}}{\pgfqpoint{4.273799in}{2.331163in}}%
\pgfusepath{clip}%
\pgfsetroundcap%
\pgfsetroundjoin%
\pgfsetlinewidth{0.301125pt}%
\definecolor{currentstroke}{rgb}{0.500000,0.500000,0.500000}%
\pgfsetstrokecolor{currentstroke}%
\pgfsetstrokeopacity{0.300000}%
\pgfsetdash{}{0pt}%
\pgfpathmoveto{\pgfqpoint{3.288026in}{2.295871in}}%
\pgfusepath{stroke}%
\end{pgfscope}%
\begin{pgfscope}%
\pgfpathrectangle{\pgfqpoint{0.647939in}{0.492442in}}{\pgfqpoint{4.273799in}{2.331163in}}%
\pgfusepath{clip}%
\pgfsetroundcap%
\pgfsetroundjoin%
\definecolor{currentfill}{rgb}{0.500000,0.500000,0.500000}%
\pgfsetfillcolor{currentfill}%
\pgfsetfillopacity{0.300000}%
\pgfsetlinewidth{0.301125pt}%
\definecolor{currentstroke}{rgb}{0.500000,0.500000,0.500000}%
\pgfsetstrokecolor{currentstroke}%
\pgfsetstrokeopacity{0.300000}%
\pgfsetdash{}{0pt}%
\pgfpathmoveto{\pgfqpoint{0.000000in}{0.000000in}}%
\pgfpathlineto{\pgfqpoint{0.000000in}{0.000000in}}%
\pgfpathclose%
\pgfusepath{stroke,fill}%
\end{pgfscope}%
\begin{pgfscope}%
\pgfpathrectangle{\pgfqpoint{0.647939in}{0.492442in}}{\pgfqpoint{4.273799in}{2.331163in}}%
\pgfusepath{clip}%
\pgfsetroundcap%
\pgfsetroundjoin%
\pgfsetlinewidth{0.301125pt}%
\definecolor{currentstroke}{rgb}{0.500000,0.500000,0.500000}%
\pgfsetstrokecolor{currentstroke}%
\pgfsetstrokeopacity{0.300000}%
\pgfsetdash{}{0pt}%
\pgfpathmoveto{\pgfqpoint{2.411480in}{1.826342in}}%
\pgfusepath{stroke}%
\end{pgfscope}%
\begin{pgfscope}%
\pgfpathrectangle{\pgfqpoint{0.647939in}{0.492442in}}{\pgfqpoint{4.273799in}{2.331163in}}%
\pgfusepath{clip}%
\pgfsetroundcap%
\pgfsetroundjoin%
\definecolor{currentfill}{rgb}{0.500000,0.500000,0.500000}%
\pgfsetfillcolor{currentfill}%
\pgfsetfillopacity{0.300000}%
\pgfsetlinewidth{0.301125pt}%
\definecolor{currentstroke}{rgb}{0.500000,0.500000,0.500000}%
\pgfsetstrokecolor{currentstroke}%
\pgfsetstrokeopacity{0.300000}%
\pgfsetdash{}{0pt}%
\pgfpathmoveto{\pgfqpoint{0.000000in}{0.000000in}}%
\pgfpathlineto{\pgfqpoint{0.000000in}{0.000000in}}%
\pgfpathclose%
\pgfusepath{stroke,fill}%
\end{pgfscope}%
\begin{pgfscope}%
\pgfpathrectangle{\pgfqpoint{0.647939in}{0.492442in}}{\pgfqpoint{4.273799in}{2.331163in}}%
\pgfusepath{clip}%
\pgfsetroundcap%
\pgfsetroundjoin%
\pgfsetlinewidth{0.301125pt}%
\definecolor{currentstroke}{rgb}{0.500000,0.500000,0.500000}%
\pgfsetstrokecolor{currentstroke}%
\pgfsetstrokeopacity{0.300000}%
\pgfsetdash{}{0pt}%
\pgfpathmoveto{\pgfqpoint{1.739464in}{1.631090in}}%
\pgfusepath{stroke}%
\end{pgfscope}%
\begin{pgfscope}%
\pgfpathrectangle{\pgfqpoint{0.647939in}{0.492442in}}{\pgfqpoint{4.273799in}{2.331163in}}%
\pgfusepath{clip}%
\pgfsetroundcap%
\pgfsetroundjoin%
\definecolor{currentfill}{rgb}{0.500000,0.500000,0.500000}%
\pgfsetfillcolor{currentfill}%
\pgfsetfillopacity{0.300000}%
\pgfsetlinewidth{0.301125pt}%
\definecolor{currentstroke}{rgb}{0.500000,0.500000,0.500000}%
\pgfsetstrokecolor{currentstroke}%
\pgfsetstrokeopacity{0.300000}%
\pgfsetdash{}{0pt}%
\pgfpathmoveto{\pgfqpoint{0.000000in}{0.000000in}}%
\pgfpathlineto{\pgfqpoint{0.000000in}{0.000000in}}%
\pgfpathclose%
\pgfusepath{stroke,fill}%
\end{pgfscope}%
\begin{pgfscope}%
\pgfpathrectangle{\pgfqpoint{0.647939in}{0.492442in}}{\pgfqpoint{4.273799in}{2.331163in}}%
\pgfusepath{clip}%
\pgfsetroundcap%
\pgfsetroundjoin%
\pgfsetlinewidth{0.301125pt}%
\definecolor{currentstroke}{rgb}{0.500000,0.500000,0.500000}%
\pgfsetstrokecolor{currentstroke}%
\pgfsetstrokeopacity{0.300000}%
\pgfsetdash{}{0pt}%
\pgfpathmoveto{\pgfqpoint{2.988474in}{1.167469in}}%
\pgfusepath{stroke}%
\end{pgfscope}%
\begin{pgfscope}%
\pgfpathrectangle{\pgfqpoint{0.647939in}{0.492442in}}{\pgfqpoint{4.273799in}{2.331163in}}%
\pgfusepath{clip}%
\pgfsetroundcap%
\pgfsetroundjoin%
\definecolor{currentfill}{rgb}{0.500000,0.500000,0.500000}%
\pgfsetfillcolor{currentfill}%
\pgfsetfillopacity{0.300000}%
\pgfsetlinewidth{0.301125pt}%
\definecolor{currentstroke}{rgb}{0.500000,0.500000,0.500000}%
\pgfsetstrokecolor{currentstroke}%
\pgfsetstrokeopacity{0.300000}%
\pgfsetdash{}{0pt}%
\pgfpathmoveto{\pgfqpoint{0.000000in}{0.000000in}}%
\pgfpathlineto{\pgfqpoint{0.000000in}{0.000000in}}%
\pgfpathclose%
\pgfusepath{stroke,fill}%
\end{pgfscope}%
\begin{pgfscope}%
\pgfpathrectangle{\pgfqpoint{0.647939in}{0.492442in}}{\pgfqpoint{4.273799in}{2.331163in}}%
\pgfusepath{clip}%
\pgfsetroundcap%
\pgfsetroundjoin%
\pgfsetlinewidth{0.301125pt}%
\definecolor{currentstroke}{rgb}{0.500000,0.500000,0.500000}%
\pgfsetstrokecolor{currentstroke}%
\pgfsetstrokeopacity{0.300000}%
\pgfsetdash{}{0pt}%
\pgfpathmoveto{\pgfqpoint{3.485401in}{2.035582in}}%
\pgfusepath{stroke}%
\end{pgfscope}%
\begin{pgfscope}%
\pgfpathrectangle{\pgfqpoint{0.647939in}{0.492442in}}{\pgfqpoint{4.273799in}{2.331163in}}%
\pgfusepath{clip}%
\pgfsetroundcap%
\pgfsetroundjoin%
\definecolor{currentfill}{rgb}{0.500000,0.500000,0.500000}%
\pgfsetfillcolor{currentfill}%
\pgfsetfillopacity{0.300000}%
\pgfsetlinewidth{0.301125pt}%
\definecolor{currentstroke}{rgb}{0.500000,0.500000,0.500000}%
\pgfsetstrokecolor{currentstroke}%
\pgfsetstrokeopacity{0.300000}%
\pgfsetdash{}{0pt}%
\pgfpathmoveto{\pgfqpoint{0.000000in}{0.000000in}}%
\pgfpathlineto{\pgfqpoint{0.000000in}{0.000000in}}%
\pgfpathclose%
\pgfusepath{stroke,fill}%
\end{pgfscope}%
\begin{pgfscope}%
\pgfpathrectangle{\pgfqpoint{0.647939in}{0.492442in}}{\pgfqpoint{4.273799in}{2.331163in}}%
\pgfusepath{clip}%
\pgfsetroundcap%
\pgfsetroundjoin%
\pgfsetlinewidth{0.301125pt}%
\definecolor{currentstroke}{rgb}{0.500000,0.500000,0.500000}%
\pgfsetstrokecolor{currentstroke}%
\pgfsetstrokeopacity{0.300000}%
\pgfsetdash{}{0pt}%
\pgfpathmoveto{\pgfqpoint{3.026619in}{2.193627in}}%
\pgfusepath{stroke}%
\end{pgfscope}%
\begin{pgfscope}%
\pgfpathrectangle{\pgfqpoint{0.647939in}{0.492442in}}{\pgfqpoint{4.273799in}{2.331163in}}%
\pgfusepath{clip}%
\pgfsetroundcap%
\pgfsetroundjoin%
\definecolor{currentfill}{rgb}{0.500000,0.500000,0.500000}%
\pgfsetfillcolor{currentfill}%
\pgfsetfillopacity{0.300000}%
\pgfsetlinewidth{0.301125pt}%
\definecolor{currentstroke}{rgb}{0.500000,0.500000,0.500000}%
\pgfsetstrokecolor{currentstroke}%
\pgfsetstrokeopacity{0.300000}%
\pgfsetdash{}{0pt}%
\pgfpathmoveto{\pgfqpoint{0.000000in}{0.000000in}}%
\pgfpathlineto{\pgfqpoint{0.000000in}{0.000000in}}%
\pgfpathclose%
\pgfusepath{stroke,fill}%
\end{pgfscope}%
\begin{pgfscope}%
\pgfpathrectangle{\pgfqpoint{0.647939in}{0.492442in}}{\pgfqpoint{4.273799in}{2.331163in}}%
\pgfusepath{clip}%
\pgfsetroundcap%
\pgfsetroundjoin%
\pgfsetlinewidth{0.301125pt}%
\definecolor{currentstroke}{rgb}{0.500000,0.500000,0.500000}%
\pgfsetstrokecolor{currentstroke}%
\pgfsetstrokeopacity{0.300000}%
\pgfsetdash{}{0pt}%
\pgfpathmoveto{\pgfqpoint{1.539301in}{1.289604in}}%
\pgfusepath{stroke}%
\end{pgfscope}%
\begin{pgfscope}%
\pgfpathrectangle{\pgfqpoint{0.647939in}{0.492442in}}{\pgfqpoint{4.273799in}{2.331163in}}%
\pgfusepath{clip}%
\pgfsetroundcap%
\pgfsetroundjoin%
\definecolor{currentfill}{rgb}{0.500000,0.500000,0.500000}%
\pgfsetfillcolor{currentfill}%
\pgfsetfillopacity{0.300000}%
\pgfsetlinewidth{0.301125pt}%
\definecolor{currentstroke}{rgb}{0.500000,0.500000,0.500000}%
\pgfsetstrokecolor{currentstroke}%
\pgfsetstrokeopacity{0.300000}%
\pgfsetdash{}{0pt}%
\pgfpathmoveto{\pgfqpoint{0.000000in}{0.000000in}}%
\pgfpathlineto{\pgfqpoint{0.000000in}{0.000000in}}%
\pgfpathclose%
\pgfusepath{stroke,fill}%
\end{pgfscope}%
\begin{pgfscope}%
\pgfpathrectangle{\pgfqpoint{0.647939in}{0.492442in}}{\pgfqpoint{4.273799in}{2.331163in}}%
\pgfusepath{clip}%
\pgfsetroundcap%
\pgfsetroundjoin%
\pgfsetlinewidth{0.301125pt}%
\definecolor{currentstroke}{rgb}{0.500000,0.500000,0.500000}%
\pgfsetstrokecolor{currentstroke}%
\pgfsetstrokeopacity{0.300000}%
\pgfsetdash{}{0pt}%
\pgfpathmoveto{\pgfqpoint{3.271552in}{1.281181in}}%
\pgfusepath{stroke}%
\end{pgfscope}%
\begin{pgfscope}%
\pgfpathrectangle{\pgfqpoint{0.647939in}{0.492442in}}{\pgfqpoint{4.273799in}{2.331163in}}%
\pgfusepath{clip}%
\pgfsetroundcap%
\pgfsetroundjoin%
\definecolor{currentfill}{rgb}{0.500000,0.500000,0.500000}%
\pgfsetfillcolor{currentfill}%
\pgfsetfillopacity{0.300000}%
\pgfsetlinewidth{0.301125pt}%
\definecolor{currentstroke}{rgb}{0.500000,0.500000,0.500000}%
\pgfsetstrokecolor{currentstroke}%
\pgfsetstrokeopacity{0.300000}%
\pgfsetdash{}{0pt}%
\pgfpathmoveto{\pgfqpoint{0.000000in}{0.000000in}}%
\pgfpathlineto{\pgfqpoint{0.000000in}{0.000000in}}%
\pgfpathclose%
\pgfusepath{stroke,fill}%
\end{pgfscope}%
\begin{pgfscope}%
\pgfpathrectangle{\pgfqpoint{0.647939in}{0.492442in}}{\pgfqpoint{4.273799in}{2.331163in}}%
\pgfusepath{clip}%
\pgfsetroundcap%
\pgfsetroundjoin%
\pgfsetlinewidth{0.301125pt}%
\definecolor{currentstroke}{rgb}{0.500000,0.500000,0.500000}%
\pgfsetstrokecolor{currentstroke}%
\pgfsetstrokeopacity{0.300000}%
\pgfsetdash{}{0pt}%
\pgfpathmoveto{\pgfqpoint{3.207935in}{1.932121in}}%
\pgfusepath{stroke}%
\end{pgfscope}%
\begin{pgfscope}%
\pgfpathrectangle{\pgfqpoint{0.647939in}{0.492442in}}{\pgfqpoint{4.273799in}{2.331163in}}%
\pgfusepath{clip}%
\pgfsetroundcap%
\pgfsetroundjoin%
\definecolor{currentfill}{rgb}{0.500000,0.500000,0.500000}%
\pgfsetfillcolor{currentfill}%
\pgfsetfillopacity{0.300000}%
\pgfsetlinewidth{0.301125pt}%
\definecolor{currentstroke}{rgb}{0.500000,0.500000,0.500000}%
\pgfsetstrokecolor{currentstroke}%
\pgfsetstrokeopacity{0.300000}%
\pgfsetdash{}{0pt}%
\pgfpathmoveto{\pgfqpoint{0.000000in}{0.000000in}}%
\pgfpathlineto{\pgfqpoint{0.000000in}{0.000000in}}%
\pgfpathclose%
\pgfusepath{stroke,fill}%
\end{pgfscope}%
\begin{pgfscope}%
\pgfpathrectangle{\pgfqpoint{0.647939in}{0.492442in}}{\pgfqpoint{4.273799in}{2.331163in}}%
\pgfusepath{clip}%
\pgfsetroundcap%
\pgfsetroundjoin%
\pgfsetlinewidth{0.301125pt}%
\definecolor{currentstroke}{rgb}{0.500000,0.500000,0.500000}%
\pgfsetstrokecolor{currentstroke}%
\pgfsetstrokeopacity{0.300000}%
\pgfsetdash{}{0pt}%
\pgfpathmoveto{\pgfqpoint{2.440418in}{1.583159in}}%
\pgfusepath{stroke}%
\end{pgfscope}%
\begin{pgfscope}%
\pgfpathrectangle{\pgfqpoint{0.647939in}{0.492442in}}{\pgfqpoint{4.273799in}{2.331163in}}%
\pgfusepath{clip}%
\pgfsetroundcap%
\pgfsetroundjoin%
\definecolor{currentfill}{rgb}{0.500000,0.500000,0.500000}%
\pgfsetfillcolor{currentfill}%
\pgfsetfillopacity{0.300000}%
\pgfsetlinewidth{0.301125pt}%
\definecolor{currentstroke}{rgb}{0.500000,0.500000,0.500000}%
\pgfsetstrokecolor{currentstroke}%
\pgfsetstrokeopacity{0.300000}%
\pgfsetdash{}{0pt}%
\pgfpathmoveto{\pgfqpoint{0.000000in}{0.000000in}}%
\pgfpathlineto{\pgfqpoint{0.000000in}{0.000000in}}%
\pgfpathclose%
\pgfusepath{stroke,fill}%
\end{pgfscope}%
\begin{pgfscope}%
\pgfpathrectangle{\pgfqpoint{0.647939in}{0.492442in}}{\pgfqpoint{4.273799in}{2.331163in}}%
\pgfusepath{clip}%
\pgfsetbuttcap%
\pgfsetroundjoin%
\pgfsetlinewidth{0.301125pt}%
\definecolor{currentstroke}{rgb}{0.500000,0.500000,0.500000}%
\pgfsetstrokecolor{currentstroke}%
\pgfsetstrokeopacity{0.300000}%
\pgfsetdash{}{0pt}%
\pgfpathmoveto{\pgfqpoint{2.293973in}{0.492442in}}%
\pgfpathlineto{\pgfqpoint{2.263141in}{0.531713in}}%
\pgfpathlineto{\pgfqpoint{2.225893in}{0.579366in}}%
\pgfpathlineto{\pgfqpoint{2.188900in}{0.627078in}}%
\pgfpathlineto{\pgfqpoint{2.152121in}{0.674840in}}%
\pgfpathlineto{\pgfqpoint{2.115507in}{0.722639in}}%
\pgfpathlineto{\pgfqpoint{2.078999in}{0.770462in}}%
\pgfpathlineto{\pgfqpoint{2.042527in}{0.818294in}}%
\pgfpathlineto{\pgfqpoint{2.006020in}{0.866117in}}%
\pgfpathlineto{\pgfqpoint{1.969392in}{0.913913in}}%
\pgfpathlineto{\pgfqpoint{1.932540in}{0.961657in}}%
\pgfpathlineto{\pgfqpoint{1.895336in}{1.009320in}}%
\pgfpathlineto{\pgfqpoint{1.857616in}{1.056862in}}%
\pgfpathlineto{\pgfqpoint{1.819156in}{1.104227in}}%
\pgfpathlineto{\pgfqpoint{1.779656in}{1.151335in}}%
\pgfpathlineto{\pgfqpoint{1.738702in}{1.198068in}}%
\pgfpathlineto{\pgfqpoint{1.695678in}{1.244240in}}%
\pgfpathlineto{\pgfqpoint{1.649608in}{1.289516in}}%
\pgfpathlineto{\pgfqpoint{1.598805in}{1.333231in}}%
\pgfpathlineto{\pgfqpoint{1.548429in}{1.368724in}}%
\pgfpathlineto{\pgfqpoint{1.503811in}{1.392522in}}%
\pgfpathlineto{\pgfqpoint{1.461606in}{1.407579in}}%
\pgfpathlineto{\pgfqpoint{1.414759in}{1.415078in}}%
\pgfpathlineto{\pgfqpoint{1.366619in}{1.412290in}}%
\pgfpathlineto{\pgfqpoint{1.366619in}{1.412290in}}%
\pgfpathlineto{\pgfqpoint{1.314541in}{1.397285in}}%
\pgfpathlineto{\pgfqpoint{1.314541in}{1.397285in}}%
\pgfpathlineto{\pgfqpoint{1.248070in}{1.360983in}}%
\pgfpathlineto{\pgfqpoint{1.193784in}{1.318677in}}%
\pgfpathlineto{\pgfqpoint{1.146695in}{1.273806in}}%
\pgfpathlineto{\pgfqpoint{1.104216in}{1.227545in}}%
\pgfpathlineto{\pgfqpoint{1.064964in}{1.180419in}}%
\pgfpathlineto{\pgfqpoint{1.028109in}{1.132703in}}%
\pgfpathlineto{\pgfqpoint{0.993115in}{1.084556in}}%
\pgfpathlineto{\pgfqpoint{0.959644in}{1.036086in}}%
\pgfpathlineto{\pgfqpoint{0.927452in}{0.987364in}}%
\pgfpathlineto{\pgfqpoint{0.896332in}{0.938434in}}%
\pgfpathlineto{\pgfqpoint{0.866125in}{0.889328in}}%
\pgfpathlineto{\pgfqpoint{0.836735in}{0.840074in}}%
\pgfpathlineto{\pgfqpoint{0.808076in}{0.790694in}}%
\pgfpathlineto{\pgfqpoint{0.780059in}{0.741202in}}%
\pgfpathlineto{\pgfqpoint{0.752633in}{0.691610in}}%
\pgfpathlineto{\pgfqpoint{0.725752in}{0.641931in}}%
\pgfpathlineto{\pgfqpoint{0.699364in}{0.592172in}}%
\pgfpathlineto{\pgfqpoint{0.673440in}{0.542340in}}%
\pgfpathlineto{\pgfqpoint{0.647939in}{0.492442in}}%
\pgfpathlineto{\pgfqpoint{0.647939in}{0.492442in}}%
\pgfusepath{stroke}%
\end{pgfscope}%
\begin{pgfscope}%
\pgfpathrectangle{\pgfqpoint{0.647939in}{0.492442in}}{\pgfqpoint{4.273799in}{2.331163in}}%
\pgfusepath{clip}%
\pgfsetbuttcap%
\pgfsetroundjoin%
\pgfsetlinewidth{0.301125pt}%
\definecolor{currentstroke}{rgb}{0.500000,0.500000,0.500000}%
\pgfsetstrokecolor{currentstroke}%
\pgfsetstrokeopacity{0.300000}%
\pgfsetdash{}{0pt}%
\pgfpathmoveto{\pgfqpoint{1.841889in}{0.492442in}}%
\pgfpathlineto{\pgfqpoint{1.830116in}{0.505883in}}%
\pgfpathlineto{\pgfqpoint{1.788737in}{0.552510in}}%
\pgfpathlineto{\pgfqpoint{1.746339in}{0.598862in}}%
\pgfpathlineto{\pgfqpoint{1.702609in}{0.644842in}}%
\pgfpathlineto{\pgfqpoint{1.657121in}{0.690310in}}%
\pgfpathlineto{\pgfqpoint{1.609276in}{0.735045in}}%
\pgfpathlineto{\pgfqpoint{1.558183in}{0.778690in}}%
\pgfpathlineto{\pgfqpoint{1.502438in}{0.820570in}}%
\pgfpathlineto{\pgfqpoint{1.442167in}{0.857988in}}%
\pgfpathlineto{\pgfqpoint{1.387778in}{0.883813in}}%
\pgfpathlineto{\pgfqpoint{1.336348in}{0.900533in}}%
\pgfpathlineto{\pgfqpoint{1.281838in}{0.909373in}}%
\pgfpathlineto{\pgfqpoint{1.217231in}{0.907266in}}%
\pgfpathlineto{\pgfqpoint{1.157913in}{0.893022in}}%
\pgfpathlineto{\pgfqpoint{1.157913in}{0.893022in}}%
\pgfpathlineto{\pgfqpoint{1.086137in}{0.859830in}}%
\pgfpathlineto{\pgfqpoint{1.026619in}{0.819639in}}%
\pgfpathlineto{\pgfqpoint{0.975320in}{0.776153in}}%
\pgfpathlineto{\pgfqpoint{0.929562in}{0.730826in}}%
\pgfpathlineto{\pgfqpoint{0.887783in}{0.684348in}}%
\pgfpathlineto{\pgfqpoint{0.849006in}{0.637089in}}%
\pgfpathlineto{\pgfqpoint{0.812575in}{0.589267in}}%
\pgfpathlineto{\pgfqpoint{0.778035in}{0.541019in}}%
\pgfpathlineto{\pgfqpoint{0.745071in}{0.492442in}}%
\pgfpathlineto{\pgfqpoint{0.745071in}{0.492442in}}%
\pgfusepath{stroke}%
\end{pgfscope}%
\begin{pgfscope}%
\pgfpathrectangle{\pgfqpoint{0.647939in}{0.492442in}}{\pgfqpoint{4.273799in}{2.331163in}}%
\pgfusepath{clip}%
\pgfsetbuttcap%
\pgfsetroundjoin%
\pgfsetlinewidth{0.301125pt}%
\definecolor{currentstroke}{rgb}{0.500000,0.500000,0.500000}%
\pgfsetstrokecolor{currentstroke}%
\pgfsetstrokeopacity{0.300000}%
\pgfsetdash{}{0pt}%
\pgfpathmoveto{\pgfqpoint{1.604532in}{0.492442in}}%
\pgfpathlineto{\pgfqpoint{1.584142in}{0.510369in}}%
\pgfpathlineto{\pgfqpoint{1.532339in}{0.553762in}}%
\pgfpathlineto{\pgfqpoint{1.476258in}{0.595519in}}%
\pgfpathlineto{\pgfqpoint{1.414005in}{0.634507in}}%
\pgfpathlineto{\pgfqpoint{1.355586in}{0.663123in}}%
\pgfpathlineto{\pgfqpoint{1.300892in}{0.682128in}}%
\pgfpathlineto{\pgfqpoint{1.245394in}{0.693036in}}%
\pgfpathlineto{\pgfqpoint{1.182685in}{0.694494in}}%
\pgfpathlineto{\pgfqpoint{1.122477in}{0.684623in}}%
\pgfpathlineto{\pgfqpoint{1.122477in}{0.684623in}}%
\pgfpathlineto{\pgfqpoint{1.057426in}{0.661201in}}%
\pgfpathlineto{\pgfqpoint{1.057426in}{0.661201in}}%
\pgfpathlineto{\pgfqpoint{0.991664in}{0.624173in}}%
\pgfpathlineto{\pgfqpoint{0.935751in}{0.582427in}}%
\pgfpathlineto{\pgfqpoint{0.886589in}{0.538188in}}%
\pgfpathlineto{\pgfqpoint{0.842203in}{0.492442in}}%
\pgfpathlineto{\pgfqpoint{0.842203in}{0.492442in}}%
\pgfusepath{stroke}%
\end{pgfscope}%
\begin{pgfscope}%
\pgfpathrectangle{\pgfqpoint{0.647939in}{0.492442in}}{\pgfqpoint{4.273799in}{2.331163in}}%
\pgfusepath{clip}%
\pgfsetbuttcap%
\pgfsetroundjoin%
\pgfsetlinewidth{0.301125pt}%
\definecolor{currentstroke}{rgb}{0.500000,0.500000,0.500000}%
\pgfsetstrokecolor{currentstroke}%
\pgfsetstrokeopacity{0.300000}%
\pgfsetdash{}{0pt}%
\pgfpathmoveto{\pgfqpoint{1.438922in}{0.492442in}}%
\pgfpathlineto{\pgfqpoint{1.377569in}{0.524286in}}%
\pgfpathlineto{\pgfqpoint{1.318907in}{0.550607in}}%
\pgfpathlineto{\pgfqpoint{1.263174in}{0.567759in}}%
\pgfpathlineto{\pgfqpoint{1.205232in}{0.576827in}}%
\pgfpathlineto{\pgfqpoint{1.138643in}{0.575473in}}%
\pgfpathlineto{\pgfqpoint{1.076557in}{0.562441in}}%
\pgfpathlineto{\pgfqpoint{1.076557in}{0.562441in}}%
\pgfpathlineto{\pgfqpoint{1.001796in}{0.531249in}}%
\pgfpathlineto{\pgfqpoint{0.939334in}{0.492442in}}%
\pgfpathlineto{\pgfqpoint{0.939334in}{0.492442in}}%
\pgfusepath{stroke}%
\end{pgfscope}%
\begin{pgfscope}%
\pgfpathrectangle{\pgfqpoint{0.647939in}{0.492442in}}{\pgfqpoint{4.273799in}{2.331163in}}%
\pgfusepath{clip}%
\pgfsetbuttcap%
\pgfsetroundjoin%
\pgfsetlinewidth{0.301125pt}%
\definecolor{currentstroke}{rgb}{0.500000,0.500000,0.500000}%
\pgfsetstrokecolor{currentstroke}%
\pgfsetstrokeopacity{0.300000}%
\pgfsetdash{}{0pt}%
\pgfpathmoveto{\pgfqpoint{1.716389in}{0.492442in}}%
\pgfpathlineto{\pgfqpoint{1.716389in}{0.492442in}}%
\pgfpathlineto{\pgfqpoint{1.671458in}{0.538077in}}%
\pgfpathlineto{\pgfqpoint{1.624578in}{0.583120in}}%
\pgfpathlineto{\pgfqpoint{1.575087in}{0.627317in}}%
\pgfpathlineto{\pgfqpoint{1.522008in}{0.670242in}}%
\pgfpathlineto{\pgfqpoint{1.463817in}{0.711124in}}%
\pgfpathlineto{\pgfqpoint{1.398021in}{0.748278in}}%
\pgfpathlineto{\pgfqpoint{1.320933in}{0.777736in}}%
\pgfpathlineto{\pgfqpoint{1.320933in}{0.777736in}}%
\pgfpathlineto{\pgfqpoint{1.261168in}{0.789021in}}%
\pgfpathlineto{\pgfqpoint{1.197091in}{0.788935in}}%
\pgfpathlineto{\pgfqpoint{1.143859in}{0.778996in}}%
\pgfpathlineto{\pgfqpoint{1.094266in}{0.761541in}}%
\pgfpathlineto{\pgfqpoint{1.043423in}{0.735473in}}%
\pgfpathlineto{\pgfqpoint{0.989225in}{0.698734in}}%
\pgfpathlineto{\pgfqpoint{0.937520in}{0.655387in}}%
\pgfusepath{stroke}%
\end{pgfscope}%
\begin{pgfscope}%
\pgfpathrectangle{\pgfqpoint{0.647939in}{0.492442in}}{\pgfqpoint{4.273799in}{2.331163in}}%
\pgfusepath{clip}%
\pgfsetbuttcap%
\pgfsetroundjoin%
\pgfsetlinewidth{0.301125pt}%
\definecolor{currentstroke}{rgb}{0.500000,0.500000,0.500000}%
\pgfsetstrokecolor{currentstroke}%
\pgfsetstrokeopacity{0.300000}%
\pgfsetdash{}{0pt}%
\pgfpathmoveto{\pgfqpoint{1.910652in}{0.492442in}}%
\pgfpathlineto{\pgfqpoint{1.910652in}{0.492442in}}%
\pgfpathlineto{\pgfqpoint{1.870900in}{0.539489in}}%
\pgfpathlineto{\pgfqpoint{1.830541in}{0.586381in}}%
\pgfpathlineto{\pgfqpoint{1.789380in}{0.633065in}}%
\pgfpathlineto{\pgfqpoint{1.747168in}{0.679467in}}%
\pgfpathlineto{\pgfqpoint{1.703572in}{0.725485in}}%
\pgfusepath{stroke}%
\end{pgfscope}%
\begin{pgfscope}%
\pgfpathrectangle{\pgfqpoint{0.647939in}{0.492442in}}{\pgfqpoint{4.273799in}{2.331163in}}%
\pgfusepath{clip}%
\pgfsetbuttcap%
\pgfsetroundjoin%
\pgfsetlinewidth{0.301125pt}%
\definecolor{currentstroke}{rgb}{0.500000,0.500000,0.500000}%
\pgfsetstrokecolor{currentstroke}%
\pgfsetstrokeopacity{0.300000}%
\pgfsetdash{}{0pt}%
\pgfpathmoveto{\pgfqpoint{2.007784in}{0.492442in}}%
\pgfpathlineto{\pgfqpoint{2.007784in}{0.492442in}}%
\pgfpathlineto{\pgfqpoint{1.969355in}{0.539815in}}%
\pgfpathlineto{\pgfqpoint{1.930659in}{0.587123in}}%
\pgfpathlineto{\pgfqpoint{1.891577in}{0.634337in}}%
\pgfpathlineto{\pgfqpoint{1.851967in}{0.681420in}}%
\pgfpathlineto{\pgfqpoint{1.811645in}{0.728322in}}%
\pgfpathlineto{\pgfqpoint{1.770376in}{0.774978in}}%
\pgfpathlineto{\pgfqpoint{1.727845in}{0.821294in}}%
\pgfpathlineto{\pgfqpoint{1.683621in}{0.867135in}}%
\pgfpathlineto{\pgfqpoint{1.637085in}{0.912285in}}%
\pgfpathlineto{\pgfqpoint{1.587295in}{0.956376in}}%
\pgfpathlineto{\pgfqpoint{1.532746in}{0.998719in}}%
\pgfpathlineto{\pgfqpoint{1.470864in}{1.037855in}}%
\pgfpathlineto{\pgfqpoint{1.397332in}{1.069917in}}%
\pgfpathlineto{\pgfqpoint{1.397332in}{1.069917in}}%
\pgfpathlineto{\pgfqpoint{1.341277in}{1.082568in}}%
\pgfpathlineto{\pgfqpoint{1.341277in}{1.082568in}}%
\pgfpathlineto{\pgfqpoint{1.288890in}{1.084401in}}%
\pgfpathlineto{\pgfqpoint{1.238325in}{1.076681in}}%
\pgfpathlineto{\pgfqpoint{1.192326in}{1.061553in}}%
\pgfpathlineto{\pgfqpoint{1.145311in}{1.038224in}}%
\pgfpathlineto{\pgfqpoint{1.094811in}{1.004618in}}%
\pgfpathlineto{\pgfqpoint{1.042920in}{0.961375in}}%
\pgfpathlineto{\pgfqpoint{0.996880in}{0.916117in}}%
\pgfpathlineto{\pgfqpoint{0.954972in}{0.869656in}}%
\pgfpathlineto{\pgfqpoint{0.916139in}{0.822404in}}%
\pgfpathlineto{\pgfqpoint{0.879683in}{0.774591in}}%
\pgfusepath{stroke}%
\end{pgfscope}%
\begin{pgfscope}%
\pgfpathrectangle{\pgfqpoint{0.647939in}{0.492442in}}{\pgfqpoint{4.273799in}{2.331163in}}%
\pgfusepath{clip}%
\pgfsetbuttcap%
\pgfsetroundjoin%
\pgfsetlinewidth{0.301125pt}%
\definecolor{currentstroke}{rgb}{0.500000,0.500000,0.500000}%
\pgfsetstrokecolor{currentstroke}%
\pgfsetstrokeopacity{0.300000}%
\pgfsetdash{}{0pt}%
\pgfpathmoveto{\pgfqpoint{2.104916in}{0.492442in}}%
\pgfpathlineto{\pgfqpoint{2.104916in}{0.492442in}}%
\pgfpathlineto{\pgfqpoint{2.067242in}{0.539995in}}%
\pgfpathlineto{\pgfqpoint{2.029542in}{0.587543in}}%
\pgfpathlineto{\pgfqpoint{1.991734in}{0.635065in}}%
\pgfpathlineto{\pgfqpoint{1.953723in}{0.682538in}}%
\pgfpathlineto{\pgfqpoint{1.915400in}{0.729937in}}%
\pgfpathlineto{\pgfqpoint{1.876631in}{0.777227in}}%
\pgfpathlineto{\pgfqpoint{1.837250in}{0.824366in}}%
\pgfpathlineto{\pgfqpoint{1.797042in}{0.871295in}}%
\pgfpathlineto{\pgfqpoint{1.755721in}{0.917935in}}%
\pgfpathlineto{\pgfqpoint{1.712895in}{0.964167in}}%
\pgfpathlineto{\pgfqpoint{1.668001in}{1.009806in}}%
\pgfpathlineto{\pgfqpoint{1.620189in}{1.054543in}}%
\pgfpathlineto{\pgfqpoint{1.568098in}{1.097809in}}%
\pgfpathlineto{\pgfqpoint{1.509367in}{1.138412in}}%
\pgfpathlineto{\pgfqpoint{1.439761in}{1.173125in}}%
\pgfpathlineto{\pgfqpoint{1.439761in}{1.173125in}}%
\pgfpathlineto{\pgfqpoint{1.384270in}{1.188997in}}%
\pgfpathlineto{\pgfqpoint{1.384270in}{1.188997in}}%
\pgfpathlineto{\pgfqpoint{1.332932in}{1.193489in}}%
\pgfpathlineto{\pgfqpoint{1.281941in}{1.187816in}}%
\pgfpathlineto{\pgfqpoint{1.237216in}{1.174547in}}%
\pgfpathlineto{\pgfqpoint{1.192120in}{1.153450in}}%
\pgfpathlineto{\pgfqpoint{1.143787in}{1.122568in}}%
\pgfpathlineto{\pgfqpoint{1.091028in}{1.079690in}}%
\pgfpathlineto{\pgfqpoint{1.044527in}{1.034601in}}%
\pgfusepath{stroke}%
\end{pgfscope}%
\begin{pgfscope}%
\pgfpathrectangle{\pgfqpoint{0.647939in}{0.492442in}}{\pgfqpoint{4.273799in}{2.331163in}}%
\pgfusepath{clip}%
\pgfsetbuttcap%
\pgfsetroundjoin%
\pgfsetlinewidth{0.301125pt}%
\definecolor{currentstroke}{rgb}{0.500000,0.500000,0.500000}%
\pgfsetstrokecolor{currentstroke}%
\pgfsetstrokeopacity{0.300000}%
\pgfsetdash{}{0pt}%
\pgfpathmoveto{\pgfqpoint{2.396312in}{0.492442in}}%
\pgfpathlineto{\pgfqpoint{2.396312in}{0.492442in}}%
\pgfpathlineto{\pgfqpoint{2.358345in}{0.539926in}}%
\pgfpathlineto{\pgfqpoint{2.320802in}{0.587510in}}%
\pgfpathlineto{\pgfqpoint{2.283651in}{0.635186in}}%
\pgfpathlineto{\pgfqpoint{2.246856in}{0.682944in}}%
\pgfpathlineto{\pgfqpoint{2.210377in}{0.730773in}}%
\pgfpathlineto{\pgfqpoint{2.174169in}{0.778665in}}%
\pgfpathlineto{\pgfqpoint{2.138190in}{0.826607in}}%
\pgfpathlineto{\pgfqpoint{2.102393in}{0.874590in}}%
\pgfpathlineto{\pgfqpoint{2.066724in}{0.922601in}}%
\pgfpathlineto{\pgfqpoint{2.031120in}{0.970627in}}%
\pgfpathlineto{\pgfqpoint{1.995501in}{1.018649in}}%
\pgfpathlineto{\pgfqpoint{1.959769in}{1.066646in}}%
\pgfpathlineto{\pgfqpoint{1.923817in}{1.114594in}}%
\pgfpathlineto{\pgfqpoint{1.887508in}{1.162461in}}%
\pgfpathlineto{\pgfqpoint{1.850663in}{1.210206in}}%
\pgfpathlineto{\pgfqpoint{1.813040in}{1.257770in}}%
\pgfpathlineto{\pgfqpoint{1.774286in}{1.305063in}}%
\pgfpathlineto{\pgfqpoint{1.733882in}{1.351939in}}%
\pgfpathlineto{\pgfqpoint{1.691026in}{1.398154in}}%
\pgfpathlineto{\pgfqpoint{1.644344in}{1.443236in}}%
\pgfpathlineto{\pgfqpoint{1.591180in}{1.486068in}}%
\pgfpathlineto{\pgfqpoint{1.525822in}{1.522971in}}%
\pgfpathlineto{\pgfqpoint{1.525822in}{1.522971in}}%
\pgfpathlineto{\pgfqpoint{1.479591in}{1.536902in}}%
\pgfpathlineto{\pgfqpoint{1.479591in}{1.536902in}}%
\pgfpathlineto{\pgfqpoint{1.436255in}{1.539896in}}%
\pgfpathlineto{\pgfqpoint{1.393938in}{1.533325in}}%
\pgfpathlineto{\pgfqpoint{1.356215in}{1.519850in}}%
\pgfpathlineto{\pgfqpoint{1.316468in}{1.498339in}}%
\pgfpathlineto{\pgfqpoint{1.271870in}{1.466102in}}%
\pgfpathlineto{\pgfqpoint{1.222799in}{1.421954in}}%
\pgfpathlineto{\pgfqpoint{1.179090in}{1.376062in}}%
\pgfpathlineto{\pgfqpoint{1.138958in}{1.329174in}}%
\pgfusepath{stroke}%
\end{pgfscope}%
\begin{pgfscope}%
\pgfpathrectangle{\pgfqpoint{0.647939in}{0.492442in}}{\pgfqpoint{4.273799in}{2.331163in}}%
\pgfusepath{clip}%
\pgfsetbuttcap%
\pgfsetroundjoin%
\pgfsetlinewidth{0.301125pt}%
\definecolor{currentstroke}{rgb}{0.500000,0.500000,0.500000}%
\pgfsetstrokecolor{currentstroke}%
\pgfsetstrokeopacity{0.300000}%
\pgfsetdash{}{0pt}%
\pgfpathmoveto{\pgfqpoint{2.493443in}{0.492442in}}%
\pgfpathlineto{\pgfqpoint{2.493443in}{0.492442in}}%
\pgfpathlineto{\pgfqpoint{2.454697in}{0.539738in}}%
\pgfpathlineto{\pgfqpoint{2.416489in}{0.587164in}}%
\pgfpathlineto{\pgfqpoint{2.378790in}{0.634712in}}%
\pgfpathlineto{\pgfqpoint{2.341568in}{0.682371in}}%
\pgfpathlineto{\pgfqpoint{2.304790in}{0.730132in}}%
\pgfpathlineto{\pgfqpoint{2.268426in}{0.777988in}}%
\pgfpathlineto{\pgfqpoint{2.232443in}{0.825930in}}%
\pgfpathlineto{\pgfqpoint{2.196808in}{0.873948in}}%
\pgfpathlineto{\pgfqpoint{2.161482in}{0.922035in}}%
\pgfpathlineto{\pgfqpoint{2.126418in}{0.970179in}}%
\pgfpathlineto{\pgfqpoint{2.091565in}{1.018368in}}%
\pgfpathlineto{\pgfqpoint{2.056873in}{1.066591in}}%
\pgfpathlineto{\pgfqpoint{2.022283in}{1.114837in}}%
\pgfpathlineto{\pgfqpoint{1.987725in}{1.163089in}}%
\pgfpathlineto{\pgfqpoint{1.953102in}{1.211328in}}%
\pgfpathlineto{\pgfqpoint{1.918295in}{1.259526in}}%
\pgfpathlineto{\pgfqpoint{1.883164in}{1.307654in}}%
\pgfpathlineto{\pgfqpoint{1.847524in}{1.355670in}}%
\pgfpathlineto{\pgfqpoint{1.811114in}{1.403514in}}%
\pgfpathlineto{\pgfqpoint{1.773538in}{1.451087in}}%
\pgfpathlineto{\pgfqpoint{1.734180in}{1.498225in}}%
\pgfpathlineto{\pgfqpoint{1.692008in}{1.544622in}}%
\pgfpathlineto{\pgfqpoint{1.645073in}{1.589601in}}%
\pgfpathlineto{\pgfqpoint{1.589002in}{1.631185in}}%
\pgfpathlineto{\pgfqpoint{1.589002in}{1.631185in}}%
\pgfpathlineto{\pgfqpoint{1.542470in}{1.653209in}}%
\pgfpathlineto{\pgfqpoint{1.542470in}{1.653209in}}%
\pgfpathlineto{\pgfqpoint{1.501351in}{1.661833in}}%
\pgfpathlineto{\pgfqpoint{1.457014in}{1.659154in}}%
\pgfpathlineto{\pgfqpoint{1.421849in}{1.648728in}}%
\pgfpathlineto{\pgfqpoint{1.386025in}{1.631089in}}%
\pgfpathlineto{\pgfqpoint{1.346139in}{1.603963in}}%
\pgfpathlineto{\pgfqpoint{1.299212in}{1.563450in}}%
\pgfpathlineto{\pgfqpoint{1.254546in}{1.517801in}}%
\pgfusepath{stroke}%
\end{pgfscope}%
\begin{pgfscope}%
\pgfpathrectangle{\pgfqpoint{0.647939in}{0.492442in}}{\pgfqpoint{4.273799in}{2.331163in}}%
\pgfusepath{clip}%
\pgfsetbuttcap%
\pgfsetroundjoin%
\pgfsetlinewidth{0.301125pt}%
\definecolor{currentstroke}{rgb}{0.500000,0.500000,0.500000}%
\pgfsetstrokecolor{currentstroke}%
\pgfsetstrokeopacity{0.300000}%
\pgfsetdash{}{0pt}%
\pgfpathmoveto{\pgfqpoint{2.590575in}{0.492442in}}%
\pgfpathlineto{\pgfqpoint{2.590575in}{0.492442in}}%
\pgfpathlineto{\pgfqpoint{2.550746in}{0.539470in}}%
\pgfpathlineto{\pgfqpoint{2.511560in}{0.586658in}}%
\pgfpathlineto{\pgfqpoint{2.472992in}{0.633997in}}%
\pgfpathlineto{\pgfqpoint{2.435015in}{0.681478in}}%
\pgfpathlineto{\pgfqpoint{2.397602in}{0.729093in}}%
\pgfpathlineto{\pgfqpoint{2.360727in}{0.776832in}}%
\pgfpathlineto{\pgfqpoint{2.324362in}{0.824687in}}%
\pgfpathlineto{\pgfqpoint{2.288482in}{0.872652in}}%
\pgfpathlineto{\pgfqpoint{2.253053in}{0.920716in}}%
\pgfpathlineto{\pgfqpoint{2.218037in}{0.968870in}}%
\pgfpathlineto{\pgfqpoint{2.183406in}{1.017106in}}%
\pgfusepath{stroke}%
\end{pgfscope}%
\begin{pgfscope}%
\pgfpathrectangle{\pgfqpoint{0.647939in}{0.492442in}}{\pgfqpoint{4.273799in}{2.331163in}}%
\pgfusepath{clip}%
\pgfsetbuttcap%
\pgfsetroundjoin%
\pgfsetlinewidth{0.301125pt}%
\definecolor{currentstroke}{rgb}{0.500000,0.500000,0.500000}%
\pgfsetstrokecolor{currentstroke}%
\pgfsetstrokeopacity{0.300000}%
\pgfsetdash{}{0pt}%
\pgfpathmoveto{\pgfqpoint{2.687707in}{0.492442in}}%
\pgfpathlineto{\pgfqpoint{2.687707in}{0.492442in}}%
\pgfpathlineto{\pgfqpoint{2.646500in}{0.539115in}}%
\pgfpathlineto{\pgfqpoint{2.606043in}{0.585982in}}%
\pgfpathlineto{\pgfqpoint{2.566314in}{0.633035in}}%
\pgfpathlineto{\pgfqpoint{2.527285in}{0.680261in}}%
\pgfpathlineto{\pgfqpoint{2.488932in}{0.727652in}}%
\pgfpathlineto{\pgfqpoint{2.451231in}{0.775199in}}%
\pgfpathlineto{\pgfqpoint{2.414157in}{0.822892in}}%
\pgfpathlineto{\pgfqpoint{2.377685in}{0.870723in}}%
\pgfusepath{stroke}%
\end{pgfscope}%
\begin{pgfscope}%
\pgfpathrectangle{\pgfqpoint{0.647939in}{0.492442in}}{\pgfqpoint{4.273799in}{2.331163in}}%
\pgfusepath{clip}%
\pgfsetbuttcap%
\pgfsetroundjoin%
\pgfsetlinewidth{0.301125pt}%
\definecolor{currentstroke}{rgb}{0.500000,0.500000,0.500000}%
\pgfsetstrokecolor{currentstroke}%
\pgfsetstrokeopacity{0.300000}%
\pgfsetdash{}{0pt}%
\pgfpathmoveto{\pgfqpoint{2.784839in}{0.492442in}}%
\pgfpathlineto{\pgfqpoint{2.784839in}{0.492442in}}%
\pgfpathlineto{\pgfqpoint{2.741981in}{0.538670in}}%
\pgfpathlineto{\pgfqpoint{2.699979in}{0.585131in}}%
\pgfpathlineto{\pgfqpoint{2.658809in}{0.631813in}}%
\pgfpathlineto{\pgfqpoint{2.618449in}{0.678705in}}%
\pgfpathlineto{\pgfqpoint{2.578873in}{0.725796in}}%
\pgfusepath{stroke}%
\end{pgfscope}%
\begin{pgfscope}%
\pgfpathrectangle{\pgfqpoint{0.647939in}{0.492442in}}{\pgfqpoint{4.273799in}{2.331163in}}%
\pgfusepath{clip}%
\pgfsetbuttcap%
\pgfsetroundjoin%
\pgfsetlinewidth{0.301125pt}%
\definecolor{currentstroke}{rgb}{0.500000,0.500000,0.500000}%
\pgfsetstrokecolor{currentstroke}%
\pgfsetstrokeopacity{0.300000}%
\pgfsetdash{}{0pt}%
\pgfpathmoveto{\pgfqpoint{2.979102in}{0.492442in}}%
\pgfpathlineto{\pgfqpoint{2.886339in}{0.582848in}}%
\pgfpathlineto{\pgfqpoint{2.797718in}{0.674483in}}%
\pgfpathlineto{\pgfqpoint{2.713091in}{0.767234in}}%
\pgfpathlineto{\pgfqpoint{2.632278in}{0.860993in}}%
\pgfpathlineto{\pgfqpoint{2.555093in}{0.955657in}}%
\pgfpathlineto{\pgfqpoint{2.481357in}{1.051136in}}%
\pgfpathlineto{\pgfqpoint{2.410932in}{1.147356in}}%
\pgfpathlineto{\pgfqpoint{2.343727in}{1.244259in}}%
\pgfpathlineto{\pgfqpoint{2.279681in}{1.341797in}}%
\pgfpathlineto{\pgfqpoint{2.218851in}{1.439946in}}%
\pgfpathlineto{\pgfqpoint{2.161388in}{1.538695in}}%
\pgfpathlineto{\pgfqpoint{2.107647in}{1.638066in}}%
\pgfpathlineto{\pgfqpoint{2.082370in}{1.687999in}}%
\pgfpathlineto{\pgfqpoint{2.058315in}{1.738112in}}%
\pgfpathlineto{\pgfqpoint{2.035674in}{1.788420in}}%
\pgfpathlineto{\pgfqpoint{2.014698in}{1.838941in}}%
\pgfpathlineto{\pgfqpoint{1.995752in}{1.889699in}}%
\pgfpathlineto{\pgfqpoint{1.979362in}{1.940717in}}%
\pgfpathlineto{\pgfqpoint{1.966286in}{1.992015in}}%
\pgfpathlineto{\pgfqpoint{1.957638in}{2.043582in}}%
\pgfpathlineto{\pgfqpoint{1.955025in}{2.095329in}}%
\pgfpathlineto{\pgfqpoint{1.960596in}{2.146970in}}%
\pgfpathlineto{\pgfqpoint{1.976762in}{2.197881in}}%
\pgfpathlineto{\pgfqpoint{2.005341in}{2.247077in}}%
\pgfpathlineto{\pgfqpoint{2.046760in}{2.293430in}}%
\pgfpathlineto{\pgfqpoint{2.100227in}{2.335933in}}%
\pgfpathlineto{\pgfqpoint{2.164642in}{2.373698in}}%
\pgfpathlineto{\pgfqpoint{2.239235in}{2.405273in}}%
\pgfpathlineto{\pgfqpoint{2.323091in}{2.428661in}}%
\pgfpathlineto{\pgfqpoint{2.407854in}{2.440858in}}%
\pgfpathlineto{\pgfqpoint{2.487991in}{2.442209in}}%
\pgfpathlineto{\pgfqpoint{2.562817in}{2.434371in}}%
\pgfpathlineto{\pgfqpoint{2.633745in}{2.418381in}}%
\pgfpathlineto{\pgfqpoint{2.701932in}{2.394397in}}%
\pgfpathlineto{\pgfqpoint{2.767300in}{2.362331in}}%
\pgfpathlineto{\pgfqpoint{2.828165in}{2.322827in}}%
\pgfpathlineto{\pgfqpoint{2.879916in}{2.279589in}}%
\pgfpathlineto{\pgfqpoint{2.922895in}{2.233543in}}%
\pgfpathlineto{\pgfqpoint{2.956764in}{2.185286in}}%
\pgfpathlineto{\pgfqpoint{2.979943in}{2.135209in}}%
\pgfpathlineto{\pgfqpoint{2.987261in}{2.083934in}}%
\pgfpathlineto{\pgfqpoint{2.987261in}{2.083934in}}%
\pgfpathlineto{\pgfqpoint{2.978727in}{2.056664in}}%
\pgfpathlineto{\pgfqpoint{2.978727in}{2.056664in}}%
\pgfpathlineto{\pgfqpoint{2.962659in}{2.042726in}}%
\pgfpathlineto{\pgfqpoint{2.962659in}{2.042726in}}%
\pgfpathlineto{\pgfqpoint{2.942705in}{2.039081in}}%
\pgfpathlineto{\pgfqpoint{2.921871in}{2.043538in}}%
\pgfpathlineto{\pgfqpoint{2.903546in}{2.053495in}}%
\pgfpathlineto{\pgfqpoint{2.885937in}{2.071095in}}%
\pgfpathlineto{\pgfqpoint{2.879977in}{2.093397in}}%
\pgfpathlineto{\pgfqpoint{2.879977in}{2.093397in}}%
\pgfpathlineto{\pgfqpoint{2.886651in}{2.096198in}}%
\pgfpathlineto{\pgfqpoint{2.895112in}{2.089515in}}%
\pgfpathlineto{\pgfqpoint{2.893084in}{2.088457in}}%
\pgfpathlineto{\pgfqpoint{2.891732in}{2.090978in}}%
\pgfpathlineto{\pgfqpoint{2.892534in}{2.084697in}}%
\pgfpathlineto{\pgfqpoint{2.892057in}{2.094493in}}%
\pgfpathlineto{\pgfqpoint{2.892057in}{2.094493in}}%
\pgfpathlineto{\pgfqpoint{2.895415in}{2.086273in}}%
\pgfpathlineto{\pgfqpoint{2.891322in}{2.089354in}}%
\pgfpathlineto{\pgfqpoint{2.895417in}{2.089818in}}%
\pgfpathlineto{\pgfqpoint{2.890631in}{2.089199in}}%
\pgfpathlineto{\pgfqpoint{2.891242in}{2.088770in}}%
\pgfpathlineto{\pgfqpoint{2.894642in}{2.091507in}}%
\pgfpathlineto{\pgfqpoint{2.890795in}{2.087875in}}%
\pgfpathlineto{\pgfqpoint{2.894642in}{2.091507in}}%
\pgfpathlineto{\pgfqpoint{2.892867in}{2.087633in}}%
\pgfpathlineto{\pgfqpoint{2.891674in}{2.091022in}}%
\pgfpathlineto{\pgfqpoint{2.892903in}{2.085405in}}%
\pgfpathlineto{\pgfqpoint{2.892903in}{2.085405in}}%
\pgfpathlineto{\pgfqpoint{2.892679in}{2.094982in}}%
\pgfpathlineto{\pgfqpoint{2.892679in}{2.094982in}}%
\pgfpathlineto{\pgfqpoint{2.895932in}{2.087608in}}%
\pgfpathlineto{\pgfqpoint{2.892803in}{2.088164in}}%
\pgfpathlineto{\pgfqpoint{2.892071in}{2.091363in}}%
\pgfpathlineto{\pgfqpoint{2.890961in}{2.085137in}}%
\pgfpathlineto{\pgfqpoint{2.891903in}{2.093243in}}%
\pgfpathlineto{\pgfqpoint{2.894296in}{2.085068in}}%
\pgfpathlineto{\pgfqpoint{2.890428in}{2.091374in}}%
\pgfpathlineto{\pgfqpoint{2.890428in}{2.091374in}}%
\pgfpathlineto{\pgfqpoint{2.895330in}{2.086326in}}%
\pgfpathlineto{\pgfqpoint{2.890604in}{2.090348in}}%
\pgfpathlineto{\pgfqpoint{2.896931in}{2.086652in}}%
\pgfpathlineto{\pgfqpoint{2.890968in}{2.089288in}}%
\pgfpathlineto{\pgfqpoint{2.896140in}{2.089244in}}%
\pgfpathlineto{\pgfqpoint{2.890260in}{2.088564in}}%
\pgfpathlineto{\pgfqpoint{2.895460in}{2.090623in}}%
\pgfpathlineto{\pgfqpoint{2.891596in}{2.088053in}}%
\pgfpathlineto{\pgfqpoint{2.892758in}{2.091213in}}%
\pgfpathlineto{\pgfqpoint{2.891161in}{2.087368in}}%
\pgfpathlineto{\pgfqpoint{2.894607in}{2.092176in}}%
\pgfpathlineto{\pgfqpoint{2.894607in}{2.092176in}}%
\pgfpathlineto{\pgfqpoint{2.893657in}{2.087230in}}%
\pgfpathlineto{\pgfqpoint{2.889680in}{2.092180in}}%
\pgfpathlineto{\pgfqpoint{2.895459in}{2.086546in}}%
\pgfpathlineto{\pgfqpoint{2.891340in}{2.089412in}}%
\pgfpathlineto{\pgfqpoint{2.895738in}{2.089243in}}%
\pgfpathlineto{\pgfqpoint{2.890403in}{2.089116in}}%
\pgfpathlineto{\pgfqpoint{2.891415in}{2.088584in}}%
\pgfpathlineto{\pgfqpoint{2.893455in}{2.090828in}}%
\pgfpathlineto{\pgfqpoint{2.890414in}{2.088385in}}%
\pgfpathlineto{\pgfqpoint{2.895139in}{2.090881in}}%
\pgfpathlineto{\pgfqpoint{2.891536in}{2.087946in}}%
\pgfpathlineto{\pgfqpoint{2.892749in}{2.091298in}}%
\pgfpathlineto{\pgfqpoint{2.891338in}{2.087271in}}%
\pgfpathlineto{\pgfqpoint{2.894333in}{2.092308in}}%
\pgfpathlineto{\pgfqpoint{2.894333in}{2.092308in}}%
\pgfpathlineto{\pgfqpoint{2.893664in}{2.087155in}}%
\pgfpathlineto{\pgfqpoint{2.891336in}{2.090291in}}%
\pgfpathlineto{\pgfqpoint{2.891336in}{2.090291in}}%
\pgfpathlineto{\pgfqpoint{2.895477in}{2.086696in}}%
\pgfpathlineto{\pgfqpoint{2.889726in}{2.091997in}}%
\pgfpathlineto{\pgfqpoint{2.891446in}{2.089309in}}%
\pgfpathlineto{\pgfqpoint{2.896121in}{2.089121in}}%
\pgfpathlineto{\pgfqpoint{2.890204in}{2.088750in}}%
\pgfpathlineto{\pgfqpoint{2.895565in}{2.090385in}}%
\pgfpathlineto{\pgfqpoint{2.891595in}{2.088200in}}%
\pgfpathlineto{\pgfqpoint{2.892895in}{2.091122in}}%
\pgfpathlineto{\pgfqpoint{2.890895in}{2.087601in}}%
\pgfpathlineto{\pgfqpoint{2.894795in}{2.091842in}}%
\pgfpathlineto{\pgfqpoint{2.894795in}{2.091842in}}%
\pgfpathlineto{\pgfqpoint{2.893381in}{2.087402in}}%
\pgfpathlineto{\pgfqpoint{2.891411in}{2.090574in}}%
\pgfpathlineto{\pgfqpoint{2.891411in}{2.090574in}}%
\pgfusepath{stroke}%
\end{pgfscope}%
\begin{pgfscope}%
\pgfpathrectangle{\pgfqpoint{0.647939in}{0.492442in}}{\pgfqpoint{4.273799in}{2.331163in}}%
\pgfusepath{clip}%
\pgfsetbuttcap%
\pgfsetroundjoin%
\pgfsetlinewidth{0.301125pt}%
\definecolor{currentstroke}{rgb}{0.500000,0.500000,0.500000}%
\pgfsetstrokecolor{currentstroke}%
\pgfsetstrokeopacity{0.300000}%
\pgfsetdash{}{0pt}%
\pgfpathmoveto{\pgfqpoint{3.173366in}{0.492442in}}%
\pgfpathlineto{\pgfqpoint{3.173366in}{0.492442in}}%
\pgfpathlineto{\pgfqpoint{3.121663in}{0.535893in}}%
\pgfpathlineto{\pgfqpoint{3.071133in}{0.579754in}}%
\pgfpathlineto{\pgfqpoint{3.021794in}{0.624016in}}%
\pgfpathlineto{\pgfqpoint{2.973654in}{0.668670in}}%
\pgfpathlineto{\pgfqpoint{2.926707in}{0.713699in}}%
\pgfpathlineto{\pgfqpoint{2.880944in}{0.759090in}}%
\pgfpathlineto{\pgfqpoint{2.836346in}{0.804824in}}%
\pgfpathlineto{\pgfqpoint{2.792895in}{0.850886in}}%
\pgfpathlineto{\pgfqpoint{2.750568in}{0.897258in}}%
\pgfpathlineto{\pgfqpoint{2.709343in}{0.943925in}}%
\pgfpathlineto{\pgfqpoint{2.669196in}{0.990871in}}%
\pgfpathlineto{\pgfqpoint{2.630107in}{1.038082in}}%
\pgfpathlineto{\pgfqpoint{2.592056in}{1.085545in}}%
\pgfpathlineto{\pgfqpoint{2.555024in}{1.133247in}}%
\pgfpathlineto{\pgfqpoint{2.518993in}{1.181177in}}%
\pgfpathlineto{\pgfqpoint{2.483955in}{1.229325in}}%
\pgfpathlineto{\pgfqpoint{2.449906in}{1.277684in}}%
\pgfpathlineto{\pgfqpoint{2.416852in}{1.326248in}}%
\pgfpathlineto{\pgfqpoint{2.384796in}{1.375010in}}%
\pgfpathlineto{\pgfqpoint{2.353747in}{1.423966in}}%
\pgfpathlineto{\pgfqpoint{2.323737in}{1.473114in}}%
\pgfpathlineto{\pgfqpoint{2.294806in}{1.522455in}}%
\pgfpathlineto{\pgfqpoint{2.267000in}{1.571987in}}%
\pgfpathlineto{\pgfqpoint{2.240394in}{1.621715in}}%
\pgfpathlineto{\pgfqpoint{2.215086in}{1.671644in}}%
\pgfpathlineto{\pgfqpoint{2.191198in}{1.721780in}}%
\pgfpathlineto{\pgfqpoint{2.168906in}{1.772134in}}%
\pgfpathlineto{\pgfqpoint{2.148432in}{1.822716in}}%
\pgfpathlineto{\pgfqpoint{2.130081in}{1.873539in}}%
\pgfpathlineto{\pgfqpoint{2.114266in}{1.924612in}}%
\pgfpathlineto{\pgfqpoint{2.101539in}{1.975940in}}%
\pgfpathlineto{\pgfqpoint{2.092651in}{2.027502in}}%
\pgfpathlineto{\pgfqpoint{2.088636in}{2.079234in}}%
\pgfpathlineto{\pgfqpoint{2.090866in}{2.130980in}}%
\pgfpathlineto{\pgfqpoint{2.101086in}{2.182414in}}%
\pgfpathlineto{\pgfqpoint{2.121295in}{2.232926in}}%
\pgfpathlineto{\pgfqpoint{2.153556in}{2.281486in}}%
\pgfpathlineto{\pgfqpoint{2.199443in}{2.326597in}}%
\pgfpathlineto{\pgfqpoint{2.259714in}{2.366182in}}%
\pgfpathlineto{\pgfqpoint{2.330631in}{2.396362in}}%
\pgfpathlineto{\pgfqpoint{2.405557in}{2.414884in}}%
\pgfpathlineto{\pgfqpoint{2.479215in}{2.422067in}}%
\pgfusepath{stroke}%
\end{pgfscope}%
\begin{pgfscope}%
\pgfpathrectangle{\pgfqpoint{0.647939in}{0.492442in}}{\pgfqpoint{4.273799in}{2.331163in}}%
\pgfusepath{clip}%
\pgfsetbuttcap%
\pgfsetroundjoin%
\pgfsetlinewidth{0.301125pt}%
\definecolor{currentstroke}{rgb}{0.500000,0.500000,0.500000}%
\pgfsetstrokecolor{currentstroke}%
\pgfsetstrokeopacity{0.300000}%
\pgfsetdash{}{0pt}%
\pgfpathmoveto{\pgfqpoint{3.367630in}{0.492442in}}%
\pgfpathlineto{\pgfqpoint{3.367630in}{0.492442in}}%
\pgfpathlineto{\pgfqpoint{3.310968in}{0.534014in}}%
\pgfpathlineto{\pgfqpoint{3.255417in}{0.576028in}}%
\pgfpathlineto{\pgfqpoint{3.201063in}{0.618506in}}%
\pgfpathlineto{\pgfqpoint{3.147970in}{0.661456in}}%
\pgfpathlineto{\pgfqpoint{3.096187in}{0.704878in}}%
\pgfpathlineto{\pgfqpoint{3.045740in}{0.748766in}}%
\pgfpathlineto{\pgfqpoint{2.996639in}{0.793105in}}%
\pgfpathlineto{\pgfqpoint{2.948883in}{0.837879in}}%
\pgfpathlineto{\pgfqpoint{2.902460in}{0.883069in}}%
\pgfpathlineto{\pgfqpoint{2.857355in}{0.928655in}}%
\pgfpathlineto{\pgfqpoint{2.813547in}{0.974616in}}%
\pgfpathlineto{\pgfqpoint{2.771013in}{1.020931in}}%
\pgfpathlineto{\pgfqpoint{2.729730in}{1.067582in}}%
\pgfpathlineto{\pgfqpoint{2.689675in}{1.114551in}}%
\pgfpathlineto{\pgfqpoint{2.650827in}{1.161820in}}%
\pgfpathlineto{\pgfqpoint{2.613170in}{1.209376in}}%
\pgfpathlineto{\pgfqpoint{2.576692in}{1.257204in}}%
\pgfusepath{stroke}%
\end{pgfscope}%
\begin{pgfscope}%
\pgfpathrectangle{\pgfqpoint{0.647939in}{0.492442in}}{\pgfqpoint{4.273799in}{2.331163in}}%
\pgfusepath{clip}%
\pgfsetbuttcap%
\pgfsetroundjoin%
\pgfsetlinewidth{0.301125pt}%
\definecolor{currentstroke}{rgb}{0.500000,0.500000,0.500000}%
\pgfsetstrokecolor{currentstroke}%
\pgfsetstrokeopacity{0.300000}%
\pgfsetdash{}{0pt}%
\pgfpathmoveto{\pgfqpoint{3.561893in}{0.492442in}}%
\pgfpathlineto{\pgfqpoint{3.561893in}{0.492442in}}%
\pgfpathlineto{\pgfqpoint{3.501112in}{0.532246in}}%
\pgfpathlineto{\pgfqpoint{3.441011in}{0.572356in}}%
\pgfpathlineto{\pgfqpoint{3.381809in}{0.612860in}}%
\pgfpathlineto{\pgfqpoint{3.323676in}{0.653823in}}%
\pgfpathlineto{\pgfqpoint{3.266760in}{0.695290in}}%
\pgfpathlineto{\pgfqpoint{3.211178in}{0.737290in}}%
\pgfpathlineto{\pgfqpoint{3.157003in}{0.779835in}}%
\pgfpathlineto{\pgfqpoint{3.104288in}{0.822922in}}%
\pgfpathlineto{\pgfqpoint{3.053067in}{0.866542in}}%
\pgfpathlineto{\pgfqpoint{3.003352in}{0.910676in}}%
\pgfpathlineto{\pgfqpoint{2.955139in}{0.955303in}}%
\pgfpathlineto{\pgfqpoint{2.908417in}{1.000400in}}%
\pgfpathlineto{\pgfqpoint{2.863166in}{1.045942in}}%
\pgfpathlineto{\pgfqpoint{2.819365in}{1.091903in}}%
\pgfpathlineto{\pgfqpoint{2.776988in}{1.138260in}}%
\pgfusepath{stroke}%
\end{pgfscope}%
\begin{pgfscope}%
\pgfpathrectangle{\pgfqpoint{0.647939in}{0.492442in}}{\pgfqpoint{4.273799in}{2.331163in}}%
\pgfusepath{clip}%
\pgfsetbuttcap%
\pgfsetroundjoin%
\pgfsetlinewidth{0.301125pt}%
\definecolor{currentstroke}{rgb}{0.500000,0.500000,0.500000}%
\pgfsetstrokecolor{currentstroke}%
\pgfsetstrokeopacity{0.300000}%
\pgfsetdash{}{0pt}%
\pgfpathmoveto{\pgfqpoint{3.756157in}{0.492442in}}%
\pgfpathlineto{\pgfqpoint{3.756157in}{0.492442in}}%
\pgfpathlineto{\pgfqpoint{3.693347in}{0.531299in}}%
\pgfpathlineto{\pgfqpoint{3.630351in}{0.570066in}}%
\pgfpathlineto{\pgfqpoint{3.567496in}{0.608901in}}%
\pgfpathlineto{\pgfqpoint{3.505102in}{0.647955in}}%
\pgfpathlineto{\pgfqpoint{3.443451in}{0.687358in}}%
\pgfpathlineto{\pgfqpoint{3.382796in}{0.727217in}}%
\pgfpathlineto{\pgfqpoint{3.323342in}{0.767610in}}%
\pgfpathlineto{\pgfqpoint{3.265257in}{0.808591in}}%
\pgfpathlineto{\pgfqpoint{3.208673in}{0.850190in}}%
\pgfpathlineto{\pgfqpoint{3.153669in}{0.892416in}}%
\pgfpathlineto{\pgfqpoint{3.100305in}{0.935264in}}%
\pgfpathlineto{\pgfqpoint{3.048613in}{0.978716in}}%
\pgfpathlineto{\pgfqpoint{2.998601in}{1.022749in}}%
\pgfpathlineto{\pgfqpoint{2.950260in}{1.067334in}}%
\pgfpathlineto{\pgfqpoint{2.903575in}{1.112442in}}%
\pgfpathlineto{\pgfqpoint{2.858523in}{1.158041in}}%
\pgfpathlineto{\pgfqpoint{2.815078in}{1.204103in}}%
\pgfpathlineto{\pgfqpoint{2.773215in}{1.250598in}}%
\pgfpathlineto{\pgfqpoint{2.732912in}{1.297502in}}%
\pgfpathlineto{\pgfqpoint{2.694151in}{1.344791in}}%
\pgfpathlineto{\pgfqpoint{2.656923in}{1.392446in}}%
\pgfpathlineto{\pgfqpoint{2.621228in}{1.440448in}}%
\pgfpathlineto{\pgfqpoint{2.587082in}{1.488785in}}%
\pgfpathlineto{\pgfqpoint{2.554509in}{1.537443in}}%
\pgfpathlineto{\pgfqpoint{2.523546in}{1.586414in}}%
\pgfpathlineto{\pgfqpoint{2.494262in}{1.635690in}}%
\pgfpathlineto{\pgfqpoint{2.466751in}{1.685269in}}%
\pgfpathlineto{\pgfqpoint{2.441134in}{1.735149in}}%
\pgfpathlineto{\pgfqpoint{2.417582in}{1.785330in}}%
\pgfpathlineto{\pgfqpoint{2.396317in}{1.835814in}}%
\pgfpathlineto{\pgfqpoint{2.377637in}{1.886599in}}%
\pgfpathlineto{\pgfqpoint{2.361943in}{1.937681in}}%
\pgfpathlineto{\pgfqpoint{2.349771in}{1.989045in}}%
\pgfpathlineto{\pgfqpoint{2.341865in}{2.040650in}}%
\pgfpathlineto{\pgfqpoint{2.339256in}{2.092407in}}%
\pgfpathlineto{\pgfqpoint{2.343399in}{2.144116in}}%
\pgfpathlineto{\pgfqpoint{2.356411in}{2.195343in}}%
\pgfpathlineto{\pgfqpoint{2.381361in}{2.245145in}}%
\pgfpathlineto{\pgfqpoint{2.422728in}{2.291343in}}%
\pgfpathlineto{\pgfqpoint{2.422728in}{2.291343in}}%
\pgfpathlineto{\pgfqpoint{2.467804in}{2.321108in}}%
\pgfpathlineto{\pgfqpoint{2.467804in}{2.321108in}}%
\pgfpathlineto{\pgfqpoint{2.517699in}{2.340177in}}%
\pgfpathlineto{\pgfqpoint{2.576245in}{2.349837in}}%
\pgfpathlineto{\pgfqpoint{2.632316in}{2.349128in}}%
\pgfpathlineto{\pgfqpoint{2.686303in}{2.340226in}}%
\pgfusepath{stroke}%
\end{pgfscope}%
\begin{pgfscope}%
\pgfpathrectangle{\pgfqpoint{0.647939in}{0.492442in}}{\pgfqpoint{4.273799in}{2.331163in}}%
\pgfusepath{clip}%
\pgfsetbuttcap%
\pgfsetroundjoin%
\pgfsetlinewidth{0.301125pt}%
\definecolor{currentstroke}{rgb}{0.500000,0.500000,0.500000}%
\pgfsetstrokecolor{currentstroke}%
\pgfsetstrokeopacity{0.300000}%
\pgfsetdash{}{0pt}%
\pgfpathmoveto{\pgfqpoint{3.853289in}{0.492442in}}%
\pgfpathlineto{\pgfqpoint{3.853289in}{0.492442in}}%
\pgfpathlineto{\pgfqpoint{3.790646in}{0.531378in}}%
\pgfpathlineto{\pgfqpoint{3.727245in}{0.569948in}}%
\pgfpathlineto{\pgfqpoint{3.663438in}{0.608318in}}%
\pgfpathlineto{\pgfqpoint{3.599590in}{0.646668in}}%
\pgfpathlineto{\pgfqpoint{3.536053in}{0.685170in}}%
\pgfusepath{stroke}%
\end{pgfscope}%
\begin{pgfscope}%
\pgfpathrectangle{\pgfqpoint{0.647939in}{0.492442in}}{\pgfqpoint{4.273799in}{2.331163in}}%
\pgfusepath{clip}%
\pgfsetbuttcap%
\pgfsetroundjoin%
\pgfsetlinewidth{0.301125pt}%
\definecolor{currentstroke}{rgb}{0.500000,0.500000,0.500000}%
\pgfsetstrokecolor{currentstroke}%
\pgfsetstrokeopacity{0.300000}%
\pgfsetdash{}{0pt}%
\pgfpathmoveto{\pgfqpoint{3.950420in}{0.492442in}}%
\pgfpathlineto{\pgfqpoint{3.950420in}{0.492442in}}%
\pgfpathlineto{\pgfqpoint{3.888886in}{0.531897in}}%
\pgfpathlineto{\pgfqpoint{3.825988in}{0.570709in}}%
\pgfpathlineto{\pgfqpoint{3.762065in}{0.609021in}}%
\pgfpathlineto{\pgfqpoint{3.697488in}{0.647006in}}%
\pgfpathlineto{\pgfqpoint{3.632650in}{0.684858in}}%
\pgfusepath{stroke}%
\end{pgfscope}%
\begin{pgfscope}%
\pgfpathrectangle{\pgfqpoint{0.647939in}{0.492442in}}{\pgfqpoint{4.273799in}{2.331163in}}%
\pgfusepath{clip}%
\pgfsetbuttcap%
\pgfsetroundjoin%
\pgfsetlinewidth{0.301125pt}%
\definecolor{currentstroke}{rgb}{0.500000,0.500000,0.500000}%
\pgfsetstrokecolor{currentstroke}%
\pgfsetstrokeopacity{0.300000}%
\pgfsetdash{}{0pt}%
\pgfpathmoveto{\pgfqpoint{4.047552in}{0.492442in}}%
\pgfpathlineto{\pgfqpoint{4.047552in}{0.492442in}}%
\pgfpathlineto{\pgfqpoint{3.988130in}{0.532846in}}%
\pgfpathlineto{\pgfqpoint{3.926784in}{0.572386in}}%
\pgfpathlineto{\pgfqpoint{3.863777in}{0.611144in}}%
\pgfpathlineto{\pgfqpoint{3.799432in}{0.649243in}}%
\pgfpathlineto{\pgfqpoint{3.734131in}{0.686858in}}%
\pgfusepath{stroke}%
\end{pgfscope}%
\begin{pgfscope}%
\pgfpathrectangle{\pgfqpoint{0.647939in}{0.492442in}}{\pgfqpoint{4.273799in}{2.331163in}}%
\pgfusepath{clip}%
\pgfsetbuttcap%
\pgfsetroundjoin%
\pgfsetlinewidth{0.301125pt}%
\definecolor{currentstroke}{rgb}{0.500000,0.500000,0.500000}%
\pgfsetstrokecolor{currentstroke}%
\pgfsetstrokeopacity{0.300000}%
\pgfsetdash{}{0pt}%
\pgfpathmoveto{\pgfqpoint{4.241816in}{0.492442in}}%
\pgfpathlineto{\pgfqpoint{4.241816in}{0.492442in}}%
\pgfpathlineto{\pgfqpoint{4.189379in}{0.535625in}}%
\pgfpathlineto{\pgfqpoint{4.134427in}{0.577866in}}%
\pgfpathlineto{\pgfqpoint{4.076934in}{0.619087in}}%
\pgfpathlineto{\pgfqpoint{4.016911in}{0.659222in}}%
\pgfpathlineto{\pgfqpoint{3.954495in}{0.698258in}}%
\pgfpathlineto{\pgfqpoint{3.889941in}{0.736246in}}%
\pgfpathlineto{\pgfqpoint{3.823595in}{0.773308in}}%
\pgfpathlineto{\pgfqpoint{3.755906in}{0.809643in}}%
\pgfpathlineto{\pgfqpoint{3.687391in}{0.845516in}}%
\pgfpathlineto{\pgfqpoint{3.618604in}{0.881234in}}%
\pgfpathlineto{\pgfqpoint{3.550071in}{0.917096in}}%
\pgfpathlineto{\pgfqpoint{3.482304in}{0.953385in}}%
\pgfpathlineto{\pgfqpoint{3.415738in}{0.990328in}}%
\pgfpathlineto{\pgfqpoint{3.350749in}{1.028094in}}%
\pgfpathlineto{\pgfqpoint{3.287624in}{1.066787in}}%
\pgfpathlineto{\pgfqpoint{3.226574in}{1.106457in}}%
\pgfpathlineto{\pgfqpoint{3.167740in}{1.147110in}}%
\pgfpathlineto{\pgfqpoint{3.111196in}{1.188719in}}%
\pgfpathlineto{\pgfqpoint{3.056974in}{1.231240in}}%
\pgfpathlineto{\pgfqpoint{3.005078in}{1.274614in}}%
\pgfpathlineto{\pgfqpoint{2.955482in}{1.318781in}}%
\pgfpathlineto{\pgfqpoint{2.908148in}{1.363682in}}%
\pgfpathlineto{\pgfqpoint{2.863038in}{1.409260in}}%
\pgfpathlineto{\pgfqpoint{2.820118in}{1.455463in}}%
\pgfpathlineto{\pgfqpoint{2.779362in}{1.502246in}}%
\pgfpathlineto{\pgfqpoint{2.740760in}{1.549570in}}%
\pgfpathlineto{\pgfqpoint{2.704320in}{1.597402in}}%
\pgfpathlineto{\pgfqpoint{2.670079in}{1.645715in}}%
\pgfpathlineto{\pgfqpoint{2.638104in}{1.694489in}}%
\pgfpathlineto{\pgfqpoint{2.608494in}{1.743704in}}%
\pgfpathlineto{\pgfqpoint{2.581404in}{1.793348in}}%
\pgfpathlineto{\pgfqpoint{2.557063in}{1.843413in}}%
\pgfpathlineto{\pgfqpoint{2.535782in}{1.893889in}}%
\pgfpathlineto{\pgfqpoint{2.518003in}{1.944764in}}%
\pgfpathlineto{\pgfqpoint{2.504362in}{1.996013in}}%
\pgfpathlineto{\pgfqpoint{2.495777in}{2.047580in}}%
\pgfpathlineto{\pgfqpoint{2.493638in}{2.099328in}}%
\pgfpathlineto{\pgfqpoint{2.500173in}{2.150921in}}%
\pgfpathlineto{\pgfqpoint{2.519194in}{2.201471in}}%
\pgfpathlineto{\pgfqpoint{2.519194in}{2.201471in}}%
\pgfpathlineto{\pgfqpoint{2.549927in}{2.241552in}}%
\pgfpathlineto{\pgfqpoint{2.549927in}{2.241552in}}%
\pgfpathlineto{\pgfqpoint{2.586755in}{2.267751in}}%
\pgfpathlineto{\pgfqpoint{2.586755in}{2.267751in}}%
\pgfpathlineto{\pgfqpoint{2.628017in}{2.282926in}}%
\pgfpathlineto{\pgfqpoint{2.677090in}{2.288384in}}%
\pgfpathlineto{\pgfqpoint{2.722151in}{2.284302in}}%
\pgfpathlineto{\pgfqpoint{2.765825in}{2.272906in}}%
\pgfpathlineto{\pgfqpoint{2.810001in}{2.253881in}}%
\pgfusepath{stroke}%
\end{pgfscope}%
\begin{pgfscope}%
\pgfpathrectangle{\pgfqpoint{0.647939in}{0.492442in}}{\pgfqpoint{4.273799in}{2.331163in}}%
\pgfusepath{clip}%
\pgfsetbuttcap%
\pgfsetroundjoin%
\pgfsetlinewidth{0.301125pt}%
\definecolor{currentstroke}{rgb}{0.500000,0.500000,0.500000}%
\pgfsetstrokecolor{currentstroke}%
\pgfsetstrokeopacity{0.300000}%
\pgfsetdash{}{0pt}%
\pgfpathmoveto{\pgfqpoint{4.436079in}{0.492442in}}%
\pgfpathlineto{\pgfqpoint{4.436079in}{0.492442in}}%
\pgfpathlineto{\pgfqpoint{4.392992in}{0.538600in}}%
\pgfpathlineto{\pgfqpoint{4.347756in}{0.584141in}}%
\pgfpathlineto{\pgfqpoint{4.300147in}{0.628955in}}%
\pgfpathlineto{\pgfqpoint{4.249933in}{0.672912in}}%
\pgfpathlineto{\pgfqpoint{4.196878in}{0.715863in}}%
\pgfpathlineto{\pgfqpoint{4.140772in}{0.757645in}}%
\pgfpathlineto{\pgfqpoint{4.081506in}{0.798106in}}%
\pgfpathlineto{\pgfqpoint{4.019050in}{0.837112in}}%
\pgfpathlineto{\pgfqpoint{3.953526in}{0.874593in}}%
\pgfpathlineto{\pgfqpoint{3.885238in}{0.910580in}}%
\pgfpathlineto{\pgfqpoint{3.814676in}{0.945243in}}%
\pgfpathlineto{\pgfqpoint{3.742491in}{0.978902in}}%
\pgfpathlineto{\pgfqpoint{3.669374in}{1.011962in}}%
\pgfpathlineto{\pgfqpoint{3.596093in}{1.044912in}}%
\pgfpathlineto{\pgfqpoint{3.523384in}{1.078233in}}%
\pgfpathlineto{\pgfqpoint{3.451874in}{1.112313in}}%
\pgfpathlineto{\pgfqpoint{3.382123in}{1.147456in}}%
\pgfpathlineto{\pgfqpoint{3.314575in}{1.183850in}}%
\pgfpathlineto{\pgfqpoint{3.249534in}{1.221577in}}%
\pgfpathlineto{\pgfqpoint{3.187206in}{1.260642in}}%
\pgfpathlineto{\pgfqpoint{3.127699in}{1.300993in}}%
\pgfpathlineto{\pgfqpoint{3.071047in}{1.342552in}}%
\pgfpathlineto{\pgfqpoint{3.017234in}{1.385222in}}%
\pgfpathlineto{\pgfqpoint{2.966221in}{1.428903in}}%
\pgfpathlineto{\pgfqpoint{2.917951in}{1.473501in}}%
\pgfpathlineto{\pgfqpoint{2.872365in}{1.518933in}}%
\pgfusepath{stroke}%
\end{pgfscope}%
\begin{pgfscope}%
\pgfpathrectangle{\pgfqpoint{0.647939in}{0.492442in}}{\pgfqpoint{4.273799in}{2.331163in}}%
\pgfusepath{clip}%
\pgfsetbuttcap%
\pgfsetroundjoin%
\pgfsetlinewidth{0.301125pt}%
\definecolor{currentstroke}{rgb}{0.500000,0.500000,0.500000}%
\pgfsetstrokecolor{currentstroke}%
\pgfsetstrokeopacity{0.300000}%
\pgfsetdash{}{0pt}%
\pgfpathmoveto{\pgfqpoint{4.533211in}{0.492442in}}%
\pgfpathlineto{\pgfqpoint{4.533211in}{0.492442in}}%
\pgfpathlineto{\pgfqpoint{4.494943in}{0.539851in}}%
\pgfpathlineto{\pgfqpoint{4.454945in}{0.586831in}}%
\pgfpathlineto{\pgfqpoint{4.413006in}{0.633303in}}%
\pgfpathlineto{\pgfqpoint{4.368881in}{0.679167in}}%
\pgfpathlineto{\pgfqpoint{4.322296in}{0.724300in}}%
\pgfpathlineto{\pgfqpoint{4.272946in}{0.768550in}}%
\pgfpathlineto{\pgfqpoint{4.220539in}{0.811738in}}%
\pgfpathlineto{\pgfqpoint{4.164788in}{0.853654in}}%
\pgfpathlineto{\pgfqpoint{4.105419in}{0.894062in}}%
\pgfpathlineto{\pgfqpoint{4.042295in}{0.932740in}}%
\pgfpathlineto{\pgfqpoint{3.975471in}{0.969520in}}%
\pgfpathlineto{\pgfqpoint{3.905230in}{1.004361in}}%
\pgfpathlineto{\pgfqpoint{3.832132in}{1.037413in}}%
\pgfpathlineto{\pgfqpoint{3.756916in}{1.069031in}}%
\pgfpathlineto{\pgfqpoint{3.680492in}{1.099782in}}%
\pgfpathlineto{\pgfqpoint{3.603801in}{1.130336in}}%
\pgfpathlineto{\pgfqpoint{3.527716in}{1.161329in}}%
\pgfpathlineto{\pgfqpoint{3.453012in}{1.193300in}}%
\pgfpathlineto{\pgfqpoint{3.380370in}{1.226648in}}%
\pgfpathlineto{\pgfqpoint{3.310325in}{1.261598in}}%
\pgfpathlineto{\pgfqpoint{3.243214in}{1.298222in}}%
\pgfpathlineto{\pgfqpoint{3.179266in}{1.336491in}}%
\pgfpathlineto{\pgfqpoint{3.118580in}{1.376309in}}%
\pgfpathlineto{\pgfqpoint{3.061161in}{1.417548in}}%
\pgfusepath{stroke}%
\end{pgfscope}%
\begin{pgfscope}%
\pgfpathrectangle{\pgfqpoint{0.647939in}{0.492442in}}{\pgfqpoint{4.273799in}{2.331163in}}%
\pgfusepath{clip}%
\pgfsetbuttcap%
\pgfsetroundjoin%
\pgfsetlinewidth{0.301125pt}%
\definecolor{currentstroke}{rgb}{0.500000,0.500000,0.500000}%
\pgfsetstrokecolor{currentstroke}%
\pgfsetstrokeopacity{0.300000}%
\pgfsetdash{}{0pt}%
\pgfpathmoveto{\pgfqpoint{4.630343in}{0.492442in}}%
\pgfpathlineto{\pgfqpoint{4.630343in}{0.492442in}}%
\pgfpathlineto{\pgfqpoint{4.596687in}{0.540880in}}%
\pgfpathlineto{\pgfqpoint{4.561739in}{0.589044in}}%
\pgfpathlineto{\pgfqpoint{4.525337in}{0.636886in}}%
\pgfpathlineto{\pgfqpoint{4.487289in}{0.684346in}}%
\pgfpathlineto{\pgfqpoint{4.447378in}{0.731349in}}%
\pgfpathlineto{\pgfqpoint{4.405347in}{0.777798in}}%
\pgfpathlineto{\pgfqpoint{4.360887in}{0.823567in}}%
\pgfpathlineto{\pgfqpoint{4.313640in}{0.868495in}}%
\pgfpathlineto{\pgfqpoint{4.263212in}{0.912376in}}%
\pgfpathlineto{\pgfqpoint{4.209162in}{0.954945in}}%
\pgfpathlineto{\pgfqpoint{4.151023in}{0.995873in}}%
\pgfpathlineto{\pgfqpoint{4.088412in}{1.034784in}}%
\pgfpathlineto{\pgfqpoint{4.021157in}{1.071309in}}%
\pgfpathlineto{\pgfqpoint{3.949389in}{1.105184in}}%
\pgfpathlineto{\pgfqpoint{3.873661in}{1.136398in}}%
\pgfpathlineto{\pgfqpoint{3.794874in}{1.165293in}}%
\pgfpathlineto{\pgfqpoint{3.714182in}{1.192599in}}%
\pgfpathlineto{\pgfqpoint{3.632798in}{1.219296in}}%
\pgfpathlineto{\pgfqpoint{3.551866in}{1.246393in}}%
\pgfpathlineto{\pgfqpoint{3.472408in}{1.274736in}}%
\pgfpathlineto{\pgfqpoint{3.395285in}{1.304915in}}%
\pgfpathlineto{\pgfqpoint{3.321162in}{1.337249in}}%
\pgfpathlineto{\pgfqpoint{3.250520in}{1.371824in}}%
\pgfpathlineto{\pgfqpoint{3.183656in}{1.408559in}}%
\pgfpathlineto{\pgfqpoint{3.120655in}{1.447279in}}%
\pgfpathlineto{\pgfqpoint{3.061506in}{1.487772in}}%
\pgfpathlineto{\pgfqpoint{3.006125in}{1.529826in}}%
\pgfpathlineto{\pgfqpoint{2.954402in}{1.573242in}}%
\pgfpathlineto{\pgfqpoint{2.906218in}{1.617857in}}%
\pgfpathlineto{\pgfqpoint{2.861481in}{1.663531in}}%
\pgfpathlineto{\pgfqpoint{2.820144in}{1.710151in}}%
\pgfpathlineto{\pgfqpoint{2.782212in}{1.757624in}}%
\pgfpathlineto{\pgfqpoint{2.747766in}{1.805881in}}%
\pgfpathlineto{\pgfqpoint{2.716992in}{1.854873in}}%
\pgfpathlineto{\pgfqpoint{2.690203in}{1.904556in}}%
\pgfpathlineto{\pgfqpoint{2.667915in}{1.954890in}}%
\pgfpathlineto{\pgfqpoint{2.650970in}{2.005831in}}%
\pgfpathlineto{\pgfqpoint{2.640778in}{2.057283in}}%
\pgfpathlineto{\pgfqpoint{2.639888in}{2.108980in}}%
\pgfpathlineto{\pgfqpoint{2.653639in}{2.159953in}}%
\pgfpathlineto{\pgfqpoint{2.653639in}{2.159953in}}%
\pgfpathlineto{\pgfqpoint{2.676862in}{2.191958in}}%
\pgfpathlineto{\pgfqpoint{2.676862in}{2.191958in}}%
\pgfpathlineto{\pgfqpoint{2.706187in}{2.211089in}}%
\pgfpathlineto{\pgfqpoint{2.706187in}{2.211089in}}%
\pgfusepath{stroke}%
\end{pgfscope}%
\begin{pgfscope}%
\pgfpathrectangle{\pgfqpoint{0.647939in}{0.492442in}}{\pgfqpoint{4.273799in}{2.331163in}}%
\pgfusepath{clip}%
\pgfsetbuttcap%
\pgfsetroundjoin%
\pgfsetlinewidth{0.301125pt}%
\definecolor{currentstroke}{rgb}{0.500000,0.500000,0.500000}%
\pgfsetstrokecolor{currentstroke}%
\pgfsetstrokeopacity{0.300000}%
\pgfsetdash{}{0pt}%
\pgfpathmoveto{\pgfqpoint{4.727475in}{0.492442in}}%
\pgfpathlineto{\pgfqpoint{4.727475in}{0.492442in}}%
\pgfpathlineto{\pgfqpoint{4.698030in}{0.541691in}}%
\pgfpathlineto{\pgfqpoint{4.667687in}{0.590777in}}%
\pgfpathlineto{\pgfqpoint{4.636351in}{0.639677in}}%
\pgfpathlineto{\pgfqpoint{4.603908in}{0.688360in}}%
\pgfpathlineto{\pgfqpoint{4.570208in}{0.736788in}}%
\pgfpathlineto{\pgfqpoint{4.535078in}{0.784914in}}%
\pgfpathlineto{\pgfqpoint{4.498322in}{0.832676in}}%
\pgfpathlineto{\pgfqpoint{4.459698in}{0.879994in}}%
\pgfpathlineto{\pgfqpoint{4.418894in}{0.926761in}}%
\pgfpathlineto{\pgfqpoint{4.375527in}{0.972834in}}%
\pgfpathlineto{\pgfqpoint{4.329122in}{1.018013in}}%
\pgfpathlineto{\pgfqpoint{4.279093in}{1.062021in}}%
\pgfpathlineto{\pgfqpoint{4.224747in}{1.104473in}}%
\pgfpathlineto{\pgfqpoint{4.165366in}{1.144847in}}%
\pgfpathlineto{\pgfqpoint{4.100204in}{1.182456in}}%
\pgfpathlineto{\pgfqpoint{4.028855in}{1.216541in}}%
\pgfpathlineto{\pgfqpoint{3.951568in}{1.246511in}}%
\pgfpathlineto{\pgfqpoint{3.869327in}{1.272299in}}%
\pgfpathlineto{\pgfqpoint{3.783719in}{1.294661in}}%
\pgfpathlineto{\pgfqpoint{3.696429in}{1.315056in}}%
\pgfpathlineto{\pgfqpoint{3.608958in}{1.335214in}}%
\pgfpathlineto{\pgfqpoint{3.522604in}{1.356708in}}%
\pgfpathlineto{\pgfqpoint{3.438528in}{1.380691in}}%
\pgfpathlineto{\pgfqpoint{3.357734in}{1.407798in}}%
\pgfpathlineto{\pgfqpoint{3.280978in}{1.438207in}}%
\pgfpathlineto{\pgfqpoint{3.208763in}{1.471767in}}%
\pgfpathlineto{\pgfqpoint{3.141307in}{1.508148in}}%
\pgfpathlineto{\pgfqpoint{3.078561in}{1.546970in}}%
\pgfpathlineto{\pgfqpoint{3.020386in}{1.587868in}}%
\pgfpathlineto{\pgfqpoint{2.966596in}{1.630525in}}%
\pgfusepath{stroke}%
\end{pgfscope}%
\begin{pgfscope}%
\pgfpathrectangle{\pgfqpoint{0.647939in}{0.492442in}}{\pgfqpoint{4.273799in}{2.331163in}}%
\pgfusepath{clip}%
\pgfsetbuttcap%
\pgfsetroundjoin%
\pgfsetlinewidth{0.301125pt}%
\definecolor{currentstroke}{rgb}{0.500000,0.500000,0.500000}%
\pgfsetstrokecolor{currentstroke}%
\pgfsetstrokeopacity{0.300000}%
\pgfsetdash{}{0pt}%
\pgfpathmoveto{\pgfqpoint{4.824607in}{0.492442in}}%
\pgfpathlineto{\pgfqpoint{4.824607in}{0.492442in}}%
\pgfpathlineto{\pgfqpoint{4.798918in}{0.542313in}}%
\pgfpathlineto{\pgfqpoint{4.772658in}{0.592095in}}%
\pgfpathlineto{\pgfqpoint{4.745770in}{0.641778in}}%
\pgfpathlineto{\pgfqpoint{4.718193in}{0.691348in}}%
\pgfpathlineto{\pgfqpoint{4.689859in}{0.740789in}}%
\pgfpathlineto{\pgfqpoint{4.660672in}{0.790084in}}%
\pgfpathlineto{\pgfqpoint{4.630530in}{0.839207in}}%
\pgfpathlineto{\pgfqpoint{4.599317in}{0.888128in}}%
\pgfpathlineto{\pgfqpoint{4.566869in}{0.936809in}}%
\pgfpathlineto{\pgfqpoint{4.532983in}{0.985200in}}%
\pgfpathlineto{\pgfqpoint{4.497425in}{1.033230in}}%
\pgfpathlineto{\pgfqpoint{4.459896in}{1.080808in}}%
\pgfpathlineto{\pgfqpoint{4.419986in}{1.127802in}}%
\pgfpathlineto{\pgfqpoint{4.377154in}{1.174017in}}%
\pgfpathlineto{\pgfqpoint{4.330672in}{1.219161in}}%
\pgfpathlineto{\pgfqpoint{4.279552in}{1.262775in}}%
\pgfpathlineto{\pgfqpoint{4.222493in}{1.304118in}}%
\pgfpathlineto{\pgfqpoint{4.158012in}{1.342003in}}%
\pgfpathlineto{\pgfqpoint{4.084725in}{1.374671in}}%
\pgfpathlineto{\pgfqpoint{4.002554in}{1.400177in}}%
\pgfpathlineto{\pgfqpoint{3.913621in}{1.417770in}}%
\pgfpathlineto{\pgfqpoint{3.821081in}{1.429004in}}%
\pgfpathlineto{\pgfqpoint{3.727280in}{1.436972in}}%
\pgfpathlineto{\pgfqpoint{3.633470in}{1.444908in}}%
\pgfpathlineto{\pgfqpoint{3.540597in}{1.455459in}}%
\pgfpathlineto{\pgfqpoint{3.449874in}{1.470444in}}%
\pgfpathlineto{\pgfqpoint{3.362746in}{1.490730in}}%
\pgfpathlineto{\pgfqpoint{3.280447in}{1.516334in}}%
\pgfpathlineto{\pgfqpoint{3.203779in}{1.546732in}}%
\pgfpathlineto{\pgfqpoint{3.133071in}{1.581164in}}%
\pgfpathlineto{\pgfqpoint{3.068177in}{1.618889in}}%
\pgfpathlineto{\pgfqpoint{3.008839in}{1.659265in}}%
\pgfpathlineto{\pgfqpoint{2.954751in}{1.701780in}}%
\pgfpathlineto{\pgfqpoint{2.905623in}{1.746062in}}%
\pgfpathlineto{\pgfqpoint{2.861273in}{1.791828in}}%
\pgfusepath{stroke}%
\end{pgfscope}%
\begin{pgfscope}%
\pgfpathrectangle{\pgfqpoint{0.647939in}{0.492442in}}{\pgfqpoint{4.273799in}{2.331163in}}%
\pgfusepath{clip}%
\pgfsetbuttcap%
\pgfsetroundjoin%
\pgfsetlinewidth{0.301125pt}%
\definecolor{currentstroke}{rgb}{0.500000,0.500000,0.500000}%
\pgfsetstrokecolor{currentstroke}%
\pgfsetstrokeopacity{0.300000}%
\pgfsetdash{}{0pt}%
\pgfpathmoveto{\pgfqpoint{4.921738in}{0.492442in}}%
\pgfpathlineto{\pgfqpoint{4.921738in}{0.492442in}}%
\pgfpathlineto{\pgfqpoint{4.899337in}{0.542783in}}%
\pgfpathlineto{\pgfqpoint{4.876607in}{0.593080in}}%
\pgfpathlineto{\pgfqpoint{4.853524in}{0.643330in}}%
\pgfpathlineto{\pgfqpoint{4.830065in}{0.693527in}}%
\pgfpathlineto{\pgfqpoint{4.806198in}{0.743666in}}%
\pgfpathlineto{\pgfqpoint{4.781890in}{0.793743in}}%
\pgfpathlineto{\pgfqpoint{4.757099in}{0.843749in}}%
\pgfpathlineto{\pgfqpoint{4.731779in}{0.893675in}}%
\pgfpathlineto{\pgfqpoint{4.705870in}{0.943512in}}%
\pgfpathlineto{\pgfqpoint{4.679308in}{0.993247in}}%
\pgfpathlineto{\pgfqpoint{4.652018in}{1.042862in}}%
\pgfpathlineto{\pgfqpoint{4.623888in}{1.092338in}}%
\pgfpathlineto{\pgfqpoint{4.594800in}{1.141648in}}%
\pgfpathlineto{\pgfqpoint{4.564594in}{1.190756in}}%
\pgfpathlineto{\pgfqpoint{4.533042in}{1.239613in}}%
\pgfpathlineto{\pgfqpoint{4.499873in}{1.288149in}}%
\pgfpathlineto{\pgfqpoint{4.464715in}{1.336259in}}%
\pgfpathlineto{\pgfqpoint{4.427002in}{1.383783in}}%
\pgfpathlineto{\pgfqpoint{4.385898in}{1.430451in}}%
\pgfpathlineto{\pgfqpoint{4.340102in}{1.475776in}}%
\pgfpathlineto{\pgfqpoint{4.287487in}{1.518778in}}%
\pgfpathlineto{\pgfqpoint{4.224476in}{1.557259in}}%
\pgfpathlineto{\pgfqpoint{4.224476in}{1.557259in}}%
\pgfpathlineto{\pgfqpoint{4.161755in}{1.582199in}}%
\pgfpathlineto{\pgfqpoint{4.087842in}{1.596601in}}%
\pgfpathlineto{\pgfqpoint{4.020597in}{1.598868in}}%
\pgfpathlineto{\pgfqpoint{3.950704in}{1.593826in}}%
\pgfpathlineto{\pgfqpoint{3.864625in}{1.582050in}}%
\pgfpathlineto{\pgfqpoint{3.773439in}{1.567747in}}%
\pgfpathlineto{\pgfqpoint{3.681346in}{1.555352in}}%
\pgfpathlineto{\pgfqpoint{3.587668in}{1.547715in}}%
\pgfpathlineto{\pgfqpoint{3.493110in}{1.547120in}}%
\pgfpathlineto{\pgfqpoint{3.399776in}{1.555017in}}%
\pgfpathlineto{\pgfqpoint{3.310361in}{1.571556in}}%
\pgfpathlineto{\pgfqpoint{3.226939in}{1.595773in}}%
\pgfusepath{stroke}%
\end{pgfscope}%
\begin{pgfscope}%
\pgfpathrectangle{\pgfqpoint{0.647939in}{0.492442in}}{\pgfqpoint{4.273799in}{2.331163in}}%
\pgfusepath{clip}%
\pgfsetbuttcap%
\pgfsetroundjoin%
\pgfsetlinewidth{0.301125pt}%
\definecolor{currentstroke}{rgb}{0.500000,0.500000,0.500000}%
\pgfsetstrokecolor{currentstroke}%
\pgfsetstrokeopacity{0.300000}%
\pgfsetdash{}{0pt}%
\pgfpathmoveto{\pgfqpoint{4.921738in}{0.704366in}}%
\pgfpathlineto{\pgfqpoint{4.921738in}{0.704366in}}%
\pgfpathlineto{\pgfqpoint{4.901059in}{0.754926in}}%
\pgfpathlineto{\pgfqpoint{4.880173in}{0.805461in}}%
\pgfpathlineto{\pgfqpoint{4.859070in}{0.855968in}}%
\pgfpathlineto{\pgfqpoint{4.837738in}{0.906447in}}%
\pgfpathlineto{\pgfqpoint{4.816166in}{0.956895in}}%
\pgfpathlineto{\pgfqpoint{4.794336in}{1.007311in}}%
\pgfpathlineto{\pgfqpoint{4.772233in}{1.057691in}}%
\pgfpathlineto{\pgfqpoint{4.749836in}{1.108032in}}%
\pgfpathlineto{\pgfqpoint{4.727122in}{1.158331in}}%
\pgfpathlineto{\pgfqpoint{4.704063in}{1.208583in}}%
\pgfpathlineto{\pgfqpoint{4.680625in}{1.258782in}}%
\pgfpathlineto{\pgfqpoint{4.656768in}{1.308922in}}%
\pgfpathlineto{\pgfqpoint{4.632437in}{1.358994in}}%
\pgfpathlineto{\pgfqpoint{4.607569in}{1.408985in}}%
\pgfpathlineto{\pgfqpoint{4.582066in}{1.458883in}}%
\pgfpathlineto{\pgfqpoint{4.555816in}{1.508662in}}%
\pgfpathlineto{\pgfqpoint{4.528643in}{1.558294in}}%
\pgfpathlineto{\pgfqpoint{4.500296in}{1.607728in}}%
\pgfpathlineto{\pgfqpoint{4.470402in}{1.656882in}}%
\pgfpathlineto{\pgfqpoint{4.438285in}{1.705610in}}%
\pgfpathlineto{\pgfqpoint{4.402657in}{1.753603in}}%
\pgfpathlineto{\pgfqpoint{4.360696in}{1.799991in}}%
\pgfpathlineto{\pgfqpoint{4.304150in}{1.840929in}}%
\pgfpathlineto{\pgfqpoint{4.304150in}{1.840929in}}%
\pgfpathlineto{\pgfqpoint{4.268026in}{1.852904in}}%
\pgfpathlineto{\pgfqpoint{4.268026in}{1.852904in}}%
\pgfpathlineto{\pgfqpoint{4.231530in}{1.854393in}}%
\pgfpathlineto{\pgfqpoint{4.196562in}{1.847953in}}%
\pgfpathlineto{\pgfqpoint{4.159728in}{1.835255in}}%
\pgfpathlineto{\pgfqpoint{4.111745in}{1.813092in}}%
\pgfpathlineto{\pgfqpoint{4.043884in}{1.777111in}}%
\pgfpathlineto{\pgfqpoint{3.976519in}{1.740707in}}%
\pgfpathlineto{\pgfqpoint{3.907156in}{1.705411in}}%
\pgfpathlineto{\pgfqpoint{3.834297in}{1.672298in}}%
\pgfpathlineto{\pgfqpoint{3.756820in}{1.642507in}}%
\pgfpathlineto{\pgfqpoint{3.673938in}{1.617542in}}%
\pgfpathlineto{\pgfqpoint{3.585486in}{1.599325in}}%
\pgfpathlineto{\pgfqpoint{3.492661in}{1.590016in}}%
\pgfpathlineto{\pgfqpoint{3.402162in}{1.591039in}}%
\pgfusepath{stroke}%
\end{pgfscope}%
\begin{pgfscope}%
\pgfpathrectangle{\pgfqpoint{0.647939in}{0.492442in}}{\pgfqpoint{4.273799in}{2.331163in}}%
\pgfusepath{clip}%
\pgfsetbuttcap%
\pgfsetroundjoin%
\pgfsetlinewidth{0.301125pt}%
\definecolor{currentstroke}{rgb}{0.500000,0.500000,0.500000}%
\pgfsetstrokecolor{currentstroke}%
\pgfsetstrokeopacity{0.300000}%
\pgfsetdash{}{0pt}%
\pgfpathmoveto{\pgfqpoint{4.921738in}{0.916290in}}%
\pgfpathlineto{\pgfqpoint{4.921738in}{0.916290in}}%
\pgfpathlineto{\pgfqpoint{4.903040in}{0.967079in}}%
\pgfpathlineto{\pgfqpoint{4.884291in}{1.017862in}}%
\pgfpathlineto{\pgfqpoint{4.865494in}{1.068640in}}%
\pgfpathlineto{\pgfqpoint{4.846660in}{1.119414in}}%
\pgfpathlineto{\pgfqpoint{4.827797in}{1.170184in}}%
\pgfpathlineto{\pgfqpoint{4.808920in}{1.220953in}}%
\pgfpathlineto{\pgfqpoint{4.790040in}{1.271721in}}%
\pgfpathlineto{\pgfqpoint{4.771178in}{1.322492in}}%
\pgfpathlineto{\pgfqpoint{4.752358in}{1.373267in}}%
\pgfpathlineto{\pgfqpoint{4.733607in}{1.424049in}}%
\pgfpathlineto{\pgfqpoint{4.714964in}{1.474843in}}%
\pgfpathlineto{\pgfqpoint{4.696468in}{1.525653in}}%
\pgfpathlineto{\pgfqpoint{4.678182in}{1.576485in}}%
\pgfpathlineto{\pgfqpoint{4.660171in}{1.627346in}}%
\pgfpathlineto{\pgfqpoint{4.642538in}{1.678247in}}%
\pgfpathlineto{\pgfqpoint{4.625410in}{1.729198in}}%
\pgfpathlineto{\pgfqpoint{4.608955in}{1.780215in}}%
\pgfpathlineto{\pgfqpoint{4.593415in}{1.831315in}}%
\pgfpathlineto{\pgfqpoint{4.579107in}{1.882524in}}%
\pgfpathlineto{\pgfqpoint{4.566494in}{1.933863in}}%
\pgfpathlineto{\pgfqpoint{4.556235in}{1.985354in}}%
\pgfpathlineto{\pgfqpoint{4.549224in}{2.037002in}}%
\pgfpathlineto{\pgfqpoint{4.546617in}{2.088765in}}%
\pgfpathlineto{\pgfqpoint{4.549677in}{2.140510in}}%
\pgfpathlineto{\pgfqpoint{4.559295in}{2.191999in}}%
\pgfpathlineto{\pgfqpoint{4.575437in}{2.242992in}}%
\pgfpathlineto{\pgfqpoint{4.597014in}{2.293385in}}%
\pgfpathlineto{\pgfqpoint{4.622512in}{2.343240in}}%
\pgfpathlineto{\pgfqpoint{4.650510in}{2.392699in}}%
\pgfpathlineto{\pgfqpoint{4.679948in}{2.441897in}}%
\pgfpathlineto{\pgfqpoint{4.710181in}{2.490973in}}%
\pgfpathlineto{\pgfqpoint{4.740742in}{2.539998in}}%
\pgfpathlineto{\pgfqpoint{4.771318in}{2.589012in}}%
\pgfpathlineto{\pgfqpoint{4.801756in}{2.638055in}}%
\pgfpathlineto{\pgfqpoint{4.831968in}{2.687152in}}%
\pgfpathlineto{\pgfqpoint{4.861858in}{2.736306in}}%
\pgfpathlineto{\pgfqpoint{4.891384in}{2.785521in}}%
\pgfpathlineto{\pgfqpoint{4.914087in}{2.823605in}}%
\pgfusepath{stroke}%
\end{pgfscope}%
\begin{pgfscope}%
\pgfpathrectangle{\pgfqpoint{0.647939in}{0.492442in}}{\pgfqpoint{4.273799in}{2.331163in}}%
\pgfusepath{clip}%
\pgfsetbuttcap%
\pgfsetroundjoin%
\pgfsetlinewidth{0.301125pt}%
\definecolor{currentstroke}{rgb}{0.500000,0.500000,0.500000}%
\pgfsetstrokecolor{currentstroke}%
\pgfsetstrokeopacity{0.300000}%
\pgfsetdash{}{0pt}%
\pgfpathmoveto{\pgfqpoint{4.921738in}{1.181195in}}%
\pgfpathlineto{\pgfqpoint{4.921738in}{1.181195in}}%
\pgfpathlineto{\pgfqpoint{4.905972in}{1.232279in}}%
\pgfpathlineto{\pgfqpoint{4.890400in}{1.283381in}}%
\pgfpathlineto{\pgfqpoint{4.875049in}{1.334502in}}%
\pgfpathlineto{\pgfqpoint{4.859964in}{1.385647in}}%
\pgfpathlineto{\pgfqpoint{4.845193in}{1.436820in}}%
\pgfpathlineto{\pgfqpoint{4.830785in}{1.488023in}}%
\pgfpathlineto{\pgfqpoint{4.816809in}{1.539261in}}%
\pgfpathlineto{\pgfqpoint{4.803340in}{1.590540in}}%
\pgfpathlineto{\pgfqpoint{4.790459in}{1.641864in}}%
\pgfpathlineto{\pgfqpoint{4.778275in}{1.693238in}}%
\pgfpathlineto{\pgfqpoint{4.766912in}{1.744668in}}%
\pgfpathlineto{\pgfqpoint{4.756506in}{1.796158in}}%
\pgfpathlineto{\pgfqpoint{4.747220in}{1.847712in}}%
\pgfpathlineto{\pgfqpoint{4.739246in}{1.899330in}}%
\pgfpathlineto{\pgfqpoint{4.732803in}{1.951012in}}%
\pgfpathlineto{\pgfqpoint{4.728123in}{2.002750in}}%
\pgfpathlineto{\pgfqpoint{4.725457in}{2.054529in}}%
\pgfpathlineto{\pgfqpoint{4.725057in}{2.106327in}}%
\pgfpathlineto{\pgfqpoint{4.727151in}{2.158112in}}%
\pgfpathlineto{\pgfqpoint{4.731918in}{2.209844in}}%
\pgfpathlineto{\pgfqpoint{4.739461in}{2.261477in}}%
\pgfpathlineto{\pgfqpoint{4.749783in}{2.312966in}}%
\pgfpathlineto{\pgfqpoint{4.762761in}{2.364275in}}%
\pgfpathlineto{\pgfqpoint{4.778179in}{2.415380in}}%
\pgfpathlineto{\pgfqpoint{4.795771in}{2.466278in}}%
\pgfpathlineto{\pgfqpoint{4.815205in}{2.516976in}}%
\pgfpathlineto{\pgfqpoint{4.836167in}{2.567496in}}%
\pgfusepath{stroke}%
\end{pgfscope}%
\begin{pgfscope}%
\pgfpathrectangle{\pgfqpoint{0.647939in}{0.492442in}}{\pgfqpoint{4.273799in}{2.331163in}}%
\pgfusepath{clip}%
\pgfsetbuttcap%
\pgfsetroundjoin%
\pgfsetlinewidth{0.301125pt}%
\definecolor{currentstroke}{rgb}{0.500000,0.500000,0.500000}%
\pgfsetstrokecolor{currentstroke}%
\pgfsetstrokeopacity{0.300000}%
\pgfsetdash{}{0pt}%
\pgfpathmoveto{\pgfqpoint{4.921738in}{1.499081in}}%
\pgfpathlineto{\pgfqpoint{4.921738in}{1.499081in}}%
\pgfpathlineto{\pgfqpoint{4.910365in}{1.550511in}}%
\pgfpathlineto{\pgfqpoint{4.899564in}{1.601978in}}%
\pgfpathlineto{\pgfqpoint{4.889403in}{1.653483in}}%
\pgfpathlineto{\pgfqpoint{4.879959in}{1.705029in}}%
\pgfpathlineto{\pgfqpoint{4.871322in}{1.756617in}}%
\pgfpathlineto{\pgfqpoint{4.863593in}{1.808248in}}%
\pgfpathlineto{\pgfqpoint{4.856875in}{1.859921in}}%
\pgfpathlineto{\pgfqpoint{4.851283in}{1.911633in}}%
\pgfpathlineto{\pgfqpoint{4.846934in}{1.963381in}}%
\pgfpathlineto{\pgfqpoint{4.843952in}{2.015157in}}%
\pgfpathlineto{\pgfqpoint{4.842460in}{2.066952in}}%
\pgfpathlineto{\pgfqpoint{4.842574in}{2.118754in}}%
\pgfpathlineto{\pgfqpoint{4.844399in}{2.170545in}}%
\pgfpathlineto{\pgfqpoint{4.848016in}{2.222308in}}%
\pgfpathlineto{\pgfqpoint{4.853481in}{2.274022in}}%
\pgfpathlineto{\pgfqpoint{4.860813in}{2.325668in}}%
\pgfusepath{stroke}%
\end{pgfscope}%
\begin{pgfscope}%
\pgfpathrectangle{\pgfqpoint{0.647939in}{0.492442in}}{\pgfqpoint{4.273799in}{2.331163in}}%
\pgfusepath{clip}%
\pgfsetbuttcap%
\pgfsetroundjoin%
\pgfsetlinewidth{0.301125pt}%
\definecolor{currentstroke}{rgb}{0.500000,0.500000,0.500000}%
\pgfsetstrokecolor{currentstroke}%
\pgfsetstrokeopacity{0.300000}%
\pgfsetdash{}{0pt}%
\pgfpathmoveto{\pgfqpoint{4.921738in}{1.763986in}}%
\pgfpathlineto{\pgfqpoint{4.921738in}{1.763986in}}%
\pgfpathlineto{\pgfqpoint{4.915009in}{1.815658in}}%
\pgfpathlineto{\pgfqpoint{4.909226in}{1.867365in}}%
\pgfpathlineto{\pgfqpoint{4.904480in}{1.919103in}}%
\pgfpathlineto{\pgfqpoint{4.900860in}{1.970867in}}%
\pgfpathlineto{\pgfqpoint{4.898460in}{2.022653in}}%
\pgfpathlineto{\pgfqpoint{4.897369in}{2.074452in}}%
\pgfpathlineto{\pgfqpoint{4.897672in}{2.126254in}}%
\pgfpathlineto{\pgfqpoint{4.899443in}{2.178046in}}%
\pgfpathlineto{\pgfqpoint{4.902743in}{2.229816in}}%
\pgfpathlineto{\pgfqpoint{4.907613in}{2.281550in}}%
\pgfpathlineto{\pgfqpoint{4.914073in}{2.333231in}}%
\pgfusepath{stroke}%
\end{pgfscope}%
\begin{pgfscope}%
\pgfpathrectangle{\pgfqpoint{0.647939in}{0.492442in}}{\pgfqpoint{4.273799in}{2.331163in}}%
\pgfusepath{clip}%
\pgfsetbuttcap%
\pgfsetroundjoin%
\pgfsetlinewidth{0.301125pt}%
\definecolor{currentstroke}{rgb}{0.500000,0.500000,0.500000}%
\pgfsetstrokecolor{currentstroke}%
\pgfsetstrokeopacity{0.300000}%
\pgfsetdash{}{0pt}%
\pgfpathmoveto{\pgfqpoint{4.337823in}{2.823605in}}%
\pgfpathlineto{\pgfqpoint{4.358427in}{2.802516in}}%
\pgfpathlineto{\pgfqpoint{4.405941in}{2.757715in}}%
\pgfpathlineto{\pgfqpoint{4.451773in}{2.722486in}}%
\pgfpathlineto{\pgfqpoint{4.491871in}{2.699223in}}%
\pgfpathlineto{\pgfqpoint{4.529581in}{2.684622in}}%
\pgfpathlineto{\pgfqpoint{4.572263in}{2.677305in}}%
\pgfpathlineto{\pgfqpoint{4.617050in}{2.680401in}}%
\pgfpathlineto{\pgfqpoint{4.617050in}{2.680401in}}%
\pgfpathlineto{\pgfqpoint{4.666179in}{2.696349in}}%
\pgfpathlineto{\pgfqpoint{4.666179in}{2.696349in}}%
\pgfpathlineto{\pgfqpoint{4.729021in}{2.734495in}}%
\pgfpathlineto{\pgfqpoint{4.780090in}{2.777940in}}%
\pgfpathlineto{\pgfqpoint{4.824607in}{2.823605in}}%
\pgfpathlineto{\pgfqpoint{4.824607in}{2.823605in}}%
\pgfusepath{stroke}%
\end{pgfscope}%
\begin{pgfscope}%
\pgfpathrectangle{\pgfqpoint{0.647939in}{0.492442in}}{\pgfqpoint{4.273799in}{2.331163in}}%
\pgfusepath{clip}%
\pgfsetbuttcap%
\pgfsetroundjoin%
\pgfsetlinewidth{0.301125pt}%
\definecolor{currentstroke}{rgb}{0.500000,0.500000,0.500000}%
\pgfsetstrokecolor{currentstroke}%
\pgfsetstrokeopacity{0.300000}%
\pgfsetdash{}{0pt}%
\pgfpathmoveto{\pgfqpoint{4.436079in}{2.823605in}}%
\pgfpathlineto{\pgfqpoint{4.436079in}{2.823605in}}%
\pgfpathlineto{\pgfqpoint{4.496710in}{2.783992in}}%
\pgfpathlineto{\pgfqpoint{4.496710in}{2.783992in}}%
\pgfpathlineto{\pgfqpoint{4.547713in}{2.763087in}}%
\pgfpathlineto{\pgfqpoint{4.547713in}{2.763087in}}%
\pgfpathlineto{\pgfqpoint{4.593430in}{2.755097in}}%
\pgfpathlineto{\pgfqpoint{4.641516in}{2.758159in}}%
\pgfpathlineto{\pgfqpoint{4.680932in}{2.769234in}}%
\pgfpathlineto{\pgfqpoint{4.720108in}{2.787635in}}%
\pgfpathlineto{\pgfqpoint{4.762818in}{2.815560in}}%
\pgfpathlineto{\pgfqpoint{4.773667in}{2.823605in}}%
\pgfusepath{stroke}%
\end{pgfscope}%
\begin{pgfscope}%
\pgfpathrectangle{\pgfqpoint{0.647939in}{0.492442in}}{\pgfqpoint{4.273799in}{2.331163in}}%
\pgfusepath{clip}%
\pgfsetbuttcap%
\pgfsetroundjoin%
\pgfsetlinewidth{0.301125pt}%
\definecolor{currentstroke}{rgb}{0.500000,0.500000,0.500000}%
\pgfsetstrokecolor{currentstroke}%
\pgfsetstrokeopacity{0.300000}%
\pgfsetdash{}{0pt}%
\pgfpathmoveto{\pgfqpoint{4.241816in}{2.823605in}}%
\pgfpathlineto{\pgfqpoint{4.241816in}{2.823605in}}%
\pgfpathlineto{\pgfqpoint{4.278842in}{2.775906in}}%
\pgfpathlineto{\pgfqpoint{4.317991in}{2.728717in}}%
\pgfpathlineto{\pgfqpoint{4.360498in}{2.682410in}}%
\pgfpathlineto{\pgfqpoint{4.408784in}{2.637865in}}%
\pgfpathlineto{\pgfqpoint{4.468452in}{2.598096in}}%
\pgfpathlineto{\pgfqpoint{4.468452in}{2.598096in}}%
\pgfpathlineto{\pgfqpoint{4.511124in}{2.581915in}}%
\pgfpathlineto{\pgfqpoint{4.511124in}{2.581915in}}%
\pgfpathlineto{\pgfqpoint{4.551147in}{2.577062in}}%
\pgfpathlineto{\pgfqpoint{4.591293in}{2.582359in}}%
\pgfpathlineto{\pgfqpoint{4.625738in}{2.594593in}}%
\pgfpathlineto{\pgfqpoint{4.662061in}{2.614632in}}%
\pgfpathlineto{\pgfqpoint{4.703050in}{2.645125in}}%
\pgfpathlineto{\pgfqpoint{4.750807in}{2.689762in}}%
\pgfusepath{stroke}%
\end{pgfscope}%
\begin{pgfscope}%
\pgfpathrectangle{\pgfqpoint{0.647939in}{0.492442in}}{\pgfqpoint{4.273799in}{2.331163in}}%
\pgfusepath{clip}%
\pgfsetbuttcap%
\pgfsetroundjoin%
\pgfsetlinewidth{0.301125pt}%
\definecolor{currentstroke}{rgb}{0.500000,0.500000,0.500000}%
\pgfsetstrokecolor{currentstroke}%
\pgfsetstrokeopacity{0.300000}%
\pgfsetdash{}{0pt}%
\pgfpathmoveto{\pgfqpoint{4.144684in}{2.823605in}}%
\pgfpathlineto{\pgfqpoint{4.144684in}{2.823605in}}%
\pgfpathlineto{\pgfqpoint{4.177110in}{2.774916in}}%
\pgfpathlineto{\pgfqpoint{4.210160in}{2.726353in}}%
\pgfpathlineto{\pgfqpoint{4.244141in}{2.677985in}}%
\pgfpathlineto{\pgfqpoint{4.279533in}{2.629920in}}%
\pgfpathlineto{\pgfqpoint{4.317167in}{2.582367in}}%
\pgfpathlineto{\pgfqpoint{4.358651in}{2.535796in}}%
\pgfpathlineto{\pgfqpoint{4.407692in}{2.491549in}}%
\pgfpathlineto{\pgfqpoint{4.407692in}{2.491549in}}%
\pgfpathlineto{\pgfqpoint{4.453244in}{2.464038in}}%
\pgfpathlineto{\pgfqpoint{4.453244in}{2.464038in}}%
\pgfpathlineto{\pgfqpoint{4.490458in}{2.453123in}}%
\pgfpathlineto{\pgfqpoint{4.490458in}{2.453123in}}%
\pgfpathlineto{\pgfqpoint{4.525033in}{2.452904in}}%
\pgfpathlineto{\pgfqpoint{4.556446in}{2.460675in}}%
\pgfpathlineto{\pgfqpoint{4.587072in}{2.474955in}}%
\pgfpathlineto{\pgfqpoint{4.622146in}{2.498409in}}%
\pgfusepath{stroke}%
\end{pgfscope}%
\begin{pgfscope}%
\pgfpathrectangle{\pgfqpoint{0.647939in}{0.492442in}}{\pgfqpoint{4.273799in}{2.331163in}}%
\pgfusepath{clip}%
\pgfsetbuttcap%
\pgfsetroundjoin%
\pgfsetlinewidth{0.301125pt}%
\definecolor{currentstroke}{rgb}{0.500000,0.500000,0.500000}%
\pgfsetstrokecolor{currentstroke}%
\pgfsetstrokeopacity{0.300000}%
\pgfsetdash{}{0pt}%
\pgfpathmoveto{\pgfqpoint{4.047552in}{2.823605in}}%
\pgfpathlineto{\pgfqpoint{4.047552in}{2.823605in}}%
\pgfpathlineto{\pgfqpoint{4.077230in}{2.774397in}}%
\pgfpathlineto{\pgfqpoint{4.106837in}{2.725175in}}%
\pgfpathlineto{\pgfqpoint{4.136446in}{2.675954in}}%
\pgfpathlineto{\pgfqpoint{4.166166in}{2.626754in}}%
\pgfpathlineto{\pgfqpoint{4.196129in}{2.577599in}}%
\pgfpathlineto{\pgfqpoint{4.226534in}{2.528525in}}%
\pgfpathlineto{\pgfqpoint{4.257727in}{2.479602in}}%
\pgfpathlineto{\pgfqpoint{4.290297in}{2.430957in}}%
\pgfpathlineto{\pgfqpoint{4.325401in}{2.382855in}}%
\pgfpathlineto{\pgfqpoint{4.366033in}{2.336123in}}%
\pgfpathlineto{\pgfqpoint{4.366033in}{2.336123in}}%
\pgfpathlineto{\pgfqpoint{4.406484in}{2.304188in}}%
\pgfpathlineto{\pgfqpoint{4.406484in}{2.304188in}}%
\pgfpathlineto{\pgfqpoint{4.434598in}{2.293413in}}%
\pgfpathlineto{\pgfqpoint{4.434598in}{2.293413in}}%
\pgfpathlineto{\pgfqpoint{4.461866in}{2.292888in}}%
\pgfpathlineto{\pgfqpoint{4.486998in}{2.300218in}}%
\pgfpathlineto{\pgfqpoint{4.511771in}{2.313694in}}%
\pgfpathlineto{\pgfqpoint{4.541220in}{2.336173in}}%
\pgfpathlineto{\pgfqpoint{4.580255in}{2.373662in}}%
\pgfusepath{stroke}%
\end{pgfscope}%
\begin{pgfscope}%
\pgfpathrectangle{\pgfqpoint{0.647939in}{0.492442in}}{\pgfqpoint{4.273799in}{2.331163in}}%
\pgfusepath{clip}%
\pgfsetbuttcap%
\pgfsetroundjoin%
\pgfsetlinewidth{0.301125pt}%
\definecolor{currentstroke}{rgb}{0.500000,0.500000,0.500000}%
\pgfsetstrokecolor{currentstroke}%
\pgfsetstrokeopacity{0.300000}%
\pgfsetdash{}{0pt}%
\pgfpathmoveto{\pgfqpoint{3.950420in}{2.823605in}}%
\pgfpathlineto{\pgfqpoint{3.950420in}{2.823605in}}%
\pgfpathlineto{\pgfqpoint{3.978454in}{2.774110in}}%
\pgfpathlineto{\pgfqpoint{4.006059in}{2.724544in}}%
\pgfpathlineto{\pgfqpoint{4.033235in}{2.674907in}}%
\pgfpathlineto{\pgfqpoint{4.059964in}{2.625198in}}%
\pgfpathlineto{\pgfqpoint{4.086233in}{2.575416in}}%
\pgfpathlineto{\pgfqpoint{4.112024in}{2.525561in}}%
\pgfpathlineto{\pgfqpoint{4.137282in}{2.475624in}}%
\pgfpathlineto{\pgfqpoint{4.161953in}{2.425601in}}%
\pgfpathlineto{\pgfqpoint{4.185924in}{2.375477in}}%
\pgfpathlineto{\pgfqpoint{4.209034in}{2.325235in}}%
\pgfpathlineto{\pgfqpoint{4.230989in}{2.274840in}}%
\pgfpathlineto{\pgfqpoint{4.251249in}{2.224239in}}%
\pgfpathlineto{\pgfqpoint{4.268703in}{2.173339in}}%
\pgfpathlineto{\pgfqpoint{4.280746in}{2.122005in}}%
\pgfpathlineto{\pgfqpoint{4.281179in}{2.070396in}}%
\pgfpathlineto{\pgfqpoint{4.262211in}{2.020185in}}%
\pgfpathlineto{\pgfqpoint{4.233611in}{1.978970in}}%
\pgfpathlineto{\pgfqpoint{4.191215in}{1.932914in}}%
\pgfpathlineto{\pgfqpoint{4.143253in}{1.888364in}}%
\pgfpathlineto{\pgfqpoint{4.091151in}{1.845188in}}%
\pgfpathlineto{\pgfqpoint{4.035300in}{1.803450in}}%
\pgfusepath{stroke}%
\end{pgfscope}%
\begin{pgfscope}%
\pgfpathrectangle{\pgfqpoint{0.647939in}{0.492442in}}{\pgfqpoint{4.273799in}{2.331163in}}%
\pgfusepath{clip}%
\pgfsetbuttcap%
\pgfsetroundjoin%
\pgfsetlinewidth{0.301125pt}%
\definecolor{currentstroke}{rgb}{0.500000,0.500000,0.500000}%
\pgfsetstrokecolor{currentstroke}%
\pgfsetstrokeopacity{0.300000}%
\pgfsetdash{}{0pt}%
\pgfpathmoveto{\pgfqpoint{3.853289in}{2.823605in}}%
\pgfpathlineto{\pgfqpoint{3.853289in}{2.823605in}}%
\pgfpathlineto{\pgfqpoint{3.880437in}{2.773964in}}%
\pgfpathlineto{\pgfqpoint{3.906941in}{2.724219in}}%
\pgfpathlineto{\pgfqpoint{3.932767in}{2.674368in}}%
\pgfpathlineto{\pgfqpoint{3.957868in}{2.624407in}}%
\pgfpathlineto{\pgfqpoint{3.982189in}{2.574332in}}%
\pgfpathlineto{\pgfqpoint{4.005644in}{2.524134in}}%
\pgfpathlineto{\pgfqpoint{4.028127in}{2.473804in}}%
\pgfpathlineto{\pgfqpoint{4.049489in}{2.423330in}}%
\pgfpathlineto{\pgfqpoint{4.069529in}{2.372695in}}%
\pgfpathlineto{\pgfqpoint{4.087965in}{2.321880in}}%
\pgfpathlineto{\pgfqpoint{4.104395in}{2.270863in}}%
\pgfpathlineto{\pgfqpoint{4.118255in}{2.219622in}}%
\pgfpathlineto{\pgfqpoint{4.128736in}{2.168148in}}%
\pgfpathlineto{\pgfqpoint{4.134678in}{2.116467in}}%
\pgfpathlineto{\pgfqpoint{4.134545in}{2.064701in}}%
\pgfpathlineto{\pgfqpoint{4.126550in}{2.013151in}}%
\pgfpathlineto{\pgfqpoint{4.109173in}{1.962329in}}%
\pgfpathlineto{\pgfqpoint{4.081866in}{1.912843in}}%
\pgfusepath{stroke}%
\end{pgfscope}%
\begin{pgfscope}%
\pgfpathrectangle{\pgfqpoint{0.647939in}{0.492442in}}{\pgfqpoint{4.273799in}{2.331163in}}%
\pgfusepath{clip}%
\pgfsetbuttcap%
\pgfsetroundjoin%
\pgfsetlinewidth{0.301125pt}%
\definecolor{currentstroke}{rgb}{0.500000,0.500000,0.500000}%
\pgfsetstrokecolor{currentstroke}%
\pgfsetstrokeopacity{0.300000}%
\pgfsetdash{}{0pt}%
\pgfpathmoveto{\pgfqpoint{3.756157in}{2.823605in}}%
\pgfpathlineto{\pgfqpoint{3.756157in}{2.823605in}}%
\pgfpathlineto{\pgfqpoint{3.782991in}{2.773913in}}%
\pgfpathlineto{\pgfqpoint{3.809031in}{2.724095in}}%
\pgfpathlineto{\pgfqpoint{3.834230in}{2.674149in}}%
\pgfpathlineto{\pgfqpoint{3.858533in}{2.624071in}}%
\pgfpathlineto{\pgfqpoint{3.881864in}{2.573856in}}%
\pgfpathlineto{\pgfqpoint{3.904134in}{2.523498in}}%
\pgfpathlineto{\pgfqpoint{3.925221in}{2.472989in}}%
\pgfpathlineto{\pgfqpoint{3.944972in}{2.422320in}}%
\pgfpathlineto{\pgfqpoint{3.963187in}{2.371480in}}%
\pgfpathlineto{\pgfqpoint{3.979603in}{2.320460in}}%
\pgfpathlineto{\pgfqpoint{3.993880in}{2.269250in}}%
\pgfpathlineto{\pgfqpoint{4.005571in}{2.217848in}}%
\pgfpathlineto{\pgfqpoint{4.014086in}{2.166262in}}%
\pgfpathlineto{\pgfqpoint{4.018681in}{2.114534in}}%
\pgfpathlineto{\pgfqpoint{4.018427in}{2.062754in}}%
\pgfpathlineto{\pgfqpoint{4.012243in}{2.011098in}}%
\pgfpathlineto{\pgfqpoint{3.998984in}{1.959855in}}%
\pgfpathlineto{\pgfqpoint{3.977655in}{1.909445in}}%
\pgfpathlineto{\pgfqpoint{3.947617in}{1.860385in}}%
\pgfpathlineto{\pgfqpoint{3.908632in}{1.813242in}}%
\pgfpathlineto{\pgfqpoint{3.860782in}{1.768610in}}%
\pgfpathlineto{\pgfqpoint{3.804226in}{1.727152in}}%
\pgfpathlineto{\pgfqpoint{3.738967in}{1.689736in}}%
\pgfpathlineto{\pgfqpoint{3.664896in}{1.657634in}}%
\pgfpathlineto{\pgfqpoint{3.582231in}{1.632739in}}%
\pgfusepath{stroke}%
\end{pgfscope}%
\begin{pgfscope}%
\pgfpathrectangle{\pgfqpoint{0.647939in}{0.492442in}}{\pgfqpoint{4.273799in}{2.331163in}}%
\pgfusepath{clip}%
\pgfsetbuttcap%
\pgfsetroundjoin%
\pgfsetlinewidth{0.301125pt}%
\definecolor{currentstroke}{rgb}{0.500000,0.500000,0.500000}%
\pgfsetstrokecolor{currentstroke}%
\pgfsetstrokeopacity{0.300000}%
\pgfsetdash{}{0pt}%
\pgfpathmoveto{\pgfqpoint{3.659025in}{2.823605in}}%
\pgfpathlineto{\pgfqpoint{3.659025in}{2.823605in}}%
\pgfpathlineto{\pgfqpoint{3.686020in}{2.773939in}}%
\pgfpathlineto{\pgfqpoint{3.712093in}{2.724127in}}%
\pgfpathlineto{\pgfqpoint{3.737203in}{2.674167in}}%
\pgfpathlineto{\pgfqpoint{3.761291in}{2.624059in}}%
\pgfpathlineto{\pgfqpoint{3.784280in}{2.573797in}}%
\pgfpathlineto{\pgfqpoint{3.806078in}{2.523378in}}%
\pgfpathlineto{\pgfqpoint{3.826565in}{2.472795in}}%
\pgfpathlineto{\pgfqpoint{3.845592in}{2.422044in}}%
\pgfpathlineto{\pgfqpoint{3.862973in}{2.371118in}}%
\pgfpathlineto{\pgfqpoint{3.878473in}{2.320012in}}%
\pgfpathlineto{\pgfqpoint{3.891788in}{2.268725in}}%
\pgfpathlineto{\pgfqpoint{3.902544in}{2.217262in}}%
\pgfpathlineto{\pgfqpoint{3.910266in}{2.165638in}}%
\pgfpathlineto{\pgfqpoint{3.914357in}{2.113895in}}%
\pgfpathlineto{\pgfqpoint{3.914090in}{2.062110in}}%
\pgfpathlineto{\pgfqpoint{3.908601in}{2.010421in}}%
\pgfpathlineto{\pgfqpoint{3.896916in}{1.959051in}}%
\pgfpathlineto{\pgfqpoint{3.878022in}{1.908336in}}%
\pgfpathlineto{\pgfqpoint{3.850918in}{1.858754in}}%
\pgfpathlineto{\pgfqpoint{3.814749in}{1.810945in}}%
\pgfpathlineto{\pgfqpoint{3.768816in}{1.765721in}}%
\pgfusepath{stroke}%
\end{pgfscope}%
\begin{pgfscope}%
\pgfpathrectangle{\pgfqpoint{0.647939in}{0.492442in}}{\pgfqpoint{4.273799in}{2.331163in}}%
\pgfusepath{clip}%
\pgfsetbuttcap%
\pgfsetroundjoin%
\pgfsetlinewidth{0.301125pt}%
\definecolor{currentstroke}{rgb}{0.500000,0.500000,0.500000}%
\pgfsetstrokecolor{currentstroke}%
\pgfsetstrokeopacity{0.300000}%
\pgfsetdash{}{0pt}%
\pgfpathmoveto{\pgfqpoint{3.561893in}{2.823605in}}%
\pgfpathlineto{\pgfqpoint{3.561893in}{2.823605in}}%
\pgfpathlineto{\pgfqpoint{3.589468in}{2.774034in}}%
\pgfpathlineto{\pgfqpoint{3.616014in}{2.724296in}}%
\pgfpathlineto{\pgfqpoint{3.641488in}{2.674392in}}%
\pgfpathlineto{\pgfqpoint{3.665829in}{2.624320in}}%
\pgfpathlineto{\pgfqpoint{3.688966in}{2.574079in}}%
\pgfpathlineto{\pgfqpoint{3.710805in}{2.523665in}}%
\pgfpathlineto{\pgfqpoint{3.731234in}{2.473076in}}%
\pgfpathlineto{\pgfqpoint{3.750113in}{2.422308in}}%
\pgfpathlineto{\pgfqpoint{3.767263in}{2.371359in}}%
\pgfpathlineto{\pgfqpoint{3.782467in}{2.320227in}}%
\pgfpathlineto{\pgfqpoint{3.795448in}{2.268915in}}%
\pgfpathlineto{\pgfqpoint{3.805866in}{2.217430in}}%
\pgfpathlineto{\pgfqpoint{3.813301in}{2.165794in}}%
\pgfpathlineto{\pgfqpoint{3.817225in}{2.114045in}}%
\pgfpathlineto{\pgfqpoint{3.816990in}{2.062259in}}%
\pgfpathlineto{\pgfqpoint{3.811813in}{2.010557in}}%
\pgfpathlineto{\pgfqpoint{3.800768in}{1.959141in}}%
\pgfpathlineto{\pgfqpoint{3.782781in}{1.908324in}}%
\pgfpathlineto{\pgfqpoint{3.756656in}{1.858584in}}%
\pgfpathlineto{\pgfqpoint{3.721126in}{1.810642in}}%
\pgfpathlineto{\pgfqpoint{3.674849in}{1.765553in}}%
\pgfpathlineto{\pgfqpoint{3.616605in}{1.724909in}}%
\pgfpathlineto{\pgfqpoint{3.545460in}{1.691152in}}%
\pgfpathlineto{\pgfqpoint{3.466882in}{1.668572in}}%
\pgfpathlineto{\pgfqpoint{3.389950in}{1.658600in}}%
\pgfpathlineto{\pgfqpoint{3.316478in}{1.659079in}}%
\pgfpathlineto{\pgfqpoint{3.245647in}{1.668363in}}%
\pgfpathlineto{\pgfqpoint{3.176437in}{1.685798in}}%
\pgfpathlineto{\pgfqpoint{3.107686in}{1.711590in}}%
\pgfpathlineto{\pgfqpoint{3.038888in}{1.746534in}}%
\pgfpathlineto{\pgfqpoint{2.977972in}{1.786088in}}%
\pgfusepath{stroke}%
\end{pgfscope}%
\begin{pgfscope}%
\pgfpathrectangle{\pgfqpoint{0.647939in}{0.492442in}}{\pgfqpoint{4.273799in}{2.331163in}}%
\pgfusepath{clip}%
\pgfsetbuttcap%
\pgfsetroundjoin%
\pgfsetlinewidth{0.301125pt}%
\definecolor{currentstroke}{rgb}{0.500000,0.500000,0.500000}%
\pgfsetstrokecolor{currentstroke}%
\pgfsetstrokeopacity{0.300000}%
\pgfsetdash{}{0pt}%
\pgfpathmoveto{\pgfqpoint{3.464761in}{2.823605in}}%
\pgfpathlineto{\pgfqpoint{3.464761in}{2.823605in}}%
\pgfpathlineto{\pgfqpoint{3.493324in}{2.774201in}}%
\pgfpathlineto{\pgfqpoint{3.520752in}{2.724606in}}%
\pgfpathlineto{\pgfqpoint{3.547000in}{2.674822in}}%
\pgfpathlineto{\pgfqpoint{3.572011in}{2.624848in}}%
\pgfpathlineto{\pgfqpoint{3.595717in}{2.574686in}}%
\pgfpathlineto{\pgfqpoint{3.618027in}{2.524334in}}%
\pgfpathlineto{\pgfqpoint{3.638833in}{2.473791in}}%
\pgfpathlineto{\pgfqpoint{3.658000in}{2.423056in}}%
\pgfpathlineto{\pgfqpoint{3.675358in}{2.372127in}}%
\pgfpathlineto{\pgfqpoint{3.690698in}{2.321008in}}%
\pgfpathlineto{\pgfqpoint{3.703762in}{2.269702in}}%
\pgfpathlineto{\pgfqpoint{3.714225in}{2.218220in}}%
\pgfpathlineto{\pgfqpoint{3.721687in}{2.166585in}}%
\pgfpathlineto{\pgfqpoint{3.725649in}{2.114838in}}%
\pgfpathlineto{\pgfqpoint{3.725487in}{2.063050in}}%
\pgfpathlineto{\pgfqpoint{3.720424in}{2.011345in}}%
\pgfusepath{stroke}%
\end{pgfscope}%
\begin{pgfscope}%
\pgfpathrectangle{\pgfqpoint{0.647939in}{0.492442in}}{\pgfqpoint{4.273799in}{2.331163in}}%
\pgfusepath{clip}%
\pgfsetbuttcap%
\pgfsetroundjoin%
\pgfsetlinewidth{0.301125pt}%
\definecolor{currentstroke}{rgb}{0.500000,0.500000,0.500000}%
\pgfsetstrokecolor{currentstroke}%
\pgfsetstrokeopacity{0.300000}%
\pgfsetdash{}{0pt}%
\pgfpathmoveto{\pgfqpoint{3.367630in}{2.823605in}}%
\pgfpathlineto{\pgfqpoint{3.367630in}{2.823605in}}%
\pgfpathlineto{\pgfqpoint{3.397604in}{2.774451in}}%
\pgfpathlineto{\pgfqpoint{3.426328in}{2.725075in}}%
\pgfpathlineto{\pgfqpoint{3.453753in}{2.675480in}}%
\pgfpathlineto{\pgfqpoint{3.479832in}{2.625670in}}%
\pgfpathlineto{\pgfqpoint{3.504500in}{2.575646in}}%
\pgfpathlineto{\pgfqpoint{3.527670in}{2.525410in}}%
\pgfpathlineto{\pgfqpoint{3.549237in}{2.474962in}}%
\pgfpathlineto{\pgfqpoint{3.569070in}{2.424303in}}%
\pgfpathlineto{\pgfqpoint{3.587004in}{2.373435in}}%
\pgfpathlineto{\pgfqpoint{3.602836in}{2.322361in}}%
\pgfusepath{stroke}%
\end{pgfscope}%
\begin{pgfscope}%
\pgfpathrectangle{\pgfqpoint{0.647939in}{0.492442in}}{\pgfqpoint{4.273799in}{2.331163in}}%
\pgfusepath{clip}%
\pgfsetbuttcap%
\pgfsetroundjoin%
\pgfsetlinewidth{0.301125pt}%
\definecolor{currentstroke}{rgb}{0.500000,0.500000,0.500000}%
\pgfsetstrokecolor{currentstroke}%
\pgfsetstrokeopacity{0.300000}%
\pgfsetdash{}{0pt}%
\pgfpathmoveto{\pgfqpoint{3.270498in}{2.823605in}}%
\pgfpathlineto{\pgfqpoint{3.270498in}{2.823605in}}%
\pgfpathlineto{\pgfqpoint{3.302341in}{2.774803in}}%
\pgfpathlineto{\pgfqpoint{3.332795in}{2.725737in}}%
\pgfpathlineto{\pgfqpoint{3.361824in}{2.676415in}}%
\pgfpathlineto{\pgfqpoint{3.389384in}{2.626842in}}%
\pgfpathlineto{\pgfqpoint{3.415411in}{2.577025in}}%
\pgfpathlineto{\pgfqpoint{3.439824in}{2.526964in}}%
\pgfpathlineto{\pgfqpoint{3.462526in}{2.476666in}}%
\pgfpathlineto{\pgfqpoint{3.483386in}{2.426130in}}%
\pgfpathlineto{\pgfqpoint{3.502243in}{2.375362in}}%
\pgfusepath{stroke}%
\end{pgfscope}%
\begin{pgfscope}%
\pgfpathrectangle{\pgfqpoint{0.647939in}{0.492442in}}{\pgfqpoint{4.273799in}{2.331163in}}%
\pgfusepath{clip}%
\pgfsetbuttcap%
\pgfsetroundjoin%
\pgfsetlinewidth{0.301125pt}%
\definecolor{currentstroke}{rgb}{0.500000,0.500000,0.500000}%
\pgfsetstrokecolor{currentstroke}%
\pgfsetstrokeopacity{0.300000}%
\pgfsetdash{}{0pt}%
\pgfpathmoveto{\pgfqpoint{3.076234in}{2.823605in}}%
\pgfpathlineto{\pgfqpoint{3.076234in}{2.823605in}}%
\pgfpathlineto{\pgfqpoint{3.113364in}{2.775929in}}%
\pgfpathlineto{\pgfqpoint{3.148764in}{2.727861in}}%
\pgfpathlineto{\pgfqpoint{3.182415in}{2.679422in}}%
\pgfpathlineto{\pgfqpoint{3.214283in}{2.630626in}}%
\pgfpathlineto{\pgfqpoint{3.244317in}{2.581485in}}%
\pgfpathlineto{\pgfqpoint{3.272449in}{2.532011in}}%
\pgfpathlineto{\pgfqpoint{3.298593in}{2.482213in}}%
\pgfpathlineto{\pgfqpoint{3.322626in}{2.432100in}}%
\pgfpathlineto{\pgfqpoint{3.344397in}{2.381681in}}%
\pgfpathlineto{\pgfqpoint{3.363705in}{2.330965in}}%
\pgfpathlineto{\pgfqpoint{3.380289in}{2.279964in}}%
\pgfpathlineto{\pgfqpoint{3.393809in}{2.228697in}}%
\pgfpathlineto{\pgfqpoint{3.403811in}{2.177194in}}%
\pgfpathlineto{\pgfqpoint{3.409690in}{2.125507in}}%
\pgfpathlineto{\pgfqpoint{3.410610in}{2.073731in}}%
\pgfpathlineto{\pgfqpoint{3.405382in}{2.022041in}}%
\pgfpathlineto{\pgfqpoint{3.392223in}{1.970804in}}%
\pgfpathlineto{\pgfqpoint{3.368342in}{1.920808in}}%
\pgfpathlineto{\pgfqpoint{3.329134in}{1.873999in}}%
\pgfpathlineto{\pgfqpoint{3.329134in}{1.873999in}}%
\pgfpathlineto{\pgfqpoint{3.286293in}{1.844131in}}%
\pgfpathlineto{\pgfqpoint{3.286293in}{1.844131in}}%
\pgfpathlineto{\pgfqpoint{3.239594in}{1.825853in}}%
\pgfpathlineto{\pgfqpoint{3.183843in}{1.817426in}}%
\pgfpathlineto{\pgfqpoint{3.132571in}{1.819584in}}%
\pgfpathlineto{\pgfqpoint{3.082945in}{1.829576in}}%
\pgfpathlineto{\pgfqpoint{3.032475in}{1.847369in}}%
\pgfpathlineto{\pgfqpoint{2.980844in}{1.873822in}}%
\pgfpathlineto{\pgfqpoint{2.929132in}{1.909929in}}%
\pgfpathlineto{\pgfqpoint{2.881590in}{1.954538in}}%
\pgfpathlineto{\pgfqpoint{2.844771in}{2.002104in}}%
\pgfpathlineto{\pgfqpoint{2.820332in}{2.051956in}}%
\pgfpathlineto{\pgfqpoint{2.820332in}{2.051956in}}%
\pgfpathlineto{\pgfqpoint{2.814142in}{2.094566in}}%
\pgfpathlineto{\pgfqpoint{2.814142in}{2.094566in}}%
\pgfpathlineto{\pgfqpoint{2.822257in}{2.118029in}}%
\pgfpathlineto{\pgfqpoint{2.822257in}{2.118029in}}%
\pgfpathlineto{\pgfqpoint{2.837187in}{2.128844in}}%
\pgfpathlineto{\pgfqpoint{2.837187in}{2.128844in}}%
\pgfpathlineto{\pgfqpoint{2.856454in}{2.130143in}}%
\pgfpathlineto{\pgfqpoint{2.873390in}{2.124757in}}%
\pgfpathlineto{\pgfqpoint{2.890257in}{2.112954in}}%
\pgfusepath{stroke}%
\end{pgfscope}%
\begin{pgfscope}%
\pgfpathrectangle{\pgfqpoint{0.647939in}{0.492442in}}{\pgfqpoint{4.273799in}{2.331163in}}%
\pgfusepath{clip}%
\pgfsetbuttcap%
\pgfsetroundjoin%
\pgfsetlinewidth{0.301125pt}%
\definecolor{currentstroke}{rgb}{0.500000,0.500000,0.500000}%
\pgfsetstrokecolor{currentstroke}%
\pgfsetstrokeopacity{0.300000}%
\pgfsetdash{}{0pt}%
\pgfpathmoveto{\pgfqpoint{2.881971in}{2.823605in}}%
\pgfpathlineto{\pgfqpoint{2.881971in}{2.823605in}}%
\pgfpathlineto{\pgfqpoint{2.926824in}{2.777951in}}%
\pgfpathlineto{\pgfqpoint{2.969492in}{2.731678in}}%
\pgfpathlineto{\pgfqpoint{3.009957in}{2.684819in}}%
\pgfpathlineto{\pgfqpoint{3.048199in}{2.637409in}}%
\pgfpathlineto{\pgfqpoint{3.084186in}{2.589476in}}%
\pgfpathlineto{\pgfqpoint{3.117874in}{2.541049in}}%
\pgfpathlineto{\pgfqpoint{3.149191in}{2.492149in}}%
\pgfpathlineto{\pgfqpoint{3.178041in}{2.442800in}}%
\pgfpathlineto{\pgfqpoint{3.204286in}{2.393021in}}%
\pgfpathlineto{\pgfqpoint{3.227732in}{2.342830in}}%
\pgfpathlineto{\pgfqpoint{3.248120in}{2.292244in}}%
\pgfpathlineto{\pgfqpoint{3.265090in}{2.241287in}}%
\pgfpathlineto{\pgfqpoint{3.278142in}{2.189990in}}%
\pgfpathlineto{\pgfqpoint{3.286566in}{2.138410in}}%
\pgfpathlineto{\pgfqpoint{3.289312in}{2.086661in}}%
\pgfpathlineto{\pgfqpoint{3.284746in}{2.034977in}}%
\pgfpathlineto{\pgfqpoint{3.270111in}{1.983915in}}%
\pgfpathlineto{\pgfqpoint{3.240257in}{1.935070in}}%
\pgfpathlineto{\pgfqpoint{3.240257in}{1.935070in}}%
\pgfpathlineto{\pgfqpoint{3.205190in}{1.904529in}}%
\pgfpathlineto{\pgfqpoint{3.205190in}{1.904529in}}%
\pgfpathlineto{\pgfqpoint{3.165743in}{1.886188in}}%
\pgfusepath{stroke}%
\end{pgfscope}%
\begin{pgfscope}%
\pgfpathrectangle{\pgfqpoint{0.647939in}{0.492442in}}{\pgfqpoint{4.273799in}{2.331163in}}%
\pgfusepath{clip}%
\pgfsetbuttcap%
\pgfsetroundjoin%
\pgfsetlinewidth{0.301125pt}%
\definecolor{currentstroke}{rgb}{0.500000,0.500000,0.500000}%
\pgfsetstrokecolor{currentstroke}%
\pgfsetstrokeopacity{0.300000}%
\pgfsetdash{}{0pt}%
\pgfpathmoveto{\pgfqpoint{2.687707in}{2.823605in}}%
\pgfpathlineto{\pgfqpoint{2.687707in}{2.823605in}}%
\pgfpathlineto{\pgfqpoint{2.742737in}{2.781401in}}%
\pgfpathlineto{\pgfqpoint{2.795118in}{2.738206in}}%
\pgfpathlineto{\pgfqpoint{2.844773in}{2.694061in}}%
\pgfpathlineto{\pgfqpoint{2.891651in}{2.649018in}}%
\pgfpathlineto{\pgfqpoint{2.935718in}{2.603138in}}%
\pgfpathlineto{\pgfqpoint{2.976944in}{2.556479in}}%
\pgfpathlineto{\pgfqpoint{3.015291in}{2.509096in}}%
\pgfpathlineto{\pgfqpoint{3.050698in}{2.461039in}}%
\pgfpathlineto{\pgfqpoint{3.083067in}{2.412351in}}%
\pgfpathlineto{\pgfqpoint{3.112244in}{2.363065in}}%
\pgfpathlineto{\pgfqpoint{3.138001in}{2.313216in}}%
\pgfpathlineto{\pgfqpoint{3.160005in}{2.262838in}}%
\pgfpathlineto{\pgfqpoint{3.177760in}{2.211966in}}%
\pgfpathlineto{\pgfqpoint{3.190519in}{2.160657in}}%
\pgfpathlineto{\pgfqpoint{3.197112in}{2.109019in}}%
\pgfpathlineto{\pgfqpoint{3.195564in}{2.057299in}}%
\pgfpathlineto{\pgfqpoint{3.182144in}{2.006194in}}%
\pgfpathlineto{\pgfqpoint{3.182144in}{2.006194in}}%
\pgfpathlineto{\pgfqpoint{3.156584in}{1.966027in}}%
\pgfpathlineto{\pgfqpoint{3.156584in}{1.966027in}}%
\pgfusepath{stroke}%
\end{pgfscope}%
\begin{pgfscope}%
\pgfpathrectangle{\pgfqpoint{0.647939in}{0.492442in}}{\pgfqpoint{4.273799in}{2.331163in}}%
\pgfusepath{clip}%
\pgfsetbuttcap%
\pgfsetroundjoin%
\pgfsetlinewidth{0.301125pt}%
\definecolor{currentstroke}{rgb}{0.500000,0.500000,0.500000}%
\pgfsetstrokecolor{currentstroke}%
\pgfsetstrokeopacity{0.300000}%
\pgfsetdash{}{0pt}%
\pgfpathmoveto{\pgfqpoint{2.493443in}{2.823605in}}%
\pgfpathlineto{\pgfqpoint{2.493443in}{2.823605in}}%
\pgfpathlineto{\pgfqpoint{2.559570in}{2.786443in}}%
\pgfpathlineto{\pgfqpoint{2.623137in}{2.747979in}}%
\pgfpathlineto{\pgfqpoint{2.683757in}{2.708127in}}%
\pgfpathlineto{\pgfqpoint{2.741165in}{2.666888in}}%
\pgfpathlineto{\pgfqpoint{2.795206in}{2.624317in}}%
\pgfpathlineto{\pgfqpoint{2.845809in}{2.580501in}}%
\pgfpathlineto{\pgfqpoint{2.892926in}{2.535540in}}%
\pgfusepath{stroke}%
\end{pgfscope}%
\begin{pgfscope}%
\pgfpathrectangle{\pgfqpoint{0.647939in}{0.492442in}}{\pgfqpoint{4.273799in}{2.331163in}}%
\pgfusepath{clip}%
\pgfsetbuttcap%
\pgfsetroundjoin%
\pgfsetlinewidth{0.301125pt}%
\definecolor{currentstroke}{rgb}{0.500000,0.500000,0.500000}%
\pgfsetstrokecolor{currentstroke}%
\pgfsetstrokeopacity{0.300000}%
\pgfsetdash{}{0pt}%
\pgfpathmoveto{\pgfqpoint{2.396312in}{2.823605in}}%
\pgfpathlineto{\pgfqpoint{2.396312in}{2.823605in}}%
\pgfpathlineto{\pgfqpoint{2.467222in}{2.789159in}}%
\pgfpathlineto{\pgfqpoint{2.536125in}{2.753530in}}%
\pgfpathlineto{\pgfqpoint{2.602339in}{2.716426in}}%
\pgfpathlineto{\pgfqpoint{2.665383in}{2.677718in}}%
\pgfpathlineto{\pgfqpoint{2.724934in}{2.637397in}}%
\pgfusepath{stroke}%
\end{pgfscope}%
\begin{pgfscope}%
\pgfpathrectangle{\pgfqpoint{0.647939in}{0.492442in}}{\pgfqpoint{4.273799in}{2.331163in}}%
\pgfusepath{clip}%
\pgfsetbuttcap%
\pgfsetroundjoin%
\pgfsetlinewidth{0.301125pt}%
\definecolor{currentstroke}{rgb}{0.500000,0.500000,0.500000}%
\pgfsetstrokecolor{currentstroke}%
\pgfsetstrokeopacity{0.300000}%
\pgfsetdash{}{0pt}%
\pgfpathmoveto{\pgfqpoint{2.202048in}{2.823605in}}%
\pgfpathlineto{\pgfqpoint{2.202048in}{2.823605in}}%
\pgfpathlineto{\pgfqpoint{2.278359in}{2.792774in}}%
\pgfpathlineto{\pgfqpoint{2.355017in}{2.762203in}}%
\pgfpathlineto{\pgfqpoint{2.430797in}{2.730999in}}%
\pgfpathlineto{\pgfqpoint{2.504598in}{2.698424in}}%
\pgfpathlineto{\pgfqpoint{2.575459in}{2.663979in}}%
\pgfusepath{stroke}%
\end{pgfscope}%
\begin{pgfscope}%
\pgfpathrectangle{\pgfqpoint{0.647939in}{0.492442in}}{\pgfqpoint{4.273799in}{2.331163in}}%
\pgfusepath{clip}%
\pgfsetbuttcap%
\pgfsetroundjoin%
\pgfsetlinewidth{0.301125pt}%
\definecolor{currentstroke}{rgb}{0.500000,0.500000,0.500000}%
\pgfsetstrokecolor{currentstroke}%
\pgfsetstrokeopacity{0.300000}%
\pgfsetdash{}{0pt}%
\pgfpathmoveto{\pgfqpoint{2.007784in}{2.823605in}}%
\pgfpathlineto{\pgfqpoint{2.007784in}{2.823605in}}%
\pgfpathlineto{\pgfqpoint{2.081146in}{2.790779in}}%
\pgfpathlineto{\pgfqpoint{2.158640in}{2.760886in}}%
\pgfpathlineto{\pgfqpoint{2.238839in}{2.733162in}}%
\pgfpathlineto{\pgfqpoint{2.320087in}{2.706344in}}%
\pgfpathlineto{\pgfqpoint{2.400788in}{2.679057in}}%
\pgfpathlineto{\pgfqpoint{2.479490in}{2.650128in}}%
\pgfpathlineto{\pgfqpoint{2.554987in}{2.618789in}}%
\pgfpathlineto{\pgfqpoint{2.626361in}{2.584697in}}%
\pgfpathlineto{\pgfqpoint{2.692957in}{2.547855in}}%
\pgfpathlineto{\pgfqpoint{2.754490in}{2.508474in}}%
\pgfpathlineto{\pgfqpoint{2.810857in}{2.466845in}}%
\pgfpathlineto{\pgfqpoint{2.862060in}{2.423269in}}%
\pgfpathlineto{\pgfqpoint{2.908115in}{2.378018in}}%
\pgfpathlineto{\pgfqpoint{2.948999in}{2.331309in}}%
\pgfpathlineto{\pgfqpoint{2.984554in}{2.283315in}}%
\pgfpathlineto{\pgfqpoint{3.014393in}{2.234174in}}%
\pgfpathlineto{\pgfqpoint{3.037748in}{2.184012in}}%
\pgfpathlineto{\pgfqpoint{3.053123in}{2.132969in}}%
\pgfpathlineto{\pgfqpoint{3.057291in}{2.081379in}}%
\pgfpathlineto{\pgfqpoint{3.057291in}{2.081379in}}%
\pgfpathlineto{\pgfqpoint{3.046463in}{2.039040in}}%
\pgfpathlineto{\pgfqpoint{3.046463in}{2.039040in}}%
\pgfusepath{stroke}%
\end{pgfscope}%
\begin{pgfscope}%
\pgfpathrectangle{\pgfqpoint{0.647939in}{0.492442in}}{\pgfqpoint{4.273799in}{2.331163in}}%
\pgfusepath{clip}%
\pgfsetbuttcap%
\pgfsetroundjoin%
\pgfsetlinewidth{0.301125pt}%
\definecolor{currentstroke}{rgb}{0.500000,0.500000,0.500000}%
\pgfsetstrokecolor{currentstroke}%
\pgfsetstrokeopacity{0.300000}%
\pgfsetdash{}{0pt}%
\pgfpathmoveto{\pgfqpoint{1.813521in}{2.823605in}}%
\pgfpathlineto{\pgfqpoint{1.813521in}{2.823605in}}%
\pgfpathlineto{\pgfqpoint{1.873808in}{2.783643in}}%
\pgfpathlineto{\pgfqpoint{1.940688in}{2.746976in}}%
\pgfpathlineto{\pgfqpoint{2.014582in}{2.714586in}}%
\pgfpathlineto{\pgfqpoint{2.094759in}{2.686996in}}%
\pgfpathlineto{\pgfqpoint{2.179450in}{2.663669in}}%
\pgfpathlineto{\pgfqpoint{2.266389in}{2.642865in}}%
\pgfpathlineto{\pgfqpoint{2.353484in}{2.622272in}}%
\pgfpathlineto{\pgfqpoint{2.438917in}{2.599782in}}%
\pgfpathlineto{\pgfqpoint{2.521060in}{2.573978in}}%
\pgfusepath{stroke}%
\end{pgfscope}%
\begin{pgfscope}%
\pgfpathrectangle{\pgfqpoint{0.647939in}{0.492442in}}{\pgfqpoint{4.273799in}{2.331163in}}%
\pgfusepath{clip}%
\pgfsetbuttcap%
\pgfsetroundjoin%
\pgfsetlinewidth{0.301125pt}%
\definecolor{currentstroke}{rgb}{0.500000,0.500000,0.500000}%
\pgfsetstrokecolor{currentstroke}%
\pgfsetstrokeopacity{0.300000}%
\pgfsetdash{}{0pt}%
\pgfpathmoveto{\pgfqpoint{1.716389in}{2.823605in}}%
\pgfpathlineto{\pgfqpoint{1.716389in}{2.823605in}}%
\pgfpathlineto{\pgfqpoint{1.767550in}{2.780000in}}%
\pgfpathlineto{\pgfqpoint{1.824465in}{2.738594in}}%
\pgfpathlineto{\pgfqpoint{1.888502in}{2.700471in}}%
\pgfpathlineto{\pgfqpoint{1.960872in}{2.667156in}}%
\pgfpathlineto{\pgfqpoint{2.041498in}{2.640114in}}%
\pgfpathlineto{\pgfqpoint{2.128370in}{2.619501in}}%
\pgfpathlineto{\pgfqpoint{2.218519in}{2.603376in}}%
\pgfpathlineto{\pgfqpoint{2.309493in}{2.588566in}}%
\pgfpathlineto{\pgfqpoint{2.399379in}{2.572020in}}%
\pgfusepath{stroke}%
\end{pgfscope}%
\begin{pgfscope}%
\pgfpathrectangle{\pgfqpoint{0.647939in}{0.492442in}}{\pgfqpoint{4.273799in}{2.331163in}}%
\pgfusepath{clip}%
\pgfsetbuttcap%
\pgfsetroundjoin%
\pgfsetlinewidth{0.301125pt}%
\definecolor{currentstroke}{rgb}{0.500000,0.500000,0.500000}%
\pgfsetstrokecolor{currentstroke}%
\pgfsetstrokeopacity{0.300000}%
\pgfsetdash{}{0pt}%
\pgfpathmoveto{\pgfqpoint{1.619257in}{2.823605in}}%
\pgfpathlineto{\pgfqpoint{1.619257in}{2.823605in}}%
\pgfpathlineto{\pgfqpoint{1.661342in}{2.777189in}}%
\pgfpathlineto{\pgfqpoint{1.707292in}{2.731890in}}%
\pgfpathlineto{\pgfqpoint{1.758380in}{2.688283in}}%
\pgfpathlineto{\pgfqpoint{1.816477in}{2.647405in}}%
\pgfpathlineto{\pgfqpoint{1.883783in}{2.611105in}}%
\pgfpathlineto{\pgfqpoint{1.961920in}{2.582237in}}%
\pgfpathlineto{\pgfqpoint{2.047265in}{2.563491in}}%
\pgfpathlineto{\pgfqpoint{2.134407in}{2.553246in}}%
\pgfpathlineto{\pgfqpoint{2.228518in}{2.546711in}}%
\pgfpathlineto{\pgfqpoint{2.322626in}{2.540009in}}%
\pgfpathlineto{\pgfqpoint{2.415406in}{2.529481in}}%
\pgfpathlineto{\pgfqpoint{2.505013in}{2.512888in}}%
\pgfpathlineto{\pgfqpoint{2.589374in}{2.489521in}}%
\pgfpathlineto{\pgfqpoint{2.666824in}{2.459864in}}%
\pgfpathlineto{\pgfqpoint{2.736575in}{2.424984in}}%
\pgfpathlineto{\pgfqpoint{2.798670in}{2.385963in}}%
\pgfpathlineto{\pgfqpoint{2.853307in}{2.343716in}}%
\pgfusepath{stroke}%
\end{pgfscope}%
\begin{pgfscope}%
\pgfpathrectangle{\pgfqpoint{0.647939in}{0.492442in}}{\pgfqpoint{4.273799in}{2.331163in}}%
\pgfusepath{clip}%
\pgfsetbuttcap%
\pgfsetroundjoin%
\pgfsetlinewidth{0.301125pt}%
\definecolor{currentstroke}{rgb}{0.500000,0.500000,0.500000}%
\pgfsetstrokecolor{currentstroke}%
\pgfsetstrokeopacity{0.300000}%
\pgfsetdash{}{0pt}%
\pgfpathmoveto{\pgfqpoint{1.522125in}{2.823605in}}%
\pgfpathlineto{\pgfqpoint{1.522125in}{2.823605in}}%
\pgfpathlineto{\pgfqpoint{1.556271in}{2.775272in}}%
\pgfpathlineto{\pgfqpoint{1.592525in}{2.727400in}}%
\pgfpathlineto{\pgfqpoint{1.631493in}{2.680175in}}%
\pgfpathlineto{\pgfqpoint{1.674120in}{2.633915in}}%
\pgfpathlineto{\pgfqpoint{1.721941in}{2.589217in}}%
\pgfpathlineto{\pgfqpoint{1.777533in}{2.547341in}}%
\pgfpathlineto{\pgfqpoint{1.844714in}{2.511196in}}%
\pgfpathlineto{\pgfqpoint{1.844714in}{2.511196in}}%
\pgfpathlineto{\pgfqpoint{1.907110in}{2.490785in}}%
\pgfpathlineto{\pgfqpoint{1.977757in}{2.480135in}}%
\pgfpathlineto{\pgfqpoint{2.046560in}{2.478615in}}%
\pgfpathlineto{\pgfqpoint{2.126105in}{2.482864in}}%
\pgfpathlineto{\pgfqpoint{2.220030in}{2.490053in}}%
\pgfpathlineto{\pgfqpoint{2.314426in}{2.494594in}}%
\pgfusepath{stroke}%
\end{pgfscope}%
\begin{pgfscope}%
\pgfpathrectangle{\pgfqpoint{0.647939in}{0.492442in}}{\pgfqpoint{4.273799in}{2.331163in}}%
\pgfusepath{clip}%
\pgfsetbuttcap%
\pgfsetroundjoin%
\pgfsetlinewidth{0.301125pt}%
\definecolor{currentstroke}{rgb}{0.500000,0.500000,0.500000}%
\pgfsetstrokecolor{currentstroke}%
\pgfsetstrokeopacity{0.300000}%
\pgfsetdash{}{0pt}%
\pgfpathmoveto{\pgfqpoint{1.424993in}{2.823605in}}%
\pgfpathlineto{\pgfqpoint{1.424993in}{2.823605in}}%
\pgfpathlineto{\pgfqpoint{1.452598in}{2.774041in}}%
\pgfpathlineto{\pgfqpoint{1.481088in}{2.724630in}}%
\pgfpathlineto{\pgfqpoint{1.510676in}{2.675409in}}%
\pgfpathlineto{\pgfqpoint{1.541613in}{2.626437in}}%
\pgfpathlineto{\pgfqpoint{1.574273in}{2.577808in}}%
\pgfpathlineto{\pgfqpoint{1.609299in}{2.529678in}}%
\pgfpathlineto{\pgfqpoint{1.647781in}{2.482349in}}%
\pgfpathlineto{\pgfqpoint{1.691777in}{2.436508in}}%
\pgfpathlineto{\pgfqpoint{1.745716in}{2.394183in}}%
\pgfpathlineto{\pgfqpoint{1.745716in}{2.394183in}}%
\pgfpathlineto{\pgfqpoint{1.793159in}{2.370604in}}%
\pgfpathlineto{\pgfqpoint{1.793159in}{2.370604in}}%
\pgfpathlineto{\pgfqpoint{1.838381in}{2.359691in}}%
\pgfpathlineto{\pgfqpoint{1.888862in}{2.358697in}}%
\pgfpathlineto{\pgfqpoint{1.935861in}{2.365291in}}%
\pgfpathlineto{\pgfqpoint{1.991414in}{2.378728in}}%
\pgfpathlineto{\pgfqpoint{2.069606in}{2.401760in}}%
\pgfusepath{stroke}%
\end{pgfscope}%
\begin{pgfscope}%
\pgfpathrectangle{\pgfqpoint{0.647939in}{0.492442in}}{\pgfqpoint{4.273799in}{2.331163in}}%
\pgfusepath{clip}%
\pgfsetbuttcap%
\pgfsetroundjoin%
\pgfsetlinewidth{0.301125pt}%
\definecolor{currentstroke}{rgb}{0.500000,0.500000,0.500000}%
\pgfsetstrokecolor{currentstroke}%
\pgfsetstrokeopacity{0.300000}%
\pgfsetdash{}{0pt}%
\pgfpathmoveto{\pgfqpoint{1.327862in}{2.823605in}}%
\pgfpathlineto{\pgfqpoint{1.327862in}{2.823605in}}%
\pgfpathlineto{\pgfqpoint{1.350279in}{2.773268in}}%
\pgfpathlineto{\pgfqpoint{1.372895in}{2.722957in}}%
\pgfpathlineto{\pgfqpoint{1.395714in}{2.672674in}}%
\pgfpathlineto{\pgfqpoint{1.418742in}{2.622420in}}%
\pgfpathlineto{\pgfqpoint{1.441980in}{2.572195in}}%
\pgfpathlineto{\pgfqpoint{1.465439in}{2.522000in}}%
\pgfpathlineto{\pgfqpoint{1.489118in}{2.471837in}}%
\pgfpathlineto{\pgfqpoint{1.513025in}{2.421707in}}%
\pgfpathlineto{\pgfqpoint{1.537159in}{2.371611in}}%
\pgfpathlineto{\pgfqpoint{1.561513in}{2.321552in}}%
\pgfpathlineto{\pgfqpoint{1.586091in}{2.271531in}}%
\pgfpathlineto{\pgfqpoint{1.610856in}{2.221556in}}%
\pgfpathlineto{\pgfqpoint{1.635778in}{2.171617in}}%
\pgfpathlineto{\pgfqpoint{1.660203in}{2.122092in}}%
\pgfpathlineto{\pgfqpoint{1.660203in}{2.122092in}}%
\pgfpathlineto{\pgfqpoint{1.667118in}{2.107180in}}%
\pgfpathlineto{\pgfqpoint{1.667118in}{2.107180in}}%
\pgfpathlineto{\pgfqpoint{1.671389in}{2.095588in}}%
\pgfpathlineto{\pgfqpoint{1.671389in}{2.095588in}}%
\pgfpathlineto{\pgfqpoint{1.671506in}{2.089262in}}%
\pgfpathlineto{\pgfqpoint{1.669485in}{2.084016in}}%
\pgfusepath{stroke}%
\end{pgfscope}%
\begin{pgfscope}%
\pgfpathrectangle{\pgfqpoint{0.647939in}{0.492442in}}{\pgfqpoint{4.273799in}{2.331163in}}%
\pgfusepath{clip}%
\pgfsetbuttcap%
\pgfsetroundjoin%
\pgfsetlinewidth{0.301125pt}%
\definecolor{currentstroke}{rgb}{0.500000,0.500000,0.500000}%
\pgfsetstrokecolor{currentstroke}%
\pgfsetstrokeopacity{0.300000}%
\pgfsetdash{}{0pt}%
\pgfpathmoveto{\pgfqpoint{1.230730in}{2.823605in}}%
\pgfpathlineto{\pgfqpoint{1.230730in}{2.823605in}}%
\pgfpathlineto{\pgfqpoint{1.249092in}{2.772780in}}%
\pgfpathlineto{\pgfqpoint{1.267286in}{2.721938in}}%
\pgfpathlineto{\pgfqpoint{1.285256in}{2.671072in}}%
\pgfpathlineto{\pgfqpoint{1.302944in}{2.620176in}}%
\pgfpathlineto{\pgfqpoint{1.320265in}{2.569243in}}%
\pgfpathlineto{\pgfqpoint{1.337123in}{2.518264in}}%
\pgfpathlineto{\pgfqpoint{1.353388in}{2.467227in}}%
\pgfpathlineto{\pgfqpoint{1.368887in}{2.416121in}}%
\pgfpathlineto{\pgfqpoint{1.383418in}{2.364930in}}%
\pgfpathlineto{\pgfqpoint{1.396696in}{2.313639in}}%
\pgfpathlineto{\pgfqpoint{1.408346in}{2.262231in}}%
\pgfpathlineto{\pgfqpoint{1.417901in}{2.210696in}}%
\pgfpathlineto{\pgfqpoint{1.424754in}{2.159035in}}%
\pgfpathlineto{\pgfqpoint{1.428162in}{2.107277in}}%
\pgfpathlineto{\pgfqpoint{1.427312in}{2.055495in}}%
\pgfpathlineto{\pgfqpoint{1.421481in}{2.003813in}}%
\pgfpathlineto{\pgfqpoint{1.410304in}{1.952398in}}%
\pgfpathlineto{\pgfqpoint{1.393954in}{1.901404in}}%
\pgfpathlineto{\pgfqpoint{1.373110in}{1.850897in}}%
\pgfpathlineto{\pgfqpoint{1.348781in}{1.800859in}}%
\pgfpathlineto{\pgfqpoint{1.321966in}{1.751202in}}%
\pgfpathlineto{\pgfqpoint{1.293477in}{1.701809in}}%
\pgfpathlineto{\pgfqpoint{1.263978in}{1.652590in}}%
\pgfpathlineto{\pgfqpoint{1.233951in}{1.603475in}}%
\pgfpathlineto{\pgfqpoint{1.203673in}{1.554396in}}%
\pgfpathlineto{\pgfqpoint{1.173363in}{1.505316in}}%
\pgfpathlineto{\pgfqpoint{1.143192in}{1.456215in}}%
\pgfusepath{stroke}%
\end{pgfscope}%
\begin{pgfscope}%
\pgfpathrectangle{\pgfqpoint{0.647939in}{0.492442in}}{\pgfqpoint{4.273799in}{2.331163in}}%
\pgfusepath{clip}%
\pgfsetbuttcap%
\pgfsetroundjoin%
\pgfsetlinewidth{0.301125pt}%
\definecolor{currentstroke}{rgb}{0.500000,0.500000,0.500000}%
\pgfsetstrokecolor{currentstroke}%
\pgfsetstrokeopacity{0.300000}%
\pgfsetdash{}{0pt}%
\pgfpathmoveto{\pgfqpoint{1.133598in}{2.823605in}}%
\pgfpathlineto{\pgfqpoint{1.133598in}{2.823605in}}%
\pgfpathlineto{\pgfqpoint{1.148781in}{2.772469in}}%
\pgfpathlineto{\pgfqpoint{1.163620in}{2.721302in}}%
\pgfpathlineto{\pgfqpoint{1.178055in}{2.670102in}}%
\pgfpathlineto{\pgfqpoint{1.192012in}{2.618862in}}%
\pgfpathlineto{\pgfqpoint{1.205415in}{2.567578in}}%
\pgfpathlineto{\pgfqpoint{1.218169in}{2.516244in}}%
\pgfpathlineto{\pgfqpoint{1.230158in}{2.464856in}}%
\pgfpathlineto{\pgfqpoint{1.241254in}{2.413409in}}%
\pgfpathlineto{\pgfqpoint{1.251311in}{2.361899in}}%
\pgfpathlineto{\pgfqpoint{1.260151in}{2.310322in}}%
\pgfpathlineto{\pgfqpoint{1.267572in}{2.258679in}}%
\pgfpathlineto{\pgfqpoint{1.273353in}{2.206974in}}%
\pgfpathlineto{\pgfqpoint{1.277252in}{2.155217in}}%
\pgfpathlineto{\pgfqpoint{1.279019in}{2.103427in}}%
\pgfpathlineto{\pgfqpoint{1.278407in}{2.051630in}}%
\pgfpathlineto{\pgfqpoint{1.275199in}{1.999862in}}%
\pgfpathlineto{\pgfqpoint{1.269239in}{1.948167in}}%
\pgfpathlineto{\pgfqpoint{1.260451in}{1.896594in}}%
\pgfpathlineto{\pgfqpoint{1.248866in}{1.845186in}}%
\pgfpathlineto{\pgfqpoint{1.234635in}{1.793977in}}%
\pgfpathlineto{\pgfqpoint{1.217994in}{1.742986in}}%
\pgfpathlineto{\pgfqpoint{1.199232in}{1.692212in}}%
\pgfpathlineto{\pgfqpoint{1.178686in}{1.641646in}}%
\pgfpathlineto{\pgfqpoint{1.156676in}{1.591263in}}%
\pgfpathlineto{\pgfqpoint{1.133507in}{1.541035in}}%
\pgfusepath{stroke}%
\end{pgfscope}%
\begin{pgfscope}%
\pgfpathrectangle{\pgfqpoint{0.647939in}{0.492442in}}{\pgfqpoint{4.273799in}{2.331163in}}%
\pgfusepath{clip}%
\pgfsetbuttcap%
\pgfsetroundjoin%
\pgfsetlinewidth{0.301125pt}%
\definecolor{currentstroke}{rgb}{0.500000,0.500000,0.500000}%
\pgfsetstrokecolor{currentstroke}%
\pgfsetstrokeopacity{0.300000}%
\pgfsetdash{}{0pt}%
\pgfpathmoveto{\pgfqpoint{1.036466in}{2.823605in}}%
\pgfpathlineto{\pgfqpoint{1.036466in}{2.823605in}}%
\pgfpathlineto{\pgfqpoint{1.049148in}{2.772266in}}%
\pgfpathlineto{\pgfqpoint{1.061409in}{2.720897in}}%
\pgfpathlineto{\pgfqpoint{1.073201in}{2.669495in}}%
\pgfpathlineto{\pgfqpoint{1.084467in}{2.618058in}}%
\pgfpathlineto{\pgfqpoint{1.095139in}{2.566583in}}%
\pgfpathlineto{\pgfqpoint{1.105142in}{2.515068in}}%
\pgfpathlineto{\pgfqpoint{1.114401in}{2.463512in}}%
\pgfpathlineto{\pgfqpoint{1.122827in}{2.411914in}}%
\pgfpathlineto{\pgfqpoint{1.130322in}{2.360273in}}%
\pgfpathlineto{\pgfqpoint{1.136778in}{2.308590in}}%
\pgfpathlineto{\pgfqpoint{1.142084in}{2.256869in}}%
\pgfpathlineto{\pgfqpoint{1.146125in}{2.205114in}}%
\pgfpathlineto{\pgfqpoint{1.148780in}{2.153332in}}%
\pgfpathlineto{\pgfqpoint{1.149932in}{2.101534in}}%
\pgfpathlineto{\pgfqpoint{1.149468in}{2.049734in}}%
\pgfpathlineto{\pgfqpoint{1.147291in}{1.997946in}}%
\pgfpathlineto{\pgfqpoint{1.143322in}{1.946191in}}%
\pgfpathlineto{\pgfqpoint{1.137509in}{1.894488in}}%
\pgfpathlineto{\pgfqpoint{1.129834in}{1.842857in}}%
\pgfpathlineto{\pgfqpoint{1.120312in}{1.791318in}}%
\pgfpathlineto{\pgfqpoint{1.109005in}{1.739886in}}%
\pgfpathlineto{\pgfqpoint{1.096006in}{1.688575in}}%
\pgfpathlineto{\pgfqpoint{1.081424in}{1.637389in}}%
\pgfpathlineto{\pgfqpoint{1.065406in}{1.586332in}}%
\pgfpathlineto{\pgfqpoint{1.048109in}{1.535400in}}%
\pgfpathlineto{\pgfqpoint{1.029683in}{1.484585in}}%
\pgfpathlineto{\pgfqpoint{1.010290in}{1.433878in}}%
\pgfpathlineto{\pgfqpoint{0.990072in}{1.383266in}}%
\pgfpathlineto{\pgfqpoint{0.969172in}{1.332737in}}%
\pgfpathlineto{\pgfqpoint{0.947710in}{1.282277in}}%
\pgfpathlineto{\pgfqpoint{0.925799in}{1.231875in}}%
\pgfpathlineto{\pgfqpoint{0.903535in}{1.181519in}}%
\pgfpathlineto{\pgfqpoint{0.881003in}{1.131199in}}%
\pgfpathlineto{\pgfqpoint{0.858274in}{1.080905in}}%
\pgfpathlineto{\pgfqpoint{0.835408in}{1.030629in}}%
\pgfpathlineto{\pgfqpoint{0.812459in}{0.980365in}}%
\pgfpathlineto{\pgfqpoint{0.789468in}{0.930106in}}%
\pgfpathlineto{\pgfqpoint{0.766475in}{0.879848in}}%
\pgfpathlineto{\pgfqpoint{0.743506in}{0.829586in}}%
\pgfusepath{stroke}%
\end{pgfscope}%
\begin{pgfscope}%
\pgfpathrectangle{\pgfqpoint{0.647939in}{0.492442in}}{\pgfqpoint{4.273799in}{2.331163in}}%
\pgfusepath{clip}%
\pgfsetbuttcap%
\pgfsetroundjoin%
\pgfsetlinewidth{0.301125pt}%
\definecolor{currentstroke}{rgb}{0.500000,0.500000,0.500000}%
\pgfsetstrokecolor{currentstroke}%
\pgfsetstrokeopacity{0.300000}%
\pgfsetdash{}{0pt}%
\pgfpathmoveto{\pgfqpoint{0.939334in}{2.823605in}}%
\pgfpathlineto{\pgfqpoint{0.939334in}{2.823605in}}%
\pgfpathlineto{\pgfqpoint{0.950025in}{2.772131in}}%
\pgfpathlineto{\pgfqpoint{0.960282in}{2.720631in}}%
\pgfpathlineto{\pgfqpoint{0.970060in}{2.669103in}}%
\pgfpathlineto{\pgfqpoint{0.979316in}{2.617547in}}%
\pgfpathlineto{\pgfqpoint{0.988003in}{2.565961in}}%
\pgfpathlineto{\pgfqpoint{0.996069in}{2.514345in}}%
\pgfpathlineto{\pgfqpoint{1.003454in}{2.462699in}}%
\pgfpathlineto{\pgfqpoint{1.010099in}{2.411023in}}%
\pgfpathlineto{\pgfqpoint{1.015942in}{2.359318in}}%
\pgfpathlineto{\pgfqpoint{1.020917in}{2.307586in}}%
\pgfpathlineto{\pgfqpoint{1.024958in}{2.255830in}}%
\pgfpathlineto{\pgfqpoint{1.027996in}{2.204054in}}%
\pgfpathlineto{\pgfqpoint{1.029964in}{2.152263in}}%
\pgfpathlineto{\pgfqpoint{1.030797in}{2.100462in}}%
\pgfpathlineto{\pgfqpoint{1.030433in}{2.048660in}}%
\pgfpathlineto{\pgfqpoint{1.028820in}{1.996865in}}%
\pgfpathlineto{\pgfqpoint{1.025913in}{1.945088in}}%
\pgfpathlineto{\pgfqpoint{1.021679in}{1.893337in}}%
\pgfpathlineto{\pgfqpoint{1.016100in}{1.841625in}}%
\pgfpathlineto{\pgfqpoint{1.009172in}{1.789961in}}%
\pgfpathlineto{\pgfqpoint{1.000906in}{1.738356in}}%
\pgfpathlineto{\pgfqpoint{0.991329in}{1.686818in}}%
\pgfpathlineto{\pgfqpoint{0.980490in}{1.635355in}}%
\pgfpathlineto{\pgfqpoint{0.968451in}{1.583972in}}%
\pgfpathlineto{\pgfqpoint{0.955277in}{1.532672in}}%
\pgfpathlineto{\pgfqpoint{0.941048in}{1.481455in}}%
\pgfpathlineto{\pgfqpoint{0.925855in}{1.430321in}}%
\pgfpathlineto{\pgfqpoint{0.909782in}{1.379267in}}%
\pgfusepath{stroke}%
\end{pgfscope}%
\begin{pgfscope}%
\pgfpathrectangle{\pgfqpoint{0.647939in}{0.492442in}}{\pgfqpoint{4.273799in}{2.331163in}}%
\pgfusepath{clip}%
\pgfsetbuttcap%
\pgfsetroundjoin%
\pgfsetlinewidth{0.301125pt}%
\definecolor{currentstroke}{rgb}{0.500000,0.500000,0.500000}%
\pgfsetstrokecolor{currentstroke}%
\pgfsetstrokeopacity{0.300000}%
\pgfsetdash{}{0pt}%
\pgfpathmoveto{\pgfqpoint{0.842203in}{2.823605in}}%
\pgfpathlineto{\pgfqpoint{0.842203in}{2.823605in}}%
\pgfpathlineto{\pgfqpoint{0.851293in}{2.772040in}}%
\pgfpathlineto{\pgfqpoint{0.859962in}{2.720453in}}%
\pgfpathlineto{\pgfqpoint{0.868176in}{2.668844in}}%
\pgfpathlineto{\pgfqpoint{0.875900in}{2.617212in}}%
\pgfpathlineto{\pgfqpoint{0.883097in}{2.565558in}}%
\pgfpathlineto{\pgfqpoint{0.889730in}{2.513881in}}%
\pgfpathlineto{\pgfqpoint{0.895761in}{2.462183in}}%
\pgfpathlineto{\pgfqpoint{0.901148in}{2.410463in}}%
\pgfpathlineto{\pgfqpoint{0.905850in}{2.358723in}}%
\pgfpathlineto{\pgfqpoint{0.909824in}{2.306965in}}%
\pgfpathlineto{\pgfqpoint{0.913026in}{2.255192in}}%
\pgfpathlineto{\pgfqpoint{0.915414in}{2.203405in}}%
\pgfpathlineto{\pgfqpoint{0.916947in}{2.151609in}}%
\pgfpathlineto{\pgfqpoint{0.917586in}{2.099807in}}%
\pgfpathlineto{\pgfqpoint{0.917293in}{2.048004in}}%
\pgfpathlineto{\pgfqpoint{0.916038in}{1.996206in}}%
\pgfpathlineto{\pgfqpoint{0.913792in}{1.944418in}}%
\pgfpathlineto{\pgfqpoint{0.910532in}{1.892646in}}%
\pgfpathlineto{\pgfqpoint{0.906245in}{1.840896in}}%
\pgfpathlineto{\pgfqpoint{0.900924in}{1.789174in}}%
\pgfpathlineto{\pgfqpoint{0.894569in}{1.737488in}}%
\pgfpathlineto{\pgfqpoint{0.887192in}{1.685842in}}%
\pgfpathlineto{\pgfqpoint{0.878808in}{1.634242in}}%
\pgfpathlineto{\pgfqpoint{0.869442in}{1.582692in}}%
\pgfpathlineto{\pgfqpoint{0.859127in}{1.531196in}}%
\pgfpathlineto{\pgfqpoint{0.847906in}{1.479757in}}%
\pgfpathlineto{\pgfqpoint{0.835825in}{1.428376in}}%
\pgfpathlineto{\pgfqpoint{0.822929in}{1.377053in}}%
\pgfpathlineto{\pgfqpoint{0.809275in}{1.325789in}}%
\pgfpathlineto{\pgfqpoint{0.794921in}{1.274583in}}%
\pgfpathlineto{\pgfqpoint{0.779917in}{1.223431in}}%
\pgfpathlineto{\pgfqpoint{0.764324in}{1.172332in}}%
\pgfpathlineto{\pgfqpoint{0.748198in}{1.121282in}}%
\pgfpathlineto{\pgfqpoint{0.731589in}{1.070278in}}%
\pgfpathlineto{\pgfqpoint{0.714550in}{1.019317in}}%
\pgfpathlineto{\pgfqpoint{0.697131in}{0.968394in}}%
\pgfpathlineto{\pgfqpoint{0.679376in}{0.917505in}}%
\pgfpathlineto{\pgfqpoint{0.661330in}{0.866647in}}%
\pgfpathlineto{\pgfqpoint{0.647939in}{0.829196in}}%
\pgfusepath{stroke}%
\end{pgfscope}%
\begin{pgfscope}%
\pgfpathrectangle{\pgfqpoint{0.647939in}{0.492442in}}{\pgfqpoint{4.273799in}{2.331163in}}%
\pgfusepath{clip}%
\pgfsetbuttcap%
\pgfsetroundjoin%
\pgfsetlinewidth{0.301125pt}%
\definecolor{currentstroke}{rgb}{0.500000,0.500000,0.500000}%
\pgfsetstrokecolor{currentstroke}%
\pgfsetstrokeopacity{0.300000}%
\pgfsetdash{}{0pt}%
\pgfpathmoveto{\pgfqpoint{0.745071in}{2.823605in}}%
\pgfpathlineto{\pgfqpoint{0.745071in}{2.823605in}}%
\pgfpathlineto{\pgfqpoint{0.752866in}{2.771977in}}%
\pgfpathlineto{\pgfqpoint{0.760263in}{2.720331in}}%
\pgfpathlineto{\pgfqpoint{0.767237in}{2.668667in}}%
\pgfpathlineto{\pgfqpoint{0.773763in}{2.616986in}}%
\pgfpathlineto{\pgfqpoint{0.779814in}{2.565288in}}%
\pgfpathlineto{\pgfqpoint{0.785362in}{2.513573in}}%
\pgfpathlineto{\pgfqpoint{0.790379in}{2.461842in}}%
\pgfpathlineto{\pgfqpoint{0.794836in}{2.410096in}}%
\pgfpathlineto{\pgfqpoint{0.798705in}{2.358336in}}%
\pgfpathlineto{\pgfqpoint{0.801957in}{2.306563in}}%
\pgfpathlineto{\pgfqpoint{0.804564in}{2.254779in}}%
\pgfpathlineto{\pgfqpoint{0.806498in}{2.202987in}}%
\pgfpathlineto{\pgfqpoint{0.807732in}{2.151188in}}%
\pgfpathlineto{\pgfqpoint{0.808240in}{2.099386in}}%
\pgfpathlineto{\pgfqpoint{0.808001in}{2.047583in}}%
\pgfpathlineto{\pgfqpoint{0.806991in}{1.995782in}}%
\pgfpathlineto{\pgfqpoint{0.805194in}{1.943989in}}%
\pgfpathlineto{\pgfqpoint{0.802594in}{1.892205in}}%
\pgfpathlineto{\pgfqpoint{0.799181in}{1.840435in}}%
\pgfpathlineto{\pgfqpoint{0.794947in}{1.788684in}}%
\pgfpathlineto{\pgfqpoint{0.789890in}{1.736954in}}%
\pgfpathlineto{\pgfqpoint{0.784012in}{1.685251in}}%
\pgfpathlineto{\pgfqpoint{0.777319in}{1.633577in}}%
\pgfpathlineto{\pgfqpoint{0.769823in}{1.581935in}}%
\pgfpathlineto{\pgfqpoint{0.761544in}{1.530330in}}%
\pgfpathlineto{\pgfqpoint{0.752500in}{1.478763in}}%
\pgfpathlineto{\pgfqpoint{0.742715in}{1.427236in}}%
\pgfpathlineto{\pgfqpoint{0.732215in}{1.375750in}}%
\pgfpathlineto{\pgfqpoint{0.721034in}{1.324308in}}%
\pgfpathlineto{\pgfqpoint{0.709208in}{1.272908in}}%
\pgfusepath{stroke}%
\end{pgfscope}%
\begin{pgfscope}%
\pgfpathrectangle{\pgfqpoint{0.647939in}{0.492442in}}{\pgfqpoint{4.273799in}{2.331163in}}%
\pgfusepath{clip}%
\pgfsetbuttcap%
\pgfsetroundjoin%
\pgfsetlinewidth{0.301125pt}%
\definecolor{currentstroke}{rgb}{0.500000,0.500000,0.500000}%
\pgfsetstrokecolor{currentstroke}%
\pgfsetstrokeopacity{0.300000}%
\pgfsetdash{}{0pt}%
\pgfpathmoveto{\pgfqpoint{0.647939in}{2.823605in}}%
\pgfpathlineto{\pgfqpoint{0.647939in}{2.823605in}}%
\pgfpathlineto{\pgfqpoint{0.654676in}{2.771932in}}%
\pgfpathlineto{\pgfqpoint{0.661043in}{2.720246in}}%
\pgfpathlineto{\pgfqpoint{0.667023in}{2.668545in}}%
\pgfpathlineto{\pgfqpoint{0.672595in}{2.616831in}}%
\pgfpathlineto{\pgfqpoint{0.677742in}{2.565103in}}%
\pgfpathlineto{\pgfqpoint{0.682443in}{2.513364in}}%
\pgfpathlineto{\pgfqpoint{0.686678in}{2.461612in}}%
\pgfpathlineto{\pgfqpoint{0.690426in}{2.409849in}}%
\pgfpathlineto{\pgfqpoint{0.693667in}{2.358075in}}%
\pgfpathlineto{\pgfqpoint{0.696381in}{2.306293in}}%
\pgfpathlineto{\pgfqpoint{0.698549in}{2.254503in}}%
\pgfpathlineto{\pgfqpoint{0.700150in}{2.202707in}}%
\pgfpathlineto{\pgfqpoint{0.701167in}{2.150907in}}%
\pgfpathlineto{\pgfqpoint{0.701584in}{2.099104in}}%
\pgfpathlineto{\pgfqpoint{0.701384in}{2.047301in}}%
\pgfpathlineto{\pgfqpoint{0.700552in}{1.995500in}}%
\pgfpathlineto{\pgfqpoint{0.699078in}{1.943703in}}%
\pgfpathlineto{\pgfqpoint{0.696949in}{1.891912in}}%
\pgfpathlineto{\pgfqpoint{0.694158in}{1.840131in}}%
\pgfpathlineto{\pgfqpoint{0.690699in}{1.788363in}}%
\pgfpathlineto{\pgfqpoint{0.686568in}{1.736608in}}%
\pgfpathlineto{\pgfqpoint{0.681767in}{1.684871in}}%
\pgfpathlineto{\pgfqpoint{0.676298in}{1.633154in}}%
\pgfpathlineto{\pgfqpoint{0.670165in}{1.581459in}}%
\pgfpathlineto{\pgfqpoint{0.663378in}{1.529789in}}%
\pgfpathlineto{\pgfqpoint{0.655945in}{1.478145in}}%
\pgfusepath{stroke}%
\end{pgfscope}%
\begin{pgfscope}%
\pgfpathrectangle{\pgfqpoint{0.647939in}{0.492442in}}{\pgfqpoint{4.273799in}{2.331163in}}%
\pgfusepath{clip}%
\pgfsetbuttcap%
\pgfsetroundjoin%
\pgfsetlinewidth{0.301125pt}%
\definecolor{currentstroke}{rgb}{0.500000,0.500000,0.500000}%
\pgfsetstrokecolor{currentstroke}%
\pgfsetstrokeopacity{0.300000}%
\pgfsetdash{}{0pt}%
\pgfpathmoveto{\pgfqpoint{0.647939in}{2.240815in}}%
\pgfpathlineto{\pgfqpoint{0.647939in}{2.240815in}}%
\pgfpathlineto{\pgfqpoint{0.649270in}{2.189016in}}%
\pgfpathlineto{\pgfqpoint{0.650059in}{2.137215in}}%
\pgfpathlineto{\pgfqpoint{0.650292in}{2.085411in}}%
\pgfpathlineto{\pgfqpoint{0.649958in}{2.033608in}}%
\pgfpathlineto{\pgfqpoint{0.649043in}{1.981807in}}%
\pgfpathlineto{\pgfqpoint{0.647939in}{1.934403in}}%
\pgfusepath{stroke}%
\end{pgfscope}%
\begin{pgfscope}%
\pgfpathrectangle{\pgfqpoint{0.647939in}{0.492442in}}{\pgfqpoint{4.273799in}{2.331163in}}%
\pgfusepath{clip}%
\pgfsetbuttcap%
\pgfsetroundjoin%
\pgfsetlinewidth{0.301125pt}%
\definecolor{currentstroke}{rgb}{0.500000,0.500000,0.500000}%
\pgfsetstrokecolor{currentstroke}%
\pgfsetstrokeopacity{0.300000}%
\pgfsetdash{}{0pt}%
\pgfpathmoveto{\pgfqpoint{4.357985in}{0.492442in}}%
\pgfpathlineto{\pgfqpoint{4.340042in}{0.509764in}}%
\pgfpathlineto{\pgfqpoint{4.292166in}{0.554492in}}%
\pgfpathlineto{\pgfqpoint{4.241816in}{0.598404in}}%
\pgfpathlineto{\pgfqpoint{4.188798in}{0.641373in}}%
\pgfpathlineto{\pgfqpoint{4.132980in}{0.683273in}}%
\pgfpathlineto{\pgfqpoint{4.074299in}{0.723989in}}%
\pgfpathlineto{\pgfqpoint{4.012742in}{0.763422in}}%
\pgfpathlineto{\pgfqpoint{3.948465in}{0.801546in}}%
\pgfpathlineto{\pgfqpoint{3.881755in}{0.838406in}}%
\pgfpathlineto{\pgfqpoint{3.813041in}{0.874160in}}%
\pgfpathlineto{\pgfqpoint{3.742879in}{0.909070in}}%
\pgfpathlineto{\pgfqpoint{3.671877in}{0.943474in}}%
\pgfpathlineto{\pgfqpoint{3.600696in}{0.977767in}}%
\pgfusepath{stroke}%
\end{pgfscope}%
\begin{pgfscope}%
\pgfpathrectangle{\pgfqpoint{0.647939in}{0.492442in}}{\pgfqpoint{4.273799in}{2.331163in}}%
\pgfusepath{clip}%
\pgfsetbuttcap%
\pgfsetroundjoin%
\pgfsetlinewidth{0.301125pt}%
\definecolor{currentstroke}{rgb}{0.500000,0.500000,0.500000}%
\pgfsetstrokecolor{currentstroke}%
\pgfsetstrokeopacity{0.300000}%
\pgfsetdash{}{0pt}%
\pgfpathmoveto{\pgfqpoint{4.630343in}{1.869948in}}%
\pgfpathlineto{\pgfqpoint{4.619558in}{1.921412in}}%
\pgfpathlineto{\pgfqpoint{4.610826in}{1.972992in}}%
\pgfpathlineto{\pgfqpoint{4.604713in}{2.024680in}}%
\pgfpathlineto{\pgfqpoint{4.601871in}{2.076450in}}%
\pgfpathlineto{\pgfqpoint{4.602991in}{2.128234in}}%
\pgfpathlineto{\pgfqpoint{4.608657in}{2.179923in}}%
\pgfpathlineto{\pgfqpoint{4.619157in}{2.231385in}}%
\pgfusepath{stroke}%
\end{pgfscope}%
\begin{pgfscope}%
\pgfpathrectangle{\pgfqpoint{0.647939in}{0.492442in}}{\pgfqpoint{4.273799in}{2.331163in}}%
\pgfusepath{clip}%
\pgfsetbuttcap%
\pgfsetroundjoin%
\pgfsetlinewidth{0.301125pt}%
\definecolor{currentstroke}{rgb}{0.500000,0.500000,0.500000}%
\pgfsetstrokecolor{currentstroke}%
\pgfsetstrokeopacity{0.300000}%
\pgfsetdash{}{0pt}%
\pgfpathmoveto{\pgfqpoint{2.007784in}{0.704366in}}%
\pgfpathlineto{\pgfqpoint{1.970286in}{0.751961in}}%
\pgfpathlineto{\pgfqpoint{1.932540in}{0.799497in}}%
\pgfpathlineto{\pgfqpoint{1.894422in}{0.846944in}}%
\pgfpathlineto{\pgfqpoint{1.855778in}{0.894265in}}%
\pgfpathlineto{\pgfqpoint{1.816402in}{0.941406in}}%
\pgfpathlineto{\pgfqpoint{1.776025in}{0.988292in}}%
\pgfusepath{stroke}%
\end{pgfscope}%
\begin{pgfscope}%
\pgfpathrectangle{\pgfqpoint{0.647939in}{0.492442in}}{\pgfqpoint{4.273799in}{2.331163in}}%
\pgfusepath{clip}%
\pgfsetbuttcap%
\pgfsetroundjoin%
\pgfsetlinewidth{0.301125pt}%
\definecolor{currentstroke}{rgb}{0.500000,0.500000,0.500000}%
\pgfsetstrokecolor{currentstroke}%
\pgfsetstrokeopacity{0.300000}%
\pgfsetdash{}{0pt}%
\pgfpathmoveto{\pgfqpoint{3.980755in}{0.627577in}}%
\pgfpathlineto{\pgfqpoint{3.917930in}{0.666419in}}%
\pgfpathlineto{\pgfqpoint{3.853289in}{0.704366in}}%
\pgfpathlineto{\pgfqpoint{3.787206in}{0.741570in}}%
\pgfpathlineto{\pgfqpoint{3.720131in}{0.778243in}}%
\pgfpathlineto{\pgfqpoint{3.652557in}{0.814644in}}%
\pgfpathlineto{\pgfqpoint{3.584976in}{0.851041in}}%
\pgfpathlineto{\pgfqpoint{3.517875in}{0.887699in}}%
\pgfpathlineto{\pgfqpoint{3.451693in}{0.924848in}}%
\pgfusepath{stroke}%
\end{pgfscope}%
\begin{pgfscope}%
\pgfpathrectangle{\pgfqpoint{0.647939in}{0.492442in}}{\pgfqpoint{4.273799in}{2.331163in}}%
\pgfusepath{clip}%
\pgfsetbuttcap%
\pgfsetroundjoin%
\pgfsetlinewidth{0.301125pt}%
\definecolor{currentstroke}{rgb}{0.500000,0.500000,0.500000}%
\pgfsetstrokecolor{currentstroke}%
\pgfsetstrokeopacity{0.300000}%
\pgfsetdash{}{0pt}%
\pgfpathmoveto{\pgfqpoint{4.533211in}{1.075233in}}%
\pgfpathlineto{\pgfqpoint{4.498400in}{1.123426in}}%
\pgfpathlineto{\pgfqpoint{4.461597in}{1.171171in}}%
\pgfpathlineto{\pgfqpoint{4.422339in}{1.218325in}}%
\pgfpathlineto{\pgfqpoint{4.379996in}{1.264670in}}%
\pgfpathlineto{\pgfqpoint{4.333681in}{1.309858in}}%
\pgfpathlineto{\pgfqpoint{4.282124in}{1.353301in}}%
\pgfpathlineto{\pgfqpoint{4.223518in}{1.393959in}}%
\pgfpathlineto{\pgfqpoint{4.155737in}{1.429986in}}%
\pgfpathlineto{\pgfqpoint{4.077147in}{1.458477in}}%
\pgfpathlineto{\pgfqpoint{3.991352in}{1.476372in}}%
\pgfpathlineto{\pgfqpoint{3.905856in}{1.484502in}}%
\pgfpathlineto{\pgfqpoint{3.811168in}{1.487046in}}%
\pgfpathlineto{\pgfqpoint{3.716245in}{1.487269in}}%
\pgfpathlineto{\pgfqpoint{3.621413in}{1.488885in}}%
\pgfpathlineto{\pgfqpoint{3.527254in}{1.494660in}}%
\pgfpathlineto{\pgfqpoint{3.435054in}{1.506335in}}%
\pgfusepath{stroke}%
\end{pgfscope}%
\begin{pgfscope}%
\pgfpathrectangle{\pgfqpoint{0.647939in}{0.492442in}}{\pgfqpoint{4.273799in}{2.331163in}}%
\pgfusepath{clip}%
\pgfsetbuttcap%
\pgfsetroundjoin%
\pgfsetlinewidth{0.301125pt}%
\definecolor{currentstroke}{rgb}{0.500000,0.500000,0.500000}%
\pgfsetstrokecolor{currentstroke}%
\pgfsetstrokeopacity{0.300000}%
\pgfsetdash{}{0pt}%
\pgfpathmoveto{\pgfqpoint{4.533211in}{1.340138in}}%
\pgfpathlineto{\pgfqpoint{4.501338in}{1.388931in}}%
\pgfpathlineto{\pgfqpoint{4.467542in}{1.437329in}}%
\pgfpathlineto{\pgfqpoint{4.431196in}{1.485167in}}%
\pgfpathlineto{\pgfqpoint{4.391322in}{1.532148in}}%
\pgfpathlineto{\pgfqpoint{4.346267in}{1.577682in}}%
\pgfpathlineto{\pgfqpoint{4.293019in}{1.620379in}}%
\pgfpathlineto{\pgfqpoint{4.225864in}{1.656339in}}%
\pgfpathlineto{\pgfqpoint{4.225864in}{1.656339in}}%
\pgfpathlineto{\pgfqpoint{4.172247in}{1.671548in}}%
\pgfpathlineto{\pgfqpoint{4.111631in}{1.676448in}}%
\pgfpathlineto{\pgfqpoint{4.055608in}{1.672171in}}%
\pgfpathlineto{\pgfqpoint{3.994003in}{1.660973in}}%
\pgfpathlineto{\pgfqpoint{3.914432in}{1.641250in}}%
\pgfusepath{stroke}%
\end{pgfscope}%
\begin{pgfscope}%
\pgfpathrectangle{\pgfqpoint{0.647939in}{0.492442in}}{\pgfqpoint{4.273799in}{2.331163in}}%
\pgfusepath{clip}%
\pgfsetbuttcap%
\pgfsetroundjoin%
\pgfsetlinewidth{0.301125pt}%
\definecolor{currentstroke}{rgb}{0.500000,0.500000,0.500000}%
\pgfsetstrokecolor{currentstroke}%
\pgfsetstrokeopacity{0.300000}%
\pgfsetdash{}{0pt}%
\pgfpathmoveto{\pgfqpoint{4.556481in}{1.660784in}}%
\pgfpathlineto{\pgfqpoint{4.533211in}{1.711005in}}%
\pgfpathlineto{\pgfqpoint{4.509588in}{1.761174in}}%
\pgfpathlineto{\pgfqpoint{4.485541in}{1.811279in}}%
\pgfpathlineto{\pgfqpoint{4.460932in}{1.861298in}}%
\pgfpathlineto{\pgfqpoint{4.435511in}{1.911178in}}%
\pgfpathlineto{\pgfqpoint{4.408564in}{1.960806in}}%
\pgfpathlineto{\pgfqpoint{4.377511in}{2.009393in}}%
\pgfpathlineto{\pgfqpoint{4.377511in}{2.009393in}}%
\pgfpathlineto{\pgfqpoint{4.361267in}{2.027070in}}%
\pgfpathlineto{\pgfqpoint{4.361267in}{2.027070in}}%
\pgfpathlineto{\pgfqpoint{4.346492in}{2.034499in}}%
\pgfpathlineto{\pgfqpoint{4.331764in}{2.033445in}}%
\pgfpathlineto{\pgfqpoint{4.315710in}{2.025414in}}%
\pgfpathlineto{\pgfqpoint{4.297117in}{2.011892in}}%
\pgfusepath{stroke}%
\end{pgfscope}%
\begin{pgfscope}%
\pgfpathrectangle{\pgfqpoint{0.647939in}{0.492442in}}{\pgfqpoint{4.273799in}{2.331163in}}%
\pgfusepath{clip}%
\pgfsetbuttcap%
\pgfsetroundjoin%
\pgfsetlinewidth{0.301125pt}%
\definecolor{currentstroke}{rgb}{0.500000,0.500000,0.500000}%
\pgfsetstrokecolor{currentstroke}%
\pgfsetstrokeopacity{0.300000}%
\pgfsetdash{}{0pt}%
\pgfpathmoveto{\pgfqpoint{2.590575in}{0.757347in}}%
\pgfpathlineto{\pgfqpoint{2.551734in}{0.804620in}}%
\pgfpathlineto{\pgfqpoint{2.513658in}{0.852077in}}%
\pgfpathlineto{\pgfqpoint{2.476324in}{0.899710in}}%
\pgfpathlineto{\pgfqpoint{2.439711in}{0.947508in}}%
\pgfpathlineto{\pgfqpoint{2.403797in}{0.995465in}}%
\pgfpathlineto{\pgfqpoint{2.368567in}{1.043572in}}%
\pgfpathlineto{\pgfqpoint{2.334002in}{1.091822in}}%
\pgfpathlineto{\pgfqpoint{2.300080in}{1.140208in}}%
\pgfpathlineto{\pgfqpoint{2.266784in}{1.188724in}}%
\pgfpathlineto{\pgfqpoint{2.234105in}{1.237364in}}%
\pgfpathlineto{\pgfqpoint{2.202038in}{1.286125in}}%
\pgfpathlineto{\pgfqpoint{2.170570in}{1.335002in}}%
\pgfpathlineto{\pgfqpoint{2.139689in}{1.383990in}}%
\pgfpathlineto{\pgfqpoint{2.109404in}{1.433089in}}%
\pgfpathlineto{\pgfqpoint{2.079721in}{1.482297in}}%
\pgfpathlineto{\pgfqpoint{2.050643in}{1.531613in}}%
\pgfpathlineto{\pgfqpoint{2.022195in}{1.581037in}}%
\pgfpathlineto{\pgfqpoint{1.994413in}{1.630574in}}%
\pgfpathlineto{\pgfqpoint{1.967339in}{1.680227in}}%
\pgfpathlineto{\pgfqpoint{1.941059in}{1.730006in}}%
\pgfpathlineto{\pgfqpoint{1.915683in}{1.779924in}}%
\pgfpathlineto{\pgfqpoint{1.891383in}{1.830000in}}%
\pgfpathlineto{\pgfqpoint{1.868456in}{1.880268in}}%
\pgfpathlineto{\pgfqpoint{1.847383in}{1.930773in}}%
\pgfpathlineto{\pgfqpoint{1.829033in}{1.981588in}}%
\pgfpathlineto{\pgfqpoint{1.815075in}{2.032799in}}%
\pgfpathlineto{\pgfqpoint{1.808807in}{2.084405in}}%
\pgfpathlineto{\pgfqpoint{1.816008in}{2.135804in}}%
\pgfpathlineto{\pgfqpoint{1.841451in}{2.185237in}}%
\pgfpathlineto{\pgfqpoint{1.877728in}{2.227325in}}%
\pgfpathlineto{\pgfqpoint{1.928236in}{2.270772in}}%
\pgfusepath{stroke}%
\end{pgfscope}%
\begin{pgfscope}%
\pgfpathrectangle{\pgfqpoint{0.647939in}{0.492442in}}{\pgfqpoint{4.273799in}{2.331163in}}%
\pgfusepath{clip}%
\pgfsetbuttcap%
\pgfsetroundjoin%
\pgfsetlinewidth{0.301125pt}%
\definecolor{currentstroke}{rgb}{0.500000,0.500000,0.500000}%
\pgfsetstrokecolor{currentstroke}%
\pgfsetstrokeopacity{0.300000}%
\pgfsetdash{}{0pt}%
\pgfpathmoveto{\pgfqpoint{3.464761in}{0.757347in}}%
\pgfpathlineto{\pgfqpoint{3.402514in}{0.796468in}}%
\pgfpathlineto{\pgfqpoint{3.341519in}{0.836171in}}%
\pgfpathlineto{\pgfqpoint{3.281972in}{0.876522in}}%
\pgfpathlineto{\pgfqpoint{3.224029in}{0.917561in}}%
\pgfpathlineto{\pgfqpoint{3.167803in}{0.959303in}}%
\pgfpathlineto{\pgfqpoint{3.113354in}{1.001741in}}%
\pgfpathlineto{\pgfqpoint{3.060720in}{1.044855in}}%
\pgfusepath{stroke}%
\end{pgfscope}%
\begin{pgfscope}%
\pgfpathrectangle{\pgfqpoint{0.647939in}{0.492442in}}{\pgfqpoint{4.273799in}{2.331163in}}%
\pgfusepath{clip}%
\pgfsetbuttcap%
\pgfsetroundjoin%
\pgfsetlinewidth{0.301125pt}%
\definecolor{currentstroke}{rgb}{0.500000,0.500000,0.500000}%
\pgfsetstrokecolor{currentstroke}%
\pgfsetstrokeopacity{0.300000}%
\pgfsetdash{}{0pt}%
\pgfpathmoveto{\pgfqpoint{1.522125in}{2.558700in}}%
\pgfpathlineto{\pgfqpoint{1.551498in}{2.509444in}}%
\pgfpathlineto{\pgfqpoint{1.582487in}{2.460490in}}%
\pgfpathlineto{\pgfqpoint{1.615813in}{2.412012in}}%
\pgfpathlineto{\pgfqpoint{1.652980in}{2.364401in}}%
\pgfpathlineto{\pgfqpoint{1.697642in}{2.318882in}}%
\pgfpathlineto{\pgfqpoint{1.697642in}{2.318882in}}%
\pgfpathlineto{\pgfqpoint{1.739730in}{2.290557in}}%
\pgfpathlineto{\pgfqpoint{1.739730in}{2.290557in}}%
\pgfpathlineto{\pgfqpoint{1.775447in}{2.278800in}}%
\pgfpathlineto{\pgfqpoint{1.775447in}{2.278800in}}%
\pgfpathlineto{\pgfqpoint{1.811314in}{2.277422in}}%
\pgfpathlineto{\pgfqpoint{1.846025in}{2.283538in}}%
\pgfusepath{stroke}%
\end{pgfscope}%
\begin{pgfscope}%
\pgfpathrectangle{\pgfqpoint{0.647939in}{0.492442in}}{\pgfqpoint{4.273799in}{2.331163in}}%
\pgfusepath{clip}%
\pgfsetbuttcap%
\pgfsetroundjoin%
\pgfsetlinewidth{0.301125pt}%
\definecolor{currentstroke}{rgb}{0.500000,0.500000,0.500000}%
\pgfsetstrokecolor{currentstroke}%
\pgfsetstrokeopacity{0.300000}%
\pgfsetdash{}{0pt}%
\pgfpathmoveto{\pgfqpoint{1.703753in}{1.084561in}}%
\pgfpathlineto{\pgfqpoint{1.658328in}{1.130038in}}%
\pgfpathlineto{\pgfqpoint{1.609338in}{1.174382in}}%
\pgfpathlineto{\pgfqpoint{1.554850in}{1.216734in}}%
\pgfpathlineto{\pgfqpoint{1.503756in}{1.248536in}}%
\pgfpathlineto{\pgfqpoint{1.457454in}{1.269761in}}%
\pgfpathlineto{\pgfqpoint{1.411927in}{1.282822in}}%
\pgfpathlineto{\pgfqpoint{1.359649in}{1.287568in}}%
\pgfpathlineto{\pgfqpoint{1.308015in}{1.281175in}}%
\pgfpathlineto{\pgfqpoint{1.308015in}{1.281175in}}%
\pgfpathlineto{\pgfqpoint{1.251303in}{1.261495in}}%
\pgfpathlineto{\pgfqpoint{1.251303in}{1.261495in}}%
\pgfpathlineto{\pgfqpoint{1.187051in}{1.223821in}}%
\pgfpathlineto{\pgfqpoint{1.133598in}{1.181195in}}%
\pgfusepath{stroke}%
\end{pgfscope}%
\begin{pgfscope}%
\pgfpathrectangle{\pgfqpoint{0.647939in}{0.492442in}}{\pgfqpoint{4.273799in}{2.331163in}}%
\pgfusepath{clip}%
\pgfsetbuttcap%
\pgfsetroundjoin%
\pgfsetlinewidth{0.301125pt}%
\definecolor{currentstroke}{rgb}{0.500000,0.500000,0.500000}%
\pgfsetstrokecolor{currentstroke}%
\pgfsetstrokeopacity{0.300000}%
\pgfsetdash{}{0pt}%
\pgfpathmoveto{\pgfqpoint{1.688481in}{0.774783in}}%
\pgfpathlineto{\pgfqpoint{1.642277in}{0.820036in}}%
\pgfpathlineto{\pgfqpoint{1.593195in}{0.864368in}}%
\pgfpathlineto{\pgfqpoint{1.540002in}{0.907237in}}%
\pgfpathlineto{\pgfqpoint{1.480663in}{0.947557in}}%
\pgfpathlineto{\pgfqpoint{1.426241in}{0.976584in}}%
\pgfpathlineto{\pgfqpoint{1.375782in}{0.995837in}}%
\pgfpathlineto{\pgfqpoint{1.324573in}{1.007123in}}%
\pgfpathlineto{\pgfqpoint{1.265670in}{1.009125in}}%
\pgfpathlineto{\pgfqpoint{1.209072in}{0.999659in}}%
\pgfpathlineto{\pgfqpoint{1.209072in}{0.999659in}}%
\pgfpathlineto{\pgfqpoint{1.133598in}{0.969271in}}%
\pgfusepath{stroke}%
\end{pgfscope}%
\begin{pgfscope}%
\pgfpathrectangle{\pgfqpoint{0.647939in}{0.492442in}}{\pgfqpoint{4.273799in}{2.331163in}}%
\pgfusepath{clip}%
\pgfsetbuttcap%
\pgfsetroundjoin%
\pgfsetlinewidth{0.301125pt}%
\definecolor{currentstroke}{rgb}{0.500000,0.500000,0.500000}%
\pgfsetstrokecolor{currentstroke}%
\pgfsetstrokeopacity{0.300000}%
\pgfsetdash{}{0pt}%
\pgfpathmoveto{\pgfqpoint{3.694657in}{0.736234in}}%
\pgfpathlineto{\pgfqpoint{3.628192in}{0.773237in}}%
\pgfpathlineto{\pgfqpoint{3.561893in}{0.810328in}}%
\pgfpathlineto{\pgfqpoint{3.496189in}{0.847732in}}%
\pgfpathlineto{\pgfqpoint{3.431462in}{0.885636in}}%
\pgfpathlineto{\pgfqpoint{3.368038in}{0.924190in}}%
\pgfpathlineto{\pgfqpoint{3.306181in}{0.963491in}}%
\pgfpathlineto{\pgfqpoint{3.246096in}{1.003601in}}%
\pgfpathlineto{\pgfqpoint{3.187918in}{1.044538in}}%
\pgfpathlineto{\pgfqpoint{3.131741in}{1.086297in}}%
\pgfpathlineto{\pgfqpoint{3.077613in}{1.128853in}}%
\pgfpathlineto{\pgfqpoint{3.025544in}{1.172170in}}%
\pgfpathlineto{\pgfqpoint{2.975530in}{1.216202in}}%
\pgfpathlineto{\pgfqpoint{2.927549in}{1.260901in}}%
\pgfpathlineto{\pgfqpoint{2.881572in}{1.306223in}}%
\pgfusepath{stroke}%
\end{pgfscope}%
\begin{pgfscope}%
\pgfpathrectangle{\pgfqpoint{0.647939in}{0.492442in}}{\pgfqpoint{4.273799in}{2.331163in}}%
\pgfusepath{clip}%
\pgfsetbuttcap%
\pgfsetroundjoin%
\pgfsetlinewidth{0.301125pt}%
\definecolor{currentstroke}{rgb}{0.500000,0.500000,0.500000}%
\pgfsetstrokecolor{currentstroke}%
\pgfsetstrokeopacity{0.300000}%
\pgfsetdash{}{0pt}%
\pgfpathmoveto{\pgfqpoint{3.984480in}{0.735441in}}%
\pgfpathlineto{\pgfqpoint{3.919948in}{0.773437in}}%
\pgfpathlineto{\pgfqpoint{3.853289in}{0.810328in}}%
\pgfpathlineto{\pgfqpoint{3.784953in}{0.846299in}}%
\pgfpathlineto{\pgfqpoint{3.715475in}{0.881615in}}%
\pgfpathlineto{\pgfqpoint{3.645451in}{0.916611in}}%
\pgfusepath{stroke}%
\end{pgfscope}%
\begin{pgfscope}%
\pgfpathrectangle{\pgfqpoint{0.647939in}{0.492442in}}{\pgfqpoint{4.273799in}{2.331163in}}%
\pgfusepath{clip}%
\pgfsetbuttcap%
\pgfsetroundjoin%
\pgfsetlinewidth{0.301125pt}%
\definecolor{currentstroke}{rgb}{0.500000,0.500000,0.500000}%
\pgfsetstrokecolor{currentstroke}%
\pgfsetstrokeopacity{0.300000}%
\pgfsetdash{}{0pt}%
\pgfpathmoveto{\pgfqpoint{4.467081in}{0.989536in}}%
\pgfpathlineto{\pgfqpoint{4.427232in}{1.036545in}}%
\pgfpathlineto{\pgfqpoint{4.384712in}{1.082848in}}%
\pgfpathlineto{\pgfqpoint{4.338948in}{1.128214in}}%
\pgfpathlineto{\pgfqpoint{4.289196in}{1.172304in}}%
\pgfpathlineto{\pgfqpoint{4.234495in}{1.214602in}}%
\pgfpathlineto{\pgfqpoint{4.173729in}{1.254335in}}%
\pgfpathlineto{\pgfqpoint{4.105820in}{1.290408in}}%
\pgfpathlineto{\pgfqpoint{4.030152in}{1.321494in}}%
\pgfpathlineto{\pgfqpoint{3.947382in}{1.346592in}}%
\pgfpathlineto{\pgfqpoint{3.859423in}{1.365857in}}%
\pgfpathlineto{\pgfqpoint{3.768626in}{1.380920in}}%
\pgfpathlineto{\pgfqpoint{3.676878in}{1.394252in}}%
\pgfpathlineto{\pgfqpoint{3.585518in}{1.408325in}}%
\pgfusepath{stroke}%
\end{pgfscope}%
\begin{pgfscope}%
\pgfpathrectangle{\pgfqpoint{0.647939in}{0.492442in}}{\pgfqpoint{4.273799in}{2.331163in}}%
\pgfusepath{clip}%
\pgfsetbuttcap%
\pgfsetroundjoin%
\pgfsetlinewidth{0.301125pt}%
\definecolor{currentstroke}{rgb}{0.500000,0.500000,0.500000}%
\pgfsetstrokecolor{currentstroke}%
\pgfsetstrokeopacity{0.300000}%
\pgfsetdash{}{0pt}%
\pgfpathmoveto{\pgfqpoint{4.424544in}{1.618866in}}%
\pgfpathlineto{\pgfqpoint{4.385192in}{1.665946in}}%
\pgfpathlineto{\pgfqpoint{4.338948in}{1.711005in}}%
\pgfpathlineto{\pgfqpoint{4.279524in}{1.750746in}}%
\pgfpathlineto{\pgfqpoint{4.279524in}{1.750746in}}%
\pgfpathlineto{\pgfqpoint{4.235275in}{1.766551in}}%
\pgfpathlineto{\pgfqpoint{4.235275in}{1.766551in}}%
\pgfpathlineto{\pgfqpoint{4.190859in}{1.771540in}}%
\pgfpathlineto{\pgfqpoint{4.145673in}{1.767563in}}%
\pgfpathlineto{\pgfqpoint{4.100098in}{1.756867in}}%
\pgfusepath{stroke}%
\end{pgfscope}%
\begin{pgfscope}%
\pgfpathrectangle{\pgfqpoint{0.647939in}{0.492442in}}{\pgfqpoint{4.273799in}{2.331163in}}%
\pgfusepath{clip}%
\pgfsetbuttcap%
\pgfsetroundjoin%
\pgfsetlinewidth{0.301125pt}%
\definecolor{currentstroke}{rgb}{0.500000,0.500000,0.500000}%
\pgfsetstrokecolor{currentstroke}%
\pgfsetstrokeopacity{0.300000}%
\pgfsetdash{}{0pt}%
\pgfpathmoveto{\pgfqpoint{1.627301in}{2.596993in}}%
\pgfpathlineto{\pgfqpoint{1.668902in}{2.550473in}}%
\pgfpathlineto{\pgfqpoint{1.716389in}{2.505719in}}%
\pgfpathlineto{\pgfqpoint{1.773618in}{2.464610in}}%
\pgfpathlineto{\pgfqpoint{1.773618in}{2.464610in}}%
\pgfpathlineto{\pgfqpoint{1.829778in}{2.438166in}}%
\pgfpathlineto{\pgfqpoint{1.829778in}{2.438166in}}%
\pgfpathlineto{\pgfqpoint{1.883581in}{2.424722in}}%
\pgfpathlineto{\pgfqpoint{1.943324in}{2.420995in}}%
\pgfusepath{stroke}%
\end{pgfscope}%
\begin{pgfscope}%
\pgfpathrectangle{\pgfqpoint{0.647939in}{0.492442in}}{\pgfqpoint{4.273799in}{2.331163in}}%
\pgfusepath{clip}%
\pgfsetbuttcap%
\pgfsetroundjoin%
\pgfsetlinewidth{0.301125pt}%
\definecolor{currentstroke}{rgb}{0.500000,0.500000,0.500000}%
\pgfsetstrokecolor{currentstroke}%
\pgfsetstrokeopacity{0.300000}%
\pgfsetdash{}{0pt}%
\pgfpathmoveto{\pgfqpoint{4.349365in}{1.361072in}}%
\pgfpathlineto{\pgfqpoint{4.299236in}{1.404997in}}%
\pgfpathlineto{\pgfqpoint{4.241816in}{1.446100in}}%
\pgfpathlineto{\pgfqpoint{4.174299in}{1.482219in}}%
\pgfpathlineto{\pgfqpoint{4.094454in}{1.509457in}}%
\pgfpathlineto{\pgfqpoint{4.014459in}{1.523111in}}%
\pgfpathlineto{\pgfqpoint{3.936287in}{1.527064in}}%
\pgfusepath{stroke}%
\end{pgfscope}%
\begin{pgfscope}%
\pgfpathrectangle{\pgfqpoint{0.647939in}{0.492442in}}{\pgfqpoint{4.273799in}{2.331163in}}%
\pgfusepath{clip}%
\pgfsetbuttcap%
\pgfsetroundjoin%
\pgfsetlinewidth{0.301125pt}%
\definecolor{currentstroke}{rgb}{0.500000,0.500000,0.500000}%
\pgfsetstrokecolor{currentstroke}%
\pgfsetstrokeopacity{0.300000}%
\pgfsetdash{}{0pt}%
\pgfpathmoveto{\pgfqpoint{1.327862in}{2.346777in}}%
\pgfpathlineto{\pgfqpoint{1.338344in}{2.295292in}}%
\pgfpathlineto{\pgfqpoint{1.347160in}{2.243715in}}%
\pgfpathlineto{\pgfqpoint{1.353945in}{2.192049in}}%
\pgfpathlineto{\pgfqpoint{1.358295in}{2.140305in}}%
\pgfpathlineto{\pgfqpoint{1.359765in}{2.088515in}}%
\pgfpathlineto{\pgfqpoint{1.357921in}{2.036732in}}%
\pgfpathlineto{\pgfqpoint{1.352403in}{1.985030in}}%
\pgfpathlineto{\pgfqpoint{1.343012in}{1.933496in}}%
\pgfusepath{stroke}%
\end{pgfscope}%
\begin{pgfscope}%
\pgfpathrectangle{\pgfqpoint{0.647939in}{0.492442in}}{\pgfqpoint{4.273799in}{2.331163in}}%
\pgfusepath{clip}%
\pgfsetbuttcap%
\pgfsetroundjoin%
\pgfsetlinewidth{0.301125pt}%
\definecolor{currentstroke}{rgb}{0.500000,0.500000,0.500000}%
\pgfsetstrokecolor{currentstroke}%
\pgfsetstrokeopacity{0.300000}%
\pgfsetdash{}{0pt}%
\pgfpathmoveto{\pgfqpoint{1.606778in}{1.387114in}}%
\pgfpathlineto{\pgfqpoint{1.557946in}{1.421798in}}%
\pgfpathlineto{\pgfqpoint{1.514813in}{1.444862in}}%
\pgfpathlineto{\pgfqpoint{1.473896in}{1.459317in}}%
\pgfpathlineto{\pgfqpoint{1.427820in}{1.466232in}}%
\pgfpathlineto{\pgfqpoint{1.379998in}{1.462599in}}%
\pgfpathlineto{\pgfqpoint{1.379998in}{1.462599in}}%
\pgfpathlineto{\pgfqpoint{1.327862in}{1.446100in}}%
\pgfpathlineto{\pgfqpoint{1.327862in}{1.446100in}}%
\pgfpathlineto{\pgfqpoint{1.327862in}{1.446100in}}%
\pgfpathlineto{\pgfqpoint{1.273302in}{1.415830in}}%
\pgfpathlineto{\pgfqpoint{1.228434in}{1.381967in}}%
\pgfusepath{stroke}%
\end{pgfscope}%
\begin{pgfscope}%
\pgfpathrectangle{\pgfqpoint{0.647939in}{0.492442in}}{\pgfqpoint{4.273799in}{2.331163in}}%
\pgfusepath{clip}%
\pgfsetbuttcap%
\pgfsetroundjoin%
\pgfsetlinewidth{0.301125pt}%
\definecolor{currentstroke}{rgb}{0.500000,0.500000,0.500000}%
\pgfsetstrokecolor{currentstroke}%
\pgfsetstrokeopacity{0.300000}%
\pgfsetdash{}{0pt}%
\pgfpathmoveto{\pgfqpoint{4.204413in}{1.035004in}}%
\pgfpathlineto{\pgfqpoint{4.144684in}{1.075233in}}%
\pgfpathlineto{\pgfqpoint{4.079837in}{1.113025in}}%
\pgfpathlineto{\pgfqpoint{4.009682in}{1.147868in}}%
\pgfpathlineto{\pgfqpoint{3.934468in}{1.179420in}}%
\pgfpathlineto{\pgfqpoint{3.855035in}{1.207748in}}%
\pgfpathlineto{\pgfqpoint{3.772634in}{1.233464in}}%
\pgfusepath{stroke}%
\end{pgfscope}%
\begin{pgfscope}%
\pgfpathrectangle{\pgfqpoint{0.647939in}{0.492442in}}{\pgfqpoint{4.273799in}{2.331163in}}%
\pgfusepath{clip}%
\pgfsetbuttcap%
\pgfsetroundjoin%
\pgfsetlinewidth{0.301125pt}%
\definecolor{currentstroke}{rgb}{0.500000,0.500000,0.500000}%
\pgfsetstrokecolor{currentstroke}%
\pgfsetstrokeopacity{0.300000}%
\pgfsetdash{}{0pt}%
\pgfpathmoveto{\pgfqpoint{1.624134in}{2.049696in}}%
\pgfpathlineto{\pgfqpoint{1.624134in}{2.049696in}}%
\pgfpathlineto{\pgfqpoint{1.587190in}{2.003269in}}%
\pgfpathlineto{\pgfqpoint{1.545840in}{1.957214in}}%
\pgfpathlineto{\pgfqpoint{1.504617in}{1.910832in}}%
\pgfpathlineto{\pgfqpoint{1.464289in}{1.864073in}}%
\pgfpathlineto{\pgfqpoint{1.424993in}{1.816967in}}%
\pgfpathlineto{\pgfqpoint{1.386800in}{1.769577in}}%
\pgfusepath{stroke}%
\end{pgfscope}%
\begin{pgfscope}%
\pgfpathrectangle{\pgfqpoint{0.647939in}{0.492442in}}{\pgfqpoint{4.273799in}{2.331163in}}%
\pgfusepath{clip}%
\pgfsetbuttcap%
\pgfsetroundjoin%
\pgfsetlinewidth{0.301125pt}%
\definecolor{currentstroke}{rgb}{0.500000,0.500000,0.500000}%
\pgfsetstrokecolor{currentstroke}%
\pgfsetstrokeopacity{0.300000}%
\pgfsetdash{}{0pt}%
\pgfpathmoveto{\pgfqpoint{2.299180in}{0.969271in}}%
\pgfpathlineto{\pgfqpoint{2.264400in}{1.017476in}}%
\pgfpathlineto{\pgfqpoint{2.230108in}{1.065784in}}%
\pgfpathlineto{\pgfqpoint{2.196282in}{1.114191in}}%
\pgfpathlineto{\pgfqpoint{2.162897in}{1.162688in}}%
\pgfpathlineto{\pgfqpoint{2.129917in}{1.211268in}}%
\pgfpathlineto{\pgfqpoint{2.097311in}{1.259923in}}%
\pgfpathlineto{\pgfqpoint{2.065053in}{1.308646in}}%
\pgfpathlineto{\pgfqpoint{2.033111in}{1.357432in}}%
\pgfpathlineto{\pgfqpoint{2.001436in}{1.406269in}}%
\pgfpathlineto{\pgfqpoint{1.969975in}{1.455147in}}%
\pgfpathlineto{\pgfqpoint{1.938681in}{1.504057in}}%
\pgfpathlineto{\pgfqpoint{1.907478in}{1.552985in}}%
\pgfpathlineto{\pgfqpoint{1.876250in}{1.601907in}}%
\pgfpathlineto{\pgfqpoint{1.844860in}{1.650796in}}%
\pgfpathlineto{\pgfqpoint{1.813111in}{1.699617in}}%
\pgfpathlineto{\pgfqpoint{1.780634in}{1.748293in}}%
\pgfpathlineto{\pgfqpoint{1.746760in}{1.796673in}}%
\pgfpathlineto{\pgfqpoint{1.710115in}{1.844415in}}%
\pgfpathlineto{\pgfqpoint{1.666796in}{1.890267in}}%
\pgfpathlineto{\pgfqpoint{1.666796in}{1.890267in}}%
\pgfpathlineto{\pgfqpoint{1.633605in}{1.913524in}}%
\pgfpathlineto{\pgfqpoint{1.633605in}{1.913524in}}%
\pgfpathlineto{\pgfqpoint{1.605770in}{1.922348in}}%
\pgfpathlineto{\pgfqpoint{1.605770in}{1.922348in}}%
\pgfpathlineto{\pgfqpoint{1.579281in}{1.921142in}}%
\pgfusepath{stroke}%
\end{pgfscope}%
\begin{pgfscope}%
\pgfpathrectangle{\pgfqpoint{0.647939in}{0.492442in}}{\pgfqpoint{4.273799in}{2.331163in}}%
\pgfusepath{clip}%
\pgfsetbuttcap%
\pgfsetroundjoin%
\pgfsetlinewidth{0.301125pt}%
\definecolor{currentstroke}{rgb}{0.500000,0.500000,0.500000}%
\pgfsetstrokecolor{currentstroke}%
\pgfsetstrokeopacity{0.300000}%
\pgfsetdash{}{0pt}%
\pgfpathmoveto{\pgfqpoint{3.464761in}{2.346777in}}%
\pgfpathlineto{\pgfqpoint{3.480596in}{2.295704in}}%
\pgfpathlineto{\pgfqpoint{3.493720in}{2.244405in}}%
\pgfpathlineto{\pgfqpoint{3.503759in}{2.192900in}}%
\pgfpathlineto{\pgfqpoint{3.510230in}{2.141229in}}%
\pgfpathlineto{\pgfqpoint{3.512493in}{2.089456in}}%
\pgfpathlineto{\pgfqpoint{3.509696in}{2.037701in}}%
\pgfpathlineto{\pgfqpoint{3.500677in}{1.986173in}}%
\pgfpathlineto{\pgfqpoint{3.483797in}{1.935263in}}%
\pgfpathlineto{\pgfqpoint{3.456657in}{1.885743in}}%
\pgfpathlineto{\pgfqpoint{3.415718in}{1.839272in}}%
\pgfpathlineto{\pgfqpoint{3.415718in}{1.839272in}}%
\pgfpathlineto{\pgfqpoint{3.368180in}{1.805670in}}%
\pgfpathlineto{\pgfqpoint{3.368180in}{1.805670in}}%
\pgfpathlineto{\pgfqpoint{3.316766in}{1.783882in}}%
\pgfpathlineto{\pgfqpoint{3.255236in}{1.771741in}}%
\pgfpathlineto{\pgfqpoint{3.197177in}{1.771025in}}%
\pgfpathlineto{\pgfqpoint{3.141568in}{1.778846in}}%
\pgfusepath{stroke}%
\end{pgfscope}%
\begin{pgfscope}%
\pgfpathrectangle{\pgfqpoint{0.647939in}{0.492442in}}{\pgfqpoint{4.273799in}{2.331163in}}%
\pgfusepath{clip}%
\pgfsetbuttcap%
\pgfsetroundjoin%
\pgfsetlinewidth{0.301125pt}%
\definecolor{currentstroke}{rgb}{0.500000,0.500000,0.500000}%
\pgfsetstrokecolor{currentstroke}%
\pgfsetstrokeopacity{0.300000}%
\pgfsetdash{}{0pt}%
\pgfpathmoveto{\pgfqpoint{2.246114in}{2.063826in}}%
\pgfpathlineto{\pgfqpoint{2.246303in}{2.115599in}}%
\pgfpathlineto{\pgfqpoint{2.253658in}{2.167197in}}%
\pgfpathlineto{\pgfqpoint{2.270268in}{2.218108in}}%
\pgfpathlineto{\pgfqpoint{2.298885in}{2.267326in}}%
\pgfpathlineto{\pgfqpoint{2.336096in}{2.307392in}}%
\pgfpathlineto{\pgfqpoint{2.396312in}{2.346777in}}%
\pgfpathlineto{\pgfqpoint{2.396312in}{2.346777in}}%
\pgfusepath{stroke}%
\end{pgfscope}%
\begin{pgfscope}%
\pgfpathrectangle{\pgfqpoint{0.647939in}{0.492442in}}{\pgfqpoint{4.273799in}{2.331163in}}%
\pgfusepath{clip}%
\pgfsetbuttcap%
\pgfsetroundjoin%
\pgfsetlinewidth{0.301125pt}%
\definecolor{currentstroke}{rgb}{0.500000,0.500000,0.500000}%
\pgfsetstrokecolor{currentstroke}%
\pgfsetstrokeopacity{0.300000}%
\pgfsetdash{}{0pt}%
\pgfpathmoveto{\pgfqpoint{1.505357in}{2.291795in}}%
\pgfpathlineto{\pgfqpoint{1.522125in}{2.240815in}}%
\pgfpathlineto{\pgfqpoint{1.536458in}{2.189619in}}%
\pgfpathlineto{\pgfqpoint{1.546757in}{2.138145in}}%
\pgfpathlineto{\pgfqpoint{1.550273in}{2.086437in}}%
\pgfpathlineto{\pgfqpoint{1.543380in}{2.034913in}}%
\pgfusepath{stroke}%
\end{pgfscope}%
\begin{pgfscope}%
\pgfpathrectangle{\pgfqpoint{0.647939in}{0.492442in}}{\pgfqpoint{4.273799in}{2.331163in}}%
\pgfusepath{clip}%
\pgfsetbuttcap%
\pgfsetroundjoin%
\pgfsetlinewidth{0.301125pt}%
\definecolor{currentstroke}{rgb}{0.500000,0.500000,0.500000}%
\pgfsetstrokecolor{currentstroke}%
\pgfsetstrokeopacity{0.300000}%
\pgfsetdash{}{0pt}%
\pgfpathmoveto{\pgfqpoint{1.823925in}{1.505249in}}%
\pgfpathlineto{\pgfqpoint{1.788252in}{1.553254in}}%
\pgfpathlineto{\pgfqpoint{1.751053in}{1.600907in}}%
\pgfpathlineto{\pgfqpoint{1.711339in}{1.647936in}}%
\pgfpathlineto{\pgfqpoint{1.667104in}{1.693693in}}%
\pgfpathlineto{\pgfqpoint{1.628253in}{1.726185in}}%
\pgfpathlineto{\pgfqpoint{1.595434in}{1.746614in}}%
\pgfpathlineto{\pgfqpoint{1.563746in}{1.759176in}}%
\pgfpathlineto{\pgfqpoint{1.522125in}{1.763986in}}%
\pgfpathlineto{\pgfqpoint{1.522125in}{1.763986in}}%
\pgfpathlineto{\pgfqpoint{1.522125in}{1.763986in}}%
\pgfpathlineto{\pgfqpoint{1.481993in}{1.756221in}}%
\pgfusepath{stroke}%
\end{pgfscope}%
\begin{pgfscope}%
\pgfpathrectangle{\pgfqpoint{0.647939in}{0.492442in}}{\pgfqpoint{4.273799in}{2.331163in}}%
\pgfusepath{clip}%
\pgfsetbuttcap%
\pgfsetroundjoin%
\pgfsetlinewidth{0.301125pt}%
\definecolor{currentstroke}{rgb}{0.500000,0.500000,0.500000}%
\pgfsetstrokecolor{currentstroke}%
\pgfsetstrokeopacity{0.300000}%
\pgfsetdash{}{0pt}%
\pgfpathmoveto{\pgfqpoint{3.561893in}{2.293796in}}%
\pgfpathlineto{\pgfqpoint{3.574218in}{2.242437in}}%
\pgfpathlineto{\pgfqpoint{3.583592in}{2.190894in}}%
\pgfpathlineto{\pgfqpoint{3.589553in}{2.139203in}}%
\pgfpathlineto{\pgfqpoint{3.591514in}{2.087426in}}%
\pgfpathlineto{\pgfqpoint{3.588707in}{2.035668in}}%
\pgfpathlineto{\pgfqpoint{3.580130in}{1.984111in}}%
\pgfpathlineto{\pgfqpoint{3.564435in}{1.933072in}}%
\pgfpathlineto{\pgfqpoint{3.539802in}{1.883126in}}%
\pgfpathlineto{\pgfqpoint{3.503723in}{1.835364in}}%
\pgfpathlineto{\pgfqpoint{3.452851in}{1.791971in}}%
\pgfpathlineto{\pgfqpoint{3.452851in}{1.791971in}}%
\pgfpathlineto{\pgfqpoint{3.399298in}{1.763139in}}%
\pgfpathlineto{\pgfqpoint{3.332688in}{1.743293in}}%
\pgfpathlineto{\pgfqpoint{3.268340in}{1.736634in}}%
\pgfusepath{stroke}%
\end{pgfscope}%
\begin{pgfscope}%
\pgfpathrectangle{\pgfqpoint{0.647939in}{0.492442in}}{\pgfqpoint{4.273799in}{2.331163in}}%
\pgfusepath{clip}%
\pgfsetbuttcap%
\pgfsetroundjoin%
\pgfsetlinewidth{0.301125pt}%
\definecolor{currentstroke}{rgb}{0.500000,0.500000,0.500000}%
\pgfsetstrokecolor{currentstroke}%
\pgfsetstrokeopacity{0.300000}%
\pgfsetdash{}{0pt}%
\pgfpathmoveto{\pgfqpoint{1.749361in}{1.689051in}}%
\pgfpathlineto{\pgfqpoint{1.710674in}{1.736325in}}%
\pgfpathlineto{\pgfqpoint{1.673799in}{1.775465in}}%
\pgfpathlineto{\pgfqpoint{1.619257in}{1.816967in}}%
\pgfpathlineto{\pgfqpoint{1.619257in}{1.816967in}}%
\pgfpathlineto{\pgfqpoint{1.586041in}{1.831064in}}%
\pgfpathlineto{\pgfqpoint{1.586041in}{1.831064in}}%
\pgfpathlineto{\pgfqpoint{1.552560in}{1.834909in}}%
\pgfpathlineto{\pgfqpoint{1.519920in}{1.828896in}}%
\pgfpathlineto{\pgfqpoint{1.492712in}{1.817340in}}%
\pgfpathlineto{\pgfqpoint{1.460566in}{1.797198in}}%
\pgfusepath{stroke}%
\end{pgfscope}%
\begin{pgfscope}%
\pgfpathrectangle{\pgfqpoint{0.647939in}{0.492442in}}{\pgfqpoint{4.273799in}{2.331163in}}%
\pgfusepath{clip}%
\pgfsetbuttcap%
\pgfsetroundjoin%
\pgfsetlinewidth{0.301125pt}%
\definecolor{currentstroke}{rgb}{0.500000,0.500000,0.500000}%
\pgfsetstrokecolor{currentstroke}%
\pgfsetstrokeopacity{0.300000}%
\pgfsetdash{}{0pt}%
\pgfpathmoveto{\pgfqpoint{3.659025in}{2.240815in}}%
\pgfpathlineto{\pgfqpoint{3.667960in}{2.189248in}}%
\pgfpathlineto{\pgfqpoint{3.673586in}{2.137545in}}%
\pgfpathlineto{\pgfqpoint{3.675338in}{2.085765in}}%
\pgfpathlineto{\pgfqpoint{3.672496in}{2.034006in}}%
\pgfpathlineto{\pgfqpoint{3.664155in}{1.982434in}}%
\pgfpathlineto{\pgfqpoint{3.649153in}{1.931325in}}%
\pgfpathlineto{\pgfqpoint{3.626034in}{1.881146in}}%
\pgfpathlineto{\pgfqpoint{3.592930in}{1.832707in}}%
\pgfpathlineto{\pgfqpoint{3.547503in}{1.787429in}}%
\pgfpathlineto{\pgfqpoint{3.487193in}{1.747861in}}%
\pgfpathlineto{\pgfqpoint{3.416048in}{1.719942in}}%
\pgfpathlineto{\pgfqpoint{3.345531in}{1.706525in}}%
\pgfusepath{stroke}%
\end{pgfscope}%
\begin{pgfscope}%
\pgfpathrectangle{\pgfqpoint{0.647939in}{0.492442in}}{\pgfqpoint{4.273799in}{2.331163in}}%
\pgfusepath{clip}%
\pgfsetbuttcap%
\pgfsetroundjoin%
\pgfsetlinewidth{0.301125pt}%
\definecolor{currentstroke}{rgb}{0.500000,0.500000,0.500000}%
\pgfsetstrokecolor{currentstroke}%
\pgfsetstrokeopacity{0.300000}%
\pgfsetdash{}{0pt}%
\pgfpathmoveto{\pgfqpoint{2.624107in}{1.922775in}}%
\pgfpathlineto{\pgfqpoint{2.605934in}{1.973603in}}%
\pgfpathlineto{\pgfqpoint{2.593024in}{2.024899in}}%
\pgfpathlineto{\pgfqpoint{2.586846in}{2.076545in}}%
\pgfpathlineto{\pgfqpoint{2.589960in}{2.128217in}}%
\pgfpathlineto{\pgfqpoint{2.602546in}{2.169762in}}%
\pgfpathlineto{\pgfqpoint{2.622128in}{2.200903in}}%
\pgfpathlineto{\pgfqpoint{2.647049in}{2.223212in}}%
\pgfpathlineto{\pgfqpoint{2.687707in}{2.240815in}}%
\pgfpathlineto{\pgfqpoint{2.687707in}{2.240815in}}%
\pgfpathlineto{\pgfqpoint{2.687707in}{2.240815in}}%
\pgfpathlineto{\pgfqpoint{2.727953in}{2.245440in}}%
\pgfusepath{stroke}%
\end{pgfscope}%
\begin{pgfscope}%
\pgfpathrectangle{\pgfqpoint{0.647939in}{0.492442in}}{\pgfqpoint{4.273799in}{2.331163in}}%
\pgfusepath{clip}%
\pgfsetbuttcap%
\pgfsetroundjoin%
\pgfsetlinewidth{0.301125pt}%
\definecolor{currentstroke}{rgb}{0.500000,0.500000,0.500000}%
\pgfsetstrokecolor{currentstroke}%
\pgfsetstrokeopacity{0.300000}%
\pgfsetdash{}{0pt}%
\pgfpathmoveto{\pgfqpoint{2.590575in}{1.287157in}}%
\pgfpathlineto{\pgfqpoint{2.555302in}{1.335253in}}%
\pgfpathlineto{\pgfqpoint{2.521262in}{1.383613in}}%
\pgfpathlineto{\pgfqpoint{2.488469in}{1.432229in}}%
\pgfpathlineto{\pgfqpoint{2.456951in}{1.481095in}}%
\pgfpathlineto{\pgfqpoint{2.426743in}{1.530207in}}%
\pgfpathlineto{\pgfqpoint{2.397890in}{1.579560in}}%
\pgfpathlineto{\pgfqpoint{2.370471in}{1.629156in}}%
\pgfpathlineto{\pgfqpoint{2.344582in}{1.678995in}}%
\pgfpathlineto{\pgfqpoint{2.320347in}{1.729081in}}%
\pgfpathlineto{\pgfqpoint{2.297942in}{1.779419in}}%
\pgfpathlineto{\pgfqpoint{2.277588in}{1.830015in}}%
\pgfpathlineto{\pgfqpoint{2.259582in}{1.880874in}}%
\pgfpathlineto{\pgfqpoint{2.244321in}{1.931997in}}%
\pgfpathlineto{\pgfqpoint{2.232327in}{1.983375in}}%
\pgfpathlineto{\pgfqpoint{2.224301in}{2.034977in}}%
\pgfusepath{stroke}%
\end{pgfscope}%
\begin{pgfscope}%
\pgfpathrectangle{\pgfqpoint{0.647939in}{0.492442in}}{\pgfqpoint{4.273799in}{2.331163in}}%
\pgfusepath{clip}%
\pgfsetbuttcap%
\pgfsetroundjoin%
\pgfsetlinewidth{0.301125pt}%
\definecolor{currentstroke}{rgb}{0.500000,0.500000,0.500000}%
\pgfsetstrokecolor{currentstroke}%
\pgfsetstrokeopacity{0.300000}%
\pgfsetdash{}{0pt}%
\pgfpathmoveto{\pgfqpoint{2.870424in}{1.247662in}}%
\pgfpathlineto{\pgfqpoint{2.826743in}{1.293655in}}%
\pgfpathlineto{\pgfqpoint{2.784839in}{1.340138in}}%
\pgfpathlineto{\pgfqpoint{2.744690in}{1.387080in}}%
\pgfpathlineto{\pgfqpoint{2.706285in}{1.434454in}}%
\pgfpathlineto{\pgfqpoint{2.669622in}{1.482238in}}%
\pgfusepath{stroke}%
\end{pgfscope}%
\begin{pgfscope}%
\pgfpathrectangle{\pgfqpoint{0.647939in}{0.492442in}}{\pgfqpoint{4.273799in}{2.331163in}}%
\pgfusepath{clip}%
\pgfsetbuttcap%
\pgfsetroundjoin%
\pgfsetlinewidth{0.301125pt}%
\definecolor{currentstroke}{rgb}{0.500000,0.500000,0.500000}%
\pgfsetstrokecolor{currentstroke}%
\pgfsetstrokeopacity{0.300000}%
\pgfsetdash{}{0pt}%
\pgfpathmoveto{\pgfqpoint{2.792841in}{1.619623in}}%
\pgfpathlineto{\pgfqpoint{2.755048in}{1.667138in}}%
\pgfpathlineto{\pgfqpoint{2.719974in}{1.715272in}}%
\pgfpathlineto{\pgfqpoint{2.687707in}{1.763986in}}%
\pgfpathlineto{\pgfqpoint{2.658396in}{1.813250in}}%
\pgfpathlineto{\pgfqpoint{2.632291in}{1.863046in}}%
\pgfusepath{stroke}%
\end{pgfscope}%
\begin{pgfscope}%
\pgfpathrectangle{\pgfqpoint{0.647939in}{0.492442in}}{\pgfqpoint{4.273799in}{2.331163in}}%
\pgfusepath{clip}%
\pgfsetroundcap%
\pgfsetroundjoin%
\pgfsetlinewidth{0.301125pt}%
\definecolor{currentstroke}{rgb}{0.500000,0.500000,0.500000}%
\pgfsetstrokecolor{currentstroke}%
\pgfsetstrokeopacity{0.300000}%
\pgfsetdash{}{0pt}%
\pgfpathmoveto{\pgfqpoint{1.461606in}{1.407579in}}%
\pgfusepath{stroke}%
\end{pgfscope}%
\begin{pgfscope}%
\pgfpathrectangle{\pgfqpoint{0.647939in}{0.492442in}}{\pgfqpoint{4.273799in}{2.331163in}}%
\pgfusepath{clip}%
\pgfsetroundcap%
\pgfsetroundjoin%
\definecolor{currentfill}{rgb}{0.500000,0.500000,0.500000}%
\pgfsetfillcolor{currentfill}%
\pgfsetfillopacity{0.300000}%
\pgfsetlinewidth{0.301125pt}%
\definecolor{currentstroke}{rgb}{0.500000,0.500000,0.500000}%
\pgfsetstrokecolor{currentstroke}%
\pgfsetstrokeopacity{0.300000}%
\pgfsetdash{}{0pt}%
\pgfpathmoveto{\pgfqpoint{0.000000in}{0.000000in}}%
\pgfpathlineto{\pgfqpoint{0.000000in}{0.000000in}}%
\pgfpathclose%
\pgfusepath{stroke,fill}%
\end{pgfscope}%
\begin{pgfscope}%
\pgfpathrectangle{\pgfqpoint{0.647939in}{0.492442in}}{\pgfqpoint{4.273799in}{2.331163in}}%
\pgfusepath{clip}%
\pgfsetroundcap%
\pgfsetroundjoin%
\pgfsetlinewidth{0.301125pt}%
\definecolor{currentstroke}{rgb}{0.500000,0.500000,0.500000}%
\pgfsetstrokecolor{currentstroke}%
\pgfsetstrokeopacity{0.300000}%
\pgfsetdash{}{0pt}%
\pgfpathmoveto{\pgfqpoint{1.336348in}{0.900533in}}%
\pgfusepath{stroke}%
\end{pgfscope}%
\begin{pgfscope}%
\pgfpathrectangle{\pgfqpoint{0.647939in}{0.492442in}}{\pgfqpoint{4.273799in}{2.331163in}}%
\pgfusepath{clip}%
\pgfsetroundcap%
\pgfsetroundjoin%
\definecolor{currentfill}{rgb}{0.500000,0.500000,0.500000}%
\pgfsetfillcolor{currentfill}%
\pgfsetfillopacity{0.300000}%
\pgfsetlinewidth{0.301125pt}%
\definecolor{currentstroke}{rgb}{0.500000,0.500000,0.500000}%
\pgfsetstrokecolor{currentstroke}%
\pgfsetstrokeopacity{0.300000}%
\pgfsetdash{}{0pt}%
\pgfpathmoveto{\pgfqpoint{0.000000in}{0.000000in}}%
\pgfpathlineto{\pgfqpoint{0.000000in}{0.000000in}}%
\pgfpathclose%
\pgfusepath{stroke,fill}%
\end{pgfscope}%
\begin{pgfscope}%
\pgfpathrectangle{\pgfqpoint{0.647939in}{0.492442in}}{\pgfqpoint{4.273799in}{2.331163in}}%
\pgfusepath{clip}%
\pgfsetroundcap%
\pgfsetroundjoin%
\pgfsetlinewidth{0.301125pt}%
\definecolor{currentstroke}{rgb}{0.500000,0.500000,0.500000}%
\pgfsetstrokecolor{currentstroke}%
\pgfsetstrokeopacity{0.300000}%
\pgfsetdash{}{0pt}%
\pgfpathmoveto{\pgfqpoint{1.217622in}{0.693681in}}%
\pgfusepath{stroke}%
\end{pgfscope}%
\begin{pgfscope}%
\pgfpathrectangle{\pgfqpoint{0.647939in}{0.492442in}}{\pgfqpoint{4.273799in}{2.331163in}}%
\pgfusepath{clip}%
\pgfsetroundcap%
\pgfsetroundjoin%
\definecolor{currentfill}{rgb}{0.500000,0.500000,0.500000}%
\pgfsetfillcolor{currentfill}%
\pgfsetfillopacity{0.300000}%
\pgfsetlinewidth{0.301125pt}%
\definecolor{currentstroke}{rgb}{0.500000,0.500000,0.500000}%
\pgfsetstrokecolor{currentstroke}%
\pgfsetstrokeopacity{0.300000}%
\pgfsetdash{}{0pt}%
\pgfpathmoveto{\pgfqpoint{0.000000in}{0.000000in}}%
\pgfpathlineto{\pgfqpoint{0.000000in}{0.000000in}}%
\pgfpathclose%
\pgfusepath{stroke,fill}%
\end{pgfscope}%
\begin{pgfscope}%
\pgfpathrectangle{\pgfqpoint{0.647939in}{0.492442in}}{\pgfqpoint{4.273799in}{2.331163in}}%
\pgfusepath{clip}%
\pgfsetroundcap%
\pgfsetroundjoin%
\pgfsetlinewidth{0.301125pt}%
\definecolor{currentstroke}{rgb}{0.500000,0.500000,0.500000}%
\pgfsetstrokecolor{currentstroke}%
\pgfsetstrokeopacity{0.300000}%
\pgfsetdash{}{0pt}%
\pgfpathmoveto{\pgfqpoint{1.177432in}{0.576262in}}%
\pgfusepath{stroke}%
\end{pgfscope}%
\begin{pgfscope}%
\pgfpathrectangle{\pgfqpoint{0.647939in}{0.492442in}}{\pgfqpoint{4.273799in}{2.331163in}}%
\pgfusepath{clip}%
\pgfsetroundcap%
\pgfsetroundjoin%
\definecolor{currentfill}{rgb}{0.500000,0.500000,0.500000}%
\pgfsetfillcolor{currentfill}%
\pgfsetfillopacity{0.300000}%
\pgfsetlinewidth{0.301125pt}%
\definecolor{currentstroke}{rgb}{0.500000,0.500000,0.500000}%
\pgfsetstrokecolor{currentstroke}%
\pgfsetstrokeopacity{0.300000}%
\pgfsetdash{}{0pt}%
\pgfpathmoveto{\pgfqpoint{0.000000in}{0.000000in}}%
\pgfpathlineto{\pgfqpoint{0.000000in}{0.000000in}}%
\pgfpathclose%
\pgfusepath{stroke,fill}%
\end{pgfscope}%
\begin{pgfscope}%
\pgfpathrectangle{\pgfqpoint{0.647939in}{0.492442in}}{\pgfqpoint{4.273799in}{2.331163in}}%
\pgfusepath{clip}%
\pgfsetroundcap%
\pgfsetroundjoin%
\pgfsetlinewidth{0.301125pt}%
\definecolor{currentstroke}{rgb}{0.500000,0.500000,0.500000}%
\pgfsetstrokecolor{currentstroke}%
\pgfsetstrokeopacity{0.300000}%
\pgfsetdash{}{0pt}%
\pgfpathmoveto{\pgfqpoint{1.372087in}{0.758189in}}%
\pgfusepath{stroke}%
\end{pgfscope}%
\begin{pgfscope}%
\pgfpathrectangle{\pgfqpoint{0.647939in}{0.492442in}}{\pgfqpoint{4.273799in}{2.331163in}}%
\pgfusepath{clip}%
\pgfsetroundcap%
\pgfsetroundjoin%
\definecolor{currentfill}{rgb}{0.500000,0.500000,0.500000}%
\pgfsetfillcolor{currentfill}%
\pgfsetfillopacity{0.300000}%
\pgfsetlinewidth{0.301125pt}%
\definecolor{currentstroke}{rgb}{0.500000,0.500000,0.500000}%
\pgfsetstrokecolor{currentstroke}%
\pgfsetstrokeopacity{0.300000}%
\pgfsetdash{}{0pt}%
\pgfpathmoveto{\pgfqpoint{0.000000in}{0.000000in}}%
\pgfpathlineto{\pgfqpoint{0.000000in}{0.000000in}}%
\pgfpathclose%
\pgfusepath{stroke,fill}%
\end{pgfscope}%
\begin{pgfscope}%
\pgfpathrectangle{\pgfqpoint{0.647939in}{0.492442in}}{\pgfqpoint{4.273799in}{2.331163in}}%
\pgfusepath{clip}%
\pgfsetroundcap%
\pgfsetroundjoin%
\pgfsetlinewidth{0.301125pt}%
\definecolor{currentstroke}{rgb}{0.500000,0.500000,0.500000}%
\pgfsetstrokecolor{currentstroke}%
\pgfsetstrokeopacity{0.300000}%
\pgfsetdash{}{0pt}%
\pgfpathmoveto{\pgfqpoint{1.812151in}{0.607238in}}%
\pgfusepath{stroke}%
\end{pgfscope}%
\begin{pgfscope}%
\pgfpathrectangle{\pgfqpoint{0.647939in}{0.492442in}}{\pgfqpoint{4.273799in}{2.331163in}}%
\pgfusepath{clip}%
\pgfsetroundcap%
\pgfsetroundjoin%
\definecolor{currentfill}{rgb}{0.500000,0.500000,0.500000}%
\pgfsetfillcolor{currentfill}%
\pgfsetfillopacity{0.300000}%
\pgfsetlinewidth{0.301125pt}%
\definecolor{currentstroke}{rgb}{0.500000,0.500000,0.500000}%
\pgfsetstrokecolor{currentstroke}%
\pgfsetstrokeopacity{0.300000}%
\pgfsetdash{}{0pt}%
\pgfpathmoveto{\pgfqpoint{0.000000in}{0.000000in}}%
\pgfpathlineto{\pgfqpoint{0.000000in}{0.000000in}}%
\pgfpathclose%
\pgfusepath{stroke,fill}%
\end{pgfscope}%
\begin{pgfscope}%
\pgfpathrectangle{\pgfqpoint{0.647939in}{0.492442in}}{\pgfqpoint{4.273799in}{2.331163in}}%
\pgfusepath{clip}%
\pgfsetroundcap%
\pgfsetroundjoin%
\pgfsetlinewidth{0.301125pt}%
\definecolor{currentstroke}{rgb}{0.500000,0.500000,0.500000}%
\pgfsetstrokecolor{currentstroke}%
\pgfsetstrokeopacity{0.300000}%
\pgfsetdash{}{0pt}%
\pgfpathmoveto{\pgfqpoint{1.509268in}{1.013567in}}%
\pgfusepath{stroke}%
\end{pgfscope}%
\begin{pgfscope}%
\pgfpathrectangle{\pgfqpoint{0.647939in}{0.492442in}}{\pgfqpoint{4.273799in}{2.331163in}}%
\pgfusepath{clip}%
\pgfsetroundcap%
\pgfsetroundjoin%
\definecolor{currentfill}{rgb}{0.500000,0.500000,0.500000}%
\pgfsetfillcolor{currentfill}%
\pgfsetfillopacity{0.300000}%
\pgfsetlinewidth{0.301125pt}%
\definecolor{currentstroke}{rgb}{0.500000,0.500000,0.500000}%
\pgfsetstrokecolor{currentstroke}%
\pgfsetstrokeopacity{0.300000}%
\pgfsetdash{}{0pt}%
\pgfpathmoveto{\pgfqpoint{0.000000in}{0.000000in}}%
\pgfpathlineto{\pgfqpoint{0.000000in}{0.000000in}}%
\pgfpathclose%
\pgfusepath{stroke,fill}%
\end{pgfscope}%
\begin{pgfscope}%
\pgfpathrectangle{\pgfqpoint{0.647939in}{0.492442in}}{\pgfqpoint{4.273799in}{2.331163in}}%
\pgfusepath{clip}%
\pgfsetroundcap%
\pgfsetroundjoin%
\pgfsetlinewidth{0.301125pt}%
\definecolor{currentstroke}{rgb}{0.500000,0.500000,0.500000}%
\pgfsetstrokecolor{currentstroke}%
\pgfsetstrokeopacity{0.300000}%
\pgfsetdash{}{0pt}%
\pgfpathmoveto{\pgfqpoint{1.647713in}{1.028789in}}%
\pgfusepath{stroke}%
\end{pgfscope}%
\begin{pgfscope}%
\pgfpathrectangle{\pgfqpoint{0.647939in}{0.492442in}}{\pgfqpoint{4.273799in}{2.331163in}}%
\pgfusepath{clip}%
\pgfsetroundcap%
\pgfsetroundjoin%
\definecolor{currentfill}{rgb}{0.500000,0.500000,0.500000}%
\pgfsetfillcolor{currentfill}%
\pgfsetfillopacity{0.300000}%
\pgfsetlinewidth{0.301125pt}%
\definecolor{currentstroke}{rgb}{0.500000,0.500000,0.500000}%
\pgfsetstrokecolor{currentstroke}%
\pgfsetstrokeopacity{0.300000}%
\pgfsetdash{}{0pt}%
\pgfpathmoveto{\pgfqpoint{0.000000in}{0.000000in}}%
\pgfpathlineto{\pgfqpoint{0.000000in}{0.000000in}}%
\pgfpathclose%
\pgfusepath{stroke,fill}%
\end{pgfscope}%
\begin{pgfscope}%
\pgfpathrectangle{\pgfqpoint{0.647939in}{0.492442in}}{\pgfqpoint{4.273799in}{2.331163in}}%
\pgfusepath{clip}%
\pgfsetroundcap%
\pgfsetroundjoin%
\pgfsetlinewidth{0.301125pt}%
\definecolor{currentstroke}{rgb}{0.500000,0.500000,0.500000}%
\pgfsetstrokecolor{currentstroke}%
\pgfsetstrokeopacity{0.300000}%
\pgfsetdash{}{0pt}%
\pgfpathmoveto{\pgfqpoint{1.833413in}{1.232014in}}%
\pgfusepath{stroke}%
\end{pgfscope}%
\begin{pgfscope}%
\pgfpathrectangle{\pgfqpoint{0.647939in}{0.492442in}}{\pgfqpoint{4.273799in}{2.331163in}}%
\pgfusepath{clip}%
\pgfsetroundcap%
\pgfsetroundjoin%
\definecolor{currentfill}{rgb}{0.500000,0.500000,0.500000}%
\pgfsetfillcolor{currentfill}%
\pgfsetfillopacity{0.300000}%
\pgfsetlinewidth{0.301125pt}%
\definecolor{currentstroke}{rgb}{0.500000,0.500000,0.500000}%
\pgfsetstrokecolor{currentstroke}%
\pgfsetstrokeopacity{0.300000}%
\pgfsetdash{}{0pt}%
\pgfpathmoveto{\pgfqpoint{0.000000in}{0.000000in}}%
\pgfpathlineto{\pgfqpoint{0.000000in}{0.000000in}}%
\pgfpathclose%
\pgfusepath{stroke,fill}%
\end{pgfscope}%
\begin{pgfscope}%
\pgfpathrectangle{\pgfqpoint{0.647939in}{0.492442in}}{\pgfqpoint{4.273799in}{2.331163in}}%
\pgfusepath{clip}%
\pgfsetroundcap%
\pgfsetroundjoin%
\pgfsetlinewidth{0.301125pt}%
\definecolor{currentstroke}{rgb}{0.500000,0.500000,0.500000}%
\pgfsetstrokecolor{currentstroke}%
\pgfsetstrokeopacity{0.300000}%
\pgfsetdash{}{0pt}%
\pgfpathmoveto{\pgfqpoint{1.936837in}{1.233850in}}%
\pgfusepath{stroke}%
\end{pgfscope}%
\begin{pgfscope}%
\pgfpathrectangle{\pgfqpoint{0.647939in}{0.492442in}}{\pgfqpoint{4.273799in}{2.331163in}}%
\pgfusepath{clip}%
\pgfsetroundcap%
\pgfsetroundjoin%
\definecolor{currentfill}{rgb}{0.500000,0.500000,0.500000}%
\pgfsetfillcolor{currentfill}%
\pgfsetfillopacity{0.300000}%
\pgfsetlinewidth{0.301125pt}%
\definecolor{currentstroke}{rgb}{0.500000,0.500000,0.500000}%
\pgfsetstrokecolor{currentstroke}%
\pgfsetstrokeopacity{0.300000}%
\pgfsetdash{}{0pt}%
\pgfpathmoveto{\pgfqpoint{0.000000in}{0.000000in}}%
\pgfpathlineto{\pgfqpoint{0.000000in}{0.000000in}}%
\pgfpathclose%
\pgfusepath{stroke,fill}%
\end{pgfscope}%
\begin{pgfscope}%
\pgfpathrectangle{\pgfqpoint{0.647939in}{0.492442in}}{\pgfqpoint{4.273799in}{2.331163in}}%
\pgfusepath{clip}%
\pgfsetroundcap%
\pgfsetroundjoin%
\pgfsetlinewidth{0.301125pt}%
\definecolor{currentstroke}{rgb}{0.500000,0.500000,0.500000}%
\pgfsetstrokecolor{currentstroke}%
\pgfsetstrokeopacity{0.300000}%
\pgfsetdash{}{0pt}%
\pgfpathmoveto{\pgfqpoint{2.380623in}{0.751074in}}%
\pgfusepath{stroke}%
\end{pgfscope}%
\begin{pgfscope}%
\pgfpathrectangle{\pgfqpoint{0.647939in}{0.492442in}}{\pgfqpoint{4.273799in}{2.331163in}}%
\pgfusepath{clip}%
\pgfsetroundcap%
\pgfsetroundjoin%
\definecolor{currentfill}{rgb}{0.500000,0.500000,0.500000}%
\pgfsetfillcolor{currentfill}%
\pgfsetfillopacity{0.300000}%
\pgfsetlinewidth{0.301125pt}%
\definecolor{currentstroke}{rgb}{0.500000,0.500000,0.500000}%
\pgfsetstrokecolor{currentstroke}%
\pgfsetstrokeopacity{0.300000}%
\pgfsetdash{}{0pt}%
\pgfpathmoveto{\pgfqpoint{0.000000in}{0.000000in}}%
\pgfpathlineto{\pgfqpoint{0.000000in}{0.000000in}}%
\pgfpathclose%
\pgfusepath{stroke,fill}%
\end{pgfscope}%
\begin{pgfscope}%
\pgfpathrectangle{\pgfqpoint{0.647939in}{0.492442in}}{\pgfqpoint{4.273799in}{2.331163in}}%
\pgfusepath{clip}%
\pgfsetroundcap%
\pgfsetroundjoin%
\pgfsetlinewidth{0.301125pt}%
\definecolor{currentstroke}{rgb}{0.500000,0.500000,0.500000}%
\pgfsetstrokecolor{currentstroke}%
\pgfsetstrokeopacity{0.300000}%
\pgfsetdash{}{0pt}%
\pgfpathmoveto{\pgfqpoint{2.548610in}{0.654457in}}%
\pgfusepath{stroke}%
\end{pgfscope}%
\begin{pgfscope}%
\pgfpathrectangle{\pgfqpoint{0.647939in}{0.492442in}}{\pgfqpoint{4.273799in}{2.331163in}}%
\pgfusepath{clip}%
\pgfsetroundcap%
\pgfsetroundjoin%
\definecolor{currentfill}{rgb}{0.500000,0.500000,0.500000}%
\pgfsetfillcolor{currentfill}%
\pgfsetfillopacity{0.300000}%
\pgfsetlinewidth{0.301125pt}%
\definecolor{currentstroke}{rgb}{0.500000,0.500000,0.500000}%
\pgfsetstrokecolor{currentstroke}%
\pgfsetstrokeopacity{0.300000}%
\pgfsetdash{}{0pt}%
\pgfpathmoveto{\pgfqpoint{0.000000in}{0.000000in}}%
\pgfpathlineto{\pgfqpoint{0.000000in}{0.000000in}}%
\pgfpathclose%
\pgfusepath{stroke,fill}%
\end{pgfscope}%
\begin{pgfscope}%
\pgfpathrectangle{\pgfqpoint{0.647939in}{0.492442in}}{\pgfqpoint{4.273799in}{2.331163in}}%
\pgfusepath{clip}%
\pgfsetroundcap%
\pgfsetroundjoin%
\pgfsetlinewidth{0.301125pt}%
\definecolor{currentstroke}{rgb}{0.500000,0.500000,0.500000}%
\pgfsetstrokecolor{currentstroke}%
\pgfsetstrokeopacity{0.300000}%
\pgfsetdash{}{0pt}%
\pgfpathmoveto{\pgfqpoint{2.681625in}{0.605942in}}%
\pgfusepath{stroke}%
\end{pgfscope}%
\begin{pgfscope}%
\pgfpathrectangle{\pgfqpoint{0.647939in}{0.492442in}}{\pgfqpoint{4.273799in}{2.331163in}}%
\pgfusepath{clip}%
\pgfsetroundcap%
\pgfsetroundjoin%
\definecolor{currentfill}{rgb}{0.500000,0.500000,0.500000}%
\pgfsetfillcolor{currentfill}%
\pgfsetfillopacity{0.300000}%
\pgfsetlinewidth{0.301125pt}%
\definecolor{currentstroke}{rgb}{0.500000,0.500000,0.500000}%
\pgfsetstrokecolor{currentstroke}%
\pgfsetstrokeopacity{0.300000}%
\pgfsetdash{}{0pt}%
\pgfpathmoveto{\pgfqpoint{0.000000in}{0.000000in}}%
\pgfpathlineto{\pgfqpoint{0.000000in}{0.000000in}}%
\pgfpathclose%
\pgfusepath{stroke,fill}%
\end{pgfscope}%
\begin{pgfscope}%
\pgfpathrectangle{\pgfqpoint{0.647939in}{0.492442in}}{\pgfqpoint{4.273799in}{2.331163in}}%
\pgfusepath{clip}%
\pgfsetroundcap%
\pgfsetroundjoin%
\pgfsetlinewidth{0.301125pt}%
\definecolor{currentstroke}{rgb}{0.500000,0.500000,0.500000}%
\pgfsetstrokecolor{currentstroke}%
\pgfsetstrokeopacity{0.300000}%
\pgfsetdash{}{0pt}%
\pgfpathmoveto{\pgfqpoint{1.955025in}{2.095329in}}%
\pgfusepath{stroke}%
\end{pgfscope}%
\begin{pgfscope}%
\pgfpathrectangle{\pgfqpoint{0.647939in}{0.492442in}}{\pgfqpoint{4.273799in}{2.331163in}}%
\pgfusepath{clip}%
\pgfsetroundcap%
\pgfsetroundjoin%
\definecolor{currentfill}{rgb}{0.500000,0.500000,0.500000}%
\pgfsetfillcolor{currentfill}%
\pgfsetfillopacity{0.300000}%
\pgfsetlinewidth{0.301125pt}%
\definecolor{currentstroke}{rgb}{0.500000,0.500000,0.500000}%
\pgfsetstrokecolor{currentstroke}%
\pgfsetstrokeopacity{0.300000}%
\pgfsetdash{}{0pt}%
\pgfpathmoveto{\pgfqpoint{0.000000in}{0.000000in}}%
\pgfpathlineto{\pgfqpoint{0.000000in}{0.000000in}}%
\pgfpathclose%
\pgfusepath{stroke,fill}%
\end{pgfscope}%
\begin{pgfscope}%
\pgfpathrectangle{\pgfqpoint{0.647939in}{0.492442in}}{\pgfqpoint{4.273799in}{2.331163in}}%
\pgfusepath{clip}%
\pgfsetroundcap%
\pgfsetroundjoin%
\pgfsetlinewidth{0.301125pt}%
\definecolor{currentstroke}{rgb}{0.500000,0.500000,0.500000}%
\pgfsetstrokecolor{currentstroke}%
\pgfsetstrokeopacity{0.300000}%
\pgfsetdash{}{0pt}%
\pgfpathmoveto{\pgfqpoint{2.339284in}{1.447652in}}%
\pgfusepath{stroke}%
\end{pgfscope}%
\begin{pgfscope}%
\pgfpathrectangle{\pgfqpoint{0.647939in}{0.492442in}}{\pgfqpoint{4.273799in}{2.331163in}}%
\pgfusepath{clip}%
\pgfsetroundcap%
\pgfsetroundjoin%
\definecolor{currentfill}{rgb}{0.500000,0.500000,0.500000}%
\pgfsetfillcolor{currentfill}%
\pgfsetfillopacity{0.300000}%
\pgfsetlinewidth{0.301125pt}%
\definecolor{currentstroke}{rgb}{0.500000,0.500000,0.500000}%
\pgfsetstrokecolor{currentstroke}%
\pgfsetstrokeopacity{0.300000}%
\pgfsetdash{}{0pt}%
\pgfpathmoveto{\pgfqpoint{0.000000in}{0.000000in}}%
\pgfpathlineto{\pgfqpoint{0.000000in}{0.000000in}}%
\pgfpathclose%
\pgfusepath{stroke,fill}%
\end{pgfscope}%
\begin{pgfscope}%
\pgfpathrectangle{\pgfqpoint{0.647939in}{0.492442in}}{\pgfqpoint{4.273799in}{2.331163in}}%
\pgfusepath{clip}%
\pgfsetroundcap%
\pgfsetroundjoin%
\pgfsetlinewidth{0.301125pt}%
\definecolor{currentstroke}{rgb}{0.500000,0.500000,0.500000}%
\pgfsetstrokecolor{currentstroke}%
\pgfsetstrokeopacity{0.300000}%
\pgfsetdash{}{0pt}%
\pgfpathmoveto{\pgfqpoint{2.928958in}{0.857275in}}%
\pgfusepath{stroke}%
\end{pgfscope}%
\begin{pgfscope}%
\pgfpathrectangle{\pgfqpoint{0.647939in}{0.492442in}}{\pgfqpoint{4.273799in}{2.331163in}}%
\pgfusepath{clip}%
\pgfsetroundcap%
\pgfsetroundjoin%
\definecolor{currentfill}{rgb}{0.500000,0.500000,0.500000}%
\pgfsetfillcolor{currentfill}%
\pgfsetfillopacity{0.300000}%
\pgfsetlinewidth{0.301125pt}%
\definecolor{currentstroke}{rgb}{0.500000,0.500000,0.500000}%
\pgfsetstrokecolor{currentstroke}%
\pgfsetstrokeopacity{0.300000}%
\pgfsetdash{}{0pt}%
\pgfpathmoveto{\pgfqpoint{0.000000in}{0.000000in}}%
\pgfpathlineto{\pgfqpoint{0.000000in}{0.000000in}}%
\pgfpathclose%
\pgfusepath{stroke,fill}%
\end{pgfscope}%
\begin{pgfscope}%
\pgfpathrectangle{\pgfqpoint{0.647939in}{0.492442in}}{\pgfqpoint{4.273799in}{2.331163in}}%
\pgfusepath{clip}%
\pgfsetroundcap%
\pgfsetroundjoin%
\pgfsetlinewidth{0.301125pt}%
\definecolor{currentstroke}{rgb}{0.500000,0.500000,0.500000}%
\pgfsetstrokecolor{currentstroke}%
\pgfsetstrokeopacity{0.300000}%
\pgfsetdash{}{0pt}%
\pgfpathmoveto{\pgfqpoint{3.135510in}{0.797402in}}%
\pgfusepath{stroke}%
\end{pgfscope}%
\begin{pgfscope}%
\pgfpathrectangle{\pgfqpoint{0.647939in}{0.492442in}}{\pgfqpoint{4.273799in}{2.331163in}}%
\pgfusepath{clip}%
\pgfsetroundcap%
\pgfsetroundjoin%
\definecolor{currentfill}{rgb}{0.500000,0.500000,0.500000}%
\pgfsetfillcolor{currentfill}%
\pgfsetfillopacity{0.300000}%
\pgfsetlinewidth{0.301125pt}%
\definecolor{currentstroke}{rgb}{0.500000,0.500000,0.500000}%
\pgfsetstrokecolor{currentstroke}%
\pgfsetstrokeopacity{0.300000}%
\pgfsetdash{}{0pt}%
\pgfpathmoveto{\pgfqpoint{0.000000in}{0.000000in}}%
\pgfpathlineto{\pgfqpoint{0.000000in}{0.000000in}}%
\pgfpathclose%
\pgfusepath{stroke,fill}%
\end{pgfscope}%
\begin{pgfscope}%
\pgfpathrectangle{\pgfqpoint{0.647939in}{0.492442in}}{\pgfqpoint{4.273799in}{2.331163in}}%
\pgfusepath{clip}%
\pgfsetroundcap%
\pgfsetroundjoin%
\pgfsetlinewidth{0.301125pt}%
\definecolor{currentstroke}{rgb}{0.500000,0.500000,0.500000}%
\pgfsetstrokecolor{currentstroke}%
\pgfsetstrokeopacity{0.300000}%
\pgfsetdash{}{0pt}%
\pgfpathmoveto{\pgfqpoint{2.715291in}{1.318999in}}%
\pgfusepath{stroke}%
\end{pgfscope}%
\begin{pgfscope}%
\pgfpathrectangle{\pgfqpoint{0.647939in}{0.492442in}}{\pgfqpoint{4.273799in}{2.331163in}}%
\pgfusepath{clip}%
\pgfsetroundcap%
\pgfsetroundjoin%
\definecolor{currentfill}{rgb}{0.500000,0.500000,0.500000}%
\pgfsetfillcolor{currentfill}%
\pgfsetfillopacity{0.300000}%
\pgfsetlinewidth{0.301125pt}%
\definecolor{currentstroke}{rgb}{0.500000,0.500000,0.500000}%
\pgfsetstrokecolor{currentstroke}%
\pgfsetstrokeopacity{0.300000}%
\pgfsetdash{}{0pt}%
\pgfpathmoveto{\pgfqpoint{0.000000in}{0.000000in}}%
\pgfpathlineto{\pgfqpoint{0.000000in}{0.000000in}}%
\pgfpathclose%
\pgfusepath{stroke,fill}%
\end{pgfscope}%
\begin{pgfscope}%
\pgfpathrectangle{\pgfqpoint{0.647939in}{0.492442in}}{\pgfqpoint{4.273799in}{2.331163in}}%
\pgfusepath{clip}%
\pgfsetroundcap%
\pgfsetroundjoin%
\pgfsetlinewidth{0.301125pt}%
\definecolor{currentstroke}{rgb}{0.500000,0.500000,0.500000}%
\pgfsetstrokecolor{currentstroke}%
\pgfsetstrokeopacity{0.300000}%
\pgfsetdash{}{0pt}%
\pgfpathmoveto{\pgfqpoint{3.703411in}{0.584280in}}%
\pgfusepath{stroke}%
\end{pgfscope}%
\begin{pgfscope}%
\pgfpathrectangle{\pgfqpoint{0.647939in}{0.492442in}}{\pgfqpoint{4.273799in}{2.331163in}}%
\pgfusepath{clip}%
\pgfsetroundcap%
\pgfsetroundjoin%
\definecolor{currentfill}{rgb}{0.500000,0.500000,0.500000}%
\pgfsetfillcolor{currentfill}%
\pgfsetfillopacity{0.300000}%
\pgfsetlinewidth{0.301125pt}%
\definecolor{currentstroke}{rgb}{0.500000,0.500000,0.500000}%
\pgfsetstrokecolor{currentstroke}%
\pgfsetstrokeopacity{0.300000}%
\pgfsetdash{}{0pt}%
\pgfpathmoveto{\pgfqpoint{0.000000in}{0.000000in}}%
\pgfpathlineto{\pgfqpoint{0.000000in}{0.000000in}}%
\pgfpathclose%
\pgfusepath{stroke,fill}%
\end{pgfscope}%
\begin{pgfscope}%
\pgfpathrectangle{\pgfqpoint{0.647939in}{0.492442in}}{\pgfqpoint{4.273799in}{2.331163in}}%
\pgfusepath{clip}%
\pgfsetroundcap%
\pgfsetroundjoin%
\pgfsetlinewidth{0.301125pt}%
\definecolor{currentstroke}{rgb}{0.500000,0.500000,0.500000}%
\pgfsetstrokecolor{currentstroke}%
\pgfsetstrokeopacity{0.300000}%
\pgfsetdash{}{0pt}%
\pgfpathmoveto{\pgfqpoint{3.802173in}{0.584982in}}%
\pgfusepath{stroke}%
\end{pgfscope}%
\begin{pgfscope}%
\pgfpathrectangle{\pgfqpoint{0.647939in}{0.492442in}}{\pgfqpoint{4.273799in}{2.331163in}}%
\pgfusepath{clip}%
\pgfsetroundcap%
\pgfsetroundjoin%
\definecolor{currentfill}{rgb}{0.500000,0.500000,0.500000}%
\pgfsetfillcolor{currentfill}%
\pgfsetfillopacity{0.300000}%
\pgfsetlinewidth{0.301125pt}%
\definecolor{currentstroke}{rgb}{0.500000,0.500000,0.500000}%
\pgfsetstrokecolor{currentstroke}%
\pgfsetstrokeopacity{0.300000}%
\pgfsetdash{}{0pt}%
\pgfpathmoveto{\pgfqpoint{0.000000in}{0.000000in}}%
\pgfpathlineto{\pgfqpoint{0.000000in}{0.000000in}}%
\pgfpathclose%
\pgfusepath{stroke,fill}%
\end{pgfscope}%
\begin{pgfscope}%
\pgfpathrectangle{\pgfqpoint{0.647939in}{0.492442in}}{\pgfqpoint{4.273799in}{2.331163in}}%
\pgfusepath{clip}%
\pgfsetroundcap%
\pgfsetroundjoin%
\pgfsetlinewidth{0.301125pt}%
\definecolor{currentstroke}{rgb}{0.500000,0.500000,0.500000}%
\pgfsetstrokecolor{currentstroke}%
\pgfsetstrokeopacity{0.300000}%
\pgfsetdash{}{0pt}%
\pgfpathmoveto{\pgfqpoint{3.903126in}{0.586939in}}%
\pgfusepath{stroke}%
\end{pgfscope}%
\begin{pgfscope}%
\pgfpathrectangle{\pgfqpoint{0.647939in}{0.492442in}}{\pgfqpoint{4.273799in}{2.331163in}}%
\pgfusepath{clip}%
\pgfsetroundcap%
\pgfsetroundjoin%
\definecolor{currentfill}{rgb}{0.500000,0.500000,0.500000}%
\pgfsetfillcolor{currentfill}%
\pgfsetfillopacity{0.300000}%
\pgfsetlinewidth{0.301125pt}%
\definecolor{currentstroke}{rgb}{0.500000,0.500000,0.500000}%
\pgfsetstrokecolor{currentstroke}%
\pgfsetstrokeopacity{0.300000}%
\pgfsetdash{}{0pt}%
\pgfpathmoveto{\pgfqpoint{0.000000in}{0.000000in}}%
\pgfpathlineto{\pgfqpoint{0.000000in}{0.000000in}}%
\pgfpathclose%
\pgfusepath{stroke,fill}%
\end{pgfscope}%
\begin{pgfscope}%
\pgfpathrectangle{\pgfqpoint{0.647939in}{0.492442in}}{\pgfqpoint{4.273799in}{2.331163in}}%
\pgfusepath{clip}%
\pgfsetroundcap%
\pgfsetroundjoin%
\pgfsetlinewidth{0.301125pt}%
\definecolor{currentstroke}{rgb}{0.500000,0.500000,0.500000}%
\pgfsetstrokecolor{currentstroke}%
\pgfsetstrokeopacity{0.300000}%
\pgfsetdash{}{0pt}%
\pgfpathmoveto{\pgfqpoint{3.035663in}{1.249052in}}%
\pgfusepath{stroke}%
\end{pgfscope}%
\begin{pgfscope}%
\pgfpathrectangle{\pgfqpoint{0.647939in}{0.492442in}}{\pgfqpoint{4.273799in}{2.331163in}}%
\pgfusepath{clip}%
\pgfsetroundcap%
\pgfsetroundjoin%
\definecolor{currentfill}{rgb}{0.500000,0.500000,0.500000}%
\pgfsetfillcolor{currentfill}%
\pgfsetfillopacity{0.300000}%
\pgfsetlinewidth{0.301125pt}%
\definecolor{currentstroke}{rgb}{0.500000,0.500000,0.500000}%
\pgfsetstrokecolor{currentstroke}%
\pgfsetstrokeopacity{0.300000}%
\pgfsetdash{}{0pt}%
\pgfpathmoveto{\pgfqpoint{0.000000in}{0.000000in}}%
\pgfpathlineto{\pgfqpoint{0.000000in}{0.000000in}}%
\pgfpathclose%
\pgfusepath{stroke,fill}%
\end{pgfscope}%
\begin{pgfscope}%
\pgfpathrectangle{\pgfqpoint{0.647939in}{0.492442in}}{\pgfqpoint{4.273799in}{2.331163in}}%
\pgfusepath{clip}%
\pgfsetroundcap%
\pgfsetroundjoin%
\pgfsetlinewidth{0.301125pt}%
\definecolor{currentstroke}{rgb}{0.500000,0.500000,0.500000}%
\pgfsetstrokecolor{currentstroke}%
\pgfsetstrokeopacity{0.300000}%
\pgfsetdash{}{0pt}%
\pgfpathmoveto{\pgfqpoint{3.644005in}{1.023369in}}%
\pgfusepath{stroke}%
\end{pgfscope}%
\begin{pgfscope}%
\pgfpathrectangle{\pgfqpoint{0.647939in}{0.492442in}}{\pgfqpoint{4.273799in}{2.331163in}}%
\pgfusepath{clip}%
\pgfsetroundcap%
\pgfsetroundjoin%
\definecolor{currentfill}{rgb}{0.500000,0.500000,0.500000}%
\pgfsetfillcolor{currentfill}%
\pgfsetfillopacity{0.300000}%
\pgfsetlinewidth{0.301125pt}%
\definecolor{currentstroke}{rgb}{0.500000,0.500000,0.500000}%
\pgfsetstrokecolor{currentstroke}%
\pgfsetstrokeopacity{0.300000}%
\pgfsetdash{}{0pt}%
\pgfpathmoveto{\pgfqpoint{0.000000in}{0.000000in}}%
\pgfpathlineto{\pgfqpoint{0.000000in}{0.000000in}}%
\pgfpathclose%
\pgfusepath{stroke,fill}%
\end{pgfscope}%
\begin{pgfscope}%
\pgfpathrectangle{\pgfqpoint{0.647939in}{0.492442in}}{\pgfqpoint{4.273799in}{2.331163in}}%
\pgfusepath{clip}%
\pgfsetroundcap%
\pgfsetroundjoin%
\pgfsetlinewidth{0.301125pt}%
\definecolor{currentstroke}{rgb}{0.500000,0.500000,0.500000}%
\pgfsetstrokecolor{currentstroke}%
\pgfsetstrokeopacity{0.300000}%
\pgfsetdash{}{0pt}%
\pgfpathmoveto{\pgfqpoint{3.879924in}{1.015803in}}%
\pgfusepath{stroke}%
\end{pgfscope}%
\begin{pgfscope}%
\pgfpathrectangle{\pgfqpoint{0.647939in}{0.492442in}}{\pgfqpoint{4.273799in}{2.331163in}}%
\pgfusepath{clip}%
\pgfsetroundcap%
\pgfsetroundjoin%
\definecolor{currentfill}{rgb}{0.500000,0.500000,0.500000}%
\pgfsetfillcolor{currentfill}%
\pgfsetfillopacity{0.300000}%
\pgfsetlinewidth{0.301125pt}%
\definecolor{currentstroke}{rgb}{0.500000,0.500000,0.500000}%
\pgfsetstrokecolor{currentstroke}%
\pgfsetstrokeopacity{0.300000}%
\pgfsetdash{}{0pt}%
\pgfpathmoveto{\pgfqpoint{0.000000in}{0.000000in}}%
\pgfpathlineto{\pgfqpoint{0.000000in}{0.000000in}}%
\pgfpathclose%
\pgfusepath{stroke,fill}%
\end{pgfscope}%
\begin{pgfscope}%
\pgfpathrectangle{\pgfqpoint{0.647939in}{0.492442in}}{\pgfqpoint{4.273799in}{2.331163in}}%
\pgfusepath{clip}%
\pgfsetroundcap%
\pgfsetroundjoin%
\pgfsetlinewidth{0.301125pt}%
\definecolor{currentstroke}{rgb}{0.500000,0.500000,0.500000}%
\pgfsetstrokecolor{currentstroke}%
\pgfsetstrokeopacity{0.300000}%
\pgfsetdash{}{0pt}%
\pgfpathmoveto{\pgfqpoint{3.525678in}{1.255735in}}%
\pgfusepath{stroke}%
\end{pgfscope}%
\begin{pgfscope}%
\pgfpathrectangle{\pgfqpoint{0.647939in}{0.492442in}}{\pgfqpoint{4.273799in}{2.331163in}}%
\pgfusepath{clip}%
\pgfsetroundcap%
\pgfsetroundjoin%
\definecolor{currentfill}{rgb}{0.500000,0.500000,0.500000}%
\pgfsetfillcolor{currentfill}%
\pgfsetfillopacity{0.300000}%
\pgfsetlinewidth{0.301125pt}%
\definecolor{currentstroke}{rgb}{0.500000,0.500000,0.500000}%
\pgfsetstrokecolor{currentstroke}%
\pgfsetstrokeopacity{0.300000}%
\pgfsetdash{}{0pt}%
\pgfpathmoveto{\pgfqpoint{0.000000in}{0.000000in}}%
\pgfpathlineto{\pgfqpoint{0.000000in}{0.000000in}}%
\pgfpathclose%
\pgfusepath{stroke,fill}%
\end{pgfscope}%
\begin{pgfscope}%
\pgfpathrectangle{\pgfqpoint{0.647939in}{0.492442in}}{\pgfqpoint{4.273799in}{2.331163in}}%
\pgfusepath{clip}%
\pgfsetroundcap%
\pgfsetroundjoin%
\pgfsetlinewidth{0.301125pt}%
\definecolor{currentstroke}{rgb}{0.500000,0.500000,0.500000}%
\pgfsetstrokecolor{currentstroke}%
\pgfsetstrokeopacity{0.300000}%
\pgfsetdash{}{0pt}%
\pgfpathmoveto{\pgfqpoint{4.002929in}{1.226595in}}%
\pgfusepath{stroke}%
\end{pgfscope}%
\begin{pgfscope}%
\pgfpathrectangle{\pgfqpoint{0.647939in}{0.492442in}}{\pgfqpoint{4.273799in}{2.331163in}}%
\pgfusepath{clip}%
\pgfsetroundcap%
\pgfsetroundjoin%
\definecolor{currentfill}{rgb}{0.500000,0.500000,0.500000}%
\pgfsetfillcolor{currentfill}%
\pgfsetfillopacity{0.300000}%
\pgfsetlinewidth{0.301125pt}%
\definecolor{currentstroke}{rgb}{0.500000,0.500000,0.500000}%
\pgfsetstrokecolor{currentstroke}%
\pgfsetstrokeopacity{0.300000}%
\pgfsetdash{}{0pt}%
\pgfpathmoveto{\pgfqpoint{0.000000in}{0.000000in}}%
\pgfpathlineto{\pgfqpoint{0.000000in}{0.000000in}}%
\pgfpathclose%
\pgfusepath{stroke,fill}%
\end{pgfscope}%
\begin{pgfscope}%
\pgfpathrectangle{\pgfqpoint{0.647939in}{0.492442in}}{\pgfqpoint{4.273799in}{2.331163in}}%
\pgfusepath{clip}%
\pgfsetroundcap%
\pgfsetroundjoin%
\pgfsetlinewidth{0.301125pt}%
\definecolor{currentstroke}{rgb}{0.500000,0.500000,0.500000}%
\pgfsetstrokecolor{currentstroke}%
\pgfsetstrokeopacity{0.300000}%
\pgfsetdash{}{0pt}%
\pgfpathmoveto{\pgfqpoint{4.058204in}{1.382903in}}%
\pgfusepath{stroke}%
\end{pgfscope}%
\begin{pgfscope}%
\pgfpathrectangle{\pgfqpoint{0.647939in}{0.492442in}}{\pgfqpoint{4.273799in}{2.331163in}}%
\pgfusepath{clip}%
\pgfsetroundcap%
\pgfsetroundjoin%
\definecolor{currentfill}{rgb}{0.500000,0.500000,0.500000}%
\pgfsetfillcolor{currentfill}%
\pgfsetfillopacity{0.300000}%
\pgfsetlinewidth{0.301125pt}%
\definecolor{currentstroke}{rgb}{0.500000,0.500000,0.500000}%
\pgfsetstrokecolor{currentstroke}%
\pgfsetstrokeopacity{0.300000}%
\pgfsetdash{}{0pt}%
\pgfpathmoveto{\pgfqpoint{0.000000in}{0.000000in}}%
\pgfpathlineto{\pgfqpoint{0.000000in}{0.000000in}}%
\pgfpathclose%
\pgfusepath{stroke,fill}%
\end{pgfscope}%
\begin{pgfscope}%
\pgfpathrectangle{\pgfqpoint{0.647939in}{0.492442in}}{\pgfqpoint{4.273799in}{2.331163in}}%
\pgfusepath{clip}%
\pgfsetroundcap%
\pgfsetroundjoin%
\pgfsetlinewidth{0.301125pt}%
\definecolor{currentstroke}{rgb}{0.500000,0.500000,0.500000}%
\pgfsetstrokecolor{currentstroke}%
\pgfsetstrokeopacity{0.300000}%
\pgfsetdash{}{0pt}%
\pgfpathmoveto{\pgfqpoint{4.318599in}{1.493350in}}%
\pgfusepath{stroke}%
\end{pgfscope}%
\begin{pgfscope}%
\pgfpathrectangle{\pgfqpoint{0.647939in}{0.492442in}}{\pgfqpoint{4.273799in}{2.331163in}}%
\pgfusepath{clip}%
\pgfsetroundcap%
\pgfsetroundjoin%
\definecolor{currentfill}{rgb}{0.500000,0.500000,0.500000}%
\pgfsetfillcolor{currentfill}%
\pgfsetfillopacity{0.300000}%
\pgfsetlinewidth{0.301125pt}%
\definecolor{currentstroke}{rgb}{0.500000,0.500000,0.500000}%
\pgfsetstrokecolor{currentstroke}%
\pgfsetstrokeopacity{0.300000}%
\pgfsetdash{}{0pt}%
\pgfpathmoveto{\pgfqpoint{0.000000in}{0.000000in}}%
\pgfpathlineto{\pgfqpoint{0.000000in}{0.000000in}}%
\pgfpathclose%
\pgfusepath{stroke,fill}%
\end{pgfscope}%
\begin{pgfscope}%
\pgfpathrectangle{\pgfqpoint{0.647939in}{0.492442in}}{\pgfqpoint{4.273799in}{2.331163in}}%
\pgfusepath{clip}%
\pgfsetroundcap%
\pgfsetroundjoin%
\pgfsetlinewidth{0.301125pt}%
\definecolor{currentstroke}{rgb}{0.500000,0.500000,0.500000}%
\pgfsetstrokecolor{currentstroke}%
\pgfsetstrokeopacity{0.300000}%
\pgfsetdash{}{0pt}%
\pgfpathmoveto{\pgfqpoint{4.421741in}{1.727896in}}%
\pgfusepath{stroke}%
\end{pgfscope}%
\begin{pgfscope}%
\pgfpathrectangle{\pgfqpoint{0.647939in}{0.492442in}}{\pgfqpoint{4.273799in}{2.331163in}}%
\pgfusepath{clip}%
\pgfsetroundcap%
\pgfsetroundjoin%
\definecolor{currentfill}{rgb}{0.500000,0.500000,0.500000}%
\pgfsetfillcolor{currentfill}%
\pgfsetfillopacity{0.300000}%
\pgfsetlinewidth{0.301125pt}%
\definecolor{currentstroke}{rgb}{0.500000,0.500000,0.500000}%
\pgfsetstrokecolor{currentstroke}%
\pgfsetstrokeopacity{0.300000}%
\pgfsetdash{}{0pt}%
\pgfpathmoveto{\pgfqpoint{0.000000in}{0.000000in}}%
\pgfpathlineto{\pgfqpoint{0.000000in}{0.000000in}}%
\pgfpathclose%
\pgfusepath{stroke,fill}%
\end{pgfscope}%
\begin{pgfscope}%
\pgfpathrectangle{\pgfqpoint{0.647939in}{0.492442in}}{\pgfqpoint{4.273799in}{2.331163in}}%
\pgfusepath{clip}%
\pgfsetroundcap%
\pgfsetroundjoin%
\pgfsetlinewidth{0.301125pt}%
\definecolor{currentstroke}{rgb}{0.500000,0.500000,0.500000}%
\pgfsetstrokecolor{currentstroke}%
\pgfsetstrokeopacity{0.300000}%
\pgfsetdash{}{0pt}%
\pgfpathmoveto{\pgfqpoint{4.579107in}{1.882524in}}%
\pgfusepath{stroke}%
\end{pgfscope}%
\begin{pgfscope}%
\pgfpathrectangle{\pgfqpoint{0.647939in}{0.492442in}}{\pgfqpoint{4.273799in}{2.331163in}}%
\pgfusepath{clip}%
\pgfsetroundcap%
\pgfsetroundjoin%
\definecolor{currentfill}{rgb}{0.500000,0.500000,0.500000}%
\pgfsetfillcolor{currentfill}%
\pgfsetfillopacity{0.300000}%
\pgfsetlinewidth{0.301125pt}%
\definecolor{currentstroke}{rgb}{0.500000,0.500000,0.500000}%
\pgfsetstrokecolor{currentstroke}%
\pgfsetstrokeopacity{0.300000}%
\pgfsetdash{}{0pt}%
\pgfpathmoveto{\pgfqpoint{0.000000in}{0.000000in}}%
\pgfpathlineto{\pgfqpoint{0.000000in}{0.000000in}}%
\pgfpathclose%
\pgfusepath{stroke,fill}%
\end{pgfscope}%
\begin{pgfscope}%
\pgfpathrectangle{\pgfqpoint{0.647939in}{0.492442in}}{\pgfqpoint{4.273799in}{2.331163in}}%
\pgfusepath{clip}%
\pgfsetroundcap%
\pgfsetroundjoin%
\pgfsetlinewidth{0.301125pt}%
\definecolor{currentstroke}{rgb}{0.500000,0.500000,0.500000}%
\pgfsetstrokecolor{currentstroke}%
\pgfsetstrokeopacity{0.300000}%
\pgfsetdash{}{0pt}%
\pgfpathmoveto{\pgfqpoint{4.747220in}{1.847712in}}%
\pgfusepath{stroke}%
\end{pgfscope}%
\begin{pgfscope}%
\pgfpathrectangle{\pgfqpoint{0.647939in}{0.492442in}}{\pgfqpoint{4.273799in}{2.331163in}}%
\pgfusepath{clip}%
\pgfsetroundcap%
\pgfsetroundjoin%
\definecolor{currentfill}{rgb}{0.500000,0.500000,0.500000}%
\pgfsetfillcolor{currentfill}%
\pgfsetfillopacity{0.300000}%
\pgfsetlinewidth{0.301125pt}%
\definecolor{currentstroke}{rgb}{0.500000,0.500000,0.500000}%
\pgfsetstrokecolor{currentstroke}%
\pgfsetstrokeopacity{0.300000}%
\pgfsetdash{}{0pt}%
\pgfpathmoveto{\pgfqpoint{0.000000in}{0.000000in}}%
\pgfpathlineto{\pgfqpoint{0.000000in}{0.000000in}}%
\pgfpathclose%
\pgfusepath{stroke,fill}%
\end{pgfscope}%
\begin{pgfscope}%
\pgfpathrectangle{\pgfqpoint{0.647939in}{0.492442in}}{\pgfqpoint{4.273799in}{2.331163in}}%
\pgfusepath{clip}%
\pgfsetroundcap%
\pgfsetroundjoin%
\pgfsetlinewidth{0.301125pt}%
\definecolor{currentstroke}{rgb}{0.500000,0.500000,0.500000}%
\pgfsetstrokecolor{currentstroke}%
\pgfsetstrokeopacity{0.300000}%
\pgfsetdash{}{0pt}%
\pgfpathmoveto{\pgfqpoint{4.856875in}{1.859921in}}%
\pgfusepath{stroke}%
\end{pgfscope}%
\begin{pgfscope}%
\pgfpathrectangle{\pgfqpoint{0.647939in}{0.492442in}}{\pgfqpoint{4.273799in}{2.331163in}}%
\pgfusepath{clip}%
\pgfsetroundcap%
\pgfsetroundjoin%
\definecolor{currentfill}{rgb}{0.500000,0.500000,0.500000}%
\pgfsetfillcolor{currentfill}%
\pgfsetfillopacity{0.300000}%
\pgfsetlinewidth{0.301125pt}%
\definecolor{currentstroke}{rgb}{0.500000,0.500000,0.500000}%
\pgfsetstrokecolor{currentstroke}%
\pgfsetstrokeopacity{0.300000}%
\pgfsetdash{}{0pt}%
\pgfpathmoveto{\pgfqpoint{0.000000in}{0.000000in}}%
\pgfpathlineto{\pgfqpoint{0.000000in}{0.000000in}}%
\pgfpathclose%
\pgfusepath{stroke,fill}%
\end{pgfscope}%
\begin{pgfscope}%
\pgfpathrectangle{\pgfqpoint{0.647939in}{0.492442in}}{\pgfqpoint{4.273799in}{2.331163in}}%
\pgfusepath{clip}%
\pgfsetroundcap%
\pgfsetroundjoin%
\pgfsetlinewidth{0.301125pt}%
\definecolor{currentstroke}{rgb}{0.500000,0.500000,0.500000}%
\pgfsetstrokecolor{currentstroke}%
\pgfsetstrokeopacity{0.300000}%
\pgfsetdash{}{0pt}%
\pgfpathmoveto{\pgfqpoint{4.898460in}{2.022653in}}%
\pgfusepath{stroke}%
\end{pgfscope}%
\begin{pgfscope}%
\pgfpathrectangle{\pgfqpoint{0.647939in}{0.492442in}}{\pgfqpoint{4.273799in}{2.331163in}}%
\pgfusepath{clip}%
\pgfsetroundcap%
\pgfsetroundjoin%
\definecolor{currentfill}{rgb}{0.500000,0.500000,0.500000}%
\pgfsetfillcolor{currentfill}%
\pgfsetfillopacity{0.300000}%
\pgfsetlinewidth{0.301125pt}%
\definecolor{currentstroke}{rgb}{0.500000,0.500000,0.500000}%
\pgfsetstrokecolor{currentstroke}%
\pgfsetstrokeopacity{0.300000}%
\pgfsetdash{}{0pt}%
\pgfpathmoveto{\pgfqpoint{0.000000in}{0.000000in}}%
\pgfpathlineto{\pgfqpoint{0.000000in}{0.000000in}}%
\pgfpathclose%
\pgfusepath{stroke,fill}%
\end{pgfscope}%
\begin{pgfscope}%
\pgfpathrectangle{\pgfqpoint{0.647939in}{0.492442in}}{\pgfqpoint{4.273799in}{2.331163in}}%
\pgfusepath{clip}%
\pgfsetroundcap%
\pgfsetroundjoin%
\pgfsetlinewidth{0.301125pt}%
\definecolor{currentstroke}{rgb}{0.500000,0.500000,0.500000}%
\pgfsetstrokecolor{currentstroke}%
\pgfsetstrokeopacity{0.300000}%
\pgfsetdash{}{0pt}%
\pgfpathmoveto{\pgfqpoint{4.572263in}{2.677305in}}%
\pgfusepath{stroke}%
\end{pgfscope}%
\begin{pgfscope}%
\pgfpathrectangle{\pgfqpoint{0.647939in}{0.492442in}}{\pgfqpoint{4.273799in}{2.331163in}}%
\pgfusepath{clip}%
\pgfsetroundcap%
\pgfsetroundjoin%
\definecolor{currentfill}{rgb}{0.500000,0.500000,0.500000}%
\pgfsetfillcolor{currentfill}%
\pgfsetfillopacity{0.300000}%
\pgfsetlinewidth{0.301125pt}%
\definecolor{currentstroke}{rgb}{0.500000,0.500000,0.500000}%
\pgfsetstrokecolor{currentstroke}%
\pgfsetstrokeopacity{0.300000}%
\pgfsetdash{}{0pt}%
\pgfpathmoveto{\pgfqpoint{0.000000in}{0.000000in}}%
\pgfpathlineto{\pgfqpoint{0.000000in}{0.000000in}}%
\pgfpathclose%
\pgfusepath{stroke,fill}%
\end{pgfscope}%
\begin{pgfscope}%
\pgfpathrectangle{\pgfqpoint{0.647939in}{0.492442in}}{\pgfqpoint{4.273799in}{2.331163in}}%
\pgfusepath{clip}%
\pgfsetroundcap%
\pgfsetroundjoin%
\pgfsetlinewidth{0.301125pt}%
\definecolor{currentstroke}{rgb}{0.500000,0.500000,0.500000}%
\pgfsetstrokecolor{currentstroke}%
\pgfsetstrokeopacity{0.300000}%
\pgfsetdash{}{0pt}%
\pgfpathmoveto{\pgfqpoint{4.593430in}{2.755097in}}%
\pgfusepath{stroke}%
\end{pgfscope}%
\begin{pgfscope}%
\pgfpathrectangle{\pgfqpoint{0.647939in}{0.492442in}}{\pgfqpoint{4.273799in}{2.331163in}}%
\pgfusepath{clip}%
\pgfsetroundcap%
\pgfsetroundjoin%
\definecolor{currentfill}{rgb}{0.500000,0.500000,0.500000}%
\pgfsetfillcolor{currentfill}%
\pgfsetfillopacity{0.300000}%
\pgfsetlinewidth{0.301125pt}%
\definecolor{currentstroke}{rgb}{0.500000,0.500000,0.500000}%
\pgfsetstrokecolor{currentstroke}%
\pgfsetstrokeopacity{0.300000}%
\pgfsetdash{}{0pt}%
\pgfpathmoveto{\pgfqpoint{0.000000in}{0.000000in}}%
\pgfpathlineto{\pgfqpoint{0.000000in}{0.000000in}}%
\pgfpathclose%
\pgfusepath{stroke,fill}%
\end{pgfscope}%
\begin{pgfscope}%
\pgfpathrectangle{\pgfqpoint{0.647939in}{0.492442in}}{\pgfqpoint{4.273799in}{2.331163in}}%
\pgfusepath{clip}%
\pgfsetroundcap%
\pgfsetroundjoin%
\pgfsetlinewidth{0.301125pt}%
\definecolor{currentstroke}{rgb}{0.500000,0.500000,0.500000}%
\pgfsetstrokecolor{currentstroke}%
\pgfsetstrokeopacity{0.300000}%
\pgfsetdash{}{0pt}%
\pgfpathmoveto{\pgfqpoint{4.431888in}{2.622466in}}%
\pgfusepath{stroke}%
\end{pgfscope}%
\begin{pgfscope}%
\pgfpathrectangle{\pgfqpoint{0.647939in}{0.492442in}}{\pgfqpoint{4.273799in}{2.331163in}}%
\pgfusepath{clip}%
\pgfsetroundcap%
\pgfsetroundjoin%
\definecolor{currentfill}{rgb}{0.500000,0.500000,0.500000}%
\pgfsetfillcolor{currentfill}%
\pgfsetfillopacity{0.300000}%
\pgfsetlinewidth{0.301125pt}%
\definecolor{currentstroke}{rgb}{0.500000,0.500000,0.500000}%
\pgfsetstrokecolor{currentstroke}%
\pgfsetstrokeopacity{0.300000}%
\pgfsetdash{}{0pt}%
\pgfpathmoveto{\pgfqpoint{0.000000in}{0.000000in}}%
\pgfpathlineto{\pgfqpoint{0.000000in}{0.000000in}}%
\pgfpathclose%
\pgfusepath{stroke,fill}%
\end{pgfscope}%
\begin{pgfscope}%
\pgfpathrectangle{\pgfqpoint{0.647939in}{0.492442in}}{\pgfqpoint{4.273799in}{2.331163in}}%
\pgfusepath{clip}%
\pgfsetroundcap%
\pgfsetroundjoin%
\pgfsetlinewidth{0.301125pt}%
\definecolor{currentstroke}{rgb}{0.500000,0.500000,0.500000}%
\pgfsetstrokecolor{currentstroke}%
\pgfsetstrokeopacity{0.300000}%
\pgfsetdash{}{0pt}%
\pgfpathmoveto{\pgfqpoint{4.335660in}{2.561606in}}%
\pgfusepath{stroke}%
\end{pgfscope}%
\begin{pgfscope}%
\pgfpathrectangle{\pgfqpoint{0.647939in}{0.492442in}}{\pgfqpoint{4.273799in}{2.331163in}}%
\pgfusepath{clip}%
\pgfsetroundcap%
\pgfsetroundjoin%
\definecolor{currentfill}{rgb}{0.500000,0.500000,0.500000}%
\pgfsetfillcolor{currentfill}%
\pgfsetfillopacity{0.300000}%
\pgfsetlinewidth{0.301125pt}%
\definecolor{currentstroke}{rgb}{0.500000,0.500000,0.500000}%
\pgfsetstrokecolor{currentstroke}%
\pgfsetstrokeopacity{0.300000}%
\pgfsetdash{}{0pt}%
\pgfpathmoveto{\pgfqpoint{0.000000in}{0.000000in}}%
\pgfpathlineto{\pgfqpoint{0.000000in}{0.000000in}}%
\pgfpathclose%
\pgfusepath{stroke,fill}%
\end{pgfscope}%
\begin{pgfscope}%
\pgfpathrectangle{\pgfqpoint{0.647939in}{0.492442in}}{\pgfqpoint{4.273799in}{2.331163in}}%
\pgfusepath{clip}%
\pgfsetroundcap%
\pgfsetroundjoin%
\pgfsetlinewidth{0.301125pt}%
\definecolor{currentstroke}{rgb}{0.500000,0.500000,0.500000}%
\pgfsetstrokecolor{currentstroke}%
\pgfsetstrokeopacity{0.300000}%
\pgfsetdash{}{0pt}%
\pgfpathmoveto{\pgfqpoint{4.273169in}{2.456538in}}%
\pgfusepath{stroke}%
\end{pgfscope}%
\begin{pgfscope}%
\pgfpathrectangle{\pgfqpoint{0.647939in}{0.492442in}}{\pgfqpoint{4.273799in}{2.331163in}}%
\pgfusepath{clip}%
\pgfsetroundcap%
\pgfsetroundjoin%
\definecolor{currentfill}{rgb}{0.500000,0.500000,0.500000}%
\pgfsetfillcolor{currentfill}%
\pgfsetfillopacity{0.300000}%
\pgfsetlinewidth{0.301125pt}%
\definecolor{currentstroke}{rgb}{0.500000,0.500000,0.500000}%
\pgfsetstrokecolor{currentstroke}%
\pgfsetstrokeopacity{0.300000}%
\pgfsetdash{}{0pt}%
\pgfpathmoveto{\pgfqpoint{0.000000in}{0.000000in}}%
\pgfpathlineto{\pgfqpoint{0.000000in}{0.000000in}}%
\pgfpathclose%
\pgfusepath{stroke,fill}%
\end{pgfscope}%
\begin{pgfscope}%
\pgfpathrectangle{\pgfqpoint{0.647939in}{0.492442in}}{\pgfqpoint{4.273799in}{2.331163in}}%
\pgfusepath{clip}%
\pgfsetroundcap%
\pgfsetroundjoin%
\pgfsetlinewidth{0.301125pt}%
\definecolor{currentstroke}{rgb}{0.500000,0.500000,0.500000}%
\pgfsetstrokecolor{currentstroke}%
\pgfsetstrokeopacity{0.300000}%
\pgfsetdash{}{0pt}%
\pgfpathmoveto{\pgfqpoint{4.209034in}{2.325235in}}%
\pgfusepath{stroke}%
\end{pgfscope}%
\begin{pgfscope}%
\pgfpathrectangle{\pgfqpoint{0.647939in}{0.492442in}}{\pgfqpoint{4.273799in}{2.331163in}}%
\pgfusepath{clip}%
\pgfsetroundcap%
\pgfsetroundjoin%
\definecolor{currentfill}{rgb}{0.500000,0.500000,0.500000}%
\pgfsetfillcolor{currentfill}%
\pgfsetfillopacity{0.300000}%
\pgfsetlinewidth{0.301125pt}%
\definecolor{currentstroke}{rgb}{0.500000,0.500000,0.500000}%
\pgfsetstrokecolor{currentstroke}%
\pgfsetstrokeopacity{0.300000}%
\pgfsetdash{}{0pt}%
\pgfpathmoveto{\pgfqpoint{0.000000in}{0.000000in}}%
\pgfpathlineto{\pgfqpoint{0.000000in}{0.000000in}}%
\pgfpathclose%
\pgfusepath{stroke,fill}%
\end{pgfscope}%
\begin{pgfscope}%
\pgfpathrectangle{\pgfqpoint{0.647939in}{0.492442in}}{\pgfqpoint{4.273799in}{2.331163in}}%
\pgfusepath{clip}%
\pgfsetroundcap%
\pgfsetroundjoin%
\pgfsetlinewidth{0.301125pt}%
\definecolor{currentstroke}{rgb}{0.500000,0.500000,0.500000}%
\pgfsetstrokecolor{currentstroke}%
\pgfsetstrokeopacity{0.300000}%
\pgfsetdash{}{0pt}%
\pgfpathmoveto{\pgfqpoint{4.049489in}{2.423330in}}%
\pgfusepath{stroke}%
\end{pgfscope}%
\begin{pgfscope}%
\pgfpathrectangle{\pgfqpoint{0.647939in}{0.492442in}}{\pgfqpoint{4.273799in}{2.331163in}}%
\pgfusepath{clip}%
\pgfsetroundcap%
\pgfsetroundjoin%
\definecolor{currentfill}{rgb}{0.500000,0.500000,0.500000}%
\pgfsetfillcolor{currentfill}%
\pgfsetfillopacity{0.300000}%
\pgfsetlinewidth{0.301125pt}%
\definecolor{currentstroke}{rgb}{0.500000,0.500000,0.500000}%
\pgfsetstrokecolor{currentstroke}%
\pgfsetstrokeopacity{0.300000}%
\pgfsetdash{}{0pt}%
\pgfpathmoveto{\pgfqpoint{0.000000in}{0.000000in}}%
\pgfpathlineto{\pgfqpoint{0.000000in}{0.000000in}}%
\pgfpathclose%
\pgfusepath{stroke,fill}%
\end{pgfscope}%
\begin{pgfscope}%
\pgfpathrectangle{\pgfqpoint{0.647939in}{0.492442in}}{\pgfqpoint{4.273799in}{2.331163in}}%
\pgfusepath{clip}%
\pgfsetroundcap%
\pgfsetroundjoin%
\pgfsetlinewidth{0.301125pt}%
\definecolor{currentstroke}{rgb}{0.500000,0.500000,0.500000}%
\pgfsetstrokecolor{currentstroke}%
\pgfsetstrokeopacity{0.300000}%
\pgfsetdash{}{0pt}%
\pgfpathmoveto{\pgfqpoint{4.014086in}{2.166262in}}%
\pgfusepath{stroke}%
\end{pgfscope}%
\begin{pgfscope}%
\pgfpathrectangle{\pgfqpoint{0.647939in}{0.492442in}}{\pgfqpoint{4.273799in}{2.331163in}}%
\pgfusepath{clip}%
\pgfsetroundcap%
\pgfsetroundjoin%
\definecolor{currentfill}{rgb}{0.500000,0.500000,0.500000}%
\pgfsetfillcolor{currentfill}%
\pgfsetfillopacity{0.300000}%
\pgfsetlinewidth{0.301125pt}%
\definecolor{currentstroke}{rgb}{0.500000,0.500000,0.500000}%
\pgfsetstrokecolor{currentstroke}%
\pgfsetstrokeopacity{0.300000}%
\pgfsetdash{}{0pt}%
\pgfpathmoveto{\pgfqpoint{0.000000in}{0.000000in}}%
\pgfpathlineto{\pgfqpoint{0.000000in}{0.000000in}}%
\pgfpathclose%
\pgfusepath{stroke,fill}%
\end{pgfscope}%
\begin{pgfscope}%
\pgfpathrectangle{\pgfqpoint{0.647939in}{0.492442in}}{\pgfqpoint{4.273799in}{2.331163in}}%
\pgfusepath{clip}%
\pgfsetroundcap%
\pgfsetroundjoin%
\pgfsetlinewidth{0.301125pt}%
\definecolor{currentstroke}{rgb}{0.500000,0.500000,0.500000}%
\pgfsetstrokecolor{currentstroke}%
\pgfsetstrokeopacity{0.300000}%
\pgfsetdash{}{0pt}%
\pgfpathmoveto{\pgfqpoint{3.878473in}{2.320012in}}%
\pgfusepath{stroke}%
\end{pgfscope}%
\begin{pgfscope}%
\pgfpathrectangle{\pgfqpoint{0.647939in}{0.492442in}}{\pgfqpoint{4.273799in}{2.331163in}}%
\pgfusepath{clip}%
\pgfsetroundcap%
\pgfsetroundjoin%
\definecolor{currentfill}{rgb}{0.500000,0.500000,0.500000}%
\pgfsetfillcolor{currentfill}%
\pgfsetfillopacity{0.300000}%
\pgfsetlinewidth{0.301125pt}%
\definecolor{currentstroke}{rgb}{0.500000,0.500000,0.500000}%
\pgfsetstrokecolor{currentstroke}%
\pgfsetstrokeopacity{0.300000}%
\pgfsetdash{}{0pt}%
\pgfpathmoveto{\pgfqpoint{0.000000in}{0.000000in}}%
\pgfpathlineto{\pgfqpoint{0.000000in}{0.000000in}}%
\pgfpathclose%
\pgfusepath{stroke,fill}%
\end{pgfscope}%
\begin{pgfscope}%
\pgfpathrectangle{\pgfqpoint{0.647939in}{0.492442in}}{\pgfqpoint{4.273799in}{2.331163in}}%
\pgfusepath{clip}%
\pgfsetroundcap%
\pgfsetroundjoin%
\pgfsetlinewidth{0.301125pt}%
\definecolor{currentstroke}{rgb}{0.500000,0.500000,0.500000}%
\pgfsetstrokecolor{currentstroke}%
\pgfsetstrokeopacity{0.300000}%
\pgfsetdash{}{0pt}%
\pgfpathmoveto{\pgfqpoint{3.800768in}{1.959141in}}%
\pgfusepath{stroke}%
\end{pgfscope}%
\begin{pgfscope}%
\pgfpathrectangle{\pgfqpoint{0.647939in}{0.492442in}}{\pgfqpoint{4.273799in}{2.331163in}}%
\pgfusepath{clip}%
\pgfsetroundcap%
\pgfsetroundjoin%
\definecolor{currentfill}{rgb}{0.500000,0.500000,0.500000}%
\pgfsetfillcolor{currentfill}%
\pgfsetfillopacity{0.300000}%
\pgfsetlinewidth{0.301125pt}%
\definecolor{currentstroke}{rgb}{0.500000,0.500000,0.500000}%
\pgfsetstrokecolor{currentstroke}%
\pgfsetstrokeopacity{0.300000}%
\pgfsetdash{}{0pt}%
\pgfpathmoveto{\pgfqpoint{0.000000in}{0.000000in}}%
\pgfpathlineto{\pgfqpoint{0.000000in}{0.000000in}}%
\pgfpathclose%
\pgfusepath{stroke,fill}%
\end{pgfscope}%
\begin{pgfscope}%
\pgfpathrectangle{\pgfqpoint{0.647939in}{0.492442in}}{\pgfqpoint{4.273799in}{2.331163in}}%
\pgfusepath{clip}%
\pgfsetroundcap%
\pgfsetroundjoin%
\pgfsetlinewidth{0.301125pt}%
\definecolor{currentstroke}{rgb}{0.500000,0.500000,0.500000}%
\pgfsetstrokecolor{currentstroke}%
\pgfsetstrokeopacity{0.300000}%
\pgfsetdash{}{0pt}%
\pgfpathmoveto{\pgfqpoint{3.638833in}{2.473791in}}%
\pgfusepath{stroke}%
\end{pgfscope}%
\begin{pgfscope}%
\pgfpathrectangle{\pgfqpoint{0.647939in}{0.492442in}}{\pgfqpoint{4.273799in}{2.331163in}}%
\pgfusepath{clip}%
\pgfsetroundcap%
\pgfsetroundjoin%
\definecolor{currentfill}{rgb}{0.500000,0.500000,0.500000}%
\pgfsetfillcolor{currentfill}%
\pgfsetfillopacity{0.300000}%
\pgfsetlinewidth{0.301125pt}%
\definecolor{currentstroke}{rgb}{0.500000,0.500000,0.500000}%
\pgfsetstrokecolor{currentstroke}%
\pgfsetstrokeopacity{0.300000}%
\pgfsetdash{}{0pt}%
\pgfpathmoveto{\pgfqpoint{0.000000in}{0.000000in}}%
\pgfpathlineto{\pgfqpoint{0.000000in}{0.000000in}}%
\pgfpathclose%
\pgfusepath{stroke,fill}%
\end{pgfscope}%
\begin{pgfscope}%
\pgfpathrectangle{\pgfqpoint{0.647939in}{0.492442in}}{\pgfqpoint{4.273799in}{2.331163in}}%
\pgfusepath{clip}%
\pgfsetroundcap%
\pgfsetroundjoin%
\pgfsetlinewidth{0.301125pt}%
\definecolor{currentstroke}{rgb}{0.500000,0.500000,0.500000}%
\pgfsetstrokecolor{currentstroke}%
\pgfsetstrokeopacity{0.300000}%
\pgfsetdash{}{0pt}%
\pgfpathmoveto{\pgfqpoint{3.492106in}{2.600780in}}%
\pgfusepath{stroke}%
\end{pgfscope}%
\begin{pgfscope}%
\pgfpathrectangle{\pgfqpoint{0.647939in}{0.492442in}}{\pgfqpoint{4.273799in}{2.331163in}}%
\pgfusepath{clip}%
\pgfsetroundcap%
\pgfsetroundjoin%
\definecolor{currentfill}{rgb}{0.500000,0.500000,0.500000}%
\pgfsetfillcolor{currentfill}%
\pgfsetfillopacity{0.300000}%
\pgfsetlinewidth{0.301125pt}%
\definecolor{currentstroke}{rgb}{0.500000,0.500000,0.500000}%
\pgfsetstrokecolor{currentstroke}%
\pgfsetstrokeopacity{0.300000}%
\pgfsetdash{}{0pt}%
\pgfpathmoveto{\pgfqpoint{0.000000in}{0.000000in}}%
\pgfpathlineto{\pgfqpoint{0.000000in}{0.000000in}}%
\pgfpathclose%
\pgfusepath{stroke,fill}%
\end{pgfscope}%
\begin{pgfscope}%
\pgfpathrectangle{\pgfqpoint{0.647939in}{0.492442in}}{\pgfqpoint{4.273799in}{2.331163in}}%
\pgfusepath{clip}%
\pgfsetroundcap%
\pgfsetroundjoin%
\pgfsetlinewidth{0.301125pt}%
\definecolor{currentstroke}{rgb}{0.500000,0.500000,0.500000}%
\pgfsetstrokecolor{currentstroke}%
\pgfsetstrokeopacity{0.300000}%
\pgfsetdash{}{0pt}%
\pgfpathmoveto{\pgfqpoint{3.402257in}{2.602201in}}%
\pgfusepath{stroke}%
\end{pgfscope}%
\begin{pgfscope}%
\pgfpathrectangle{\pgfqpoint{0.647939in}{0.492442in}}{\pgfqpoint{4.273799in}{2.331163in}}%
\pgfusepath{clip}%
\pgfsetroundcap%
\pgfsetroundjoin%
\definecolor{currentfill}{rgb}{0.500000,0.500000,0.500000}%
\pgfsetfillcolor{currentfill}%
\pgfsetfillopacity{0.300000}%
\pgfsetlinewidth{0.301125pt}%
\definecolor{currentstroke}{rgb}{0.500000,0.500000,0.500000}%
\pgfsetstrokecolor{currentstroke}%
\pgfsetstrokeopacity{0.300000}%
\pgfsetdash{}{0pt}%
\pgfpathmoveto{\pgfqpoint{0.000000in}{0.000000in}}%
\pgfpathlineto{\pgfqpoint{0.000000in}{0.000000in}}%
\pgfpathclose%
\pgfusepath{stroke,fill}%
\end{pgfscope}%
\begin{pgfscope}%
\pgfpathrectangle{\pgfqpoint{0.647939in}{0.492442in}}{\pgfqpoint{4.273799in}{2.331163in}}%
\pgfusepath{clip}%
\pgfsetroundcap%
\pgfsetroundjoin%
\pgfsetlinewidth{0.301125pt}%
\definecolor{currentstroke}{rgb}{0.500000,0.500000,0.500000}%
\pgfsetstrokecolor{currentstroke}%
\pgfsetstrokeopacity{0.300000}%
\pgfsetdash{}{0pt}%
\pgfpathmoveto{\pgfqpoint{3.405382in}{2.022041in}}%
\pgfusepath{stroke}%
\end{pgfscope}%
\begin{pgfscope}%
\pgfpathrectangle{\pgfqpoint{0.647939in}{0.492442in}}{\pgfqpoint{4.273799in}{2.331163in}}%
\pgfusepath{clip}%
\pgfsetroundcap%
\pgfsetroundjoin%
\definecolor{currentfill}{rgb}{0.500000,0.500000,0.500000}%
\pgfsetfillcolor{currentfill}%
\pgfsetfillopacity{0.300000}%
\pgfsetlinewidth{0.301125pt}%
\definecolor{currentstroke}{rgb}{0.500000,0.500000,0.500000}%
\pgfsetstrokecolor{currentstroke}%
\pgfsetstrokeopacity{0.300000}%
\pgfsetdash{}{0pt}%
\pgfpathmoveto{\pgfqpoint{0.000000in}{0.000000in}}%
\pgfpathlineto{\pgfqpoint{0.000000in}{0.000000in}}%
\pgfpathclose%
\pgfusepath{stroke,fill}%
\end{pgfscope}%
\begin{pgfscope}%
\pgfpathrectangle{\pgfqpoint{0.647939in}{0.492442in}}{\pgfqpoint{4.273799in}{2.331163in}}%
\pgfusepath{clip}%
\pgfsetroundcap%
\pgfsetroundjoin%
\pgfsetlinewidth{0.301125pt}%
\definecolor{currentstroke}{rgb}{0.500000,0.500000,0.500000}%
\pgfsetstrokecolor{currentstroke}%
\pgfsetstrokeopacity{0.300000}%
\pgfsetdash{}{0pt}%
\pgfpathmoveto{\pgfqpoint{3.204286in}{2.393021in}}%
\pgfusepath{stroke}%
\end{pgfscope}%
\begin{pgfscope}%
\pgfpathrectangle{\pgfqpoint{0.647939in}{0.492442in}}{\pgfqpoint{4.273799in}{2.331163in}}%
\pgfusepath{clip}%
\pgfsetroundcap%
\pgfsetroundjoin%
\definecolor{currentfill}{rgb}{0.500000,0.500000,0.500000}%
\pgfsetfillcolor{currentfill}%
\pgfsetfillopacity{0.300000}%
\pgfsetlinewidth{0.301125pt}%
\definecolor{currentstroke}{rgb}{0.500000,0.500000,0.500000}%
\pgfsetstrokecolor{currentstroke}%
\pgfsetstrokeopacity{0.300000}%
\pgfsetdash{}{0pt}%
\pgfpathmoveto{\pgfqpoint{0.000000in}{0.000000in}}%
\pgfpathlineto{\pgfqpoint{0.000000in}{0.000000in}}%
\pgfpathclose%
\pgfusepath{stroke,fill}%
\end{pgfscope}%
\begin{pgfscope}%
\pgfpathrectangle{\pgfqpoint{0.647939in}{0.492442in}}{\pgfqpoint{4.273799in}{2.331163in}}%
\pgfusepath{clip}%
\pgfsetroundcap%
\pgfsetroundjoin%
\pgfsetlinewidth{0.301125pt}%
\definecolor{currentstroke}{rgb}{0.500000,0.500000,0.500000}%
\pgfsetstrokecolor{currentstroke}%
\pgfsetstrokeopacity{0.300000}%
\pgfsetdash{}{0pt}%
\pgfpathmoveto{\pgfqpoint{3.066076in}{2.437907in}}%
\pgfusepath{stroke}%
\end{pgfscope}%
\begin{pgfscope}%
\pgfpathrectangle{\pgfqpoint{0.647939in}{0.492442in}}{\pgfqpoint{4.273799in}{2.331163in}}%
\pgfusepath{clip}%
\pgfsetroundcap%
\pgfsetroundjoin%
\definecolor{currentfill}{rgb}{0.500000,0.500000,0.500000}%
\pgfsetfillcolor{currentfill}%
\pgfsetfillopacity{0.300000}%
\pgfsetlinewidth{0.301125pt}%
\definecolor{currentstroke}{rgb}{0.500000,0.500000,0.500000}%
\pgfsetstrokecolor{currentstroke}%
\pgfsetstrokeopacity{0.300000}%
\pgfsetdash{}{0pt}%
\pgfpathmoveto{\pgfqpoint{0.000000in}{0.000000in}}%
\pgfpathlineto{\pgfqpoint{0.000000in}{0.000000in}}%
\pgfpathclose%
\pgfusepath{stroke,fill}%
\end{pgfscope}%
\begin{pgfscope}%
\pgfpathrectangle{\pgfqpoint{0.647939in}{0.492442in}}{\pgfqpoint{4.273799in}{2.331163in}}%
\pgfusepath{clip}%
\pgfsetroundcap%
\pgfsetroundjoin%
\pgfsetlinewidth{0.301125pt}%
\definecolor{currentstroke}{rgb}{0.500000,0.500000,0.500000}%
\pgfsetstrokecolor{currentstroke}%
\pgfsetstrokeopacity{0.300000}%
\pgfsetdash{}{0pt}%
\pgfpathmoveto{\pgfqpoint{2.706322in}{2.691918in}}%
\pgfusepath{stroke}%
\end{pgfscope}%
\begin{pgfscope}%
\pgfpathrectangle{\pgfqpoint{0.647939in}{0.492442in}}{\pgfqpoint{4.273799in}{2.331163in}}%
\pgfusepath{clip}%
\pgfsetroundcap%
\pgfsetroundjoin%
\definecolor{currentfill}{rgb}{0.500000,0.500000,0.500000}%
\pgfsetfillcolor{currentfill}%
\pgfsetfillopacity{0.300000}%
\pgfsetlinewidth{0.301125pt}%
\definecolor{currentstroke}{rgb}{0.500000,0.500000,0.500000}%
\pgfsetstrokecolor{currentstroke}%
\pgfsetstrokeopacity{0.300000}%
\pgfsetdash{}{0pt}%
\pgfpathmoveto{\pgfqpoint{0.000000in}{0.000000in}}%
\pgfpathlineto{\pgfqpoint{0.000000in}{0.000000in}}%
\pgfpathclose%
\pgfusepath{stroke,fill}%
\end{pgfscope}%
\begin{pgfscope}%
\pgfpathrectangle{\pgfqpoint{0.647939in}{0.492442in}}{\pgfqpoint{4.273799in}{2.331163in}}%
\pgfusepath{clip}%
\pgfsetroundcap%
\pgfsetroundjoin%
\pgfsetlinewidth{0.301125pt}%
\definecolor{currentstroke}{rgb}{0.500000,0.500000,0.500000}%
\pgfsetstrokecolor{currentstroke}%
\pgfsetstrokeopacity{0.300000}%
\pgfsetdash{}{0pt}%
\pgfpathmoveto{\pgfqpoint{2.560341in}{2.739960in}}%
\pgfusepath{stroke}%
\end{pgfscope}%
\begin{pgfscope}%
\pgfpathrectangle{\pgfqpoint{0.647939in}{0.492442in}}{\pgfqpoint{4.273799in}{2.331163in}}%
\pgfusepath{clip}%
\pgfsetroundcap%
\pgfsetroundjoin%
\definecolor{currentfill}{rgb}{0.500000,0.500000,0.500000}%
\pgfsetfillcolor{currentfill}%
\pgfsetfillopacity{0.300000}%
\pgfsetlinewidth{0.301125pt}%
\definecolor{currentstroke}{rgb}{0.500000,0.500000,0.500000}%
\pgfsetstrokecolor{currentstroke}%
\pgfsetstrokeopacity{0.300000}%
\pgfsetdash{}{0pt}%
\pgfpathmoveto{\pgfqpoint{0.000000in}{0.000000in}}%
\pgfpathlineto{\pgfqpoint{0.000000in}{0.000000in}}%
\pgfpathclose%
\pgfusepath{stroke,fill}%
\end{pgfscope}%
\begin{pgfscope}%
\pgfpathrectangle{\pgfqpoint{0.647939in}{0.492442in}}{\pgfqpoint{4.273799in}{2.331163in}}%
\pgfusepath{clip}%
\pgfsetroundcap%
\pgfsetroundjoin%
\pgfsetlinewidth{0.301125pt}%
\definecolor{currentstroke}{rgb}{0.500000,0.500000,0.500000}%
\pgfsetstrokecolor{currentstroke}%
\pgfsetstrokeopacity{0.300000}%
\pgfsetdash{}{0pt}%
\pgfpathmoveto{\pgfqpoint{2.380733in}{2.751613in}}%
\pgfusepath{stroke}%
\end{pgfscope}%
\begin{pgfscope}%
\pgfpathrectangle{\pgfqpoint{0.647939in}{0.492442in}}{\pgfqpoint{4.273799in}{2.331163in}}%
\pgfusepath{clip}%
\pgfsetroundcap%
\pgfsetroundjoin%
\definecolor{currentfill}{rgb}{0.500000,0.500000,0.500000}%
\pgfsetfillcolor{currentfill}%
\pgfsetfillopacity{0.300000}%
\pgfsetlinewidth{0.301125pt}%
\definecolor{currentstroke}{rgb}{0.500000,0.500000,0.500000}%
\pgfsetstrokecolor{currentstroke}%
\pgfsetstrokeopacity{0.300000}%
\pgfsetdash{}{0pt}%
\pgfpathmoveto{\pgfqpoint{0.000000in}{0.000000in}}%
\pgfpathlineto{\pgfqpoint{0.000000in}{0.000000in}}%
\pgfpathclose%
\pgfusepath{stroke,fill}%
\end{pgfscope}%
\begin{pgfscope}%
\pgfpathrectangle{\pgfqpoint{0.647939in}{0.492442in}}{\pgfqpoint{4.273799in}{2.331163in}}%
\pgfusepath{clip}%
\pgfsetroundcap%
\pgfsetroundjoin%
\pgfsetlinewidth{0.301125pt}%
\definecolor{currentstroke}{rgb}{0.500000,0.500000,0.500000}%
\pgfsetstrokecolor{currentstroke}%
\pgfsetstrokeopacity{0.300000}%
\pgfsetdash{}{0pt}%
\pgfpathmoveto{\pgfqpoint{2.650652in}{2.571259in}}%
\pgfusepath{stroke}%
\end{pgfscope}%
\begin{pgfscope}%
\pgfpathrectangle{\pgfqpoint{0.647939in}{0.492442in}}{\pgfqpoint{4.273799in}{2.331163in}}%
\pgfusepath{clip}%
\pgfsetroundcap%
\pgfsetroundjoin%
\definecolor{currentfill}{rgb}{0.500000,0.500000,0.500000}%
\pgfsetfillcolor{currentfill}%
\pgfsetfillopacity{0.300000}%
\pgfsetlinewidth{0.301125pt}%
\definecolor{currentstroke}{rgb}{0.500000,0.500000,0.500000}%
\pgfsetstrokecolor{currentstroke}%
\pgfsetstrokeopacity{0.300000}%
\pgfsetdash{}{0pt}%
\pgfpathmoveto{\pgfqpoint{0.000000in}{0.000000in}}%
\pgfpathlineto{\pgfqpoint{0.000000in}{0.000000in}}%
\pgfpathclose%
\pgfusepath{stroke,fill}%
\end{pgfscope}%
\begin{pgfscope}%
\pgfpathrectangle{\pgfqpoint{0.647939in}{0.492442in}}{\pgfqpoint{4.273799in}{2.331163in}}%
\pgfusepath{clip}%
\pgfsetroundcap%
\pgfsetroundjoin%
\pgfsetlinewidth{0.301125pt}%
\definecolor{currentstroke}{rgb}{0.500000,0.500000,0.500000}%
\pgfsetstrokecolor{currentstroke}%
\pgfsetstrokeopacity{0.300000}%
\pgfsetdash{}{0pt}%
\pgfpathmoveto{\pgfqpoint{2.121514in}{2.679626in}}%
\pgfusepath{stroke}%
\end{pgfscope}%
\begin{pgfscope}%
\pgfpathrectangle{\pgfqpoint{0.647939in}{0.492442in}}{\pgfqpoint{4.273799in}{2.331163in}}%
\pgfusepath{clip}%
\pgfsetroundcap%
\pgfsetroundjoin%
\definecolor{currentfill}{rgb}{0.500000,0.500000,0.500000}%
\pgfsetfillcolor{currentfill}%
\pgfsetfillopacity{0.300000}%
\pgfsetlinewidth{0.301125pt}%
\definecolor{currentstroke}{rgb}{0.500000,0.500000,0.500000}%
\pgfsetstrokecolor{currentstroke}%
\pgfsetstrokeopacity{0.300000}%
\pgfsetdash{}{0pt}%
\pgfpathmoveto{\pgfqpoint{0.000000in}{0.000000in}}%
\pgfpathlineto{\pgfqpoint{0.000000in}{0.000000in}}%
\pgfpathclose%
\pgfusepath{stroke,fill}%
\end{pgfscope}%
\begin{pgfscope}%
\pgfpathrectangle{\pgfqpoint{0.647939in}{0.492442in}}{\pgfqpoint{4.273799in}{2.331163in}}%
\pgfusepath{clip}%
\pgfsetroundcap%
\pgfsetroundjoin%
\pgfsetlinewidth{0.301125pt}%
\definecolor{currentstroke}{rgb}{0.500000,0.500000,0.500000}%
\pgfsetstrokecolor{currentstroke}%
\pgfsetstrokeopacity{0.300000}%
\pgfsetdash{}{0pt}%
\pgfpathmoveto{\pgfqpoint{1.987210in}{2.658322in}}%
\pgfusepath{stroke}%
\end{pgfscope}%
\begin{pgfscope}%
\pgfpathrectangle{\pgfqpoint{0.647939in}{0.492442in}}{\pgfqpoint{4.273799in}{2.331163in}}%
\pgfusepath{clip}%
\pgfsetroundcap%
\pgfsetroundjoin%
\definecolor{currentfill}{rgb}{0.500000,0.500000,0.500000}%
\pgfsetfillcolor{currentfill}%
\pgfsetfillopacity{0.300000}%
\pgfsetlinewidth{0.301125pt}%
\definecolor{currentstroke}{rgb}{0.500000,0.500000,0.500000}%
\pgfsetstrokecolor{currentstroke}%
\pgfsetstrokeopacity{0.300000}%
\pgfsetdash{}{0pt}%
\pgfpathmoveto{\pgfqpoint{0.000000in}{0.000000in}}%
\pgfpathlineto{\pgfqpoint{0.000000in}{0.000000in}}%
\pgfpathclose%
\pgfusepath{stroke,fill}%
\end{pgfscope}%
\begin{pgfscope}%
\pgfpathrectangle{\pgfqpoint{0.647939in}{0.492442in}}{\pgfqpoint{4.273799in}{2.331163in}}%
\pgfusepath{clip}%
\pgfsetroundcap%
\pgfsetroundjoin%
\pgfsetlinewidth{0.301125pt}%
\definecolor{currentstroke}{rgb}{0.500000,0.500000,0.500000}%
\pgfsetstrokecolor{currentstroke}%
\pgfsetstrokeopacity{0.300000}%
\pgfsetdash{}{0pt}%
\pgfpathmoveto{\pgfqpoint{2.162117in}{2.551322in}}%
\pgfusepath{stroke}%
\end{pgfscope}%
\begin{pgfscope}%
\pgfpathrectangle{\pgfqpoint{0.647939in}{0.492442in}}{\pgfqpoint{4.273799in}{2.331163in}}%
\pgfusepath{clip}%
\pgfsetroundcap%
\pgfsetroundjoin%
\definecolor{currentfill}{rgb}{0.500000,0.500000,0.500000}%
\pgfsetfillcolor{currentfill}%
\pgfsetfillopacity{0.300000}%
\pgfsetlinewidth{0.301125pt}%
\definecolor{currentstroke}{rgb}{0.500000,0.500000,0.500000}%
\pgfsetstrokecolor{currentstroke}%
\pgfsetstrokeopacity{0.300000}%
\pgfsetdash{}{0pt}%
\pgfpathmoveto{\pgfqpoint{0.000000in}{0.000000in}}%
\pgfpathlineto{\pgfqpoint{0.000000in}{0.000000in}}%
\pgfpathclose%
\pgfusepath{stroke,fill}%
\end{pgfscope}%
\begin{pgfscope}%
\pgfpathrectangle{\pgfqpoint{0.647939in}{0.492442in}}{\pgfqpoint{4.273799in}{2.331163in}}%
\pgfusepath{clip}%
\pgfsetroundcap%
\pgfsetroundjoin%
\pgfsetlinewidth{0.301125pt}%
\definecolor{currentstroke}{rgb}{0.500000,0.500000,0.500000}%
\pgfsetstrokecolor{currentstroke}%
\pgfsetstrokeopacity{0.300000}%
\pgfsetdash{}{0pt}%
\pgfpathmoveto{\pgfqpoint{1.871129in}{2.502555in}}%
\pgfusepath{stroke}%
\end{pgfscope}%
\begin{pgfscope}%
\pgfpathrectangle{\pgfqpoint{0.647939in}{0.492442in}}{\pgfqpoint{4.273799in}{2.331163in}}%
\pgfusepath{clip}%
\pgfsetroundcap%
\pgfsetroundjoin%
\definecolor{currentfill}{rgb}{0.500000,0.500000,0.500000}%
\pgfsetfillcolor{currentfill}%
\pgfsetfillopacity{0.300000}%
\pgfsetlinewidth{0.301125pt}%
\definecolor{currentstroke}{rgb}{0.500000,0.500000,0.500000}%
\pgfsetstrokecolor{currentstroke}%
\pgfsetstrokeopacity{0.300000}%
\pgfsetdash{}{0pt}%
\pgfpathmoveto{\pgfqpoint{0.000000in}{0.000000in}}%
\pgfpathlineto{\pgfqpoint{0.000000in}{0.000000in}}%
\pgfpathclose%
\pgfusepath{stroke,fill}%
\end{pgfscope}%
\begin{pgfscope}%
\pgfpathrectangle{\pgfqpoint{0.647939in}{0.492442in}}{\pgfqpoint{4.273799in}{2.331163in}}%
\pgfusepath{clip}%
\pgfsetroundcap%
\pgfsetroundjoin%
\pgfsetlinewidth{0.301125pt}%
\definecolor{currentstroke}{rgb}{0.500000,0.500000,0.500000}%
\pgfsetstrokecolor{currentstroke}%
\pgfsetstrokeopacity{0.300000}%
\pgfsetdash{}{0pt}%
\pgfpathmoveto{\pgfqpoint{1.667008in}{2.462316in}}%
\pgfusepath{stroke}%
\end{pgfscope}%
\begin{pgfscope}%
\pgfpathrectangle{\pgfqpoint{0.647939in}{0.492442in}}{\pgfqpoint{4.273799in}{2.331163in}}%
\pgfusepath{clip}%
\pgfsetroundcap%
\pgfsetroundjoin%
\definecolor{currentfill}{rgb}{0.500000,0.500000,0.500000}%
\pgfsetfillcolor{currentfill}%
\pgfsetfillopacity{0.300000}%
\pgfsetlinewidth{0.301125pt}%
\definecolor{currentstroke}{rgb}{0.500000,0.500000,0.500000}%
\pgfsetstrokecolor{currentstroke}%
\pgfsetstrokeopacity{0.300000}%
\pgfsetdash{}{0pt}%
\pgfpathmoveto{\pgfqpoint{0.000000in}{0.000000in}}%
\pgfpathlineto{\pgfqpoint{0.000000in}{0.000000in}}%
\pgfpathclose%
\pgfusepath{stroke,fill}%
\end{pgfscope}%
\begin{pgfscope}%
\pgfpathrectangle{\pgfqpoint{0.647939in}{0.492442in}}{\pgfqpoint{4.273799in}{2.331163in}}%
\pgfusepath{clip}%
\pgfsetroundcap%
\pgfsetroundjoin%
\pgfsetlinewidth{0.301125pt}%
\definecolor{currentstroke}{rgb}{0.500000,0.500000,0.500000}%
\pgfsetstrokecolor{currentstroke}%
\pgfsetstrokeopacity{0.300000}%
\pgfsetdash{}{0pt}%
\pgfpathmoveto{\pgfqpoint{1.489118in}{2.471837in}}%
\pgfusepath{stroke}%
\end{pgfscope}%
\begin{pgfscope}%
\pgfpathrectangle{\pgfqpoint{0.647939in}{0.492442in}}{\pgfqpoint{4.273799in}{2.331163in}}%
\pgfusepath{clip}%
\pgfsetroundcap%
\pgfsetroundjoin%
\definecolor{currentfill}{rgb}{0.500000,0.500000,0.500000}%
\pgfsetfillcolor{currentfill}%
\pgfsetfillopacity{0.300000}%
\pgfsetlinewidth{0.301125pt}%
\definecolor{currentstroke}{rgb}{0.500000,0.500000,0.500000}%
\pgfsetstrokecolor{currentstroke}%
\pgfsetstrokeopacity{0.300000}%
\pgfsetdash{}{0pt}%
\pgfpathmoveto{\pgfqpoint{0.000000in}{0.000000in}}%
\pgfpathlineto{\pgfqpoint{0.000000in}{0.000000in}}%
\pgfpathclose%
\pgfusepath{stroke,fill}%
\end{pgfscope}%
\begin{pgfscope}%
\pgfpathrectangle{\pgfqpoint{0.647939in}{0.492442in}}{\pgfqpoint{4.273799in}{2.331163in}}%
\pgfusepath{clip}%
\pgfsetroundcap%
\pgfsetroundjoin%
\pgfsetlinewidth{0.301125pt}%
\definecolor{currentstroke}{rgb}{0.500000,0.500000,0.500000}%
\pgfsetstrokecolor{currentstroke}%
\pgfsetstrokeopacity{0.300000}%
\pgfsetdash{}{0pt}%
\pgfpathmoveto{\pgfqpoint{1.424754in}{2.159035in}}%
\pgfusepath{stroke}%
\end{pgfscope}%
\begin{pgfscope}%
\pgfpathrectangle{\pgfqpoint{0.647939in}{0.492442in}}{\pgfqpoint{4.273799in}{2.331163in}}%
\pgfusepath{clip}%
\pgfsetroundcap%
\pgfsetroundjoin%
\definecolor{currentfill}{rgb}{0.500000,0.500000,0.500000}%
\pgfsetfillcolor{currentfill}%
\pgfsetfillopacity{0.300000}%
\pgfsetlinewidth{0.301125pt}%
\definecolor{currentstroke}{rgb}{0.500000,0.500000,0.500000}%
\pgfsetstrokecolor{currentstroke}%
\pgfsetstrokeopacity{0.300000}%
\pgfsetdash{}{0pt}%
\pgfpathmoveto{\pgfqpoint{0.000000in}{0.000000in}}%
\pgfpathlineto{\pgfqpoint{0.000000in}{0.000000in}}%
\pgfpathclose%
\pgfusepath{stroke,fill}%
\end{pgfscope}%
\begin{pgfscope}%
\pgfpathrectangle{\pgfqpoint{0.647939in}{0.492442in}}{\pgfqpoint{4.273799in}{2.331163in}}%
\pgfusepath{clip}%
\pgfsetroundcap%
\pgfsetroundjoin%
\pgfsetlinewidth{0.301125pt}%
\definecolor{currentstroke}{rgb}{0.500000,0.500000,0.500000}%
\pgfsetstrokecolor{currentstroke}%
\pgfsetstrokeopacity{0.300000}%
\pgfsetdash{}{0pt}%
\pgfpathmoveto{\pgfqpoint{1.273353in}{2.206974in}}%
\pgfusepath{stroke}%
\end{pgfscope}%
\begin{pgfscope}%
\pgfpathrectangle{\pgfqpoint{0.647939in}{0.492442in}}{\pgfqpoint{4.273799in}{2.331163in}}%
\pgfusepath{clip}%
\pgfsetroundcap%
\pgfsetroundjoin%
\definecolor{currentfill}{rgb}{0.500000,0.500000,0.500000}%
\pgfsetfillcolor{currentfill}%
\pgfsetfillopacity{0.300000}%
\pgfsetlinewidth{0.301125pt}%
\definecolor{currentstroke}{rgb}{0.500000,0.500000,0.500000}%
\pgfsetstrokecolor{currentstroke}%
\pgfsetstrokeopacity{0.300000}%
\pgfsetdash{}{0pt}%
\pgfpathmoveto{\pgfqpoint{0.000000in}{0.000000in}}%
\pgfpathlineto{\pgfqpoint{0.000000in}{0.000000in}}%
\pgfpathclose%
\pgfusepath{stroke,fill}%
\end{pgfscope}%
\begin{pgfscope}%
\pgfpathrectangle{\pgfqpoint{0.647939in}{0.492442in}}{\pgfqpoint{4.273799in}{2.331163in}}%
\pgfusepath{clip}%
\pgfsetroundcap%
\pgfsetroundjoin%
\pgfsetlinewidth{0.301125pt}%
\definecolor{currentstroke}{rgb}{0.500000,0.500000,0.500000}%
\pgfsetstrokecolor{currentstroke}%
\pgfsetstrokeopacity{0.300000}%
\pgfsetdash{}{0pt}%
\pgfpathmoveto{\pgfqpoint{1.129834in}{1.842857in}}%
\pgfusepath{stroke}%
\end{pgfscope}%
\begin{pgfscope}%
\pgfpathrectangle{\pgfqpoint{0.647939in}{0.492442in}}{\pgfqpoint{4.273799in}{2.331163in}}%
\pgfusepath{clip}%
\pgfsetroundcap%
\pgfsetroundjoin%
\definecolor{currentfill}{rgb}{0.500000,0.500000,0.500000}%
\pgfsetfillcolor{currentfill}%
\pgfsetfillopacity{0.300000}%
\pgfsetlinewidth{0.301125pt}%
\definecolor{currentstroke}{rgb}{0.500000,0.500000,0.500000}%
\pgfsetstrokecolor{currentstroke}%
\pgfsetstrokeopacity{0.300000}%
\pgfsetdash{}{0pt}%
\pgfpathmoveto{\pgfqpoint{0.000000in}{0.000000in}}%
\pgfpathlineto{\pgfqpoint{0.000000in}{0.000000in}}%
\pgfpathclose%
\pgfusepath{stroke,fill}%
\end{pgfscope}%
\begin{pgfscope}%
\pgfpathrectangle{\pgfqpoint{0.647939in}{0.492442in}}{\pgfqpoint{4.273799in}{2.331163in}}%
\pgfusepath{clip}%
\pgfsetroundcap%
\pgfsetroundjoin%
\pgfsetlinewidth{0.301125pt}%
\definecolor{currentstroke}{rgb}{0.500000,0.500000,0.500000}%
\pgfsetstrokecolor{currentstroke}%
\pgfsetstrokeopacity{0.300000}%
\pgfsetdash{}{0pt}%
\pgfpathmoveto{\pgfqpoint{1.030797in}{2.100462in}}%
\pgfusepath{stroke}%
\end{pgfscope}%
\begin{pgfscope}%
\pgfpathrectangle{\pgfqpoint{0.647939in}{0.492442in}}{\pgfqpoint{4.273799in}{2.331163in}}%
\pgfusepath{clip}%
\pgfsetroundcap%
\pgfsetroundjoin%
\definecolor{currentfill}{rgb}{0.500000,0.500000,0.500000}%
\pgfsetfillcolor{currentfill}%
\pgfsetfillopacity{0.300000}%
\pgfsetlinewidth{0.301125pt}%
\definecolor{currentstroke}{rgb}{0.500000,0.500000,0.500000}%
\pgfsetstrokecolor{currentstroke}%
\pgfsetstrokeopacity{0.300000}%
\pgfsetdash{}{0pt}%
\pgfpathmoveto{\pgfqpoint{0.000000in}{0.000000in}}%
\pgfpathlineto{\pgfqpoint{0.000000in}{0.000000in}}%
\pgfpathclose%
\pgfusepath{stroke,fill}%
\end{pgfscope}%
\begin{pgfscope}%
\pgfpathrectangle{\pgfqpoint{0.647939in}{0.492442in}}{\pgfqpoint{4.273799in}{2.331163in}}%
\pgfusepath{clip}%
\pgfsetroundcap%
\pgfsetroundjoin%
\pgfsetlinewidth{0.301125pt}%
\definecolor{currentstroke}{rgb}{0.500000,0.500000,0.500000}%
\pgfsetstrokecolor{currentstroke}%
\pgfsetstrokeopacity{0.300000}%
\pgfsetdash{}{0pt}%
\pgfpathmoveto{\pgfqpoint{0.906245in}{1.840896in}}%
\pgfusepath{stroke}%
\end{pgfscope}%
\begin{pgfscope}%
\pgfpathrectangle{\pgfqpoint{0.647939in}{0.492442in}}{\pgfqpoint{4.273799in}{2.331163in}}%
\pgfusepath{clip}%
\pgfsetroundcap%
\pgfsetroundjoin%
\definecolor{currentfill}{rgb}{0.500000,0.500000,0.500000}%
\pgfsetfillcolor{currentfill}%
\pgfsetfillopacity{0.300000}%
\pgfsetlinewidth{0.301125pt}%
\definecolor{currentstroke}{rgb}{0.500000,0.500000,0.500000}%
\pgfsetstrokecolor{currentstroke}%
\pgfsetstrokeopacity{0.300000}%
\pgfsetdash{}{0pt}%
\pgfpathmoveto{\pgfqpoint{0.000000in}{0.000000in}}%
\pgfpathlineto{\pgfqpoint{0.000000in}{0.000000in}}%
\pgfpathclose%
\pgfusepath{stroke,fill}%
\end{pgfscope}%
\begin{pgfscope}%
\pgfpathrectangle{\pgfqpoint{0.647939in}{0.492442in}}{\pgfqpoint{4.273799in}{2.331163in}}%
\pgfusepath{clip}%
\pgfsetroundcap%
\pgfsetroundjoin%
\pgfsetlinewidth{0.301125pt}%
\definecolor{currentstroke}{rgb}{0.500000,0.500000,0.500000}%
\pgfsetstrokecolor{currentstroke}%
\pgfsetstrokeopacity{0.300000}%
\pgfsetdash{}{0pt}%
\pgfpathmoveto{\pgfqpoint{0.808001in}{2.047583in}}%
\pgfusepath{stroke}%
\end{pgfscope}%
\begin{pgfscope}%
\pgfpathrectangle{\pgfqpoint{0.647939in}{0.492442in}}{\pgfqpoint{4.273799in}{2.331163in}}%
\pgfusepath{clip}%
\pgfsetroundcap%
\pgfsetroundjoin%
\definecolor{currentfill}{rgb}{0.500000,0.500000,0.500000}%
\pgfsetfillcolor{currentfill}%
\pgfsetfillopacity{0.300000}%
\pgfsetlinewidth{0.301125pt}%
\definecolor{currentstroke}{rgb}{0.500000,0.500000,0.500000}%
\pgfsetstrokecolor{currentstroke}%
\pgfsetstrokeopacity{0.300000}%
\pgfsetdash{}{0pt}%
\pgfpathmoveto{\pgfqpoint{0.000000in}{0.000000in}}%
\pgfpathlineto{\pgfqpoint{0.000000in}{0.000000in}}%
\pgfpathclose%
\pgfusepath{stroke,fill}%
\end{pgfscope}%
\begin{pgfscope}%
\pgfpathrectangle{\pgfqpoint{0.647939in}{0.492442in}}{\pgfqpoint{4.273799in}{2.331163in}}%
\pgfusepath{clip}%
\pgfsetroundcap%
\pgfsetroundjoin%
\pgfsetlinewidth{0.301125pt}%
\definecolor{currentstroke}{rgb}{0.500000,0.500000,0.500000}%
\pgfsetstrokecolor{currentstroke}%
\pgfsetstrokeopacity{0.300000}%
\pgfsetdash{}{0pt}%
\pgfpathmoveto{\pgfqpoint{0.700150in}{2.202707in}}%
\pgfusepath{stroke}%
\end{pgfscope}%
\begin{pgfscope}%
\pgfpathrectangle{\pgfqpoint{0.647939in}{0.492442in}}{\pgfqpoint{4.273799in}{2.331163in}}%
\pgfusepath{clip}%
\pgfsetroundcap%
\pgfsetroundjoin%
\definecolor{currentfill}{rgb}{0.500000,0.500000,0.500000}%
\pgfsetfillcolor{currentfill}%
\pgfsetfillopacity{0.300000}%
\pgfsetlinewidth{0.301125pt}%
\definecolor{currentstroke}{rgb}{0.500000,0.500000,0.500000}%
\pgfsetstrokecolor{currentstroke}%
\pgfsetstrokeopacity{0.300000}%
\pgfsetdash{}{0pt}%
\pgfpathmoveto{\pgfqpoint{0.000000in}{0.000000in}}%
\pgfpathlineto{\pgfqpoint{0.000000in}{0.000000in}}%
\pgfpathclose%
\pgfusepath{stroke,fill}%
\end{pgfscope}%
\begin{pgfscope}%
\pgfpathrectangle{\pgfqpoint{0.647939in}{0.492442in}}{\pgfqpoint{4.273799in}{2.331163in}}%
\pgfusepath{clip}%
\pgfsetroundcap%
\pgfsetroundjoin%
\pgfsetlinewidth{0.301125pt}%
\definecolor{currentstroke}{rgb}{0.500000,0.500000,0.500000}%
\pgfsetstrokecolor{currentstroke}%
\pgfsetstrokeopacity{0.300000}%
\pgfsetdash{}{0pt}%
\pgfpathmoveto{\pgfqpoint{0.650059in}{2.137215in}}%
\pgfusepath{stroke}%
\end{pgfscope}%
\begin{pgfscope}%
\pgfpathrectangle{\pgfqpoint{0.647939in}{0.492442in}}{\pgfqpoint{4.273799in}{2.331163in}}%
\pgfusepath{clip}%
\pgfsetroundcap%
\pgfsetroundjoin%
\definecolor{currentfill}{rgb}{0.500000,0.500000,0.500000}%
\pgfsetfillcolor{currentfill}%
\pgfsetfillopacity{0.300000}%
\pgfsetlinewidth{0.301125pt}%
\definecolor{currentstroke}{rgb}{0.500000,0.500000,0.500000}%
\pgfsetstrokecolor{currentstroke}%
\pgfsetstrokeopacity{0.300000}%
\pgfsetdash{}{0pt}%
\pgfpathmoveto{\pgfqpoint{0.000000in}{0.000000in}}%
\pgfpathlineto{\pgfqpoint{0.000000in}{0.000000in}}%
\pgfpathclose%
\pgfusepath{stroke,fill}%
\end{pgfscope}%
\begin{pgfscope}%
\pgfpathrectangle{\pgfqpoint{0.647939in}{0.492442in}}{\pgfqpoint{4.273799in}{2.331163in}}%
\pgfusepath{clip}%
\pgfsetroundcap%
\pgfsetroundjoin%
\pgfsetlinewidth{0.301125pt}%
\definecolor{currentstroke}{rgb}{0.500000,0.500000,0.500000}%
\pgfsetstrokecolor{currentstroke}%
\pgfsetstrokeopacity{0.300000}%
\pgfsetdash{}{0pt}%
\pgfpathmoveto{\pgfqpoint{3.988858in}{0.777588in}}%
\pgfusepath{stroke}%
\end{pgfscope}%
\begin{pgfscope}%
\pgfpathrectangle{\pgfqpoint{0.647939in}{0.492442in}}{\pgfqpoint{4.273799in}{2.331163in}}%
\pgfusepath{clip}%
\pgfsetroundcap%
\pgfsetroundjoin%
\definecolor{currentfill}{rgb}{0.500000,0.500000,0.500000}%
\pgfsetfillcolor{currentfill}%
\pgfsetfillopacity{0.300000}%
\pgfsetlinewidth{0.301125pt}%
\definecolor{currentstroke}{rgb}{0.500000,0.500000,0.500000}%
\pgfsetstrokecolor{currentstroke}%
\pgfsetstrokeopacity{0.300000}%
\pgfsetdash{}{0pt}%
\pgfpathmoveto{\pgfqpoint{0.000000in}{0.000000in}}%
\pgfpathlineto{\pgfqpoint{0.000000in}{0.000000in}}%
\pgfpathclose%
\pgfusepath{stroke,fill}%
\end{pgfscope}%
\begin{pgfscope}%
\pgfpathrectangle{\pgfqpoint{0.647939in}{0.492442in}}{\pgfqpoint{4.273799in}{2.331163in}}%
\pgfusepath{clip}%
\pgfsetroundcap%
\pgfsetroundjoin%
\pgfsetlinewidth{0.301125pt}%
\definecolor{currentstroke}{rgb}{0.500000,0.500000,0.500000}%
\pgfsetstrokecolor{currentstroke}%
\pgfsetstrokeopacity{0.300000}%
\pgfsetdash{}{0pt}%
\pgfpathmoveto{\pgfqpoint{4.604713in}{2.024680in}}%
\pgfusepath{stroke}%
\end{pgfscope}%
\begin{pgfscope}%
\pgfpathrectangle{\pgfqpoint{0.647939in}{0.492442in}}{\pgfqpoint{4.273799in}{2.331163in}}%
\pgfusepath{clip}%
\pgfsetroundcap%
\pgfsetroundjoin%
\definecolor{currentfill}{rgb}{0.500000,0.500000,0.500000}%
\pgfsetfillcolor{currentfill}%
\pgfsetfillopacity{0.300000}%
\pgfsetlinewidth{0.301125pt}%
\definecolor{currentstroke}{rgb}{0.500000,0.500000,0.500000}%
\pgfsetstrokecolor{currentstroke}%
\pgfsetstrokeopacity{0.300000}%
\pgfsetdash{}{0pt}%
\pgfpathmoveto{\pgfqpoint{0.000000in}{0.000000in}}%
\pgfpathlineto{\pgfqpoint{0.000000in}{0.000000in}}%
\pgfpathclose%
\pgfusepath{stroke,fill}%
\end{pgfscope}%
\begin{pgfscope}%
\pgfpathrectangle{\pgfqpoint{0.647939in}{0.492442in}}{\pgfqpoint{4.273799in}{2.331163in}}%
\pgfusepath{clip}%
\pgfsetroundcap%
\pgfsetroundjoin%
\pgfsetlinewidth{0.301125pt}%
\definecolor{currentstroke}{rgb}{0.500000,0.500000,0.500000}%
\pgfsetstrokecolor{currentstroke}%
\pgfsetstrokeopacity{0.300000}%
\pgfsetdash{}{0pt}%
\pgfpathmoveto{\pgfqpoint{1.876855in}{0.868456in}}%
\pgfusepath{stroke}%
\end{pgfscope}%
\begin{pgfscope}%
\pgfpathrectangle{\pgfqpoint{0.647939in}{0.492442in}}{\pgfqpoint{4.273799in}{2.331163in}}%
\pgfusepath{clip}%
\pgfsetroundcap%
\pgfsetroundjoin%
\definecolor{currentfill}{rgb}{0.500000,0.500000,0.500000}%
\pgfsetfillcolor{currentfill}%
\pgfsetfillopacity{0.300000}%
\pgfsetlinewidth{0.301125pt}%
\definecolor{currentstroke}{rgb}{0.500000,0.500000,0.500000}%
\pgfsetstrokecolor{currentstroke}%
\pgfsetstrokeopacity{0.300000}%
\pgfsetdash{}{0pt}%
\pgfpathmoveto{\pgfqpoint{0.000000in}{0.000000in}}%
\pgfpathlineto{\pgfqpoint{0.000000in}{0.000000in}}%
\pgfpathclose%
\pgfusepath{stroke,fill}%
\end{pgfscope}%
\begin{pgfscope}%
\pgfpathrectangle{\pgfqpoint{0.647939in}{0.492442in}}{\pgfqpoint{4.273799in}{2.331163in}}%
\pgfusepath{clip}%
\pgfsetroundcap%
\pgfsetroundjoin%
\pgfsetlinewidth{0.301125pt}%
\definecolor{currentstroke}{rgb}{0.500000,0.500000,0.500000}%
\pgfsetstrokecolor{currentstroke}%
\pgfsetstrokeopacity{0.300000}%
\pgfsetdash{}{0pt}%
\pgfpathmoveto{\pgfqpoint{3.695681in}{0.791413in}}%
\pgfusepath{stroke}%
\end{pgfscope}%
\begin{pgfscope}%
\pgfpathrectangle{\pgfqpoint{0.647939in}{0.492442in}}{\pgfqpoint{4.273799in}{2.331163in}}%
\pgfusepath{clip}%
\pgfsetroundcap%
\pgfsetroundjoin%
\definecolor{currentfill}{rgb}{0.500000,0.500000,0.500000}%
\pgfsetfillcolor{currentfill}%
\pgfsetfillopacity{0.300000}%
\pgfsetlinewidth{0.301125pt}%
\definecolor{currentstroke}{rgb}{0.500000,0.500000,0.500000}%
\pgfsetstrokecolor{currentstroke}%
\pgfsetstrokeopacity{0.300000}%
\pgfsetdash{}{0pt}%
\pgfpathmoveto{\pgfqpoint{0.000000in}{0.000000in}}%
\pgfpathlineto{\pgfqpoint{0.000000in}{0.000000in}}%
\pgfpathclose%
\pgfusepath{stroke,fill}%
\end{pgfscope}%
\begin{pgfscope}%
\pgfpathrectangle{\pgfqpoint{0.647939in}{0.492442in}}{\pgfqpoint{4.273799in}{2.331163in}}%
\pgfusepath{clip}%
\pgfsetroundcap%
\pgfsetroundjoin%
\pgfsetlinewidth{0.301125pt}%
\definecolor{currentstroke}{rgb}{0.500000,0.500000,0.500000}%
\pgfsetstrokecolor{currentstroke}%
\pgfsetstrokeopacity{0.300000}%
\pgfsetdash{}{0pt}%
\pgfpathmoveto{\pgfqpoint{4.049959in}{1.464148in}}%
\pgfusepath{stroke}%
\end{pgfscope}%
\begin{pgfscope}%
\pgfpathrectangle{\pgfqpoint{0.647939in}{0.492442in}}{\pgfqpoint{4.273799in}{2.331163in}}%
\pgfusepath{clip}%
\pgfsetroundcap%
\pgfsetroundjoin%
\definecolor{currentfill}{rgb}{0.500000,0.500000,0.500000}%
\pgfsetfillcolor{currentfill}%
\pgfsetfillopacity{0.300000}%
\pgfsetlinewidth{0.301125pt}%
\definecolor{currentstroke}{rgb}{0.500000,0.500000,0.500000}%
\pgfsetstrokecolor{currentstroke}%
\pgfsetstrokeopacity{0.300000}%
\pgfsetdash{}{0pt}%
\pgfpathmoveto{\pgfqpoint{0.000000in}{0.000000in}}%
\pgfpathlineto{\pgfqpoint{0.000000in}{0.000000in}}%
\pgfpathclose%
\pgfusepath{stroke,fill}%
\end{pgfscope}%
\begin{pgfscope}%
\pgfpathrectangle{\pgfqpoint{0.647939in}{0.492442in}}{\pgfqpoint{4.273799in}{2.331163in}}%
\pgfusepath{clip}%
\pgfsetroundcap%
\pgfsetroundjoin%
\pgfsetlinewidth{0.301125pt}%
\definecolor{currentstroke}{rgb}{0.500000,0.500000,0.500000}%
\pgfsetstrokecolor{currentstroke}%
\pgfsetstrokeopacity{0.300000}%
\pgfsetdash{}{0pt}%
\pgfpathmoveto{\pgfqpoint{4.268524in}{1.633495in}}%
\pgfusepath{stroke}%
\end{pgfscope}%
\begin{pgfscope}%
\pgfpathrectangle{\pgfqpoint{0.647939in}{0.492442in}}{\pgfqpoint{4.273799in}{2.331163in}}%
\pgfusepath{clip}%
\pgfsetroundcap%
\pgfsetroundjoin%
\definecolor{currentfill}{rgb}{0.500000,0.500000,0.500000}%
\pgfsetfillcolor{currentfill}%
\pgfsetfillopacity{0.300000}%
\pgfsetlinewidth{0.301125pt}%
\definecolor{currentstroke}{rgb}{0.500000,0.500000,0.500000}%
\pgfsetstrokecolor{currentstroke}%
\pgfsetstrokeopacity{0.300000}%
\pgfsetdash{}{0pt}%
\pgfpathmoveto{\pgfqpoint{0.000000in}{0.000000in}}%
\pgfpathlineto{\pgfqpoint{0.000000in}{0.000000in}}%
\pgfpathclose%
\pgfusepath{stroke,fill}%
\end{pgfscope}%
\begin{pgfscope}%
\pgfpathrectangle{\pgfqpoint{0.647939in}{0.492442in}}{\pgfqpoint{4.273799in}{2.331163in}}%
\pgfusepath{clip}%
\pgfsetroundcap%
\pgfsetroundjoin%
\pgfsetlinewidth{0.301125pt}%
\definecolor{currentstroke}{rgb}{0.500000,0.500000,0.500000}%
\pgfsetstrokecolor{currentstroke}%
\pgfsetstrokeopacity{0.300000}%
\pgfsetdash{}{0pt}%
\pgfpathmoveto{\pgfqpoint{4.448308in}{1.886068in}}%
\pgfusepath{stroke}%
\end{pgfscope}%
\begin{pgfscope}%
\pgfpathrectangle{\pgfqpoint{0.647939in}{0.492442in}}{\pgfqpoint{4.273799in}{2.331163in}}%
\pgfusepath{clip}%
\pgfsetroundcap%
\pgfsetroundjoin%
\definecolor{currentfill}{rgb}{0.500000,0.500000,0.500000}%
\pgfsetfillcolor{currentfill}%
\pgfsetfillopacity{0.300000}%
\pgfsetlinewidth{0.301125pt}%
\definecolor{currentstroke}{rgb}{0.500000,0.500000,0.500000}%
\pgfsetstrokecolor{currentstroke}%
\pgfsetstrokeopacity{0.300000}%
\pgfsetdash{}{0pt}%
\pgfpathmoveto{\pgfqpoint{0.000000in}{0.000000in}}%
\pgfpathlineto{\pgfqpoint{0.000000in}{0.000000in}}%
\pgfpathclose%
\pgfusepath{stroke,fill}%
\end{pgfscope}%
\begin{pgfscope}%
\pgfpathrectangle{\pgfqpoint{0.647939in}{0.492442in}}{\pgfqpoint{4.273799in}{2.331163in}}%
\pgfusepath{clip}%
\pgfsetroundcap%
\pgfsetroundjoin%
\pgfsetlinewidth{0.301125pt}%
\definecolor{currentstroke}{rgb}{0.500000,0.500000,0.500000}%
\pgfsetstrokecolor{currentstroke}%
\pgfsetstrokeopacity{0.300000}%
\pgfsetdash{}{0pt}%
\pgfpathmoveto{\pgfqpoint{2.065622in}{1.506208in}}%
\pgfusepath{stroke}%
\end{pgfscope}%
\begin{pgfscope}%
\pgfpathrectangle{\pgfqpoint{0.647939in}{0.492442in}}{\pgfqpoint{4.273799in}{2.331163in}}%
\pgfusepath{clip}%
\pgfsetroundcap%
\pgfsetroundjoin%
\definecolor{currentfill}{rgb}{0.500000,0.500000,0.500000}%
\pgfsetfillcolor{currentfill}%
\pgfsetfillopacity{0.300000}%
\pgfsetlinewidth{0.301125pt}%
\definecolor{currentstroke}{rgb}{0.500000,0.500000,0.500000}%
\pgfsetstrokecolor{currentstroke}%
\pgfsetstrokeopacity{0.300000}%
\pgfsetdash{}{0pt}%
\pgfpathmoveto{\pgfqpoint{0.000000in}{0.000000in}}%
\pgfpathlineto{\pgfqpoint{0.000000in}{0.000000in}}%
\pgfpathclose%
\pgfusepath{stroke,fill}%
\end{pgfscope}%
\begin{pgfscope}%
\pgfpathrectangle{\pgfqpoint{0.647939in}{0.492442in}}{\pgfqpoint{4.273799in}{2.331163in}}%
\pgfusepath{clip}%
\pgfsetroundcap%
\pgfsetroundjoin%
\pgfsetlinewidth{0.301125pt}%
\definecolor{currentstroke}{rgb}{0.500000,0.500000,0.500000}%
\pgfsetstrokecolor{currentstroke}%
\pgfsetstrokeopacity{0.300000}%
\pgfsetdash{}{0pt}%
\pgfpathmoveto{\pgfqpoint{3.259310in}{0.892573in}}%
\pgfusepath{stroke}%
\end{pgfscope}%
\begin{pgfscope}%
\pgfpathrectangle{\pgfqpoint{0.647939in}{0.492442in}}{\pgfqpoint{4.273799in}{2.331163in}}%
\pgfusepath{clip}%
\pgfsetroundcap%
\pgfsetroundjoin%
\definecolor{currentfill}{rgb}{0.500000,0.500000,0.500000}%
\pgfsetfillcolor{currentfill}%
\pgfsetfillopacity{0.300000}%
\pgfsetlinewidth{0.301125pt}%
\definecolor{currentstroke}{rgb}{0.500000,0.500000,0.500000}%
\pgfsetstrokecolor{currentstroke}%
\pgfsetstrokeopacity{0.300000}%
\pgfsetdash{}{0pt}%
\pgfpathmoveto{\pgfqpoint{0.000000in}{0.000000in}}%
\pgfpathlineto{\pgfqpoint{0.000000in}{0.000000in}}%
\pgfpathclose%
\pgfusepath{stroke,fill}%
\end{pgfscope}%
\begin{pgfscope}%
\pgfpathrectangle{\pgfqpoint{0.647939in}{0.492442in}}{\pgfqpoint{4.273799in}{2.331163in}}%
\pgfusepath{clip}%
\pgfsetroundcap%
\pgfsetroundjoin%
\pgfsetlinewidth{0.301125pt}%
\definecolor{currentstroke}{rgb}{0.500000,0.500000,0.500000}%
\pgfsetstrokecolor{currentstroke}%
\pgfsetstrokeopacity{0.300000}%
\pgfsetdash{}{0pt}%
\pgfpathmoveto{\pgfqpoint{1.632890in}{2.390136in}}%
\pgfusepath{stroke}%
\end{pgfscope}%
\begin{pgfscope}%
\pgfpathrectangle{\pgfqpoint{0.647939in}{0.492442in}}{\pgfqpoint{4.273799in}{2.331163in}}%
\pgfusepath{clip}%
\pgfsetroundcap%
\pgfsetroundjoin%
\definecolor{currentfill}{rgb}{0.500000,0.500000,0.500000}%
\pgfsetfillcolor{currentfill}%
\pgfsetfillopacity{0.300000}%
\pgfsetlinewidth{0.301125pt}%
\definecolor{currentstroke}{rgb}{0.500000,0.500000,0.500000}%
\pgfsetstrokecolor{currentstroke}%
\pgfsetstrokeopacity{0.300000}%
\pgfsetdash{}{0pt}%
\pgfpathmoveto{\pgfqpoint{0.000000in}{0.000000in}}%
\pgfpathlineto{\pgfqpoint{0.000000in}{0.000000in}}%
\pgfpathclose%
\pgfusepath{stroke,fill}%
\end{pgfscope}%
\begin{pgfscope}%
\pgfpathrectangle{\pgfqpoint{0.647939in}{0.492442in}}{\pgfqpoint{4.273799in}{2.331163in}}%
\pgfusepath{clip}%
\pgfsetroundcap%
\pgfsetroundjoin%
\pgfsetlinewidth{0.301125pt}%
\definecolor{currentstroke}{rgb}{0.500000,0.500000,0.500000}%
\pgfsetstrokecolor{currentstroke}%
\pgfsetstrokeopacity{0.300000}%
\pgfsetdash{}{0pt}%
\pgfpathmoveto{\pgfqpoint{1.457454in}{1.269761in}}%
\pgfusepath{stroke}%
\end{pgfscope}%
\begin{pgfscope}%
\pgfpathrectangle{\pgfqpoint{0.647939in}{0.492442in}}{\pgfqpoint{4.273799in}{2.331163in}}%
\pgfusepath{clip}%
\pgfsetroundcap%
\pgfsetroundjoin%
\definecolor{currentfill}{rgb}{0.500000,0.500000,0.500000}%
\pgfsetfillcolor{currentfill}%
\pgfsetfillopacity{0.300000}%
\pgfsetlinewidth{0.301125pt}%
\definecolor{currentstroke}{rgb}{0.500000,0.500000,0.500000}%
\pgfsetstrokecolor{currentstroke}%
\pgfsetstrokeopacity{0.300000}%
\pgfsetdash{}{0pt}%
\pgfpathmoveto{\pgfqpoint{0.000000in}{0.000000in}}%
\pgfpathlineto{\pgfqpoint{0.000000in}{0.000000in}}%
\pgfpathclose%
\pgfusepath{stroke,fill}%
\end{pgfscope}%
\begin{pgfscope}%
\pgfpathrectangle{\pgfqpoint{0.647939in}{0.492442in}}{\pgfqpoint{4.273799in}{2.331163in}}%
\pgfusepath{clip}%
\pgfsetroundcap%
\pgfsetroundjoin%
\pgfsetlinewidth{0.301125pt}%
\definecolor{currentstroke}{rgb}{0.500000,0.500000,0.500000}%
\pgfsetstrokecolor{currentstroke}%
\pgfsetstrokeopacity{0.300000}%
\pgfsetdash{}{0pt}%
\pgfpathmoveto{\pgfqpoint{1.456136in}{0.960639in}}%
\pgfusepath{stroke}%
\end{pgfscope}%
\begin{pgfscope}%
\pgfpathrectangle{\pgfqpoint{0.647939in}{0.492442in}}{\pgfqpoint{4.273799in}{2.331163in}}%
\pgfusepath{clip}%
\pgfsetroundcap%
\pgfsetroundjoin%
\definecolor{currentfill}{rgb}{0.500000,0.500000,0.500000}%
\pgfsetfillcolor{currentfill}%
\pgfsetfillopacity{0.300000}%
\pgfsetlinewidth{0.301125pt}%
\definecolor{currentstroke}{rgb}{0.500000,0.500000,0.500000}%
\pgfsetstrokecolor{currentstroke}%
\pgfsetstrokeopacity{0.300000}%
\pgfsetdash{}{0pt}%
\pgfpathmoveto{\pgfqpoint{0.000000in}{0.000000in}}%
\pgfpathlineto{\pgfqpoint{0.000000in}{0.000000in}}%
\pgfpathclose%
\pgfusepath{stroke,fill}%
\end{pgfscope}%
\begin{pgfscope}%
\pgfpathrectangle{\pgfqpoint{0.647939in}{0.492442in}}{\pgfqpoint{4.273799in}{2.331163in}}%
\pgfusepath{clip}%
\pgfsetroundcap%
\pgfsetroundjoin%
\pgfsetlinewidth{0.301125pt}%
\definecolor{currentstroke}{rgb}{0.500000,0.500000,0.500000}%
\pgfsetstrokecolor{currentstroke}%
\pgfsetstrokeopacity{0.300000}%
\pgfsetdash{}{0pt}%
\pgfpathmoveto{\pgfqpoint{3.283091in}{0.978904in}}%
\pgfusepath{stroke}%
\end{pgfscope}%
\begin{pgfscope}%
\pgfpathrectangle{\pgfqpoint{0.647939in}{0.492442in}}{\pgfqpoint{4.273799in}{2.331163in}}%
\pgfusepath{clip}%
\pgfsetroundcap%
\pgfsetroundjoin%
\definecolor{currentfill}{rgb}{0.500000,0.500000,0.500000}%
\pgfsetfillcolor{currentfill}%
\pgfsetfillopacity{0.300000}%
\pgfsetlinewidth{0.301125pt}%
\definecolor{currentstroke}{rgb}{0.500000,0.500000,0.500000}%
\pgfsetstrokecolor{currentstroke}%
\pgfsetstrokeopacity{0.300000}%
\pgfsetdash{}{0pt}%
\pgfpathmoveto{\pgfqpoint{0.000000in}{0.000000in}}%
\pgfpathlineto{\pgfqpoint{0.000000in}{0.000000in}}%
\pgfpathclose%
\pgfusepath{stroke,fill}%
\end{pgfscope}%
\begin{pgfscope}%
\pgfpathrectangle{\pgfqpoint{0.647939in}{0.492442in}}{\pgfqpoint{4.273799in}{2.331163in}}%
\pgfusepath{clip}%
\pgfsetroundcap%
\pgfsetroundjoin%
\pgfsetlinewidth{0.301125pt}%
\definecolor{currentstroke}{rgb}{0.500000,0.500000,0.500000}%
\pgfsetstrokecolor{currentstroke}%
\pgfsetstrokeopacity{0.300000}%
\pgfsetdash{}{0pt}%
\pgfpathmoveto{\pgfqpoint{3.828697in}{0.823273in}}%
\pgfusepath{stroke}%
\end{pgfscope}%
\begin{pgfscope}%
\pgfpathrectangle{\pgfqpoint{0.647939in}{0.492442in}}{\pgfqpoint{4.273799in}{2.331163in}}%
\pgfusepath{clip}%
\pgfsetroundcap%
\pgfsetroundjoin%
\definecolor{currentfill}{rgb}{0.500000,0.500000,0.500000}%
\pgfsetfillcolor{currentfill}%
\pgfsetfillopacity{0.300000}%
\pgfsetlinewidth{0.301125pt}%
\definecolor{currentstroke}{rgb}{0.500000,0.500000,0.500000}%
\pgfsetstrokecolor{currentstroke}%
\pgfsetstrokeopacity{0.300000}%
\pgfsetdash{}{0pt}%
\pgfpathmoveto{\pgfqpoint{0.000000in}{0.000000in}}%
\pgfpathlineto{\pgfqpoint{0.000000in}{0.000000in}}%
\pgfpathclose%
\pgfusepath{stroke,fill}%
\end{pgfscope}%
\begin{pgfscope}%
\pgfpathrectangle{\pgfqpoint{0.647939in}{0.492442in}}{\pgfqpoint{4.273799in}{2.331163in}}%
\pgfusepath{clip}%
\pgfsetroundcap%
\pgfsetroundjoin%
\pgfsetlinewidth{0.301125pt}%
\definecolor{currentstroke}{rgb}{0.500000,0.500000,0.500000}%
\pgfsetstrokecolor{currentstroke}%
\pgfsetstrokeopacity{0.300000}%
\pgfsetdash{}{0pt}%
\pgfpathmoveto{\pgfqpoint{4.080142in}{1.300957in}}%
\pgfusepath{stroke}%
\end{pgfscope}%
\begin{pgfscope}%
\pgfpathrectangle{\pgfqpoint{0.647939in}{0.492442in}}{\pgfqpoint{4.273799in}{2.331163in}}%
\pgfusepath{clip}%
\pgfsetroundcap%
\pgfsetroundjoin%
\definecolor{currentfill}{rgb}{0.500000,0.500000,0.500000}%
\pgfsetfillcolor{currentfill}%
\pgfsetfillopacity{0.300000}%
\pgfsetlinewidth{0.301125pt}%
\definecolor{currentstroke}{rgb}{0.500000,0.500000,0.500000}%
\pgfsetstrokecolor{currentstroke}%
\pgfsetstrokeopacity{0.300000}%
\pgfsetdash{}{0pt}%
\pgfpathmoveto{\pgfqpoint{0.000000in}{0.000000in}}%
\pgfpathlineto{\pgfqpoint{0.000000in}{0.000000in}}%
\pgfpathclose%
\pgfusepath{stroke,fill}%
\end{pgfscope}%
\begin{pgfscope}%
\pgfpathrectangle{\pgfqpoint{0.647939in}{0.492442in}}{\pgfqpoint{4.273799in}{2.331163in}}%
\pgfusepath{clip}%
\pgfsetroundcap%
\pgfsetroundjoin%
\pgfsetlinewidth{0.301125pt}%
\definecolor{currentstroke}{rgb}{0.500000,0.500000,0.500000}%
\pgfsetstrokecolor{currentstroke}%
\pgfsetstrokeopacity{0.300000}%
\pgfsetdash{}{0pt}%
\pgfpathmoveto{\pgfqpoint{4.315880in}{1.726432in}}%
\pgfusepath{stroke}%
\end{pgfscope}%
\begin{pgfscope}%
\pgfpathrectangle{\pgfqpoint{0.647939in}{0.492442in}}{\pgfqpoint{4.273799in}{2.331163in}}%
\pgfusepath{clip}%
\pgfsetroundcap%
\pgfsetroundjoin%
\definecolor{currentfill}{rgb}{0.500000,0.500000,0.500000}%
\pgfsetfillcolor{currentfill}%
\pgfsetfillopacity{0.300000}%
\pgfsetlinewidth{0.301125pt}%
\definecolor{currentstroke}{rgb}{0.500000,0.500000,0.500000}%
\pgfsetstrokecolor{currentstroke}%
\pgfsetstrokeopacity{0.300000}%
\pgfsetdash{}{0pt}%
\pgfpathmoveto{\pgfqpoint{0.000000in}{0.000000in}}%
\pgfpathlineto{\pgfqpoint{0.000000in}{0.000000in}}%
\pgfpathclose%
\pgfusepath{stroke,fill}%
\end{pgfscope}%
\begin{pgfscope}%
\pgfpathrectangle{\pgfqpoint{0.647939in}{0.492442in}}{\pgfqpoint{4.273799in}{2.331163in}}%
\pgfusepath{clip}%
\pgfsetroundcap%
\pgfsetroundjoin%
\pgfsetlinewidth{0.301125pt}%
\definecolor{currentstroke}{rgb}{0.500000,0.500000,0.500000}%
\pgfsetstrokecolor{currentstroke}%
\pgfsetstrokeopacity{0.300000}%
\pgfsetdash{}{0pt}%
\pgfpathmoveto{\pgfqpoint{1.738940in}{2.489521in}}%
\pgfusepath{stroke}%
\end{pgfscope}%
\begin{pgfscope}%
\pgfpathrectangle{\pgfqpoint{0.647939in}{0.492442in}}{\pgfqpoint{4.273799in}{2.331163in}}%
\pgfusepath{clip}%
\pgfsetroundcap%
\pgfsetroundjoin%
\definecolor{currentfill}{rgb}{0.500000,0.500000,0.500000}%
\pgfsetfillcolor{currentfill}%
\pgfsetfillopacity{0.300000}%
\pgfsetlinewidth{0.301125pt}%
\definecolor{currentstroke}{rgb}{0.500000,0.500000,0.500000}%
\pgfsetstrokecolor{currentstroke}%
\pgfsetstrokeopacity{0.300000}%
\pgfsetdash{}{0pt}%
\pgfpathmoveto{\pgfqpoint{0.000000in}{0.000000in}}%
\pgfpathlineto{\pgfqpoint{0.000000in}{0.000000in}}%
\pgfpathclose%
\pgfusepath{stroke,fill}%
\end{pgfscope}%
\begin{pgfscope}%
\pgfpathrectangle{\pgfqpoint{0.647939in}{0.492442in}}{\pgfqpoint{4.273799in}{2.331163in}}%
\pgfusepath{clip}%
\pgfsetroundcap%
\pgfsetroundjoin%
\pgfsetlinewidth{0.301125pt}%
\definecolor{currentstroke}{rgb}{0.500000,0.500000,0.500000}%
\pgfsetstrokecolor{currentstroke}%
\pgfsetstrokeopacity{0.300000}%
\pgfsetdash{}{0pt}%
\pgfpathmoveto{\pgfqpoint{4.147983in}{1.491196in}}%
\pgfusepath{stroke}%
\end{pgfscope}%
\begin{pgfscope}%
\pgfpathrectangle{\pgfqpoint{0.647939in}{0.492442in}}{\pgfqpoint{4.273799in}{2.331163in}}%
\pgfusepath{clip}%
\pgfsetroundcap%
\pgfsetroundjoin%
\definecolor{currentfill}{rgb}{0.500000,0.500000,0.500000}%
\pgfsetfillcolor{currentfill}%
\pgfsetfillopacity{0.300000}%
\pgfsetlinewidth{0.301125pt}%
\definecolor{currentstroke}{rgb}{0.500000,0.500000,0.500000}%
\pgfsetstrokecolor{currentstroke}%
\pgfsetstrokeopacity{0.300000}%
\pgfsetdash{}{0pt}%
\pgfpathmoveto{\pgfqpoint{0.000000in}{0.000000in}}%
\pgfpathlineto{\pgfqpoint{0.000000in}{0.000000in}}%
\pgfpathclose%
\pgfusepath{stroke,fill}%
\end{pgfscope}%
\begin{pgfscope}%
\pgfpathrectangle{\pgfqpoint{0.647939in}{0.492442in}}{\pgfqpoint{4.273799in}{2.331163in}}%
\pgfusepath{clip}%
\pgfsetroundcap%
\pgfsetroundjoin%
\pgfsetlinewidth{0.301125pt}%
\definecolor{currentstroke}{rgb}{0.500000,0.500000,0.500000}%
\pgfsetstrokecolor{currentstroke}%
\pgfsetstrokeopacity{0.300000}%
\pgfsetdash{}{0pt}%
\pgfpathmoveto{\pgfqpoint{1.353945in}{2.192049in}}%
\pgfusepath{stroke}%
\end{pgfscope}%
\begin{pgfscope}%
\pgfpathrectangle{\pgfqpoint{0.647939in}{0.492442in}}{\pgfqpoint{4.273799in}{2.331163in}}%
\pgfusepath{clip}%
\pgfsetroundcap%
\pgfsetroundjoin%
\definecolor{currentfill}{rgb}{0.500000,0.500000,0.500000}%
\pgfsetfillcolor{currentfill}%
\pgfsetfillopacity{0.300000}%
\pgfsetlinewidth{0.301125pt}%
\definecolor{currentstroke}{rgb}{0.500000,0.500000,0.500000}%
\pgfsetstrokecolor{currentstroke}%
\pgfsetstrokeopacity{0.300000}%
\pgfsetdash{}{0pt}%
\pgfpathmoveto{\pgfqpoint{0.000000in}{0.000000in}}%
\pgfpathlineto{\pgfqpoint{0.000000in}{0.000000in}}%
\pgfpathclose%
\pgfusepath{stroke,fill}%
\end{pgfscope}%
\begin{pgfscope}%
\pgfpathrectangle{\pgfqpoint{0.647939in}{0.492442in}}{\pgfqpoint{4.273799in}{2.331163in}}%
\pgfusepath{clip}%
\pgfsetroundcap%
\pgfsetroundjoin%
\pgfsetlinewidth{0.301125pt}%
\definecolor{currentstroke}{rgb}{0.500000,0.500000,0.500000}%
\pgfsetstrokecolor{currentstroke}%
\pgfsetstrokeopacity{0.300000}%
\pgfsetdash{}{0pt}%
\pgfpathmoveto{\pgfqpoint{1.427820in}{1.466232in}}%
\pgfusepath{stroke}%
\end{pgfscope}%
\begin{pgfscope}%
\pgfpathrectangle{\pgfqpoint{0.647939in}{0.492442in}}{\pgfqpoint{4.273799in}{2.331163in}}%
\pgfusepath{clip}%
\pgfsetroundcap%
\pgfsetroundjoin%
\definecolor{currentfill}{rgb}{0.500000,0.500000,0.500000}%
\pgfsetfillcolor{currentfill}%
\pgfsetfillopacity{0.300000}%
\pgfsetlinewidth{0.301125pt}%
\definecolor{currentstroke}{rgb}{0.500000,0.500000,0.500000}%
\pgfsetstrokecolor{currentstroke}%
\pgfsetstrokeopacity{0.300000}%
\pgfsetdash{}{0pt}%
\pgfpathmoveto{\pgfqpoint{0.000000in}{0.000000in}}%
\pgfpathlineto{\pgfqpoint{0.000000in}{0.000000in}}%
\pgfpathclose%
\pgfusepath{stroke,fill}%
\end{pgfscope}%
\begin{pgfscope}%
\pgfpathrectangle{\pgfqpoint{0.647939in}{0.492442in}}{\pgfqpoint{4.273799in}{2.331163in}}%
\pgfusepath{clip}%
\pgfsetroundcap%
\pgfsetroundjoin%
\pgfsetlinewidth{0.301125pt}%
\definecolor{currentstroke}{rgb}{0.500000,0.500000,0.500000}%
\pgfsetstrokecolor{currentstroke}%
\pgfsetstrokeopacity{0.300000}%
\pgfsetdash{}{0pt}%
\pgfpathmoveto{\pgfqpoint{3.984084in}{1.158606in}}%
\pgfusepath{stroke}%
\end{pgfscope}%
\begin{pgfscope}%
\pgfpathrectangle{\pgfqpoint{0.647939in}{0.492442in}}{\pgfqpoint{4.273799in}{2.331163in}}%
\pgfusepath{clip}%
\pgfsetroundcap%
\pgfsetroundjoin%
\definecolor{currentfill}{rgb}{0.500000,0.500000,0.500000}%
\pgfsetfillcolor{currentfill}%
\pgfsetfillopacity{0.300000}%
\pgfsetlinewidth{0.301125pt}%
\definecolor{currentstroke}{rgb}{0.500000,0.500000,0.500000}%
\pgfsetstrokecolor{currentstroke}%
\pgfsetstrokeopacity{0.300000}%
\pgfsetdash{}{0pt}%
\pgfpathmoveto{\pgfqpoint{0.000000in}{0.000000in}}%
\pgfpathlineto{\pgfqpoint{0.000000in}{0.000000in}}%
\pgfpathclose%
\pgfusepath{stroke,fill}%
\end{pgfscope}%
\begin{pgfscope}%
\pgfpathrectangle{\pgfqpoint{0.647939in}{0.492442in}}{\pgfqpoint{4.273799in}{2.331163in}}%
\pgfusepath{clip}%
\pgfsetroundcap%
\pgfsetroundjoin%
\pgfsetlinewidth{0.301125pt}%
\definecolor{currentstroke}{rgb}{0.500000,0.500000,0.500000}%
\pgfsetstrokecolor{currentstroke}%
\pgfsetstrokeopacity{0.300000}%
\pgfsetdash{}{0pt}%
\pgfpathmoveto{\pgfqpoint{1.486481in}{1.889804in}}%
\pgfusepath{stroke}%
\end{pgfscope}%
\begin{pgfscope}%
\pgfpathrectangle{\pgfqpoint{0.647939in}{0.492442in}}{\pgfqpoint{4.273799in}{2.331163in}}%
\pgfusepath{clip}%
\pgfsetroundcap%
\pgfsetroundjoin%
\definecolor{currentfill}{rgb}{0.500000,0.500000,0.500000}%
\pgfsetfillcolor{currentfill}%
\pgfsetfillopacity{0.300000}%
\pgfsetlinewidth{0.301125pt}%
\definecolor{currentstroke}{rgb}{0.500000,0.500000,0.500000}%
\pgfsetstrokecolor{currentstroke}%
\pgfsetstrokeopacity{0.300000}%
\pgfsetdash{}{0pt}%
\pgfpathmoveto{\pgfqpoint{0.000000in}{0.000000in}}%
\pgfpathlineto{\pgfqpoint{0.000000in}{0.000000in}}%
\pgfpathclose%
\pgfusepath{stroke,fill}%
\end{pgfscope}%
\begin{pgfscope}%
\pgfpathrectangle{\pgfqpoint{0.647939in}{0.492442in}}{\pgfqpoint{4.273799in}{2.331163in}}%
\pgfusepath{clip}%
\pgfsetroundcap%
\pgfsetroundjoin%
\pgfsetlinewidth{0.301125pt}%
\definecolor{currentstroke}{rgb}{0.500000,0.500000,0.500000}%
\pgfsetstrokecolor{currentstroke}%
\pgfsetstrokeopacity{0.300000}%
\pgfsetdash{}{0pt}%
\pgfpathmoveto{\pgfqpoint{1.955016in}{1.478528in}}%
\pgfusepath{stroke}%
\end{pgfscope}%
\begin{pgfscope}%
\pgfpathrectangle{\pgfqpoint{0.647939in}{0.492442in}}{\pgfqpoint{4.273799in}{2.331163in}}%
\pgfusepath{clip}%
\pgfsetroundcap%
\pgfsetroundjoin%
\definecolor{currentfill}{rgb}{0.500000,0.500000,0.500000}%
\pgfsetfillcolor{currentfill}%
\pgfsetfillopacity{0.300000}%
\pgfsetlinewidth{0.301125pt}%
\definecolor{currentstroke}{rgb}{0.500000,0.500000,0.500000}%
\pgfsetstrokecolor{currentstroke}%
\pgfsetstrokeopacity{0.300000}%
\pgfsetdash{}{0pt}%
\pgfpathmoveto{\pgfqpoint{0.000000in}{0.000000in}}%
\pgfpathlineto{\pgfqpoint{0.000000in}{0.000000in}}%
\pgfpathclose%
\pgfusepath{stroke,fill}%
\end{pgfscope}%
\begin{pgfscope}%
\pgfpathrectangle{\pgfqpoint{0.647939in}{0.492442in}}{\pgfqpoint{4.273799in}{2.331163in}}%
\pgfusepath{clip}%
\pgfsetroundcap%
\pgfsetroundjoin%
\pgfsetlinewidth{0.301125pt}%
\definecolor{currentstroke}{rgb}{0.500000,0.500000,0.500000}%
\pgfsetstrokecolor{currentstroke}%
\pgfsetstrokeopacity{0.300000}%
\pgfsetdash{}{0pt}%
\pgfpathmoveto{\pgfqpoint{3.500677in}{1.986173in}}%
\pgfusepath{stroke}%
\end{pgfscope}%
\begin{pgfscope}%
\pgfpathrectangle{\pgfqpoint{0.647939in}{0.492442in}}{\pgfqpoint{4.273799in}{2.331163in}}%
\pgfusepath{clip}%
\pgfsetroundcap%
\pgfsetroundjoin%
\definecolor{currentfill}{rgb}{0.500000,0.500000,0.500000}%
\pgfsetfillcolor{currentfill}%
\pgfsetfillopacity{0.300000}%
\pgfsetlinewidth{0.301125pt}%
\definecolor{currentstroke}{rgb}{0.500000,0.500000,0.500000}%
\pgfsetstrokecolor{currentstroke}%
\pgfsetstrokeopacity{0.300000}%
\pgfsetdash{}{0pt}%
\pgfpathmoveto{\pgfqpoint{0.000000in}{0.000000in}}%
\pgfpathlineto{\pgfqpoint{0.000000in}{0.000000in}}%
\pgfpathclose%
\pgfusepath{stroke,fill}%
\end{pgfscope}%
\begin{pgfscope}%
\pgfpathrectangle{\pgfqpoint{0.647939in}{0.492442in}}{\pgfqpoint{4.273799in}{2.331163in}}%
\pgfusepath{clip}%
\pgfsetroundcap%
\pgfsetroundjoin%
\pgfsetlinewidth{0.301125pt}%
\definecolor{currentstroke}{rgb}{0.500000,0.500000,0.500000}%
\pgfsetstrokecolor{currentstroke}%
\pgfsetstrokeopacity{0.300000}%
\pgfsetdash{}{0pt}%
\pgfpathmoveto{\pgfqpoint{2.284227in}{2.242116in}}%
\pgfusepath{stroke}%
\end{pgfscope}%
\begin{pgfscope}%
\pgfpathrectangle{\pgfqpoint{0.647939in}{0.492442in}}{\pgfqpoint{4.273799in}{2.331163in}}%
\pgfusepath{clip}%
\pgfsetroundcap%
\pgfsetroundjoin%
\definecolor{currentfill}{rgb}{0.500000,0.500000,0.500000}%
\pgfsetfillcolor{currentfill}%
\pgfsetfillopacity{0.300000}%
\pgfsetlinewidth{0.301125pt}%
\definecolor{currentstroke}{rgb}{0.500000,0.500000,0.500000}%
\pgfsetstrokecolor{currentstroke}%
\pgfsetstrokeopacity{0.300000}%
\pgfsetdash{}{0pt}%
\pgfpathmoveto{\pgfqpoint{0.000000in}{0.000000in}}%
\pgfpathlineto{\pgfqpoint{0.000000in}{0.000000in}}%
\pgfpathclose%
\pgfusepath{stroke,fill}%
\end{pgfscope}%
\begin{pgfscope}%
\pgfpathrectangle{\pgfqpoint{0.647939in}{0.492442in}}{\pgfqpoint{4.273799in}{2.331163in}}%
\pgfusepath{clip}%
\pgfsetroundcap%
\pgfsetroundjoin%
\pgfsetlinewidth{0.301125pt}%
\definecolor{currentstroke}{rgb}{0.500000,0.500000,0.500000}%
\pgfsetstrokecolor{currentstroke}%
\pgfsetstrokeopacity{0.300000}%
\pgfsetdash{}{0pt}%
\pgfpathmoveto{\pgfqpoint{1.536458in}{2.189619in}}%
\pgfusepath{stroke}%
\end{pgfscope}%
\begin{pgfscope}%
\pgfpathrectangle{\pgfqpoint{0.647939in}{0.492442in}}{\pgfqpoint{4.273799in}{2.331163in}}%
\pgfusepath{clip}%
\pgfsetroundcap%
\pgfsetroundjoin%
\definecolor{currentfill}{rgb}{0.500000,0.500000,0.500000}%
\pgfsetfillcolor{currentfill}%
\pgfsetfillopacity{0.300000}%
\pgfsetlinewidth{0.301125pt}%
\definecolor{currentstroke}{rgb}{0.500000,0.500000,0.500000}%
\pgfsetstrokecolor{currentstroke}%
\pgfsetstrokeopacity{0.300000}%
\pgfsetdash{}{0pt}%
\pgfpathmoveto{\pgfqpoint{0.000000in}{0.000000in}}%
\pgfpathlineto{\pgfqpoint{0.000000in}{0.000000in}}%
\pgfpathclose%
\pgfusepath{stroke,fill}%
\end{pgfscope}%
\begin{pgfscope}%
\pgfpathrectangle{\pgfqpoint{0.647939in}{0.492442in}}{\pgfqpoint{4.273799in}{2.331163in}}%
\pgfusepath{clip}%
\pgfsetroundcap%
\pgfsetroundjoin%
\pgfsetlinewidth{0.301125pt}%
\definecolor{currentstroke}{rgb}{0.500000,0.500000,0.500000}%
\pgfsetstrokecolor{currentstroke}%
\pgfsetstrokeopacity{0.300000}%
\pgfsetdash{}{0pt}%
\pgfpathmoveto{\pgfqpoint{1.692051in}{1.667888in}}%
\pgfusepath{stroke}%
\end{pgfscope}%
\begin{pgfscope}%
\pgfpathrectangle{\pgfqpoint{0.647939in}{0.492442in}}{\pgfqpoint{4.273799in}{2.331163in}}%
\pgfusepath{clip}%
\pgfsetroundcap%
\pgfsetroundjoin%
\definecolor{currentfill}{rgb}{0.500000,0.500000,0.500000}%
\pgfsetfillcolor{currentfill}%
\pgfsetfillopacity{0.300000}%
\pgfsetlinewidth{0.301125pt}%
\definecolor{currentstroke}{rgb}{0.500000,0.500000,0.500000}%
\pgfsetstrokecolor{currentstroke}%
\pgfsetstrokeopacity{0.300000}%
\pgfsetdash{}{0pt}%
\pgfpathmoveto{\pgfqpoint{0.000000in}{0.000000in}}%
\pgfpathlineto{\pgfqpoint{0.000000in}{0.000000in}}%
\pgfpathclose%
\pgfusepath{stroke,fill}%
\end{pgfscope}%
\begin{pgfscope}%
\pgfpathrectangle{\pgfqpoint{0.647939in}{0.492442in}}{\pgfqpoint{4.273799in}{2.331163in}}%
\pgfusepath{clip}%
\pgfsetroundcap%
\pgfsetroundjoin%
\pgfsetlinewidth{0.301125pt}%
\definecolor{currentstroke}{rgb}{0.500000,0.500000,0.500000}%
\pgfsetstrokecolor{currentstroke}%
\pgfsetstrokeopacity{0.300000}%
\pgfsetdash{}{0pt}%
\pgfpathmoveto{\pgfqpoint{3.552155in}{1.908172in}}%
\pgfusepath{stroke}%
\end{pgfscope}%
\begin{pgfscope}%
\pgfpathrectangle{\pgfqpoint{0.647939in}{0.492442in}}{\pgfqpoint{4.273799in}{2.331163in}}%
\pgfusepath{clip}%
\pgfsetroundcap%
\pgfsetroundjoin%
\definecolor{currentfill}{rgb}{0.500000,0.500000,0.500000}%
\pgfsetfillcolor{currentfill}%
\pgfsetfillopacity{0.300000}%
\pgfsetlinewidth{0.301125pt}%
\definecolor{currentstroke}{rgb}{0.500000,0.500000,0.500000}%
\pgfsetstrokecolor{currentstroke}%
\pgfsetstrokeopacity{0.300000}%
\pgfsetdash{}{0pt}%
\pgfpathmoveto{\pgfqpoint{0.000000in}{0.000000in}}%
\pgfpathlineto{\pgfqpoint{0.000000in}{0.000000in}}%
\pgfpathclose%
\pgfusepath{stroke,fill}%
\end{pgfscope}%
\begin{pgfscope}%
\pgfpathrectangle{\pgfqpoint{0.647939in}{0.492442in}}{\pgfqpoint{4.273799in}{2.331163in}}%
\pgfusepath{clip}%
\pgfsetroundcap%
\pgfsetroundjoin%
\pgfsetlinewidth{0.301125pt}%
\definecolor{currentstroke}{rgb}{0.500000,0.500000,0.500000}%
\pgfsetstrokecolor{currentstroke}%
\pgfsetstrokeopacity{0.300000}%
\pgfsetdash{}{0pt}%
\pgfpathmoveto{\pgfqpoint{1.651668in}{1.792305in}}%
\pgfusepath{stroke}%
\end{pgfscope}%
\begin{pgfscope}%
\pgfpathrectangle{\pgfqpoint{0.647939in}{0.492442in}}{\pgfqpoint{4.273799in}{2.331163in}}%
\pgfusepath{clip}%
\pgfsetroundcap%
\pgfsetroundjoin%
\definecolor{currentfill}{rgb}{0.500000,0.500000,0.500000}%
\pgfsetfillcolor{currentfill}%
\pgfsetfillopacity{0.300000}%
\pgfsetlinewidth{0.301125pt}%
\definecolor{currentstroke}{rgb}{0.500000,0.500000,0.500000}%
\pgfsetstrokecolor{currentstroke}%
\pgfsetstrokeopacity{0.300000}%
\pgfsetdash{}{0pt}%
\pgfpathmoveto{\pgfqpoint{0.000000in}{0.000000in}}%
\pgfpathlineto{\pgfqpoint{0.000000in}{0.000000in}}%
\pgfpathclose%
\pgfusepath{stroke,fill}%
\end{pgfscope}%
\begin{pgfscope}%
\pgfpathrectangle{\pgfqpoint{0.647939in}{0.492442in}}{\pgfqpoint{4.273799in}{2.331163in}}%
\pgfusepath{clip}%
\pgfsetroundcap%
\pgfsetroundjoin%
\pgfsetlinewidth{0.301125pt}%
\definecolor{currentstroke}{rgb}{0.500000,0.500000,0.500000}%
\pgfsetstrokecolor{currentstroke}%
\pgfsetstrokeopacity{0.300000}%
\pgfsetdash{}{0pt}%
\pgfpathmoveto{\pgfqpoint{3.649153in}{1.931325in}}%
\pgfusepath{stroke}%
\end{pgfscope}%
\begin{pgfscope}%
\pgfpathrectangle{\pgfqpoint{0.647939in}{0.492442in}}{\pgfqpoint{4.273799in}{2.331163in}}%
\pgfusepath{clip}%
\pgfsetroundcap%
\pgfsetroundjoin%
\definecolor{currentfill}{rgb}{0.500000,0.500000,0.500000}%
\pgfsetfillcolor{currentfill}%
\pgfsetfillopacity{0.300000}%
\pgfsetlinewidth{0.301125pt}%
\definecolor{currentstroke}{rgb}{0.500000,0.500000,0.500000}%
\pgfsetstrokecolor{currentstroke}%
\pgfsetstrokeopacity{0.300000}%
\pgfsetdash{}{0pt}%
\pgfpathmoveto{\pgfqpoint{0.000000in}{0.000000in}}%
\pgfpathlineto{\pgfqpoint{0.000000in}{0.000000in}}%
\pgfpathclose%
\pgfusepath{stroke,fill}%
\end{pgfscope}%
\begin{pgfscope}%
\pgfpathrectangle{\pgfqpoint{0.647939in}{0.492442in}}{\pgfqpoint{4.273799in}{2.331163in}}%
\pgfusepath{clip}%
\pgfsetroundcap%
\pgfsetroundjoin%
\pgfsetlinewidth{0.301125pt}%
\definecolor{currentstroke}{rgb}{0.500000,0.500000,0.500000}%
\pgfsetstrokecolor{currentstroke}%
\pgfsetstrokeopacity{0.300000}%
\pgfsetdash{}{0pt}%
\pgfpathmoveto{\pgfqpoint{2.586846in}{2.076545in}}%
\pgfusepath{stroke}%
\end{pgfscope}%
\begin{pgfscope}%
\pgfpathrectangle{\pgfqpoint{0.647939in}{0.492442in}}{\pgfqpoint{4.273799in}{2.331163in}}%
\pgfusepath{clip}%
\pgfsetroundcap%
\pgfsetroundjoin%
\definecolor{currentfill}{rgb}{0.500000,0.500000,0.500000}%
\pgfsetfillcolor{currentfill}%
\pgfsetfillopacity{0.300000}%
\pgfsetlinewidth{0.301125pt}%
\definecolor{currentstroke}{rgb}{0.500000,0.500000,0.500000}%
\pgfsetstrokecolor{currentstroke}%
\pgfsetstrokeopacity{0.300000}%
\pgfsetdash{}{0pt}%
\pgfpathmoveto{\pgfqpoint{0.000000in}{0.000000in}}%
\pgfpathlineto{\pgfqpoint{0.000000in}{0.000000in}}%
\pgfpathclose%
\pgfusepath{stroke,fill}%
\end{pgfscope}%
\begin{pgfscope}%
\pgfpathrectangle{\pgfqpoint{0.647939in}{0.492442in}}{\pgfqpoint{4.273799in}{2.331163in}}%
\pgfusepath{clip}%
\pgfsetroundcap%
\pgfsetroundjoin%
\pgfsetlinewidth{0.301125pt}%
\definecolor{currentstroke}{rgb}{0.500000,0.500000,0.500000}%
\pgfsetstrokecolor{currentstroke}%
\pgfsetstrokeopacity{0.300000}%
\pgfsetdash{}{0pt}%
\pgfpathmoveto{\pgfqpoint{2.357665in}{1.653808in}}%
\pgfusepath{stroke}%
\end{pgfscope}%
\begin{pgfscope}%
\pgfpathrectangle{\pgfqpoint{0.647939in}{0.492442in}}{\pgfqpoint{4.273799in}{2.331163in}}%
\pgfusepath{clip}%
\pgfsetroundcap%
\pgfsetroundjoin%
\definecolor{currentfill}{rgb}{0.500000,0.500000,0.500000}%
\pgfsetfillcolor{currentfill}%
\pgfsetfillopacity{0.300000}%
\pgfsetlinewidth{0.301125pt}%
\definecolor{currentstroke}{rgb}{0.500000,0.500000,0.500000}%
\pgfsetstrokecolor{currentstroke}%
\pgfsetstrokeopacity{0.300000}%
\pgfsetdash{}{0pt}%
\pgfpathmoveto{\pgfqpoint{0.000000in}{0.000000in}}%
\pgfpathlineto{\pgfqpoint{0.000000in}{0.000000in}}%
\pgfpathclose%
\pgfusepath{stroke,fill}%
\end{pgfscope}%
\begin{pgfscope}%
\pgfpathrectangle{\pgfqpoint{0.647939in}{0.492442in}}{\pgfqpoint{4.273799in}{2.331163in}}%
\pgfusepath{clip}%
\pgfsetroundcap%
\pgfsetroundjoin%
\pgfsetlinewidth{0.301125pt}%
\definecolor{currentstroke}{rgb}{0.500000,0.500000,0.500000}%
\pgfsetstrokecolor{currentstroke}%
\pgfsetstrokeopacity{0.300000}%
\pgfsetdash{}{0pt}%
\pgfpathmoveto{\pgfqpoint{2.766784in}{1.361248in}}%
\pgfusepath{stroke}%
\end{pgfscope}%
\begin{pgfscope}%
\pgfpathrectangle{\pgfqpoint{0.647939in}{0.492442in}}{\pgfqpoint{4.273799in}{2.331163in}}%
\pgfusepath{clip}%
\pgfsetroundcap%
\pgfsetroundjoin%
\definecolor{currentfill}{rgb}{0.500000,0.500000,0.500000}%
\pgfsetfillcolor{currentfill}%
\pgfsetfillopacity{0.300000}%
\pgfsetlinewidth{0.301125pt}%
\definecolor{currentstroke}{rgb}{0.500000,0.500000,0.500000}%
\pgfsetstrokecolor{currentstroke}%
\pgfsetstrokeopacity{0.300000}%
\pgfsetdash{}{0pt}%
\pgfpathmoveto{\pgfqpoint{0.000000in}{0.000000in}}%
\pgfpathlineto{\pgfqpoint{0.000000in}{0.000000in}}%
\pgfpathclose%
\pgfusepath{stroke,fill}%
\end{pgfscope}%
\begin{pgfscope}%
\pgfpathrectangle{\pgfqpoint{0.647939in}{0.492442in}}{\pgfqpoint{4.273799in}{2.331163in}}%
\pgfusepath{clip}%
\pgfsetroundcap%
\pgfsetroundjoin%
\pgfsetlinewidth{0.301125pt}%
\definecolor{currentstroke}{rgb}{0.500000,0.500000,0.500000}%
\pgfsetstrokecolor{currentstroke}%
\pgfsetstrokeopacity{0.300000}%
\pgfsetdash{}{0pt}%
\pgfpathmoveto{\pgfqpoint{2.704644in}{1.738416in}}%
\pgfusepath{stroke}%
\end{pgfscope}%
\begin{pgfscope}%
\pgfpathrectangle{\pgfqpoint{0.647939in}{0.492442in}}{\pgfqpoint{4.273799in}{2.331163in}}%
\pgfusepath{clip}%
\pgfsetroundcap%
\pgfsetroundjoin%
\definecolor{currentfill}{rgb}{0.500000,0.500000,0.500000}%
\pgfsetfillcolor{currentfill}%
\pgfsetfillopacity{0.300000}%
\pgfsetlinewidth{0.301125pt}%
\definecolor{currentstroke}{rgb}{0.500000,0.500000,0.500000}%
\pgfsetstrokecolor{currentstroke}%
\pgfsetstrokeopacity{0.300000}%
\pgfsetdash{}{0pt}%
\pgfpathmoveto{\pgfqpoint{0.000000in}{0.000000in}}%
\pgfpathlineto{\pgfqpoint{0.000000in}{0.000000in}}%
\pgfpathclose%
\pgfusepath{stroke,fill}%
\end{pgfscope}%
\begin{pgfscope}%
\pgfpathrectangle{\pgfqpoint{0.647939in}{0.492442in}}{\pgfqpoint{4.273799in}{2.331163in}}%
\pgfusepath{clip}%
\pgfsetbuttcap%
\pgfsetroundjoin%
\pgfsetlinewidth{0.301125pt}%
\definecolor{currentstroke}{rgb}{0.500000,0.500000,0.500000}%
\pgfsetstrokecolor{currentstroke}%
\pgfsetstrokeopacity{0.300000}%
\pgfsetdash{}{0pt}%
\pgfpathmoveto{\pgfqpoint{2.280996in}{0.492442in}}%
\pgfpathlineto{\pgfqpoint{2.273668in}{0.501913in}}%
\pgfpathlineto{\pgfqpoint{2.236891in}{0.549675in}}%
\pgfpathlineto{\pgfqpoint{2.200395in}{0.597501in}}%
\pgfpathlineto{\pgfqpoint{2.164139in}{0.645381in}}%
\pgfpathlineto{\pgfqpoint{2.128073in}{0.693305in}}%
\pgfpathlineto{\pgfqpoint{2.092151in}{0.741259in}}%
\pgfpathlineto{\pgfqpoint{2.056317in}{0.789234in}}%
\pgfpathlineto{\pgfqpoint{2.020513in}{0.837216in}}%
\pgfpathlineto{\pgfqpoint{1.984669in}{0.885188in}}%
\pgfpathlineto{\pgfqpoint{1.948691in}{0.933131in}}%
\pgfpathlineto{\pgfqpoint{1.912468in}{0.981018in}}%
\pgfpathlineto{\pgfqpoint{1.875865in}{1.028819in}}%
\pgfpathlineto{\pgfqpoint{1.838715in}{1.076493in}}%
\pgfpathlineto{\pgfqpoint{1.800795in}{1.123986in}}%
\pgfpathlineto{\pgfqpoint{1.761801in}{1.171218in}}%
\pgfpathlineto{\pgfqpoint{1.721291in}{1.218067in}}%
\pgfpathlineto{\pgfqpoint{1.678599in}{1.264330in}}%
\pgfpathlineto{\pgfqpoint{1.632636in}{1.309635in}}%
\pgfpathlineto{\pgfqpoint{1.581463in}{1.353187in}}%
\pgfpathlineto{\pgfqpoint{1.533589in}{1.385993in}}%
\pgfpathlineto{\pgfqpoint{1.490874in}{1.407717in}}%
\pgfpathlineto{\pgfqpoint{1.449487in}{1.421178in}}%
\pgfpathlineto{\pgfqpoint{1.401400in}{1.426720in}}%
\pgfpathlineto{\pgfqpoint{1.353367in}{1.421155in}}%
\pgfpathlineto{\pgfqpoint{1.353367in}{1.421155in}}%
\pgfpathlineto{\pgfqpoint{1.300170in}{1.402240in}}%
\pgfpathlineto{\pgfqpoint{1.300170in}{1.402240in}}%
\pgfpathlineto{\pgfqpoint{1.237332in}{1.363803in}}%
\pgfpathlineto{\pgfqpoint{1.185510in}{1.320600in}}%
\pgfpathlineto{\pgfqpoint{1.140059in}{1.275221in}}%
\pgfpathlineto{\pgfqpoint{1.098776in}{1.228634in}}%
\pgfpathlineto{\pgfqpoint{1.060448in}{1.181282in}}%
\pgfpathlineto{\pgfqpoint{1.024335in}{1.133400in}}%
\pgfpathlineto{\pgfqpoint{0.989952in}{1.085127in}}%
\pgfpathlineto{\pgfqpoint{0.956984in}{1.036555in}}%
\pgfpathlineto{\pgfqpoint{0.925215in}{0.987749in}}%
\pgfpathlineto{\pgfqpoint{0.894461in}{0.938750in}}%
\pgfpathlineto{\pgfqpoint{0.864572in}{0.889585in}}%
\pgfpathlineto{\pgfqpoint{0.835459in}{0.840282in}}%
\pgfpathlineto{\pgfqpoint{0.807047in}{0.790860in}}%
\pgfpathlineto{\pgfqpoint{0.779249in}{0.741330in}}%
\pgfpathlineto{\pgfqpoint{0.752020in}{0.691706in}}%
\pgfpathlineto{\pgfqpoint{0.725315in}{0.641999in}}%
\pgfpathlineto{\pgfqpoint{0.699087in}{0.592214in}}%
\pgfpathlineto{\pgfqpoint{0.673308in}{0.542360in}}%
\pgfpathlineto{\pgfqpoint{0.647939in}{0.492442in}}%
\pgfpathlineto{\pgfqpoint{0.647939in}{0.492442in}}%
\pgfusepath{stroke}%
\end{pgfscope}%
\begin{pgfscope}%
\pgfpathrectangle{\pgfqpoint{0.647939in}{0.492442in}}{\pgfqpoint{4.273799in}{2.331163in}}%
\pgfusepath{clip}%
\pgfsetbuttcap%
\pgfsetroundjoin%
\pgfsetlinewidth{0.301125pt}%
\definecolor{currentstroke}{rgb}{0.500000,0.500000,0.500000}%
\pgfsetstrokecolor{currentstroke}%
\pgfsetstrokeopacity{0.300000}%
\pgfsetdash{}{0pt}%
\pgfpathmoveto{\pgfqpoint{1.829232in}{0.492442in}}%
\pgfpathlineto{\pgfqpoint{1.821528in}{0.501367in}}%
\pgfpathlineto{\pgfqpoint{1.780638in}{0.548122in}}%
\pgfpathlineto{\pgfqpoint{1.738743in}{0.594610in}}%
\pgfpathlineto{\pgfqpoint{1.695537in}{0.640739in}}%
\pgfpathlineto{\pgfqpoint{1.650602in}{0.686370in}}%
\pgfpathlineto{\pgfqpoint{1.603348in}{0.731295in}}%
\pgfpathlineto{\pgfqpoint{1.552900in}{0.775165in}}%
\pgfpathlineto{\pgfqpoint{1.497869in}{0.817327in}}%
\pgfpathlineto{\pgfqpoint{1.437911in}{0.855365in}}%
\pgfpathlineto{\pgfqpoint{1.384006in}{0.881617in}}%
\pgfpathlineto{\pgfqpoint{1.333126in}{0.898698in}}%
\pgfpathlineto{\pgfqpoint{1.279592in}{0.907902in}}%
\pgfpathlineto{\pgfqpoint{1.216160in}{0.906457in}}%
\pgfpathlineto{\pgfqpoint{1.157555in}{0.892920in}}%
\pgfpathlineto{\pgfqpoint{1.157555in}{0.892920in}}%
\pgfpathlineto{\pgfqpoint{1.085454in}{0.859979in}}%
\pgfpathlineto{\pgfqpoint{1.025868in}{0.819820in}}%
\pgfpathlineto{\pgfqpoint{0.974629in}{0.776317in}}%
\pgfpathlineto{\pgfqpoint{0.928975in}{0.730960in}}%
\pgfpathlineto{\pgfqpoint{0.887317in}{0.684450in}}%
\pgfpathlineto{\pgfqpoint{0.848662in}{0.637162in}}%
\pgfpathlineto{\pgfqpoint{0.812352in}{0.589313in}}%
\pgfpathlineto{\pgfqpoint{0.777926in}{0.541041in}}%
\pgfpathlineto{\pgfqpoint{0.745071in}{0.492442in}}%
\pgfpathlineto{\pgfqpoint{0.745071in}{0.492442in}}%
\pgfusepath{stroke}%
\end{pgfscope}%
\begin{pgfscope}%
\pgfpathrectangle{\pgfqpoint{0.647939in}{0.492442in}}{\pgfqpoint{4.273799in}{2.331163in}}%
\pgfusepath{clip}%
\pgfsetbuttcap%
\pgfsetroundjoin%
\pgfsetlinewidth{0.301125pt}%
\definecolor{currentstroke}{rgb}{0.500000,0.500000,0.500000}%
\pgfsetstrokecolor{currentstroke}%
\pgfsetstrokeopacity{0.300000}%
\pgfsetdash{}{0pt}%
\pgfpathmoveto{\pgfqpoint{1.593607in}{0.492442in}}%
\pgfpathlineto{\pgfqpoint{1.578791in}{0.505666in}}%
\pgfpathlineto{\pgfqpoint{1.527560in}{0.549261in}}%
\pgfpathlineto{\pgfqpoint{1.472115in}{0.591275in}}%
\pgfpathlineto{\pgfqpoint{1.410578in}{0.630603in}}%
\pgfpathlineto{\pgfqpoint{1.352463in}{0.659761in}}%
\pgfpathlineto{\pgfqpoint{1.298243in}{0.679195in}}%
\pgfpathlineto{\pgfqpoint{1.243477in}{0.690518in}}%
\pgfpathlineto{\pgfqpoint{1.181830in}{0.692582in}}%
\pgfpathlineto{\pgfqpoint{1.122242in}{0.683411in}}%
\pgfpathlineto{\pgfqpoint{1.122242in}{0.683411in}}%
\pgfpathlineto{\pgfqpoint{1.058017in}{0.660886in}}%
\pgfpathlineto{\pgfqpoint{1.058017in}{0.660886in}}%
\pgfpathlineto{\pgfqpoint{0.991859in}{0.624088in}}%
\pgfpathlineto{\pgfqpoint{0.935782in}{0.582413in}}%
\pgfpathlineto{\pgfqpoint{0.886575in}{0.538191in}}%
\pgfpathlineto{\pgfqpoint{0.842203in}{0.492442in}}%
\pgfpathlineto{\pgfqpoint{0.842203in}{0.492442in}}%
\pgfusepath{stroke}%
\end{pgfscope}%
\begin{pgfscope}%
\pgfpathrectangle{\pgfqpoint{0.647939in}{0.492442in}}{\pgfqpoint{4.273799in}{2.331163in}}%
\pgfusepath{clip}%
\pgfsetbuttcap%
\pgfsetroundjoin%
\pgfsetlinewidth{0.301125pt}%
\definecolor{currentstroke}{rgb}{0.500000,0.500000,0.500000}%
\pgfsetstrokecolor{currentstroke}%
\pgfsetstrokeopacity{0.300000}%
\pgfsetdash{}{0pt}%
\pgfpathmoveto{\pgfqpoint{1.427957in}{0.492442in}}%
\pgfpathlineto{\pgfqpoint{1.374928in}{0.520620in}}%
\pgfpathlineto{\pgfqpoint{1.316555in}{0.547503in}}%
\pgfpathlineto{\pgfqpoint{1.261299in}{0.565120in}}%
\pgfpathlineto{\pgfqpoint{1.204186in}{0.574672in}}%
\pgfpathlineto{\pgfqpoint{1.138764in}{0.574075in}}%
\pgfpathlineto{\pgfqpoint{1.077341in}{0.561882in}}%
\pgfpathlineto{\pgfqpoint{1.077341in}{0.561882in}}%
\pgfpathlineto{\pgfqpoint{1.002051in}{0.531115in}}%
\pgfpathlineto{\pgfqpoint{0.939334in}{0.492442in}}%
\pgfpathlineto{\pgfqpoint{0.939334in}{0.492442in}}%
\pgfusepath{stroke}%
\end{pgfscope}%
\begin{pgfscope}%
\pgfpathrectangle{\pgfqpoint{0.647939in}{0.492442in}}{\pgfqpoint{4.273799in}{2.331163in}}%
\pgfusepath{clip}%
\pgfsetbuttcap%
\pgfsetroundjoin%
\pgfsetlinewidth{0.301125pt}%
\definecolor{currentstroke}{rgb}{0.500000,0.500000,0.500000}%
\pgfsetstrokecolor{currentstroke}%
\pgfsetstrokeopacity{0.300000}%
\pgfsetdash{}{0pt}%
\pgfpathmoveto{\pgfqpoint{1.716389in}{0.492442in}}%
\pgfpathlineto{\pgfqpoint{1.716389in}{0.492442in}}%
\pgfpathlineto{\pgfqpoint{1.672230in}{0.538301in}}%
\pgfpathlineto{\pgfqpoint{1.626221in}{0.583611in}}%
\pgfpathlineto{\pgfqpoint{1.577740in}{0.628141in}}%
\pgfpathlineto{\pgfqpoint{1.525869in}{0.671507in}}%
\pgfpathlineto{\pgfqpoint{1.469184in}{0.713018in}}%
\pgfpathlineto{\pgfqpoint{1.405353in}{0.751214in}}%
\pgfpathlineto{\pgfqpoint{1.330658in}{0.782656in}}%
\pgfpathlineto{\pgfqpoint{1.330658in}{0.782656in}}%
\pgfpathlineto{\pgfqpoint{1.270619in}{0.796383in}}%
\pgfpathlineto{\pgfqpoint{1.270619in}{0.796383in}}%
\pgfpathlineto{\pgfqpoint{1.214366in}{0.799134in}}%
\pgfpathlineto{\pgfqpoint{1.159217in}{0.791930in}}%
\pgfpathlineto{\pgfqpoint{1.109884in}{0.777044in}}%
\pgfpathlineto{\pgfqpoint{1.060500in}{0.754138in}}%
\pgfpathlineto{\pgfqpoint{1.008188in}{0.721293in}}%
\pgfpathlineto{\pgfqpoint{0.954482in}{0.678710in}}%
\pgfpathlineto{\pgfqpoint{0.906982in}{0.633905in}}%
\pgfusepath{stroke}%
\end{pgfscope}%
\begin{pgfscope}%
\pgfpathrectangle{\pgfqpoint{0.647939in}{0.492442in}}{\pgfqpoint{4.273799in}{2.331163in}}%
\pgfusepath{clip}%
\pgfsetbuttcap%
\pgfsetroundjoin%
\pgfsetlinewidth{0.301125pt}%
\definecolor{currentstroke}{rgb}{0.500000,0.500000,0.500000}%
\pgfsetstrokecolor{currentstroke}%
\pgfsetstrokeopacity{0.300000}%
\pgfsetdash{}{0pt}%
\pgfpathmoveto{\pgfqpoint{1.910652in}{0.492442in}}%
\pgfpathlineto{\pgfqpoint{1.910652in}{0.492442in}}%
\pgfpathlineto{\pgfqpoint{1.871568in}{0.539655in}}%
\pgfpathlineto{\pgfqpoint{1.831923in}{0.586728in}}%
\pgfpathlineto{\pgfqpoint{1.791539in}{0.633614in}}%
\pgfpathlineto{\pgfqpoint{1.750183in}{0.680245in}}%
\pgfpathlineto{\pgfqpoint{1.707549in}{0.726531in}}%
\pgfusepath{stroke}%
\end{pgfscope}%
\begin{pgfscope}%
\pgfpathrectangle{\pgfqpoint{0.647939in}{0.492442in}}{\pgfqpoint{4.273799in}{2.331163in}}%
\pgfusepath{clip}%
\pgfsetbuttcap%
\pgfsetroundjoin%
\pgfsetlinewidth{0.301125pt}%
\definecolor{currentstroke}{rgb}{0.500000,0.500000,0.500000}%
\pgfsetstrokecolor{currentstroke}%
\pgfsetstrokeopacity{0.300000}%
\pgfsetdash{}{0pt}%
\pgfpathmoveto{\pgfqpoint{2.007784in}{0.492442in}}%
\pgfpathlineto{\pgfqpoint{2.007784in}{0.492442in}}%
\pgfpathlineto{\pgfqpoint{1.969996in}{0.539968in}}%
\pgfpathlineto{\pgfqpoint{1.931975in}{0.587439in}}%
\pgfpathlineto{\pgfqpoint{1.893613in}{0.634828in}}%
\pgfpathlineto{\pgfqpoint{1.854779in}{0.682103in}}%
\pgfpathlineto{\pgfqpoint{1.815301in}{0.729219in}}%
\pgfpathlineto{\pgfqpoint{1.774962in}{0.776116in}}%
\pgfpathlineto{\pgfqpoint{1.733476in}{0.822713in}}%
\pgfpathlineto{\pgfqpoint{1.690458in}{0.868893in}}%
\pgfpathlineto{\pgfqpoint{1.645359in}{0.914475in}}%
\pgfpathlineto{\pgfqpoint{1.597364in}{0.959163in}}%
\pgfpathlineto{\pgfqpoint{1.545161in}{1.002401in}}%
\pgfpathlineto{\pgfqpoint{1.486508in}{1.043017in}}%
\pgfpathlineto{\pgfqpoint{1.417441in}{1.078184in}}%
\pgfpathlineto{\pgfqpoint{1.417441in}{1.078184in}}%
\pgfpathlineto{\pgfqpoint{1.360339in}{1.095493in}}%
\pgfpathlineto{\pgfqpoint{1.360339in}{1.095493in}}%
\pgfpathlineto{\pgfqpoint{1.307680in}{1.101112in}}%
\pgfpathlineto{\pgfqpoint{1.254235in}{1.096234in}}%
\pgfpathlineto{\pgfqpoint{1.208013in}{1.083372in}}%
\pgfpathlineto{\pgfqpoint{1.162555in}{1.062956in}}%
\pgfpathlineto{\pgfqpoint{1.114223in}{1.033050in}}%
\pgfpathlineto{\pgfqpoint{1.061068in}{0.990899in}}%
\pgfpathlineto{\pgfqpoint{1.013959in}{0.946023in}}%
\pgfusepath{stroke}%
\end{pgfscope}%
\begin{pgfscope}%
\pgfpathrectangle{\pgfqpoint{0.647939in}{0.492442in}}{\pgfqpoint{4.273799in}{2.331163in}}%
\pgfusepath{clip}%
\pgfsetbuttcap%
\pgfsetroundjoin%
\pgfsetlinewidth{0.301125pt}%
\definecolor{currentstroke}{rgb}{0.500000,0.500000,0.500000}%
\pgfsetstrokecolor{currentstroke}%
\pgfsetstrokeopacity{0.300000}%
\pgfsetdash{}{0pt}%
\pgfpathmoveto{\pgfqpoint{2.104916in}{0.492442in}}%
\pgfpathlineto{\pgfqpoint{2.104916in}{0.492442in}}%
\pgfpathlineto{\pgfqpoint{2.067866in}{0.540141in}}%
\pgfpathlineto{\pgfqpoint{2.030815in}{0.587840in}}%
\pgfpathlineto{\pgfqpoint{1.993688in}{0.635521in}}%
\pgfpathlineto{\pgfqpoint{1.956397in}{0.683164in}}%
\pgfpathlineto{\pgfqpoint{1.918842in}{0.730745in}}%
\pgfpathlineto{\pgfqpoint{1.880905in}{0.778236in}}%
\pgfpathlineto{\pgfqpoint{1.842434in}{0.825598in}}%
\pgfpathlineto{\pgfqpoint{1.803239in}{0.872783in}}%
\pgfpathlineto{\pgfqpoint{1.763066in}{0.919721in}}%
\pgfpathlineto{\pgfqpoint{1.721571in}{0.966315in}}%
\pgfpathlineto{\pgfqpoint{1.678265in}{1.012413in}}%
\pgfpathlineto{\pgfqpoint{1.632420in}{1.057767in}}%
\pgfpathlineto{\pgfqpoint{1.582883in}{1.101931in}}%
\pgfpathlineto{\pgfqpoint{1.527695in}{1.143996in}}%
\pgfpathlineto{\pgfqpoint{1.463241in}{1.181791in}}%
\pgfpathlineto{\pgfqpoint{1.463241in}{1.181791in}}%
\pgfpathlineto{\pgfqpoint{1.405293in}{1.203681in}}%
\pgfpathlineto{\pgfqpoint{1.405293in}{1.203681in}}%
\pgfpathlineto{\pgfqpoint{1.353617in}{1.212473in}}%
\pgfpathlineto{\pgfqpoint{1.298893in}{1.210133in}}%
\pgfpathlineto{\pgfqpoint{1.253435in}{1.199070in}}%
\pgfpathlineto{\pgfqpoint{1.209843in}{1.180753in}}%
\pgfpathlineto{\pgfqpoint{1.163671in}{1.153400in}}%
\pgfpathlineto{\pgfqpoint{1.112839in}{1.114363in}}%
\pgfusepath{stroke}%
\end{pgfscope}%
\begin{pgfscope}%
\pgfpathrectangle{\pgfqpoint{0.647939in}{0.492442in}}{\pgfqpoint{4.273799in}{2.331163in}}%
\pgfusepath{clip}%
\pgfsetbuttcap%
\pgfsetroundjoin%
\pgfsetlinewidth{0.301125pt}%
\definecolor{currentstroke}{rgb}{0.500000,0.500000,0.500000}%
\pgfsetstrokecolor{currentstroke}%
\pgfsetstrokeopacity{0.300000}%
\pgfsetdash{}{0pt}%
\pgfpathmoveto{\pgfqpoint{2.396312in}{0.492442in}}%
\pgfpathlineto{\pgfqpoint{2.396312in}{0.492442in}}%
\pgfpathlineto{\pgfqpoint{2.358971in}{0.540074in}}%
\pgfpathlineto{\pgfqpoint{2.322064in}{0.587805in}}%
\pgfpathlineto{\pgfqpoint{2.285560in}{0.635630in}}%
\pgfpathlineto{\pgfqpoint{2.249428in}{0.683538in}}%
\pgfpathlineto{\pgfqpoint{2.213629in}{0.731520in}}%
\pgfpathlineto{\pgfqpoint{2.178127in}{0.779568in}}%
\pgfpathlineto{\pgfqpoint{2.142883in}{0.827673in}}%
\pgfpathlineto{\pgfqpoint{2.107859in}{0.875825in}}%
\pgfpathlineto{\pgfqpoint{2.073009in}{0.924015in}}%
\pgfpathlineto{\pgfqpoint{2.038272in}{0.972229in}}%
\pgfpathlineto{\pgfqpoint{2.003579in}{1.020452in}}%
\pgfpathlineto{\pgfqpoint{1.968856in}{1.068669in}}%
\pgfpathlineto{\pgfqpoint{1.934018in}{1.116861in}}%
\pgfpathlineto{\pgfqpoint{1.898954in}{1.165005in}}%
\pgfpathlineto{\pgfqpoint{1.863512in}{1.213066in}}%
\pgfpathlineto{\pgfqpoint{1.827488in}{1.260996in}}%
\pgfpathlineto{\pgfqpoint{1.790617in}{1.308733in}}%
\pgfpathlineto{\pgfqpoint{1.752519in}{1.356180in}}%
\pgfpathlineto{\pgfqpoint{1.712611in}{1.403177in}}%
\pgfpathlineto{\pgfqpoint{1.669931in}{1.449431in}}%
\pgfpathlineto{\pgfqpoint{1.622721in}{1.494317in}}%
\pgfpathlineto{\pgfqpoint{1.567311in}{1.536152in}}%
\pgfpathlineto{\pgfqpoint{1.567311in}{1.536152in}}%
\pgfpathlineto{\pgfqpoint{1.515666in}{1.562221in}}%
\pgfpathlineto{\pgfqpoint{1.515666in}{1.562221in}}%
\pgfpathlineto{\pgfqpoint{1.472169in}{1.573045in}}%
\pgfpathlineto{\pgfqpoint{1.472169in}{1.573045in}}%
\pgfpathlineto{\pgfqpoint{1.431226in}{1.573318in}}%
\pgfpathlineto{\pgfqpoint{1.393037in}{1.565021in}}%
\pgfpathlineto{\pgfqpoint{1.357083in}{1.550033in}}%
\pgfpathlineto{\pgfqpoint{1.317720in}{1.526283in}}%
\pgfpathlineto{\pgfqpoint{1.272699in}{1.490800in}}%
\pgfpathlineto{\pgfqpoint{1.225922in}{1.445896in}}%
\pgfpathlineto{\pgfqpoint{1.183817in}{1.399560in}}%
\pgfpathlineto{\pgfqpoint{1.144861in}{1.352379in}}%
\pgfusepath{stroke}%
\end{pgfscope}%
\begin{pgfscope}%
\pgfpathrectangle{\pgfqpoint{0.647939in}{0.492442in}}{\pgfqpoint{4.273799in}{2.331163in}}%
\pgfusepath{clip}%
\pgfsetbuttcap%
\pgfsetroundjoin%
\pgfsetlinewidth{0.301125pt}%
\definecolor{currentstroke}{rgb}{0.500000,0.500000,0.500000}%
\pgfsetstrokecolor{currentstroke}%
\pgfsetstrokeopacity{0.300000}%
\pgfsetdash{}{0pt}%
\pgfpathmoveto{\pgfqpoint{2.493443in}{0.492442in}}%
\pgfpathlineto{\pgfqpoint{2.493443in}{0.492442in}}%
\pgfpathlineto{\pgfqpoint{2.455336in}{0.539892in}}%
\pgfpathlineto{\pgfqpoint{2.417774in}{0.587472in}}%
\pgfpathlineto{\pgfqpoint{2.380728in}{0.635172in}}%
\pgfpathlineto{\pgfqpoint{2.344167in}{0.682983in}}%
\pgfpathlineto{\pgfqpoint{2.308066in}{0.730898in}}%
\pgfpathlineto{\pgfqpoint{2.272396in}{0.778909in}}%
\pgfpathlineto{\pgfqpoint{2.237132in}{0.827009in}}%
\pgfpathlineto{\pgfqpoint{2.202245in}{0.875191in}}%
\pgfpathlineto{\pgfqpoint{2.167696in}{0.923445in}}%
\pgfpathlineto{\pgfqpoint{2.133445in}{0.971762in}}%
\pgfpathlineto{\pgfqpoint{2.099455in}{1.020134in}}%
\pgfpathlineto{\pgfqpoint{2.065687in}{1.068553in}}%
\pgfpathlineto{\pgfqpoint{2.032094in}{1.117008in}}%
\pgfpathlineto{\pgfqpoint{1.998610in}{1.165485in}}%
\pgfpathlineto{\pgfqpoint{1.965162in}{1.213969in}}%
\pgfpathlineto{\pgfqpoint{1.931670in}{1.262444in}}%
\pgfpathlineto{\pgfqpoint{1.898036in}{1.310890in}}%
\pgfpathlineto{\pgfqpoint{1.864114in}{1.359276in}}%
\pgfpathlineto{\pgfqpoint{1.829703in}{1.407558in}}%
\pgfpathlineto{\pgfqpoint{1.794538in}{1.455675in}}%
\pgfpathlineto{\pgfqpoint{1.758224in}{1.503536in}}%
\pgfpathlineto{\pgfqpoint{1.720124in}{1.550978in}}%
\pgfpathlineto{\pgfqpoint{1.679101in}{1.597679in}}%
\pgfpathlineto{\pgfqpoint{1.632830in}{1.642848in}}%
\pgfpathlineto{\pgfqpoint{1.575596in}{1.683670in}}%
\pgfpathlineto{\pgfqpoint{1.575596in}{1.683670in}}%
\pgfpathlineto{\pgfqpoint{1.535976in}{1.699945in}}%
\pgfpathlineto{\pgfqpoint{1.535976in}{1.699945in}}%
\pgfpathlineto{\pgfqpoint{1.498369in}{1.705094in}}%
\pgfpathlineto{\pgfqpoint{1.460289in}{1.700021in}}%
\pgfpathlineto{\pgfqpoint{1.427861in}{1.688203in}}%
\pgfpathlineto{\pgfqpoint{1.393419in}{1.668775in}}%
\pgfpathlineto{\pgfqpoint{1.353713in}{1.638796in}}%
\pgfpathlineto{\pgfqpoint{1.306145in}{1.594202in}}%
\pgfpathlineto{\pgfqpoint{1.263603in}{1.547997in}}%
\pgfusepath{stroke}%
\end{pgfscope}%
\begin{pgfscope}%
\pgfpathrectangle{\pgfqpoint{0.647939in}{0.492442in}}{\pgfqpoint{4.273799in}{2.331163in}}%
\pgfusepath{clip}%
\pgfsetbuttcap%
\pgfsetroundjoin%
\pgfsetlinewidth{0.301125pt}%
\definecolor{currentstroke}{rgb}{0.500000,0.500000,0.500000}%
\pgfsetstrokecolor{currentstroke}%
\pgfsetstrokeopacity{0.300000}%
\pgfsetdash{}{0pt}%
\pgfpathmoveto{\pgfqpoint{2.590575in}{0.492442in}}%
\pgfpathlineto{\pgfqpoint{2.590575in}{0.492442in}}%
\pgfpathlineto{\pgfqpoint{2.551404in}{0.539634in}}%
\pgfpathlineto{\pgfqpoint{2.512879in}{0.586984in}}%
\pgfpathlineto{\pgfqpoint{2.474974in}{0.634482in}}%
\pgfpathlineto{\pgfqpoint{2.437667in}{0.682121in}}%
\pgfpathlineto{\pgfqpoint{2.400934in}{0.729893in}}%
\pgfpathlineto{\pgfqpoint{2.364752in}{0.777789in}}%
\pgfpathlineto{\pgfqpoint{2.329099in}{0.825804in}}%
\pgfpathlineto{\pgfqpoint{2.293947in}{0.873928in}}%
\pgfpathlineto{\pgfqpoint{2.259268in}{0.922155in}}%
\pgfpathlineto{\pgfqpoint{2.225036in}{0.970476in}}%
\pgfpathlineto{\pgfqpoint{2.191227in}{1.018886in}}%
\pgfpathlineto{\pgfqpoint{2.157819in}{1.067379in}}%
\pgfpathlineto{\pgfqpoint{2.124778in}{1.115946in}}%
\pgfpathlineto{\pgfqpoint{2.092065in}{1.164579in}}%
\pgfpathlineto{\pgfqpoint{2.059651in}{1.213272in}}%
\pgfpathlineto{\pgfqpoint{2.027504in}{1.262018in}}%
\pgfpathlineto{\pgfqpoint{1.995576in}{1.310806in}}%
\pgfpathlineto{\pgfqpoint{1.963803in}{1.359624in}}%
\pgfpathlineto{\pgfqpoint{1.932128in}{1.408461in}}%
\pgfpathlineto{\pgfqpoint{1.900481in}{1.457304in}}%
\pgfpathlineto{\pgfqpoint{1.868742in}{1.506129in}}%
\pgfpathlineto{\pgfqpoint{1.836750in}{1.554902in}}%
\pgfpathlineto{\pgfqpoint{1.804303in}{1.603585in}}%
\pgfpathlineto{\pgfqpoint{1.771076in}{1.652112in}}%
\pgfpathlineto{\pgfqpoint{1.736483in}{1.700349in}}%
\pgfpathlineto{\pgfqpoint{1.699403in}{1.748013in}}%
\pgfpathlineto{\pgfqpoint{1.657296in}{1.794335in}}%
\pgfpathlineto{\pgfqpoint{1.602026in}{1.835586in}}%
\pgfpathlineto{\pgfqpoint{1.602026in}{1.835586in}}%
\pgfpathlineto{\pgfqpoint{1.571350in}{1.845958in}}%
\pgfpathlineto{\pgfqpoint{1.571350in}{1.845958in}}%
\pgfpathlineto{\pgfqpoint{1.541166in}{1.846310in}}%
\pgfpathlineto{\pgfqpoint{1.513058in}{1.838629in}}%
\pgfpathlineto{\pgfqpoint{1.485721in}{1.824753in}}%
\pgfpathlineto{\pgfqpoint{1.453995in}{1.802137in}}%
\pgfpathlineto{\pgfqpoint{1.412907in}{1.765152in}}%
\pgfpathlineto{\pgfqpoint{1.369398in}{1.719295in}}%
\pgfusepath{stroke}%
\end{pgfscope}%
\begin{pgfscope}%
\pgfpathrectangle{\pgfqpoint{0.647939in}{0.492442in}}{\pgfqpoint{4.273799in}{2.331163in}}%
\pgfusepath{clip}%
\pgfsetbuttcap%
\pgfsetroundjoin%
\pgfsetlinewidth{0.301125pt}%
\definecolor{currentstroke}{rgb}{0.500000,0.500000,0.500000}%
\pgfsetstrokecolor{currentstroke}%
\pgfsetstrokeopacity{0.300000}%
\pgfsetdash{}{0pt}%
\pgfpathmoveto{\pgfqpoint{2.687707in}{0.492442in}}%
\pgfpathlineto{\pgfqpoint{2.687707in}{0.492442in}}%
\pgfpathlineto{\pgfqpoint{2.647183in}{0.539292in}}%
\pgfpathlineto{\pgfqpoint{2.607408in}{0.586333in}}%
\pgfpathlineto{\pgfqpoint{2.568359in}{0.633555in}}%
\pgfpathlineto{\pgfqpoint{2.530013in}{0.680947in}}%
\pgfpathlineto{\pgfqpoint{2.492349in}{0.728503in}}%
\pgfpathlineto{\pgfqpoint{2.455346in}{0.776212in}}%
\pgfpathlineto{\pgfqpoint{2.418983in}{0.824068in}}%
\pgfpathlineto{\pgfqpoint{2.383234in}{0.872061in}}%
\pgfpathlineto{\pgfqpoint{2.348075in}{0.920184in}}%
\pgfpathlineto{\pgfqpoint{2.313488in}{0.968430in}}%
\pgfpathlineto{\pgfqpoint{2.279455in}{1.016793in}}%
\pgfusepath{stroke}%
\end{pgfscope}%
\begin{pgfscope}%
\pgfpathrectangle{\pgfqpoint{0.647939in}{0.492442in}}{\pgfqpoint{4.273799in}{2.331163in}}%
\pgfusepath{clip}%
\pgfsetbuttcap%
\pgfsetroundjoin%
\pgfsetlinewidth{0.301125pt}%
\definecolor{currentstroke}{rgb}{0.500000,0.500000,0.500000}%
\pgfsetstrokecolor{currentstroke}%
\pgfsetstrokeopacity{0.300000}%
\pgfsetdash{}{0pt}%
\pgfpathmoveto{\pgfqpoint{2.784839in}{0.492442in}}%
\pgfpathlineto{\pgfqpoint{2.784839in}{0.492442in}}%
\pgfpathlineto{\pgfqpoint{2.742693in}{0.538864in}}%
\pgfpathlineto{\pgfqpoint{2.701398in}{0.585513in}}%
\pgfpathlineto{\pgfqpoint{2.660931in}{0.632378in}}%
\pgfpathlineto{\pgfqpoint{2.621272in}{0.679448in}}%
\pgfpathlineto{\pgfqpoint{2.582400in}{0.726713in}}%
\pgfpathlineto{\pgfqpoint{2.544294in}{0.774163in}}%
\pgfpathlineto{\pgfqpoint{2.506932in}{0.821789in}}%
\pgfpathlineto{\pgfqpoint{2.470291in}{0.869582in}}%
\pgfusepath{stroke}%
\end{pgfscope}%
\begin{pgfscope}%
\pgfpathrectangle{\pgfqpoint{0.647939in}{0.492442in}}{\pgfqpoint{4.273799in}{2.331163in}}%
\pgfusepath{clip}%
\pgfsetbuttcap%
\pgfsetroundjoin%
\pgfsetlinewidth{0.301125pt}%
\definecolor{currentstroke}{rgb}{0.500000,0.500000,0.500000}%
\pgfsetstrokecolor{currentstroke}%
\pgfsetstrokeopacity{0.300000}%
\pgfsetdash{}{0pt}%
\pgfpathmoveto{\pgfqpoint{2.979102in}{0.492442in}}%
\pgfpathlineto{\pgfqpoint{2.887890in}{0.583317in}}%
\pgfpathlineto{\pgfqpoint{2.800778in}{0.675381in}}%
\pgfpathlineto{\pgfqpoint{2.717633in}{0.768531in}}%
\pgfpathlineto{\pgfqpoint{2.638300in}{0.862665in}}%
\pgfpathlineto{\pgfqpoint{2.562611in}{0.957688in}}%
\pgfpathlineto{\pgfqpoint{2.490425in}{1.053519in}}%
\pgfpathlineto{\pgfqpoint{2.421644in}{1.150092in}}%
\pgfpathlineto{\pgfqpoint{2.356197in}{1.247352in}}%
\pgfpathlineto{\pgfqpoint{2.294101in}{1.345264in}}%
\pgfpathlineto{\pgfqpoint{2.235455in}{1.443807in}}%
\pgfpathlineto{\pgfqpoint{2.180517in}{1.542982in}}%
\pgfpathlineto{\pgfqpoint{2.129759in}{1.642817in}}%
\pgfpathlineto{\pgfqpoint{2.106191in}{1.692999in}}%
\pgfpathlineto{\pgfqpoint{2.084032in}{1.743371in}}%
\pgfpathlineto{\pgfqpoint{2.063494in}{1.793946in}}%
\pgfpathlineto{\pgfqpoint{2.044867in}{1.844739in}}%
\pgfpathlineto{\pgfqpoint{2.028548in}{1.895766in}}%
\pgfpathlineto{\pgfqpoint{2.015077in}{1.947038in}}%
\pgfpathlineto{\pgfqpoint{2.005190in}{1.998547in}}%
\pgfpathlineto{\pgfqpoint{1.999915in}{2.050250in}}%
\pgfpathlineto{\pgfqpoint{2.000609in}{2.102014in}}%
\pgfpathlineto{\pgfqpoint{2.008953in}{2.153552in}}%
\pgfpathlineto{\pgfqpoint{2.026758in}{2.204334in}}%
\pgfpathlineto{\pgfqpoint{2.055567in}{2.253551in}}%
\pgfpathlineto{\pgfqpoint{2.096231in}{2.300175in}}%
\pgfpathlineto{\pgfqpoint{2.148785in}{2.343063in}}%
\pgfpathlineto{\pgfqpoint{2.212843in}{2.380941in}}%
\pgfpathlineto{\pgfqpoint{2.288007in}{2.412050in}}%
\pgfpathlineto{\pgfqpoint{2.371141in}{2.433625in}}%
\pgfpathlineto{\pgfqpoint{2.452846in}{2.443435in}}%
\pgfpathlineto{\pgfqpoint{2.530355in}{2.442819in}}%
\pgfpathlineto{\pgfqpoint{2.603442in}{2.433347in}}%
\pgfpathlineto{\pgfqpoint{2.673630in}{2.415768in}}%
\pgfpathlineto{\pgfqpoint{2.741671in}{2.390074in}}%
\pgfpathlineto{\pgfqpoint{2.807597in}{2.355949in}}%
\pgfpathlineto{\pgfqpoint{2.867103in}{2.315784in}}%
\pgfpathlineto{\pgfqpoint{2.917964in}{2.272213in}}%
\pgfpathlineto{\pgfqpoint{2.960559in}{2.226051in}}%
\pgfpathlineto{\pgfqpoint{2.994641in}{2.177823in}}%
\pgfpathlineto{\pgfqpoint{3.018930in}{2.127874in}}%
\pgfpathlineto{\pgfqpoint{3.029545in}{2.076650in}}%
\pgfpathlineto{\pgfqpoint{3.029545in}{2.076650in}}%
\pgfpathlineto{\pgfqpoint{3.023408in}{2.042879in}}%
\pgfpathlineto{\pgfqpoint{3.023408in}{2.042879in}}%
\pgfpathlineto{\pgfqpoint{3.008097in}{2.024670in}}%
\pgfpathlineto{\pgfqpoint{3.008097in}{2.024670in}}%
\pgfpathlineto{\pgfqpoint{2.988323in}{2.017968in}}%
\pgfpathlineto{\pgfqpoint{2.965504in}{2.020358in}}%
\pgfpathlineto{\pgfqpoint{2.946219in}{2.028667in}}%
\pgfpathlineto{\pgfqpoint{2.926884in}{2.044030in}}%
\pgfpathlineto{\pgfqpoint{2.913358in}{2.066722in}}%
\pgfpathlineto{\pgfqpoint{2.913358in}{2.066722in}}%
\pgfpathlineto{\pgfqpoint{2.914776in}{2.077688in}}%
\pgfpathlineto{\pgfqpoint{2.921819in}{2.079267in}}%
\pgfpathlineto{\pgfqpoint{2.928179in}{2.074796in}}%
\pgfpathlineto{\pgfqpoint{2.925333in}{2.068871in}}%
\pgfpathlineto{\pgfqpoint{2.925333in}{2.068871in}}%
\pgfpathlineto{\pgfqpoint{2.926073in}{2.075099in}}%
\pgfpathlineto{\pgfqpoint{2.929731in}{2.068895in}}%
\pgfpathlineto{\pgfqpoint{2.924316in}{2.076861in}}%
\pgfpathlineto{\pgfqpoint{2.929920in}{2.069091in}}%
\pgfpathlineto{\pgfqpoint{2.924261in}{2.076122in}}%
\pgfpathlineto{\pgfqpoint{2.929163in}{2.071576in}}%
\pgfpathlineto{\pgfqpoint{2.924045in}{2.074495in}}%
\pgfpathlineto{\pgfqpoint{2.928910in}{2.072583in}}%
\pgfpathlineto{\pgfqpoint{2.923754in}{2.072646in}}%
\pgfpathlineto{\pgfqpoint{2.928331in}{2.073388in}}%
\pgfpathlineto{\pgfqpoint{2.925325in}{2.071568in}}%
\pgfpathlineto{\pgfqpoint{2.928981in}{2.073382in}}%
\pgfpathlineto{\pgfqpoint{2.922697in}{2.072220in}}%
\pgfpathlineto{\pgfqpoint{2.927549in}{2.074304in}}%
\pgfpathlineto{\pgfqpoint{2.927079in}{2.070673in}}%
\pgfpathlineto{\pgfqpoint{2.928582in}{2.075178in}}%
\pgfpathlineto{\pgfqpoint{2.928582in}{2.075178in}}%
\pgfpathlineto{\pgfqpoint{2.924462in}{2.069235in}}%
\pgfpathlineto{\pgfqpoint{2.926323in}{2.075232in}}%
\pgfpathlineto{\pgfqpoint{2.929588in}{2.067054in}}%
\pgfpathlineto{\pgfqpoint{2.924511in}{2.076728in}}%
\pgfpathlineto{\pgfqpoint{2.929974in}{2.068125in}}%
\pgfpathlineto{\pgfqpoint{2.924016in}{2.076272in}}%
\pgfpathlineto{\pgfqpoint{2.929356in}{2.071088in}}%
\pgfpathlineto{\pgfqpoint{2.928909in}{2.071808in}}%
\pgfpathlineto{\pgfqpoint{2.923688in}{2.074658in}}%
\pgfpathlineto{\pgfqpoint{2.929302in}{2.071576in}}%
\pgfpathlineto{\pgfqpoint{2.924598in}{2.073227in}}%
\pgfusepath{stroke}%
\end{pgfscope}%
\begin{pgfscope}%
\pgfpathrectangle{\pgfqpoint{0.647939in}{0.492442in}}{\pgfqpoint{4.273799in}{2.331163in}}%
\pgfusepath{clip}%
\pgfsetbuttcap%
\pgfsetroundjoin%
\pgfsetlinewidth{0.301125pt}%
\definecolor{currentstroke}{rgb}{0.500000,0.500000,0.500000}%
\pgfsetstrokecolor{currentstroke}%
\pgfsetstrokeopacity{0.300000}%
\pgfsetdash{}{0pt}%
\pgfpathmoveto{\pgfqpoint{3.173366in}{0.492442in}}%
\pgfpathlineto{\pgfqpoint{3.173366in}{0.492442in}}%
\pgfpathlineto{\pgfqpoint{3.122522in}{0.536194in}}%
\pgfpathlineto{\pgfqpoint{3.072834in}{0.580340in}}%
\pgfpathlineto{\pgfqpoint{3.024319in}{0.624872in}}%
\pgfpathlineto{\pgfqpoint{2.976987in}{0.669782in}}%
\pgfpathlineto{\pgfqpoint{2.930834in}{0.715056in}}%
\pgfpathlineto{\pgfqpoint{2.885853in}{0.760679in}}%
\pgfpathlineto{\pgfqpoint{2.842030in}{0.806636in}}%
\pgfpathlineto{\pgfqpoint{2.799347in}{0.852911in}}%
\pgfpathlineto{\pgfqpoint{2.757787in}{0.899490in}}%
\pgfpathlineto{\pgfqpoint{2.717329in}{0.946356in}}%
\pgfpathlineto{\pgfqpoint{2.677956in}{0.993497in}}%
\pgfpathlineto{\pgfqpoint{2.639648in}{1.040898in}}%
\pgfpathlineto{\pgfqpoint{2.602387in}{1.088547in}}%
\pgfpathlineto{\pgfqpoint{2.566161in}{1.136433in}}%
\pgfpathlineto{\pgfqpoint{2.530961in}{1.184545in}}%
\pgfpathlineto{\pgfqpoint{2.496784in}{1.232877in}}%
\pgfpathlineto{\pgfqpoint{2.463633in}{1.281422in}}%
\pgfpathlineto{\pgfqpoint{2.431510in}{1.330171in}}%
\pgfpathlineto{\pgfqpoint{2.400427in}{1.379120in}}%
\pgfpathlineto{\pgfqpoint{2.370413in}{1.428268in}}%
\pgfpathlineto{\pgfqpoint{2.341501in}{1.477612in}}%
\pgfpathlineto{\pgfqpoint{2.313733in}{1.527150in}}%
\pgfpathlineto{\pgfqpoint{2.287181in}{1.576887in}}%
\pgfpathlineto{\pgfqpoint{2.261926in}{1.626824in}}%
\pgfpathlineto{\pgfqpoint{2.238074in}{1.676965in}}%
\pgfpathlineto{\pgfqpoint{2.215774in}{1.727318in}}%
\pgfpathlineto{\pgfqpoint{2.195212in}{1.777890in}}%
\pgfpathlineto{\pgfqpoint{2.176634in}{1.828689in}}%
\pgfpathlineto{\pgfqpoint{2.160365in}{1.879721in}}%
\pgfpathlineto{\pgfqpoint{2.146826in}{1.930989in}}%
\pgfpathlineto{\pgfqpoint{2.136572in}{1.982479in}}%
\pgfpathlineto{\pgfqpoint{2.130352in}{2.034154in}}%
\pgfpathlineto{\pgfqpoint{2.129142in}{2.085927in}}%
\pgfpathlineto{\pgfqpoint{2.134203in}{2.137618in}}%
\pgfpathlineto{\pgfqpoint{2.147104in}{2.188881in}}%
\pgfpathlineto{\pgfqpoint{2.169653in}{2.239102in}}%
\pgfpathlineto{\pgfqpoint{2.203773in}{2.287276in}}%
\pgfpathlineto{\pgfqpoint{2.251196in}{2.331881in}}%
\pgfpathlineto{\pgfqpoint{2.313308in}{2.370608in}}%
\pgfpathlineto{\pgfqpoint{2.384890in}{2.398751in}}%
\pgfpathlineto{\pgfqpoint{2.458898in}{2.414515in}}%
\pgfpathlineto{\pgfqpoint{2.530719in}{2.419027in}}%
\pgfpathlineto{\pgfqpoint{2.599159in}{2.414186in}}%
\pgfusepath{stroke}%
\end{pgfscope}%
\begin{pgfscope}%
\pgfpathrectangle{\pgfqpoint{0.647939in}{0.492442in}}{\pgfqpoint{4.273799in}{2.331163in}}%
\pgfusepath{clip}%
\pgfsetbuttcap%
\pgfsetroundjoin%
\pgfsetlinewidth{0.301125pt}%
\definecolor{currentstroke}{rgb}{0.500000,0.500000,0.500000}%
\pgfsetstrokecolor{currentstroke}%
\pgfsetstrokeopacity{0.300000}%
\pgfsetdash{}{0pt}%
\pgfpathmoveto{\pgfqpoint{3.367630in}{0.492442in}}%
\pgfpathlineto{\pgfqpoint{3.367630in}{0.492442in}}%
\pgfpathlineto{\pgfqpoint{3.311898in}{0.534386in}}%
\pgfpathlineto{\pgfqpoint{3.257260in}{0.576755in}}%
\pgfpathlineto{\pgfqpoint{3.203799in}{0.619569in}}%
\pgfpathlineto{\pgfqpoint{3.151578in}{0.662836in}}%
\pgfpathlineto{\pgfqpoint{3.100646in}{0.706557in}}%
\pgfpathlineto{\pgfqpoint{3.051031in}{0.750727in}}%
\pgfpathlineto{\pgfqpoint{3.002746in}{0.795332in}}%
\pgfpathlineto{\pgfqpoint{2.955791in}{0.840358in}}%
\pgfpathlineto{\pgfqpoint{2.910159in}{0.885786in}}%
\pgfpathlineto{\pgfqpoint{2.865838in}{0.931600in}}%
\pgfpathlineto{\pgfqpoint{2.822810in}{0.977779in}}%
\pgfpathlineto{\pgfqpoint{2.781056in}{1.024305in}}%
\pgfpathlineto{\pgfqpoint{2.740559in}{1.071160in}}%
\pgfpathlineto{\pgfqpoint{2.701299in}{1.118328in}}%
\pgfpathlineto{\pgfqpoint{2.663263in}{1.165793in}}%
\pgfpathlineto{\pgfqpoint{2.626439in}{1.213542in}}%
\pgfpathlineto{\pgfqpoint{2.590822in}{1.261564in}}%
\pgfpathlineto{\pgfqpoint{2.556409in}{1.309845in}}%
\pgfpathlineto{\pgfqpoint{2.523201in}{1.358377in}}%
\pgfpathlineto{\pgfqpoint{2.491216in}{1.407153in}}%
\pgfpathlineto{\pgfqpoint{2.460485in}{1.456168in}}%
\pgfpathlineto{\pgfqpoint{2.431042in}{1.505418in}}%
\pgfpathlineto{\pgfqpoint{2.402934in}{1.554899in}}%
\pgfpathlineto{\pgfqpoint{2.376242in}{1.604612in}}%
\pgfpathlineto{\pgfqpoint{2.351054in}{1.654559in}}%
\pgfpathlineto{\pgfqpoint{2.327495in}{1.704740in}}%
\pgfpathlineto{\pgfqpoint{2.305728in}{1.755162in}}%
\pgfpathlineto{\pgfqpoint{2.285959in}{1.805827in}}%
\pgfpathlineto{\pgfqpoint{2.268463in}{1.856739in}}%
\pgfpathlineto{\pgfqpoint{2.253589in}{1.907897in}}%
\pgfpathlineto{\pgfqpoint{2.241808in}{1.959291in}}%
\pgfpathlineto{\pgfqpoint{2.233731in}{2.010893in}}%
\pgfpathlineto{\pgfqpoint{2.230169in}{2.062639in}}%
\pgfpathlineto{\pgfqpoint{2.232201in}{2.114399in}}%
\pgfpathlineto{\pgfqpoint{2.241252in}{2.165919in}}%
\pgfpathlineto{\pgfqpoint{2.259176in}{2.216707in}}%
\pgfpathlineto{\pgfqpoint{2.288326in}{2.265853in}}%
\pgfpathlineto{\pgfqpoint{2.331472in}{2.311698in}}%
\pgfpathlineto{\pgfqpoint{2.391441in}{2.351256in}}%
\pgfusepath{stroke}%
\end{pgfscope}%
\begin{pgfscope}%
\pgfpathrectangle{\pgfqpoint{0.647939in}{0.492442in}}{\pgfqpoint{4.273799in}{2.331163in}}%
\pgfusepath{clip}%
\pgfsetbuttcap%
\pgfsetroundjoin%
\pgfsetlinewidth{0.301125pt}%
\definecolor{currentstroke}{rgb}{0.500000,0.500000,0.500000}%
\pgfsetstrokecolor{currentstroke}%
\pgfsetstrokeopacity{0.300000}%
\pgfsetdash{}{0pt}%
\pgfpathmoveto{\pgfqpoint{3.561893in}{0.492442in}}%
\pgfpathlineto{\pgfqpoint{3.561893in}{0.492442in}}%
\pgfpathlineto{\pgfqpoint{3.502094in}{0.532686in}}%
\pgfpathlineto{\pgfqpoint{3.442965in}{0.573224in}}%
\pgfpathlineto{\pgfqpoint{3.384720in}{0.614139in}}%
\pgfpathlineto{\pgfqpoint{3.327524in}{0.655492in}}%
\pgfpathlineto{\pgfqpoint{3.271521in}{0.697327in}}%
\pgfpathlineto{\pgfqpoint{3.216826in}{0.739674in}}%
\pgfpathlineto{\pgfqpoint{3.163516in}{0.782541in}}%
\pgfpathlineto{\pgfqpoint{3.111643in}{0.825932in}}%
\pgfpathlineto{\pgfqpoint{3.061241in}{0.869835in}}%
\pgfpathlineto{\pgfqpoint{3.012330in}{0.914237in}}%
\pgfpathlineto{\pgfqpoint{2.964910in}{0.959117in}}%
\pgfpathlineto{\pgfqpoint{2.918973in}{1.004453in}}%
\pgfpathlineto{\pgfqpoint{2.874505in}{1.050223in}}%
\pgfusepath{stroke}%
\end{pgfscope}%
\begin{pgfscope}%
\pgfpathrectangle{\pgfqpoint{0.647939in}{0.492442in}}{\pgfqpoint{4.273799in}{2.331163in}}%
\pgfusepath{clip}%
\pgfsetbuttcap%
\pgfsetroundjoin%
\pgfsetlinewidth{0.301125pt}%
\definecolor{currentstroke}{rgb}{0.500000,0.500000,0.500000}%
\pgfsetstrokecolor{currentstroke}%
\pgfsetstrokeopacity{0.300000}%
\pgfsetdash{}{0pt}%
\pgfpathmoveto{\pgfqpoint{3.756157in}{0.492442in}}%
\pgfpathlineto{\pgfqpoint{3.756157in}{0.492442in}}%
\pgfpathlineto{\pgfqpoint{3.694354in}{0.531776in}}%
\pgfpathlineto{\pgfqpoint{3.632376in}{0.571028in}}%
\pgfpathlineto{\pgfqpoint{3.570535in}{0.610345in}}%
\pgfpathlineto{\pgfqpoint{3.509142in}{0.649868in}}%
\pgfpathlineto{\pgfqpoint{3.448468in}{0.689720in}}%
\pgfpathlineto{\pgfqpoint{3.388762in}{0.730003in}}%
\pgfpathlineto{\pgfqpoint{3.330231in}{0.770795in}}%
\pgfpathlineto{\pgfqpoint{3.273036in}{0.812147in}}%
\pgfpathlineto{\pgfqpoint{3.217310in}{0.854090in}}%
\pgfpathlineto{\pgfqpoint{3.163142in}{0.896636in}}%
\pgfpathlineto{\pgfqpoint{3.110588in}{0.939780in}}%
\pgfpathlineto{\pgfqpoint{3.059684in}{0.983510in}}%
\pgfpathlineto{\pgfqpoint{3.010449in}{1.027804in}}%
\pgfpathlineto{\pgfqpoint{2.962879in}{1.072635in}}%
\pgfpathlineto{\pgfqpoint{2.916964in}{1.117977in}}%
\pgfpathlineto{\pgfqpoint{2.872687in}{1.163801in}}%
\pgfpathlineto{\pgfqpoint{2.830028in}{1.210080in}}%
\pgfpathlineto{\pgfqpoint{2.788968in}{1.256788in}}%
\pgfpathlineto{\pgfqpoint{2.749493in}{1.303900in}}%
\pgfpathlineto{\pgfqpoint{2.711592in}{1.351396in}}%
\pgfpathlineto{\pgfqpoint{2.675265in}{1.399258in}}%
\pgfpathlineto{\pgfqpoint{2.640518in}{1.447467in}}%
\pgfpathlineto{\pgfqpoint{2.607368in}{1.496009in}}%
\pgfpathlineto{\pgfqpoint{2.575851in}{1.544874in}}%
\pgfpathlineto{\pgfqpoint{2.546027in}{1.594054in}}%
\pgfpathlineto{\pgfqpoint{2.517970in}{1.643542in}}%
\pgfpathlineto{\pgfqpoint{2.491779in}{1.693334in}}%
\pgfpathlineto{\pgfqpoint{2.467605in}{1.743427in}}%
\pgfpathlineto{\pgfqpoint{2.445630in}{1.793820in}}%
\pgfpathlineto{\pgfqpoint{2.426104in}{1.844511in}}%
\pgfpathlineto{\pgfqpoint{2.409356in}{1.895494in}}%
\pgfpathlineto{\pgfqpoint{2.395821in}{1.946759in}}%
\pgfpathlineto{\pgfqpoint{2.386088in}{1.998276in}}%
\pgfpathlineto{\pgfqpoint{2.380965in}{2.049984in}}%
\pgfpathlineto{\pgfqpoint{2.381571in}{2.101753in}}%
\pgfpathlineto{\pgfqpoint{2.389509in}{2.153315in}}%
\pgfpathlineto{\pgfqpoint{2.407137in}{2.204108in}}%
\pgfpathlineto{\pgfqpoint{2.437989in}{2.252873in}}%
\pgfpathlineto{\pgfqpoint{2.487200in}{2.296560in}}%
\pgfpathlineto{\pgfqpoint{2.487200in}{2.296560in}}%
\pgfpathlineto{\pgfqpoint{2.534472in}{2.320550in}}%
\pgfpathlineto{\pgfqpoint{2.593671in}{2.335403in}}%
\pgfpathlineto{\pgfqpoint{2.650138in}{2.338129in}}%
\pgfpathlineto{\pgfqpoint{2.703254in}{2.332083in}}%
\pgfpathlineto{\pgfqpoint{2.756048in}{2.318295in}}%
\pgfpathlineto{\pgfqpoint{2.809029in}{2.296482in}}%
\pgfpathlineto{\pgfqpoint{2.861739in}{2.265915in}}%
\pgfpathlineto{\pgfqpoint{2.912334in}{2.226027in}}%
\pgfusepath{stroke}%
\end{pgfscope}%
\begin{pgfscope}%
\pgfpathrectangle{\pgfqpoint{0.647939in}{0.492442in}}{\pgfqpoint{4.273799in}{2.331163in}}%
\pgfusepath{clip}%
\pgfsetbuttcap%
\pgfsetroundjoin%
\pgfsetlinewidth{0.301125pt}%
\definecolor{currentstroke}{rgb}{0.500000,0.500000,0.500000}%
\pgfsetstrokecolor{currentstroke}%
\pgfsetstrokeopacity{0.300000}%
\pgfsetdash{}{0pt}%
\pgfpathmoveto{\pgfqpoint{3.853289in}{0.492442in}}%
\pgfpathlineto{\pgfqpoint{3.853289in}{0.492442in}}%
\pgfpathlineto{\pgfqpoint{3.791656in}{0.531854in}}%
\pgfpathlineto{\pgfqpoint{3.729288in}{0.570922in}}%
\pgfpathlineto{\pgfqpoint{3.666519in}{0.609799in}}%
\pgfpathlineto{\pgfqpoint{3.603703in}{0.648652in}}%
\pgfpathlineto{\pgfqpoint{3.541180in}{0.687646in}}%
\pgfusepath{stroke}%
\end{pgfscope}%
\begin{pgfscope}%
\pgfpathrectangle{\pgfqpoint{0.647939in}{0.492442in}}{\pgfqpoint{4.273799in}{2.331163in}}%
\pgfusepath{clip}%
\pgfsetbuttcap%
\pgfsetroundjoin%
\pgfsetlinewidth{0.301125pt}%
\definecolor{currentstroke}{rgb}{0.500000,0.500000,0.500000}%
\pgfsetstrokecolor{currentstroke}%
\pgfsetstrokeopacity{0.300000}%
\pgfsetdash{}{0pt}%
\pgfpathmoveto{\pgfqpoint{3.950420in}{0.492442in}}%
\pgfpathlineto{\pgfqpoint{3.950420in}{0.492442in}}%
\pgfpathlineto{\pgfqpoint{3.889891in}{0.532357in}}%
\pgfpathlineto{\pgfqpoint{3.828038in}{0.571666in}}%
\pgfpathlineto{\pgfqpoint{3.765184in}{0.610500in}}%
\pgfpathlineto{\pgfqpoint{3.701678in}{0.649019in}}%
\pgfpathlineto{\pgfqpoint{3.637902in}{0.687405in}}%
\pgfusepath{stroke}%
\end{pgfscope}%
\begin{pgfscope}%
\pgfpathrectangle{\pgfqpoint{0.647939in}{0.492442in}}{\pgfqpoint{4.273799in}{2.331163in}}%
\pgfusepath{clip}%
\pgfsetbuttcap%
\pgfsetroundjoin%
\pgfsetlinewidth{0.301125pt}%
\definecolor{currentstroke}{rgb}{0.500000,0.500000,0.500000}%
\pgfsetstrokecolor{currentstroke}%
\pgfsetstrokeopacity{0.300000}%
\pgfsetdash{}{0pt}%
\pgfpathmoveto{\pgfqpoint{4.047552in}{0.492442in}}%
\pgfpathlineto{\pgfqpoint{4.047552in}{0.492442in}}%
\pgfpathlineto{\pgfqpoint{3.989117in}{0.533272in}}%
\pgfpathlineto{\pgfqpoint{3.928819in}{0.573290in}}%
\pgfpathlineto{\pgfqpoint{3.866898in}{0.612566in}}%
\pgfpathlineto{\pgfqpoint{3.803669in}{0.651217in}}%
\pgfpathlineto{\pgfqpoint{3.739486in}{0.689399in}}%
\pgfusepath{stroke}%
\end{pgfscope}%
\begin{pgfscope}%
\pgfpathrectangle{\pgfqpoint{0.647939in}{0.492442in}}{\pgfqpoint{4.273799in}{2.331163in}}%
\pgfusepath{clip}%
\pgfsetbuttcap%
\pgfsetroundjoin%
\pgfsetlinewidth{0.301125pt}%
\definecolor{currentstroke}{rgb}{0.500000,0.500000,0.500000}%
\pgfsetstrokecolor{currentstroke}%
\pgfsetstrokeopacity{0.300000}%
\pgfsetdash{}{0pt}%
\pgfpathmoveto{\pgfqpoint{4.241816in}{0.492442in}}%
\pgfpathlineto{\pgfqpoint{4.241816in}{0.492442in}}%
\pgfpathlineto{\pgfqpoint{4.190280in}{0.535947in}}%
\pgfpathlineto{\pgfqpoint{4.136325in}{0.578570in}}%
\pgfpathlineto{\pgfqpoint{4.079919in}{0.620237in}}%
\pgfpathlineto{\pgfqpoint{4.021071in}{0.660886in}}%
\pgfpathlineto{\pgfqpoint{3.959887in}{0.700498in}}%
\pgfpathlineto{\pgfqpoint{3.896596in}{0.739114in}}%
\pgfpathlineto{\pgfqpoint{3.831514in}{0.776837in}}%
\pgfpathlineto{\pgfqpoint{3.765061in}{0.813844in}}%
\pgfpathlineto{\pgfqpoint{3.697728in}{0.850376in}}%
\pgfpathlineto{\pgfqpoint{3.630033in}{0.886711in}}%
\pgfpathlineto{\pgfqpoint{3.562514in}{0.923140in}}%
\pgfpathlineto{\pgfqpoint{3.495667in}{0.959935in}}%
\pgfpathlineto{\pgfqpoint{3.429942in}{0.997324in}}%
\pgfpathlineto{\pgfqpoint{3.365721in}{1.035480in}}%
\pgfpathlineto{\pgfqpoint{3.303305in}{1.074515in}}%
\pgfpathlineto{\pgfqpoint{3.242920in}{1.114487in}}%
\pgfpathlineto{\pgfqpoint{3.184718in}{1.155412in}}%
\pgfpathlineto{\pgfqpoint{3.128798in}{1.197270in}}%
\pgfpathlineto{\pgfqpoint{3.075200in}{1.240023in}}%
\pgfpathlineto{\pgfqpoint{3.023932in}{1.283621in}}%
\pgfpathlineto{\pgfqpoint{2.974986in}{1.328007in}}%
\pgfpathlineto{\pgfqpoint{2.928339in}{1.373123in}}%
\pgfpathlineto{\pgfqpoint{2.883962in}{1.418916in}}%
\pgfpathlineto{\pgfqpoint{2.841830in}{1.465335in}}%
\pgfpathlineto{\pgfqpoint{2.801930in}{1.512338in}}%
\pgfpathlineto{\pgfqpoint{2.764263in}{1.559888in}}%
\pgfpathlineto{\pgfqpoint{2.728851in}{1.607950in}}%
\pgfpathlineto{\pgfqpoint{2.695738in}{1.656496in}}%
\pgfpathlineto{\pgfqpoint{2.665013in}{1.705506in}}%
\pgfpathlineto{\pgfqpoint{2.636807in}{1.754965in}}%
\pgfpathlineto{\pgfqpoint{2.611302in}{1.804858in}}%
\pgfpathlineto{\pgfqpoint{2.588762in}{1.855172in}}%
\pgfpathlineto{\pgfqpoint{2.569557in}{1.905895in}}%
\pgfpathlineto{\pgfqpoint{2.554208in}{1.957003in}}%
\pgfpathlineto{\pgfqpoint{2.543471in}{2.008452in}}%
\pgfpathlineto{\pgfqpoint{2.538462in}{2.060148in}}%
\pgfpathlineto{\pgfqpoint{2.540926in}{2.111873in}}%
\pgfpathlineto{\pgfqpoint{2.553784in}{2.163083in}}%
\pgfpathlineto{\pgfqpoint{2.582413in}{2.212142in}}%
\pgfpathlineto{\pgfqpoint{2.582413in}{2.212142in}}%
\pgfpathlineto{\pgfqpoint{2.616625in}{2.242663in}}%
\pgfpathlineto{\pgfqpoint{2.616625in}{2.242663in}}%
\pgfpathlineto{\pgfqpoint{2.655884in}{2.261425in}}%
\pgfpathlineto{\pgfqpoint{2.705657in}{2.270339in}}%
\pgfusepath{stroke}%
\end{pgfscope}%
\begin{pgfscope}%
\pgfpathrectangle{\pgfqpoint{0.647939in}{0.492442in}}{\pgfqpoint{4.273799in}{2.331163in}}%
\pgfusepath{clip}%
\pgfsetbuttcap%
\pgfsetroundjoin%
\pgfsetlinewidth{0.301125pt}%
\definecolor{currentstroke}{rgb}{0.500000,0.500000,0.500000}%
\pgfsetstrokecolor{currentstroke}%
\pgfsetstrokeopacity{0.300000}%
\pgfsetdash{}{0pt}%
\pgfpathmoveto{\pgfqpoint{4.338948in}{0.492442in}}%
\pgfpathlineto{\pgfqpoint{4.338948in}{0.492442in}}%
\pgfpathlineto{\pgfqpoint{4.291872in}{0.537424in}}%
\pgfpathlineto{\pgfqpoint{4.242459in}{0.581654in}}%
\pgfpathlineto{\pgfqpoint{4.190535in}{0.625019in}}%
\pgfpathlineto{\pgfqpoint{4.135968in}{0.667408in}}%
\pgfpathlineto{\pgfqpoint{4.078696in}{0.708719in}}%
\pgfpathlineto{\pgfqpoint{4.018703in}{0.748864in}}%
\pgfpathlineto{\pgfqpoint{3.956101in}{0.787809in}}%
\pgfpathlineto{\pgfqpoint{3.891139in}{0.825588in}}%
\pgfpathlineto{\pgfqpoint{3.824189in}{0.862324in}}%
\pgfpathlineto{\pgfqpoint{3.755737in}{0.898230in}}%
\pgfpathlineto{\pgfqpoint{3.686365in}{0.933609in}}%
\pgfpathlineto{\pgfqpoint{3.616699in}{0.968816in}}%
\pgfpathlineto{\pgfqpoint{3.547331in}{1.004196in}}%
\pgfusepath{stroke}%
\end{pgfscope}%
\begin{pgfscope}%
\pgfpathrectangle{\pgfqpoint{0.647939in}{0.492442in}}{\pgfqpoint{4.273799in}{2.331163in}}%
\pgfusepath{clip}%
\pgfsetbuttcap%
\pgfsetroundjoin%
\pgfsetlinewidth{0.301125pt}%
\definecolor{currentstroke}{rgb}{0.500000,0.500000,0.500000}%
\pgfsetstrokecolor{currentstroke}%
\pgfsetstrokeopacity{0.300000}%
\pgfsetdash{}{0pt}%
\pgfpathmoveto{\pgfqpoint{4.436079in}{0.492442in}}%
\pgfpathlineto{\pgfqpoint{4.436079in}{0.492442in}}%
\pgfpathlineto{\pgfqpoint{4.393732in}{0.538804in}}%
\pgfpathlineto{\pgfqpoint{4.349335in}{0.584591in}}%
\pgfpathlineto{\pgfqpoint{4.302679in}{0.629703in}}%
\pgfpathlineto{\pgfqpoint{4.253541in}{0.674022in}}%
\pgfpathlineto{\pgfqpoint{4.201698in}{0.717412in}}%
\pgfpathlineto{\pgfqpoint{4.146935in}{0.759721in}}%
\pgfpathlineto{\pgfqpoint{4.089115in}{0.800802in}}%
\pgfpathlineto{\pgfqpoint{4.028204in}{0.840530in}}%
\pgfpathlineto{\pgfqpoint{3.964260in}{0.878814in}}%
\pgfpathlineto{\pgfqpoint{3.897540in}{0.915666in}}%
\pgfpathlineto{\pgfqpoint{3.828473in}{0.951213in}}%
\pgfpathlineto{\pgfqpoint{3.757627in}{0.985707in}}%
\pgfpathlineto{\pgfqpoint{3.685703in}{1.019536in}}%
\pgfpathlineto{\pgfqpoint{3.613452in}{1.053156in}}%
\pgfpathlineto{\pgfqpoint{3.541576in}{1.087014in}}%
\pgfpathlineto{\pgfqpoint{3.470763in}{1.121525in}}%
\pgfpathlineto{\pgfqpoint{3.401598in}{1.157010in}}%
\pgfpathlineto{\pgfqpoint{3.334533in}{1.193672in}}%
\pgfpathlineto{\pgfqpoint{3.269923in}{1.231618in}}%
\pgfpathlineto{\pgfqpoint{3.208000in}{1.270874in}}%
\pgfpathlineto{\pgfqpoint{3.148893in}{1.311402in}}%
\pgfpathlineto{\pgfqpoint{3.092667in}{1.353133in}}%
\pgfpathlineto{\pgfqpoint{3.039329in}{1.395975in}}%
\pgfpathlineto{\pgfqpoint{2.988847in}{1.439837in}}%
\pgfpathlineto{\pgfqpoint{2.941176in}{1.484629in}}%
\pgfpathlineto{\pgfqpoint{2.896276in}{1.530267in}}%
\pgfpathlineto{\pgfqpoint{2.854116in}{1.576677in}}%
\pgfpathlineto{\pgfqpoint{2.814685in}{1.623794in}}%
\pgfpathlineto{\pgfqpoint{2.778003in}{1.671567in}}%
\pgfpathlineto{\pgfqpoint{2.744131in}{1.719952in}}%
\pgfpathlineto{\pgfqpoint{2.713185in}{1.768918in}}%
\pgfpathlineto{\pgfqpoint{2.685356in}{1.818436in}}%
\pgfpathlineto{\pgfqpoint{2.660920in}{1.868482in}}%
\pgfpathlineto{\pgfqpoint{2.640302in}{1.919034in}}%
\pgfusepath{stroke}%
\end{pgfscope}%
\begin{pgfscope}%
\pgfpathrectangle{\pgfqpoint{0.647939in}{0.492442in}}{\pgfqpoint{4.273799in}{2.331163in}}%
\pgfusepath{clip}%
\pgfsetbuttcap%
\pgfsetroundjoin%
\pgfsetlinewidth{0.301125pt}%
\definecolor{currentstroke}{rgb}{0.500000,0.500000,0.500000}%
\pgfsetstrokecolor{currentstroke}%
\pgfsetstrokeopacity{0.300000}%
\pgfsetdash{}{0pt}%
\pgfpathmoveto{\pgfqpoint{4.533211in}{0.492442in}}%
\pgfpathlineto{\pgfqpoint{4.533211in}{0.492442in}}%
\pgfpathlineto{\pgfqpoint{4.495591in}{0.540005in}}%
\pgfpathlineto{\pgfqpoint{4.456332in}{0.587170in}}%
\pgfpathlineto{\pgfqpoint{4.415233in}{0.633865in}}%
\pgfpathlineto{\pgfqpoint{4.372068in}{0.680000in}}%
\pgfpathlineto{\pgfqpoint{4.326581in}{0.725465in}}%
\pgfpathlineto{\pgfqpoint{4.278486in}{0.770124in}}%
\pgfpathlineto{\pgfqpoint{4.227479in}{0.813809in}}%
\pgfpathlineto{\pgfqpoint{4.173284in}{0.856333in}}%
\pgfpathlineto{\pgfqpoint{4.115642in}{0.897480in}}%
\pgfpathlineto{\pgfqpoint{4.054344in}{0.937023in}}%
\pgfpathlineto{\pgfqpoint{3.989391in}{0.974790in}}%
\pgfpathlineto{\pgfqpoint{3.920977in}{1.010694in}}%
\pgfpathlineto{\pgfqpoint{3.849547in}{1.044814in}}%
\pgfpathlineto{\pgfqpoint{3.775800in}{1.077444in}}%
\pgfpathlineto{\pgfqpoint{3.700588in}{1.109071in}}%
\pgfpathlineto{\pgfqpoint{3.624813in}{1.140297in}}%
\pgfpathlineto{\pgfqpoint{3.549398in}{1.171778in}}%
\pgfpathlineto{\pgfqpoint{3.475196in}{1.204095in}}%
\pgfpathlineto{\pgfqpoint{3.402907in}{1.237667in}}%
\pgfpathlineto{\pgfqpoint{3.333089in}{1.272758in}}%
\pgfpathlineto{\pgfqpoint{3.266173in}{1.309485in}}%
\pgfpathlineto{\pgfqpoint{3.202408in}{1.347845in}}%
\pgfpathlineto{\pgfqpoint{3.141934in}{1.387760in}}%
\pgfpathlineto{\pgfqpoint{3.084793in}{1.429109in}}%
\pgfpathlineto{\pgfqpoint{3.030959in}{1.471761in}}%
\pgfusepath{stroke}%
\end{pgfscope}%
\begin{pgfscope}%
\pgfpathrectangle{\pgfqpoint{0.647939in}{0.492442in}}{\pgfqpoint{4.273799in}{2.331163in}}%
\pgfusepath{clip}%
\pgfsetbuttcap%
\pgfsetroundjoin%
\pgfsetlinewidth{0.301125pt}%
\definecolor{currentstroke}{rgb}{0.500000,0.500000,0.500000}%
\pgfsetstrokecolor{currentstroke}%
\pgfsetstrokeopacity{0.300000}%
\pgfsetdash{}{0pt}%
\pgfpathmoveto{\pgfqpoint{4.630343in}{0.492442in}}%
\pgfpathlineto{\pgfqpoint{4.630343in}{0.492442in}}%
\pgfpathlineto{\pgfqpoint{4.597244in}{0.540994in}}%
\pgfpathlineto{\pgfqpoint{4.562925in}{0.589293in}}%
\pgfpathlineto{\pgfqpoint{4.527232in}{0.637295in}}%
\pgfpathlineto{\pgfqpoint{4.489992in}{0.684946in}}%
\pgfpathlineto{\pgfqpoint{4.451007in}{0.732180in}}%
\pgfpathlineto{\pgfqpoint{4.410046in}{0.778911in}}%
\pgfpathlineto{\pgfqpoint{4.366820in}{0.825029in}}%
\pgfpathlineto{\pgfqpoint{4.320993in}{0.870392in}}%
\pgfpathlineto{\pgfqpoint{4.272169in}{0.914813in}}%
\pgfpathlineto{\pgfqpoint{4.219941in}{0.958061in}}%
\pgfpathlineto{\pgfqpoint{4.163873in}{0.999843in}}%
\pgfpathlineto{\pgfqpoint{4.103519in}{1.039804in}}%
\pgfpathlineto{\pgfqpoint{4.038602in}{1.077573in}}%
\pgfpathlineto{\pgfqpoint{3.969112in}{1.112832in}}%
\pgfpathlineto{\pgfqpoint{3.895421in}{1.145465in}}%
\pgfpathlineto{\pgfqpoint{3.818328in}{1.175684in}}%
\pgfpathlineto{\pgfqpoint{3.738908in}{1.204074in}}%
\pgfpathlineto{\pgfqpoint{3.658366in}{1.231517in}}%
\pgfpathlineto{\pgfqpoint{3.577891in}{1.259014in}}%
\pgfpathlineto{\pgfqpoint{3.498574in}{1.287485in}}%
\pgfpathlineto{\pgfqpoint{3.421389in}{1.317633in}}%
\pgfpathlineto{\pgfqpoint{3.347122in}{1.349866in}}%
\pgfpathlineto{\pgfqpoint{3.276301in}{1.384325in}}%
\pgfpathlineto{\pgfqpoint{3.209265in}{1.420970in}}%
\pgfpathlineto{\pgfqpoint{3.146185in}{1.459648in}}%
\pgfpathlineto{\pgfqpoint{3.087074in}{1.500154in}}%
\pgfpathlineto{\pgfqpoint{3.031876in}{1.542273in}}%
\pgfpathlineto{\pgfqpoint{2.980483in}{1.585812in}}%
\pgfpathlineto{\pgfqpoint{2.932802in}{1.630594in}}%
\pgfpathlineto{\pgfqpoint{2.888758in}{1.676471in}}%
\pgfpathlineto{\pgfqpoint{2.848323in}{1.723329in}}%
\pgfpathlineto{\pgfqpoint{2.811531in}{1.771072in}}%
\pgfpathlineto{\pgfqpoint{2.778503in}{1.819623in}}%
\pgfpathlineto{\pgfqpoint{2.749483in}{1.868927in}}%
\pgfusepath{stroke}%
\end{pgfscope}%
\begin{pgfscope}%
\pgfpathrectangle{\pgfqpoint{0.647939in}{0.492442in}}{\pgfqpoint{4.273799in}{2.331163in}}%
\pgfusepath{clip}%
\pgfsetbuttcap%
\pgfsetroundjoin%
\pgfsetlinewidth{0.301125pt}%
\definecolor{currentstroke}{rgb}{0.500000,0.500000,0.500000}%
\pgfsetstrokecolor{currentstroke}%
\pgfsetstrokeopacity{0.300000}%
\pgfsetdash{}{0pt}%
\pgfpathmoveto{\pgfqpoint{4.727475in}{0.492442in}}%
\pgfpathlineto{\pgfqpoint{4.727475in}{0.492442in}}%
\pgfpathlineto{\pgfqpoint{4.698505in}{0.541774in}}%
\pgfpathlineto{\pgfqpoint{4.668690in}{0.590957in}}%
\pgfpathlineto{\pgfqpoint{4.637948in}{0.639969in}}%
\pgfpathlineto{\pgfqpoint{4.606170in}{0.688783in}}%
\pgfpathlineto{\pgfqpoint{4.573218in}{0.737365in}}%
\pgfpathlineto{\pgfqpoint{4.538945in}{0.785675in}}%
\pgfpathlineto{\pgfqpoint{4.503175in}{0.833658in}}%
\pgfpathlineto{\pgfqpoint{4.465683in}{0.881246in}}%
\pgfpathlineto{\pgfqpoint{4.426190in}{0.928348in}}%
\pgfpathlineto{\pgfqpoint{4.384350in}{0.974840in}}%
\pgfpathlineto{\pgfqpoint{4.339738in}{1.020555in}}%
\pgfpathlineto{\pgfqpoint{4.291828in}{1.065259in}}%
\pgfpathlineto{\pgfqpoint{4.239971in}{1.108626in}}%
\pgfpathlineto{\pgfqpoint{4.183398in}{1.150196in}}%
\pgfpathlineto{\pgfqpoint{4.121377in}{1.189362in}}%
\pgfpathlineto{\pgfqpoint{4.053255in}{1.225363in}}%
\pgfpathlineto{\pgfqpoint{3.978895in}{1.257464in}}%
\pgfpathlineto{\pgfqpoint{3.898957in}{1.285302in}}%
\pgfpathlineto{\pgfqpoint{3.814854in}{1.309267in}}%
\pgfpathlineto{\pgfqpoint{3.728331in}{1.330582in}}%
\pgfpathlineto{\pgfqpoint{3.641045in}{1.350971in}}%
\pgfpathlineto{\pgfqpoint{3.554435in}{1.372171in}}%
\pgfpathlineto{\pgfqpoint{3.469786in}{1.395559in}}%
\pgfpathlineto{\pgfqpoint{3.388236in}{1.421978in}}%
\pgfpathlineto{\pgfqpoint{3.310687in}{1.451749in}}%
\pgfpathlineto{\pgfqpoint{3.237709in}{1.484801in}}%
\pgfpathlineto{\pgfqpoint{3.169615in}{1.520832in}}%
\pgfusepath{stroke}%
\end{pgfscope}%
\begin{pgfscope}%
\pgfpathrectangle{\pgfqpoint{0.647939in}{0.492442in}}{\pgfqpoint{4.273799in}{2.331163in}}%
\pgfusepath{clip}%
\pgfsetbuttcap%
\pgfsetroundjoin%
\pgfsetlinewidth{0.301125pt}%
\definecolor{currentstroke}{rgb}{0.500000,0.500000,0.500000}%
\pgfsetstrokecolor{currentstroke}%
\pgfsetstrokeopacity{0.300000}%
\pgfsetdash{}{0pt}%
\pgfpathmoveto{\pgfqpoint{4.824607in}{0.492442in}}%
\pgfpathlineto{\pgfqpoint{4.824607in}{0.492442in}}%
\pgfpathlineto{\pgfqpoint{4.799320in}{0.542374in}}%
\pgfpathlineto{\pgfqpoint{4.773502in}{0.592225in}}%
\pgfpathlineto{\pgfqpoint{4.747100in}{0.641986in}}%
\pgfpathlineto{\pgfqpoint{4.720063in}{0.691644in}}%
\pgfpathlineto{\pgfqpoint{4.692330in}{0.741187in}}%
\pgfpathlineto{\pgfqpoint{4.663813in}{0.790598in}}%
\pgfpathlineto{\pgfqpoint{4.634427in}{0.839858in}}%
\pgfpathlineto{\pgfqpoint{4.604069in}{0.888939in}}%
\pgfpathlineto{\pgfqpoint{4.572589in}{0.937811in}}%
\pgfpathlineto{\pgfqpoint{4.539824in}{0.986430in}}%
\pgfpathlineto{\pgfqpoint{4.505574in}{1.034741in}}%
\pgfpathlineto{\pgfqpoint{4.469573in}{1.082669in}}%
\pgfpathlineto{\pgfqpoint{4.431465in}{1.130108in}}%
\pgfpathlineto{\pgfqpoint{4.390790in}{1.176904in}}%
\pgfpathlineto{\pgfqpoint{4.346929in}{1.222829in}}%
\pgfpathlineto{\pgfqpoint{4.299044in}{1.267529in}}%
\pgfpathlineto{\pgfqpoint{4.245987in}{1.310432in}}%
\pgfpathlineto{\pgfqpoint{4.186198in}{1.350582in}}%
\pgfpathlineto{\pgfqpoint{4.118018in}{1.386446in}}%
\pgfpathlineto{\pgfqpoint{4.040410in}{1.415899in}}%
\pgfpathlineto{\pgfqpoint{3.954331in}{1.437156in}}%
\pgfpathlineto{\pgfqpoint{3.862916in}{1.450587in}}%
\pgfpathlineto{\pgfqpoint{3.769289in}{1.458960in}}%
\pgfpathlineto{\pgfqpoint{3.675193in}{1.465872in}}%
\pgfpathlineto{\pgfqpoint{3.581649in}{1.474514in}}%
\pgfpathlineto{\pgfqpoint{3.489774in}{1.487231in}}%
\pgfpathlineto{\pgfqpoint{3.401067in}{1.505325in}}%
\pgfpathlineto{\pgfqpoint{3.316990in}{1.529094in}}%
\pgfpathlineto{\pgfqpoint{3.238594in}{1.558116in}}%
\pgfpathlineto{\pgfqpoint{3.166398in}{1.591615in}}%
\pgfpathlineto{\pgfqpoint{3.100437in}{1.628750in}}%
\pgfpathlineto{\pgfqpoint{3.040424in}{1.668805in}}%
\pgfpathlineto{\pgfqpoint{2.986043in}{1.711205in}}%
\pgfpathlineto{\pgfqpoint{2.937005in}{1.755522in}}%
\pgfpathlineto{\pgfqpoint{2.893137in}{1.801435in}}%
\pgfpathlineto{\pgfqpoint{2.854420in}{1.848708in}}%
\pgfpathlineto{\pgfqpoint{2.821036in}{1.897166in}}%
\pgfpathlineto{\pgfqpoint{2.793476in}{1.946694in}}%
\pgfpathlineto{\pgfqpoint{2.772753in}{1.997194in}}%
\pgfpathlineto{\pgfqpoint{2.760915in}{2.048506in}}%
\pgfpathlineto{\pgfqpoint{2.762692in}{2.100054in}}%
\pgfpathlineto{\pgfqpoint{2.762692in}{2.100054in}}%
\pgfpathlineto{\pgfqpoint{2.777522in}{2.133556in}}%
\pgfpathlineto{\pgfqpoint{2.777522in}{2.133556in}}%
\pgfpathlineto{\pgfqpoint{2.799288in}{2.152404in}}%
\pgfpathlineto{\pgfqpoint{2.799288in}{2.152404in}}%
\pgfpathlineto{\pgfqpoint{2.825606in}{2.160449in}}%
\pgfpathlineto{\pgfqpoint{2.855467in}{2.158973in}}%
\pgfpathlineto{\pgfqpoint{2.881852in}{2.150643in}}%
\pgfpathlineto{\pgfqpoint{2.907907in}{2.135764in}}%
\pgfusepath{stroke}%
\end{pgfscope}%
\begin{pgfscope}%
\pgfpathrectangle{\pgfqpoint{0.647939in}{0.492442in}}{\pgfqpoint{4.273799in}{2.331163in}}%
\pgfusepath{clip}%
\pgfsetbuttcap%
\pgfsetroundjoin%
\pgfsetlinewidth{0.301125pt}%
\definecolor{currentstroke}{rgb}{0.500000,0.500000,0.500000}%
\pgfsetstrokecolor{currentstroke}%
\pgfsetstrokeopacity{0.300000}%
\pgfsetdash{}{0pt}%
\pgfpathmoveto{\pgfqpoint{4.921738in}{0.492442in}}%
\pgfpathlineto{\pgfqpoint{4.921738in}{0.492442in}}%
\pgfpathlineto{\pgfqpoint{4.899677in}{0.542828in}}%
\pgfpathlineto{\pgfqpoint{4.877315in}{0.593174in}}%
\pgfpathlineto{\pgfqpoint{4.854633in}{0.643478in}}%
\pgfpathlineto{\pgfqpoint{4.831611in}{0.693735in}}%
\pgfpathlineto{\pgfqpoint{4.808222in}{0.743942in}}%
\pgfpathlineto{\pgfqpoint{4.784439in}{0.794094in}}%
\pgfpathlineto{\pgfqpoint{4.760225in}{0.844184in}}%
\pgfpathlineto{\pgfqpoint{4.735546in}{0.894206in}}%
\pgfpathlineto{\pgfqpoint{4.710351in}{0.944151in}}%
\pgfpathlineto{\pgfqpoint{4.684588in}{0.994010in}}%
\pgfpathlineto{\pgfqpoint{4.658190in}{1.043770in}}%
\pgfpathlineto{\pgfqpoint{4.631075in}{1.093415in}}%
\pgfpathlineto{\pgfqpoint{4.603149in}{1.142924in}}%
\pgfpathlineto{\pgfqpoint{4.574273in}{1.192272in}}%
\pgfpathlineto{\pgfqpoint{4.544288in}{1.241422in}}%
\pgfpathlineto{\pgfqpoint{4.512989in}{1.290324in}}%
\pgfpathlineto{\pgfqpoint{4.480069in}{1.338906in}}%
\pgfpathlineto{\pgfqpoint{4.445091in}{1.387060in}}%
\pgfpathlineto{\pgfqpoint{4.407455in}{1.434610in}}%
\pgfpathlineto{\pgfqpoint{4.366223in}{1.481251in}}%
\pgfpathlineto{\pgfqpoint{4.319833in}{1.526401in}}%
\pgfpathlineto{\pgfqpoint{4.265572in}{1.568790in}}%
\pgfpathlineto{\pgfqpoint{4.199077in}{1.605229in}}%
\pgfpathlineto{\pgfqpoint{4.199077in}{1.605229in}}%
\pgfpathlineto{\pgfqpoint{4.140586in}{1.623906in}}%
\pgfpathlineto{\pgfqpoint{4.074039in}{1.632312in}}%
\pgfpathlineto{\pgfqpoint{4.011640in}{1.630875in}}%
\pgfpathlineto{\pgfqpoint{3.943727in}{1.622636in}}%
\pgfpathlineto{\pgfqpoint{3.855600in}{1.606968in}}%
\pgfpathlineto{\pgfqpoint{3.765857in}{1.590216in}}%
\pgfpathlineto{\pgfqpoint{3.674579in}{1.576237in}}%
\pgfpathlineto{\pgfqpoint{3.581178in}{1.567813in}}%
\pgfpathlineto{\pgfqpoint{3.486742in}{1.567265in}}%
\pgfpathlineto{\pgfqpoint{3.393762in}{1.575905in}}%
\pgfpathlineto{\pgfqpoint{3.307172in}{1.593105in}}%
\pgfpathlineto{\pgfqpoint{3.225676in}{1.618265in}}%
\pgfpathlineto{\pgfqpoint{3.150869in}{1.649876in}}%
\pgfpathlineto{\pgfqpoint{3.083524in}{1.686226in}}%
\pgfusepath{stroke}%
\end{pgfscope}%
\begin{pgfscope}%
\pgfpathrectangle{\pgfqpoint{0.647939in}{0.492442in}}{\pgfqpoint{4.273799in}{2.331163in}}%
\pgfusepath{clip}%
\pgfsetbuttcap%
\pgfsetroundjoin%
\pgfsetlinewidth{0.301125pt}%
\definecolor{currentstroke}{rgb}{0.500000,0.500000,0.500000}%
\pgfsetstrokecolor{currentstroke}%
\pgfsetstrokeopacity{0.300000}%
\pgfsetdash{}{0pt}%
\pgfpathmoveto{\pgfqpoint{4.921738in}{0.704366in}}%
\pgfpathlineto{\pgfqpoint{4.921738in}{0.704366in}}%
\pgfpathlineto{\pgfqpoint{4.901415in}{0.754969in}}%
\pgfpathlineto{\pgfqpoint{4.880916in}{0.805551in}}%
\pgfpathlineto{\pgfqpoint{4.860234in}{0.856110in}}%
\pgfpathlineto{\pgfqpoint{4.839360in}{0.906646in}}%
\pgfpathlineto{\pgfqpoint{4.818288in}{0.957157in}}%
\pgfpathlineto{\pgfqpoint{4.797008in}{1.007642in}}%
\pgfpathlineto{\pgfqpoint{4.775509in}{1.058100in}}%
\pgfpathlineto{\pgfqpoint{4.753779in}{1.108528in}}%
\pgfpathlineto{\pgfqpoint{4.731805in}{1.158924in}}%
\pgfpathlineto{\pgfqpoint{4.709571in}{1.209286in}}%
\pgfpathlineto{\pgfqpoint{4.687058in}{1.259612in}}%
\pgfpathlineto{\pgfqpoint{4.664241in}{1.309896in}}%
\pgfpathlineto{\pgfqpoint{4.641095in}{1.360135in}}%
\pgfpathlineto{\pgfqpoint{4.617584in}{1.410324in}}%
\pgfpathlineto{\pgfqpoint{4.593663in}{1.460455in}}%
\pgfpathlineto{\pgfqpoint{4.569279in}{1.510517in}}%
\pgfpathlineto{\pgfqpoint{4.544345in}{1.560499in}}%
\pgfpathlineto{\pgfqpoint{4.518755in}{1.610379in}}%
\pgfpathlineto{\pgfqpoint{4.492327in}{1.660130in}}%
\pgfpathlineto{\pgfqpoint{4.464805in}{1.709699in}}%
\pgfpathlineto{\pgfqpoint{4.435746in}{1.758996in}}%
\pgfpathlineto{\pgfqpoint{4.404207in}{1.807837in}}%
\pgfpathlineto{\pgfqpoint{4.368085in}{1.855673in}}%
\pgfpathlineto{\pgfqpoint{4.320243in}{1.899804in}}%
\pgfpathlineto{\pgfqpoint{4.320243in}{1.899804in}}%
\pgfpathlineto{\pgfqpoint{4.292680in}{1.912557in}}%
\pgfpathlineto{\pgfqpoint{4.292680in}{1.912557in}}%
\pgfpathlineto{\pgfqpoint{4.263660in}{1.915468in}}%
\pgfpathlineto{\pgfqpoint{4.235175in}{1.909776in}}%
\pgfpathlineto{\pgfqpoint{4.207182in}{1.898491in}}%
\pgfpathlineto{\pgfqpoint{4.169482in}{1.877974in}}%
\pgfpathlineto{\pgfqpoint{4.113370in}{1.842611in}}%
\pgfpathlineto{\pgfqpoint{4.052319in}{1.803159in}}%
\pgfpathlineto{\pgfqpoint{3.989446in}{1.764447in}}%
\pgfpathlineto{\pgfqpoint{3.923474in}{1.727291in}}%
\pgfpathlineto{\pgfqpoint{3.853340in}{1.692495in}}%
\pgfpathlineto{\pgfqpoint{3.778113in}{1.661052in}}%
\pgfpathlineto{\pgfqpoint{3.697131in}{1.634325in}}%
\pgfpathlineto{\pgfqpoint{3.610171in}{1.614166in}}%
\pgfpathlineto{\pgfqpoint{3.518189in}{1.602817in}}%
\pgfpathlineto{\pgfqpoint{3.428001in}{1.601961in}}%
\pgfpathlineto{\pgfqpoint{3.344751in}{1.610585in}}%
\pgfusepath{stroke}%
\end{pgfscope}%
\begin{pgfscope}%
\pgfpathrectangle{\pgfqpoint{0.647939in}{0.492442in}}{\pgfqpoint{4.273799in}{2.331163in}}%
\pgfusepath{clip}%
\pgfsetbuttcap%
\pgfsetroundjoin%
\pgfsetlinewidth{0.301125pt}%
\definecolor{currentstroke}{rgb}{0.500000,0.500000,0.500000}%
\pgfsetstrokecolor{currentstroke}%
\pgfsetstrokeopacity{0.300000}%
\pgfsetdash{}{0pt}%
\pgfpathmoveto{\pgfqpoint{4.921738in}{0.916290in}}%
\pgfpathlineto{\pgfqpoint{4.921738in}{0.916290in}}%
\pgfpathlineto{\pgfqpoint{4.903414in}{0.967120in}}%
\pgfpathlineto{\pgfqpoint{4.885069in}{1.017947in}}%
\pgfpathlineto{\pgfqpoint{4.866711in}{1.068772in}}%
\pgfpathlineto{\pgfqpoint{4.848353in}{1.119598in}}%
\pgfpathlineto{\pgfqpoint{4.830008in}{1.170425in}}%
\pgfpathlineto{\pgfqpoint{4.811695in}{1.221255in}}%
\pgfpathlineto{\pgfqpoint{4.793432in}{1.272091in}}%
\pgfpathlineto{\pgfqpoint{4.775247in}{1.322935in}}%
\pgfpathlineto{\pgfqpoint{4.757174in}{1.373790in}}%
\pgfpathlineto{\pgfqpoint{4.739245in}{1.424661in}}%
\pgfpathlineto{\pgfqpoint{4.721510in}{1.475552in}}%
\pgfpathlineto{\pgfqpoint{4.704021in}{1.526468in}}%
\pgfpathlineto{\pgfqpoint{4.686854in}{1.577417in}}%
\pgfpathlineto{\pgfqpoint{4.670102in}{1.628406in}}%
\pgfpathlineto{\pgfqpoint{4.653875in}{1.679446in}}%
\pgfpathlineto{\pgfqpoint{4.638328in}{1.730547in}}%
\pgfpathlineto{\pgfqpoint{4.623650in}{1.781726in}}%
\pgfpathlineto{\pgfqpoint{4.610095in}{1.832996in}}%
\pgfpathlineto{\pgfqpoint{4.598008in}{1.884374in}}%
\pgfpathlineto{\pgfqpoint{4.587825in}{1.935873in}}%
\pgfpathlineto{\pgfqpoint{4.580116in}{1.987498in}}%
\pgfpathlineto{\pgfqpoint{4.575594in}{2.039232in}}%
\pgfpathlineto{\pgfqpoint{4.575074in}{2.091019in}}%
\pgfpathlineto{\pgfqpoint{4.579326in}{2.142750in}}%
\pgfpathlineto{\pgfqpoint{4.588819in}{2.194265in}}%
\pgfpathlineto{\pgfqpoint{4.603492in}{2.245408in}}%
\pgfpathlineto{\pgfqpoint{4.622765in}{2.296102in}}%
\pgfpathlineto{\pgfqpoint{4.645648in}{2.346345in}}%
\pgfpathlineto{\pgfqpoint{4.671148in}{2.396213in}}%
\pgfpathlineto{\pgfqpoint{4.698417in}{2.445810in}}%
\pgfpathlineto{\pgfqpoint{4.726778in}{2.495220in}}%
\pgfpathlineto{\pgfqpoint{4.755774in}{2.544525in}}%
\pgfpathlineto{\pgfqpoint{4.785086in}{2.593782in}}%
\pgfpathlineto{\pgfqpoint{4.814463in}{2.643021in}}%
\pgfpathlineto{\pgfqpoint{4.843770in}{2.692275in}}%
\pgfpathlineto{\pgfqpoint{4.872919in}{2.741565in}}%
\pgfpathlineto{\pgfqpoint{4.901821in}{2.790896in}}%
\pgfpathlineto{\pgfqpoint{4.920889in}{2.823605in}}%
\pgfusepath{stroke}%
\end{pgfscope}%
\begin{pgfscope}%
\pgfpathrectangle{\pgfqpoint{0.647939in}{0.492442in}}{\pgfqpoint{4.273799in}{2.331163in}}%
\pgfusepath{clip}%
\pgfsetbuttcap%
\pgfsetroundjoin%
\pgfsetlinewidth{0.301125pt}%
\definecolor{currentstroke}{rgb}{0.500000,0.500000,0.500000}%
\pgfsetstrokecolor{currentstroke}%
\pgfsetstrokeopacity{0.300000}%
\pgfsetdash{}{0pt}%
\pgfpathmoveto{\pgfqpoint{4.921738in}{1.128214in}}%
\pgfpathlineto{\pgfqpoint{4.921738in}{1.128214in}}%
\pgfpathlineto{\pgfqpoint{4.905731in}{1.179276in}}%
\pgfpathlineto{\pgfqpoint{4.889894in}{1.230353in}}%
\pgfpathlineto{\pgfqpoint{4.874251in}{1.281449in}}%
\pgfpathlineto{\pgfqpoint{4.858844in}{1.332565in}}%
\pgfpathlineto{\pgfqpoint{4.843714in}{1.383706in}}%
\pgfpathlineto{\pgfqpoint{4.828908in}{1.434875in}}%
\pgfpathlineto{\pgfqpoint{4.814487in}{1.486077in}}%
\pgfpathlineto{\pgfqpoint{4.800516in}{1.537316in}}%
\pgfpathlineto{\pgfqpoint{4.787072in}{1.588597in}}%
\pgfpathlineto{\pgfqpoint{4.774252in}{1.639925in}}%
\pgfpathlineto{\pgfqpoint{4.762167in}{1.691306in}}%
\pgfpathlineto{\pgfqpoint{4.750937in}{1.742745in}}%
\pgfpathlineto{\pgfqpoint{4.740717in}{1.794246in}}%
\pgfpathlineto{\pgfqpoint{4.731686in}{1.845813in}}%
\pgfpathlineto{\pgfqpoint{4.724043in}{1.897446in}}%
\pgfpathlineto{\pgfqpoint{4.718012in}{1.949142in}}%
\pgfpathlineto{\pgfqpoint{4.713840in}{2.000893in}}%
\pgfpathlineto{\pgfqpoint{4.711789in}{2.052680in}}%
\pgfpathlineto{\pgfqpoint{4.712115in}{2.104478in}}%
\pgfpathlineto{\pgfqpoint{4.715043in}{2.156251in}}%
\pgfpathlineto{\pgfqpoint{4.720737in}{2.207954in}}%
\pgfpathlineto{\pgfqpoint{4.729268in}{2.259540in}}%
\pgfpathlineto{\pgfqpoint{4.740603in}{2.310964in}}%
\pgfpathlineto{\pgfqpoint{4.754595in}{2.362193in}}%
\pgfpathlineto{\pgfqpoint{4.770986in}{2.413208in}}%
\pgfpathlineto{\pgfqpoint{4.789476in}{2.464010in}}%
\pgfpathlineto{\pgfqpoint{4.809727in}{2.514613in}}%
\pgfpathlineto{\pgfqpoint{4.831410in}{2.565039in}}%
\pgfusepath{stroke}%
\end{pgfscope}%
\begin{pgfscope}%
\pgfpathrectangle{\pgfqpoint{0.647939in}{0.492442in}}{\pgfqpoint{4.273799in}{2.331163in}}%
\pgfusepath{clip}%
\pgfsetbuttcap%
\pgfsetroundjoin%
\pgfsetlinewidth{0.301125pt}%
\definecolor{currentstroke}{rgb}{0.500000,0.500000,0.500000}%
\pgfsetstrokecolor{currentstroke}%
\pgfsetstrokeopacity{0.300000}%
\pgfsetdash{}{0pt}%
\pgfpathmoveto{\pgfqpoint{4.921738in}{1.393119in}}%
\pgfpathlineto{\pgfqpoint{4.921738in}{1.393119in}}%
\pgfpathlineto{\pgfqpoint{4.909190in}{1.444468in}}%
\pgfpathlineto{\pgfqpoint{4.897104in}{1.495849in}}%
\pgfpathlineto{\pgfqpoint{4.885536in}{1.547266in}}%
\pgfpathlineto{\pgfqpoint{4.874555in}{1.598722in}}%
\pgfpathlineto{\pgfqpoint{4.864240in}{1.650218in}}%
\pgfpathlineto{\pgfqpoint{4.854679in}{1.701757in}}%
\pgfpathlineto{\pgfqpoint{4.845964in}{1.753341in}}%
\pgfpathlineto{\pgfqpoint{4.838201in}{1.804970in}}%
\pgfpathlineto{\pgfqpoint{4.831505in}{1.856644in}}%
\pgfpathlineto{\pgfqpoint{4.826004in}{1.908359in}}%
\pgfpathlineto{\pgfqpoint{4.821829in}{1.960111in}}%
\pgfpathlineto{\pgfqpoint{4.819114in}{2.011892in}}%
\pgfpathlineto{\pgfqpoint{4.817993in}{2.063689in}}%
\pgfpathlineto{\pgfqpoint{4.818590in}{2.115489in}}%
\pgfpathlineto{\pgfqpoint{4.821015in}{2.167273in}}%
\pgfpathlineto{\pgfqpoint{4.825352in}{2.219019in}}%
\pgfpathlineto{\pgfqpoint{4.831651in}{2.270705in}}%
\pgfpathlineto{\pgfqpoint{4.839922in}{2.322308in}}%
\pgfpathlineto{\pgfqpoint{4.850124in}{2.373808in}}%
\pgfpathlineto{\pgfqpoint{4.862175in}{2.425188in}}%
\pgfpathlineto{\pgfqpoint{4.875968in}{2.476438in}}%
\pgfpathlineto{\pgfqpoint{4.891349in}{2.527553in}}%
\pgfpathlineto{\pgfqpoint{4.908147in}{2.578533in}}%
\pgfpathlineto{\pgfqpoint{4.921738in}{2.618204in}}%
\pgfusepath{stroke}%
\end{pgfscope}%
\begin{pgfscope}%
\pgfpathrectangle{\pgfqpoint{0.647939in}{0.492442in}}{\pgfqpoint{4.273799in}{2.331163in}}%
\pgfusepath{clip}%
\pgfsetbuttcap%
\pgfsetroundjoin%
\pgfsetlinewidth{0.301125pt}%
\definecolor{currentstroke}{rgb}{0.500000,0.500000,0.500000}%
\pgfsetstrokecolor{currentstroke}%
\pgfsetstrokeopacity{0.300000}%
\pgfsetdash{}{0pt}%
\pgfpathmoveto{\pgfqpoint{4.921738in}{1.658024in}}%
\pgfpathlineto{\pgfqpoint{4.921738in}{1.658024in}}%
\pgfpathlineto{\pgfqpoint{4.913456in}{1.709629in}}%
\pgfpathlineto{\pgfqpoint{4.905986in}{1.761272in}}%
\pgfpathlineto{\pgfqpoint{4.899411in}{1.812950in}}%
\pgfpathlineto{\pgfqpoint{4.893818in}{1.864663in}}%
\pgfpathlineto{\pgfqpoint{4.889301in}{1.916406in}}%
\pgfpathlineto{\pgfqpoint{4.885954in}{1.968176in}}%
\pgfpathlineto{\pgfqpoint{4.883874in}{2.019966in}}%
\pgfpathlineto{\pgfqpoint{4.883153in}{2.071767in}}%
\pgfpathlineto{\pgfqpoint{4.883877in}{2.123567in}}%
\pgfpathlineto{\pgfqpoint{4.886120in}{2.175354in}}%
\pgfpathlineto{\pgfqpoint{4.889938in}{2.227114in}}%
\pgfpathlineto{\pgfqpoint{4.895368in}{2.278830in}}%
\pgfpathlineto{\pgfqpoint{4.902423in}{2.330488in}}%
\pgfpathlineto{\pgfqpoint{4.911088in}{2.382072in}}%
\pgfusepath{stroke}%
\end{pgfscope}%
\begin{pgfscope}%
\pgfpathrectangle{\pgfqpoint{0.647939in}{0.492442in}}{\pgfqpoint{4.273799in}{2.331163in}}%
\pgfusepath{clip}%
\pgfsetbuttcap%
\pgfsetroundjoin%
\pgfsetlinewidth{0.301125pt}%
\definecolor{currentstroke}{rgb}{0.500000,0.500000,0.500000}%
\pgfsetstrokecolor{currentstroke}%
\pgfsetstrokeopacity{0.300000}%
\pgfsetdash{}{0pt}%
\pgfpathmoveto{\pgfqpoint{4.323639in}{2.823605in}}%
\pgfpathlineto{\pgfqpoint{4.352125in}{2.795306in}}%
\pgfpathlineto{\pgfqpoint{4.400812in}{2.750891in}}%
\pgfpathlineto{\pgfqpoint{4.446816in}{2.716879in}}%
\pgfpathlineto{\pgfqpoint{4.487376in}{2.694435in}}%
\pgfpathlineto{\pgfqpoint{4.525996in}{2.680452in}}%
\pgfpathlineto{\pgfqpoint{4.570484in}{2.673957in}}%
\pgfpathlineto{\pgfqpoint{4.616293in}{2.678264in}}%
\pgfpathlineto{\pgfqpoint{4.616293in}{2.678264in}}%
\pgfpathlineto{\pgfqpoint{4.667023in}{2.695868in}}%
\pgfpathlineto{\pgfqpoint{4.667023in}{2.695868in}}%
\pgfpathlineto{\pgfqpoint{4.729149in}{2.734429in}}%
\pgfpathlineto{\pgfqpoint{4.780061in}{2.777944in}}%
\pgfpathlineto{\pgfqpoint{4.824607in}{2.823605in}}%
\pgfpathlineto{\pgfqpoint{4.824607in}{2.823605in}}%
\pgfusepath{stroke}%
\end{pgfscope}%
\begin{pgfscope}%
\pgfpathrectangle{\pgfqpoint{0.647939in}{0.492442in}}{\pgfqpoint{4.273799in}{2.331163in}}%
\pgfusepath{clip}%
\pgfsetbuttcap%
\pgfsetroundjoin%
\pgfsetlinewidth{0.301125pt}%
\definecolor{currentstroke}{rgb}{0.500000,0.500000,0.500000}%
\pgfsetstrokecolor{currentstroke}%
\pgfsetstrokeopacity{0.300000}%
\pgfsetdash{}{0pt}%
\pgfpathmoveto{\pgfqpoint{4.241816in}{2.823605in}}%
\pgfpathlineto{\pgfqpoint{4.241816in}{2.823605in}}%
\pgfpathlineto{\pgfqpoint{4.280260in}{2.776243in}}%
\pgfpathlineto{\pgfqpoint{4.321172in}{2.729505in}}%
\pgfpathlineto{\pgfqpoint{4.366021in}{2.683874in}}%
\pgfpathlineto{\pgfqpoint{4.417768in}{2.640577in}}%
\pgfpathlineto{\pgfqpoint{4.483217in}{2.603952in}}%
\pgfpathlineto{\pgfqpoint{4.483217in}{2.603952in}}%
\pgfpathlineto{\pgfqpoint{4.525883in}{2.592650in}}%
\pgfpathlineto{\pgfqpoint{4.525883in}{2.592650in}}%
\pgfpathlineto{\pgfqpoint{4.566271in}{2.591882in}}%
\pgfpathlineto{\pgfqpoint{4.604176in}{2.599849in}}%
\pgfpathlineto{\pgfqpoint{4.639565in}{2.614520in}}%
\pgfpathlineto{\pgfqpoint{4.678139in}{2.637860in}}%
\pgfpathlineto{\pgfqpoint{4.722138in}{2.672806in}}%
\pgfusepath{stroke}%
\end{pgfscope}%
\begin{pgfscope}%
\pgfpathrectangle{\pgfqpoint{0.647939in}{0.492442in}}{\pgfqpoint{4.273799in}{2.331163in}}%
\pgfusepath{clip}%
\pgfsetbuttcap%
\pgfsetroundjoin%
\pgfsetlinewidth{0.301125pt}%
\definecolor{currentstroke}{rgb}{0.500000,0.500000,0.500000}%
\pgfsetstrokecolor{currentstroke}%
\pgfsetstrokeopacity{0.300000}%
\pgfsetdash{}{0pt}%
\pgfpathmoveto{\pgfqpoint{4.144684in}{2.823605in}}%
\pgfpathlineto{\pgfqpoint{4.144684in}{2.823605in}}%
\pgfpathlineto{\pgfqpoint{4.178307in}{2.775158in}}%
\pgfpathlineto{\pgfqpoint{4.212738in}{2.726882in}}%
\pgfpathlineto{\pgfqpoint{4.248365in}{2.678869in}}%
\pgfpathlineto{\pgfqpoint{4.285809in}{2.631275in}}%
\pgfpathlineto{\pgfqpoint{4.326168in}{2.584407in}}%
\pgfpathlineto{\pgfqpoint{4.371645in}{2.539007in}}%
\pgfpathlineto{\pgfqpoint{4.427557in}{2.497590in}}%
\pgfpathlineto{\pgfqpoint{4.427557in}{2.497590in}}%
\pgfpathlineto{\pgfqpoint{4.468266in}{2.479661in}}%
\pgfpathlineto{\pgfqpoint{4.468266in}{2.479661in}}%
\pgfpathlineto{\pgfqpoint{4.506083in}{2.473487in}}%
\pgfpathlineto{\pgfqpoint{4.545135in}{2.477905in}}%
\pgfpathlineto{\pgfqpoint{4.577504in}{2.489283in}}%
\pgfpathlineto{\pgfqpoint{4.611679in}{2.508233in}}%
\pgfpathlineto{\pgfqpoint{4.650364in}{2.537271in}}%
\pgfusepath{stroke}%
\end{pgfscope}%
\begin{pgfscope}%
\pgfpathrectangle{\pgfqpoint{0.647939in}{0.492442in}}{\pgfqpoint{4.273799in}{2.331163in}}%
\pgfusepath{clip}%
\pgfsetbuttcap%
\pgfsetroundjoin%
\pgfsetlinewidth{0.301125pt}%
\definecolor{currentstroke}{rgb}{0.500000,0.500000,0.500000}%
\pgfsetstrokecolor{currentstroke}%
\pgfsetstrokeopacity{0.300000}%
\pgfsetdash{}{0pt}%
\pgfpathmoveto{\pgfqpoint{4.047552in}{2.823605in}}%
\pgfpathlineto{\pgfqpoint{4.047552in}{2.823605in}}%
\pgfpathlineto{\pgfqpoint{4.078300in}{2.774593in}}%
\pgfpathlineto{\pgfqpoint{4.109091in}{2.725588in}}%
\pgfpathlineto{\pgfqpoint{4.140027in}{2.676611in}}%
\pgfpathlineto{\pgfqpoint{4.171269in}{2.627692in}}%
\pgfpathlineto{\pgfqpoint{4.203033in}{2.578876in}}%
\pgfpathlineto{\pgfqpoint{4.235634in}{2.530225in}}%
\pgfpathlineto{\pgfqpoint{4.269621in}{2.481859in}}%
\pgfpathlineto{\pgfqpoint{4.306040in}{2.434038in}}%
\pgfpathlineto{\pgfqpoint{4.347177in}{2.387435in}}%
\pgfpathlineto{\pgfqpoint{4.399804in}{2.344877in}}%
\pgfpathlineto{\pgfqpoint{4.399804in}{2.344877in}}%
\pgfpathlineto{\pgfqpoint{4.431444in}{2.331289in}}%
\pgfpathlineto{\pgfqpoint{4.431444in}{2.331289in}}%
\pgfpathlineto{\pgfqpoint{4.462374in}{2.328095in}}%
\pgfpathlineto{\pgfqpoint{4.492635in}{2.334206in}}%
\pgfpathlineto{\pgfqpoint{4.519609in}{2.346416in}}%
\pgfpathlineto{\pgfqpoint{4.549597in}{2.366431in}}%
\pgfusepath{stroke}%
\end{pgfscope}%
\begin{pgfscope}%
\pgfpathrectangle{\pgfqpoint{0.647939in}{0.492442in}}{\pgfqpoint{4.273799in}{2.331163in}}%
\pgfusepath{clip}%
\pgfsetbuttcap%
\pgfsetroundjoin%
\pgfsetlinewidth{0.301125pt}%
\definecolor{currentstroke}{rgb}{0.500000,0.500000,0.500000}%
\pgfsetstrokecolor{currentstroke}%
\pgfsetstrokeopacity{0.300000}%
\pgfsetdash{}{0pt}%
\pgfpathmoveto{\pgfqpoint{3.950420in}{2.823605in}}%
\pgfpathlineto{\pgfqpoint{3.950420in}{2.823605in}}%
\pgfpathlineto{\pgfqpoint{3.979449in}{2.774281in}}%
\pgfpathlineto{\pgfqpoint{4.008122in}{2.724895in}}%
\pgfpathlineto{\pgfqpoint{4.036461in}{2.675452in}}%
\pgfpathlineto{\pgfqpoint{4.064474in}{2.625954in}}%
\pgfpathlineto{\pgfqpoint{4.092173in}{2.576404in}}%
\pgfpathlineto{\pgfqpoint{4.119582in}{2.526806in}}%
\pgfpathlineto{\pgfqpoint{4.146706in}{2.477161in}}%
\pgfpathlineto{\pgfqpoint{4.173571in}{2.427476in}}%
\pgfpathlineto{\pgfqpoint{4.200216in}{2.377757in}}%
\pgfpathlineto{\pgfqpoint{4.226662in}{2.328007in}}%
\pgfpathlineto{\pgfqpoint{4.252990in}{2.278244in}}%
\pgfpathlineto{\pgfqpoint{4.279333in}{2.228492in}}%
\pgfpathlineto{\pgfqpoint{4.306042in}{2.178822in}}%
\pgfpathlineto{\pgfqpoint{4.334614in}{2.129929in}}%
\pgfpathlineto{\pgfqpoint{4.334614in}{2.129929in}}%
\pgfpathlineto{\pgfqpoint{4.345786in}{2.114482in}}%
\pgfpathlineto{\pgfqpoint{4.345786in}{2.114482in}}%
\pgfpathlineto{\pgfqpoint{4.360793in}{2.103349in}}%
\pgfpathlineto{\pgfqpoint{4.360793in}{2.103349in}}%
\pgfpathlineto{\pgfqpoint{4.360793in}{2.103349in}}%
\pgfpathlineto{\pgfqpoint{4.371075in}{2.104631in}}%
\pgfusepath{stroke}%
\end{pgfscope}%
\begin{pgfscope}%
\pgfpathrectangle{\pgfqpoint{0.647939in}{0.492442in}}{\pgfqpoint{4.273799in}{2.331163in}}%
\pgfusepath{clip}%
\pgfsetbuttcap%
\pgfsetroundjoin%
\pgfsetlinewidth{0.301125pt}%
\definecolor{currentstroke}{rgb}{0.500000,0.500000,0.500000}%
\pgfsetstrokecolor{currentstroke}%
\pgfsetstrokeopacity{0.300000}%
\pgfsetdash{}{0pt}%
\pgfpathmoveto{\pgfqpoint{3.853289in}{2.823605in}}%
\pgfpathlineto{\pgfqpoint{3.853289in}{2.823605in}}%
\pgfpathlineto{\pgfqpoint{3.881389in}{2.774121in}}%
\pgfpathlineto{\pgfqpoint{3.908900in}{2.724539in}}%
\pgfpathlineto{\pgfqpoint{3.935800in}{2.674858in}}%
\pgfpathlineto{\pgfqpoint{3.962053in}{2.625073in}}%
\pgfpathlineto{\pgfqpoint{3.987620in}{2.575183in}}%
\pgfpathlineto{\pgfqpoint{4.012433in}{2.525179in}}%
\pgfpathlineto{\pgfqpoint{4.036412in}{2.475055in}}%
\pgfpathlineto{\pgfqpoint{4.059436in}{2.424799in}}%
\pgfpathlineto{\pgfqpoint{4.081340in}{2.374393in}}%
\pgfpathlineto{\pgfqpoint{4.101882in}{2.323819in}}%
\pgfpathlineto{\pgfqpoint{4.120706in}{2.273047in}}%
\pgfpathlineto{\pgfqpoint{4.137283in}{2.222047in}}%
\pgfpathlineto{\pgfqpoint{4.150779in}{2.170781in}}%
\pgfpathlineto{\pgfqpoint{4.159919in}{2.119240in}}%
\pgfpathlineto{\pgfqpoint{4.162780in}{2.067502in}}%
\pgfpathlineto{\pgfqpoint{4.156832in}{2.015891in}}%
\pgfpathlineto{\pgfqpoint{4.139836in}{1.965073in}}%
\pgfpathlineto{\pgfqpoint{4.111204in}{1.915878in}}%
\pgfpathlineto{\pgfqpoint{4.072372in}{1.868803in}}%
\pgfusepath{stroke}%
\end{pgfscope}%
\begin{pgfscope}%
\pgfpathrectangle{\pgfqpoint{0.647939in}{0.492442in}}{\pgfqpoint{4.273799in}{2.331163in}}%
\pgfusepath{clip}%
\pgfsetbuttcap%
\pgfsetroundjoin%
\pgfsetlinewidth{0.301125pt}%
\definecolor{currentstroke}{rgb}{0.500000,0.500000,0.500000}%
\pgfsetstrokecolor{currentstroke}%
\pgfsetstrokeopacity{0.300000}%
\pgfsetdash{}{0pt}%
\pgfpathmoveto{\pgfqpoint{3.756157in}{2.823605in}}%
\pgfpathlineto{\pgfqpoint{3.756157in}{2.823605in}}%
\pgfpathlineto{\pgfqpoint{3.783927in}{2.774066in}}%
\pgfpathlineto{\pgfqpoint{3.810946in}{2.724404in}}%
\pgfpathlineto{\pgfqpoint{3.837169in}{2.674615in}}%
\pgfpathlineto{\pgfqpoint{3.862554in}{2.624697in}}%
\pgfpathlineto{\pgfqpoint{3.887035in}{2.574644in}}%
\pgfpathlineto{\pgfqpoint{3.910532in}{2.524452in}}%
\pgfpathlineto{\pgfqpoint{3.932940in}{2.474113in}}%
\pgfpathlineto{\pgfqpoint{3.954117in}{2.423615in}}%
\pgfpathlineto{\pgfqpoint{3.973877in}{2.372947in}}%
\pgfpathlineto{\pgfqpoint{3.991972in}{2.322095in}}%
\pgfpathlineto{\pgfqpoint{4.008074in}{2.271046in}}%
\pgfpathlineto{\pgfqpoint{4.021732in}{2.219788in}}%
\pgfpathlineto{\pgfqpoint{4.032347in}{2.168319in}}%
\pgfpathlineto{\pgfqpoint{4.039111in}{2.116661in}}%
\pgfpathlineto{\pgfqpoint{4.040967in}{2.064891in}}%
\pgfpathlineto{\pgfqpoint{4.036631in}{2.013182in}}%
\pgfpathlineto{\pgfqpoint{4.024716in}{1.961852in}}%
\pgfpathlineto{\pgfqpoint{4.004034in}{1.911378in}}%
\pgfpathlineto{\pgfqpoint{3.973896in}{1.862357in}}%
\pgfpathlineto{\pgfqpoint{3.934277in}{1.815395in}}%
\pgfpathlineto{\pgfqpoint{3.885554in}{1.771067in}}%
\pgfusepath{stroke}%
\end{pgfscope}%
\begin{pgfscope}%
\pgfpathrectangle{\pgfqpoint{0.647939in}{0.492442in}}{\pgfqpoint{4.273799in}{2.331163in}}%
\pgfusepath{clip}%
\pgfsetbuttcap%
\pgfsetroundjoin%
\pgfsetlinewidth{0.301125pt}%
\definecolor{currentstroke}{rgb}{0.500000,0.500000,0.500000}%
\pgfsetstrokecolor{currentstroke}%
\pgfsetstrokeopacity{0.300000}%
\pgfsetdash{}{0pt}%
\pgfpathmoveto{\pgfqpoint{3.659025in}{2.823605in}}%
\pgfpathlineto{\pgfqpoint{3.659025in}{2.823605in}}%
\pgfpathlineto{\pgfqpoint{3.686961in}{2.774094in}}%
\pgfpathlineto{\pgfqpoint{3.714007in}{2.724436in}}%
\pgfpathlineto{\pgfqpoint{3.740125in}{2.674631in}}%
\pgfpathlineto{\pgfqpoint{3.765266in}{2.624676in}}%
\pgfpathlineto{\pgfqpoint{3.789358in}{2.574568in}}%
\pgfpathlineto{\pgfqpoint{3.812319in}{2.524303in}}%
\pgfpathlineto{\pgfqpoint{3.834037in}{2.473873in}}%
\pgfpathlineto{\pgfqpoint{3.854372in}{2.423273in}}%
\pgfpathlineto{\pgfqpoint{3.873147in}{2.372494in}}%
\pgfpathlineto{\pgfqpoint{3.890129in}{2.321528in}}%
\pgfpathlineto{\pgfqpoint{3.905027in}{2.270371in}}%
\pgfpathlineto{\pgfqpoint{3.917464in}{2.219019in}}%
\pgfpathlineto{\pgfqpoint{3.926944in}{2.167483in}}%
\pgfpathlineto{\pgfqpoint{3.932847in}{2.115791in}}%
\pgfpathlineto{\pgfqpoint{3.934393in}{2.064013in}}%
\pgfpathlineto{\pgfqpoint{3.930629in}{2.012279in}}%
\pgfpathlineto{\pgfqpoint{3.920460in}{1.960817in}}%
\pgfpathlineto{\pgfqpoint{3.902713in}{1.909987in}}%
\pgfpathlineto{\pgfqpoint{3.876330in}{1.860300in}}%
\pgfpathlineto{\pgfqpoint{3.840449in}{1.812434in}}%
\pgfpathlineto{\pgfqpoint{3.794500in}{1.767231in}}%
\pgfpathlineto{\pgfqpoint{3.738041in}{1.725784in}}%
\pgfpathlineto{\pgfqpoint{3.670854in}{1.689505in}}%
\pgfpathlineto{\pgfqpoint{3.593034in}{1.660405in}}%
\pgfpathlineto{\pgfqpoint{3.508690in}{1.641620in}}%
\pgfpathlineto{\pgfqpoint{3.426812in}{1.634589in}}%
\pgfpathlineto{\pgfqpoint{3.349332in}{1.637618in}}%
\pgfpathlineto{\pgfqpoint{3.275240in}{1.649296in}}%
\pgfusepath{stroke}%
\end{pgfscope}%
\begin{pgfscope}%
\pgfpathrectangle{\pgfqpoint{0.647939in}{0.492442in}}{\pgfqpoint{4.273799in}{2.331163in}}%
\pgfusepath{clip}%
\pgfsetbuttcap%
\pgfsetroundjoin%
\pgfsetlinewidth{0.301125pt}%
\definecolor{currentstroke}{rgb}{0.500000,0.500000,0.500000}%
\pgfsetstrokecolor{currentstroke}%
\pgfsetstrokeopacity{0.300000}%
\pgfsetdash{}{0pt}%
\pgfpathmoveto{\pgfqpoint{3.561893in}{2.823605in}}%
\pgfpathlineto{\pgfqpoint{3.561893in}{2.823605in}}%
\pgfpathlineto{\pgfqpoint{3.590431in}{2.774197in}}%
\pgfpathlineto{\pgfqpoint{3.617963in}{2.724619in}}%
\pgfpathlineto{\pgfqpoint{3.644452in}{2.674872in}}%
\pgfpathlineto{\pgfqpoint{3.669841in}{2.624955in}}%
\pgfpathlineto{\pgfqpoint{3.694065in}{2.574866in}}%
\pgfpathlineto{\pgfqpoint{3.717041in}{2.524603in}}%
\pgfpathlineto{\pgfqpoint{3.738661in}{2.474161in}}%
\pgfpathlineto{\pgfqpoint{3.758789in}{2.423536in}}%
\pgfpathlineto{\pgfqpoint{3.777256in}{2.372723in}}%
\pgfpathlineto{\pgfqpoint{3.793851in}{2.321720in}}%
\pgfpathlineto{\pgfqpoint{3.808300in}{2.270524in}}%
\pgfpathlineto{\pgfqpoint{3.820268in}{2.219140in}}%
\pgfpathlineto{\pgfqpoint{3.829324in}{2.167581in}}%
\pgfpathlineto{\pgfqpoint{3.834923in}{2.115878in}}%
\pgfpathlineto{\pgfqpoint{3.836391in}{2.064097in}}%
\pgfpathlineto{\pgfqpoint{3.832896in}{2.012353in}}%
\pgfpathlineto{\pgfqpoint{3.823435in}{1.960845in}}%
\pgfpathlineto{\pgfqpoint{3.806853in}{1.909890in}}%
\pgfpathlineto{\pgfqpoint{3.781883in}{1.859981in}}%
\pgfpathlineto{\pgfqpoint{3.747208in}{1.811856in}}%
\pgfpathlineto{\pgfqpoint{3.701530in}{1.766595in}}%
\pgfusepath{stroke}%
\end{pgfscope}%
\begin{pgfscope}%
\pgfpathrectangle{\pgfqpoint{0.647939in}{0.492442in}}{\pgfqpoint{4.273799in}{2.331163in}}%
\pgfusepath{clip}%
\pgfsetbuttcap%
\pgfsetroundjoin%
\pgfsetlinewidth{0.301125pt}%
\definecolor{currentstroke}{rgb}{0.500000,0.500000,0.500000}%
\pgfsetstrokecolor{currentstroke}%
\pgfsetstrokeopacity{0.300000}%
\pgfsetdash{}{0pt}%
\pgfpathmoveto{\pgfqpoint{3.464761in}{2.823605in}}%
\pgfpathlineto{\pgfqpoint{3.464761in}{2.823605in}}%
\pgfpathlineto{\pgfqpoint{3.494327in}{2.774377in}}%
\pgfpathlineto{\pgfqpoint{3.522773in}{2.724953in}}%
\pgfpathlineto{\pgfqpoint{3.550062in}{2.675335in}}%
\pgfpathlineto{\pgfqpoint{3.576139in}{2.625524in}}%
\pgfpathlineto{\pgfqpoint{3.600944in}{2.575520in}}%
\pgfpathlineto{\pgfqpoint{3.624390in}{2.525322in}}%
\pgfpathlineto{\pgfqpoint{3.646378in}{2.474928in}}%
\pgfpathlineto{\pgfqpoint{3.666777in}{2.424335in}}%
\pgfpathlineto{\pgfqpoint{3.685426in}{2.373543in}}%
\pgfpathlineto{\pgfqpoint{3.702124in}{2.322550in}}%
\pgfpathlineto{\pgfqpoint{3.716611in}{2.271358in}}%
\pgfpathlineto{\pgfqpoint{3.728572in}{2.219973in}}%
\pgfpathlineto{\pgfqpoint{3.737605in}{2.168412in}}%
\pgfpathlineto{\pgfqpoint{3.743202in}{2.116709in}}%
\pgfpathlineto{\pgfqpoint{3.744723in}{2.064928in}}%
\pgfpathlineto{\pgfqpoint{3.741369in}{2.013180in}}%
\pgfpathlineto{\pgfqpoint{3.732139in}{1.961658in}}%
\pgfpathlineto{\pgfqpoint{3.715812in}{1.910679in}}%
\pgfpathlineto{\pgfqpoint{3.690894in}{1.860769in}}%
\pgfpathlineto{\pgfqpoint{3.655615in}{1.812794in}}%
\pgfpathlineto{\pgfqpoint{3.607980in}{1.768199in}}%
\pgfpathlineto{\pgfqpoint{3.546042in}{1.729405in}}%
\pgfpathlineto{\pgfqpoint{3.473337in}{1.701373in}}%
\pgfpathlineto{\pgfqpoint{3.400550in}{1.687102in}}%
\pgfusepath{stroke}%
\end{pgfscope}%
\begin{pgfscope}%
\pgfpathrectangle{\pgfqpoint{0.647939in}{0.492442in}}{\pgfqpoint{4.273799in}{2.331163in}}%
\pgfusepath{clip}%
\pgfsetbuttcap%
\pgfsetroundjoin%
\pgfsetlinewidth{0.301125pt}%
\definecolor{currentstroke}{rgb}{0.500000,0.500000,0.500000}%
\pgfsetstrokecolor{currentstroke}%
\pgfsetstrokeopacity{0.300000}%
\pgfsetdash{}{0pt}%
\pgfpathmoveto{\pgfqpoint{3.367630in}{2.823605in}}%
\pgfpathlineto{\pgfqpoint{3.367630in}{2.823605in}}%
\pgfpathlineto{\pgfqpoint{3.398664in}{2.774647in}}%
\pgfpathlineto{\pgfqpoint{3.428457in}{2.725460in}}%
\pgfpathlineto{\pgfqpoint{3.456967in}{2.676048in}}%
\pgfpathlineto{\pgfqpoint{3.484150in}{2.626413in}}%
\pgfpathlineto{\pgfqpoint{3.509949in}{2.576559in}}%
\pgfpathlineto{\pgfqpoint{3.534280in}{2.526487in}}%
\pgfpathlineto{\pgfqpoint{3.557047in}{2.476196in}}%
\pgfpathlineto{\pgfqpoint{3.578126in}{2.425687in}}%
\pgfpathlineto{\pgfqpoint{3.597359in}{2.374960in}}%
\pgfusepath{stroke}%
\end{pgfscope}%
\begin{pgfscope}%
\pgfpathrectangle{\pgfqpoint{0.647939in}{0.492442in}}{\pgfqpoint{4.273799in}{2.331163in}}%
\pgfusepath{clip}%
\pgfsetbuttcap%
\pgfsetroundjoin%
\pgfsetlinewidth{0.301125pt}%
\definecolor{currentstroke}{rgb}{0.500000,0.500000,0.500000}%
\pgfsetstrokecolor{currentstroke}%
\pgfsetstrokeopacity{0.300000}%
\pgfsetdash{}{0pt}%
\pgfpathmoveto{\pgfqpoint{3.173366in}{2.823605in}}%
\pgfpathlineto{\pgfqpoint{3.173366in}{2.823605in}}%
\pgfpathlineto{\pgfqpoint{3.208807in}{2.775547in}}%
\pgfpathlineto{\pgfqpoint{3.242694in}{2.727157in}}%
\pgfpathlineto{\pgfqpoint{3.275006in}{2.678446in}}%
\pgfpathlineto{\pgfqpoint{3.305710in}{2.629427in}}%
\pgfpathlineto{\pgfqpoint{3.334758in}{2.580110in}}%
\pgfpathlineto{\pgfqpoint{3.362080in}{2.530499in}}%
\pgfpathlineto{\pgfqpoint{3.387592in}{2.480603in}}%
\pgfpathlineto{\pgfqpoint{3.411176in}{2.430426in}}%
\pgfpathlineto{\pgfqpoint{3.432686in}{2.379973in}}%
\pgfpathlineto{\pgfqpoint{3.451932in}{2.329249in}}%
\pgfpathlineto{\pgfqpoint{3.468670in}{2.278262in}}%
\pgfpathlineto{\pgfqpoint{3.482588in}{2.227025in}}%
\pgfpathlineto{\pgfqpoint{3.493272in}{2.175561in}}%
\pgfpathlineto{\pgfqpoint{3.500184in}{2.123909in}}%
\pgfpathlineto{\pgfqpoint{3.502607in}{2.072141in}}%
\pgfpathlineto{\pgfqpoint{3.499556in}{2.020393in}}%
\pgfpathlineto{\pgfqpoint{3.489646in}{1.968921in}}%
\pgfpathlineto{\pgfqpoint{3.470849in}{1.918235in}}%
\pgfpathlineto{\pgfqpoint{3.440117in}{1.869411in}}%
\pgfpathlineto{\pgfqpoint{3.392834in}{1.824977in}}%
\pgfpathlineto{\pgfqpoint{3.392834in}{1.824977in}}%
\pgfpathlineto{\pgfqpoint{3.345296in}{1.798611in}}%
\pgfpathlineto{\pgfqpoint{3.345296in}{1.798611in}}%
\pgfpathlineto{\pgfqpoint{3.294225in}{1.783059in}}%
\pgfpathlineto{\pgfqpoint{3.236367in}{1.777218in}}%
\pgfpathlineto{\pgfqpoint{3.181973in}{1.781048in}}%
\pgfpathlineto{\pgfqpoint{3.129058in}{1.792649in}}%
\pgfpathlineto{\pgfqpoint{3.075295in}{1.812228in}}%
\pgfpathlineto{\pgfqpoint{3.020670in}{1.840612in}}%
\pgfpathlineto{\pgfqpoint{2.966425in}{1.878651in}}%
\pgfpathlineto{\pgfqpoint{2.918510in}{1.923147in}}%
\pgfpathlineto{\pgfqpoint{2.880554in}{1.970452in}}%
\pgfusepath{stroke}%
\end{pgfscope}%
\begin{pgfscope}%
\pgfpathrectangle{\pgfqpoint{0.647939in}{0.492442in}}{\pgfqpoint{4.273799in}{2.331163in}}%
\pgfusepath{clip}%
\pgfsetbuttcap%
\pgfsetroundjoin%
\pgfsetlinewidth{0.301125pt}%
\definecolor{currentstroke}{rgb}{0.500000,0.500000,0.500000}%
\pgfsetstrokecolor{currentstroke}%
\pgfsetstrokeopacity{0.300000}%
\pgfsetdash{}{0pt}%
\pgfpathmoveto{\pgfqpoint{3.076234in}{2.823605in}}%
\pgfpathlineto{\pgfqpoint{3.076234in}{2.823605in}}%
\pgfpathlineto{\pgfqpoint{3.114714in}{2.776249in}}%
\pgfpathlineto{\pgfqpoint{3.151446in}{2.728480in}}%
\pgfpathlineto{\pgfqpoint{3.186414in}{2.680319in}}%
\pgfpathlineto{\pgfqpoint{3.219597in}{2.631784in}}%
\pgfpathlineto{\pgfqpoint{3.250952in}{2.582890in}}%
\pgfpathlineto{\pgfqpoint{3.280418in}{2.533646in}}%
\pgfpathlineto{\pgfqpoint{3.307919in}{2.484066in}}%
\pgfusepath{stroke}%
\end{pgfscope}%
\begin{pgfscope}%
\pgfpathrectangle{\pgfqpoint{0.647939in}{0.492442in}}{\pgfqpoint{4.273799in}{2.331163in}}%
\pgfusepath{clip}%
\pgfsetbuttcap%
\pgfsetroundjoin%
\pgfsetlinewidth{0.301125pt}%
\definecolor{currentstroke}{rgb}{0.500000,0.500000,0.500000}%
\pgfsetstrokecolor{currentstroke}%
\pgfsetstrokeopacity{0.300000}%
\pgfsetdash{}{0pt}%
\pgfpathmoveto{\pgfqpoint{2.881971in}{2.823605in}}%
\pgfpathlineto{\pgfqpoint{2.881971in}{2.823605in}}%
\pgfpathlineto{\pgfqpoint{2.928473in}{2.778445in}}%
\pgfpathlineto{\pgfqpoint{2.972737in}{2.732620in}}%
\pgfpathlineto{\pgfqpoint{3.014754in}{2.686169in}}%
\pgfpathlineto{\pgfqpoint{3.054516in}{2.639132in}}%
\pgfpathlineto{\pgfqpoint{3.092006in}{2.591543in}}%
\pgfpathlineto{\pgfqpoint{3.127192in}{2.543434in}}%
\pgfpathlineto{\pgfqpoint{3.160019in}{2.494831in}}%
\pgfpathlineto{\pgfqpoint{3.190404in}{2.445757in}}%
\pgfpathlineto{\pgfqpoint{3.218230in}{2.396234in}}%
\pgfpathlineto{\pgfqpoint{3.243326in}{2.346280in}}%
\pgfpathlineto{\pgfqpoint{3.265459in}{2.295912in}}%
\pgfpathlineto{\pgfqpoint{3.284304in}{2.245150in}}%
\pgfpathlineto{\pgfqpoint{3.299410in}{2.194021in}}%
\pgfpathlineto{\pgfqpoint{3.310141in}{2.142568in}}%
\pgfpathlineto{\pgfqpoint{3.315557in}{2.090875in}}%
\pgfpathlineto{\pgfqpoint{3.314225in}{2.039125in}}%
\pgfpathlineto{\pgfqpoint{3.303785in}{1.987734in}}%
\pgfpathlineto{\pgfqpoint{3.279974in}{1.937828in}}%
\pgfpathlineto{\pgfqpoint{3.279974in}{1.937828in}}%
\pgfpathlineto{\pgfqpoint{3.247089in}{1.902014in}}%
\pgfpathlineto{\pgfqpoint{3.247089in}{1.902014in}}%
\pgfpathlineto{\pgfqpoint{3.209364in}{1.879707in}}%
\pgfpathlineto{\pgfqpoint{3.209364in}{1.879707in}}%
\pgfpathlineto{\pgfqpoint{3.168346in}{1.868304in}}%
\pgfpathlineto{\pgfqpoint{3.122402in}{1.866566in}}%
\pgfpathlineto{\pgfqpoint{3.079807in}{1.873242in}}%
\pgfpathlineto{\pgfqpoint{3.037269in}{1.887209in}}%
\pgfpathlineto{\pgfqpoint{2.993581in}{1.909308in}}%
\pgfpathlineto{\pgfqpoint{2.949796in}{1.940650in}}%
\pgfusepath{stroke}%
\end{pgfscope}%
\begin{pgfscope}%
\pgfpathrectangle{\pgfqpoint{0.647939in}{0.492442in}}{\pgfqpoint{4.273799in}{2.331163in}}%
\pgfusepath{clip}%
\pgfsetbuttcap%
\pgfsetroundjoin%
\pgfsetlinewidth{0.301125pt}%
\definecolor{currentstroke}{rgb}{0.500000,0.500000,0.500000}%
\pgfsetstrokecolor{currentstroke}%
\pgfsetstrokeopacity{0.300000}%
\pgfsetdash{}{0pt}%
\pgfpathmoveto{\pgfqpoint{2.687707in}{2.823605in}}%
\pgfpathlineto{\pgfqpoint{2.687707in}{2.823605in}}%
\pgfpathlineto{\pgfqpoint{2.744711in}{2.782190in}}%
\pgfpathlineto{\pgfqpoint{2.798978in}{2.739694in}}%
\pgfpathlineto{\pgfqpoint{2.850436in}{2.696167in}}%
\pgfpathlineto{\pgfqpoint{2.899043in}{2.651675in}}%
\pgfpathlineto{\pgfqpoint{2.944781in}{2.606289in}}%
\pgfpathlineto{\pgfqpoint{2.987646in}{2.560075in}}%
\pgfpathlineto{\pgfqpoint{3.027621in}{2.513096in}}%
\pgfpathlineto{\pgfqpoint{3.064668in}{2.465406in}}%
\pgfpathlineto{\pgfqpoint{3.098714in}{2.417056in}}%
\pgfpathlineto{\pgfqpoint{3.129639in}{2.368089in}}%
\pgfpathlineto{\pgfqpoint{3.157253in}{2.318537in}}%
\pgfpathlineto{\pgfqpoint{3.181271in}{2.268432in}}%
\pgfpathlineto{\pgfqpoint{3.201273in}{2.217808in}}%
\pgfpathlineto{\pgfqpoint{3.216627in}{2.166708in}}%
\pgfusepath{stroke}%
\end{pgfscope}%
\begin{pgfscope}%
\pgfpathrectangle{\pgfqpoint{0.647939in}{0.492442in}}{\pgfqpoint{4.273799in}{2.331163in}}%
\pgfusepath{clip}%
\pgfsetbuttcap%
\pgfsetroundjoin%
\pgfsetlinewidth{0.301125pt}%
\definecolor{currentstroke}{rgb}{0.500000,0.500000,0.500000}%
\pgfsetstrokecolor{currentstroke}%
\pgfsetstrokeopacity{0.300000}%
\pgfsetdash{}{0pt}%
\pgfpathmoveto{\pgfqpoint{2.493443in}{2.823605in}}%
\pgfpathlineto{\pgfqpoint{2.493443in}{2.823605in}}%
\pgfpathlineto{\pgfqpoint{2.561743in}{2.787634in}}%
\pgfpathlineto{\pgfqpoint{2.627393in}{2.750228in}}%
\pgfpathlineto{\pgfqpoint{2.689984in}{2.711297in}}%
\pgfpathlineto{\pgfqpoint{2.749256in}{2.670853in}}%
\pgfpathlineto{\pgfqpoint{2.805082in}{2.628969in}}%
\pgfpathlineto{\pgfqpoint{2.857393in}{2.585755in}}%
\pgfpathlineto{\pgfqpoint{2.906165in}{2.541329in}}%
\pgfusepath{stroke}%
\end{pgfscope}%
\begin{pgfscope}%
\pgfpathrectangle{\pgfqpoint{0.647939in}{0.492442in}}{\pgfqpoint{4.273799in}{2.331163in}}%
\pgfusepath{clip}%
\pgfsetbuttcap%
\pgfsetroundjoin%
\pgfsetlinewidth{0.301125pt}%
\definecolor{currentstroke}{rgb}{0.500000,0.500000,0.500000}%
\pgfsetstrokecolor{currentstroke}%
\pgfsetstrokeopacity{0.300000}%
\pgfsetdash{}{0pt}%
\pgfpathmoveto{\pgfqpoint{2.396312in}{2.823605in}}%
\pgfpathlineto{\pgfqpoint{2.396312in}{2.823605in}}%
\pgfpathlineto{\pgfqpoint{2.469407in}{2.790549in}}%
\pgfpathlineto{\pgfqpoint{2.540416in}{2.756174in}}%
\pgfpathlineto{\pgfqpoint{2.608629in}{2.720168in}}%
\pgfpathlineto{\pgfqpoint{2.673569in}{2.682403in}}%
\pgfpathlineto{\pgfqpoint{2.734906in}{2.642890in}}%
\pgfusepath{stroke}%
\end{pgfscope}%
\begin{pgfscope}%
\pgfpathrectangle{\pgfqpoint{0.647939in}{0.492442in}}{\pgfqpoint{4.273799in}{2.331163in}}%
\pgfusepath{clip}%
\pgfsetbuttcap%
\pgfsetroundjoin%
\pgfsetlinewidth{0.301125pt}%
\definecolor{currentstroke}{rgb}{0.500000,0.500000,0.500000}%
\pgfsetstrokecolor{currentstroke}%
\pgfsetstrokeopacity{0.300000}%
\pgfsetdash{}{0pt}%
\pgfpathmoveto{\pgfqpoint{2.202048in}{2.823605in}}%
\pgfpathlineto{\pgfqpoint{2.202048in}{2.823605in}}%
\pgfpathlineto{\pgfqpoint{2.280503in}{2.794418in}}%
\pgfpathlineto{\pgfqpoint{2.359219in}{2.765446in}}%
\pgfpathlineto{\pgfqpoint{2.436934in}{2.735699in}}%
\pgfpathlineto{\pgfqpoint{2.512537in}{2.704391in}}%
\pgfpathlineto{\pgfqpoint{2.585110in}{2.671025in}}%
\pgfpathlineto{\pgfqpoint{2.653934in}{2.635391in}}%
\pgfpathlineto{\pgfqpoint{2.718602in}{2.597506in}}%
\pgfpathlineto{\pgfqpoint{2.778885in}{2.557527in}}%
\pgfpathlineto{\pgfqpoint{2.834701in}{2.515666in}}%
\pgfpathlineto{\pgfqpoint{2.886085in}{2.472141in}}%
\pgfpathlineto{\pgfqpoint{2.933074in}{2.427151in}}%
\pgfpathlineto{\pgfqpoint{2.975675in}{2.380876in}}%
\pgfpathlineto{\pgfqpoint{3.013830in}{2.333464in}}%
\pgfpathlineto{\pgfqpoint{3.047377in}{2.285030in}}%
\pgfpathlineto{\pgfqpoint{3.075982in}{2.235665in}}%
\pgfpathlineto{\pgfqpoint{3.099036in}{2.185446in}}%
\pgfpathlineto{\pgfqpoint{3.115451in}{2.134472in}}%
\pgfusepath{stroke}%
\end{pgfscope}%
\begin{pgfscope}%
\pgfpathrectangle{\pgfqpoint{0.647939in}{0.492442in}}{\pgfqpoint{4.273799in}{2.331163in}}%
\pgfusepath{clip}%
\pgfsetbuttcap%
\pgfsetroundjoin%
\pgfsetlinewidth{0.301125pt}%
\definecolor{currentstroke}{rgb}{0.500000,0.500000,0.500000}%
\pgfsetstrokecolor{currentstroke}%
\pgfsetstrokeopacity{0.300000}%
\pgfsetdash{}{0pt}%
\pgfpathmoveto{\pgfqpoint{2.007784in}{2.823605in}}%
\pgfpathlineto{\pgfqpoint{2.007784in}{2.823605in}}%
\pgfpathlineto{\pgfqpoint{2.083389in}{2.792338in}}%
\pgfpathlineto{\pgfqpoint{2.163057in}{2.764202in}}%
\pgfpathlineto{\pgfqpoint{2.245235in}{2.738263in}}%
\pgfpathlineto{\pgfqpoint{2.328246in}{2.713111in}}%
\pgfpathlineto{\pgfqpoint{2.410505in}{2.687251in}}%
\pgfpathlineto{\pgfqpoint{2.490608in}{2.659487in}}%
\pgfpathlineto{\pgfqpoint{2.567374in}{2.629085in}}%
\pgfusepath{stroke}%
\end{pgfscope}%
\begin{pgfscope}%
\pgfpathrectangle{\pgfqpoint{0.647939in}{0.492442in}}{\pgfqpoint{4.273799in}{2.331163in}}%
\pgfusepath{clip}%
\pgfsetbuttcap%
\pgfsetroundjoin%
\pgfsetlinewidth{0.301125pt}%
\definecolor{currentstroke}{rgb}{0.500000,0.500000,0.500000}%
\pgfsetstrokecolor{currentstroke}%
\pgfsetstrokeopacity{0.300000}%
\pgfsetdash{}{0pt}%
\pgfpathmoveto{\pgfqpoint{1.813521in}{2.823605in}}%
\pgfpathlineto{\pgfqpoint{1.813521in}{2.823605in}}%
\pgfpathlineto{\pgfqpoint{1.876056in}{2.784697in}}%
\pgfpathlineto{\pgfqpoint{1.945549in}{2.749521in}}%
\pgfpathlineto{\pgfqpoint{2.022133in}{2.719063in}}%
\pgfpathlineto{\pgfqpoint{2.104697in}{2.693652in}}%
\pgfpathlineto{\pgfqpoint{2.191197in}{2.672388in}}%
\pgfpathlineto{\pgfqpoint{2.279421in}{2.653257in}}%
\pgfpathlineto{\pgfqpoint{2.367466in}{2.633878in}}%
\pgfpathlineto{\pgfqpoint{2.453654in}{2.612226in}}%
\pgfpathlineto{\pgfqpoint{2.536453in}{2.587010in}}%
\pgfpathlineto{\pgfqpoint{2.614572in}{2.557739in}}%
\pgfpathlineto{\pgfqpoint{2.687181in}{2.524523in}}%
\pgfpathlineto{\pgfqpoint{2.753909in}{2.487784in}}%
\pgfpathlineto{\pgfqpoint{2.814648in}{2.448063in}}%
\pgfpathlineto{\pgfqpoint{2.869517in}{2.405864in}}%
\pgfpathlineto{\pgfqpoint{2.918662in}{2.361610in}}%
\pgfpathlineto{\pgfqpoint{2.962179in}{2.315628in}}%
\pgfpathlineto{\pgfqpoint{3.000003in}{2.268168in}}%
\pgfusepath{stroke}%
\end{pgfscope}%
\begin{pgfscope}%
\pgfpathrectangle{\pgfqpoint{0.647939in}{0.492442in}}{\pgfqpoint{4.273799in}{2.331163in}}%
\pgfusepath{clip}%
\pgfsetbuttcap%
\pgfsetroundjoin%
\pgfsetlinewidth{0.301125pt}%
\definecolor{currentstroke}{rgb}{0.500000,0.500000,0.500000}%
\pgfsetstrokecolor{currentstroke}%
\pgfsetstrokeopacity{0.300000}%
\pgfsetdash{}{0pt}%
\pgfpathmoveto{\pgfqpoint{1.716389in}{2.823605in}}%
\pgfpathlineto{\pgfqpoint{1.716389in}{2.823605in}}%
\pgfpathlineto{\pgfqpoint{1.769562in}{2.780729in}}%
\pgfpathlineto{\pgfqpoint{1.828997in}{2.740405in}}%
\pgfpathlineto{\pgfqpoint{1.896066in}{2.703890in}}%
\pgfpathlineto{\pgfqpoint{1.971674in}{2.672811in}}%
\pgfpathlineto{\pgfqpoint{2.055043in}{2.648384in}}%
\pgfpathlineto{\pgfqpoint{2.143722in}{2.630166in}}%
\pgfpathlineto{\pgfqpoint{2.234890in}{2.615761in}}%
\pgfpathlineto{\pgfqpoint{2.326389in}{2.601968in}}%
\pgfusepath{stroke}%
\end{pgfscope}%
\begin{pgfscope}%
\pgfpathrectangle{\pgfqpoint{0.647939in}{0.492442in}}{\pgfqpoint{4.273799in}{2.331163in}}%
\pgfusepath{clip}%
\pgfsetbuttcap%
\pgfsetroundjoin%
\pgfsetlinewidth{0.301125pt}%
\definecolor{currentstroke}{rgb}{0.500000,0.500000,0.500000}%
\pgfsetstrokecolor{currentstroke}%
\pgfsetstrokeopacity{0.300000}%
\pgfsetdash{}{0pt}%
\pgfpathmoveto{\pgfqpoint{1.619257in}{2.823605in}}%
\pgfpathlineto{\pgfqpoint{1.619257in}{2.823605in}}%
\pgfpathlineto{\pgfqpoint{1.662992in}{2.777650in}}%
\pgfpathlineto{\pgfqpoint{1.711058in}{2.733016in}}%
\pgfpathlineto{\pgfqpoint{1.764909in}{2.690421in}}%
\pgfpathlineto{\pgfqpoint{1.826518in}{2.651140in}}%
\pgfpathlineto{\pgfqpoint{1.897989in}{2.617357in}}%
\pgfpathlineto{\pgfqpoint{1.980086in}{2.591960in}}%
\pgfpathlineto{\pgfqpoint{2.066143in}{2.576739in}}%
\pgfpathlineto{\pgfqpoint{2.158290in}{2.568227in}}%
\pgfpathlineto{\pgfqpoint{2.252655in}{2.562575in}}%
\pgfpathlineto{\pgfqpoint{2.346660in}{2.555644in}}%
\pgfpathlineto{\pgfqpoint{2.439055in}{2.544344in}}%
\pgfpathlineto{\pgfqpoint{2.528080in}{2.526924in}}%
\pgfpathlineto{\pgfqpoint{2.611792in}{2.502970in}}%
\pgfpathlineto{\pgfqpoint{2.688837in}{2.473044in}}%
\pgfusepath{stroke}%
\end{pgfscope}%
\begin{pgfscope}%
\pgfpathrectangle{\pgfqpoint{0.647939in}{0.492442in}}{\pgfqpoint{4.273799in}{2.331163in}}%
\pgfusepath{clip}%
\pgfsetbuttcap%
\pgfsetroundjoin%
\pgfsetlinewidth{0.301125pt}%
\definecolor{currentstroke}{rgb}{0.500000,0.500000,0.500000}%
\pgfsetstrokecolor{currentstroke}%
\pgfsetstrokeopacity{0.300000}%
\pgfsetdash{}{0pt}%
\pgfpathmoveto{\pgfqpoint{1.522125in}{2.823605in}}%
\pgfpathlineto{\pgfqpoint{1.522125in}{2.823605in}}%
\pgfpathlineto{\pgfqpoint{1.557564in}{2.775551in}}%
\pgfpathlineto{\pgfqpoint{1.595437in}{2.728055in}}%
\pgfpathlineto{\pgfqpoint{1.636484in}{2.681357in}}%
\pgfpathlineto{\pgfqpoint{1.681859in}{2.635883in}}%
\pgfpathlineto{\pgfqpoint{1.733429in}{2.592449in}}%
\pgfpathlineto{\pgfqpoint{1.794039in}{2.552797in}}%
\pgfpathlineto{\pgfqpoint{1.867618in}{2.520731in}}%
\pgfpathlineto{\pgfqpoint{1.867618in}{2.520731in}}%
\pgfpathlineto{\pgfqpoint{1.933038in}{2.504991in}}%
\pgfpathlineto{\pgfqpoint{2.004169in}{2.498334in}}%
\pgfpathlineto{\pgfqpoint{2.078552in}{2.498951in}}%
\pgfpathlineto{\pgfqpoint{2.171036in}{2.504620in}}%
\pgfpathlineto{\pgfqpoint{2.265262in}{2.510448in}}%
\pgfpathlineto{\pgfqpoint{2.359905in}{2.512154in}}%
\pgfusepath{stroke}%
\end{pgfscope}%
\begin{pgfscope}%
\pgfpathrectangle{\pgfqpoint{0.647939in}{0.492442in}}{\pgfqpoint{4.273799in}{2.331163in}}%
\pgfusepath{clip}%
\pgfsetbuttcap%
\pgfsetroundjoin%
\pgfsetlinewidth{0.301125pt}%
\definecolor{currentstroke}{rgb}{0.500000,0.500000,0.500000}%
\pgfsetstrokecolor{currentstroke}%
\pgfsetstrokeopacity{0.300000}%
\pgfsetdash{}{0pt}%
\pgfpathmoveto{\pgfqpoint{1.424993in}{2.823605in}}%
\pgfpathlineto{\pgfqpoint{1.424993in}{2.823605in}}%
\pgfpathlineto{\pgfqpoint{1.453596in}{2.774210in}}%
\pgfpathlineto{\pgfqpoint{1.483277in}{2.725009in}}%
\pgfpathlineto{\pgfqpoint{1.514307in}{2.676055in}}%
\pgfpathlineto{\pgfqpoint{1.547037in}{2.627433in}}%
\pgfpathlineto{\pgfqpoint{1.581990in}{2.579283in}}%
\pgfpathlineto{\pgfqpoint{1.620042in}{2.531853in}}%
\pgfpathlineto{\pgfqpoint{1.662743in}{2.485649in}}%
\pgfpathlineto{\pgfqpoint{1.713105in}{2.441909in}}%
\pgfpathlineto{\pgfqpoint{1.777621in}{2.404524in}}%
\pgfpathlineto{\pgfqpoint{1.777621in}{2.404524in}}%
\pgfpathlineto{\pgfqpoint{1.826690in}{2.389679in}}%
\pgfpathlineto{\pgfqpoint{1.883408in}{2.385117in}}%
\pgfpathlineto{\pgfqpoint{1.934726in}{2.389542in}}%
\pgfpathlineto{\pgfqpoint{1.992104in}{2.400623in}}%
\pgfpathlineto{\pgfqpoint{2.068736in}{2.420098in}}%
\pgfpathlineto{\pgfqpoint{2.154375in}{2.442345in}}%
\pgfpathlineto{\pgfqpoint{2.242343in}{2.461507in}}%
\pgfpathlineto{\pgfqpoint{2.333726in}{2.474660in}}%
\pgfusepath{stroke}%
\end{pgfscope}%
\begin{pgfscope}%
\pgfpathrectangle{\pgfqpoint{0.647939in}{0.492442in}}{\pgfqpoint{4.273799in}{2.331163in}}%
\pgfusepath{clip}%
\pgfsetbuttcap%
\pgfsetroundjoin%
\pgfsetlinewidth{0.301125pt}%
\definecolor{currentstroke}{rgb}{0.500000,0.500000,0.500000}%
\pgfsetstrokecolor{currentstroke}%
\pgfsetstrokeopacity{0.300000}%
\pgfsetdash{}{0pt}%
\pgfpathmoveto{\pgfqpoint{1.327862in}{2.823605in}}%
\pgfpathlineto{\pgfqpoint{1.327862in}{2.823605in}}%
\pgfpathlineto{\pgfqpoint{1.351050in}{2.773372in}}%
\pgfpathlineto{\pgfqpoint{1.374550in}{2.723182in}}%
\pgfpathlineto{\pgfqpoint{1.398388in}{2.673041in}}%
\pgfpathlineto{\pgfqpoint{1.422607in}{2.622953in}}%
\pgfpathlineto{\pgfqpoint{1.447252in}{2.572928in}}%
\pgfpathlineto{\pgfqpoint{1.472390in}{2.522977in}}%
\pgfpathlineto{\pgfqpoint{1.498122in}{2.473117in}}%
\pgfpathlineto{\pgfqpoint{1.524585in}{2.423376in}}%
\pgfpathlineto{\pgfqpoint{1.552049in}{2.373796in}}%
\pgfpathlineto{\pgfqpoint{1.580943in}{2.324467in}}%
\pgfpathlineto{\pgfqpoint{1.612187in}{2.275600in}}%
\pgfpathlineto{\pgfqpoint{1.648495in}{2.227856in}}%
\pgfpathlineto{\pgfqpoint{1.648495in}{2.227856in}}%
\pgfpathlineto{\pgfqpoint{1.684231in}{2.195733in}}%
\pgfpathlineto{\pgfqpoint{1.684231in}{2.195733in}}%
\pgfpathlineto{\pgfqpoint{1.708021in}{2.185426in}}%
\pgfpathlineto{\pgfqpoint{1.708021in}{2.185426in}}%
\pgfpathlineto{\pgfqpoint{1.733612in}{2.184614in}}%
\pgfpathlineto{\pgfqpoint{1.757706in}{2.191100in}}%
\pgfpathlineto{\pgfqpoint{1.783588in}{2.203089in}}%
\pgfpathlineto{\pgfqpoint{1.822047in}{2.225754in}}%
\pgfpathlineto{\pgfqpoint{1.883198in}{2.264879in}}%
\pgfpathlineto{\pgfqpoint{1.945544in}{2.303715in}}%
\pgfpathlineto{\pgfqpoint{2.011086in}{2.341032in}}%
\pgfusepath{stroke}%
\end{pgfscope}%
\begin{pgfscope}%
\pgfpathrectangle{\pgfqpoint{0.647939in}{0.492442in}}{\pgfqpoint{4.273799in}{2.331163in}}%
\pgfusepath{clip}%
\pgfsetbuttcap%
\pgfsetroundjoin%
\pgfsetlinewidth{0.301125pt}%
\definecolor{currentstroke}{rgb}{0.500000,0.500000,0.500000}%
\pgfsetstrokecolor{currentstroke}%
\pgfsetstrokeopacity{0.300000}%
\pgfsetdash{}{0pt}%
\pgfpathmoveto{\pgfqpoint{1.230730in}{2.823605in}}%
\pgfpathlineto{\pgfqpoint{1.230730in}{2.823605in}}%
\pgfpathlineto{\pgfqpoint{1.249695in}{2.772846in}}%
\pgfpathlineto{\pgfqpoint{1.268557in}{2.722076in}}%
\pgfpathlineto{\pgfqpoint{1.287270in}{2.671290in}}%
\pgfpathlineto{\pgfqpoint{1.305785in}{2.620482in}}%
\pgfpathlineto{\pgfqpoint{1.324031in}{2.569646in}}%
\pgfpathlineto{\pgfqpoint{1.341926in}{2.518772in}}%
\pgfpathlineto{\pgfqpoint{1.359355in}{2.467850in}}%
\pgfpathlineto{\pgfqpoint{1.376169in}{2.416868in}}%
\pgfpathlineto{\pgfqpoint{1.392175in}{2.365808in}}%
\pgfpathlineto{\pgfqpoint{1.407104in}{2.314653in}}%
\pgfpathlineto{\pgfqpoint{1.420601in}{2.263380in}}%
\pgfpathlineto{\pgfqpoint{1.432171in}{2.211968in}}%
\pgfpathlineto{\pgfqpoint{1.441131in}{2.160404in}}%
\pgfpathlineto{\pgfqpoint{1.446594in}{2.108700in}}%
\pgfpathlineto{\pgfqpoint{1.447485in}{2.056922in}}%
\pgfpathlineto{\pgfqpoint{1.442735in}{2.005218in}}%
\pgfpathlineto{\pgfqpoint{1.431679in}{1.953810in}}%
\pgfpathlineto{\pgfqpoint{1.414486in}{1.902906in}}%
\pgfpathlineto{\pgfqpoint{1.392184in}{1.852604in}}%
\pgfpathlineto{\pgfqpoint{1.366133in}{1.802841in}}%
\pgfpathlineto{\pgfqpoint{1.337578in}{1.753467in}}%
\pgfusepath{stroke}%
\end{pgfscope}%
\begin{pgfscope}%
\pgfpathrectangle{\pgfqpoint{0.647939in}{0.492442in}}{\pgfqpoint{4.273799in}{2.331163in}}%
\pgfusepath{clip}%
\pgfsetbuttcap%
\pgfsetroundjoin%
\pgfsetlinewidth{0.301125pt}%
\definecolor{currentstroke}{rgb}{0.500000,0.500000,0.500000}%
\pgfsetstrokecolor{currentstroke}%
\pgfsetstrokeopacity{0.300000}%
\pgfsetdash{}{0pt}%
\pgfpathmoveto{\pgfqpoint{1.133598in}{2.823605in}}%
\pgfpathlineto{\pgfqpoint{1.133598in}{2.823605in}}%
\pgfpathlineto{\pgfqpoint{1.149259in}{2.772512in}}%
\pgfpathlineto{\pgfqpoint{1.164616in}{2.721391in}}%
\pgfpathlineto{\pgfqpoint{1.179610in}{2.670238in}}%
\pgfpathlineto{\pgfqpoint{1.194172in}{2.619048in}}%
\pgfpathlineto{\pgfqpoint{1.208230in}{2.567816in}}%
\pgfpathlineto{\pgfqpoint{1.221688in}{2.516536in}}%
\pgfpathlineto{\pgfqpoint{1.234434in}{2.465203in}}%
\pgfpathlineto{\pgfqpoint{1.246343in}{2.413810in}}%
\pgfpathlineto{\pgfqpoint{1.257264in}{2.362351in}}%
\pgfpathlineto{\pgfqpoint{1.267015in}{2.310824in}}%
\pgfpathlineto{\pgfqpoint{1.275385in}{2.259224in}}%
\pgfpathlineto{\pgfqpoint{1.282141in}{2.207555in}}%
\pgfpathlineto{\pgfqpoint{1.287015in}{2.155823in}}%
\pgfpathlineto{\pgfqpoint{1.289722in}{2.104045in}}%
\pgfpathlineto{\pgfqpoint{1.289973in}{2.052247in}}%
\pgfpathlineto{\pgfqpoint{1.287502in}{2.000468in}}%
\pgfpathlineto{\pgfqpoint{1.282105in}{1.948755in}}%
\pgfpathlineto{\pgfqpoint{1.273676in}{1.897166in}}%
\pgfpathlineto{\pgfqpoint{1.262231in}{1.845752in}}%
\pgfpathlineto{\pgfqpoint{1.247912in}{1.794550in}}%
\pgfpathlineto{\pgfqpoint{1.230998in}{1.743586in}}%
\pgfpathlineto{\pgfqpoint{1.211824in}{1.692860in}}%
\pgfpathlineto{\pgfqpoint{1.190767in}{1.642355in}}%
\pgfpathlineto{\pgfqpoint{1.168196in}{1.592047in}}%
\pgfpathlineto{\pgfqpoint{1.144451in}{1.541899in}}%
\pgfpathlineto{\pgfqpoint{1.119821in}{1.491878in}}%
\pgfpathlineto{\pgfqpoint{1.094552in}{1.441951in}}%
\pgfpathlineto{\pgfqpoint{1.068850in}{1.392092in}}%
\pgfpathlineto{\pgfqpoint{1.042859in}{1.342275in}}%
\pgfpathlineto{\pgfqpoint{1.016715in}{1.292482in}}%
\pgfpathlineto{\pgfqpoint{0.990507in}{1.242697in}}%
\pgfpathlineto{\pgfqpoint{0.964314in}{1.192909in}}%
\pgfpathlineto{\pgfqpoint{0.938202in}{1.143111in}}%
\pgfpathlineto{\pgfqpoint{0.912202in}{1.093293in}}%
\pgfpathlineto{\pgfqpoint{0.886356in}{1.043451in}}%
\pgfpathlineto{\pgfqpoint{0.860684in}{0.993581in}}%
\pgfpathlineto{\pgfqpoint{0.835207in}{0.943681in}}%
\pgfpathlineto{\pgfqpoint{0.809942in}{0.893750in}}%
\pgfpathlineto{\pgfqpoint{0.784892in}{0.843786in}}%
\pgfpathlineto{\pgfqpoint{0.760068in}{0.793788in}}%
\pgfusepath{stroke}%
\end{pgfscope}%
\begin{pgfscope}%
\pgfpathrectangle{\pgfqpoint{0.647939in}{0.492442in}}{\pgfqpoint{4.273799in}{2.331163in}}%
\pgfusepath{clip}%
\pgfsetbuttcap%
\pgfsetroundjoin%
\pgfsetlinewidth{0.301125pt}%
\definecolor{currentstroke}{rgb}{0.500000,0.500000,0.500000}%
\pgfsetstrokecolor{currentstroke}%
\pgfsetstrokeopacity{0.300000}%
\pgfsetdash{}{0pt}%
\pgfpathmoveto{\pgfqpoint{1.036466in}{2.823605in}}%
\pgfpathlineto{\pgfqpoint{1.036466in}{2.823605in}}%
\pgfpathlineto{\pgfqpoint{1.049533in}{2.772295in}}%
\pgfpathlineto{\pgfqpoint{1.062204in}{2.720955in}}%
\pgfpathlineto{\pgfqpoint{1.074430in}{2.669583in}}%
\pgfpathlineto{\pgfqpoint{1.086156in}{2.618177in}}%
\pgfpathlineto{\pgfqpoint{1.097312in}{2.566733in}}%
\pgfpathlineto{\pgfqpoint{1.107828in}{2.515248in}}%
\pgfpathlineto{\pgfqpoint{1.117624in}{2.463722in}}%
\pgfpathlineto{\pgfqpoint{1.126611in}{2.412152in}}%
\pgfpathlineto{\pgfqpoint{1.134686in}{2.360537in}}%
\pgfpathlineto{\pgfqpoint{1.141741in}{2.308878in}}%
\pgfpathlineto{\pgfqpoint{1.147658in}{2.257176in}}%
\pgfpathlineto{\pgfqpoint{1.152315in}{2.205436in}}%
\pgfpathlineto{\pgfqpoint{1.155581in}{2.153665in}}%
\pgfpathlineto{\pgfqpoint{1.157327in}{2.101872in}}%
\pgfpathlineto{\pgfqpoint{1.157428in}{2.050071in}}%
\pgfpathlineto{\pgfqpoint{1.155771in}{1.998278in}}%
\pgfpathlineto{\pgfqpoint{1.152262in}{1.946513in}}%
\pgfpathlineto{\pgfqpoint{1.146838in}{1.894797in}}%
\pgfpathlineto{\pgfqpoint{1.139471in}{1.843154in}}%
\pgfpathlineto{\pgfqpoint{1.130169in}{1.791604in}}%
\pgfpathlineto{\pgfqpoint{1.118985in}{1.740164in}}%
\pgfpathlineto{\pgfqpoint{1.106020in}{1.688850in}}%
\pgfpathlineto{\pgfqpoint{1.091401in}{1.637669in}}%
\pgfpathlineto{\pgfqpoint{1.075270in}{1.586623in}}%
\pgfpathlineto{\pgfqpoint{1.057802in}{1.535708in}}%
\pgfpathlineto{\pgfqpoint{1.039164in}{1.484917in}}%
\pgfpathlineto{\pgfqpoint{1.019525in}{1.434237in}}%
\pgfusepath{stroke}%
\end{pgfscope}%
\begin{pgfscope}%
\pgfpathrectangle{\pgfqpoint{0.647939in}{0.492442in}}{\pgfqpoint{4.273799in}{2.331163in}}%
\pgfusepath{clip}%
\pgfsetbuttcap%
\pgfsetroundjoin%
\pgfsetlinewidth{0.301125pt}%
\definecolor{currentstroke}{rgb}{0.500000,0.500000,0.500000}%
\pgfsetstrokecolor{currentstroke}%
\pgfsetstrokeopacity{0.300000}%
\pgfsetdash{}{0pt}%
\pgfpathmoveto{\pgfqpoint{0.939334in}{2.823605in}}%
\pgfpathlineto{\pgfqpoint{0.939334in}{2.823605in}}%
\pgfpathlineto{\pgfqpoint{0.950340in}{2.772151in}}%
\pgfpathlineto{\pgfqpoint{0.960927in}{2.720671in}}%
\pgfpathlineto{\pgfqpoint{0.971051in}{2.669163in}}%
\pgfpathlineto{\pgfqpoint{0.980668in}{2.617626in}}%
\pgfpathlineto{\pgfqpoint{0.989731in}{2.566059in}}%
\pgfpathlineto{\pgfqpoint{0.998186in}{2.514462in}}%
\pgfpathlineto{\pgfqpoint{1.005974in}{2.462833in}}%
\pgfpathlineto{\pgfqpoint{1.013034in}{2.411174in}}%
\pgfpathlineto{\pgfqpoint{1.019301in}{2.359484in}}%
\pgfpathlineto{\pgfqpoint{1.024709in}{2.307765in}}%
\pgfpathlineto{\pgfqpoint{1.029188in}{2.256020in}}%
\pgfpathlineto{\pgfqpoint{1.032664in}{2.204252in}}%
\pgfpathlineto{\pgfqpoint{1.035067in}{2.152466in}}%
\pgfpathlineto{\pgfqpoint{1.036326in}{2.100668in}}%
\pgfpathlineto{\pgfqpoint{1.036373in}{2.048866in}}%
\pgfpathlineto{\pgfqpoint{1.035150in}{1.997068in}}%
\pgfpathlineto{\pgfqpoint{1.032606in}{1.945284in}}%
\pgfpathlineto{\pgfqpoint{1.028704in}{1.893526in}}%
\pgfpathlineto{\pgfqpoint{1.023418in}{1.841805in}}%
\pgfpathlineto{\pgfqpoint{1.016742in}{1.790131in}}%
\pgfpathlineto{\pgfqpoint{1.008685in}{1.738516in}}%
\pgfpathlineto{\pgfqpoint{0.999271in}{1.686970in}}%
\pgfpathlineto{\pgfqpoint{0.988546in}{1.635500in}}%
\pgfpathlineto{\pgfqpoint{0.976574in}{1.584112in}}%
\pgfpathlineto{\pgfqpoint{0.963429in}{1.532810in}}%
\pgfpathlineto{\pgfqpoint{0.949186in}{1.481594in}}%
\pgfpathlineto{\pgfqpoint{0.933943in}{1.430464in}}%
\pgfpathlineto{\pgfqpoint{0.917797in}{1.379418in}}%
\pgfpathlineto{\pgfqpoint{0.900835in}{1.328450in}}%
\pgfpathlineto{\pgfqpoint{0.883156in}{1.277555in}}%
\pgfpathlineto{\pgfqpoint{0.864844in}{1.226726in}}%
\pgfpathlineto{\pgfqpoint{0.845986in}{1.175956in}}%
\pgfpathlineto{\pgfqpoint{0.826664in}{1.125238in}}%
\pgfpathlineto{\pgfqpoint{0.806946in}{1.074566in}}%
\pgfpathlineto{\pgfqpoint{0.786900in}{1.023933in}}%
\pgfpathlineto{\pgfqpoint{0.766581in}{0.973332in}}%
\pgfpathlineto{\pgfqpoint{0.746044in}{0.922757in}}%
\pgfusepath{stroke}%
\end{pgfscope}%
\begin{pgfscope}%
\pgfpathrectangle{\pgfqpoint{0.647939in}{0.492442in}}{\pgfqpoint{4.273799in}{2.331163in}}%
\pgfusepath{clip}%
\pgfsetbuttcap%
\pgfsetroundjoin%
\pgfsetlinewidth{0.301125pt}%
\definecolor{currentstroke}{rgb}{0.500000,0.500000,0.500000}%
\pgfsetstrokecolor{currentstroke}%
\pgfsetstrokeopacity{0.300000}%
\pgfsetdash{}{0pt}%
\pgfpathmoveto{\pgfqpoint{0.842203in}{2.823605in}}%
\pgfpathlineto{\pgfqpoint{0.842203in}{2.823605in}}%
\pgfpathlineto{\pgfqpoint{0.851555in}{2.772054in}}%
\pgfpathlineto{\pgfqpoint{0.860495in}{2.720481in}}%
\pgfpathlineto{\pgfqpoint{0.868990in}{2.668885in}}%
\pgfpathlineto{\pgfqpoint{0.877004in}{2.617266in}}%
\pgfpathlineto{\pgfqpoint{0.884500in}{2.565625in}}%
\pgfpathlineto{\pgfqpoint{0.891441in}{2.513960in}}%
\pgfpathlineto{\pgfqpoint{0.897786in}{2.462273in}}%
\pgfpathlineto{\pgfqpoint{0.903496in}{2.410563in}}%
\pgfpathlineto{\pgfqpoint{0.908525in}{2.358833in}}%
\pgfpathlineto{\pgfqpoint{0.912829in}{2.307083in}}%
\pgfpathlineto{\pgfqpoint{0.916363in}{2.255316in}}%
\pgfpathlineto{\pgfqpoint{0.919083in}{2.203534in}}%
\pgfpathlineto{\pgfqpoint{0.920946in}{2.151741in}}%
\pgfpathlineto{\pgfqpoint{0.921909in}{2.099941in}}%
\pgfpathlineto{\pgfqpoint{0.921933in}{2.048138in}}%
\pgfpathlineto{\pgfqpoint{0.920984in}{1.996338in}}%
\pgfpathlineto{\pgfqpoint{0.919031in}{1.944546in}}%
\pgfpathlineto{\pgfqpoint{0.916048in}{1.892769in}}%
\pgfpathlineto{\pgfqpoint{0.912018in}{1.841013in}}%
\pgfpathlineto{\pgfqpoint{0.906931in}{1.789284in}}%
\pgfpathlineto{\pgfqpoint{0.900785in}{1.737590in}}%
\pgfpathlineto{\pgfqpoint{0.893591in}{1.685937in}}%
\pgfpathlineto{\pgfqpoint{0.885365in}{1.634329in}}%
\pgfpathlineto{\pgfqpoint{0.876131in}{1.582773in}}%
\pgfpathlineto{\pgfqpoint{0.865920in}{1.531271in}}%
\pgfpathlineto{\pgfqpoint{0.854774in}{1.479826in}}%
\pgfpathlineto{\pgfqpoint{0.842743in}{1.428441in}}%
\pgfpathlineto{\pgfqpoint{0.829877in}{1.377117in}}%
\pgfpathlineto{\pgfqpoint{0.816228in}{1.325853in}}%
\pgfusepath{stroke}%
\end{pgfscope}%
\begin{pgfscope}%
\pgfpathrectangle{\pgfqpoint{0.647939in}{0.492442in}}{\pgfqpoint{4.273799in}{2.331163in}}%
\pgfusepath{clip}%
\pgfsetbuttcap%
\pgfsetroundjoin%
\pgfsetlinewidth{0.301125pt}%
\definecolor{currentstroke}{rgb}{0.500000,0.500000,0.500000}%
\pgfsetstrokecolor{currentstroke}%
\pgfsetstrokeopacity{0.300000}%
\pgfsetdash{}{0pt}%
\pgfpathmoveto{\pgfqpoint{0.745071in}{2.823605in}}%
\pgfpathlineto{\pgfqpoint{0.745071in}{2.823605in}}%
\pgfpathlineto{\pgfqpoint{0.753086in}{2.771987in}}%
\pgfpathlineto{\pgfqpoint{0.760708in}{2.720351in}}%
\pgfpathlineto{\pgfqpoint{0.767915in}{2.668697in}}%
\pgfpathlineto{\pgfqpoint{0.774679in}{2.617025in}}%
\pgfpathlineto{\pgfqpoint{0.780975in}{2.565335in}}%
\pgfpathlineto{\pgfqpoint{0.786772in}{2.513629in}}%
\pgfpathlineto{\pgfqpoint{0.792042in}{2.461905in}}%
\pgfpathlineto{\pgfqpoint{0.796756in}{2.410166in}}%
\pgfpathlineto{\pgfqpoint{0.800885in}{2.358412in}}%
\pgfpathlineto{\pgfqpoint{0.804400in}{2.306644in}}%
\pgfpathlineto{\pgfqpoint{0.807270in}{2.254864in}}%
\pgfpathlineto{\pgfqpoint{0.809466in}{2.203075in}}%
\pgfpathlineto{\pgfqpoint{0.810962in}{2.151278in}}%
\pgfpathlineto{\pgfqpoint{0.811729in}{2.099477in}}%
\pgfpathlineto{\pgfqpoint{0.811742in}{2.047673in}}%
\pgfpathlineto{\pgfqpoint{0.810980in}{1.995872in}}%
\pgfpathlineto{\pgfqpoint{0.809422in}{1.944076in}}%
\pgfpathlineto{\pgfqpoint{0.807052in}{1.892289in}}%
\pgfpathlineto{\pgfqpoint{0.803858in}{1.840515in}}%
\pgfpathlineto{\pgfqpoint{0.799830in}{1.788759in}}%
\pgfpathlineto{\pgfqpoint{0.794966in}{1.737024in}}%
\pgfpathlineto{\pgfqpoint{0.789266in}{1.685314in}}%
\pgfpathlineto{\pgfqpoint{0.782736in}{1.633634in}}%
\pgfpathlineto{\pgfqpoint{0.775385in}{1.581987in}}%
\pgfpathlineto{\pgfqpoint{0.767232in}{1.530375in}}%
\pgfpathlineto{\pgfqpoint{0.758298in}{1.478802in}}%
\pgfpathlineto{\pgfqpoint{0.748606in}{1.427270in}}%
\pgfpathlineto{\pgfqpoint{0.738183in}{1.375780in}}%
\pgfpathlineto{\pgfqpoint{0.727062in}{1.324334in}}%
\pgfpathlineto{\pgfqpoint{0.715278in}{1.272931in}}%
\pgfpathlineto{\pgfqpoint{0.702869in}{1.221573in}}%
\pgfpathlineto{\pgfqpoint{0.689867in}{1.170258in}}%
\pgfpathlineto{\pgfqpoint{0.676315in}{1.118985in}}%
\pgfpathlineto{\pgfqpoint{0.662254in}{1.067754in}}%
\pgfusepath{stroke}%
\end{pgfscope}%
\begin{pgfscope}%
\pgfpathrectangle{\pgfqpoint{0.647939in}{0.492442in}}{\pgfqpoint{4.273799in}{2.331163in}}%
\pgfusepath{clip}%
\pgfsetbuttcap%
\pgfsetroundjoin%
\pgfsetlinewidth{0.301125pt}%
\definecolor{currentstroke}{rgb}{0.500000,0.500000,0.500000}%
\pgfsetstrokecolor{currentstroke}%
\pgfsetstrokeopacity{0.300000}%
\pgfsetdash{}{0pt}%
\pgfpathmoveto{\pgfqpoint{0.647939in}{2.823605in}}%
\pgfpathlineto{\pgfqpoint{0.647939in}{2.823605in}}%
\pgfpathlineto{\pgfqpoint{0.654862in}{2.771940in}}%
\pgfpathlineto{\pgfqpoint{0.661420in}{2.720260in}}%
\pgfpathlineto{\pgfqpoint{0.667595in}{2.668566in}}%
\pgfpathlineto{\pgfqpoint{0.673366in}{2.616859in}}%
\pgfpathlineto{\pgfqpoint{0.678716in}{2.565137in}}%
\pgfpathlineto{\pgfqpoint{0.683623in}{2.513403in}}%
\pgfpathlineto{\pgfqpoint{0.688066in}{2.461656in}}%
\pgfpathlineto{\pgfqpoint{0.692026in}{2.409898in}}%
\pgfpathlineto{\pgfqpoint{0.695480in}{2.358129in}}%
\pgfpathlineto{\pgfqpoint{0.698407in}{2.306350in}}%
\pgfpathlineto{\pgfqpoint{0.700789in}{2.254563in}}%
\pgfpathlineto{\pgfqpoint{0.702604in}{2.202769in}}%
\pgfpathlineto{\pgfqpoint{0.703835in}{2.150970in}}%
\pgfpathlineto{\pgfqpoint{0.704462in}{2.099168in}}%
\pgfpathlineto{\pgfqpoint{0.704470in}{2.047365in}}%
\pgfpathlineto{\pgfqpoint{0.703843in}{1.995563in}}%
\pgfpathlineto{\pgfqpoint{0.702568in}{1.943764in}}%
\pgfpathlineto{\pgfqpoint{0.700633in}{1.891971in}}%
\pgfpathlineto{\pgfqpoint{0.698029in}{1.840188in}}%
\pgfpathlineto{\pgfqpoint{0.694750in}{1.788415in}}%
\pgfpathlineto{\pgfqpoint{0.690790in}{1.736657in}}%
\pgfpathlineto{\pgfqpoint{0.686150in}{1.684916in}}%
\pgfpathlineto{\pgfqpoint{0.680832in}{1.633194in}}%
\pgfpathlineto{\pgfqpoint{0.674840in}{1.581494in}}%
\pgfpathlineto{\pgfqpoint{0.668182in}{1.529818in}}%
\pgfpathlineto{\pgfqpoint{0.660869in}{1.478169in}}%
\pgfpathlineto{\pgfqpoint{0.652912in}{1.426548in}}%
\pgfpathlineto{\pgfqpoint{0.647939in}{1.395527in}}%
\pgfusepath{stroke}%
\end{pgfscope}%
\begin{pgfscope}%
\pgfpathrectangle{\pgfqpoint{0.647939in}{0.492442in}}{\pgfqpoint{4.273799in}{2.331163in}}%
\pgfusepath{clip}%
\pgfsetbuttcap%
\pgfsetroundjoin%
\pgfsetlinewidth{0.301125pt}%
\definecolor{currentstroke}{rgb}{0.500000,0.500000,0.500000}%
\pgfsetstrokecolor{currentstroke}%
\pgfsetstrokeopacity{0.300000}%
\pgfsetdash{}{0pt}%
\pgfpathmoveto{\pgfqpoint{0.647939in}{2.293796in}}%
\pgfpathlineto{\pgfqpoint{0.647939in}{2.293796in}}%
\pgfpathlineto{\pgfqpoint{0.650003in}{2.242004in}}%
\pgfpathlineto{\pgfqpoint{0.651540in}{2.190208in}}%
\pgfpathlineto{\pgfqpoint{0.652535in}{2.138407in}}%
\pgfpathlineto{\pgfqpoint{0.652973in}{2.086604in}}%
\pgfpathlineto{\pgfqpoint{0.652841in}{2.034801in}}%
\pgfpathlineto{\pgfqpoint{0.652126in}{1.982999in}}%
\pgfpathlineto{\pgfqpoint{0.650817in}{1.931201in}}%
\pgfpathlineto{\pgfqpoint{0.648905in}{1.879408in}}%
\pgfpathlineto{\pgfqpoint{0.647939in}{1.856818in}}%
\pgfusepath{stroke}%
\end{pgfscope}%
\begin{pgfscope}%
\pgfpathrectangle{\pgfqpoint{0.647939in}{0.492442in}}{\pgfqpoint{4.273799in}{2.331163in}}%
\pgfusepath{clip}%
\pgfsetbuttcap%
\pgfsetroundjoin%
\pgfsetlinewidth{0.301125pt}%
\definecolor{currentstroke}{rgb}{0.500000,0.500000,0.500000}%
\pgfsetstrokecolor{currentstroke}%
\pgfsetstrokeopacity{0.300000}%
\pgfsetdash{}{0pt}%
\pgfpathmoveto{\pgfqpoint{2.104916in}{0.598404in}}%
\pgfpathlineto{\pgfqpoint{2.068374in}{0.646220in}}%
\pgfpathlineto{\pgfqpoint{2.031853in}{0.694040in}}%
\pgfpathlineto{\pgfqpoint{1.995277in}{0.741848in}}%
\pgfpathlineto{\pgfqpoint{1.958558in}{0.789623in}}%
\pgfpathlineto{\pgfqpoint{1.921599in}{0.837342in}}%
\pgfpathlineto{\pgfqpoint{1.884277in}{0.884978in}}%
\pgfpathlineto{\pgfqpoint{1.846442in}{0.932492in}}%
\pgfpathlineto{\pgfqpoint{1.807897in}{0.979836in}}%
\pgfpathlineto{\pgfqpoint{1.768378in}{1.026940in}}%
\pgfpathlineto{\pgfqpoint{1.727520in}{1.073702in}}%
\pgfpathlineto{\pgfqpoint{1.684795in}{1.119962in}}%
\pgfpathlineto{\pgfqpoint{1.639403in}{1.165456in}}%
\pgfpathlineto{\pgfqpoint{1.590026in}{1.209675in}}%
\pgfpathlineto{\pgfqpoint{1.534316in}{1.251509in}}%
\pgfpathlineto{\pgfqpoint{1.467756in}{1.288023in}}%
\pgfpathlineto{\pgfqpoint{1.467756in}{1.288023in}}%
\pgfpathlineto{\pgfqpoint{1.415219in}{1.304993in}}%
\pgfpathlineto{\pgfqpoint{1.415219in}{1.304993in}}%
\pgfpathlineto{\pgfqpoint{1.366594in}{1.310364in}}%
\pgfpathlineto{\pgfqpoint{1.317644in}{1.305388in}}%
\pgfpathlineto{\pgfqpoint{1.275333in}{1.292799in}}%
\pgfpathlineto{\pgfqpoint{1.232688in}{1.272536in}}%
\pgfpathlineto{\pgfqpoint{1.186537in}{1.242487in}}%
\pgfpathlineto{\pgfqpoint{1.134875in}{1.199617in}}%
\pgfusepath{stroke}%
\end{pgfscope}%
\begin{pgfscope}%
\pgfpathrectangle{\pgfqpoint{0.647939in}{0.492442in}}{\pgfqpoint{4.273799in}{2.331163in}}%
\pgfusepath{clip}%
\pgfsetbuttcap%
\pgfsetroundjoin%
\pgfsetlinewidth{0.301125pt}%
\definecolor{currentstroke}{rgb}{0.500000,0.500000,0.500000}%
\pgfsetstrokecolor{currentstroke}%
\pgfsetstrokeopacity{0.300000}%
\pgfsetdash{}{0pt}%
\pgfpathmoveto{\pgfqpoint{3.978525in}{0.626491in}}%
\pgfpathlineto{\pgfqpoint{3.916780in}{0.665845in}}%
\pgfpathlineto{\pgfqpoint{3.853289in}{0.704366in}}%
\pgfpathlineto{\pgfqpoint{3.788401in}{0.742191in}}%
\pgfpathlineto{\pgfqpoint{3.722545in}{0.779516in}}%
\pgfpathlineto{\pgfqpoint{3.656189in}{0.816577in}}%
\pgfpathlineto{\pgfqpoint{3.589809in}{0.853625in}}%
\pgfusepath{stroke}%
\end{pgfscope}%
\begin{pgfscope}%
\pgfpathrectangle{\pgfqpoint{0.647939in}{0.492442in}}{\pgfqpoint{4.273799in}{2.331163in}}%
\pgfusepath{clip}%
\pgfsetbuttcap%
\pgfsetroundjoin%
\pgfsetlinewidth{0.301125pt}%
\definecolor{currentstroke}{rgb}{0.500000,0.500000,0.500000}%
\pgfsetstrokecolor{currentstroke}%
\pgfsetstrokeopacity{0.300000}%
\pgfsetdash{}{0pt}%
\pgfpathmoveto{\pgfqpoint{4.561847in}{1.343730in}}%
\pgfpathlineto{\pgfqpoint{4.533211in}{1.393119in}}%
\pgfpathlineto{\pgfqpoint{4.503327in}{1.442285in}}%
\pgfpathlineto{\pgfqpoint{4.471882in}{1.491155in}}%
\pgfpathlineto{\pgfqpoint{4.438355in}{1.539607in}}%
\pgfpathlineto{\pgfqpoint{4.401916in}{1.587430in}}%
\pgfpathlineto{\pgfqpoint{4.361159in}{1.634189in}}%
\pgfpathlineto{\pgfqpoint{4.313251in}{1.678836in}}%
\pgfpathlineto{\pgfqpoint{4.252290in}{1.717943in}}%
\pgfpathlineto{\pgfqpoint{4.252290in}{1.717943in}}%
\pgfpathlineto{\pgfqpoint{4.205849in}{1.734286in}}%
\pgfpathlineto{\pgfqpoint{4.205849in}{1.734286in}}%
\pgfpathlineto{\pgfqpoint{4.158754in}{1.740033in}}%
\pgfpathlineto{\pgfqpoint{4.110242in}{1.736801in}}%
\pgfpathlineto{\pgfqpoint{4.060872in}{1.726680in}}%
\pgfpathlineto{\pgfqpoint{4.000542in}{1.708695in}}%
\pgfpathlineto{\pgfqpoint{3.920620in}{1.680948in}}%
\pgfpathlineto{\pgfqpoint{3.840556in}{1.653222in}}%
\pgfusepath{stroke}%
\end{pgfscope}%
\begin{pgfscope}%
\pgfpathrectangle{\pgfqpoint{0.647939in}{0.492442in}}{\pgfqpoint{4.273799in}{2.331163in}}%
\pgfusepath{clip}%
\pgfsetbuttcap%
\pgfsetroundjoin%
\pgfsetlinewidth{0.301125pt}%
\definecolor{currentstroke}{rgb}{0.500000,0.500000,0.500000}%
\pgfsetstrokecolor{currentstroke}%
\pgfsetstrokeopacity{0.300000}%
\pgfsetdash{}{0pt}%
\pgfpathmoveto{\pgfqpoint{4.533211in}{1.711005in}}%
\pgfpathlineto{\pgfqpoint{4.511227in}{1.761396in}}%
\pgfpathlineto{\pgfqpoint{4.489299in}{1.811794in}}%
\pgfpathlineto{\pgfqpoint{4.467560in}{1.862212in}}%
\pgfpathlineto{\pgfqpoint{4.446257in}{1.912684in}}%
\pgfpathlineto{\pgfqpoint{4.425947in}{1.963264in}}%
\pgfpathlineto{\pgfqpoint{4.408124in}{2.014105in}}%
\pgfpathlineto{\pgfqpoint{4.398380in}{2.065478in}}%
\pgfpathlineto{\pgfqpoint{4.398380in}{2.065478in}}%
\pgfpathlineto{\pgfqpoint{4.403343in}{2.099122in}}%
\pgfpathlineto{\pgfqpoint{4.422949in}{2.135834in}}%
\pgfpathlineto{\pgfqpoint{4.448891in}{2.172183in}}%
\pgfpathlineto{\pgfqpoint{4.486141in}{2.219541in}}%
\pgfpathlineto{\pgfqpoint{4.523790in}{2.266819in}}%
\pgfpathlineto{\pgfqpoint{4.560986in}{2.314248in}}%
\pgfusepath{stroke}%
\end{pgfscope}%
\begin{pgfscope}%
\pgfpathrectangle{\pgfqpoint{0.647939in}{0.492442in}}{\pgfqpoint{4.273799in}{2.331163in}}%
\pgfusepath{clip}%
\pgfsetbuttcap%
\pgfsetroundjoin%
\pgfsetlinewidth{0.301125pt}%
\definecolor{currentstroke}{rgb}{0.500000,0.500000,0.500000}%
\pgfsetstrokecolor{currentstroke}%
\pgfsetstrokeopacity{0.300000}%
\pgfsetdash{}{0pt}%
\pgfpathmoveto{\pgfqpoint{3.464761in}{0.757347in}}%
\pgfpathlineto{\pgfqpoint{3.403702in}{0.797022in}}%
\pgfpathlineto{\pgfqpoint{3.343871in}{0.837248in}}%
\pgfpathlineto{\pgfqpoint{3.285458in}{0.878089in}}%
\pgfpathlineto{\pgfqpoint{3.228614in}{0.919583in}}%
\pgfpathlineto{\pgfqpoint{3.173449in}{0.961745in}}%
\pgfpathlineto{\pgfqpoint{3.120031in}{1.004570in}}%
\pgfpathlineto{\pgfqpoint{3.068397in}{1.048043in}}%
\pgfusepath{stroke}%
\end{pgfscope}%
\begin{pgfscope}%
\pgfpathrectangle{\pgfqpoint{0.647939in}{0.492442in}}{\pgfqpoint{4.273799in}{2.331163in}}%
\pgfusepath{clip}%
\pgfsetbuttcap%
\pgfsetroundjoin%
\pgfsetlinewidth{0.301125pt}%
\definecolor{currentstroke}{rgb}{0.500000,0.500000,0.500000}%
\pgfsetstrokecolor{currentstroke}%
\pgfsetstrokeopacity{0.300000}%
\pgfsetdash{}{0pt}%
\pgfpathmoveto{\pgfqpoint{3.659025in}{0.757347in}}%
\pgfpathlineto{\pgfqpoint{3.593806in}{0.795004in}}%
\pgfpathlineto{\pgfqpoint{3.528980in}{0.832861in}}%
\pgfpathlineto{\pgfqpoint{3.464937in}{0.871111in}}%
\pgfpathlineto{\pgfqpoint{3.402021in}{0.909912in}}%
\pgfpathlineto{\pgfqpoint{3.340522in}{0.949380in}}%
\pgfpathlineto{\pgfqpoint{3.280662in}{0.989591in}}%
\pgfpathlineto{\pgfqpoint{3.222613in}{1.030584in}}%
\pgfpathlineto{\pgfqpoint{3.166499in}{1.072368in}}%
\pgfpathlineto{\pgfqpoint{3.112381in}{1.114931in}}%
\pgfpathlineto{\pgfqpoint{3.060299in}{1.158243in}}%
\pgfpathlineto{\pgfqpoint{3.010264in}{1.202266in}}%
\pgfpathlineto{\pgfqpoint{2.962269in}{1.246960in}}%
\pgfpathlineto{\pgfqpoint{2.916293in}{1.292280in}}%
\pgfpathlineto{\pgfqpoint{2.872314in}{1.338187in}}%
\pgfpathlineto{\pgfqpoint{2.830309in}{1.384641in}}%
\pgfpathlineto{\pgfqpoint{2.790262in}{1.431608in}}%
\pgfusepath{stroke}%
\end{pgfscope}%
\begin{pgfscope}%
\pgfpathrectangle{\pgfqpoint{0.647939in}{0.492442in}}{\pgfqpoint{4.273799in}{2.331163in}}%
\pgfusepath{clip}%
\pgfsetbuttcap%
\pgfsetroundjoin%
\pgfsetlinewidth{0.301125pt}%
\definecolor{currentstroke}{rgb}{0.500000,0.500000,0.500000}%
\pgfsetstrokecolor{currentstroke}%
\pgfsetstrokeopacity{0.300000}%
\pgfsetdash{}{0pt}%
\pgfpathmoveto{\pgfqpoint{4.436079in}{1.234176in}}%
\pgfpathlineto{\pgfqpoint{4.396360in}{1.281216in}}%
\pgfpathlineto{\pgfqpoint{4.353241in}{1.327351in}}%
\pgfpathlineto{\pgfqpoint{4.305652in}{1.372141in}}%
\pgfpathlineto{\pgfqpoint{4.252009in}{1.414800in}}%
\pgfpathlineto{\pgfqpoint{4.189995in}{1.453850in}}%
\pgfpathlineto{\pgfqpoint{4.117008in}{1.486546in}}%
\pgfpathlineto{\pgfqpoint{4.032503in}{1.509052in}}%
\pgfpathlineto{\pgfqpoint{3.952009in}{1.518902in}}%
\pgfpathlineto{\pgfqpoint{3.867244in}{1.521434in}}%
\pgfpathlineto{\pgfqpoint{3.772426in}{1.519788in}}%
\pgfpathlineto{\pgfqpoint{3.677606in}{1.517847in}}%
\pgfusepath{stroke}%
\end{pgfscope}%
\begin{pgfscope}%
\pgfpathrectangle{\pgfqpoint{0.647939in}{0.492442in}}{\pgfqpoint{4.273799in}{2.331163in}}%
\pgfusepath{clip}%
\pgfsetbuttcap%
\pgfsetroundjoin%
\pgfsetlinewidth{0.301125pt}%
\definecolor{currentstroke}{rgb}{0.500000,0.500000,0.500000}%
\pgfsetstrokecolor{currentstroke}%
\pgfsetstrokeopacity{0.300000}%
\pgfsetdash{}{0pt}%
\pgfpathmoveto{\pgfqpoint{1.666713in}{2.602744in}}%
\pgfpathlineto{\pgfqpoint{1.716389in}{2.558700in}}%
\pgfpathlineto{\pgfqpoint{1.775398in}{2.518309in}}%
\pgfpathlineto{\pgfqpoint{1.848312in}{2.485976in}}%
\pgfpathlineto{\pgfqpoint{1.848312in}{2.485976in}}%
\pgfpathlineto{\pgfqpoint{1.909288in}{2.472104in}}%
\pgfpathlineto{\pgfqpoint{1.974985in}{2.467559in}}%
\pgfpathlineto{\pgfqpoint{2.042514in}{2.470431in}}%
\pgfusepath{stroke}%
\end{pgfscope}%
\begin{pgfscope}%
\pgfpathrectangle{\pgfqpoint{0.647939in}{0.492442in}}{\pgfqpoint{4.273799in}{2.331163in}}%
\pgfusepath{clip}%
\pgfsetbuttcap%
\pgfsetroundjoin%
\pgfsetlinewidth{0.301125pt}%
\definecolor{currentstroke}{rgb}{0.500000,0.500000,0.500000}%
\pgfsetstrokecolor{currentstroke}%
\pgfsetstrokeopacity{0.300000}%
\pgfsetdash{}{0pt}%
\pgfpathmoveto{\pgfqpoint{1.674850in}{0.776611in}}%
\pgfpathlineto{\pgfqpoint{1.628879in}{0.821930in}}%
\pgfpathlineto{\pgfqpoint{1.579934in}{0.866300in}}%
\pgfpathlineto{\pgfqpoint{1.526723in}{0.909154in}}%
\pgfpathlineto{\pgfqpoint{1.467193in}{0.949276in}}%
\pgfpathlineto{\pgfqpoint{1.414033in}{0.977099in}}%
\pgfpathlineto{\pgfqpoint{1.364365in}{0.995417in}}%
\pgfpathlineto{\pgfqpoint{1.313429in}{1.005820in}}%
\pgfpathlineto{\pgfqpoint{1.253217in}{1.006507in}}%
\pgfpathlineto{\pgfqpoint{1.196433in}{0.995268in}}%
\pgfpathlineto{\pgfqpoint{1.196433in}{0.995268in}}%
\pgfpathlineto{\pgfqpoint{1.133598in}{0.969271in}}%
\pgfpathlineto{\pgfqpoint{1.133598in}{0.969271in}}%
\pgfpathlineto{\pgfqpoint{1.071821in}{0.930390in}}%
\pgfpathlineto{\pgfqpoint{1.019227in}{0.887414in}}%
\pgfpathlineto{\pgfqpoint{0.972747in}{0.842328in}}%
\pgfpathlineto{\pgfqpoint{0.930533in}{0.795977in}}%
\pgfusepath{stroke}%
\end{pgfscope}%
\begin{pgfscope}%
\pgfpathrectangle{\pgfqpoint{0.647939in}{0.492442in}}{\pgfqpoint{4.273799in}{2.331163in}}%
\pgfusepath{clip}%
\pgfsetbuttcap%
\pgfsetroundjoin%
\pgfsetlinewidth{0.301125pt}%
\definecolor{currentstroke}{rgb}{0.500000,0.500000,0.500000}%
\pgfsetstrokecolor{currentstroke}%
\pgfsetstrokeopacity{0.300000}%
\pgfsetdash{}{0pt}%
\pgfpathmoveto{\pgfqpoint{4.338948in}{1.128214in}}%
\pgfpathlineto{\pgfqpoint{4.290657in}{1.172789in}}%
\pgfpathlineto{\pgfqpoint{4.237768in}{1.215772in}}%
\pgfpathlineto{\pgfqpoint{4.179193in}{1.256484in}}%
\pgfpathlineto{\pgfqpoint{4.113834in}{1.293949in}}%
\pgfpathlineto{\pgfqpoint{4.040845in}{1.326897in}}%
\pgfpathlineto{\pgfqpoint{3.960395in}{1.354156in}}%
\pgfpathlineto{\pgfqpoint{3.874083in}{1.375493in}}%
\pgfpathlineto{\pgfqpoint{3.784284in}{1.392191in}}%
\pgfpathlineto{\pgfqpoint{3.693092in}{1.406614in}}%
\pgfpathlineto{\pgfqpoint{3.602070in}{1.421314in}}%
\pgfpathlineto{\pgfqpoint{3.512523in}{1.438403in}}%
\pgfpathlineto{\pgfqpoint{3.425767in}{1.459265in}}%
\pgfusepath{stroke}%
\end{pgfscope}%
\begin{pgfscope}%
\pgfpathrectangle{\pgfqpoint{0.647939in}{0.492442in}}{\pgfqpoint{4.273799in}{2.331163in}}%
\pgfusepath{clip}%
\pgfsetbuttcap%
\pgfsetroundjoin%
\pgfsetlinewidth{0.301125pt}%
\definecolor{currentstroke}{rgb}{0.500000,0.500000,0.500000}%
\pgfsetstrokecolor{currentstroke}%
\pgfsetstrokeopacity{0.300000}%
\pgfsetdash{}{0pt}%
\pgfpathmoveto{\pgfqpoint{4.298094in}{1.457529in}}%
\pgfpathlineto{\pgfqpoint{4.241816in}{1.499081in}}%
\pgfpathlineto{\pgfqpoint{4.174549in}{1.535268in}}%
\pgfpathlineto{\pgfqpoint{4.174549in}{1.535268in}}%
\pgfpathlineto{\pgfqpoint{4.109196in}{1.557496in}}%
\pgfpathlineto{\pgfqpoint{4.034550in}{1.569844in}}%
\pgfpathlineto{\pgfqpoint{3.962733in}{1.572100in}}%
\pgfpathlineto{\pgfqpoint{3.883617in}{1.567765in}}%
\pgfpathlineto{\pgfqpoint{3.790200in}{1.558757in}}%
\pgfusepath{stroke}%
\end{pgfscope}%
\begin{pgfscope}%
\pgfpathrectangle{\pgfqpoint{0.647939in}{0.492442in}}{\pgfqpoint{4.273799in}{2.331163in}}%
\pgfusepath{clip}%
\pgfsetbuttcap%
\pgfsetroundjoin%
\pgfsetlinewidth{0.301125pt}%
\definecolor{currentstroke}{rgb}{0.500000,0.500000,0.500000}%
\pgfsetstrokecolor{currentstroke}%
\pgfsetstrokeopacity{0.300000}%
\pgfsetdash{}{0pt}%
\pgfpathmoveto{\pgfqpoint{1.619257in}{2.452738in}}%
\pgfpathlineto{\pgfqpoint{1.661203in}{2.406353in}}%
\pgfpathlineto{\pgfqpoint{1.712709in}{2.363159in}}%
\pgfpathlineto{\pgfqpoint{1.712709in}{2.363159in}}%
\pgfpathlineto{\pgfqpoint{1.757413in}{2.339368in}}%
\pgfpathlineto{\pgfqpoint{1.757413in}{2.339368in}}%
\pgfpathlineto{\pgfqpoint{1.799162in}{2.328639in}}%
\pgfpathlineto{\pgfqpoint{1.846706in}{2.328064in}}%
\pgfpathlineto{\pgfqpoint{1.889340in}{2.335165in}}%
\pgfpathlineto{\pgfqpoint{1.938827in}{2.349185in}}%
\pgfusepath{stroke}%
\end{pgfscope}%
\begin{pgfscope}%
\pgfpathrectangle{\pgfqpoint{0.647939in}{0.492442in}}{\pgfqpoint{4.273799in}{2.331163in}}%
\pgfusepath{clip}%
\pgfsetbuttcap%
\pgfsetroundjoin%
\pgfsetlinewidth{0.301125pt}%
\definecolor{currentstroke}{rgb}{0.500000,0.500000,0.500000}%
\pgfsetstrokecolor{currentstroke}%
\pgfsetstrokeopacity{0.300000}%
\pgfsetdash{}{0pt}%
\pgfpathmoveto{\pgfqpoint{2.897942in}{0.684092in}}%
\pgfpathlineto{\pgfqpoint{2.853745in}{0.729943in}}%
\pgfpathlineto{\pgfqpoint{2.810642in}{0.776103in}}%
\pgfpathlineto{\pgfqpoint{2.768615in}{0.822557in}}%
\pgfpathlineto{\pgfqpoint{2.727643in}{0.869290in}}%
\pgfpathlineto{\pgfqpoint{2.687707in}{0.916290in}}%
\pgfusepath{stroke}%
\end{pgfscope}%
\begin{pgfscope}%
\pgfpathrectangle{\pgfqpoint{0.647939in}{0.492442in}}{\pgfqpoint{4.273799in}{2.331163in}}%
\pgfusepath{clip}%
\pgfsetbuttcap%
\pgfsetroundjoin%
\pgfsetlinewidth{0.301125pt}%
\definecolor{currentstroke}{rgb}{0.500000,0.500000,0.500000}%
\pgfsetstrokecolor{currentstroke}%
\pgfsetstrokeopacity{0.300000}%
\pgfsetdash{}{0pt}%
\pgfpathmoveto{\pgfqpoint{4.202970in}{1.034374in}}%
\pgfpathlineto{\pgfqpoint{4.144684in}{1.075233in}}%
\pgfpathlineto{\pgfqpoint{4.081507in}{1.113864in}}%
\pgfpathlineto{\pgfqpoint{4.013182in}{1.149781in}}%
\pgfpathlineto{\pgfqpoint{3.939850in}{1.182630in}}%
\pgfpathlineto{\pgfqpoint{3.862209in}{1.212399in}}%
\pgfpathlineto{\pgfqpoint{3.781377in}{1.239554in}}%
\pgfpathlineto{\pgfqpoint{3.698715in}{1.265046in}}%
\pgfpathlineto{\pgfqpoint{3.615594in}{1.290096in}}%
\pgfpathlineto{\pgfqpoint{3.533274in}{1.315906in}}%
\pgfusepath{stroke}%
\end{pgfscope}%
\begin{pgfscope}%
\pgfpathrectangle{\pgfqpoint{0.647939in}{0.492442in}}{\pgfqpoint{4.273799in}{2.331163in}}%
\pgfusepath{clip}%
\pgfsetbuttcap%
\pgfsetroundjoin%
\pgfsetlinewidth{0.301125pt}%
\definecolor{currentstroke}{rgb}{0.500000,0.500000,0.500000}%
\pgfsetstrokecolor{currentstroke}%
\pgfsetstrokeopacity{0.300000}%
\pgfsetdash{}{0pt}%
\pgfpathmoveto{\pgfqpoint{1.600486in}{1.013017in}}%
\pgfpathlineto{\pgfqpoint{1.548155in}{1.056197in}}%
\pgfpathlineto{\pgfqpoint{1.494303in}{1.093342in}}%
\pgfpathlineto{\pgfqpoint{1.424993in}{1.128214in}}%
\pgfpathlineto{\pgfqpoint{1.424993in}{1.128214in}}%
\pgfpathlineto{\pgfqpoint{1.369452in}{1.144605in}}%
\pgfpathlineto{\pgfqpoint{1.369452in}{1.144605in}}%
\pgfpathlineto{\pgfqpoint{1.317902in}{1.149545in}}%
\pgfpathlineto{\pgfqpoint{1.266240in}{1.144197in}}%
\pgfpathlineto{\pgfqpoint{1.221047in}{1.131074in}}%
\pgfusepath{stroke}%
\end{pgfscope}%
\begin{pgfscope}%
\pgfpathrectangle{\pgfqpoint{0.647939in}{0.492442in}}{\pgfqpoint{4.273799in}{2.331163in}}%
\pgfusepath{clip}%
\pgfsetbuttcap%
\pgfsetroundjoin%
\pgfsetlinewidth{0.301125pt}%
\definecolor{currentstroke}{rgb}{0.500000,0.500000,0.500000}%
\pgfsetstrokecolor{currentstroke}%
\pgfsetstrokeopacity{0.300000}%
\pgfsetdash{}{0pt}%
\pgfpathmoveto{\pgfqpoint{2.396312in}{0.969271in}}%
\pgfpathlineto{\pgfqpoint{2.361772in}{1.017527in}}%
\pgfpathlineto{\pgfqpoint{2.327886in}{1.065920in}}%
\pgfpathlineto{\pgfqpoint{2.294638in}{1.114446in}}%
\pgfpathlineto{\pgfqpoint{2.262011in}{1.163096in}}%
\pgfpathlineto{\pgfqpoint{2.229996in}{1.211867in}}%
\pgfpathlineto{\pgfqpoint{2.198594in}{1.260757in}}%
\pgfpathlineto{\pgfqpoint{2.167798in}{1.309761in}}%
\pgfpathlineto{\pgfqpoint{2.137602in}{1.358876in}}%
\pgfpathlineto{\pgfqpoint{2.108015in}{1.408102in}}%
\pgfpathlineto{\pgfqpoint{2.079053in}{1.457437in}}%
\pgfpathlineto{\pgfqpoint{2.050726in}{1.506882in}}%
\pgfpathlineto{\pgfqpoint{2.023066in}{1.556440in}}%
\pgfpathlineto{\pgfqpoint{1.996119in}{1.606114in}}%
\pgfpathlineto{\pgfqpoint{1.969938in}{1.655909in}}%
\pgfpathlineto{\pgfqpoint{1.944627in}{1.705837in}}%
\pgfpathlineto{\pgfqpoint{1.920307in}{1.755912in}}%
\pgfpathlineto{\pgfqpoint{1.897177in}{1.806153in}}%
\pgfpathlineto{\pgfqpoint{1.875540in}{1.856590in}}%
\pgfpathlineto{\pgfqpoint{1.855876in}{1.907266in}}%
\pgfpathlineto{\pgfqpoint{1.838981in}{1.958233in}}%
\pgfpathlineto{\pgfqpoint{1.826205in}{2.009541in}}%
\pgfpathlineto{\pgfqpoint{1.819937in}{2.061176in}}%
\pgfpathlineto{\pgfqpoint{1.823918in}{2.112804in}}%
\pgfusepath{stroke}%
\end{pgfscope}%
\begin{pgfscope}%
\pgfpathrectangle{\pgfqpoint{0.647939in}{0.492442in}}{\pgfqpoint{4.273799in}{2.331163in}}%
\pgfusepath{clip}%
\pgfsetbuttcap%
\pgfsetroundjoin%
\pgfsetlinewidth{0.301125pt}%
\definecolor{currentstroke}{rgb}{0.500000,0.500000,0.500000}%
\pgfsetstrokecolor{currentstroke}%
\pgfsetstrokeopacity{0.300000}%
\pgfsetdash{}{0pt}%
\pgfpathmoveto{\pgfqpoint{3.561893in}{2.346777in}}%
\pgfpathlineto{\pgfqpoint{3.578189in}{2.295746in}}%
\pgfpathlineto{\pgfqpoint{3.591976in}{2.244498in}}%
\pgfpathlineto{\pgfqpoint{3.602900in}{2.193046in}}%
\pgfpathlineto{\pgfqpoint{3.610507in}{2.141419in}}%
\pgfpathlineto{\pgfqpoint{3.614211in}{2.089668in}}%
\pgfpathlineto{\pgfqpoint{3.613252in}{2.037889in}}%
\pgfpathlineto{\pgfqpoint{3.606635in}{1.986245in}}%
\pgfpathlineto{\pgfqpoint{3.593030in}{1.935028in}}%
\pgfpathlineto{\pgfqpoint{3.570649in}{1.884767in}}%
\pgfpathlineto{\pgfqpoint{3.537043in}{1.836461in}}%
\pgfpathlineto{\pgfqpoint{3.488973in}{1.792098in}}%
\pgfpathlineto{\pgfqpoint{3.488973in}{1.792098in}}%
\pgfpathlineto{\pgfqpoint{3.436475in}{1.761361in}}%
\pgfpathlineto{\pgfqpoint{3.370104in}{1.739307in}}%
\pgfpathlineto{\pgfqpoint{3.305213in}{1.730865in}}%
\pgfpathlineto{\pgfqpoint{3.244340in}{1.732710in}}%
\pgfusepath{stroke}%
\end{pgfscope}%
\begin{pgfscope}%
\pgfpathrectangle{\pgfqpoint{0.647939in}{0.492442in}}{\pgfqpoint{4.273799in}{2.331163in}}%
\pgfusepath{clip}%
\pgfsetbuttcap%
\pgfsetroundjoin%
\pgfsetlinewidth{0.301125pt}%
\definecolor{currentstroke}{rgb}{0.500000,0.500000,0.500000}%
\pgfsetstrokecolor{currentstroke}%
\pgfsetstrokeopacity{0.300000}%
\pgfsetdash{}{0pt}%
\pgfpathmoveto{\pgfqpoint{1.522125in}{2.346777in}}%
\pgfpathlineto{\pgfqpoint{1.546024in}{2.296652in}}%
\pgfpathlineto{\pgfqpoint{1.570049in}{2.246555in}}%
\pgfpathlineto{\pgfqpoint{1.594135in}{2.196470in}}%
\pgfpathlineto{\pgfqpoint{1.617954in}{2.146402in}}%
\pgfpathlineto{\pgfqpoint{1.639745in}{2.096316in}}%
\pgfpathlineto{\pgfqpoint{1.639745in}{2.096316in}}%
\pgfpathlineto{\pgfqpoint{1.644386in}{2.080854in}}%
\pgfpathlineto{\pgfqpoint{1.644386in}{2.080854in}}%
\pgfpathlineto{\pgfqpoint{1.643988in}{2.068418in}}%
\pgfpathlineto{\pgfqpoint{1.633783in}{2.053555in}}%
\pgfpathlineto{\pgfqpoint{1.620103in}{2.036262in}}%
\pgfpathlineto{\pgfqpoint{1.596810in}{2.010900in}}%
\pgfpathlineto{\pgfqpoint{1.555830in}{1.966628in}}%
\pgfpathlineto{\pgfqpoint{1.514188in}{1.920411in}}%
\pgfusepath{stroke}%
\end{pgfscope}%
\begin{pgfscope}%
\pgfpathrectangle{\pgfqpoint{0.647939in}{0.492442in}}{\pgfqpoint{4.273799in}{2.331163in}}%
\pgfusepath{clip}%
\pgfsetbuttcap%
\pgfsetroundjoin%
\pgfsetlinewidth{0.301125pt}%
\definecolor{currentstroke}{rgb}{0.500000,0.500000,0.500000}%
\pgfsetstrokecolor{currentstroke}%
\pgfsetstrokeopacity{0.300000}%
\pgfsetdash{}{0pt}%
\pgfpathmoveto{\pgfqpoint{3.302085in}{2.444980in}}%
\pgfpathlineto{\pgfqpoint{3.326427in}{2.394912in}}%
\pgfpathlineto{\pgfqpoint{3.348353in}{2.344514in}}%
\pgfpathlineto{\pgfqpoint{3.367630in}{2.293796in}}%
\pgfpathlineto{\pgfqpoint{3.383953in}{2.242772in}}%
\pgfpathlineto{\pgfqpoint{3.396915in}{2.191464in}}%
\pgfpathlineto{\pgfqpoint{3.405959in}{2.139910in}}%
\pgfpathlineto{\pgfqpoint{3.410319in}{2.088182in}}%
\pgfpathlineto{\pgfqpoint{3.408906in}{2.036419in}}%
\pgfpathlineto{\pgfqpoint{3.400090in}{1.984900in}}%
\pgfpathlineto{\pgfqpoint{3.381291in}{1.934240in}}%
\pgfpathlineto{\pgfqpoint{3.348187in}{1.885966in}}%
\pgfpathlineto{\pgfqpoint{3.348187in}{1.885966in}}%
\pgfpathlineto{\pgfqpoint{3.308551in}{1.852763in}}%
\pgfpathlineto{\pgfqpoint{3.308551in}{1.852763in}}%
\pgfpathlineto{\pgfqpoint{3.264487in}{1.831777in}}%
\pgfusepath{stroke}%
\end{pgfscope}%
\begin{pgfscope}%
\pgfpathrectangle{\pgfqpoint{0.647939in}{0.492442in}}{\pgfqpoint{4.273799in}{2.331163in}}%
\pgfusepath{clip}%
\pgfsetbuttcap%
\pgfsetroundjoin%
\pgfsetlinewidth{0.301125pt}%
\definecolor{currentstroke}{rgb}{0.500000,0.500000,0.500000}%
\pgfsetstrokecolor{currentstroke}%
\pgfsetstrokeopacity{0.300000}%
\pgfsetdash{}{0pt}%
\pgfpathmoveto{\pgfqpoint{1.810304in}{1.726628in}}%
\pgfpathlineto{\pgfqpoint{1.780833in}{1.775870in}}%
\pgfpathlineto{\pgfqpoint{1.750857in}{1.825014in}}%
\pgfpathlineto{\pgfqpoint{1.719752in}{1.873935in}}%
\pgfpathlineto{\pgfqpoint{1.685769in}{1.922271in}}%
\pgfpathlineto{\pgfqpoint{1.661142in}{1.950836in}}%
\pgfpathlineto{\pgfqpoint{1.643346in}{1.965961in}}%
\pgfpathlineto{\pgfqpoint{1.619257in}{1.975910in}}%
\pgfpathlineto{\pgfqpoint{1.619257in}{1.975910in}}%
\pgfpathlineto{\pgfqpoint{1.619257in}{1.975910in}}%
\pgfpathlineto{\pgfqpoint{1.619257in}{1.975910in}}%
\pgfpathlineto{\pgfqpoint{1.619257in}{1.975910in}}%
\pgfpathlineto{\pgfqpoint{1.597091in}{1.973660in}}%
\pgfpathlineto{\pgfqpoint{1.577923in}{1.965026in}}%
\pgfusepath{stroke}%
\end{pgfscope}%
\begin{pgfscope}%
\pgfpathrectangle{\pgfqpoint{0.647939in}{0.492442in}}{\pgfqpoint{4.273799in}{2.331163in}}%
\pgfusepath{clip}%
\pgfsetbuttcap%
\pgfsetroundjoin%
\pgfsetlinewidth{0.301125pt}%
\definecolor{currentstroke}{rgb}{0.500000,0.500000,0.500000}%
\pgfsetstrokecolor{currentstroke}%
\pgfsetstrokeopacity{0.300000}%
\pgfsetdash{}{0pt}%
\pgfpathmoveto{\pgfqpoint{2.784839in}{1.128214in}}%
\pgfpathlineto{\pgfqpoint{2.744733in}{1.175169in}}%
\pgfpathlineto{\pgfqpoint{2.705998in}{1.222465in}}%
\pgfpathlineto{\pgfqpoint{2.668621in}{1.270085in}}%
\pgfpathlineto{\pgfqpoint{2.632600in}{1.318016in}}%
\pgfpathlineto{\pgfqpoint{2.597934in}{1.366244in}}%
\pgfpathlineto{\pgfqpoint{2.564630in}{1.414756in}}%
\pgfusepath{stroke}%
\end{pgfscope}%
\begin{pgfscope}%
\pgfpathrectangle{\pgfqpoint{0.647939in}{0.492442in}}{\pgfqpoint{4.273799in}{2.331163in}}%
\pgfusepath{clip}%
\pgfsetbuttcap%
\pgfsetroundjoin%
\pgfsetlinewidth{0.301125pt}%
\definecolor{currentstroke}{rgb}{0.500000,0.500000,0.500000}%
\pgfsetstrokecolor{currentstroke}%
\pgfsetstrokeopacity{0.300000}%
\pgfsetdash{}{0pt}%
\pgfpathmoveto{\pgfqpoint{3.502785in}{1.055448in}}%
\pgfpathlineto{\pgfqpoint{3.434343in}{1.091354in}}%
\pgfpathlineto{\pgfqpoint{3.367630in}{1.128214in}}%
\pgfpathlineto{\pgfqpoint{3.303023in}{1.166169in}}%
\pgfpathlineto{\pgfqpoint{3.240786in}{1.205284in}}%
\pgfpathlineto{\pgfqpoint{3.181090in}{1.245558in}}%
\pgfpathlineto{\pgfqpoint{3.124036in}{1.286953in}}%
\pgfusepath{stroke}%
\end{pgfscope}%
\begin{pgfscope}%
\pgfpathrectangle{\pgfqpoint{0.647939in}{0.492442in}}{\pgfqpoint{4.273799in}{2.331163in}}%
\pgfusepath{clip}%
\pgfsetbuttcap%
\pgfsetroundjoin%
\pgfsetlinewidth{0.301125pt}%
\definecolor{currentstroke}{rgb}{0.500000,0.500000,0.500000}%
\pgfsetstrokecolor{currentstroke}%
\pgfsetstrokeopacity{0.300000}%
\pgfsetdash{}{0pt}%
\pgfpathmoveto{\pgfqpoint{2.717149in}{1.985646in}}%
\pgfpathlineto{\pgfqpoint{2.705767in}{2.037021in}}%
\pgfpathlineto{\pgfqpoint{2.704190in}{2.088054in}}%
\pgfpathlineto{\pgfqpoint{2.712591in}{2.126490in}}%
\pgfpathlineto{\pgfqpoint{2.727821in}{2.154256in}}%
\pgfpathlineto{\pgfqpoint{2.748123in}{2.173486in}}%
\pgfpathlineto{\pgfqpoint{2.784839in}{2.187834in}}%
\pgfpathlineto{\pgfqpoint{2.784839in}{2.187834in}}%
\pgfpathlineto{\pgfqpoint{2.784839in}{2.187834in}}%
\pgfpathlineto{\pgfqpoint{2.821818in}{2.188561in}}%
\pgfpathlineto{\pgfqpoint{2.853103in}{2.181335in}}%
\pgfpathlineto{\pgfqpoint{2.884500in}{2.167214in}}%
\pgfusepath{stroke}%
\end{pgfscope}%
\begin{pgfscope}%
\pgfpathrectangle{\pgfqpoint{0.647939in}{0.492442in}}{\pgfqpoint{4.273799in}{2.331163in}}%
\pgfusepath{clip}%
\pgfsetbuttcap%
\pgfsetroundjoin%
\pgfsetlinewidth{0.301125pt}%
\definecolor{currentstroke}{rgb}{0.500000,0.500000,0.500000}%
\pgfsetstrokecolor{currentstroke}%
\pgfsetstrokeopacity{0.300000}%
\pgfsetdash{}{0pt}%
\pgfpathmoveto{\pgfqpoint{1.906970in}{1.931076in}}%
\pgfpathlineto{\pgfqpoint{1.893871in}{1.982370in}}%
\pgfpathlineto{\pgfqpoint{1.885493in}{2.033944in}}%
\pgfpathlineto{\pgfqpoint{1.883801in}{2.085685in}}%
\pgfpathlineto{\pgfqpoint{1.891358in}{2.137233in}}%
\pgfpathlineto{\pgfqpoint{1.910652in}{2.187834in}}%
\pgfusepath{stroke}%
\end{pgfscope}%
\begin{pgfscope}%
\pgfpathrectangle{\pgfqpoint{0.647939in}{0.492442in}}{\pgfqpoint{4.273799in}{2.331163in}}%
\pgfusepath{clip}%
\pgfsetbuttcap%
\pgfsetroundjoin%
\pgfsetlinewidth{0.301125pt}%
\definecolor{currentstroke}{rgb}{0.500000,0.500000,0.500000}%
\pgfsetstrokecolor{currentstroke}%
\pgfsetstrokeopacity{0.300000}%
\pgfsetdash{}{0pt}%
\pgfpathmoveto{\pgfqpoint{3.195701in}{2.125099in}}%
\pgfpathlineto{\pgfqpoint{3.200055in}{2.073407in}}%
\pgfpathlineto{\pgfqpoint{3.195028in}{2.025960in}}%
\pgfpathlineto{\pgfqpoint{3.173366in}{1.975910in}}%
\pgfpathlineto{\pgfqpoint{3.173366in}{1.975910in}}%
\pgfpathlineto{\pgfqpoint{3.146638in}{1.948096in}}%
\pgfpathlineto{\pgfqpoint{3.146638in}{1.948096in}}%
\pgfpathlineto{\pgfqpoint{3.114658in}{1.932114in}}%
\pgfpathlineto{\pgfqpoint{3.114658in}{1.932114in}}%
\pgfpathlineto{\pgfqpoint{3.079906in}{1.926433in}}%
\pgfpathlineto{\pgfqpoint{3.044010in}{1.929465in}}%
\pgfusepath{stroke}%
\end{pgfscope}%
\begin{pgfscope}%
\pgfpathrectangle{\pgfqpoint{0.647939in}{0.492442in}}{\pgfqpoint{4.273799in}{2.331163in}}%
\pgfusepath{clip}%
\pgfsetroundcap%
\pgfsetroundjoin%
\pgfsetlinewidth{0.301125pt}%
\definecolor{currentstroke}{rgb}{0.500000,0.500000,0.500000}%
\pgfsetstrokecolor{currentstroke}%
\pgfsetstrokeopacity{0.300000}%
\pgfsetdash{}{0pt}%
\pgfpathmoveto{\pgfqpoint{1.449487in}{1.421178in}}%
\pgfusepath{stroke}%
\end{pgfscope}%
\begin{pgfscope}%
\pgfpathrectangle{\pgfqpoint{0.647939in}{0.492442in}}{\pgfqpoint{4.273799in}{2.331163in}}%
\pgfusepath{clip}%
\pgfsetroundcap%
\pgfsetroundjoin%
\definecolor{currentfill}{rgb}{0.500000,0.500000,0.500000}%
\pgfsetfillcolor{currentfill}%
\pgfsetfillopacity{0.300000}%
\pgfsetlinewidth{0.301125pt}%
\definecolor{currentstroke}{rgb}{0.500000,0.500000,0.500000}%
\pgfsetstrokecolor{currentstroke}%
\pgfsetstrokeopacity{0.300000}%
\pgfsetdash{}{0pt}%
\pgfpathmoveto{\pgfqpoint{0.000000in}{0.000000in}}%
\pgfpathlineto{\pgfqpoint{0.000000in}{0.000000in}}%
\pgfpathclose%
\pgfusepath{stroke,fill}%
\end{pgfscope}%
\begin{pgfscope}%
\pgfpathrectangle{\pgfqpoint{0.647939in}{0.492442in}}{\pgfqpoint{4.273799in}{2.331163in}}%
\pgfusepath{clip}%
\pgfsetroundcap%
\pgfsetroundjoin%
\pgfsetlinewidth{0.301125pt}%
\definecolor{currentstroke}{rgb}{0.500000,0.500000,0.500000}%
\pgfsetstrokecolor{currentstroke}%
\pgfsetstrokeopacity{0.300000}%
\pgfsetdash{}{0pt}%
\pgfpathmoveto{\pgfqpoint{1.251809in}{0.907269in}}%
\pgfusepath{stroke}%
\end{pgfscope}%
\begin{pgfscope}%
\pgfpathrectangle{\pgfqpoint{0.647939in}{0.492442in}}{\pgfqpoint{4.273799in}{2.331163in}}%
\pgfusepath{clip}%
\pgfsetroundcap%
\pgfsetroundjoin%
\definecolor{currentfill}{rgb}{0.500000,0.500000,0.500000}%
\pgfsetfillcolor{currentfill}%
\pgfsetfillopacity{0.300000}%
\pgfsetlinewidth{0.301125pt}%
\definecolor{currentstroke}{rgb}{0.500000,0.500000,0.500000}%
\pgfsetstrokecolor{currentstroke}%
\pgfsetstrokeopacity{0.300000}%
\pgfsetdash{}{0pt}%
\pgfpathmoveto{\pgfqpoint{0.000000in}{0.000000in}}%
\pgfpathlineto{\pgfqpoint{0.000000in}{0.000000in}}%
\pgfpathclose%
\pgfusepath{stroke,fill}%
\end{pgfscope}%
\begin{pgfscope}%
\pgfpathrectangle{\pgfqpoint{0.647939in}{0.492442in}}{\pgfqpoint{4.273799in}{2.331163in}}%
\pgfusepath{clip}%
\pgfsetroundcap%
\pgfsetroundjoin%
\pgfsetlinewidth{0.301125pt}%
\definecolor{currentstroke}{rgb}{0.500000,0.500000,0.500000}%
\pgfsetstrokecolor{currentstroke}%
\pgfsetstrokeopacity{0.300000}%
\pgfsetdash{}{0pt}%
\pgfpathmoveto{\pgfqpoint{1.215694in}{0.691449in}}%
\pgfusepath{stroke}%
\end{pgfscope}%
\begin{pgfscope}%
\pgfpathrectangle{\pgfqpoint{0.647939in}{0.492442in}}{\pgfqpoint{4.273799in}{2.331163in}}%
\pgfusepath{clip}%
\pgfsetroundcap%
\pgfsetroundjoin%
\definecolor{currentfill}{rgb}{0.500000,0.500000,0.500000}%
\pgfsetfillcolor{currentfill}%
\pgfsetfillopacity{0.300000}%
\pgfsetlinewidth{0.301125pt}%
\definecolor{currentstroke}{rgb}{0.500000,0.500000,0.500000}%
\pgfsetstrokecolor{currentstroke}%
\pgfsetstrokeopacity{0.300000}%
\pgfsetdash{}{0pt}%
\pgfpathmoveto{\pgfqpoint{0.000000in}{0.000000in}}%
\pgfpathlineto{\pgfqpoint{0.000000in}{0.000000in}}%
\pgfpathclose%
\pgfusepath{stroke,fill}%
\end{pgfscope}%
\begin{pgfscope}%
\pgfpathrectangle{\pgfqpoint{0.647939in}{0.492442in}}{\pgfqpoint{4.273799in}{2.331163in}}%
\pgfusepath{clip}%
\pgfsetroundcap%
\pgfsetroundjoin%
\pgfsetlinewidth{0.301125pt}%
\definecolor{currentstroke}{rgb}{0.500000,0.500000,0.500000}%
\pgfsetstrokecolor{currentstroke}%
\pgfsetstrokeopacity{0.300000}%
\pgfsetdash{}{0pt}%
\pgfpathmoveto{\pgfqpoint{1.176426in}{0.574418in}}%
\pgfusepath{stroke}%
\end{pgfscope}%
\begin{pgfscope}%
\pgfpathrectangle{\pgfqpoint{0.647939in}{0.492442in}}{\pgfqpoint{4.273799in}{2.331163in}}%
\pgfusepath{clip}%
\pgfsetroundcap%
\pgfsetroundjoin%
\definecolor{currentfill}{rgb}{0.500000,0.500000,0.500000}%
\pgfsetfillcolor{currentfill}%
\pgfsetfillopacity{0.300000}%
\pgfsetlinewidth{0.301125pt}%
\definecolor{currentstroke}{rgb}{0.500000,0.500000,0.500000}%
\pgfsetstrokecolor{currentstroke}%
\pgfsetstrokeopacity{0.300000}%
\pgfsetdash{}{0pt}%
\pgfpathmoveto{\pgfqpoint{0.000000in}{0.000000in}}%
\pgfpathlineto{\pgfqpoint{0.000000in}{0.000000in}}%
\pgfpathclose%
\pgfusepath{stroke,fill}%
\end{pgfscope}%
\begin{pgfscope}%
\pgfpathrectangle{\pgfqpoint{0.647939in}{0.492442in}}{\pgfqpoint{4.273799in}{2.331163in}}%
\pgfusepath{clip}%
\pgfsetroundcap%
\pgfsetroundjoin%
\pgfsetlinewidth{0.301125pt}%
\definecolor{currentstroke}{rgb}{0.500000,0.500000,0.500000}%
\pgfsetstrokecolor{currentstroke}%
\pgfsetstrokeopacity{0.300000}%
\pgfsetdash{}{0pt}%
\pgfpathmoveto{\pgfqpoint{1.379786in}{0.761976in}}%
\pgfusepath{stroke}%
\end{pgfscope}%
\begin{pgfscope}%
\pgfpathrectangle{\pgfqpoint{0.647939in}{0.492442in}}{\pgfqpoint{4.273799in}{2.331163in}}%
\pgfusepath{clip}%
\pgfsetroundcap%
\pgfsetroundjoin%
\definecolor{currentfill}{rgb}{0.500000,0.500000,0.500000}%
\pgfsetfillcolor{currentfill}%
\pgfsetfillopacity{0.300000}%
\pgfsetlinewidth{0.301125pt}%
\definecolor{currentstroke}{rgb}{0.500000,0.500000,0.500000}%
\pgfsetstrokecolor{currentstroke}%
\pgfsetstrokeopacity{0.300000}%
\pgfsetdash{}{0pt}%
\pgfpathmoveto{\pgfqpoint{0.000000in}{0.000000in}}%
\pgfpathlineto{\pgfqpoint{0.000000in}{0.000000in}}%
\pgfpathclose%
\pgfusepath{stroke,fill}%
\end{pgfscope}%
\begin{pgfscope}%
\pgfpathrectangle{\pgfqpoint{0.647939in}{0.492442in}}{\pgfqpoint{4.273799in}{2.331163in}}%
\pgfusepath{clip}%
\pgfsetroundcap%
\pgfsetroundjoin%
\pgfsetlinewidth{0.301125pt}%
\definecolor{currentstroke}{rgb}{0.500000,0.500000,0.500000}%
\pgfsetstrokecolor{currentstroke}%
\pgfsetstrokeopacity{0.300000}%
\pgfsetdash{}{0pt}%
\pgfpathmoveto{\pgfqpoint{1.813801in}{0.607767in}}%
\pgfusepath{stroke}%
\end{pgfscope}%
\begin{pgfscope}%
\pgfpathrectangle{\pgfqpoint{0.647939in}{0.492442in}}{\pgfqpoint{4.273799in}{2.331163in}}%
\pgfusepath{clip}%
\pgfsetroundcap%
\pgfsetroundjoin%
\definecolor{currentfill}{rgb}{0.500000,0.500000,0.500000}%
\pgfsetfillcolor{currentfill}%
\pgfsetfillopacity{0.300000}%
\pgfsetlinewidth{0.301125pt}%
\definecolor{currentstroke}{rgb}{0.500000,0.500000,0.500000}%
\pgfsetstrokecolor{currentstroke}%
\pgfsetstrokeopacity{0.300000}%
\pgfsetdash{}{0pt}%
\pgfpathmoveto{\pgfqpoint{0.000000in}{0.000000in}}%
\pgfpathlineto{\pgfqpoint{0.000000in}{0.000000in}}%
\pgfpathclose%
\pgfusepath{stroke,fill}%
\end{pgfscope}%
\begin{pgfscope}%
\pgfpathrectangle{\pgfqpoint{0.647939in}{0.492442in}}{\pgfqpoint{4.273799in}{2.331163in}}%
\pgfusepath{clip}%
\pgfsetroundcap%
\pgfsetroundjoin%
\pgfsetlinewidth{0.301125pt}%
\definecolor{currentstroke}{rgb}{0.500000,0.500000,0.500000}%
\pgfsetstrokecolor{currentstroke}%
\pgfsetstrokeopacity{0.300000}%
\pgfsetdash{}{0pt}%
\pgfpathmoveto{\pgfqpoint{1.575978in}{0.976877in}}%
\pgfusepath{stroke}%
\end{pgfscope}%
\begin{pgfscope}%
\pgfpathrectangle{\pgfqpoint{0.647939in}{0.492442in}}{\pgfqpoint{4.273799in}{2.331163in}}%
\pgfusepath{clip}%
\pgfsetroundcap%
\pgfsetroundjoin%
\definecolor{currentfill}{rgb}{0.500000,0.500000,0.500000}%
\pgfsetfillcolor{currentfill}%
\pgfsetfillopacity{0.300000}%
\pgfsetlinewidth{0.301125pt}%
\definecolor{currentstroke}{rgb}{0.500000,0.500000,0.500000}%
\pgfsetstrokecolor{currentstroke}%
\pgfsetstrokeopacity{0.300000}%
\pgfsetdash{}{0pt}%
\pgfpathmoveto{\pgfqpoint{0.000000in}{0.000000in}}%
\pgfpathlineto{\pgfqpoint{0.000000in}{0.000000in}}%
\pgfpathclose%
\pgfusepath{stroke,fill}%
\end{pgfscope}%
\begin{pgfscope}%
\pgfpathrectangle{\pgfqpoint{0.647939in}{0.492442in}}{\pgfqpoint{4.273799in}{2.331163in}}%
\pgfusepath{clip}%
\pgfsetroundcap%
\pgfsetroundjoin%
\pgfsetlinewidth{0.301125pt}%
\definecolor{currentstroke}{rgb}{0.500000,0.500000,0.500000}%
\pgfsetstrokecolor{currentstroke}%
\pgfsetstrokeopacity{0.300000}%
\pgfsetdash{}{0pt}%
\pgfpathmoveto{\pgfqpoint{1.702561in}{0.986550in}}%
\pgfusepath{stroke}%
\end{pgfscope}%
\begin{pgfscope}%
\pgfpathrectangle{\pgfqpoint{0.647939in}{0.492442in}}{\pgfqpoint{4.273799in}{2.331163in}}%
\pgfusepath{clip}%
\pgfsetroundcap%
\pgfsetroundjoin%
\definecolor{currentfill}{rgb}{0.500000,0.500000,0.500000}%
\pgfsetfillcolor{currentfill}%
\pgfsetfillopacity{0.300000}%
\pgfsetlinewidth{0.301125pt}%
\definecolor{currentstroke}{rgb}{0.500000,0.500000,0.500000}%
\pgfsetstrokecolor{currentstroke}%
\pgfsetstrokeopacity{0.300000}%
\pgfsetdash{}{0pt}%
\pgfpathmoveto{\pgfqpoint{0.000000in}{0.000000in}}%
\pgfpathlineto{\pgfqpoint{0.000000in}{0.000000in}}%
\pgfpathclose%
\pgfusepath{stroke,fill}%
\end{pgfscope}%
\begin{pgfscope}%
\pgfpathrectangle{\pgfqpoint{0.647939in}{0.492442in}}{\pgfqpoint{4.273799in}{2.331163in}}%
\pgfusepath{clip}%
\pgfsetroundcap%
\pgfsetroundjoin%
\pgfsetlinewidth{0.301125pt}%
\definecolor{currentstroke}{rgb}{0.500000,0.500000,0.500000}%
\pgfsetstrokecolor{currentstroke}%
\pgfsetstrokeopacity{0.300000}%
\pgfsetdash{}{0pt}%
\pgfpathmoveto{\pgfqpoint{1.846820in}{1.235276in}}%
\pgfusepath{stroke}%
\end{pgfscope}%
\begin{pgfscope}%
\pgfpathrectangle{\pgfqpoint{0.647939in}{0.492442in}}{\pgfqpoint{4.273799in}{2.331163in}}%
\pgfusepath{clip}%
\pgfsetroundcap%
\pgfsetroundjoin%
\definecolor{currentfill}{rgb}{0.500000,0.500000,0.500000}%
\pgfsetfillcolor{currentfill}%
\pgfsetfillopacity{0.300000}%
\pgfsetlinewidth{0.301125pt}%
\definecolor{currentstroke}{rgb}{0.500000,0.500000,0.500000}%
\pgfsetstrokecolor{currentstroke}%
\pgfsetstrokeopacity{0.300000}%
\pgfsetdash{}{0pt}%
\pgfpathmoveto{\pgfqpoint{0.000000in}{0.000000in}}%
\pgfpathlineto{\pgfqpoint{0.000000in}{0.000000in}}%
\pgfpathclose%
\pgfusepath{stroke,fill}%
\end{pgfscope}%
\begin{pgfscope}%
\pgfpathrectangle{\pgfqpoint{0.647939in}{0.492442in}}{\pgfqpoint{4.273799in}{2.331163in}}%
\pgfusepath{clip}%
\pgfsetroundcap%
\pgfsetroundjoin%
\pgfsetlinewidth{0.301125pt}%
\definecolor{currentstroke}{rgb}{0.500000,0.500000,0.500000}%
\pgfsetstrokecolor{currentstroke}%
\pgfsetstrokeopacity{0.300000}%
\pgfsetdash{}{0pt}%
\pgfpathmoveto{\pgfqpoint{1.949381in}{1.236810in}}%
\pgfusepath{stroke}%
\end{pgfscope}%
\begin{pgfscope}%
\pgfpathrectangle{\pgfqpoint{0.647939in}{0.492442in}}{\pgfqpoint{4.273799in}{2.331163in}}%
\pgfusepath{clip}%
\pgfsetroundcap%
\pgfsetroundjoin%
\definecolor{currentfill}{rgb}{0.500000,0.500000,0.500000}%
\pgfsetfillcolor{currentfill}%
\pgfsetfillopacity{0.300000}%
\pgfsetlinewidth{0.301125pt}%
\definecolor{currentstroke}{rgb}{0.500000,0.500000,0.500000}%
\pgfsetstrokecolor{currentstroke}%
\pgfsetstrokeopacity{0.300000}%
\pgfsetdash{}{0pt}%
\pgfpathmoveto{\pgfqpoint{0.000000in}{0.000000in}}%
\pgfpathlineto{\pgfqpoint{0.000000in}{0.000000in}}%
\pgfpathclose%
\pgfusepath{stroke,fill}%
\end{pgfscope}%
\begin{pgfscope}%
\pgfpathrectangle{\pgfqpoint{0.647939in}{0.492442in}}{\pgfqpoint{4.273799in}{2.331163in}}%
\pgfusepath{clip}%
\pgfsetroundcap%
\pgfsetroundjoin%
\pgfsetlinewidth{0.301125pt}%
\definecolor{currentstroke}{rgb}{0.500000,0.500000,0.500000}%
\pgfsetstrokecolor{currentstroke}%
\pgfsetstrokeopacity{0.300000}%
\pgfsetdash{}{0pt}%
\pgfpathmoveto{\pgfqpoint{2.012304in}{1.285245in}}%
\pgfusepath{stroke}%
\end{pgfscope}%
\begin{pgfscope}%
\pgfpathrectangle{\pgfqpoint{0.647939in}{0.492442in}}{\pgfqpoint{4.273799in}{2.331163in}}%
\pgfusepath{clip}%
\pgfsetroundcap%
\pgfsetroundjoin%
\definecolor{currentfill}{rgb}{0.500000,0.500000,0.500000}%
\pgfsetfillcolor{currentfill}%
\pgfsetfillopacity{0.300000}%
\pgfsetlinewidth{0.301125pt}%
\definecolor{currentstroke}{rgb}{0.500000,0.500000,0.500000}%
\pgfsetstrokecolor{currentstroke}%
\pgfsetstrokeopacity{0.300000}%
\pgfsetdash{}{0pt}%
\pgfpathmoveto{\pgfqpoint{0.000000in}{0.000000in}}%
\pgfpathlineto{\pgfqpoint{0.000000in}{0.000000in}}%
\pgfpathclose%
\pgfusepath{stroke,fill}%
\end{pgfscope}%
\begin{pgfscope}%
\pgfpathrectangle{\pgfqpoint{0.647939in}{0.492442in}}{\pgfqpoint{4.273799in}{2.331163in}}%
\pgfusepath{clip}%
\pgfsetroundcap%
\pgfsetroundjoin%
\pgfsetlinewidth{0.301125pt}%
\definecolor{currentstroke}{rgb}{0.500000,0.500000,0.500000}%
\pgfsetstrokecolor{currentstroke}%
\pgfsetstrokeopacity{0.300000}%
\pgfsetdash{}{0pt}%
\pgfpathmoveto{\pgfqpoint{2.475311in}{0.750470in}}%
\pgfusepath{stroke}%
\end{pgfscope}%
\begin{pgfscope}%
\pgfpathrectangle{\pgfqpoint{0.647939in}{0.492442in}}{\pgfqpoint{4.273799in}{2.331163in}}%
\pgfusepath{clip}%
\pgfsetroundcap%
\pgfsetroundjoin%
\definecolor{currentfill}{rgb}{0.500000,0.500000,0.500000}%
\pgfsetfillcolor{currentfill}%
\pgfsetfillopacity{0.300000}%
\pgfsetlinewidth{0.301125pt}%
\definecolor{currentstroke}{rgb}{0.500000,0.500000,0.500000}%
\pgfsetstrokecolor{currentstroke}%
\pgfsetstrokeopacity{0.300000}%
\pgfsetdash{}{0pt}%
\pgfpathmoveto{\pgfqpoint{0.000000in}{0.000000in}}%
\pgfpathlineto{\pgfqpoint{0.000000in}{0.000000in}}%
\pgfpathclose%
\pgfusepath{stroke,fill}%
\end{pgfscope}%
\begin{pgfscope}%
\pgfpathrectangle{\pgfqpoint{0.647939in}{0.492442in}}{\pgfqpoint{4.273799in}{2.331163in}}%
\pgfusepath{clip}%
\pgfsetroundcap%
\pgfsetroundjoin%
\pgfsetlinewidth{0.301125pt}%
\definecolor{currentstroke}{rgb}{0.500000,0.500000,0.500000}%
\pgfsetstrokecolor{currentstroke}%
\pgfsetstrokeopacity{0.300000}%
\pgfsetdash{}{0pt}%
\pgfpathmoveto{\pgfqpoint{2.643019in}{0.653638in}}%
\pgfusepath{stroke}%
\end{pgfscope}%
\begin{pgfscope}%
\pgfpathrectangle{\pgfqpoint{0.647939in}{0.492442in}}{\pgfqpoint{4.273799in}{2.331163in}}%
\pgfusepath{clip}%
\pgfsetroundcap%
\pgfsetroundjoin%
\definecolor{currentfill}{rgb}{0.500000,0.500000,0.500000}%
\pgfsetfillcolor{currentfill}%
\pgfsetfillopacity{0.300000}%
\pgfsetlinewidth{0.301125pt}%
\definecolor{currentstroke}{rgb}{0.500000,0.500000,0.500000}%
\pgfsetstrokecolor{currentstroke}%
\pgfsetstrokeopacity{0.300000}%
\pgfsetdash{}{0pt}%
\pgfpathmoveto{\pgfqpoint{0.000000in}{0.000000in}}%
\pgfpathlineto{\pgfqpoint{0.000000in}{0.000000in}}%
\pgfpathclose%
\pgfusepath{stroke,fill}%
\end{pgfscope}%
\begin{pgfscope}%
\pgfpathrectangle{\pgfqpoint{0.647939in}{0.492442in}}{\pgfqpoint{4.273799in}{2.331163in}}%
\pgfusepath{clip}%
\pgfsetroundcap%
\pgfsetroundjoin%
\pgfsetlinewidth{0.301125pt}%
\definecolor{currentstroke}{rgb}{0.500000,0.500000,0.500000}%
\pgfsetstrokecolor{currentstroke}%
\pgfsetstrokeopacity{0.300000}%
\pgfsetdash{}{0pt}%
\pgfpathmoveto{\pgfqpoint{2.000609in}{2.102014in}}%
\pgfusepath{stroke}%
\end{pgfscope}%
\begin{pgfscope}%
\pgfpathrectangle{\pgfqpoint{0.647939in}{0.492442in}}{\pgfqpoint{4.273799in}{2.331163in}}%
\pgfusepath{clip}%
\pgfsetroundcap%
\pgfsetroundjoin%
\definecolor{currentfill}{rgb}{0.500000,0.500000,0.500000}%
\pgfsetfillcolor{currentfill}%
\pgfsetfillopacity{0.300000}%
\pgfsetlinewidth{0.301125pt}%
\definecolor{currentstroke}{rgb}{0.500000,0.500000,0.500000}%
\pgfsetstrokecolor{currentstroke}%
\pgfsetstrokeopacity{0.300000}%
\pgfsetdash{}{0pt}%
\pgfpathmoveto{\pgfqpoint{0.000000in}{0.000000in}}%
\pgfpathlineto{\pgfqpoint{0.000000in}{0.000000in}}%
\pgfpathclose%
\pgfusepath{stroke,fill}%
\end{pgfscope}%
\begin{pgfscope}%
\pgfpathrectangle{\pgfqpoint{0.647939in}{0.492442in}}{\pgfqpoint{4.273799in}{2.331163in}}%
\pgfusepath{clip}%
\pgfsetroundcap%
\pgfsetroundjoin%
\pgfsetlinewidth{0.301125pt}%
\definecolor{currentstroke}{rgb}{0.500000,0.500000,0.500000}%
\pgfsetstrokecolor{currentstroke}%
\pgfsetstrokeopacity{0.300000}%
\pgfsetdash{}{0pt}%
\pgfpathmoveto{\pgfqpoint{2.327929in}{1.501825in}}%
\pgfusepath{stroke}%
\end{pgfscope}%
\begin{pgfscope}%
\pgfpathrectangle{\pgfqpoint{0.647939in}{0.492442in}}{\pgfqpoint{4.273799in}{2.331163in}}%
\pgfusepath{clip}%
\pgfsetroundcap%
\pgfsetroundjoin%
\definecolor{currentfill}{rgb}{0.500000,0.500000,0.500000}%
\pgfsetfillcolor{currentfill}%
\pgfsetfillopacity{0.300000}%
\pgfsetlinewidth{0.301125pt}%
\definecolor{currentstroke}{rgb}{0.500000,0.500000,0.500000}%
\pgfsetstrokecolor{currentstroke}%
\pgfsetstrokeopacity{0.300000}%
\pgfsetdash{}{0pt}%
\pgfpathmoveto{\pgfqpoint{0.000000in}{0.000000in}}%
\pgfpathlineto{\pgfqpoint{0.000000in}{0.000000in}}%
\pgfpathclose%
\pgfusepath{stroke,fill}%
\end{pgfscope}%
\begin{pgfscope}%
\pgfpathrectangle{\pgfqpoint{0.647939in}{0.492442in}}{\pgfqpoint{4.273799in}{2.331163in}}%
\pgfusepath{clip}%
\pgfsetroundcap%
\pgfsetroundjoin%
\pgfsetlinewidth{0.301125pt}%
\definecolor{currentstroke}{rgb}{0.500000,0.500000,0.500000}%
\pgfsetstrokecolor{currentstroke}%
\pgfsetstrokeopacity{0.300000}%
\pgfsetdash{}{0pt}%
\pgfpathmoveto{\pgfqpoint{2.540729in}{1.332761in}}%
\pgfusepath{stroke}%
\end{pgfscope}%
\begin{pgfscope}%
\pgfpathrectangle{\pgfqpoint{0.647939in}{0.492442in}}{\pgfqpoint{4.273799in}{2.331163in}}%
\pgfusepath{clip}%
\pgfsetroundcap%
\pgfsetroundjoin%
\definecolor{currentfill}{rgb}{0.500000,0.500000,0.500000}%
\pgfsetfillcolor{currentfill}%
\pgfsetfillopacity{0.300000}%
\pgfsetlinewidth{0.301125pt}%
\definecolor{currentstroke}{rgb}{0.500000,0.500000,0.500000}%
\pgfsetstrokecolor{currentstroke}%
\pgfsetstrokeopacity{0.300000}%
\pgfsetdash{}{0pt}%
\pgfpathmoveto{\pgfqpoint{0.000000in}{0.000000in}}%
\pgfpathlineto{\pgfqpoint{0.000000in}{0.000000in}}%
\pgfpathclose%
\pgfusepath{stroke,fill}%
\end{pgfscope}%
\begin{pgfscope}%
\pgfpathrectangle{\pgfqpoint{0.647939in}{0.492442in}}{\pgfqpoint{4.273799in}{2.331163in}}%
\pgfusepath{clip}%
\pgfsetroundcap%
\pgfsetroundjoin%
\pgfsetlinewidth{0.301125pt}%
\definecolor{currentstroke}{rgb}{0.500000,0.500000,0.500000}%
\pgfsetstrokecolor{currentstroke}%
\pgfsetstrokeopacity{0.300000}%
\pgfsetdash{}{0pt}%
\pgfpathmoveto{\pgfqpoint{3.195195in}{0.757068in}}%
\pgfusepath{stroke}%
\end{pgfscope}%
\begin{pgfscope}%
\pgfpathrectangle{\pgfqpoint{0.647939in}{0.492442in}}{\pgfqpoint{4.273799in}{2.331163in}}%
\pgfusepath{clip}%
\pgfsetroundcap%
\pgfsetroundjoin%
\definecolor{currentfill}{rgb}{0.500000,0.500000,0.500000}%
\pgfsetfillcolor{currentfill}%
\pgfsetfillopacity{0.300000}%
\pgfsetlinewidth{0.301125pt}%
\definecolor{currentstroke}{rgb}{0.500000,0.500000,0.500000}%
\pgfsetstrokecolor{currentstroke}%
\pgfsetstrokeopacity{0.300000}%
\pgfsetdash{}{0pt}%
\pgfpathmoveto{\pgfqpoint{0.000000in}{0.000000in}}%
\pgfpathlineto{\pgfqpoint{0.000000in}{0.000000in}}%
\pgfpathclose%
\pgfusepath{stroke,fill}%
\end{pgfscope}%
\begin{pgfscope}%
\pgfpathrectangle{\pgfqpoint{0.647939in}{0.492442in}}{\pgfqpoint{4.273799in}{2.331163in}}%
\pgfusepath{clip}%
\pgfsetroundcap%
\pgfsetroundjoin%
\pgfsetlinewidth{0.301125pt}%
\definecolor{currentstroke}{rgb}{0.500000,0.500000,0.500000}%
\pgfsetstrokecolor{currentstroke}%
\pgfsetstrokeopacity{0.300000}%
\pgfsetdash{}{0pt}%
\pgfpathmoveto{\pgfqpoint{2.659028in}{1.421785in}}%
\pgfusepath{stroke}%
\end{pgfscope}%
\begin{pgfscope}%
\pgfpathrectangle{\pgfqpoint{0.647939in}{0.492442in}}{\pgfqpoint{4.273799in}{2.331163in}}%
\pgfusepath{clip}%
\pgfsetroundcap%
\pgfsetroundjoin%
\definecolor{currentfill}{rgb}{0.500000,0.500000,0.500000}%
\pgfsetfillcolor{currentfill}%
\pgfsetfillopacity{0.300000}%
\pgfsetlinewidth{0.301125pt}%
\definecolor{currentstroke}{rgb}{0.500000,0.500000,0.500000}%
\pgfsetstrokecolor{currentstroke}%
\pgfsetstrokeopacity{0.300000}%
\pgfsetdash{}{0pt}%
\pgfpathmoveto{\pgfqpoint{0.000000in}{0.000000in}}%
\pgfpathlineto{\pgfqpoint{0.000000in}{0.000000in}}%
\pgfpathclose%
\pgfusepath{stroke,fill}%
\end{pgfscope}%
\begin{pgfscope}%
\pgfpathrectangle{\pgfqpoint{0.647939in}{0.492442in}}{\pgfqpoint{4.273799in}{2.331163in}}%
\pgfusepath{clip}%
\pgfsetroundcap%
\pgfsetroundjoin%
\pgfsetlinewidth{0.301125pt}%
\definecolor{currentstroke}{rgb}{0.500000,0.500000,0.500000}%
\pgfsetstrokecolor{currentstroke}%
\pgfsetstrokeopacity{0.300000}%
\pgfsetdash{}{0pt}%
\pgfpathmoveto{\pgfqpoint{3.705657in}{0.585558in}}%
\pgfusepath{stroke}%
\end{pgfscope}%
\begin{pgfscope}%
\pgfpathrectangle{\pgfqpoint{0.647939in}{0.492442in}}{\pgfqpoint{4.273799in}{2.331163in}}%
\pgfusepath{clip}%
\pgfsetroundcap%
\pgfsetroundjoin%
\definecolor{currentfill}{rgb}{0.500000,0.500000,0.500000}%
\pgfsetfillcolor{currentfill}%
\pgfsetfillopacity{0.300000}%
\pgfsetlinewidth{0.301125pt}%
\definecolor{currentstroke}{rgb}{0.500000,0.500000,0.500000}%
\pgfsetstrokecolor{currentstroke}%
\pgfsetstrokeopacity{0.300000}%
\pgfsetdash{}{0pt}%
\pgfpathmoveto{\pgfqpoint{0.000000in}{0.000000in}}%
\pgfpathlineto{\pgfqpoint{0.000000in}{0.000000in}}%
\pgfpathclose%
\pgfusepath{stroke,fill}%
\end{pgfscope}%
\begin{pgfscope}%
\pgfpathrectangle{\pgfqpoint{0.647939in}{0.492442in}}{\pgfqpoint{4.273799in}{2.331163in}}%
\pgfusepath{clip}%
\pgfsetroundcap%
\pgfsetroundjoin%
\pgfsetlinewidth{0.301125pt}%
\definecolor{currentstroke}{rgb}{0.500000,0.500000,0.500000}%
\pgfsetstrokecolor{currentstroke}%
\pgfsetstrokeopacity{0.300000}%
\pgfsetdash{}{0pt}%
\pgfpathmoveto{\pgfqpoint{3.804437in}{0.586248in}}%
\pgfusepath{stroke}%
\end{pgfscope}%
\begin{pgfscope}%
\pgfpathrectangle{\pgfqpoint{0.647939in}{0.492442in}}{\pgfqpoint{4.273799in}{2.331163in}}%
\pgfusepath{clip}%
\pgfsetroundcap%
\pgfsetroundjoin%
\definecolor{currentfill}{rgb}{0.500000,0.500000,0.500000}%
\pgfsetfillcolor{currentfill}%
\pgfsetfillopacity{0.300000}%
\pgfsetlinewidth{0.301125pt}%
\definecolor{currentstroke}{rgb}{0.500000,0.500000,0.500000}%
\pgfsetstrokecolor{currentstroke}%
\pgfsetstrokeopacity{0.300000}%
\pgfsetdash{}{0pt}%
\pgfpathmoveto{\pgfqpoint{0.000000in}{0.000000in}}%
\pgfpathlineto{\pgfqpoint{0.000000in}{0.000000in}}%
\pgfpathclose%
\pgfusepath{stroke,fill}%
\end{pgfscope}%
\begin{pgfscope}%
\pgfpathrectangle{\pgfqpoint{0.647939in}{0.492442in}}{\pgfqpoint{4.273799in}{2.331163in}}%
\pgfusepath{clip}%
\pgfsetroundcap%
\pgfsetroundjoin%
\pgfsetlinewidth{0.301125pt}%
\definecolor{currentstroke}{rgb}{0.500000,0.500000,0.500000}%
\pgfsetstrokecolor{currentstroke}%
\pgfsetstrokeopacity{0.300000}%
\pgfsetdash{}{0pt}%
\pgfpathmoveto{\pgfqpoint{3.905387in}{0.588152in}}%
\pgfusepath{stroke}%
\end{pgfscope}%
\begin{pgfscope}%
\pgfpathrectangle{\pgfqpoint{0.647939in}{0.492442in}}{\pgfqpoint{4.273799in}{2.331163in}}%
\pgfusepath{clip}%
\pgfsetroundcap%
\pgfsetroundjoin%
\definecolor{currentfill}{rgb}{0.500000,0.500000,0.500000}%
\pgfsetfillcolor{currentfill}%
\pgfsetfillopacity{0.300000}%
\pgfsetlinewidth{0.301125pt}%
\definecolor{currentstroke}{rgb}{0.500000,0.500000,0.500000}%
\pgfsetstrokecolor{currentstroke}%
\pgfsetstrokeopacity{0.300000}%
\pgfsetdash{}{0pt}%
\pgfpathmoveto{\pgfqpoint{0.000000in}{0.000000in}}%
\pgfpathlineto{\pgfqpoint{0.000000in}{0.000000in}}%
\pgfpathclose%
\pgfusepath{stroke,fill}%
\end{pgfscope}%
\begin{pgfscope}%
\pgfpathrectangle{\pgfqpoint{0.647939in}{0.492442in}}{\pgfqpoint{4.273799in}{2.331163in}}%
\pgfusepath{clip}%
\pgfsetroundcap%
\pgfsetroundjoin%
\pgfsetlinewidth{0.301125pt}%
\definecolor{currentstroke}{rgb}{0.500000,0.500000,0.500000}%
\pgfsetstrokecolor{currentstroke}%
\pgfsetstrokeopacity{0.300000}%
\pgfsetdash{}{0pt}%
\pgfpathmoveto{\pgfqpoint{3.107102in}{1.214576in}}%
\pgfusepath{stroke}%
\end{pgfscope}%
\begin{pgfscope}%
\pgfpathrectangle{\pgfqpoint{0.647939in}{0.492442in}}{\pgfqpoint{4.273799in}{2.331163in}}%
\pgfusepath{clip}%
\pgfsetroundcap%
\pgfsetroundjoin%
\definecolor{currentfill}{rgb}{0.500000,0.500000,0.500000}%
\pgfsetfillcolor{currentfill}%
\pgfsetfillopacity{0.300000}%
\pgfsetlinewidth{0.301125pt}%
\definecolor{currentstroke}{rgb}{0.500000,0.500000,0.500000}%
\pgfsetstrokecolor{currentstroke}%
\pgfsetstrokeopacity{0.300000}%
\pgfsetdash{}{0pt}%
\pgfpathmoveto{\pgfqpoint{0.000000in}{0.000000in}}%
\pgfpathlineto{\pgfqpoint{0.000000in}{0.000000in}}%
\pgfpathclose%
\pgfusepath{stroke,fill}%
\end{pgfscope}%
\begin{pgfscope}%
\pgfpathrectangle{\pgfqpoint{0.647939in}{0.492442in}}{\pgfqpoint{4.273799in}{2.331163in}}%
\pgfusepath{clip}%
\pgfsetroundcap%
\pgfsetroundjoin%
\pgfsetlinewidth{0.301125pt}%
\definecolor{currentstroke}{rgb}{0.500000,0.500000,0.500000}%
\pgfsetstrokecolor{currentstroke}%
\pgfsetstrokeopacity{0.300000}%
\pgfsetdash{}{0pt}%
\pgfpathmoveto{\pgfqpoint{3.995135in}{0.763526in}}%
\pgfusepath{stroke}%
\end{pgfscope}%
\begin{pgfscope}%
\pgfpathrectangle{\pgfqpoint{0.647939in}{0.492442in}}{\pgfqpoint{4.273799in}{2.331163in}}%
\pgfusepath{clip}%
\pgfsetroundcap%
\pgfsetroundjoin%
\definecolor{currentfill}{rgb}{0.500000,0.500000,0.500000}%
\pgfsetfillcolor{currentfill}%
\pgfsetfillopacity{0.300000}%
\pgfsetlinewidth{0.301125pt}%
\definecolor{currentstroke}{rgb}{0.500000,0.500000,0.500000}%
\pgfsetstrokecolor{currentstroke}%
\pgfsetstrokeopacity{0.300000}%
\pgfsetdash{}{0pt}%
\pgfpathmoveto{\pgfqpoint{0.000000in}{0.000000in}}%
\pgfpathlineto{\pgfqpoint{0.000000in}{0.000000in}}%
\pgfpathclose%
\pgfusepath{stroke,fill}%
\end{pgfscope}%
\begin{pgfscope}%
\pgfpathrectangle{\pgfqpoint{0.647939in}{0.492442in}}{\pgfqpoint{4.273799in}{2.331163in}}%
\pgfusepath{clip}%
\pgfsetroundcap%
\pgfsetroundjoin%
\pgfsetlinewidth{0.301125pt}%
\definecolor{currentstroke}{rgb}{0.500000,0.500000,0.500000}%
\pgfsetstrokecolor{currentstroke}%
\pgfsetstrokeopacity{0.300000}%
\pgfsetdash{}{0pt}%
\pgfpathmoveto{\pgfqpoint{3.446075in}{1.134191in}}%
\pgfusepath{stroke}%
\end{pgfscope}%
\begin{pgfscope}%
\pgfpathrectangle{\pgfqpoint{0.647939in}{0.492442in}}{\pgfqpoint{4.273799in}{2.331163in}}%
\pgfusepath{clip}%
\pgfsetroundcap%
\pgfsetroundjoin%
\definecolor{currentfill}{rgb}{0.500000,0.500000,0.500000}%
\pgfsetfillcolor{currentfill}%
\pgfsetfillopacity{0.300000}%
\pgfsetlinewidth{0.301125pt}%
\definecolor{currentstroke}{rgb}{0.500000,0.500000,0.500000}%
\pgfsetstrokecolor{currentstroke}%
\pgfsetstrokeopacity{0.300000}%
\pgfsetdash{}{0pt}%
\pgfpathmoveto{\pgfqpoint{0.000000in}{0.000000in}}%
\pgfpathlineto{\pgfqpoint{0.000000in}{0.000000in}}%
\pgfpathclose%
\pgfusepath{stroke,fill}%
\end{pgfscope}%
\begin{pgfscope}%
\pgfpathrectangle{\pgfqpoint{0.647939in}{0.492442in}}{\pgfqpoint{4.273799in}{2.331163in}}%
\pgfusepath{clip}%
\pgfsetroundcap%
\pgfsetroundjoin%
\pgfsetlinewidth{0.301125pt}%
\definecolor{currentstroke}{rgb}{0.500000,0.500000,0.500000}%
\pgfsetstrokecolor{currentstroke}%
\pgfsetstrokeopacity{0.300000}%
\pgfsetdash{}{0pt}%
\pgfpathmoveto{\pgfqpoint{3.824160in}{1.056047in}}%
\pgfusepath{stroke}%
\end{pgfscope}%
\begin{pgfscope}%
\pgfpathrectangle{\pgfqpoint{0.647939in}{0.492442in}}{\pgfqpoint{4.273799in}{2.331163in}}%
\pgfusepath{clip}%
\pgfsetroundcap%
\pgfsetroundjoin%
\definecolor{currentfill}{rgb}{0.500000,0.500000,0.500000}%
\pgfsetfillcolor{currentfill}%
\pgfsetfillopacity{0.300000}%
\pgfsetlinewidth{0.301125pt}%
\definecolor{currentstroke}{rgb}{0.500000,0.500000,0.500000}%
\pgfsetstrokecolor{currentstroke}%
\pgfsetstrokeopacity{0.300000}%
\pgfsetdash{}{0pt}%
\pgfpathmoveto{\pgfqpoint{0.000000in}{0.000000in}}%
\pgfpathlineto{\pgfqpoint{0.000000in}{0.000000in}}%
\pgfpathclose%
\pgfusepath{stroke,fill}%
\end{pgfscope}%
\begin{pgfscope}%
\pgfpathrectangle{\pgfqpoint{0.647939in}{0.492442in}}{\pgfqpoint{4.273799in}{2.331163in}}%
\pgfusepath{clip}%
\pgfsetroundcap%
\pgfsetroundjoin%
\pgfsetlinewidth{0.301125pt}%
\definecolor{currentstroke}{rgb}{0.500000,0.500000,0.500000}%
\pgfsetstrokecolor{currentstroke}%
\pgfsetstrokeopacity{0.300000}%
\pgfsetdash{}{0pt}%
\pgfpathmoveto{\pgfqpoint{3.712598in}{1.213039in}}%
\pgfusepath{stroke}%
\end{pgfscope}%
\begin{pgfscope}%
\pgfpathrectangle{\pgfqpoint{0.647939in}{0.492442in}}{\pgfqpoint{4.273799in}{2.331163in}}%
\pgfusepath{clip}%
\pgfsetroundcap%
\pgfsetroundjoin%
\definecolor{currentfill}{rgb}{0.500000,0.500000,0.500000}%
\pgfsetfillcolor{currentfill}%
\pgfsetfillopacity{0.300000}%
\pgfsetlinewidth{0.301125pt}%
\definecolor{currentstroke}{rgb}{0.500000,0.500000,0.500000}%
\pgfsetstrokecolor{currentstroke}%
\pgfsetstrokeopacity{0.300000}%
\pgfsetdash{}{0pt}%
\pgfpathmoveto{\pgfqpoint{0.000000in}{0.000000in}}%
\pgfpathlineto{\pgfqpoint{0.000000in}{0.000000in}}%
\pgfpathclose%
\pgfusepath{stroke,fill}%
\end{pgfscope}%
\begin{pgfscope}%
\pgfpathrectangle{\pgfqpoint{0.647939in}{0.492442in}}{\pgfqpoint{4.273799in}{2.331163in}}%
\pgfusepath{clip}%
\pgfsetroundcap%
\pgfsetroundjoin%
\pgfsetlinewidth{0.301125pt}%
\definecolor{currentstroke}{rgb}{0.500000,0.500000,0.500000}%
\pgfsetstrokecolor{currentstroke}%
\pgfsetstrokeopacity{0.300000}%
\pgfsetdash{}{0pt}%
\pgfpathmoveto{\pgfqpoint{4.096796in}{1.202352in}}%
\pgfusepath{stroke}%
\end{pgfscope}%
\begin{pgfscope}%
\pgfpathrectangle{\pgfqpoint{0.647939in}{0.492442in}}{\pgfqpoint{4.273799in}{2.331163in}}%
\pgfusepath{clip}%
\pgfsetroundcap%
\pgfsetroundjoin%
\definecolor{currentfill}{rgb}{0.500000,0.500000,0.500000}%
\pgfsetfillcolor{currentfill}%
\pgfsetfillopacity{0.300000}%
\pgfsetlinewidth{0.301125pt}%
\definecolor{currentstroke}{rgb}{0.500000,0.500000,0.500000}%
\pgfsetstrokecolor{currentstroke}%
\pgfsetstrokeopacity{0.300000}%
\pgfsetdash{}{0pt}%
\pgfpathmoveto{\pgfqpoint{0.000000in}{0.000000in}}%
\pgfpathlineto{\pgfqpoint{0.000000in}{0.000000in}}%
\pgfpathclose%
\pgfusepath{stroke,fill}%
\end{pgfscope}%
\begin{pgfscope}%
\pgfpathrectangle{\pgfqpoint{0.647939in}{0.492442in}}{\pgfqpoint{4.273799in}{2.331163in}}%
\pgfusepath{clip}%
\pgfsetroundcap%
\pgfsetroundjoin%
\pgfsetlinewidth{0.301125pt}%
\definecolor{currentstroke}{rgb}{0.500000,0.500000,0.500000}%
\pgfsetstrokecolor{currentstroke}%
\pgfsetstrokeopacity{0.300000}%
\pgfsetdash{}{0pt}%
\pgfpathmoveto{\pgfqpoint{3.835257in}{1.453061in}}%
\pgfusepath{stroke}%
\end{pgfscope}%
\begin{pgfscope}%
\pgfpathrectangle{\pgfqpoint{0.647939in}{0.492442in}}{\pgfqpoint{4.273799in}{2.331163in}}%
\pgfusepath{clip}%
\pgfsetroundcap%
\pgfsetroundjoin%
\definecolor{currentfill}{rgb}{0.500000,0.500000,0.500000}%
\pgfsetfillcolor{currentfill}%
\pgfsetfillopacity{0.300000}%
\pgfsetlinewidth{0.301125pt}%
\definecolor{currentstroke}{rgb}{0.500000,0.500000,0.500000}%
\pgfsetstrokecolor{currentstroke}%
\pgfsetstrokeopacity{0.300000}%
\pgfsetdash{}{0pt}%
\pgfpathmoveto{\pgfqpoint{0.000000in}{0.000000in}}%
\pgfpathlineto{\pgfqpoint{0.000000in}{0.000000in}}%
\pgfpathclose%
\pgfusepath{stroke,fill}%
\end{pgfscope}%
\begin{pgfscope}%
\pgfpathrectangle{\pgfqpoint{0.647939in}{0.492442in}}{\pgfqpoint{4.273799in}{2.331163in}}%
\pgfusepath{clip}%
\pgfsetroundcap%
\pgfsetroundjoin%
\pgfsetlinewidth{0.301125pt}%
\definecolor{currentstroke}{rgb}{0.500000,0.500000,0.500000}%
\pgfsetstrokecolor{currentstroke}%
\pgfsetstrokeopacity{0.300000}%
\pgfsetdash{}{0pt}%
\pgfpathmoveto{\pgfqpoint{4.297922in}{1.543518in}}%
\pgfusepath{stroke}%
\end{pgfscope}%
\begin{pgfscope}%
\pgfpathrectangle{\pgfqpoint{0.647939in}{0.492442in}}{\pgfqpoint{4.273799in}{2.331163in}}%
\pgfusepath{clip}%
\pgfsetroundcap%
\pgfsetroundjoin%
\definecolor{currentfill}{rgb}{0.500000,0.500000,0.500000}%
\pgfsetfillcolor{currentfill}%
\pgfsetfillopacity{0.300000}%
\pgfsetlinewidth{0.301125pt}%
\definecolor{currentstroke}{rgb}{0.500000,0.500000,0.500000}%
\pgfsetstrokecolor{currentstroke}%
\pgfsetstrokeopacity{0.300000}%
\pgfsetdash{}{0pt}%
\pgfpathmoveto{\pgfqpoint{0.000000in}{0.000000in}}%
\pgfpathlineto{\pgfqpoint{0.000000in}{0.000000in}}%
\pgfpathclose%
\pgfusepath{stroke,fill}%
\end{pgfscope}%
\begin{pgfscope}%
\pgfpathrectangle{\pgfqpoint{0.647939in}{0.492442in}}{\pgfqpoint{4.273799in}{2.331163in}}%
\pgfusepath{clip}%
\pgfsetroundcap%
\pgfsetroundjoin%
\pgfsetlinewidth{0.301125pt}%
\definecolor{currentstroke}{rgb}{0.500000,0.500000,0.500000}%
\pgfsetstrokecolor{currentstroke}%
\pgfsetstrokeopacity{0.300000}%
\pgfsetdash{}{0pt}%
\pgfpathmoveto{\pgfqpoint{4.420669in}{1.782344in}}%
\pgfusepath{stroke}%
\end{pgfscope}%
\begin{pgfscope}%
\pgfpathrectangle{\pgfqpoint{0.647939in}{0.492442in}}{\pgfqpoint{4.273799in}{2.331163in}}%
\pgfusepath{clip}%
\pgfsetroundcap%
\pgfsetroundjoin%
\definecolor{currentfill}{rgb}{0.500000,0.500000,0.500000}%
\pgfsetfillcolor{currentfill}%
\pgfsetfillopacity{0.300000}%
\pgfsetlinewidth{0.301125pt}%
\definecolor{currentstroke}{rgb}{0.500000,0.500000,0.500000}%
\pgfsetstrokecolor{currentstroke}%
\pgfsetstrokeopacity{0.300000}%
\pgfsetdash{}{0pt}%
\pgfpathmoveto{\pgfqpoint{0.000000in}{0.000000in}}%
\pgfpathlineto{\pgfqpoint{0.000000in}{0.000000in}}%
\pgfpathclose%
\pgfusepath{stroke,fill}%
\end{pgfscope}%
\begin{pgfscope}%
\pgfpathrectangle{\pgfqpoint{0.647939in}{0.492442in}}{\pgfqpoint{4.273799in}{2.331163in}}%
\pgfusepath{clip}%
\pgfsetroundcap%
\pgfsetroundjoin%
\pgfsetlinewidth{0.301125pt}%
\definecolor{currentstroke}{rgb}{0.500000,0.500000,0.500000}%
\pgfsetstrokecolor{currentstroke}%
\pgfsetstrokeopacity{0.300000}%
\pgfsetdash{}{0pt}%
\pgfpathmoveto{\pgfqpoint{4.598008in}{1.884374in}}%
\pgfusepath{stroke}%
\end{pgfscope}%
\begin{pgfscope}%
\pgfpathrectangle{\pgfqpoint{0.647939in}{0.492442in}}{\pgfqpoint{4.273799in}{2.331163in}}%
\pgfusepath{clip}%
\pgfsetroundcap%
\pgfsetroundjoin%
\definecolor{currentfill}{rgb}{0.500000,0.500000,0.500000}%
\pgfsetfillcolor{currentfill}%
\pgfsetfillopacity{0.300000}%
\pgfsetlinewidth{0.301125pt}%
\definecolor{currentstroke}{rgb}{0.500000,0.500000,0.500000}%
\pgfsetstrokecolor{currentstroke}%
\pgfsetstrokeopacity{0.300000}%
\pgfsetdash{}{0pt}%
\pgfpathmoveto{\pgfqpoint{0.000000in}{0.000000in}}%
\pgfpathlineto{\pgfqpoint{0.000000in}{0.000000in}}%
\pgfpathclose%
\pgfusepath{stroke,fill}%
\end{pgfscope}%
\begin{pgfscope}%
\pgfpathrectangle{\pgfqpoint{0.647939in}{0.492442in}}{\pgfqpoint{4.273799in}{2.331163in}}%
\pgfusepath{clip}%
\pgfsetroundcap%
\pgfsetroundjoin%
\pgfsetlinewidth{0.301125pt}%
\definecolor{currentstroke}{rgb}{0.500000,0.500000,0.500000}%
\pgfsetstrokecolor{currentstroke}%
\pgfsetstrokeopacity{0.300000}%
\pgfsetdash{}{0pt}%
\pgfpathmoveto{\pgfqpoint{4.740717in}{1.794246in}}%
\pgfusepath{stroke}%
\end{pgfscope}%
\begin{pgfscope}%
\pgfpathrectangle{\pgfqpoint{0.647939in}{0.492442in}}{\pgfqpoint{4.273799in}{2.331163in}}%
\pgfusepath{clip}%
\pgfsetroundcap%
\pgfsetroundjoin%
\definecolor{currentfill}{rgb}{0.500000,0.500000,0.500000}%
\pgfsetfillcolor{currentfill}%
\pgfsetfillopacity{0.300000}%
\pgfsetlinewidth{0.301125pt}%
\definecolor{currentstroke}{rgb}{0.500000,0.500000,0.500000}%
\pgfsetstrokecolor{currentstroke}%
\pgfsetstrokeopacity{0.300000}%
\pgfsetdash{}{0pt}%
\pgfpathmoveto{\pgfqpoint{0.000000in}{0.000000in}}%
\pgfpathlineto{\pgfqpoint{0.000000in}{0.000000in}}%
\pgfpathclose%
\pgfusepath{stroke,fill}%
\end{pgfscope}%
\begin{pgfscope}%
\pgfpathrectangle{\pgfqpoint{0.647939in}{0.492442in}}{\pgfqpoint{4.273799in}{2.331163in}}%
\pgfusepath{clip}%
\pgfsetroundcap%
\pgfsetroundjoin%
\pgfsetlinewidth{0.301125pt}%
\definecolor{currentstroke}{rgb}{0.500000,0.500000,0.500000}%
\pgfsetstrokecolor{currentstroke}%
\pgfsetstrokeopacity{0.300000}%
\pgfsetdash{}{0pt}%
\pgfpathmoveto{\pgfqpoint{4.821829in}{1.960111in}}%
\pgfusepath{stroke}%
\end{pgfscope}%
\begin{pgfscope}%
\pgfpathrectangle{\pgfqpoint{0.647939in}{0.492442in}}{\pgfqpoint{4.273799in}{2.331163in}}%
\pgfusepath{clip}%
\pgfsetroundcap%
\pgfsetroundjoin%
\definecolor{currentfill}{rgb}{0.500000,0.500000,0.500000}%
\pgfsetfillcolor{currentfill}%
\pgfsetfillopacity{0.300000}%
\pgfsetlinewidth{0.301125pt}%
\definecolor{currentstroke}{rgb}{0.500000,0.500000,0.500000}%
\pgfsetstrokecolor{currentstroke}%
\pgfsetstrokeopacity{0.300000}%
\pgfsetdash{}{0pt}%
\pgfpathmoveto{\pgfqpoint{0.000000in}{0.000000in}}%
\pgfpathlineto{\pgfqpoint{0.000000in}{0.000000in}}%
\pgfpathclose%
\pgfusepath{stroke,fill}%
\end{pgfscope}%
\begin{pgfscope}%
\pgfpathrectangle{\pgfqpoint{0.647939in}{0.492442in}}{\pgfqpoint{4.273799in}{2.331163in}}%
\pgfusepath{clip}%
\pgfsetroundcap%
\pgfsetroundjoin%
\pgfsetlinewidth{0.301125pt}%
\definecolor{currentstroke}{rgb}{0.500000,0.500000,0.500000}%
\pgfsetstrokecolor{currentstroke}%
\pgfsetstrokeopacity{0.300000}%
\pgfsetdash{}{0pt}%
\pgfpathmoveto{\pgfqpoint{4.885954in}{1.968176in}}%
\pgfusepath{stroke}%
\end{pgfscope}%
\begin{pgfscope}%
\pgfpathrectangle{\pgfqpoint{0.647939in}{0.492442in}}{\pgfqpoint{4.273799in}{2.331163in}}%
\pgfusepath{clip}%
\pgfsetroundcap%
\pgfsetroundjoin%
\definecolor{currentfill}{rgb}{0.500000,0.500000,0.500000}%
\pgfsetfillcolor{currentfill}%
\pgfsetfillopacity{0.300000}%
\pgfsetlinewidth{0.301125pt}%
\definecolor{currentstroke}{rgb}{0.500000,0.500000,0.500000}%
\pgfsetstrokecolor{currentstroke}%
\pgfsetstrokeopacity{0.300000}%
\pgfsetdash{}{0pt}%
\pgfpathmoveto{\pgfqpoint{0.000000in}{0.000000in}}%
\pgfpathlineto{\pgfqpoint{0.000000in}{0.000000in}}%
\pgfpathclose%
\pgfusepath{stroke,fill}%
\end{pgfscope}%
\begin{pgfscope}%
\pgfpathrectangle{\pgfqpoint{0.647939in}{0.492442in}}{\pgfqpoint{4.273799in}{2.331163in}}%
\pgfusepath{clip}%
\pgfsetroundcap%
\pgfsetroundjoin%
\pgfsetlinewidth{0.301125pt}%
\definecolor{currentstroke}{rgb}{0.500000,0.500000,0.500000}%
\pgfsetstrokecolor{currentstroke}%
\pgfsetstrokeopacity{0.300000}%
\pgfsetdash{}{0pt}%
\pgfpathmoveto{\pgfqpoint{4.570484in}{2.673957in}}%
\pgfusepath{stroke}%
\end{pgfscope}%
\begin{pgfscope}%
\pgfpathrectangle{\pgfqpoint{0.647939in}{0.492442in}}{\pgfqpoint{4.273799in}{2.331163in}}%
\pgfusepath{clip}%
\pgfsetroundcap%
\pgfsetroundjoin%
\definecolor{currentfill}{rgb}{0.500000,0.500000,0.500000}%
\pgfsetfillcolor{currentfill}%
\pgfsetfillopacity{0.300000}%
\pgfsetlinewidth{0.301125pt}%
\definecolor{currentstroke}{rgb}{0.500000,0.500000,0.500000}%
\pgfsetstrokecolor{currentstroke}%
\pgfsetstrokeopacity{0.300000}%
\pgfsetdash{}{0pt}%
\pgfpathmoveto{\pgfqpoint{0.000000in}{0.000000in}}%
\pgfpathlineto{\pgfqpoint{0.000000in}{0.000000in}}%
\pgfpathclose%
\pgfusepath{stroke,fill}%
\end{pgfscope}%
\begin{pgfscope}%
\pgfpathrectangle{\pgfqpoint{0.647939in}{0.492442in}}{\pgfqpoint{4.273799in}{2.331163in}}%
\pgfusepath{clip}%
\pgfsetroundcap%
\pgfsetroundjoin%
\pgfsetlinewidth{0.301125pt}%
\definecolor{currentstroke}{rgb}{0.500000,0.500000,0.500000}%
\pgfsetstrokecolor{currentstroke}%
\pgfsetstrokeopacity{0.300000}%
\pgfsetdash{}{0pt}%
\pgfpathmoveto{\pgfqpoint{4.442024in}{2.627003in}}%
\pgfusepath{stroke}%
\end{pgfscope}%
\begin{pgfscope}%
\pgfpathrectangle{\pgfqpoint{0.647939in}{0.492442in}}{\pgfqpoint{4.273799in}{2.331163in}}%
\pgfusepath{clip}%
\pgfsetroundcap%
\pgfsetroundjoin%
\definecolor{currentfill}{rgb}{0.500000,0.500000,0.500000}%
\pgfsetfillcolor{currentfill}%
\pgfsetfillopacity{0.300000}%
\pgfsetlinewidth{0.301125pt}%
\definecolor{currentstroke}{rgb}{0.500000,0.500000,0.500000}%
\pgfsetstrokecolor{currentstroke}%
\pgfsetstrokeopacity{0.300000}%
\pgfsetdash{}{0pt}%
\pgfpathmoveto{\pgfqpoint{0.000000in}{0.000000in}}%
\pgfpathlineto{\pgfqpoint{0.000000in}{0.000000in}}%
\pgfpathclose%
\pgfusepath{stroke,fill}%
\end{pgfscope}%
\begin{pgfscope}%
\pgfpathrectangle{\pgfqpoint{0.647939in}{0.492442in}}{\pgfqpoint{4.273799in}{2.331163in}}%
\pgfusepath{clip}%
\pgfsetroundcap%
\pgfsetroundjoin%
\pgfsetlinewidth{0.301125pt}%
\definecolor{currentstroke}{rgb}{0.500000,0.500000,0.500000}%
\pgfsetstrokecolor{currentstroke}%
\pgfsetstrokeopacity{0.300000}%
\pgfsetdash{}{0pt}%
\pgfpathmoveto{\pgfqpoint{4.345820in}{2.564788in}}%
\pgfusepath{stroke}%
\end{pgfscope}%
\begin{pgfscope}%
\pgfpathrectangle{\pgfqpoint{0.647939in}{0.492442in}}{\pgfqpoint{4.273799in}{2.331163in}}%
\pgfusepath{clip}%
\pgfsetroundcap%
\pgfsetroundjoin%
\definecolor{currentfill}{rgb}{0.500000,0.500000,0.500000}%
\pgfsetfillcolor{currentfill}%
\pgfsetfillopacity{0.300000}%
\pgfsetlinewidth{0.301125pt}%
\definecolor{currentstroke}{rgb}{0.500000,0.500000,0.500000}%
\pgfsetstrokecolor{currentstroke}%
\pgfsetstrokeopacity{0.300000}%
\pgfsetdash{}{0pt}%
\pgfpathmoveto{\pgfqpoint{0.000000in}{0.000000in}}%
\pgfpathlineto{\pgfqpoint{0.000000in}{0.000000in}}%
\pgfpathclose%
\pgfusepath{stroke,fill}%
\end{pgfscope}%
\begin{pgfscope}%
\pgfpathrectangle{\pgfqpoint{0.647939in}{0.492442in}}{\pgfqpoint{4.273799in}{2.331163in}}%
\pgfusepath{clip}%
\pgfsetroundcap%
\pgfsetroundjoin%
\pgfsetlinewidth{0.301125pt}%
\definecolor{currentstroke}{rgb}{0.500000,0.500000,0.500000}%
\pgfsetstrokecolor{currentstroke}%
\pgfsetstrokeopacity{0.300000}%
\pgfsetdash{}{0pt}%
\pgfpathmoveto{\pgfqpoint{4.251615in}{2.507483in}}%
\pgfusepath{stroke}%
\end{pgfscope}%
\begin{pgfscope}%
\pgfpathrectangle{\pgfqpoint{0.647939in}{0.492442in}}{\pgfqpoint{4.273799in}{2.331163in}}%
\pgfusepath{clip}%
\pgfsetroundcap%
\pgfsetroundjoin%
\definecolor{currentfill}{rgb}{0.500000,0.500000,0.500000}%
\pgfsetfillcolor{currentfill}%
\pgfsetfillopacity{0.300000}%
\pgfsetlinewidth{0.301125pt}%
\definecolor{currentstroke}{rgb}{0.500000,0.500000,0.500000}%
\pgfsetstrokecolor{currentstroke}%
\pgfsetstrokeopacity{0.300000}%
\pgfsetdash{}{0pt}%
\pgfpathmoveto{\pgfqpoint{0.000000in}{0.000000in}}%
\pgfpathlineto{\pgfqpoint{0.000000in}{0.000000in}}%
\pgfpathclose%
\pgfusepath{stroke,fill}%
\end{pgfscope}%
\begin{pgfscope}%
\pgfpathrectangle{\pgfqpoint{0.647939in}{0.492442in}}{\pgfqpoint{4.273799in}{2.331163in}}%
\pgfusepath{clip}%
\pgfsetroundcap%
\pgfsetroundjoin%
\pgfsetlinewidth{0.301125pt}%
\definecolor{currentstroke}{rgb}{0.500000,0.500000,0.500000}%
\pgfsetstrokecolor{currentstroke}%
\pgfsetstrokeopacity{0.300000}%
\pgfsetdash{}{0pt}%
\pgfpathmoveto{\pgfqpoint{4.159915in}{2.452731in}}%
\pgfusepath{stroke}%
\end{pgfscope}%
\begin{pgfscope}%
\pgfpathrectangle{\pgfqpoint{0.647939in}{0.492442in}}{\pgfqpoint{4.273799in}{2.331163in}}%
\pgfusepath{clip}%
\pgfsetroundcap%
\pgfsetroundjoin%
\definecolor{currentfill}{rgb}{0.500000,0.500000,0.500000}%
\pgfsetfillcolor{currentfill}%
\pgfsetfillopacity{0.300000}%
\pgfsetlinewidth{0.301125pt}%
\definecolor{currentstroke}{rgb}{0.500000,0.500000,0.500000}%
\pgfsetstrokecolor{currentstroke}%
\pgfsetstrokeopacity{0.300000}%
\pgfsetdash{}{0pt}%
\pgfpathmoveto{\pgfqpoint{0.000000in}{0.000000in}}%
\pgfpathlineto{\pgfqpoint{0.000000in}{0.000000in}}%
\pgfpathclose%
\pgfusepath{stroke,fill}%
\end{pgfscope}%
\begin{pgfscope}%
\pgfpathrectangle{\pgfqpoint{0.647939in}{0.492442in}}{\pgfqpoint{4.273799in}{2.331163in}}%
\pgfusepath{clip}%
\pgfsetroundcap%
\pgfsetroundjoin%
\pgfsetlinewidth{0.301125pt}%
\definecolor{currentstroke}{rgb}{0.500000,0.500000,0.500000}%
\pgfsetstrokecolor{currentstroke}%
\pgfsetstrokeopacity{0.300000}%
\pgfsetdash{}{0pt}%
\pgfpathmoveto{\pgfqpoint{4.081340in}{2.374393in}}%
\pgfusepath{stroke}%
\end{pgfscope}%
\begin{pgfscope}%
\pgfpathrectangle{\pgfqpoint{0.647939in}{0.492442in}}{\pgfqpoint{4.273799in}{2.331163in}}%
\pgfusepath{clip}%
\pgfsetroundcap%
\pgfsetroundjoin%
\definecolor{currentfill}{rgb}{0.500000,0.500000,0.500000}%
\pgfsetfillcolor{currentfill}%
\pgfsetfillopacity{0.300000}%
\pgfsetlinewidth{0.301125pt}%
\definecolor{currentstroke}{rgb}{0.500000,0.500000,0.500000}%
\pgfsetstrokecolor{currentstroke}%
\pgfsetstrokeopacity{0.300000}%
\pgfsetdash{}{0pt}%
\pgfpathmoveto{\pgfqpoint{0.000000in}{0.000000in}}%
\pgfpathlineto{\pgfqpoint{0.000000in}{0.000000in}}%
\pgfpathclose%
\pgfusepath{stroke,fill}%
\end{pgfscope}%
\begin{pgfscope}%
\pgfpathrectangle{\pgfqpoint{0.647939in}{0.492442in}}{\pgfqpoint{4.273799in}{2.331163in}}%
\pgfusepath{clip}%
\pgfsetroundcap%
\pgfsetroundjoin%
\pgfsetlinewidth{0.301125pt}%
\definecolor{currentstroke}{rgb}{0.500000,0.500000,0.500000}%
\pgfsetstrokecolor{currentstroke}%
\pgfsetstrokeopacity{0.300000}%
\pgfsetdash{}{0pt}%
\pgfpathmoveto{\pgfqpoint{3.991972in}{2.322095in}}%
\pgfusepath{stroke}%
\end{pgfscope}%
\begin{pgfscope}%
\pgfpathrectangle{\pgfqpoint{0.647939in}{0.492442in}}{\pgfqpoint{4.273799in}{2.331163in}}%
\pgfusepath{clip}%
\pgfsetroundcap%
\pgfsetroundjoin%
\definecolor{currentfill}{rgb}{0.500000,0.500000,0.500000}%
\pgfsetfillcolor{currentfill}%
\pgfsetfillopacity{0.300000}%
\pgfsetlinewidth{0.301125pt}%
\definecolor{currentstroke}{rgb}{0.500000,0.500000,0.500000}%
\pgfsetstrokecolor{currentstroke}%
\pgfsetstrokeopacity{0.300000}%
\pgfsetdash{}{0pt}%
\pgfpathmoveto{\pgfqpoint{0.000000in}{0.000000in}}%
\pgfpathlineto{\pgfqpoint{0.000000in}{0.000000in}}%
\pgfpathclose%
\pgfusepath{stroke,fill}%
\end{pgfscope}%
\begin{pgfscope}%
\pgfpathrectangle{\pgfqpoint{0.647939in}{0.492442in}}{\pgfqpoint{4.273799in}{2.331163in}}%
\pgfusepath{clip}%
\pgfsetroundcap%
\pgfsetroundjoin%
\pgfsetlinewidth{0.301125pt}%
\definecolor{currentstroke}{rgb}{0.500000,0.500000,0.500000}%
\pgfsetstrokecolor{currentstroke}%
\pgfsetstrokeopacity{0.300000}%
\pgfsetdash{}{0pt}%
\pgfpathmoveto{\pgfqpoint{3.934393in}{2.064013in}}%
\pgfusepath{stroke}%
\end{pgfscope}%
\begin{pgfscope}%
\pgfpathrectangle{\pgfqpoint{0.647939in}{0.492442in}}{\pgfqpoint{4.273799in}{2.331163in}}%
\pgfusepath{clip}%
\pgfsetroundcap%
\pgfsetroundjoin%
\definecolor{currentfill}{rgb}{0.500000,0.500000,0.500000}%
\pgfsetfillcolor{currentfill}%
\pgfsetfillopacity{0.300000}%
\pgfsetlinewidth{0.301125pt}%
\definecolor{currentstroke}{rgb}{0.500000,0.500000,0.500000}%
\pgfsetstrokecolor{currentstroke}%
\pgfsetstrokeopacity{0.300000}%
\pgfsetdash{}{0pt}%
\pgfpathmoveto{\pgfqpoint{0.000000in}{0.000000in}}%
\pgfpathlineto{\pgfqpoint{0.000000in}{0.000000in}}%
\pgfpathclose%
\pgfusepath{stroke,fill}%
\end{pgfscope}%
\begin{pgfscope}%
\pgfpathrectangle{\pgfqpoint{0.647939in}{0.492442in}}{\pgfqpoint{4.273799in}{2.331163in}}%
\pgfusepath{clip}%
\pgfsetroundcap%
\pgfsetroundjoin%
\pgfsetlinewidth{0.301125pt}%
\definecolor{currentstroke}{rgb}{0.500000,0.500000,0.500000}%
\pgfsetstrokecolor{currentstroke}%
\pgfsetstrokeopacity{0.300000}%
\pgfsetdash{}{0pt}%
\pgfpathmoveto{\pgfqpoint{3.793851in}{2.321720in}}%
\pgfusepath{stroke}%
\end{pgfscope}%
\begin{pgfscope}%
\pgfpathrectangle{\pgfqpoint{0.647939in}{0.492442in}}{\pgfqpoint{4.273799in}{2.331163in}}%
\pgfusepath{clip}%
\pgfsetroundcap%
\pgfsetroundjoin%
\definecolor{currentfill}{rgb}{0.500000,0.500000,0.500000}%
\pgfsetfillcolor{currentfill}%
\pgfsetfillopacity{0.300000}%
\pgfsetlinewidth{0.301125pt}%
\definecolor{currentstroke}{rgb}{0.500000,0.500000,0.500000}%
\pgfsetstrokecolor{currentstroke}%
\pgfsetstrokeopacity{0.300000}%
\pgfsetdash{}{0pt}%
\pgfpathmoveto{\pgfqpoint{0.000000in}{0.000000in}}%
\pgfpathlineto{\pgfqpoint{0.000000in}{0.000000in}}%
\pgfpathclose%
\pgfusepath{stroke,fill}%
\end{pgfscope}%
\begin{pgfscope}%
\pgfpathrectangle{\pgfqpoint{0.647939in}{0.492442in}}{\pgfqpoint{4.273799in}{2.331163in}}%
\pgfusepath{clip}%
\pgfsetroundcap%
\pgfsetroundjoin%
\pgfsetlinewidth{0.301125pt}%
\definecolor{currentstroke}{rgb}{0.500000,0.500000,0.500000}%
\pgfsetstrokecolor{currentstroke}%
\pgfsetstrokeopacity{0.300000}%
\pgfsetdash{}{0pt}%
\pgfpathmoveto{\pgfqpoint{3.728572in}{2.219973in}}%
\pgfusepath{stroke}%
\end{pgfscope}%
\begin{pgfscope}%
\pgfpathrectangle{\pgfqpoint{0.647939in}{0.492442in}}{\pgfqpoint{4.273799in}{2.331163in}}%
\pgfusepath{clip}%
\pgfsetroundcap%
\pgfsetroundjoin%
\definecolor{currentfill}{rgb}{0.500000,0.500000,0.500000}%
\pgfsetfillcolor{currentfill}%
\pgfsetfillopacity{0.300000}%
\pgfsetlinewidth{0.301125pt}%
\definecolor{currentstroke}{rgb}{0.500000,0.500000,0.500000}%
\pgfsetstrokecolor{currentstroke}%
\pgfsetstrokeopacity{0.300000}%
\pgfsetdash{}{0pt}%
\pgfpathmoveto{\pgfqpoint{0.000000in}{0.000000in}}%
\pgfpathlineto{\pgfqpoint{0.000000in}{0.000000in}}%
\pgfpathclose%
\pgfusepath{stroke,fill}%
\end{pgfscope}%
\begin{pgfscope}%
\pgfpathrectangle{\pgfqpoint{0.647939in}{0.492442in}}{\pgfqpoint{4.273799in}{2.331163in}}%
\pgfusepath{clip}%
\pgfsetroundcap%
\pgfsetroundjoin%
\pgfsetlinewidth{0.301125pt}%
\definecolor{currentstroke}{rgb}{0.500000,0.500000,0.500000}%
\pgfsetstrokecolor{currentstroke}%
\pgfsetstrokeopacity{0.300000}%
\pgfsetdash{}{0pt}%
\pgfpathmoveto{\pgfqpoint{3.496911in}{2.601754in}}%
\pgfusepath{stroke}%
\end{pgfscope}%
\begin{pgfscope}%
\pgfpathrectangle{\pgfqpoint{0.647939in}{0.492442in}}{\pgfqpoint{4.273799in}{2.331163in}}%
\pgfusepath{clip}%
\pgfsetroundcap%
\pgfsetroundjoin%
\definecolor{currentfill}{rgb}{0.500000,0.500000,0.500000}%
\pgfsetfillcolor{currentfill}%
\pgfsetfillopacity{0.300000}%
\pgfsetlinewidth{0.301125pt}%
\definecolor{currentstroke}{rgb}{0.500000,0.500000,0.500000}%
\pgfsetstrokecolor{currentstroke}%
\pgfsetstrokeopacity{0.300000}%
\pgfsetdash{}{0pt}%
\pgfpathmoveto{\pgfqpoint{0.000000in}{0.000000in}}%
\pgfpathlineto{\pgfqpoint{0.000000in}{0.000000in}}%
\pgfpathclose%
\pgfusepath{stroke,fill}%
\end{pgfscope}%
\begin{pgfscope}%
\pgfpathrectangle{\pgfqpoint{0.647939in}{0.492442in}}{\pgfqpoint{4.273799in}{2.331163in}}%
\pgfusepath{clip}%
\pgfsetroundcap%
\pgfsetroundjoin%
\pgfsetlinewidth{0.301125pt}%
\definecolor{currentstroke}{rgb}{0.500000,0.500000,0.500000}%
\pgfsetstrokecolor{currentstroke}%
\pgfsetstrokeopacity{0.300000}%
\pgfsetdash{}{0pt}%
\pgfpathmoveto{\pgfqpoint{3.502607in}{2.072141in}}%
\pgfusepath{stroke}%
\end{pgfscope}%
\begin{pgfscope}%
\pgfpathrectangle{\pgfqpoint{0.647939in}{0.492442in}}{\pgfqpoint{4.273799in}{2.331163in}}%
\pgfusepath{clip}%
\pgfsetroundcap%
\pgfsetroundjoin%
\definecolor{currentfill}{rgb}{0.500000,0.500000,0.500000}%
\pgfsetfillcolor{currentfill}%
\pgfsetfillopacity{0.300000}%
\pgfsetlinewidth{0.301125pt}%
\definecolor{currentstroke}{rgb}{0.500000,0.500000,0.500000}%
\pgfsetstrokecolor{currentstroke}%
\pgfsetstrokeopacity{0.300000}%
\pgfsetdash{}{0pt}%
\pgfpathmoveto{\pgfqpoint{0.000000in}{0.000000in}}%
\pgfpathlineto{\pgfqpoint{0.000000in}{0.000000in}}%
\pgfpathclose%
\pgfusepath{stroke,fill}%
\end{pgfscope}%
\begin{pgfscope}%
\pgfpathrectangle{\pgfqpoint{0.647939in}{0.492442in}}{\pgfqpoint{4.273799in}{2.331163in}}%
\pgfusepath{clip}%
\pgfsetroundcap%
\pgfsetroundjoin%
\pgfsetlinewidth{0.301125pt}%
\definecolor{currentstroke}{rgb}{0.500000,0.500000,0.500000}%
\pgfsetstrokecolor{currentstroke}%
\pgfsetstrokeopacity{0.300000}%
\pgfsetdash{}{0pt}%
\pgfpathmoveto{\pgfqpoint{3.202082in}{2.657402in}}%
\pgfusepath{stroke}%
\end{pgfscope}%
\begin{pgfscope}%
\pgfpathrectangle{\pgfqpoint{0.647939in}{0.492442in}}{\pgfqpoint{4.273799in}{2.331163in}}%
\pgfusepath{clip}%
\pgfsetroundcap%
\pgfsetroundjoin%
\definecolor{currentfill}{rgb}{0.500000,0.500000,0.500000}%
\pgfsetfillcolor{currentfill}%
\pgfsetfillopacity{0.300000}%
\pgfsetlinewidth{0.301125pt}%
\definecolor{currentstroke}{rgb}{0.500000,0.500000,0.500000}%
\pgfsetstrokecolor{currentstroke}%
\pgfsetstrokeopacity{0.300000}%
\pgfsetdash{}{0pt}%
\pgfpathmoveto{\pgfqpoint{0.000000in}{0.000000in}}%
\pgfpathlineto{\pgfqpoint{0.000000in}{0.000000in}}%
\pgfpathclose%
\pgfusepath{stroke,fill}%
\end{pgfscope}%
\begin{pgfscope}%
\pgfpathrectangle{\pgfqpoint{0.647939in}{0.492442in}}{\pgfqpoint{4.273799in}{2.331163in}}%
\pgfusepath{clip}%
\pgfsetroundcap%
\pgfsetroundjoin%
\pgfsetlinewidth{0.301125pt}%
\definecolor{currentstroke}{rgb}{0.500000,0.500000,0.500000}%
\pgfsetstrokecolor{currentstroke}%
\pgfsetstrokeopacity{0.300000}%
\pgfsetdash{}{0pt}%
\pgfpathmoveto{\pgfqpoint{3.265459in}{2.295912in}}%
\pgfusepath{stroke}%
\end{pgfscope}%
\begin{pgfscope}%
\pgfpathrectangle{\pgfqpoint{0.647939in}{0.492442in}}{\pgfqpoint{4.273799in}{2.331163in}}%
\pgfusepath{clip}%
\pgfsetroundcap%
\pgfsetroundjoin%
\definecolor{currentfill}{rgb}{0.500000,0.500000,0.500000}%
\pgfsetfillcolor{currentfill}%
\pgfsetfillopacity{0.300000}%
\pgfsetlinewidth{0.301125pt}%
\definecolor{currentstroke}{rgb}{0.500000,0.500000,0.500000}%
\pgfsetstrokecolor{currentstroke}%
\pgfsetstrokeopacity{0.300000}%
\pgfsetdash{}{0pt}%
\pgfpathmoveto{\pgfqpoint{0.000000in}{0.000000in}}%
\pgfpathlineto{\pgfqpoint{0.000000in}{0.000000in}}%
\pgfpathclose%
\pgfusepath{stroke,fill}%
\end{pgfscope}%
\begin{pgfscope}%
\pgfpathrectangle{\pgfqpoint{0.647939in}{0.492442in}}{\pgfqpoint{4.273799in}{2.331163in}}%
\pgfusepath{clip}%
\pgfsetroundcap%
\pgfsetroundjoin%
\pgfsetlinewidth{0.301125pt}%
\definecolor{currentstroke}{rgb}{0.500000,0.500000,0.500000}%
\pgfsetstrokecolor{currentstroke}%
\pgfsetstrokeopacity{0.300000}%
\pgfsetdash{}{0pt}%
\pgfpathmoveto{\pgfqpoint{3.005662in}{2.538902in}}%
\pgfusepath{stroke}%
\end{pgfscope}%
\begin{pgfscope}%
\pgfpathrectangle{\pgfqpoint{0.647939in}{0.492442in}}{\pgfqpoint{4.273799in}{2.331163in}}%
\pgfusepath{clip}%
\pgfsetroundcap%
\pgfsetroundjoin%
\definecolor{currentfill}{rgb}{0.500000,0.500000,0.500000}%
\pgfsetfillcolor{currentfill}%
\pgfsetfillopacity{0.300000}%
\pgfsetlinewidth{0.301125pt}%
\definecolor{currentstroke}{rgb}{0.500000,0.500000,0.500000}%
\pgfsetstrokecolor{currentstroke}%
\pgfsetstrokeopacity{0.300000}%
\pgfsetdash{}{0pt}%
\pgfpathmoveto{\pgfqpoint{0.000000in}{0.000000in}}%
\pgfpathlineto{\pgfqpoint{0.000000in}{0.000000in}}%
\pgfpathclose%
\pgfusepath{stroke,fill}%
\end{pgfscope}%
\begin{pgfscope}%
\pgfpathrectangle{\pgfqpoint{0.647939in}{0.492442in}}{\pgfqpoint{4.273799in}{2.331163in}}%
\pgfusepath{clip}%
\pgfsetroundcap%
\pgfsetroundjoin%
\pgfsetlinewidth{0.301125pt}%
\definecolor{currentstroke}{rgb}{0.500000,0.500000,0.500000}%
\pgfsetstrokecolor{currentstroke}%
\pgfsetstrokeopacity{0.300000}%
\pgfsetdash{}{0pt}%
\pgfpathmoveto{\pgfqpoint{2.712934in}{2.695636in}}%
\pgfusepath{stroke}%
\end{pgfscope}%
\begin{pgfscope}%
\pgfpathrectangle{\pgfqpoint{0.647939in}{0.492442in}}{\pgfqpoint{4.273799in}{2.331163in}}%
\pgfusepath{clip}%
\pgfsetroundcap%
\pgfsetroundjoin%
\definecolor{currentfill}{rgb}{0.500000,0.500000,0.500000}%
\pgfsetfillcolor{currentfill}%
\pgfsetfillopacity{0.300000}%
\pgfsetlinewidth{0.301125pt}%
\definecolor{currentstroke}{rgb}{0.500000,0.500000,0.500000}%
\pgfsetstrokecolor{currentstroke}%
\pgfsetstrokeopacity{0.300000}%
\pgfsetdash{}{0pt}%
\pgfpathmoveto{\pgfqpoint{0.000000in}{0.000000in}}%
\pgfpathlineto{\pgfqpoint{0.000000in}{0.000000in}}%
\pgfpathclose%
\pgfusepath{stroke,fill}%
\end{pgfscope}%
\begin{pgfscope}%
\pgfpathrectangle{\pgfqpoint{0.647939in}{0.492442in}}{\pgfqpoint{4.273799in}{2.331163in}}%
\pgfusepath{clip}%
\pgfsetroundcap%
\pgfsetroundjoin%
\pgfsetlinewidth{0.301125pt}%
\definecolor{currentstroke}{rgb}{0.500000,0.500000,0.500000}%
\pgfsetstrokecolor{currentstroke}%
\pgfsetstrokeopacity{0.300000}%
\pgfsetdash{}{0pt}%
\pgfpathmoveto{\pgfqpoint{2.564963in}{2.743217in}}%
\pgfusepath{stroke}%
\end{pgfscope}%
\begin{pgfscope}%
\pgfpathrectangle{\pgfqpoint{0.647939in}{0.492442in}}{\pgfqpoint{4.273799in}{2.331163in}}%
\pgfusepath{clip}%
\pgfsetroundcap%
\pgfsetroundjoin%
\definecolor{currentfill}{rgb}{0.500000,0.500000,0.500000}%
\pgfsetfillcolor{currentfill}%
\pgfsetfillopacity{0.300000}%
\pgfsetlinewidth{0.301125pt}%
\definecolor{currentstroke}{rgb}{0.500000,0.500000,0.500000}%
\pgfsetstrokecolor{currentstroke}%
\pgfsetstrokeopacity{0.300000}%
\pgfsetdash{}{0pt}%
\pgfpathmoveto{\pgfqpoint{0.000000in}{0.000000in}}%
\pgfpathlineto{\pgfqpoint{0.000000in}{0.000000in}}%
\pgfpathclose%
\pgfusepath{stroke,fill}%
\end{pgfscope}%
\begin{pgfscope}%
\pgfpathrectangle{\pgfqpoint{0.647939in}{0.492442in}}{\pgfqpoint{4.273799in}{2.331163in}}%
\pgfusepath{clip}%
\pgfsetroundcap%
\pgfsetroundjoin%
\pgfsetlinewidth{0.301125pt}%
\definecolor{currentstroke}{rgb}{0.500000,0.500000,0.500000}%
\pgfsetstrokecolor{currentstroke}%
\pgfsetstrokeopacity{0.300000}%
\pgfsetdash{}{0pt}%
\pgfpathmoveto{\pgfqpoint{2.741768in}{2.582143in}}%
\pgfusepath{stroke}%
\end{pgfscope}%
\begin{pgfscope}%
\pgfpathrectangle{\pgfqpoint{0.647939in}{0.492442in}}{\pgfqpoint{4.273799in}{2.331163in}}%
\pgfusepath{clip}%
\pgfsetroundcap%
\pgfsetroundjoin%
\definecolor{currentfill}{rgb}{0.500000,0.500000,0.500000}%
\pgfsetfillcolor{currentfill}%
\pgfsetfillopacity{0.300000}%
\pgfsetlinewidth{0.301125pt}%
\definecolor{currentstroke}{rgb}{0.500000,0.500000,0.500000}%
\pgfsetstrokecolor{currentstroke}%
\pgfsetstrokeopacity{0.300000}%
\pgfsetdash{}{0pt}%
\pgfpathmoveto{\pgfqpoint{0.000000in}{0.000000in}}%
\pgfpathlineto{\pgfqpoint{0.000000in}{0.000000in}}%
\pgfpathclose%
\pgfusepath{stroke,fill}%
\end{pgfscope}%
\begin{pgfscope}%
\pgfpathrectangle{\pgfqpoint{0.647939in}{0.492442in}}{\pgfqpoint{4.273799in}{2.331163in}}%
\pgfusepath{clip}%
\pgfsetroundcap%
\pgfsetroundjoin%
\pgfsetlinewidth{0.301125pt}%
\definecolor{currentstroke}{rgb}{0.500000,0.500000,0.500000}%
\pgfsetstrokecolor{currentstroke}%
\pgfsetstrokeopacity{0.300000}%
\pgfsetdash{}{0pt}%
\pgfpathmoveto{\pgfqpoint{2.271784in}{2.730219in}}%
\pgfusepath{stroke}%
\end{pgfscope}%
\begin{pgfscope}%
\pgfpathrectangle{\pgfqpoint{0.647939in}{0.492442in}}{\pgfqpoint{4.273799in}{2.331163in}}%
\pgfusepath{clip}%
\pgfsetroundcap%
\pgfsetroundjoin%
\definecolor{currentfill}{rgb}{0.500000,0.500000,0.500000}%
\pgfsetfillcolor{currentfill}%
\pgfsetfillopacity{0.300000}%
\pgfsetlinewidth{0.301125pt}%
\definecolor{currentstroke}{rgb}{0.500000,0.500000,0.500000}%
\pgfsetstrokecolor{currentstroke}%
\pgfsetstrokeopacity{0.300000}%
\pgfsetdash{}{0pt}%
\pgfpathmoveto{\pgfqpoint{0.000000in}{0.000000in}}%
\pgfpathlineto{\pgfqpoint{0.000000in}{0.000000in}}%
\pgfpathclose%
\pgfusepath{stroke,fill}%
\end{pgfscope}%
\begin{pgfscope}%
\pgfpathrectangle{\pgfqpoint{0.647939in}{0.492442in}}{\pgfqpoint{4.273799in}{2.331163in}}%
\pgfusepath{clip}%
\pgfsetroundcap%
\pgfsetroundjoin%
\pgfsetlinewidth{0.301125pt}%
\definecolor{currentstroke}{rgb}{0.500000,0.500000,0.500000}%
\pgfsetstrokecolor{currentstroke}%
\pgfsetstrokeopacity{0.300000}%
\pgfsetdash{}{0pt}%
\pgfpathmoveto{\pgfqpoint{2.394442in}{2.627101in}}%
\pgfusepath{stroke}%
\end{pgfscope}%
\begin{pgfscope}%
\pgfpathrectangle{\pgfqpoint{0.647939in}{0.492442in}}{\pgfqpoint{4.273799in}{2.331163in}}%
\pgfusepath{clip}%
\pgfsetroundcap%
\pgfsetroundjoin%
\definecolor{currentfill}{rgb}{0.500000,0.500000,0.500000}%
\pgfsetfillcolor{currentfill}%
\pgfsetfillopacity{0.300000}%
\pgfsetlinewidth{0.301125pt}%
\definecolor{currentstroke}{rgb}{0.500000,0.500000,0.500000}%
\pgfsetstrokecolor{currentstroke}%
\pgfsetstrokeopacity{0.300000}%
\pgfsetdash{}{0pt}%
\pgfpathmoveto{\pgfqpoint{0.000000in}{0.000000in}}%
\pgfpathlineto{\pgfqpoint{0.000000in}{0.000000in}}%
\pgfpathclose%
\pgfusepath{stroke,fill}%
\end{pgfscope}%
\begin{pgfscope}%
\pgfpathrectangle{\pgfqpoint{0.647939in}{0.492442in}}{\pgfqpoint{4.273799in}{2.331163in}}%
\pgfusepath{clip}%
\pgfsetroundcap%
\pgfsetroundjoin%
\pgfsetlinewidth{0.301125pt}%
\definecolor{currentstroke}{rgb}{0.500000,0.500000,0.500000}%
\pgfsetstrokecolor{currentstroke}%
\pgfsetstrokeopacity{0.300000}%
\pgfsetdash{}{0pt}%
\pgfpathmoveto{\pgfqpoint{1.998338in}{2.664999in}}%
\pgfusepath{stroke}%
\end{pgfscope}%
\begin{pgfscope}%
\pgfpathrectangle{\pgfqpoint{0.647939in}{0.492442in}}{\pgfqpoint{4.273799in}{2.331163in}}%
\pgfusepath{clip}%
\pgfsetroundcap%
\pgfsetroundjoin%
\definecolor{currentfill}{rgb}{0.500000,0.500000,0.500000}%
\pgfsetfillcolor{currentfill}%
\pgfsetfillopacity{0.300000}%
\pgfsetlinewidth{0.301125pt}%
\definecolor{currentstroke}{rgb}{0.500000,0.500000,0.500000}%
\pgfsetstrokecolor{currentstroke}%
\pgfsetstrokeopacity{0.300000}%
\pgfsetdash{}{0pt}%
\pgfpathmoveto{\pgfqpoint{0.000000in}{0.000000in}}%
\pgfpathlineto{\pgfqpoint{0.000000in}{0.000000in}}%
\pgfpathclose%
\pgfusepath{stroke,fill}%
\end{pgfscope}%
\begin{pgfscope}%
\pgfpathrectangle{\pgfqpoint{0.647939in}{0.492442in}}{\pgfqpoint{4.273799in}{2.331163in}}%
\pgfusepath{clip}%
\pgfsetroundcap%
\pgfsetroundjoin%
\pgfsetlinewidth{0.301125pt}%
\definecolor{currentstroke}{rgb}{0.500000,0.500000,0.500000}%
\pgfsetstrokecolor{currentstroke}%
\pgfsetstrokeopacity{0.300000}%
\pgfsetdash{}{0pt}%
\pgfpathmoveto{\pgfqpoint{2.093814in}{2.574183in}}%
\pgfusepath{stroke}%
\end{pgfscope}%
\begin{pgfscope}%
\pgfpathrectangle{\pgfqpoint{0.647939in}{0.492442in}}{\pgfqpoint{4.273799in}{2.331163in}}%
\pgfusepath{clip}%
\pgfsetroundcap%
\pgfsetroundjoin%
\definecolor{currentfill}{rgb}{0.500000,0.500000,0.500000}%
\pgfsetfillcolor{currentfill}%
\pgfsetfillopacity{0.300000}%
\pgfsetlinewidth{0.301125pt}%
\definecolor{currentstroke}{rgb}{0.500000,0.500000,0.500000}%
\pgfsetstrokecolor{currentstroke}%
\pgfsetstrokeopacity{0.300000}%
\pgfsetdash{}{0pt}%
\pgfpathmoveto{\pgfqpoint{0.000000in}{0.000000in}}%
\pgfpathlineto{\pgfqpoint{0.000000in}{0.000000in}}%
\pgfpathclose%
\pgfusepath{stroke,fill}%
\end{pgfscope}%
\begin{pgfscope}%
\pgfpathrectangle{\pgfqpoint{0.647939in}{0.492442in}}{\pgfqpoint{4.273799in}{2.331163in}}%
\pgfusepath{clip}%
\pgfsetroundcap%
\pgfsetroundjoin%
\pgfsetlinewidth{0.301125pt}%
\definecolor{currentstroke}{rgb}{0.500000,0.500000,0.500000}%
\pgfsetstrokecolor{currentstroke}%
\pgfsetstrokeopacity{0.300000}%
\pgfsetdash{}{0pt}%
\pgfpathmoveto{\pgfqpoint{1.894610in}{2.514237in}}%
\pgfusepath{stroke}%
\end{pgfscope}%
\begin{pgfscope}%
\pgfpathrectangle{\pgfqpoint{0.647939in}{0.492442in}}{\pgfqpoint{4.273799in}{2.331163in}}%
\pgfusepath{clip}%
\pgfsetroundcap%
\pgfsetroundjoin%
\definecolor{currentfill}{rgb}{0.500000,0.500000,0.500000}%
\pgfsetfillcolor{currentfill}%
\pgfsetfillopacity{0.300000}%
\pgfsetlinewidth{0.301125pt}%
\definecolor{currentstroke}{rgb}{0.500000,0.500000,0.500000}%
\pgfsetstrokecolor{currentstroke}%
\pgfsetstrokeopacity{0.300000}%
\pgfsetdash{}{0pt}%
\pgfpathmoveto{\pgfqpoint{0.000000in}{0.000000in}}%
\pgfpathlineto{\pgfqpoint{0.000000in}{0.000000in}}%
\pgfpathclose%
\pgfusepath{stroke,fill}%
\end{pgfscope}%
\begin{pgfscope}%
\pgfpathrectangle{\pgfqpoint{0.647939in}{0.492442in}}{\pgfqpoint{4.273799in}{2.331163in}}%
\pgfusepath{clip}%
\pgfsetroundcap%
\pgfsetroundjoin%
\pgfsetlinewidth{0.301125pt}%
\definecolor{currentstroke}{rgb}{0.500000,0.500000,0.500000}%
\pgfsetstrokecolor{currentstroke}%
\pgfsetstrokeopacity{0.300000}%
\pgfsetdash{}{0pt}%
\pgfpathmoveto{\pgfqpoint{1.777621in}{2.404524in}}%
\pgfusepath{stroke}%
\end{pgfscope}%
\begin{pgfscope}%
\pgfpathrectangle{\pgfqpoint{0.647939in}{0.492442in}}{\pgfqpoint{4.273799in}{2.331163in}}%
\pgfusepath{clip}%
\pgfsetroundcap%
\pgfsetroundjoin%
\definecolor{currentfill}{rgb}{0.500000,0.500000,0.500000}%
\pgfsetfillcolor{currentfill}%
\pgfsetfillopacity{0.300000}%
\pgfsetlinewidth{0.301125pt}%
\definecolor{currentstroke}{rgb}{0.500000,0.500000,0.500000}%
\pgfsetstrokecolor{currentstroke}%
\pgfsetstrokeopacity{0.300000}%
\pgfsetdash{}{0pt}%
\pgfpathmoveto{\pgfqpoint{0.000000in}{0.000000in}}%
\pgfpathlineto{\pgfqpoint{0.000000in}{0.000000in}}%
\pgfpathclose%
\pgfusepath{stroke,fill}%
\end{pgfscope}%
\begin{pgfscope}%
\pgfpathrectangle{\pgfqpoint{0.647939in}{0.492442in}}{\pgfqpoint{4.273799in}{2.331163in}}%
\pgfusepath{clip}%
\pgfsetroundcap%
\pgfsetroundjoin%
\pgfsetlinewidth{0.301125pt}%
\definecolor{currentstroke}{rgb}{0.500000,0.500000,0.500000}%
\pgfsetstrokecolor{currentstroke}%
\pgfsetstrokeopacity{0.300000}%
\pgfsetdash{}{0pt}%
\pgfpathmoveto{\pgfqpoint{1.566087in}{2.349830in}}%
\pgfusepath{stroke}%
\end{pgfscope}%
\begin{pgfscope}%
\pgfpathrectangle{\pgfqpoint{0.647939in}{0.492442in}}{\pgfqpoint{4.273799in}{2.331163in}}%
\pgfusepath{clip}%
\pgfsetroundcap%
\pgfsetroundjoin%
\definecolor{currentfill}{rgb}{0.500000,0.500000,0.500000}%
\pgfsetfillcolor{currentfill}%
\pgfsetfillopacity{0.300000}%
\pgfsetlinewidth{0.301125pt}%
\definecolor{currentstroke}{rgb}{0.500000,0.500000,0.500000}%
\pgfsetstrokecolor{currentstroke}%
\pgfsetstrokeopacity{0.300000}%
\pgfsetdash{}{0pt}%
\pgfpathmoveto{\pgfqpoint{0.000000in}{0.000000in}}%
\pgfpathlineto{\pgfqpoint{0.000000in}{0.000000in}}%
\pgfpathclose%
\pgfusepath{stroke,fill}%
\end{pgfscope}%
\begin{pgfscope}%
\pgfpathrectangle{\pgfqpoint{0.647939in}{0.492442in}}{\pgfqpoint{4.273799in}{2.331163in}}%
\pgfusepath{clip}%
\pgfsetroundcap%
\pgfsetroundjoin%
\pgfsetlinewidth{0.301125pt}%
\definecolor{currentstroke}{rgb}{0.500000,0.500000,0.500000}%
\pgfsetstrokecolor{currentstroke}%
\pgfsetstrokeopacity{0.300000}%
\pgfsetdash{}{0pt}%
\pgfpathmoveto{\pgfqpoint{1.407104in}{2.314653in}}%
\pgfusepath{stroke}%
\end{pgfscope}%
\begin{pgfscope}%
\pgfpathrectangle{\pgfqpoint{0.647939in}{0.492442in}}{\pgfqpoint{4.273799in}{2.331163in}}%
\pgfusepath{clip}%
\pgfsetroundcap%
\pgfsetroundjoin%
\definecolor{currentfill}{rgb}{0.500000,0.500000,0.500000}%
\pgfsetfillcolor{currentfill}%
\pgfsetfillopacity{0.300000}%
\pgfsetlinewidth{0.301125pt}%
\definecolor{currentstroke}{rgb}{0.500000,0.500000,0.500000}%
\pgfsetstrokecolor{currentstroke}%
\pgfsetstrokeopacity{0.300000}%
\pgfsetdash{}{0pt}%
\pgfpathmoveto{\pgfqpoint{0.000000in}{0.000000in}}%
\pgfpathlineto{\pgfqpoint{0.000000in}{0.000000in}}%
\pgfpathclose%
\pgfusepath{stroke,fill}%
\end{pgfscope}%
\begin{pgfscope}%
\pgfpathrectangle{\pgfqpoint{0.647939in}{0.492442in}}{\pgfqpoint{4.273799in}{2.331163in}}%
\pgfusepath{clip}%
\pgfsetroundcap%
\pgfsetroundjoin%
\pgfsetlinewidth{0.301125pt}%
\definecolor{currentstroke}{rgb}{0.500000,0.500000,0.500000}%
\pgfsetstrokecolor{currentstroke}%
\pgfsetstrokeopacity{0.300000}%
\pgfsetdash{}{0pt}%
\pgfpathmoveto{\pgfqpoint{1.247912in}{1.794550in}}%
\pgfusepath{stroke}%
\end{pgfscope}%
\begin{pgfscope}%
\pgfpathrectangle{\pgfqpoint{0.647939in}{0.492442in}}{\pgfqpoint{4.273799in}{2.331163in}}%
\pgfusepath{clip}%
\pgfsetroundcap%
\pgfsetroundjoin%
\definecolor{currentfill}{rgb}{0.500000,0.500000,0.500000}%
\pgfsetfillcolor{currentfill}%
\pgfsetfillopacity{0.300000}%
\pgfsetlinewidth{0.301125pt}%
\definecolor{currentstroke}{rgb}{0.500000,0.500000,0.500000}%
\pgfsetstrokecolor{currentstroke}%
\pgfsetstrokeopacity{0.300000}%
\pgfsetdash{}{0pt}%
\pgfpathmoveto{\pgfqpoint{0.000000in}{0.000000in}}%
\pgfpathlineto{\pgfqpoint{0.000000in}{0.000000in}}%
\pgfpathclose%
\pgfusepath{stroke,fill}%
\end{pgfscope}%
\begin{pgfscope}%
\pgfpathrectangle{\pgfqpoint{0.647939in}{0.492442in}}{\pgfqpoint{4.273799in}{2.331163in}}%
\pgfusepath{clip}%
\pgfsetroundcap%
\pgfsetroundjoin%
\pgfsetlinewidth{0.301125pt}%
\definecolor{currentstroke}{rgb}{0.500000,0.500000,0.500000}%
\pgfsetstrokecolor{currentstroke}%
\pgfsetstrokeopacity{0.300000}%
\pgfsetdash{}{0pt}%
\pgfpathmoveto{\pgfqpoint{1.155581in}{2.153665in}}%
\pgfusepath{stroke}%
\end{pgfscope}%
\begin{pgfscope}%
\pgfpathrectangle{\pgfqpoint{0.647939in}{0.492442in}}{\pgfqpoint{4.273799in}{2.331163in}}%
\pgfusepath{clip}%
\pgfsetroundcap%
\pgfsetroundjoin%
\definecolor{currentfill}{rgb}{0.500000,0.500000,0.500000}%
\pgfsetfillcolor{currentfill}%
\pgfsetfillopacity{0.300000}%
\pgfsetlinewidth{0.301125pt}%
\definecolor{currentstroke}{rgb}{0.500000,0.500000,0.500000}%
\pgfsetstrokecolor{currentstroke}%
\pgfsetstrokeopacity{0.300000}%
\pgfsetdash{}{0pt}%
\pgfpathmoveto{\pgfqpoint{0.000000in}{0.000000in}}%
\pgfpathlineto{\pgfqpoint{0.000000in}{0.000000in}}%
\pgfpathclose%
\pgfusepath{stroke,fill}%
\end{pgfscope}%
\begin{pgfscope}%
\pgfpathrectangle{\pgfqpoint{0.647939in}{0.492442in}}{\pgfqpoint{4.273799in}{2.331163in}}%
\pgfusepath{clip}%
\pgfsetroundcap%
\pgfsetroundjoin%
\pgfsetlinewidth{0.301125pt}%
\definecolor{currentstroke}{rgb}{0.500000,0.500000,0.500000}%
\pgfsetstrokecolor{currentstroke}%
\pgfsetstrokeopacity{0.300000}%
\pgfsetdash{}{0pt}%
\pgfpathmoveto{\pgfqpoint{1.028704in}{1.893526in}}%
\pgfusepath{stroke}%
\end{pgfscope}%
\begin{pgfscope}%
\pgfpathrectangle{\pgfqpoint{0.647939in}{0.492442in}}{\pgfqpoint{4.273799in}{2.331163in}}%
\pgfusepath{clip}%
\pgfsetroundcap%
\pgfsetroundjoin%
\definecolor{currentfill}{rgb}{0.500000,0.500000,0.500000}%
\pgfsetfillcolor{currentfill}%
\pgfsetfillopacity{0.300000}%
\pgfsetlinewidth{0.301125pt}%
\definecolor{currentstroke}{rgb}{0.500000,0.500000,0.500000}%
\pgfsetstrokecolor{currentstroke}%
\pgfsetstrokeopacity{0.300000}%
\pgfsetdash{}{0pt}%
\pgfpathmoveto{\pgfqpoint{0.000000in}{0.000000in}}%
\pgfpathlineto{\pgfqpoint{0.000000in}{0.000000in}}%
\pgfpathclose%
\pgfusepath{stroke,fill}%
\end{pgfscope}%
\begin{pgfscope}%
\pgfpathrectangle{\pgfqpoint{0.647939in}{0.492442in}}{\pgfqpoint{4.273799in}{2.331163in}}%
\pgfusepath{clip}%
\pgfsetroundcap%
\pgfsetroundjoin%
\pgfsetlinewidth{0.301125pt}%
\definecolor{currentstroke}{rgb}{0.500000,0.500000,0.500000}%
\pgfsetstrokecolor{currentstroke}%
\pgfsetstrokeopacity{0.300000}%
\pgfsetdash{}{0pt}%
\pgfpathmoveto{\pgfqpoint{0.921909in}{2.099941in}}%
\pgfusepath{stroke}%
\end{pgfscope}%
\begin{pgfscope}%
\pgfpathrectangle{\pgfqpoint{0.647939in}{0.492442in}}{\pgfqpoint{4.273799in}{2.331163in}}%
\pgfusepath{clip}%
\pgfsetroundcap%
\pgfsetroundjoin%
\definecolor{currentfill}{rgb}{0.500000,0.500000,0.500000}%
\pgfsetfillcolor{currentfill}%
\pgfsetfillopacity{0.300000}%
\pgfsetlinewidth{0.301125pt}%
\definecolor{currentstroke}{rgb}{0.500000,0.500000,0.500000}%
\pgfsetstrokecolor{currentstroke}%
\pgfsetstrokeopacity{0.300000}%
\pgfsetdash{}{0pt}%
\pgfpathmoveto{\pgfqpoint{0.000000in}{0.000000in}}%
\pgfpathlineto{\pgfqpoint{0.000000in}{0.000000in}}%
\pgfpathclose%
\pgfusepath{stroke,fill}%
\end{pgfscope}%
\begin{pgfscope}%
\pgfpathrectangle{\pgfqpoint{0.647939in}{0.492442in}}{\pgfqpoint{4.273799in}{2.331163in}}%
\pgfusepath{clip}%
\pgfsetroundcap%
\pgfsetroundjoin%
\pgfsetlinewidth{0.301125pt}%
\definecolor{currentstroke}{rgb}{0.500000,0.500000,0.500000}%
\pgfsetstrokecolor{currentstroke}%
\pgfsetstrokeopacity{0.300000}%
\pgfsetdash{}{0pt}%
\pgfpathmoveto{\pgfqpoint{0.809422in}{1.944076in}}%
\pgfusepath{stroke}%
\end{pgfscope}%
\begin{pgfscope}%
\pgfpathrectangle{\pgfqpoint{0.647939in}{0.492442in}}{\pgfqpoint{4.273799in}{2.331163in}}%
\pgfusepath{clip}%
\pgfsetroundcap%
\pgfsetroundjoin%
\definecolor{currentfill}{rgb}{0.500000,0.500000,0.500000}%
\pgfsetfillcolor{currentfill}%
\pgfsetfillopacity{0.300000}%
\pgfsetlinewidth{0.301125pt}%
\definecolor{currentstroke}{rgb}{0.500000,0.500000,0.500000}%
\pgfsetstrokecolor{currentstroke}%
\pgfsetstrokeopacity{0.300000}%
\pgfsetdash{}{0pt}%
\pgfpathmoveto{\pgfqpoint{0.000000in}{0.000000in}}%
\pgfpathlineto{\pgfqpoint{0.000000in}{0.000000in}}%
\pgfpathclose%
\pgfusepath{stroke,fill}%
\end{pgfscope}%
\begin{pgfscope}%
\pgfpathrectangle{\pgfqpoint{0.647939in}{0.492442in}}{\pgfqpoint{4.273799in}{2.331163in}}%
\pgfusepath{clip}%
\pgfsetroundcap%
\pgfsetroundjoin%
\pgfsetlinewidth{0.301125pt}%
\definecolor{currentstroke}{rgb}{0.500000,0.500000,0.500000}%
\pgfsetstrokecolor{currentstroke}%
\pgfsetstrokeopacity{0.300000}%
\pgfsetdash{}{0pt}%
\pgfpathmoveto{\pgfqpoint{0.703835in}{2.150970in}}%
\pgfusepath{stroke}%
\end{pgfscope}%
\begin{pgfscope}%
\pgfpathrectangle{\pgfqpoint{0.647939in}{0.492442in}}{\pgfqpoint{4.273799in}{2.331163in}}%
\pgfusepath{clip}%
\pgfsetroundcap%
\pgfsetroundjoin%
\definecolor{currentfill}{rgb}{0.500000,0.500000,0.500000}%
\pgfsetfillcolor{currentfill}%
\pgfsetfillopacity{0.300000}%
\pgfsetlinewidth{0.301125pt}%
\definecolor{currentstroke}{rgb}{0.500000,0.500000,0.500000}%
\pgfsetstrokecolor{currentstroke}%
\pgfsetstrokeopacity{0.300000}%
\pgfsetdash{}{0pt}%
\pgfpathmoveto{\pgfqpoint{0.000000in}{0.000000in}}%
\pgfpathlineto{\pgfqpoint{0.000000in}{0.000000in}}%
\pgfpathclose%
\pgfusepath{stroke,fill}%
\end{pgfscope}%
\begin{pgfscope}%
\pgfpathrectangle{\pgfqpoint{0.647939in}{0.492442in}}{\pgfqpoint{4.273799in}{2.331163in}}%
\pgfusepath{clip}%
\pgfsetroundcap%
\pgfsetroundjoin%
\pgfsetlinewidth{0.301125pt}%
\definecolor{currentstroke}{rgb}{0.500000,0.500000,0.500000}%
\pgfsetstrokecolor{currentstroke}%
\pgfsetstrokeopacity{0.300000}%
\pgfsetdash{}{0pt}%
\pgfpathmoveto{\pgfqpoint{0.652973in}{2.086604in}}%
\pgfusepath{stroke}%
\end{pgfscope}%
\begin{pgfscope}%
\pgfpathrectangle{\pgfqpoint{0.647939in}{0.492442in}}{\pgfqpoint{4.273799in}{2.331163in}}%
\pgfusepath{clip}%
\pgfsetroundcap%
\pgfsetroundjoin%
\definecolor{currentfill}{rgb}{0.500000,0.500000,0.500000}%
\pgfsetfillcolor{currentfill}%
\pgfsetfillopacity{0.300000}%
\pgfsetlinewidth{0.301125pt}%
\definecolor{currentstroke}{rgb}{0.500000,0.500000,0.500000}%
\pgfsetstrokecolor{currentstroke}%
\pgfsetstrokeopacity{0.300000}%
\pgfsetdash{}{0pt}%
\pgfpathmoveto{\pgfqpoint{0.000000in}{0.000000in}}%
\pgfpathlineto{\pgfqpoint{0.000000in}{0.000000in}}%
\pgfpathclose%
\pgfusepath{stroke,fill}%
\end{pgfscope}%
\begin{pgfscope}%
\pgfpathrectangle{\pgfqpoint{0.647939in}{0.492442in}}{\pgfqpoint{4.273799in}{2.331163in}}%
\pgfusepath{clip}%
\pgfsetroundcap%
\pgfsetroundjoin%
\pgfsetlinewidth{0.301125pt}%
\definecolor{currentstroke}{rgb}{0.500000,0.500000,0.500000}%
\pgfsetstrokecolor{currentstroke}%
\pgfsetstrokeopacity{0.300000}%
\pgfsetdash{}{0pt}%
\pgfpathmoveto{\pgfqpoint{1.708682in}{1.094099in}}%
\pgfusepath{stroke}%
\end{pgfscope}%
\begin{pgfscope}%
\pgfpathrectangle{\pgfqpoint{0.647939in}{0.492442in}}{\pgfqpoint{4.273799in}{2.331163in}}%
\pgfusepath{clip}%
\pgfsetroundcap%
\pgfsetroundjoin%
\definecolor{currentfill}{rgb}{0.500000,0.500000,0.500000}%
\pgfsetfillcolor{currentfill}%
\pgfsetfillopacity{0.300000}%
\pgfsetlinewidth{0.301125pt}%
\definecolor{currentstroke}{rgb}{0.500000,0.500000,0.500000}%
\pgfsetstrokecolor{currentstroke}%
\pgfsetstrokeopacity{0.300000}%
\pgfsetdash{}{0pt}%
\pgfpathmoveto{\pgfqpoint{0.000000in}{0.000000in}}%
\pgfpathlineto{\pgfqpoint{0.000000in}{0.000000in}}%
\pgfpathclose%
\pgfusepath{stroke,fill}%
\end{pgfscope}%
\begin{pgfscope}%
\pgfpathrectangle{\pgfqpoint{0.647939in}{0.492442in}}{\pgfqpoint{4.273799in}{2.331163in}}%
\pgfusepath{clip}%
\pgfsetroundcap%
\pgfsetroundjoin%
\pgfsetlinewidth{0.301125pt}%
\definecolor{currentstroke}{rgb}{0.500000,0.500000,0.500000}%
\pgfsetstrokecolor{currentstroke}%
\pgfsetstrokeopacity{0.300000}%
\pgfsetdash{}{0pt}%
\pgfpathmoveto{\pgfqpoint{3.764251in}{0.755878in}}%
\pgfusepath{stroke}%
\end{pgfscope}%
\begin{pgfscope}%
\pgfpathrectangle{\pgfqpoint{0.647939in}{0.492442in}}{\pgfqpoint{4.273799in}{2.331163in}}%
\pgfusepath{clip}%
\pgfsetroundcap%
\pgfsetroundjoin%
\definecolor{currentfill}{rgb}{0.500000,0.500000,0.500000}%
\pgfsetfillcolor{currentfill}%
\pgfsetfillopacity{0.300000}%
\pgfsetlinewidth{0.301125pt}%
\definecolor{currentstroke}{rgb}{0.500000,0.500000,0.500000}%
\pgfsetstrokecolor{currentstroke}%
\pgfsetstrokeopacity{0.300000}%
\pgfsetdash{}{0pt}%
\pgfpathmoveto{\pgfqpoint{0.000000in}{0.000000in}}%
\pgfpathlineto{\pgfqpoint{0.000000in}{0.000000in}}%
\pgfpathclose%
\pgfusepath{stroke,fill}%
\end{pgfscope}%
\begin{pgfscope}%
\pgfpathrectangle{\pgfqpoint{0.647939in}{0.492442in}}{\pgfqpoint{4.273799in}{2.331163in}}%
\pgfusepath{clip}%
\pgfsetroundcap%
\pgfsetroundjoin%
\pgfsetlinewidth{0.301125pt}%
\definecolor{currentstroke}{rgb}{0.500000,0.500000,0.500000}%
\pgfsetstrokecolor{currentstroke}%
\pgfsetstrokeopacity{0.300000}%
\pgfsetdash{}{0pt}%
\pgfpathmoveto{\pgfqpoint{4.289885in}{1.693825in}}%
\pgfusepath{stroke}%
\end{pgfscope}%
\begin{pgfscope}%
\pgfpathrectangle{\pgfqpoint{0.647939in}{0.492442in}}{\pgfqpoint{4.273799in}{2.331163in}}%
\pgfusepath{clip}%
\pgfsetroundcap%
\pgfsetroundjoin%
\definecolor{currentfill}{rgb}{0.500000,0.500000,0.500000}%
\pgfsetfillcolor{currentfill}%
\pgfsetfillopacity{0.300000}%
\pgfsetlinewidth{0.301125pt}%
\definecolor{currentstroke}{rgb}{0.500000,0.500000,0.500000}%
\pgfsetstrokecolor{currentstroke}%
\pgfsetstrokeopacity{0.300000}%
\pgfsetdash{}{0pt}%
\pgfpathmoveto{\pgfqpoint{0.000000in}{0.000000in}}%
\pgfpathlineto{\pgfqpoint{0.000000in}{0.000000in}}%
\pgfpathclose%
\pgfusepath{stroke,fill}%
\end{pgfscope}%
\begin{pgfscope}%
\pgfpathrectangle{\pgfqpoint{0.647939in}{0.492442in}}{\pgfqpoint{4.273799in}{2.331163in}}%
\pgfusepath{clip}%
\pgfsetroundcap%
\pgfsetroundjoin%
\pgfsetlinewidth{0.301125pt}%
\definecolor{currentstroke}{rgb}{0.500000,0.500000,0.500000}%
\pgfsetstrokecolor{currentstroke}%
\pgfsetstrokeopacity{0.300000}%
\pgfsetdash{}{0pt}%
\pgfpathmoveto{\pgfqpoint{4.408124in}{2.014105in}}%
\pgfusepath{stroke}%
\end{pgfscope}%
\begin{pgfscope}%
\pgfpathrectangle{\pgfqpoint{0.647939in}{0.492442in}}{\pgfqpoint{4.273799in}{2.331163in}}%
\pgfusepath{clip}%
\pgfsetroundcap%
\pgfsetroundjoin%
\definecolor{currentfill}{rgb}{0.500000,0.500000,0.500000}%
\pgfsetfillcolor{currentfill}%
\pgfsetfillopacity{0.300000}%
\pgfsetlinewidth{0.301125pt}%
\definecolor{currentstroke}{rgb}{0.500000,0.500000,0.500000}%
\pgfsetstrokecolor{currentstroke}%
\pgfsetstrokeopacity{0.300000}%
\pgfsetdash{}{0pt}%
\pgfpathmoveto{\pgfqpoint{0.000000in}{0.000000in}}%
\pgfpathlineto{\pgfqpoint{0.000000in}{0.000000in}}%
\pgfpathclose%
\pgfusepath{stroke,fill}%
\end{pgfscope}%
\begin{pgfscope}%
\pgfpathrectangle{\pgfqpoint{0.647939in}{0.492442in}}{\pgfqpoint{4.273799in}{2.331163in}}%
\pgfusepath{clip}%
\pgfsetroundcap%
\pgfsetroundjoin%
\pgfsetlinewidth{0.301125pt}%
\definecolor{currentstroke}{rgb}{0.500000,0.500000,0.500000}%
\pgfsetstrokecolor{currentstroke}%
\pgfsetstrokeopacity{0.300000}%
\pgfsetdash{}{0pt}%
\pgfpathmoveto{\pgfqpoint{3.263003in}{0.894480in}}%
\pgfusepath{stroke}%
\end{pgfscope}%
\begin{pgfscope}%
\pgfpathrectangle{\pgfqpoint{0.647939in}{0.492442in}}{\pgfqpoint{4.273799in}{2.331163in}}%
\pgfusepath{clip}%
\pgfsetroundcap%
\pgfsetroundjoin%
\definecolor{currentfill}{rgb}{0.500000,0.500000,0.500000}%
\pgfsetfillcolor{currentfill}%
\pgfsetfillopacity{0.300000}%
\pgfsetlinewidth{0.301125pt}%
\definecolor{currentstroke}{rgb}{0.500000,0.500000,0.500000}%
\pgfsetstrokecolor{currentstroke}%
\pgfsetstrokeopacity{0.300000}%
\pgfsetdash{}{0pt}%
\pgfpathmoveto{\pgfqpoint{0.000000in}{0.000000in}}%
\pgfpathlineto{\pgfqpoint{0.000000in}{0.000000in}}%
\pgfpathclose%
\pgfusepath{stroke,fill}%
\end{pgfscope}%
\begin{pgfscope}%
\pgfpathrectangle{\pgfqpoint{0.647939in}{0.492442in}}{\pgfqpoint{4.273799in}{2.331163in}}%
\pgfusepath{clip}%
\pgfsetroundcap%
\pgfsetroundjoin%
\pgfsetlinewidth{0.301125pt}%
\definecolor{currentstroke}{rgb}{0.500000,0.500000,0.500000}%
\pgfsetstrokecolor{currentstroke}%
\pgfsetstrokeopacity{0.300000}%
\pgfsetdash{}{0pt}%
\pgfpathmoveto{\pgfqpoint{3.200337in}{1.047171in}}%
\pgfusepath{stroke}%
\end{pgfscope}%
\begin{pgfscope}%
\pgfpathrectangle{\pgfqpoint{0.647939in}{0.492442in}}{\pgfqpoint{4.273799in}{2.331163in}}%
\pgfusepath{clip}%
\pgfsetroundcap%
\pgfsetroundjoin%
\definecolor{currentfill}{rgb}{0.500000,0.500000,0.500000}%
\pgfsetfillcolor{currentfill}%
\pgfsetfillopacity{0.300000}%
\pgfsetlinewidth{0.301125pt}%
\definecolor{currentstroke}{rgb}{0.500000,0.500000,0.500000}%
\pgfsetstrokecolor{currentstroke}%
\pgfsetstrokeopacity{0.300000}%
\pgfsetdash{}{0pt}%
\pgfpathmoveto{\pgfqpoint{0.000000in}{0.000000in}}%
\pgfpathlineto{\pgfqpoint{0.000000in}{0.000000in}}%
\pgfpathclose%
\pgfusepath{stroke,fill}%
\end{pgfscope}%
\begin{pgfscope}%
\pgfpathrectangle{\pgfqpoint{0.647939in}{0.492442in}}{\pgfqpoint{4.273799in}{2.331163in}}%
\pgfusepath{clip}%
\pgfsetroundcap%
\pgfsetroundjoin%
\pgfsetlinewidth{0.301125pt}%
\definecolor{currentstroke}{rgb}{0.500000,0.500000,0.500000}%
\pgfsetstrokecolor{currentstroke}%
\pgfsetstrokeopacity{0.300000}%
\pgfsetdash{}{0pt}%
\pgfpathmoveto{\pgfqpoint{4.090146in}{1.493700in}}%
\pgfusepath{stroke}%
\end{pgfscope}%
\begin{pgfscope}%
\pgfpathrectangle{\pgfqpoint{0.647939in}{0.492442in}}{\pgfqpoint{4.273799in}{2.331163in}}%
\pgfusepath{clip}%
\pgfsetroundcap%
\pgfsetroundjoin%
\definecolor{currentfill}{rgb}{0.500000,0.500000,0.500000}%
\pgfsetfillcolor{currentfill}%
\pgfsetfillopacity{0.300000}%
\pgfsetlinewidth{0.301125pt}%
\definecolor{currentstroke}{rgb}{0.500000,0.500000,0.500000}%
\pgfsetstrokecolor{currentstroke}%
\pgfsetstrokeopacity{0.300000}%
\pgfsetdash{}{0pt}%
\pgfpathmoveto{\pgfqpoint{0.000000in}{0.000000in}}%
\pgfpathlineto{\pgfqpoint{0.000000in}{0.000000in}}%
\pgfpathclose%
\pgfusepath{stroke,fill}%
\end{pgfscope}%
\begin{pgfscope}%
\pgfpathrectangle{\pgfqpoint{0.647939in}{0.492442in}}{\pgfqpoint{4.273799in}{2.331163in}}%
\pgfusepath{clip}%
\pgfsetroundcap%
\pgfsetroundjoin%
\pgfsetlinewidth{0.301125pt}%
\definecolor{currentstroke}{rgb}{0.500000,0.500000,0.500000}%
\pgfsetstrokecolor{currentstroke}%
\pgfsetstrokeopacity{0.300000}%
\pgfsetdash{}{0pt}%
\pgfpathmoveto{\pgfqpoint{1.800783in}{2.507052in}}%
\pgfusepath{stroke}%
\end{pgfscope}%
\begin{pgfscope}%
\pgfpathrectangle{\pgfqpoint{0.647939in}{0.492442in}}{\pgfqpoint{4.273799in}{2.331163in}}%
\pgfusepath{clip}%
\pgfsetroundcap%
\pgfsetroundjoin%
\definecolor{currentfill}{rgb}{0.500000,0.500000,0.500000}%
\pgfsetfillcolor{currentfill}%
\pgfsetfillopacity{0.300000}%
\pgfsetlinewidth{0.301125pt}%
\definecolor{currentstroke}{rgb}{0.500000,0.500000,0.500000}%
\pgfsetstrokecolor{currentstroke}%
\pgfsetstrokeopacity{0.300000}%
\pgfsetdash{}{0pt}%
\pgfpathmoveto{\pgfqpoint{0.000000in}{0.000000in}}%
\pgfpathlineto{\pgfqpoint{0.000000in}{0.000000in}}%
\pgfpathclose%
\pgfusepath{stroke,fill}%
\end{pgfscope}%
\begin{pgfscope}%
\pgfpathrectangle{\pgfqpoint{0.647939in}{0.492442in}}{\pgfqpoint{4.273799in}{2.331163in}}%
\pgfusepath{clip}%
\pgfsetroundcap%
\pgfsetroundjoin%
\pgfsetlinewidth{0.301125pt}%
\definecolor{currentstroke}{rgb}{0.500000,0.500000,0.500000}%
\pgfsetstrokecolor{currentstroke}%
\pgfsetstrokeopacity{0.300000}%
\pgfsetdash{}{0pt}%
\pgfpathmoveto{\pgfqpoint{1.285646in}{1.006137in}}%
\pgfusepath{stroke}%
\end{pgfscope}%
\begin{pgfscope}%
\pgfpathrectangle{\pgfqpoint{0.647939in}{0.492442in}}{\pgfqpoint{4.273799in}{2.331163in}}%
\pgfusepath{clip}%
\pgfsetroundcap%
\pgfsetroundjoin%
\definecolor{currentfill}{rgb}{0.500000,0.500000,0.500000}%
\pgfsetfillcolor{currentfill}%
\pgfsetfillopacity{0.300000}%
\pgfsetlinewidth{0.301125pt}%
\definecolor{currentstroke}{rgb}{0.500000,0.500000,0.500000}%
\pgfsetstrokecolor{currentstroke}%
\pgfsetstrokeopacity{0.300000}%
\pgfsetdash{}{0pt}%
\pgfpathmoveto{\pgfqpoint{0.000000in}{0.000000in}}%
\pgfpathlineto{\pgfqpoint{0.000000in}{0.000000in}}%
\pgfpathclose%
\pgfusepath{stroke,fill}%
\end{pgfscope}%
\begin{pgfscope}%
\pgfpathrectangle{\pgfqpoint{0.647939in}{0.492442in}}{\pgfqpoint{4.273799in}{2.331163in}}%
\pgfusepath{clip}%
\pgfsetroundcap%
\pgfsetroundjoin%
\pgfsetlinewidth{0.301125pt}%
\definecolor{currentstroke}{rgb}{0.500000,0.500000,0.500000}%
\pgfsetstrokecolor{currentstroke}%
\pgfsetstrokeopacity{0.300000}%
\pgfsetdash{}{0pt}%
\pgfpathmoveto{\pgfqpoint{3.933465in}{1.360813in}}%
\pgfusepath{stroke}%
\end{pgfscope}%
\begin{pgfscope}%
\pgfpathrectangle{\pgfqpoint{0.647939in}{0.492442in}}{\pgfqpoint{4.273799in}{2.331163in}}%
\pgfusepath{clip}%
\pgfsetroundcap%
\pgfsetroundjoin%
\definecolor{currentfill}{rgb}{0.500000,0.500000,0.500000}%
\pgfsetfillcolor{currentfill}%
\pgfsetfillopacity{0.300000}%
\pgfsetlinewidth{0.301125pt}%
\definecolor{currentstroke}{rgb}{0.500000,0.500000,0.500000}%
\pgfsetstrokecolor{currentstroke}%
\pgfsetstrokeopacity{0.300000}%
\pgfsetdash{}{0pt}%
\pgfpathmoveto{\pgfqpoint{0.000000in}{0.000000in}}%
\pgfpathlineto{\pgfqpoint{0.000000in}{0.000000in}}%
\pgfpathclose%
\pgfusepath{stroke,fill}%
\end{pgfscope}%
\begin{pgfscope}%
\pgfpathrectangle{\pgfqpoint{0.647939in}{0.492442in}}{\pgfqpoint{4.273799in}{2.331163in}}%
\pgfusepath{clip}%
\pgfsetroundcap%
\pgfsetroundjoin%
\pgfsetlinewidth{0.301125pt}%
\definecolor{currentstroke}{rgb}{0.500000,0.500000,0.500000}%
\pgfsetstrokecolor{currentstroke}%
\pgfsetstrokeopacity{0.300000}%
\pgfsetdash{}{0pt}%
\pgfpathmoveto{\pgfqpoint{4.081824in}{1.562024in}}%
\pgfusepath{stroke}%
\end{pgfscope}%
\begin{pgfscope}%
\pgfpathrectangle{\pgfqpoint{0.647939in}{0.492442in}}{\pgfqpoint{4.273799in}{2.331163in}}%
\pgfusepath{clip}%
\pgfsetroundcap%
\pgfsetroundjoin%
\definecolor{currentfill}{rgb}{0.500000,0.500000,0.500000}%
\pgfsetfillcolor{currentfill}%
\pgfsetfillopacity{0.300000}%
\pgfsetlinewidth{0.301125pt}%
\definecolor{currentstroke}{rgb}{0.500000,0.500000,0.500000}%
\pgfsetstrokecolor{currentstroke}%
\pgfsetstrokeopacity{0.300000}%
\pgfsetdash{}{0pt}%
\pgfpathmoveto{\pgfqpoint{0.000000in}{0.000000in}}%
\pgfpathlineto{\pgfqpoint{0.000000in}{0.000000in}}%
\pgfpathclose%
\pgfusepath{stroke,fill}%
\end{pgfscope}%
\begin{pgfscope}%
\pgfpathrectangle{\pgfqpoint{0.647939in}{0.492442in}}{\pgfqpoint{4.273799in}{2.331163in}}%
\pgfusepath{clip}%
\pgfsetroundcap%
\pgfsetroundjoin%
\pgfsetlinewidth{0.301125pt}%
\definecolor{currentstroke}{rgb}{0.500000,0.500000,0.500000}%
\pgfsetstrokecolor{currentstroke}%
\pgfsetstrokeopacity{0.300000}%
\pgfsetdash{}{0pt}%
\pgfpathmoveto{\pgfqpoint{1.757413in}{2.339368in}}%
\pgfusepath{stroke}%
\end{pgfscope}%
\begin{pgfscope}%
\pgfpathrectangle{\pgfqpoint{0.647939in}{0.492442in}}{\pgfqpoint{4.273799in}{2.331163in}}%
\pgfusepath{clip}%
\pgfsetroundcap%
\pgfsetroundjoin%
\definecolor{currentfill}{rgb}{0.500000,0.500000,0.500000}%
\pgfsetfillcolor{currentfill}%
\pgfsetfillopacity{0.300000}%
\pgfsetlinewidth{0.301125pt}%
\definecolor{currentstroke}{rgb}{0.500000,0.500000,0.500000}%
\pgfsetstrokecolor{currentstroke}%
\pgfsetstrokeopacity{0.300000}%
\pgfsetdash{}{0pt}%
\pgfpathmoveto{\pgfqpoint{0.000000in}{0.000000in}}%
\pgfpathlineto{\pgfqpoint{0.000000in}{0.000000in}}%
\pgfpathclose%
\pgfusepath{stroke,fill}%
\end{pgfscope}%
\begin{pgfscope}%
\pgfpathrectangle{\pgfqpoint{0.647939in}{0.492442in}}{\pgfqpoint{4.273799in}{2.331163in}}%
\pgfusepath{clip}%
\pgfsetroundcap%
\pgfsetroundjoin%
\pgfsetlinewidth{0.301125pt}%
\definecolor{currentstroke}{rgb}{0.500000,0.500000,0.500000}%
\pgfsetstrokecolor{currentstroke}%
\pgfsetstrokeopacity{0.300000}%
\pgfsetdash{}{0pt}%
\pgfpathmoveto{\pgfqpoint{2.791988in}{0.796721in}}%
\pgfusepath{stroke}%
\end{pgfscope}%
\begin{pgfscope}%
\pgfpathrectangle{\pgfqpoint{0.647939in}{0.492442in}}{\pgfqpoint{4.273799in}{2.331163in}}%
\pgfusepath{clip}%
\pgfsetroundcap%
\pgfsetroundjoin%
\definecolor{currentfill}{rgb}{0.500000,0.500000,0.500000}%
\pgfsetfillcolor{currentfill}%
\pgfsetfillopacity{0.300000}%
\pgfsetlinewidth{0.301125pt}%
\definecolor{currentstroke}{rgb}{0.500000,0.500000,0.500000}%
\pgfsetstrokecolor{currentstroke}%
\pgfsetstrokeopacity{0.300000}%
\pgfsetdash{}{0pt}%
\pgfpathmoveto{\pgfqpoint{0.000000in}{0.000000in}}%
\pgfpathlineto{\pgfqpoint{0.000000in}{0.000000in}}%
\pgfpathclose%
\pgfusepath{stroke,fill}%
\end{pgfscope}%
\begin{pgfscope}%
\pgfpathrectangle{\pgfqpoint{0.647939in}{0.492442in}}{\pgfqpoint{4.273799in}{2.331163in}}%
\pgfusepath{clip}%
\pgfsetroundcap%
\pgfsetroundjoin%
\pgfsetlinewidth{0.301125pt}%
\definecolor{currentstroke}{rgb}{0.500000,0.500000,0.500000}%
\pgfsetstrokecolor{currentstroke}%
\pgfsetstrokeopacity{0.300000}%
\pgfsetdash{}{0pt}%
\pgfpathmoveto{\pgfqpoint{3.913881in}{1.192587in}}%
\pgfusepath{stroke}%
\end{pgfscope}%
\begin{pgfscope}%
\pgfpathrectangle{\pgfqpoint{0.647939in}{0.492442in}}{\pgfqpoint{4.273799in}{2.331163in}}%
\pgfusepath{clip}%
\pgfsetroundcap%
\pgfsetroundjoin%
\definecolor{currentfill}{rgb}{0.500000,0.500000,0.500000}%
\pgfsetfillcolor{currentfill}%
\pgfsetfillopacity{0.300000}%
\pgfsetlinewidth{0.301125pt}%
\definecolor{currentstroke}{rgb}{0.500000,0.500000,0.500000}%
\pgfsetstrokecolor{currentstroke}%
\pgfsetstrokeopacity{0.300000}%
\pgfsetdash{}{0pt}%
\pgfpathmoveto{\pgfqpoint{0.000000in}{0.000000in}}%
\pgfpathlineto{\pgfqpoint{0.000000in}{0.000000in}}%
\pgfpathclose%
\pgfusepath{stroke,fill}%
\end{pgfscope}%
\begin{pgfscope}%
\pgfpathrectangle{\pgfqpoint{0.647939in}{0.492442in}}{\pgfqpoint{4.273799in}{2.331163in}}%
\pgfusepath{clip}%
\pgfsetroundcap%
\pgfsetroundjoin%
\pgfsetlinewidth{0.301125pt}%
\definecolor{currentstroke}{rgb}{0.500000,0.500000,0.500000}%
\pgfsetstrokecolor{currentstroke}%
\pgfsetstrokeopacity{0.300000}%
\pgfsetdash{}{0pt}%
\pgfpathmoveto{\pgfqpoint{1.469496in}{1.105823in}}%
\pgfusepath{stroke}%
\end{pgfscope}%
\begin{pgfscope}%
\pgfpathrectangle{\pgfqpoint{0.647939in}{0.492442in}}{\pgfqpoint{4.273799in}{2.331163in}}%
\pgfusepath{clip}%
\pgfsetroundcap%
\pgfsetroundjoin%
\definecolor{currentfill}{rgb}{0.500000,0.500000,0.500000}%
\pgfsetfillcolor{currentfill}%
\pgfsetfillopacity{0.300000}%
\pgfsetlinewidth{0.301125pt}%
\definecolor{currentstroke}{rgb}{0.500000,0.500000,0.500000}%
\pgfsetstrokecolor{currentstroke}%
\pgfsetstrokeopacity{0.300000}%
\pgfsetdash{}{0pt}%
\pgfpathmoveto{\pgfqpoint{0.000000in}{0.000000in}}%
\pgfpathlineto{\pgfqpoint{0.000000in}{0.000000in}}%
\pgfpathclose%
\pgfusepath{stroke,fill}%
\end{pgfscope}%
\begin{pgfscope}%
\pgfpathrectangle{\pgfqpoint{0.647939in}{0.492442in}}{\pgfqpoint{4.273799in}{2.331163in}}%
\pgfusepath{clip}%
\pgfsetroundcap%
\pgfsetroundjoin%
\pgfsetlinewidth{0.301125pt}%
\definecolor{currentstroke}{rgb}{0.500000,0.500000,0.500000}%
\pgfsetstrokecolor{currentstroke}%
\pgfsetstrokeopacity{0.300000}%
\pgfsetdash{}{0pt}%
\pgfpathmoveto{\pgfqpoint{2.037179in}{1.531153in}}%
\pgfusepath{stroke}%
\end{pgfscope}%
\begin{pgfscope}%
\pgfpathrectangle{\pgfqpoint{0.647939in}{0.492442in}}{\pgfqpoint{4.273799in}{2.331163in}}%
\pgfusepath{clip}%
\pgfsetroundcap%
\pgfsetroundjoin%
\definecolor{currentfill}{rgb}{0.500000,0.500000,0.500000}%
\pgfsetfillcolor{currentfill}%
\pgfsetfillopacity{0.300000}%
\pgfsetlinewidth{0.301125pt}%
\definecolor{currentstroke}{rgb}{0.500000,0.500000,0.500000}%
\pgfsetstrokecolor{currentstroke}%
\pgfsetstrokeopacity{0.300000}%
\pgfsetdash{}{0pt}%
\pgfpathmoveto{\pgfqpoint{0.000000in}{0.000000in}}%
\pgfpathlineto{\pgfqpoint{0.000000in}{0.000000in}}%
\pgfpathclose%
\pgfusepath{stroke,fill}%
\end{pgfscope}%
\begin{pgfscope}%
\pgfpathrectangle{\pgfqpoint{0.647939in}{0.492442in}}{\pgfqpoint{4.273799in}{2.331163in}}%
\pgfusepath{clip}%
\pgfsetroundcap%
\pgfsetroundjoin%
\pgfsetlinewidth{0.301125pt}%
\definecolor{currentstroke}{rgb}{0.500000,0.500000,0.500000}%
\pgfsetstrokecolor{currentstroke}%
\pgfsetstrokeopacity{0.300000}%
\pgfsetdash{}{0pt}%
\pgfpathmoveto{\pgfqpoint{3.593030in}{1.935028in}}%
\pgfusepath{stroke}%
\end{pgfscope}%
\begin{pgfscope}%
\pgfpathrectangle{\pgfqpoint{0.647939in}{0.492442in}}{\pgfqpoint{4.273799in}{2.331163in}}%
\pgfusepath{clip}%
\pgfsetroundcap%
\pgfsetroundjoin%
\definecolor{currentfill}{rgb}{0.500000,0.500000,0.500000}%
\pgfsetfillcolor{currentfill}%
\pgfsetfillopacity{0.300000}%
\pgfsetlinewidth{0.301125pt}%
\definecolor{currentstroke}{rgb}{0.500000,0.500000,0.500000}%
\pgfsetstrokecolor{currentstroke}%
\pgfsetstrokeopacity{0.300000}%
\pgfsetdash{}{0pt}%
\pgfpathmoveto{\pgfqpoint{0.000000in}{0.000000in}}%
\pgfpathlineto{\pgfqpoint{0.000000in}{0.000000in}}%
\pgfpathclose%
\pgfusepath{stroke,fill}%
\end{pgfscope}%
\begin{pgfscope}%
\pgfpathrectangle{\pgfqpoint{0.647939in}{0.492442in}}{\pgfqpoint{4.273799in}{2.331163in}}%
\pgfusepath{clip}%
\pgfsetroundcap%
\pgfsetroundjoin%
\pgfsetlinewidth{0.301125pt}%
\definecolor{currentstroke}{rgb}{0.500000,0.500000,0.500000}%
\pgfsetstrokecolor{currentstroke}%
\pgfsetstrokeopacity{0.300000}%
\pgfsetdash{}{0pt}%
\pgfpathmoveto{\pgfqpoint{1.617954in}{2.146402in}}%
\pgfusepath{stroke}%
\end{pgfscope}%
\begin{pgfscope}%
\pgfpathrectangle{\pgfqpoint{0.647939in}{0.492442in}}{\pgfqpoint{4.273799in}{2.331163in}}%
\pgfusepath{clip}%
\pgfsetroundcap%
\pgfsetroundjoin%
\definecolor{currentfill}{rgb}{0.500000,0.500000,0.500000}%
\pgfsetfillcolor{currentfill}%
\pgfsetfillopacity{0.300000}%
\pgfsetlinewidth{0.301125pt}%
\definecolor{currentstroke}{rgb}{0.500000,0.500000,0.500000}%
\pgfsetstrokecolor{currentstroke}%
\pgfsetstrokeopacity{0.300000}%
\pgfsetdash{}{0pt}%
\pgfpathmoveto{\pgfqpoint{0.000000in}{0.000000in}}%
\pgfpathlineto{\pgfqpoint{0.000000in}{0.000000in}}%
\pgfpathclose%
\pgfusepath{stroke,fill}%
\end{pgfscope}%
\begin{pgfscope}%
\pgfpathrectangle{\pgfqpoint{0.647939in}{0.492442in}}{\pgfqpoint{4.273799in}{2.331163in}}%
\pgfusepath{clip}%
\pgfsetroundcap%
\pgfsetroundjoin%
\pgfsetlinewidth{0.301125pt}%
\definecolor{currentstroke}{rgb}{0.500000,0.500000,0.500000}%
\pgfsetstrokecolor{currentstroke}%
\pgfsetstrokeopacity{0.300000}%
\pgfsetdash{}{0pt}%
\pgfpathmoveto{\pgfqpoint{3.405959in}{2.139910in}}%
\pgfusepath{stroke}%
\end{pgfscope}%
\begin{pgfscope}%
\pgfpathrectangle{\pgfqpoint{0.647939in}{0.492442in}}{\pgfqpoint{4.273799in}{2.331163in}}%
\pgfusepath{clip}%
\pgfsetroundcap%
\pgfsetroundjoin%
\definecolor{currentfill}{rgb}{0.500000,0.500000,0.500000}%
\pgfsetfillcolor{currentfill}%
\pgfsetfillopacity{0.300000}%
\pgfsetlinewidth{0.301125pt}%
\definecolor{currentstroke}{rgb}{0.500000,0.500000,0.500000}%
\pgfsetstrokecolor{currentstroke}%
\pgfsetstrokeopacity{0.300000}%
\pgfsetdash{}{0pt}%
\pgfpathmoveto{\pgfqpoint{0.000000in}{0.000000in}}%
\pgfpathlineto{\pgfqpoint{0.000000in}{0.000000in}}%
\pgfpathclose%
\pgfusepath{stroke,fill}%
\end{pgfscope}%
\begin{pgfscope}%
\pgfpathrectangle{\pgfqpoint{0.647939in}{0.492442in}}{\pgfqpoint{4.273799in}{2.331163in}}%
\pgfusepath{clip}%
\pgfsetroundcap%
\pgfsetroundjoin%
\pgfsetlinewidth{0.301125pt}%
\definecolor{currentstroke}{rgb}{0.500000,0.500000,0.500000}%
\pgfsetstrokecolor{currentstroke}%
\pgfsetstrokeopacity{0.300000}%
\pgfsetdash{}{0pt}%
\pgfpathmoveto{\pgfqpoint{1.703773in}{1.896663in}}%
\pgfusepath{stroke}%
\end{pgfscope}%
\begin{pgfscope}%
\pgfpathrectangle{\pgfqpoint{0.647939in}{0.492442in}}{\pgfqpoint{4.273799in}{2.331163in}}%
\pgfusepath{clip}%
\pgfsetroundcap%
\pgfsetroundjoin%
\definecolor{currentfill}{rgb}{0.500000,0.500000,0.500000}%
\pgfsetfillcolor{currentfill}%
\pgfsetfillopacity{0.300000}%
\pgfsetlinewidth{0.301125pt}%
\definecolor{currentstroke}{rgb}{0.500000,0.500000,0.500000}%
\pgfsetstrokecolor{currentstroke}%
\pgfsetstrokeopacity{0.300000}%
\pgfsetdash{}{0pt}%
\pgfpathmoveto{\pgfqpoint{0.000000in}{0.000000in}}%
\pgfpathlineto{\pgfqpoint{0.000000in}{0.000000in}}%
\pgfpathclose%
\pgfusepath{stroke,fill}%
\end{pgfscope}%
\begin{pgfscope}%
\pgfpathrectangle{\pgfqpoint{0.647939in}{0.492442in}}{\pgfqpoint{4.273799in}{2.331163in}}%
\pgfusepath{clip}%
\pgfsetroundcap%
\pgfsetroundjoin%
\pgfsetlinewidth{0.301125pt}%
\definecolor{currentstroke}{rgb}{0.500000,0.500000,0.500000}%
\pgfsetstrokecolor{currentstroke}%
\pgfsetstrokeopacity{0.300000}%
\pgfsetdash{}{0pt}%
\pgfpathmoveto{\pgfqpoint{2.688860in}{1.244299in}}%
\pgfusepath{stroke}%
\end{pgfscope}%
\begin{pgfscope}%
\pgfpathrectangle{\pgfqpoint{0.647939in}{0.492442in}}{\pgfqpoint{4.273799in}{2.331163in}}%
\pgfusepath{clip}%
\pgfsetroundcap%
\pgfsetroundjoin%
\definecolor{currentfill}{rgb}{0.500000,0.500000,0.500000}%
\pgfsetfillcolor{currentfill}%
\pgfsetfillopacity{0.300000}%
\pgfsetlinewidth{0.301125pt}%
\definecolor{currentstroke}{rgb}{0.500000,0.500000,0.500000}%
\pgfsetstrokecolor{currentstroke}%
\pgfsetstrokeopacity{0.300000}%
\pgfsetdash{}{0pt}%
\pgfpathmoveto{\pgfqpoint{0.000000in}{0.000000in}}%
\pgfpathlineto{\pgfqpoint{0.000000in}{0.000000in}}%
\pgfpathclose%
\pgfusepath{stroke,fill}%
\end{pgfscope}%
\begin{pgfscope}%
\pgfpathrectangle{\pgfqpoint{0.647939in}{0.492442in}}{\pgfqpoint{4.273799in}{2.331163in}}%
\pgfusepath{clip}%
\pgfsetroundcap%
\pgfsetroundjoin%
\pgfsetlinewidth{0.301125pt}%
\definecolor{currentstroke}{rgb}{0.500000,0.500000,0.500000}%
\pgfsetstrokecolor{currentstroke}%
\pgfsetstrokeopacity{0.300000}%
\pgfsetdash{}{0pt}%
\pgfpathmoveto{\pgfqpoint{3.343686in}{1.142280in}}%
\pgfusepath{stroke}%
\end{pgfscope}%
\begin{pgfscope}%
\pgfpathrectangle{\pgfqpoint{0.647939in}{0.492442in}}{\pgfqpoint{4.273799in}{2.331163in}}%
\pgfusepath{clip}%
\pgfsetroundcap%
\pgfsetroundjoin%
\definecolor{currentfill}{rgb}{0.500000,0.500000,0.500000}%
\pgfsetfillcolor{currentfill}%
\pgfsetfillopacity{0.300000}%
\pgfsetlinewidth{0.301125pt}%
\definecolor{currentstroke}{rgb}{0.500000,0.500000,0.500000}%
\pgfsetstrokecolor{currentstroke}%
\pgfsetstrokeopacity{0.300000}%
\pgfsetdash{}{0pt}%
\pgfpathmoveto{\pgfqpoint{0.000000in}{0.000000in}}%
\pgfpathlineto{\pgfqpoint{0.000000in}{0.000000in}}%
\pgfpathclose%
\pgfusepath{stroke,fill}%
\end{pgfscope}%
\begin{pgfscope}%
\pgfpathrectangle{\pgfqpoint{0.647939in}{0.492442in}}{\pgfqpoint{4.273799in}{2.331163in}}%
\pgfusepath{clip}%
\pgfsetroundcap%
\pgfsetroundjoin%
\pgfsetlinewidth{0.301125pt}%
\definecolor{currentstroke}{rgb}{0.500000,0.500000,0.500000}%
\pgfsetstrokecolor{currentstroke}%
\pgfsetstrokeopacity{0.300000}%
\pgfsetdash{}{0pt}%
\pgfpathmoveto{\pgfqpoint{2.712591in}{2.126490in}}%
\pgfusepath{stroke}%
\end{pgfscope}%
\begin{pgfscope}%
\pgfpathrectangle{\pgfqpoint{0.647939in}{0.492442in}}{\pgfqpoint{4.273799in}{2.331163in}}%
\pgfusepath{clip}%
\pgfsetroundcap%
\pgfsetroundjoin%
\definecolor{currentfill}{rgb}{0.500000,0.500000,0.500000}%
\pgfsetfillcolor{currentfill}%
\pgfsetfillopacity{0.300000}%
\pgfsetlinewidth{0.301125pt}%
\definecolor{currentstroke}{rgb}{0.500000,0.500000,0.500000}%
\pgfsetstrokecolor{currentstroke}%
\pgfsetstrokeopacity{0.300000}%
\pgfsetdash{}{0pt}%
\pgfpathmoveto{\pgfqpoint{0.000000in}{0.000000in}}%
\pgfpathlineto{\pgfqpoint{0.000000in}{0.000000in}}%
\pgfpathclose%
\pgfusepath{stroke,fill}%
\end{pgfscope}%
\begin{pgfscope}%
\pgfpathrectangle{\pgfqpoint{0.647939in}{0.492442in}}{\pgfqpoint{4.273799in}{2.331163in}}%
\pgfusepath{clip}%
\pgfsetroundcap%
\pgfsetroundjoin%
\pgfsetlinewidth{0.301125pt}%
\definecolor{currentstroke}{rgb}{0.500000,0.500000,0.500000}%
\pgfsetstrokecolor{currentstroke}%
\pgfsetstrokeopacity{0.300000}%
\pgfsetdash{}{0pt}%
\pgfpathmoveto{\pgfqpoint{1.885493in}{2.033944in}}%
\pgfusepath{stroke}%
\end{pgfscope}%
\begin{pgfscope}%
\pgfpathrectangle{\pgfqpoint{0.647939in}{0.492442in}}{\pgfqpoint{4.273799in}{2.331163in}}%
\pgfusepath{clip}%
\pgfsetroundcap%
\pgfsetroundjoin%
\definecolor{currentfill}{rgb}{0.500000,0.500000,0.500000}%
\pgfsetfillcolor{currentfill}%
\pgfsetfillopacity{0.300000}%
\pgfsetlinewidth{0.301125pt}%
\definecolor{currentstroke}{rgb}{0.500000,0.500000,0.500000}%
\pgfsetstrokecolor{currentstroke}%
\pgfsetstrokeopacity{0.300000}%
\pgfsetdash{}{0pt}%
\pgfpathmoveto{\pgfqpoint{0.000000in}{0.000000in}}%
\pgfpathlineto{\pgfqpoint{0.000000in}{0.000000in}}%
\pgfpathclose%
\pgfusepath{stroke,fill}%
\end{pgfscope}%
\begin{pgfscope}%
\pgfpathrectangle{\pgfqpoint{0.647939in}{0.492442in}}{\pgfqpoint{4.273799in}{2.331163in}}%
\pgfusepath{clip}%
\pgfsetroundcap%
\pgfsetroundjoin%
\pgfsetlinewidth{0.301125pt}%
\definecolor{currentstroke}{rgb}{0.500000,0.500000,0.500000}%
\pgfsetstrokecolor{currentstroke}%
\pgfsetstrokeopacity{0.300000}%
\pgfsetdash{}{0pt}%
\pgfpathmoveto{\pgfqpoint{3.195028in}{2.025960in}}%
\pgfusepath{stroke}%
\end{pgfscope}%
\begin{pgfscope}%
\pgfpathrectangle{\pgfqpoint{0.647939in}{0.492442in}}{\pgfqpoint{4.273799in}{2.331163in}}%
\pgfusepath{clip}%
\pgfsetroundcap%
\pgfsetroundjoin%
\definecolor{currentfill}{rgb}{0.500000,0.500000,0.500000}%
\pgfsetfillcolor{currentfill}%
\pgfsetfillopacity{0.300000}%
\pgfsetlinewidth{0.301125pt}%
\definecolor{currentstroke}{rgb}{0.500000,0.500000,0.500000}%
\pgfsetstrokecolor{currentstroke}%
\pgfsetstrokeopacity{0.300000}%
\pgfsetdash{}{0pt}%
\pgfpathmoveto{\pgfqpoint{0.000000in}{0.000000in}}%
\pgfpathlineto{\pgfqpoint{0.000000in}{0.000000in}}%
\pgfpathclose%
\pgfusepath{stroke,fill}%
\end{pgfscope}%
\begin{pgfscope}%
\pgfpathrectangle{\pgfqpoint{0.647939in}{0.492442in}}{\pgfqpoint{4.273799in}{2.331163in}}%
\pgfusepath{clip}%
\pgfsetbuttcap%
\pgfsetroundjoin%
\pgfsetlinewidth{0.301125pt}%
\definecolor{currentstroke}{rgb}{0.500000,0.500000,0.500000}%
\pgfsetstrokecolor{currentstroke}%
\pgfsetstrokeopacity{0.300000}%
\pgfsetdash{}{0pt}%
\pgfpathmoveto{\pgfqpoint{2.268589in}{0.492442in}}%
\pgfpathlineto{\pgfqpoint{2.247975in}{0.519742in}}%
\pgfpathlineto{\pgfqpoint{2.211944in}{0.567673in}}%
\pgfpathlineto{\pgfqpoint{2.176173in}{0.615661in}}%
\pgfpathlineto{\pgfqpoint{2.140625in}{0.663699in}}%
\pgfpathlineto{\pgfqpoint{2.105258in}{0.711777in}}%
\pgfpathlineto{\pgfqpoint{2.070030in}{0.759885in}}%
\pgfpathlineto{\pgfqpoint{2.034883in}{0.808011in}}%
\pgfpathlineto{\pgfqpoint{1.999749in}{0.856139in}}%
\pgfpathlineto{\pgfqpoint{1.964556in}{0.904255in}}%
\pgfpathlineto{\pgfqpoint{1.929218in}{0.952338in}}%
\pgfpathlineto{\pgfqpoint{1.893633in}{1.000368in}}%
\pgfpathlineto{\pgfqpoint{1.857670in}{1.048313in}}%
\pgfpathlineto{\pgfqpoint{1.821146in}{1.096132in}}%
\pgfpathlineto{\pgfqpoint{1.783823in}{1.143766in}}%
\pgfpathlineto{\pgfqpoint{1.745384in}{1.191132in}}%
\pgfpathlineto{\pgfqpoint{1.705365in}{1.238103in}}%
\pgfpathlineto{\pgfqpoint{1.663054in}{1.284464in}}%
\pgfpathlineto{\pgfqpoint{1.617262in}{1.329809in}}%
\pgfpathlineto{\pgfqpoint{1.565803in}{1.373243in}}%
\pgfpathlineto{\pgfqpoint{1.520319in}{1.403662in}}%
\pgfpathlineto{\pgfqpoint{1.479473in}{1.423554in}}%
\pgfpathlineto{\pgfqpoint{1.438823in}{1.435600in}}%
\pgfpathlineto{\pgfqpoint{1.389118in}{1.439128in}}%
\pgfpathlineto{\pgfqpoint{1.340917in}{1.430670in}}%
\pgfpathlineto{\pgfqpoint{1.340917in}{1.430670in}}%
\pgfpathlineto{\pgfqpoint{1.285434in}{1.406812in}}%
\pgfpathlineto{\pgfqpoint{1.285434in}{1.406812in}}%
\pgfpathlineto{\pgfqpoint{1.226609in}{1.366581in}}%
\pgfpathlineto{\pgfqpoint{1.177165in}{1.322517in}}%
\pgfpathlineto{\pgfqpoint{1.133334in}{1.276638in}}%
\pgfpathlineto{\pgfqpoint{1.093249in}{1.229727in}}%
\pgfpathlineto{\pgfqpoint{1.055854in}{1.182149in}}%
\pgfpathlineto{\pgfqpoint{1.020493in}{1.134101in}}%
\pgfpathlineto{\pgfqpoint{0.986730in}{1.085701in}}%
\pgfpathlineto{\pgfqpoint{0.954275in}{1.037027in}}%
\pgfpathlineto{\pgfqpoint{0.922936in}{0.988136in}}%
\pgfpathlineto{\pgfqpoint{0.892556in}{0.939068in}}%
\pgfpathlineto{\pgfqpoint{0.862992in}{0.889845in}}%
\pgfpathlineto{\pgfqpoint{0.834164in}{0.840491in}}%
\pgfpathlineto{\pgfqpoint{0.806003in}{0.791026in}}%
\pgfpathlineto{\pgfqpoint{0.778428in}{0.741459in}}%
\pgfpathlineto{\pgfqpoint{0.751397in}{0.691803in}}%
\pgfpathlineto{\pgfqpoint{0.724873in}{0.642067in}}%
\pgfpathlineto{\pgfqpoint{0.698807in}{0.592257in}}%
\pgfpathlineto{\pgfqpoint{0.673174in}{0.542381in}}%
\pgfpathlineto{\pgfqpoint{0.647939in}{0.492442in}}%
\pgfpathlineto{\pgfqpoint{0.647939in}{0.492442in}}%
\pgfusepath{stroke}%
\end{pgfscope}%
\begin{pgfscope}%
\pgfpathrectangle{\pgfqpoint{0.647939in}{0.492442in}}{\pgfqpoint{4.273799in}{2.331163in}}%
\pgfusepath{clip}%
\pgfsetbuttcap%
\pgfsetroundjoin%
\pgfsetlinewidth{0.301125pt}%
\definecolor{currentstroke}{rgb}{0.500000,0.500000,0.500000}%
\pgfsetstrokecolor{currentstroke}%
\pgfsetstrokeopacity{0.300000}%
\pgfsetdash{}{0pt}%
\pgfpathmoveto{\pgfqpoint{1.816757in}{0.492442in}}%
\pgfpathlineto{\pgfqpoint{1.812943in}{0.496926in}}%
\pgfpathlineto{\pgfqpoint{1.772539in}{0.543807in}}%
\pgfpathlineto{\pgfqpoint{1.731146in}{0.590430in}}%
\pgfpathlineto{\pgfqpoint{1.688462in}{0.636704in}}%
\pgfpathlineto{\pgfqpoint{1.644077in}{0.682497in}}%
\pgfpathlineto{\pgfqpoint{1.597414in}{0.727608in}}%
\pgfpathlineto{\pgfqpoint{1.547600in}{0.771692in}}%
\pgfpathlineto{\pgfqpoint{1.493270in}{0.814122in}}%
\pgfpathlineto{\pgfqpoint{1.433655in}{0.852769in}}%
\pgfpathlineto{\pgfqpoint{1.380243in}{0.879438in}}%
\pgfpathlineto{\pgfqpoint{1.329907in}{0.896886in}}%
\pgfpathlineto{\pgfqpoint{1.277328in}{0.906450in}}%
\pgfpathlineto{\pgfqpoint{1.215067in}{0.905657in}}%
\pgfpathlineto{\pgfqpoint{1.157168in}{0.892822in}}%
\pgfpathlineto{\pgfqpoint{1.157168in}{0.892822in}}%
\pgfpathlineto{\pgfqpoint{1.084742in}{0.860137in}}%
\pgfpathlineto{\pgfqpoint{1.025091in}{0.820007in}}%
\pgfpathlineto{\pgfqpoint{0.973918in}{0.776485in}}%
\pgfpathlineto{\pgfqpoint{0.928374in}{0.731097in}}%
\pgfpathlineto{\pgfqpoint{0.886840in}{0.684555in}}%
\pgfpathlineto{\pgfqpoint{0.848312in}{0.637236in}}%
\pgfpathlineto{\pgfqpoint{0.812125in}{0.589359in}}%
\pgfpathlineto{\pgfqpoint{0.777816in}{0.541063in}}%
\pgfpathlineto{\pgfqpoint{0.745071in}{0.492442in}}%
\pgfpathlineto{\pgfqpoint{0.745071in}{0.492442in}}%
\pgfusepath{stroke}%
\end{pgfscope}%
\begin{pgfscope}%
\pgfpathrectangle{\pgfqpoint{0.647939in}{0.492442in}}{\pgfqpoint{4.273799in}{2.331163in}}%
\pgfusepath{clip}%
\pgfsetbuttcap%
\pgfsetroundjoin%
\pgfsetlinewidth{0.301125pt}%
\definecolor{currentstroke}{rgb}{0.500000,0.500000,0.500000}%
\pgfsetstrokecolor{currentstroke}%
\pgfsetstrokeopacity{0.300000}%
\pgfsetdash{}{0pt}%
\pgfpathmoveto{\pgfqpoint{1.582830in}{0.492442in}}%
\pgfpathlineto{\pgfqpoint{1.573491in}{0.500904in}}%
\pgfpathlineto{\pgfqpoint{1.522834in}{0.544701in}}%
\pgfpathlineto{\pgfqpoint{1.468030in}{0.586970in}}%
\pgfpathlineto{\pgfqpoint{1.407215in}{0.626637in}}%
\pgfpathlineto{\pgfqpoint{1.349390in}{0.656359in}}%
\pgfpathlineto{\pgfqpoint{1.295633in}{0.676233in}}%
\pgfpathlineto{\pgfqpoint{1.241585in}{0.687978in}}%
\pgfpathlineto{\pgfqpoint{1.180996in}{0.690645in}}%
\pgfpathlineto{\pgfqpoint{1.122027in}{0.682178in}}%
\pgfpathlineto{\pgfqpoint{1.122027in}{0.682178in}}%
\pgfpathlineto{\pgfqpoint{1.058628in}{0.660555in}}%
\pgfpathlineto{\pgfqpoint{1.058628in}{0.660555in}}%
\pgfpathlineto{\pgfqpoint{0.992061in}{0.624001in}}%
\pgfpathlineto{\pgfqpoint{0.935814in}{0.582398in}}%
\pgfpathlineto{\pgfqpoint{0.886562in}{0.538194in}}%
\pgfpathlineto{\pgfqpoint{0.842203in}{0.492442in}}%
\pgfpathlineto{\pgfqpoint{0.842203in}{0.492442in}}%
\pgfusepath{stroke}%
\end{pgfscope}%
\begin{pgfscope}%
\pgfpathrectangle{\pgfqpoint{0.647939in}{0.492442in}}{\pgfqpoint{4.273799in}{2.331163in}}%
\pgfusepath{clip}%
\pgfsetbuttcap%
\pgfsetroundjoin%
\pgfsetlinewidth{0.301125pt}%
\definecolor{currentstroke}{rgb}{0.500000,0.500000,0.500000}%
\pgfsetstrokecolor{currentstroke}%
\pgfsetstrokeopacity{0.300000}%
\pgfsetdash{}{0pt}%
\pgfpathmoveto{\pgfqpoint{1.417285in}{0.492442in}}%
\pgfpathlineto{\pgfqpoint{1.372361in}{0.516887in}}%
\pgfpathlineto{\pgfqpoint{1.314260in}{0.544355in}}%
\pgfpathlineto{\pgfqpoint{1.259463in}{0.562449in}}%
\pgfpathlineto{\pgfqpoint{1.203162in}{0.572488in}}%
\pgfpathlineto{\pgfqpoint{1.138910in}{0.572644in}}%
\pgfpathlineto{\pgfqpoint{1.078146in}{0.561295in}}%
\pgfpathlineto{\pgfqpoint{1.078146in}{0.561295in}}%
\pgfpathlineto{\pgfqpoint{1.002313in}{0.530977in}}%
\pgfpathlineto{\pgfqpoint{0.939334in}{0.492442in}}%
\pgfpathlineto{\pgfqpoint{0.939334in}{0.492442in}}%
\pgfusepath{stroke}%
\end{pgfscope}%
\begin{pgfscope}%
\pgfpathrectangle{\pgfqpoint{0.647939in}{0.492442in}}{\pgfqpoint{4.273799in}{2.331163in}}%
\pgfusepath{clip}%
\pgfsetbuttcap%
\pgfsetroundjoin%
\pgfsetlinewidth{0.301125pt}%
\definecolor{currentstroke}{rgb}{0.500000,0.500000,0.500000}%
\pgfsetstrokecolor{currentstroke}%
\pgfsetstrokeopacity{0.300000}%
\pgfsetdash{}{0pt}%
\pgfpathmoveto{\pgfqpoint{1.716389in}{0.492442in}}%
\pgfpathlineto{\pgfqpoint{1.716389in}{0.492442in}}%
\pgfpathlineto{\pgfqpoint{1.672996in}{0.538518in}}%
\pgfpathlineto{\pgfqpoint{1.627850in}{0.584087in}}%
\pgfpathlineto{\pgfqpoint{1.580367in}{0.628938in}}%
\pgfpathlineto{\pgfqpoint{1.529689in}{0.672723in}}%
\pgfpathlineto{\pgfqpoint{1.474487in}{0.714827in}}%
\pgfpathlineto{\pgfqpoint{1.412587in}{0.753985in}}%
\pgfpathlineto{\pgfqpoint{1.340424in}{0.787198in}}%
\pgfpathlineto{\pgfqpoint{1.340424in}{0.787198in}}%
\pgfpathlineto{\pgfqpoint{1.279943in}{0.803518in}}%
\pgfpathlineto{\pgfqpoint{1.279943in}{0.803518in}}%
\pgfpathlineto{\pgfqpoint{1.223834in}{0.808392in}}%
\pgfpathlineto{\pgfqpoint{1.167497in}{0.802903in}}%
\pgfpathlineto{\pgfqpoint{1.117948in}{0.789329in}}%
\pgfpathlineto{\pgfqpoint{1.069511in}{0.768097in}}%
\pgfpathlineto{\pgfqpoint{1.018466in}{0.737329in}}%
\pgfpathlineto{\pgfqpoint{0.963812in}{0.695157in}}%
\pgfpathlineto{\pgfqpoint{0.915739in}{0.650564in}}%
\pgfusepath{stroke}%
\end{pgfscope}%
\begin{pgfscope}%
\pgfpathrectangle{\pgfqpoint{0.647939in}{0.492442in}}{\pgfqpoint{4.273799in}{2.331163in}}%
\pgfusepath{clip}%
\pgfsetbuttcap%
\pgfsetroundjoin%
\pgfsetlinewidth{0.301125pt}%
\definecolor{currentstroke}{rgb}{0.500000,0.500000,0.500000}%
\pgfsetstrokecolor{currentstroke}%
\pgfsetstrokeopacity{0.300000}%
\pgfsetdash{}{0pt}%
\pgfpathmoveto{\pgfqpoint{1.910652in}{0.492442in}}%
\pgfpathlineto{\pgfqpoint{1.910652in}{0.492442in}}%
\pgfpathlineto{\pgfqpoint{1.872229in}{0.539816in}}%
\pgfpathlineto{\pgfqpoint{1.833293in}{0.587066in}}%
\pgfpathlineto{\pgfqpoint{1.793678in}{0.634146in}}%
\pgfpathlineto{\pgfqpoint{1.753168in}{0.680998in}}%
\pgfpathlineto{\pgfqpoint{1.711481in}{0.727541in}}%
\pgfpathlineto{\pgfqpoint{1.668235in}{0.773656in}}%
\pgfpathlineto{\pgfqpoint{1.622886in}{0.819160in}}%
\pgfpathlineto{\pgfqpoint{1.574629in}{0.863756in}}%
\pgfpathlineto{\pgfqpoint{1.522201in}{0.906906in}}%
\pgfpathlineto{\pgfqpoint{1.463485in}{0.947530in}}%
\pgfpathlineto{\pgfqpoint{1.394791in}{0.982884in}}%
\pgfpathlineto{\pgfqpoint{1.394791in}{0.982884in}}%
\pgfpathlineto{\pgfqpoint{1.335848in}{1.001427in}}%
\pgfpathlineto{\pgfqpoint{1.335848in}{1.001427in}}%
\pgfpathlineto{\pgfqpoint{1.282238in}{1.007890in}}%
\pgfpathlineto{\pgfqpoint{1.227697in}{1.003753in}}%
\pgfpathlineto{\pgfqpoint{1.180587in}{0.991430in}}%
\pgfpathlineto{\pgfqpoint{1.134333in}{0.971471in}}%
\pgfusepath{stroke}%
\end{pgfscope}%
\begin{pgfscope}%
\pgfpathrectangle{\pgfqpoint{0.647939in}{0.492442in}}{\pgfqpoint{4.273799in}{2.331163in}}%
\pgfusepath{clip}%
\pgfsetbuttcap%
\pgfsetroundjoin%
\pgfsetlinewidth{0.301125pt}%
\definecolor{currentstroke}{rgb}{0.500000,0.500000,0.500000}%
\pgfsetstrokecolor{currentstroke}%
\pgfsetstrokeopacity{0.300000}%
\pgfsetdash{}{0pt}%
\pgfpathmoveto{\pgfqpoint{2.007784in}{0.492442in}}%
\pgfpathlineto{\pgfqpoint{2.007784in}{0.492442in}}%
\pgfpathlineto{\pgfqpoint{1.970631in}{0.540117in}}%
\pgfpathlineto{\pgfqpoint{1.933279in}{0.587746in}}%
\pgfpathlineto{\pgfqpoint{1.895630in}{0.635305in}}%
\pgfpathlineto{\pgfqpoint{1.857562in}{0.682765in}}%
\pgfpathlineto{\pgfqpoint{1.818915in}{0.730085in}}%
\pgfpathlineto{\pgfqpoint{1.779489in}{0.777212in}}%
\pgfpathlineto{\pgfqpoint{1.739030in}{0.824077in}}%
\pgfpathlineto{\pgfqpoint{1.697191in}{0.870578in}}%
\pgfpathlineto{\pgfqpoint{1.653485in}{0.916562in}}%
\pgfpathlineto{\pgfqpoint{1.607195in}{0.961780in}}%
\pgfpathlineto{\pgfqpoint{1.557194in}{1.005794in}}%
\pgfpathlineto{\pgfqpoint{1.501578in}{1.047711in}}%
\pgfpathlineto{\pgfqpoint{1.436917in}{1.085348in}}%
\pgfpathlineto{\pgfqpoint{1.436917in}{1.085348in}}%
\pgfpathlineto{\pgfqpoint{1.377185in}{1.108009in}}%
\pgfpathlineto{\pgfqpoint{1.377185in}{1.108009in}}%
\pgfpathlineto{\pgfqpoint{1.324427in}{1.117256in}}%
\pgfpathlineto{\pgfqpoint{1.268695in}{1.115344in}}%
\pgfpathlineto{\pgfqpoint{1.222286in}{1.104565in}}%
\pgfpathlineto{\pgfqpoint{1.177763in}{1.086414in}}%
\pgfpathlineto{\pgfqpoint{1.130887in}{1.059278in}}%
\pgfpathlineto{\pgfqpoint{1.079540in}{1.020571in}}%
\pgfpathlineto{\pgfqpoint{1.031118in}{0.976078in}}%
\pgfusepath{stroke}%
\end{pgfscope}%
\begin{pgfscope}%
\pgfpathrectangle{\pgfqpoint{0.647939in}{0.492442in}}{\pgfqpoint{4.273799in}{2.331163in}}%
\pgfusepath{clip}%
\pgfsetbuttcap%
\pgfsetroundjoin%
\pgfsetlinewidth{0.301125pt}%
\definecolor{currentstroke}{rgb}{0.500000,0.500000,0.500000}%
\pgfsetstrokecolor{currentstroke}%
\pgfsetstrokeopacity{0.300000}%
\pgfsetdash{}{0pt}%
\pgfpathmoveto{\pgfqpoint{2.104916in}{0.492442in}}%
\pgfpathlineto{\pgfqpoint{2.104916in}{0.492442in}}%
\pgfpathlineto{\pgfqpoint{2.068485in}{0.540283in}}%
\pgfpathlineto{\pgfqpoint{2.032078in}{0.588129in}}%
\pgfpathlineto{\pgfqpoint{1.995625in}{0.635965in}}%
\pgfpathlineto{\pgfqpoint{1.959046in}{0.683772in}}%
\pgfpathlineto{\pgfqpoint{1.922252in}{0.731529in}}%
\pgfpathlineto{\pgfqpoint{1.885135in}{0.779212in}}%
\pgfpathlineto{\pgfqpoint{1.847561in}{0.826789in}}%
\pgfpathlineto{\pgfqpoint{1.809361in}{0.874216in}}%
\pgfpathlineto{\pgfqpoint{1.770311in}{0.921437in}}%
\pgfpathlineto{\pgfqpoint{1.730105in}{0.968367in}}%
\pgfpathlineto{\pgfqpoint{1.688317in}{1.014882in}}%
\pgfpathlineto{\pgfqpoint{1.644331in}{1.060786in}}%
\pgfpathlineto{\pgfqpoint{1.597195in}{1.105743in}}%
\pgfpathlineto{\pgfqpoint{1.545294in}{1.149077in}}%
\pgfpathlineto{\pgfqpoint{1.485686in}{1.189209in}}%
\pgfpathlineto{\pgfqpoint{1.412924in}{1.221651in}}%
\pgfpathlineto{\pgfqpoint{1.412924in}{1.221651in}}%
\pgfpathlineto{\pgfqpoint{1.362080in}{1.232188in}}%
\pgfpathlineto{\pgfqpoint{1.306683in}{1.231364in}}%
\pgfpathlineto{\pgfqpoint{1.261656in}{1.221261in}}%
\pgfpathlineto{\pgfqpoint{1.219047in}{1.203983in}}%
\pgfpathlineto{\pgfqpoint{1.173971in}{1.177835in}}%
\pgfpathlineto{\pgfqpoint{1.124209in}{1.140145in}}%
\pgfpathlineto{\pgfqpoint{1.076040in}{1.095579in}}%
\pgfpathlineto{\pgfqpoint{1.032864in}{1.049480in}}%
\pgfusepath{stroke}%
\end{pgfscope}%
\begin{pgfscope}%
\pgfpathrectangle{\pgfqpoint{0.647939in}{0.492442in}}{\pgfqpoint{4.273799in}{2.331163in}}%
\pgfusepath{clip}%
\pgfsetbuttcap%
\pgfsetroundjoin%
\pgfsetlinewidth{0.301125pt}%
\definecolor{currentstroke}{rgb}{0.500000,0.500000,0.500000}%
\pgfsetstrokecolor{currentstroke}%
\pgfsetstrokeopacity{0.300000}%
\pgfsetdash{}{0pt}%
\pgfpathmoveto{\pgfqpoint{2.396312in}{0.492442in}}%
\pgfpathlineto{\pgfqpoint{2.396312in}{0.492442in}}%
\pgfpathlineto{\pgfqpoint{2.359592in}{0.540217in}}%
\pgfpathlineto{\pgfqpoint{2.323315in}{0.588092in}}%
\pgfpathlineto{\pgfqpoint{2.287454in}{0.636061in}}%
\pgfpathlineto{\pgfqpoint{2.251979in}{0.684115in}}%
\pgfpathlineto{\pgfqpoint{2.216854in}{0.732245in}}%
\pgfpathlineto{\pgfqpoint{2.182051in}{0.780445in}}%
\pgfpathlineto{\pgfqpoint{2.147536in}{0.828707in}}%
\pgfpathlineto{\pgfqpoint{2.113277in}{0.877022in}}%
\pgfpathlineto{\pgfqpoint{2.079229in}{0.925383in}}%
\pgfpathlineto{\pgfqpoint{2.045341in}{0.973776in}}%
\pgfpathlineto{\pgfqpoint{2.011559in}{1.022192in}}%
\pgfpathlineto{\pgfqpoint{1.977828in}{1.070618in}}%
\pgfpathlineto{\pgfqpoint{1.944077in}{1.119040in}}%
\pgfpathlineto{\pgfqpoint{1.910207in}{1.167437in}}%
\pgfpathlineto{\pgfqpoint{1.876095in}{1.215783in}}%
\pgfpathlineto{\pgfqpoint{1.841597in}{1.264046in}}%
\pgfpathlineto{\pgfqpoint{1.806519in}{1.312185in}}%
\pgfpathlineto{\pgfqpoint{1.770584in}{1.360135in}}%
\pgfpathlineto{\pgfqpoint{1.733365in}{1.407792in}}%
\pgfpathlineto{\pgfqpoint{1.694192in}{1.454973in}}%
\pgfpathlineto{\pgfqpoint{1.651932in}{1.501338in}}%
\pgfpathlineto{\pgfqpoint{1.604381in}{1.546104in}}%
\pgfpathlineto{\pgfqpoint{1.546402in}{1.586833in}}%
\pgfpathlineto{\pgfqpoint{1.546402in}{1.586833in}}%
\pgfpathlineto{\pgfqpoint{1.502644in}{1.605296in}}%
\pgfpathlineto{\pgfqpoint{1.502644in}{1.605296in}}%
\pgfpathlineto{\pgfqpoint{1.462220in}{1.611736in}}%
\pgfpathlineto{\pgfqpoint{1.420520in}{1.607410in}}%
\pgfpathlineto{\pgfqpoint{1.386192in}{1.596002in}}%
\pgfpathlineto{\pgfqpoint{1.350676in}{1.577229in}}%
\pgfpathlineto{\pgfqpoint{1.310348in}{1.548271in}}%
\pgfpathlineto{\pgfqpoint{1.262620in}{1.505117in}}%
\pgfpathlineto{\pgfqpoint{1.219311in}{1.459115in}}%
\pgfpathlineto{\pgfqpoint{1.179591in}{1.412124in}}%
\pgfusepath{stroke}%
\end{pgfscope}%
\begin{pgfscope}%
\pgfpathrectangle{\pgfqpoint{0.647939in}{0.492442in}}{\pgfqpoint{4.273799in}{2.331163in}}%
\pgfusepath{clip}%
\pgfsetbuttcap%
\pgfsetroundjoin%
\pgfsetlinewidth{0.301125pt}%
\definecolor{currentstroke}{rgb}{0.500000,0.500000,0.500000}%
\pgfsetstrokecolor{currentstroke}%
\pgfsetstrokeopacity{0.300000}%
\pgfsetdash{}{0pt}%
\pgfpathmoveto{\pgfqpoint{2.590575in}{0.492442in}}%
\pgfpathlineto{\pgfqpoint{2.590575in}{0.492442in}}%
\pgfpathlineto{\pgfqpoint{2.552058in}{0.539794in}}%
\pgfpathlineto{\pgfqpoint{2.514187in}{0.587300in}}%
\pgfpathlineto{\pgfqpoint{2.476941in}{0.634953in}}%
\pgfpathlineto{\pgfqpoint{2.440300in}{0.682746in}}%
\pgfpathlineto{\pgfqpoint{2.404242in}{0.730670in}}%
\pgfpathlineto{\pgfqpoint{2.368748in}{0.778720in}}%
\pgfpathlineto{\pgfqpoint{2.333798in}{0.826888in}}%
\pgfpathlineto{\pgfqpoint{2.299368in}{0.875167in}}%
\pgfpathlineto{\pgfqpoint{2.265433in}{0.923551in}}%
\pgfpathlineto{\pgfqpoint{2.231976in}{0.972034in}}%
\pgfpathlineto{\pgfqpoint{2.198982in}{1.020610in}}%
\pgfpathlineto{\pgfqpoint{2.166425in}{1.069275in}}%
\pgfpathlineto{\pgfqpoint{2.134280in}{1.118021in}}%
\pgfpathlineto{\pgfqpoint{2.102526in}{1.166843in}}%
\pgfpathlineto{\pgfqpoint{2.071149in}{1.215738in}}%
\pgfpathlineto{\pgfqpoint{2.040119in}{1.264698in}}%
\pgfpathlineto{\pgfqpoint{2.009401in}{1.313717in}}%
\pgfpathlineto{\pgfqpoint{1.978975in}{1.362790in}}%
\pgfpathlineto{\pgfqpoint{1.948809in}{1.411911in}}%
\pgfpathlineto{\pgfqpoint{1.918854in}{1.461069in}}%
\pgfpathlineto{\pgfqpoint{1.889061in}{1.510257in}}%
\pgfpathlineto{\pgfqpoint{1.859377in}{1.559464in}}%
\pgfpathlineto{\pgfqpoint{1.829702in}{1.608673in}}%
\pgfpathlineto{\pgfqpoint{1.799901in}{1.657856in}}%
\pgfpathlineto{\pgfqpoint{1.769800in}{1.706986in}}%
\pgfpathlineto{\pgfqpoint{1.739047in}{1.755993in}}%
\pgfpathlineto{\pgfqpoint{1.706965in}{1.804729in}}%
\pgfpathlineto{\pgfqpoint{1.672044in}{1.852840in}}%
\pgfpathlineto{\pgfqpoint{1.629115in}{1.898659in}}%
\pgfpathlineto{\pgfqpoint{1.629115in}{1.898659in}}%
\pgfpathlineto{\pgfqpoint{1.604177in}{1.914567in}}%
\pgfpathlineto{\pgfqpoint{1.604177in}{1.914567in}}%
\pgfpathlineto{\pgfqpoint{1.580576in}{1.919904in}}%
\pgfpathlineto{\pgfqpoint{1.555580in}{1.915567in}}%
\pgfpathlineto{\pgfqpoint{1.534198in}{1.905387in}}%
\pgfpathlineto{\pgfqpoint{1.509794in}{1.888132in}}%
\pgfpathlineto{\pgfqpoint{1.475106in}{1.856888in}}%
\pgfpathlineto{\pgfqpoint{1.431824in}{1.811058in}}%
\pgfpathlineto{\pgfqpoint{1.391580in}{1.764287in}}%
\pgfpathlineto{\pgfqpoint{1.353280in}{1.717002in}}%
\pgfpathlineto{\pgfqpoint{1.316408in}{1.669359in}}%
\pgfpathlineto{\pgfqpoint{1.280682in}{1.621431in}}%
\pgfpathlineto{\pgfqpoint{1.245921in}{1.573265in}}%
\pgfusepath{stroke}%
\end{pgfscope}%
\begin{pgfscope}%
\pgfpathrectangle{\pgfqpoint{0.647939in}{0.492442in}}{\pgfqpoint{4.273799in}{2.331163in}}%
\pgfusepath{clip}%
\pgfsetbuttcap%
\pgfsetroundjoin%
\pgfsetlinewidth{0.301125pt}%
\definecolor{currentstroke}{rgb}{0.500000,0.500000,0.500000}%
\pgfsetstrokecolor{currentstroke}%
\pgfsetstrokeopacity{0.300000}%
\pgfsetdash{}{0pt}%
\pgfpathmoveto{\pgfqpoint{2.687707in}{0.492442in}}%
\pgfpathlineto{\pgfqpoint{2.687707in}{0.492442in}}%
\pgfpathlineto{\pgfqpoint{2.647861in}{0.539465in}}%
\pgfpathlineto{\pgfqpoint{2.608762in}{0.586674in}}%
\pgfpathlineto{\pgfqpoint{2.570389in}{0.634060in}}%
\pgfpathlineto{\pgfqpoint{2.532722in}{0.681615in}}%
\pgfpathlineto{\pgfqpoint{2.495743in}{0.729330in}}%
\pgfpathlineto{\pgfqpoint{2.459434in}{0.777198in}}%
\pgfpathlineto{\pgfqpoint{2.423773in}{0.825210in}}%
\pgfpathlineto{\pgfqpoint{2.388739in}{0.873360in}}%
\pgfpathlineto{\pgfqpoint{2.354315in}{0.921641in}}%
\pgfpathlineto{\pgfqpoint{2.320487in}{0.970046in}}%
\pgfpathlineto{\pgfqpoint{2.287243in}{1.018572in}}%
\pgfpathlineto{\pgfqpoint{2.254567in}{1.067213in}}%
\pgfpathlineto{\pgfqpoint{2.222439in}{1.115962in}}%
\pgfpathlineto{\pgfqpoint{2.190853in}{1.164817in}}%
\pgfpathlineto{\pgfqpoint{2.159805in}{1.213774in}}%
\pgfpathlineto{\pgfqpoint{2.129283in}{1.262829in}}%
\pgfpathlineto{\pgfqpoint{2.099276in}{1.311979in}}%
\pgfpathlineto{\pgfqpoint{2.069792in}{1.361222in}}%
\pgfpathlineto{\pgfqpoint{2.040831in}{1.410558in}}%
\pgfpathlineto{\pgfqpoint{2.012392in}{1.459984in}}%
\pgfpathlineto{\pgfqpoint{1.984499in}{1.509503in}}%
\pgfpathlineto{\pgfqpoint{1.957171in}{1.559115in}}%
\pgfpathlineto{\pgfqpoint{1.930431in}{1.608822in}}%
\pgfpathlineto{\pgfqpoint{1.904346in}{1.658632in}}%
\pgfpathlineto{\pgfqpoint{1.878984in}{1.708552in}}%
\pgfpathlineto{\pgfqpoint{1.854469in}{1.758598in}}%
\pgfpathlineto{\pgfqpoint{1.830989in}{1.808790in}}%
\pgfpathlineto{\pgfqpoint{1.808850in}{1.859162in}}%
\pgfpathlineto{\pgfqpoint{1.788596in}{1.909766in}}%
\pgfpathlineto{\pgfqpoint{1.771241in}{1.960682in}}%
\pgfpathlineto{\pgfqpoint{1.758807in}{2.012001in}}%
\pgfpathlineto{\pgfqpoint{1.755437in}{2.063652in}}%
\pgfpathlineto{\pgfqpoint{1.767639in}{2.114671in}}%
\pgfpathlineto{\pgfqpoint{1.796076in}{2.161805in}}%
\pgfpathlineto{\pgfqpoint{1.834795in}{2.204842in}}%
\pgfpathlineto{\pgfqpoint{1.884432in}{2.248666in}}%
\pgfpathlineto{\pgfqpoint{1.939985in}{2.290424in}}%
\pgfpathlineto{\pgfqpoint{2.000791in}{2.330055in}}%
\pgfpathlineto{\pgfqpoint{2.066871in}{2.367105in}}%
\pgfpathlineto{\pgfqpoint{2.138601in}{2.400847in}}%
\pgfpathlineto{\pgfqpoint{2.216599in}{2.430117in}}%
\pgfpathlineto{\pgfqpoint{2.301105in}{2.453172in}}%
\pgfpathlineto{\pgfqpoint{2.391555in}{2.467714in}}%
\pgfpathlineto{\pgfqpoint{2.481101in}{2.471560in}}%
\pgfpathlineto{\pgfqpoint{2.563348in}{2.465494in}}%
\pgfpathlineto{\pgfqpoint{2.640192in}{2.450982in}}%
\pgfpathlineto{\pgfqpoint{2.713758in}{2.428419in}}%
\pgfpathlineto{\pgfqpoint{2.784865in}{2.397619in}}%
\pgfpathlineto{\pgfqpoint{2.850282in}{2.360325in}}%
\pgfpathlineto{\pgfqpoint{2.907552in}{2.319147in}}%
\pgfpathlineto{\pgfqpoint{2.957152in}{2.275102in}}%
\pgfpathlineto{\pgfqpoint{2.999387in}{2.228812in}}%
\pgfpathlineto{\pgfqpoint{3.034094in}{2.180687in}}%
\pgfpathlineto{\pgfqpoint{3.060386in}{2.131000in}}%
\pgfpathlineto{\pgfqpoint{3.075871in}{2.080044in}}%
\pgfpathlineto{\pgfqpoint{3.073401in}{2.028752in}}%
\pgfpathlineto{\pgfqpoint{3.073401in}{2.028752in}}%
\pgfpathlineto{\pgfqpoint{3.059398in}{2.005933in}}%
\pgfpathlineto{\pgfqpoint{3.059398in}{2.005933in}}%
\pgfpathlineto{\pgfqpoint{3.039583in}{1.995067in}}%
\pgfpathlineto{\pgfqpoint{3.039583in}{1.995067in}}%
\pgfpathlineto{\pgfqpoint{3.016069in}{1.993902in}}%
\pgfpathlineto{\pgfqpoint{2.994292in}{1.999998in}}%
\pgfpathlineto{\pgfqpoint{2.973546in}{2.012065in}}%
\pgfpathlineto{\pgfqpoint{2.953860in}{2.032166in}}%
\pgfpathlineto{\pgfqpoint{2.945429in}{2.057446in}}%
\pgfpathlineto{\pgfqpoint{2.945429in}{2.057446in}}%
\pgfpathlineto{\pgfqpoint{2.951856in}{2.063560in}}%
\pgfpathlineto{\pgfqpoint{2.958873in}{2.061763in}}%
\pgfpathlineto{\pgfqpoint{2.963646in}{2.056631in}}%
\pgfusepath{stroke}%
\end{pgfscope}%
\begin{pgfscope}%
\pgfpathrectangle{\pgfqpoint{0.647939in}{0.492442in}}{\pgfqpoint{4.273799in}{2.331163in}}%
\pgfusepath{clip}%
\pgfsetbuttcap%
\pgfsetroundjoin%
\pgfsetlinewidth{0.301125pt}%
\definecolor{currentstroke}{rgb}{0.500000,0.500000,0.500000}%
\pgfsetstrokecolor{currentstroke}%
\pgfsetstrokeopacity{0.300000}%
\pgfsetdash{}{0pt}%
\pgfpathmoveto{\pgfqpoint{2.784839in}{0.492442in}}%
\pgfpathlineto{\pgfqpoint{2.784839in}{0.492442in}}%
\pgfpathlineto{\pgfqpoint{2.743401in}{0.539053in}}%
\pgfpathlineto{\pgfqpoint{2.702807in}{0.585885in}}%
\pgfpathlineto{\pgfqpoint{2.663039in}{0.632928in}}%
\pgfpathlineto{\pgfqpoint{2.624077in}{0.680171in}}%
\pgfpathlineto{\pgfqpoint{2.585904in}{0.727605in}}%
\pgfpathlineto{\pgfqpoint{2.548502in}{0.775221in}}%
\pgfpathlineto{\pgfqpoint{2.511848in}{0.823011in}}%
\pgfpathlineto{\pgfqpoint{2.475925in}{0.870965in}}%
\pgfusepath{stroke}%
\end{pgfscope}%
\begin{pgfscope}%
\pgfpathrectangle{\pgfqpoint{0.647939in}{0.492442in}}{\pgfqpoint{4.273799in}{2.331163in}}%
\pgfusepath{clip}%
\pgfsetbuttcap%
\pgfsetroundjoin%
\pgfsetlinewidth{0.301125pt}%
\definecolor{currentstroke}{rgb}{0.500000,0.500000,0.500000}%
\pgfsetstrokecolor{currentstroke}%
\pgfsetstrokeopacity{0.300000}%
\pgfsetdash{}{0pt}%
\pgfpathmoveto{\pgfqpoint{2.881971in}{0.492442in}}%
\pgfpathlineto{\pgfqpoint{2.881971in}{0.492442in}}%
\pgfpathlineto{\pgfqpoint{2.838691in}{0.538553in}}%
\pgfpathlineto{\pgfqpoint{2.796350in}{0.584922in}}%
\pgfpathlineto{\pgfqpoint{2.754932in}{0.631538in}}%
\pgfpathlineto{\pgfqpoint{2.714419in}{0.678391in}}%
\pgfpathlineto{\pgfqpoint{2.674795in}{0.725470in}}%
\pgfusepath{stroke}%
\end{pgfscope}%
\begin{pgfscope}%
\pgfpathrectangle{\pgfqpoint{0.647939in}{0.492442in}}{\pgfqpoint{4.273799in}{2.331163in}}%
\pgfusepath{clip}%
\pgfsetbuttcap%
\pgfsetroundjoin%
\pgfsetlinewidth{0.301125pt}%
\definecolor{currentstroke}{rgb}{0.500000,0.500000,0.500000}%
\pgfsetstrokecolor{currentstroke}%
\pgfsetstrokeopacity{0.300000}%
\pgfsetdash{}{0pt}%
\pgfpathmoveto{\pgfqpoint{3.076234in}{0.492442in}}%
\pgfpathlineto{\pgfqpoint{3.076234in}{0.492442in}}%
\pgfpathlineto{\pgfqpoint{3.028631in}{0.537266in}}%
\pgfpathlineto{\pgfqpoint{2.982123in}{0.582432in}}%
\pgfpathlineto{\pgfqpoint{2.936713in}{0.627929in}}%
\pgfpathlineto{\pgfqpoint{2.892393in}{0.673745in}}%
\pgfpathlineto{\pgfqpoint{2.849156in}{0.719867in}}%
\pgfpathlineto{\pgfqpoint{2.806986in}{0.766283in}}%
\pgfpathlineto{\pgfqpoint{2.765868in}{0.812979in}}%
\pgfpathlineto{\pgfqpoint{2.725784in}{0.859941in}}%
\pgfpathlineto{\pgfqpoint{2.686718in}{0.907158in}}%
\pgfpathlineto{\pgfqpoint{2.648653in}{0.954617in}}%
\pgfpathlineto{\pgfqpoint{2.611574in}{1.002308in}}%
\pgfpathlineto{\pgfqpoint{2.575470in}{1.050221in}}%
\pgfpathlineto{\pgfqpoint{2.540332in}{1.098348in}}%
\pgfpathlineto{\pgfqpoint{2.506156in}{1.146680in}}%
\pgfpathlineto{\pgfqpoint{2.472936in}{1.195210in}}%
\pgfpathlineto{\pgfqpoint{2.440671in}{1.243932in}}%
\pgfpathlineto{\pgfqpoint{2.409373in}{1.292841in}}%
\pgfpathlineto{\pgfqpoint{2.379062in}{1.341934in}}%
\pgfpathlineto{\pgfqpoint{2.349754in}{1.391209in}}%
\pgfpathlineto{\pgfqpoint{2.321483in}{1.440663in}}%
\pgfpathlineto{\pgfqpoint{2.294300in}{1.490298in}}%
\pgfpathlineto{\pgfqpoint{2.268258in}{1.540115in}}%
\pgfpathlineto{\pgfqpoint{2.243439in}{1.590117in}}%
\pgfpathlineto{\pgfqpoint{2.219944in}{1.640309in}}%
\pgfpathlineto{\pgfqpoint{2.197901in}{1.690695in}}%
\pgfpathlineto{\pgfqpoint{2.177479in}{1.741285in}}%
\pgfpathlineto{\pgfqpoint{2.158897in}{1.792084in}}%
\pgfpathlineto{\pgfqpoint{2.142435in}{1.843099in}}%
\pgfpathlineto{\pgfqpoint{2.128452in}{1.894332in}}%
\pgfpathlineto{\pgfqpoint{2.117436in}{1.945776in}}%
\pgfpathlineto{\pgfqpoint{2.110002in}{1.997409in}}%
\pgfpathlineto{\pgfqpoint{2.106952in}{2.049168in}}%
\pgfpathlineto{\pgfqpoint{2.109314in}{2.100926in}}%
\pgfpathlineto{\pgfqpoint{2.118346in}{2.152447in}}%
\pgfpathlineto{\pgfqpoint{2.135542in}{2.203320in}}%
\pgfpathlineto{\pgfqpoint{2.162480in}{2.252886in}}%
\pgfpathlineto{\pgfqpoint{2.200710in}{2.300142in}}%
\pgfpathlineto{\pgfqpoint{2.251497in}{2.343650in}}%
\pgfpathlineto{\pgfqpoint{2.315802in}{2.381295in}}%
\pgfpathlineto{\pgfqpoint{2.390719in}{2.409431in}}%
\pgfpathlineto{\pgfqpoint{2.467409in}{2.425154in}}%
\pgfusepath{stroke}%
\end{pgfscope}%
\begin{pgfscope}%
\pgfpathrectangle{\pgfqpoint{0.647939in}{0.492442in}}{\pgfqpoint{4.273799in}{2.331163in}}%
\pgfusepath{clip}%
\pgfsetbuttcap%
\pgfsetroundjoin%
\pgfsetlinewidth{0.301125pt}%
\definecolor{currentstroke}{rgb}{0.500000,0.500000,0.500000}%
\pgfsetstrokecolor{currentstroke}%
\pgfsetstrokeopacity{0.300000}%
\pgfsetdash{}{0pt}%
\pgfpathmoveto{\pgfqpoint{3.270498in}{0.492442in}}%
\pgfpathlineto{\pgfqpoint{3.270498in}{0.492442in}}%
\pgfpathlineto{\pgfqpoint{3.218074in}{0.535636in}}%
\pgfpathlineto{\pgfqpoint{3.166789in}{0.579234in}}%
\pgfpathlineto{\pgfqpoint{3.116687in}{0.623240in}}%
\pgfpathlineto{\pgfqpoint{3.067800in}{0.667651in}}%
\pgfpathlineto{\pgfqpoint{3.020146in}{0.712460in}}%
\pgfpathlineto{\pgfqpoint{2.973732in}{0.757653in}}%
\pgfpathlineto{\pgfqpoint{2.928558in}{0.803219in}}%
\pgfpathlineto{\pgfqpoint{2.884614in}{0.849142in}}%
\pgfpathlineto{\pgfqpoint{2.841890in}{0.895405in}}%
\pgfpathlineto{\pgfqpoint{2.800370in}{0.941994in}}%
\pgfpathlineto{\pgfqpoint{2.760040in}{0.988893in}}%
\pgfpathlineto{\pgfqpoint{2.720885in}{1.036088in}}%
\pgfpathlineto{\pgfqpoint{2.682891in}{1.083564in}}%
\pgfpathlineto{\pgfqpoint{2.646043in}{1.131308in}}%
\pgfpathlineto{\pgfqpoint{2.610335in}{1.179309in}}%
\pgfpathlineto{\pgfqpoint{2.575765in}{1.227557in}}%
\pgfpathlineto{\pgfqpoint{2.542337in}{1.276044in}}%
\pgfpathlineto{\pgfqpoint{2.510060in}{1.324763in}}%
\pgfpathlineto{\pgfqpoint{2.478947in}{1.373707in}}%
\pgfpathlineto{\pgfqpoint{2.449024in}{1.422870in}}%
\pgfpathlineto{\pgfqpoint{2.420335in}{1.472252in}}%
\pgfpathlineto{\pgfqpoint{2.392926in}{1.521850in}}%
\pgfpathlineto{\pgfqpoint{2.366863in}{1.571663in}}%
\pgfpathlineto{\pgfqpoint{2.342241in}{1.621693in}}%
\pgfpathlineto{\pgfqpoint{2.319168in}{1.671943in}}%
\pgfpathlineto{\pgfqpoint{2.297796in}{1.722415in}}%
\pgfpathlineto{\pgfqpoint{2.278311in}{1.773113in}}%
\pgfpathlineto{\pgfqpoint{2.260952in}{1.824039in}}%
\pgfpathlineto{\pgfqpoint{2.246034in}{1.875194in}}%
\pgfpathlineto{\pgfqpoint{2.233954in}{1.926569in}}%
\pgfpathlineto{\pgfqpoint{2.225225in}{1.978143in}}%
\pgfpathlineto{\pgfqpoint{2.220522in}{2.029867in}}%
\pgfpathlineto{\pgfqpoint{2.220721in}{2.081646in}}%
\pgfpathlineto{\pgfqpoint{2.226952in}{2.133300in}}%
\pgfpathlineto{\pgfqpoint{2.240661in}{2.184503in}}%
\pgfpathlineto{\pgfqpoint{2.263667in}{2.234675in}}%
\pgfpathlineto{\pgfqpoint{2.298169in}{2.282786in}}%
\pgfusepath{stroke}%
\end{pgfscope}%
\begin{pgfscope}%
\pgfpathrectangle{\pgfqpoint{0.647939in}{0.492442in}}{\pgfqpoint{4.273799in}{2.331163in}}%
\pgfusepath{clip}%
\pgfsetbuttcap%
\pgfsetroundjoin%
\pgfsetlinewidth{0.301125pt}%
\definecolor{currentstroke}{rgb}{0.500000,0.500000,0.500000}%
\pgfsetstrokecolor{currentstroke}%
\pgfsetstrokeopacity{0.300000}%
\pgfsetdash{}{0pt}%
\pgfpathmoveto{\pgfqpoint{3.464761in}{0.492442in}}%
\pgfpathlineto{\pgfqpoint{3.464761in}{0.492442in}}%
\pgfpathlineto{\pgfqpoint{3.407776in}{0.533883in}}%
\pgfpathlineto{\pgfqpoint{3.351712in}{0.575695in}}%
\pgfpathlineto{\pgfqpoint{3.296710in}{0.617924in}}%
\pgfpathlineto{\pgfqpoint{3.242875in}{0.660598in}}%
\pgfpathlineto{\pgfqpoint{3.190294in}{0.703735in}}%
\pgfpathlineto{\pgfqpoint{3.139031in}{0.747340in}}%
\pgfpathlineto{\pgfqpoint{3.089129in}{0.791412in}}%
\pgfpathlineto{\pgfqpoint{3.040609in}{0.835941in}}%
\pgfpathlineto{\pgfqpoint{2.993484in}{0.880914in}}%
\pgfpathlineto{\pgfqpoint{2.947753in}{0.926313in}}%
\pgfpathlineto{\pgfqpoint{2.903412in}{0.972120in}}%
\pgfpathlineto{\pgfqpoint{2.860447in}{1.018317in}}%
\pgfpathlineto{\pgfqpoint{2.818846in}{1.064883in}}%
\pgfpathlineto{\pgfqpoint{2.778594in}{1.111801in}}%
\pgfpathlineto{\pgfqpoint{2.739677in}{1.159053in}}%
\pgfpathlineto{\pgfqpoint{2.702087in}{1.206623in}}%
\pgfpathlineto{\pgfqpoint{2.665816in}{1.254498in}}%
\pgfusepath{stroke}%
\end{pgfscope}%
\begin{pgfscope}%
\pgfpathrectangle{\pgfqpoint{0.647939in}{0.492442in}}{\pgfqpoint{4.273799in}{2.331163in}}%
\pgfusepath{clip}%
\pgfsetbuttcap%
\pgfsetroundjoin%
\pgfsetlinewidth{0.301125pt}%
\definecolor{currentstroke}{rgb}{0.500000,0.500000,0.500000}%
\pgfsetstrokecolor{currentstroke}%
\pgfsetstrokeopacity{0.300000}%
\pgfsetdash{}{0pt}%
\pgfpathmoveto{\pgfqpoint{3.659025in}{0.492442in}}%
\pgfpathlineto{\pgfqpoint{3.659025in}{0.492442in}}%
\pgfpathlineto{\pgfqpoint{3.598886in}{0.532536in}}%
\pgfpathlineto{\pgfqpoint{3.539042in}{0.572761in}}%
\pgfpathlineto{\pgfqpoint{3.479758in}{0.613231in}}%
\pgfpathlineto{\pgfqpoint{3.421270in}{0.654044in}}%
\pgfpathlineto{\pgfqpoint{3.363772in}{0.695272in}}%
\pgfpathlineto{\pgfqpoint{3.307439in}{0.736975in}}%
\pgfpathlineto{\pgfqpoint{3.252408in}{0.779191in}}%
\pgfpathlineto{\pgfqpoint{3.198774in}{0.821939in}}%
\pgfpathlineto{\pgfqpoint{3.146612in}{0.865227in}}%
\pgfpathlineto{\pgfqpoint{3.095976in}{0.909048in}}%
\pgfpathlineto{\pgfqpoint{3.046893in}{0.953392in}}%
\pgfpathlineto{\pgfqpoint{2.999374in}{0.998240in}}%
\pgfpathlineto{\pgfqpoint{2.953419in}{1.043570in}}%
\pgfpathlineto{\pgfqpoint{2.909020in}{1.089360in}}%
\pgfpathlineto{\pgfqpoint{2.866167in}{1.135586in}}%
\pgfusepath{stroke}%
\end{pgfscope}%
\begin{pgfscope}%
\pgfpathrectangle{\pgfqpoint{0.647939in}{0.492442in}}{\pgfqpoint{4.273799in}{2.331163in}}%
\pgfusepath{clip}%
\pgfsetbuttcap%
\pgfsetroundjoin%
\pgfsetlinewidth{0.301125pt}%
\definecolor{currentstroke}{rgb}{0.500000,0.500000,0.500000}%
\pgfsetstrokecolor{currentstroke}%
\pgfsetstrokeopacity{0.300000}%
\pgfsetdash{}{0pt}%
\pgfpathmoveto{\pgfqpoint{3.853289in}{0.492442in}}%
\pgfpathlineto{\pgfqpoint{3.853289in}{0.492442in}}%
\pgfpathlineto{\pgfqpoint{3.792667in}{0.532318in}}%
\pgfpathlineto{\pgfqpoint{3.731335in}{0.571871in}}%
\pgfpathlineto{\pgfqpoint{3.669610in}{0.611242in}}%
\pgfpathlineto{\pgfqpoint{3.607834in}{0.650588in}}%
\pgfpathlineto{\pgfqpoint{3.546341in}{0.690066in}}%
\pgfpathlineto{\pgfqpoint{3.485434in}{0.729812in}}%
\pgfpathlineto{\pgfqpoint{3.425399in}{0.769950in}}%
\pgfpathlineto{\pgfqpoint{3.366469in}{0.810572in}}%
\pgfpathlineto{\pgfqpoint{3.308848in}{0.851747in}}%
\pgfpathlineto{\pgfqpoint{3.252700in}{0.893522in}}%
\pgfpathlineto{\pgfqpoint{3.198132in}{0.935916in}}%
\pgfpathlineto{\pgfqpoint{3.145230in}{0.978933in}}%
\pgfpathlineto{\pgfqpoint{3.094049in}{1.022565in}}%
\pgfpathlineto{\pgfqpoint{3.044618in}{1.066792in}}%
\pgfpathlineto{\pgfqpoint{2.996944in}{1.111590in}}%
\pgfpathlineto{\pgfqpoint{2.951026in}{1.156930in}}%
\pgfpathlineto{\pgfqpoint{2.906854in}{1.202784in}}%
\pgfpathlineto{\pgfqpoint{2.864415in}{1.249122in}}%
\pgfpathlineto{\pgfqpoint{2.823697in}{1.295918in}}%
\pgfpathlineto{\pgfqpoint{2.784691in}{1.343146in}}%
\pgfpathlineto{\pgfqpoint{2.747394in}{1.390783in}}%
\pgfpathlineto{\pgfqpoint{2.711815in}{1.438810in}}%
\pgfpathlineto{\pgfqpoint{2.677974in}{1.487210in}}%
\pgfpathlineto{\pgfqpoint{2.645902in}{1.535967in}}%
\pgfpathlineto{\pgfqpoint{2.615649in}{1.585068in}}%
\pgfpathlineto{\pgfqpoint{2.587296in}{1.634506in}}%
\pgfpathlineto{\pgfqpoint{2.560949in}{1.684271in}}%
\pgfpathlineto{\pgfqpoint{2.536744in}{1.734359in}}%
\pgfpathlineto{\pgfqpoint{2.514874in}{1.784765in}}%
\pgfpathlineto{\pgfqpoint{2.495586in}{1.835483in}}%
\pgfpathlineto{\pgfqpoint{2.479213in}{1.886503in}}%
\pgfpathlineto{\pgfqpoint{2.466196in}{1.937806in}}%
\pgfpathlineto{\pgfqpoint{2.457134in}{1.989357in}}%
\pgfpathlineto{\pgfqpoint{2.452850in}{2.041085in}}%
\pgfpathlineto{\pgfqpoint{2.454516in}{2.092847in}}%
\pgfpathlineto{\pgfqpoint{2.463845in}{2.144340in}}%
\pgfpathlineto{\pgfqpoint{2.483483in}{2.194897in}}%
\pgfpathlineto{\pgfqpoint{2.517630in}{2.242931in}}%
\pgfpathlineto{\pgfqpoint{2.517630in}{2.242931in}}%
\pgfpathlineto{\pgfqpoint{2.558561in}{2.276450in}}%
\pgfpathlineto{\pgfqpoint{2.558561in}{2.276450in}}%
\pgfpathlineto{\pgfqpoint{2.604295in}{2.298139in}}%
\pgfpathlineto{\pgfqpoint{2.660952in}{2.310255in}}%
\pgfpathlineto{\pgfqpoint{2.714104in}{2.310756in}}%
\pgfpathlineto{\pgfqpoint{2.764646in}{2.302911in}}%
\pgfpathlineto{\pgfqpoint{2.814817in}{2.287488in}}%
\pgfpathlineto{\pgfqpoint{2.865396in}{2.263966in}}%
\pgfpathlineto{\pgfqpoint{2.916032in}{2.231255in}}%
\pgfpathlineto{\pgfqpoint{2.964169in}{2.188837in}}%
\pgfpathlineto{\pgfqpoint{3.002441in}{2.141683in}}%
\pgfusepath{stroke}%
\end{pgfscope}%
\begin{pgfscope}%
\pgfpathrectangle{\pgfqpoint{0.647939in}{0.492442in}}{\pgfqpoint{4.273799in}{2.331163in}}%
\pgfusepath{clip}%
\pgfsetbuttcap%
\pgfsetroundjoin%
\pgfsetlinewidth{0.301125pt}%
\definecolor{currentstroke}{rgb}{0.500000,0.500000,0.500000}%
\pgfsetstrokecolor{currentstroke}%
\pgfsetstrokeopacity{0.300000}%
\pgfsetdash{}{0pt}%
\pgfpathmoveto{\pgfqpoint{3.950420in}{0.492442in}}%
\pgfpathlineto{\pgfqpoint{3.950420in}{0.492442in}}%
\pgfpathlineto{\pgfqpoint{3.890897in}{0.532805in}}%
\pgfpathlineto{\pgfqpoint{3.830092in}{0.572597in}}%
\pgfpathlineto{\pgfqpoint{3.768310in}{0.611941in}}%
\pgfpathlineto{\pgfqpoint{3.705884in}{0.650981in}}%
\pgfpathlineto{\pgfqpoint{3.643180in}{0.689888in}}%
\pgfusepath{stroke}%
\end{pgfscope}%
\begin{pgfscope}%
\pgfpathrectangle{\pgfqpoint{0.647939in}{0.492442in}}{\pgfqpoint{4.273799in}{2.331163in}}%
\pgfusepath{clip}%
\pgfsetbuttcap%
\pgfsetroundjoin%
\pgfsetlinewidth{0.301125pt}%
\definecolor{currentstroke}{rgb}{0.500000,0.500000,0.500000}%
\pgfsetstrokecolor{currentstroke}%
\pgfsetstrokeopacity{0.300000}%
\pgfsetdash{}{0pt}%
\pgfpathmoveto{\pgfqpoint{4.144684in}{0.492442in}}%
\pgfpathlineto{\pgfqpoint{4.144684in}{0.492442in}}%
\pgfpathlineto{\pgfqpoint{4.090229in}{0.534878in}}%
\pgfpathlineto{\pgfqpoint{4.033628in}{0.576468in}}%
\pgfpathlineto{\pgfqpoint{3.974947in}{0.617193in}}%
\pgfpathlineto{\pgfqpoint{3.914357in}{0.657079in}}%
\pgfpathlineto{\pgfqpoint{3.852124in}{0.696207in}}%
\pgfpathlineto{\pgfqpoint{3.788568in}{0.734699in}}%
\pgfpathlineto{\pgfqpoint{3.724092in}{0.772734in}}%
\pgfpathlineto{\pgfqpoint{3.659131in}{0.810524in}}%
\pgfpathlineto{\pgfqpoint{3.594139in}{0.848297in}}%
\pgfpathlineto{\pgfqpoint{3.529560in}{0.886279in}}%
\pgfpathlineto{\pgfqpoint{3.465804in}{0.924671in}}%
\pgfpathlineto{\pgfqpoint{3.403232in}{0.963637in}}%
\pgfpathlineto{\pgfqpoint{3.342148in}{1.003296in}}%
\pgfpathlineto{\pgfqpoint{3.282786in}{1.043726in}}%
\pgfpathlineto{\pgfqpoint{3.225328in}{1.084965in}}%
\pgfpathlineto{\pgfqpoint{3.169901in}{1.127020in}}%
\pgfpathlineto{\pgfqpoint{3.116574in}{1.169876in}}%
\pgfpathlineto{\pgfqpoint{3.065385in}{1.213504in}}%
\pgfpathlineto{\pgfqpoint{3.016350in}{1.257862in}}%
\pgfpathlineto{\pgfqpoint{2.969468in}{1.302906in}}%
\pgfpathlineto{\pgfqpoint{2.924724in}{1.348592in}}%
\pgfpathlineto{\pgfqpoint{2.882103in}{1.394879in}}%
\pgfpathlineto{\pgfqpoint{2.841595in}{1.441727in}}%
\pgfpathlineto{\pgfqpoint{2.803199in}{1.489102in}}%
\pgfpathlineto{\pgfqpoint{2.766929in}{1.536974in}}%
\pgfpathlineto{\pgfqpoint{2.732813in}{1.585314in}}%
\pgfpathlineto{\pgfqpoint{2.700906in}{1.634101in}}%
\pgfpathlineto{\pgfqpoint{2.671301in}{1.683316in}}%
\pgfpathlineto{\pgfqpoint{2.644126in}{1.732948in}}%
\pgfpathlineto{\pgfqpoint{2.619556in}{1.782980in}}%
\pgfpathlineto{\pgfqpoint{2.597844in}{1.833403in}}%
\pgfpathlineto{\pgfqpoint{2.579330in}{1.884203in}}%
\pgfpathlineto{\pgfqpoint{2.564488in}{1.935357in}}%
\pgfpathlineto{\pgfqpoint{2.553990in}{1.986823in}}%
\pgfpathlineto{\pgfqpoint{2.548803in}{2.038519in}}%
\pgfpathlineto{\pgfqpoint{2.550391in}{2.090268in}}%
\pgfpathlineto{\pgfqpoint{2.561112in}{2.141653in}}%
\pgfpathlineto{\pgfqpoint{2.585069in}{2.191559in}}%
\pgfpathlineto{\pgfqpoint{2.585069in}{2.191559in}}%
\pgfpathlineto{\pgfqpoint{2.618761in}{2.228586in}}%
\pgfpathlineto{\pgfqpoint{2.618761in}{2.228586in}}%
\pgfpathlineto{\pgfqpoint{2.657750in}{2.252321in}}%
\pgfpathlineto{\pgfqpoint{2.657750in}{2.252321in}}%
\pgfpathlineto{\pgfqpoint{2.700335in}{2.265142in}}%
\pgfusepath{stroke}%
\end{pgfscope}%
\begin{pgfscope}%
\pgfpathrectangle{\pgfqpoint{0.647939in}{0.492442in}}{\pgfqpoint{4.273799in}{2.331163in}}%
\pgfusepath{clip}%
\pgfsetbuttcap%
\pgfsetroundjoin%
\pgfsetlinewidth{0.301125pt}%
\definecolor{currentstroke}{rgb}{0.500000,0.500000,0.500000}%
\pgfsetstrokecolor{currentstroke}%
\pgfsetstrokeopacity{0.300000}%
\pgfsetdash{}{0pt}%
\pgfpathmoveto{\pgfqpoint{4.338948in}{0.492442in}}%
\pgfpathlineto{\pgfqpoint{4.338948in}{0.492442in}}%
\pgfpathlineto{\pgfqpoint{4.292693in}{0.537678in}}%
\pgfpathlineto{\pgfqpoint{4.244206in}{0.582212in}}%
\pgfpathlineto{\pgfqpoint{4.193321in}{0.625942in}}%
\pgfpathlineto{\pgfqpoint{4.139898in}{0.668763in}}%
\pgfpathlineto{\pgfqpoint{4.083870in}{0.710580in}}%
\pgfpathlineto{\pgfqpoint{4.025219in}{0.751312in}}%
\pgfpathlineto{\pgfqpoint{3.964011in}{0.790911in}}%
\pgfpathlineto{\pgfqpoint{3.900467in}{0.829401in}}%
\pgfpathlineto{\pgfqpoint{3.834912in}{0.866878in}}%
\pgfpathlineto{\pgfqpoint{3.767800in}{0.903529in}}%
\pgfpathlineto{\pgfqpoint{3.699678in}{0.939624in}}%
\pgfpathlineto{\pgfqpoint{3.631131in}{0.975479in}}%
\pgfpathlineto{\pgfqpoint{3.562769in}{1.011437in}}%
\pgfpathlineto{\pgfqpoint{3.495160in}{1.047815in}}%
\pgfpathlineto{\pgfqpoint{3.428816in}{1.084875in}}%
\pgfpathlineto{\pgfqpoint{3.364169in}{1.122813in}}%
\pgfpathlineto{\pgfqpoint{3.301554in}{1.161750in}}%
\pgfpathlineto{\pgfqpoint{3.241218in}{1.201742in}}%
\pgfpathlineto{\pgfqpoint{3.183325in}{1.242794in}}%
\pgfpathlineto{\pgfqpoint{3.127978in}{1.284874in}}%
\pgfpathlineto{\pgfqpoint{3.075213in}{1.327932in}}%
\pgfpathlineto{\pgfqpoint{3.025037in}{1.371904in}}%
\pgfpathlineto{\pgfqpoint{2.977441in}{1.416723in}}%
\pgfpathlineto{\pgfqpoint{2.932404in}{1.462323in}}%
\pgfpathlineto{\pgfqpoint{2.889910in}{1.508644in}}%
\pgfpathlineto{\pgfqpoint{2.849950in}{1.555631in}}%
\pgfpathlineto{\pgfqpoint{2.812538in}{1.603239in}}%
\pgfpathlineto{\pgfqpoint{2.777711in}{1.651425in}}%
\pgfpathlineto{\pgfqpoint{2.745546in}{1.700157in}}%
\pgfpathlineto{\pgfqpoint{2.716172in}{1.749411in}}%
\pgfpathlineto{\pgfqpoint{2.689777in}{1.799164in}}%
\pgfpathlineto{\pgfqpoint{2.666634in}{1.849394in}}%
\pgfpathlineto{\pgfqpoint{2.647143in}{1.900080in}}%
\pgfusepath{stroke}%
\end{pgfscope}%
\begin{pgfscope}%
\pgfpathrectangle{\pgfqpoint{0.647939in}{0.492442in}}{\pgfqpoint{4.273799in}{2.331163in}}%
\pgfusepath{clip}%
\pgfsetbuttcap%
\pgfsetroundjoin%
\pgfsetlinewidth{0.301125pt}%
\definecolor{currentstroke}{rgb}{0.500000,0.500000,0.500000}%
\pgfsetstrokecolor{currentstroke}%
\pgfsetstrokeopacity{0.300000}%
\pgfsetdash{}{0pt}%
\pgfpathmoveto{\pgfqpoint{4.533211in}{0.492442in}}%
\pgfpathlineto{\pgfqpoint{4.533211in}{0.492442in}}%
\pgfpathlineto{\pgfqpoint{4.496234in}{0.540155in}}%
\pgfpathlineto{\pgfqpoint{4.457704in}{0.587499in}}%
\pgfpathlineto{\pgfqpoint{4.417435in}{0.634409in}}%
\pgfpathlineto{\pgfqpoint{4.375216in}{0.680804in}}%
\pgfpathlineto{\pgfqpoint{4.330808in}{0.726585in}}%
\pgfpathlineto{\pgfqpoint{4.283945in}{0.771632in}}%
\pgfpathlineto{\pgfqpoint{4.234335in}{0.815793in}}%
\pgfpathlineto{\pgfqpoint{4.181686in}{0.858894in}}%
\pgfpathlineto{\pgfqpoint{4.125751in}{0.900739in}}%
\pgfpathlineto{\pgfqpoint{4.066314in}{0.941119in}}%
\pgfpathlineto{\pgfqpoint{4.003278in}{0.979842in}}%
\pgfpathlineto{\pgfqpoint{3.936768in}{1.016795in}}%
\pgfpathlineto{\pgfqpoint{3.867134in}{1.052001in}}%
\pgfpathlineto{\pgfqpoint{3.794967in}{1.085663in}}%
\pgfpathlineto{\pgfqpoint{3.721066in}{1.118194in}}%
\pgfpathlineto{\pgfqpoint{3.646352in}{1.150172in}}%
\pgfpathlineto{\pgfqpoint{3.571741in}{1.182219in}}%
\pgfpathlineto{\pgfqpoint{3.498081in}{1.214909in}}%
\pgfpathlineto{\pgfqpoint{3.426161in}{1.248723in}}%
\pgfpathlineto{\pgfqpoint{3.356617in}{1.283972in}}%
\pgfpathlineto{\pgfqpoint{3.289891in}{1.320805in}}%
\pgfpathlineto{\pgfqpoint{3.226306in}{1.359252in}}%
\pgfpathlineto{\pgfqpoint{3.166038in}{1.399256in}}%
\pgfpathlineto{\pgfqpoint{3.109150in}{1.440713in}}%
\pgfpathlineto{\pgfqpoint{3.055651in}{1.483494in}}%
\pgfpathlineto{\pgfqpoint{3.005514in}{1.527467in}}%
\pgfpathlineto{\pgfqpoint{2.958687in}{1.572513in}}%
\pgfpathlineto{\pgfqpoint{2.915128in}{1.618526in}}%
\pgfpathlineto{\pgfqpoint{2.874828in}{1.665415in}}%
\pgfpathlineto{\pgfqpoint{2.837816in}{1.713106in}}%
\pgfpathlineto{\pgfqpoint{2.804185in}{1.761538in}}%
\pgfpathlineto{\pgfqpoint{2.774108in}{1.810661in}}%
\pgfpathlineto{\pgfqpoint{2.747866in}{1.860430in}}%
\pgfusepath{stroke}%
\end{pgfscope}%
\begin{pgfscope}%
\pgfpathrectangle{\pgfqpoint{0.647939in}{0.492442in}}{\pgfqpoint{4.273799in}{2.331163in}}%
\pgfusepath{clip}%
\pgfsetbuttcap%
\pgfsetroundjoin%
\pgfsetlinewidth{0.301125pt}%
\definecolor{currentstroke}{rgb}{0.500000,0.500000,0.500000}%
\pgfsetstrokecolor{currentstroke}%
\pgfsetstrokeopacity{0.300000}%
\pgfsetdash{}{0pt}%
\pgfpathmoveto{\pgfqpoint{4.630343in}{0.492442in}}%
\pgfpathlineto{\pgfqpoint{4.630343in}{0.492442in}}%
\pgfpathlineto{\pgfqpoint{4.597797in}{0.541105in}}%
\pgfpathlineto{\pgfqpoint{4.564098in}{0.589534in}}%
\pgfpathlineto{\pgfqpoint{4.529106in}{0.637690in}}%
\pgfpathlineto{\pgfqpoint{4.492665in}{0.685525in}}%
\pgfpathlineto{\pgfqpoint{4.454595in}{0.732980in}}%
\pgfpathlineto{\pgfqpoint{4.414681in}{0.779980in}}%
\pgfpathlineto{\pgfqpoint{4.372661in}{0.826427in}}%
\pgfpathlineto{\pgfqpoint{4.328223in}{0.872198in}}%
\pgfpathlineto{\pgfqpoint{4.281006in}{0.917131in}}%
\pgfpathlineto{\pgfqpoint{4.230589in}{0.961017in}}%
\pgfpathlineto{\pgfqpoint{4.176548in}{1.003595in}}%
\pgfpathlineto{\pgfqpoint{4.118473in}{1.044551in}}%
\pgfpathlineto{\pgfqpoint{4.055964in}{1.083512in}}%
\pgfpathlineto{\pgfqpoint{3.988901in}{1.120148in}}%
\pgfpathlineto{\pgfqpoint{3.917483in}{1.154244in}}%
\pgfpathlineto{\pgfqpoint{3.842283in}{1.185844in}}%
\pgfpathlineto{\pgfqpoint{3.764299in}{1.215389in}}%
\pgfpathlineto{\pgfqpoint{3.684754in}{1.243683in}}%
\pgfpathlineto{\pgfqpoint{3.604877in}{1.271699in}}%
\pgfpathlineto{\pgfqpoint{3.525835in}{1.300396in}}%
\pgfpathlineto{\pgfqpoint{3.448658in}{1.330547in}}%
\pgfpathlineto{\pgfqpoint{3.374204in}{1.362663in}}%
\pgfpathlineto{\pgfqpoint{3.303153in}{1.396980in}}%
\pgfpathlineto{\pgfqpoint{3.235905in}{1.433506in}}%
\pgfpathlineto{\pgfqpoint{3.172679in}{1.472115in}}%
\pgfusepath{stroke}%
\end{pgfscope}%
\begin{pgfscope}%
\pgfpathrectangle{\pgfqpoint{0.647939in}{0.492442in}}{\pgfqpoint{4.273799in}{2.331163in}}%
\pgfusepath{clip}%
\pgfsetbuttcap%
\pgfsetroundjoin%
\pgfsetlinewidth{0.301125pt}%
\definecolor{currentstroke}{rgb}{0.500000,0.500000,0.500000}%
\pgfsetstrokecolor{currentstroke}%
\pgfsetstrokeopacity{0.300000}%
\pgfsetdash{}{0pt}%
\pgfpathmoveto{\pgfqpoint{4.727475in}{0.492442in}}%
\pgfpathlineto{\pgfqpoint{4.727475in}{0.492442in}}%
\pgfpathlineto{\pgfqpoint{4.698975in}{0.541856in}}%
\pgfpathlineto{\pgfqpoint{4.669684in}{0.591132in}}%
\pgfpathlineto{\pgfqpoint{4.639526in}{0.640252in}}%
\pgfpathlineto{\pgfqpoint{4.608403in}{0.689192in}}%
\pgfpathlineto{\pgfqpoint{4.576188in}{0.737922in}}%
\pgfpathlineto{\pgfqpoint{4.542753in}{0.786406in}}%
\pgfpathlineto{\pgfqpoint{4.507943in}{0.834599in}}%
\pgfpathlineto{\pgfqpoint{4.471551in}{0.882441in}}%
\pgfpathlineto{\pgfqpoint{4.433325in}{0.929854in}}%
\pgfpathlineto{\pgfqpoint{4.392955in}{0.976735in}}%
\pgfpathlineto{\pgfqpoint{4.350061in}{1.022941in}}%
\pgfpathlineto{\pgfqpoint{4.304173in}{1.068278in}}%
\pgfpathlineto{\pgfqpoint{4.254708in}{1.112474in}}%
\pgfpathlineto{\pgfqpoint{4.200954in}{1.155140in}}%
\pgfpathlineto{\pgfqpoint{4.142070in}{1.195735in}}%
\pgfpathlineto{\pgfqpoint{4.077339in}{1.233563in}}%
\pgfpathlineto{\pgfqpoint{4.006290in}{1.267824in}}%
\pgfpathlineto{\pgfqpoint{3.929111in}{1.297879in}}%
\pgfpathlineto{\pgfqpoint{3.846912in}{1.323712in}}%
\pgfpathlineto{\pgfqpoint{3.761445in}{1.346233in}}%
\pgfpathlineto{\pgfqpoint{3.674506in}{1.367063in}}%
\pgfpathlineto{\pgfqpoint{3.587702in}{1.388062in}}%
\pgfpathlineto{\pgfqpoint{3.502459in}{1.410845in}}%
\pgfpathlineto{\pgfqpoint{3.420061in}{1.436506in}}%
\pgfpathlineto{\pgfqpoint{3.341581in}{1.465548in}}%
\pgfpathlineto{\pgfqpoint{3.267749in}{1.498000in}}%
\pgfpathlineto{\pgfqpoint{3.198920in}{1.533603in}}%
\pgfpathlineto{\pgfqpoint{3.135249in}{1.571961in}}%
\pgfpathlineto{\pgfqpoint{3.076660in}{1.612663in}}%
\pgfpathlineto{\pgfqpoint{3.022979in}{1.655350in}}%
\pgfpathlineto{\pgfqpoint{2.974051in}{1.699715in}}%
\pgfpathlineto{\pgfqpoint{2.929772in}{1.745510in}}%
\pgfpathlineto{\pgfqpoint{2.890114in}{1.792552in}}%
\pgfpathlineto{\pgfqpoint{2.855190in}{1.840699in}}%
\pgfusepath{stroke}%
\end{pgfscope}%
\begin{pgfscope}%
\pgfpathrectangle{\pgfqpoint{0.647939in}{0.492442in}}{\pgfqpoint{4.273799in}{2.331163in}}%
\pgfusepath{clip}%
\pgfsetbuttcap%
\pgfsetroundjoin%
\pgfsetlinewidth{0.301125pt}%
\definecolor{currentstroke}{rgb}{0.500000,0.500000,0.500000}%
\pgfsetstrokecolor{currentstroke}%
\pgfsetstrokeopacity{0.300000}%
\pgfsetdash{}{0pt}%
\pgfpathmoveto{\pgfqpoint{4.824607in}{0.492442in}}%
\pgfpathlineto{\pgfqpoint{4.824607in}{0.492442in}}%
\pgfpathlineto{\pgfqpoint{4.799718in}{0.542434in}}%
\pgfpathlineto{\pgfqpoint{4.774338in}{0.592352in}}%
\pgfpathlineto{\pgfqpoint{4.748417in}{0.642188in}}%
\pgfpathlineto{\pgfqpoint{4.721912in}{0.691931in}}%
\pgfpathlineto{\pgfqpoint{4.694769in}{0.741572in}}%
\pgfpathlineto{\pgfqpoint{4.666913in}{0.791095in}}%
\pgfpathlineto{\pgfqpoint{4.638268in}{0.840484in}}%
\pgfpathlineto{\pgfqpoint{4.608742in}{0.889717in}}%
\pgfpathlineto{\pgfqpoint{4.578206in}{0.938767in}}%
\pgfpathlineto{\pgfqpoint{4.546525in}{0.987600in}}%
\pgfpathlineto{\pgfqpoint{4.513529in}{1.036170in}}%
\pgfpathlineto{\pgfqpoint{4.478981in}{1.084417in}}%
\pgfpathlineto{\pgfqpoint{4.442576in}{1.132256in}}%
\pgfpathlineto{\pgfqpoint{4.403926in}{1.179569in}}%
\pgfpathlineto{\pgfqpoint{4.362514in}{1.226178in}}%
\pgfpathlineto{\pgfqpoint{4.317628in}{1.271813in}}%
\pgfpathlineto{\pgfqpoint{4.268269in}{1.316040in}}%
\pgfpathlineto{\pgfqpoint{4.213113in}{1.358138in}}%
\pgfpathlineto{\pgfqpoint{4.150365in}{1.396871in}}%
\pgfpathlineto{\pgfqpoint{4.078207in}{1.430251in}}%
\pgfpathlineto{\pgfqpoint{3.996246in}{1.455835in}}%
\pgfpathlineto{\pgfqpoint{3.907067in}{1.472449in}}%
\pgfpathlineto{\pgfqpoint{3.813951in}{1.482026in}}%
\pgfpathlineto{\pgfqpoint{3.719716in}{1.488276in}}%
\pgfpathlineto{\pgfqpoint{3.625621in}{1.494988in}}%
\pgfpathlineto{\pgfqpoint{3.532667in}{1.505148in}}%
\pgfpathlineto{\pgfqpoint{3.442320in}{1.520627in}}%
\pgfpathlineto{\pgfqpoint{3.356279in}{1.542113in}}%
\pgfpathlineto{\pgfqpoint{3.275890in}{1.569376in}}%
\pgfpathlineto{\pgfqpoint{3.201893in}{1.601654in}}%
\pgfpathlineto{\pgfqpoint{3.134535in}{1.638013in}}%
\pgfpathlineto{\pgfqpoint{3.073580in}{1.677625in}}%
\pgfusepath{stroke}%
\end{pgfscope}%
\begin{pgfscope}%
\pgfpathrectangle{\pgfqpoint{0.647939in}{0.492442in}}{\pgfqpoint{4.273799in}{2.331163in}}%
\pgfusepath{clip}%
\pgfsetbuttcap%
\pgfsetroundjoin%
\pgfsetlinewidth{0.301125pt}%
\definecolor{currentstroke}{rgb}{0.500000,0.500000,0.500000}%
\pgfsetstrokecolor{currentstroke}%
\pgfsetstrokeopacity{0.300000}%
\pgfsetdash{}{0pt}%
\pgfpathmoveto{\pgfqpoint{4.921738in}{0.492442in}}%
\pgfpathlineto{\pgfqpoint{4.921738in}{0.492442in}}%
\pgfpathlineto{\pgfqpoint{4.900014in}{0.542872in}}%
\pgfpathlineto{\pgfqpoint{4.878017in}{0.593266in}}%
\pgfpathlineto{\pgfqpoint{4.855732in}{0.643622in}}%
\pgfpathlineto{\pgfqpoint{4.833141in}{0.693938in}}%
\pgfpathlineto{\pgfqpoint{4.810223in}{0.744209in}}%
\pgfpathlineto{\pgfqpoint{4.786955in}{0.794433in}}%
\pgfpathlineto{\pgfqpoint{4.763309in}{0.844604in}}%
\pgfpathlineto{\pgfqpoint{4.739255in}{0.894717in}}%
\pgfpathlineto{\pgfqpoint{4.714754in}{0.944765in}}%
\pgfpathlineto{\pgfqpoint{4.689765in}{0.994741in}}%
\pgfpathlineto{\pgfqpoint{4.664231in}{1.044635in}}%
\pgfpathlineto{\pgfqpoint{4.638091in}{1.094436in}}%
\pgfpathlineto{\pgfqpoint{4.611273in}{1.144128in}}%
\pgfpathlineto{\pgfqpoint{4.583671in}{1.193692in}}%
\pgfpathlineto{\pgfqpoint{4.555169in}{1.243103in}}%
\pgfpathlineto{\pgfqpoint{4.525599in}{1.292326in}}%
\pgfpathlineto{\pgfqpoint{4.494735in}{1.341313in}}%
\pgfpathlineto{\pgfqpoint{4.462290in}{1.389991in}}%
\pgfpathlineto{\pgfqpoint{4.427836in}{1.438250in}}%
\pgfpathlineto{\pgfqpoint{4.390703in}{1.485907in}}%
\pgfpathlineto{\pgfqpoint{4.349830in}{1.532631in}}%
\pgfpathlineto{\pgfqpoint{4.303408in}{1.577747in}}%
\pgfpathlineto{\pgfqpoint{4.248103in}{1.619622in}}%
\pgfpathlineto{\pgfqpoint{4.248103in}{1.619622in}}%
\pgfpathlineto{\pgfqpoint{4.190479in}{1.649118in}}%
\pgfpathlineto{\pgfqpoint{4.190479in}{1.649118in}}%
\pgfpathlineto{\pgfqpoint{4.136030in}{1.664457in}}%
\pgfpathlineto{\pgfqpoint{4.075086in}{1.669569in}}%
\pgfpathlineto{\pgfqpoint{4.017792in}{1.665740in}}%
\pgfpathlineto{\pgfqpoint{3.954066in}{1.655224in}}%
\pgfpathlineto{\pgfqpoint{3.868693in}{1.636418in}}%
\pgfpathlineto{\pgfqpoint{3.781058in}{1.616655in}}%
\pgfpathlineto{\pgfqpoint{3.691387in}{1.599892in}}%
\pgfpathlineto{\pgfqpoint{3.598936in}{1.588922in}}%
\pgfpathlineto{\pgfqpoint{3.504690in}{1.586144in}}%
\pgfpathlineto{\pgfqpoint{3.411296in}{1.593084in}}%
\pgfpathlineto{\pgfqpoint{3.326773in}{1.608614in}}%
\pgfpathlineto{\pgfqpoint{3.247643in}{1.632060in}}%
\pgfusepath{stroke}%
\end{pgfscope}%
\begin{pgfscope}%
\pgfpathrectangle{\pgfqpoint{0.647939in}{0.492442in}}{\pgfqpoint{4.273799in}{2.331163in}}%
\pgfusepath{clip}%
\pgfsetbuttcap%
\pgfsetroundjoin%
\pgfsetlinewidth{0.301125pt}%
\definecolor{currentstroke}{rgb}{0.500000,0.500000,0.500000}%
\pgfsetstrokecolor{currentstroke}%
\pgfsetstrokeopacity{0.300000}%
\pgfsetdash{}{0pt}%
\pgfpathmoveto{\pgfqpoint{4.921738in}{0.704366in}}%
\pgfpathlineto{\pgfqpoint{4.921738in}{0.704366in}}%
\pgfpathlineto{\pgfqpoint{4.901769in}{0.755011in}}%
\pgfpathlineto{\pgfqpoint{4.881653in}{0.805638in}}%
\pgfpathlineto{\pgfqpoint{4.861385in}{0.856248in}}%
\pgfpathlineto{\pgfqpoint{4.840964in}{0.906838in}}%
\pgfpathlineto{\pgfqpoint{4.820383in}{0.957410in}}%
\pgfpathlineto{\pgfqpoint{4.799641in}{1.007962in}}%
\pgfpathlineto{\pgfqpoint{4.778732in}{1.058493in}}%
\pgfpathlineto{\pgfqpoint{4.757652in}{1.109003in}}%
\pgfpathlineto{\pgfqpoint{4.736396in}{1.159491in}}%
\pgfpathlineto{\pgfqpoint{4.714959in}{1.209956in}}%
\pgfpathlineto{\pgfqpoint{4.693335in}{1.260397in}}%
\pgfpathlineto{\pgfqpoint{4.671517in}{1.310813in}}%
\pgfpathlineto{\pgfqpoint{4.649499in}{1.361202in}}%
\pgfpathlineto{\pgfqpoint{4.627271in}{1.411565in}}%
\pgfpathlineto{\pgfqpoint{4.604826in}{1.461898in}}%
\pgfpathlineto{\pgfqpoint{4.582149in}{1.512200in}}%
\pgfpathlineto{\pgfqpoint{4.559230in}{1.562468in}}%
\pgfpathlineto{\pgfqpoint{4.536045in}{1.612700in}}%
\pgfpathlineto{\pgfqpoint{4.512575in}{1.662892in}}%
\pgfpathlineto{\pgfqpoint{4.488780in}{1.713037in}}%
\pgfpathlineto{\pgfqpoint{4.464612in}{1.763128in}}%
\pgfpathlineto{\pgfqpoint{4.439993in}{1.813147in}}%
\pgfpathlineto{\pgfqpoint{4.414748in}{1.863066in}}%
\pgfpathlineto{\pgfqpoint{4.388531in}{1.912807in}}%
\pgfpathlineto{\pgfqpoint{4.360060in}{1.962172in}}%
\pgfpathlineto{\pgfqpoint{4.321490in}{2.005912in}}%
\pgfpathlineto{\pgfqpoint{4.321490in}{2.005912in}}%
\pgfpathlineto{\pgfqpoint{4.321490in}{2.005912in}}%
\pgfpathlineto{\pgfqpoint{4.307750in}{2.008090in}}%
\pgfpathlineto{\pgfqpoint{4.295723in}{2.004587in}}%
\pgfpathlineto{\pgfqpoint{4.280555in}{1.995424in}}%
\pgfpathlineto{\pgfqpoint{4.258140in}{1.977990in}}%
\pgfpathlineto{\pgfqpoint{4.216976in}{1.943296in}}%
\pgfpathlineto{\pgfqpoint{4.165322in}{1.900539in}}%
\pgfpathlineto{\pgfqpoint{4.111352in}{1.858266in}}%
\pgfpathlineto{\pgfqpoint{4.054486in}{1.816920in}}%
\pgfpathlineto{\pgfqpoint{3.994184in}{1.777065in}}%
\pgfpathlineto{\pgfqpoint{3.929878in}{1.739093in}}%
\pgfpathlineto{\pgfqpoint{3.860853in}{1.703652in}}%
\pgfpathlineto{\pgfqpoint{3.786365in}{1.671701in}}%
\pgfpathlineto{\pgfqpoint{3.705811in}{1.644559in}}%
\pgfusepath{stroke}%
\end{pgfscope}%
\begin{pgfscope}%
\pgfpathrectangle{\pgfqpoint{0.647939in}{0.492442in}}{\pgfqpoint{4.273799in}{2.331163in}}%
\pgfusepath{clip}%
\pgfsetbuttcap%
\pgfsetroundjoin%
\pgfsetlinewidth{0.301125pt}%
\definecolor{currentstroke}{rgb}{0.500000,0.500000,0.500000}%
\pgfsetstrokecolor{currentstroke}%
\pgfsetstrokeopacity{0.300000}%
\pgfsetdash{}{0pt}%
\pgfpathmoveto{\pgfqpoint{4.921738in}{0.969271in}}%
\pgfpathlineto{\pgfqpoint{4.921738in}{0.969271in}}%
\pgfpathlineto{\pgfqpoint{4.904336in}{1.020197in}}%
\pgfpathlineto{\pgfqpoint{4.886986in}{1.071128in}}%
\pgfpathlineto{\pgfqpoint{4.869704in}{1.122066in}}%
\pgfpathlineto{\pgfqpoint{4.852512in}{1.173012in}}%
\pgfpathlineto{\pgfqpoint{4.835431in}{1.223971in}}%
\pgfpathlineto{\pgfqpoint{4.818494in}{1.274943in}}%
\pgfpathlineto{\pgfqpoint{4.801739in}{1.325933in}}%
\pgfpathlineto{\pgfqpoint{4.785200in}{1.376944in}}%
\pgfpathlineto{\pgfqpoint{4.768931in}{1.427981in}}%
\pgfpathlineto{\pgfqpoint{4.752987in}{1.479048in}}%
\pgfpathlineto{\pgfqpoint{4.737439in}{1.530151in}}%
\pgfpathlineto{\pgfqpoint{4.722376in}{1.581297in}}%
\pgfpathlineto{\pgfqpoint{4.707894in}{1.632494in}}%
\pgfpathlineto{\pgfqpoint{4.694123in}{1.683748in}}%
\pgfpathlineto{\pgfqpoint{4.681223in}{1.735070in}}%
\pgfpathlineto{\pgfqpoint{4.669377in}{1.786466in}}%
\pgfpathlineto{\pgfqpoint{4.658814in}{1.837946in}}%
\pgfpathlineto{\pgfqpoint{4.649820in}{1.889514in}}%
\pgfpathlineto{\pgfqpoint{4.642729in}{1.941168in}}%
\pgfpathlineto{\pgfqpoint{4.637924in}{1.992900in}}%
\pgfpathlineto{\pgfqpoint{4.635824in}{2.044684in}}%
\pgfpathlineto{\pgfqpoint{4.636850in}{2.096476in}}%
\pgfpathlineto{\pgfqpoint{4.641361in}{2.148209in}}%
\pgfpathlineto{\pgfqpoint{4.649576in}{2.199803in}}%
\pgfpathlineto{\pgfqpoint{4.661513in}{2.251180in}}%
\pgfpathlineto{\pgfqpoint{4.676942in}{2.302280in}}%
\pgfpathlineto{\pgfqpoint{4.695437in}{2.353074in}}%
\pgfpathlineto{\pgfqpoint{4.716490in}{2.403573in}}%
\pgfpathlineto{\pgfqpoint{4.739551in}{2.453808in}}%
\pgfpathlineto{\pgfqpoint{4.764120in}{2.503829in}}%
\pgfpathlineto{\pgfqpoint{4.789785in}{2.553691in}}%
\pgfpathlineto{\pgfqpoint{4.816193in}{2.603438in}}%
\pgfpathlineto{\pgfqpoint{4.843074in}{2.653107in}}%
\pgfpathlineto{\pgfqpoint{4.870239in}{2.702735in}}%
\pgfpathlineto{\pgfqpoint{4.897523in}{2.752341in}}%
\pgfpathlineto{\pgfqpoint{4.921738in}{2.796326in}}%
\pgfusepath{stroke}%
\end{pgfscope}%
\begin{pgfscope}%
\pgfpathrectangle{\pgfqpoint{0.647939in}{0.492442in}}{\pgfqpoint{4.273799in}{2.331163in}}%
\pgfusepath{clip}%
\pgfsetbuttcap%
\pgfsetroundjoin%
\pgfsetlinewidth{0.301125pt}%
\definecolor{currentstroke}{rgb}{0.500000,0.500000,0.500000}%
\pgfsetstrokecolor{currentstroke}%
\pgfsetstrokeopacity{0.300000}%
\pgfsetdash{}{0pt}%
\pgfpathmoveto{\pgfqpoint{4.921738in}{1.234176in}}%
\pgfpathlineto{\pgfqpoint{4.921738in}{1.234176in}}%
\pgfpathlineto{\pgfqpoint{4.907427in}{1.285387in}}%
\pgfpathlineto{\pgfqpoint{4.893424in}{1.336624in}}%
\pgfpathlineto{\pgfqpoint{4.879766in}{1.387888in}}%
\pgfpathlineto{\pgfqpoint{4.866507in}{1.439184in}}%
\pgfpathlineto{\pgfqpoint{4.853711in}{1.490514in}}%
\pgfpathlineto{\pgfqpoint{4.841439in}{1.541883in}}%
\pgfpathlineto{\pgfqpoint{4.829768in}{1.593293in}}%
\pgfpathlineto{\pgfqpoint{4.818787in}{1.644748in}}%
\pgfpathlineto{\pgfqpoint{4.808595in}{1.696251in}}%
\pgfpathlineto{\pgfqpoint{4.799302in}{1.747805in}}%
\pgfpathlineto{\pgfqpoint{4.791033in}{1.799411in}}%
\pgfpathlineto{\pgfqpoint{4.783928in}{1.851068in}}%
\pgfpathlineto{\pgfqpoint{4.778143in}{1.902773in}}%
\pgfpathlineto{\pgfqpoint{4.773839in}{1.954522in}}%
\pgfpathlineto{\pgfqpoint{4.771185in}{2.006302in}}%
\pgfpathlineto{\pgfqpoint{4.770348in}{2.058101in}}%
\pgfpathlineto{\pgfqpoint{4.771484in}{2.109897in}}%
\pgfpathlineto{\pgfqpoint{4.774725in}{2.161667in}}%
\pgfpathlineto{\pgfqpoint{4.780165in}{2.213381in}}%
\pgfpathlineto{\pgfqpoint{4.787849in}{2.265010in}}%
\pgfpathlineto{\pgfqpoint{4.797761in}{2.316525in}}%
\pgfpathlineto{\pgfqpoint{4.809815in}{2.367904in}}%
\pgfpathlineto{\pgfqpoint{4.823878in}{2.419129in}}%
\pgfpathlineto{\pgfqpoint{4.839781in}{2.470196in}}%
\pgfpathlineto{\pgfqpoint{4.857306in}{2.521104in}}%
\pgfpathlineto{\pgfqpoint{4.876239in}{2.571862in}}%
\pgfpathlineto{\pgfqpoint{4.896359in}{2.622483in}}%
\pgfpathlineto{\pgfqpoint{4.917458in}{2.672985in}}%
\pgfpathlineto{\pgfqpoint{4.921738in}{2.683014in}}%
\pgfusepath{stroke}%
\end{pgfscope}%
\begin{pgfscope}%
\pgfpathrectangle{\pgfqpoint{0.647939in}{0.492442in}}{\pgfqpoint{4.273799in}{2.331163in}}%
\pgfusepath{clip}%
\pgfsetbuttcap%
\pgfsetroundjoin%
\pgfsetlinewidth{0.301125pt}%
\definecolor{currentstroke}{rgb}{0.500000,0.500000,0.500000}%
\pgfsetstrokecolor{currentstroke}%
\pgfsetstrokeopacity{0.300000}%
\pgfsetdash{}{0pt}%
\pgfpathmoveto{\pgfqpoint{4.921738in}{1.499081in}}%
\pgfpathlineto{\pgfqpoint{4.921738in}{1.499081in}}%
\pgfpathlineto{\pgfqpoint{4.911202in}{1.550564in}}%
\pgfpathlineto{\pgfqpoint{4.901285in}{1.602084in}}%
\pgfpathlineto{\pgfqpoint{4.892057in}{1.653641in}}%
\pgfpathlineto{\pgfqpoint{4.883594in}{1.705238in}}%
\pgfpathlineto{\pgfqpoint{4.875983in}{1.756874in}}%
\pgfusepath{stroke}%
\end{pgfscope}%
\begin{pgfscope}%
\pgfpathrectangle{\pgfqpoint{0.647939in}{0.492442in}}{\pgfqpoint{4.273799in}{2.331163in}}%
\pgfusepath{clip}%
\pgfsetbuttcap%
\pgfsetroundjoin%
\pgfsetlinewidth{0.301125pt}%
\definecolor{currentstroke}{rgb}{0.500000,0.500000,0.500000}%
\pgfsetstrokecolor{currentstroke}%
\pgfsetstrokeopacity{0.300000}%
\pgfsetdash{}{0pt}%
\pgfpathmoveto{\pgfqpoint{4.921738in}{1.816967in}}%
\pgfpathlineto{\pgfqpoint{4.921738in}{1.816967in}}%
\pgfpathlineto{\pgfqpoint{4.916941in}{1.868703in}}%
\pgfpathlineto{\pgfqpoint{4.913193in}{1.920465in}}%
\pgfpathlineto{\pgfqpoint{4.910574in}{1.972248in}}%
\pgfpathlineto{\pgfqpoint{4.909164in}{2.024045in}}%
\pgfpathlineto{\pgfqpoint{4.909039in}{2.075847in}}%
\pgfpathlineto{\pgfqpoint{4.910268in}{2.127645in}}%
\pgfpathlineto{\pgfqpoint{4.912907in}{2.179427in}}%
\pgfpathlineto{\pgfqpoint{4.917002in}{2.231180in}}%
\pgfpathlineto{\pgfqpoint{4.921738in}{2.281966in}}%
\pgfusepath{stroke}%
\end{pgfscope}%
\begin{pgfscope}%
\pgfpathrectangle{\pgfqpoint{0.647939in}{0.492442in}}{\pgfqpoint{4.273799in}{2.331163in}}%
\pgfusepath{clip}%
\pgfsetbuttcap%
\pgfsetroundjoin%
\pgfsetlinewidth{0.301125pt}%
\definecolor{currentstroke}{rgb}{0.500000,0.500000,0.500000}%
\pgfsetstrokecolor{currentstroke}%
\pgfsetstrokeopacity{0.300000}%
\pgfsetdash{}{0pt}%
\pgfpathmoveto{\pgfqpoint{4.309006in}{2.823605in}}%
\pgfpathlineto{\pgfqpoint{4.345556in}{2.788338in}}%
\pgfpathlineto{\pgfqpoint{4.395399in}{2.744324in}}%
\pgfpathlineto{\pgfqpoint{4.441640in}{2.711445in}}%
\pgfpathlineto{\pgfqpoint{4.482691in}{2.689768in}}%
\pgfpathlineto{\pgfqpoint{4.522286in}{2.676356in}}%
\pgfpathlineto{\pgfqpoint{4.568606in}{2.670692in}}%
\pgfpathlineto{\pgfqpoint{4.615430in}{2.676173in}}%
\pgfpathlineto{\pgfqpoint{4.615430in}{2.676173in}}%
\pgfpathlineto{\pgfqpoint{4.667806in}{2.695433in}}%
\pgfpathlineto{\pgfqpoint{4.667806in}{2.695433in}}%
\pgfpathlineto{\pgfqpoint{4.729271in}{2.734367in}}%
\pgfpathlineto{\pgfqpoint{4.780034in}{2.777948in}}%
\pgfpathlineto{\pgfqpoint{4.824607in}{2.823605in}}%
\pgfpathlineto{\pgfqpoint{4.824607in}{2.823605in}}%
\pgfusepath{stroke}%
\end{pgfscope}%
\begin{pgfscope}%
\pgfpathrectangle{\pgfqpoint{0.647939in}{0.492442in}}{\pgfqpoint{4.273799in}{2.331163in}}%
\pgfusepath{clip}%
\pgfsetbuttcap%
\pgfsetroundjoin%
\pgfsetlinewidth{0.301125pt}%
\definecolor{currentstroke}{rgb}{0.500000,0.500000,0.500000}%
\pgfsetstrokecolor{currentstroke}%
\pgfsetstrokeopacity{0.300000}%
\pgfsetdash{}{0pt}%
\pgfpathmoveto{\pgfqpoint{4.144684in}{2.823605in}}%
\pgfpathlineto{\pgfqpoint{4.144684in}{2.823605in}}%
\pgfpathlineto{\pgfqpoint{4.179528in}{2.775416in}}%
\pgfpathlineto{\pgfqpoint{4.215374in}{2.727448in}}%
\pgfpathlineto{\pgfqpoint{4.252701in}{2.679822in}}%
\pgfpathlineto{\pgfqpoint{4.292293in}{2.632754in}}%
\pgfpathlineto{\pgfqpoint{4.335565in}{2.586686in}}%
\pgfpathlineto{\pgfqpoint{4.385469in}{2.542764in}}%
\pgfpathlineto{\pgfqpoint{4.449365in}{2.505240in}}%
\pgfpathlineto{\pgfqpoint{4.449365in}{2.505240in}}%
\pgfpathlineto{\pgfqpoint{4.489729in}{2.494261in}}%
\pgfpathlineto{\pgfqpoint{4.489729in}{2.494261in}}%
\pgfpathlineto{\pgfqpoint{4.527925in}{2.493771in}}%
\pgfpathlineto{\pgfqpoint{4.563246in}{2.501695in}}%
\pgfpathlineto{\pgfqpoint{4.596634in}{2.516136in}}%
\pgfpathlineto{\pgfqpoint{4.633911in}{2.539464in}}%
\pgfpathlineto{\pgfqpoint{4.677123in}{2.574735in}}%
\pgfpathlineto{\pgfqpoint{4.722765in}{2.619985in}}%
\pgfpathlineto{\pgfqpoint{4.763922in}{2.666574in}}%
\pgfusepath{stroke}%
\end{pgfscope}%
\begin{pgfscope}%
\pgfpathrectangle{\pgfqpoint{0.647939in}{0.492442in}}{\pgfqpoint{4.273799in}{2.331163in}}%
\pgfusepath{clip}%
\pgfsetbuttcap%
\pgfsetroundjoin%
\pgfsetlinewidth{0.301125pt}%
\definecolor{currentstroke}{rgb}{0.500000,0.500000,0.500000}%
\pgfsetstrokecolor{currentstroke}%
\pgfsetstrokeopacity{0.300000}%
\pgfsetdash{}{0pt}%
\pgfpathmoveto{\pgfqpoint{4.047552in}{2.823605in}}%
\pgfpathlineto{\pgfqpoint{4.047552in}{2.823605in}}%
\pgfpathlineto{\pgfqpoint{4.079390in}{2.774800in}}%
\pgfpathlineto{\pgfqpoint{4.111391in}{2.726027in}}%
\pgfpathlineto{\pgfqpoint{4.143692in}{2.677312in}}%
\pgfpathlineto{\pgfqpoint{4.176508in}{2.628701in}}%
\pgfpathlineto{\pgfqpoint{4.210157in}{2.580263in}}%
\pgfpathlineto{\pgfqpoint{4.245109in}{2.532106in}}%
\pgfpathlineto{\pgfqpoint{4.282193in}{2.484432in}}%
\pgfpathlineto{\pgfqpoint{4.323065in}{2.437710in}}%
\pgfpathlineto{\pgfqpoint{4.371792in}{2.393447in}}%
\pgfpathlineto{\pgfqpoint{4.371792in}{2.393447in}}%
\pgfpathlineto{\pgfqpoint{4.413360in}{2.368718in}}%
\pgfpathlineto{\pgfqpoint{4.413360in}{2.368718in}}%
\pgfpathlineto{\pgfqpoint{4.448725in}{2.359015in}}%
\pgfpathlineto{\pgfqpoint{4.448725in}{2.359015in}}%
\pgfpathlineto{\pgfqpoint{4.481514in}{2.359875in}}%
\pgfpathlineto{\pgfqpoint{4.510748in}{2.368148in}}%
\pgfpathlineto{\pgfqpoint{4.540488in}{2.383068in}}%
\pgfpathlineto{\pgfqpoint{4.575426in}{2.407649in}}%
\pgfpathlineto{\pgfqpoint{4.617528in}{2.445400in}}%
\pgfusepath{stroke}%
\end{pgfscope}%
\begin{pgfscope}%
\pgfpathrectangle{\pgfqpoint{0.647939in}{0.492442in}}{\pgfqpoint{4.273799in}{2.331163in}}%
\pgfusepath{clip}%
\pgfsetbuttcap%
\pgfsetroundjoin%
\pgfsetlinewidth{0.301125pt}%
\definecolor{currentstroke}{rgb}{0.500000,0.500000,0.500000}%
\pgfsetstrokecolor{currentstroke}%
\pgfsetstrokeopacity{0.300000}%
\pgfsetdash{}{0pt}%
\pgfpathmoveto{\pgfqpoint{3.950420in}{2.823605in}}%
\pgfpathlineto{\pgfqpoint{3.950420in}{2.823605in}}%
\pgfpathlineto{\pgfqpoint{3.980461in}{2.774462in}}%
\pgfpathlineto{\pgfqpoint{4.010225in}{2.725268in}}%
\pgfpathlineto{\pgfqpoint{4.039755in}{2.676033in}}%
\pgfpathlineto{\pgfqpoint{4.069090in}{2.626763in}}%
\pgfpathlineto{\pgfqpoint{4.098269in}{2.577466in}}%
\pgfpathlineto{\pgfqpoint{4.127369in}{2.528156in}}%
\pgfpathlineto{\pgfqpoint{4.156485in}{2.478849in}}%
\pgfpathlineto{\pgfqpoint{4.185735in}{2.429566in}}%
\pgfpathlineto{\pgfqpoint{4.215349in}{2.380350in}}%
\pgfpathlineto{\pgfqpoint{4.245725in}{2.331281in}}%
\pgfpathlineto{\pgfqpoint{4.277618in}{2.282503in}}%
\pgfpathlineto{\pgfqpoint{4.313089in}{2.234504in}}%
\pgfpathlineto{\pgfqpoint{4.360959in}{2.190592in}}%
\pgfpathlineto{\pgfqpoint{4.360959in}{2.190592in}}%
\pgfpathlineto{\pgfqpoint{4.384621in}{2.182675in}}%
\pgfpathlineto{\pgfqpoint{4.410466in}{2.185885in}}%
\pgfpathlineto{\pgfqpoint{4.430903in}{2.195218in}}%
\pgfpathlineto{\pgfqpoint{4.454015in}{2.211404in}}%
\pgfpathlineto{\pgfqpoint{4.486524in}{2.240674in}}%
\pgfpathlineto{\pgfqpoint{4.529671in}{2.286598in}}%
\pgfusepath{stroke}%
\end{pgfscope}%
\begin{pgfscope}%
\pgfpathrectangle{\pgfqpoint{0.647939in}{0.492442in}}{\pgfqpoint{4.273799in}{2.331163in}}%
\pgfusepath{clip}%
\pgfsetbuttcap%
\pgfsetroundjoin%
\pgfsetlinewidth{0.301125pt}%
\definecolor{currentstroke}{rgb}{0.500000,0.500000,0.500000}%
\pgfsetstrokecolor{currentstroke}%
\pgfsetstrokeopacity{0.300000}%
\pgfsetdash{}{0pt}%
\pgfpathmoveto{\pgfqpoint{3.853289in}{2.823605in}}%
\pgfpathlineto{\pgfqpoint{3.853289in}{2.823605in}}%
\pgfpathlineto{\pgfqpoint{3.882357in}{2.774288in}}%
\pgfpathlineto{\pgfqpoint{3.910895in}{2.724879in}}%
\pgfpathlineto{\pgfqpoint{3.938893in}{2.675378in}}%
\pgfpathlineto{\pgfqpoint{3.966326in}{2.625783in}}%
\pgfpathlineto{\pgfqpoint{3.993176in}{2.576093in}}%
\pgfpathlineto{\pgfqpoint{4.019402in}{2.526305in}}%
\pgfpathlineto{\pgfqpoint{4.044948in}{2.476412in}}%
\pgfpathlineto{\pgfqpoint{4.069741in}{2.426406in}}%
\pgfpathlineto{\pgfqpoint{4.093653in}{2.376273in}}%
\pgfpathlineto{\pgfqpoint{4.116514in}{2.325995in}}%
\pgfpathlineto{\pgfqpoint{4.138043in}{2.275544in}}%
\pgfpathlineto{\pgfqpoint{4.157799in}{2.224881in}}%
\pgfpathlineto{\pgfqpoint{4.175034in}{2.173949in}}%
\pgfpathlineto{\pgfqpoint{4.188411in}{2.122684in}}%
\pgfpathlineto{\pgfqpoint{4.195539in}{2.071072in}}%
\pgfpathlineto{\pgfqpoint{4.192498in}{2.019423in}}%
\pgfpathlineto{\pgfqpoint{4.175237in}{1.968751in}}%
\pgfusepath{stroke}%
\end{pgfscope}%
\begin{pgfscope}%
\pgfpathrectangle{\pgfqpoint{0.647939in}{0.492442in}}{\pgfqpoint{4.273799in}{2.331163in}}%
\pgfusepath{clip}%
\pgfsetbuttcap%
\pgfsetroundjoin%
\pgfsetlinewidth{0.301125pt}%
\definecolor{currentstroke}{rgb}{0.500000,0.500000,0.500000}%
\pgfsetstrokecolor{currentstroke}%
\pgfsetstrokeopacity{0.300000}%
\pgfsetdash{}{0pt}%
\pgfpathmoveto{\pgfqpoint{3.756157in}{2.823605in}}%
\pgfpathlineto{\pgfqpoint{3.756157in}{2.823605in}}%
\pgfpathlineto{\pgfqpoint{3.784880in}{2.774228in}}%
\pgfpathlineto{\pgfqpoint{3.812894in}{2.724730in}}%
\pgfpathlineto{\pgfqpoint{3.840163in}{2.675108in}}%
\pgfpathlineto{\pgfqpoint{3.866654in}{2.625361in}}%
\pgfpathlineto{\pgfqpoint{3.892319in}{2.575486in}}%
\pgfpathlineto{\pgfqpoint{3.917084in}{2.525475in}}%
\pgfpathlineto{\pgfqpoint{3.940863in}{2.475323in}}%
\pgfpathlineto{\pgfqpoint{3.963530in}{2.425018in}}%
\pgfpathlineto{\pgfqpoint{3.984921in}{2.374547in}}%
\pgfpathlineto{\pgfqpoint{4.004810in}{2.323895in}}%
\pgfpathlineto{\pgfqpoint{4.022886in}{2.273042in}}%
\pgfpathlineto{\pgfqpoint{4.038717in}{2.221970in}}%
\pgfpathlineto{\pgfqpoint{4.051690in}{2.170661in}}%
\pgfpathlineto{\pgfqpoint{4.060942in}{2.119118in}}%
\pgfpathlineto{\pgfqpoint{4.065286in}{2.067393in}}%
\pgfpathlineto{\pgfqpoint{4.063174in}{2.015647in}}%
\pgfpathlineto{\pgfqpoint{4.052845in}{1.964226in}}%
\pgfpathlineto{\pgfqpoint{4.032727in}{1.913714in}}%
\pgfpathlineto{\pgfqpoint{4.002147in}{1.864809in}}%
\pgfpathlineto{\pgfqpoint{3.961417in}{1.818163in}}%
\pgfusepath{stroke}%
\end{pgfscope}%
\begin{pgfscope}%
\pgfpathrectangle{\pgfqpoint{0.647939in}{0.492442in}}{\pgfqpoint{4.273799in}{2.331163in}}%
\pgfusepath{clip}%
\pgfsetbuttcap%
\pgfsetroundjoin%
\pgfsetlinewidth{0.301125pt}%
\definecolor{currentstroke}{rgb}{0.500000,0.500000,0.500000}%
\pgfsetstrokecolor{currentstroke}%
\pgfsetstrokeopacity{0.300000}%
\pgfsetdash{}{0pt}%
\pgfpathmoveto{\pgfqpoint{3.659025in}{2.823605in}}%
\pgfpathlineto{\pgfqpoint{3.659025in}{2.823605in}}%
\pgfpathlineto{\pgfqpoint{3.687919in}{2.774258in}}%
\pgfpathlineto{\pgfqpoint{3.715954in}{2.724764in}}%
\pgfpathlineto{\pgfqpoint{3.743100in}{2.675122in}}%
\pgfpathlineto{\pgfqpoint{3.769317in}{2.625332in}}%
\pgfpathlineto{\pgfqpoint{3.794538in}{2.575389in}}%
\pgfpathlineto{\pgfqpoint{3.818691in}{2.525290in}}%
\pgfpathlineto{\pgfqpoint{3.841672in}{2.475027in}}%
\pgfpathlineto{\pgfqpoint{3.863358in}{2.424594in}}%
\pgfpathlineto{\pgfqpoint{3.883579in}{2.373980in}}%
\pgfpathlineto{\pgfqpoint{3.902119in}{2.323176in}}%
\pgfpathlineto{\pgfqpoint{3.918695in}{2.272172in}}%
\pgfpathlineto{\pgfqpoint{3.932924in}{2.220959in}}%
\pgfpathlineto{\pgfqpoint{3.944316in}{2.169538in}}%
\pgfpathlineto{\pgfqpoint{3.952221in}{2.117925in}}%
\pgfpathlineto{\pgfqpoint{3.955789in}{2.066176in}}%
\pgfpathlineto{\pgfqpoint{3.953958in}{2.014412in}}%
\pgfpathlineto{\pgfqpoint{3.945467in}{1.962864in}}%
\pgfpathlineto{\pgfqpoint{3.928988in}{1.911916in}}%
\pgfpathlineto{\pgfqpoint{3.903296in}{1.862139in}}%
\pgfpathlineto{\pgfqpoint{3.867512in}{1.814274in}}%
\pgfpathlineto{\pgfqpoint{3.821218in}{1.769200in}}%
\pgfpathlineto{\pgfqpoint{3.764306in}{1.727959in}}%
\pgfpathlineto{\pgfqpoint{3.696773in}{1.691875in}}%
\pgfusepath{stroke}%
\end{pgfscope}%
\begin{pgfscope}%
\pgfpathrectangle{\pgfqpoint{0.647939in}{0.492442in}}{\pgfqpoint{4.273799in}{2.331163in}}%
\pgfusepath{clip}%
\pgfsetbuttcap%
\pgfsetroundjoin%
\pgfsetlinewidth{0.301125pt}%
\definecolor{currentstroke}{rgb}{0.500000,0.500000,0.500000}%
\pgfsetstrokecolor{currentstroke}%
\pgfsetstrokeopacity{0.300000}%
\pgfsetdash{}{0pt}%
\pgfpathmoveto{\pgfqpoint{3.561893in}{2.823605in}}%
\pgfpathlineto{\pgfqpoint{3.561893in}{2.823605in}}%
\pgfpathlineto{\pgfqpoint{3.591411in}{2.774368in}}%
\pgfpathlineto{\pgfqpoint{3.619946in}{2.724959in}}%
\pgfpathlineto{\pgfqpoint{3.647468in}{2.675379in}}%
\pgfpathlineto{\pgfqpoint{3.673928in}{2.625628in}}%
\pgfpathlineto{\pgfqpoint{3.699265in}{2.575703in}}%
\pgfpathlineto{\pgfqpoint{3.723406in}{2.525602in}}%
\pgfpathlineto{\pgfqpoint{3.746247in}{2.475321in}}%
\pgfpathlineto{\pgfqpoint{3.767665in}{2.424854in}}%
\pgfpathlineto{\pgfqpoint{3.787498in}{2.374194in}}%
\pgfpathlineto{\pgfqpoint{3.805539in}{2.323337in}}%
\pgfpathlineto{\pgfqpoint{3.821528in}{2.272277in}}%
\pgfpathlineto{\pgfqpoint{3.835126in}{2.221013in}}%
\pgfpathlineto{\pgfqpoint{3.845899in}{2.169552in}}%
\pgfpathlineto{\pgfqpoint{3.853294in}{2.117916in}}%
\pgfpathlineto{\pgfqpoint{3.856604in}{2.066160in}}%
\pgfpathlineto{\pgfqpoint{3.854938in}{2.014389in}}%
\pgfpathlineto{\pgfqpoint{3.847213in}{1.962796in}}%
\pgfpathlineto{\pgfqpoint{3.832169in}{1.911704in}}%
\pgfpathlineto{\pgfqpoint{3.808411in}{1.861629in}}%
\pgfpathlineto{\pgfqpoint{3.774562in}{1.813342in}}%
\pgfpathlineto{\pgfqpoint{3.729356in}{1.767957in}}%
\pgfusepath{stroke}%
\end{pgfscope}%
\begin{pgfscope}%
\pgfpathrectangle{\pgfqpoint{0.647939in}{0.492442in}}{\pgfqpoint{4.273799in}{2.331163in}}%
\pgfusepath{clip}%
\pgfsetbuttcap%
\pgfsetroundjoin%
\pgfsetlinewidth{0.301125pt}%
\definecolor{currentstroke}{rgb}{0.500000,0.500000,0.500000}%
\pgfsetstrokecolor{currentstroke}%
\pgfsetstrokeopacity{0.300000}%
\pgfsetdash{}{0pt}%
\pgfpathmoveto{\pgfqpoint{3.464761in}{2.823605in}}%
\pgfpathlineto{\pgfqpoint{3.464761in}{2.823605in}}%
\pgfpathlineto{\pgfqpoint{3.495346in}{2.774563in}}%
\pgfpathlineto{\pgfqpoint{3.524828in}{2.725319in}}%
\pgfpathlineto{\pgfqpoint{3.553175in}{2.675879in}}%
\pgfpathlineto{\pgfqpoint{3.580338in}{2.626240in}}%
\pgfpathlineto{\pgfqpoint{3.606262in}{2.576405in}}%
\pgfpathlineto{\pgfqpoint{3.630871in}{2.526373in}}%
\pgfpathlineto{\pgfqpoint{3.654069in}{2.476140in}}%
\pgfpathlineto{\pgfqpoint{3.675736in}{2.425705in}}%
\pgfpathlineto{\pgfqpoint{3.695716in}{2.375063in}}%
\pgfpathlineto{\pgfqpoint{3.713811in}{2.324212in}}%
\pgfpathlineto{\pgfqpoint{3.729777in}{2.273150in}}%
\pgfpathlineto{\pgfqpoint{3.743301in}{2.221880in}}%
\pgfpathlineto{\pgfqpoint{3.753977in}{2.170413in}}%
\pgfpathlineto{\pgfqpoint{3.761294in}{2.118774in}}%
\pgfpathlineto{\pgfqpoint{3.764604in}{2.067017in}}%
\pgfpathlineto{\pgfqpoint{3.763074in}{2.015243in}}%
\pgfpathlineto{\pgfqpoint{3.755665in}{1.963634in}}%
\pgfpathlineto{\pgfqpoint{3.741081in}{1.912500in}}%
\pgfpathlineto{\pgfqpoint{3.717756in}{1.862367in}}%
\pgfpathlineto{\pgfqpoint{3.683877in}{1.814102in}}%
\pgfpathlineto{\pgfqpoint{3.637437in}{1.769143in}}%
\pgfpathlineto{\pgfqpoint{3.576608in}{1.729822in}}%
\pgfpathlineto{\pgfqpoint{3.504573in}{1.700996in}}%
\pgfpathlineto{\pgfqpoint{3.431910in}{1.685812in}}%
\pgfpathlineto{\pgfqpoint{3.361933in}{1.682050in}}%
\pgfpathlineto{\pgfqpoint{3.294783in}{1.687565in}}%
\pgfpathlineto{\pgfqpoint{3.229675in}{1.701222in}}%
\pgfpathlineto{\pgfqpoint{3.165108in}{1.723026in}}%
\pgfpathlineto{\pgfqpoint{3.100995in}{1.753448in}}%
\pgfpathlineto{\pgfqpoint{3.038958in}{1.792406in}}%
\pgfpathlineto{\pgfqpoint{2.985659in}{1.835117in}}%
\pgfpathlineto{\pgfqpoint{2.940488in}{1.880584in}}%
\pgfpathlineto{\pgfqpoint{2.903403in}{1.928180in}}%
\pgfpathlineto{\pgfqpoint{2.875140in}{1.977525in}}%
\pgfpathlineto{\pgfqpoint{2.858069in}{2.028321in}}%
\pgfpathlineto{\pgfqpoint{2.859813in}{2.079540in}}%
\pgfpathlineto{\pgfqpoint{2.859813in}{2.079540in}}%
\pgfpathlineto{\pgfqpoint{2.873134in}{2.100577in}}%
\pgfpathlineto{\pgfqpoint{2.873134in}{2.100577in}}%
\pgfpathlineto{\pgfqpoint{2.893192in}{2.110585in}}%
\pgfpathlineto{\pgfqpoint{2.893192in}{2.110585in}}%
\pgfpathlineto{\pgfqpoint{2.915022in}{2.110648in}}%
\pgfusepath{stroke}%
\end{pgfscope}%
\begin{pgfscope}%
\pgfpathrectangle{\pgfqpoint{0.647939in}{0.492442in}}{\pgfqpoint{4.273799in}{2.331163in}}%
\pgfusepath{clip}%
\pgfsetbuttcap%
\pgfsetroundjoin%
\pgfsetlinewidth{0.301125pt}%
\definecolor{currentstroke}{rgb}{0.500000,0.500000,0.500000}%
\pgfsetstrokecolor{currentstroke}%
\pgfsetstrokeopacity{0.300000}%
\pgfsetdash{}{0pt}%
\pgfpathmoveto{\pgfqpoint{3.270498in}{2.823605in}}%
\pgfpathlineto{\pgfqpoint{3.270498in}{2.823605in}}%
\pgfpathlineto{\pgfqpoint{3.304632in}{2.775266in}}%
\pgfpathlineto{\pgfqpoint{3.337381in}{2.726642in}}%
\pgfpathlineto{\pgfqpoint{3.368716in}{2.677741in}}%
\pgfpathlineto{\pgfqpoint{3.398608in}{2.628573in}}%
\pgfpathlineto{\pgfqpoint{3.427010in}{2.579142in}}%
\pgfpathlineto{\pgfqpoint{3.453851in}{2.529453in}}%
\pgfpathlineto{\pgfqpoint{3.479048in}{2.479509in}}%
\pgfpathlineto{\pgfqpoint{3.502491in}{2.429311in}}%
\pgfpathlineto{\pgfqpoint{3.524035in}{2.378862in}}%
\pgfpathlineto{\pgfqpoint{3.543498in}{2.328161in}}%
\pgfpathlineto{\pgfqpoint{3.560644in}{2.277214in}}%
\pgfpathlineto{\pgfqpoint{3.575177in}{2.226027in}}%
\pgfpathlineto{\pgfqpoint{3.586706in}{2.174616in}}%
\pgfpathlineto{\pgfqpoint{3.594733in}{2.123010in}}%
\pgfpathlineto{\pgfqpoint{3.598600in}{2.071265in}}%
\pgfpathlineto{\pgfqpoint{3.597432in}{2.019489in}}%
\pgfpathlineto{\pgfqpoint{3.590052in}{1.967882in}}%
\pgfpathlineto{\pgfqpoint{3.574847in}{1.916816in}}%
\pgfpathlineto{\pgfqpoint{3.549542in}{1.867010in}}%
\pgfpathlineto{\pgfqpoint{3.510966in}{1.819929in}}%
\pgfpathlineto{\pgfqpoint{3.455034in}{1.778701in}}%
\pgfpathlineto{\pgfqpoint{3.455034in}{1.778701in}}%
\pgfpathlineto{\pgfqpoint{3.402405in}{1.755798in}}%
\pgfpathlineto{\pgfqpoint{3.339316in}{1.742273in}}%
\pgfpathlineto{\pgfqpoint{3.278970in}{1.740264in}}%
\pgfpathlineto{\pgfqpoint{3.221343in}{1.747061in}}%
\pgfpathlineto{\pgfqpoint{3.164818in}{1.761598in}}%
\pgfusepath{stroke}%
\end{pgfscope}%
\begin{pgfscope}%
\pgfpathrectangle{\pgfqpoint{0.647939in}{0.492442in}}{\pgfqpoint{4.273799in}{2.331163in}}%
\pgfusepath{clip}%
\pgfsetbuttcap%
\pgfsetroundjoin%
\pgfsetlinewidth{0.301125pt}%
\definecolor{currentstroke}{rgb}{0.500000,0.500000,0.500000}%
\pgfsetstrokecolor{currentstroke}%
\pgfsetstrokeopacity{0.300000}%
\pgfsetdash{}{0pt}%
\pgfpathmoveto{\pgfqpoint{3.173366in}{2.823605in}}%
\pgfpathlineto{\pgfqpoint{3.173366in}{2.823605in}}%
\pgfpathlineto{\pgfqpoint{3.210060in}{2.775828in}}%
\pgfpathlineto{\pgfqpoint{3.245191in}{2.727702in}}%
\pgfpathlineto{\pgfqpoint{3.278743in}{2.679242in}}%
\pgfpathlineto{\pgfqpoint{3.310691in}{2.630460in}}%
\pgfpathlineto{\pgfqpoint{3.340998in}{2.581368in}}%
\pgfpathlineto{\pgfqpoint{3.369600in}{2.531973in}}%
\pgfpathlineto{\pgfqpoint{3.396419in}{2.482281in}}%
\pgfusepath{stroke}%
\end{pgfscope}%
\begin{pgfscope}%
\pgfpathrectangle{\pgfqpoint{0.647939in}{0.492442in}}{\pgfqpoint{4.273799in}{2.331163in}}%
\pgfusepath{clip}%
\pgfsetbuttcap%
\pgfsetroundjoin%
\pgfsetlinewidth{0.301125pt}%
\definecolor{currentstroke}{rgb}{0.500000,0.500000,0.500000}%
\pgfsetstrokecolor{currentstroke}%
\pgfsetstrokeopacity{0.300000}%
\pgfsetdash{}{0pt}%
\pgfpathmoveto{\pgfqpoint{2.979102in}{2.823605in}}%
\pgfpathlineto{\pgfqpoint{2.979102in}{2.823605in}}%
\pgfpathlineto{\pgfqpoint{3.022772in}{2.777607in}}%
\pgfpathlineto{\pgfqpoint{3.064423in}{2.731056in}}%
\pgfpathlineto{\pgfqpoint{3.104058in}{2.683983in}}%
\pgfpathlineto{\pgfqpoint{3.141673in}{2.636421in}}%
\pgfpathlineto{\pgfqpoint{3.177250in}{2.588396in}}%
\pgfpathlineto{\pgfqpoint{3.210755in}{2.539929in}}%
\pgfpathlineto{\pgfqpoint{3.242130in}{2.491039in}}%
\pgfpathlineto{\pgfqpoint{3.271292in}{2.441744in}}%
\pgfpathlineto{\pgfqpoint{3.298127in}{2.392057in}}%
\pgfpathlineto{\pgfqpoint{3.322468in}{2.341991in}}%
\pgfpathlineto{\pgfqpoint{3.344096in}{2.291556in}}%
\pgfpathlineto{\pgfqpoint{3.362713in}{2.240767in}}%
\pgfpathlineto{\pgfqpoint{3.377911in}{2.189642in}}%
\pgfpathlineto{\pgfqpoint{3.389129in}{2.138216in}}%
\pgfpathlineto{\pgfqpoint{3.395573in}{2.086555in}}%
\pgfpathlineto{\pgfqpoint{3.396079in}{2.034790in}}%
\pgfpathlineto{\pgfqpoint{3.388865in}{1.983203in}}%
\pgfpathlineto{\pgfqpoint{3.370997in}{1.932456in}}%
\pgfpathlineto{\pgfqpoint{3.337264in}{1.884380in}}%
\pgfpathlineto{\pgfqpoint{3.337264in}{1.884380in}}%
\pgfpathlineto{\pgfqpoint{3.299382in}{1.854746in}}%
\pgfpathlineto{\pgfqpoint{3.299382in}{1.854746in}}%
\pgfpathlineto{\pgfqpoint{3.257115in}{1.836790in}}%
\pgfpathlineto{\pgfqpoint{3.205498in}{1.828752in}}%
\pgfpathlineto{\pgfqpoint{3.158235in}{1.831304in}}%
\pgfusepath{stroke}%
\end{pgfscope}%
\begin{pgfscope}%
\pgfpathrectangle{\pgfqpoint{0.647939in}{0.492442in}}{\pgfqpoint{4.273799in}{2.331163in}}%
\pgfusepath{clip}%
\pgfsetbuttcap%
\pgfsetroundjoin%
\pgfsetlinewidth{0.301125pt}%
\definecolor{currentstroke}{rgb}{0.500000,0.500000,0.500000}%
\pgfsetstrokecolor{currentstroke}%
\pgfsetstrokeopacity{0.300000}%
\pgfsetdash{}{0pt}%
\pgfpathmoveto{\pgfqpoint{2.784839in}{2.823605in}}%
\pgfpathlineto{\pgfqpoint{2.784839in}{2.823605in}}%
\pgfpathlineto{\pgfqpoint{2.838160in}{2.780753in}}%
\pgfpathlineto{\pgfqpoint{2.888897in}{2.736973in}}%
\pgfpathlineto{\pgfqpoint{2.937033in}{2.692326in}}%
\pgfpathlineto{\pgfqpoint{2.982572in}{2.646875in}}%
\pgfpathlineto{\pgfqpoint{3.025517in}{2.600679in}}%
\pgfpathlineto{\pgfqpoint{3.065869in}{2.553792in}}%
\pgfpathlineto{\pgfqpoint{3.103610in}{2.506262in}}%
\pgfpathlineto{\pgfqpoint{3.138701in}{2.458134in}}%
\pgfpathlineto{\pgfqpoint{3.171065in}{2.409441in}}%
\pgfpathlineto{\pgfqpoint{3.200577in}{2.360212in}}%
\pgfpathlineto{\pgfqpoint{3.227050in}{2.310472in}}%
\pgfpathlineto{\pgfqpoint{3.250215in}{2.260245in}}%
\pgfpathlineto{\pgfqpoint{3.269681in}{2.209556in}}%
\pgfpathlineto{\pgfqpoint{3.284878in}{2.158438in}}%
\pgfpathlineto{\pgfqpoint{3.294960in}{2.106955in}}%
\pgfpathlineto{\pgfqpoint{3.298610in}{2.055235in}}%
\pgfpathlineto{\pgfqpoint{3.293619in}{2.003586in}}%
\pgfpathlineto{\pgfqpoint{3.275886in}{1.952879in}}%
\pgfpathlineto{\pgfqpoint{3.275886in}{1.952879in}}%
\pgfpathlineto{\pgfqpoint{3.247401in}{1.915141in}}%
\pgfpathlineto{\pgfqpoint{3.247401in}{1.915141in}}%
\pgfpathlineto{\pgfqpoint{3.213493in}{1.891776in}}%
\pgfpathlineto{\pgfqpoint{3.213493in}{1.891776in}}%
\pgfpathlineto{\pgfqpoint{3.175764in}{1.879563in}}%
\pgfpathlineto{\pgfqpoint{3.132159in}{1.877223in}}%
\pgfpathlineto{\pgfqpoint{3.092696in}{1.883442in}}%
\pgfpathlineto{\pgfqpoint{3.053403in}{1.896802in}}%
\pgfpathlineto{\pgfqpoint{3.013147in}{1.918162in}}%
\pgfpathlineto{\pgfqpoint{2.973244in}{1.948633in}}%
\pgfusepath{stroke}%
\end{pgfscope}%
\begin{pgfscope}%
\pgfpathrectangle{\pgfqpoint{0.647939in}{0.492442in}}{\pgfqpoint{4.273799in}{2.331163in}}%
\pgfusepath{clip}%
\pgfsetbuttcap%
\pgfsetroundjoin%
\pgfsetlinewidth{0.301125pt}%
\definecolor{currentstroke}{rgb}{0.500000,0.500000,0.500000}%
\pgfsetstrokecolor{currentstroke}%
\pgfsetstrokeopacity{0.300000}%
\pgfsetdash{}{0pt}%
\pgfpathmoveto{\pgfqpoint{2.590575in}{2.823605in}}%
\pgfpathlineto{\pgfqpoint{2.590575in}{2.823605in}}%
\pgfpathlineto{\pgfqpoint{2.655425in}{2.785782in}}%
\pgfpathlineto{\pgfqpoint{2.717337in}{2.746523in}}%
\pgfpathlineto{\pgfqpoint{2.776120in}{2.705859in}}%
\pgfpathlineto{\pgfqpoint{2.831669in}{2.663868in}}%
\pgfpathlineto{\pgfqpoint{2.883961in}{2.620648in}}%
\pgfpathlineto{\pgfqpoint{2.933004in}{2.576303in}}%
\pgfpathlineto{\pgfqpoint{2.978815in}{2.530938in}}%
\pgfpathlineto{\pgfqpoint{3.021405in}{2.484647in}}%
\pgfpathlineto{\pgfqpoint{3.060760in}{2.437513in}}%
\pgfpathlineto{\pgfqpoint{3.096831in}{2.389606in}}%
\pgfusepath{stroke}%
\end{pgfscope}%
\begin{pgfscope}%
\pgfpathrectangle{\pgfqpoint{0.647939in}{0.492442in}}{\pgfqpoint{4.273799in}{2.331163in}}%
\pgfusepath{clip}%
\pgfsetbuttcap%
\pgfsetroundjoin%
\pgfsetlinewidth{0.301125pt}%
\definecolor{currentstroke}{rgb}{0.500000,0.500000,0.500000}%
\pgfsetstrokecolor{currentstroke}%
\pgfsetstrokeopacity{0.300000}%
\pgfsetdash{}{0pt}%
\pgfpathmoveto{\pgfqpoint{2.299180in}{2.823605in}}%
\pgfpathlineto{\pgfqpoint{2.299180in}{2.823605in}}%
\pgfpathlineto{\pgfqpoint{2.377871in}{2.794608in}}%
\pgfpathlineto{\pgfqpoint{2.455383in}{2.764695in}}%
\pgfpathlineto{\pgfqpoint{2.530674in}{2.733163in}}%
\pgfpathlineto{\pgfqpoint{2.602939in}{2.699607in}}%
\pgfpathlineto{\pgfqpoint{2.671617in}{2.663878in}}%
\pgfpathlineto{\pgfqpoint{2.736319in}{2.626010in}}%
\pgfpathlineto{\pgfqpoint{2.796875in}{2.586153in}}%
\pgfpathlineto{\pgfqpoint{2.853252in}{2.544504in}}%
\pgfpathlineto{\pgfqpoint{2.905479in}{2.501271in}}%
\pgfpathlineto{\pgfqpoint{2.953609in}{2.456648in}}%
\pgfpathlineto{\pgfqpoint{2.997696in}{2.410796in}}%
\pgfpathlineto{\pgfqpoint{3.037741in}{2.363852in}}%
\pgfpathlineto{\pgfqpoint{3.073665in}{2.315924in}}%
\pgfpathlineto{\pgfqpoint{3.105277in}{2.267096in}}%
\pgfpathlineto{\pgfqpoint{3.132218in}{2.217447in}}%
\pgfpathlineto{\pgfqpoint{3.153872in}{2.167040in}}%
\pgfpathlineto{\pgfqpoint{3.169169in}{2.115954in}}%
\pgfpathlineto{\pgfqpoint{3.176165in}{2.064369in}}%
\pgfpathlineto{\pgfqpoint{3.170806in}{2.012852in}}%
\pgfpathlineto{\pgfqpoint{3.170806in}{2.012852in}}%
\pgfpathlineto{\pgfqpoint{3.152490in}{1.974807in}}%
\pgfpathlineto{\pgfqpoint{3.152490in}{1.974807in}}%
\pgfpathlineto{\pgfqpoint{3.127549in}{1.952743in}}%
\pgfpathlineto{\pgfqpoint{3.127549in}{1.952743in}}%
\pgfusepath{stroke}%
\end{pgfscope}%
\begin{pgfscope}%
\pgfpathrectangle{\pgfqpoint{0.647939in}{0.492442in}}{\pgfqpoint{4.273799in}{2.331163in}}%
\pgfusepath{clip}%
\pgfsetbuttcap%
\pgfsetroundjoin%
\pgfsetlinewidth{0.301125pt}%
\definecolor{currentstroke}{rgb}{0.500000,0.500000,0.500000}%
\pgfsetstrokecolor{currentstroke}%
\pgfsetstrokeopacity{0.300000}%
\pgfsetdash{}{0pt}%
\pgfpathmoveto{\pgfqpoint{2.007784in}{2.823605in}}%
\pgfpathlineto{\pgfqpoint{2.007784in}{2.823605in}}%
\pgfpathlineto{\pgfqpoint{2.085536in}{2.793955in}}%
\pgfpathlineto{\pgfqpoint{2.167222in}{2.767598in}}%
\pgfpathlineto{\pgfqpoint{2.251188in}{2.743423in}}%
\pgfpathlineto{\pgfqpoint{2.335762in}{2.719876in}}%
\pgfpathlineto{\pgfqpoint{2.419432in}{2.695408in}}%
\pgfpathlineto{\pgfqpoint{2.500852in}{2.668806in}}%
\pgfpathlineto{\pgfqpoint{2.578852in}{2.639349in}}%
\pgfpathlineto{\pgfqpoint{2.652578in}{2.606805in}}%
\pgfpathlineto{\pgfqpoint{2.721545in}{2.571283in}}%
\pgfpathlineto{\pgfqpoint{2.785513in}{2.533067in}}%
\pgfusepath{stroke}%
\end{pgfscope}%
\begin{pgfscope}%
\pgfpathrectangle{\pgfqpoint{0.647939in}{0.492442in}}{\pgfqpoint{4.273799in}{2.331163in}}%
\pgfusepath{clip}%
\pgfsetbuttcap%
\pgfsetroundjoin%
\pgfsetlinewidth{0.301125pt}%
\definecolor{currentstroke}{rgb}{0.500000,0.500000,0.500000}%
\pgfsetstrokecolor{currentstroke}%
\pgfsetstrokeopacity{0.300000}%
\pgfsetdash{}{0pt}%
\pgfpathmoveto{\pgfqpoint{1.813521in}{2.823605in}}%
\pgfpathlineto{\pgfqpoint{1.813521in}{2.823605in}}%
\pgfpathlineto{\pgfqpoint{1.878290in}{2.785815in}}%
\pgfpathlineto{\pgfqpoint{1.950290in}{2.752194in}}%
\pgfpathlineto{\pgfqpoint{2.029373in}{2.723715in}}%
\pgfpathlineto{\pgfqpoint{2.114075in}{2.700473in}}%
\pgfpathlineto{\pgfqpoint{2.202157in}{2.681196in}}%
\pgfpathlineto{\pgfqpoint{2.291492in}{2.663638in}}%
\pgfpathlineto{\pgfqpoint{2.380322in}{2.645387in}}%
\pgfpathlineto{\pgfqpoint{2.467134in}{2.624533in}}%
\pgfpathlineto{\pgfqpoint{2.550523in}{2.599923in}}%
\pgfpathlineto{\pgfqpoint{2.629312in}{2.571155in}}%
\pgfusepath{stroke}%
\end{pgfscope}%
\begin{pgfscope}%
\pgfpathrectangle{\pgfqpoint{0.647939in}{0.492442in}}{\pgfqpoint{4.273799in}{2.331163in}}%
\pgfusepath{clip}%
\pgfsetbuttcap%
\pgfsetroundjoin%
\pgfsetlinewidth{0.301125pt}%
\definecolor{currentstroke}{rgb}{0.500000,0.500000,0.500000}%
\pgfsetstrokecolor{currentstroke}%
\pgfsetstrokeopacity{0.300000}%
\pgfsetdash{}{0pt}%
\pgfpathmoveto{\pgfqpoint{1.716389in}{2.823605in}}%
\pgfpathlineto{\pgfqpoint{1.716389in}{2.823605in}}%
\pgfpathlineto{\pgfqpoint{1.771596in}{2.781508in}}%
\pgfpathlineto{\pgfqpoint{1.833546in}{2.742346in}}%
\pgfpathlineto{\pgfqpoint{1.903590in}{2.707552in}}%
\pgfpathlineto{\pgfqpoint{1.982190in}{2.678784in}}%
\pgfpathlineto{\pgfqpoint{2.067970in}{2.656933in}}%
\pgfusepath{stroke}%
\end{pgfscope}%
\begin{pgfscope}%
\pgfpathrectangle{\pgfqpoint{0.647939in}{0.492442in}}{\pgfqpoint{4.273799in}{2.331163in}}%
\pgfusepath{clip}%
\pgfsetbuttcap%
\pgfsetroundjoin%
\pgfsetlinewidth{0.301125pt}%
\definecolor{currentstroke}{rgb}{0.500000,0.500000,0.500000}%
\pgfsetstrokecolor{currentstroke}%
\pgfsetstrokeopacity{0.300000}%
\pgfsetdash{}{0pt}%
\pgfpathmoveto{\pgfqpoint{1.522125in}{2.823605in}}%
\pgfpathlineto{\pgfqpoint{1.522125in}{2.823605in}}%
\pgfpathlineto{\pgfqpoint{1.558885in}{2.775850in}}%
\pgfpathlineto{\pgfqpoint{1.598424in}{2.728760in}}%
\pgfpathlineto{\pgfqpoint{1.641630in}{2.682646in}}%
\pgfpathlineto{\pgfqpoint{1.689865in}{2.638062in}}%
\pgfpathlineto{\pgfqpoint{1.745234in}{2.596096in}}%
\pgfpathlineto{\pgfqpoint{1.810968in}{2.559044in}}%
\pgfpathlineto{\pgfqpoint{1.890132in}{2.531428in}}%
\pgfpathlineto{\pgfqpoint{1.966875in}{2.518944in}}%
\pgfpathlineto{\pgfqpoint{2.040544in}{2.516189in}}%
\pgfpathlineto{\pgfqpoint{2.124770in}{2.519297in}}%
\pgfpathlineto{\pgfqpoint{2.219062in}{2.525177in}}%
\pgfpathlineto{\pgfqpoint{2.313579in}{2.529078in}}%
\pgfpathlineto{\pgfqpoint{2.408155in}{2.527656in}}%
\pgfpathlineto{\pgfqpoint{2.501234in}{2.518746in}}%
\pgfpathlineto{\pgfqpoint{2.590314in}{2.501558in}}%
\pgfusepath{stroke}%
\end{pgfscope}%
\begin{pgfscope}%
\pgfpathrectangle{\pgfqpoint{0.647939in}{0.492442in}}{\pgfqpoint{4.273799in}{2.331163in}}%
\pgfusepath{clip}%
\pgfsetbuttcap%
\pgfsetroundjoin%
\pgfsetlinewidth{0.301125pt}%
\definecolor{currentstroke}{rgb}{0.500000,0.500000,0.500000}%
\pgfsetstrokecolor{currentstroke}%
\pgfsetstrokeopacity{0.300000}%
\pgfsetdash{}{0pt}%
\pgfpathmoveto{\pgfqpoint{1.424993in}{2.823605in}}%
\pgfpathlineto{\pgfqpoint{1.424993in}{2.823605in}}%
\pgfpathlineto{\pgfqpoint{1.454615in}{2.774390in}}%
\pgfpathlineto{\pgfqpoint{1.485518in}{2.725414in}}%
\pgfpathlineto{\pgfqpoint{1.518039in}{2.676752in}}%
\pgfpathlineto{\pgfqpoint{1.552652in}{2.628520in}}%
\pgfpathlineto{\pgfqpoint{1.590058in}{2.580921in}}%
\pgfpathlineto{\pgfqpoint{1.631424in}{2.534330in}}%
\pgfpathlineto{\pgfqpoint{1.678892in}{2.489551in}}%
\pgfpathlineto{\pgfqpoint{1.736651in}{2.448686in}}%
\pgfpathlineto{\pgfqpoint{1.736651in}{2.448686in}}%
\pgfpathlineto{\pgfqpoint{1.790848in}{2.424073in}}%
\pgfpathlineto{\pgfqpoint{1.790848in}{2.424073in}}%
\pgfpathlineto{\pgfqpoint{1.843129in}{2.412029in}}%
\pgfpathlineto{\pgfqpoint{1.900700in}{2.409777in}}%
\pgfpathlineto{\pgfqpoint{1.956076in}{2.415310in}}%
\pgfpathlineto{\pgfqpoint{2.021756in}{2.427788in}}%
\pgfpathlineto{\pgfqpoint{2.108902in}{2.448216in}}%
\pgfusepath{stroke}%
\end{pgfscope}%
\begin{pgfscope}%
\pgfpathrectangle{\pgfqpoint{0.647939in}{0.492442in}}{\pgfqpoint{4.273799in}{2.331163in}}%
\pgfusepath{clip}%
\pgfsetbuttcap%
\pgfsetroundjoin%
\pgfsetlinewidth{0.301125pt}%
\definecolor{currentstroke}{rgb}{0.500000,0.500000,0.500000}%
\pgfsetstrokecolor{currentstroke}%
\pgfsetstrokeopacity{0.300000}%
\pgfsetdash{}{0pt}%
\pgfpathmoveto{\pgfqpoint{1.327862in}{2.823605in}}%
\pgfpathlineto{\pgfqpoint{1.327862in}{2.823605in}}%
\pgfpathlineto{\pgfqpoint{1.351834in}{2.773482in}}%
\pgfpathlineto{\pgfqpoint{1.376239in}{2.723421in}}%
\pgfpathlineto{\pgfqpoint{1.401128in}{2.673432in}}%
\pgfpathlineto{\pgfqpoint{1.426584in}{2.623527in}}%
\pgfpathlineto{\pgfqpoint{1.452707in}{2.573726in}}%
\pgfpathlineto{\pgfqpoint{1.479631in}{2.524056in}}%
\pgfpathlineto{\pgfqpoint{1.507597in}{2.474557in}}%
\pgfpathlineto{\pgfqpoint{1.536940in}{2.425301in}}%
\pgfpathlineto{\pgfqpoint{1.568259in}{2.376422in}}%
\pgfpathlineto{\pgfqpoint{1.602813in}{2.328205in}}%
\pgfpathlineto{\pgfqpoint{1.643618in}{2.281510in}}%
\pgfpathlineto{\pgfqpoint{1.643618in}{2.281510in}}%
\pgfpathlineto{\pgfqpoint{1.686748in}{2.247970in}}%
\pgfpathlineto{\pgfqpoint{1.686748in}{2.247970in}}%
\pgfpathlineto{\pgfqpoint{1.718699in}{2.235323in}}%
\pgfpathlineto{\pgfqpoint{1.718699in}{2.235323in}}%
\pgfpathlineto{\pgfqpoint{1.750849in}{2.232850in}}%
\pgfpathlineto{\pgfqpoint{1.782494in}{2.238376in}}%
\pgfpathlineto{\pgfqpoint{1.815639in}{2.250092in}}%
\pgfpathlineto{\pgfqpoint{1.858853in}{2.270629in}}%
\pgfusepath{stroke}%
\end{pgfscope}%
\begin{pgfscope}%
\pgfpathrectangle{\pgfqpoint{0.647939in}{0.492442in}}{\pgfqpoint{4.273799in}{2.331163in}}%
\pgfusepath{clip}%
\pgfsetbuttcap%
\pgfsetroundjoin%
\pgfsetlinewidth{0.301125pt}%
\definecolor{currentstroke}{rgb}{0.500000,0.500000,0.500000}%
\pgfsetstrokecolor{currentstroke}%
\pgfsetstrokeopacity{0.300000}%
\pgfsetdash{}{0pt}%
\pgfpathmoveto{\pgfqpoint{1.230730in}{2.823605in}}%
\pgfpathlineto{\pgfqpoint{1.230730in}{2.823605in}}%
\pgfpathlineto{\pgfqpoint{1.250308in}{2.772916in}}%
\pgfpathlineto{\pgfqpoint{1.269851in}{2.722222in}}%
\pgfpathlineto{\pgfqpoint{1.289326in}{2.671521in}}%
\pgfpathlineto{\pgfqpoint{1.308694in}{2.620808in}}%
\pgfpathlineto{\pgfqpoint{1.327900in}{2.570078in}}%
\pgfpathlineto{\pgfqpoint{1.346880in}{2.519321in}}%
\pgfpathlineto{\pgfqpoint{1.365540in}{2.468531in}}%
\pgfpathlineto{\pgfqpoint{1.383764in}{2.417692in}}%
\pgfpathlineto{\pgfqpoint{1.401382in}{2.366791in}}%
\pgfpathlineto{\pgfqpoint{1.418158in}{2.315807in}}%
\pgfpathlineto{\pgfqpoint{1.433766in}{2.264713in}}%
\pgfpathlineto{\pgfqpoint{1.447713in}{2.213478in}}%
\pgfpathlineto{\pgfqpoint{1.459290in}{2.162070in}}%
\pgfpathlineto{\pgfqpoint{1.467453in}{2.110473in}}%
\pgfpathlineto{\pgfqpoint{1.470740in}{2.058728in}}%
\pgfpathlineto{\pgfqpoint{1.467447in}{2.007004in}}%
\pgfpathlineto{\pgfqpoint{1.456359in}{1.955622in}}%
\pgfpathlineto{\pgfqpoint{1.437725in}{1.904908in}}%
\pgfpathlineto{\pgfqpoint{1.413147in}{1.854941in}}%
\pgfusepath{stroke}%
\end{pgfscope}%
\begin{pgfscope}%
\pgfpathrectangle{\pgfqpoint{0.647939in}{0.492442in}}{\pgfqpoint{4.273799in}{2.331163in}}%
\pgfusepath{clip}%
\pgfsetbuttcap%
\pgfsetroundjoin%
\pgfsetlinewidth{0.301125pt}%
\definecolor{currentstroke}{rgb}{0.500000,0.500000,0.500000}%
\pgfsetstrokecolor{currentstroke}%
\pgfsetstrokeopacity{0.300000}%
\pgfsetdash{}{0pt}%
\pgfpathmoveto{\pgfqpoint{1.133598in}{2.823605in}}%
\pgfpathlineto{\pgfqpoint{1.133598in}{2.823605in}}%
\pgfpathlineto{\pgfqpoint{1.149743in}{2.772557in}}%
\pgfpathlineto{\pgfqpoint{1.165626in}{2.721484in}}%
\pgfpathlineto{\pgfqpoint{1.181192in}{2.670381in}}%
\pgfpathlineto{\pgfqpoint{1.196372in}{2.619245in}}%
\pgfpathlineto{\pgfqpoint{1.211102in}{2.568070in}}%
\pgfpathlineto{\pgfqpoint{1.225288in}{2.516848in}}%
\pgfpathlineto{\pgfqpoint{1.238821in}{2.465575in}}%
\pgfpathlineto{\pgfqpoint{1.251584in}{2.414243in}}%
\pgfpathlineto{\pgfqpoint{1.263419in}{2.362844in}}%
\pgfpathlineto{\pgfqpoint{1.274139in}{2.311374in}}%
\pgfpathlineto{\pgfqpoint{1.283534in}{2.259827in}}%
\pgfpathlineto{\pgfqpoint{1.291354in}{2.208202in}}%
\pgfpathlineto{\pgfqpoint{1.297305in}{2.156504in}}%
\pgfpathlineto{\pgfqpoint{1.301062in}{2.104746in}}%
\pgfpathlineto{\pgfqpoint{1.302287in}{2.052953in}}%
\pgfpathlineto{\pgfqpoint{1.300654in}{2.001164in}}%
\pgfpathlineto{\pgfqpoint{1.295895in}{1.949434in}}%
\pgfpathlineto{\pgfqpoint{1.287854in}{1.897826in}}%
\pgfpathlineto{\pgfqpoint{1.276533in}{1.846405in}}%
\pgfpathlineto{\pgfqpoint{1.262089in}{1.795219in}}%
\pgfpathlineto{\pgfqpoint{1.244818in}{1.744292in}}%
\pgfpathlineto{\pgfqpoint{1.225134in}{1.693627in}}%
\pgfpathlineto{\pgfqpoint{1.203464in}{1.643202in}}%
\pgfpathlineto{\pgfqpoint{1.180237in}{1.592984in}}%
\pgfpathlineto{\pgfqpoint{1.155831in}{1.542935in}}%
\pgfpathlineto{\pgfqpoint{1.130559in}{1.493012in}}%
\pgfpathlineto{\pgfqpoint{1.104690in}{1.443181in}}%
\pgfpathlineto{\pgfqpoint{1.078415in}{1.393408in}}%
\pgfpathlineto{\pgfqpoint{1.051908in}{1.343671in}}%
\pgfpathlineto{\pgfqpoint{1.025301in}{1.293954in}}%
\pgfpathlineto{\pgfqpoint{0.998672in}{1.244236in}}%
\pgfpathlineto{\pgfqpoint{0.972103in}{1.194509in}}%
\pgfpathlineto{\pgfqpoint{0.945645in}{1.144765in}}%
\pgfusepath{stroke}%
\end{pgfscope}%
\begin{pgfscope}%
\pgfpathrectangle{\pgfqpoint{0.647939in}{0.492442in}}{\pgfqpoint{4.273799in}{2.331163in}}%
\pgfusepath{clip}%
\pgfsetbuttcap%
\pgfsetroundjoin%
\pgfsetlinewidth{0.301125pt}%
\definecolor{currentstroke}{rgb}{0.500000,0.500000,0.500000}%
\pgfsetstrokecolor{currentstroke}%
\pgfsetstrokeopacity{0.300000}%
\pgfsetdash{}{0pt}%
\pgfpathmoveto{\pgfqpoint{1.036466in}{2.823605in}}%
\pgfpathlineto{\pgfqpoint{1.036466in}{2.823605in}}%
\pgfpathlineto{\pgfqpoint{1.049923in}{2.772325in}}%
\pgfpathlineto{\pgfqpoint{1.063008in}{2.721016in}}%
\pgfpathlineto{\pgfqpoint{1.075677in}{2.669676in}}%
\pgfpathlineto{\pgfqpoint{1.087871in}{2.618302in}}%
\pgfpathlineto{\pgfqpoint{1.099523in}{2.566891in}}%
\pgfpathlineto{\pgfqpoint{1.110563in}{2.515439in}}%
\pgfpathlineto{\pgfqpoint{1.120913in}{2.463945in}}%
\pgfpathlineto{\pgfqpoint{1.130480in}{2.412406in}}%
\pgfpathlineto{\pgfqpoint{1.139158in}{2.360820in}}%
\pgfpathlineto{\pgfqpoint{1.146837in}{2.309187in}}%
\pgfpathlineto{\pgfqpoint{1.153396in}{2.257509in}}%
\pgfpathlineto{\pgfqpoint{1.158703in}{2.205788in}}%
\pgfpathlineto{\pgfqpoint{1.162618in}{2.154030in}}%
\pgfpathlineto{\pgfqpoint{1.164999in}{2.102245in}}%
\pgfpathlineto{\pgfqpoint{1.165704in}{2.050445in}}%
\pgfpathlineto{\pgfqpoint{1.164605in}{1.998648in}}%
\pgfpathlineto{\pgfqpoint{1.161592in}{1.946873in}}%
\pgfpathlineto{\pgfqpoint{1.156585in}{1.895145in}}%
\pgfpathlineto{\pgfqpoint{1.149544in}{1.843489in}}%
\pgfpathlineto{\pgfqpoint{1.140473in}{1.791927in}}%
\pgfpathlineto{\pgfqpoint{1.129420in}{1.740480in}}%
\pgfpathlineto{\pgfqpoint{1.116480in}{1.689164in}}%
\pgfpathlineto{\pgfqpoint{1.101799in}{1.637988in}}%
\pgfusepath{stroke}%
\end{pgfscope}%
\begin{pgfscope}%
\pgfpathrectangle{\pgfqpoint{0.647939in}{0.492442in}}{\pgfqpoint{4.273799in}{2.331163in}}%
\pgfusepath{clip}%
\pgfsetbuttcap%
\pgfsetroundjoin%
\pgfsetlinewidth{0.301125pt}%
\definecolor{currentstroke}{rgb}{0.500000,0.500000,0.500000}%
\pgfsetstrokecolor{currentstroke}%
\pgfsetstrokeopacity{0.300000}%
\pgfsetdash{}{0pt}%
\pgfpathmoveto{\pgfqpoint{0.939334in}{2.823605in}}%
\pgfpathlineto{\pgfqpoint{0.939334in}{2.823605in}}%
\pgfpathlineto{\pgfqpoint{0.950659in}{2.772172in}}%
\pgfpathlineto{\pgfqpoint{0.961580in}{2.720712in}}%
\pgfpathlineto{\pgfqpoint{0.972053in}{2.669225in}}%
\pgfpathlineto{\pgfqpoint{0.982037in}{2.617709in}}%
\pgfpathlineto{\pgfqpoint{0.991483in}{2.566163in}}%
\pgfpathlineto{\pgfqpoint{1.000335in}{2.514585in}}%
\pgfpathlineto{\pgfqpoint{1.008535in}{2.462976in}}%
\pgfpathlineto{\pgfqpoint{1.016020in}{2.411334in}}%
\pgfpathlineto{\pgfqpoint{1.022724in}{2.359660in}}%
\pgfpathlineto{\pgfqpoint{1.028578in}{2.307956in}}%
\pgfpathlineto{\pgfqpoint{1.033509in}{2.256223in}}%
\pgfpathlineto{\pgfqpoint{1.037441in}{2.204465in}}%
\pgfpathlineto{\pgfqpoint{1.040295in}{2.152686in}}%
\pgfpathlineto{\pgfqpoint{1.041997in}{2.100891in}}%
\pgfpathlineto{\pgfqpoint{1.042475in}{2.049090in}}%
\pgfpathlineto{\pgfqpoint{1.041662in}{1.997289in}}%
\pgfpathlineto{\pgfqpoint{1.039501in}{1.945501in}}%
\pgfpathlineto{\pgfqpoint{1.035946in}{1.893735in}}%
\pgfpathlineto{\pgfqpoint{1.030968in}{1.842005in}}%
\pgfpathlineto{\pgfqpoint{1.024556in}{1.790321in}}%
\pgfpathlineto{\pgfqpoint{1.016715in}{1.738697in}}%
\pgfpathlineto{\pgfqpoint{1.007471in}{1.687141in}}%
\pgfpathlineto{\pgfqpoint{0.996864in}{1.635664in}}%
\pgfpathlineto{\pgfqpoint{0.984958in}{1.584272in}}%
\pgfpathlineto{\pgfqpoint{0.971834in}{1.532968in}}%
\pgfpathlineto{\pgfqpoint{0.957576in}{1.481754in}}%
\pgfpathlineto{\pgfqpoint{0.942278in}{1.430630in}}%
\pgfpathlineto{\pgfqpoint{0.926044in}{1.379592in}}%
\pgfpathlineto{\pgfqpoint{0.908969in}{1.328635in}}%
\pgfpathlineto{\pgfqpoint{0.891155in}{1.277754in}}%
\pgfpathlineto{\pgfqpoint{0.872698in}{1.226941in}}%
\pgfpathlineto{\pgfqpoint{0.853683in}{1.176189in}}%
\pgfpathlineto{\pgfqpoint{0.834196in}{1.125491in}}%
\pgfpathlineto{\pgfqpoint{0.814308in}{1.074839in}}%
\pgfpathlineto{\pgfqpoint{0.794089in}{1.024226in}}%
\pgfpathlineto{\pgfqpoint{0.773598in}{0.973645in}}%
\pgfpathlineto{\pgfqpoint{0.752891in}{0.923091in}}%
\pgfpathlineto{\pgfqpoint{0.732014in}{0.872557in}}%
\pgfpathlineto{\pgfqpoint{0.711010in}{0.822038in}}%
\pgfpathlineto{\pgfqpoint{0.689914in}{0.771531in}}%
\pgfpathlineto{\pgfqpoint{0.668760in}{0.721031in}}%
\pgfpathlineto{\pgfqpoint{0.647939in}{0.671366in}}%
\pgfusepath{stroke}%
\end{pgfscope}%
\begin{pgfscope}%
\pgfpathrectangle{\pgfqpoint{0.647939in}{0.492442in}}{\pgfqpoint{4.273799in}{2.331163in}}%
\pgfusepath{clip}%
\pgfsetbuttcap%
\pgfsetroundjoin%
\pgfsetlinewidth{0.301125pt}%
\definecolor{currentstroke}{rgb}{0.500000,0.500000,0.500000}%
\pgfsetstrokecolor{currentstroke}%
\pgfsetstrokeopacity{0.300000}%
\pgfsetdash{}{0pt}%
\pgfpathmoveto{\pgfqpoint{0.842203in}{2.823605in}}%
\pgfpathlineto{\pgfqpoint{0.842203in}{2.823605in}}%
\pgfpathlineto{\pgfqpoint{0.851818in}{2.772068in}}%
\pgfpathlineto{\pgfqpoint{0.861032in}{2.720509in}}%
\pgfpathlineto{\pgfqpoint{0.869812in}{2.668928in}}%
\pgfpathlineto{\pgfqpoint{0.878119in}{2.617323in}}%
\pgfpathlineto{\pgfqpoint{0.885919in}{2.565695in}}%
\pgfpathlineto{\pgfqpoint{0.893172in}{2.514043in}}%
\pgfpathlineto{\pgfqpoint{0.899838in}{2.462368in}}%
\pgfpathlineto{\pgfqpoint{0.905876in}{2.410669in}}%
\pgfpathlineto{\pgfqpoint{0.911239in}{2.358949in}}%
\pgfpathlineto{\pgfqpoint{0.915881in}{2.307208in}}%
\pgfpathlineto{\pgfqpoint{0.919755in}{2.255448in}}%
\pgfpathlineto{\pgfqpoint{0.922816in}{2.203672in}}%
\pgfpathlineto{\pgfqpoint{0.925018in}{2.151883in}}%
\pgfpathlineto{\pgfqpoint{0.926316in}{2.100085in}}%
\pgfpathlineto{\pgfqpoint{0.926667in}{2.048282in}}%
\pgfpathlineto{\pgfqpoint{0.926035in}{1.996480in}}%
\pgfpathlineto{\pgfqpoint{0.924384in}{1.944685in}}%
\pgfpathlineto{\pgfqpoint{0.921688in}{1.892904in}}%
\pgfpathlineto{\pgfqpoint{0.917925in}{1.841142in}}%
\pgfpathlineto{\pgfqpoint{0.913082in}{1.789407in}}%
\pgfpathlineto{\pgfqpoint{0.907155in}{1.737705in}}%
\pgfpathlineto{\pgfqpoint{0.900150in}{1.686044in}}%
\pgfpathlineto{\pgfqpoint{0.892086in}{1.634428in}}%
\pgfpathlineto{\pgfqpoint{0.882987in}{1.582865in}}%
\pgfpathlineto{\pgfqpoint{0.872883in}{1.531357in}}%
\pgfpathlineto{\pgfqpoint{0.861814in}{1.479907in}}%
\pgfpathlineto{\pgfqpoint{0.849832in}{1.428519in}}%
\pgfpathlineto{\pgfqpoint{0.836992in}{1.377193in}}%
\pgfpathlineto{\pgfqpoint{0.823346in}{1.325928in}}%
\pgfpathlineto{\pgfqpoint{0.808956in}{1.274724in}}%
\pgfusepath{stroke}%
\end{pgfscope}%
\begin{pgfscope}%
\pgfpathrectangle{\pgfqpoint{0.647939in}{0.492442in}}{\pgfqpoint{4.273799in}{2.331163in}}%
\pgfusepath{clip}%
\pgfsetbuttcap%
\pgfsetroundjoin%
\pgfsetlinewidth{0.301125pt}%
\definecolor{currentstroke}{rgb}{0.500000,0.500000,0.500000}%
\pgfsetstrokecolor{currentstroke}%
\pgfsetstrokeopacity{0.300000}%
\pgfsetdash{}{0pt}%
\pgfpathmoveto{\pgfqpoint{0.745071in}{2.823605in}}%
\pgfpathlineto{\pgfqpoint{0.745071in}{2.823605in}}%
\pgfpathlineto{\pgfqpoint{0.753307in}{2.771997in}}%
\pgfpathlineto{\pgfqpoint{0.761157in}{2.720371in}}%
\pgfpathlineto{\pgfqpoint{0.768598in}{2.668727in}}%
\pgfpathlineto{\pgfqpoint{0.775604in}{2.617065in}}%
\pgfpathlineto{\pgfqpoint{0.782146in}{2.565385in}}%
\pgfpathlineto{\pgfqpoint{0.788195in}{2.513686in}}%
\pgfpathlineto{\pgfqpoint{0.793722in}{2.461971in}}%
\pgfpathlineto{\pgfqpoint{0.798698in}{2.410239in}}%
\pgfpathlineto{\pgfqpoint{0.803091in}{2.358491in}}%
\pgfpathlineto{\pgfqpoint{0.806873in}{2.306729in}}%
\pgfpathlineto{\pgfqpoint{0.810011in}{2.254954in}}%
\pgfpathlineto{\pgfqpoint{0.812476in}{2.203168in}}%
\pgfpathlineto{\pgfqpoint{0.814239in}{2.151374in}}%
\pgfpathlineto{\pgfqpoint{0.815270in}{2.099574in}}%
\pgfpathlineto{\pgfqpoint{0.815543in}{2.047771in}}%
\pgfpathlineto{\pgfqpoint{0.815035in}{1.995968in}}%
\pgfpathlineto{\pgfqpoint{0.813722in}{1.944170in}}%
\pgfpathlineto{\pgfqpoint{0.811589in}{1.892380in}}%
\pgfpathlineto{\pgfqpoint{0.808619in}{1.840603in}}%
\pgfpathlineto{\pgfqpoint{0.804804in}{1.788841in}}%
\pgfpathlineto{\pgfqpoint{0.800138in}{1.737101in}}%
\pgfpathlineto{\pgfqpoint{0.794621in}{1.685386in}}%
\pgfpathlineto{\pgfqpoint{0.788257in}{1.633699in}}%
\pgfpathlineto{\pgfqpoint{0.781056in}{1.582046in}}%
\pgfpathlineto{\pgfqpoint{0.773034in}{1.530428in}}%
\pgfpathlineto{\pgfqpoint{0.764211in}{1.478849in}}%
\pgfpathlineto{\pgfqpoint{0.754615in}{1.427312in}}%
\pgfpathlineto{\pgfqpoint{0.744270in}{1.375817in}}%
\pgfpathlineto{\pgfqpoint{0.733208in}{1.324367in}}%
\pgfpathlineto{\pgfqpoint{0.721467in}{1.272962in}}%
\pgfpathlineto{\pgfqpoint{0.709085in}{1.221602in}}%
\pgfpathlineto{\pgfqpoint{0.696099in}{1.170286in}}%
\pgfpathlineto{\pgfqpoint{0.682548in}{1.119013in}}%
\pgfpathlineto{\pgfqpoint{0.668475in}{1.067783in}}%
\pgfpathlineto{\pgfqpoint{0.653918in}{1.016592in}}%
\pgfpathlineto{\pgfqpoint{0.647939in}{0.995901in}}%
\pgfusepath{stroke}%
\end{pgfscope}%
\begin{pgfscope}%
\pgfpathrectangle{\pgfqpoint{0.647939in}{0.492442in}}{\pgfqpoint{4.273799in}{2.331163in}}%
\pgfusepath{clip}%
\pgfsetbuttcap%
\pgfsetroundjoin%
\pgfsetlinewidth{0.301125pt}%
\definecolor{currentstroke}{rgb}{0.500000,0.500000,0.500000}%
\pgfsetstrokecolor{currentstroke}%
\pgfsetstrokeopacity{0.300000}%
\pgfsetdash{}{0pt}%
\pgfpathmoveto{\pgfqpoint{0.647939in}{2.823605in}}%
\pgfpathlineto{\pgfqpoint{0.647939in}{2.823605in}}%
\pgfpathlineto{\pgfqpoint{0.655049in}{2.771947in}}%
\pgfpathlineto{\pgfqpoint{0.661800in}{2.720275in}}%
\pgfpathlineto{\pgfqpoint{0.668171in}{2.668588in}}%
\pgfpathlineto{\pgfqpoint{0.674143in}{2.616887in}}%
\pgfpathlineto{\pgfqpoint{0.679697in}{2.565173in}}%
\pgfpathlineto{\pgfqpoint{0.684812in}{2.513444in}}%
\pgfpathlineto{\pgfqpoint{0.689467in}{2.461703in}}%
\pgfpathlineto{\pgfqpoint{0.693640in}{2.409950in}}%
\pgfpathlineto{\pgfqpoint{0.697309in}{2.358185in}}%
\pgfpathlineto{\pgfqpoint{0.700454in}{2.306410in}}%
\pgfpathlineto{\pgfqpoint{0.703053in}{2.254626in}}%
\pgfpathlineto{\pgfqpoint{0.705086in}{2.202835in}}%
\pgfpathlineto{\pgfqpoint{0.706533in}{2.151037in}}%
\pgfpathlineto{\pgfqpoint{0.707375in}{2.099236in}}%
\pgfpathlineto{\pgfqpoint{0.707595in}{2.047433in}}%
\pgfpathlineto{\pgfqpoint{0.707177in}{1.995630in}}%
\pgfpathlineto{\pgfqpoint{0.706105in}{1.943830in}}%
\pgfpathlineto{\pgfqpoint{0.704368in}{1.892036in}}%
\pgfpathlineto{\pgfqpoint{0.701956in}{1.840249in}}%
\pgfpathlineto{\pgfqpoint{0.698860in}{1.788473in}}%
\pgfpathlineto{\pgfqpoint{0.695076in}{1.736711in}}%
\pgfpathlineto{\pgfqpoint{0.690601in}{1.684966in}}%
\pgfpathlineto{\pgfqpoint{0.685437in}{1.633239in}}%
\pgfpathlineto{\pgfqpoint{0.679589in}{1.581534in}}%
\pgfpathlineto{\pgfqpoint{0.673065in}{1.529854in}}%
\pgfpathlineto{\pgfqpoint{0.665873in}{1.478199in}}%
\pgfpathlineto{\pgfqpoint{0.658026in}{1.426574in}}%
\pgfpathlineto{\pgfqpoint{0.649539in}{1.374978in}}%
\pgfpathlineto{\pgfqpoint{0.647939in}{1.365602in}}%
\pgfusepath{stroke}%
\end{pgfscope}%
\begin{pgfscope}%
\pgfpathrectangle{\pgfqpoint{0.647939in}{0.492442in}}{\pgfqpoint{4.273799in}{2.331163in}}%
\pgfusepath{clip}%
\pgfsetbuttcap%
\pgfsetroundjoin%
\pgfsetlinewidth{0.301125pt}%
\definecolor{currentstroke}{rgb}{0.500000,0.500000,0.500000}%
\pgfsetstrokecolor{currentstroke}%
\pgfsetstrokeopacity{0.300000}%
\pgfsetdash{}{0pt}%
\pgfpathmoveto{\pgfqpoint{0.647939in}{2.346777in}}%
\pgfpathlineto{\pgfqpoint{0.647939in}{2.346777in}}%
\pgfpathlineto{\pgfqpoint{0.650726in}{2.294995in}}%
\pgfpathlineto{\pgfqpoint{0.653003in}{2.243207in}}%
\pgfpathlineto{\pgfqpoint{0.654752in}{2.191412in}}%
\pgfpathlineto{\pgfqpoint{0.655959in}{2.139613in}}%
\pgfpathlineto{\pgfqpoint{0.656606in}{2.087811in}}%
\pgfpathlineto{\pgfqpoint{0.656679in}{2.036008in}}%
\pgfpathlineto{\pgfqpoint{0.656167in}{1.984205in}}%
\pgfpathlineto{\pgfqpoint{0.655055in}{1.932405in}}%
\pgfpathlineto{\pgfqpoint{0.653335in}{1.880610in}}%
\pgfpathlineto{\pgfqpoint{0.650998in}{1.828823in}}%
\pgfpathlineto{\pgfqpoint{0.648038in}{1.777045in}}%
\pgfpathlineto{\pgfqpoint{0.647939in}{1.775482in}}%
\pgfusepath{stroke}%
\end{pgfscope}%
\begin{pgfscope}%
\pgfpathrectangle{\pgfqpoint{0.647939in}{0.492442in}}{\pgfqpoint{4.273799in}{2.331163in}}%
\pgfusepath{clip}%
\pgfsetbuttcap%
\pgfsetroundjoin%
\pgfsetlinewidth{0.301125pt}%
\definecolor{currentstroke}{rgb}{0.500000,0.500000,0.500000}%
\pgfsetstrokecolor{currentstroke}%
\pgfsetstrokeopacity{0.300000}%
\pgfsetdash{}{0pt}%
\pgfpathmoveto{\pgfqpoint{0.745522in}{1.169521in}}%
\pgfpathlineto{\pgfqpoint{0.730391in}{1.118380in}}%
\pgfpathlineto{\pgfqpoint{0.714744in}{1.067286in}}%
\pgfpathlineto{\pgfqpoint{0.698630in}{1.016235in}}%
\pgfpathlineto{\pgfqpoint{0.682094in}{0.965224in}}%
\pgfpathlineto{\pgfqpoint{0.665185in}{0.914250in}}%
\pgfpathlineto{\pgfqpoint{0.647939in}{0.863309in}}%
\pgfpathlineto{\pgfqpoint{0.647939in}{0.863309in}}%
\pgfusepath{stroke}%
\end{pgfscope}%
\begin{pgfscope}%
\pgfpathrectangle{\pgfqpoint{0.647939in}{0.492442in}}{\pgfqpoint{4.273799in}{2.331163in}}%
\pgfusepath{clip}%
\pgfsetbuttcap%
\pgfsetroundjoin%
\pgfsetlinewidth{0.301125pt}%
\definecolor{currentstroke}{rgb}{0.500000,0.500000,0.500000}%
\pgfsetstrokecolor{currentstroke}%
\pgfsetstrokeopacity{0.300000}%
\pgfsetdash{}{0pt}%
\pgfpathmoveto{\pgfqpoint{1.499074in}{0.492442in}}%
\pgfpathlineto{\pgfqpoint{1.483175in}{0.504543in}}%
\pgfpathlineto{\pgfqpoint{1.424993in}{0.545423in}}%
\pgfpathlineto{\pgfqpoint{1.359484in}{0.582751in}}%
\pgfpathlineto{\pgfqpoint{1.283128in}{0.612901in}}%
\pgfpathlineto{\pgfqpoint{1.283128in}{0.612901in}}%
\pgfpathlineto{\pgfqpoint{1.221884in}{0.625595in}}%
\pgfpathlineto{\pgfqpoint{1.155506in}{0.626790in}}%
\pgfusepath{stroke}%
\end{pgfscope}%
\begin{pgfscope}%
\pgfpathrectangle{\pgfqpoint{0.647939in}{0.492442in}}{\pgfqpoint{4.273799in}{2.331163in}}%
\pgfusepath{clip}%
\pgfsetbuttcap%
\pgfsetroundjoin%
\pgfsetlinewidth{0.301125pt}%
\definecolor{currentstroke}{rgb}{0.500000,0.500000,0.500000}%
\pgfsetstrokecolor{currentstroke}%
\pgfsetstrokeopacity{0.300000}%
\pgfsetdash{}{0pt}%
\pgfpathmoveto{\pgfqpoint{4.338948in}{0.598404in}}%
\pgfpathlineto{\pgfqpoint{4.292632in}{0.643620in}}%
\pgfpathlineto{\pgfqpoint{4.243893in}{0.688070in}}%
\pgfpathlineto{\pgfqpoint{4.192509in}{0.731624in}}%
\pgfpathlineto{\pgfqpoint{4.138290in}{0.774144in}}%
\pgfpathlineto{\pgfqpoint{4.081119in}{0.815494in}}%
\pgfpathlineto{\pgfqpoint{4.020934in}{0.855550in}}%
\pgfpathlineto{\pgfqpoint{3.957805in}{0.894237in}}%
\pgfpathlineto{\pgfqpoint{3.891979in}{0.931565in}}%
\pgfpathlineto{\pgfqpoint{3.823863in}{0.967654in}}%
\pgfpathlineto{\pgfqpoint{3.754016in}{1.002750in}}%
\pgfpathlineto{\pgfqpoint{3.683122in}{1.037219in}}%
\pgfpathlineto{\pgfqpoint{3.611921in}{1.071500in}}%
\pgfpathlineto{\pgfqpoint{3.541112in}{1.106020in}}%
\pgfpathlineto{\pgfqpoint{3.471378in}{1.141179in}}%
\pgfpathlineto{\pgfqpoint{3.403308in}{1.177289in}}%
\pgfpathlineto{\pgfqpoint{3.337354in}{1.214546in}}%
\pgfpathlineto{\pgfqpoint{3.273875in}{1.253056in}}%
\pgfusepath{stroke}%
\end{pgfscope}%
\begin{pgfscope}%
\pgfpathrectangle{\pgfqpoint{0.647939in}{0.492442in}}{\pgfqpoint{4.273799in}{2.331163in}}%
\pgfusepath{clip}%
\pgfsetbuttcap%
\pgfsetroundjoin%
\pgfsetlinewidth{0.301125pt}%
\definecolor{currentstroke}{rgb}{0.500000,0.500000,0.500000}%
\pgfsetstrokecolor{currentstroke}%
\pgfsetstrokeopacity{0.300000}%
\pgfsetdash{}{0pt}%
\pgfpathmoveto{\pgfqpoint{1.604559in}{2.823605in}}%
\pgfpathlineto{\pgfqpoint{1.620357in}{2.806874in}}%
\pgfpathlineto{\pgfqpoint{1.665883in}{2.761449in}}%
\pgfpathlineto{\pgfqpoint{1.716389in}{2.717643in}}%
\pgfpathlineto{\pgfqpoint{1.773635in}{2.676399in}}%
\pgfpathlineto{\pgfqpoint{1.839740in}{2.639426in}}%
\pgfpathlineto{\pgfqpoint{1.916531in}{2.609465in}}%
\pgfpathlineto{\pgfqpoint{2.002630in}{2.589352in}}%
\pgfpathlineto{\pgfqpoint{2.088427in}{2.579109in}}%
\pgfpathlineto{\pgfqpoint{2.182755in}{2.573866in}}%
\pgfpathlineto{\pgfqpoint{2.277412in}{2.570117in}}%
\pgfpathlineto{\pgfqpoint{2.371603in}{2.564020in}}%
\pgfpathlineto{\pgfqpoint{2.464088in}{2.552864in}}%
\pgfusepath{stroke}%
\end{pgfscope}%
\begin{pgfscope}%
\pgfpathrectangle{\pgfqpoint{0.647939in}{0.492442in}}{\pgfqpoint{4.273799in}{2.331163in}}%
\pgfusepath{clip}%
\pgfsetbuttcap%
\pgfsetroundjoin%
\pgfsetlinewidth{0.301125pt}%
\definecolor{currentstroke}{rgb}{0.500000,0.500000,0.500000}%
\pgfsetstrokecolor{currentstroke}%
\pgfsetstrokeopacity{0.300000}%
\pgfsetdash{}{0pt}%
\pgfpathmoveto{\pgfqpoint{4.247801in}{0.492442in}}%
\pgfpathlineto{\pgfqpoint{4.210890in}{0.524137in}}%
\pgfpathlineto{\pgfqpoint{4.158824in}{0.567451in}}%
\pgfpathlineto{\pgfqpoint{4.104380in}{0.609888in}}%
\pgfpathlineto{\pgfqpoint{4.047552in}{0.651385in}}%
\pgfpathlineto{\pgfqpoint{3.988406in}{0.691907in}}%
\pgfpathlineto{\pgfqpoint{3.927076in}{0.731453in}}%
\pgfpathlineto{\pgfqpoint{3.863831in}{0.770094in}}%
\pgfpathlineto{\pgfqpoint{3.799026in}{0.807959in}}%
\pgfpathlineto{\pgfqpoint{3.733098in}{0.845246in}}%
\pgfpathlineto{\pgfqpoint{3.666551in}{0.882205in}}%
\pgfpathlineto{\pgfqpoint{3.599898in}{0.919107in}}%
\pgfpathlineto{\pgfqpoint{3.533655in}{0.956227in}}%
\pgfpathlineto{\pgfqpoint{3.468293in}{0.993806in}}%
\pgfusepath{stroke}%
\end{pgfscope}%
\begin{pgfscope}%
\pgfpathrectangle{\pgfqpoint{0.647939in}{0.492442in}}{\pgfqpoint{4.273799in}{2.331163in}}%
\pgfusepath{clip}%
\pgfsetbuttcap%
\pgfsetroundjoin%
\pgfsetlinewidth{0.301125pt}%
\definecolor{currentstroke}{rgb}{0.500000,0.500000,0.500000}%
\pgfsetstrokecolor{currentstroke}%
\pgfsetstrokeopacity{0.300000}%
\pgfsetdash{}{0pt}%
\pgfpathmoveto{\pgfqpoint{4.648571in}{1.554205in}}%
\pgfpathlineto{\pgfqpoint{4.630343in}{1.605043in}}%
\pgfpathlineto{\pgfqpoint{4.612513in}{1.655923in}}%
\pgfpathlineto{\pgfqpoint{4.595224in}{1.706857in}}%
\pgfpathlineto{\pgfqpoint{4.578656in}{1.757863in}}%
\pgfpathlineto{\pgfqpoint{4.563069in}{1.808961in}}%
\pgfpathlineto{\pgfqpoint{4.548833in}{1.860174in}}%
\pgfpathlineto{\pgfqpoint{4.536441in}{1.911528in}}%
\pgfpathlineto{\pgfqpoint{4.526620in}{1.963045in}}%
\pgfpathlineto{\pgfqpoint{4.520372in}{2.014720in}}%
\pgfpathlineto{\pgfqpoint{4.518959in}{2.066493in}}%
\pgfpathlineto{\pgfqpoint{4.523680in}{2.118195in}}%
\pgfpathlineto{\pgfqpoint{4.535323in}{2.169558in}}%
\pgfusepath{stroke}%
\end{pgfscope}%
\begin{pgfscope}%
\pgfpathrectangle{\pgfqpoint{0.647939in}{0.492442in}}{\pgfqpoint{4.273799in}{2.331163in}}%
\pgfusepath{clip}%
\pgfsetbuttcap%
\pgfsetroundjoin%
\pgfsetlinewidth{0.301125pt}%
\definecolor{currentstroke}{rgb}{0.500000,0.500000,0.500000}%
\pgfsetstrokecolor{currentstroke}%
\pgfsetstrokeopacity{0.300000}%
\pgfsetdash{}{0pt}%
\pgfpathmoveto{\pgfqpoint{4.229067in}{2.823605in}}%
\pgfpathlineto{\pgfqpoint{4.242402in}{2.807313in}}%
\pgfpathlineto{\pgfqpoint{4.282227in}{2.760295in}}%
\pgfpathlineto{\pgfqpoint{4.324954in}{2.714053in}}%
\pgfpathlineto{\pgfqpoint{4.372412in}{2.669253in}}%
\pgfpathlineto{\pgfqpoint{4.417510in}{2.634625in}}%
\pgfpathlineto{\pgfqpoint{4.456926in}{2.611878in}}%
\pgfpathlineto{\pgfqpoint{4.493877in}{2.597747in}}%
\pgfpathlineto{\pgfqpoint{4.536874in}{2.590810in}}%
\pgfpathlineto{\pgfqpoint{4.581189in}{2.594648in}}%
\pgfpathlineto{\pgfqpoint{4.581189in}{2.594648in}}%
\pgfpathlineto{\pgfqpoint{4.630343in}{2.611681in}}%
\pgfpathlineto{\pgfqpoint{4.630343in}{2.611681in}}%
\pgfpathlineto{\pgfqpoint{4.630343in}{2.611681in}}%
\pgfusepath{stroke}%
\end{pgfscope}%
\begin{pgfscope}%
\pgfpathrectangle{\pgfqpoint{0.647939in}{0.492442in}}{\pgfqpoint{4.273799in}{2.331163in}}%
\pgfusepath{clip}%
\pgfsetbuttcap%
\pgfsetroundjoin%
\pgfsetlinewidth{0.301125pt}%
\definecolor{currentstroke}{rgb}{0.500000,0.500000,0.500000}%
\pgfsetstrokecolor{currentstroke}%
\pgfsetstrokeopacity{0.300000}%
\pgfsetdash{}{0pt}%
\pgfpathmoveto{\pgfqpoint{2.007784in}{0.704366in}}%
\pgfpathlineto{\pgfqpoint{1.971718in}{0.752289in}}%
\pgfpathlineto{\pgfqpoint{1.935491in}{0.800175in}}%
\pgfpathlineto{\pgfqpoint{1.899005in}{0.848003in}}%
\pgfpathlineto{\pgfqpoint{1.862125in}{0.895741in}}%
\pgfpathlineto{\pgfqpoint{1.824684in}{0.943349in}}%
\pgfpathlineto{\pgfqpoint{1.786472in}{0.990772in}}%
\pgfusepath{stroke}%
\end{pgfscope}%
\begin{pgfscope}%
\pgfpathrectangle{\pgfqpoint{0.647939in}{0.492442in}}{\pgfqpoint{4.273799in}{2.331163in}}%
\pgfusepath{clip}%
\pgfsetbuttcap%
\pgfsetroundjoin%
\pgfsetlinewidth{0.301125pt}%
\definecolor{currentstroke}{rgb}{0.500000,0.500000,0.500000}%
\pgfsetstrokecolor{currentstroke}%
\pgfsetstrokeopacity{0.300000}%
\pgfsetdash{}{0pt}%
\pgfpathmoveto{\pgfqpoint{3.881570in}{0.626561in}}%
\pgfpathlineto{\pgfqpoint{3.819386in}{0.665713in}}%
\pgfpathlineto{\pgfqpoint{3.756157in}{0.704366in}}%
\pgfpathlineto{\pgfqpoint{3.692274in}{0.742699in}}%
\pgfpathlineto{\pgfqpoint{3.628156in}{0.780914in}}%
\pgfpathlineto{\pgfqpoint{3.564217in}{0.819218in}}%
\pgfpathlineto{\pgfqpoint{3.500856in}{0.857806in}}%
\pgfusepath{stroke}%
\end{pgfscope}%
\begin{pgfscope}%
\pgfpathrectangle{\pgfqpoint{0.647939in}{0.492442in}}{\pgfqpoint{4.273799in}{2.331163in}}%
\pgfusepath{clip}%
\pgfsetbuttcap%
\pgfsetroundjoin%
\pgfsetlinewidth{0.301125pt}%
\definecolor{currentstroke}{rgb}{0.500000,0.500000,0.500000}%
\pgfsetstrokecolor{currentstroke}%
\pgfsetstrokeopacity{0.300000}%
\pgfsetdash{}{0pt}%
\pgfpathmoveto{\pgfqpoint{4.564263in}{1.079261in}}%
\pgfpathlineto{\pgfqpoint{4.533211in}{1.128214in}}%
\pgfpathlineto{\pgfqpoint{4.500800in}{1.176903in}}%
\pgfpathlineto{\pgfqpoint{4.466776in}{1.225259in}}%
\pgfpathlineto{\pgfqpoint{4.430762in}{1.273181in}}%
\pgfpathlineto{\pgfqpoint{4.392243in}{1.320516in}}%
\pgfpathlineto{\pgfqpoint{4.350483in}{1.367019in}}%
\pgfpathlineto{\pgfqpoint{4.304399in}{1.412273in}}%
\pgfpathlineto{\pgfqpoint{4.252330in}{1.455508in}}%
\pgfpathlineto{\pgfqpoint{4.191709in}{1.495188in}}%
\pgfpathlineto{\pgfqpoint{4.119329in}{1.528199in}}%
\pgfpathlineto{\pgfqpoint{4.119329in}{1.528199in}}%
\pgfpathlineto{\pgfqpoint{4.050316in}{1.546922in}}%
\pgfpathlineto{\pgfqpoint{3.975073in}{1.556218in}}%
\pgfpathlineto{\pgfqpoint{3.897747in}{1.557608in}}%
\pgfpathlineto{\pgfqpoint{3.804220in}{1.553639in}}%
\pgfpathlineto{\pgfqpoint{3.709775in}{1.548427in}}%
\pgfpathlineto{\pgfqpoint{3.615082in}{1.546090in}}%
\pgfpathlineto{\pgfqpoint{3.520677in}{1.549515in}}%
\pgfpathlineto{\pgfqpoint{3.428314in}{1.560410in}}%
\pgfusepath{stroke}%
\end{pgfscope}%
\begin{pgfscope}%
\pgfpathrectangle{\pgfqpoint{0.647939in}{0.492442in}}{\pgfqpoint{4.273799in}{2.331163in}}%
\pgfusepath{clip}%
\pgfsetbuttcap%
\pgfsetroundjoin%
\pgfsetlinewidth{0.301125pt}%
\definecolor{currentstroke}{rgb}{0.500000,0.500000,0.500000}%
\pgfsetstrokecolor{currentstroke}%
\pgfsetstrokeopacity{0.300000}%
\pgfsetdash{}{0pt}%
\pgfpathmoveto{\pgfqpoint{4.560002in}{1.396403in}}%
\pgfpathlineto{\pgfqpoint{4.533211in}{1.446100in}}%
\pgfpathlineto{\pgfqpoint{4.505497in}{1.495644in}}%
\pgfpathlineto{\pgfqpoint{4.476643in}{1.544989in}}%
\pgfpathlineto{\pgfqpoint{4.446263in}{1.594063in}}%
\pgfpathlineto{\pgfqpoint{4.413804in}{1.642733in}}%
\pgfpathlineto{\pgfqpoint{4.378286in}{1.690738in}}%
\pgfpathlineto{\pgfqpoint{4.337642in}{1.737457in}}%
\pgfpathlineto{\pgfqpoint{4.286786in}{1.780803in}}%
\pgfpathlineto{\pgfqpoint{4.286786in}{1.780803in}}%
\pgfpathlineto{\pgfqpoint{4.246743in}{1.801546in}}%
\pgfpathlineto{\pgfqpoint{4.246743in}{1.801546in}}%
\pgfpathlineto{\pgfqpoint{4.208239in}{1.810263in}}%
\pgfpathlineto{\pgfqpoint{4.165751in}{1.808877in}}%
\pgfpathlineto{\pgfqpoint{4.127609in}{1.800415in}}%
\pgfpathlineto{\pgfqpoint{4.082590in}{1.784649in}}%
\pgfpathlineto{\pgfqpoint{4.021461in}{1.758088in}}%
\pgfpathlineto{\pgfqpoint{3.949083in}{1.724738in}}%
\pgfusepath{stroke}%
\end{pgfscope}%
\begin{pgfscope}%
\pgfpathrectangle{\pgfqpoint{0.647939in}{0.492442in}}{\pgfqpoint{4.273799in}{2.331163in}}%
\pgfusepath{clip}%
\pgfsetbuttcap%
\pgfsetroundjoin%
\pgfsetlinewidth{0.301125pt}%
\definecolor{currentstroke}{rgb}{0.500000,0.500000,0.500000}%
\pgfsetstrokecolor{currentstroke}%
\pgfsetstrokeopacity{0.300000}%
\pgfsetdash{}{0pt}%
\pgfpathmoveto{\pgfqpoint{1.397958in}{0.705955in}}%
\pgfpathlineto{\pgfqpoint{1.342849in}{0.731689in}}%
\pgfpathlineto{\pgfqpoint{1.290764in}{0.748306in}}%
\pgfpathlineto{\pgfqpoint{1.230730in}{0.757347in}}%
\pgfpathlineto{\pgfqpoint{1.230730in}{0.757347in}}%
\pgfpathlineto{\pgfqpoint{1.230730in}{0.757347in}}%
\pgfpathlineto{\pgfqpoint{1.174981in}{0.755533in}}%
\pgfpathlineto{\pgfqpoint{1.123035in}{0.744751in}}%
\pgfpathlineto{\pgfqpoint{1.073984in}{0.726432in}}%
\pgfpathlineto{\pgfqpoint{1.023785in}{0.699519in}}%
\pgfusepath{stroke}%
\end{pgfscope}%
\begin{pgfscope}%
\pgfpathrectangle{\pgfqpoint{0.647939in}{0.492442in}}{\pgfqpoint{4.273799in}{2.331163in}}%
\pgfusepath{clip}%
\pgfsetbuttcap%
\pgfsetroundjoin%
\pgfsetlinewidth{0.301125pt}%
\definecolor{currentstroke}{rgb}{0.500000,0.500000,0.500000}%
\pgfsetstrokecolor{currentstroke}%
\pgfsetstrokeopacity{0.300000}%
\pgfsetdash{}{0pt}%
\pgfpathmoveto{\pgfqpoint{2.687707in}{0.757347in}}%
\pgfpathlineto{\pgfqpoint{2.648872in}{0.804621in}}%
\pgfpathlineto{\pgfqpoint{2.610922in}{0.852108in}}%
\pgfpathlineto{\pgfqpoint{2.573840in}{0.899798in}}%
\pgfpathlineto{\pgfqpoint{2.537610in}{0.947684in}}%
\pgfpathlineto{\pgfqpoint{2.502223in}{0.995756in}}%
\pgfpathlineto{\pgfqpoint{2.467665in}{1.044008in}}%
\pgfpathlineto{\pgfqpoint{2.433923in}{1.092431in}}%
\pgfpathlineto{\pgfqpoint{2.400992in}{1.141020in}}%
\pgfpathlineto{\pgfqpoint{2.368876in}{1.189771in}}%
\pgfpathlineto{\pgfqpoint{2.337576in}{1.238680in}}%
\pgfpathlineto{\pgfqpoint{2.307097in}{1.287743in}}%
\pgfpathlineto{\pgfqpoint{2.277451in}{1.336957in}}%
\pgfpathlineto{\pgfqpoint{2.248669in}{1.386324in}}%
\pgfpathlineto{\pgfqpoint{2.220773in}{1.435842in}}%
\pgfpathlineto{\pgfqpoint{2.193805in}{1.485512in}}%
\pgfpathlineto{\pgfqpoint{2.167826in}{1.535339in}}%
\pgfpathlineto{\pgfqpoint{2.142906in}{1.585327in}}%
\pgfpathlineto{\pgfqpoint{2.119148in}{1.635482in}}%
\pgfpathlineto{\pgfqpoint{2.096675in}{1.685812in}}%
\pgfpathlineto{\pgfqpoint{2.075658in}{1.736329in}}%
\pgfpathlineto{\pgfqpoint{2.056320in}{1.787045in}}%
\pgfpathlineto{\pgfqpoint{2.038954in}{1.837971in}}%
\pgfpathlineto{\pgfqpoint{2.023965in}{1.889120in}}%
\pgfpathlineto{\pgfqpoint{2.011884in}{1.940493in}}%
\pgfpathlineto{\pgfqpoint{2.003432in}{1.992077in}}%
\pgfpathlineto{\pgfqpoint{1.999569in}{2.043817in}}%
\pgfpathlineto{\pgfqpoint{2.001524in}{2.095577in}}%
\pgfpathlineto{\pgfqpoint{2.010772in}{2.147079in}}%
\pgfpathlineto{\pgfqpoint{2.028850in}{2.197846in}}%
\pgfpathlineto{\pgfqpoint{2.057085in}{2.247177in}}%
\pgfpathlineto{\pgfqpoint{2.096290in}{2.294186in}}%
\pgfusepath{stroke}%
\end{pgfscope}%
\begin{pgfscope}%
\pgfpathrectangle{\pgfqpoint{0.647939in}{0.492442in}}{\pgfqpoint{4.273799in}{2.331163in}}%
\pgfusepath{clip}%
\pgfsetbuttcap%
\pgfsetroundjoin%
\pgfsetlinewidth{0.301125pt}%
\definecolor{currentstroke}{rgb}{0.500000,0.500000,0.500000}%
\pgfsetstrokecolor{currentstroke}%
\pgfsetstrokeopacity{0.300000}%
\pgfsetdash{}{0pt}%
\pgfpathmoveto{\pgfqpoint{3.561893in}{0.757347in}}%
\pgfpathlineto{\pgfqpoint{3.499644in}{0.796470in}}%
\pgfpathlineto{\pgfqpoint{3.438270in}{0.836000in}}%
\pgfpathlineto{\pgfqpoint{3.378056in}{0.876056in}}%
\pgfpathlineto{\pgfqpoint{3.319228in}{0.916720in}}%
\pgfpathlineto{\pgfqpoint{3.261973in}{0.958045in}}%
\pgfpathlineto{\pgfqpoint{3.206428in}{1.000057in}}%
\pgfpathlineto{\pgfqpoint{3.152680in}{1.042761in}}%
\pgfusepath{stroke}%
\end{pgfscope}%
\begin{pgfscope}%
\pgfpathrectangle{\pgfqpoint{0.647939in}{0.492442in}}{\pgfqpoint{4.273799in}{2.331163in}}%
\pgfusepath{clip}%
\pgfsetbuttcap%
\pgfsetroundjoin%
\pgfsetlinewidth{0.301125pt}%
\definecolor{currentstroke}{rgb}{0.500000,0.500000,0.500000}%
\pgfsetstrokecolor{currentstroke}%
\pgfsetstrokeopacity{0.300000}%
\pgfsetdash{}{0pt}%
\pgfpathmoveto{\pgfqpoint{1.526952in}{0.974394in}}%
\pgfpathlineto{\pgfqpoint{1.469554in}{1.013692in}}%
\pgfpathlineto{\pgfqpoint{1.418453in}{1.040729in}}%
\pgfpathlineto{\pgfqpoint{1.370639in}{1.058416in}}%
\pgfpathlineto{\pgfqpoint{1.320710in}{1.068329in}}%
\pgfpathlineto{\pgfqpoint{1.260702in}{1.068002in}}%
\pgfpathlineto{\pgfqpoint{1.205036in}{1.055523in}}%
\pgfpathlineto{\pgfqpoint{1.205036in}{1.055523in}}%
\pgfpathlineto{\pgfqpoint{1.133598in}{1.022252in}}%
\pgfusepath{stroke}%
\end{pgfscope}%
\begin{pgfscope}%
\pgfpathrectangle{\pgfqpoint{0.647939in}{0.492442in}}{\pgfqpoint{4.273799in}{2.331163in}}%
\pgfusepath{clip}%
\pgfsetbuttcap%
\pgfsetroundjoin%
\pgfsetlinewidth{0.301125pt}%
\definecolor{currentstroke}{rgb}{0.500000,0.500000,0.500000}%
\pgfsetstrokecolor{currentstroke}%
\pgfsetstrokeopacity{0.300000}%
\pgfsetdash{}{0pt}%
\pgfpathmoveto{\pgfqpoint{4.338948in}{1.128214in}}%
\pgfpathlineto{\pgfqpoint{4.292100in}{1.173249in}}%
\pgfpathlineto{\pgfqpoint{4.241000in}{1.216875in}}%
\pgfpathlineto{\pgfqpoint{4.184600in}{1.258503in}}%
\pgfpathlineto{\pgfqpoint{4.121808in}{1.297273in}}%
\pgfpathlineto{\pgfqpoint{4.051597in}{1.331997in}}%
\pgfpathlineto{\pgfqpoint{3.973724in}{1.361422in}}%
\pgfpathlineto{\pgfqpoint{3.889346in}{1.384952in}}%
\pgfpathlineto{\pgfqpoint{3.800743in}{1.403430in}}%
\pgfpathlineto{\pgfqpoint{3.710216in}{1.419016in}}%
\pgfpathlineto{\pgfqpoint{3.619532in}{1.434359in}}%
\pgfpathlineto{\pgfqpoint{3.530152in}{1.451739in}}%
\pgfusepath{stroke}%
\end{pgfscope}%
\begin{pgfscope}%
\pgfpathrectangle{\pgfqpoint{0.647939in}{0.492442in}}{\pgfqpoint{4.273799in}{2.331163in}}%
\pgfusepath{clip}%
\pgfsetbuttcap%
\pgfsetroundjoin%
\pgfsetlinewidth{0.301125pt}%
\definecolor{currentstroke}{rgb}{0.500000,0.500000,0.500000}%
\pgfsetstrokecolor{currentstroke}%
\pgfsetstrokeopacity{0.300000}%
\pgfsetdash{}{0pt}%
\pgfpathmoveto{\pgfqpoint{4.380196in}{1.611445in}}%
\pgfpathlineto{\pgfqpoint{4.338948in}{1.658024in}}%
\pgfpathlineto{\pgfqpoint{4.289753in}{1.702126in}}%
\pgfpathlineto{\pgfqpoint{4.289753in}{1.702126in}}%
\pgfpathlineto{\pgfqpoint{4.238254in}{1.733440in}}%
\pgfpathlineto{\pgfqpoint{4.238254in}{1.733440in}}%
\pgfpathlineto{\pgfqpoint{4.193189in}{1.748224in}}%
\pgfpathlineto{\pgfqpoint{4.193189in}{1.748224in}}%
\pgfpathlineto{\pgfqpoint{4.147601in}{1.752421in}}%
\pgfpathlineto{\pgfqpoint{4.101631in}{1.748049in}}%
\pgfpathlineto{\pgfqpoint{4.053431in}{1.737019in}}%
\pgfusepath{stroke}%
\end{pgfscope}%
\begin{pgfscope}%
\pgfpathrectangle{\pgfqpoint{0.647939in}{0.492442in}}{\pgfqpoint{4.273799in}{2.331163in}}%
\pgfusepath{clip}%
\pgfsetbuttcap%
\pgfsetroundjoin%
\pgfsetlinewidth{0.301125pt}%
\definecolor{currentstroke}{rgb}{0.500000,0.500000,0.500000}%
\pgfsetstrokecolor{currentstroke}%
\pgfsetstrokeopacity{0.300000}%
\pgfsetdash{}{0pt}%
\pgfpathmoveto{\pgfqpoint{4.238729in}{2.251127in}}%
\pgfpathlineto{\pgfqpoint{4.265175in}{2.201391in}}%
\pgfpathlineto{\pgfqpoint{4.291927in}{2.151757in}}%
\pgfpathlineto{\pgfqpoint{4.313682in}{2.113525in}}%
\pgfpathlineto{\pgfqpoint{4.326365in}{2.094192in}}%
\pgfpathlineto{\pgfqpoint{4.338948in}{2.081872in}}%
\pgfpathlineto{\pgfqpoint{4.338948in}{2.081872in}}%
\pgfpathlineto{\pgfqpoint{4.338948in}{2.081872in}}%
\pgfpathlineto{\pgfqpoint{4.338948in}{2.081872in}}%
\pgfpathlineto{\pgfqpoint{4.349330in}{2.082211in}}%
\pgfpathlineto{\pgfqpoint{4.362891in}{2.090805in}}%
\pgfpathlineto{\pgfqpoint{4.377322in}{2.103524in}}%
\pgfpathlineto{\pgfqpoint{4.402399in}{2.129208in}}%
\pgfusepath{stroke}%
\end{pgfscope}%
\begin{pgfscope}%
\pgfpathrectangle{\pgfqpoint{0.647939in}{0.492442in}}{\pgfqpoint{4.273799in}{2.331163in}}%
\pgfusepath{clip}%
\pgfsetbuttcap%
\pgfsetroundjoin%
\pgfsetlinewidth{0.301125pt}%
\definecolor{currentstroke}{rgb}{0.500000,0.500000,0.500000}%
\pgfsetstrokecolor{currentstroke}%
\pgfsetstrokeopacity{0.300000}%
\pgfsetdash{}{0pt}%
\pgfpathmoveto{\pgfqpoint{1.230730in}{2.240815in}}%
\pgfpathlineto{\pgfqpoint{1.236751in}{2.189118in}}%
\pgfpathlineto{\pgfqpoint{1.241048in}{2.137370in}}%
\pgfpathlineto{\pgfqpoint{1.243394in}{2.085586in}}%
\pgfpathlineto{\pgfqpoint{1.243565in}{2.033787in}}%
\pgfpathlineto{\pgfqpoint{1.241359in}{1.982002in}}%
\pgfpathlineto{\pgfqpoint{1.236614in}{1.930269in}}%
\pgfusepath{stroke}%
\end{pgfscope}%
\begin{pgfscope}%
\pgfpathrectangle{\pgfqpoint{0.647939in}{0.492442in}}{\pgfqpoint{4.273799in}{2.331163in}}%
\pgfusepath{clip}%
\pgfsetbuttcap%
\pgfsetroundjoin%
\pgfsetlinewidth{0.301125pt}%
\definecolor{currentstroke}{rgb}{0.500000,0.500000,0.500000}%
\pgfsetstrokecolor{currentstroke}%
\pgfsetstrokeopacity{0.300000}%
\pgfsetdash{}{0pt}%
\pgfpathmoveto{\pgfqpoint{1.719768in}{1.090644in}}%
\pgfpathlineto{\pgfqpoint{1.677846in}{1.137120in}}%
\pgfpathlineto{\pgfqpoint{1.633288in}{1.182851in}}%
\pgfpathlineto{\pgfqpoint{1.584762in}{1.227336in}}%
\pgfpathlineto{\pgfqpoint{1.531124in}{1.268602in}}%
\pgfpathlineto{\pgfqpoint{1.484239in}{1.296699in}}%
\pgfpathlineto{\pgfqpoint{1.441246in}{1.314999in}}%
\pgfpathlineto{\pgfqpoint{1.396943in}{1.325668in}}%
\pgfpathlineto{\pgfqpoint{1.341647in}{1.326639in}}%
\pgfpathlineto{\pgfqpoint{1.290053in}{1.315064in}}%
\pgfpathlineto{\pgfqpoint{1.290053in}{1.315064in}}%
\pgfpathlineto{\pgfqpoint{1.230730in}{1.287157in}}%
\pgfpathlineto{\pgfqpoint{1.230730in}{1.287157in}}%
\pgfusepath{stroke}%
\end{pgfscope}%
\begin{pgfscope}%
\pgfpathrectangle{\pgfqpoint{0.647939in}{0.492442in}}{\pgfqpoint{4.273799in}{2.331163in}}%
\pgfusepath{clip}%
\pgfsetbuttcap%
\pgfsetroundjoin%
\pgfsetlinewidth{0.301125pt}%
\definecolor{currentstroke}{rgb}{0.500000,0.500000,0.500000}%
\pgfsetstrokecolor{currentstroke}%
\pgfsetstrokeopacity{0.300000}%
\pgfsetdash{}{0pt}%
\pgfpathmoveto{\pgfqpoint{2.270919in}{0.766753in}}%
\pgfpathlineto{\pgfqpoint{2.236289in}{0.814989in}}%
\pgfpathlineto{\pgfqpoint{2.202048in}{0.863309in}}%
\pgfpathlineto{\pgfqpoint{2.168162in}{0.911703in}}%
\pgfpathlineto{\pgfqpoint{2.134595in}{0.960163in}}%
\pgfpathlineto{\pgfqpoint{2.101318in}{1.008682in}}%
\pgfpathlineto{\pgfqpoint{2.068298in}{1.057254in}}%
\pgfpathlineto{\pgfqpoint{2.035493in}{1.105869in}}%
\pgfpathlineto{\pgfqpoint{2.002849in}{1.154517in}}%
\pgfpathlineto{\pgfqpoint{1.970310in}{1.203185in}}%
\pgfusepath{stroke}%
\end{pgfscope}%
\begin{pgfscope}%
\pgfpathrectangle{\pgfqpoint{0.647939in}{0.492442in}}{\pgfqpoint{4.273799in}{2.331163in}}%
\pgfusepath{clip}%
\pgfsetbuttcap%
\pgfsetroundjoin%
\pgfsetlinewidth{0.301125pt}%
\definecolor{currentstroke}{rgb}{0.500000,0.500000,0.500000}%
\pgfsetstrokecolor{currentstroke}%
\pgfsetstrokeopacity{0.300000}%
\pgfsetdash{}{0pt}%
\pgfpathmoveto{\pgfqpoint{4.338397in}{0.933107in}}%
\pgfpathlineto{\pgfqpoint{4.291774in}{0.978219in}}%
\pgfpathlineto{\pgfqpoint{4.241816in}{1.022252in}}%
\pgfpathlineto{\pgfqpoint{4.187964in}{1.064896in}}%
\pgfpathlineto{\pgfqpoint{4.129676in}{1.105760in}}%
\pgfpathlineto{\pgfqpoint{4.066484in}{1.144377in}}%
\pgfpathlineto{\pgfqpoint{3.998095in}{1.180255in}}%
\pgfpathlineto{\pgfqpoint{3.924670in}{1.213038in}}%
\pgfpathlineto{\pgfqpoint{3.846939in}{1.242743in}}%
\pgfpathlineto{\pgfqpoint{3.766124in}{1.269922in}}%
\pgfpathlineto{\pgfqpoint{3.683657in}{1.295607in}}%
\pgfpathlineto{\pgfqpoint{3.600957in}{1.321065in}}%
\pgfpathlineto{\pgfqpoint{3.519328in}{1.347504in}}%
\pgfusepath{stroke}%
\end{pgfscope}%
\begin{pgfscope}%
\pgfpathrectangle{\pgfqpoint{0.647939in}{0.492442in}}{\pgfqpoint{4.273799in}{2.331163in}}%
\pgfusepath{clip}%
\pgfsetbuttcap%
\pgfsetroundjoin%
\pgfsetlinewidth{0.301125pt}%
\definecolor{currentstroke}{rgb}{0.500000,0.500000,0.500000}%
\pgfsetstrokecolor{currentstroke}%
\pgfsetstrokeopacity{0.300000}%
\pgfsetdash{}{0pt}%
\pgfpathmoveto{\pgfqpoint{4.294381in}{1.350058in}}%
\pgfpathlineto{\pgfqpoint{4.241816in}{1.393119in}}%
\pgfpathlineto{\pgfqpoint{4.181607in}{1.433053in}}%
\pgfpathlineto{\pgfqpoint{4.111465in}{1.467647in}}%
\pgfpathlineto{\pgfqpoint{4.030146in}{1.493639in}}%
\pgfpathlineto{\pgfqpoint{3.946475in}{1.508103in}}%
\pgfusepath{stroke}%
\end{pgfscope}%
\begin{pgfscope}%
\pgfpathrectangle{\pgfqpoint{0.647939in}{0.492442in}}{\pgfqpoint{4.273799in}{2.331163in}}%
\pgfusepath{clip}%
\pgfsetbuttcap%
\pgfsetroundjoin%
\pgfsetlinewidth{0.301125pt}%
\definecolor{currentstroke}{rgb}{0.500000,0.500000,0.500000}%
\pgfsetstrokecolor{currentstroke}%
\pgfsetstrokeopacity{0.300000}%
\pgfsetdash{}{0pt}%
\pgfpathmoveto{\pgfqpoint{4.343824in}{1.465088in}}%
\pgfpathlineto{\pgfqpoint{4.296717in}{1.509987in}}%
\pgfpathlineto{\pgfqpoint{4.241816in}{1.552062in}}%
\pgfpathlineto{\pgfqpoint{4.174867in}{1.588331in}}%
\pgfpathlineto{\pgfqpoint{4.174867in}{1.588331in}}%
\pgfpathlineto{\pgfqpoint{4.114933in}{1.607755in}}%
\pgfpathlineto{\pgfqpoint{4.045977in}{1.617217in}}%
\pgfpathlineto{\pgfqpoint{3.980928in}{1.616874in}}%
\pgfpathlineto{\pgfqpoint{3.908962in}{1.609925in}}%
\pgfpathlineto{\pgfqpoint{3.817262in}{1.596739in}}%
\pgfusepath{stroke}%
\end{pgfscope}%
\begin{pgfscope}%
\pgfpathrectangle{\pgfqpoint{0.647939in}{0.492442in}}{\pgfqpoint{4.273799in}{2.331163in}}%
\pgfusepath{clip}%
\pgfsetbuttcap%
\pgfsetroundjoin%
\pgfsetlinewidth{0.301125pt}%
\definecolor{currentstroke}{rgb}{0.500000,0.500000,0.500000}%
\pgfsetstrokecolor{currentstroke}%
\pgfsetstrokeopacity{0.300000}%
\pgfsetdash{}{0pt}%
\pgfpathmoveto{\pgfqpoint{1.544904in}{2.547964in}}%
\pgfpathlineto{\pgfqpoint{1.580165in}{2.499891in}}%
\pgfpathlineto{\pgfqpoint{1.619257in}{2.452738in}}%
\pgfpathlineto{\pgfqpoint{1.664661in}{2.407371in}}%
\pgfpathlineto{\pgfqpoint{1.722129in}{2.366646in}}%
\pgfpathlineto{\pgfqpoint{1.722129in}{2.366646in}}%
\pgfpathlineto{\pgfqpoint{1.766890in}{2.348508in}}%
\pgfpathlineto{\pgfqpoint{1.766890in}{2.348508in}}%
\pgfpathlineto{\pgfqpoint{1.811657in}{2.341492in}}%
\pgfpathlineto{\pgfqpoint{1.858441in}{2.343775in}}%
\pgfpathlineto{\pgfqpoint{1.904635in}{2.352840in}}%
\pgfusepath{stroke}%
\end{pgfscope}%
\begin{pgfscope}%
\pgfpathrectangle{\pgfqpoint{0.647939in}{0.492442in}}{\pgfqpoint{4.273799in}{2.331163in}}%
\pgfusepath{clip}%
\pgfsetbuttcap%
\pgfsetroundjoin%
\pgfsetlinewidth{0.301125pt}%
\definecolor{currentstroke}{rgb}{0.500000,0.500000,0.500000}%
\pgfsetstrokecolor{currentstroke}%
\pgfsetstrokeopacity{0.300000}%
\pgfsetdash{}{0pt}%
\pgfpathmoveto{\pgfqpoint{4.144684in}{0.969271in}}%
\pgfpathlineto{\pgfqpoint{4.085170in}{1.009615in}}%
\pgfpathlineto{\pgfqpoint{4.021580in}{1.048056in}}%
\pgfpathlineto{\pgfqpoint{3.953921in}{1.084373in}}%
\pgfpathlineto{\pgfqpoint{3.882526in}{1.118501in}}%
\pgfpathlineto{\pgfqpoint{3.808075in}{1.150641in}}%
\pgfpathlineto{\pgfqpoint{3.731526in}{1.181293in}}%
\pgfusepath{stroke}%
\end{pgfscope}%
\begin{pgfscope}%
\pgfpathrectangle{\pgfqpoint{0.647939in}{0.492442in}}{\pgfqpoint{4.273799in}{2.331163in}}%
\pgfusepath{clip}%
\pgfsetbuttcap%
\pgfsetroundjoin%
\pgfsetlinewidth{0.301125pt}%
\definecolor{currentstroke}{rgb}{0.500000,0.500000,0.500000}%
\pgfsetstrokecolor{currentstroke}%
\pgfsetstrokeopacity{0.300000}%
\pgfsetdash{}{0pt}%
\pgfpathmoveto{\pgfqpoint{1.830112in}{1.416010in}}%
\pgfpathlineto{\pgfqpoint{1.796525in}{1.464463in}}%
\pgfpathlineto{\pgfqpoint{1.762075in}{1.512733in}}%
\pgfpathlineto{\pgfqpoint{1.726273in}{1.560710in}}%
\pgfpathlineto{\pgfqpoint{1.688241in}{1.608167in}}%
\pgfpathlineto{\pgfqpoint{1.646245in}{1.654586in}}%
\pgfpathlineto{\pgfqpoint{1.607070in}{1.690208in}}%
\pgfpathlineto{\pgfqpoint{1.573997in}{1.712924in}}%
\pgfpathlineto{\pgfqpoint{1.544339in}{1.726551in}}%
\pgfpathlineto{\pgfqpoint{1.508096in}{1.733653in}}%
\pgfpathlineto{\pgfqpoint{1.469873in}{1.729658in}}%
\pgfpathlineto{\pgfqpoint{1.469873in}{1.729658in}}%
\pgfpathlineto{\pgfqpoint{1.424993in}{1.711005in}}%
\pgfpathlineto{\pgfqpoint{1.424993in}{1.711005in}}%
\pgfusepath{stroke}%
\end{pgfscope}%
\begin{pgfscope}%
\pgfpathrectangle{\pgfqpoint{0.647939in}{0.492442in}}{\pgfqpoint{4.273799in}{2.331163in}}%
\pgfusepath{clip}%
\pgfsetbuttcap%
\pgfsetroundjoin%
\pgfsetlinewidth{0.301125pt}%
\definecolor{currentstroke}{rgb}{0.500000,0.500000,0.500000}%
\pgfsetstrokecolor{currentstroke}%
\pgfsetstrokeopacity{0.300000}%
\pgfsetdash{}{0pt}%
\pgfpathmoveto{\pgfqpoint{4.242934in}{1.281523in}}%
\pgfpathlineto{\pgfqpoint{4.185592in}{1.322734in}}%
\pgfpathlineto{\pgfqpoint{4.120866in}{1.360484in}}%
\pgfpathlineto{\pgfqpoint{4.047552in}{1.393119in}}%
\pgfpathlineto{\pgfqpoint{3.965544in}{1.418837in}}%
\pgfpathlineto{\pgfqpoint{3.877089in}{1.437244in}}%
\pgfpathlineto{\pgfqpoint{3.785229in}{1.450137in}}%
\pgfusepath{stroke}%
\end{pgfscope}%
\begin{pgfscope}%
\pgfpathrectangle{\pgfqpoint{0.647939in}{0.492442in}}{\pgfqpoint{4.273799in}{2.331163in}}%
\pgfusepath{clip}%
\pgfsetbuttcap%
\pgfsetroundjoin%
\pgfsetlinewidth{0.301125pt}%
\definecolor{currentstroke}{rgb}{0.500000,0.500000,0.500000}%
\pgfsetstrokecolor{currentstroke}%
\pgfsetstrokeopacity{0.300000}%
\pgfsetdash{}{0pt}%
\pgfpathmoveto{\pgfqpoint{2.432404in}{1.750351in}}%
\pgfpathlineto{\pgfqpoint{2.412858in}{1.801041in}}%
\pgfpathlineto{\pgfqpoint{2.395799in}{1.851996in}}%
\pgfpathlineto{\pgfqpoint{2.381579in}{1.903208in}}%
\pgfpathlineto{\pgfqpoint{2.370663in}{1.954658in}}%
\pgfpathlineto{\pgfqpoint{2.363655in}{2.006306in}}%
\pgfpathlineto{\pgfqpoint{2.361377in}{2.058071in}}%
\pgfpathlineto{\pgfqpoint{2.364948in}{2.109802in}}%
\pgfpathlineto{\pgfqpoint{2.375919in}{2.161202in}}%
\pgfpathlineto{\pgfqpoint{2.396521in}{2.211677in}}%
\pgfpathlineto{\pgfqpoint{2.427085in}{2.256767in}}%
\pgfpathlineto{\pgfqpoint{2.464607in}{2.292420in}}%
\pgfpathlineto{\pgfqpoint{2.508430in}{2.319490in}}%
\pgfpathlineto{\pgfqpoint{2.559913in}{2.338594in}}%
\pgfpathlineto{\pgfqpoint{2.621897in}{2.348552in}}%
\pgfpathlineto{\pgfqpoint{2.687707in}{2.346777in}}%
\pgfpathlineto{\pgfqpoint{2.687707in}{2.346777in}}%
\pgfpathlineto{\pgfqpoint{2.687707in}{2.346777in}}%
\pgfpathlineto{\pgfqpoint{2.749336in}{2.334893in}}%
\pgfusepath{stroke}%
\end{pgfscope}%
\begin{pgfscope}%
\pgfpathrectangle{\pgfqpoint{0.647939in}{0.492442in}}{\pgfqpoint{4.273799in}{2.331163in}}%
\pgfusepath{clip}%
\pgfsetbuttcap%
\pgfsetroundjoin%
\pgfsetlinewidth{0.301125pt}%
\definecolor{currentstroke}{rgb}{0.500000,0.500000,0.500000}%
\pgfsetstrokecolor{currentstroke}%
\pgfsetstrokeopacity{0.300000}%
\pgfsetdash{}{0pt}%
\pgfpathmoveto{\pgfqpoint{1.522125in}{2.346777in}}%
\pgfpathlineto{\pgfqpoint{1.548544in}{2.297032in}}%
\pgfpathlineto{\pgfqpoint{1.576258in}{2.247516in}}%
\pgfpathlineto{\pgfqpoint{1.606441in}{2.198475in}}%
\pgfpathlineto{\pgfqpoint{1.644065in}{2.151126in}}%
\pgfpathlineto{\pgfqpoint{1.644065in}{2.151126in}}%
\pgfpathlineto{\pgfqpoint{1.662216in}{2.137577in}}%
\pgfpathlineto{\pgfqpoint{1.662216in}{2.137577in}}%
\pgfpathlineto{\pgfqpoint{1.682826in}{2.132473in}}%
\pgfpathlineto{\pgfqpoint{1.705029in}{2.137447in}}%
\pgfpathlineto{\pgfqpoint{1.722878in}{2.146395in}}%
\pgfpathlineto{\pgfqpoint{1.750775in}{2.164800in}}%
\pgfusepath{stroke}%
\end{pgfscope}%
\begin{pgfscope}%
\pgfpathrectangle{\pgfqpoint{0.647939in}{0.492442in}}{\pgfqpoint{4.273799in}{2.331163in}}%
\pgfusepath{clip}%
\pgfsetbuttcap%
\pgfsetroundjoin%
\pgfsetlinewidth{0.301125pt}%
\definecolor{currentstroke}{rgb}{0.500000,0.500000,0.500000}%
\pgfsetstrokecolor{currentstroke}%
\pgfsetstrokeopacity{0.300000}%
\pgfsetdash{}{0pt}%
\pgfpathmoveto{\pgfqpoint{1.522125in}{2.240815in}}%
\pgfpathlineto{\pgfqpoint{1.542163in}{2.190195in}}%
\pgfpathlineto{\pgfqpoint{1.560179in}{2.139358in}}%
\pgfpathlineto{\pgfqpoint{1.573402in}{2.088121in}}%
\pgfpathlineto{\pgfqpoint{1.573644in}{2.036687in}}%
\pgfpathlineto{\pgfqpoint{1.573644in}{2.036687in}}%
\pgfpathlineto{\pgfqpoint{1.559146in}{1.996943in}}%
\pgfpathlineto{\pgfqpoint{1.534337in}{1.957423in}}%
\pgfusepath{stroke}%
\end{pgfscope}%
\begin{pgfscope}%
\pgfpathrectangle{\pgfqpoint{0.647939in}{0.492442in}}{\pgfqpoint{4.273799in}{2.331163in}}%
\pgfusepath{clip}%
\pgfsetbuttcap%
\pgfsetroundjoin%
\pgfsetlinewidth{0.301125pt}%
\definecolor{currentstroke}{rgb}{0.500000,0.500000,0.500000}%
\pgfsetstrokecolor{currentstroke}%
\pgfsetstrokeopacity{0.300000}%
\pgfsetdash{}{0pt}%
\pgfpathmoveto{\pgfqpoint{1.629597in}{1.418829in}}%
\pgfpathlineto{\pgfqpoint{1.582772in}{1.459762in}}%
\pgfpathlineto{\pgfqpoint{1.522125in}{1.499081in}}%
\pgfpathlineto{\pgfqpoint{1.522125in}{1.499081in}}%
\pgfpathlineto{\pgfqpoint{1.475353in}{1.517445in}}%
\pgfpathlineto{\pgfqpoint{1.475353in}{1.517445in}}%
\pgfpathlineto{\pgfqpoint{1.432515in}{1.523808in}}%
\pgfpathlineto{\pgfqpoint{1.388287in}{1.519562in}}%
\pgfpathlineto{\pgfqpoint{1.351576in}{1.508011in}}%
\pgfpathlineto{\pgfqpoint{1.314043in}{1.489056in}}%
\pgfusepath{stroke}%
\end{pgfscope}%
\begin{pgfscope}%
\pgfpathrectangle{\pgfqpoint{0.647939in}{0.492442in}}{\pgfqpoint{4.273799in}{2.331163in}}%
\pgfusepath{clip}%
\pgfsetbuttcap%
\pgfsetroundjoin%
\pgfsetlinewidth{0.301125pt}%
\definecolor{currentstroke}{rgb}{0.500000,0.500000,0.500000}%
\pgfsetstrokecolor{currentstroke}%
\pgfsetstrokeopacity{0.300000}%
\pgfsetdash{}{0pt}%
\pgfpathmoveto{\pgfqpoint{3.641925in}{2.344748in}}%
\pgfpathlineto{\pgfqpoint{3.659025in}{2.293796in}}%
\pgfpathlineto{\pgfqpoint{3.673752in}{2.242624in}}%
\pgfpathlineto{\pgfqpoint{3.685752in}{2.191243in}}%
\pgfpathlineto{\pgfqpoint{3.694574in}{2.139672in}}%
\pgfpathlineto{\pgfqpoint{3.699632in}{2.087956in}}%
\pgfpathlineto{\pgfqpoint{3.700177in}{2.036174in}}%
\pgfpathlineto{\pgfqpoint{3.695241in}{1.984471in}}%
\pgfpathlineto{\pgfqpoint{3.683578in}{1.933106in}}%
\pgfpathlineto{\pgfqpoint{3.663603in}{1.882530in}}%
\pgfusepath{stroke}%
\end{pgfscope}%
\begin{pgfscope}%
\pgfpathrectangle{\pgfqpoint{0.647939in}{0.492442in}}{\pgfqpoint{4.273799in}{2.331163in}}%
\pgfusepath{clip}%
\pgfsetbuttcap%
\pgfsetroundjoin%
\pgfsetlinewidth{0.301125pt}%
\definecolor{currentstroke}{rgb}{0.500000,0.500000,0.500000}%
\pgfsetstrokecolor{currentstroke}%
\pgfsetstrokeopacity{0.300000}%
\pgfsetdash{}{0pt}%
\pgfpathmoveto{\pgfqpoint{3.400328in}{2.445133in}}%
\pgfpathlineto{\pgfqpoint{3.424058in}{2.394977in}}%
\pgfpathlineto{\pgfqpoint{3.445608in}{2.344529in}}%
\pgfpathlineto{\pgfqpoint{3.464761in}{2.293796in}}%
\pgfpathlineto{\pgfqpoint{3.481242in}{2.242785in}}%
\pgfpathlineto{\pgfqpoint{3.494675in}{2.191511in}}%
\pgfpathlineto{\pgfqpoint{3.504576in}{2.140001in}}%
\pgfpathlineto{\pgfqpoint{3.510296in}{2.088308in}}%
\pgfpathlineto{\pgfqpoint{3.510940in}{2.036532in}}%
\pgfpathlineto{\pgfqpoint{3.505253in}{1.984865in}}%
\pgfpathlineto{\pgfqpoint{3.491421in}{1.933686in}}%
\pgfpathlineto{\pgfqpoint{3.466664in}{1.883810in}}%
\pgfusepath{stroke}%
\end{pgfscope}%
\begin{pgfscope}%
\pgfpathrectangle{\pgfqpoint{0.647939in}{0.492442in}}{\pgfqpoint{4.273799in}{2.331163in}}%
\pgfusepath{clip}%
\pgfsetbuttcap%
\pgfsetroundjoin%
\pgfsetlinewidth{0.301125pt}%
\definecolor{currentstroke}{rgb}{0.500000,0.500000,0.500000}%
\pgfsetstrokecolor{currentstroke}%
\pgfsetstrokeopacity{0.300000}%
\pgfsetdash{}{0pt}%
\pgfpathmoveto{\pgfqpoint{1.950642in}{1.791273in}}%
\pgfpathlineto{\pgfqpoint{1.931800in}{1.842043in}}%
\pgfpathlineto{\pgfqpoint{1.915239in}{1.893046in}}%
\pgfpathlineto{\pgfqpoint{1.901654in}{1.944305in}}%
\pgfpathlineto{\pgfqpoint{1.892038in}{1.995825in}}%
\pgfpathlineto{\pgfqpoint{1.887808in}{2.047546in}}%
\pgfpathlineto{\pgfqpoint{1.890878in}{2.099267in}}%
\pgfpathlineto{\pgfqpoint{1.903386in}{2.150524in}}%
\pgfpathlineto{\pgfqpoint{1.927004in}{2.200565in}}%
\pgfpathlineto{\pgfqpoint{1.962112in}{2.248541in}}%
\pgfpathlineto{\pgfqpoint{2.007784in}{2.293796in}}%
\pgfusepath{stroke}%
\end{pgfscope}%
\begin{pgfscope}%
\pgfpathrectangle{\pgfqpoint{0.647939in}{0.492442in}}{\pgfqpoint{4.273799in}{2.331163in}}%
\pgfusepath{clip}%
\pgfsetbuttcap%
\pgfsetroundjoin%
\pgfsetlinewidth{0.301125pt}%
\definecolor{currentstroke}{rgb}{0.500000,0.500000,0.500000}%
\pgfsetstrokecolor{currentstroke}%
\pgfsetstrokeopacity{0.300000}%
\pgfsetdash{}{0pt}%
\pgfpathmoveto{\pgfqpoint{2.693863in}{1.959830in}}%
\pgfpathlineto{\pgfqpoint{2.683254in}{2.011272in}}%
\pgfpathlineto{\pgfqpoint{2.680394in}{2.062985in}}%
\pgfpathlineto{\pgfqpoint{2.687515in}{2.110568in}}%
\pgfpathlineto{\pgfqpoint{2.702705in}{2.146436in}}%
\pgfpathlineto{\pgfqpoint{2.723538in}{2.172371in}}%
\pgfpathlineto{\pgfqpoint{2.750258in}{2.190825in}}%
\pgfpathlineto{\pgfqpoint{2.787920in}{2.202194in}}%
\pgfpathlineto{\pgfqpoint{2.831264in}{2.202032in}}%
\pgfpathlineto{\pgfqpoint{2.831264in}{2.202032in}}%
\pgfpathlineto{\pgfqpoint{2.881971in}{2.187834in}}%
\pgfpathlineto{\pgfqpoint{2.881971in}{2.187834in}}%
\pgfpathlineto{\pgfqpoint{2.881971in}{2.187834in}}%
\pgfpathlineto{\pgfqpoint{2.927167in}{2.162956in}}%
\pgfusepath{stroke}%
\end{pgfscope}%
\begin{pgfscope}%
\pgfpathrectangle{\pgfqpoint{0.647939in}{0.492442in}}{\pgfqpoint{4.273799in}{2.331163in}}%
\pgfusepath{clip}%
\pgfsetbuttcap%
\pgfsetroundjoin%
\pgfsetlinewidth{0.301125pt}%
\definecolor{currentstroke}{rgb}{0.500000,0.500000,0.500000}%
\pgfsetstrokecolor{currentstroke}%
\pgfsetstrokeopacity{0.300000}%
\pgfsetdash{}{0pt}%
\pgfpathmoveto{\pgfqpoint{2.302325in}{1.878607in}}%
\pgfpathlineto{\pgfqpoint{2.290284in}{1.929985in}}%
\pgfpathlineto{\pgfqpoint{2.281695in}{1.981565in}}%
\pgfpathlineto{\pgfqpoint{2.277236in}{2.033294in}}%
\pgfpathlineto{\pgfqpoint{2.277807in}{2.085071in}}%
\pgfpathlineto{\pgfqpoint{2.284594in}{2.136703in}}%
\pgfpathlineto{\pgfqpoint{2.299180in}{2.187834in}}%
\pgfusepath{stroke}%
\end{pgfscope}%
\begin{pgfscope}%
\pgfpathrectangle{\pgfqpoint{0.647939in}{0.492442in}}{\pgfqpoint{4.273799in}{2.331163in}}%
\pgfusepath{clip}%
\pgfsetbuttcap%
\pgfsetroundjoin%
\pgfsetlinewidth{0.301125pt}%
\definecolor{currentstroke}{rgb}{0.500000,0.500000,0.500000}%
\pgfsetstrokecolor{currentstroke}%
\pgfsetstrokeopacity{0.300000}%
\pgfsetdash{}{0pt}%
\pgfpathmoveto{\pgfqpoint{2.797416in}{1.906638in}}%
\pgfpathlineto{\pgfqpoint{2.776529in}{1.957139in}}%
\pgfpathlineto{\pgfqpoint{2.762654in}{2.008334in}}%
\pgfpathlineto{\pgfqpoint{2.758371in}{2.059975in}}%
\pgfpathlineto{\pgfqpoint{2.764079in}{2.097382in}}%
\pgfpathlineto{\pgfqpoint{2.784839in}{2.134853in}}%
\pgfpathlineto{\pgfqpoint{2.784839in}{2.134853in}}%
\pgfpathlineto{\pgfqpoint{2.784839in}{2.134853in}}%
\pgfpathlineto{\pgfqpoint{2.809025in}{2.153682in}}%
\pgfpathlineto{\pgfqpoint{2.809025in}{2.153682in}}%
\pgfusepath{stroke}%
\end{pgfscope}%
\begin{pgfscope}%
\pgfpathrectangle{\pgfqpoint{0.647939in}{0.492442in}}{\pgfqpoint{4.273799in}{2.331163in}}%
\pgfusepath{clip}%
\pgfsetbuttcap%
\pgfsetroundjoin%
\pgfsetlinewidth{0.301125pt}%
\definecolor{currentstroke}{rgb}{0.500000,0.500000,0.500000}%
\pgfsetstrokecolor{currentstroke}%
\pgfsetstrokeopacity{0.300000}%
\pgfsetdash{}{0pt}%
\pgfpathmoveto{\pgfqpoint{2.131456in}{1.460965in}}%
\pgfpathlineto{\pgfqpoint{2.104762in}{1.510680in}}%
\pgfpathlineto{\pgfqpoint{2.078957in}{1.560533in}}%
\pgfpathlineto{\pgfqpoint{2.054114in}{1.610532in}}%
\pgfpathlineto{\pgfqpoint{2.030343in}{1.660686in}}%
\pgfpathlineto{\pgfqpoint{2.007784in}{1.711005in}}%
\pgfusepath{stroke}%
\end{pgfscope}%
\begin{pgfscope}%
\pgfpathrectangle{\pgfqpoint{0.647939in}{0.492442in}}{\pgfqpoint{4.273799in}{2.331163in}}%
\pgfusepath{clip}%
\pgfsetbuttcap%
\pgfsetroundjoin%
\pgfsetlinewidth{0.301125pt}%
\definecolor{currentstroke}{rgb}{0.500000,0.500000,0.500000}%
\pgfsetstrokecolor{currentstroke}%
\pgfsetstrokeopacity{0.300000}%
\pgfsetdash{}{0pt}%
\pgfpathmoveto{\pgfqpoint{2.687707in}{1.287157in}}%
\pgfpathlineto{\pgfqpoint{2.652588in}{1.335286in}}%
\pgfpathlineto{\pgfqpoint{2.618883in}{1.383715in}}%
\pgfpathlineto{\pgfqpoint{2.586615in}{1.432434in}}%
\pgfpathlineto{\pgfqpoint{2.555813in}{1.481436in}}%
\pgfpathlineto{\pgfqpoint{2.526520in}{1.530711in}}%
\pgfpathlineto{\pgfqpoint{2.498802in}{1.580257in}}%
\pgfpathlineto{\pgfqpoint{2.472745in}{1.630070in}}%
\pgfpathlineto{\pgfqpoint{2.448457in}{1.680147in}}%
\pgfpathlineto{\pgfqpoint{2.426088in}{1.730490in}}%
\pgfusepath{stroke}%
\end{pgfscope}%
\begin{pgfscope}%
\pgfpathrectangle{\pgfqpoint{0.647939in}{0.492442in}}{\pgfqpoint{4.273799in}{2.331163in}}%
\pgfusepath{clip}%
\pgfsetroundcap%
\pgfsetroundjoin%
\pgfsetlinewidth{0.301125pt}%
\definecolor{currentstroke}{rgb}{0.500000,0.500000,0.500000}%
\pgfsetstrokecolor{currentstroke}%
\pgfsetstrokeopacity{0.300000}%
\pgfsetdash{}{0pt}%
\pgfpathmoveto{\pgfqpoint{1.438823in}{1.435600in}}%
\pgfusepath{stroke}%
\end{pgfscope}%
\begin{pgfscope}%
\pgfpathrectangle{\pgfqpoint{0.647939in}{0.492442in}}{\pgfqpoint{4.273799in}{2.331163in}}%
\pgfusepath{clip}%
\pgfsetroundcap%
\pgfsetroundjoin%
\definecolor{currentfill}{rgb}{0.500000,0.500000,0.500000}%
\pgfsetfillcolor{currentfill}%
\pgfsetfillopacity{0.300000}%
\pgfsetlinewidth{0.301125pt}%
\definecolor{currentstroke}{rgb}{0.500000,0.500000,0.500000}%
\pgfsetstrokecolor{currentstroke}%
\pgfsetstrokeopacity{0.300000}%
\pgfsetdash{}{0pt}%
\pgfpathmoveto{\pgfqpoint{0.000000in}{0.000000in}}%
\pgfpathlineto{\pgfqpoint{0.000000in}{0.000000in}}%
\pgfpathclose%
\pgfusepath{stroke,fill}%
\end{pgfscope}%
\begin{pgfscope}%
\pgfpathrectangle{\pgfqpoint{0.647939in}{0.492442in}}{\pgfqpoint{4.273799in}{2.331163in}}%
\pgfusepath{clip}%
\pgfsetroundcap%
\pgfsetroundjoin%
\pgfsetlinewidth{0.301125pt}%
\definecolor{currentstroke}{rgb}{0.500000,0.500000,0.500000}%
\pgfsetstrokecolor{currentstroke}%
\pgfsetstrokeopacity{0.300000}%
\pgfsetdash{}{0pt}%
\pgfpathmoveto{\pgfqpoint{1.249572in}{0.906097in}}%
\pgfusepath{stroke}%
\end{pgfscope}%
\begin{pgfscope}%
\pgfpathrectangle{\pgfqpoint{0.647939in}{0.492442in}}{\pgfqpoint{4.273799in}{2.331163in}}%
\pgfusepath{clip}%
\pgfsetroundcap%
\pgfsetroundjoin%
\definecolor{currentfill}{rgb}{0.500000,0.500000,0.500000}%
\pgfsetfillcolor{currentfill}%
\pgfsetfillopacity{0.300000}%
\pgfsetlinewidth{0.301125pt}%
\definecolor{currentstroke}{rgb}{0.500000,0.500000,0.500000}%
\pgfsetstrokecolor{currentstroke}%
\pgfsetstrokeopacity{0.300000}%
\pgfsetdash{}{0pt}%
\pgfpathmoveto{\pgfqpoint{0.000000in}{0.000000in}}%
\pgfpathlineto{\pgfqpoint{0.000000in}{0.000000in}}%
\pgfpathclose%
\pgfusepath{stroke,fill}%
\end{pgfscope}%
\begin{pgfscope}%
\pgfpathrectangle{\pgfqpoint{0.647939in}{0.492442in}}{\pgfqpoint{4.273799in}{2.331163in}}%
\pgfusepath{clip}%
\pgfsetroundcap%
\pgfsetroundjoin%
\pgfsetlinewidth{0.301125pt}%
\definecolor{currentstroke}{rgb}{0.500000,0.500000,0.500000}%
\pgfsetstrokecolor{currentstroke}%
\pgfsetstrokeopacity{0.300000}%
\pgfsetdash{}{0pt}%
\pgfpathmoveto{\pgfqpoint{1.213805in}{0.689201in}}%
\pgfusepath{stroke}%
\end{pgfscope}%
\begin{pgfscope}%
\pgfpathrectangle{\pgfqpoint{0.647939in}{0.492442in}}{\pgfqpoint{4.273799in}{2.331163in}}%
\pgfusepath{clip}%
\pgfsetroundcap%
\pgfsetroundjoin%
\definecolor{currentfill}{rgb}{0.500000,0.500000,0.500000}%
\pgfsetfillcolor{currentfill}%
\pgfsetfillopacity{0.300000}%
\pgfsetlinewidth{0.301125pt}%
\definecolor{currentstroke}{rgb}{0.500000,0.500000,0.500000}%
\pgfsetstrokecolor{currentstroke}%
\pgfsetstrokeopacity{0.300000}%
\pgfsetdash{}{0pt}%
\pgfpathmoveto{\pgfqpoint{0.000000in}{0.000000in}}%
\pgfpathlineto{\pgfqpoint{0.000000in}{0.000000in}}%
\pgfpathclose%
\pgfusepath{stroke,fill}%
\end{pgfscope}%
\begin{pgfscope}%
\pgfpathrectangle{\pgfqpoint{0.647939in}{0.492442in}}{\pgfqpoint{4.273799in}{2.331163in}}%
\pgfusepath{clip}%
\pgfsetroundcap%
\pgfsetroundjoin%
\pgfsetlinewidth{0.301125pt}%
\definecolor{currentstroke}{rgb}{0.500000,0.500000,0.500000}%
\pgfsetstrokecolor{currentstroke}%
\pgfsetstrokeopacity{0.300000}%
\pgfsetdash{}{0pt}%
\pgfpathmoveto{\pgfqpoint{1.175397in}{0.572555in}}%
\pgfusepath{stroke}%
\end{pgfscope}%
\begin{pgfscope}%
\pgfpathrectangle{\pgfqpoint{0.647939in}{0.492442in}}{\pgfqpoint{4.273799in}{2.331163in}}%
\pgfusepath{clip}%
\pgfsetroundcap%
\pgfsetroundjoin%
\definecolor{currentfill}{rgb}{0.500000,0.500000,0.500000}%
\pgfsetfillcolor{currentfill}%
\pgfsetfillopacity{0.300000}%
\pgfsetlinewidth{0.301125pt}%
\definecolor{currentstroke}{rgb}{0.500000,0.500000,0.500000}%
\pgfsetstrokecolor{currentstroke}%
\pgfsetstrokeopacity{0.300000}%
\pgfsetdash{}{0pt}%
\pgfpathmoveto{\pgfqpoint{0.000000in}{0.000000in}}%
\pgfpathlineto{\pgfqpoint{0.000000in}{0.000000in}}%
\pgfpathclose%
\pgfusepath{stroke,fill}%
\end{pgfscope}%
\begin{pgfscope}%
\pgfpathrectangle{\pgfqpoint{0.647939in}{0.492442in}}{\pgfqpoint{4.273799in}{2.331163in}}%
\pgfusepath{clip}%
\pgfsetroundcap%
\pgfsetroundjoin%
\pgfsetlinewidth{0.301125pt}%
\definecolor{currentstroke}{rgb}{0.500000,0.500000,0.500000}%
\pgfsetstrokecolor{currentstroke}%
\pgfsetstrokeopacity{0.300000}%
\pgfsetdash{}{0pt}%
\pgfpathmoveto{\pgfqpoint{1.387323in}{0.765613in}}%
\pgfusepath{stroke}%
\end{pgfscope}%
\begin{pgfscope}%
\pgfpathrectangle{\pgfqpoint{0.647939in}{0.492442in}}{\pgfqpoint{4.273799in}{2.331163in}}%
\pgfusepath{clip}%
\pgfsetroundcap%
\pgfsetroundjoin%
\definecolor{currentfill}{rgb}{0.500000,0.500000,0.500000}%
\pgfsetfillcolor{currentfill}%
\pgfsetfillopacity{0.300000}%
\pgfsetlinewidth{0.301125pt}%
\definecolor{currentstroke}{rgb}{0.500000,0.500000,0.500000}%
\pgfsetstrokecolor{currentstroke}%
\pgfsetstrokeopacity{0.300000}%
\pgfsetdash{}{0pt}%
\pgfpathmoveto{\pgfqpoint{0.000000in}{0.000000in}}%
\pgfpathlineto{\pgfqpoint{0.000000in}{0.000000in}}%
\pgfpathclose%
\pgfusepath{stroke,fill}%
\end{pgfscope}%
\begin{pgfscope}%
\pgfpathrectangle{\pgfqpoint{0.647939in}{0.492442in}}{\pgfqpoint{4.273799in}{2.331163in}}%
\pgfusepath{clip}%
\pgfsetroundcap%
\pgfsetroundjoin%
\pgfsetlinewidth{0.301125pt}%
\definecolor{currentstroke}{rgb}{0.500000,0.500000,0.500000}%
\pgfsetstrokecolor{currentstroke}%
\pgfsetstrokeopacity{0.300000}%
\pgfsetdash{}{0pt}%
\pgfpathmoveto{\pgfqpoint{1.602504in}{0.837996in}}%
\pgfusepath{stroke}%
\end{pgfscope}%
\begin{pgfscope}%
\pgfpathrectangle{\pgfqpoint{0.647939in}{0.492442in}}{\pgfqpoint{4.273799in}{2.331163in}}%
\pgfusepath{clip}%
\pgfsetroundcap%
\pgfsetroundjoin%
\definecolor{currentfill}{rgb}{0.500000,0.500000,0.500000}%
\pgfsetfillcolor{currentfill}%
\pgfsetfillopacity{0.300000}%
\pgfsetlinewidth{0.301125pt}%
\definecolor{currentstroke}{rgb}{0.500000,0.500000,0.500000}%
\pgfsetstrokecolor{currentstroke}%
\pgfsetstrokeopacity{0.300000}%
\pgfsetdash{}{0pt}%
\pgfpathmoveto{\pgfqpoint{0.000000in}{0.000000in}}%
\pgfpathlineto{\pgfqpoint{0.000000in}{0.000000in}}%
\pgfpathclose%
\pgfusepath{stroke,fill}%
\end{pgfscope}%
\begin{pgfscope}%
\pgfpathrectangle{\pgfqpoint{0.647939in}{0.492442in}}{\pgfqpoint{4.273799in}{2.331163in}}%
\pgfusepath{clip}%
\pgfsetroundcap%
\pgfsetroundjoin%
\pgfsetlinewidth{0.301125pt}%
\definecolor{currentstroke}{rgb}{0.500000,0.500000,0.500000}%
\pgfsetstrokecolor{currentstroke}%
\pgfsetstrokeopacity{0.300000}%
\pgfsetdash{}{0pt}%
\pgfpathmoveto{\pgfqpoint{1.586321in}{0.980155in}}%
\pgfusepath{stroke}%
\end{pgfscope}%
\begin{pgfscope}%
\pgfpathrectangle{\pgfqpoint{0.647939in}{0.492442in}}{\pgfqpoint{4.273799in}{2.331163in}}%
\pgfusepath{clip}%
\pgfsetroundcap%
\pgfsetroundjoin%
\definecolor{currentfill}{rgb}{0.500000,0.500000,0.500000}%
\pgfsetfillcolor{currentfill}%
\pgfsetfillopacity{0.300000}%
\pgfsetlinewidth{0.301125pt}%
\definecolor{currentstroke}{rgb}{0.500000,0.500000,0.500000}%
\pgfsetstrokecolor{currentstroke}%
\pgfsetstrokeopacity{0.300000}%
\pgfsetdash{}{0pt}%
\pgfpathmoveto{\pgfqpoint{0.000000in}{0.000000in}}%
\pgfpathlineto{\pgfqpoint{0.000000in}{0.000000in}}%
\pgfpathclose%
\pgfusepath{stroke,fill}%
\end{pgfscope}%
\begin{pgfscope}%
\pgfpathrectangle{\pgfqpoint{0.647939in}{0.492442in}}{\pgfqpoint{4.273799in}{2.331163in}}%
\pgfusepath{clip}%
\pgfsetroundcap%
\pgfsetroundjoin%
\pgfsetlinewidth{0.301125pt}%
\definecolor{currentstroke}{rgb}{0.500000,0.500000,0.500000}%
\pgfsetstrokecolor{currentstroke}%
\pgfsetstrokeopacity{0.300000}%
\pgfsetdash{}{0pt}%
\pgfpathmoveto{\pgfqpoint{1.669095in}{1.034943in}}%
\pgfusepath{stroke}%
\end{pgfscope}%
\begin{pgfscope}%
\pgfpathrectangle{\pgfqpoint{0.647939in}{0.492442in}}{\pgfqpoint{4.273799in}{2.331163in}}%
\pgfusepath{clip}%
\pgfsetroundcap%
\pgfsetroundjoin%
\definecolor{currentfill}{rgb}{0.500000,0.500000,0.500000}%
\pgfsetfillcolor{currentfill}%
\pgfsetfillopacity{0.300000}%
\pgfsetlinewidth{0.301125pt}%
\definecolor{currentstroke}{rgb}{0.500000,0.500000,0.500000}%
\pgfsetstrokecolor{currentstroke}%
\pgfsetstrokeopacity{0.300000}%
\pgfsetdash{}{0pt}%
\pgfpathmoveto{\pgfqpoint{0.000000in}{0.000000in}}%
\pgfpathlineto{\pgfqpoint{0.000000in}{0.000000in}}%
\pgfpathclose%
\pgfusepath{stroke,fill}%
\end{pgfscope}%
\begin{pgfscope}%
\pgfpathrectangle{\pgfqpoint{0.647939in}{0.492442in}}{\pgfqpoint{4.273799in}{2.331163in}}%
\pgfusepath{clip}%
\pgfsetroundcap%
\pgfsetroundjoin%
\pgfsetlinewidth{0.301125pt}%
\definecolor{currentstroke}{rgb}{0.500000,0.500000,0.500000}%
\pgfsetstrokecolor{currentstroke}%
\pgfsetstrokeopacity{0.300000}%
\pgfsetdash{}{0pt}%
\pgfpathmoveto{\pgfqpoint{1.859940in}{1.238383in}}%
\pgfusepath{stroke}%
\end{pgfscope}%
\begin{pgfscope}%
\pgfpathrectangle{\pgfqpoint{0.647939in}{0.492442in}}{\pgfqpoint{4.273799in}{2.331163in}}%
\pgfusepath{clip}%
\pgfsetroundcap%
\pgfsetroundjoin%
\definecolor{currentfill}{rgb}{0.500000,0.500000,0.500000}%
\pgfsetfillcolor{currentfill}%
\pgfsetfillopacity{0.300000}%
\pgfsetlinewidth{0.301125pt}%
\definecolor{currentstroke}{rgb}{0.500000,0.500000,0.500000}%
\pgfsetstrokecolor{currentstroke}%
\pgfsetstrokeopacity{0.300000}%
\pgfsetdash{}{0pt}%
\pgfpathmoveto{\pgfqpoint{0.000000in}{0.000000in}}%
\pgfpathlineto{\pgfqpoint{0.000000in}{0.000000in}}%
\pgfpathclose%
\pgfusepath{stroke,fill}%
\end{pgfscope}%
\begin{pgfscope}%
\pgfpathrectangle{\pgfqpoint{0.647939in}{0.492442in}}{\pgfqpoint{4.273799in}{2.331163in}}%
\pgfusepath{clip}%
\pgfsetroundcap%
\pgfsetroundjoin%
\pgfsetlinewidth{0.301125pt}%
\definecolor{currentstroke}{rgb}{0.500000,0.500000,0.500000}%
\pgfsetstrokecolor{currentstroke}%
\pgfsetstrokeopacity{0.300000}%
\pgfsetdash{}{0pt}%
\pgfpathmoveto{\pgfqpoint{1.934344in}{1.435650in}}%
\pgfusepath{stroke}%
\end{pgfscope}%
\begin{pgfscope}%
\pgfpathrectangle{\pgfqpoint{0.647939in}{0.492442in}}{\pgfqpoint{4.273799in}{2.331163in}}%
\pgfusepath{clip}%
\pgfsetroundcap%
\pgfsetroundjoin%
\definecolor{currentfill}{rgb}{0.500000,0.500000,0.500000}%
\pgfsetfillcolor{currentfill}%
\pgfsetfillopacity{0.300000}%
\pgfsetlinewidth{0.301125pt}%
\definecolor{currentstroke}{rgb}{0.500000,0.500000,0.500000}%
\pgfsetstrokecolor{currentstroke}%
\pgfsetstrokeopacity{0.300000}%
\pgfsetdash{}{0pt}%
\pgfpathmoveto{\pgfqpoint{0.000000in}{0.000000in}}%
\pgfpathlineto{\pgfqpoint{0.000000in}{0.000000in}}%
\pgfpathclose%
\pgfusepath{stroke,fill}%
\end{pgfscope}%
\begin{pgfscope}%
\pgfpathrectangle{\pgfqpoint{0.647939in}{0.492442in}}{\pgfqpoint{4.273799in}{2.331163in}}%
\pgfusepath{clip}%
\pgfsetroundcap%
\pgfsetroundjoin%
\pgfsetlinewidth{0.301125pt}%
\definecolor{currentstroke}{rgb}{0.500000,0.500000,0.500000}%
\pgfsetstrokecolor{currentstroke}%
\pgfsetstrokeopacity{0.300000}%
\pgfsetdash{}{0pt}%
\pgfpathmoveto{\pgfqpoint{1.755437in}{2.063652in}}%
\pgfusepath{stroke}%
\end{pgfscope}%
\begin{pgfscope}%
\pgfpathrectangle{\pgfqpoint{0.647939in}{0.492442in}}{\pgfqpoint{4.273799in}{2.331163in}}%
\pgfusepath{clip}%
\pgfsetroundcap%
\pgfsetroundjoin%
\definecolor{currentfill}{rgb}{0.500000,0.500000,0.500000}%
\pgfsetfillcolor{currentfill}%
\pgfsetfillopacity{0.300000}%
\pgfsetlinewidth{0.301125pt}%
\definecolor{currentstroke}{rgb}{0.500000,0.500000,0.500000}%
\pgfsetstrokecolor{currentstroke}%
\pgfsetstrokeopacity{0.300000}%
\pgfsetdash{}{0pt}%
\pgfpathmoveto{\pgfqpoint{0.000000in}{0.000000in}}%
\pgfpathlineto{\pgfqpoint{0.000000in}{0.000000in}}%
\pgfpathclose%
\pgfusepath{stroke,fill}%
\end{pgfscope}%
\begin{pgfscope}%
\pgfpathrectangle{\pgfqpoint{0.647939in}{0.492442in}}{\pgfqpoint{4.273799in}{2.331163in}}%
\pgfusepath{clip}%
\pgfsetroundcap%
\pgfsetroundjoin%
\pgfsetlinewidth{0.301125pt}%
\definecolor{currentstroke}{rgb}{0.500000,0.500000,0.500000}%
\pgfsetstrokecolor{currentstroke}%
\pgfsetstrokeopacity{0.300000}%
\pgfsetdash{}{0pt}%
\pgfpathmoveto{\pgfqpoint{2.645365in}{0.654358in}}%
\pgfusepath{stroke}%
\end{pgfscope}%
\begin{pgfscope}%
\pgfpathrectangle{\pgfqpoint{0.647939in}{0.492442in}}{\pgfqpoint{4.273799in}{2.331163in}}%
\pgfusepath{clip}%
\pgfsetroundcap%
\pgfsetroundjoin%
\definecolor{currentfill}{rgb}{0.500000,0.500000,0.500000}%
\pgfsetfillcolor{currentfill}%
\pgfsetfillopacity{0.300000}%
\pgfsetlinewidth{0.301125pt}%
\definecolor{currentstroke}{rgb}{0.500000,0.500000,0.500000}%
\pgfsetstrokecolor{currentstroke}%
\pgfsetstrokeopacity{0.300000}%
\pgfsetdash{}{0pt}%
\pgfpathmoveto{\pgfqpoint{0.000000in}{0.000000in}}%
\pgfpathlineto{\pgfqpoint{0.000000in}{0.000000in}}%
\pgfpathclose%
\pgfusepath{stroke,fill}%
\end{pgfscope}%
\begin{pgfscope}%
\pgfpathrectangle{\pgfqpoint{0.647939in}{0.492442in}}{\pgfqpoint{4.273799in}{2.331163in}}%
\pgfusepath{clip}%
\pgfsetroundcap%
\pgfsetroundjoin%
\pgfsetlinewidth{0.301125pt}%
\definecolor{currentstroke}{rgb}{0.500000,0.500000,0.500000}%
\pgfsetstrokecolor{currentstroke}%
\pgfsetstrokeopacity{0.300000}%
\pgfsetdash{}{0pt}%
\pgfpathmoveto{\pgfqpoint{2.777886in}{0.605704in}}%
\pgfusepath{stroke}%
\end{pgfscope}%
\begin{pgfscope}%
\pgfpathrectangle{\pgfqpoint{0.647939in}{0.492442in}}{\pgfqpoint{4.273799in}{2.331163in}}%
\pgfusepath{clip}%
\pgfsetroundcap%
\pgfsetroundjoin%
\definecolor{currentfill}{rgb}{0.500000,0.500000,0.500000}%
\pgfsetfillcolor{currentfill}%
\pgfsetfillopacity{0.300000}%
\pgfsetlinewidth{0.301125pt}%
\definecolor{currentstroke}{rgb}{0.500000,0.500000,0.500000}%
\pgfsetstrokecolor{currentstroke}%
\pgfsetstrokeopacity{0.300000}%
\pgfsetdash{}{0pt}%
\pgfpathmoveto{\pgfqpoint{0.000000in}{0.000000in}}%
\pgfpathlineto{\pgfqpoint{0.000000in}{0.000000in}}%
\pgfpathclose%
\pgfusepath{stroke,fill}%
\end{pgfscope}%
\begin{pgfscope}%
\pgfpathrectangle{\pgfqpoint{0.647939in}{0.492442in}}{\pgfqpoint{4.273799in}{2.331163in}}%
\pgfusepath{clip}%
\pgfsetroundcap%
\pgfsetroundjoin%
\pgfsetlinewidth{0.301125pt}%
\definecolor{currentstroke}{rgb}{0.500000,0.500000,0.500000}%
\pgfsetstrokecolor{currentstroke}%
\pgfsetstrokeopacity{0.300000}%
\pgfsetdash{}{0pt}%
\pgfpathmoveto{\pgfqpoint{2.308144in}{1.465020in}}%
\pgfusepath{stroke}%
\end{pgfscope}%
\begin{pgfscope}%
\pgfpathrectangle{\pgfqpoint{0.647939in}{0.492442in}}{\pgfqpoint{4.273799in}{2.331163in}}%
\pgfusepath{clip}%
\pgfsetroundcap%
\pgfsetroundjoin%
\definecolor{currentfill}{rgb}{0.500000,0.500000,0.500000}%
\pgfsetfillcolor{currentfill}%
\pgfsetfillopacity{0.300000}%
\pgfsetlinewidth{0.301125pt}%
\definecolor{currentstroke}{rgb}{0.500000,0.500000,0.500000}%
\pgfsetstrokecolor{currentstroke}%
\pgfsetstrokeopacity{0.300000}%
\pgfsetdash{}{0pt}%
\pgfpathmoveto{\pgfqpoint{0.000000in}{0.000000in}}%
\pgfpathlineto{\pgfqpoint{0.000000in}{0.000000in}}%
\pgfpathclose%
\pgfusepath{stroke,fill}%
\end{pgfscope}%
\begin{pgfscope}%
\pgfpathrectangle{\pgfqpoint{0.647939in}{0.492442in}}{\pgfqpoint{4.273799in}{2.331163in}}%
\pgfusepath{clip}%
\pgfsetroundcap%
\pgfsetroundjoin%
\pgfsetlinewidth{0.301125pt}%
\definecolor{currentstroke}{rgb}{0.500000,0.500000,0.500000}%
\pgfsetstrokecolor{currentstroke}%
\pgfsetstrokeopacity{0.300000}%
\pgfsetdash{}{0pt}%
\pgfpathmoveto{\pgfqpoint{2.527002in}{1.299191in}}%
\pgfusepath{stroke}%
\end{pgfscope}%
\begin{pgfscope}%
\pgfpathrectangle{\pgfqpoint{0.647939in}{0.492442in}}{\pgfqpoint{4.273799in}{2.331163in}}%
\pgfusepath{clip}%
\pgfsetroundcap%
\pgfsetroundjoin%
\definecolor{currentfill}{rgb}{0.500000,0.500000,0.500000}%
\pgfsetfillcolor{currentfill}%
\pgfsetfillopacity{0.300000}%
\pgfsetlinewidth{0.301125pt}%
\definecolor{currentstroke}{rgb}{0.500000,0.500000,0.500000}%
\pgfsetstrokecolor{currentstroke}%
\pgfsetstrokeopacity{0.300000}%
\pgfsetdash{}{0pt}%
\pgfpathmoveto{\pgfqpoint{0.000000in}{0.000000in}}%
\pgfpathlineto{\pgfqpoint{0.000000in}{0.000000in}}%
\pgfpathclose%
\pgfusepath{stroke,fill}%
\end{pgfscope}%
\begin{pgfscope}%
\pgfpathrectangle{\pgfqpoint{0.647939in}{0.492442in}}{\pgfqpoint{4.273799in}{2.331163in}}%
\pgfusepath{clip}%
\pgfsetroundcap%
\pgfsetroundjoin%
\pgfsetlinewidth{0.301125pt}%
\definecolor{currentstroke}{rgb}{0.500000,0.500000,0.500000}%
\pgfsetstrokecolor{currentstroke}%
\pgfsetstrokeopacity{0.300000}%
\pgfsetdash{}{0pt}%
\pgfpathmoveto{\pgfqpoint{3.020521in}{0.855112in}}%
\pgfusepath{stroke}%
\end{pgfscope}%
\begin{pgfscope}%
\pgfpathrectangle{\pgfqpoint{0.647939in}{0.492442in}}{\pgfqpoint{4.273799in}{2.331163in}}%
\pgfusepath{clip}%
\pgfsetroundcap%
\pgfsetroundjoin%
\definecolor{currentfill}{rgb}{0.500000,0.500000,0.500000}%
\pgfsetfillcolor{currentfill}%
\pgfsetfillopacity{0.300000}%
\pgfsetlinewidth{0.301125pt}%
\definecolor{currentstroke}{rgb}{0.500000,0.500000,0.500000}%
\pgfsetstrokecolor{currentstroke}%
\pgfsetstrokeopacity{0.300000}%
\pgfsetdash{}{0pt}%
\pgfpathmoveto{\pgfqpoint{0.000000in}{0.000000in}}%
\pgfpathlineto{\pgfqpoint{0.000000in}{0.000000in}}%
\pgfpathclose%
\pgfusepath{stroke,fill}%
\end{pgfscope}%
\begin{pgfscope}%
\pgfpathrectangle{\pgfqpoint{0.647939in}{0.492442in}}{\pgfqpoint{4.273799in}{2.331163in}}%
\pgfusepath{clip}%
\pgfsetroundcap%
\pgfsetroundjoin%
\pgfsetlinewidth{0.301125pt}%
\definecolor{currentstroke}{rgb}{0.500000,0.500000,0.500000}%
\pgfsetstrokecolor{currentstroke}%
\pgfsetstrokeopacity{0.300000}%
\pgfsetdash{}{0pt}%
\pgfpathmoveto{\pgfqpoint{3.230697in}{0.796495in}}%
\pgfusepath{stroke}%
\end{pgfscope}%
\begin{pgfscope}%
\pgfpathrectangle{\pgfqpoint{0.647939in}{0.492442in}}{\pgfqpoint{4.273799in}{2.331163in}}%
\pgfusepath{clip}%
\pgfsetroundcap%
\pgfsetroundjoin%
\definecolor{currentfill}{rgb}{0.500000,0.500000,0.500000}%
\pgfsetfillcolor{currentfill}%
\pgfsetfillopacity{0.300000}%
\pgfsetlinewidth{0.301125pt}%
\definecolor{currentstroke}{rgb}{0.500000,0.500000,0.500000}%
\pgfsetstrokecolor{currentstroke}%
\pgfsetstrokeopacity{0.300000}%
\pgfsetdash{}{0pt}%
\pgfpathmoveto{\pgfqpoint{0.000000in}{0.000000in}}%
\pgfpathlineto{\pgfqpoint{0.000000in}{0.000000in}}%
\pgfpathclose%
\pgfusepath{stroke,fill}%
\end{pgfscope}%
\begin{pgfscope}%
\pgfpathrectangle{\pgfqpoint{0.647939in}{0.492442in}}{\pgfqpoint{4.273799in}{2.331163in}}%
\pgfusepath{clip}%
\pgfsetroundcap%
\pgfsetroundjoin%
\pgfsetlinewidth{0.301125pt}%
\definecolor{currentstroke}{rgb}{0.500000,0.500000,0.500000}%
\pgfsetstrokecolor{currentstroke}%
\pgfsetstrokeopacity{0.300000}%
\pgfsetdash{}{0pt}%
\pgfpathmoveto{\pgfqpoint{2.730873in}{1.413085in}}%
\pgfusepath{stroke}%
\end{pgfscope}%
\begin{pgfscope}%
\pgfpathrectangle{\pgfqpoint{0.647939in}{0.492442in}}{\pgfqpoint{4.273799in}{2.331163in}}%
\pgfusepath{clip}%
\pgfsetroundcap%
\pgfsetroundjoin%
\definecolor{currentfill}{rgb}{0.500000,0.500000,0.500000}%
\pgfsetfillcolor{currentfill}%
\pgfsetfillopacity{0.300000}%
\pgfsetlinewidth{0.301125pt}%
\definecolor{currentstroke}{rgb}{0.500000,0.500000,0.500000}%
\pgfsetstrokecolor{currentstroke}%
\pgfsetstrokeopacity{0.300000}%
\pgfsetdash{}{0pt}%
\pgfpathmoveto{\pgfqpoint{0.000000in}{0.000000in}}%
\pgfpathlineto{\pgfqpoint{0.000000in}{0.000000in}}%
\pgfpathclose%
\pgfusepath{stroke,fill}%
\end{pgfscope}%
\begin{pgfscope}%
\pgfpathrectangle{\pgfqpoint{0.647939in}{0.492442in}}{\pgfqpoint{4.273799in}{2.331163in}}%
\pgfusepath{clip}%
\pgfsetroundcap%
\pgfsetroundjoin%
\pgfsetlinewidth{0.301125pt}%
\definecolor{currentstroke}{rgb}{0.500000,0.500000,0.500000}%
\pgfsetstrokecolor{currentstroke}%
\pgfsetstrokeopacity{0.300000}%
\pgfsetdash{}{0pt}%
\pgfpathmoveto{\pgfqpoint{3.806652in}{0.587524in}}%
\pgfusepath{stroke}%
\end{pgfscope}%
\begin{pgfscope}%
\pgfpathrectangle{\pgfqpoint{0.647939in}{0.492442in}}{\pgfqpoint{4.273799in}{2.331163in}}%
\pgfusepath{clip}%
\pgfsetroundcap%
\pgfsetroundjoin%
\definecolor{currentfill}{rgb}{0.500000,0.500000,0.500000}%
\pgfsetfillcolor{currentfill}%
\pgfsetfillopacity{0.300000}%
\pgfsetlinewidth{0.301125pt}%
\definecolor{currentstroke}{rgb}{0.500000,0.500000,0.500000}%
\pgfsetstrokecolor{currentstroke}%
\pgfsetstrokeopacity{0.300000}%
\pgfsetdash{}{0pt}%
\pgfpathmoveto{\pgfqpoint{0.000000in}{0.000000in}}%
\pgfpathlineto{\pgfqpoint{0.000000in}{0.000000in}}%
\pgfpathclose%
\pgfusepath{stroke,fill}%
\end{pgfscope}%
\begin{pgfscope}%
\pgfpathrectangle{\pgfqpoint{0.647939in}{0.492442in}}{\pgfqpoint{4.273799in}{2.331163in}}%
\pgfusepath{clip}%
\pgfsetroundcap%
\pgfsetroundjoin%
\pgfsetlinewidth{0.301125pt}%
\definecolor{currentstroke}{rgb}{0.500000,0.500000,0.500000}%
\pgfsetstrokecolor{currentstroke}%
\pgfsetstrokeopacity{0.300000}%
\pgfsetdash{}{0pt}%
\pgfpathmoveto{\pgfqpoint{3.095454in}{1.187877in}}%
\pgfusepath{stroke}%
\end{pgfscope}%
\begin{pgfscope}%
\pgfpathrectangle{\pgfqpoint{0.647939in}{0.492442in}}{\pgfqpoint{4.273799in}{2.331163in}}%
\pgfusepath{clip}%
\pgfsetroundcap%
\pgfsetroundjoin%
\definecolor{currentfill}{rgb}{0.500000,0.500000,0.500000}%
\pgfsetfillcolor{currentfill}%
\pgfsetfillopacity{0.300000}%
\pgfsetlinewidth{0.301125pt}%
\definecolor{currentstroke}{rgb}{0.500000,0.500000,0.500000}%
\pgfsetstrokecolor{currentstroke}%
\pgfsetstrokeopacity{0.300000}%
\pgfsetdash{}{0pt}%
\pgfpathmoveto{\pgfqpoint{0.000000in}{0.000000in}}%
\pgfpathlineto{\pgfqpoint{0.000000in}{0.000000in}}%
\pgfpathclose%
\pgfusepath{stroke,fill}%
\end{pgfscope}%
\begin{pgfscope}%
\pgfpathrectangle{\pgfqpoint{0.647939in}{0.492442in}}{\pgfqpoint{4.273799in}{2.331163in}}%
\pgfusepath{clip}%
\pgfsetroundcap%
\pgfsetroundjoin%
\pgfsetlinewidth{0.301125pt}%
\definecolor{currentstroke}{rgb}{0.500000,0.500000,0.500000}%
\pgfsetstrokecolor{currentstroke}%
\pgfsetstrokeopacity{0.300000}%
\pgfsetdash{}{0pt}%
\pgfpathmoveto{\pgfqpoint{3.404858in}{1.098935in}}%
\pgfusepath{stroke}%
\end{pgfscope}%
\begin{pgfscope}%
\pgfpathrectangle{\pgfqpoint{0.647939in}{0.492442in}}{\pgfqpoint{4.273799in}{2.331163in}}%
\pgfusepath{clip}%
\pgfsetroundcap%
\pgfsetroundjoin%
\definecolor{currentfill}{rgb}{0.500000,0.500000,0.500000}%
\pgfsetfillcolor{currentfill}%
\pgfsetfillopacity{0.300000}%
\pgfsetlinewidth{0.301125pt}%
\definecolor{currentstroke}{rgb}{0.500000,0.500000,0.500000}%
\pgfsetstrokecolor{currentstroke}%
\pgfsetstrokeopacity{0.300000}%
\pgfsetdash{}{0pt}%
\pgfpathmoveto{\pgfqpoint{0.000000in}{0.000000in}}%
\pgfpathlineto{\pgfqpoint{0.000000in}{0.000000in}}%
\pgfpathclose%
\pgfusepath{stroke,fill}%
\end{pgfscope}%
\begin{pgfscope}%
\pgfpathrectangle{\pgfqpoint{0.647939in}{0.492442in}}{\pgfqpoint{4.273799in}{2.331163in}}%
\pgfusepath{clip}%
\pgfsetroundcap%
\pgfsetroundjoin%
\pgfsetlinewidth{0.301125pt}%
\definecolor{currentstroke}{rgb}{0.500000,0.500000,0.500000}%
\pgfsetstrokecolor{currentstroke}%
\pgfsetstrokeopacity{0.300000}%
\pgfsetdash{}{0pt}%
\pgfpathmoveto{\pgfqpoint{3.620814in}{1.161141in}}%
\pgfusepath{stroke}%
\end{pgfscope}%
\begin{pgfscope}%
\pgfpathrectangle{\pgfqpoint{0.647939in}{0.492442in}}{\pgfqpoint{4.273799in}{2.331163in}}%
\pgfusepath{clip}%
\pgfsetroundcap%
\pgfsetroundjoin%
\definecolor{currentfill}{rgb}{0.500000,0.500000,0.500000}%
\pgfsetfillcolor{currentfill}%
\pgfsetfillopacity{0.300000}%
\pgfsetlinewidth{0.301125pt}%
\definecolor{currentstroke}{rgb}{0.500000,0.500000,0.500000}%
\pgfsetstrokecolor{currentstroke}%
\pgfsetstrokeopacity{0.300000}%
\pgfsetdash{}{0pt}%
\pgfpathmoveto{\pgfqpoint{0.000000in}{0.000000in}}%
\pgfpathlineto{\pgfqpoint{0.000000in}{0.000000in}}%
\pgfpathclose%
\pgfusepath{stroke,fill}%
\end{pgfscope}%
\begin{pgfscope}%
\pgfpathrectangle{\pgfqpoint{0.647939in}{0.492442in}}{\pgfqpoint{4.273799in}{2.331163in}}%
\pgfusepath{clip}%
\pgfsetroundcap%
\pgfsetroundjoin%
\pgfsetlinewidth{0.301125pt}%
\definecolor{currentstroke}{rgb}{0.500000,0.500000,0.500000}%
\pgfsetstrokecolor{currentstroke}%
\pgfsetstrokeopacity{0.300000}%
\pgfsetdash{}{0pt}%
\pgfpathmoveto{\pgfqpoint{4.031569in}{1.096839in}}%
\pgfusepath{stroke}%
\end{pgfscope}%
\begin{pgfscope}%
\pgfpathrectangle{\pgfqpoint{0.647939in}{0.492442in}}{\pgfqpoint{4.273799in}{2.331163in}}%
\pgfusepath{clip}%
\pgfsetroundcap%
\pgfsetroundjoin%
\definecolor{currentfill}{rgb}{0.500000,0.500000,0.500000}%
\pgfsetfillcolor{currentfill}%
\pgfsetfillopacity{0.300000}%
\pgfsetlinewidth{0.301125pt}%
\definecolor{currentstroke}{rgb}{0.500000,0.500000,0.500000}%
\pgfsetstrokecolor{currentstroke}%
\pgfsetstrokeopacity{0.300000}%
\pgfsetdash{}{0pt}%
\pgfpathmoveto{\pgfqpoint{0.000000in}{0.000000in}}%
\pgfpathlineto{\pgfqpoint{0.000000in}{0.000000in}}%
\pgfpathclose%
\pgfusepath{stroke,fill}%
\end{pgfscope}%
\begin{pgfscope}%
\pgfpathrectangle{\pgfqpoint{0.647939in}{0.492442in}}{\pgfqpoint{4.273799in}{2.331163in}}%
\pgfusepath{clip}%
\pgfsetroundcap%
\pgfsetroundjoin%
\pgfsetlinewidth{0.301125pt}%
\definecolor{currentstroke}{rgb}{0.500000,0.500000,0.500000}%
\pgfsetstrokecolor{currentstroke}%
\pgfsetstrokeopacity{0.300000}%
\pgfsetdash{}{0pt}%
\pgfpathmoveto{\pgfqpoint{3.902581in}{1.306217in}}%
\pgfusepath{stroke}%
\end{pgfscope}%
\begin{pgfscope}%
\pgfpathrectangle{\pgfqpoint{0.647939in}{0.492442in}}{\pgfqpoint{4.273799in}{2.331163in}}%
\pgfusepath{clip}%
\pgfsetroundcap%
\pgfsetroundjoin%
\definecolor{currentfill}{rgb}{0.500000,0.500000,0.500000}%
\pgfsetfillcolor{currentfill}%
\pgfsetfillopacity{0.300000}%
\pgfsetlinewidth{0.301125pt}%
\definecolor{currentstroke}{rgb}{0.500000,0.500000,0.500000}%
\pgfsetstrokecolor{currentstroke}%
\pgfsetstrokeopacity{0.300000}%
\pgfsetdash{}{0pt}%
\pgfpathmoveto{\pgfqpoint{0.000000in}{0.000000in}}%
\pgfpathlineto{\pgfqpoint{0.000000in}{0.000000in}}%
\pgfpathclose%
\pgfusepath{stroke,fill}%
\end{pgfscope}%
\begin{pgfscope}%
\pgfpathrectangle{\pgfqpoint{0.647939in}{0.492442in}}{\pgfqpoint{4.273799in}{2.331163in}}%
\pgfusepath{clip}%
\pgfsetroundcap%
\pgfsetroundjoin%
\pgfsetlinewidth{0.301125pt}%
\definecolor{currentstroke}{rgb}{0.500000,0.500000,0.500000}%
\pgfsetstrokecolor{currentstroke}%
\pgfsetstrokeopacity{0.300000}%
\pgfsetdash{}{0pt}%
\pgfpathmoveto{\pgfqpoint{4.189490in}{1.372720in}}%
\pgfusepath{stroke}%
\end{pgfscope}%
\begin{pgfscope}%
\pgfpathrectangle{\pgfqpoint{0.647939in}{0.492442in}}{\pgfqpoint{4.273799in}{2.331163in}}%
\pgfusepath{clip}%
\pgfsetroundcap%
\pgfsetroundjoin%
\definecolor{currentfill}{rgb}{0.500000,0.500000,0.500000}%
\pgfsetfillcolor{currentfill}%
\pgfsetfillopacity{0.300000}%
\pgfsetlinewidth{0.301125pt}%
\definecolor{currentstroke}{rgb}{0.500000,0.500000,0.500000}%
\pgfsetstrokecolor{currentstroke}%
\pgfsetstrokeopacity{0.300000}%
\pgfsetdash{}{0pt}%
\pgfpathmoveto{\pgfqpoint{0.000000in}{0.000000in}}%
\pgfpathlineto{\pgfqpoint{0.000000in}{0.000000in}}%
\pgfpathclose%
\pgfusepath{stroke,fill}%
\end{pgfscope}%
\begin{pgfscope}%
\pgfpathrectangle{\pgfqpoint{0.647939in}{0.492442in}}{\pgfqpoint{4.273799in}{2.331163in}}%
\pgfusepath{clip}%
\pgfsetroundcap%
\pgfsetroundjoin%
\pgfsetlinewidth{0.301125pt}%
\definecolor{currentstroke}{rgb}{0.500000,0.500000,0.500000}%
\pgfsetstrokecolor{currentstroke}%
\pgfsetstrokeopacity{0.300000}%
\pgfsetdash{}{0pt}%
\pgfpathmoveto{\pgfqpoint{4.372402in}{1.506828in}}%
\pgfusepath{stroke}%
\end{pgfscope}%
\begin{pgfscope}%
\pgfpathrectangle{\pgfqpoint{0.647939in}{0.492442in}}{\pgfqpoint{4.273799in}{2.331163in}}%
\pgfusepath{clip}%
\pgfsetroundcap%
\pgfsetroundjoin%
\definecolor{currentfill}{rgb}{0.500000,0.500000,0.500000}%
\pgfsetfillcolor{currentfill}%
\pgfsetfillopacity{0.300000}%
\pgfsetlinewidth{0.301125pt}%
\definecolor{currentstroke}{rgb}{0.500000,0.500000,0.500000}%
\pgfsetstrokecolor{currentstroke}%
\pgfsetstrokeopacity{0.300000}%
\pgfsetdash{}{0pt}%
\pgfpathmoveto{\pgfqpoint{0.000000in}{0.000000in}}%
\pgfpathlineto{\pgfqpoint{0.000000in}{0.000000in}}%
\pgfpathclose%
\pgfusepath{stroke,fill}%
\end{pgfscope}%
\begin{pgfscope}%
\pgfpathrectangle{\pgfqpoint{0.647939in}{0.492442in}}{\pgfqpoint{4.273799in}{2.331163in}}%
\pgfusepath{clip}%
\pgfsetroundcap%
\pgfsetroundjoin%
\pgfsetlinewidth{0.301125pt}%
\definecolor{currentstroke}{rgb}{0.500000,0.500000,0.500000}%
\pgfsetstrokecolor{currentstroke}%
\pgfsetstrokeopacity{0.300000}%
\pgfsetdash{}{0pt}%
\pgfpathmoveto{\pgfqpoint{4.512575in}{1.662892in}}%
\pgfusepath{stroke}%
\end{pgfscope}%
\begin{pgfscope}%
\pgfpathrectangle{\pgfqpoint{0.647939in}{0.492442in}}{\pgfqpoint{4.273799in}{2.331163in}}%
\pgfusepath{clip}%
\pgfsetroundcap%
\pgfsetroundjoin%
\definecolor{currentfill}{rgb}{0.500000,0.500000,0.500000}%
\pgfsetfillcolor{currentfill}%
\pgfsetfillopacity{0.300000}%
\pgfsetlinewidth{0.301125pt}%
\definecolor{currentstroke}{rgb}{0.500000,0.500000,0.500000}%
\pgfsetstrokecolor{currentstroke}%
\pgfsetstrokeopacity{0.300000}%
\pgfsetdash{}{0pt}%
\pgfpathmoveto{\pgfqpoint{0.000000in}{0.000000in}}%
\pgfpathlineto{\pgfqpoint{0.000000in}{0.000000in}}%
\pgfpathclose%
\pgfusepath{stroke,fill}%
\end{pgfscope}%
\begin{pgfscope}%
\pgfpathrectangle{\pgfqpoint{0.647939in}{0.492442in}}{\pgfqpoint{4.273799in}{2.331163in}}%
\pgfusepath{clip}%
\pgfsetroundcap%
\pgfsetroundjoin%
\pgfsetlinewidth{0.301125pt}%
\definecolor{currentstroke}{rgb}{0.500000,0.500000,0.500000}%
\pgfsetstrokecolor{currentstroke}%
\pgfsetstrokeopacity{0.300000}%
\pgfsetdash{}{0pt}%
\pgfpathmoveto{\pgfqpoint{4.649820in}{1.889514in}}%
\pgfusepath{stroke}%
\end{pgfscope}%
\begin{pgfscope}%
\pgfpathrectangle{\pgfqpoint{0.647939in}{0.492442in}}{\pgfqpoint{4.273799in}{2.331163in}}%
\pgfusepath{clip}%
\pgfsetroundcap%
\pgfsetroundjoin%
\definecolor{currentfill}{rgb}{0.500000,0.500000,0.500000}%
\pgfsetfillcolor{currentfill}%
\pgfsetfillopacity{0.300000}%
\pgfsetlinewidth{0.301125pt}%
\definecolor{currentstroke}{rgb}{0.500000,0.500000,0.500000}%
\pgfsetstrokecolor{currentstroke}%
\pgfsetstrokeopacity{0.300000}%
\pgfsetdash{}{0pt}%
\pgfpathmoveto{\pgfqpoint{0.000000in}{0.000000in}}%
\pgfpathlineto{\pgfqpoint{0.000000in}{0.000000in}}%
\pgfpathclose%
\pgfusepath{stroke,fill}%
\end{pgfscope}%
\begin{pgfscope}%
\pgfpathrectangle{\pgfqpoint{0.647939in}{0.492442in}}{\pgfqpoint{4.273799in}{2.331163in}}%
\pgfusepath{clip}%
\pgfsetroundcap%
\pgfsetroundjoin%
\pgfsetlinewidth{0.301125pt}%
\definecolor{currentstroke}{rgb}{0.500000,0.500000,0.500000}%
\pgfsetstrokecolor{currentstroke}%
\pgfsetstrokeopacity{0.300000}%
\pgfsetdash{}{0pt}%
\pgfpathmoveto{\pgfqpoint{4.773839in}{1.954522in}}%
\pgfusepath{stroke}%
\end{pgfscope}%
\begin{pgfscope}%
\pgfpathrectangle{\pgfqpoint{0.647939in}{0.492442in}}{\pgfqpoint{4.273799in}{2.331163in}}%
\pgfusepath{clip}%
\pgfsetroundcap%
\pgfsetroundjoin%
\definecolor{currentfill}{rgb}{0.500000,0.500000,0.500000}%
\pgfsetfillcolor{currentfill}%
\pgfsetfillopacity{0.300000}%
\pgfsetlinewidth{0.301125pt}%
\definecolor{currentstroke}{rgb}{0.500000,0.500000,0.500000}%
\pgfsetstrokecolor{currentstroke}%
\pgfsetstrokeopacity{0.300000}%
\pgfsetdash{}{0pt}%
\pgfpathmoveto{\pgfqpoint{0.000000in}{0.000000in}}%
\pgfpathlineto{\pgfqpoint{0.000000in}{0.000000in}}%
\pgfpathclose%
\pgfusepath{stroke,fill}%
\end{pgfscope}%
\begin{pgfscope}%
\pgfpathrectangle{\pgfqpoint{0.647939in}{0.492442in}}{\pgfqpoint{4.273799in}{2.331163in}}%
\pgfusepath{clip}%
\pgfsetroundcap%
\pgfsetroundjoin%
\pgfsetlinewidth{0.301125pt}%
\definecolor{currentstroke}{rgb}{0.500000,0.500000,0.500000}%
\pgfsetstrokecolor{currentstroke}%
\pgfsetstrokeopacity{0.300000}%
\pgfsetdash{}{0pt}%
\pgfpathmoveto{\pgfqpoint{4.901285in}{1.602084in}}%
\pgfusepath{stroke}%
\end{pgfscope}%
\begin{pgfscope}%
\pgfpathrectangle{\pgfqpoint{0.647939in}{0.492442in}}{\pgfqpoint{4.273799in}{2.331163in}}%
\pgfusepath{clip}%
\pgfsetroundcap%
\pgfsetroundjoin%
\definecolor{currentfill}{rgb}{0.500000,0.500000,0.500000}%
\pgfsetfillcolor{currentfill}%
\pgfsetfillopacity{0.300000}%
\pgfsetlinewidth{0.301125pt}%
\definecolor{currentstroke}{rgb}{0.500000,0.500000,0.500000}%
\pgfsetstrokecolor{currentstroke}%
\pgfsetstrokeopacity{0.300000}%
\pgfsetdash{}{0pt}%
\pgfpathmoveto{\pgfqpoint{0.000000in}{0.000000in}}%
\pgfpathlineto{\pgfqpoint{0.000000in}{0.000000in}}%
\pgfpathclose%
\pgfusepath{stroke,fill}%
\end{pgfscope}%
\begin{pgfscope}%
\pgfpathrectangle{\pgfqpoint{0.647939in}{0.492442in}}{\pgfqpoint{4.273799in}{2.331163in}}%
\pgfusepath{clip}%
\pgfsetroundcap%
\pgfsetroundjoin%
\pgfsetlinewidth{0.301125pt}%
\definecolor{currentstroke}{rgb}{0.500000,0.500000,0.500000}%
\pgfsetstrokecolor{currentstroke}%
\pgfsetstrokeopacity{0.300000}%
\pgfsetdash{}{0pt}%
\pgfpathmoveto{\pgfqpoint{4.909164in}{2.024045in}}%
\pgfusepath{stroke}%
\end{pgfscope}%
\begin{pgfscope}%
\pgfpathrectangle{\pgfqpoint{0.647939in}{0.492442in}}{\pgfqpoint{4.273799in}{2.331163in}}%
\pgfusepath{clip}%
\pgfsetroundcap%
\pgfsetroundjoin%
\definecolor{currentfill}{rgb}{0.500000,0.500000,0.500000}%
\pgfsetfillcolor{currentfill}%
\pgfsetfillopacity{0.300000}%
\pgfsetlinewidth{0.301125pt}%
\definecolor{currentstroke}{rgb}{0.500000,0.500000,0.500000}%
\pgfsetstrokecolor{currentstroke}%
\pgfsetstrokeopacity{0.300000}%
\pgfsetdash{}{0pt}%
\pgfpathmoveto{\pgfqpoint{0.000000in}{0.000000in}}%
\pgfpathlineto{\pgfqpoint{0.000000in}{0.000000in}}%
\pgfpathclose%
\pgfusepath{stroke,fill}%
\end{pgfscope}%
\begin{pgfscope}%
\pgfpathrectangle{\pgfqpoint{0.647939in}{0.492442in}}{\pgfqpoint{4.273799in}{2.331163in}}%
\pgfusepath{clip}%
\pgfsetroundcap%
\pgfsetroundjoin%
\pgfsetlinewidth{0.301125pt}%
\definecolor{currentstroke}{rgb}{0.500000,0.500000,0.500000}%
\pgfsetstrokecolor{currentstroke}%
\pgfsetstrokeopacity{0.300000}%
\pgfsetdash{}{0pt}%
\pgfpathmoveto{\pgfqpoint{4.522286in}{2.676356in}}%
\pgfusepath{stroke}%
\end{pgfscope}%
\begin{pgfscope}%
\pgfpathrectangle{\pgfqpoint{0.647939in}{0.492442in}}{\pgfqpoint{4.273799in}{2.331163in}}%
\pgfusepath{clip}%
\pgfsetroundcap%
\pgfsetroundjoin%
\definecolor{currentfill}{rgb}{0.500000,0.500000,0.500000}%
\pgfsetfillcolor{currentfill}%
\pgfsetfillopacity{0.300000}%
\pgfsetlinewidth{0.301125pt}%
\definecolor{currentstroke}{rgb}{0.500000,0.500000,0.500000}%
\pgfsetstrokecolor{currentstroke}%
\pgfsetstrokeopacity{0.300000}%
\pgfsetdash{}{0pt}%
\pgfpathmoveto{\pgfqpoint{0.000000in}{0.000000in}}%
\pgfpathlineto{\pgfqpoint{0.000000in}{0.000000in}}%
\pgfpathclose%
\pgfusepath{stroke,fill}%
\end{pgfscope}%
\begin{pgfscope}%
\pgfpathrectangle{\pgfqpoint{0.647939in}{0.492442in}}{\pgfqpoint{4.273799in}{2.331163in}}%
\pgfusepath{clip}%
\pgfsetroundcap%
\pgfsetroundjoin%
\pgfsetlinewidth{0.301125pt}%
\definecolor{currentstroke}{rgb}{0.500000,0.500000,0.500000}%
\pgfsetstrokecolor{currentstroke}%
\pgfsetstrokeopacity{0.300000}%
\pgfsetdash{}{0pt}%
\pgfpathmoveto{\pgfqpoint{4.409399in}{2.528711in}}%
\pgfusepath{stroke}%
\end{pgfscope}%
\begin{pgfscope}%
\pgfpathrectangle{\pgfqpoint{0.647939in}{0.492442in}}{\pgfqpoint{4.273799in}{2.331163in}}%
\pgfusepath{clip}%
\pgfsetroundcap%
\pgfsetroundjoin%
\definecolor{currentfill}{rgb}{0.500000,0.500000,0.500000}%
\pgfsetfillcolor{currentfill}%
\pgfsetfillopacity{0.300000}%
\pgfsetlinewidth{0.301125pt}%
\definecolor{currentstroke}{rgb}{0.500000,0.500000,0.500000}%
\pgfsetstrokecolor{currentstroke}%
\pgfsetstrokeopacity{0.300000}%
\pgfsetdash{}{0pt}%
\pgfpathmoveto{\pgfqpoint{0.000000in}{0.000000in}}%
\pgfpathlineto{\pgfqpoint{0.000000in}{0.000000in}}%
\pgfpathclose%
\pgfusepath{stroke,fill}%
\end{pgfscope}%
\begin{pgfscope}%
\pgfpathrectangle{\pgfqpoint{0.647939in}{0.492442in}}{\pgfqpoint{4.273799in}{2.331163in}}%
\pgfusepath{clip}%
\pgfsetroundcap%
\pgfsetroundjoin%
\pgfsetlinewidth{0.301125pt}%
\definecolor{currentstroke}{rgb}{0.500000,0.500000,0.500000}%
\pgfsetstrokecolor{currentstroke}%
\pgfsetstrokeopacity{0.300000}%
\pgfsetdash{}{0pt}%
\pgfpathmoveto{\pgfqpoint{4.262148in}{2.510201in}}%
\pgfusepath{stroke}%
\end{pgfscope}%
\begin{pgfscope}%
\pgfpathrectangle{\pgfqpoint{0.647939in}{0.492442in}}{\pgfqpoint{4.273799in}{2.331163in}}%
\pgfusepath{clip}%
\pgfsetroundcap%
\pgfsetroundjoin%
\definecolor{currentfill}{rgb}{0.500000,0.500000,0.500000}%
\pgfsetfillcolor{currentfill}%
\pgfsetfillopacity{0.300000}%
\pgfsetlinewidth{0.301125pt}%
\definecolor{currentstroke}{rgb}{0.500000,0.500000,0.500000}%
\pgfsetstrokecolor{currentstroke}%
\pgfsetstrokeopacity{0.300000}%
\pgfsetdash{}{0pt}%
\pgfpathmoveto{\pgfqpoint{0.000000in}{0.000000in}}%
\pgfpathlineto{\pgfqpoint{0.000000in}{0.000000in}}%
\pgfpathclose%
\pgfusepath{stroke,fill}%
\end{pgfscope}%
\begin{pgfscope}%
\pgfpathrectangle{\pgfqpoint{0.647939in}{0.492442in}}{\pgfqpoint{4.273799in}{2.331163in}}%
\pgfusepath{clip}%
\pgfsetroundcap%
\pgfsetroundjoin%
\pgfsetlinewidth{0.301125pt}%
\definecolor{currentstroke}{rgb}{0.500000,0.500000,0.500000}%
\pgfsetstrokecolor{currentstroke}%
\pgfsetstrokeopacity{0.300000}%
\pgfsetdash{}{0pt}%
\pgfpathmoveto{\pgfqpoint{4.200065in}{2.405751in}}%
\pgfusepath{stroke}%
\end{pgfscope}%
\begin{pgfscope}%
\pgfpathrectangle{\pgfqpoint{0.647939in}{0.492442in}}{\pgfqpoint{4.273799in}{2.331163in}}%
\pgfusepath{clip}%
\pgfsetroundcap%
\pgfsetroundjoin%
\definecolor{currentfill}{rgb}{0.500000,0.500000,0.500000}%
\pgfsetfillcolor{currentfill}%
\pgfsetfillopacity{0.300000}%
\pgfsetlinewidth{0.301125pt}%
\definecolor{currentstroke}{rgb}{0.500000,0.500000,0.500000}%
\pgfsetstrokecolor{currentstroke}%
\pgfsetstrokeopacity{0.300000}%
\pgfsetdash{}{0pt}%
\pgfpathmoveto{\pgfqpoint{0.000000in}{0.000000in}}%
\pgfpathlineto{\pgfqpoint{0.000000in}{0.000000in}}%
\pgfpathclose%
\pgfusepath{stroke,fill}%
\end{pgfscope}%
\begin{pgfscope}%
\pgfpathrectangle{\pgfqpoint{0.647939in}{0.492442in}}{\pgfqpoint{4.273799in}{2.331163in}}%
\pgfusepath{clip}%
\pgfsetroundcap%
\pgfsetroundjoin%
\pgfsetlinewidth{0.301125pt}%
\definecolor{currentstroke}{rgb}{0.500000,0.500000,0.500000}%
\pgfsetstrokecolor{currentstroke}%
\pgfsetstrokeopacity{0.300000}%
\pgfsetdash{}{0pt}%
\pgfpathmoveto{\pgfqpoint{4.069741in}{2.426406in}}%
\pgfusepath{stroke}%
\end{pgfscope}%
\begin{pgfscope}%
\pgfpathrectangle{\pgfqpoint{0.647939in}{0.492442in}}{\pgfqpoint{4.273799in}{2.331163in}}%
\pgfusepath{clip}%
\pgfsetroundcap%
\pgfsetroundjoin%
\definecolor{currentfill}{rgb}{0.500000,0.500000,0.500000}%
\pgfsetfillcolor{currentfill}%
\pgfsetfillopacity{0.300000}%
\pgfsetlinewidth{0.301125pt}%
\definecolor{currentstroke}{rgb}{0.500000,0.500000,0.500000}%
\pgfsetstrokecolor{currentstroke}%
\pgfsetstrokeopacity{0.300000}%
\pgfsetdash{}{0pt}%
\pgfpathmoveto{\pgfqpoint{0.000000in}{0.000000in}}%
\pgfpathlineto{\pgfqpoint{0.000000in}{0.000000in}}%
\pgfpathclose%
\pgfusepath{stroke,fill}%
\end{pgfscope}%
\begin{pgfscope}%
\pgfpathrectangle{\pgfqpoint{0.647939in}{0.492442in}}{\pgfqpoint{4.273799in}{2.331163in}}%
\pgfusepath{clip}%
\pgfsetroundcap%
\pgfsetroundjoin%
\pgfsetlinewidth{0.301125pt}%
\definecolor{currentstroke}{rgb}{0.500000,0.500000,0.500000}%
\pgfsetstrokecolor{currentstroke}%
\pgfsetstrokeopacity{0.300000}%
\pgfsetdash{}{0pt}%
\pgfpathmoveto{\pgfqpoint{3.984921in}{2.374547in}}%
\pgfusepath{stroke}%
\end{pgfscope}%
\begin{pgfscope}%
\pgfpathrectangle{\pgfqpoint{0.647939in}{0.492442in}}{\pgfqpoint{4.273799in}{2.331163in}}%
\pgfusepath{clip}%
\pgfsetroundcap%
\pgfsetroundjoin%
\definecolor{currentfill}{rgb}{0.500000,0.500000,0.500000}%
\pgfsetfillcolor{currentfill}%
\pgfsetfillopacity{0.300000}%
\pgfsetlinewidth{0.301125pt}%
\definecolor{currentstroke}{rgb}{0.500000,0.500000,0.500000}%
\pgfsetstrokecolor{currentstroke}%
\pgfsetstrokeopacity{0.300000}%
\pgfsetdash{}{0pt}%
\pgfpathmoveto{\pgfqpoint{0.000000in}{0.000000in}}%
\pgfpathlineto{\pgfqpoint{0.000000in}{0.000000in}}%
\pgfpathclose%
\pgfusepath{stroke,fill}%
\end{pgfscope}%
\begin{pgfscope}%
\pgfpathrectangle{\pgfqpoint{0.647939in}{0.492442in}}{\pgfqpoint{4.273799in}{2.331163in}}%
\pgfusepath{clip}%
\pgfsetroundcap%
\pgfsetroundjoin%
\pgfsetlinewidth{0.301125pt}%
\definecolor{currentstroke}{rgb}{0.500000,0.500000,0.500000}%
\pgfsetstrokecolor{currentstroke}%
\pgfsetstrokeopacity{0.300000}%
\pgfsetdash{}{0pt}%
\pgfpathmoveto{\pgfqpoint{3.918695in}{2.272172in}}%
\pgfusepath{stroke}%
\end{pgfscope}%
\begin{pgfscope}%
\pgfpathrectangle{\pgfqpoint{0.647939in}{0.492442in}}{\pgfqpoint{4.273799in}{2.331163in}}%
\pgfusepath{clip}%
\pgfsetroundcap%
\pgfsetroundjoin%
\definecolor{currentfill}{rgb}{0.500000,0.500000,0.500000}%
\pgfsetfillcolor{currentfill}%
\pgfsetfillopacity{0.300000}%
\pgfsetlinewidth{0.301125pt}%
\definecolor{currentstroke}{rgb}{0.500000,0.500000,0.500000}%
\pgfsetstrokecolor{currentstroke}%
\pgfsetstrokeopacity{0.300000}%
\pgfsetdash{}{0pt}%
\pgfpathmoveto{\pgfqpoint{0.000000in}{0.000000in}}%
\pgfpathlineto{\pgfqpoint{0.000000in}{0.000000in}}%
\pgfpathclose%
\pgfusepath{stroke,fill}%
\end{pgfscope}%
\begin{pgfscope}%
\pgfpathrectangle{\pgfqpoint{0.647939in}{0.492442in}}{\pgfqpoint{4.273799in}{2.331163in}}%
\pgfusepath{clip}%
\pgfsetroundcap%
\pgfsetroundjoin%
\pgfsetlinewidth{0.301125pt}%
\definecolor{currentstroke}{rgb}{0.500000,0.500000,0.500000}%
\pgfsetstrokecolor{currentstroke}%
\pgfsetstrokeopacity{0.300000}%
\pgfsetdash{}{0pt}%
\pgfpathmoveto{\pgfqpoint{3.805539in}{2.323337in}}%
\pgfusepath{stroke}%
\end{pgfscope}%
\begin{pgfscope}%
\pgfpathrectangle{\pgfqpoint{0.647939in}{0.492442in}}{\pgfqpoint{4.273799in}{2.331163in}}%
\pgfusepath{clip}%
\pgfsetroundcap%
\pgfsetroundjoin%
\definecolor{currentfill}{rgb}{0.500000,0.500000,0.500000}%
\pgfsetfillcolor{currentfill}%
\pgfsetfillopacity{0.300000}%
\pgfsetlinewidth{0.301125pt}%
\definecolor{currentstroke}{rgb}{0.500000,0.500000,0.500000}%
\pgfsetstrokecolor{currentstroke}%
\pgfsetstrokeopacity{0.300000}%
\pgfsetdash{}{0pt}%
\pgfpathmoveto{\pgfqpoint{0.000000in}{0.000000in}}%
\pgfpathlineto{\pgfqpoint{0.000000in}{0.000000in}}%
\pgfpathclose%
\pgfusepath{stroke,fill}%
\end{pgfscope}%
\begin{pgfscope}%
\pgfpathrectangle{\pgfqpoint{0.647939in}{0.492442in}}{\pgfqpoint{4.273799in}{2.331163in}}%
\pgfusepath{clip}%
\pgfsetroundcap%
\pgfsetroundjoin%
\pgfsetlinewidth{0.301125pt}%
\definecolor{currentstroke}{rgb}{0.500000,0.500000,0.500000}%
\pgfsetstrokecolor{currentstroke}%
\pgfsetstrokeopacity{0.300000}%
\pgfsetdash{}{0pt}%
\pgfpathmoveto{\pgfqpoint{3.663900in}{1.794762in}}%
\pgfusepath{stroke}%
\end{pgfscope}%
\begin{pgfscope}%
\pgfpathrectangle{\pgfqpoint{0.647939in}{0.492442in}}{\pgfqpoint{4.273799in}{2.331163in}}%
\pgfusepath{clip}%
\pgfsetroundcap%
\pgfsetroundjoin%
\definecolor{currentfill}{rgb}{0.500000,0.500000,0.500000}%
\pgfsetfillcolor{currentfill}%
\pgfsetfillopacity{0.300000}%
\pgfsetlinewidth{0.301125pt}%
\definecolor{currentstroke}{rgb}{0.500000,0.500000,0.500000}%
\pgfsetstrokecolor{currentstroke}%
\pgfsetstrokeopacity{0.300000}%
\pgfsetdash{}{0pt}%
\pgfpathmoveto{\pgfqpoint{0.000000in}{0.000000in}}%
\pgfpathlineto{\pgfqpoint{0.000000in}{0.000000in}}%
\pgfpathclose%
\pgfusepath{stroke,fill}%
\end{pgfscope}%
\begin{pgfscope}%
\pgfpathrectangle{\pgfqpoint{0.647939in}{0.492442in}}{\pgfqpoint{4.273799in}{2.331163in}}%
\pgfusepath{clip}%
\pgfsetroundcap%
\pgfsetroundjoin%
\pgfsetlinewidth{0.301125pt}%
\definecolor{currentstroke}{rgb}{0.500000,0.500000,0.500000}%
\pgfsetstrokecolor{currentstroke}%
\pgfsetstrokeopacity{0.300000}%
\pgfsetdash{}{0pt}%
\pgfpathmoveto{\pgfqpoint{3.586706in}{2.174616in}}%
\pgfusepath{stroke}%
\end{pgfscope}%
\begin{pgfscope}%
\pgfpathrectangle{\pgfqpoint{0.647939in}{0.492442in}}{\pgfqpoint{4.273799in}{2.331163in}}%
\pgfusepath{clip}%
\pgfsetroundcap%
\pgfsetroundjoin%
\definecolor{currentfill}{rgb}{0.500000,0.500000,0.500000}%
\pgfsetfillcolor{currentfill}%
\pgfsetfillopacity{0.300000}%
\pgfsetlinewidth{0.301125pt}%
\definecolor{currentstroke}{rgb}{0.500000,0.500000,0.500000}%
\pgfsetstrokecolor{currentstroke}%
\pgfsetstrokeopacity{0.300000}%
\pgfsetdash{}{0pt}%
\pgfpathmoveto{\pgfqpoint{0.000000in}{0.000000in}}%
\pgfpathlineto{\pgfqpoint{0.000000in}{0.000000in}}%
\pgfpathclose%
\pgfusepath{stroke,fill}%
\end{pgfscope}%
\begin{pgfscope}%
\pgfpathrectangle{\pgfqpoint{0.647939in}{0.492442in}}{\pgfqpoint{4.273799in}{2.331163in}}%
\pgfusepath{clip}%
\pgfsetroundcap%
\pgfsetroundjoin%
\pgfsetlinewidth{0.301125pt}%
\definecolor{currentstroke}{rgb}{0.500000,0.500000,0.500000}%
\pgfsetstrokecolor{currentstroke}%
\pgfsetstrokeopacity{0.300000}%
\pgfsetdash{}{0pt}%
\pgfpathmoveto{\pgfqpoint{3.293952in}{2.656018in}}%
\pgfusepath{stroke}%
\end{pgfscope}%
\begin{pgfscope}%
\pgfpathrectangle{\pgfqpoint{0.647939in}{0.492442in}}{\pgfqpoint{4.273799in}{2.331163in}}%
\pgfusepath{clip}%
\pgfsetroundcap%
\pgfsetroundjoin%
\definecolor{currentfill}{rgb}{0.500000,0.500000,0.500000}%
\pgfsetfillcolor{currentfill}%
\pgfsetfillopacity{0.300000}%
\pgfsetlinewidth{0.301125pt}%
\definecolor{currentstroke}{rgb}{0.500000,0.500000,0.500000}%
\pgfsetstrokecolor{currentstroke}%
\pgfsetstrokeopacity{0.300000}%
\pgfsetdash{}{0pt}%
\pgfpathmoveto{\pgfqpoint{0.000000in}{0.000000in}}%
\pgfpathlineto{\pgfqpoint{0.000000in}{0.000000in}}%
\pgfpathclose%
\pgfusepath{stroke,fill}%
\end{pgfscope}%
\begin{pgfscope}%
\pgfpathrectangle{\pgfqpoint{0.647939in}{0.492442in}}{\pgfqpoint{4.273799in}{2.331163in}}%
\pgfusepath{clip}%
\pgfsetroundcap%
\pgfsetroundjoin%
\pgfsetlinewidth{0.301125pt}%
\definecolor{currentstroke}{rgb}{0.500000,0.500000,0.500000}%
\pgfsetstrokecolor{currentstroke}%
\pgfsetstrokeopacity{0.300000}%
\pgfsetdash{}{0pt}%
\pgfpathmoveto{\pgfqpoint{3.322468in}{2.341991in}}%
\pgfusepath{stroke}%
\end{pgfscope}%
\begin{pgfscope}%
\pgfpathrectangle{\pgfqpoint{0.647939in}{0.492442in}}{\pgfqpoint{4.273799in}{2.331163in}}%
\pgfusepath{clip}%
\pgfsetroundcap%
\pgfsetroundjoin%
\definecolor{currentfill}{rgb}{0.500000,0.500000,0.500000}%
\pgfsetfillcolor{currentfill}%
\pgfsetfillopacity{0.300000}%
\pgfsetlinewidth{0.301125pt}%
\definecolor{currentstroke}{rgb}{0.500000,0.500000,0.500000}%
\pgfsetstrokecolor{currentstroke}%
\pgfsetstrokeopacity{0.300000}%
\pgfsetdash{}{0pt}%
\pgfpathmoveto{\pgfqpoint{0.000000in}{0.000000in}}%
\pgfpathlineto{\pgfqpoint{0.000000in}{0.000000in}}%
\pgfpathclose%
\pgfusepath{stroke,fill}%
\end{pgfscope}%
\begin{pgfscope}%
\pgfpathrectangle{\pgfqpoint{0.647939in}{0.492442in}}{\pgfqpoint{4.273799in}{2.331163in}}%
\pgfusepath{clip}%
\pgfsetroundcap%
\pgfsetroundjoin%
\pgfsetlinewidth{0.301125pt}%
\definecolor{currentstroke}{rgb}{0.500000,0.500000,0.500000}%
\pgfsetstrokecolor{currentstroke}%
\pgfsetstrokeopacity{0.300000}%
\pgfsetdash{}{0pt}%
\pgfpathmoveto{\pgfqpoint{3.227050in}{2.310472in}}%
\pgfusepath{stroke}%
\end{pgfscope}%
\begin{pgfscope}%
\pgfpathrectangle{\pgfqpoint{0.647939in}{0.492442in}}{\pgfqpoint{4.273799in}{2.331163in}}%
\pgfusepath{clip}%
\pgfsetroundcap%
\pgfsetroundjoin%
\definecolor{currentfill}{rgb}{0.500000,0.500000,0.500000}%
\pgfsetfillcolor{currentfill}%
\pgfsetfillopacity{0.300000}%
\pgfsetlinewidth{0.301125pt}%
\definecolor{currentstroke}{rgb}{0.500000,0.500000,0.500000}%
\pgfsetstrokecolor{currentstroke}%
\pgfsetstrokeopacity{0.300000}%
\pgfsetdash{}{0pt}%
\pgfpathmoveto{\pgfqpoint{0.000000in}{0.000000in}}%
\pgfpathlineto{\pgfqpoint{0.000000in}{0.000000in}}%
\pgfpathclose%
\pgfusepath{stroke,fill}%
\end{pgfscope}%
\begin{pgfscope}%
\pgfpathrectangle{\pgfqpoint{0.647939in}{0.492442in}}{\pgfqpoint{4.273799in}{2.331163in}}%
\pgfusepath{clip}%
\pgfsetroundcap%
\pgfsetroundjoin%
\pgfsetlinewidth{0.301125pt}%
\definecolor{currentstroke}{rgb}{0.500000,0.500000,0.500000}%
\pgfsetstrokecolor{currentstroke}%
\pgfsetstrokeopacity{0.300000}%
\pgfsetdash{}{0pt}%
\pgfpathmoveto{\pgfqpoint{2.853091in}{2.646162in}}%
\pgfusepath{stroke}%
\end{pgfscope}%
\begin{pgfscope}%
\pgfpathrectangle{\pgfqpoint{0.647939in}{0.492442in}}{\pgfqpoint{4.273799in}{2.331163in}}%
\pgfusepath{clip}%
\pgfsetroundcap%
\pgfsetroundjoin%
\definecolor{currentfill}{rgb}{0.500000,0.500000,0.500000}%
\pgfsetfillcolor{currentfill}%
\pgfsetfillopacity{0.300000}%
\pgfsetlinewidth{0.301125pt}%
\definecolor{currentstroke}{rgb}{0.500000,0.500000,0.500000}%
\pgfsetstrokecolor{currentstroke}%
\pgfsetstrokeopacity{0.300000}%
\pgfsetdash{}{0pt}%
\pgfpathmoveto{\pgfqpoint{0.000000in}{0.000000in}}%
\pgfpathlineto{\pgfqpoint{0.000000in}{0.000000in}}%
\pgfpathclose%
\pgfusepath{stroke,fill}%
\end{pgfscope}%
\begin{pgfscope}%
\pgfpathrectangle{\pgfqpoint{0.647939in}{0.492442in}}{\pgfqpoint{4.273799in}{2.331163in}}%
\pgfusepath{clip}%
\pgfsetroundcap%
\pgfsetroundjoin%
\pgfsetlinewidth{0.301125pt}%
\definecolor{currentstroke}{rgb}{0.500000,0.500000,0.500000}%
\pgfsetstrokecolor{currentstroke}%
\pgfsetstrokeopacity{0.300000}%
\pgfsetdash{}{0pt}%
\pgfpathmoveto{\pgfqpoint{2.874648in}{2.526793in}}%
\pgfusepath{stroke}%
\end{pgfscope}%
\begin{pgfscope}%
\pgfpathrectangle{\pgfqpoint{0.647939in}{0.492442in}}{\pgfqpoint{4.273799in}{2.331163in}}%
\pgfusepath{clip}%
\pgfsetroundcap%
\pgfsetroundjoin%
\definecolor{currentfill}{rgb}{0.500000,0.500000,0.500000}%
\pgfsetfillcolor{currentfill}%
\pgfsetfillopacity{0.300000}%
\pgfsetlinewidth{0.301125pt}%
\definecolor{currentstroke}{rgb}{0.500000,0.500000,0.500000}%
\pgfsetstrokecolor{currentstroke}%
\pgfsetstrokeopacity{0.300000}%
\pgfsetdash{}{0pt}%
\pgfpathmoveto{\pgfqpoint{0.000000in}{0.000000in}}%
\pgfpathlineto{\pgfqpoint{0.000000in}{0.000000in}}%
\pgfpathclose%
\pgfusepath{stroke,fill}%
\end{pgfscope}%
\begin{pgfscope}%
\pgfpathrectangle{\pgfqpoint{0.647939in}{0.492442in}}{\pgfqpoint{4.273799in}{2.331163in}}%
\pgfusepath{clip}%
\pgfsetroundcap%
\pgfsetroundjoin%
\pgfsetlinewidth{0.301125pt}%
\definecolor{currentstroke}{rgb}{0.500000,0.500000,0.500000}%
\pgfsetstrokecolor{currentstroke}%
\pgfsetstrokeopacity{0.300000}%
\pgfsetdash{}{0pt}%
\pgfpathmoveto{\pgfqpoint{2.362440in}{2.712075in}}%
\pgfusepath{stroke}%
\end{pgfscope}%
\begin{pgfscope}%
\pgfpathrectangle{\pgfqpoint{0.647939in}{0.492442in}}{\pgfqpoint{4.273799in}{2.331163in}}%
\pgfusepath{clip}%
\pgfsetroundcap%
\pgfsetroundjoin%
\definecolor{currentfill}{rgb}{0.500000,0.500000,0.500000}%
\pgfsetfillcolor{currentfill}%
\pgfsetfillopacity{0.300000}%
\pgfsetlinewidth{0.301125pt}%
\definecolor{currentstroke}{rgb}{0.500000,0.500000,0.500000}%
\pgfsetstrokecolor{currentstroke}%
\pgfsetstrokeopacity{0.300000}%
\pgfsetdash{}{0pt}%
\pgfpathmoveto{\pgfqpoint{0.000000in}{0.000000in}}%
\pgfpathlineto{\pgfqpoint{0.000000in}{0.000000in}}%
\pgfpathclose%
\pgfusepath{stroke,fill}%
\end{pgfscope}%
\begin{pgfscope}%
\pgfpathrectangle{\pgfqpoint{0.647939in}{0.492442in}}{\pgfqpoint{4.273799in}{2.331163in}}%
\pgfusepath{clip}%
\pgfsetroundcap%
\pgfsetroundjoin%
\pgfsetlinewidth{0.301125pt}%
\definecolor{currentstroke}{rgb}{0.500000,0.500000,0.500000}%
\pgfsetstrokecolor{currentstroke}%
\pgfsetstrokeopacity{0.300000}%
\pgfsetdash{}{0pt}%
\pgfpathmoveto{\pgfqpoint{2.229420in}{2.675838in}}%
\pgfusepath{stroke}%
\end{pgfscope}%
\begin{pgfscope}%
\pgfpathrectangle{\pgfqpoint{0.647939in}{0.492442in}}{\pgfqpoint{4.273799in}{2.331163in}}%
\pgfusepath{clip}%
\pgfsetroundcap%
\pgfsetroundjoin%
\definecolor{currentfill}{rgb}{0.500000,0.500000,0.500000}%
\pgfsetfillcolor{currentfill}%
\pgfsetfillopacity{0.300000}%
\pgfsetlinewidth{0.301125pt}%
\definecolor{currentstroke}{rgb}{0.500000,0.500000,0.500000}%
\pgfsetstrokecolor{currentstroke}%
\pgfsetstrokeopacity{0.300000}%
\pgfsetdash{}{0pt}%
\pgfpathmoveto{\pgfqpoint{0.000000in}{0.000000in}}%
\pgfpathlineto{\pgfqpoint{0.000000in}{0.000000in}}%
\pgfpathclose%
\pgfusepath{stroke,fill}%
\end{pgfscope}%
\begin{pgfscope}%
\pgfpathrectangle{\pgfqpoint{0.647939in}{0.492442in}}{\pgfqpoint{4.273799in}{2.331163in}}%
\pgfusepath{clip}%
\pgfsetroundcap%
\pgfsetroundjoin%
\pgfsetlinewidth{0.301125pt}%
\definecolor{currentstroke}{rgb}{0.500000,0.500000,0.500000}%
\pgfsetstrokecolor{currentstroke}%
\pgfsetstrokeopacity{0.300000}%
\pgfsetdash{}{0pt}%
\pgfpathmoveto{\pgfqpoint{1.858410in}{2.729995in}}%
\pgfusepath{stroke}%
\end{pgfscope}%
\begin{pgfscope}%
\pgfpathrectangle{\pgfqpoint{0.647939in}{0.492442in}}{\pgfqpoint{4.273799in}{2.331163in}}%
\pgfusepath{clip}%
\pgfsetroundcap%
\pgfsetroundjoin%
\definecolor{currentfill}{rgb}{0.500000,0.500000,0.500000}%
\pgfsetfillcolor{currentfill}%
\pgfsetfillopacity{0.300000}%
\pgfsetlinewidth{0.301125pt}%
\definecolor{currentstroke}{rgb}{0.500000,0.500000,0.500000}%
\pgfsetstrokecolor{currentstroke}%
\pgfsetstrokeopacity{0.300000}%
\pgfsetdash{}{0pt}%
\pgfpathmoveto{\pgfqpoint{0.000000in}{0.000000in}}%
\pgfpathlineto{\pgfqpoint{0.000000in}{0.000000in}}%
\pgfpathclose%
\pgfusepath{stroke,fill}%
\end{pgfscope}%
\begin{pgfscope}%
\pgfpathrectangle{\pgfqpoint{0.647939in}{0.492442in}}{\pgfqpoint{4.273799in}{2.331163in}}%
\pgfusepath{clip}%
\pgfsetroundcap%
\pgfsetroundjoin%
\pgfsetlinewidth{0.301125pt}%
\definecolor{currentstroke}{rgb}{0.500000,0.500000,0.500000}%
\pgfsetstrokecolor{currentstroke}%
\pgfsetstrokeopacity{0.300000}%
\pgfsetdash{}{0pt}%
\pgfpathmoveto{\pgfqpoint{1.994609in}{2.517907in}}%
\pgfusepath{stroke}%
\end{pgfscope}%
\begin{pgfscope}%
\pgfpathrectangle{\pgfqpoint{0.647939in}{0.492442in}}{\pgfqpoint{4.273799in}{2.331163in}}%
\pgfusepath{clip}%
\pgfsetroundcap%
\pgfsetroundjoin%
\definecolor{currentfill}{rgb}{0.500000,0.500000,0.500000}%
\pgfsetfillcolor{currentfill}%
\pgfsetfillopacity{0.300000}%
\pgfsetlinewidth{0.301125pt}%
\definecolor{currentstroke}{rgb}{0.500000,0.500000,0.500000}%
\pgfsetstrokecolor{currentstroke}%
\pgfsetstrokeopacity{0.300000}%
\pgfsetdash{}{0pt}%
\pgfpathmoveto{\pgfqpoint{0.000000in}{0.000000in}}%
\pgfpathlineto{\pgfqpoint{0.000000in}{0.000000in}}%
\pgfpathclose%
\pgfusepath{stroke,fill}%
\end{pgfscope}%
\begin{pgfscope}%
\pgfpathrectangle{\pgfqpoint{0.647939in}{0.492442in}}{\pgfqpoint{4.273799in}{2.331163in}}%
\pgfusepath{clip}%
\pgfsetroundcap%
\pgfsetroundjoin%
\pgfsetlinewidth{0.301125pt}%
\definecolor{currentstroke}{rgb}{0.500000,0.500000,0.500000}%
\pgfsetstrokecolor{currentstroke}%
\pgfsetstrokeopacity{0.300000}%
\pgfsetdash{}{0pt}%
\pgfpathmoveto{\pgfqpoint{1.701595in}{2.473488in}}%
\pgfusepath{stroke}%
\end{pgfscope}%
\begin{pgfscope}%
\pgfpathrectangle{\pgfqpoint{0.647939in}{0.492442in}}{\pgfqpoint{4.273799in}{2.331163in}}%
\pgfusepath{clip}%
\pgfsetroundcap%
\pgfsetroundjoin%
\definecolor{currentfill}{rgb}{0.500000,0.500000,0.500000}%
\pgfsetfillcolor{currentfill}%
\pgfsetfillopacity{0.300000}%
\pgfsetlinewidth{0.301125pt}%
\definecolor{currentstroke}{rgb}{0.500000,0.500000,0.500000}%
\pgfsetstrokecolor{currentstroke}%
\pgfsetstrokeopacity{0.300000}%
\pgfsetdash{}{0pt}%
\pgfpathmoveto{\pgfqpoint{0.000000in}{0.000000in}}%
\pgfpathlineto{\pgfqpoint{0.000000in}{0.000000in}}%
\pgfpathclose%
\pgfusepath{stroke,fill}%
\end{pgfscope}%
\begin{pgfscope}%
\pgfpathrectangle{\pgfqpoint{0.647939in}{0.492442in}}{\pgfqpoint{4.273799in}{2.331163in}}%
\pgfusepath{clip}%
\pgfsetroundcap%
\pgfsetroundjoin%
\pgfsetlinewidth{0.301125pt}%
\definecolor{currentstroke}{rgb}{0.500000,0.500000,0.500000}%
\pgfsetstrokecolor{currentstroke}%
\pgfsetstrokeopacity{0.300000}%
\pgfsetdash{}{0pt}%
\pgfpathmoveto{\pgfqpoint{1.521824in}{2.450675in}}%
\pgfusepath{stroke}%
\end{pgfscope}%
\begin{pgfscope}%
\pgfpathrectangle{\pgfqpoint{0.647939in}{0.492442in}}{\pgfqpoint{4.273799in}{2.331163in}}%
\pgfusepath{clip}%
\pgfsetroundcap%
\pgfsetroundjoin%
\definecolor{currentfill}{rgb}{0.500000,0.500000,0.500000}%
\pgfsetfillcolor{currentfill}%
\pgfsetfillopacity{0.300000}%
\pgfsetlinewidth{0.301125pt}%
\definecolor{currentstroke}{rgb}{0.500000,0.500000,0.500000}%
\pgfsetstrokecolor{currentstroke}%
\pgfsetstrokeopacity{0.300000}%
\pgfsetdash{}{0pt}%
\pgfpathmoveto{\pgfqpoint{0.000000in}{0.000000in}}%
\pgfpathlineto{\pgfqpoint{0.000000in}{0.000000in}}%
\pgfpathclose%
\pgfusepath{stroke,fill}%
\end{pgfscope}%
\begin{pgfscope}%
\pgfpathrectangle{\pgfqpoint{0.647939in}{0.492442in}}{\pgfqpoint{4.273799in}{2.331163in}}%
\pgfusepath{clip}%
\pgfsetroundcap%
\pgfsetroundjoin%
\pgfsetlinewidth{0.301125pt}%
\definecolor{currentstroke}{rgb}{0.500000,0.500000,0.500000}%
\pgfsetstrokecolor{currentstroke}%
\pgfsetstrokeopacity{0.300000}%
\pgfsetdash{}{0pt}%
\pgfpathmoveto{\pgfqpoint{1.401382in}{2.366791in}}%
\pgfusepath{stroke}%
\end{pgfscope}%
\begin{pgfscope}%
\pgfpathrectangle{\pgfqpoint{0.647939in}{0.492442in}}{\pgfqpoint{4.273799in}{2.331163in}}%
\pgfusepath{clip}%
\pgfsetroundcap%
\pgfsetroundjoin%
\definecolor{currentfill}{rgb}{0.500000,0.500000,0.500000}%
\pgfsetfillcolor{currentfill}%
\pgfsetfillopacity{0.300000}%
\pgfsetlinewidth{0.301125pt}%
\definecolor{currentstroke}{rgb}{0.500000,0.500000,0.500000}%
\pgfsetstrokecolor{currentstroke}%
\pgfsetstrokeopacity{0.300000}%
\pgfsetdash{}{0pt}%
\pgfpathmoveto{\pgfqpoint{0.000000in}{0.000000in}}%
\pgfpathlineto{\pgfqpoint{0.000000in}{0.000000in}}%
\pgfpathclose%
\pgfusepath{stroke,fill}%
\end{pgfscope}%
\begin{pgfscope}%
\pgfpathrectangle{\pgfqpoint{0.647939in}{0.492442in}}{\pgfqpoint{4.273799in}{2.331163in}}%
\pgfusepath{clip}%
\pgfsetroundcap%
\pgfsetroundjoin%
\pgfsetlinewidth{0.301125pt}%
\definecolor{currentstroke}{rgb}{0.500000,0.500000,0.500000}%
\pgfsetstrokecolor{currentstroke}%
\pgfsetstrokeopacity{0.300000}%
\pgfsetdash{}{0pt}%
\pgfpathmoveto{\pgfqpoint{1.300654in}{2.001164in}}%
\pgfusepath{stroke}%
\end{pgfscope}%
\begin{pgfscope}%
\pgfpathrectangle{\pgfqpoint{0.647939in}{0.492442in}}{\pgfqpoint{4.273799in}{2.331163in}}%
\pgfusepath{clip}%
\pgfsetroundcap%
\pgfsetroundjoin%
\definecolor{currentfill}{rgb}{0.500000,0.500000,0.500000}%
\pgfsetfillcolor{currentfill}%
\pgfsetfillopacity{0.300000}%
\pgfsetlinewidth{0.301125pt}%
\definecolor{currentstroke}{rgb}{0.500000,0.500000,0.500000}%
\pgfsetstrokecolor{currentstroke}%
\pgfsetstrokeopacity{0.300000}%
\pgfsetdash{}{0pt}%
\pgfpathmoveto{\pgfqpoint{0.000000in}{0.000000in}}%
\pgfpathlineto{\pgfqpoint{0.000000in}{0.000000in}}%
\pgfpathclose%
\pgfusepath{stroke,fill}%
\end{pgfscope}%
\begin{pgfscope}%
\pgfpathrectangle{\pgfqpoint{0.647939in}{0.492442in}}{\pgfqpoint{4.273799in}{2.331163in}}%
\pgfusepath{clip}%
\pgfsetroundcap%
\pgfsetroundjoin%
\pgfsetlinewidth{0.301125pt}%
\definecolor{currentstroke}{rgb}{0.500000,0.500000,0.500000}%
\pgfsetstrokecolor{currentstroke}%
\pgfsetstrokeopacity{0.300000}%
\pgfsetdash{}{0pt}%
\pgfpathmoveto{\pgfqpoint{1.153396in}{2.257509in}}%
\pgfusepath{stroke}%
\end{pgfscope}%
\begin{pgfscope}%
\pgfpathrectangle{\pgfqpoint{0.647939in}{0.492442in}}{\pgfqpoint{4.273799in}{2.331163in}}%
\pgfusepath{clip}%
\pgfsetroundcap%
\pgfsetroundjoin%
\definecolor{currentfill}{rgb}{0.500000,0.500000,0.500000}%
\pgfsetfillcolor{currentfill}%
\pgfsetfillopacity{0.300000}%
\pgfsetlinewidth{0.301125pt}%
\definecolor{currentstroke}{rgb}{0.500000,0.500000,0.500000}%
\pgfsetstrokecolor{currentstroke}%
\pgfsetstrokeopacity{0.300000}%
\pgfsetdash{}{0pt}%
\pgfpathmoveto{\pgfqpoint{0.000000in}{0.000000in}}%
\pgfpathlineto{\pgfqpoint{0.000000in}{0.000000in}}%
\pgfpathclose%
\pgfusepath{stroke,fill}%
\end{pgfscope}%
\begin{pgfscope}%
\pgfpathrectangle{\pgfqpoint{0.647939in}{0.492442in}}{\pgfqpoint{4.273799in}{2.331163in}}%
\pgfusepath{clip}%
\pgfsetroundcap%
\pgfsetroundjoin%
\pgfsetlinewidth{0.301125pt}%
\definecolor{currentstroke}{rgb}{0.500000,0.500000,0.500000}%
\pgfsetstrokecolor{currentstroke}%
\pgfsetstrokeopacity{0.300000}%
\pgfsetdash{}{0pt}%
\pgfpathmoveto{\pgfqpoint{1.016715in}{1.738697in}}%
\pgfusepath{stroke}%
\end{pgfscope}%
\begin{pgfscope}%
\pgfpathrectangle{\pgfqpoint{0.647939in}{0.492442in}}{\pgfqpoint{4.273799in}{2.331163in}}%
\pgfusepath{clip}%
\pgfsetroundcap%
\pgfsetroundjoin%
\definecolor{currentfill}{rgb}{0.500000,0.500000,0.500000}%
\pgfsetfillcolor{currentfill}%
\pgfsetfillopacity{0.300000}%
\pgfsetlinewidth{0.301125pt}%
\definecolor{currentstroke}{rgb}{0.500000,0.500000,0.500000}%
\pgfsetstrokecolor{currentstroke}%
\pgfsetstrokeopacity{0.300000}%
\pgfsetdash{}{0pt}%
\pgfpathmoveto{\pgfqpoint{0.000000in}{0.000000in}}%
\pgfpathlineto{\pgfqpoint{0.000000in}{0.000000in}}%
\pgfpathclose%
\pgfusepath{stroke,fill}%
\end{pgfscope}%
\begin{pgfscope}%
\pgfpathrectangle{\pgfqpoint{0.647939in}{0.492442in}}{\pgfqpoint{4.273799in}{2.331163in}}%
\pgfusepath{clip}%
\pgfsetroundcap%
\pgfsetroundjoin%
\pgfsetlinewidth{0.301125pt}%
\definecolor{currentstroke}{rgb}{0.500000,0.500000,0.500000}%
\pgfsetstrokecolor{currentstroke}%
\pgfsetstrokeopacity{0.300000}%
\pgfsetdash{}{0pt}%
\pgfpathmoveto{\pgfqpoint{0.926667in}{2.048282in}}%
\pgfusepath{stroke}%
\end{pgfscope}%
\begin{pgfscope}%
\pgfpathrectangle{\pgfqpoint{0.647939in}{0.492442in}}{\pgfqpoint{4.273799in}{2.331163in}}%
\pgfusepath{clip}%
\pgfsetroundcap%
\pgfsetroundjoin%
\definecolor{currentfill}{rgb}{0.500000,0.500000,0.500000}%
\pgfsetfillcolor{currentfill}%
\pgfsetfillopacity{0.300000}%
\pgfsetlinewidth{0.301125pt}%
\definecolor{currentstroke}{rgb}{0.500000,0.500000,0.500000}%
\pgfsetstrokecolor{currentstroke}%
\pgfsetstrokeopacity{0.300000}%
\pgfsetdash{}{0pt}%
\pgfpathmoveto{\pgfqpoint{0.000000in}{0.000000in}}%
\pgfpathlineto{\pgfqpoint{0.000000in}{0.000000in}}%
\pgfpathclose%
\pgfusepath{stroke,fill}%
\end{pgfscope}%
\begin{pgfscope}%
\pgfpathrectangle{\pgfqpoint{0.647939in}{0.492442in}}{\pgfqpoint{4.273799in}{2.331163in}}%
\pgfusepath{clip}%
\pgfsetroundcap%
\pgfsetroundjoin%
\pgfsetlinewidth{0.301125pt}%
\definecolor{currentstroke}{rgb}{0.500000,0.500000,0.500000}%
\pgfsetstrokecolor{currentstroke}%
\pgfsetstrokeopacity{0.300000}%
\pgfsetdash{}{0pt}%
\pgfpathmoveto{\pgfqpoint{0.813722in}{1.944170in}}%
\pgfusepath{stroke}%
\end{pgfscope}%
\begin{pgfscope}%
\pgfpathrectangle{\pgfqpoint{0.647939in}{0.492442in}}{\pgfqpoint{4.273799in}{2.331163in}}%
\pgfusepath{clip}%
\pgfsetroundcap%
\pgfsetroundjoin%
\definecolor{currentfill}{rgb}{0.500000,0.500000,0.500000}%
\pgfsetfillcolor{currentfill}%
\pgfsetfillopacity{0.300000}%
\pgfsetlinewidth{0.301125pt}%
\definecolor{currentstroke}{rgb}{0.500000,0.500000,0.500000}%
\pgfsetstrokecolor{currentstroke}%
\pgfsetstrokeopacity{0.300000}%
\pgfsetdash{}{0pt}%
\pgfpathmoveto{\pgfqpoint{0.000000in}{0.000000in}}%
\pgfpathlineto{\pgfqpoint{0.000000in}{0.000000in}}%
\pgfpathclose%
\pgfusepath{stroke,fill}%
\end{pgfscope}%
\begin{pgfscope}%
\pgfpathrectangle{\pgfqpoint{0.647939in}{0.492442in}}{\pgfqpoint{4.273799in}{2.331163in}}%
\pgfusepath{clip}%
\pgfsetroundcap%
\pgfsetroundjoin%
\pgfsetlinewidth{0.301125pt}%
\definecolor{currentstroke}{rgb}{0.500000,0.500000,0.500000}%
\pgfsetstrokecolor{currentstroke}%
\pgfsetstrokeopacity{0.300000}%
\pgfsetdash{}{0pt}%
\pgfpathmoveto{\pgfqpoint{0.707375in}{2.099236in}}%
\pgfusepath{stroke}%
\end{pgfscope}%
\begin{pgfscope}%
\pgfpathrectangle{\pgfqpoint{0.647939in}{0.492442in}}{\pgfqpoint{4.273799in}{2.331163in}}%
\pgfusepath{clip}%
\pgfsetroundcap%
\pgfsetroundjoin%
\definecolor{currentfill}{rgb}{0.500000,0.500000,0.500000}%
\pgfsetfillcolor{currentfill}%
\pgfsetfillopacity{0.300000}%
\pgfsetlinewidth{0.301125pt}%
\definecolor{currentstroke}{rgb}{0.500000,0.500000,0.500000}%
\pgfsetstrokecolor{currentstroke}%
\pgfsetstrokeopacity{0.300000}%
\pgfsetdash{}{0pt}%
\pgfpathmoveto{\pgfqpoint{0.000000in}{0.000000in}}%
\pgfpathlineto{\pgfqpoint{0.000000in}{0.000000in}}%
\pgfpathclose%
\pgfusepath{stroke,fill}%
\end{pgfscope}%
\begin{pgfscope}%
\pgfpathrectangle{\pgfqpoint{0.647939in}{0.492442in}}{\pgfqpoint{4.273799in}{2.331163in}}%
\pgfusepath{clip}%
\pgfsetroundcap%
\pgfsetroundjoin%
\pgfsetlinewidth{0.301125pt}%
\definecolor{currentstroke}{rgb}{0.500000,0.500000,0.500000}%
\pgfsetstrokecolor{currentstroke}%
\pgfsetstrokeopacity{0.300000}%
\pgfsetdash{}{0pt}%
\pgfpathmoveto{\pgfqpoint{0.656606in}{2.087811in}}%
\pgfusepath{stroke}%
\end{pgfscope}%
\begin{pgfscope}%
\pgfpathrectangle{\pgfqpoint{0.647939in}{0.492442in}}{\pgfqpoint{4.273799in}{2.331163in}}%
\pgfusepath{clip}%
\pgfsetroundcap%
\pgfsetroundjoin%
\definecolor{currentfill}{rgb}{0.500000,0.500000,0.500000}%
\pgfsetfillcolor{currentfill}%
\pgfsetfillopacity{0.300000}%
\pgfsetlinewidth{0.301125pt}%
\definecolor{currentstroke}{rgb}{0.500000,0.500000,0.500000}%
\pgfsetstrokecolor{currentstroke}%
\pgfsetstrokeopacity{0.300000}%
\pgfsetdash{}{0pt}%
\pgfpathmoveto{\pgfqpoint{0.000000in}{0.000000in}}%
\pgfpathlineto{\pgfqpoint{0.000000in}{0.000000in}}%
\pgfpathclose%
\pgfusepath{stroke,fill}%
\end{pgfscope}%
\begin{pgfscope}%
\pgfpathrectangle{\pgfqpoint{0.647939in}{0.492442in}}{\pgfqpoint{4.273799in}{2.331163in}}%
\pgfusepath{clip}%
\pgfsetroundcap%
\pgfsetroundjoin%
\pgfsetlinewidth{0.301125pt}%
\definecolor{currentstroke}{rgb}{0.500000,0.500000,0.500000}%
\pgfsetstrokecolor{currentstroke}%
\pgfsetstrokeopacity{0.300000}%
\pgfsetdash{}{0pt}%
\pgfpathmoveto{\pgfqpoint{0.698630in}{1.016235in}}%
\pgfusepath{stroke}%
\end{pgfscope}%
\begin{pgfscope}%
\pgfpathrectangle{\pgfqpoint{0.647939in}{0.492442in}}{\pgfqpoint{4.273799in}{2.331163in}}%
\pgfusepath{clip}%
\pgfsetroundcap%
\pgfsetroundjoin%
\definecolor{currentfill}{rgb}{0.500000,0.500000,0.500000}%
\pgfsetfillcolor{currentfill}%
\pgfsetfillopacity{0.300000}%
\pgfsetlinewidth{0.301125pt}%
\definecolor{currentstroke}{rgb}{0.500000,0.500000,0.500000}%
\pgfsetstrokecolor{currentstroke}%
\pgfsetstrokeopacity{0.300000}%
\pgfsetdash{}{0pt}%
\pgfpathmoveto{\pgfqpoint{0.000000in}{0.000000in}}%
\pgfpathlineto{\pgfqpoint{0.000000in}{0.000000in}}%
\pgfpathclose%
\pgfusepath{stroke,fill}%
\end{pgfscope}%
\begin{pgfscope}%
\pgfpathrectangle{\pgfqpoint{0.647939in}{0.492442in}}{\pgfqpoint{4.273799in}{2.331163in}}%
\pgfusepath{clip}%
\pgfsetroundcap%
\pgfsetroundjoin%
\pgfsetlinewidth{0.301125pt}%
\definecolor{currentstroke}{rgb}{0.500000,0.500000,0.500000}%
\pgfsetstrokecolor{currentstroke}%
\pgfsetstrokeopacity{0.300000}%
\pgfsetdash{}{0pt}%
\pgfpathmoveto{\pgfqpoint{1.333647in}{0.592953in}}%
\pgfusepath{stroke}%
\end{pgfscope}%
\begin{pgfscope}%
\pgfpathrectangle{\pgfqpoint{0.647939in}{0.492442in}}{\pgfqpoint{4.273799in}{2.331163in}}%
\pgfusepath{clip}%
\pgfsetroundcap%
\pgfsetroundjoin%
\definecolor{currentfill}{rgb}{0.500000,0.500000,0.500000}%
\pgfsetfillcolor{currentfill}%
\pgfsetfillopacity{0.300000}%
\pgfsetlinewidth{0.301125pt}%
\definecolor{currentstroke}{rgb}{0.500000,0.500000,0.500000}%
\pgfsetstrokecolor{currentstroke}%
\pgfsetstrokeopacity{0.300000}%
\pgfsetdash{}{0pt}%
\pgfpathmoveto{\pgfqpoint{0.000000in}{0.000000in}}%
\pgfpathlineto{\pgfqpoint{0.000000in}{0.000000in}}%
\pgfpathclose%
\pgfusepath{stroke,fill}%
\end{pgfscope}%
\begin{pgfscope}%
\pgfpathrectangle{\pgfqpoint{0.647939in}{0.492442in}}{\pgfqpoint{4.273799in}{2.331163in}}%
\pgfusepath{clip}%
\pgfsetroundcap%
\pgfsetroundjoin%
\pgfsetlinewidth{0.301125pt}%
\definecolor{currentstroke}{rgb}{0.500000,0.500000,0.500000}%
\pgfsetstrokecolor{currentstroke}%
\pgfsetstrokeopacity{0.300000}%
\pgfsetdash{}{0pt}%
\pgfpathmoveto{\pgfqpoint{3.867466in}{0.944552in}}%
\pgfusepath{stroke}%
\end{pgfscope}%
\begin{pgfscope}%
\pgfpathrectangle{\pgfqpoint{0.647939in}{0.492442in}}{\pgfqpoint{4.273799in}{2.331163in}}%
\pgfusepath{clip}%
\pgfsetroundcap%
\pgfsetroundjoin%
\definecolor{currentfill}{rgb}{0.500000,0.500000,0.500000}%
\pgfsetfillcolor{currentfill}%
\pgfsetfillopacity{0.300000}%
\pgfsetlinewidth{0.301125pt}%
\definecolor{currentstroke}{rgb}{0.500000,0.500000,0.500000}%
\pgfsetstrokecolor{currentstroke}%
\pgfsetstrokeopacity{0.300000}%
\pgfsetdash{}{0pt}%
\pgfpathmoveto{\pgfqpoint{0.000000in}{0.000000in}}%
\pgfpathlineto{\pgfqpoint{0.000000in}{0.000000in}}%
\pgfpathclose%
\pgfusepath{stroke,fill}%
\end{pgfscope}%
\begin{pgfscope}%
\pgfpathrectangle{\pgfqpoint{0.647939in}{0.492442in}}{\pgfqpoint{4.273799in}{2.331163in}}%
\pgfusepath{clip}%
\pgfsetroundcap%
\pgfsetroundjoin%
\pgfsetlinewidth{0.301125pt}%
\definecolor{currentstroke}{rgb}{0.500000,0.500000,0.500000}%
\pgfsetstrokecolor{currentstroke}%
\pgfsetstrokeopacity{0.300000}%
\pgfsetdash{}{0pt}%
\pgfpathmoveto{\pgfqpoint{1.943563in}{2.603150in}}%
\pgfusepath{stroke}%
\end{pgfscope}%
\begin{pgfscope}%
\pgfpathrectangle{\pgfqpoint{0.647939in}{0.492442in}}{\pgfqpoint{4.273799in}{2.331163in}}%
\pgfusepath{clip}%
\pgfsetroundcap%
\pgfsetroundjoin%
\definecolor{currentfill}{rgb}{0.500000,0.500000,0.500000}%
\pgfsetfillcolor{currentfill}%
\pgfsetfillopacity{0.300000}%
\pgfsetlinewidth{0.301125pt}%
\definecolor{currentstroke}{rgb}{0.500000,0.500000,0.500000}%
\pgfsetstrokecolor{currentstroke}%
\pgfsetstrokeopacity{0.300000}%
\pgfsetdash{}{0pt}%
\pgfpathmoveto{\pgfqpoint{0.000000in}{0.000000in}}%
\pgfpathlineto{\pgfqpoint{0.000000in}{0.000000in}}%
\pgfpathclose%
\pgfusepath{stroke,fill}%
\end{pgfscope}%
\begin{pgfscope}%
\pgfpathrectangle{\pgfqpoint{0.647939in}{0.492442in}}{\pgfqpoint{4.273799in}{2.331163in}}%
\pgfusepath{clip}%
\pgfsetroundcap%
\pgfsetroundjoin%
\pgfsetlinewidth{0.301125pt}%
\definecolor{currentstroke}{rgb}{0.500000,0.500000,0.500000}%
\pgfsetstrokecolor{currentstroke}%
\pgfsetstrokeopacity{0.300000}%
\pgfsetdash{}{0pt}%
\pgfpathmoveto{\pgfqpoint{3.903390in}{0.745924in}}%
\pgfusepath{stroke}%
\end{pgfscope}%
\begin{pgfscope}%
\pgfpathrectangle{\pgfqpoint{0.647939in}{0.492442in}}{\pgfqpoint{4.273799in}{2.331163in}}%
\pgfusepath{clip}%
\pgfsetroundcap%
\pgfsetroundjoin%
\definecolor{currentfill}{rgb}{0.500000,0.500000,0.500000}%
\pgfsetfillcolor{currentfill}%
\pgfsetfillopacity{0.300000}%
\pgfsetlinewidth{0.301125pt}%
\definecolor{currentstroke}{rgb}{0.500000,0.500000,0.500000}%
\pgfsetstrokecolor{currentstroke}%
\pgfsetstrokeopacity{0.300000}%
\pgfsetdash{}{0pt}%
\pgfpathmoveto{\pgfqpoint{0.000000in}{0.000000in}}%
\pgfpathlineto{\pgfqpoint{0.000000in}{0.000000in}}%
\pgfpathclose%
\pgfusepath{stroke,fill}%
\end{pgfscope}%
\begin{pgfscope}%
\pgfpathrectangle{\pgfqpoint{0.647939in}{0.492442in}}{\pgfqpoint{4.273799in}{2.331163in}}%
\pgfusepath{clip}%
\pgfsetroundcap%
\pgfsetroundjoin%
\pgfsetlinewidth{0.301125pt}%
\definecolor{currentstroke}{rgb}{0.500000,0.500000,0.500000}%
\pgfsetstrokecolor{currentstroke}%
\pgfsetstrokeopacity{0.300000}%
\pgfsetdash{}{0pt}%
\pgfpathmoveto{\pgfqpoint{4.563069in}{1.808961in}}%
\pgfusepath{stroke}%
\end{pgfscope}%
\begin{pgfscope}%
\pgfpathrectangle{\pgfqpoint{0.647939in}{0.492442in}}{\pgfqpoint{4.273799in}{2.331163in}}%
\pgfusepath{clip}%
\pgfsetroundcap%
\pgfsetroundjoin%
\definecolor{currentfill}{rgb}{0.500000,0.500000,0.500000}%
\pgfsetfillcolor{currentfill}%
\pgfsetfillopacity{0.300000}%
\pgfsetlinewidth{0.301125pt}%
\definecolor{currentstroke}{rgb}{0.500000,0.500000,0.500000}%
\pgfsetstrokecolor{currentstroke}%
\pgfsetstrokeopacity{0.300000}%
\pgfsetdash{}{0pt}%
\pgfpathmoveto{\pgfqpoint{0.000000in}{0.000000in}}%
\pgfpathlineto{\pgfqpoint{0.000000in}{0.000000in}}%
\pgfpathclose%
\pgfusepath{stroke,fill}%
\end{pgfscope}%
\begin{pgfscope}%
\pgfpathrectangle{\pgfqpoint{0.647939in}{0.492442in}}{\pgfqpoint{4.273799in}{2.331163in}}%
\pgfusepath{clip}%
\pgfsetroundcap%
\pgfsetroundjoin%
\pgfsetlinewidth{0.301125pt}%
\definecolor{currentstroke}{rgb}{0.500000,0.500000,0.500000}%
\pgfsetstrokecolor{currentstroke}%
\pgfsetstrokeopacity{0.300000}%
\pgfsetdash{}{0pt}%
\pgfpathmoveto{\pgfqpoint{4.394455in}{2.652328in}}%
\pgfusepath{stroke}%
\end{pgfscope}%
\begin{pgfscope}%
\pgfpathrectangle{\pgfqpoint{0.647939in}{0.492442in}}{\pgfqpoint{4.273799in}{2.331163in}}%
\pgfusepath{clip}%
\pgfsetroundcap%
\pgfsetroundjoin%
\definecolor{currentfill}{rgb}{0.500000,0.500000,0.500000}%
\pgfsetfillcolor{currentfill}%
\pgfsetfillopacity{0.300000}%
\pgfsetlinewidth{0.301125pt}%
\definecolor{currentstroke}{rgb}{0.500000,0.500000,0.500000}%
\pgfsetstrokecolor{currentstroke}%
\pgfsetstrokeopacity{0.300000}%
\pgfsetdash{}{0pt}%
\pgfpathmoveto{\pgfqpoint{0.000000in}{0.000000in}}%
\pgfpathlineto{\pgfqpoint{0.000000in}{0.000000in}}%
\pgfpathclose%
\pgfusepath{stroke,fill}%
\end{pgfscope}%
\begin{pgfscope}%
\pgfpathrectangle{\pgfqpoint{0.647939in}{0.492442in}}{\pgfqpoint{4.273799in}{2.331163in}}%
\pgfusepath{clip}%
\pgfsetroundcap%
\pgfsetroundjoin%
\pgfsetlinewidth{0.301125pt}%
\definecolor{currentstroke}{rgb}{0.500000,0.500000,0.500000}%
\pgfsetstrokecolor{currentstroke}%
\pgfsetstrokeopacity{0.300000}%
\pgfsetdash{}{0pt}%
\pgfpathmoveto{\pgfqpoint{1.882024in}{0.869984in}}%
\pgfusepath{stroke}%
\end{pgfscope}%
\begin{pgfscope}%
\pgfpathrectangle{\pgfqpoint{0.647939in}{0.492442in}}{\pgfqpoint{4.273799in}{2.331163in}}%
\pgfusepath{clip}%
\pgfsetroundcap%
\pgfsetroundjoin%
\definecolor{currentfill}{rgb}{0.500000,0.500000,0.500000}%
\pgfsetfillcolor{currentfill}%
\pgfsetfillopacity{0.300000}%
\pgfsetlinewidth{0.301125pt}%
\definecolor{currentstroke}{rgb}{0.500000,0.500000,0.500000}%
\pgfsetstrokecolor{currentstroke}%
\pgfsetstrokeopacity{0.300000}%
\pgfsetdash{}{0pt}%
\pgfpathmoveto{\pgfqpoint{0.000000in}{0.000000in}}%
\pgfpathlineto{\pgfqpoint{0.000000in}{0.000000in}}%
\pgfpathclose%
\pgfusepath{stroke,fill}%
\end{pgfscope}%
\begin{pgfscope}%
\pgfpathrectangle{\pgfqpoint{0.647939in}{0.492442in}}{\pgfqpoint{4.273799in}{2.331163in}}%
\pgfusepath{clip}%
\pgfsetroundcap%
\pgfsetroundjoin%
\pgfsetlinewidth{0.301125pt}%
\definecolor{currentstroke}{rgb}{0.500000,0.500000,0.500000}%
\pgfsetstrokecolor{currentstroke}%
\pgfsetstrokeopacity{0.300000}%
\pgfsetdash{}{0pt}%
\pgfpathmoveto{\pgfqpoint{3.668386in}{0.756936in}}%
\pgfusepath{stroke}%
\end{pgfscope}%
\begin{pgfscope}%
\pgfpathrectangle{\pgfqpoint{0.647939in}{0.492442in}}{\pgfqpoint{4.273799in}{2.331163in}}%
\pgfusepath{clip}%
\pgfsetroundcap%
\pgfsetroundjoin%
\definecolor{currentfill}{rgb}{0.500000,0.500000,0.500000}%
\pgfsetfillcolor{currentfill}%
\pgfsetfillopacity{0.300000}%
\pgfsetlinewidth{0.301125pt}%
\definecolor{currentstroke}{rgb}{0.500000,0.500000,0.500000}%
\pgfsetstrokecolor{currentstroke}%
\pgfsetstrokeopacity{0.300000}%
\pgfsetdash{}{0pt}%
\pgfpathmoveto{\pgfqpoint{0.000000in}{0.000000in}}%
\pgfpathlineto{\pgfqpoint{0.000000in}{0.000000in}}%
\pgfpathclose%
\pgfusepath{stroke,fill}%
\end{pgfscope}%
\begin{pgfscope}%
\pgfpathrectangle{\pgfqpoint{0.647939in}{0.492442in}}{\pgfqpoint{4.273799in}{2.331163in}}%
\pgfusepath{clip}%
\pgfsetroundcap%
\pgfsetroundjoin%
\pgfsetlinewidth{0.301125pt}%
\definecolor{currentstroke}{rgb}{0.500000,0.500000,0.500000}%
\pgfsetstrokecolor{currentstroke}%
\pgfsetstrokeopacity{0.300000}%
\pgfsetdash{}{0pt}%
\pgfpathmoveto{\pgfqpoint{4.092539in}{1.535467in}}%
\pgfusepath{stroke}%
\end{pgfscope}%
\begin{pgfscope}%
\pgfpathrectangle{\pgfqpoint{0.647939in}{0.492442in}}{\pgfqpoint{4.273799in}{2.331163in}}%
\pgfusepath{clip}%
\pgfsetroundcap%
\pgfsetroundjoin%
\definecolor{currentfill}{rgb}{0.500000,0.500000,0.500000}%
\pgfsetfillcolor{currentfill}%
\pgfsetfillopacity{0.300000}%
\pgfsetlinewidth{0.301125pt}%
\definecolor{currentstroke}{rgb}{0.500000,0.500000,0.500000}%
\pgfsetstrokecolor{currentstroke}%
\pgfsetstrokeopacity{0.300000}%
\pgfsetdash{}{0pt}%
\pgfpathmoveto{\pgfqpoint{0.000000in}{0.000000in}}%
\pgfpathlineto{\pgfqpoint{0.000000in}{0.000000in}}%
\pgfpathclose%
\pgfusepath{stroke,fill}%
\end{pgfscope}%
\begin{pgfscope}%
\pgfpathrectangle{\pgfqpoint{0.647939in}{0.492442in}}{\pgfqpoint{4.273799in}{2.331163in}}%
\pgfusepath{clip}%
\pgfsetroundcap%
\pgfsetroundjoin%
\pgfsetlinewidth{0.301125pt}%
\definecolor{currentstroke}{rgb}{0.500000,0.500000,0.500000}%
\pgfsetstrokecolor{currentstroke}%
\pgfsetstrokeopacity{0.300000}%
\pgfsetdash{}{0pt}%
\pgfpathmoveto{\pgfqpoint{4.316510in}{1.755468in}}%
\pgfusepath{stroke}%
\end{pgfscope}%
\begin{pgfscope}%
\pgfpathrectangle{\pgfqpoint{0.647939in}{0.492442in}}{\pgfqpoint{4.273799in}{2.331163in}}%
\pgfusepath{clip}%
\pgfsetroundcap%
\pgfsetroundjoin%
\definecolor{currentfill}{rgb}{0.500000,0.500000,0.500000}%
\pgfsetfillcolor{currentfill}%
\pgfsetfillopacity{0.300000}%
\pgfsetlinewidth{0.301125pt}%
\definecolor{currentstroke}{rgb}{0.500000,0.500000,0.500000}%
\pgfsetstrokecolor{currentstroke}%
\pgfsetstrokeopacity{0.300000}%
\pgfsetdash{}{0pt}%
\pgfpathmoveto{\pgfqpoint{0.000000in}{0.000000in}}%
\pgfpathlineto{\pgfqpoint{0.000000in}{0.000000in}}%
\pgfpathclose%
\pgfusepath{stroke,fill}%
\end{pgfscope}%
\begin{pgfscope}%
\pgfpathrectangle{\pgfqpoint{0.647939in}{0.492442in}}{\pgfqpoint{4.273799in}{2.331163in}}%
\pgfusepath{clip}%
\pgfsetroundcap%
\pgfsetroundjoin%
\pgfsetlinewidth{0.301125pt}%
\definecolor{currentstroke}{rgb}{0.500000,0.500000,0.500000}%
\pgfsetstrokecolor{currentstroke}%
\pgfsetstrokeopacity{0.300000}%
\pgfsetdash{}{0pt}%
\pgfpathmoveto{\pgfqpoint{1.202992in}{0.756444in}}%
\pgfusepath{stroke}%
\end{pgfscope}%
\begin{pgfscope}%
\pgfpathrectangle{\pgfqpoint{0.647939in}{0.492442in}}{\pgfqpoint{4.273799in}{2.331163in}}%
\pgfusepath{clip}%
\pgfsetroundcap%
\pgfsetroundjoin%
\definecolor{currentfill}{rgb}{0.500000,0.500000,0.500000}%
\pgfsetfillcolor{currentfill}%
\pgfsetfillopacity{0.300000}%
\pgfsetlinewidth{0.301125pt}%
\definecolor{currentstroke}{rgb}{0.500000,0.500000,0.500000}%
\pgfsetstrokecolor{currentstroke}%
\pgfsetstrokeopacity{0.300000}%
\pgfsetdash{}{0pt}%
\pgfpathmoveto{\pgfqpoint{0.000000in}{0.000000in}}%
\pgfpathlineto{\pgfqpoint{0.000000in}{0.000000in}}%
\pgfpathclose%
\pgfusepath{stroke,fill}%
\end{pgfscope}%
\begin{pgfscope}%
\pgfpathrectangle{\pgfqpoint{0.647939in}{0.492442in}}{\pgfqpoint{4.273799in}{2.331163in}}%
\pgfusepath{clip}%
\pgfsetroundcap%
\pgfsetroundjoin%
\pgfsetlinewidth{0.301125pt}%
\definecolor{currentstroke}{rgb}{0.500000,0.500000,0.500000}%
\pgfsetstrokecolor{currentstroke}%
\pgfsetstrokeopacity{0.300000}%
\pgfsetdash{}{0pt}%
\pgfpathmoveto{\pgfqpoint{2.207513in}{1.460265in}}%
\pgfusepath{stroke}%
\end{pgfscope}%
\begin{pgfscope}%
\pgfpathrectangle{\pgfqpoint{0.647939in}{0.492442in}}{\pgfqpoint{4.273799in}{2.331163in}}%
\pgfusepath{clip}%
\pgfsetroundcap%
\pgfsetroundjoin%
\definecolor{currentfill}{rgb}{0.500000,0.500000,0.500000}%
\pgfsetfillcolor{currentfill}%
\pgfsetfillopacity{0.300000}%
\pgfsetlinewidth{0.301125pt}%
\definecolor{currentstroke}{rgb}{0.500000,0.500000,0.500000}%
\pgfsetstrokecolor{currentstroke}%
\pgfsetstrokeopacity{0.300000}%
\pgfsetdash{}{0pt}%
\pgfpathmoveto{\pgfqpoint{0.000000in}{0.000000in}}%
\pgfpathlineto{\pgfqpoint{0.000000in}{0.000000in}}%
\pgfpathclose%
\pgfusepath{stroke,fill}%
\end{pgfscope}%
\begin{pgfscope}%
\pgfpathrectangle{\pgfqpoint{0.647939in}{0.492442in}}{\pgfqpoint{4.273799in}{2.331163in}}%
\pgfusepath{clip}%
\pgfsetroundcap%
\pgfsetroundjoin%
\pgfsetlinewidth{0.301125pt}%
\definecolor{currentstroke}{rgb}{0.500000,0.500000,0.500000}%
\pgfsetstrokecolor{currentstroke}%
\pgfsetstrokeopacity{0.300000}%
\pgfsetdash{}{0pt}%
\pgfpathmoveto{\pgfqpoint{3.355220in}{0.891841in}}%
\pgfusepath{stroke}%
\end{pgfscope}%
\begin{pgfscope}%
\pgfpathrectangle{\pgfqpoint{0.647939in}{0.492442in}}{\pgfqpoint{4.273799in}{2.331163in}}%
\pgfusepath{clip}%
\pgfsetroundcap%
\pgfsetroundjoin%
\definecolor{currentfill}{rgb}{0.500000,0.500000,0.500000}%
\pgfsetfillcolor{currentfill}%
\pgfsetfillopacity{0.300000}%
\pgfsetlinewidth{0.301125pt}%
\definecolor{currentstroke}{rgb}{0.500000,0.500000,0.500000}%
\pgfsetstrokecolor{currentstroke}%
\pgfsetstrokeopacity{0.300000}%
\pgfsetdash{}{0pt}%
\pgfpathmoveto{\pgfqpoint{0.000000in}{0.000000in}}%
\pgfpathlineto{\pgfqpoint{0.000000in}{0.000000in}}%
\pgfpathclose%
\pgfusepath{stroke,fill}%
\end{pgfscope}%
\begin{pgfscope}%
\pgfpathrectangle{\pgfqpoint{0.647939in}{0.492442in}}{\pgfqpoint{4.273799in}{2.331163in}}%
\pgfusepath{clip}%
\pgfsetroundcap%
\pgfsetroundjoin%
\pgfsetlinewidth{0.301125pt}%
\definecolor{currentstroke}{rgb}{0.500000,0.500000,0.500000}%
\pgfsetstrokecolor{currentstroke}%
\pgfsetstrokeopacity{0.300000}%
\pgfsetdash{}{0pt}%
\pgfpathmoveto{\pgfqpoint{1.370639in}{1.058416in}}%
\pgfusepath{stroke}%
\end{pgfscope}%
\begin{pgfscope}%
\pgfpathrectangle{\pgfqpoint{0.647939in}{0.492442in}}{\pgfqpoint{4.273799in}{2.331163in}}%
\pgfusepath{clip}%
\pgfsetroundcap%
\pgfsetroundjoin%
\definecolor{currentfill}{rgb}{0.500000,0.500000,0.500000}%
\pgfsetfillcolor{currentfill}%
\pgfsetfillopacity{0.300000}%
\pgfsetlinewidth{0.301125pt}%
\definecolor{currentstroke}{rgb}{0.500000,0.500000,0.500000}%
\pgfsetstrokecolor{currentstroke}%
\pgfsetstrokeopacity{0.300000}%
\pgfsetdash{}{0pt}%
\pgfpathmoveto{\pgfqpoint{0.000000in}{0.000000in}}%
\pgfpathlineto{\pgfqpoint{0.000000in}{0.000000in}}%
\pgfpathclose%
\pgfusepath{stroke,fill}%
\end{pgfscope}%
\begin{pgfscope}%
\pgfpathrectangle{\pgfqpoint{0.647939in}{0.492442in}}{\pgfqpoint{4.273799in}{2.331163in}}%
\pgfusepath{clip}%
\pgfsetroundcap%
\pgfsetroundjoin%
\pgfsetlinewidth{0.301125pt}%
\definecolor{currentstroke}{rgb}{0.500000,0.500000,0.500000}%
\pgfsetstrokecolor{currentstroke}%
\pgfsetstrokeopacity{0.300000}%
\pgfsetdash{}{0pt}%
\pgfpathmoveto{\pgfqpoint{3.946985in}{1.368879in}}%
\pgfusepath{stroke}%
\end{pgfscope}%
\begin{pgfscope}%
\pgfpathrectangle{\pgfqpoint{0.647939in}{0.492442in}}{\pgfqpoint{4.273799in}{2.331163in}}%
\pgfusepath{clip}%
\pgfsetroundcap%
\pgfsetroundjoin%
\definecolor{currentfill}{rgb}{0.500000,0.500000,0.500000}%
\pgfsetfillcolor{currentfill}%
\pgfsetfillopacity{0.300000}%
\pgfsetlinewidth{0.301125pt}%
\definecolor{currentstroke}{rgb}{0.500000,0.500000,0.500000}%
\pgfsetstrokecolor{currentstroke}%
\pgfsetstrokeopacity{0.300000}%
\pgfsetdash{}{0pt}%
\pgfpathmoveto{\pgfqpoint{0.000000in}{0.000000in}}%
\pgfpathlineto{\pgfqpoint{0.000000in}{0.000000in}}%
\pgfpathclose%
\pgfusepath{stroke,fill}%
\end{pgfscope}%
\begin{pgfscope}%
\pgfpathrectangle{\pgfqpoint{0.647939in}{0.492442in}}{\pgfqpoint{4.273799in}{2.331163in}}%
\pgfusepath{clip}%
\pgfsetroundcap%
\pgfsetroundjoin%
\pgfsetlinewidth{0.301125pt}%
\definecolor{currentstroke}{rgb}{0.500000,0.500000,0.500000}%
\pgfsetstrokecolor{currentstroke}%
\pgfsetstrokeopacity{0.300000}%
\pgfsetdash{}{0pt}%
\pgfpathmoveto{\pgfqpoint{4.238254in}{1.733440in}}%
\pgfusepath{stroke}%
\end{pgfscope}%
\begin{pgfscope}%
\pgfpathrectangle{\pgfqpoint{0.647939in}{0.492442in}}{\pgfqpoint{4.273799in}{2.331163in}}%
\pgfusepath{clip}%
\pgfsetroundcap%
\pgfsetroundjoin%
\definecolor{currentfill}{rgb}{0.500000,0.500000,0.500000}%
\pgfsetfillcolor{currentfill}%
\pgfsetfillopacity{0.300000}%
\pgfsetlinewidth{0.301125pt}%
\definecolor{currentstroke}{rgb}{0.500000,0.500000,0.500000}%
\pgfsetstrokecolor{currentstroke}%
\pgfsetstrokeopacity{0.300000}%
\pgfsetdash{}{0pt}%
\pgfpathmoveto{\pgfqpoint{0.000000in}{0.000000in}}%
\pgfpathlineto{\pgfqpoint{0.000000in}{0.000000in}}%
\pgfpathclose%
\pgfusepath{stroke,fill}%
\end{pgfscope}%
\begin{pgfscope}%
\pgfpathrectangle{\pgfqpoint{0.647939in}{0.492442in}}{\pgfqpoint{4.273799in}{2.331163in}}%
\pgfusepath{clip}%
\pgfsetroundcap%
\pgfsetroundjoin%
\pgfsetlinewidth{0.301125pt}%
\definecolor{currentstroke}{rgb}{0.500000,0.500000,0.500000}%
\pgfsetstrokecolor{currentstroke}%
\pgfsetstrokeopacity{0.300000}%
\pgfsetdash{}{0pt}%
\pgfpathmoveto{\pgfqpoint{4.291927in}{2.151757in}}%
\pgfusepath{stroke}%
\end{pgfscope}%
\begin{pgfscope}%
\pgfpathrectangle{\pgfqpoint{0.647939in}{0.492442in}}{\pgfqpoint{4.273799in}{2.331163in}}%
\pgfusepath{clip}%
\pgfsetroundcap%
\pgfsetroundjoin%
\definecolor{currentfill}{rgb}{0.500000,0.500000,0.500000}%
\pgfsetfillcolor{currentfill}%
\pgfsetfillopacity{0.300000}%
\pgfsetlinewidth{0.301125pt}%
\definecolor{currentstroke}{rgb}{0.500000,0.500000,0.500000}%
\pgfsetstrokecolor{currentstroke}%
\pgfsetstrokeopacity{0.300000}%
\pgfsetdash{}{0pt}%
\pgfpathmoveto{\pgfqpoint{0.000000in}{0.000000in}}%
\pgfpathlineto{\pgfqpoint{0.000000in}{0.000000in}}%
\pgfpathclose%
\pgfusepath{stroke,fill}%
\end{pgfscope}%
\begin{pgfscope}%
\pgfpathrectangle{\pgfqpoint{0.647939in}{0.492442in}}{\pgfqpoint{4.273799in}{2.331163in}}%
\pgfusepath{clip}%
\pgfsetroundcap%
\pgfsetroundjoin%
\pgfsetlinewidth{0.301125pt}%
\definecolor{currentstroke}{rgb}{0.500000,0.500000,0.500000}%
\pgfsetstrokecolor{currentstroke}%
\pgfsetstrokeopacity{0.300000}%
\pgfsetdash{}{0pt}%
\pgfpathmoveto{\pgfqpoint{1.241048in}{2.137370in}}%
\pgfusepath{stroke}%
\end{pgfscope}%
\begin{pgfscope}%
\pgfpathrectangle{\pgfqpoint{0.647939in}{0.492442in}}{\pgfqpoint{4.273799in}{2.331163in}}%
\pgfusepath{clip}%
\pgfsetroundcap%
\pgfsetroundjoin%
\definecolor{currentfill}{rgb}{0.500000,0.500000,0.500000}%
\pgfsetfillcolor{currentfill}%
\pgfsetfillopacity{0.300000}%
\pgfsetlinewidth{0.301125pt}%
\definecolor{currentstroke}{rgb}{0.500000,0.500000,0.500000}%
\pgfsetstrokecolor{currentstroke}%
\pgfsetstrokeopacity{0.300000}%
\pgfsetdash{}{0pt}%
\pgfpathmoveto{\pgfqpoint{0.000000in}{0.000000in}}%
\pgfpathlineto{\pgfqpoint{0.000000in}{0.000000in}}%
\pgfpathclose%
\pgfusepath{stroke,fill}%
\end{pgfscope}%
\begin{pgfscope}%
\pgfpathrectangle{\pgfqpoint{0.647939in}{0.492442in}}{\pgfqpoint{4.273799in}{2.331163in}}%
\pgfusepath{clip}%
\pgfsetroundcap%
\pgfsetroundjoin%
\pgfsetlinewidth{0.301125pt}%
\definecolor{currentstroke}{rgb}{0.500000,0.500000,0.500000}%
\pgfsetstrokecolor{currentstroke}%
\pgfsetstrokeopacity{0.300000}%
\pgfsetdash{}{0pt}%
\pgfpathmoveto{\pgfqpoint{1.531124in}{1.268602in}}%
\pgfusepath{stroke}%
\end{pgfscope}%
\begin{pgfscope}%
\pgfpathrectangle{\pgfqpoint{0.647939in}{0.492442in}}{\pgfqpoint{4.273799in}{2.331163in}}%
\pgfusepath{clip}%
\pgfsetroundcap%
\pgfsetroundjoin%
\definecolor{currentfill}{rgb}{0.500000,0.500000,0.500000}%
\pgfsetfillcolor{currentfill}%
\pgfsetfillopacity{0.300000}%
\pgfsetlinewidth{0.301125pt}%
\definecolor{currentstroke}{rgb}{0.500000,0.500000,0.500000}%
\pgfsetstrokecolor{currentstroke}%
\pgfsetstrokeopacity{0.300000}%
\pgfsetdash{}{0pt}%
\pgfpathmoveto{\pgfqpoint{0.000000in}{0.000000in}}%
\pgfpathlineto{\pgfqpoint{0.000000in}{0.000000in}}%
\pgfpathclose%
\pgfusepath{stroke,fill}%
\end{pgfscope}%
\begin{pgfscope}%
\pgfpathrectangle{\pgfqpoint{0.647939in}{0.492442in}}{\pgfqpoint{4.273799in}{2.331163in}}%
\pgfusepath{clip}%
\pgfsetroundcap%
\pgfsetroundjoin%
\pgfsetlinewidth{0.301125pt}%
\definecolor{currentstroke}{rgb}{0.500000,0.500000,0.500000}%
\pgfsetstrokecolor{currentstroke}%
\pgfsetstrokeopacity{0.300000}%
\pgfsetdash{}{0pt}%
\pgfpathmoveto{\pgfqpoint{2.118883in}{0.983072in}}%
\pgfusepath{stroke}%
\end{pgfscope}%
\begin{pgfscope}%
\pgfpathrectangle{\pgfqpoint{0.647939in}{0.492442in}}{\pgfqpoint{4.273799in}{2.331163in}}%
\pgfusepath{clip}%
\pgfsetroundcap%
\pgfsetroundjoin%
\definecolor{currentfill}{rgb}{0.500000,0.500000,0.500000}%
\pgfsetfillcolor{currentfill}%
\pgfsetfillopacity{0.300000}%
\pgfsetlinewidth{0.301125pt}%
\definecolor{currentstroke}{rgb}{0.500000,0.500000,0.500000}%
\pgfsetstrokecolor{currentstroke}%
\pgfsetstrokeopacity{0.300000}%
\pgfsetdash{}{0pt}%
\pgfpathmoveto{\pgfqpoint{0.000000in}{0.000000in}}%
\pgfpathlineto{\pgfqpoint{0.000000in}{0.000000in}}%
\pgfpathclose%
\pgfusepath{stroke,fill}%
\end{pgfscope}%
\begin{pgfscope}%
\pgfpathrectangle{\pgfqpoint{0.647939in}{0.492442in}}{\pgfqpoint{4.273799in}{2.331163in}}%
\pgfusepath{clip}%
\pgfsetroundcap%
\pgfsetroundjoin%
\pgfsetlinewidth{0.301125pt}%
\definecolor{currentstroke}{rgb}{0.500000,0.500000,0.500000}%
\pgfsetstrokecolor{currentstroke}%
\pgfsetstrokeopacity{0.300000}%
\pgfsetdash{}{0pt}%
\pgfpathmoveto{\pgfqpoint{3.972747in}{1.191572in}}%
\pgfusepath{stroke}%
\end{pgfscope}%
\begin{pgfscope}%
\pgfpathrectangle{\pgfqpoint{0.647939in}{0.492442in}}{\pgfqpoint{4.273799in}{2.331163in}}%
\pgfusepath{clip}%
\pgfsetroundcap%
\pgfsetroundjoin%
\definecolor{currentfill}{rgb}{0.500000,0.500000,0.500000}%
\pgfsetfillcolor{currentfill}%
\pgfsetfillopacity{0.300000}%
\pgfsetlinewidth{0.301125pt}%
\definecolor{currentstroke}{rgb}{0.500000,0.500000,0.500000}%
\pgfsetstrokecolor{currentstroke}%
\pgfsetstrokeopacity{0.300000}%
\pgfsetdash{}{0pt}%
\pgfpathmoveto{\pgfqpoint{0.000000in}{0.000000in}}%
\pgfpathlineto{\pgfqpoint{0.000000in}{0.000000in}}%
\pgfpathclose%
\pgfusepath{stroke,fill}%
\end{pgfscope}%
\begin{pgfscope}%
\pgfpathrectangle{\pgfqpoint{0.647939in}{0.492442in}}{\pgfqpoint{4.273799in}{2.331163in}}%
\pgfusepath{clip}%
\pgfsetroundcap%
\pgfsetroundjoin%
\pgfsetlinewidth{0.301125pt}%
\definecolor{currentstroke}{rgb}{0.500000,0.500000,0.500000}%
\pgfsetstrokecolor{currentstroke}%
\pgfsetstrokeopacity{0.300000}%
\pgfsetdash{}{0pt}%
\pgfpathmoveto{\pgfqpoint{4.156708in}{1.445333in}}%
\pgfusepath{stroke}%
\end{pgfscope}%
\begin{pgfscope}%
\pgfpathrectangle{\pgfqpoint{0.647939in}{0.492442in}}{\pgfqpoint{4.273799in}{2.331163in}}%
\pgfusepath{clip}%
\pgfsetroundcap%
\pgfsetroundjoin%
\definecolor{currentfill}{rgb}{0.500000,0.500000,0.500000}%
\pgfsetfillcolor{currentfill}%
\pgfsetfillopacity{0.300000}%
\pgfsetlinewidth{0.301125pt}%
\definecolor{currentstroke}{rgb}{0.500000,0.500000,0.500000}%
\pgfsetstrokecolor{currentstroke}%
\pgfsetstrokeopacity{0.300000}%
\pgfsetdash{}{0pt}%
\pgfpathmoveto{\pgfqpoint{0.000000in}{0.000000in}}%
\pgfpathlineto{\pgfqpoint{0.000000in}{0.000000in}}%
\pgfpathclose%
\pgfusepath{stroke,fill}%
\end{pgfscope}%
\begin{pgfscope}%
\pgfpathrectangle{\pgfqpoint{0.647939in}{0.492442in}}{\pgfqpoint{4.273799in}{2.331163in}}%
\pgfusepath{clip}%
\pgfsetroundcap%
\pgfsetroundjoin%
\pgfsetlinewidth{0.301125pt}%
\definecolor{currentstroke}{rgb}{0.500000,0.500000,0.500000}%
\pgfsetstrokecolor{currentstroke}%
\pgfsetstrokeopacity{0.300000}%
\pgfsetdash{}{0pt}%
\pgfpathmoveto{\pgfqpoint{4.087425in}{1.611529in}}%
\pgfusepath{stroke}%
\end{pgfscope}%
\begin{pgfscope}%
\pgfpathrectangle{\pgfqpoint{0.647939in}{0.492442in}}{\pgfqpoint{4.273799in}{2.331163in}}%
\pgfusepath{clip}%
\pgfsetroundcap%
\pgfsetroundjoin%
\definecolor{currentfill}{rgb}{0.500000,0.500000,0.500000}%
\pgfsetfillcolor{currentfill}%
\pgfsetfillopacity{0.300000}%
\pgfsetlinewidth{0.301125pt}%
\definecolor{currentstroke}{rgb}{0.500000,0.500000,0.500000}%
\pgfsetstrokecolor{currentstroke}%
\pgfsetstrokeopacity{0.300000}%
\pgfsetdash{}{0pt}%
\pgfpathmoveto{\pgfqpoint{0.000000in}{0.000000in}}%
\pgfpathlineto{\pgfqpoint{0.000000in}{0.000000in}}%
\pgfpathclose%
\pgfusepath{stroke,fill}%
\end{pgfscope}%
\begin{pgfscope}%
\pgfpathrectangle{\pgfqpoint{0.647939in}{0.492442in}}{\pgfqpoint{4.273799in}{2.331163in}}%
\pgfusepath{clip}%
\pgfsetroundcap%
\pgfsetroundjoin%
\pgfsetlinewidth{0.301125pt}%
\definecolor{currentstroke}{rgb}{0.500000,0.500000,0.500000}%
\pgfsetstrokecolor{currentstroke}%
\pgfsetstrokeopacity{0.300000}%
\pgfsetdash{}{0pt}%
\pgfpathmoveto{\pgfqpoint{1.687306in}{2.391323in}}%
\pgfusepath{stroke}%
\end{pgfscope}%
\begin{pgfscope}%
\pgfpathrectangle{\pgfqpoint{0.647939in}{0.492442in}}{\pgfqpoint{4.273799in}{2.331163in}}%
\pgfusepath{clip}%
\pgfsetroundcap%
\pgfsetroundjoin%
\definecolor{currentfill}{rgb}{0.500000,0.500000,0.500000}%
\pgfsetfillcolor{currentfill}%
\pgfsetfillopacity{0.300000}%
\pgfsetlinewidth{0.301125pt}%
\definecolor{currentstroke}{rgb}{0.500000,0.500000,0.500000}%
\pgfsetstrokecolor{currentstroke}%
\pgfsetstrokeopacity{0.300000}%
\pgfsetdash{}{0pt}%
\pgfpathmoveto{\pgfqpoint{0.000000in}{0.000000in}}%
\pgfpathlineto{\pgfqpoint{0.000000in}{0.000000in}}%
\pgfpathclose%
\pgfusepath{stroke,fill}%
\end{pgfscope}%
\begin{pgfscope}%
\pgfpathrectangle{\pgfqpoint{0.647939in}{0.492442in}}{\pgfqpoint{4.273799in}{2.331163in}}%
\pgfusepath{clip}%
\pgfsetroundcap%
\pgfsetroundjoin%
\pgfsetlinewidth{0.301125pt}%
\definecolor{currentstroke}{rgb}{0.500000,0.500000,0.500000}%
\pgfsetstrokecolor{currentstroke}%
\pgfsetstrokeopacity{0.300000}%
\pgfsetdash{}{0pt}%
\pgfpathmoveto{\pgfqpoint{3.928856in}{1.096354in}}%
\pgfusepath{stroke}%
\end{pgfscope}%
\begin{pgfscope}%
\pgfpathrectangle{\pgfqpoint{0.647939in}{0.492442in}}{\pgfqpoint{4.273799in}{2.331163in}}%
\pgfusepath{clip}%
\pgfsetroundcap%
\pgfsetroundjoin%
\definecolor{currentfill}{rgb}{0.500000,0.500000,0.500000}%
\pgfsetfillcolor{currentfill}%
\pgfsetfillopacity{0.300000}%
\pgfsetlinewidth{0.301125pt}%
\definecolor{currentstroke}{rgb}{0.500000,0.500000,0.500000}%
\pgfsetstrokecolor{currentstroke}%
\pgfsetstrokeopacity{0.300000}%
\pgfsetdash{}{0pt}%
\pgfpathmoveto{\pgfqpoint{0.000000in}{0.000000in}}%
\pgfpathlineto{\pgfqpoint{0.000000in}{0.000000in}}%
\pgfpathclose%
\pgfusepath{stroke,fill}%
\end{pgfscope}%
\begin{pgfscope}%
\pgfpathrectangle{\pgfqpoint{0.647939in}{0.492442in}}{\pgfqpoint{4.273799in}{2.331163in}}%
\pgfusepath{clip}%
\pgfsetroundcap%
\pgfsetroundjoin%
\pgfsetlinewidth{0.301125pt}%
\definecolor{currentstroke}{rgb}{0.500000,0.500000,0.500000}%
\pgfsetstrokecolor{currentstroke}%
\pgfsetstrokeopacity{0.300000}%
\pgfsetdash{}{0pt}%
\pgfpathmoveto{\pgfqpoint{1.669601in}{1.628770in}}%
\pgfusepath{stroke}%
\end{pgfscope}%
\begin{pgfscope}%
\pgfpathrectangle{\pgfqpoint{0.647939in}{0.492442in}}{\pgfqpoint{4.273799in}{2.331163in}}%
\pgfusepath{clip}%
\pgfsetroundcap%
\pgfsetroundjoin%
\definecolor{currentfill}{rgb}{0.500000,0.500000,0.500000}%
\pgfsetfillcolor{currentfill}%
\pgfsetfillopacity{0.300000}%
\pgfsetlinewidth{0.301125pt}%
\definecolor{currentstroke}{rgb}{0.500000,0.500000,0.500000}%
\pgfsetstrokecolor{currentstroke}%
\pgfsetstrokeopacity{0.300000}%
\pgfsetdash{}{0pt}%
\pgfpathmoveto{\pgfqpoint{0.000000in}{0.000000in}}%
\pgfpathlineto{\pgfqpoint{0.000000in}{0.000000in}}%
\pgfpathclose%
\pgfusepath{stroke,fill}%
\end{pgfscope}%
\begin{pgfscope}%
\pgfpathrectangle{\pgfqpoint{0.647939in}{0.492442in}}{\pgfqpoint{4.273799in}{2.331163in}}%
\pgfusepath{clip}%
\pgfsetroundcap%
\pgfsetroundjoin%
\pgfsetlinewidth{0.301125pt}%
\definecolor{currentstroke}{rgb}{0.500000,0.500000,0.500000}%
\pgfsetstrokecolor{currentstroke}%
\pgfsetstrokeopacity{0.300000}%
\pgfsetdash{}{0pt}%
\pgfpathmoveto{\pgfqpoint{4.021084in}{1.401420in}}%
\pgfusepath{stroke}%
\end{pgfscope}%
\begin{pgfscope}%
\pgfpathrectangle{\pgfqpoint{0.647939in}{0.492442in}}{\pgfqpoint{4.273799in}{2.331163in}}%
\pgfusepath{clip}%
\pgfsetroundcap%
\pgfsetroundjoin%
\definecolor{currentfill}{rgb}{0.500000,0.500000,0.500000}%
\pgfsetfillcolor{currentfill}%
\pgfsetfillopacity{0.300000}%
\pgfsetlinewidth{0.301125pt}%
\definecolor{currentstroke}{rgb}{0.500000,0.500000,0.500000}%
\pgfsetstrokecolor{currentstroke}%
\pgfsetstrokeopacity{0.300000}%
\pgfsetdash{}{0pt}%
\pgfpathmoveto{\pgfqpoint{0.000000in}{0.000000in}}%
\pgfpathlineto{\pgfqpoint{0.000000in}{0.000000in}}%
\pgfpathclose%
\pgfusepath{stroke,fill}%
\end{pgfscope}%
\begin{pgfscope}%
\pgfpathrectangle{\pgfqpoint{0.647939in}{0.492442in}}{\pgfqpoint{4.273799in}{2.331163in}}%
\pgfusepath{clip}%
\pgfsetroundcap%
\pgfsetroundjoin%
\pgfsetlinewidth{0.301125pt}%
\definecolor{currentstroke}{rgb}{0.500000,0.500000,0.500000}%
\pgfsetstrokecolor{currentstroke}%
\pgfsetstrokeopacity{0.300000}%
\pgfsetdash{}{0pt}%
\pgfpathmoveto{\pgfqpoint{2.375919in}{2.161202in}}%
\pgfusepath{stroke}%
\end{pgfscope}%
\begin{pgfscope}%
\pgfpathrectangle{\pgfqpoint{0.647939in}{0.492442in}}{\pgfqpoint{4.273799in}{2.331163in}}%
\pgfusepath{clip}%
\pgfsetroundcap%
\pgfsetroundjoin%
\definecolor{currentfill}{rgb}{0.500000,0.500000,0.500000}%
\pgfsetfillcolor{currentfill}%
\pgfsetfillopacity{0.300000}%
\pgfsetlinewidth{0.301125pt}%
\definecolor{currentstroke}{rgb}{0.500000,0.500000,0.500000}%
\pgfsetstrokecolor{currentstroke}%
\pgfsetstrokeopacity{0.300000}%
\pgfsetdash{}{0pt}%
\pgfpathmoveto{\pgfqpoint{0.000000in}{0.000000in}}%
\pgfpathlineto{\pgfqpoint{0.000000in}{0.000000in}}%
\pgfpathclose%
\pgfusepath{stroke,fill}%
\end{pgfscope}%
\begin{pgfscope}%
\pgfpathrectangle{\pgfqpoint{0.647939in}{0.492442in}}{\pgfqpoint{4.273799in}{2.331163in}}%
\pgfusepath{clip}%
\pgfsetroundcap%
\pgfsetroundjoin%
\pgfsetlinewidth{0.301125pt}%
\definecolor{currentstroke}{rgb}{0.500000,0.500000,0.500000}%
\pgfsetstrokecolor{currentstroke}%
\pgfsetstrokeopacity{0.300000}%
\pgfsetdash{}{0pt}%
\pgfpathmoveto{\pgfqpoint{1.623728in}{2.176719in}}%
\pgfusepath{stroke}%
\end{pgfscope}%
\begin{pgfscope}%
\pgfpathrectangle{\pgfqpoint{0.647939in}{0.492442in}}{\pgfqpoint{4.273799in}{2.331163in}}%
\pgfusepath{clip}%
\pgfsetroundcap%
\pgfsetroundjoin%
\definecolor{currentfill}{rgb}{0.500000,0.500000,0.500000}%
\pgfsetfillcolor{currentfill}%
\pgfsetfillopacity{0.300000}%
\pgfsetlinewidth{0.301125pt}%
\definecolor{currentstroke}{rgb}{0.500000,0.500000,0.500000}%
\pgfsetstrokecolor{currentstroke}%
\pgfsetstrokeopacity{0.300000}%
\pgfsetdash{}{0pt}%
\pgfpathmoveto{\pgfqpoint{0.000000in}{0.000000in}}%
\pgfpathlineto{\pgfqpoint{0.000000in}{0.000000in}}%
\pgfpathclose%
\pgfusepath{stroke,fill}%
\end{pgfscope}%
\begin{pgfscope}%
\pgfpathrectangle{\pgfqpoint{0.647939in}{0.492442in}}{\pgfqpoint{4.273799in}{2.331163in}}%
\pgfusepath{clip}%
\pgfsetroundcap%
\pgfsetroundjoin%
\pgfsetlinewidth{0.301125pt}%
\definecolor{currentstroke}{rgb}{0.500000,0.500000,0.500000}%
\pgfsetstrokecolor{currentstroke}%
\pgfsetstrokeopacity{0.300000}%
\pgfsetdash{}{0pt}%
\pgfpathmoveto{\pgfqpoint{1.560179in}{2.139358in}}%
\pgfusepath{stroke}%
\end{pgfscope}%
\begin{pgfscope}%
\pgfpathrectangle{\pgfqpoint{0.647939in}{0.492442in}}{\pgfqpoint{4.273799in}{2.331163in}}%
\pgfusepath{clip}%
\pgfsetroundcap%
\pgfsetroundjoin%
\definecolor{currentfill}{rgb}{0.500000,0.500000,0.500000}%
\pgfsetfillcolor{currentfill}%
\pgfsetfillopacity{0.300000}%
\pgfsetlinewidth{0.301125pt}%
\definecolor{currentstroke}{rgb}{0.500000,0.500000,0.500000}%
\pgfsetstrokecolor{currentstroke}%
\pgfsetstrokeopacity{0.300000}%
\pgfsetdash{}{0pt}%
\pgfpathmoveto{\pgfqpoint{0.000000in}{0.000000in}}%
\pgfpathlineto{\pgfqpoint{0.000000in}{0.000000in}}%
\pgfpathclose%
\pgfusepath{stroke,fill}%
\end{pgfscope}%
\begin{pgfscope}%
\pgfpathrectangle{\pgfqpoint{0.647939in}{0.492442in}}{\pgfqpoint{4.273799in}{2.331163in}}%
\pgfusepath{clip}%
\pgfsetroundcap%
\pgfsetroundjoin%
\pgfsetlinewidth{0.301125pt}%
\definecolor{currentstroke}{rgb}{0.500000,0.500000,0.500000}%
\pgfsetstrokecolor{currentstroke}%
\pgfsetstrokeopacity{0.300000}%
\pgfsetdash{}{0pt}%
\pgfpathmoveto{\pgfqpoint{1.522125in}{1.499081in}}%
\pgfusepath{stroke}%
\end{pgfscope}%
\begin{pgfscope}%
\pgfpathrectangle{\pgfqpoint{0.647939in}{0.492442in}}{\pgfqpoint{4.273799in}{2.331163in}}%
\pgfusepath{clip}%
\pgfsetroundcap%
\pgfsetroundjoin%
\definecolor{currentfill}{rgb}{0.500000,0.500000,0.500000}%
\pgfsetfillcolor{currentfill}%
\pgfsetfillopacity{0.300000}%
\pgfsetlinewidth{0.301125pt}%
\definecolor{currentstroke}{rgb}{0.500000,0.500000,0.500000}%
\pgfsetstrokecolor{currentstroke}%
\pgfsetstrokeopacity{0.300000}%
\pgfsetdash{}{0pt}%
\pgfpathmoveto{\pgfqpoint{0.000000in}{0.000000in}}%
\pgfpathlineto{\pgfqpoint{0.000000in}{0.000000in}}%
\pgfpathclose%
\pgfusepath{stroke,fill}%
\end{pgfscope}%
\begin{pgfscope}%
\pgfpathrectangle{\pgfqpoint{0.647939in}{0.492442in}}{\pgfqpoint{4.273799in}{2.331163in}}%
\pgfusepath{clip}%
\pgfsetroundcap%
\pgfsetroundjoin%
\pgfsetlinewidth{0.301125pt}%
\definecolor{currentstroke}{rgb}{0.500000,0.500000,0.500000}%
\pgfsetstrokecolor{currentstroke}%
\pgfsetstrokeopacity{0.300000}%
\pgfsetdash{}{0pt}%
\pgfpathmoveto{\pgfqpoint{3.694574in}{2.139672in}}%
\pgfusepath{stroke}%
\end{pgfscope}%
\begin{pgfscope}%
\pgfpathrectangle{\pgfqpoint{0.647939in}{0.492442in}}{\pgfqpoint{4.273799in}{2.331163in}}%
\pgfusepath{clip}%
\pgfsetroundcap%
\pgfsetroundjoin%
\definecolor{currentfill}{rgb}{0.500000,0.500000,0.500000}%
\pgfsetfillcolor{currentfill}%
\pgfsetfillopacity{0.300000}%
\pgfsetlinewidth{0.301125pt}%
\definecolor{currentstroke}{rgb}{0.500000,0.500000,0.500000}%
\pgfsetstrokecolor{currentstroke}%
\pgfsetstrokeopacity{0.300000}%
\pgfsetdash{}{0pt}%
\pgfpathmoveto{\pgfqpoint{0.000000in}{0.000000in}}%
\pgfpathlineto{\pgfqpoint{0.000000in}{0.000000in}}%
\pgfpathclose%
\pgfusepath{stroke,fill}%
\end{pgfscope}%
\begin{pgfscope}%
\pgfpathrectangle{\pgfqpoint{0.647939in}{0.492442in}}{\pgfqpoint{4.273799in}{2.331163in}}%
\pgfusepath{clip}%
\pgfsetroundcap%
\pgfsetroundjoin%
\pgfsetlinewidth{0.301125pt}%
\definecolor{currentstroke}{rgb}{0.500000,0.500000,0.500000}%
\pgfsetstrokecolor{currentstroke}%
\pgfsetstrokeopacity{0.300000}%
\pgfsetdash{}{0pt}%
\pgfpathmoveto{\pgfqpoint{3.494675in}{2.191511in}}%
\pgfusepath{stroke}%
\end{pgfscope}%
\begin{pgfscope}%
\pgfpathrectangle{\pgfqpoint{0.647939in}{0.492442in}}{\pgfqpoint{4.273799in}{2.331163in}}%
\pgfusepath{clip}%
\pgfsetroundcap%
\pgfsetroundjoin%
\definecolor{currentfill}{rgb}{0.500000,0.500000,0.500000}%
\pgfsetfillcolor{currentfill}%
\pgfsetfillopacity{0.300000}%
\pgfsetlinewidth{0.301125pt}%
\definecolor{currentstroke}{rgb}{0.500000,0.500000,0.500000}%
\pgfsetstrokecolor{currentstroke}%
\pgfsetstrokeopacity{0.300000}%
\pgfsetdash{}{0pt}%
\pgfpathmoveto{\pgfqpoint{0.000000in}{0.000000in}}%
\pgfpathlineto{\pgfqpoint{0.000000in}{0.000000in}}%
\pgfpathclose%
\pgfusepath{stroke,fill}%
\end{pgfscope}%
\begin{pgfscope}%
\pgfpathrectangle{\pgfqpoint{0.647939in}{0.492442in}}{\pgfqpoint{4.273799in}{2.331163in}}%
\pgfusepath{clip}%
\pgfsetroundcap%
\pgfsetroundjoin%
\pgfsetlinewidth{0.301125pt}%
\definecolor{currentstroke}{rgb}{0.500000,0.500000,0.500000}%
\pgfsetstrokecolor{currentstroke}%
\pgfsetstrokeopacity{0.300000}%
\pgfsetdash{}{0pt}%
\pgfpathmoveto{\pgfqpoint{1.887808in}{2.047546in}}%
\pgfusepath{stroke}%
\end{pgfscope}%
\begin{pgfscope}%
\pgfpathrectangle{\pgfqpoint{0.647939in}{0.492442in}}{\pgfqpoint{4.273799in}{2.331163in}}%
\pgfusepath{clip}%
\pgfsetroundcap%
\pgfsetroundjoin%
\definecolor{currentfill}{rgb}{0.500000,0.500000,0.500000}%
\pgfsetfillcolor{currentfill}%
\pgfsetfillopacity{0.300000}%
\pgfsetlinewidth{0.301125pt}%
\definecolor{currentstroke}{rgb}{0.500000,0.500000,0.500000}%
\pgfsetstrokecolor{currentstroke}%
\pgfsetstrokeopacity{0.300000}%
\pgfsetdash{}{0pt}%
\pgfpathmoveto{\pgfqpoint{0.000000in}{0.000000in}}%
\pgfpathlineto{\pgfqpoint{0.000000in}{0.000000in}}%
\pgfpathclose%
\pgfusepath{stroke,fill}%
\end{pgfscope}%
\begin{pgfscope}%
\pgfpathrectangle{\pgfqpoint{0.647939in}{0.492442in}}{\pgfqpoint{4.273799in}{2.331163in}}%
\pgfusepath{clip}%
\pgfsetroundcap%
\pgfsetroundjoin%
\pgfsetlinewidth{0.301125pt}%
\definecolor{currentstroke}{rgb}{0.500000,0.500000,0.500000}%
\pgfsetstrokecolor{currentstroke}%
\pgfsetstrokeopacity{0.300000}%
\pgfsetdash{}{0pt}%
\pgfpathmoveto{\pgfqpoint{2.702705in}{2.146436in}}%
\pgfusepath{stroke}%
\end{pgfscope}%
\begin{pgfscope}%
\pgfpathrectangle{\pgfqpoint{0.647939in}{0.492442in}}{\pgfqpoint{4.273799in}{2.331163in}}%
\pgfusepath{clip}%
\pgfsetroundcap%
\pgfsetroundjoin%
\definecolor{currentfill}{rgb}{0.500000,0.500000,0.500000}%
\pgfsetfillcolor{currentfill}%
\pgfsetfillopacity{0.300000}%
\pgfsetlinewidth{0.301125pt}%
\definecolor{currentstroke}{rgb}{0.500000,0.500000,0.500000}%
\pgfsetstrokecolor{currentstroke}%
\pgfsetstrokeopacity{0.300000}%
\pgfsetdash{}{0pt}%
\pgfpathmoveto{\pgfqpoint{0.000000in}{0.000000in}}%
\pgfpathlineto{\pgfqpoint{0.000000in}{0.000000in}}%
\pgfpathclose%
\pgfusepath{stroke,fill}%
\end{pgfscope}%
\begin{pgfscope}%
\pgfpathrectangle{\pgfqpoint{0.647939in}{0.492442in}}{\pgfqpoint{4.273799in}{2.331163in}}%
\pgfusepath{clip}%
\pgfsetroundcap%
\pgfsetroundjoin%
\pgfsetlinewidth{0.301125pt}%
\definecolor{currentstroke}{rgb}{0.500000,0.500000,0.500000}%
\pgfsetstrokecolor{currentstroke}%
\pgfsetstrokeopacity{0.300000}%
\pgfsetdash{}{0pt}%
\pgfpathmoveto{\pgfqpoint{2.277236in}{2.033294in}}%
\pgfusepath{stroke}%
\end{pgfscope}%
\begin{pgfscope}%
\pgfpathrectangle{\pgfqpoint{0.647939in}{0.492442in}}{\pgfqpoint{4.273799in}{2.331163in}}%
\pgfusepath{clip}%
\pgfsetroundcap%
\pgfsetroundjoin%
\definecolor{currentfill}{rgb}{0.500000,0.500000,0.500000}%
\pgfsetfillcolor{currentfill}%
\pgfsetfillopacity{0.300000}%
\pgfsetlinewidth{0.301125pt}%
\definecolor{currentstroke}{rgb}{0.500000,0.500000,0.500000}%
\pgfsetstrokecolor{currentstroke}%
\pgfsetstrokeopacity{0.300000}%
\pgfsetdash{}{0pt}%
\pgfpathmoveto{\pgfqpoint{0.000000in}{0.000000in}}%
\pgfpathlineto{\pgfqpoint{0.000000in}{0.000000in}}%
\pgfpathclose%
\pgfusepath{stroke,fill}%
\end{pgfscope}%
\begin{pgfscope}%
\pgfpathrectangle{\pgfqpoint{0.647939in}{0.492442in}}{\pgfqpoint{4.273799in}{2.331163in}}%
\pgfusepath{clip}%
\pgfsetroundcap%
\pgfsetroundjoin%
\pgfsetlinewidth{0.301125pt}%
\definecolor{currentstroke}{rgb}{0.500000,0.500000,0.500000}%
\pgfsetstrokecolor{currentstroke}%
\pgfsetstrokeopacity{0.300000}%
\pgfsetdash{}{0pt}%
\pgfpathmoveto{\pgfqpoint{2.762654in}{2.008334in}}%
\pgfusepath{stroke}%
\end{pgfscope}%
\begin{pgfscope}%
\pgfpathrectangle{\pgfqpoint{0.647939in}{0.492442in}}{\pgfqpoint{4.273799in}{2.331163in}}%
\pgfusepath{clip}%
\pgfsetroundcap%
\pgfsetroundjoin%
\definecolor{currentfill}{rgb}{0.500000,0.500000,0.500000}%
\pgfsetfillcolor{currentfill}%
\pgfsetfillopacity{0.300000}%
\pgfsetlinewidth{0.301125pt}%
\definecolor{currentstroke}{rgb}{0.500000,0.500000,0.500000}%
\pgfsetstrokecolor{currentstroke}%
\pgfsetstrokeopacity{0.300000}%
\pgfsetdash{}{0pt}%
\pgfpathmoveto{\pgfqpoint{0.000000in}{0.000000in}}%
\pgfpathlineto{\pgfqpoint{0.000000in}{0.000000in}}%
\pgfpathclose%
\pgfusepath{stroke,fill}%
\end{pgfscope}%
\begin{pgfscope}%
\pgfpathrectangle{\pgfqpoint{0.647939in}{0.492442in}}{\pgfqpoint{4.273799in}{2.331163in}}%
\pgfusepath{clip}%
\pgfsetroundcap%
\pgfsetroundjoin%
\pgfsetlinewidth{0.301125pt}%
\definecolor{currentstroke}{rgb}{0.500000,0.500000,0.500000}%
\pgfsetstrokecolor{currentstroke}%
\pgfsetstrokeopacity{0.300000}%
\pgfsetdash{}{0pt}%
\pgfpathmoveto{\pgfqpoint{2.066596in}{1.585411in}}%
\pgfusepath{stroke}%
\end{pgfscope}%
\begin{pgfscope}%
\pgfpathrectangle{\pgfqpoint{0.647939in}{0.492442in}}{\pgfqpoint{4.273799in}{2.331163in}}%
\pgfusepath{clip}%
\pgfsetroundcap%
\pgfsetroundjoin%
\definecolor{currentfill}{rgb}{0.500000,0.500000,0.500000}%
\pgfsetfillcolor{currentfill}%
\pgfsetfillopacity{0.300000}%
\pgfsetlinewidth{0.301125pt}%
\definecolor{currentstroke}{rgb}{0.500000,0.500000,0.500000}%
\pgfsetstrokecolor{currentstroke}%
\pgfsetstrokeopacity{0.300000}%
\pgfsetdash{}{0pt}%
\pgfpathmoveto{\pgfqpoint{0.000000in}{0.000000in}}%
\pgfpathlineto{\pgfqpoint{0.000000in}{0.000000in}}%
\pgfpathclose%
\pgfusepath{stroke,fill}%
\end{pgfscope}%
\begin{pgfscope}%
\pgfpathrectangle{\pgfqpoint{0.647939in}{0.492442in}}{\pgfqpoint{4.273799in}{2.331163in}}%
\pgfusepath{clip}%
\pgfsetroundcap%
\pgfsetroundjoin%
\pgfsetlinewidth{0.301125pt}%
\definecolor{currentstroke}{rgb}{0.500000,0.500000,0.500000}%
\pgfsetstrokecolor{currentstroke}%
\pgfsetstrokeopacity{0.300000}%
\pgfsetdash{}{0pt}%
\pgfpathmoveto{\pgfqpoint{2.541610in}{1.505327in}}%
\pgfusepath{stroke}%
\end{pgfscope}%
\begin{pgfscope}%
\pgfpathrectangle{\pgfqpoint{0.647939in}{0.492442in}}{\pgfqpoint{4.273799in}{2.331163in}}%
\pgfusepath{clip}%
\pgfsetroundcap%
\pgfsetroundjoin%
\definecolor{currentfill}{rgb}{0.500000,0.500000,0.500000}%
\pgfsetfillcolor{currentfill}%
\pgfsetfillopacity{0.300000}%
\pgfsetlinewidth{0.301125pt}%
\definecolor{currentstroke}{rgb}{0.500000,0.500000,0.500000}%
\pgfsetstrokecolor{currentstroke}%
\pgfsetstrokeopacity{0.300000}%
\pgfsetdash{}{0pt}%
\pgfpathmoveto{\pgfqpoint{0.000000in}{0.000000in}}%
\pgfpathlineto{\pgfqpoint{0.000000in}{0.000000in}}%
\pgfpathclose%
\pgfusepath{stroke,fill}%
\end{pgfscope}%
\begin{pgfscope}%
\pgfpathrectangle{\pgfqpoint{0.647939in}{0.492442in}}{\pgfqpoint{4.273799in}{2.331163in}}%
\pgfusepath{clip}%
\pgfsetbuttcap%
\pgfsetroundjoin%
\pgfsetlinewidth{0.301125pt}%
\definecolor{currentstroke}{rgb}{0.500000,0.500000,0.500000}%
\pgfsetstrokecolor{currentstroke}%
\pgfsetstrokeopacity{0.300000}%
\pgfsetdash{}{0pt}%
\pgfpathmoveto{\pgfqpoint{2.256982in}{0.492442in}}%
\pgfpathlineto{\pgfqpoint{2.250549in}{0.501102in}}%
\pgfpathlineto{\pgfqpoint{2.215030in}{0.549146in}}%
\pgfpathlineto{\pgfqpoint{2.179786in}{0.597250in}}%
\pgfpathlineto{\pgfqpoint{2.144782in}{0.645407in}}%
\pgfpathlineto{\pgfqpoint{2.109985in}{0.693608in}}%
\pgfpathlineto{\pgfqpoint{2.075352in}{0.741845in}}%
\pgfpathlineto{\pgfqpoint{2.040831in}{0.790105in}}%
\pgfpathlineto{\pgfqpoint{2.006363in}{0.838377in}}%
\pgfpathlineto{\pgfqpoint{1.971890in}{0.886647in}}%
\pgfpathlineto{\pgfqpoint{1.937340in}{0.934901in}}%
\pgfpathlineto{\pgfqpoint{1.902623in}{0.983119in}}%
\pgfpathlineto{\pgfqpoint{1.867621in}{1.031276in}}%
\pgfpathlineto{\pgfqpoint{1.832184in}{1.079337in}}%
\pgfpathlineto{\pgfqpoint{1.796129in}{1.127260in}}%
\pgfpathlineto{\pgfqpoint{1.759206in}{1.174985in}}%
\pgfpathlineto{\pgfqpoint{1.721060in}{1.222422in}}%
\pgfpathlineto{\pgfqpoint{1.681159in}{1.269425in}}%
\pgfpathlineto{\pgfqpoint{1.638651in}{1.315737in}}%
\pgfpathlineto{\pgfqpoint{1.592046in}{1.360833in}}%
\pgfpathlineto{\pgfqpoint{1.543366in}{1.400017in}}%
\pgfpathlineto{\pgfqpoint{1.501124in}{1.426220in}}%
\pgfpathlineto{\pgfqpoint{1.462577in}{1.442904in}}%
\pgfpathlineto{\pgfqpoint{1.421507in}{1.452373in}}%
\pgfpathlineto{\pgfqpoint{1.376965in}{1.452503in}}%
\pgfpathlineto{\pgfqpoint{1.376965in}{1.452503in}}%
\pgfpathlineto{\pgfqpoint{1.328861in}{1.440747in}}%
\pgfpathlineto{\pgfqpoint{1.328861in}{1.440747in}}%
\pgfpathlineto{\pgfqpoint{1.271043in}{1.411189in}}%
\pgfpathlineto{\pgfqpoint{1.271043in}{1.411189in}}%
\pgfpathlineto{\pgfqpoint{1.215915in}{1.369315in}}%
\pgfpathlineto{\pgfqpoint{1.168762in}{1.324424in}}%
\pgfpathlineto{\pgfqpoint{1.126530in}{1.278056in}}%
\pgfpathlineto{\pgfqpoint{1.087642in}{1.230823in}}%
\pgfpathlineto{\pgfqpoint{1.051185in}{1.183020in}}%
\pgfpathlineto{\pgfqpoint{1.016584in}{1.134806in}}%
\pgfpathlineto{\pgfqpoint{0.983451in}{1.086279in}}%
\pgfpathlineto{\pgfqpoint{0.951517in}{1.037501in}}%
\pgfpathlineto{\pgfqpoint{0.920616in}{0.988525in}}%
\pgfpathlineto{\pgfqpoint{0.890617in}{0.939387in}}%
\pgfpathlineto{\pgfqpoint{0.861387in}{0.890105in}}%
\pgfpathlineto{\pgfqpoint{0.832849in}{0.840702in}}%
\pgfpathlineto{\pgfqpoint{0.804943in}{0.791194in}}%
\pgfpathlineto{\pgfqpoint{0.777595in}{0.741589in}}%
\pgfpathlineto{\pgfqpoint{0.750766in}{0.691900in}}%
\pgfpathlineto{\pgfqpoint{0.724424in}{0.642135in}}%
\pgfpathlineto{\pgfqpoint{0.698523in}{0.592300in}}%
\pgfpathlineto{\pgfqpoint{0.673039in}{0.542401in}}%
\pgfpathlineto{\pgfqpoint{0.647939in}{0.492442in}}%
\pgfpathlineto{\pgfqpoint{0.647939in}{0.492442in}}%
\pgfusepath{stroke}%
\end{pgfscope}%
\begin{pgfscope}%
\pgfpathrectangle{\pgfqpoint{0.647939in}{0.492442in}}{\pgfqpoint{4.273799in}{2.331163in}}%
\pgfusepath{clip}%
\pgfsetbuttcap%
\pgfsetroundjoin%
\pgfsetlinewidth{0.301125pt}%
\definecolor{currentstroke}{rgb}{0.500000,0.500000,0.500000}%
\pgfsetstrokecolor{currentstroke}%
\pgfsetstrokeopacity{0.300000}%
\pgfsetdash{}{0pt}%
\pgfpathmoveto{\pgfqpoint{1.804555in}{0.492442in}}%
\pgfpathlineto{\pgfqpoint{1.804436in}{0.492584in}}%
\pgfpathlineto{\pgfqpoint{1.764498in}{0.539583in}}%
\pgfpathlineto{\pgfqpoint{1.723592in}{0.586334in}}%
\pgfpathlineto{\pgfqpoint{1.681419in}{0.632747in}}%
\pgfpathlineto{\pgfqpoint{1.637575in}{0.678695in}}%
\pgfpathlineto{\pgfqpoint{1.591484in}{0.723979in}}%
\pgfpathlineto{\pgfqpoint{1.542293in}{0.768268in}}%
\pgfpathlineto{\pgfqpoint{1.488659in}{0.810961in}}%
\pgfpathlineto{\pgfqpoint{1.429445in}{0.850191in}}%
\pgfpathlineto{\pgfqpoint{1.376471in}{0.877301in}}%
\pgfpathlineto{\pgfqpoint{1.326682in}{0.895104in}}%
\pgfpathlineto{\pgfqpoint{1.275039in}{0.905024in}}%
\pgfpathlineto{\pgfqpoint{1.213940in}{0.904868in}}%
\pgfpathlineto{\pgfqpoint{1.156737in}{0.892727in}}%
\pgfpathlineto{\pgfqpoint{1.156737in}{0.892727in}}%
\pgfpathlineto{\pgfqpoint{1.083988in}{0.860302in}}%
\pgfpathlineto{\pgfqpoint{1.024287in}{0.820201in}}%
\pgfpathlineto{\pgfqpoint{0.973187in}{0.776658in}}%
\pgfpathlineto{\pgfqpoint{0.927758in}{0.731237in}}%
\pgfpathlineto{\pgfqpoint{0.886353in}{0.684661in}}%
\pgfpathlineto{\pgfqpoint{0.847955in}{0.637311in}}%
\pgfpathlineto{\pgfqpoint{0.811894in}{0.589407in}}%
\pgfpathlineto{\pgfqpoint{0.777704in}{0.541085in}}%
\pgfpathlineto{\pgfqpoint{0.745071in}{0.492442in}}%
\pgfpathlineto{\pgfqpoint{0.745071in}{0.492442in}}%
\pgfusepath{stroke}%
\end{pgfscope}%
\begin{pgfscope}%
\pgfpathrectangle{\pgfqpoint{0.647939in}{0.492442in}}{\pgfqpoint{4.273799in}{2.331163in}}%
\pgfusepath{clip}%
\pgfsetbuttcap%
\pgfsetroundjoin%
\pgfsetlinewidth{0.301125pt}%
\definecolor{currentstroke}{rgb}{0.500000,0.500000,0.500000}%
\pgfsetstrokecolor{currentstroke}%
\pgfsetstrokeopacity{0.300000}%
\pgfsetdash{}{0pt}%
\pgfpathmoveto{\pgfqpoint{1.572232in}{0.492442in}}%
\pgfpathlineto{\pgfqpoint{1.568207in}{0.496146in}}%
\pgfpathlineto{\pgfqpoint{1.518123in}{0.540140in}}%
\pgfpathlineto{\pgfqpoint{1.463952in}{0.582650in}}%
\pgfpathlineto{\pgfqpoint{1.403861in}{0.622650in}}%
\pgfpathlineto{\pgfqpoint{1.346343in}{0.652937in}}%
\pgfpathlineto{\pgfqpoint{1.293065in}{0.673240in}}%
\pgfpathlineto{\pgfqpoint{1.239719in}{0.685413in}}%
\pgfpathlineto{\pgfqpoint{1.180187in}{0.688681in}}%
\pgfpathlineto{\pgfqpoint{1.121833in}{0.680925in}}%
\pgfpathlineto{\pgfqpoint{1.121833in}{0.680925in}}%
\pgfpathlineto{\pgfqpoint{1.059260in}{0.660208in}}%
\pgfpathlineto{\pgfqpoint{1.059260in}{0.660208in}}%
\pgfpathlineto{\pgfqpoint{0.992268in}{0.623910in}}%
\pgfpathlineto{\pgfqpoint{0.935847in}{0.582383in}}%
\pgfpathlineto{\pgfqpoint{0.886547in}{0.538197in}}%
\pgfpathlineto{\pgfqpoint{0.842203in}{0.492442in}}%
\pgfpathlineto{\pgfqpoint{0.842203in}{0.492442in}}%
\pgfusepath{stroke}%
\end{pgfscope}%
\begin{pgfscope}%
\pgfpathrectangle{\pgfqpoint{0.647939in}{0.492442in}}{\pgfqpoint{4.273799in}{2.331163in}}%
\pgfusepath{clip}%
\pgfsetbuttcap%
\pgfsetroundjoin%
\pgfsetlinewidth{0.301125pt}%
\definecolor{currentstroke}{rgb}{0.500000,0.500000,0.500000}%
\pgfsetstrokecolor{currentstroke}%
\pgfsetstrokeopacity{0.300000}%
\pgfsetdash{}{0pt}%
\pgfpathmoveto{\pgfqpoint{1.406902in}{0.492442in}}%
\pgfpathlineto{\pgfqpoint{1.369872in}{0.513079in}}%
\pgfpathlineto{\pgfqpoint{1.312024in}{0.541158in}}%
\pgfpathlineto{\pgfqpoint{1.257671in}{0.559745in}}%
\pgfpathlineto{\pgfqpoint{1.202161in}{0.570275in}}%
\pgfpathlineto{\pgfqpoint{1.139084in}{0.571177in}}%
\pgfpathlineto{\pgfqpoint{1.078974in}{0.560679in}}%
\pgfpathlineto{\pgfqpoint{1.078974in}{0.560679in}}%
\pgfpathlineto{\pgfqpoint{1.002583in}{0.530832in}}%
\pgfpathlineto{\pgfqpoint{0.939334in}{0.492442in}}%
\pgfpathlineto{\pgfqpoint{0.939334in}{0.492442in}}%
\pgfusepath{stroke}%
\end{pgfscope}%
\begin{pgfscope}%
\pgfpathrectangle{\pgfqpoint{0.647939in}{0.492442in}}{\pgfqpoint{4.273799in}{2.331163in}}%
\pgfusepath{clip}%
\pgfsetbuttcap%
\pgfsetroundjoin%
\pgfsetlinewidth{0.301125pt}%
\definecolor{currentstroke}{rgb}{0.500000,0.500000,0.500000}%
\pgfsetstrokecolor{currentstroke}%
\pgfsetstrokeopacity{0.300000}%
\pgfsetdash{}{0pt}%
\pgfpathmoveto{\pgfqpoint{1.910652in}{0.492442in}}%
\pgfpathlineto{\pgfqpoint{1.910652in}{0.492442in}}%
\pgfpathlineto{\pgfqpoint{1.872886in}{0.539973in}}%
\pgfpathlineto{\pgfqpoint{1.834652in}{0.587393in}}%
\pgfpathlineto{\pgfqpoint{1.795796in}{0.634661in}}%
\pgfpathlineto{\pgfqpoint{1.756122in}{0.681726in}}%
\pgfpathlineto{\pgfqpoint{1.715370in}{0.728515in}}%
\pgfpathlineto{\pgfqpoint{1.673192in}{0.774924in}}%
\pgfpathlineto{\pgfqpoint{1.629096in}{0.820797in}}%
\pgfpathlineto{\pgfqpoint{1.582360in}{0.865876in}}%
\pgfpathlineto{\pgfqpoint{1.531863in}{0.909712in}}%
\pgfpathlineto{\pgfqpoint{1.475744in}{0.951418in}}%
\pgfpathlineto{\pgfqpoint{1.410715in}{0.988898in}}%
\pgfpathlineto{\pgfqpoint{1.410715in}{0.988898in}}%
\pgfpathlineto{\pgfqpoint{1.349701in}{1.011917in}}%
\pgfpathlineto{\pgfqpoint{1.349701in}{1.011917in}}%
\pgfpathlineto{\pgfqpoint{1.295611in}{1.021496in}}%
\pgfpathlineto{\pgfqpoint{1.238267in}{1.019766in}}%
\pgfpathlineto{\pgfqpoint{1.191290in}{1.009125in}}%
\pgfpathlineto{\pgfqpoint{1.146065in}{0.991046in}}%
\pgfpathlineto{\pgfqpoint{1.098674in}{0.964049in}}%
\pgfpathlineto{\pgfqpoint{1.047008in}{0.925641in}}%
\pgfusepath{stroke}%
\end{pgfscope}%
\begin{pgfscope}%
\pgfpathrectangle{\pgfqpoint{0.647939in}{0.492442in}}{\pgfqpoint{4.273799in}{2.331163in}}%
\pgfusepath{clip}%
\pgfsetbuttcap%
\pgfsetroundjoin%
\pgfsetlinewidth{0.301125pt}%
\definecolor{currentstroke}{rgb}{0.500000,0.500000,0.500000}%
\pgfsetstrokecolor{currentstroke}%
\pgfsetstrokeopacity{0.300000}%
\pgfsetdash{}{0pt}%
\pgfpathmoveto{\pgfqpoint{2.007784in}{0.492442in}}%
\pgfpathlineto{\pgfqpoint{2.007784in}{0.492442in}}%
\pgfpathlineto{\pgfqpoint{1.971260in}{0.540262in}}%
\pgfpathlineto{\pgfqpoint{1.934572in}{0.588044in}}%
\pgfpathlineto{\pgfqpoint{1.897629in}{0.635767in}}%
\pgfpathlineto{\pgfqpoint{1.860315in}{0.683405in}}%
\pgfpathlineto{\pgfqpoint{1.822485in}{0.730921in}}%
\pgfpathlineto{\pgfqpoint{1.783959in}{0.778269in}}%
\pgfpathlineto{\pgfqpoint{1.744509in}{0.825389in}}%
\pgfpathlineto{\pgfqpoint{1.703824in}{0.872195in}}%
\pgfpathlineto{\pgfqpoint{1.661475in}{0.918555in}}%
\pgfpathlineto{\pgfqpoint{1.616832in}{0.964264in}}%
\pgfpathlineto{\pgfqpoint{1.568923in}{1.008965in}}%
\pgfpathlineto{\pgfqpoint{1.516138in}{1.051973in}}%
\pgfpathlineto{\pgfqpoint{1.455580in}{1.091700in}}%
\pgfpathlineto{\pgfqpoint{1.382043in}{1.123611in}}%
\pgfpathlineto{\pgfqpoint{1.382043in}{1.123611in}}%
\pgfpathlineto{\pgfqpoint{1.330018in}{1.134166in}}%
\pgfpathlineto{\pgfqpoint{1.273845in}{1.133371in}}%
\pgfpathlineto{\pgfqpoint{1.227979in}{1.123318in}}%
\pgfpathlineto{\pgfqpoint{1.184402in}{1.105983in}}%
\pgfpathlineto{\pgfqpoint{1.138581in}{1.079820in}}%
\pgfpathlineto{\pgfqpoint{1.088333in}{1.042280in}}%
\pgfpathlineto{\pgfqpoint{1.039771in}{0.997854in}}%
\pgfpathlineto{\pgfqpoint{0.996293in}{0.951833in}}%
\pgfpathlineto{\pgfqpoint{0.956425in}{0.904840in}}%
\pgfpathlineto{\pgfqpoint{0.919256in}{0.857193in}}%
\pgfusepath{stroke}%
\end{pgfscope}%
\begin{pgfscope}%
\pgfpathrectangle{\pgfqpoint{0.647939in}{0.492442in}}{\pgfqpoint{4.273799in}{2.331163in}}%
\pgfusepath{clip}%
\pgfsetbuttcap%
\pgfsetroundjoin%
\pgfsetlinewidth{0.301125pt}%
\definecolor{currentstroke}{rgb}{0.500000,0.500000,0.500000}%
\pgfsetstrokecolor{currentstroke}%
\pgfsetstrokeopacity{0.300000}%
\pgfsetdash{}{0pt}%
\pgfpathmoveto{\pgfqpoint{2.104916in}{0.492442in}}%
\pgfpathlineto{\pgfqpoint{2.104916in}{0.492442in}}%
\pgfpathlineto{\pgfqpoint{2.069099in}{0.540421in}}%
\pgfpathlineto{\pgfqpoint{2.033331in}{0.588410in}}%
\pgfpathlineto{\pgfqpoint{1.997545in}{0.636395in}}%
\pgfpathlineto{\pgfqpoint{1.961670in}{0.684361in}}%
\pgfpathlineto{\pgfqpoint{1.925628in}{0.732289in}}%
\pgfpathlineto{\pgfqpoint{1.889321in}{0.780157in}}%
\pgfpathlineto{\pgfqpoint{1.852630in}{0.827938in}}%
\pgfpathlineto{\pgfqpoint{1.815404in}{0.875596in}}%
\pgfpathlineto{\pgfqpoint{1.777437in}{0.923079in}}%
\pgfpathlineto{\pgfqpoint{1.738462in}{0.970317in}}%
\pgfpathlineto{\pgfqpoint{1.698118in}{1.017209in}}%
\pgfpathlineto{\pgfqpoint{1.655882in}{1.063598in}}%
\pgfpathlineto{\pgfqpoint{1.610956in}{1.109220in}}%
\pgfpathlineto{\pgfqpoint{1.562034in}{1.153583in}}%
\pgfpathlineto{\pgfqpoint{1.506773in}{1.195628in}}%
\pgfpathlineto{\pgfqpoint{1.440655in}{1.232321in}}%
\pgfpathlineto{\pgfqpoint{1.440655in}{1.232321in}}%
\pgfpathlineto{\pgfqpoint{1.388294in}{1.249371in}}%
\pgfpathlineto{\pgfqpoint{1.388294in}{1.249371in}}%
\pgfpathlineto{\pgfqpoint{1.339839in}{1.254794in}}%
\pgfpathlineto{\pgfqpoint{1.290919in}{1.249768in}}%
\pgfpathlineto{\pgfqpoint{1.249196in}{1.237235in}}%
\pgfpathlineto{\pgfqpoint{1.206996in}{1.217038in}}%
\pgfpathlineto{\pgfqpoint{1.161345in}{1.187097in}}%
\pgfpathlineto{\pgfqpoint{1.110314in}{1.144406in}}%
\pgfpathlineto{\pgfqpoint{1.064882in}{1.099012in}}%
\pgfpathlineto{\pgfqpoint{1.023665in}{1.052399in}}%
\pgfusepath{stroke}%
\end{pgfscope}%
\begin{pgfscope}%
\pgfpathrectangle{\pgfqpoint{0.647939in}{0.492442in}}{\pgfqpoint{4.273799in}{2.331163in}}%
\pgfusepath{clip}%
\pgfsetbuttcap%
\pgfsetroundjoin%
\pgfsetlinewidth{0.301125pt}%
\definecolor{currentstroke}{rgb}{0.500000,0.500000,0.500000}%
\pgfsetstrokecolor{currentstroke}%
\pgfsetstrokeopacity{0.300000}%
\pgfsetdash{}{0pt}%
\pgfpathmoveto{\pgfqpoint{2.396312in}{0.492442in}}%
\pgfpathlineto{\pgfqpoint{2.396312in}{0.492442in}}%
\pgfpathlineto{\pgfqpoint{2.360208in}{0.540357in}}%
\pgfpathlineto{\pgfqpoint{2.324556in}{0.588371in}}%
\pgfpathlineto{\pgfqpoint{2.289332in}{0.636480in}}%
\pgfpathlineto{\pgfqpoint{2.254507in}{0.684675in}}%
\pgfpathlineto{\pgfqpoint{2.220052in}{0.732949in}}%
\pgfpathlineto{\pgfqpoint{2.185940in}{0.781296in}}%
\pgfpathlineto{\pgfqpoint{2.152146in}{0.829709in}}%
\pgfpathlineto{\pgfqpoint{2.118641in}{0.878182in}}%
\pgfpathlineto{\pgfqpoint{2.085383in}{0.926706in}}%
\pgfpathlineto{\pgfqpoint{2.052332in}{0.975271in}}%
\pgfpathlineto{\pgfqpoint{2.019448in}{1.023870in}}%
\pgfpathlineto{\pgfqpoint{1.986689in}{1.072494in}}%
\pgfpathlineto{\pgfqpoint{1.953988in}{1.121130in}}%
\pgfpathlineto{\pgfqpoint{1.921267in}{1.169762in}}%
\pgfpathlineto{\pgfqpoint{1.888443in}{1.218372in}}%
\pgfpathlineto{\pgfqpoint{1.855415in}{1.266942in}}%
\pgfpathlineto{\pgfqpoint{1.822027in}{1.315438in}}%
\pgfpathlineto{\pgfqpoint{1.788066in}{1.363814in}}%
\pgfpathlineto{\pgfqpoint{1.753246in}{1.412006in}}%
\pgfpathlineto{\pgfqpoint{1.717136in}{1.459911in}}%
\pgfpathlineto{\pgfqpoint{1.679027in}{1.507350in}}%
\pgfpathlineto{\pgfqpoint{1.637626in}{1.553951in}}%
\pgfpathlineto{\pgfqpoint{1.590196in}{1.598744in}}%
\pgfpathlineto{\pgfqpoint{1.529792in}{1.637902in}}%
\pgfpathlineto{\pgfqpoint{1.529792in}{1.637902in}}%
\pgfpathlineto{\pgfqpoint{1.491511in}{1.650279in}}%
\pgfpathlineto{\pgfqpoint{1.491511in}{1.650279in}}%
\pgfpathlineto{\pgfqpoint{1.454674in}{1.652142in}}%
\pgfpathlineto{\pgfqpoint{1.419504in}{1.644957in}}%
\pgfpathlineto{\pgfqpoint{1.387217in}{1.631258in}}%
\pgfpathlineto{\pgfqpoint{1.352001in}{1.609308in}}%
\pgfpathlineto{\pgfqpoint{1.310613in}{1.575543in}}%
\pgfpathlineto{\pgfqpoint{1.265034in}{1.530288in}}%
\pgfpathlineto{\pgfqpoint{1.223942in}{1.483686in}}%
\pgfpathlineto{\pgfqpoint{1.185775in}{1.436319in}}%
\pgfpathlineto{\pgfqpoint{1.149693in}{1.388451in}}%
\pgfusepath{stroke}%
\end{pgfscope}%
\begin{pgfscope}%
\pgfpathrectangle{\pgfqpoint{0.647939in}{0.492442in}}{\pgfqpoint{4.273799in}{2.331163in}}%
\pgfusepath{clip}%
\pgfsetbuttcap%
\pgfsetroundjoin%
\pgfsetlinewidth{0.301125pt}%
\definecolor{currentstroke}{rgb}{0.500000,0.500000,0.500000}%
\pgfsetstrokecolor{currentstroke}%
\pgfsetstrokeopacity{0.300000}%
\pgfsetdash{}{0pt}%
\pgfpathmoveto{\pgfqpoint{2.493443in}{0.492442in}}%
\pgfpathlineto{\pgfqpoint{2.493443in}{0.492442in}}%
\pgfpathlineto{\pgfqpoint{2.456601in}{0.540189in}}%
\pgfpathlineto{\pgfqpoint{2.420312in}{0.588061in}}%
\pgfpathlineto{\pgfqpoint{2.384554in}{0.636053in}}%
\pgfpathlineto{\pgfqpoint{2.349302in}{0.684155in}}%
\pgfpathlineto{\pgfqpoint{2.314537in}{0.732363in}}%
\pgfpathlineto{\pgfqpoint{2.280241in}{0.780671in}}%
\pgfpathlineto{\pgfqpoint{2.246394in}{0.829073in}}%
\pgfpathlineto{\pgfqpoint{2.212970in}{0.877562in}}%
\pgfpathlineto{\pgfqpoint{2.179941in}{0.926132in}}%
\pgfpathlineto{\pgfqpoint{2.147288in}{0.974778in}}%
\pgfpathlineto{\pgfqpoint{2.114991in}{1.023494in}}%
\pgfpathlineto{\pgfqpoint{2.083020in}{1.072274in}}%
\pgfpathlineto{\pgfqpoint{2.051339in}{1.121110in}}%
\pgfpathlineto{\pgfqpoint{2.019922in}{1.169997in}}%
\pgfpathlineto{\pgfqpoint{1.988740in}{1.218929in}}%
\pgfpathlineto{\pgfqpoint{1.957742in}{1.267895in}}%
\pgfpathlineto{\pgfqpoint{1.926874in}{1.316886in}}%
\pgfpathlineto{\pgfqpoint{1.896087in}{1.365892in}}%
\pgfpathlineto{\pgfqpoint{1.865302in}{1.414897in}}%
\pgfpathlineto{\pgfqpoint{1.834403in}{1.463881in}}%
\pgfpathlineto{\pgfqpoint{1.803256in}{1.512817in}}%
\pgfpathlineto{\pgfqpoint{1.771668in}{1.561670in}}%
\pgfpathlineto{\pgfqpoint{1.739307in}{1.610369in}}%
\pgfpathlineto{\pgfqpoint{1.705614in}{1.658790in}}%
\pgfpathlineto{\pgfqpoint{1.669577in}{1.706689in}}%
\pgfpathlineto{\pgfqpoint{1.628851in}{1.753393in}}%
\pgfpathlineto{\pgfqpoint{1.576128in}{1.795652in}}%
\pgfpathlineto{\pgfqpoint{1.576128in}{1.795652in}}%
\pgfpathlineto{\pgfqpoint{1.545665in}{1.807749in}}%
\pgfpathlineto{\pgfqpoint{1.545665in}{1.807749in}}%
\pgfpathlineto{\pgfqpoint{1.515722in}{1.809498in}}%
\pgfpathlineto{\pgfqpoint{1.487766in}{1.802655in}}%
\pgfpathlineto{\pgfqpoint{1.462353in}{1.790220in}}%
\pgfpathlineto{\pgfqpoint{1.430906in}{1.768224in}}%
\pgfpathlineto{\pgfqpoint{1.392553in}{1.733801in}}%
\pgfpathlineto{\pgfqpoint{1.349328in}{1.687900in}}%
\pgfpathlineto{\pgfqpoint{1.309595in}{1.640978in}}%
\pgfpathlineto{\pgfqpoint{1.272112in}{1.593476in}}%
\pgfusepath{stroke}%
\end{pgfscope}%
\begin{pgfscope}%
\pgfpathrectangle{\pgfqpoint{0.647939in}{0.492442in}}{\pgfqpoint{4.273799in}{2.331163in}}%
\pgfusepath{clip}%
\pgfsetbuttcap%
\pgfsetroundjoin%
\pgfsetlinewidth{0.301125pt}%
\definecolor{currentstroke}{rgb}{0.500000,0.500000,0.500000}%
\pgfsetstrokecolor{currentstroke}%
\pgfsetstrokeopacity{0.300000}%
\pgfsetdash{}{0pt}%
\pgfpathmoveto{\pgfqpoint{2.687707in}{0.492442in}}%
\pgfpathlineto{\pgfqpoint{2.687707in}{0.492442in}}%
\pgfpathlineto{\pgfqpoint{2.648534in}{0.539633in}}%
\pgfpathlineto{\pgfqpoint{2.610106in}{0.587006in}}%
\pgfpathlineto{\pgfqpoint{2.572404in}{0.634552in}}%
\pgfpathlineto{\pgfqpoint{2.535412in}{0.682264in}}%
\pgfpathlineto{\pgfqpoint{2.499113in}{0.730134in}}%
\pgfpathlineto{\pgfqpoint{2.463491in}{0.778155in}}%
\pgfpathlineto{\pgfqpoint{2.428527in}{0.826320in}}%
\pgfpathlineto{\pgfqpoint{2.394203in}{0.874622in}}%
\pgfpathlineto{\pgfqpoint{2.360508in}{0.923055in}}%
\pgfpathlineto{\pgfqpoint{2.327434in}{0.971615in}}%
\pgfpathlineto{\pgfqpoint{2.294967in}{1.020298in}}%
\pgfpathlineto{\pgfqpoint{2.263095in}{1.069097in}}%
\pgfpathlineto{\pgfqpoint{2.231811in}{1.118009in}}%
\pgfpathlineto{\pgfqpoint{2.201118in}{1.167032in}}%
\pgfpathlineto{\pgfqpoint{2.171008in}{1.216163in}}%
\pgfpathlineto{\pgfqpoint{2.141479in}{1.265399in}}%
\pgfpathlineto{\pgfqpoint{2.112543in}{1.314739in}}%
\pgfpathlineto{\pgfqpoint{2.084212in}{1.364184in}}%
\pgfpathlineto{\pgfqpoint{2.056494in}{1.413732in}}%
\pgfpathlineto{\pgfqpoint{2.029428in}{1.463387in}}%
\pgfpathlineto{\pgfqpoint{2.003044in}{1.513151in}}%
\pgfpathlineto{\pgfqpoint{1.977386in}{1.563027in}}%
\pgfpathlineto{\pgfqpoint{1.952536in}{1.613025in}}%
\pgfpathlineto{\pgfqpoint{1.928588in}{1.663153in}}%
\pgfpathlineto{\pgfqpoint{1.905689in}{1.713426in}}%
\pgfpathlineto{\pgfqpoint{1.884037in}{1.763863in}}%
\pgfpathlineto{\pgfqpoint{1.863933in}{1.814490in}}%
\pgfpathlineto{\pgfqpoint{1.845825in}{1.865337in}}%
\pgfpathlineto{\pgfqpoint{1.830399in}{1.916442in}}%
\pgfpathlineto{\pgfqpoint{1.818747in}{1.967836in}}%
\pgfpathlineto{\pgfqpoint{1.812562in}{2.019495in}}%
\pgfpathlineto{\pgfqpoint{1.814289in}{2.071224in}}%
\pgfpathlineto{\pgfqpoint{1.826711in}{2.122451in}}%
\pgfpathlineto{\pgfqpoint{1.851432in}{2.172262in}}%
\pgfpathlineto{\pgfqpoint{1.887643in}{2.219925in}}%
\pgfpathlineto{\pgfqpoint{1.933282in}{2.265144in}}%
\pgfpathlineto{\pgfqpoint{1.986528in}{2.307882in}}%
\pgfpathlineto{\pgfqpoint{2.046477in}{2.347935in}}%
\pgfpathlineto{\pgfqpoint{2.112823in}{2.384818in}}%
\pgfpathlineto{\pgfqpoint{2.185769in}{2.417761in}}%
\pgfpathlineto{\pgfqpoint{2.265677in}{2.445417in}}%
\pgfpathlineto{\pgfqpoint{2.352431in}{2.465868in}}%
\pgfpathlineto{\pgfqpoint{2.444468in}{2.476730in}}%
\pgfpathlineto{\pgfqpoint{2.531005in}{2.476697in}}%
\pgfpathlineto{\pgfqpoint{2.611320in}{2.467413in}}%
\pgfpathlineto{\pgfqpoint{2.687413in}{2.449902in}}%
\pgfpathlineto{\pgfqpoint{2.761055in}{2.424236in}}%
\pgfpathlineto{\pgfqpoint{2.832325in}{2.390300in}}%
\pgfpathlineto{\pgfqpoint{2.895324in}{2.351744in}}%
\pgfpathlineto{\pgfqpoint{2.950720in}{2.309790in}}%
\pgfpathlineto{\pgfqpoint{2.998975in}{2.265282in}}%
\pgfpathlineto{\pgfqpoint{3.040341in}{2.218743in}}%
\pgfpathlineto{\pgfqpoint{3.074644in}{2.170517in}}%
\pgfpathlineto{\pgfqpoint{3.101074in}{2.120838in}}%
\pgfpathlineto{\pgfqpoint{3.117571in}{2.069936in}}%
\pgfpathlineto{\pgfqpoint{3.118482in}{2.018472in}}%
\pgfpathlineto{\pgfqpoint{3.118482in}{2.018472in}}%
\pgfpathlineto{\pgfqpoint{3.106203in}{1.990854in}}%
\pgfpathlineto{\pgfqpoint{3.106203in}{1.990854in}}%
\pgfpathlineto{\pgfqpoint{3.086797in}{1.976029in}}%
\pgfpathlineto{\pgfqpoint{3.086797in}{1.976029in}}%
\pgfpathlineto{\pgfqpoint{3.062820in}{1.971730in}}%
\pgfpathlineto{\pgfqpoint{3.039165in}{1.975889in}}%
\pgfpathlineto{\pgfqpoint{3.018029in}{1.985648in}}%
\pgfpathlineto{\pgfqpoint{2.996021in}{2.003316in}}%
\pgfpathlineto{\pgfqpoint{2.980224in}{2.027963in}}%
\pgfpathlineto{\pgfqpoint{2.980224in}{2.027963in}}%
\pgfpathlineto{\pgfqpoint{2.981425in}{2.045730in}}%
\pgfpathlineto{\pgfqpoint{2.981425in}{2.045730in}}%
\pgfpathlineto{\pgfqpoint{2.990367in}{2.046823in}}%
\pgfpathlineto{\pgfqpoint{2.995306in}{2.043918in}}%
\pgfpathlineto{\pgfqpoint{2.998953in}{2.037784in}}%
\pgfpathlineto{\pgfqpoint{2.995220in}{2.037354in}}%
\pgfpathlineto{\pgfqpoint{2.997543in}{2.038482in}}%
\pgfpathlineto{\pgfqpoint{2.994250in}{2.037594in}}%
\pgfpathlineto{\pgfqpoint{2.994250in}{2.037594in}}%
\pgfpathlineto{\pgfqpoint{2.998592in}{2.038216in}}%
\pgfpathlineto{\pgfqpoint{2.994814in}{2.037376in}}%
\pgfpathlineto{\pgfqpoint{2.998443in}{2.038316in}}%
\pgfpathlineto{\pgfqpoint{2.993685in}{2.037768in}}%
\pgfpathlineto{\pgfqpoint{2.993685in}{2.037768in}}%
\pgfpathlineto{\pgfqpoint{2.998120in}{2.038491in}}%
\pgfpathlineto{\pgfqpoint{2.995859in}{2.037057in}}%
\pgfpathlineto{\pgfqpoint{2.997488in}{2.038847in}}%
\pgfpathlineto{\pgfqpoint{2.994207in}{2.037096in}}%
\pgfpathlineto{\pgfqpoint{2.994207in}{2.037096in}}%
\pgfpathlineto{\pgfqpoint{2.998267in}{2.038822in}}%
\pgfpathlineto{\pgfqpoint{2.995371in}{2.036859in}}%
\pgfpathlineto{\pgfqpoint{2.997972in}{2.038969in}}%
\pgfpathlineto{\pgfqpoint{2.993403in}{2.037101in}}%
\pgfpathlineto{\pgfqpoint{2.993403in}{2.037101in}}%
\pgfpathlineto{\pgfqpoint{2.997343in}{2.039297in}}%
\pgfpathlineto{\pgfqpoint{2.996632in}{2.036809in}}%
\pgfpathlineto{\pgfqpoint{2.996457in}{2.039578in}}%
\pgfpathlineto{\pgfqpoint{2.995610in}{2.035723in}}%
\pgfpathlineto{\pgfqpoint{2.999701in}{2.040811in}}%
\pgfpathlineto{\pgfqpoint{2.999701in}{2.040811in}}%
\pgfpathlineto{\pgfqpoint{2.997475in}{2.035321in}}%
\pgfpathlineto{\pgfqpoint{2.995285in}{2.038747in}}%
\pgfpathlineto{\pgfqpoint{2.998061in}{2.036228in}}%
\pgfpathlineto{\pgfqpoint{2.994171in}{2.040809in}}%
\pgfpathlineto{\pgfqpoint{2.999165in}{2.032842in}}%
\pgfpathlineto{\pgfqpoint{2.993561in}{2.043920in}}%
\pgfpathlineto{\pgfqpoint{2.999289in}{2.033189in}}%
\pgfpathlineto{\pgfqpoint{2.993913in}{2.040395in}}%
\pgfpathlineto{\pgfqpoint{2.998605in}{2.035943in}}%
\pgfpathlineto{\pgfqpoint{2.994380in}{2.039545in}}%
\pgfpathlineto{\pgfqpoint{2.998982in}{2.035353in}}%
\pgfpathlineto{\pgfqpoint{2.993466in}{2.041152in}}%
\pgfpathlineto{\pgfqpoint{2.999531in}{2.034166in}}%
\pgfpathlineto{\pgfqpoint{2.993870in}{2.040109in}}%
\pgfpathlineto{\pgfqpoint{2.998627in}{2.036176in}}%
\pgfusepath{stroke}%
\end{pgfscope}%
\begin{pgfscope}%
\pgfpathrectangle{\pgfqpoint{0.647939in}{0.492442in}}{\pgfqpoint{4.273799in}{2.331163in}}%
\pgfusepath{clip}%
\pgfsetbuttcap%
\pgfsetroundjoin%
\pgfsetlinewidth{0.301125pt}%
\definecolor{currentstroke}{rgb}{0.500000,0.500000,0.500000}%
\pgfsetstrokecolor{currentstroke}%
\pgfsetstrokeopacity{0.300000}%
\pgfsetdash{}{0pt}%
\pgfpathmoveto{\pgfqpoint{2.784839in}{0.492442in}}%
\pgfpathlineto{\pgfqpoint{2.784839in}{0.492442in}}%
\pgfpathlineto{\pgfqpoint{2.744103in}{0.539237in}}%
\pgfpathlineto{\pgfqpoint{2.704206in}{0.586247in}}%
\pgfpathlineto{\pgfqpoint{2.665132in}{0.633462in}}%
\pgfpathlineto{\pgfqpoint{2.626864in}{0.680873in}}%
\pgfpathlineto{\pgfqpoint{2.589385in}{0.728472in}}%
\pgfpathlineto{\pgfqpoint{2.552680in}{0.776250in}}%
\pgfpathlineto{\pgfqpoint{2.516730in}{0.824198in}}%
\pgfpathlineto{\pgfqpoint{2.481520in}{0.872310in}}%
\pgfpathlineto{\pgfqpoint{2.447039in}{0.920578in}}%
\pgfpathlineto{\pgfqpoint{2.413278in}{0.968997in}}%
\pgfpathlineto{\pgfqpoint{2.380229in}{1.017563in}}%
\pgfpathlineto{\pgfqpoint{2.347880in}{1.066268in}}%
\pgfpathlineto{\pgfqpoint{2.316226in}{1.115109in}}%
\pgfpathlineto{\pgfqpoint{2.285273in}{1.164084in}}%
\pgfpathlineto{\pgfqpoint{2.255021in}{1.213189in}}%
\pgfpathlineto{\pgfqpoint{2.225473in}{1.262421in}}%
\pgfpathlineto{\pgfqpoint{2.196648in}{1.311780in}}%
\pgfpathlineto{\pgfqpoint{2.168565in}{1.361267in}}%
\pgfusepath{stroke}%
\end{pgfscope}%
\begin{pgfscope}%
\pgfpathrectangle{\pgfqpoint{0.647939in}{0.492442in}}{\pgfqpoint{4.273799in}{2.331163in}}%
\pgfusepath{clip}%
\pgfsetbuttcap%
\pgfsetroundjoin%
\pgfsetlinewidth{0.301125pt}%
\definecolor{currentstroke}{rgb}{0.500000,0.500000,0.500000}%
\pgfsetstrokecolor{currentstroke}%
\pgfsetstrokeopacity{0.300000}%
\pgfsetdash{}{0pt}%
\pgfpathmoveto{\pgfqpoint{2.881971in}{0.492442in}}%
\pgfpathlineto{\pgfqpoint{2.881971in}{0.492442in}}%
\pgfpathlineto{\pgfqpoint{2.839427in}{0.538756in}}%
\pgfpathlineto{\pgfqpoint{2.797813in}{0.585321in}}%
\pgfpathlineto{\pgfqpoint{2.757115in}{0.632126in}}%
\pgfpathlineto{\pgfqpoint{2.717318in}{0.679161in}}%
\pgfpathlineto{\pgfqpoint{2.678408in}{0.726417in}}%
\pgfpathlineto{\pgfqpoint{2.640366in}{0.773882in}}%
\pgfpathlineto{\pgfqpoint{2.603176in}{0.821548in}}%
\pgfpathlineto{\pgfqpoint{2.566823in}{0.869405in}}%
\pgfusepath{stroke}%
\end{pgfscope}%
\begin{pgfscope}%
\pgfpathrectangle{\pgfqpoint{0.647939in}{0.492442in}}{\pgfqpoint{4.273799in}{2.331163in}}%
\pgfusepath{clip}%
\pgfsetbuttcap%
\pgfsetroundjoin%
\pgfsetlinewidth{0.301125pt}%
\definecolor{currentstroke}{rgb}{0.500000,0.500000,0.500000}%
\pgfsetstrokecolor{currentstroke}%
\pgfsetstrokeopacity{0.300000}%
\pgfsetdash{}{0pt}%
\pgfpathmoveto{\pgfqpoint{3.076234in}{0.492442in}}%
\pgfpathlineto{\pgfqpoint{3.076234in}{0.492442in}}%
\pgfpathlineto{\pgfqpoint{3.029442in}{0.537520in}}%
\pgfpathlineto{\pgfqpoint{2.983731in}{0.582927in}}%
\pgfpathlineto{\pgfqpoint{2.939104in}{0.628654in}}%
\pgfpathlineto{\pgfqpoint{2.895557in}{0.674689in}}%
\pgfpathlineto{\pgfqpoint{2.853081in}{0.721022in}}%
\pgfpathlineto{\pgfqpoint{2.811666in}{0.767639in}}%
\pgfpathlineto{\pgfqpoint{2.771298in}{0.814529in}}%
\pgfpathlineto{\pgfqpoint{2.731962in}{0.861678in}}%
\pgfpathlineto{\pgfqpoint{2.693645in}{0.909077in}}%
\pgfpathlineto{\pgfqpoint{2.656334in}{0.956714in}}%
\pgfpathlineto{\pgfqpoint{2.620018in}{1.004580in}}%
\pgfpathlineto{\pgfqpoint{2.584691in}{1.052665in}}%
\pgfpathlineto{\pgfqpoint{2.550346in}{1.100962in}}%
\pgfpathlineto{\pgfqpoint{2.516979in}{1.149462in}}%
\pgfpathlineto{\pgfqpoint{2.484589in}{1.198158in}}%
\pgfpathlineto{\pgfqpoint{2.453188in}{1.247048in}}%
\pgfpathlineto{\pgfqpoint{2.422791in}{1.296125in}}%
\pgfpathlineto{\pgfqpoint{2.393414in}{1.345387in}}%
\pgfpathlineto{\pgfqpoint{2.365088in}{1.394832in}}%
\pgfpathlineto{\pgfqpoint{2.337856in}{1.444459in}}%
\pgfpathlineto{\pgfqpoint{2.311768in}{1.494269in}}%
\pgfpathlineto{\pgfqpoint{2.286895in}{1.544263in}}%
\pgfpathlineto{\pgfqpoint{2.263323in}{1.594444in}}%
\pgfpathlineto{\pgfqpoint{2.241162in}{1.644816in}}%
\pgfpathlineto{\pgfqpoint{2.220555in}{1.695383in}}%
\pgfpathlineto{\pgfqpoint{2.201683in}{1.746151in}}%
\pgfpathlineto{\pgfqpoint{2.184774in}{1.797123in}}%
\pgfpathlineto{\pgfqpoint{2.170114in}{1.848300in}}%
\pgfpathlineto{\pgfqpoint{2.158083in}{1.899680in}}%
\pgfpathlineto{\pgfqpoint{2.149152in}{1.951245in}}%
\pgfpathlineto{\pgfqpoint{2.143924in}{2.002957in}}%
\pgfpathlineto{\pgfqpoint{2.143172in}{2.054739in}}%
\pgfpathlineto{\pgfqpoint{2.147856in}{2.106449in}}%
\pgfpathlineto{\pgfqpoint{2.159139in}{2.157840in}}%
\pgfpathlineto{\pgfqpoint{2.178401in}{2.208501in}}%
\pgfpathlineto{\pgfqpoint{2.207156in}{2.257779in}}%
\pgfpathlineto{\pgfqpoint{2.247003in}{2.304658in}}%
\pgfpathlineto{\pgfqpoint{2.299450in}{2.347594in}}%
\pgfpathlineto{\pgfqpoint{2.365680in}{2.384200in}}%
\pgfpathlineto{\pgfqpoint{2.440829in}{2.410007in}}%
\pgfpathlineto{\pgfqpoint{2.516129in}{2.423057in}}%
\pgfpathlineto{\pgfqpoint{2.588730in}{2.425174in}}%
\pgfpathlineto{\pgfqpoint{2.658079in}{2.418176in}}%
\pgfpathlineto{\pgfqpoint{2.725090in}{2.403098in}}%
\pgfusepath{stroke}%
\end{pgfscope}%
\begin{pgfscope}%
\pgfpathrectangle{\pgfqpoint{0.647939in}{0.492442in}}{\pgfqpoint{4.273799in}{2.331163in}}%
\pgfusepath{clip}%
\pgfsetbuttcap%
\pgfsetroundjoin%
\pgfsetlinewidth{0.301125pt}%
\definecolor{currentstroke}{rgb}{0.500000,0.500000,0.500000}%
\pgfsetstrokecolor{currentstroke}%
\pgfsetstrokeopacity{0.300000}%
\pgfsetdash{}{0pt}%
\pgfpathmoveto{\pgfqpoint{3.270498in}{0.492442in}}%
\pgfpathlineto{\pgfqpoint{3.270498in}{0.492442in}}%
\pgfpathlineto{\pgfqpoint{3.218964in}{0.535953in}}%
\pgfpathlineto{\pgfqpoint{3.168551in}{0.579854in}}%
\pgfpathlineto{\pgfqpoint{3.119305in}{0.624146in}}%
\pgfpathlineto{\pgfqpoint{3.071256in}{0.668828in}}%
\pgfpathlineto{\pgfqpoint{3.024424in}{0.713893in}}%
\pgfpathlineto{\pgfqpoint{2.978817in}{0.759331in}}%
\pgfpathlineto{\pgfqpoint{2.934436in}{0.805129in}}%
\pgfpathlineto{\pgfqpoint{2.891278in}{0.851274in}}%
\pgfpathlineto{\pgfqpoint{2.849333in}{0.897750in}}%
\pgfpathlineto{\pgfqpoint{2.808591in}{0.944543in}}%
\pgfpathlineto{\pgfqpoint{2.769038in}{0.991638in}}%
\pgfpathlineto{\pgfqpoint{2.730662in}{1.039023in}}%
\pgfpathlineto{\pgfqpoint{2.693454in}{1.086683in}}%
\pgfpathlineto{\pgfqpoint{2.657406in}{1.134608in}}%
\pgfpathlineto{\pgfqpoint{2.622517in}{1.182788in}}%
\pgfpathlineto{\pgfqpoint{2.588790in}{1.231214in}}%
\pgfpathlineto{\pgfqpoint{2.556230in}{1.279876in}}%
\pgfpathlineto{\pgfqpoint{2.524848in}{1.328768in}}%
\pgfpathlineto{\pgfqpoint{2.494669in}{1.377885in}}%
\pgfpathlineto{\pgfqpoint{2.465729in}{1.427224in}}%
\pgfpathlineto{\pgfqpoint{2.438067in}{1.476780in}}%
\pgfpathlineto{\pgfqpoint{2.411744in}{1.526552in}}%
\pgfpathlineto{\pgfqpoint{2.386838in}{1.576541in}}%
\pgfpathlineto{\pgfqpoint{2.363444in}{1.626746in}}%
\pgfpathlineto{\pgfqpoint{2.341688in}{1.677169in}}%
\pgfpathlineto{\pgfqpoint{2.321728in}{1.727813in}}%
\pgfpathlineto{\pgfqpoint{2.303763in}{1.778677in}}%
\pgfpathlineto{\pgfqpoint{2.288049in}{1.829762in}}%
\pgfpathlineto{\pgfqpoint{2.274907in}{1.881061in}}%
\pgfpathlineto{\pgfqpoint{2.264742in}{1.932559in}}%
\pgfpathlineto{\pgfqpoint{2.258085in}{1.984223in}}%
\pgfpathlineto{\pgfqpoint{2.255614in}{2.035991in}}%
\pgfpathlineto{\pgfqpoint{2.258200in}{2.087750in}}%
\pgfpathlineto{\pgfqpoint{2.266966in}{2.139295in}}%
\pgfpathlineto{\pgfqpoint{2.283353in}{2.190264in}}%
\pgfpathlineto{\pgfqpoint{2.309193in}{2.240012in}}%
\pgfpathlineto{\pgfqpoint{2.346772in}{2.287401in}}%
\pgfpathlineto{\pgfqpoint{2.398770in}{2.330386in}}%
\pgfpathlineto{\pgfqpoint{2.466152in}{2.364850in}}%
\pgfusepath{stroke}%
\end{pgfscope}%
\begin{pgfscope}%
\pgfpathrectangle{\pgfqpoint{0.647939in}{0.492442in}}{\pgfqpoint{4.273799in}{2.331163in}}%
\pgfusepath{clip}%
\pgfsetbuttcap%
\pgfsetroundjoin%
\pgfsetlinewidth{0.301125pt}%
\definecolor{currentstroke}{rgb}{0.500000,0.500000,0.500000}%
\pgfsetstrokecolor{currentstroke}%
\pgfsetstrokeopacity{0.300000}%
\pgfsetdash{}{0pt}%
\pgfpathmoveto{\pgfqpoint{3.464761in}{0.492442in}}%
\pgfpathlineto{\pgfqpoint{3.464761in}{0.492442in}}%
\pgfpathlineto{\pgfqpoint{3.408732in}{0.534269in}}%
\pgfpathlineto{\pgfqpoint{3.353611in}{0.576453in}}%
\pgfpathlineto{\pgfqpoint{3.299536in}{0.619036in}}%
\pgfpathlineto{\pgfqpoint{3.246610in}{0.662047in}}%
\pgfpathlineto{\pgfqpoint{3.194914in}{0.705502in}}%
\pgfpathlineto{\pgfqpoint{3.144515in}{0.749406in}}%
\pgfpathlineto{\pgfqpoint{3.095456in}{0.793760in}}%
\pgfpathlineto{\pgfqpoint{3.047765in}{0.838555in}}%
\pgfpathlineto{\pgfqpoint{3.001452in}{0.883778in}}%
\pgfpathlineto{\pgfqpoint{2.956524in}{0.929414in}}%
\pgfpathlineto{\pgfqpoint{2.912977in}{0.975448in}}%
\pgfpathlineto{\pgfqpoint{2.870803in}{1.021860in}}%
\pgfpathlineto{\pgfqpoint{2.829993in}{1.068634in}}%
\pgfpathlineto{\pgfqpoint{2.790538in}{1.115753in}}%
\pgfpathlineto{\pgfqpoint{2.752430in}{1.163200in}}%
\pgfpathlineto{\pgfqpoint{2.715663in}{1.210962in}}%
\pgfpathlineto{\pgfqpoint{2.680239in}{1.259026in}}%
\pgfpathlineto{\pgfqpoint{2.646158in}{1.307377in}}%
\pgfpathlineto{\pgfqpoint{2.613433in}{1.356006in}}%
\pgfpathlineto{\pgfqpoint{2.582088in}{1.404904in}}%
\pgfpathlineto{\pgfqpoint{2.552159in}{1.454066in}}%
\pgfpathlineto{\pgfqpoint{2.523687in}{1.503485in}}%
\pgfpathlineto{\pgfqpoint{2.496735in}{1.553156in}}%
\pgfpathlineto{\pgfqpoint{2.471389in}{1.603079in}}%
\pgfpathlineto{\pgfqpoint{2.447751in}{1.653249in}}%
\pgfpathlineto{\pgfqpoint{2.425961in}{1.703667in}}%
\pgfpathlineto{\pgfqpoint{2.406191in}{1.754332in}}%
\pgfpathlineto{\pgfqpoint{2.388664in}{1.805241in}}%
\pgfpathlineto{\pgfqpoint{2.373666in}{1.856388in}}%
\pgfpathlineto{\pgfqpoint{2.361561in}{1.907762in}}%
\pgfpathlineto{\pgfqpoint{2.352818in}{1.959335in}}%
\pgfpathlineto{\pgfqpoint{2.348056in}{2.011057in}}%
\pgfpathlineto{\pgfqpoint{2.348088in}{2.062837in}}%
\pgfpathlineto{\pgfqpoint{2.354003in}{2.114506in}}%
\pgfpathlineto{\pgfqpoint{2.367275in}{2.165748in}}%
\pgfpathlineto{\pgfqpoint{2.389934in}{2.215957in}}%
\pgfpathlineto{\pgfqpoint{2.424793in}{2.263949in}}%
\pgfpathlineto{\pgfqpoint{2.475620in}{2.307259in}}%
\pgfpathlineto{\pgfqpoint{2.475620in}{2.307259in}}%
\pgfusepath{stroke}%
\end{pgfscope}%
\begin{pgfscope}%
\pgfpathrectangle{\pgfqpoint{0.647939in}{0.492442in}}{\pgfqpoint{4.273799in}{2.331163in}}%
\pgfusepath{clip}%
\pgfsetbuttcap%
\pgfsetroundjoin%
\pgfsetlinewidth{0.301125pt}%
\definecolor{currentstroke}{rgb}{0.500000,0.500000,0.500000}%
\pgfsetstrokecolor{currentstroke}%
\pgfsetstrokeopacity{0.300000}%
\pgfsetdash{}{0pt}%
\pgfpathmoveto{\pgfqpoint{3.659025in}{0.492442in}}%
\pgfpathlineto{\pgfqpoint{3.659025in}{0.492442in}}%
\pgfpathlineto{\pgfqpoint{3.599886in}{0.532976in}}%
\pgfpathlineto{\pgfqpoint{3.541042in}{0.573638in}}%
\pgfpathlineto{\pgfqpoint{3.482749in}{0.614535in}}%
\pgfpathlineto{\pgfqpoint{3.425239in}{0.655758in}}%
\pgfpathlineto{\pgfqpoint{3.368696in}{0.697378in}}%
\pgfpathlineto{\pgfqpoint{3.313291in}{0.739450in}}%
\pgfpathlineto{\pgfqpoint{3.259164in}{0.782013in}}%
\pgfpathlineto{\pgfqpoint{3.206411in}{0.825085in}}%
\pgfpathlineto{\pgfqpoint{3.155105in}{0.868676in}}%
\pgfpathlineto{\pgfqpoint{3.105301in}{0.912781in}}%
\pgfpathlineto{\pgfqpoint{3.057032in}{0.957391in}}%
\pgfpathlineto{\pgfqpoint{3.010315in}{1.002490in}}%
\pgfpathlineto{\pgfqpoint{2.965155in}{1.048058in}}%
\pgfpathlineto{\pgfqpoint{2.921550in}{1.094074in}}%
\pgfpathlineto{\pgfqpoint{2.879492in}{1.140517in}}%
\pgfpathlineto{\pgfqpoint{2.838974in}{1.187366in}}%
\pgfpathlineto{\pgfqpoint{2.799990in}{1.234601in}}%
\pgfpathlineto{\pgfqpoint{2.762537in}{1.282204in}}%
\pgfpathlineto{\pgfqpoint{2.726616in}{1.330156in}}%
\pgfpathlineto{\pgfqpoint{2.692233in}{1.378443in}}%
\pgfpathlineto{\pgfqpoint{2.659410in}{1.427051in}}%
\pgfpathlineto{\pgfqpoint{2.628179in}{1.475971in}}%
\pgfpathlineto{\pgfqpoint{2.598588in}{1.525193in}}%
\pgfpathlineto{\pgfqpoint{2.570695in}{1.574709in}}%
\pgfpathlineto{\pgfqpoint{2.544593in}{1.624514in}}%
\pgfusepath{stroke}%
\end{pgfscope}%
\begin{pgfscope}%
\pgfpathrectangle{\pgfqpoint{0.647939in}{0.492442in}}{\pgfqpoint{4.273799in}{2.331163in}}%
\pgfusepath{clip}%
\pgfsetbuttcap%
\pgfsetroundjoin%
\pgfsetlinewidth{0.301125pt}%
\definecolor{currentstroke}{rgb}{0.500000,0.500000,0.500000}%
\pgfsetstrokecolor{currentstroke}%
\pgfsetstrokeopacity{0.300000}%
\pgfsetdash{}{0pt}%
\pgfpathmoveto{\pgfqpoint{3.853289in}{0.492442in}}%
\pgfpathlineto{\pgfqpoint{3.853289in}{0.492442in}}%
\pgfpathlineto{\pgfqpoint{3.793679in}{0.532770in}}%
\pgfpathlineto{\pgfqpoint{3.733386in}{0.572795in}}%
\pgfpathlineto{\pgfqpoint{3.672715in}{0.612650in}}%
\pgfpathlineto{\pgfqpoint{3.611989in}{0.652480in}}%
\pgfpathlineto{\pgfqpoint{3.551532in}{0.692432in}}%
\pgfpathlineto{\pgfqpoint{3.491637in}{0.732633in}}%
\pgfpathlineto{\pgfqpoint{3.432581in}{0.773200in}}%
\pgfpathlineto{\pgfqpoint{3.374601in}{0.814227in}}%
\pgfpathlineto{\pgfqpoint{3.317893in}{0.855778in}}%
\pgfpathlineto{\pgfqpoint{3.262621in}{0.897899in}}%
\pgfpathlineto{\pgfqpoint{3.208905in}{0.940615in}}%
\pgfpathlineto{\pgfqpoint{3.156827in}{0.983930in}}%
\pgfpathlineto{\pgfqpoint{3.106446in}{1.027839in}}%
\pgfpathlineto{\pgfqpoint{3.057799in}{1.072326in}}%
\pgfpathlineto{\pgfqpoint{3.010904in}{1.117369in}}%
\pgfpathlineto{\pgfqpoint{2.965765in}{1.162941in}}%
\pgfpathlineto{\pgfqpoint{2.922377in}{1.209018in}}%
\pgfpathlineto{\pgfqpoint{2.880736in}{1.255571in}}%
\pgfpathlineto{\pgfqpoint{2.840836in}{1.302577in}}%
\pgfpathlineto{\pgfqpoint{2.802676in}{1.350011in}}%
\pgfpathlineto{\pgfqpoint{2.766262in}{1.397852in}}%
\pgfusepath{stroke}%
\end{pgfscope}%
\begin{pgfscope}%
\pgfpathrectangle{\pgfqpoint{0.647939in}{0.492442in}}{\pgfqpoint{4.273799in}{2.331163in}}%
\pgfusepath{clip}%
\pgfsetbuttcap%
\pgfsetroundjoin%
\pgfsetlinewidth{0.301125pt}%
\definecolor{currentstroke}{rgb}{0.500000,0.500000,0.500000}%
\pgfsetstrokecolor{currentstroke}%
\pgfsetstrokeopacity{0.300000}%
\pgfsetdash{}{0pt}%
\pgfpathmoveto{\pgfqpoint{4.047552in}{0.492442in}}%
\pgfpathlineto{\pgfqpoint{4.047552in}{0.492442in}}%
\pgfpathlineto{\pgfqpoint{3.991088in}{0.534090in}}%
\pgfpathlineto{\pgfqpoint{3.932891in}{0.575024in}}%
\pgfpathlineto{\pgfqpoint{3.873164in}{0.615298in}}%
\pgfpathlineto{\pgfqpoint{3.812184in}{0.655010in}}%
\pgfpathlineto{\pgfqpoint{3.750263in}{0.694288in}}%
\pgfpathlineto{\pgfqpoint{3.687773in}{0.733298in}}%
\pgfpathlineto{\pgfqpoint{3.625096in}{0.772219in}}%
\pgfpathlineto{\pgfqpoint{3.562625in}{0.811236in}}%
\pgfpathlineto{\pgfqpoint{3.500732in}{0.850527in}}%
\pgfpathlineto{\pgfqpoint{3.439754in}{0.890239in}}%
\pgfpathlineto{\pgfqpoint{3.379989in}{0.930494in}}%
\pgfpathlineto{\pgfqpoint{3.321673in}{0.971376in}}%
\pgfpathlineto{\pgfqpoint{3.265003in}{1.012940in}}%
\pgfpathlineto{\pgfqpoint{3.210124in}{1.055211in}}%
\pgfpathlineto{\pgfqpoint{3.157130in}{1.098193in}}%
\pgfpathlineto{\pgfqpoint{3.106085in}{1.141871in}}%
\pgfpathlineto{\pgfqpoint{3.057027in}{1.186222in}}%
\pgfpathlineto{\pgfqpoint{3.009974in}{1.231213in}}%
\pgfpathlineto{\pgfqpoint{2.964927in}{1.276811in}}%
\pgfpathlineto{\pgfqpoint{2.921881in}{1.322981in}}%
\pgfpathlineto{\pgfqpoint{2.880832in}{1.369689in}}%
\pgfpathlineto{\pgfqpoint{2.841779in}{1.416905in}}%
\pgfpathlineto{\pgfqpoint{2.804732in}{1.464600in}}%
\pgfpathlineto{\pgfqpoint{2.769712in}{1.512748in}}%
\pgfpathlineto{\pgfqpoint{2.736755in}{1.561327in}}%
\pgfpathlineto{\pgfqpoint{2.705921in}{1.610319in}}%
\pgfpathlineto{\pgfqpoint{2.677305in}{1.659709in}}%
\pgfpathlineto{\pgfqpoint{2.651028in}{1.709485in}}%
\pgfpathlineto{\pgfqpoint{2.627260in}{1.759633in}}%
\pgfpathlineto{\pgfqpoint{2.606236in}{1.810144in}}%
\pgfpathlineto{\pgfqpoint{2.588270in}{1.861003in}}%
\pgfpathlineto{\pgfqpoint{2.573790in}{1.912189in}}%
\pgfpathlineto{\pgfqpoint{2.563385in}{1.963664in}}%
\pgfpathlineto{\pgfqpoint{2.557894in}{2.015355in}}%
\pgfpathlineto{\pgfqpoint{2.558546in}{2.067119in}}%
\pgfpathlineto{\pgfqpoint{2.567244in}{2.118639in}}%
\pgfpathlineto{\pgfqpoint{2.587134in}{2.169144in}}%
\pgfpathlineto{\pgfqpoint{2.623787in}{2.216485in}}%
\pgfpathlineto{\pgfqpoint{2.623787in}{2.216485in}}%
\pgfpathlineto{\pgfqpoint{2.662699in}{2.244000in}}%
\pgfpathlineto{\pgfqpoint{2.662699in}{2.244000in}}%
\pgfpathlineto{\pgfqpoint{2.705727in}{2.260203in}}%
\pgfpathlineto{\pgfqpoint{2.756961in}{2.266527in}}%
\pgfpathlineto{\pgfqpoint{2.804036in}{2.262885in}}%
\pgfpathlineto{\pgfqpoint{2.849645in}{2.251712in}}%
\pgfpathlineto{\pgfqpoint{2.895858in}{2.232856in}}%
\pgfpathlineto{\pgfqpoint{2.942690in}{2.205335in}}%
\pgfpathlineto{\pgfqpoint{2.988401in}{2.168206in}}%
\pgfusepath{stroke}%
\end{pgfscope}%
\begin{pgfscope}%
\pgfpathrectangle{\pgfqpoint{0.647939in}{0.492442in}}{\pgfqpoint{4.273799in}{2.331163in}}%
\pgfusepath{clip}%
\pgfsetbuttcap%
\pgfsetroundjoin%
\pgfsetlinewidth{0.301125pt}%
\definecolor{currentstroke}{rgb}{0.500000,0.500000,0.500000}%
\pgfsetstrokecolor{currentstroke}%
\pgfsetstrokeopacity{0.300000}%
\pgfsetdash{}{0pt}%
\pgfpathmoveto{\pgfqpoint{4.241816in}{0.492442in}}%
\pgfpathlineto{\pgfqpoint{4.241816in}{0.492442in}}%
\pgfpathlineto{\pgfqpoint{4.192066in}{0.536562in}}%
\pgfpathlineto{\pgfqpoint{4.140090in}{0.579912in}}%
\pgfpathlineto{\pgfqpoint{4.085848in}{0.622427in}}%
\pgfpathlineto{\pgfqpoint{4.029343in}{0.664055in}}%
\pgfpathlineto{\pgfqpoint{3.970627in}{0.704765in}}%
\pgfpathlineto{\pgfqpoint{3.909885in}{0.744581in}}%
\pgfpathlineto{\pgfqpoint{3.847384in}{0.783581in}}%
\pgfpathlineto{\pgfqpoint{3.783473in}{0.821898in}}%
\pgfpathlineto{\pgfqpoint{3.718589in}{0.859726in}}%
\pgfpathlineto{\pgfqpoint{3.653211in}{0.897301in}}%
\pgfpathlineto{\pgfqpoint{3.587841in}{0.934879in}}%
\pgfpathlineto{\pgfqpoint{3.522969in}{0.972712in}}%
\pgfpathlineto{\pgfqpoint{3.459049in}{1.011022in}}%
\pgfpathlineto{\pgfqpoint{3.396474in}{1.049985in}}%
\pgfpathlineto{\pgfqpoint{3.335576in}{1.089726in}}%
\pgfpathlineto{\pgfqpoint{3.276599in}{1.130321in}}%
\pgfpathlineto{\pgfqpoint{3.219736in}{1.171802in}}%
\pgfpathlineto{\pgfqpoint{3.165113in}{1.214167in}}%
\pgfpathlineto{\pgfqpoint{3.112798in}{1.257392in}}%
\pgfpathlineto{\pgfqpoint{3.062832in}{1.301438in}}%
\pgfpathlineto{\pgfqpoint{3.015232in}{1.346256in}}%
\pgfpathlineto{\pgfqpoint{2.969998in}{1.391797in}}%
\pgfpathlineto{\pgfqpoint{2.927123in}{1.438013in}}%
\pgfpathlineto{\pgfqpoint{2.886607in}{1.484858in}}%
\pgfpathlineto{\pgfqpoint{2.848458in}{1.532292in}}%
\pgfpathlineto{\pgfqpoint{2.812706in}{1.580278in}}%
\pgfpathlineto{\pgfqpoint{2.779399in}{1.628783in}}%
\pgfpathlineto{\pgfqpoint{2.748623in}{1.677783in}}%
\pgfusepath{stroke}%
\end{pgfscope}%
\begin{pgfscope}%
\pgfpathrectangle{\pgfqpoint{0.647939in}{0.492442in}}{\pgfqpoint{4.273799in}{2.331163in}}%
\pgfusepath{clip}%
\pgfsetbuttcap%
\pgfsetroundjoin%
\pgfsetlinewidth{0.301125pt}%
\definecolor{currentstroke}{rgb}{0.500000,0.500000,0.500000}%
\pgfsetstrokecolor{currentstroke}%
\pgfsetstrokeopacity{0.300000}%
\pgfsetdash{}{0pt}%
\pgfpathmoveto{\pgfqpoint{4.338948in}{0.492442in}}%
\pgfpathlineto{\pgfqpoint{4.338948in}{0.492442in}}%
\pgfpathlineto{\pgfqpoint{4.293508in}{0.537923in}}%
\pgfpathlineto{\pgfqpoint{4.245939in}{0.582752in}}%
\pgfpathlineto{\pgfqpoint{4.196084in}{0.626834in}}%
\pgfpathlineto{\pgfqpoint{4.143799in}{0.670072in}}%
\pgfpathlineto{\pgfqpoint{4.089007in}{0.712375in}}%
\pgfpathlineto{\pgfqpoint{4.031693in}{0.753671in}}%
\pgfpathlineto{\pgfqpoint{3.971887in}{0.793903in}}%
\pgfpathlineto{\pgfqpoint{3.909770in}{0.833081in}}%
\pgfpathlineto{\pgfqpoint{3.845643in}{0.871286in}}%
\pgfpathlineto{\pgfqpoint{3.779909in}{0.908672in}}%
\pgfpathlineto{\pgfqpoint{3.713073in}{0.945474in}}%
\pgfpathlineto{\pgfqpoint{3.645708in}{0.981990in}}%
\pgfpathlineto{\pgfqpoint{3.578410in}{1.018541in}}%
\pgfpathlineto{\pgfqpoint{3.511744in}{1.055433in}}%
\pgfpathlineto{\pgfqpoint{3.446234in}{1.092932in}}%
\pgfpathlineto{\pgfqpoint{3.382321in}{1.131240in}}%
\pgfpathlineto{\pgfqpoint{3.320364in}{1.170489in}}%
\pgfpathlineto{\pgfqpoint{3.260634in}{1.210751in}}%
\pgfusepath{stroke}%
\end{pgfscope}%
\begin{pgfscope}%
\pgfpathrectangle{\pgfqpoint{0.647939in}{0.492442in}}{\pgfqpoint{4.273799in}{2.331163in}}%
\pgfusepath{clip}%
\pgfsetbuttcap%
\pgfsetroundjoin%
\pgfsetlinewidth{0.301125pt}%
\definecolor{currentstroke}{rgb}{0.500000,0.500000,0.500000}%
\pgfsetstrokecolor{currentstroke}%
\pgfsetstrokeopacity{0.300000}%
\pgfsetdash{}{0pt}%
\pgfpathmoveto{\pgfqpoint{4.533211in}{0.492442in}}%
\pgfpathlineto{\pgfqpoint{4.533211in}{0.492442in}}%
\pgfpathlineto{\pgfqpoint{4.496871in}{0.540301in}}%
\pgfpathlineto{\pgfqpoint{4.459062in}{0.587818in}}%
\pgfpathlineto{\pgfqpoint{4.419612in}{0.634934in}}%
\pgfpathlineto{\pgfqpoint{4.378324in}{0.681578in}}%
\pgfpathlineto{\pgfqpoint{4.334978in}{0.727662in}}%
\pgfpathlineto{\pgfqpoint{4.289324in}{0.773077in}}%
\pgfpathlineto{\pgfqpoint{4.241092in}{0.817689in}}%
\pgfpathlineto{\pgfqpoint{4.189988in}{0.861340in}}%
\pgfpathlineto{\pgfqpoint{4.135744in}{0.903847in}}%
\pgfpathlineto{\pgfqpoint{4.078160in}{0.945021in}}%
\pgfpathlineto{\pgfqpoint{4.017076in}{0.984664in}}%
\pgfpathlineto{\pgfqpoint{3.952530in}{1.022643in}}%
\pgfpathlineto{\pgfqpoint{3.884779in}{1.058924in}}%
\pgfpathlineto{\pgfqpoint{3.814325in}{1.093648in}}%
\pgfpathlineto{\pgfqpoint{3.741908in}{1.127156in}}%
\pgfpathlineto{\pgfqpoint{3.668363in}{1.159931in}}%
\pgfpathlineto{\pgfqpoint{3.594636in}{1.192582in}}%
\pgfpathlineto{\pgfqpoint{3.521644in}{1.225714in}}%
\pgfpathlineto{\pgfqpoint{3.450175in}{1.259813in}}%
\pgfpathlineto{\pgfqpoint{3.380916in}{1.295234in}}%
\pgfpathlineto{\pgfqpoint{3.314405in}{1.332182in}}%
\pgfpathlineto{\pgfqpoint{3.250998in}{1.370716in}}%
\pgfpathlineto{\pgfqpoint{3.190918in}{1.410806in}}%
\pgfpathlineto{\pgfqpoint{3.134279in}{1.452361in}}%
\pgfpathlineto{\pgfqpoint{3.081120in}{1.495259in}}%
\pgfpathlineto{\pgfqpoint{3.031420in}{1.539379in}}%
\pgfpathlineto{\pgfqpoint{2.985153in}{1.584599in}}%
\pgfpathlineto{\pgfqpoint{2.942303in}{1.630813in}}%
\pgfpathlineto{\pgfqpoint{2.902883in}{1.677926in}}%
\pgfpathlineto{\pgfqpoint{2.866951in}{1.725861in}}%
\pgfpathlineto{\pgfqpoint{2.834635in}{1.774554in}}%
\pgfpathlineto{\pgfqpoint{2.806166in}{1.823957in}}%
\pgfpathlineto{\pgfqpoint{2.781914in}{1.874024in}}%
\pgfpathlineto{\pgfqpoint{2.762467in}{1.924704in}}%
\pgfpathlineto{\pgfqpoint{2.748779in}{1.975934in}}%
\pgfpathlineto{\pgfqpoint{2.742465in}{2.027566in}}%
\pgfpathlineto{\pgfqpoint{2.746574in}{2.079187in}}%
\pgfpathlineto{\pgfqpoint{2.767893in}{2.129214in}}%
\pgfpathlineto{\pgfqpoint{2.767893in}{2.129214in}}%
\pgfpathlineto{\pgfqpoint{2.794260in}{2.155189in}}%
\pgfpathlineto{\pgfqpoint{2.794260in}{2.155189in}}%
\pgfpathlineto{\pgfqpoint{2.825466in}{2.169377in}}%
\pgfpathlineto{\pgfqpoint{2.865171in}{2.173279in}}%
\pgfpathlineto{\pgfqpoint{2.899392in}{2.167871in}}%
\pgfpathlineto{\pgfqpoint{2.933057in}{2.155511in}}%
\pgfpathlineto{\pgfqpoint{2.966894in}{2.135649in}}%
\pgfpathlineto{\pgfqpoint{2.999693in}{2.106931in}}%
\pgfpathlineto{\pgfqpoint{3.026072in}{2.069272in}}%
\pgfpathlineto{\pgfqpoint{3.026072in}{2.069272in}}%
\pgfusepath{stroke}%
\end{pgfscope}%
\begin{pgfscope}%
\pgfpathrectangle{\pgfqpoint{0.647939in}{0.492442in}}{\pgfqpoint{4.273799in}{2.331163in}}%
\pgfusepath{clip}%
\pgfsetbuttcap%
\pgfsetroundjoin%
\pgfsetlinewidth{0.301125pt}%
\definecolor{currentstroke}{rgb}{0.500000,0.500000,0.500000}%
\pgfsetstrokecolor{currentstroke}%
\pgfsetstrokeopacity{0.300000}%
\pgfsetdash{}{0pt}%
\pgfpathmoveto{\pgfqpoint{4.630343in}{0.492442in}}%
\pgfpathlineto{\pgfqpoint{4.630343in}{0.492442in}}%
\pgfpathlineto{\pgfqpoint{4.598344in}{0.541213in}}%
\pgfpathlineto{\pgfqpoint{4.565260in}{0.589769in}}%
\pgfpathlineto{\pgfqpoint{4.530959in}{0.638073in}}%
\pgfpathlineto{\pgfqpoint{4.495303in}{0.686084in}}%
\pgfpathlineto{\pgfqpoint{4.458132in}{0.733750in}}%
\pgfpathlineto{\pgfqpoint{4.419243in}{0.781004in}}%
\pgfpathlineto{\pgfqpoint{4.378397in}{0.827763in}}%
\pgfpathlineto{\pgfqpoint{4.335311in}{0.873917in}}%
\pgfpathlineto{\pgfqpoint{4.289654in}{0.919327in}}%
\pgfpathlineto{\pgfqpoint{4.241040in}{0.963812in}}%
\pgfpathlineto{\pgfqpoint{4.189031in}{1.007138in}}%
\pgfpathlineto{\pgfqpoint{4.133196in}{1.049019in}}%
\pgfpathlineto{\pgfqpoint{4.073166in}{1.089124in}}%
\pgfpathlineto{\pgfqpoint{4.008641in}{1.127092in}}%
\pgfpathlineto{\pgfqpoint{3.939657in}{1.162651in}}%
\pgfpathlineto{\pgfqpoint{3.866635in}{1.195731in}}%
\pgfpathlineto{\pgfqpoint{3.790392in}{1.226590in}}%
\pgfpathlineto{\pgfqpoint{3.712047in}{1.255858in}}%
\pgfpathlineto{\pgfqpoint{3.632891in}{1.284475in}}%
\pgfpathlineto{\pgfqpoint{3.554186in}{1.313453in}}%
\pgfpathlineto{\pgfqpoint{3.477062in}{1.343648in}}%
\pgfpathlineto{\pgfqpoint{3.402468in}{1.375661in}}%
\pgfpathlineto{\pgfqpoint{3.331141in}{1.409814in}}%
\pgfpathlineto{\pgfqpoint{3.263628in}{1.446190in}}%
\pgfpathlineto{\pgfqpoint{3.200204in}{1.484697in}}%
\pgfpathlineto{\pgfqpoint{3.140996in}{1.525156in}}%
\pgfusepath{stroke}%
\end{pgfscope}%
\begin{pgfscope}%
\pgfpathrectangle{\pgfqpoint{0.647939in}{0.492442in}}{\pgfqpoint{4.273799in}{2.331163in}}%
\pgfusepath{clip}%
\pgfsetbuttcap%
\pgfsetroundjoin%
\pgfsetlinewidth{0.301125pt}%
\definecolor{currentstroke}{rgb}{0.500000,0.500000,0.500000}%
\pgfsetstrokecolor{currentstroke}%
\pgfsetstrokeopacity{0.300000}%
\pgfsetdash{}{0pt}%
\pgfpathmoveto{\pgfqpoint{4.727475in}{0.492442in}}%
\pgfpathlineto{\pgfqpoint{4.727475in}{0.492442in}}%
\pgfpathlineto{\pgfqpoint{4.699441in}{0.541935in}}%
\pgfpathlineto{\pgfqpoint{4.670668in}{0.591303in}}%
\pgfpathlineto{\pgfqpoint{4.641088in}{0.640527in}}%
\pgfpathlineto{\pgfqpoint{4.610609in}{0.689587in}}%
\pgfpathlineto{\pgfqpoint{4.579122in}{0.738459in}}%
\pgfpathlineto{\pgfqpoint{4.546513in}{0.787110in}}%
\pgfpathlineto{\pgfqpoint{4.512640in}{0.835501in}}%
\pgfpathlineto{\pgfqpoint{4.477316in}{0.883582in}}%
\pgfpathlineto{\pgfqpoint{4.440315in}{0.931286in}}%
\pgfpathlineto{\pgfqpoint{4.401361in}{0.978526in}}%
\pgfpathlineto{\pgfqpoint{4.360119in}{1.025183in}}%
\pgfpathlineto{\pgfqpoint{4.316164in}{1.071096in}}%
\pgfpathlineto{\pgfqpoint{4.268975in}{1.116038in}}%
\pgfpathlineto{\pgfqpoint{4.217911in}{1.159688in}}%
\pgfpathlineto{\pgfqpoint{4.162197in}{1.201594in}}%
\pgfpathlineto{\pgfqpoint{4.100948in}{1.241120in}}%
\pgfpathlineto{\pgfqpoint{4.033473in}{1.277481in}}%
\pgfpathlineto{\pgfqpoint{3.959584in}{1.309892in}}%
\pgfpathlineto{\pgfqpoint{3.879884in}{1.337930in}}%
\pgfpathlineto{\pgfqpoint{3.795908in}{1.362039in}}%
\pgfpathlineto{\pgfqpoint{3.709595in}{1.383621in}}%
\pgfpathlineto{\pgfqpoint{3.622768in}{1.404589in}}%
\pgfpathlineto{\pgfqpoint{3.537015in}{1.426790in}}%
\pgfpathlineto{\pgfqpoint{3.453739in}{1.451597in}}%
\pgfpathlineto{\pgfqpoint{3.374171in}{1.479768in}}%
\pgfpathlineto{\pgfqpoint{3.299276in}{1.511490in}}%
\pgfpathlineto{\pgfqpoint{3.229568in}{1.546550in}}%
\pgfpathlineto{\pgfqpoint{3.165230in}{1.584562in}}%
\pgfpathlineto{\pgfqpoint{3.106256in}{1.625094in}}%
\pgfpathlineto{\pgfqpoint{3.052511in}{1.667749in}}%
\pgfpathlineto{\pgfqpoint{3.003864in}{1.712185in}}%
\pgfpathlineto{\pgfqpoint{2.960211in}{1.758142in}}%
\pgfpathlineto{\pgfqpoint{2.921567in}{1.805419in}}%
\pgfpathlineto{\pgfqpoint{2.888118in}{1.853864in}}%
\pgfpathlineto{\pgfqpoint{2.860309in}{1.903360in}}%
\pgfpathlineto{\pgfqpoint{2.839012in}{1.953795in}}%
\pgfusepath{stroke}%
\end{pgfscope}%
\begin{pgfscope}%
\pgfpathrectangle{\pgfqpoint{0.647939in}{0.492442in}}{\pgfqpoint{4.273799in}{2.331163in}}%
\pgfusepath{clip}%
\pgfsetbuttcap%
\pgfsetroundjoin%
\pgfsetlinewidth{0.301125pt}%
\definecolor{currentstroke}{rgb}{0.500000,0.500000,0.500000}%
\pgfsetstrokecolor{currentstroke}%
\pgfsetstrokeopacity{0.300000}%
\pgfsetdash{}{0pt}%
\pgfpathmoveto{\pgfqpoint{4.824607in}{0.492442in}}%
\pgfpathlineto{\pgfqpoint{4.824607in}{0.492442in}}%
\pgfpathlineto{\pgfqpoint{4.800114in}{0.542492in}}%
\pgfpathlineto{\pgfqpoint{4.775166in}{0.592475in}}%
\pgfpathlineto{\pgfqpoint{4.749722in}{0.642384in}}%
\pgfpathlineto{\pgfqpoint{4.723742in}{0.692210in}}%
\pgfpathlineto{\pgfqpoint{4.697180in}{0.741944in}}%
\pgfpathlineto{\pgfqpoint{4.669970in}{0.791574in}}%
\pgfpathlineto{\pgfqpoint{4.642050in}{0.841086in}}%
\pgfpathlineto{\pgfqpoint{4.613334in}{0.890462in}}%
\pgfpathlineto{\pgfqpoint{4.583719in}{0.939680in}}%
\pgfpathlineto{\pgfqpoint{4.553091in}{0.988712in}}%
\pgfpathlineto{\pgfqpoint{4.521298in}{1.037521in}}%
\pgfpathlineto{\pgfqpoint{4.488133in}{1.086059in}}%
\pgfpathlineto{\pgfqpoint{4.453361in}{1.134261in}}%
\pgfpathlineto{\pgfqpoint{4.416673in}{1.182034in}}%
\pgfpathlineto{\pgfqpoint{4.377640in}{1.229245in}}%
\pgfpathlineto{\pgfqpoint{4.335668in}{1.275692in}}%
\pgfpathlineto{\pgfqpoint{4.289931in}{1.321058in}}%
\pgfpathlineto{\pgfqpoint{4.239246in}{1.364813in}}%
\pgfpathlineto{\pgfqpoint{4.181927in}{1.406029in}}%
\pgfpathlineto{\pgfqpoint{4.115969in}{1.443087in}}%
\pgfpathlineto{\pgfqpoint{4.039621in}{1.473424in}}%
\pgfpathlineto{\pgfqpoint{3.953566in}{1.494479in}}%
\pgfpathlineto{\pgfqpoint{3.866297in}{1.505953in}}%
\pgfpathlineto{\pgfqpoint{3.772179in}{1.512333in}}%
\pgfpathlineto{\pgfqpoint{3.677696in}{1.517200in}}%
\pgfpathlineto{\pgfqpoint{3.583761in}{1.524309in}}%
\pgfpathlineto{\pgfqpoint{3.491640in}{1.536331in}}%
\pgfpathlineto{\pgfqpoint{3.403164in}{1.554627in}}%
\pgfpathlineto{\pgfqpoint{3.320071in}{1.579323in}}%
\pgfpathlineto{\pgfqpoint{3.243520in}{1.609736in}}%
\pgfpathlineto{\pgfqpoint{3.174034in}{1.644844in}}%
\pgfusepath{stroke}%
\end{pgfscope}%
\begin{pgfscope}%
\pgfpathrectangle{\pgfqpoint{0.647939in}{0.492442in}}{\pgfqpoint{4.273799in}{2.331163in}}%
\pgfusepath{clip}%
\pgfsetbuttcap%
\pgfsetroundjoin%
\pgfsetlinewidth{0.301125pt}%
\definecolor{currentstroke}{rgb}{0.500000,0.500000,0.500000}%
\pgfsetstrokecolor{currentstroke}%
\pgfsetstrokeopacity{0.300000}%
\pgfsetdash{}{0pt}%
\pgfpathmoveto{\pgfqpoint{4.921738in}{0.492442in}}%
\pgfpathlineto{\pgfqpoint{4.921738in}{0.492442in}}%
\pgfpathlineto{\pgfqpoint{4.900349in}{0.542914in}}%
\pgfpathlineto{\pgfqpoint{4.878714in}{0.593355in}}%
\pgfpathlineto{\pgfqpoint{4.856821in}{0.643763in}}%
\pgfpathlineto{\pgfqpoint{4.834656in}{0.694135in}}%
\pgfpathlineto{\pgfqpoint{4.812201in}{0.744469in}}%
\pgfpathlineto{\pgfqpoint{4.789440in}{0.794762in}}%
\pgfpathlineto{\pgfqpoint{4.766350in}{0.845010in}}%
\pgfpathlineto{\pgfqpoint{4.742907in}{0.895209in}}%
\pgfpathlineto{\pgfqpoint{4.719082in}{0.945354in}}%
\pgfpathlineto{\pgfqpoint{4.694844in}{0.995440in}}%
\pgfpathlineto{\pgfqpoint{4.670149in}{1.045460in}}%
\pgfpathlineto{\pgfqpoint{4.644953in}{1.095405in}}%
\pgfpathlineto{\pgfqpoint{4.619192in}{1.145264in}}%
\pgfpathlineto{\pgfqpoint{4.592798in}{1.195025in}}%
\pgfpathlineto{\pgfqpoint{4.565689in}{1.244669in}}%
\pgfpathlineto{\pgfqpoint{4.537735in}{1.294174in}}%
\pgfpathlineto{\pgfqpoint{4.508791in}{1.343508in}}%
\pgfpathlineto{\pgfqpoint{4.478652in}{1.392626in}}%
\pgfpathlineto{\pgfqpoint{4.447000in}{1.441462in}}%
\pgfpathlineto{\pgfqpoint{4.413416in}{1.489909in}}%
\pgfpathlineto{\pgfqpoint{4.377251in}{1.537787in}}%
\pgfpathlineto{\pgfqpoint{4.337368in}{1.584755in}}%
\pgfpathlineto{\pgfqpoint{4.291690in}{1.630071in}}%
\pgfpathlineto{\pgfqpoint{4.235965in}{1.671712in}}%
\pgfpathlineto{\pgfqpoint{4.235965in}{1.671712in}}%
\pgfpathlineto{\pgfqpoint{4.184189in}{1.696339in}}%
\pgfpathlineto{\pgfqpoint{4.184189in}{1.696339in}}%
\pgfpathlineto{\pgfqpoint{4.134640in}{1.707993in}}%
\pgfpathlineto{\pgfqpoint{4.080479in}{1.709675in}}%
\pgfpathlineto{\pgfqpoint{4.029115in}{1.703508in}}%
\pgfpathlineto{\pgfqpoint{3.969453in}{1.690368in}}%
\pgfpathlineto{\pgfqpoint{3.888498in}{1.668076in}}%
\pgfpathlineto{\pgfqpoint{3.803795in}{1.644739in}}%
\pgfpathlineto{\pgfqpoint{3.716518in}{1.624521in}}%
\pgfpathlineto{\pgfqpoint{3.625682in}{1.610103in}}%
\pgfpathlineto{\pgfqpoint{3.531961in}{1.604071in}}%
\pgfpathlineto{\pgfqpoint{3.440333in}{1.608060in}}%
\pgfusepath{stroke}%
\end{pgfscope}%
\begin{pgfscope}%
\pgfpathrectangle{\pgfqpoint{0.647939in}{0.492442in}}{\pgfqpoint{4.273799in}{2.331163in}}%
\pgfusepath{clip}%
\pgfsetbuttcap%
\pgfsetroundjoin%
\pgfsetlinewidth{0.301125pt}%
\definecolor{currentstroke}{rgb}{0.500000,0.500000,0.500000}%
\pgfsetstrokecolor{currentstroke}%
\pgfsetstrokeopacity{0.300000}%
\pgfsetdash{}{0pt}%
\pgfpathmoveto{\pgfqpoint{4.921738in}{0.704366in}}%
\pgfpathlineto{\pgfqpoint{4.921738in}{0.704366in}}%
\pgfpathlineto{\pgfqpoint{4.902120in}{0.755052in}}%
\pgfpathlineto{\pgfqpoint{4.882382in}{0.805723in}}%
\pgfpathlineto{\pgfqpoint{4.862525in}{0.856381in}}%
\pgfpathlineto{\pgfqpoint{4.842549in}{0.907025in}}%
\pgfpathlineto{\pgfqpoint{4.822452in}{0.957654in}}%
\pgfpathlineto{\pgfqpoint{4.802238in}{1.008270in}}%
\pgfpathlineto{\pgfqpoint{4.781906in}{1.058871in}}%
\pgfpathlineto{\pgfqpoint{4.761459in}{1.109458in}}%
\pgfpathlineto{\pgfqpoint{4.740898in}{1.160032in}}%
\pgfpathlineto{\pgfqpoint{4.720230in}{1.210592in}}%
\pgfpathlineto{\pgfqpoint{4.699458in}{1.261140in}}%
\pgfpathlineto{\pgfqpoint{4.678591in}{1.311676in}}%
\pgfpathlineto{\pgfqpoint{4.657638in}{1.362201in}}%
\pgfpathlineto{\pgfqpoint{4.636614in}{1.412717in}}%
\pgfpathlineto{\pgfqpoint{4.615535in}{1.463227in}}%
\pgfpathlineto{\pgfqpoint{4.594426in}{1.513732in}}%
\pgfpathlineto{\pgfqpoint{4.573321in}{1.564237in}}%
\pgfpathlineto{\pgfqpoint{4.552268in}{1.614749in}}%
\pgfpathlineto{\pgfqpoint{4.531331in}{1.665274in}}%
\pgfpathlineto{\pgfqpoint{4.510611in}{1.715826in}}%
\pgfpathlineto{\pgfqpoint{4.490254in}{1.766421in}}%
\pgfpathlineto{\pgfqpoint{4.470502in}{1.817086in}}%
\pgfpathlineto{\pgfqpoint{4.451762in}{1.867861in}}%
\pgfpathlineto{\pgfqpoint{4.434752in}{1.918812in}}%
\pgfpathlineto{\pgfqpoint{4.420901in}{1.970034in}}%
\pgfpathlineto{\pgfqpoint{4.413117in}{2.021609in}}%
\pgfpathlineto{\pgfqpoint{4.416714in}{2.073199in}}%
\pgfpathlineto{\pgfqpoint{4.435131in}{2.123591in}}%
\pgfpathlineto{\pgfqpoint{4.464013in}{2.172609in}}%
\pgfpathlineto{\pgfqpoint{4.497852in}{2.220807in}}%
\pgfpathlineto{\pgfqpoint{4.533577in}{2.268718in}}%
\pgfpathlineto{\pgfqpoint{4.569655in}{2.316574in}}%
\pgfpathlineto{\pgfqpoint{4.605413in}{2.364466in}}%
\pgfpathlineto{\pgfqpoint{4.640648in}{2.412479in}}%
\pgfpathlineto{\pgfqpoint{4.675284in}{2.460645in}}%
\pgfpathlineto{\pgfqpoint{4.709302in}{2.508975in}}%
\pgfpathlineto{\pgfqpoint{4.742671in}{2.557454in}}%
\pgfpathlineto{\pgfqpoint{4.775363in}{2.606060in}}%
\pgfpathlineto{\pgfqpoint{4.807418in}{2.654792in}}%
\pgfpathlineto{\pgfqpoint{4.838891in}{2.703650in}}%
\pgfpathlineto{\pgfqpoint{4.869792in}{2.752623in}}%
\pgfpathlineto{\pgfqpoint{4.900124in}{2.801696in}}%
\pgfpathlineto{\pgfqpoint{4.913538in}{2.823605in}}%
\pgfusepath{stroke}%
\end{pgfscope}%
\begin{pgfscope}%
\pgfpathrectangle{\pgfqpoint{0.647939in}{0.492442in}}{\pgfqpoint{4.273799in}{2.331163in}}%
\pgfusepath{clip}%
\pgfsetbuttcap%
\pgfsetroundjoin%
\pgfsetlinewidth{0.301125pt}%
\definecolor{currentstroke}{rgb}{0.500000,0.500000,0.500000}%
\pgfsetstrokecolor{currentstroke}%
\pgfsetstrokeopacity{0.300000}%
\pgfsetdash{}{0pt}%
\pgfpathmoveto{\pgfqpoint{4.921738in}{0.969271in}}%
\pgfpathlineto{\pgfqpoint{4.921738in}{0.969271in}}%
\pgfpathlineto{\pgfqpoint{4.904707in}{1.020234in}}%
\pgfpathlineto{\pgfqpoint{4.887757in}{1.071205in}}%
\pgfpathlineto{\pgfqpoint{4.870905in}{1.122186in}}%
\pgfpathlineto{\pgfqpoint{4.854177in}{1.173179in}}%
\pgfpathlineto{\pgfqpoint{4.837600in}{1.224186in}}%
\pgfpathlineto{\pgfqpoint{4.821207in}{1.275212in}}%
\pgfpathlineto{\pgfqpoint{4.805040in}{1.326258in}}%
\pgfpathlineto{\pgfqpoint{4.789138in}{1.377329in}}%
\pgfpathlineto{\pgfqpoint{4.773559in}{1.428430in}}%
\pgfpathlineto{\pgfqpoint{4.758368in}{1.479566in}}%
\pgfpathlineto{\pgfqpoint{4.743636in}{1.530741in}}%
\pgfpathlineto{\pgfqpoint{4.729456in}{1.581963in}}%
\pgfpathlineto{\pgfqpoint{4.715938in}{1.633237in}}%
\pgfpathlineto{\pgfqpoint{4.703206in}{1.684572in}}%
\pgfpathlineto{\pgfqpoint{4.691421in}{1.735973in}}%
\pgfpathlineto{\pgfqpoint{4.680773in}{1.787448in}}%
\pgfpathlineto{\pgfqpoint{4.671477in}{1.839000in}}%
\pgfpathlineto{\pgfqpoint{4.663794in}{1.890631in}}%
\pgfpathlineto{\pgfqpoint{4.658024in}{1.942335in}}%
\pgfpathlineto{\pgfqpoint{4.654502in}{1.994098in}}%
\pgfpathlineto{\pgfqpoint{4.653571in}{2.045893in}}%
\pgfpathlineto{\pgfqpoint{4.655558in}{2.097678in}}%
\pgfpathlineto{\pgfqpoint{4.660725in}{2.149395in}}%
\pgfpathlineto{\pgfqpoint{4.669218in}{2.200980in}}%
\pgfpathlineto{\pgfqpoint{4.681010in}{2.252370in}}%
\pgfpathlineto{\pgfqpoint{4.695921in}{2.303517in}}%
\pgfpathlineto{\pgfqpoint{4.713630in}{2.354400in}}%
\pgfpathlineto{\pgfqpoint{4.733707in}{2.405018in}}%
\pgfpathlineto{\pgfqpoint{4.755717in}{2.455398in}}%
\pgfpathlineto{\pgfqpoint{4.779231in}{2.505574in}}%
\pgfpathlineto{\pgfqpoint{4.803873in}{2.555588in}}%
\pgfpathlineto{\pgfqpoint{4.829338in}{2.605483in}}%
\pgfusepath{stroke}%
\end{pgfscope}%
\begin{pgfscope}%
\pgfpathrectangle{\pgfqpoint{0.647939in}{0.492442in}}{\pgfqpoint{4.273799in}{2.331163in}}%
\pgfusepath{clip}%
\pgfsetbuttcap%
\pgfsetroundjoin%
\pgfsetlinewidth{0.301125pt}%
\definecolor{currentstroke}{rgb}{0.500000,0.500000,0.500000}%
\pgfsetstrokecolor{currentstroke}%
\pgfsetstrokeopacity{0.300000}%
\pgfsetdash{}{0pt}%
\pgfpathmoveto{\pgfqpoint{4.921738in}{1.234176in}}%
\pgfpathlineto{\pgfqpoint{4.921738in}{1.234176in}}%
\pgfpathlineto{\pgfqpoint{4.907819in}{1.285420in}}%
\pgfpathlineto{\pgfqpoint{4.894234in}{1.336690in}}%
\pgfpathlineto{\pgfqpoint{4.881023in}{1.387989in}}%
\pgfpathlineto{\pgfqpoint{4.868242in}{1.439321in}}%
\pgfpathlineto{\pgfqpoint{4.855953in}{1.490688in}}%
\pgfpathlineto{\pgfqpoint{4.844223in}{1.542094in}}%
\pgfpathlineto{\pgfqpoint{4.833124in}{1.593542in}}%
\pgfpathlineto{\pgfqpoint{4.822749in}{1.645034in}}%
\pgfpathlineto{\pgfqpoint{4.813196in}{1.696574in}}%
\pgfpathlineto{\pgfqpoint{4.804574in}{1.748163in}}%
\pgfpathlineto{\pgfqpoint{4.797001in}{1.799800in}}%
\pgfpathlineto{\pgfqpoint{4.790611in}{1.851485in}}%
\pgfpathlineto{\pgfqpoint{4.785550in}{1.903213in}}%
\pgfpathlineto{\pgfqpoint{4.781969in}{1.954978in}}%
\pgfpathlineto{\pgfqpoint{4.780019in}{2.006768in}}%
\pgfpathlineto{\pgfqpoint{4.779849in}{2.058569in}}%
\pgfpathlineto{\pgfqpoint{4.781593in}{2.110360in}}%
\pgfpathlineto{\pgfqpoint{4.785360in}{2.162119in}}%
\pgfpathlineto{\pgfqpoint{4.791227in}{2.213820in}}%
\pgfpathlineto{\pgfqpoint{4.799218in}{2.265436in}}%
\pgfpathlineto{\pgfqpoint{4.809303in}{2.316941in}}%
\pgfpathlineto{\pgfqpoint{4.821407in}{2.368316in}}%
\pgfpathlineto{\pgfqpoint{4.835413in}{2.419549in}}%
\pgfpathlineto{\pgfqpoint{4.851152in}{2.470632in}}%
\pgfusepath{stroke}%
\end{pgfscope}%
\begin{pgfscope}%
\pgfpathrectangle{\pgfqpoint{0.647939in}{0.492442in}}{\pgfqpoint{4.273799in}{2.331163in}}%
\pgfusepath{clip}%
\pgfsetbuttcap%
\pgfsetroundjoin%
\pgfsetlinewidth{0.301125pt}%
\definecolor{currentstroke}{rgb}{0.500000,0.500000,0.500000}%
\pgfsetstrokecolor{currentstroke}%
\pgfsetstrokeopacity{0.300000}%
\pgfsetdash{}{0pt}%
\pgfpathmoveto{\pgfqpoint{4.921738in}{1.446100in}}%
\pgfpathlineto{\pgfqpoint{4.921738in}{1.446100in}}%
\pgfpathlineto{\pgfqpoint{4.910790in}{1.497558in}}%
\pgfpathlineto{\pgfqpoint{4.900417in}{1.549050in}}%
\pgfpathlineto{\pgfqpoint{4.890685in}{1.600580in}}%
\pgfpathlineto{\pgfqpoint{4.881663in}{1.652149in}}%
\pgfpathlineto{\pgfqpoint{4.873436in}{1.703757in}}%
\pgfusepath{stroke}%
\end{pgfscope}%
\begin{pgfscope}%
\pgfpathrectangle{\pgfqpoint{0.647939in}{0.492442in}}{\pgfqpoint{4.273799in}{2.331163in}}%
\pgfusepath{clip}%
\pgfsetbuttcap%
\pgfsetroundjoin%
\pgfsetlinewidth{0.301125pt}%
\definecolor{currentstroke}{rgb}{0.500000,0.500000,0.500000}%
\pgfsetstrokecolor{currentstroke}%
\pgfsetstrokeopacity{0.300000}%
\pgfsetdash{}{0pt}%
\pgfpathmoveto{\pgfqpoint{4.921738in}{1.763986in}}%
\pgfpathlineto{\pgfqpoint{4.921738in}{1.763986in}}%
\pgfpathlineto{\pgfqpoint{4.916298in}{1.815704in}}%
\pgfpathlineto{\pgfqpoint{4.911846in}{1.867449in}}%
\pgfpathlineto{\pgfqpoint{4.908462in}{1.919219in}}%
\pgfpathlineto{\pgfqpoint{4.906225in}{1.971007in}}%
\pgfpathlineto{\pgfqpoint{4.905212in}{2.022806in}}%
\pgfpathlineto{\pgfqpoint{4.905497in}{2.074608in}}%
\pgfpathlineto{\pgfqpoint{4.907143in}{2.126403in}}%
\pgfpathlineto{\pgfqpoint{4.910205in}{2.178177in}}%
\pgfpathlineto{\pgfqpoint{4.914721in}{2.229920in}}%
\pgfpathlineto{\pgfqpoint{4.920712in}{2.281619in}}%
\pgfpathlineto{\pgfqpoint{4.921738in}{2.289508in}}%
\pgfusepath{stroke}%
\end{pgfscope}%
\begin{pgfscope}%
\pgfpathrectangle{\pgfqpoint{0.647939in}{0.492442in}}{\pgfqpoint{4.273799in}{2.331163in}}%
\pgfusepath{clip}%
\pgfsetbuttcap%
\pgfsetroundjoin%
\pgfsetlinewidth{0.301125pt}%
\definecolor{currentstroke}{rgb}{0.500000,0.500000,0.500000}%
\pgfsetstrokecolor{currentstroke}%
\pgfsetstrokeopacity{0.300000}%
\pgfsetdash{}{0pt}%
\pgfpathmoveto{\pgfqpoint{4.338948in}{2.823605in}}%
\pgfpathlineto{\pgfqpoint{4.338948in}{2.823605in}}%
\pgfpathlineto{\pgfqpoint{4.389550in}{2.779841in}}%
\pgfpathlineto{\pgfqpoint{4.448769in}{2.739546in}}%
\pgfpathlineto{\pgfqpoint{4.448769in}{2.739546in}}%
\pgfpathlineto{\pgfqpoint{4.503851in}{2.714632in}}%
\pgfpathlineto{\pgfqpoint{4.503851in}{2.714632in}}%
\pgfpathlineto{\pgfqpoint{4.551457in}{2.704151in}}%
\pgfpathlineto{\pgfqpoint{4.603504in}{2.705190in}}%
\pgfpathlineto{\pgfqpoint{4.644993in}{2.715208in}}%
\pgfpathlineto{\pgfqpoint{4.684799in}{2.732295in}}%
\pgfpathlineto{\pgfqpoint{4.727709in}{2.758450in}}%
\pgfpathlineto{\pgfqpoint{4.775845in}{2.796530in}}%
\pgfpathlineto{\pgfqpoint{4.806718in}{2.823605in}}%
\pgfusepath{stroke}%
\end{pgfscope}%
\begin{pgfscope}%
\pgfpathrectangle{\pgfqpoint{0.647939in}{0.492442in}}{\pgfqpoint{4.273799in}{2.331163in}}%
\pgfusepath{clip}%
\pgfsetbuttcap%
\pgfsetroundjoin%
\pgfsetlinewidth{0.301125pt}%
\definecolor{currentstroke}{rgb}{0.500000,0.500000,0.500000}%
\pgfsetstrokecolor{currentstroke}%
\pgfsetstrokeopacity{0.300000}%
\pgfsetdash{}{0pt}%
\pgfpathmoveto{\pgfqpoint{4.241816in}{2.823605in}}%
\pgfpathlineto{\pgfqpoint{4.241816in}{2.823605in}}%
\pgfpathlineto{\pgfqpoint{4.283182in}{2.776986in}}%
\pgfpathlineto{\pgfqpoint{4.327769in}{2.731273in}}%
\pgfpathlineto{\pgfqpoint{4.377594in}{2.687252in}}%
\pgfpathlineto{\pgfqpoint{4.436828in}{2.647084in}}%
\pgfpathlineto{\pgfqpoint{4.436828in}{2.647084in}}%
\pgfpathlineto{\pgfqpoint{4.487474in}{2.625171in}}%
\pgfpathlineto{\pgfqpoint{4.487474in}{2.625171in}}%
\pgfpathlineto{\pgfqpoint{4.532362in}{2.616543in}}%
\pgfpathlineto{\pgfqpoint{4.580360in}{2.619166in}}%
\pgfpathlineto{\pgfqpoint{4.618750in}{2.629800in}}%
\pgfpathlineto{\pgfqpoint{4.657019in}{2.647639in}}%
\pgfpathlineto{\pgfqpoint{4.698786in}{2.674780in}}%
\pgfusepath{stroke}%
\end{pgfscope}%
\begin{pgfscope}%
\pgfpathrectangle{\pgfqpoint{0.647939in}{0.492442in}}{\pgfqpoint{4.273799in}{2.331163in}}%
\pgfusepath{clip}%
\pgfsetbuttcap%
\pgfsetroundjoin%
\pgfsetlinewidth{0.301125pt}%
\definecolor{currentstroke}{rgb}{0.500000,0.500000,0.500000}%
\pgfsetstrokecolor{currentstroke}%
\pgfsetstrokeopacity{0.300000}%
\pgfsetdash{}{0pt}%
\pgfpathmoveto{\pgfqpoint{4.144684in}{2.823605in}}%
\pgfpathlineto{\pgfqpoint{4.144684in}{2.823605in}}%
\pgfpathlineto{\pgfqpoint{4.180772in}{2.775689in}}%
\pgfpathlineto{\pgfqpoint{4.218067in}{2.728052in}}%
\pgfpathlineto{\pgfqpoint{4.257150in}{2.680849in}}%
\pgfpathlineto{\pgfqpoint{4.298986in}{2.634367in}}%
\pgfpathlineto{\pgfqpoint{4.345366in}{2.589226in}}%
\pgfpathlineto{\pgfqpoint{4.400150in}{2.547126in}}%
\pgfpathlineto{\pgfqpoint{4.400150in}{2.547126in}}%
\pgfpathlineto{\pgfqpoint{4.449699in}{2.521913in}}%
\pgfpathlineto{\pgfqpoint{4.449699in}{2.521913in}}%
\pgfpathlineto{\pgfqpoint{4.491986in}{2.511573in}}%
\pgfpathlineto{\pgfqpoint{4.491986in}{2.511573in}}%
\pgfpathlineto{\pgfqpoint{4.531861in}{2.511815in}}%
\pgfpathlineto{\pgfqpoint{4.568648in}{2.520445in}}%
\pgfpathlineto{\pgfqpoint{4.603368in}{2.535605in}}%
\pgfpathlineto{\pgfqpoint{4.642004in}{2.559767in}}%
\pgfusepath{stroke}%
\end{pgfscope}%
\begin{pgfscope}%
\pgfpathrectangle{\pgfqpoint{0.647939in}{0.492442in}}{\pgfqpoint{4.273799in}{2.331163in}}%
\pgfusepath{clip}%
\pgfsetbuttcap%
\pgfsetroundjoin%
\pgfsetlinewidth{0.301125pt}%
\definecolor{currentstroke}{rgb}{0.500000,0.500000,0.500000}%
\pgfsetstrokecolor{currentstroke}%
\pgfsetstrokeopacity{0.300000}%
\pgfsetdash{}{0pt}%
\pgfpathmoveto{\pgfqpoint{4.047552in}{2.823605in}}%
\pgfpathlineto{\pgfqpoint{4.047552in}{2.823605in}}%
\pgfpathlineto{\pgfqpoint{4.080500in}{2.775020in}}%
\pgfpathlineto{\pgfqpoint{4.113739in}{2.726494in}}%
\pgfpathlineto{\pgfqpoint{4.147447in}{2.678065in}}%
\pgfpathlineto{\pgfqpoint{4.181900in}{2.629792in}}%
\pgfpathlineto{\pgfqpoint{4.217521in}{2.581776in}}%
\pgfpathlineto{\pgfqpoint{4.254979in}{2.534186in}}%
\pgfpathlineto{\pgfqpoint{4.295480in}{2.487365in}}%
\pgfpathlineto{\pgfqpoint{4.341575in}{2.442188in}}%
\pgfpathlineto{\pgfqpoint{4.399981in}{2.401971in}}%
\pgfpathlineto{\pgfqpoint{4.399981in}{2.401971in}}%
\pgfpathlineto{\pgfqpoint{4.437422in}{2.388459in}}%
\pgfpathlineto{\pgfqpoint{4.437422in}{2.388459in}}%
\pgfpathlineto{\pgfqpoint{4.472969in}{2.385620in}}%
\pgfpathlineto{\pgfqpoint{4.507255in}{2.391895in}}%
\pgfpathlineto{\pgfqpoint{4.538487in}{2.404664in}}%
\pgfpathlineto{\pgfqpoint{4.573131in}{2.425761in}}%
\pgfusepath{stroke}%
\end{pgfscope}%
\begin{pgfscope}%
\pgfpathrectangle{\pgfqpoint{0.647939in}{0.492442in}}{\pgfqpoint{4.273799in}{2.331163in}}%
\pgfusepath{clip}%
\pgfsetbuttcap%
\pgfsetroundjoin%
\pgfsetlinewidth{0.301125pt}%
\definecolor{currentstroke}{rgb}{0.500000,0.500000,0.500000}%
\pgfsetstrokecolor{currentstroke}%
\pgfsetstrokeopacity{0.300000}%
\pgfsetdash{}{0pt}%
\pgfpathmoveto{\pgfqpoint{3.950420in}{2.823605in}}%
\pgfpathlineto{\pgfqpoint{3.950420in}{2.823605in}}%
\pgfpathlineto{\pgfqpoint{3.981491in}{2.774653in}}%
\pgfpathlineto{\pgfqpoint{4.012369in}{2.725664in}}%
\pgfpathlineto{\pgfqpoint{4.043118in}{2.676651in}}%
\pgfpathlineto{\pgfqpoint{4.073813in}{2.627629in}}%
\pgfpathlineto{\pgfqpoint{4.104532in}{2.578611in}}%
\pgfpathlineto{\pgfqpoint{4.135405in}{2.529622in}}%
\pgfpathlineto{\pgfqpoint{4.166633in}{2.480703in}}%
\pgfpathlineto{\pgfqpoint{4.198496in}{2.431908in}}%
\pgfpathlineto{\pgfqpoint{4.231461in}{2.383330in}}%
\pgfpathlineto{\pgfqpoint{4.266495in}{2.335196in}}%
\pgfpathlineto{\pgfqpoint{4.305845in}{2.288128in}}%
\pgfpathlineto{\pgfqpoint{4.356899in}{2.245101in}}%
\pgfpathlineto{\pgfqpoint{4.356899in}{2.245101in}}%
\pgfpathlineto{\pgfqpoint{4.385078in}{2.233351in}}%
\pgfpathlineto{\pgfqpoint{4.385078in}{2.233351in}}%
\pgfpathlineto{\pgfqpoint{4.413431in}{2.231622in}}%
\pgfpathlineto{\pgfqpoint{4.440089in}{2.238458in}}%
\pgfpathlineto{\pgfqpoint{4.464677in}{2.250947in}}%
\pgfusepath{stroke}%
\end{pgfscope}%
\begin{pgfscope}%
\pgfpathrectangle{\pgfqpoint{0.647939in}{0.492442in}}{\pgfqpoint{4.273799in}{2.331163in}}%
\pgfusepath{clip}%
\pgfsetbuttcap%
\pgfsetroundjoin%
\pgfsetlinewidth{0.301125pt}%
\definecolor{currentstroke}{rgb}{0.500000,0.500000,0.500000}%
\pgfsetstrokecolor{currentstroke}%
\pgfsetstrokeopacity{0.300000}%
\pgfsetdash{}{0pt}%
\pgfpathmoveto{\pgfqpoint{3.853289in}{2.823605in}}%
\pgfpathlineto{\pgfqpoint{3.853289in}{2.823605in}}%
\pgfpathlineto{\pgfqpoint{3.883343in}{2.774464in}}%
\pgfpathlineto{\pgfqpoint{3.912926in}{2.725238in}}%
\pgfpathlineto{\pgfqpoint{3.942045in}{2.675930in}}%
\pgfpathlineto{\pgfqpoint{3.970688in}{2.626539in}}%
\pgfpathlineto{\pgfqpoint{3.998858in}{2.577067in}}%
\pgfpathlineto{\pgfqpoint{4.026552in}{2.527516in}}%
\pgfpathlineto{\pgfqpoint{4.053738in}{2.477881in}}%
\pgfpathlineto{\pgfqpoint{4.080395in}{2.428161in}}%
\pgfpathlineto{\pgfqpoint{4.106470in}{2.378350in}}%
\pgfpathlineto{\pgfqpoint{4.131877in}{2.328437in}}%
\pgfpathlineto{\pgfqpoint{4.156489in}{2.278408in}}%
\pgfpathlineto{\pgfqpoint{4.180051in}{2.228230in}}%
\pgfpathlineto{\pgfqpoint{4.202106in}{2.177854in}}%
\pgfpathlineto{\pgfqpoint{4.221652in}{2.127181in}}%
\pgfpathlineto{\pgfqpoint{4.236181in}{2.076041in}}%
\pgfpathlineto{\pgfqpoint{4.238508in}{2.024494in}}%
\pgfpathlineto{\pgfqpoint{4.238508in}{2.024494in}}%
\pgfpathlineto{\pgfqpoint{4.226032in}{1.986475in}}%
\pgfpathlineto{\pgfqpoint{4.199804in}{1.948916in}}%
\pgfpathlineto{\pgfqpoint{4.163045in}{1.910647in}}%
\pgfpathlineto{\pgfqpoint{4.112863in}{1.867151in}}%
\pgfpathlineto{\pgfqpoint{4.057939in}{1.825179in}}%
\pgfpathlineto{\pgfqpoint{3.998619in}{1.784894in}}%
\pgfpathlineto{\pgfqpoint{3.934782in}{1.746701in}}%
\pgfpathlineto{\pgfqpoint{3.865986in}{1.711184in}}%
\pgfusepath{stroke}%
\end{pgfscope}%
\begin{pgfscope}%
\pgfpathrectangle{\pgfqpoint{0.647939in}{0.492442in}}{\pgfqpoint{4.273799in}{2.331163in}}%
\pgfusepath{clip}%
\pgfsetbuttcap%
\pgfsetroundjoin%
\pgfsetlinewidth{0.301125pt}%
\definecolor{currentstroke}{rgb}{0.500000,0.500000,0.500000}%
\pgfsetstrokecolor{currentstroke}%
\pgfsetstrokeopacity{0.300000}%
\pgfsetdash{}{0pt}%
\pgfpathmoveto{\pgfqpoint{3.756157in}{2.823605in}}%
\pgfpathlineto{\pgfqpoint{3.756157in}{2.823605in}}%
\pgfpathlineto{\pgfqpoint{3.785848in}{2.774399in}}%
\pgfpathlineto{\pgfqpoint{3.814877in}{2.725075in}}%
\pgfpathlineto{\pgfqpoint{3.843214in}{2.675631in}}%
\pgfpathlineto{\pgfqpoint{3.870837in}{2.626068in}}%
\pgfpathlineto{\pgfqpoint{3.897714in}{2.576383in}}%
\pgfpathlineto{\pgfqpoint{3.923785in}{2.526570in}}%
\pgfpathlineto{\pgfqpoint{3.948982in}{2.476624in}}%
\pgfpathlineto{\pgfqpoint{3.973201in}{2.426534in}}%
\pgfpathlineto{\pgfqpoint{3.996307in}{2.376288in}}%
\pgfpathlineto{\pgfqpoint{4.018108in}{2.325870in}}%
\pgfpathlineto{\pgfqpoint{4.038327in}{2.275258in}}%
\pgfpathlineto{\pgfqpoint{4.056559in}{2.224423in}}%
\pgfpathlineto{\pgfqpoint{4.072216in}{2.173336in}}%
\pgfpathlineto{\pgfqpoint{4.084415in}{2.121977in}}%
\pgfpathlineto{\pgfqpoint{4.091812in}{2.070354in}}%
\pgfpathlineto{\pgfqpoint{4.092491in}{2.018600in}}%
\pgfpathlineto{\pgfqpoint{4.084047in}{1.967099in}}%
\pgfpathlineto{\pgfqpoint{4.064349in}{1.916584in}}%
\pgfpathlineto{\pgfqpoint{4.032732in}{1.867941in}}%
\pgfpathlineto{\pgfqpoint{3.990181in}{1.821849in}}%
\pgfusepath{stroke}%
\end{pgfscope}%
\begin{pgfscope}%
\pgfpathrectangle{\pgfqpoint{0.647939in}{0.492442in}}{\pgfqpoint{4.273799in}{2.331163in}}%
\pgfusepath{clip}%
\pgfsetbuttcap%
\pgfsetroundjoin%
\pgfsetlinewidth{0.301125pt}%
\definecolor{currentstroke}{rgb}{0.500000,0.500000,0.500000}%
\pgfsetstrokecolor{currentstroke}%
\pgfsetstrokeopacity{0.300000}%
\pgfsetdash{}{0pt}%
\pgfpathmoveto{\pgfqpoint{3.659025in}{2.823605in}}%
\pgfpathlineto{\pgfqpoint{3.659025in}{2.823605in}}%
\pgfpathlineto{\pgfqpoint{3.688892in}{2.774431in}}%
\pgfpathlineto{\pgfqpoint{3.717936in}{2.725109in}}%
\pgfpathlineto{\pgfqpoint{3.746129in}{2.675642in}}%
\pgfpathlineto{\pgfqpoint{3.773443in}{2.626027in}}%
\pgfpathlineto{\pgfqpoint{3.799818in}{2.576262in}}%
\pgfpathlineto{\pgfqpoint{3.825193in}{2.526343in}}%
\pgfpathlineto{\pgfqpoint{3.849485in}{2.476264in}}%
\pgfpathlineto{\pgfqpoint{3.872575in}{2.426016in}}%
\pgfpathlineto{\pgfqpoint{3.894312in}{2.375589in}}%
\pgfpathlineto{\pgfqpoint{3.914494in}{2.324971in}}%
\pgfpathlineto{\pgfqpoint{3.932848in}{2.274148in}}%
\pgfpathlineto{\pgfqpoint{3.949011in}{2.223106in}}%
\pgfpathlineto{\pgfqpoint{3.962491in}{2.171834in}}%
\pgfpathlineto{\pgfqpoint{3.972607in}{2.120337in}}%
\pgfpathlineto{\pgfqpoint{3.978453in}{2.068649in}}%
\pgfpathlineto{\pgfqpoint{3.978839in}{2.016876in}}%
\pgfpathlineto{\pgfqpoint{3.972295in}{1.965248in}}%
\pgfpathlineto{\pgfqpoint{3.957209in}{1.914185in}}%
\pgfpathlineto{\pgfqpoint{3.932150in}{1.864334in}}%
\pgfpathlineto{\pgfqpoint{3.896281in}{1.816519in}}%
\pgfpathlineto{\pgfqpoint{3.849422in}{1.771652in}}%
\pgfpathlineto{\pgfqpoint{3.791871in}{1.730699in}}%
\pgfpathlineto{\pgfqpoint{3.723774in}{1.694918in}}%
\pgfpathlineto{\pgfqpoint{3.645576in}{1.666078in}}%
\pgfpathlineto{\pgfqpoint{3.559706in}{1.646818in}}%
\pgfpathlineto{\pgfqpoint{3.476450in}{1.639290in}}%
\pgfpathlineto{\pgfqpoint{3.398148in}{1.641936in}}%
\pgfpathlineto{\pgfqpoint{3.323936in}{1.653288in}}%
\pgfpathlineto{\pgfqpoint{3.252638in}{1.672738in}}%
\pgfpathlineto{\pgfqpoint{3.183592in}{1.700269in}}%
\pgfpathlineto{\pgfqpoint{3.116200in}{1.736473in}}%
\pgfpathlineto{\pgfqpoint{3.057466in}{1.776988in}}%
\pgfpathlineto{\pgfqpoint{3.006752in}{1.820649in}}%
\pgfpathlineto{\pgfqpoint{2.963715in}{1.866712in}}%
\pgfusepath{stroke}%
\end{pgfscope}%
\begin{pgfscope}%
\pgfpathrectangle{\pgfqpoint{0.647939in}{0.492442in}}{\pgfqpoint{4.273799in}{2.331163in}}%
\pgfusepath{clip}%
\pgfsetbuttcap%
\pgfsetroundjoin%
\pgfsetlinewidth{0.301125pt}%
\definecolor{currentstroke}{rgb}{0.500000,0.500000,0.500000}%
\pgfsetstrokecolor{currentstroke}%
\pgfsetstrokeopacity{0.300000}%
\pgfsetdash{}{0pt}%
\pgfpathmoveto{\pgfqpoint{3.561893in}{2.823605in}}%
\pgfpathlineto{\pgfqpoint{3.561893in}{2.823605in}}%
\pgfpathlineto{\pgfqpoint{3.592408in}{2.774549in}}%
\pgfpathlineto{\pgfqpoint{3.621963in}{2.725319in}}%
\pgfpathlineto{\pgfqpoint{3.650536in}{2.675916in}}%
\pgfpathlineto{\pgfqpoint{3.678087in}{2.626341in}}%
\pgfpathlineto{\pgfqpoint{3.704559in}{2.576592in}}%
\pgfpathlineto{\pgfqpoint{3.729889in}{2.526666in}}%
\pgfpathlineto{\pgfqpoint{3.753982in}{2.476559in}}%
\pgfpathlineto{\pgfqpoint{3.776724in}{2.426264in}}%
\pgfpathlineto{\pgfqpoint{3.797965in}{2.375774in}}%
\pgfpathlineto{\pgfqpoint{3.817510in}{2.325082in}}%
\pgfpathlineto{\pgfqpoint{3.835111in}{2.274180in}}%
\pgfpathlineto{\pgfqpoint{3.850431in}{2.223060in}}%
\pgfpathlineto{\pgfqpoint{3.863043in}{2.171723in}}%
\pgfpathlineto{\pgfqpoint{3.872380in}{2.120181in}}%
\pgfpathlineto{\pgfqpoint{3.877698in}{2.068474in}}%
\pgfpathlineto{\pgfqpoint{3.878048in}{2.016695in}}%
\pgfpathlineto{\pgfqpoint{3.872238in}{1.965029in}}%
\pgfpathlineto{\pgfqpoint{3.858856in}{1.913805in}}%
\pgfpathlineto{\pgfqpoint{3.836395in}{1.863563in}}%
\pgfpathlineto{\pgfqpoint{3.803363in}{1.815126in}}%
\pgfpathlineto{\pgfqpoint{3.758571in}{1.769643in}}%
\pgfpathlineto{\pgfqpoint{3.701217in}{1.728676in}}%
\pgfpathlineto{\pgfqpoint{3.630882in}{1.694405in}}%
\pgfusepath{stroke}%
\end{pgfscope}%
\begin{pgfscope}%
\pgfpathrectangle{\pgfqpoint{0.647939in}{0.492442in}}{\pgfqpoint{4.273799in}{2.331163in}}%
\pgfusepath{clip}%
\pgfsetbuttcap%
\pgfsetroundjoin%
\pgfsetlinewidth{0.301125pt}%
\definecolor{currentstroke}{rgb}{0.500000,0.500000,0.500000}%
\pgfsetstrokecolor{currentstroke}%
\pgfsetstrokeopacity{0.300000}%
\pgfsetdash{}{0pt}%
\pgfpathmoveto{\pgfqpoint{3.367630in}{2.823605in}}%
\pgfpathlineto{\pgfqpoint{3.367630in}{2.823605in}}%
\pgfpathlineto{\pgfqpoint{3.400838in}{2.775073in}}%
\pgfpathlineto{\pgfqpoint{3.432822in}{2.726297in}}%
\pgfpathlineto{\pgfqpoint{3.463555in}{2.677283in}}%
\pgfpathlineto{\pgfqpoint{3.493000in}{2.628034in}}%
\pgfpathlineto{\pgfqpoint{3.521115in}{2.578554in}}%
\pgfpathlineto{\pgfqpoint{3.547835in}{2.528845in}}%
\pgfpathlineto{\pgfqpoint{3.573075in}{2.478906in}}%
\pgfpathlineto{\pgfqpoint{3.596729in}{2.428738in}}%
\pgfpathlineto{\pgfqpoint{3.618655in}{2.378336in}}%
\pgfpathlineto{\pgfqpoint{3.638676in}{2.327700in}}%
\pgfpathlineto{\pgfqpoint{3.656564in}{2.276828in}}%
\pgfpathlineto{\pgfqpoint{3.672020in}{2.225721in}}%
\pgfpathlineto{\pgfqpoint{3.684667in}{2.174387in}}%
\pgfpathlineto{\pgfqpoint{3.694016in}{2.122845in}}%
\pgfpathlineto{\pgfqpoint{3.699420in}{2.071140in}}%
\pgfpathlineto{\pgfqpoint{3.700040in}{2.019359in}}%
\pgfpathlineto{\pgfqpoint{3.694772in}{1.967671in}}%
\pgfpathlineto{\pgfqpoint{3.682169in}{1.916385in}}%
\pgfpathlineto{\pgfqpoint{3.660366in}{1.866063in}}%
\pgfpathlineto{\pgfqpoint{3.626952in}{1.817744in}}%
\pgfpathlineto{\pgfqpoint{3.579072in}{1.773350in}}%
\pgfpathlineto{\pgfqpoint{3.579072in}{1.773350in}}%
\pgfpathlineto{\pgfqpoint{3.525533in}{1.741446in}}%
\pgfpathlineto{\pgfqpoint{3.458276in}{1.717895in}}%
\pgfpathlineto{\pgfqpoint{3.391307in}{1.707520in}}%
\pgfpathlineto{\pgfqpoint{3.327654in}{1.707667in}}%
\pgfpathlineto{\pgfqpoint{3.265940in}{1.716363in}}%
\pgfusepath{stroke}%
\end{pgfscope}%
\begin{pgfscope}%
\pgfpathrectangle{\pgfqpoint{0.647939in}{0.492442in}}{\pgfqpoint{4.273799in}{2.331163in}}%
\pgfusepath{clip}%
\pgfsetbuttcap%
\pgfsetroundjoin%
\pgfsetlinewidth{0.301125pt}%
\definecolor{currentstroke}{rgb}{0.500000,0.500000,0.500000}%
\pgfsetstrokecolor{currentstroke}%
\pgfsetstrokeopacity{0.300000}%
\pgfsetdash{}{0pt}%
\pgfpathmoveto{\pgfqpoint{3.270498in}{2.823605in}}%
\pgfpathlineto{\pgfqpoint{3.270498in}{2.823605in}}%
\pgfpathlineto{\pgfqpoint{3.305806in}{2.775518in}}%
\pgfpathlineto{\pgfqpoint{3.339729in}{2.727134in}}%
\pgfpathlineto{\pgfqpoint{3.372246in}{2.678463in}}%
\pgfpathlineto{\pgfqpoint{3.403330in}{2.629515in}}%
\pgfpathlineto{\pgfqpoint{3.432945in}{2.580297in}}%
\pgfpathlineto{\pgfqpoint{3.461031in}{2.530813in}}%
\pgfpathlineto{\pgfqpoint{3.487508in}{2.481066in}}%
\pgfpathlineto{\pgfqpoint{3.512279in}{2.431058in}}%
\pgfpathlineto{\pgfqpoint{3.535204in}{2.380790in}}%
\pgfpathlineto{\pgfqpoint{3.556116in}{2.330262in}}%
\pgfpathlineto{\pgfqpoint{3.574792in}{2.279474in}}%
\pgfpathlineto{\pgfqpoint{3.590941in}{2.228431in}}%
\pgfpathlineto{\pgfqpoint{3.604193in}{2.177143in}}%
\pgfusepath{stroke}%
\end{pgfscope}%
\begin{pgfscope}%
\pgfpathrectangle{\pgfqpoint{0.647939in}{0.492442in}}{\pgfqpoint{4.273799in}{2.331163in}}%
\pgfusepath{clip}%
\pgfsetbuttcap%
\pgfsetroundjoin%
\pgfsetlinewidth{0.301125pt}%
\definecolor{currentstroke}{rgb}{0.500000,0.500000,0.500000}%
\pgfsetstrokecolor{currentstroke}%
\pgfsetstrokeopacity{0.300000}%
\pgfsetdash{}{0pt}%
\pgfpathmoveto{\pgfqpoint{3.173366in}{2.823605in}}%
\pgfpathlineto{\pgfqpoint{3.173366in}{2.823605in}}%
\pgfpathlineto{\pgfqpoint{3.211333in}{2.776125in}}%
\pgfpathlineto{\pgfqpoint{3.247726in}{2.728279in}}%
\pgfpathlineto{\pgfqpoint{3.282535in}{2.680084in}}%
\pgfpathlineto{\pgfqpoint{3.315743in}{2.631553in}}%
\pgfpathlineto{\pgfqpoint{3.347323in}{2.582700in}}%
\pgfpathlineto{\pgfqpoint{3.377222in}{2.533534in}}%
\pgfpathlineto{\pgfqpoint{3.405368in}{2.484061in}}%
\pgfusepath{stroke}%
\end{pgfscope}%
\begin{pgfscope}%
\pgfpathrectangle{\pgfqpoint{0.647939in}{0.492442in}}{\pgfqpoint{4.273799in}{2.331163in}}%
\pgfusepath{clip}%
\pgfsetbuttcap%
\pgfsetroundjoin%
\pgfsetlinewidth{0.301125pt}%
\definecolor{currentstroke}{rgb}{0.500000,0.500000,0.500000}%
\pgfsetstrokecolor{currentstroke}%
\pgfsetstrokeopacity{0.300000}%
\pgfsetdash{}{0pt}%
\pgfpathmoveto{\pgfqpoint{2.979102in}{2.823605in}}%
\pgfpathlineto{\pgfqpoint{2.979102in}{2.823605in}}%
\pgfpathlineto{\pgfqpoint{3.024304in}{2.778050in}}%
\pgfpathlineto{\pgfqpoint{3.067443in}{2.731905in}}%
\pgfpathlineto{\pgfqpoint{3.108535in}{2.685205in}}%
\pgfpathlineto{\pgfqpoint{3.147583in}{2.637987in}}%
\pgfpathlineto{\pgfqpoint{3.184584in}{2.590282in}}%
\pgfpathlineto{\pgfqpoint{3.219516in}{2.542116in}}%
\pgfpathlineto{\pgfqpoint{3.252338in}{2.493511in}}%
\pgfpathlineto{\pgfqpoint{3.282976in}{2.444482in}}%
\pgfpathlineto{\pgfqpoint{3.311330in}{2.395047in}}%
\pgfpathlineto{\pgfqpoint{3.337261in}{2.345218in}}%
\pgfpathlineto{\pgfqpoint{3.360563in}{2.295005in}}%
\pgfpathlineto{\pgfqpoint{3.380966in}{2.244420in}}%
\pgfpathlineto{\pgfqpoint{3.398097in}{2.193478in}}%
\pgfpathlineto{\pgfqpoint{3.411436in}{2.142201in}}%
\pgfpathlineto{\pgfqpoint{3.420257in}{2.090641in}}%
\pgfpathlineto{\pgfqpoint{3.423503in}{2.038900in}}%
\pgfpathlineto{\pgfqpoint{3.419569in}{1.987198in}}%
\pgfpathlineto{\pgfqpoint{3.405872in}{1.936048in}}%
\pgfpathlineto{\pgfqpoint{3.377882in}{1.886822in}}%
\pgfpathlineto{\pgfqpoint{3.377882in}{1.886822in}}%
\pgfpathlineto{\pgfqpoint{3.342275in}{1.852844in}}%
\pgfpathlineto{\pgfqpoint{3.342275in}{1.852844in}}%
\pgfpathlineto{\pgfqpoint{3.301630in}{1.831272in}}%
\pgfpathlineto{\pgfqpoint{3.301630in}{1.831272in}}%
\pgfpathlineto{\pgfqpoint{3.257657in}{1.820257in}}%
\pgfpathlineto{\pgfqpoint{3.209300in}{1.818765in}}%
\pgfpathlineto{\pgfqpoint{3.163814in}{1.825648in}}%
\pgfpathlineto{\pgfqpoint{3.118467in}{1.839958in}}%
\pgfpathlineto{\pgfqpoint{3.072321in}{1.862384in}}%
\pgfusepath{stroke}%
\end{pgfscope}%
\begin{pgfscope}%
\pgfpathrectangle{\pgfqpoint{0.647939in}{0.492442in}}{\pgfqpoint{4.273799in}{2.331163in}}%
\pgfusepath{clip}%
\pgfsetbuttcap%
\pgfsetroundjoin%
\pgfsetlinewidth{0.301125pt}%
\definecolor{currentstroke}{rgb}{0.500000,0.500000,0.500000}%
\pgfsetstrokecolor{currentstroke}%
\pgfsetstrokeopacity{0.300000}%
\pgfsetdash{}{0pt}%
\pgfpathmoveto{\pgfqpoint{2.784839in}{2.823605in}}%
\pgfpathlineto{\pgfqpoint{2.784839in}{2.823605in}}%
\pgfpathlineto{\pgfqpoint{2.839999in}{2.781452in}}%
\pgfpathlineto{\pgfqpoint{2.892491in}{2.738293in}}%
\pgfpathlineto{\pgfqpoint{2.942309in}{2.694200in}}%
\pgfpathlineto{\pgfqpoint{2.989468in}{2.649246in}}%
\pgfpathlineto{\pgfqpoint{3.033992in}{2.603500in}}%
\pgfpathlineto{\pgfqpoint{3.075899in}{2.557024in}}%
\pgfpathlineto{\pgfqpoint{3.115193in}{2.509872in}}%
\pgfpathlineto{\pgfqpoint{3.151848in}{2.462091in}}%
\pgfpathlineto{\pgfqpoint{3.185813in}{2.413721in}}%
\pgfpathlineto{\pgfqpoint{3.216990in}{2.364799in}}%
\pgfpathlineto{\pgfqpoint{3.245223in}{2.315348in}}%
\pgfpathlineto{\pgfqpoint{3.270275in}{2.265391in}}%
\pgfpathlineto{\pgfqpoint{3.291810in}{2.214950in}}%
\pgfpathlineto{\pgfqpoint{3.309332in}{2.164052in}}%
\pgfusepath{stroke}%
\end{pgfscope}%
\begin{pgfscope}%
\pgfpathrectangle{\pgfqpoint{0.647939in}{0.492442in}}{\pgfqpoint{4.273799in}{2.331163in}}%
\pgfusepath{clip}%
\pgfsetbuttcap%
\pgfsetroundjoin%
\pgfsetlinewidth{0.301125pt}%
\definecolor{currentstroke}{rgb}{0.500000,0.500000,0.500000}%
\pgfsetstrokecolor{currentstroke}%
\pgfsetstrokeopacity{0.300000}%
\pgfsetdash{}{0pt}%
\pgfpathmoveto{\pgfqpoint{2.590575in}{2.823605in}}%
\pgfpathlineto{\pgfqpoint{2.590575in}{2.823605in}}%
\pgfpathlineto{\pgfqpoint{2.657485in}{2.786868in}}%
\pgfpathlineto{\pgfqpoint{2.721351in}{2.748556in}}%
\pgfpathlineto{\pgfqpoint{2.781986in}{2.708712in}}%
\pgfpathlineto{\pgfqpoint{2.839291in}{2.667430in}}%
\pgfpathlineto{\pgfqpoint{2.893265in}{2.624827in}}%
\pgfpathlineto{\pgfqpoint{2.943930in}{2.581029in}}%
\pgfpathlineto{\pgfqpoint{2.991325in}{2.536157in}}%
\pgfusepath{stroke}%
\end{pgfscope}%
\begin{pgfscope}%
\pgfpathrectangle{\pgfqpoint{0.647939in}{0.492442in}}{\pgfqpoint{4.273799in}{2.331163in}}%
\pgfusepath{clip}%
\pgfsetbuttcap%
\pgfsetroundjoin%
\pgfsetlinewidth{0.301125pt}%
\definecolor{currentstroke}{rgb}{0.500000,0.500000,0.500000}%
\pgfsetstrokecolor{currentstroke}%
\pgfsetstrokeopacity{0.300000}%
\pgfsetdash{}{0pt}%
\pgfpathmoveto{\pgfqpoint{2.493443in}{2.823605in}}%
\pgfpathlineto{\pgfqpoint{2.493443in}{2.823605in}}%
\pgfpathlineto{\pgfqpoint{2.565953in}{2.790183in}}%
\pgfpathlineto{\pgfqpoint{2.635626in}{2.755016in}}%
\pgfpathlineto{\pgfqpoint{2.702032in}{2.718018in}}%
\pgfpathlineto{\pgfqpoint{2.764913in}{2.679233in}}%
\pgfpathlineto{\pgfqpoint{2.824158in}{2.638777in}}%
\pgfusepath{stroke}%
\end{pgfscope}%
\begin{pgfscope}%
\pgfpathrectangle{\pgfqpoint{0.647939in}{0.492442in}}{\pgfqpoint{4.273799in}{2.331163in}}%
\pgfusepath{clip}%
\pgfsetbuttcap%
\pgfsetroundjoin%
\pgfsetlinewidth{0.301125pt}%
\definecolor{currentstroke}{rgb}{0.500000,0.500000,0.500000}%
\pgfsetstrokecolor{currentstroke}%
\pgfsetstrokeopacity{0.300000}%
\pgfsetdash{}{0pt}%
\pgfpathmoveto{\pgfqpoint{2.299180in}{2.823605in}}%
\pgfpathlineto{\pgfqpoint{2.299180in}{2.823605in}}%
\pgfpathlineto{\pgfqpoint{2.379826in}{2.796254in}}%
\pgfpathlineto{\pgfqpoint{2.459226in}{2.767854in}}%
\pgfpathlineto{\pgfqpoint{2.536329in}{2.737653in}}%
\pgfpathlineto{\pgfqpoint{2.610306in}{2.705231in}}%
\pgfpathlineto{\pgfqpoint{2.680599in}{2.670458in}}%
\pgfpathlineto{\pgfqpoint{2.746854in}{2.633398in}}%
\pgfpathlineto{\pgfqpoint{2.808903in}{2.594230in}}%
\pgfpathlineto{\pgfqpoint{2.866741in}{2.553183in}}%
\pgfpathlineto{\pgfqpoint{2.920419in}{2.510490in}}%
\pgfpathlineto{\pgfqpoint{2.970044in}{2.466354in}}%
\pgfpathlineto{\pgfqpoint{3.015696in}{2.420954in}}%
\pgfpathlineto{\pgfqpoint{3.057415in}{2.374440in}}%
\pgfpathlineto{\pgfqpoint{3.095173in}{2.326930in}}%
\pgfpathlineto{\pgfqpoint{3.128849in}{2.278518in}}%
\pgfpathlineto{\pgfqpoint{3.158186in}{2.229272in}}%
\pgfpathlineto{\pgfqpoint{3.182721in}{2.179254in}}%
\pgfpathlineto{\pgfqpoint{3.201677in}{2.128530in}}%
\pgfpathlineto{\pgfqpoint{3.213693in}{2.077198in}}%
\pgfpathlineto{\pgfqpoint{3.216203in}{2.025514in}}%
\pgfpathlineto{\pgfqpoint{3.203483in}{1.974480in}}%
\pgfpathlineto{\pgfqpoint{3.203483in}{1.974480in}}%
\pgfpathlineto{\pgfqpoint{3.181785in}{1.944684in}}%
\pgfpathlineto{\pgfqpoint{3.181785in}{1.944684in}}%
\pgfpathlineto{\pgfqpoint{3.154128in}{1.927575in}}%
\pgfpathlineto{\pgfqpoint{3.154128in}{1.927575in}}%
\pgfpathlineto{\pgfqpoint{3.123528in}{1.921083in}}%
\pgfpathlineto{\pgfqpoint{3.091288in}{1.923297in}}%
\pgfpathlineto{\pgfqpoint{3.060439in}{1.932470in}}%
\pgfpathlineto{\pgfqpoint{3.028966in}{1.948932in}}%
\pgfusepath{stroke}%
\end{pgfscope}%
\begin{pgfscope}%
\pgfpathrectangle{\pgfqpoint{0.647939in}{0.492442in}}{\pgfqpoint{4.273799in}{2.331163in}}%
\pgfusepath{clip}%
\pgfsetbuttcap%
\pgfsetroundjoin%
\pgfsetlinewidth{0.301125pt}%
\definecolor{currentstroke}{rgb}{0.500000,0.500000,0.500000}%
\pgfsetstrokecolor{currentstroke}%
\pgfsetstrokeopacity{0.300000}%
\pgfsetdash{}{0pt}%
\pgfpathmoveto{\pgfqpoint{2.007784in}{2.823605in}}%
\pgfpathlineto{\pgfqpoint{2.007784in}{2.823605in}}%
\pgfpathlineto{\pgfqpoint{2.087576in}{2.795623in}}%
\pgfpathlineto{\pgfqpoint{2.171131in}{2.771070in}}%
\pgfpathlineto{\pgfqpoint{2.256722in}{2.748648in}}%
\pgfpathlineto{\pgfqpoint{2.342711in}{2.726674in}}%
\pgfpathlineto{\pgfqpoint{2.427647in}{2.703545in}}%
\pgfpathlineto{\pgfqpoint{2.510241in}{2.678060in}}%
\pgfpathlineto{\pgfqpoint{2.589412in}{2.649541in}}%
\pgfpathlineto{\pgfqpoint{2.664300in}{2.617783in}}%
\pgfpathlineto{\pgfqpoint{2.734391in}{2.582932in}}%
\pgfpathlineto{\pgfqpoint{2.799524in}{2.545310in}}%
\pgfusepath{stroke}%
\end{pgfscope}%
\begin{pgfscope}%
\pgfpathrectangle{\pgfqpoint{0.647939in}{0.492442in}}{\pgfqpoint{4.273799in}{2.331163in}}%
\pgfusepath{clip}%
\pgfsetbuttcap%
\pgfsetroundjoin%
\pgfsetlinewidth{0.301125pt}%
\definecolor{currentstroke}{rgb}{0.500000,0.500000,0.500000}%
\pgfsetstrokecolor{currentstroke}%
\pgfsetstrokeopacity{0.300000}%
\pgfsetdash{}{0pt}%
\pgfpathmoveto{\pgfqpoint{1.813521in}{2.823605in}}%
\pgfpathlineto{\pgfqpoint{1.813521in}{2.823605in}}%
\pgfpathlineto{\pgfqpoint{1.880500in}{2.786996in}}%
\pgfpathlineto{\pgfqpoint{1.954908in}{2.754997in}}%
\pgfpathlineto{\pgfqpoint{2.036270in}{2.728510in}}%
\pgfpathlineto{\pgfqpoint{2.122789in}{2.707331in}}%
\pgfpathlineto{\pgfqpoint{2.212145in}{2.689866in}}%
\pgfpathlineto{\pgfqpoint{2.302355in}{2.673721in}}%
\pgfpathlineto{\pgfqpoint{2.391879in}{2.656523in}}%
\pgfpathlineto{\pgfqpoint{2.479303in}{2.636456in}}%
\pgfpathlineto{\pgfqpoint{2.563278in}{2.612467in}}%
\pgfusepath{stroke}%
\end{pgfscope}%
\begin{pgfscope}%
\pgfpathrectangle{\pgfqpoint{0.647939in}{0.492442in}}{\pgfqpoint{4.273799in}{2.331163in}}%
\pgfusepath{clip}%
\pgfsetbuttcap%
\pgfsetroundjoin%
\pgfsetlinewidth{0.301125pt}%
\definecolor{currentstroke}{rgb}{0.500000,0.500000,0.500000}%
\pgfsetstrokecolor{currentstroke}%
\pgfsetstrokeopacity{0.300000}%
\pgfsetdash{}{0pt}%
\pgfpathmoveto{\pgfqpoint{1.716389in}{2.823605in}}%
\pgfpathlineto{\pgfqpoint{1.716389in}{2.823605in}}%
\pgfpathlineto{\pgfqpoint{1.773644in}{2.782339in}}%
\pgfpathlineto{\pgfqpoint{1.838094in}{2.744420in}}%
\pgfpathlineto{\pgfqpoint{1.910941in}{2.711423in}}%
\pgfpathlineto{\pgfqpoint{1.992221in}{2.684985in}}%
\pgfpathlineto{\pgfqpoint{2.079976in}{2.665565in}}%
\pgfusepath{stroke}%
\end{pgfscope}%
\begin{pgfscope}%
\pgfpathrectangle{\pgfqpoint{0.647939in}{0.492442in}}{\pgfqpoint{4.273799in}{2.331163in}}%
\pgfusepath{clip}%
\pgfsetbuttcap%
\pgfsetroundjoin%
\pgfsetlinewidth{0.301125pt}%
\definecolor{currentstroke}{rgb}{0.500000,0.500000,0.500000}%
\pgfsetstrokecolor{currentstroke}%
\pgfsetstrokeopacity{0.300000}%
\pgfsetdash{}{0pt}%
\pgfpathmoveto{\pgfqpoint{1.619257in}{2.823605in}}%
\pgfpathlineto{\pgfqpoint{1.619257in}{2.823605in}}%
\pgfpathlineto{\pgfqpoint{1.666383in}{2.778669in}}%
\pgfpathlineto{\pgfqpoint{1.718825in}{2.735544in}}%
\pgfpathlineto{\pgfqpoint{1.778342in}{2.695284in}}%
\pgfpathlineto{\pgfqpoint{1.846898in}{2.659686in}}%
\pgfpathlineto{\pgfqpoint{1.925789in}{2.631381in}}%
\pgfpathlineto{\pgfqpoint{2.013473in}{2.612497in}}%
\pgfpathlineto{\pgfqpoint{2.103061in}{2.602415in}}%
\pgfpathlineto{\pgfqpoint{2.197390in}{2.596947in}}%
\pgfpathlineto{\pgfqpoint{2.291936in}{2.592377in}}%
\pgfpathlineto{\pgfqpoint{2.385872in}{2.585222in}}%
\pgfpathlineto{\pgfqpoint{2.477934in}{2.573118in}}%
\pgfpathlineto{\pgfqpoint{2.566443in}{2.554892in}}%
\pgfpathlineto{\pgfqpoint{2.649757in}{2.530454in}}%
\pgfpathlineto{\pgfqpoint{2.726803in}{2.500457in}}%
\pgfpathlineto{\pgfqpoint{2.797183in}{2.465849in}}%
\pgfpathlineto{\pgfqpoint{2.860911in}{2.427580in}}%
\pgfpathlineto{\pgfqpoint{2.918320in}{2.386415in}}%
\pgfpathlineto{\pgfqpoint{2.969743in}{2.342941in}}%
\pgfusepath{stroke}%
\end{pgfscope}%
\begin{pgfscope}%
\pgfpathrectangle{\pgfqpoint{0.647939in}{0.492442in}}{\pgfqpoint{4.273799in}{2.331163in}}%
\pgfusepath{clip}%
\pgfsetbuttcap%
\pgfsetroundjoin%
\pgfsetlinewidth{0.301125pt}%
\definecolor{currentstroke}{rgb}{0.500000,0.500000,0.500000}%
\pgfsetstrokecolor{currentstroke}%
\pgfsetstrokeopacity{0.300000}%
\pgfsetdash{}{0pt}%
\pgfpathmoveto{\pgfqpoint{1.522125in}{2.823605in}}%
\pgfpathlineto{\pgfqpoint{1.522125in}{2.823605in}}%
\pgfpathlineto{\pgfqpoint{1.560232in}{2.776167in}}%
\pgfpathlineto{\pgfqpoint{1.601473in}{2.729519in}}%
\pgfpathlineto{\pgfqpoint{1.646878in}{2.684049in}}%
\pgfpathlineto{\pgfqpoint{1.698017in}{2.640468in}}%
\pgfpathlineto{\pgfqpoint{1.757309in}{2.600185in}}%
\pgfpathlineto{\pgfqpoint{1.828047in}{2.566088in}}%
\pgfpathlineto{\pgfqpoint{1.828047in}{2.566088in}}%
\pgfpathlineto{\pgfqpoint{1.896182in}{2.545998in}}%
\pgfpathlineto{\pgfqpoint{1.971724in}{2.535408in}}%
\pgfpathlineto{\pgfqpoint{2.048050in}{2.533311in}}%
\pgfpathlineto{\pgfqpoint{2.137485in}{2.536840in}}%
\pgfpathlineto{\pgfqpoint{2.231861in}{2.542414in}}%
\pgfpathlineto{\pgfqpoint{2.326508in}{2.545733in}}%
\pgfpathlineto{\pgfqpoint{2.421109in}{2.543713in}}%
\pgfusepath{stroke}%
\end{pgfscope}%
\begin{pgfscope}%
\pgfpathrectangle{\pgfqpoint{0.647939in}{0.492442in}}{\pgfqpoint{4.273799in}{2.331163in}}%
\pgfusepath{clip}%
\pgfsetbuttcap%
\pgfsetroundjoin%
\pgfsetlinewidth{0.301125pt}%
\definecolor{currentstroke}{rgb}{0.500000,0.500000,0.500000}%
\pgfsetstrokecolor{currentstroke}%
\pgfsetstrokeopacity{0.300000}%
\pgfsetdash{}{0pt}%
\pgfpathmoveto{\pgfqpoint{1.424993in}{2.823605in}}%
\pgfpathlineto{\pgfqpoint{1.424993in}{2.823605in}}%
\pgfpathlineto{\pgfqpoint{1.455654in}{2.774580in}}%
\pgfpathlineto{\pgfqpoint{1.487813in}{2.725847in}}%
\pgfpathlineto{\pgfqpoint{1.521880in}{2.677504in}}%
\pgfpathlineto{\pgfqpoint{1.558458in}{2.629710in}}%
\pgfpathlineto{\pgfqpoint{1.598467in}{2.582746in}}%
\pgfpathlineto{\pgfqpoint{1.643460in}{2.537160in}}%
\pgfpathlineto{\pgfqpoint{1.696155in}{2.494197in}}%
\pgfpathlineto{\pgfqpoint{1.761483in}{2.457190in}}%
\pgfpathlineto{\pgfqpoint{1.761483in}{2.457190in}}%
\pgfpathlineto{\pgfqpoint{1.816733in}{2.439406in}}%
\pgfpathlineto{\pgfqpoint{1.880225in}{2.431986in}}%
\pgfpathlineto{\pgfqpoint{1.938691in}{2.434370in}}%
\pgfpathlineto{\pgfqpoint{2.002213in}{2.443497in}}%
\pgfpathlineto{\pgfqpoint{2.085052in}{2.460383in}}%
\pgfpathlineto{\pgfqpoint{2.173292in}{2.479327in}}%
\pgfpathlineto{\pgfqpoint{2.263266in}{2.495574in}}%
\pgfpathlineto{\pgfqpoint{2.355845in}{2.506252in}}%
\pgfpathlineto{\pgfqpoint{2.450120in}{2.508920in}}%
\pgfpathlineto{\pgfqpoint{2.543559in}{2.502018in}}%
\pgfpathlineto{\pgfqpoint{2.628705in}{2.486473in}}%
\pgfusepath{stroke}%
\end{pgfscope}%
\begin{pgfscope}%
\pgfpathrectangle{\pgfqpoint{0.647939in}{0.492442in}}{\pgfqpoint{4.273799in}{2.331163in}}%
\pgfusepath{clip}%
\pgfsetbuttcap%
\pgfsetroundjoin%
\pgfsetlinewidth{0.301125pt}%
\definecolor{currentstroke}{rgb}{0.500000,0.500000,0.500000}%
\pgfsetstrokecolor{currentstroke}%
\pgfsetstrokeopacity{0.300000}%
\pgfsetdash{}{0pt}%
\pgfpathmoveto{\pgfqpoint{1.327862in}{2.823605in}}%
\pgfpathlineto{\pgfqpoint{1.327862in}{2.823605in}}%
\pgfpathlineto{\pgfqpoint{1.352632in}{2.773598in}}%
\pgfpathlineto{\pgfqpoint{1.377963in}{2.723674in}}%
\pgfpathlineto{\pgfqpoint{1.403937in}{2.673850in}}%
\pgfpathlineto{\pgfqpoint{1.430675in}{2.624147in}}%
\pgfpathlineto{\pgfqpoint{1.458347in}{2.574597in}}%
\pgfpathlineto{\pgfqpoint{1.487184in}{2.525250in}}%
\pgfpathlineto{\pgfqpoint{1.517576in}{2.476183in}}%
\pgfpathlineto{\pgfqpoint{1.550158in}{2.427538in}}%
\pgfpathlineto{\pgfqpoint{1.586057in}{2.379616in}}%
\pgfpathlineto{\pgfqpoint{1.627717in}{2.333182in}}%
\pgfpathlineto{\pgfqpoint{1.681825in}{2.291178in}}%
\pgfpathlineto{\pgfqpoint{1.681825in}{2.291178in}}%
\pgfpathlineto{\pgfqpoint{1.719088in}{2.275655in}}%
\pgfpathlineto{\pgfqpoint{1.719088in}{2.275655in}}%
\pgfpathlineto{\pgfqpoint{1.756318in}{2.270715in}}%
\pgfpathlineto{\pgfqpoint{1.794563in}{2.274667in}}%
\pgfpathlineto{\pgfqpoint{1.832354in}{2.285052in}}%
\pgfpathlineto{\pgfqpoint{1.879229in}{2.303530in}}%
\pgfpathlineto{\pgfqpoint{1.945623in}{2.334518in}}%
\pgfusepath{stroke}%
\end{pgfscope}%
\begin{pgfscope}%
\pgfpathrectangle{\pgfqpoint{0.647939in}{0.492442in}}{\pgfqpoint{4.273799in}{2.331163in}}%
\pgfusepath{clip}%
\pgfsetbuttcap%
\pgfsetroundjoin%
\pgfsetlinewidth{0.301125pt}%
\definecolor{currentstroke}{rgb}{0.500000,0.500000,0.500000}%
\pgfsetstrokecolor{currentstroke}%
\pgfsetstrokeopacity{0.300000}%
\pgfsetdash{}{0pt}%
\pgfpathmoveto{\pgfqpoint{1.230730in}{2.823605in}}%
\pgfpathlineto{\pgfqpoint{1.230730in}{2.823605in}}%
\pgfpathlineto{\pgfqpoint{1.250929in}{2.772988in}}%
\pgfpathlineto{\pgfqpoint{1.271168in}{2.722376in}}%
\pgfpathlineto{\pgfqpoint{1.291423in}{2.671766in}}%
\pgfpathlineto{\pgfqpoint{1.311672in}{2.621156in}}%
\pgfpathlineto{\pgfqpoint{1.331877in}{2.570541in}}%
\pgfpathlineto{\pgfqpoint{1.351996in}{2.519916in}}%
\pgfpathlineto{\pgfqpoint{1.371964in}{2.469273in}}%
\pgfpathlineto{\pgfqpoint{1.391699in}{2.418604in}}%
\pgfpathlineto{\pgfqpoint{1.411077in}{2.367895in}}%
\pgfpathlineto{\pgfqpoint{1.429927in}{2.317127in}}%
\pgfpathlineto{\pgfqpoint{1.447979in}{2.266275in}}%
\pgfpathlineto{\pgfqpoint{1.464821in}{2.215299in}}%
\pgfpathlineto{\pgfqpoint{1.479775in}{2.164151in}}%
\pgfpathlineto{\pgfqpoint{1.491679in}{2.112774in}}%
\pgfpathlineto{\pgfqpoint{1.498597in}{2.061144in}}%
\pgfpathlineto{\pgfqpoint{1.497608in}{2.009425in}}%
\pgfpathlineto{\pgfqpoint{1.486074in}{1.958162in}}%
\pgfpathlineto{\pgfqpoint{1.464438in}{1.907861in}}%
\pgfpathlineto{\pgfqpoint{1.436124in}{1.858533in}}%
\pgfpathlineto{\pgfqpoint{1.404309in}{1.809850in}}%
\pgfpathlineto{\pgfqpoint{1.370911in}{1.761453in}}%
\pgfusepath{stroke}%
\end{pgfscope}%
\begin{pgfscope}%
\pgfpathrectangle{\pgfqpoint{0.647939in}{0.492442in}}{\pgfqpoint{4.273799in}{2.331163in}}%
\pgfusepath{clip}%
\pgfsetbuttcap%
\pgfsetroundjoin%
\pgfsetlinewidth{0.301125pt}%
\definecolor{currentstroke}{rgb}{0.500000,0.500000,0.500000}%
\pgfsetstrokecolor{currentstroke}%
\pgfsetstrokeopacity{0.300000}%
\pgfsetdash{}{0pt}%
\pgfpathmoveto{\pgfqpoint{1.133598in}{2.823605in}}%
\pgfpathlineto{\pgfqpoint{1.133598in}{2.823605in}}%
\pgfpathlineto{\pgfqpoint{1.150234in}{2.772604in}}%
\pgfpathlineto{\pgfqpoint{1.166652in}{2.721581in}}%
\pgfpathlineto{\pgfqpoint{1.182798in}{2.670533in}}%
\pgfpathlineto{\pgfqpoint{1.198613in}{2.619454in}}%
\pgfpathlineto{\pgfqpoint{1.214034in}{2.568339in}}%
\pgfpathlineto{\pgfqpoint{1.228972in}{2.517181in}}%
\pgfpathlineto{\pgfqpoint{1.243327in}{2.465975in}}%
\pgfpathlineto{\pgfqpoint{1.256980in}{2.414711in}}%
\pgfpathlineto{\pgfqpoint{1.269778in}{2.363381in}}%
\pgfpathlineto{\pgfqpoint{1.281537in}{2.311979in}}%
\pgfpathlineto{\pgfqpoint{1.292043in}{2.260496in}}%
\pgfpathlineto{\pgfqpoint{1.301026in}{2.208927in}}%
\pgfpathlineto{\pgfqpoint{1.308167in}{2.157274in}}%
\pgfpathlineto{\pgfqpoint{1.313103in}{2.105546in}}%
\pgfpathlineto{\pgfqpoint{1.315439in}{2.053764in}}%
\pgfpathlineto{\pgfqpoint{1.314776in}{2.001970in}}%
\pgfpathlineto{\pgfqpoint{1.310763in}{1.950223in}}%
\pgfpathlineto{\pgfqpoint{1.303172in}{1.898598in}}%
\pgfpathlineto{\pgfqpoint{1.291954in}{1.847171in}}%
\pgfpathlineto{\pgfqpoint{1.277299in}{1.796004in}}%
\pgfpathlineto{\pgfqpoint{1.259567in}{1.745130in}}%
\pgfpathlineto{\pgfqpoint{1.239230in}{1.694543in}}%
\pgfpathlineto{\pgfqpoint{1.216813in}{1.644219in}}%
\pgfusepath{stroke}%
\end{pgfscope}%
\begin{pgfscope}%
\pgfpathrectangle{\pgfqpoint{0.647939in}{0.492442in}}{\pgfqpoint{4.273799in}{2.331163in}}%
\pgfusepath{clip}%
\pgfsetbuttcap%
\pgfsetroundjoin%
\pgfsetlinewidth{0.301125pt}%
\definecolor{currentstroke}{rgb}{0.500000,0.500000,0.500000}%
\pgfsetstrokecolor{currentstroke}%
\pgfsetstrokeopacity{0.300000}%
\pgfsetdash{}{0pt}%
\pgfpathmoveto{\pgfqpoint{1.036466in}{2.823605in}}%
\pgfpathlineto{\pgfqpoint{1.036466in}{2.823605in}}%
\pgfpathlineto{\pgfqpoint{1.050317in}{2.772356in}}%
\pgfpathlineto{\pgfqpoint{1.063823in}{2.721080in}}%
\pgfpathlineto{\pgfqpoint{1.076940in}{2.669773in}}%
\pgfpathlineto{\pgfqpoint{1.089612in}{2.618434in}}%
\pgfpathlineto{\pgfqpoint{1.101771in}{2.567057in}}%
\pgfpathlineto{\pgfqpoint{1.113350in}{2.515641in}}%
\pgfpathlineto{\pgfqpoint{1.124270in}{2.464182in}}%
\pgfpathlineto{\pgfqpoint{1.134435in}{2.412677in}}%
\pgfpathlineto{\pgfqpoint{1.143741in}{2.361124in}}%
\pgfpathlineto{\pgfqpoint{1.152073in}{2.309521in}}%
\pgfpathlineto{\pgfqpoint{1.159306in}{2.257870in}}%
\pgfpathlineto{\pgfqpoint{1.165301in}{2.206171in}}%
\pgfpathlineto{\pgfqpoint{1.169906in}{2.154430in}}%
\pgfpathlineto{\pgfqpoint{1.172963in}{2.102655in}}%
\pgfpathlineto{\pgfqpoint{1.174319in}{2.050860in}}%
\pgfpathlineto{\pgfqpoint{1.173825in}{1.999060in}}%
\pgfpathlineto{\pgfqpoint{1.171351in}{1.947277in}}%
\pgfpathlineto{\pgfqpoint{1.166796in}{1.895537in}}%
\pgfpathlineto{\pgfqpoint{1.160104in}{1.843866in}}%
\pgfpathlineto{\pgfqpoint{1.151273in}{1.792292in}}%
\pgfpathlineto{\pgfqpoint{1.140350in}{1.740838in}}%
\pgfpathlineto{\pgfqpoint{1.127431in}{1.689521in}}%
\pgfpathlineto{\pgfqpoint{1.112663in}{1.638353in}}%
\pgfpathlineto{\pgfqpoint{1.096235in}{1.587337in}}%
\pgfpathlineto{\pgfqpoint{1.078331in}{1.536467in}}%
\pgfpathlineto{\pgfqpoint{1.059166in}{1.485735in}}%
\pgfpathlineto{\pgfqpoint{1.038936in}{1.435125in}}%
\pgfpathlineto{\pgfqpoint{1.017827in}{1.384623in}}%
\pgfpathlineto{\pgfqpoint{0.996007in}{1.334210in}}%
\pgfpathlineto{\pgfqpoint{0.973625in}{1.283871in}}%
\pgfpathlineto{\pgfqpoint{0.950805in}{1.233590in}}%
\pgfpathlineto{\pgfqpoint{0.927660in}{1.183353in}}%
\pgfpathlineto{\pgfqpoint{0.904283in}{1.133148in}}%
\pgfpathlineto{\pgfqpoint{0.880749in}{1.082965in}}%
\pgfpathlineto{\pgfqpoint{0.857124in}{1.032795in}}%
\pgfpathlineto{\pgfqpoint{0.833459in}{0.982630in}}%
\pgfusepath{stroke}%
\end{pgfscope}%
\begin{pgfscope}%
\pgfpathrectangle{\pgfqpoint{0.647939in}{0.492442in}}{\pgfqpoint{4.273799in}{2.331163in}}%
\pgfusepath{clip}%
\pgfsetbuttcap%
\pgfsetroundjoin%
\pgfsetlinewidth{0.301125pt}%
\definecolor{currentstroke}{rgb}{0.500000,0.500000,0.500000}%
\pgfsetstrokecolor{currentstroke}%
\pgfsetstrokeopacity{0.300000}%
\pgfsetdash{}{0pt}%
\pgfpathmoveto{\pgfqpoint{0.939334in}{2.823605in}}%
\pgfpathlineto{\pgfqpoint{0.939334in}{2.823605in}}%
\pgfpathlineto{\pgfqpoint{0.950980in}{2.772193in}}%
\pgfpathlineto{\pgfqpoint{0.962239in}{2.720755in}}%
\pgfpathlineto{\pgfqpoint{0.973068in}{2.669290in}}%
\pgfpathlineto{\pgfqpoint{0.983423in}{2.617796in}}%
\pgfpathlineto{\pgfqpoint{0.993258in}{2.566271in}}%
\pgfpathlineto{\pgfqpoint{1.002517in}{2.514715in}}%
\pgfpathlineto{\pgfqpoint{1.011137in}{2.463126in}}%
\pgfpathlineto{\pgfqpoint{1.019058in}{2.411503in}}%
\pgfpathlineto{\pgfqpoint{1.026211in}{2.359848in}}%
\pgfpathlineto{\pgfqpoint{1.032526in}{2.308159in}}%
\pgfpathlineto{\pgfqpoint{1.037925in}{2.256440in}}%
\pgfpathlineto{\pgfqpoint{1.042328in}{2.204693in}}%
\pgfpathlineto{\pgfqpoint{1.045652in}{2.152923in}}%
\pgfpathlineto{\pgfqpoint{1.047817in}{2.101134in}}%
\pgfpathlineto{\pgfqpoint{1.048745in}{2.049334in}}%
\pgfpathlineto{\pgfqpoint{1.048363in}{1.997532in}}%
\pgfpathlineto{\pgfqpoint{1.046606in}{1.945739in}}%
\pgfpathlineto{\pgfqpoint{1.043421in}{1.893966in}}%
\pgfpathlineto{\pgfqpoint{1.038770in}{1.842227in}}%
\pgfpathlineto{\pgfqpoint{1.032635in}{1.790533in}}%
\pgfpathlineto{\pgfqpoint{1.025020in}{1.738899in}}%
\pgfpathlineto{\pgfqpoint{1.015949in}{1.687334in}}%
\pgfpathlineto{\pgfqpoint{1.005464in}{1.635850in}}%
\pgfpathlineto{\pgfqpoint{0.993624in}{1.584453in}}%
\pgfpathlineto{\pgfqpoint{0.980514in}{1.533148in}}%
\pgfpathlineto{\pgfqpoint{0.966229in}{1.481937in}}%
\pgfpathlineto{\pgfqpoint{0.950863in}{1.430818in}}%
\pgfpathlineto{\pgfqpoint{0.934526in}{1.379790in}}%
\pgfpathlineto{\pgfqpoint{0.917328in}{1.328846in}}%
\pgfusepath{stroke}%
\end{pgfscope}%
\begin{pgfscope}%
\pgfpathrectangle{\pgfqpoint{0.647939in}{0.492442in}}{\pgfqpoint{4.273799in}{2.331163in}}%
\pgfusepath{clip}%
\pgfsetbuttcap%
\pgfsetroundjoin%
\pgfsetlinewidth{0.301125pt}%
\definecolor{currentstroke}{rgb}{0.500000,0.500000,0.500000}%
\pgfsetstrokecolor{currentstroke}%
\pgfsetstrokeopacity{0.300000}%
\pgfsetdash{}{0pt}%
\pgfpathmoveto{\pgfqpoint{0.842203in}{2.823605in}}%
\pgfpathlineto{\pgfqpoint{0.842203in}{2.823605in}}%
\pgfpathlineto{\pgfqpoint{0.852084in}{2.772083in}}%
\pgfpathlineto{\pgfqpoint{0.861575in}{2.720539in}}%
\pgfpathlineto{\pgfqpoint{0.870641in}{2.668973in}}%
\pgfpathlineto{\pgfqpoint{0.879246in}{2.617382in}}%
\pgfpathlineto{\pgfqpoint{0.887353in}{2.565768in}}%
\pgfpathlineto{\pgfqpoint{0.894924in}{2.514130in}}%
\pgfpathlineto{\pgfqpoint{0.901917in}{2.462468in}}%
\pgfpathlineto{\pgfqpoint{0.908289in}{2.410781in}}%
\pgfpathlineto{\pgfqpoint{0.913992in}{2.359071in}}%
\pgfpathlineto{\pgfqpoint{0.918980in}{2.307340in}}%
\pgfpathlineto{\pgfqpoint{0.923203in}{2.255588in}}%
\pgfpathlineto{\pgfqpoint{0.926615in}{2.203818in}}%
\pgfpathlineto{\pgfqpoint{0.929166in}{2.152034in}}%
\pgfpathlineto{\pgfqpoint{0.930809in}{2.100239in}}%
\pgfpathlineto{\pgfqpoint{0.931499in}{2.048438in}}%
\pgfpathlineto{\pgfqpoint{0.931194in}{1.996635in}}%
\pgfpathlineto{\pgfqpoint{0.929858in}{1.944837in}}%
\pgfpathlineto{\pgfqpoint{0.927459in}{1.893051in}}%
\pgfpathlineto{\pgfqpoint{0.923972in}{1.841284in}}%
\pgfpathlineto{\pgfqpoint{0.919383in}{1.789542in}}%
\pgfpathlineto{\pgfqpoint{0.913684in}{1.737833in}}%
\pgfpathlineto{\pgfqpoint{0.906878in}{1.686163in}}%
\pgfpathlineto{\pgfqpoint{0.898981in}{1.634540in}}%
\pgfpathlineto{\pgfqpoint{0.890019in}{1.582970in}}%
\pgfpathlineto{\pgfqpoint{0.880024in}{1.531455in}}%
\pgfpathlineto{\pgfqpoint{0.869035in}{1.480001in}}%
\pgfpathlineto{\pgfqpoint{0.857102in}{1.428609in}}%
\pgfpathlineto{\pgfqpoint{0.844284in}{1.377282in}}%
\pgfpathlineto{\pgfqpoint{0.830639in}{1.326018in}}%
\pgfpathlineto{\pgfqpoint{0.816225in}{1.274816in}}%
\pgfpathlineto{\pgfqpoint{0.801111in}{1.223674in}}%
\pgfpathlineto{\pgfqpoint{0.785361in}{1.172590in}}%
\pgfpathlineto{\pgfqpoint{0.769033in}{1.121559in}}%
\pgfpathlineto{\pgfqpoint{0.752193in}{1.070579in}}%
\pgfpathlineto{\pgfqpoint{0.734893in}{1.019643in}}%
\pgfpathlineto{\pgfqpoint{0.717191in}{0.968749in}}%
\pgfpathlineto{\pgfqpoint{0.699140in}{0.917892in}}%
\pgfpathlineto{\pgfqpoint{0.680784in}{0.867067in}}%
\pgfpathlineto{\pgfqpoint{0.662168in}{0.816270in}}%
\pgfpathlineto{\pgfqpoint{0.647939in}{0.777695in}}%
\pgfusepath{stroke}%
\end{pgfscope}%
\begin{pgfscope}%
\pgfpathrectangle{\pgfqpoint{0.647939in}{0.492442in}}{\pgfqpoint{4.273799in}{2.331163in}}%
\pgfusepath{clip}%
\pgfsetbuttcap%
\pgfsetroundjoin%
\pgfsetlinewidth{0.301125pt}%
\definecolor{currentstroke}{rgb}{0.500000,0.500000,0.500000}%
\pgfsetstrokecolor{currentstroke}%
\pgfsetstrokeopacity{0.300000}%
\pgfsetdash{}{0pt}%
\pgfpathmoveto{\pgfqpoint{0.745071in}{2.823605in}}%
\pgfpathlineto{\pgfqpoint{0.745071in}{2.823605in}}%
\pgfpathlineto{\pgfqpoint{0.753530in}{2.772008in}}%
\pgfpathlineto{\pgfqpoint{0.761610in}{2.720392in}}%
\pgfpathlineto{\pgfqpoint{0.769288in}{2.668759in}}%
\pgfpathlineto{\pgfqpoint{0.776537in}{2.617106in}}%
\pgfpathlineto{\pgfqpoint{0.783328in}{2.565436in}}%
\pgfpathlineto{\pgfqpoint{0.789633in}{2.513747in}}%
\pgfpathlineto{\pgfqpoint{0.795420in}{2.462040in}}%
\pgfpathlineto{\pgfqpoint{0.800661in}{2.410315in}}%
\pgfpathlineto{\pgfqpoint{0.805324in}{2.358575in}}%
\pgfpathlineto{\pgfqpoint{0.809377in}{2.306818in}}%
\pgfpathlineto{\pgfqpoint{0.812789in}{2.255049in}}%
\pgfpathlineto{\pgfqpoint{0.815529in}{2.203267in}}%
\pgfpathlineto{\pgfqpoint{0.817564in}{2.151476in}}%
\pgfpathlineto{\pgfqpoint{0.818866in}{2.099677in}}%
\pgfpathlineto{\pgfqpoint{0.819406in}{2.047875in}}%
\pgfpathlineto{\pgfqpoint{0.819158in}{1.996072in}}%
\pgfpathlineto{\pgfqpoint{0.818098in}{1.944272in}}%
\pgfpathlineto{\pgfqpoint{0.816207in}{1.892480in}}%
\pgfpathlineto{\pgfqpoint{0.813469in}{1.840698in}}%
\pgfpathlineto{\pgfqpoint{0.809872in}{1.788932in}}%
\pgfpathlineto{\pgfqpoint{0.805410in}{1.737187in}}%
\pgfpathlineto{\pgfqpoint{0.800081in}{1.685465in}}%
\pgfpathlineto{\pgfqpoint{0.793889in}{1.633773in}}%
\pgfpathlineto{\pgfqpoint{0.786842in}{1.582113in}}%
\pgfpathlineto{\pgfqpoint{0.778954in}{1.530489in}}%
\pgfpathlineto{\pgfqpoint{0.770246in}{1.478904in}}%
\pgfpathlineto{\pgfqpoint{0.760745in}{1.427362in}}%
\pgfpathlineto{\pgfqpoint{0.750480in}{1.375863in}}%
\pgfpathlineto{\pgfqpoint{0.739480in}{1.324409in}}%
\pgfpathlineto{\pgfqpoint{0.727781in}{1.273001in}}%
\pgfpathlineto{\pgfqpoint{0.715426in}{1.221638in}}%
\pgfusepath{stroke}%
\end{pgfscope}%
\begin{pgfscope}%
\pgfpathrectangle{\pgfqpoint{0.647939in}{0.492442in}}{\pgfqpoint{4.273799in}{2.331163in}}%
\pgfusepath{clip}%
\pgfsetbuttcap%
\pgfsetroundjoin%
\pgfsetlinewidth{0.301125pt}%
\definecolor{currentstroke}{rgb}{0.500000,0.500000,0.500000}%
\pgfsetstrokecolor{currentstroke}%
\pgfsetstrokeopacity{0.300000}%
\pgfsetdash{}{0pt}%
\pgfpathmoveto{\pgfqpoint{0.647939in}{2.823605in}}%
\pgfpathlineto{\pgfqpoint{0.647939in}{2.823605in}}%
\pgfpathlineto{\pgfqpoint{0.655238in}{2.771955in}}%
\pgfpathlineto{\pgfqpoint{0.662182in}{2.720290in}}%
\pgfpathlineto{\pgfqpoint{0.668751in}{2.668611in}}%
\pgfpathlineto{\pgfqpoint{0.674926in}{2.616917in}}%
\pgfpathlineto{\pgfqpoint{0.680686in}{2.565209in}}%
\pgfpathlineto{\pgfqpoint{0.686011in}{2.513487in}}%
\pgfpathlineto{\pgfqpoint{0.690880in}{2.461752in}}%
\pgfpathlineto{\pgfqpoint{0.695269in}{2.410004in}}%
\pgfpathlineto{\pgfqpoint{0.699157in}{2.358244in}}%
\pgfpathlineto{\pgfqpoint{0.702522in}{2.306473in}}%
\pgfpathlineto{\pgfqpoint{0.705342in}{2.254693in}}%
\pgfpathlineto{\pgfqpoint{0.707596in}{2.202904in}}%
\pgfpathlineto{\pgfqpoint{0.709263in}{2.151109in}}%
\pgfpathlineto{\pgfqpoint{0.710324in}{2.099309in}}%
\pgfpathlineto{\pgfqpoint{0.710760in}{2.047506in}}%
\pgfpathlineto{\pgfqpoint{0.710554in}{1.995703in}}%
\pgfpathlineto{\pgfqpoint{0.709691in}{1.943901in}}%
\pgfpathlineto{\pgfqpoint{0.708156in}{1.892105in}}%
\pgfpathlineto{\pgfqpoint{0.705939in}{1.840316in}}%
\pgfpathlineto{\pgfqpoint{0.703031in}{1.788537in}}%
\pgfpathlineto{\pgfqpoint{0.699427in}{1.736771in}}%
\pgfpathlineto{\pgfqpoint{0.695122in}{1.685021in}}%
\pgfpathlineto{\pgfqpoint{0.690117in}{1.633290in}}%
\pgfpathlineto{\pgfqpoint{0.684416in}{1.581580in}}%
\pgfpathlineto{\pgfqpoint{0.678027in}{1.529894in}}%
\pgfpathlineto{\pgfqpoint{0.670959in}{1.478235in}}%
\pgfpathlineto{\pgfqpoint{0.663224in}{1.426604in}}%
\pgfpathlineto{\pgfqpoint{0.654838in}{1.375004in}}%
\pgfpathlineto{\pgfqpoint{0.647939in}{1.334125in}}%
\pgfusepath{stroke}%
\end{pgfscope}%
\begin{pgfscope}%
\pgfpathrectangle{\pgfqpoint{0.647939in}{0.492442in}}{\pgfqpoint{4.273799in}{2.331163in}}%
\pgfusepath{clip}%
\pgfsetbuttcap%
\pgfsetroundjoin%
\pgfsetlinewidth{0.301125pt}%
\definecolor{currentstroke}{rgb}{0.500000,0.500000,0.500000}%
\pgfsetstrokecolor{currentstroke}%
\pgfsetstrokeopacity{0.300000}%
\pgfsetdash{}{0pt}%
\pgfpathmoveto{\pgfqpoint{0.647939in}{2.346777in}}%
\pgfpathlineto{\pgfqpoint{0.647939in}{2.346777in}}%
\pgfpathlineto{\pgfqpoint{0.650918in}{2.294999in}}%
\pgfpathlineto{\pgfqpoint{0.653388in}{2.243213in}}%
\pgfpathlineto{\pgfqpoint{0.655333in}{2.191420in}}%
\pgfpathlineto{\pgfqpoint{0.656735in}{2.139622in}}%
\pgfpathlineto{\pgfqpoint{0.657578in}{2.087821in}}%
\pgfpathlineto{\pgfqpoint{0.657847in}{2.036018in}}%
\pgfpathlineto{\pgfqpoint{0.657528in}{1.984215in}}%
\pgfpathlineto{\pgfqpoint{0.656609in}{1.932414in}}%
\pgfpathlineto{\pgfqpoint{0.655078in}{1.880617in}}%
\pgfpathlineto{\pgfqpoint{0.652927in}{1.828827in}}%
\pgfpathlineto{\pgfqpoint{0.650148in}{1.777046in}}%
\pgfpathlineto{\pgfqpoint{0.647939in}{1.740066in}}%
\pgfusepath{stroke}%
\end{pgfscope}%
\begin{pgfscope}%
\pgfpathrectangle{\pgfqpoint{0.647939in}{0.492442in}}{\pgfqpoint{4.273799in}{2.331163in}}%
\pgfusepath{clip}%
\pgfsetbuttcap%
\pgfsetroundjoin%
\pgfsetlinewidth{0.301125pt}%
\definecolor{currentstroke}{rgb}{0.500000,0.500000,0.500000}%
\pgfsetstrokecolor{currentstroke}%
\pgfsetstrokeopacity{0.300000}%
\pgfsetdash{}{0pt}%
\pgfpathmoveto{\pgfqpoint{1.670976in}{0.492442in}}%
\pgfpathlineto{\pgfqpoint{1.664019in}{0.499742in}}%
\pgfpathlineto{\pgfqpoint{1.619257in}{0.545423in}}%
\pgfpathlineto{\pgfqpoint{1.572223in}{0.590415in}}%
\pgfpathlineto{\pgfqpoint{1.522091in}{0.634390in}}%
\pgfpathlineto{\pgfqpoint{1.467597in}{0.676779in}}%
\pgfpathlineto{\pgfqpoint{1.406665in}{0.716395in}}%
\pgfpathlineto{\pgfqpoint{1.335844in}{0.750535in}}%
\pgfpathlineto{\pgfqpoint{1.335844in}{0.750535in}}%
\pgfpathlineto{\pgfqpoint{1.274435in}{0.768527in}}%
\pgfpathlineto{\pgfqpoint{1.274435in}{0.768527in}}%
\pgfpathlineto{\pgfqpoint{1.217953in}{0.774669in}}%
\pgfpathlineto{\pgfqpoint{1.160578in}{0.770175in}}%
\pgfpathlineto{\pgfqpoint{1.110974in}{0.757413in}}%
\pgfpathlineto{\pgfqpoint{1.062704in}{0.737031in}}%
\pgfpathlineto{\pgfqpoint{1.012006in}{0.707285in}}%
\pgfpathlineto{\pgfqpoint{0.957089in}{0.665731in}}%
\pgfusepath{stroke}%
\end{pgfscope}%
\begin{pgfscope}%
\pgfpathrectangle{\pgfqpoint{0.647939in}{0.492442in}}{\pgfqpoint{4.273799in}{2.331163in}}%
\pgfusepath{clip}%
\pgfsetbuttcap%
\pgfsetroundjoin%
\pgfsetlinewidth{0.301125pt}%
\definecolor{currentstroke}{rgb}{0.500000,0.500000,0.500000}%
\pgfsetstrokecolor{currentstroke}%
\pgfsetstrokeopacity{0.300000}%
\pgfsetdash{}{0pt}%
\pgfpathmoveto{\pgfqpoint{1.498716in}{0.492442in}}%
\pgfpathlineto{\pgfqpoint{1.456781in}{0.523058in}}%
\pgfpathlineto{\pgfqpoint{1.396626in}{0.563033in}}%
\pgfpathlineto{\pgfqpoint{1.327862in}{0.598404in}}%
\pgfpathlineto{\pgfqpoint{1.327862in}{0.598404in}}%
\pgfpathlineto{\pgfqpoint{1.261222in}{0.621025in}}%
\pgfpathlineto{\pgfqpoint{1.261222in}{0.621025in}}%
\pgfpathlineto{\pgfqpoint{1.201775in}{0.630491in}}%
\pgfpathlineto{\pgfqpoint{1.139505in}{0.628786in}}%
\pgfpathlineto{\pgfqpoint{1.086827in}{0.617835in}}%
\pgfusepath{stroke}%
\end{pgfscope}%
\begin{pgfscope}%
\pgfpathrectangle{\pgfqpoint{0.647939in}{0.492442in}}{\pgfqpoint{4.273799in}{2.331163in}}%
\pgfusepath{clip}%
\pgfsetbuttcap%
\pgfsetroundjoin%
\pgfsetlinewidth{0.301125pt}%
\definecolor{currentstroke}{rgb}{0.500000,0.500000,0.500000}%
\pgfsetstrokecolor{currentstroke}%
\pgfsetstrokeopacity{0.300000}%
\pgfsetdash{}{0pt}%
\pgfpathmoveto{\pgfqpoint{4.727475in}{1.816967in}}%
\pgfpathlineto{\pgfqpoint{4.720296in}{1.868620in}}%
\pgfpathlineto{\pgfqpoint{4.714709in}{1.920331in}}%
\pgfpathlineto{\pgfqpoint{4.710942in}{1.972090in}}%
\pgfpathlineto{\pgfqpoint{4.709230in}{2.023882in}}%
\pgfpathlineto{\pgfqpoint{4.709800in}{2.075680in}}%
\pgfpathlineto{\pgfqpoint{4.712854in}{2.127451in}}%
\pgfpathlineto{\pgfqpoint{4.718533in}{2.179155in}}%
\pgfusepath{stroke}%
\end{pgfscope}%
\begin{pgfscope}%
\pgfpathrectangle{\pgfqpoint{0.647939in}{0.492442in}}{\pgfqpoint{4.273799in}{2.331163in}}%
\pgfusepath{clip}%
\pgfsetbuttcap%
\pgfsetroundjoin%
\pgfsetlinewidth{0.301125pt}%
\definecolor{currentstroke}{rgb}{0.500000,0.500000,0.500000}%
\pgfsetstrokecolor{currentstroke}%
\pgfsetstrokeopacity{0.300000}%
\pgfsetdash{}{0pt}%
\pgfpathmoveto{\pgfqpoint{3.950420in}{0.651385in}}%
\pgfpathlineto{\pgfqpoint{3.890017in}{0.691356in}}%
\pgfpathlineto{\pgfqpoint{3.828102in}{0.730635in}}%
\pgfpathlineto{\pgfqpoint{3.765031in}{0.769363in}}%
\pgfpathlineto{\pgfqpoint{3.701201in}{0.807722in}}%
\pgfpathlineto{\pgfqpoint{3.637055in}{0.845924in}}%
\pgfpathlineto{\pgfqpoint{3.573040in}{0.884190in}}%
\pgfusepath{stroke}%
\end{pgfscope}%
\begin{pgfscope}%
\pgfpathrectangle{\pgfqpoint{0.647939in}{0.492442in}}{\pgfqpoint{4.273799in}{2.331163in}}%
\pgfusepath{clip}%
\pgfsetbuttcap%
\pgfsetroundjoin%
\pgfsetlinewidth{0.301125pt}%
\definecolor{currentstroke}{rgb}{0.500000,0.500000,0.500000}%
\pgfsetstrokecolor{currentstroke}%
\pgfsetstrokeopacity{0.300000}%
\pgfsetdash{}{0pt}%
\pgfpathmoveto{\pgfqpoint{4.647733in}{1.554118in}}%
\pgfpathlineto{\pgfqpoint{4.630343in}{1.605043in}}%
\pgfpathlineto{\pgfqpoint{4.613471in}{1.656020in}}%
\pgfpathlineto{\pgfqpoint{4.597280in}{1.707062in}}%
\pgfpathlineto{\pgfqpoint{4.581980in}{1.758186in}}%
\pgfpathlineto{\pgfqpoint{4.567849in}{1.809409in}}%
\pgfpathlineto{\pgfqpoint{4.555275in}{1.860752in}}%
\pgfpathlineto{\pgfqpoint{4.544771in}{1.912231in}}%
\pgfpathlineto{\pgfqpoint{4.537015in}{1.963852in}}%
\pgfpathlineto{\pgfqpoint{4.532882in}{2.015592in}}%
\pgfpathlineto{\pgfqpoint{4.533372in}{2.067373in}}%
\pgfpathlineto{\pgfqpoint{4.539392in}{2.119044in}}%
\pgfpathlineto{\pgfqpoint{4.551361in}{2.170394in}}%
\pgfusepath{stroke}%
\end{pgfscope}%
\begin{pgfscope}%
\pgfpathrectangle{\pgfqpoint{0.647939in}{0.492442in}}{\pgfqpoint{4.273799in}{2.331163in}}%
\pgfusepath{clip}%
\pgfsetbuttcap%
\pgfsetroundjoin%
\pgfsetlinewidth{0.301125pt}%
\definecolor{currentstroke}{rgb}{0.500000,0.500000,0.500000}%
\pgfsetstrokecolor{currentstroke}%
\pgfsetstrokeopacity{0.300000}%
\pgfsetdash{}{0pt}%
\pgfpathmoveto{\pgfqpoint{1.572518in}{0.660485in}}%
\pgfpathlineto{\pgfqpoint{1.522125in}{0.704366in}}%
\pgfpathlineto{\pgfqpoint{1.467050in}{0.746524in}}%
\pgfpathlineto{\pgfqpoint{1.404922in}{0.785553in}}%
\pgfpathlineto{\pgfqpoint{1.331819in}{0.818119in}}%
\pgfpathlineto{\pgfqpoint{1.331819in}{0.818119in}}%
\pgfpathlineto{\pgfqpoint{1.273203in}{0.832578in}}%
\pgfpathlineto{\pgfqpoint{1.273203in}{0.832578in}}%
\pgfpathlineto{\pgfqpoint{1.218420in}{0.835926in}}%
\pgfpathlineto{\pgfqpoint{1.164351in}{0.829223in}}%
\pgfpathlineto{\pgfqpoint{1.116188in}{0.814779in}}%
\pgfusepath{stroke}%
\end{pgfscope}%
\begin{pgfscope}%
\pgfpathrectangle{\pgfqpoint{0.647939in}{0.492442in}}{\pgfqpoint{4.273799in}{2.331163in}}%
\pgfusepath{clip}%
\pgfsetbuttcap%
\pgfsetroundjoin%
\pgfsetlinewidth{0.301125pt}%
\definecolor{currentstroke}{rgb}{0.500000,0.500000,0.500000}%
\pgfsetstrokecolor{currentstroke}%
\pgfsetstrokeopacity{0.300000}%
\pgfsetdash{}{0pt}%
\pgfpathmoveto{\pgfqpoint{3.782425in}{0.625602in}}%
\pgfpathlineto{\pgfqpoint{3.720906in}{0.665068in}}%
\pgfpathlineto{\pgfqpoint{3.659025in}{0.704366in}}%
\pgfpathlineto{\pgfqpoint{3.597151in}{0.743667in}}%
\pgfpathlineto{\pgfqpoint{3.535635in}{0.783134in}}%
\pgfpathlineto{\pgfqpoint{3.474814in}{0.822918in}}%
\pgfpathlineto{\pgfqpoint{3.414983in}{0.863146in}}%
\pgfusepath{stroke}%
\end{pgfscope}%
\begin{pgfscope}%
\pgfpathrectangle{\pgfqpoint{0.647939in}{0.492442in}}{\pgfqpoint{4.273799in}{2.331163in}}%
\pgfusepath{clip}%
\pgfsetbuttcap%
\pgfsetroundjoin%
\pgfsetlinewidth{0.301125pt}%
\definecolor{currentstroke}{rgb}{0.500000,0.500000,0.500000}%
\pgfsetstrokecolor{currentstroke}%
\pgfsetstrokeopacity{0.300000}%
\pgfsetdash{}{0pt}%
\pgfpathmoveto{\pgfqpoint{4.241816in}{0.704366in}}%
\pgfpathlineto{\pgfqpoint{4.191294in}{0.748221in}}%
\pgfpathlineto{\pgfqpoint{4.138021in}{0.791096in}}%
\pgfpathlineto{\pgfqpoint{4.081874in}{0.832862in}}%
\pgfpathlineto{\pgfqpoint{4.022773in}{0.873396in}}%
\pgfpathlineto{\pgfqpoint{3.960758in}{0.912616in}}%
\pgfpathlineto{\pgfqpoint{3.896048in}{0.950520in}}%
\pgfpathlineto{\pgfqpoint{3.829013in}{0.987206in}}%
\pgfpathlineto{\pgfqpoint{3.760185in}{1.022896in}}%
\pgfpathlineto{\pgfqpoint{3.690232in}{1.057933in}}%
\pgfpathlineto{\pgfqpoint{3.619883in}{1.092733in}}%
\pgfpathlineto{\pgfqpoint{3.549861in}{1.127727in}}%
\pgfpathlineto{\pgfqpoint{3.480865in}{1.163317in}}%
\pgfusepath{stroke}%
\end{pgfscope}%
\begin{pgfscope}%
\pgfpathrectangle{\pgfqpoint{0.647939in}{0.492442in}}{\pgfqpoint{4.273799in}{2.331163in}}%
\pgfusepath{clip}%
\pgfsetbuttcap%
\pgfsetroundjoin%
\pgfsetlinewidth{0.301125pt}%
\definecolor{currentstroke}{rgb}{0.500000,0.500000,0.500000}%
\pgfsetstrokecolor{currentstroke}%
\pgfsetstrokeopacity{0.300000}%
\pgfsetdash{}{0pt}%
\pgfpathmoveto{\pgfqpoint{4.533211in}{1.128214in}}%
\pgfpathlineto{\pgfqpoint{4.501736in}{1.177086in}}%
\pgfpathlineto{\pgfqpoint{4.468825in}{1.225672in}}%
\pgfpathlineto{\pgfqpoint{4.434156in}{1.273890in}}%
\pgfpathlineto{\pgfqpoint{4.397288in}{1.321619in}}%
\pgfpathlineto{\pgfqpoint{4.357599in}{1.368669in}}%
\pgfpathlineto{\pgfqpoint{4.314178in}{1.414721in}}%
\pgfpathlineto{\pgfqpoint{4.265634in}{1.459198in}}%
\pgfpathlineto{\pgfqpoint{4.209782in}{1.500959in}}%
\pgfpathlineto{\pgfqpoint{4.143269in}{1.537587in}}%
\pgfpathlineto{\pgfqpoint{4.143269in}{1.537587in}}%
\pgfpathlineto{\pgfqpoint{4.076832in}{1.561072in}}%
\pgfpathlineto{\pgfqpoint{4.000954in}{1.574658in}}%
\pgfpathlineto{\pgfqpoint{3.927967in}{1.578118in}}%
\pgfpathlineto{\pgfqpoint{3.846549in}{1.575471in}}%
\pgfpathlineto{\pgfqpoint{3.752277in}{1.569423in}}%
\pgfpathlineto{\pgfqpoint{3.657840in}{1.564848in}}%
\pgfpathlineto{\pgfqpoint{3.563223in}{1.565275in}}%
\pgfusepath{stroke}%
\end{pgfscope}%
\begin{pgfscope}%
\pgfpathrectangle{\pgfqpoint{0.647939in}{0.492442in}}{\pgfqpoint{4.273799in}{2.331163in}}%
\pgfusepath{clip}%
\pgfsetbuttcap%
\pgfsetroundjoin%
\pgfsetlinewidth{0.301125pt}%
\definecolor{currentstroke}{rgb}{0.500000,0.500000,0.500000}%
\pgfsetstrokecolor{currentstroke}%
\pgfsetstrokeopacity{0.300000}%
\pgfsetdash{}{0pt}%
\pgfpathmoveto{\pgfqpoint{4.559009in}{1.396247in}}%
\pgfpathlineto{\pgfqpoint{4.533211in}{1.446100in}}%
\pgfpathlineto{\pgfqpoint{4.506697in}{1.495840in}}%
\pgfpathlineto{\pgfqpoint{4.479319in}{1.545436in}}%
\pgfpathlineto{\pgfqpoint{4.450820in}{1.594845in}}%
\pgfpathlineto{\pgfqpoint{4.420845in}{1.643987in}}%
\pgfpathlineto{\pgfqpoint{4.388770in}{1.692720in}}%
\pgfpathlineto{\pgfqpoint{4.353331in}{1.740741in}}%
\pgfpathlineto{\pgfqpoint{4.311619in}{1.787188in}}%
\pgfpathlineto{\pgfqpoint{4.311619in}{1.787188in}}%
\pgfpathlineto{\pgfqpoint{4.265214in}{1.822612in}}%
\pgfpathlineto{\pgfqpoint{4.265214in}{1.822612in}}%
\pgfpathlineto{\pgfqpoint{4.230394in}{1.836444in}}%
\pgfpathlineto{\pgfqpoint{4.230394in}{1.836444in}}%
\pgfpathlineto{\pgfqpoint{4.195974in}{1.839648in}}%
\pgfpathlineto{\pgfqpoint{4.161853in}{1.834704in}}%
\pgfpathlineto{\pgfqpoint{4.126569in}{1.823585in}}%
\pgfusepath{stroke}%
\end{pgfscope}%
\begin{pgfscope}%
\pgfpathrectangle{\pgfqpoint{0.647939in}{0.492442in}}{\pgfqpoint{4.273799in}{2.331163in}}%
\pgfusepath{clip}%
\pgfsetbuttcap%
\pgfsetroundjoin%
\pgfsetlinewidth{0.301125pt}%
\definecolor{currentstroke}{rgb}{0.500000,0.500000,0.500000}%
\pgfsetstrokecolor{currentstroke}%
\pgfsetstrokeopacity{0.300000}%
\pgfsetdash{}{0pt}%
\pgfpathmoveto{\pgfqpoint{1.661819in}{2.600923in}}%
\pgfpathlineto{\pgfqpoint{1.716389in}{2.558700in}}%
\pgfpathlineto{\pgfqpoint{1.782329in}{2.521813in}}%
\pgfpathlineto{\pgfqpoint{1.782329in}{2.521813in}}%
\pgfpathlineto{\pgfqpoint{1.843920in}{2.500408in}}%
\pgfpathlineto{\pgfqpoint{1.915263in}{2.489003in}}%
\pgfpathlineto{\pgfqpoint{1.982343in}{2.487971in}}%
\pgfpathlineto{\pgfqpoint{2.055491in}{2.493583in}}%
\pgfpathlineto{\pgfqpoint{2.147964in}{2.505083in}}%
\pgfusepath{stroke}%
\end{pgfscope}%
\begin{pgfscope}%
\pgfpathrectangle{\pgfqpoint{0.647939in}{0.492442in}}{\pgfqpoint{4.273799in}{2.331163in}}%
\pgfusepath{clip}%
\pgfsetbuttcap%
\pgfsetroundjoin%
\pgfsetlinewidth{0.301125pt}%
\definecolor{currentstroke}{rgb}{0.500000,0.500000,0.500000}%
\pgfsetstrokecolor{currentstroke}%
\pgfsetstrokeopacity{0.300000}%
\pgfsetdash{}{0pt}%
\pgfpathmoveto{\pgfqpoint{1.619257in}{2.505719in}}%
\pgfpathlineto{\pgfqpoint{1.667766in}{2.461320in}}%
\pgfpathlineto{\pgfqpoint{1.728104in}{2.421742in}}%
\pgfpathlineto{\pgfqpoint{1.728104in}{2.421742in}}%
\pgfpathlineto{\pgfqpoint{1.779295in}{2.401823in}}%
\pgfpathlineto{\pgfqpoint{1.779295in}{2.401823in}}%
\pgfpathlineto{\pgfqpoint{1.830435in}{2.393171in}}%
\pgfpathlineto{\pgfqpoint{1.884188in}{2.393862in}}%
\pgfpathlineto{\pgfqpoint{1.938276in}{2.401797in}}%
\pgfpathlineto{\pgfqpoint{2.003101in}{2.416979in}}%
\pgfusepath{stroke}%
\end{pgfscope}%
\begin{pgfscope}%
\pgfpathrectangle{\pgfqpoint{0.647939in}{0.492442in}}{\pgfqpoint{4.273799in}{2.331163in}}%
\pgfusepath{clip}%
\pgfsetbuttcap%
\pgfsetroundjoin%
\pgfsetlinewidth{0.301125pt}%
\definecolor{currentstroke}{rgb}{0.500000,0.500000,0.500000}%
\pgfsetstrokecolor{currentstroke}%
\pgfsetstrokeopacity{0.300000}%
\pgfsetdash{}{0pt}%
\pgfpathmoveto{\pgfqpoint{1.424993in}{2.505719in}}%
\pgfpathlineto{\pgfqpoint{1.449403in}{2.455661in}}%
\pgfpathlineto{\pgfqpoint{1.474336in}{2.405682in}}%
\pgfpathlineto{\pgfqpoint{1.499943in}{2.355807in}}%
\pgfpathlineto{\pgfqpoint{1.526451in}{2.306080in}}%
\pgfpathlineto{\pgfqpoint{1.554370in}{2.256581in}}%
\pgfpathlineto{\pgfqpoint{1.584780in}{2.207554in}}%
\pgfpathlineto{\pgfqpoint{1.621286in}{2.160089in}}%
\pgfpathlineto{\pgfqpoint{1.621286in}{2.160089in}}%
\pgfpathlineto{\pgfqpoint{1.644969in}{2.140247in}}%
\pgfpathlineto{\pgfqpoint{1.644969in}{2.140247in}}%
\pgfpathlineto{\pgfqpoint{1.665297in}{2.132658in}}%
\pgfpathlineto{\pgfqpoint{1.665297in}{2.132658in}}%
\pgfpathlineto{\pgfqpoint{1.687759in}{2.134244in}}%
\pgfpathlineto{\pgfqpoint{1.708413in}{2.142064in}}%
\pgfpathlineto{\pgfqpoint{1.732858in}{2.156197in}}%
\pgfusepath{stroke}%
\end{pgfscope}%
\begin{pgfscope}%
\pgfpathrectangle{\pgfqpoint{0.647939in}{0.492442in}}{\pgfqpoint{4.273799in}{2.331163in}}%
\pgfusepath{clip}%
\pgfsetbuttcap%
\pgfsetroundjoin%
\pgfsetlinewidth{0.301125pt}%
\definecolor{currentstroke}{rgb}{0.500000,0.500000,0.500000}%
\pgfsetstrokecolor{currentstroke}%
\pgfsetstrokeopacity{0.300000}%
\pgfsetdash{}{0pt}%
\pgfpathmoveto{\pgfqpoint{1.230730in}{2.293796in}}%
\pgfpathlineto{\pgfqpoint{1.239202in}{2.242201in}}%
\pgfpathlineto{\pgfqpoint{1.246191in}{2.190540in}}%
\pgfpathlineto{\pgfqpoint{1.251461in}{2.138819in}}%
\pgfpathlineto{\pgfqpoint{1.254754in}{2.087050in}}%
\pgfpathlineto{\pgfqpoint{1.255808in}{2.035255in}}%
\pgfpathlineto{\pgfqpoint{1.254377in}{1.983463in}}%
\pgfpathlineto{\pgfqpoint{1.250255in}{1.931714in}}%
\pgfpathlineto{\pgfqpoint{1.243314in}{1.880056in}}%
\pgfusepath{stroke}%
\end{pgfscope}%
\begin{pgfscope}%
\pgfpathrectangle{\pgfqpoint{0.647939in}{0.492442in}}{\pgfqpoint{4.273799in}{2.331163in}}%
\pgfusepath{clip}%
\pgfsetbuttcap%
\pgfsetroundjoin%
\pgfsetlinewidth{0.301125pt}%
\definecolor{currentstroke}{rgb}{0.500000,0.500000,0.500000}%
\pgfsetstrokecolor{currentstroke}%
\pgfsetstrokeopacity{0.300000}%
\pgfsetdash{}{0pt}%
\pgfpathmoveto{\pgfqpoint{1.711091in}{1.084931in}}%
\pgfpathlineto{\pgfqpoint{1.669848in}{1.131585in}}%
\pgfpathlineto{\pgfqpoint{1.626043in}{1.177529in}}%
\pgfpathlineto{\pgfqpoint{1.578384in}{1.222291in}}%
\pgfpathlineto{\pgfqpoint{1.525071in}{1.264439in}}%
\pgfpathlineto{\pgfqpoint{1.478674in}{1.293111in}}%
\pgfpathlineto{\pgfqpoint{1.436346in}{1.311818in}}%
\pgfpathlineto{\pgfqpoint{1.393213in}{1.322851in}}%
\pgfpathlineto{\pgfqpoint{1.339211in}{1.324564in}}%
\pgfpathlineto{\pgfqpoint{1.288647in}{1.313817in}}%
\pgfpathlineto{\pgfqpoint{1.288647in}{1.313817in}}%
\pgfpathlineto{\pgfqpoint{1.230730in}{1.287157in}}%
\pgfpathlineto{\pgfqpoint{1.230730in}{1.287157in}}%
\pgfusepath{stroke}%
\end{pgfscope}%
\begin{pgfscope}%
\pgfpathrectangle{\pgfqpoint{0.647939in}{0.492442in}}{\pgfqpoint{4.273799in}{2.331163in}}%
\pgfusepath{clip}%
\pgfsetbuttcap%
\pgfsetroundjoin%
\pgfsetlinewidth{0.301125pt}%
\definecolor{currentstroke}{rgb}{0.500000,0.500000,0.500000}%
\pgfsetstrokecolor{currentstroke}%
\pgfsetstrokeopacity{0.300000}%
\pgfsetdash{}{0pt}%
\pgfpathmoveto{\pgfqpoint{2.367277in}{0.766590in}}%
\pgfpathlineto{\pgfqpoint{2.332965in}{0.814894in}}%
\pgfpathlineto{\pgfqpoint{2.299180in}{0.863309in}}%
\pgfpathlineto{\pgfqpoint{2.265898in}{0.911827in}}%
\pgfpathlineto{\pgfqpoint{2.233108in}{0.960445in}}%
\pgfpathlineto{\pgfqpoint{2.200797in}{1.009159in}}%
\pgfpathlineto{\pgfqpoint{2.168946in}{1.057962in}}%
\pgfpathlineto{\pgfqpoint{2.137532in}{1.106849in}}%
\pgfpathlineto{\pgfqpoint{2.106544in}{1.155818in}}%
\pgfpathlineto{\pgfqpoint{2.075973in}{1.204864in}}%
\pgfusepath{stroke}%
\end{pgfscope}%
\begin{pgfscope}%
\pgfpathrectangle{\pgfqpoint{0.647939in}{0.492442in}}{\pgfqpoint{4.273799in}{2.331163in}}%
\pgfusepath{clip}%
\pgfsetbuttcap%
\pgfsetroundjoin%
\pgfsetlinewidth{0.301125pt}%
\definecolor{currentstroke}{rgb}{0.500000,0.500000,0.500000}%
\pgfsetstrokecolor{currentstroke}%
\pgfsetstrokeopacity{0.300000}%
\pgfsetdash{}{0pt}%
\pgfpathmoveto{\pgfqpoint{4.337010in}{1.144622in}}%
\pgfpathlineto{\pgfqpoint{4.291429in}{1.190042in}}%
\pgfpathlineto{\pgfqpoint{4.241816in}{1.234176in}}%
\pgfpathlineto{\pgfqpoint{4.187141in}{1.276486in}}%
\pgfpathlineto{\pgfqpoint{4.126242in}{1.316155in}}%
\pgfpathlineto{\pgfqpoint{4.057961in}{1.352011in}}%
\pgfpathlineto{\pgfqpoint{3.981726in}{1.382682in}}%
\pgfpathlineto{\pgfqpoint{3.898463in}{1.407312in}}%
\pgfpathlineto{\pgfqpoint{3.810379in}{1.426458in}}%
\pgfpathlineto{\pgfqpoint{3.719977in}{1.442233in}}%
\pgfpathlineto{\pgfqpoint{3.629201in}{1.457420in}}%
\pgfusepath{stroke}%
\end{pgfscope}%
\begin{pgfscope}%
\pgfpathrectangle{\pgfqpoint{0.647939in}{0.492442in}}{\pgfqpoint{4.273799in}{2.331163in}}%
\pgfusepath{clip}%
\pgfsetbuttcap%
\pgfsetroundjoin%
\pgfsetlinewidth{0.301125pt}%
\definecolor{currentstroke}{rgb}{0.500000,0.500000,0.500000}%
\pgfsetstrokecolor{currentstroke}%
\pgfsetstrokeopacity{0.300000}%
\pgfsetdash{}{0pt}%
\pgfpathmoveto{\pgfqpoint{4.339516in}{1.463546in}}%
\pgfpathlineto{\pgfqpoint{4.294174in}{1.508996in}}%
\pgfpathlineto{\pgfqpoint{4.241816in}{1.552062in}}%
\pgfpathlineto{\pgfqpoint{4.178446in}{1.590310in}}%
\pgfpathlineto{\pgfqpoint{4.178446in}{1.590310in}}%
\pgfpathlineto{\pgfqpoint{4.118789in}{1.613052in}}%
\pgfpathlineto{\pgfqpoint{4.048223in}{1.625499in}}%
\pgfpathlineto{\pgfqpoint{3.983929in}{1.626705in}}%
\pgfpathlineto{\pgfqpoint{3.914961in}{1.621098in}}%
\pgfpathlineto{\pgfqpoint{3.826623in}{1.609083in}}%
\pgfusepath{stroke}%
\end{pgfscope}%
\begin{pgfscope}%
\pgfpathrectangle{\pgfqpoint{0.647939in}{0.492442in}}{\pgfqpoint{4.273799in}{2.331163in}}%
\pgfusepath{clip}%
\pgfsetbuttcap%
\pgfsetroundjoin%
\pgfsetlinewidth{0.301125pt}%
\definecolor{currentstroke}{rgb}{0.500000,0.500000,0.500000}%
\pgfsetstrokecolor{currentstroke}%
\pgfsetstrokeopacity{0.300000}%
\pgfsetdash{}{0pt}%
\pgfpathmoveto{\pgfqpoint{4.366567in}{1.650310in}}%
\pgfpathlineto{\pgfqpoint{4.326379in}{1.697175in}}%
\pgfpathlineto{\pgfqpoint{4.289670in}{1.731984in}}%
\pgfpathlineto{\pgfqpoint{4.241816in}{1.763986in}}%
\pgfpathlineto{\pgfqpoint{4.241816in}{1.763986in}}%
\pgfpathlineto{\pgfqpoint{4.241816in}{1.763986in}}%
\pgfpathlineto{\pgfqpoint{4.201689in}{1.778944in}}%
\pgfpathlineto{\pgfqpoint{4.201689in}{1.778944in}}%
\pgfpathlineto{\pgfqpoint{4.160628in}{1.783650in}}%
\pgfpathlineto{\pgfqpoint{4.118749in}{1.779623in}}%
\pgfpathlineto{\pgfqpoint{4.076146in}{1.769095in}}%
\pgfpathlineto{\pgfqpoint{4.022578in}{1.750351in}}%
\pgfusepath{stroke}%
\end{pgfscope}%
\begin{pgfscope}%
\pgfpathrectangle{\pgfqpoint{0.647939in}{0.492442in}}{\pgfqpoint{4.273799in}{2.331163in}}%
\pgfusepath{clip}%
\pgfsetbuttcap%
\pgfsetroundjoin%
\pgfsetlinewidth{0.301125pt}%
\definecolor{currentstroke}{rgb}{0.500000,0.500000,0.500000}%
\pgfsetstrokecolor{currentstroke}%
\pgfsetstrokeopacity{0.300000}%
\pgfsetdash{}{0pt}%
\pgfpathmoveto{\pgfqpoint{3.367630in}{2.452738in}}%
\pgfpathlineto{\pgfqpoint{3.393884in}{2.402958in}}%
\pgfpathlineto{\pgfqpoint{3.417963in}{2.352852in}}%
\pgfpathlineto{\pgfqpoint{3.439667in}{2.302424in}}%
\pgfpathlineto{\pgfqpoint{3.458741in}{2.251683in}}%
\pgfpathlineto{\pgfqpoint{3.474840in}{2.200638in}}%
\pgfpathlineto{\pgfqpoint{3.487507in}{2.149310in}}%
\pgfusepath{stroke}%
\end{pgfscope}%
\begin{pgfscope}%
\pgfpathrectangle{\pgfqpoint{0.647939in}{0.492442in}}{\pgfqpoint{4.273799in}{2.331163in}}%
\pgfusepath{clip}%
\pgfsetbuttcap%
\pgfsetroundjoin%
\pgfsetlinewidth{0.301125pt}%
\definecolor{currentstroke}{rgb}{0.500000,0.500000,0.500000}%
\pgfsetstrokecolor{currentstroke}%
\pgfsetstrokeopacity{0.300000}%
\pgfsetdash{}{0pt}%
\pgfpathmoveto{\pgfqpoint{1.675308in}{1.359419in}}%
\pgfpathlineto{\pgfqpoint{1.632624in}{1.405669in}}%
\pgfpathlineto{\pgfqpoint{1.584975in}{1.450399in}}%
\pgfpathlineto{\pgfqpoint{1.541224in}{1.483515in}}%
\pgfpathlineto{\pgfqpoint{1.502957in}{1.505079in}}%
\pgfpathlineto{\pgfqpoint{1.466541in}{1.518381in}}%
\pgfpathlineto{\pgfqpoint{1.422435in}{1.524289in}}%
\pgfpathlineto{\pgfqpoint{1.378082in}{1.518763in}}%
\pgfpathlineto{\pgfqpoint{1.378082in}{1.518763in}}%
\pgfpathlineto{\pgfqpoint{1.327862in}{1.499081in}}%
\pgfpathlineto{\pgfqpoint{1.327862in}{1.499081in}}%
\pgfusepath{stroke}%
\end{pgfscope}%
\begin{pgfscope}%
\pgfpathrectangle{\pgfqpoint{0.647939in}{0.492442in}}{\pgfqpoint{4.273799in}{2.331163in}}%
\pgfusepath{clip}%
\pgfsetbuttcap%
\pgfsetroundjoin%
\pgfsetlinewidth{0.301125pt}%
\definecolor{currentstroke}{rgb}{0.500000,0.500000,0.500000}%
\pgfsetstrokecolor{currentstroke}%
\pgfsetstrokeopacity{0.300000}%
\pgfsetdash{}{0pt}%
\pgfpathmoveto{\pgfqpoint{4.047552in}{0.916290in}}%
\pgfpathlineto{\pgfqpoint{3.985680in}{0.955572in}}%
\pgfpathlineto{\pgfqpoint{3.920712in}{0.993339in}}%
\pgfpathlineto{\pgfqpoint{3.852995in}{1.029647in}}%
\pgfpathlineto{\pgfqpoint{3.783080in}{1.064698in}}%
\pgfpathlineto{\pgfqpoint{3.711658in}{1.098839in}}%
\pgfpathlineto{\pgfqpoint{3.639553in}{1.132552in}}%
\pgfusepath{stroke}%
\end{pgfscope}%
\begin{pgfscope}%
\pgfpathrectangle{\pgfqpoint{0.647939in}{0.492442in}}{\pgfqpoint{4.273799in}{2.331163in}}%
\pgfusepath{clip}%
\pgfsetbuttcap%
\pgfsetroundjoin%
\pgfsetlinewidth{0.301125pt}%
\definecolor{currentstroke}{rgb}{0.500000,0.500000,0.500000}%
\pgfsetstrokecolor{currentstroke}%
\pgfsetstrokeopacity{0.300000}%
\pgfsetdash{}{0pt}%
\pgfpathmoveto{\pgfqpoint{4.252504in}{1.043031in}}%
\pgfpathlineto{\pgfqpoint{4.200734in}{1.086431in}}%
\pgfpathlineto{\pgfqpoint{4.144684in}{1.128214in}}%
\pgfpathlineto{\pgfqpoint{4.083780in}{1.167918in}}%
\pgfpathlineto{\pgfqpoint{4.017612in}{1.205013in}}%
\pgfpathlineto{\pgfqpoint{3.946098in}{1.239026in}}%
\pgfpathlineto{\pgfqpoint{3.869783in}{1.269785in}}%
\pgfpathlineto{\pgfqpoint{3.789750in}{1.297625in}}%
\pgfpathlineto{\pgfqpoint{3.707480in}{1.323484in}}%
\pgfpathlineto{\pgfqpoint{3.624523in}{1.348691in}}%
\pgfpathlineto{\pgfqpoint{3.542328in}{1.374616in}}%
\pgfpathlineto{\pgfqpoint{3.462205in}{1.402372in}}%
\pgfpathlineto{\pgfqpoint{3.385247in}{1.432643in}}%
\pgfusepath{stroke}%
\end{pgfscope}%
\begin{pgfscope}%
\pgfpathrectangle{\pgfqpoint{0.647939in}{0.492442in}}{\pgfqpoint{4.273799in}{2.331163in}}%
\pgfusepath{clip}%
\pgfsetbuttcap%
\pgfsetroundjoin%
\pgfsetlinewidth{0.301125pt}%
\definecolor{currentstroke}{rgb}{0.500000,0.500000,0.500000}%
\pgfsetstrokecolor{currentstroke}%
\pgfsetstrokeopacity{0.300000}%
\pgfsetdash{}{0pt}%
\pgfpathmoveto{\pgfqpoint{2.572641in}{0.926214in}}%
\pgfpathlineto{\pgfqpoint{2.537442in}{0.974328in}}%
\pgfpathlineto{\pgfqpoint{2.503108in}{1.022627in}}%
\pgfpathlineto{\pgfqpoint{2.469628in}{1.071104in}}%
\pgfpathlineto{\pgfqpoint{2.436996in}{1.119753in}}%
\pgfpathlineto{\pgfqpoint{2.405216in}{1.168570in}}%
\pgfpathlineto{\pgfqpoint{2.374298in}{1.217551in}}%
\pgfpathlineto{\pgfqpoint{2.344248in}{1.266692in}}%
\pgfpathlineto{\pgfqpoint{2.315079in}{1.315991in}}%
\pgfpathlineto{\pgfqpoint{2.286825in}{1.365449in}}%
\pgfpathlineto{\pgfqpoint{2.259515in}{1.415064in}}%
\pgfpathlineto{\pgfqpoint{2.233191in}{1.464837in}}%
\pgfpathlineto{\pgfqpoint{2.207919in}{1.514771in}}%
\pgfpathlineto{\pgfqpoint{2.183770in}{1.564871in}}%
\pgfpathlineto{\pgfqpoint{2.160848in}{1.615142in}}%
\pgfpathlineto{\pgfqpoint{2.139275in}{1.665589in}}%
\pgfpathlineto{\pgfqpoint{2.119213in}{1.716222in}}%
\pgfpathlineto{\pgfqpoint{2.100873in}{1.767048in}}%
\pgfpathlineto{\pgfqpoint{2.084521in}{1.818073in}}%
\pgfpathlineto{\pgfqpoint{2.070516in}{1.869305in}}%
\pgfpathlineto{\pgfqpoint{2.059307in}{1.920738in}}%
\pgfpathlineto{\pgfqpoint{2.051481in}{1.972354in}}%
\pgfpathlineto{\pgfqpoint{2.047803in}{2.024103in}}%
\pgfpathlineto{\pgfqpoint{2.049222in}{2.075875in}}%
\pgfpathlineto{\pgfqpoint{2.056885in}{2.127470in}}%
\pgfpathlineto{\pgfqpoint{2.072063in}{2.178548in}}%
\pgfpathlineto{\pgfqpoint{2.096082in}{2.228585in}}%
\pgfpathlineto{\pgfqpoint{2.130088in}{2.276846in}}%
\pgfpathlineto{\pgfqpoint{2.174929in}{2.322366in}}%
\pgfpathlineto{\pgfqpoint{2.231176in}{2.363893in}}%
\pgfpathlineto{\pgfqpoint{2.299180in}{2.399758in}}%
\pgfusepath{stroke}%
\end{pgfscope}%
\begin{pgfscope}%
\pgfpathrectangle{\pgfqpoint{0.647939in}{0.492442in}}{\pgfqpoint{4.273799in}{2.331163in}}%
\pgfusepath{clip}%
\pgfsetbuttcap%
\pgfsetroundjoin%
\pgfsetlinewidth{0.301125pt}%
\definecolor{currentstroke}{rgb}{0.500000,0.500000,0.500000}%
\pgfsetstrokecolor{currentstroke}%
\pgfsetstrokeopacity{0.300000}%
\pgfsetdash{}{0pt}%
\pgfpathmoveto{\pgfqpoint{4.107819in}{0.982244in}}%
\pgfpathlineto{\pgfqpoint{4.047552in}{1.022252in}}%
\pgfpathlineto{\pgfqpoint{3.983420in}{1.060429in}}%
\pgfpathlineto{\pgfqpoint{3.915568in}{1.096645in}}%
\pgfpathlineto{\pgfqpoint{3.844429in}{1.130941in}}%
\pgfpathlineto{\pgfqpoint{3.770747in}{1.163612in}}%
\pgfusepath{stroke}%
\end{pgfscope}%
\begin{pgfscope}%
\pgfpathrectangle{\pgfqpoint{0.647939in}{0.492442in}}{\pgfqpoint{4.273799in}{2.331163in}}%
\pgfusepath{clip}%
\pgfsetbuttcap%
\pgfsetroundjoin%
\pgfsetlinewidth{0.301125pt}%
\definecolor{currentstroke}{rgb}{0.500000,0.500000,0.500000}%
\pgfsetstrokecolor{currentstroke}%
\pgfsetstrokeopacity{0.300000}%
\pgfsetdash{}{0pt}%
\pgfpathmoveto{\pgfqpoint{2.548537in}{2.361885in}}%
\pgfpathlineto{\pgfqpoint{2.616947in}{2.371533in}}%
\pgfpathlineto{\pgfqpoint{2.699061in}{2.367468in}}%
\pgfpathlineto{\pgfqpoint{2.784839in}{2.346777in}}%
\pgfpathlineto{\pgfqpoint{2.858982in}{2.315275in}}%
\pgfpathlineto{\pgfqpoint{2.916917in}{2.280828in}}%
\pgfusepath{stroke}%
\end{pgfscope}%
\begin{pgfscope}%
\pgfpathrectangle{\pgfqpoint{0.647939in}{0.492442in}}{\pgfqpoint{4.273799in}{2.331163in}}%
\pgfusepath{clip}%
\pgfsetbuttcap%
\pgfsetroundjoin%
\pgfsetlinewidth{0.301125pt}%
\definecolor{currentstroke}{rgb}{0.500000,0.500000,0.500000}%
\pgfsetstrokecolor{currentstroke}%
\pgfsetstrokeopacity{0.300000}%
\pgfsetdash{}{0pt}%
\pgfpathmoveto{\pgfqpoint{3.737896in}{2.344629in}}%
\pgfpathlineto{\pgfqpoint{3.756157in}{2.293796in}}%
\pgfpathlineto{\pgfqpoint{3.772210in}{2.242742in}}%
\pgfpathlineto{\pgfqpoint{3.785694in}{2.191470in}}%
\pgfpathlineto{\pgfqpoint{3.796126in}{2.139988in}}%
\pgfpathlineto{\pgfqpoint{3.802893in}{2.088329in}}%
\pgfusepath{stroke}%
\end{pgfscope}%
\begin{pgfscope}%
\pgfpathrectangle{\pgfqpoint{0.647939in}{0.492442in}}{\pgfqpoint{4.273799in}{2.331163in}}%
\pgfusepath{clip}%
\pgfsetbuttcap%
\pgfsetroundjoin%
\pgfsetlinewidth{0.301125pt}%
\definecolor{currentstroke}{rgb}{0.500000,0.500000,0.500000}%
\pgfsetstrokecolor{currentstroke}%
\pgfsetstrokeopacity{0.300000}%
\pgfsetdash{}{0pt}%
\pgfpathmoveto{\pgfqpoint{2.544891in}{1.675234in}}%
\pgfpathlineto{\pgfqpoint{2.522322in}{1.725550in}}%
\pgfpathlineto{\pgfqpoint{2.502043in}{1.776153in}}%
\pgfpathlineto{\pgfqpoint{2.484304in}{1.827039in}}%
\pgfpathlineto{\pgfqpoint{2.469429in}{1.878195in}}%
\pgfpathlineto{\pgfqpoint{2.457850in}{1.929603in}}%
\pgfpathlineto{\pgfqpoint{2.450134in}{1.981221in}}%
\pgfpathlineto{\pgfqpoint{2.447053in}{2.032975in}}%
\pgfpathlineto{\pgfqpoint{2.449682in}{2.084725in}}%
\pgfpathlineto{\pgfqpoint{2.459553in}{2.136192in}}%
\pgfpathlineto{\pgfqpoint{2.478969in}{2.186803in}}%
\pgfpathlineto{\pgfqpoint{2.507409in}{2.230615in}}%
\pgfpathlineto{\pgfqpoint{2.541809in}{2.264436in}}%
\pgfpathlineto{\pgfqpoint{2.590575in}{2.293796in}}%
\pgfpathlineto{\pgfqpoint{2.590575in}{2.293796in}}%
\pgfpathlineto{\pgfqpoint{2.590575in}{2.293796in}}%
\pgfpathlineto{\pgfqpoint{2.639291in}{2.310005in}}%
\pgfpathlineto{\pgfqpoint{2.696062in}{2.316833in}}%
\pgfpathlineto{\pgfqpoint{2.749378in}{2.313820in}}%
\pgfpathlineto{\pgfqpoint{2.801631in}{2.302909in}}%
\pgfusepath{stroke}%
\end{pgfscope}%
\begin{pgfscope}%
\pgfpathrectangle{\pgfqpoint{0.647939in}{0.492442in}}{\pgfqpoint{4.273799in}{2.331163in}}%
\pgfusepath{clip}%
\pgfsetbuttcap%
\pgfsetroundjoin%
\pgfsetlinewidth{0.301125pt}%
\definecolor{currentstroke}{rgb}{0.500000,0.500000,0.500000}%
\pgfsetstrokecolor{currentstroke}%
\pgfsetstrokeopacity{0.300000}%
\pgfsetdash{}{0pt}%
\pgfpathmoveto{\pgfqpoint{2.078804in}{1.213109in}}%
\pgfpathlineto{\pgfqpoint{2.048757in}{1.262251in}}%
\pgfpathlineto{\pgfqpoint{2.019106in}{1.311465in}}%
\pgfpathlineto{\pgfqpoint{1.989846in}{1.360749in}}%
\pgfpathlineto{\pgfqpoint{1.960957in}{1.410097in}}%
\pgfpathlineto{\pgfqpoint{1.932425in}{1.459507in}}%
\pgfpathlineto{\pgfqpoint{1.904250in}{1.508978in}}%
\pgfpathlineto{\pgfqpoint{1.876411in}{1.558505in}}%
\pgfpathlineto{\pgfqpoint{1.848897in}{1.608086in}}%
\pgfpathlineto{\pgfqpoint{1.821706in}{1.657719in}}%
\pgfpathlineto{\pgfqpoint{1.794813in}{1.707399in}}%
\pgfpathlineto{\pgfqpoint{1.768222in}{1.757127in}}%
\pgfpathlineto{\pgfqpoint{1.741910in}{1.806897in}}%
\pgfpathlineto{\pgfqpoint{1.715843in}{1.856697in}}%
\pgfpathlineto{\pgfqpoint{1.690010in}{1.906531in}}%
\pgfpathlineto{\pgfqpoint{1.664305in}{1.956325in}}%
\pgfpathlineto{\pgfqpoint{1.645957in}{1.991601in}}%
\pgfpathlineto{\pgfqpoint{1.633870in}{2.013876in}}%
\pgfpathlineto{\pgfqpoint{1.625471in}{2.026418in}}%
\pgfpathlineto{\pgfqpoint{1.625471in}{2.026418in}}%
\pgfpathlineto{\pgfqpoint{1.619257in}{2.028891in}}%
\pgfpathlineto{\pgfqpoint{1.619257in}{2.028891in}}%
\pgfpathlineto{\pgfqpoint{1.619257in}{2.028891in}}%
\pgfpathlineto{\pgfqpoint{1.611044in}{2.021445in}}%
\pgfpathlineto{\pgfqpoint{1.611044in}{2.021445in}}%
\pgfpathlineto{\pgfqpoint{1.599001in}{2.010161in}}%
\pgfusepath{stroke}%
\end{pgfscope}%
\begin{pgfscope}%
\pgfpathrectangle{\pgfqpoint{0.647939in}{0.492442in}}{\pgfqpoint{4.273799in}{2.331163in}}%
\pgfusepath{clip}%
\pgfsetbuttcap%
\pgfsetroundjoin%
\pgfsetlinewidth{0.301125pt}%
\definecolor{currentstroke}{rgb}{0.500000,0.500000,0.500000}%
\pgfsetstrokecolor{currentstroke}%
\pgfsetstrokeopacity{0.300000}%
\pgfsetdash{}{0pt}%
\pgfpathmoveto{\pgfqpoint{2.129103in}{1.479054in}}%
\pgfpathlineto{\pgfqpoint{2.103926in}{1.529004in}}%
\pgfpathlineto{\pgfqpoint{2.079752in}{1.579100in}}%
\pgfpathlineto{\pgfqpoint{2.056680in}{1.629350in}}%
\pgfpathlineto{\pgfqpoint{2.034848in}{1.679765in}}%
\pgfpathlineto{\pgfqpoint{2.014435in}{1.730355in}}%
\pgfpathlineto{\pgfqpoint{1.995677in}{1.781135in}}%
\pgfpathlineto{\pgfqpoint{1.978901in}{1.832120in}}%
\pgfpathlineto{\pgfqpoint{1.964534in}{1.883321in}}%
\pgfpathlineto{\pgfqpoint{1.953170in}{1.934743in}}%
\pgfpathlineto{\pgfqpoint{1.945612in}{1.986366in}}%
\pgfpathlineto{\pgfqpoint{1.942906in}{2.038123in}}%
\pgfpathlineto{\pgfqpoint{1.946364in}{2.089854in}}%
\pgfpathlineto{\pgfqpoint{1.957483in}{2.141247in}}%
\pgfpathlineto{\pgfqpoint{1.977647in}{2.191794in}}%
\pgfpathlineto{\pgfqpoint{2.007784in}{2.240815in}}%
\pgfusepath{stroke}%
\end{pgfscope}%
\begin{pgfscope}%
\pgfpathrectangle{\pgfqpoint{0.647939in}{0.492442in}}{\pgfqpoint{4.273799in}{2.331163in}}%
\pgfusepath{clip}%
\pgfsetbuttcap%
\pgfsetroundjoin%
\pgfsetlinewidth{0.301125pt}%
\definecolor{currentstroke}{rgb}{0.500000,0.500000,0.500000}%
\pgfsetstrokecolor{currentstroke}%
\pgfsetstrokeopacity{0.300000}%
\pgfsetdash{}{0pt}%
\pgfpathmoveto{\pgfqpoint{3.561893in}{2.134853in}}%
\pgfpathlineto{\pgfqpoint{3.568738in}{2.083200in}}%
\pgfpathlineto{\pgfqpoint{3.570694in}{2.031433in}}%
\pgfpathlineto{\pgfqpoint{3.566572in}{1.979715in}}%
\pgfpathlineto{\pgfqpoint{3.554681in}{1.928379in}}%
\pgfpathlineto{\pgfqpoint{3.532581in}{1.878113in}}%
\pgfpathlineto{\pgfqpoint{3.496625in}{1.830432in}}%
\pgfpathlineto{\pgfqpoint{3.496625in}{1.830432in}}%
\pgfpathlineto{\pgfqpoint{3.452543in}{1.794944in}}%
\pgfpathlineto{\pgfqpoint{3.452543in}{1.794944in}}%
\pgfpathlineto{\pgfqpoint{3.403569in}{1.771227in}}%
\pgfpathlineto{\pgfqpoint{3.343255in}{1.756871in}}%
\pgfpathlineto{\pgfqpoint{3.285825in}{1.754442in}}%
\pgfpathlineto{\pgfqpoint{3.231155in}{1.760820in}}%
\pgfpathlineto{\pgfqpoint{3.177443in}{1.774890in}}%
\pgfusepath{stroke}%
\end{pgfscope}%
\begin{pgfscope}%
\pgfpathrectangle{\pgfqpoint{0.647939in}{0.492442in}}{\pgfqpoint{4.273799in}{2.331163in}}%
\pgfusepath{clip}%
\pgfsetbuttcap%
\pgfsetroundjoin%
\pgfsetlinewidth{0.301125pt}%
\definecolor{currentstroke}{rgb}{0.500000,0.500000,0.500000}%
\pgfsetstrokecolor{currentstroke}%
\pgfsetstrokeopacity{0.300000}%
\pgfsetdash{}{0pt}%
\pgfpathmoveto{\pgfqpoint{3.270498in}{2.134853in}}%
\pgfpathlineto{\pgfqpoint{3.281160in}{2.083413in}}%
\pgfpathlineto{\pgfqpoint{3.284299in}{2.031703in}}%
\pgfpathlineto{\pgfqpoint{3.276726in}{1.980199in}}%
\pgfpathlineto{\pgfqpoint{3.251581in}{1.930672in}}%
\pgfpathlineto{\pgfqpoint{3.251581in}{1.930672in}}%
\pgfpathlineto{\pgfqpoint{3.222457in}{1.904739in}}%
\pgfpathlineto{\pgfqpoint{3.222457in}{1.904739in}}%
\pgfpathlineto{\pgfqpoint{3.188498in}{1.890360in}}%
\pgfpathlineto{\pgfqpoint{3.146232in}{1.886257in}}%
\pgfusepath{stroke}%
\end{pgfscope}%
\begin{pgfscope}%
\pgfpathrectangle{\pgfqpoint{0.647939in}{0.492442in}}{\pgfqpoint{4.273799in}{2.331163in}}%
\pgfusepath{clip}%
\pgfsetbuttcap%
\pgfsetroundjoin%
\pgfsetlinewidth{0.301125pt}%
\definecolor{currentstroke}{rgb}{0.500000,0.500000,0.500000}%
\pgfsetstrokecolor{currentstroke}%
\pgfsetstrokeopacity{0.300000}%
\pgfsetdash{}{0pt}%
\pgfpathmoveto{\pgfqpoint{2.769198in}{1.727834in}}%
\pgfpathlineto{\pgfqpoint{2.742037in}{1.777464in}}%
\pgfpathlineto{\pgfqpoint{2.718195in}{1.827597in}}%
\pgfpathlineto{\pgfqpoint{2.698046in}{1.878208in}}%
\pgfpathlineto{\pgfqpoint{2.682145in}{1.929263in}}%
\pgfpathlineto{\pgfqpoint{2.671322in}{1.980705in}}%
\pgfpathlineto{\pgfqpoint{2.666867in}{2.032408in}}%
\pgfpathlineto{\pgfqpoint{2.670958in}{2.084077in}}%
\pgfpathlineto{\pgfqpoint{2.687707in}{2.134853in}}%
\pgfpathlineto{\pgfqpoint{2.687707in}{2.134853in}}%
\pgfpathlineto{\pgfqpoint{2.715431in}{2.173281in}}%
\pgfpathlineto{\pgfqpoint{2.715431in}{2.173281in}}%
\pgfusepath{stroke}%
\end{pgfscope}%
\begin{pgfscope}%
\pgfpathrectangle{\pgfqpoint{0.647939in}{0.492442in}}{\pgfqpoint{4.273799in}{2.331163in}}%
\pgfusepath{clip}%
\pgfsetbuttcap%
\pgfsetroundjoin%
\pgfsetlinewidth{0.301125pt}%
\definecolor{currentstroke}{rgb}{0.500000,0.500000,0.500000}%
\pgfsetstrokecolor{currentstroke}%
\pgfsetstrokeopacity{0.300000}%
\pgfsetdash{}{0pt}%
\pgfpathmoveto{\pgfqpoint{3.230491in}{1.298773in}}%
\pgfpathlineto{\pgfqpoint{3.173366in}{1.340138in}}%
\pgfpathlineto{\pgfqpoint{3.119144in}{1.382648in}}%
\pgfpathlineto{\pgfqpoint{3.067849in}{1.426231in}}%
\pgfpathlineto{\pgfqpoint{3.019489in}{1.470803in}}%
\pgfpathlineto{\pgfqpoint{2.974057in}{1.516282in}}%
\pgfpathlineto{\pgfqpoint{2.931542in}{1.562593in}}%
\pgfpathlineto{\pgfqpoint{2.891952in}{1.609671in}}%
\pgfusepath{stroke}%
\end{pgfscope}%
\begin{pgfscope}%
\pgfpathrectangle{\pgfqpoint{0.647939in}{0.492442in}}{\pgfqpoint{4.273799in}{2.331163in}}%
\pgfusepath{clip}%
\pgfsetroundcap%
\pgfsetroundjoin%
\pgfsetlinewidth{0.301125pt}%
\definecolor{currentstroke}{rgb}{0.500000,0.500000,0.500000}%
\pgfsetstrokecolor{currentstroke}%
\pgfsetstrokeopacity{0.300000}%
\pgfsetdash{}{0pt}%
\pgfpathmoveto{\pgfqpoint{1.462577in}{1.442904in}}%
\pgfusepath{stroke}%
\end{pgfscope}%
\begin{pgfscope}%
\pgfpathrectangle{\pgfqpoint{0.647939in}{0.492442in}}{\pgfqpoint{4.273799in}{2.331163in}}%
\pgfusepath{clip}%
\pgfsetroundcap%
\pgfsetroundjoin%
\definecolor{currentfill}{rgb}{0.500000,0.500000,0.500000}%
\pgfsetfillcolor{currentfill}%
\pgfsetfillopacity{0.300000}%
\pgfsetlinewidth{0.301125pt}%
\definecolor{currentstroke}{rgb}{0.500000,0.500000,0.500000}%
\pgfsetstrokecolor{currentstroke}%
\pgfsetstrokeopacity{0.300000}%
\pgfsetdash{}{0pt}%
\pgfpathmoveto{\pgfqpoint{0.000000in}{0.000000in}}%
\pgfpathlineto{\pgfqpoint{0.000000in}{0.000000in}}%
\pgfpathclose%
\pgfusepath{stroke,fill}%
\end{pgfscope}%
\begin{pgfscope}%
\pgfpathrectangle{\pgfqpoint{0.647939in}{0.492442in}}{\pgfqpoint{4.273799in}{2.331163in}}%
\pgfusepath{clip}%
\pgfsetroundcap%
\pgfsetroundjoin%
\pgfsetlinewidth{0.301125pt}%
\definecolor{currentstroke}{rgb}{0.500000,0.500000,0.500000}%
\pgfsetstrokecolor{currentstroke}%
\pgfsetstrokeopacity{0.300000}%
\pgfsetdash{}{0pt}%
\pgfpathmoveto{\pgfqpoint{1.247264in}{0.904953in}}%
\pgfusepath{stroke}%
\end{pgfscope}%
\begin{pgfscope}%
\pgfpathrectangle{\pgfqpoint{0.647939in}{0.492442in}}{\pgfqpoint{4.273799in}{2.331163in}}%
\pgfusepath{clip}%
\pgfsetroundcap%
\pgfsetroundjoin%
\definecolor{currentfill}{rgb}{0.500000,0.500000,0.500000}%
\pgfsetfillcolor{currentfill}%
\pgfsetfillopacity{0.300000}%
\pgfsetlinewidth{0.301125pt}%
\definecolor{currentstroke}{rgb}{0.500000,0.500000,0.500000}%
\pgfsetstrokecolor{currentstroke}%
\pgfsetstrokeopacity{0.300000}%
\pgfsetdash{}{0pt}%
\pgfpathmoveto{\pgfqpoint{0.000000in}{0.000000in}}%
\pgfpathlineto{\pgfqpoint{0.000000in}{0.000000in}}%
\pgfpathclose%
\pgfusepath{stroke,fill}%
\end{pgfscope}%
\begin{pgfscope}%
\pgfpathrectangle{\pgfqpoint{0.647939in}{0.492442in}}{\pgfqpoint{4.273799in}{2.331163in}}%
\pgfusepath{clip}%
\pgfsetroundcap%
\pgfsetroundjoin%
\pgfsetlinewidth{0.301125pt}%
\definecolor{currentstroke}{rgb}{0.500000,0.500000,0.500000}%
\pgfsetstrokecolor{currentstroke}%
\pgfsetstrokeopacity{0.300000}%
\pgfsetdash{}{0pt}%
\pgfpathmoveto{\pgfqpoint{1.211958in}{0.686937in}}%
\pgfusepath{stroke}%
\end{pgfscope}%
\begin{pgfscope}%
\pgfpathrectangle{\pgfqpoint{0.647939in}{0.492442in}}{\pgfqpoint{4.273799in}{2.331163in}}%
\pgfusepath{clip}%
\pgfsetroundcap%
\pgfsetroundjoin%
\definecolor{currentfill}{rgb}{0.500000,0.500000,0.500000}%
\pgfsetfillcolor{currentfill}%
\pgfsetfillopacity{0.300000}%
\pgfsetlinewidth{0.301125pt}%
\definecolor{currentstroke}{rgb}{0.500000,0.500000,0.500000}%
\pgfsetstrokecolor{currentstroke}%
\pgfsetstrokeopacity{0.300000}%
\pgfsetdash{}{0pt}%
\pgfpathmoveto{\pgfqpoint{0.000000in}{0.000000in}}%
\pgfpathlineto{\pgfqpoint{0.000000in}{0.000000in}}%
\pgfpathclose%
\pgfusepath{stroke,fill}%
\end{pgfscope}%
\begin{pgfscope}%
\pgfpathrectangle{\pgfqpoint{0.647939in}{0.492442in}}{\pgfqpoint{4.273799in}{2.331163in}}%
\pgfusepath{clip}%
\pgfsetroundcap%
\pgfsetroundjoin%
\pgfsetlinewidth{0.301125pt}%
\definecolor{currentstroke}{rgb}{0.500000,0.500000,0.500000}%
\pgfsetstrokecolor{currentstroke}%
\pgfsetstrokeopacity{0.300000}%
\pgfsetdash{}{0pt}%
\pgfpathmoveto{\pgfqpoint{1.174411in}{0.570672in}}%
\pgfusepath{stroke}%
\end{pgfscope}%
\begin{pgfscope}%
\pgfpathrectangle{\pgfqpoint{0.647939in}{0.492442in}}{\pgfqpoint{4.273799in}{2.331163in}}%
\pgfusepath{clip}%
\pgfsetroundcap%
\pgfsetroundjoin%
\definecolor{currentfill}{rgb}{0.500000,0.500000,0.500000}%
\pgfsetfillcolor{currentfill}%
\pgfsetfillopacity{0.300000}%
\pgfsetlinewidth{0.301125pt}%
\definecolor{currentstroke}{rgb}{0.500000,0.500000,0.500000}%
\pgfsetstrokecolor{currentstroke}%
\pgfsetstrokeopacity{0.300000}%
\pgfsetdash{}{0pt}%
\pgfpathmoveto{\pgfqpoint{0.000000in}{0.000000in}}%
\pgfpathlineto{\pgfqpoint{0.000000in}{0.000000in}}%
\pgfpathclose%
\pgfusepath{stroke,fill}%
\end{pgfscope}%
\begin{pgfscope}%
\pgfpathrectangle{\pgfqpoint{0.647939in}{0.492442in}}{\pgfqpoint{4.273799in}{2.331163in}}%
\pgfusepath{clip}%
\pgfsetroundcap%
\pgfsetroundjoin%
\pgfsetlinewidth{0.301125pt}%
\definecolor{currentstroke}{rgb}{0.500000,0.500000,0.500000}%
\pgfsetstrokecolor{currentstroke}%
\pgfsetstrokeopacity{0.300000}%
\pgfsetdash{}{0pt}%
\pgfpathmoveto{\pgfqpoint{1.561377in}{0.884091in}}%
\pgfusepath{stroke}%
\end{pgfscope}%
\begin{pgfscope}%
\pgfpathrectangle{\pgfqpoint{0.647939in}{0.492442in}}{\pgfqpoint{4.273799in}{2.331163in}}%
\pgfusepath{clip}%
\pgfsetroundcap%
\pgfsetroundjoin%
\definecolor{currentfill}{rgb}{0.500000,0.500000,0.500000}%
\pgfsetfillcolor{currentfill}%
\pgfsetfillopacity{0.300000}%
\pgfsetlinewidth{0.301125pt}%
\definecolor{currentstroke}{rgb}{0.500000,0.500000,0.500000}%
\pgfsetstrokecolor{currentstroke}%
\pgfsetstrokeopacity{0.300000}%
\pgfsetdash{}{0pt}%
\pgfpathmoveto{\pgfqpoint{0.000000in}{0.000000in}}%
\pgfpathlineto{\pgfqpoint{0.000000in}{0.000000in}}%
\pgfpathclose%
\pgfusepath{stroke,fill}%
\end{pgfscope}%
\begin{pgfscope}%
\pgfpathrectangle{\pgfqpoint{0.647939in}{0.492442in}}{\pgfqpoint{4.273799in}{2.331163in}}%
\pgfusepath{clip}%
\pgfsetroundcap%
\pgfsetroundjoin%
\pgfsetlinewidth{0.301125pt}%
\definecolor{currentstroke}{rgb}{0.500000,0.500000,0.500000}%
\pgfsetstrokecolor{currentstroke}%
\pgfsetstrokeopacity{0.300000}%
\pgfsetdash{}{0pt}%
\pgfpathmoveto{\pgfqpoint{1.547402in}{1.026500in}}%
\pgfusepath{stroke}%
\end{pgfscope}%
\begin{pgfscope}%
\pgfpathrectangle{\pgfqpoint{0.647939in}{0.492442in}}{\pgfqpoint{4.273799in}{2.331163in}}%
\pgfusepath{clip}%
\pgfsetroundcap%
\pgfsetroundjoin%
\definecolor{currentfill}{rgb}{0.500000,0.500000,0.500000}%
\pgfsetfillcolor{currentfill}%
\pgfsetfillopacity{0.300000}%
\pgfsetlinewidth{0.301125pt}%
\definecolor{currentstroke}{rgb}{0.500000,0.500000,0.500000}%
\pgfsetstrokecolor{currentstroke}%
\pgfsetstrokeopacity{0.300000}%
\pgfsetdash{}{0pt}%
\pgfpathmoveto{\pgfqpoint{0.000000in}{0.000000in}}%
\pgfpathlineto{\pgfqpoint{0.000000in}{0.000000in}}%
\pgfpathclose%
\pgfusepath{stroke,fill}%
\end{pgfscope}%
\begin{pgfscope}%
\pgfpathrectangle{\pgfqpoint{0.647939in}{0.492442in}}{\pgfqpoint{4.273799in}{2.331163in}}%
\pgfusepath{clip}%
\pgfsetroundcap%
\pgfsetroundjoin%
\pgfsetlinewidth{0.301125pt}%
\definecolor{currentstroke}{rgb}{0.500000,0.500000,0.500000}%
\pgfsetstrokecolor{currentstroke}%
\pgfsetstrokeopacity{0.300000}%
\pgfsetdash{}{0pt}%
\pgfpathmoveto{\pgfqpoint{1.636381in}{1.083402in}}%
\pgfusepath{stroke}%
\end{pgfscope}%
\begin{pgfscope}%
\pgfpathrectangle{\pgfqpoint{0.647939in}{0.492442in}}{\pgfqpoint{4.273799in}{2.331163in}}%
\pgfusepath{clip}%
\pgfsetroundcap%
\pgfsetroundjoin%
\definecolor{currentfill}{rgb}{0.500000,0.500000,0.500000}%
\pgfsetfillcolor{currentfill}%
\pgfsetfillopacity{0.300000}%
\pgfsetlinewidth{0.301125pt}%
\definecolor{currentstroke}{rgb}{0.500000,0.500000,0.500000}%
\pgfsetstrokecolor{currentstroke}%
\pgfsetstrokeopacity{0.300000}%
\pgfsetdash{}{0pt}%
\pgfpathmoveto{\pgfqpoint{0.000000in}{0.000000in}}%
\pgfpathlineto{\pgfqpoint{0.000000in}{0.000000in}}%
\pgfpathclose%
\pgfusepath{stroke,fill}%
\end{pgfscope}%
\begin{pgfscope}%
\pgfpathrectangle{\pgfqpoint{0.647939in}{0.492442in}}{\pgfqpoint{4.273799in}{2.331163in}}%
\pgfusepath{clip}%
\pgfsetroundcap%
\pgfsetroundjoin%
\pgfsetlinewidth{0.301125pt}%
\definecolor{currentstroke}{rgb}{0.500000,0.500000,0.500000}%
\pgfsetstrokecolor{currentstroke}%
\pgfsetstrokeopacity{0.300000}%
\pgfsetdash{}{0pt}%
\pgfpathmoveto{\pgfqpoint{1.839651in}{1.289840in}}%
\pgfusepath{stroke}%
\end{pgfscope}%
\begin{pgfscope}%
\pgfpathrectangle{\pgfqpoint{0.647939in}{0.492442in}}{\pgfqpoint{4.273799in}{2.331163in}}%
\pgfusepath{clip}%
\pgfsetroundcap%
\pgfsetroundjoin%
\definecolor{currentfill}{rgb}{0.500000,0.500000,0.500000}%
\pgfsetfillcolor{currentfill}%
\pgfsetfillopacity{0.300000}%
\pgfsetlinewidth{0.301125pt}%
\definecolor{currentstroke}{rgb}{0.500000,0.500000,0.500000}%
\pgfsetstrokecolor{currentstroke}%
\pgfsetstrokeopacity{0.300000}%
\pgfsetdash{}{0pt}%
\pgfpathmoveto{\pgfqpoint{0.000000in}{0.000000in}}%
\pgfpathlineto{\pgfqpoint{0.000000in}{0.000000in}}%
\pgfpathclose%
\pgfusepath{stroke,fill}%
\end{pgfscope}%
\begin{pgfscope}%
\pgfpathrectangle{\pgfqpoint{0.647939in}{0.492442in}}{\pgfqpoint{4.273799in}{2.331163in}}%
\pgfusepath{clip}%
\pgfsetroundcap%
\pgfsetroundjoin%
\pgfsetlinewidth{0.301125pt}%
\definecolor{currentstroke}{rgb}{0.500000,0.500000,0.500000}%
\pgfsetstrokecolor{currentstroke}%
\pgfsetstrokeopacity{0.300000}%
\pgfsetdash{}{0pt}%
\pgfpathmoveto{\pgfqpoint{1.942926in}{1.291410in}}%
\pgfusepath{stroke}%
\end{pgfscope}%
\begin{pgfscope}%
\pgfpathrectangle{\pgfqpoint{0.647939in}{0.492442in}}{\pgfqpoint{4.273799in}{2.331163in}}%
\pgfusepath{clip}%
\pgfsetroundcap%
\pgfsetroundjoin%
\definecolor{currentfill}{rgb}{0.500000,0.500000,0.500000}%
\pgfsetfillcolor{currentfill}%
\pgfsetfillopacity{0.300000}%
\pgfsetlinewidth{0.301125pt}%
\definecolor{currentstroke}{rgb}{0.500000,0.500000,0.500000}%
\pgfsetstrokecolor{currentstroke}%
\pgfsetstrokeopacity{0.300000}%
\pgfsetdash{}{0pt}%
\pgfpathmoveto{\pgfqpoint{0.000000in}{0.000000in}}%
\pgfpathlineto{\pgfqpoint{0.000000in}{0.000000in}}%
\pgfpathclose%
\pgfusepath{stroke,fill}%
\end{pgfscope}%
\begin{pgfscope}%
\pgfpathrectangle{\pgfqpoint{0.647939in}{0.492442in}}{\pgfqpoint{4.273799in}{2.331163in}}%
\pgfusepath{clip}%
\pgfsetroundcap%
\pgfsetroundjoin%
\pgfsetlinewidth{0.301125pt}%
\definecolor{currentstroke}{rgb}{0.500000,0.500000,0.500000}%
\pgfsetstrokecolor{currentstroke}%
\pgfsetstrokeopacity{0.300000}%
\pgfsetdash{}{0pt}%
\pgfpathmoveto{\pgfqpoint{1.868247in}{2.194394in}}%
\pgfusepath{stroke}%
\end{pgfscope}%
\begin{pgfscope}%
\pgfpathrectangle{\pgfqpoint{0.647939in}{0.492442in}}{\pgfqpoint{4.273799in}{2.331163in}}%
\pgfusepath{clip}%
\pgfsetroundcap%
\pgfsetroundjoin%
\definecolor{currentfill}{rgb}{0.500000,0.500000,0.500000}%
\pgfsetfillcolor{currentfill}%
\pgfsetfillopacity{0.300000}%
\pgfsetlinewidth{0.301125pt}%
\definecolor{currentstroke}{rgb}{0.500000,0.500000,0.500000}%
\pgfsetstrokecolor{currentstroke}%
\pgfsetstrokeopacity{0.300000}%
\pgfsetdash{}{0pt}%
\pgfpathmoveto{\pgfqpoint{0.000000in}{0.000000in}}%
\pgfpathlineto{\pgfqpoint{0.000000in}{0.000000in}}%
\pgfpathclose%
\pgfusepath{stroke,fill}%
\end{pgfscope}%
\begin{pgfscope}%
\pgfpathrectangle{\pgfqpoint{0.647939in}{0.492442in}}{\pgfqpoint{4.273799in}{2.331163in}}%
\pgfusepath{clip}%
\pgfsetroundcap%
\pgfsetroundjoin%
\pgfsetlinewidth{0.301125pt}%
\definecolor{currentstroke}{rgb}{0.500000,0.500000,0.500000}%
\pgfsetstrokecolor{currentstroke}%
\pgfsetstrokeopacity{0.300000}%
\pgfsetdash{}{0pt}%
\pgfpathmoveto{\pgfqpoint{2.465374in}{0.894912in}}%
\pgfusepath{stroke}%
\end{pgfscope}%
\begin{pgfscope}%
\pgfpathrectangle{\pgfqpoint{0.647939in}{0.492442in}}{\pgfqpoint{4.273799in}{2.331163in}}%
\pgfusepath{clip}%
\pgfsetroundcap%
\pgfsetroundjoin%
\definecolor{currentfill}{rgb}{0.500000,0.500000,0.500000}%
\pgfsetfillcolor{currentfill}%
\pgfsetfillopacity{0.300000}%
\pgfsetlinewidth{0.301125pt}%
\definecolor{currentstroke}{rgb}{0.500000,0.500000,0.500000}%
\pgfsetstrokecolor{currentstroke}%
\pgfsetstrokeopacity{0.300000}%
\pgfsetdash{}{0pt}%
\pgfpathmoveto{\pgfqpoint{0.000000in}{0.000000in}}%
\pgfpathlineto{\pgfqpoint{0.000000in}{0.000000in}}%
\pgfpathclose%
\pgfusepath{stroke,fill}%
\end{pgfscope}%
\begin{pgfscope}%
\pgfpathrectangle{\pgfqpoint{0.647939in}{0.492442in}}{\pgfqpoint{4.273799in}{2.331163in}}%
\pgfusepath{clip}%
\pgfsetroundcap%
\pgfsetroundjoin%
\pgfsetlinewidth{0.301125pt}%
\definecolor{currentstroke}{rgb}{0.500000,0.500000,0.500000}%
\pgfsetstrokecolor{currentstroke}%
\pgfsetstrokeopacity{0.300000}%
\pgfsetdash{}{0pt}%
\pgfpathmoveto{\pgfqpoint{2.739179in}{0.653324in}}%
\pgfusepath{stroke}%
\end{pgfscope}%
\begin{pgfscope}%
\pgfpathrectangle{\pgfqpoint{0.647939in}{0.492442in}}{\pgfqpoint{4.273799in}{2.331163in}}%
\pgfusepath{clip}%
\pgfsetroundcap%
\pgfsetroundjoin%
\definecolor{currentfill}{rgb}{0.500000,0.500000,0.500000}%
\pgfsetfillcolor{currentfill}%
\pgfsetfillopacity{0.300000}%
\pgfsetlinewidth{0.301125pt}%
\definecolor{currentstroke}{rgb}{0.500000,0.500000,0.500000}%
\pgfsetstrokecolor{currentstroke}%
\pgfsetstrokeopacity{0.300000}%
\pgfsetdash{}{0pt}%
\pgfpathmoveto{\pgfqpoint{0.000000in}{0.000000in}}%
\pgfpathlineto{\pgfqpoint{0.000000in}{0.000000in}}%
\pgfpathclose%
\pgfusepath{stroke,fill}%
\end{pgfscope}%
\begin{pgfscope}%
\pgfpathrectangle{\pgfqpoint{0.647939in}{0.492442in}}{\pgfqpoint{4.273799in}{2.331163in}}%
\pgfusepath{clip}%
\pgfsetroundcap%
\pgfsetroundjoin%
\pgfsetlinewidth{0.301125pt}%
\definecolor{currentstroke}{rgb}{0.500000,0.500000,0.500000}%
\pgfsetstrokecolor{currentstroke}%
\pgfsetstrokeopacity{0.300000}%
\pgfsetdash{}{0pt}%
\pgfpathmoveto{\pgfqpoint{2.286895in}{1.544263in}}%
\pgfusepath{stroke}%
\end{pgfscope}%
\begin{pgfscope}%
\pgfpathrectangle{\pgfqpoint{0.647939in}{0.492442in}}{\pgfqpoint{4.273799in}{2.331163in}}%
\pgfusepath{clip}%
\pgfsetroundcap%
\pgfsetroundjoin%
\definecolor{currentfill}{rgb}{0.500000,0.500000,0.500000}%
\pgfsetfillcolor{currentfill}%
\pgfsetfillopacity{0.300000}%
\pgfsetlinewidth{0.301125pt}%
\definecolor{currentstroke}{rgb}{0.500000,0.500000,0.500000}%
\pgfsetstrokecolor{currentstroke}%
\pgfsetstrokeopacity{0.300000}%
\pgfsetdash{}{0pt}%
\pgfpathmoveto{\pgfqpoint{0.000000in}{0.000000in}}%
\pgfpathlineto{\pgfqpoint{0.000000in}{0.000000in}}%
\pgfpathclose%
\pgfusepath{stroke,fill}%
\end{pgfscope}%
\begin{pgfscope}%
\pgfpathrectangle{\pgfqpoint{0.647939in}{0.492442in}}{\pgfqpoint{4.273799in}{2.331163in}}%
\pgfusepath{clip}%
\pgfsetroundcap%
\pgfsetroundjoin%
\pgfsetlinewidth{0.301125pt}%
\definecolor{currentstroke}{rgb}{0.500000,0.500000,0.500000}%
\pgfsetstrokecolor{currentstroke}%
\pgfsetstrokeopacity{0.300000}%
\pgfsetdash{}{0pt}%
\pgfpathmoveto{\pgfqpoint{2.510304in}{1.352440in}}%
\pgfusepath{stroke}%
\end{pgfscope}%
\begin{pgfscope}%
\pgfpathrectangle{\pgfqpoint{0.647939in}{0.492442in}}{\pgfqpoint{4.273799in}{2.331163in}}%
\pgfusepath{clip}%
\pgfsetroundcap%
\pgfsetroundjoin%
\definecolor{currentfill}{rgb}{0.500000,0.500000,0.500000}%
\pgfsetfillcolor{currentfill}%
\pgfsetfillopacity{0.300000}%
\pgfsetlinewidth{0.301125pt}%
\definecolor{currentstroke}{rgb}{0.500000,0.500000,0.500000}%
\pgfsetstrokecolor{currentstroke}%
\pgfsetstrokeopacity{0.300000}%
\pgfsetdash{}{0pt}%
\pgfpathmoveto{\pgfqpoint{0.000000in}{0.000000in}}%
\pgfpathlineto{\pgfqpoint{0.000000in}{0.000000in}}%
\pgfpathclose%
\pgfusepath{stroke,fill}%
\end{pgfscope}%
\begin{pgfscope}%
\pgfpathrectangle{\pgfqpoint{0.647939in}{0.492442in}}{\pgfqpoint{4.273799in}{2.331163in}}%
\pgfusepath{clip}%
\pgfsetroundcap%
\pgfsetroundjoin%
\pgfsetlinewidth{0.301125pt}%
\definecolor{currentstroke}{rgb}{0.500000,0.500000,0.500000}%
\pgfsetstrokecolor{currentstroke}%
\pgfsetstrokeopacity{0.300000}%
\pgfsetdash{}{0pt}%
\pgfpathmoveto{\pgfqpoint{2.664247in}{1.281714in}}%
\pgfusepath{stroke}%
\end{pgfscope}%
\begin{pgfscope}%
\pgfpathrectangle{\pgfqpoint{0.647939in}{0.492442in}}{\pgfqpoint{4.273799in}{2.331163in}}%
\pgfusepath{clip}%
\pgfsetroundcap%
\pgfsetroundjoin%
\definecolor{currentfill}{rgb}{0.500000,0.500000,0.500000}%
\pgfsetfillcolor{currentfill}%
\pgfsetfillopacity{0.300000}%
\pgfsetlinewidth{0.301125pt}%
\definecolor{currentstroke}{rgb}{0.500000,0.500000,0.500000}%
\pgfsetstrokecolor{currentstroke}%
\pgfsetstrokeopacity{0.300000}%
\pgfsetdash{}{0pt}%
\pgfpathmoveto{\pgfqpoint{0.000000in}{0.000000in}}%
\pgfpathlineto{\pgfqpoint{0.000000in}{0.000000in}}%
\pgfpathclose%
\pgfusepath{stroke,fill}%
\end{pgfscope}%
\begin{pgfscope}%
\pgfpathrectangle{\pgfqpoint{0.647939in}{0.492442in}}{\pgfqpoint{4.273799in}{2.331163in}}%
\pgfusepath{clip}%
\pgfsetroundcap%
\pgfsetroundjoin%
\pgfsetlinewidth{0.301125pt}%
\definecolor{currentstroke}{rgb}{0.500000,0.500000,0.500000}%
\pgfsetstrokecolor{currentstroke}%
\pgfsetstrokeopacity{0.300000}%
\pgfsetdash{}{0pt}%
\pgfpathmoveto{\pgfqpoint{3.037027in}{0.976704in}}%
\pgfusepath{stroke}%
\end{pgfscope}%
\begin{pgfscope}%
\pgfpathrectangle{\pgfqpoint{0.647939in}{0.492442in}}{\pgfqpoint{4.273799in}{2.331163in}}%
\pgfusepath{clip}%
\pgfsetroundcap%
\pgfsetroundjoin%
\definecolor{currentfill}{rgb}{0.500000,0.500000,0.500000}%
\pgfsetfillcolor{currentfill}%
\pgfsetfillopacity{0.300000}%
\pgfsetlinewidth{0.301125pt}%
\definecolor{currentstroke}{rgb}{0.500000,0.500000,0.500000}%
\pgfsetstrokecolor{currentstroke}%
\pgfsetstrokeopacity{0.300000}%
\pgfsetdash{}{0pt}%
\pgfpathmoveto{\pgfqpoint{0.000000in}{0.000000in}}%
\pgfpathlineto{\pgfqpoint{0.000000in}{0.000000in}}%
\pgfpathclose%
\pgfusepath{stroke,fill}%
\end{pgfscope}%
\begin{pgfscope}%
\pgfpathrectangle{\pgfqpoint{0.647939in}{0.492442in}}{\pgfqpoint{4.273799in}{2.331163in}}%
\pgfusepath{clip}%
\pgfsetroundcap%
\pgfsetroundjoin%
\pgfsetlinewidth{0.301125pt}%
\definecolor{currentstroke}{rgb}{0.500000,0.500000,0.500000}%
\pgfsetstrokecolor{currentstroke}%
\pgfsetstrokeopacity{0.300000}%
\pgfsetdash{}{0pt}%
\pgfpathmoveto{\pgfqpoint{3.295790in}{0.872622in}}%
\pgfusepath{stroke}%
\end{pgfscope}%
\begin{pgfscope}%
\pgfpathrectangle{\pgfqpoint{0.647939in}{0.492442in}}{\pgfqpoint{4.273799in}{2.331163in}}%
\pgfusepath{clip}%
\pgfsetroundcap%
\pgfsetroundjoin%
\definecolor{currentfill}{rgb}{0.500000,0.500000,0.500000}%
\pgfsetfillcolor{currentfill}%
\pgfsetfillopacity{0.300000}%
\pgfsetlinewidth{0.301125pt}%
\definecolor{currentstroke}{rgb}{0.500000,0.500000,0.500000}%
\pgfsetstrokecolor{currentstroke}%
\pgfsetstrokeopacity{0.300000}%
\pgfsetdash{}{0pt}%
\pgfpathmoveto{\pgfqpoint{0.000000in}{0.000000in}}%
\pgfpathlineto{\pgfqpoint{0.000000in}{0.000000in}}%
\pgfpathclose%
\pgfusepath{stroke,fill}%
\end{pgfscope}%
\begin{pgfscope}%
\pgfpathrectangle{\pgfqpoint{0.647939in}{0.492442in}}{\pgfqpoint{4.273799in}{2.331163in}}%
\pgfusepath{clip}%
\pgfsetroundcap%
\pgfsetroundjoin%
\pgfsetlinewidth{0.301125pt}%
\definecolor{currentstroke}{rgb}{0.500000,0.500000,0.500000}%
\pgfsetstrokecolor{currentstroke}%
\pgfsetstrokeopacity{0.300000}%
\pgfsetdash{}{0pt}%
\pgfpathmoveto{\pgfqpoint{2.903541in}{1.343849in}}%
\pgfusepath{stroke}%
\end{pgfscope}%
\begin{pgfscope}%
\pgfpathrectangle{\pgfqpoint{0.647939in}{0.492442in}}{\pgfqpoint{4.273799in}{2.331163in}}%
\pgfusepath{clip}%
\pgfsetroundcap%
\pgfsetroundjoin%
\definecolor{currentfill}{rgb}{0.500000,0.500000,0.500000}%
\pgfsetfillcolor{currentfill}%
\pgfsetfillopacity{0.300000}%
\pgfsetlinewidth{0.301125pt}%
\definecolor{currentstroke}{rgb}{0.500000,0.500000,0.500000}%
\pgfsetstrokecolor{currentstroke}%
\pgfsetstrokeopacity{0.300000}%
\pgfsetdash{}{0pt}%
\pgfpathmoveto{\pgfqpoint{0.000000in}{0.000000in}}%
\pgfpathlineto{\pgfqpoint{0.000000in}{0.000000in}}%
\pgfpathclose%
\pgfusepath{stroke,fill}%
\end{pgfscope}%
\begin{pgfscope}%
\pgfpathrectangle{\pgfqpoint{0.647939in}{0.492442in}}{\pgfqpoint{4.273799in}{2.331163in}}%
\pgfusepath{clip}%
\pgfsetroundcap%
\pgfsetroundjoin%
\pgfsetlinewidth{0.301125pt}%
\definecolor{currentstroke}{rgb}{0.500000,0.500000,0.500000}%
\pgfsetstrokecolor{currentstroke}%
\pgfsetstrokeopacity{0.300000}%
\pgfsetdash{}{0pt}%
\pgfpathmoveto{\pgfqpoint{3.435492in}{1.025690in}}%
\pgfusepath{stroke}%
\end{pgfscope}%
\begin{pgfscope}%
\pgfpathrectangle{\pgfqpoint{0.647939in}{0.492442in}}{\pgfqpoint{4.273799in}{2.331163in}}%
\pgfusepath{clip}%
\pgfsetroundcap%
\pgfsetroundjoin%
\definecolor{currentfill}{rgb}{0.500000,0.500000,0.500000}%
\pgfsetfillcolor{currentfill}%
\pgfsetfillopacity{0.300000}%
\pgfsetlinewidth{0.301125pt}%
\definecolor{currentstroke}{rgb}{0.500000,0.500000,0.500000}%
\pgfsetstrokecolor{currentstroke}%
\pgfsetstrokeopacity{0.300000}%
\pgfsetdash{}{0pt}%
\pgfpathmoveto{\pgfqpoint{0.000000in}{0.000000in}}%
\pgfpathlineto{\pgfqpoint{0.000000in}{0.000000in}}%
\pgfpathclose%
\pgfusepath{stroke,fill}%
\end{pgfscope}%
\begin{pgfscope}%
\pgfpathrectangle{\pgfqpoint{0.647939in}{0.492442in}}{\pgfqpoint{4.273799in}{2.331163in}}%
\pgfusepath{clip}%
\pgfsetroundcap%
\pgfsetroundjoin%
\pgfsetlinewidth{0.301125pt}%
\definecolor{currentstroke}{rgb}{0.500000,0.500000,0.500000}%
\pgfsetstrokecolor{currentstroke}%
\pgfsetstrokeopacity{0.300000}%
\pgfsetdash{}{0pt}%
\pgfpathmoveto{\pgfqpoint{3.821474in}{0.885032in}}%
\pgfusepath{stroke}%
\end{pgfscope}%
\begin{pgfscope}%
\pgfpathrectangle{\pgfqpoint{0.647939in}{0.492442in}}{\pgfqpoint{4.273799in}{2.331163in}}%
\pgfusepath{clip}%
\pgfsetroundcap%
\pgfsetroundjoin%
\definecolor{currentfill}{rgb}{0.500000,0.500000,0.500000}%
\pgfsetfillcolor{currentfill}%
\pgfsetfillopacity{0.300000}%
\pgfsetlinewidth{0.301125pt}%
\definecolor{currentstroke}{rgb}{0.500000,0.500000,0.500000}%
\pgfsetstrokecolor{currentstroke}%
\pgfsetstrokeopacity{0.300000}%
\pgfsetdash{}{0pt}%
\pgfpathmoveto{\pgfqpoint{0.000000in}{0.000000in}}%
\pgfpathlineto{\pgfqpoint{0.000000in}{0.000000in}}%
\pgfpathclose%
\pgfusepath{stroke,fill}%
\end{pgfscope}%
\begin{pgfscope}%
\pgfpathrectangle{\pgfqpoint{0.647939in}{0.492442in}}{\pgfqpoint{4.273799in}{2.331163in}}%
\pgfusepath{clip}%
\pgfsetroundcap%
\pgfsetroundjoin%
\pgfsetlinewidth{0.301125pt}%
\definecolor{currentstroke}{rgb}{0.500000,0.500000,0.500000}%
\pgfsetstrokecolor{currentstroke}%
\pgfsetstrokeopacity{0.300000}%
\pgfsetdash{}{0pt}%
\pgfpathmoveto{\pgfqpoint{3.425454in}{1.272456in}}%
\pgfusepath{stroke}%
\end{pgfscope}%
\begin{pgfscope}%
\pgfpathrectangle{\pgfqpoint{0.647939in}{0.492442in}}{\pgfqpoint{4.273799in}{2.331163in}}%
\pgfusepath{clip}%
\pgfsetroundcap%
\pgfsetroundjoin%
\definecolor{currentfill}{rgb}{0.500000,0.500000,0.500000}%
\pgfsetfillcolor{currentfill}%
\pgfsetfillopacity{0.300000}%
\pgfsetlinewidth{0.301125pt}%
\definecolor{currentstroke}{rgb}{0.500000,0.500000,0.500000}%
\pgfsetstrokecolor{currentstroke}%
\pgfsetstrokeopacity{0.300000}%
\pgfsetdash{}{0pt}%
\pgfpathmoveto{\pgfqpoint{0.000000in}{0.000000in}}%
\pgfpathlineto{\pgfqpoint{0.000000in}{0.000000in}}%
\pgfpathclose%
\pgfusepath{stroke,fill}%
\end{pgfscope}%
\begin{pgfscope}%
\pgfpathrectangle{\pgfqpoint{0.647939in}{0.492442in}}{\pgfqpoint{4.273799in}{2.331163in}}%
\pgfusepath{clip}%
\pgfsetroundcap%
\pgfsetroundjoin%
\pgfsetlinewidth{0.301125pt}%
\definecolor{currentstroke}{rgb}{0.500000,0.500000,0.500000}%
\pgfsetstrokecolor{currentstroke}%
\pgfsetstrokeopacity{0.300000}%
\pgfsetdash{}{0pt}%
\pgfpathmoveto{\pgfqpoint{3.983951in}{1.139819in}}%
\pgfusepath{stroke}%
\end{pgfscope}%
\begin{pgfscope}%
\pgfpathrectangle{\pgfqpoint{0.647939in}{0.492442in}}{\pgfqpoint{4.273799in}{2.331163in}}%
\pgfusepath{clip}%
\pgfsetroundcap%
\pgfsetroundjoin%
\definecolor{currentfill}{rgb}{0.500000,0.500000,0.500000}%
\pgfsetfillcolor{currentfill}%
\pgfsetfillopacity{0.300000}%
\pgfsetlinewidth{0.301125pt}%
\definecolor{currentstroke}{rgb}{0.500000,0.500000,0.500000}%
\pgfsetstrokecolor{currentstroke}%
\pgfsetstrokeopacity{0.300000}%
\pgfsetdash{}{0pt}%
\pgfpathmoveto{\pgfqpoint{0.000000in}{0.000000in}}%
\pgfpathlineto{\pgfqpoint{0.000000in}{0.000000in}}%
\pgfpathclose%
\pgfusepath{stroke,fill}%
\end{pgfscope}%
\begin{pgfscope}%
\pgfpathrectangle{\pgfqpoint{0.647939in}{0.492442in}}{\pgfqpoint{4.273799in}{2.331163in}}%
\pgfusepath{clip}%
\pgfsetroundcap%
\pgfsetroundjoin%
\pgfsetlinewidth{0.301125pt}%
\definecolor{currentstroke}{rgb}{0.500000,0.500000,0.500000}%
\pgfsetstrokecolor{currentstroke}%
\pgfsetstrokeopacity{0.300000}%
\pgfsetdash{}{0pt}%
\pgfpathmoveto{\pgfqpoint{3.853191in}{1.345594in}}%
\pgfusepath{stroke}%
\end{pgfscope}%
\begin{pgfscope}%
\pgfpathrectangle{\pgfqpoint{0.647939in}{0.492442in}}{\pgfqpoint{4.273799in}{2.331163in}}%
\pgfusepath{clip}%
\pgfsetroundcap%
\pgfsetroundjoin%
\definecolor{currentfill}{rgb}{0.500000,0.500000,0.500000}%
\pgfsetfillcolor{currentfill}%
\pgfsetfillopacity{0.300000}%
\pgfsetlinewidth{0.301125pt}%
\definecolor{currentstroke}{rgb}{0.500000,0.500000,0.500000}%
\pgfsetstrokecolor{currentstroke}%
\pgfsetstrokeopacity{0.300000}%
\pgfsetdash{}{0pt}%
\pgfpathmoveto{\pgfqpoint{0.000000in}{0.000000in}}%
\pgfpathlineto{\pgfqpoint{0.000000in}{0.000000in}}%
\pgfpathclose%
\pgfusepath{stroke,fill}%
\end{pgfscope}%
\begin{pgfscope}%
\pgfpathrectangle{\pgfqpoint{0.647939in}{0.492442in}}{\pgfqpoint{4.273799in}{2.331163in}}%
\pgfusepath{clip}%
\pgfsetroundcap%
\pgfsetroundjoin%
\pgfsetlinewidth{0.301125pt}%
\definecolor{currentstroke}{rgb}{0.500000,0.500000,0.500000}%
\pgfsetstrokecolor{currentstroke}%
\pgfsetstrokeopacity{0.300000}%
\pgfsetdash{}{0pt}%
\pgfpathmoveto{\pgfqpoint{4.216716in}{1.381014in}}%
\pgfusepath{stroke}%
\end{pgfscope}%
\begin{pgfscope}%
\pgfpathrectangle{\pgfqpoint{0.647939in}{0.492442in}}{\pgfqpoint{4.273799in}{2.331163in}}%
\pgfusepath{clip}%
\pgfsetroundcap%
\pgfsetroundjoin%
\definecolor{currentfill}{rgb}{0.500000,0.500000,0.500000}%
\pgfsetfillcolor{currentfill}%
\pgfsetfillopacity{0.300000}%
\pgfsetlinewidth{0.301125pt}%
\definecolor{currentstroke}{rgb}{0.500000,0.500000,0.500000}%
\pgfsetstrokecolor{currentstroke}%
\pgfsetstrokeopacity{0.300000}%
\pgfsetdash{}{0pt}%
\pgfpathmoveto{\pgfqpoint{0.000000in}{0.000000in}}%
\pgfpathlineto{\pgfqpoint{0.000000in}{0.000000in}}%
\pgfpathclose%
\pgfusepath{stroke,fill}%
\end{pgfscope}%
\begin{pgfscope}%
\pgfpathrectangle{\pgfqpoint{0.647939in}{0.492442in}}{\pgfqpoint{4.273799in}{2.331163in}}%
\pgfusepath{clip}%
\pgfsetroundcap%
\pgfsetroundjoin%
\pgfsetlinewidth{0.301125pt}%
\definecolor{currentstroke}{rgb}{0.500000,0.500000,0.500000}%
\pgfsetstrokecolor{currentstroke}%
\pgfsetstrokeopacity{0.300000}%
\pgfsetdash{}{0pt}%
\pgfpathmoveto{\pgfqpoint{4.431176in}{1.464290in}}%
\pgfusepath{stroke}%
\end{pgfscope}%
\begin{pgfscope}%
\pgfpathrectangle{\pgfqpoint{0.647939in}{0.492442in}}{\pgfqpoint{4.273799in}{2.331163in}}%
\pgfusepath{clip}%
\pgfsetroundcap%
\pgfsetroundjoin%
\definecolor{currentfill}{rgb}{0.500000,0.500000,0.500000}%
\pgfsetfillcolor{currentfill}%
\pgfsetfillopacity{0.300000}%
\pgfsetlinewidth{0.301125pt}%
\definecolor{currentstroke}{rgb}{0.500000,0.500000,0.500000}%
\pgfsetstrokecolor{currentstroke}%
\pgfsetstrokeopacity{0.300000}%
\pgfsetdash{}{0pt}%
\pgfpathmoveto{\pgfqpoint{0.000000in}{0.000000in}}%
\pgfpathlineto{\pgfqpoint{0.000000in}{0.000000in}}%
\pgfpathclose%
\pgfusepath{stroke,fill}%
\end{pgfscope}%
\begin{pgfscope}%
\pgfpathrectangle{\pgfqpoint{0.647939in}{0.492442in}}{\pgfqpoint{4.273799in}{2.331163in}}%
\pgfusepath{clip}%
\pgfsetroundcap%
\pgfsetroundjoin%
\pgfsetlinewidth{0.301125pt}%
\definecolor{currentstroke}{rgb}{0.500000,0.500000,0.500000}%
\pgfsetstrokecolor{currentstroke}%
\pgfsetstrokeopacity{0.300000}%
\pgfsetdash{}{0pt}%
\pgfpathmoveto{\pgfqpoint{4.490254in}{1.766421in}}%
\pgfusepath{stroke}%
\end{pgfscope}%
\begin{pgfscope}%
\pgfpathrectangle{\pgfqpoint{0.647939in}{0.492442in}}{\pgfqpoint{4.273799in}{2.331163in}}%
\pgfusepath{clip}%
\pgfsetroundcap%
\pgfsetroundjoin%
\definecolor{currentfill}{rgb}{0.500000,0.500000,0.500000}%
\pgfsetfillcolor{currentfill}%
\pgfsetfillopacity{0.300000}%
\pgfsetlinewidth{0.301125pt}%
\definecolor{currentstroke}{rgb}{0.500000,0.500000,0.500000}%
\pgfsetstrokecolor{currentstroke}%
\pgfsetstrokeopacity{0.300000}%
\pgfsetdash{}{0pt}%
\pgfpathmoveto{\pgfqpoint{0.000000in}{0.000000in}}%
\pgfpathlineto{\pgfqpoint{0.000000in}{0.000000in}}%
\pgfpathclose%
\pgfusepath{stroke,fill}%
\end{pgfscope}%
\begin{pgfscope}%
\pgfpathrectangle{\pgfqpoint{0.647939in}{0.492442in}}{\pgfqpoint{4.273799in}{2.331163in}}%
\pgfusepath{clip}%
\pgfsetroundcap%
\pgfsetroundjoin%
\pgfsetlinewidth{0.301125pt}%
\definecolor{currentstroke}{rgb}{0.500000,0.500000,0.500000}%
\pgfsetstrokecolor{currentstroke}%
\pgfsetstrokeopacity{0.300000}%
\pgfsetdash{}{0pt}%
\pgfpathmoveto{\pgfqpoint{4.691421in}{1.735973in}}%
\pgfusepath{stroke}%
\end{pgfscope}%
\begin{pgfscope}%
\pgfpathrectangle{\pgfqpoint{0.647939in}{0.492442in}}{\pgfqpoint{4.273799in}{2.331163in}}%
\pgfusepath{clip}%
\pgfsetroundcap%
\pgfsetroundjoin%
\definecolor{currentfill}{rgb}{0.500000,0.500000,0.500000}%
\pgfsetfillcolor{currentfill}%
\pgfsetfillopacity{0.300000}%
\pgfsetlinewidth{0.301125pt}%
\definecolor{currentstroke}{rgb}{0.500000,0.500000,0.500000}%
\pgfsetstrokecolor{currentstroke}%
\pgfsetstrokeopacity{0.300000}%
\pgfsetdash{}{0pt}%
\pgfpathmoveto{\pgfqpoint{0.000000in}{0.000000in}}%
\pgfpathlineto{\pgfqpoint{0.000000in}{0.000000in}}%
\pgfpathclose%
\pgfusepath{stroke,fill}%
\end{pgfscope}%
\begin{pgfscope}%
\pgfpathrectangle{\pgfqpoint{0.647939in}{0.492442in}}{\pgfqpoint{4.273799in}{2.331163in}}%
\pgfusepath{clip}%
\pgfsetroundcap%
\pgfsetroundjoin%
\pgfsetlinewidth{0.301125pt}%
\definecolor{currentstroke}{rgb}{0.500000,0.500000,0.500000}%
\pgfsetstrokecolor{currentstroke}%
\pgfsetstrokeopacity{0.300000}%
\pgfsetdash{}{0pt}%
\pgfpathmoveto{\pgfqpoint{4.797001in}{1.799800in}}%
\pgfusepath{stroke}%
\end{pgfscope}%
\begin{pgfscope}%
\pgfpathrectangle{\pgfqpoint{0.647939in}{0.492442in}}{\pgfqpoint{4.273799in}{2.331163in}}%
\pgfusepath{clip}%
\pgfsetroundcap%
\pgfsetroundjoin%
\definecolor{currentfill}{rgb}{0.500000,0.500000,0.500000}%
\pgfsetfillcolor{currentfill}%
\pgfsetfillopacity{0.300000}%
\pgfsetlinewidth{0.301125pt}%
\definecolor{currentstroke}{rgb}{0.500000,0.500000,0.500000}%
\pgfsetstrokecolor{currentstroke}%
\pgfsetstrokeopacity{0.300000}%
\pgfsetdash{}{0pt}%
\pgfpathmoveto{\pgfqpoint{0.000000in}{0.000000in}}%
\pgfpathlineto{\pgfqpoint{0.000000in}{0.000000in}}%
\pgfpathclose%
\pgfusepath{stroke,fill}%
\end{pgfscope}%
\begin{pgfscope}%
\pgfpathrectangle{\pgfqpoint{0.647939in}{0.492442in}}{\pgfqpoint{4.273799in}{2.331163in}}%
\pgfusepath{clip}%
\pgfsetroundcap%
\pgfsetroundjoin%
\pgfsetlinewidth{0.301125pt}%
\definecolor{currentstroke}{rgb}{0.500000,0.500000,0.500000}%
\pgfsetstrokecolor{currentstroke}%
\pgfsetstrokeopacity{0.300000}%
\pgfsetdash{}{0pt}%
\pgfpathmoveto{\pgfqpoint{4.900417in}{1.549050in}}%
\pgfusepath{stroke}%
\end{pgfscope}%
\begin{pgfscope}%
\pgfpathrectangle{\pgfqpoint{0.647939in}{0.492442in}}{\pgfqpoint{4.273799in}{2.331163in}}%
\pgfusepath{clip}%
\pgfsetroundcap%
\pgfsetroundjoin%
\definecolor{currentfill}{rgb}{0.500000,0.500000,0.500000}%
\pgfsetfillcolor{currentfill}%
\pgfsetfillopacity{0.300000}%
\pgfsetlinewidth{0.301125pt}%
\definecolor{currentstroke}{rgb}{0.500000,0.500000,0.500000}%
\pgfsetstrokecolor{currentstroke}%
\pgfsetstrokeopacity{0.300000}%
\pgfsetdash{}{0pt}%
\pgfpathmoveto{\pgfqpoint{0.000000in}{0.000000in}}%
\pgfpathlineto{\pgfqpoint{0.000000in}{0.000000in}}%
\pgfpathclose%
\pgfusepath{stroke,fill}%
\end{pgfscope}%
\begin{pgfscope}%
\pgfpathrectangle{\pgfqpoint{0.647939in}{0.492442in}}{\pgfqpoint{4.273799in}{2.331163in}}%
\pgfusepath{clip}%
\pgfsetroundcap%
\pgfsetroundjoin%
\pgfsetlinewidth{0.301125pt}%
\definecolor{currentstroke}{rgb}{0.500000,0.500000,0.500000}%
\pgfsetstrokecolor{currentstroke}%
\pgfsetstrokeopacity{0.300000}%
\pgfsetdash{}{0pt}%
\pgfpathmoveto{\pgfqpoint{4.905212in}{2.022806in}}%
\pgfusepath{stroke}%
\end{pgfscope}%
\begin{pgfscope}%
\pgfpathrectangle{\pgfqpoint{0.647939in}{0.492442in}}{\pgfqpoint{4.273799in}{2.331163in}}%
\pgfusepath{clip}%
\pgfsetroundcap%
\pgfsetroundjoin%
\definecolor{currentfill}{rgb}{0.500000,0.500000,0.500000}%
\pgfsetfillcolor{currentfill}%
\pgfsetfillopacity{0.300000}%
\pgfsetlinewidth{0.301125pt}%
\definecolor{currentstroke}{rgb}{0.500000,0.500000,0.500000}%
\pgfsetstrokecolor{currentstroke}%
\pgfsetstrokeopacity{0.300000}%
\pgfsetdash{}{0pt}%
\pgfpathmoveto{\pgfqpoint{0.000000in}{0.000000in}}%
\pgfpathlineto{\pgfqpoint{0.000000in}{0.000000in}}%
\pgfpathclose%
\pgfusepath{stroke,fill}%
\end{pgfscope}%
\begin{pgfscope}%
\pgfpathrectangle{\pgfqpoint{0.647939in}{0.492442in}}{\pgfqpoint{4.273799in}{2.331163in}}%
\pgfusepath{clip}%
\pgfsetroundcap%
\pgfsetroundjoin%
\pgfsetlinewidth{0.301125pt}%
\definecolor{currentstroke}{rgb}{0.500000,0.500000,0.500000}%
\pgfsetstrokecolor{currentstroke}%
\pgfsetstrokeopacity{0.300000}%
\pgfsetdash{}{0pt}%
\pgfpathmoveto{\pgfqpoint{4.551457in}{2.704151in}}%
\pgfusepath{stroke}%
\end{pgfscope}%
\begin{pgfscope}%
\pgfpathrectangle{\pgfqpoint{0.647939in}{0.492442in}}{\pgfqpoint{4.273799in}{2.331163in}}%
\pgfusepath{clip}%
\pgfsetroundcap%
\pgfsetroundjoin%
\definecolor{currentfill}{rgb}{0.500000,0.500000,0.500000}%
\pgfsetfillcolor{currentfill}%
\pgfsetfillopacity{0.300000}%
\pgfsetlinewidth{0.301125pt}%
\definecolor{currentstroke}{rgb}{0.500000,0.500000,0.500000}%
\pgfsetstrokecolor{currentstroke}%
\pgfsetstrokeopacity{0.300000}%
\pgfsetdash{}{0pt}%
\pgfpathmoveto{\pgfqpoint{0.000000in}{0.000000in}}%
\pgfpathlineto{\pgfqpoint{0.000000in}{0.000000in}}%
\pgfpathclose%
\pgfusepath{stroke,fill}%
\end{pgfscope}%
\begin{pgfscope}%
\pgfpathrectangle{\pgfqpoint{0.647939in}{0.492442in}}{\pgfqpoint{4.273799in}{2.331163in}}%
\pgfusepath{clip}%
\pgfsetroundcap%
\pgfsetroundjoin%
\pgfsetlinewidth{0.301125pt}%
\definecolor{currentstroke}{rgb}{0.500000,0.500000,0.500000}%
\pgfsetstrokecolor{currentstroke}%
\pgfsetstrokeopacity{0.300000}%
\pgfsetdash{}{0pt}%
\pgfpathmoveto{\pgfqpoint{4.436828in}{2.647084in}}%
\pgfusepath{stroke}%
\end{pgfscope}%
\begin{pgfscope}%
\pgfpathrectangle{\pgfqpoint{0.647939in}{0.492442in}}{\pgfqpoint{4.273799in}{2.331163in}}%
\pgfusepath{clip}%
\pgfsetroundcap%
\pgfsetroundjoin%
\definecolor{currentfill}{rgb}{0.500000,0.500000,0.500000}%
\pgfsetfillcolor{currentfill}%
\pgfsetfillopacity{0.300000}%
\pgfsetlinewidth{0.301125pt}%
\definecolor{currentstroke}{rgb}{0.500000,0.500000,0.500000}%
\pgfsetstrokecolor{currentstroke}%
\pgfsetstrokeopacity{0.300000}%
\pgfsetdash{}{0pt}%
\pgfpathmoveto{\pgfqpoint{0.000000in}{0.000000in}}%
\pgfpathlineto{\pgfqpoint{0.000000in}{0.000000in}}%
\pgfpathclose%
\pgfusepath{stroke,fill}%
\end{pgfscope}%
\begin{pgfscope}%
\pgfpathrectangle{\pgfqpoint{0.647939in}{0.492442in}}{\pgfqpoint{4.273799in}{2.331163in}}%
\pgfusepath{clip}%
\pgfsetroundcap%
\pgfsetroundjoin%
\pgfsetlinewidth{0.301125pt}%
\definecolor{currentstroke}{rgb}{0.500000,0.500000,0.500000}%
\pgfsetstrokecolor{currentstroke}%
\pgfsetstrokeopacity{0.300000}%
\pgfsetdash{}{0pt}%
\pgfpathmoveto{\pgfqpoint{4.367382in}{2.572308in}}%
\pgfusepath{stroke}%
\end{pgfscope}%
\begin{pgfscope}%
\pgfpathrectangle{\pgfqpoint{0.647939in}{0.492442in}}{\pgfqpoint{4.273799in}{2.331163in}}%
\pgfusepath{clip}%
\pgfsetroundcap%
\pgfsetroundjoin%
\definecolor{currentfill}{rgb}{0.500000,0.500000,0.500000}%
\pgfsetfillcolor{currentfill}%
\pgfsetfillopacity{0.300000}%
\pgfsetlinewidth{0.301125pt}%
\definecolor{currentstroke}{rgb}{0.500000,0.500000,0.500000}%
\pgfsetstrokecolor{currentstroke}%
\pgfsetstrokeopacity{0.300000}%
\pgfsetdash{}{0pt}%
\pgfpathmoveto{\pgfqpoint{0.000000in}{0.000000in}}%
\pgfpathlineto{\pgfqpoint{0.000000in}{0.000000in}}%
\pgfpathclose%
\pgfusepath{stroke,fill}%
\end{pgfscope}%
\begin{pgfscope}%
\pgfpathrectangle{\pgfqpoint{0.647939in}{0.492442in}}{\pgfqpoint{4.273799in}{2.331163in}}%
\pgfusepath{clip}%
\pgfsetroundcap%
\pgfsetroundjoin%
\pgfsetlinewidth{0.301125pt}%
\definecolor{currentstroke}{rgb}{0.500000,0.500000,0.500000}%
\pgfsetstrokecolor{currentstroke}%
\pgfsetstrokeopacity{0.300000}%
\pgfsetdash{}{0pt}%
\pgfpathmoveto{\pgfqpoint{4.273153in}{2.513176in}}%
\pgfusepath{stroke}%
\end{pgfscope}%
\begin{pgfscope}%
\pgfpathrectangle{\pgfqpoint{0.647939in}{0.492442in}}{\pgfqpoint{4.273799in}{2.331163in}}%
\pgfusepath{clip}%
\pgfsetroundcap%
\pgfsetroundjoin%
\definecolor{currentfill}{rgb}{0.500000,0.500000,0.500000}%
\pgfsetfillcolor{currentfill}%
\pgfsetfillopacity{0.300000}%
\pgfsetlinewidth{0.301125pt}%
\definecolor{currentstroke}{rgb}{0.500000,0.500000,0.500000}%
\pgfsetstrokecolor{currentstroke}%
\pgfsetstrokeopacity{0.300000}%
\pgfsetdash{}{0pt}%
\pgfpathmoveto{\pgfqpoint{0.000000in}{0.000000in}}%
\pgfpathlineto{\pgfqpoint{0.000000in}{0.000000in}}%
\pgfpathclose%
\pgfusepath{stroke,fill}%
\end{pgfscope}%
\begin{pgfscope}%
\pgfpathrectangle{\pgfqpoint{0.647939in}{0.492442in}}{\pgfqpoint{4.273799in}{2.331163in}}%
\pgfusepath{clip}%
\pgfsetroundcap%
\pgfsetroundjoin%
\pgfsetlinewidth{0.301125pt}%
\definecolor{currentstroke}{rgb}{0.500000,0.500000,0.500000}%
\pgfsetstrokecolor{currentstroke}%
\pgfsetstrokeopacity{0.300000}%
\pgfsetdash{}{0pt}%
\pgfpathmoveto{\pgfqpoint{4.181833in}{2.457425in}}%
\pgfusepath{stroke}%
\end{pgfscope}%
\begin{pgfscope}%
\pgfpathrectangle{\pgfqpoint{0.647939in}{0.492442in}}{\pgfqpoint{4.273799in}{2.331163in}}%
\pgfusepath{clip}%
\pgfsetroundcap%
\pgfsetroundjoin%
\definecolor{currentfill}{rgb}{0.500000,0.500000,0.500000}%
\pgfsetfillcolor{currentfill}%
\pgfsetfillopacity{0.300000}%
\pgfsetlinewidth{0.301125pt}%
\definecolor{currentstroke}{rgb}{0.500000,0.500000,0.500000}%
\pgfsetstrokecolor{currentstroke}%
\pgfsetstrokeopacity{0.300000}%
\pgfsetdash{}{0pt}%
\pgfpathmoveto{\pgfqpoint{0.000000in}{0.000000in}}%
\pgfpathlineto{\pgfqpoint{0.000000in}{0.000000in}}%
\pgfpathclose%
\pgfusepath{stroke,fill}%
\end{pgfscope}%
\begin{pgfscope}%
\pgfpathrectangle{\pgfqpoint{0.647939in}{0.492442in}}{\pgfqpoint{4.273799in}{2.331163in}}%
\pgfusepath{clip}%
\pgfsetroundcap%
\pgfsetroundjoin%
\pgfsetlinewidth{0.301125pt}%
\definecolor{currentstroke}{rgb}{0.500000,0.500000,0.500000}%
\pgfsetstrokecolor{currentstroke}%
\pgfsetstrokeopacity{0.300000}%
\pgfsetdash{}{0pt}%
\pgfpathmoveto{\pgfqpoint{4.180051in}{2.228230in}}%
\pgfusepath{stroke}%
\end{pgfscope}%
\begin{pgfscope}%
\pgfpathrectangle{\pgfqpoint{0.647939in}{0.492442in}}{\pgfqpoint{4.273799in}{2.331163in}}%
\pgfusepath{clip}%
\pgfsetroundcap%
\pgfsetroundjoin%
\definecolor{currentfill}{rgb}{0.500000,0.500000,0.500000}%
\pgfsetfillcolor{currentfill}%
\pgfsetfillopacity{0.300000}%
\pgfsetlinewidth{0.301125pt}%
\definecolor{currentstroke}{rgb}{0.500000,0.500000,0.500000}%
\pgfsetstrokecolor{currentstroke}%
\pgfsetstrokeopacity{0.300000}%
\pgfsetdash{}{0pt}%
\pgfpathmoveto{\pgfqpoint{0.000000in}{0.000000in}}%
\pgfpathlineto{\pgfqpoint{0.000000in}{0.000000in}}%
\pgfpathclose%
\pgfusepath{stroke,fill}%
\end{pgfscope}%
\begin{pgfscope}%
\pgfpathrectangle{\pgfqpoint{0.647939in}{0.492442in}}{\pgfqpoint{4.273799in}{2.331163in}}%
\pgfusepath{clip}%
\pgfsetroundcap%
\pgfsetroundjoin%
\pgfsetlinewidth{0.301125pt}%
\definecolor{currentstroke}{rgb}{0.500000,0.500000,0.500000}%
\pgfsetstrokecolor{currentstroke}%
\pgfsetstrokeopacity{0.300000}%
\pgfsetdash{}{0pt}%
\pgfpathmoveto{\pgfqpoint{3.996307in}{2.376288in}}%
\pgfusepath{stroke}%
\end{pgfscope}%
\begin{pgfscope}%
\pgfpathrectangle{\pgfqpoint{0.647939in}{0.492442in}}{\pgfqpoint{4.273799in}{2.331163in}}%
\pgfusepath{clip}%
\pgfsetroundcap%
\pgfsetroundjoin%
\definecolor{currentfill}{rgb}{0.500000,0.500000,0.500000}%
\pgfsetfillcolor{currentfill}%
\pgfsetfillopacity{0.300000}%
\pgfsetlinewidth{0.301125pt}%
\definecolor{currentstroke}{rgb}{0.500000,0.500000,0.500000}%
\pgfsetstrokecolor{currentstroke}%
\pgfsetstrokeopacity{0.300000}%
\pgfsetdash{}{0pt}%
\pgfpathmoveto{\pgfqpoint{0.000000in}{0.000000in}}%
\pgfpathlineto{\pgfqpoint{0.000000in}{0.000000in}}%
\pgfpathclose%
\pgfusepath{stroke,fill}%
\end{pgfscope}%
\begin{pgfscope}%
\pgfpathrectangle{\pgfqpoint{0.647939in}{0.492442in}}{\pgfqpoint{4.273799in}{2.331163in}}%
\pgfusepath{clip}%
\pgfsetroundcap%
\pgfsetroundjoin%
\pgfsetlinewidth{0.301125pt}%
\definecolor{currentstroke}{rgb}{0.500000,0.500000,0.500000}%
\pgfsetstrokecolor{currentstroke}%
\pgfsetstrokeopacity{0.300000}%
\pgfsetdash{}{0pt}%
\pgfpathmoveto{\pgfqpoint{3.915494in}{1.842131in}}%
\pgfusepath{stroke}%
\end{pgfscope}%
\begin{pgfscope}%
\pgfpathrectangle{\pgfqpoint{0.647939in}{0.492442in}}{\pgfqpoint{4.273799in}{2.331163in}}%
\pgfusepath{clip}%
\pgfsetroundcap%
\pgfsetroundjoin%
\definecolor{currentfill}{rgb}{0.500000,0.500000,0.500000}%
\pgfsetfillcolor{currentfill}%
\pgfsetfillopacity{0.300000}%
\pgfsetlinewidth{0.301125pt}%
\definecolor{currentstroke}{rgb}{0.500000,0.500000,0.500000}%
\pgfsetstrokecolor{currentstroke}%
\pgfsetstrokeopacity{0.300000}%
\pgfsetdash{}{0pt}%
\pgfpathmoveto{\pgfqpoint{0.000000in}{0.000000in}}%
\pgfpathlineto{\pgfqpoint{0.000000in}{0.000000in}}%
\pgfpathclose%
\pgfusepath{stroke,fill}%
\end{pgfscope}%
\begin{pgfscope}%
\pgfpathrectangle{\pgfqpoint{0.647939in}{0.492442in}}{\pgfqpoint{4.273799in}{2.331163in}}%
\pgfusepath{clip}%
\pgfsetroundcap%
\pgfsetroundjoin%
\pgfsetlinewidth{0.301125pt}%
\definecolor{currentstroke}{rgb}{0.500000,0.500000,0.500000}%
\pgfsetstrokecolor{currentstroke}%
\pgfsetstrokeopacity{0.300000}%
\pgfsetdash{}{0pt}%
\pgfpathmoveto{\pgfqpoint{3.835111in}{2.274180in}}%
\pgfusepath{stroke}%
\end{pgfscope}%
\begin{pgfscope}%
\pgfpathrectangle{\pgfqpoint{0.647939in}{0.492442in}}{\pgfqpoint{4.273799in}{2.331163in}}%
\pgfusepath{clip}%
\pgfsetroundcap%
\pgfsetroundjoin%
\definecolor{currentfill}{rgb}{0.500000,0.500000,0.500000}%
\pgfsetfillcolor{currentfill}%
\pgfsetfillopacity{0.300000}%
\pgfsetlinewidth{0.301125pt}%
\definecolor{currentstroke}{rgb}{0.500000,0.500000,0.500000}%
\pgfsetstrokecolor{currentstroke}%
\pgfsetstrokeopacity{0.300000}%
\pgfsetdash{}{0pt}%
\pgfpathmoveto{\pgfqpoint{0.000000in}{0.000000in}}%
\pgfpathlineto{\pgfqpoint{0.000000in}{0.000000in}}%
\pgfpathclose%
\pgfusepath{stroke,fill}%
\end{pgfscope}%
\begin{pgfscope}%
\pgfpathrectangle{\pgfqpoint{0.647939in}{0.492442in}}{\pgfqpoint{4.273799in}{2.331163in}}%
\pgfusepath{clip}%
\pgfsetroundcap%
\pgfsetroundjoin%
\pgfsetlinewidth{0.301125pt}%
\definecolor{currentstroke}{rgb}{0.500000,0.500000,0.500000}%
\pgfsetstrokecolor{currentstroke}%
\pgfsetstrokeopacity{0.300000}%
\pgfsetdash{}{0pt}%
\pgfpathmoveto{\pgfqpoint{3.684667in}{2.174387in}}%
\pgfusepath{stroke}%
\end{pgfscope}%
\begin{pgfscope}%
\pgfpathrectangle{\pgfqpoint{0.647939in}{0.492442in}}{\pgfqpoint{4.273799in}{2.331163in}}%
\pgfusepath{clip}%
\pgfsetroundcap%
\pgfsetroundjoin%
\definecolor{currentfill}{rgb}{0.500000,0.500000,0.500000}%
\pgfsetfillcolor{currentfill}%
\pgfsetfillopacity{0.300000}%
\pgfsetlinewidth{0.301125pt}%
\definecolor{currentstroke}{rgb}{0.500000,0.500000,0.500000}%
\pgfsetstrokecolor{currentstroke}%
\pgfsetstrokeopacity{0.300000}%
\pgfsetdash{}{0pt}%
\pgfpathmoveto{\pgfqpoint{0.000000in}{0.000000in}}%
\pgfpathlineto{\pgfqpoint{0.000000in}{0.000000in}}%
\pgfpathclose%
\pgfusepath{stroke,fill}%
\end{pgfscope}%
\begin{pgfscope}%
\pgfpathrectangle{\pgfqpoint{0.647939in}{0.492442in}}{\pgfqpoint{4.273799in}{2.331163in}}%
\pgfusepath{clip}%
\pgfsetroundcap%
\pgfsetroundjoin%
\pgfsetlinewidth{0.301125pt}%
\definecolor{currentstroke}{rgb}{0.500000,0.500000,0.500000}%
\pgfsetstrokecolor{currentstroke}%
\pgfsetstrokeopacity{0.300000}%
\pgfsetdash{}{0pt}%
\pgfpathmoveto{\pgfqpoint{3.474076in}{2.506304in}}%
\pgfusepath{stroke}%
\end{pgfscope}%
\begin{pgfscope}%
\pgfpathrectangle{\pgfqpoint{0.647939in}{0.492442in}}{\pgfqpoint{4.273799in}{2.331163in}}%
\pgfusepath{clip}%
\pgfsetroundcap%
\pgfsetroundjoin%
\definecolor{currentfill}{rgb}{0.500000,0.500000,0.500000}%
\pgfsetfillcolor{currentfill}%
\pgfsetfillopacity{0.300000}%
\pgfsetlinewidth{0.301125pt}%
\definecolor{currentstroke}{rgb}{0.500000,0.500000,0.500000}%
\pgfsetstrokecolor{currentstroke}%
\pgfsetstrokeopacity{0.300000}%
\pgfsetdash{}{0pt}%
\pgfpathmoveto{\pgfqpoint{0.000000in}{0.000000in}}%
\pgfpathlineto{\pgfqpoint{0.000000in}{0.000000in}}%
\pgfpathclose%
\pgfusepath{stroke,fill}%
\end{pgfscope}%
\begin{pgfscope}%
\pgfpathrectangle{\pgfqpoint{0.647939in}{0.492442in}}{\pgfqpoint{4.273799in}{2.331163in}}%
\pgfusepath{clip}%
\pgfsetroundcap%
\pgfsetroundjoin%
\pgfsetlinewidth{0.301125pt}%
\definecolor{currentstroke}{rgb}{0.500000,0.500000,0.500000}%
\pgfsetstrokecolor{currentstroke}%
\pgfsetstrokeopacity{0.300000}%
\pgfsetdash{}{0pt}%
\pgfpathmoveto{\pgfqpoint{3.298215in}{2.657169in}}%
\pgfusepath{stroke}%
\end{pgfscope}%
\begin{pgfscope}%
\pgfpathrectangle{\pgfqpoint{0.647939in}{0.492442in}}{\pgfqpoint{4.273799in}{2.331163in}}%
\pgfusepath{clip}%
\pgfsetroundcap%
\pgfsetroundjoin%
\definecolor{currentfill}{rgb}{0.500000,0.500000,0.500000}%
\pgfsetfillcolor{currentfill}%
\pgfsetfillopacity{0.300000}%
\pgfsetlinewidth{0.301125pt}%
\definecolor{currentstroke}{rgb}{0.500000,0.500000,0.500000}%
\pgfsetstrokecolor{currentstroke}%
\pgfsetstrokeopacity{0.300000}%
\pgfsetdash{}{0pt}%
\pgfpathmoveto{\pgfqpoint{0.000000in}{0.000000in}}%
\pgfpathlineto{\pgfqpoint{0.000000in}{0.000000in}}%
\pgfpathclose%
\pgfusepath{stroke,fill}%
\end{pgfscope}%
\begin{pgfscope}%
\pgfpathrectangle{\pgfqpoint{0.647939in}{0.492442in}}{\pgfqpoint{4.273799in}{2.331163in}}%
\pgfusepath{clip}%
\pgfsetroundcap%
\pgfsetroundjoin%
\pgfsetlinewidth{0.301125pt}%
\definecolor{currentstroke}{rgb}{0.500000,0.500000,0.500000}%
\pgfsetstrokecolor{currentstroke}%
\pgfsetstrokeopacity{0.300000}%
\pgfsetdash{}{0pt}%
\pgfpathmoveto{\pgfqpoint{3.360563in}{2.295005in}}%
\pgfusepath{stroke}%
\end{pgfscope}%
\begin{pgfscope}%
\pgfpathrectangle{\pgfqpoint{0.647939in}{0.492442in}}{\pgfqpoint{4.273799in}{2.331163in}}%
\pgfusepath{clip}%
\pgfsetroundcap%
\pgfsetroundjoin%
\definecolor{currentfill}{rgb}{0.500000,0.500000,0.500000}%
\pgfsetfillcolor{currentfill}%
\pgfsetfillopacity{0.300000}%
\pgfsetlinewidth{0.301125pt}%
\definecolor{currentstroke}{rgb}{0.500000,0.500000,0.500000}%
\pgfsetstrokecolor{currentstroke}%
\pgfsetstrokeopacity{0.300000}%
\pgfsetdash{}{0pt}%
\pgfpathmoveto{\pgfqpoint{0.000000in}{0.000000in}}%
\pgfpathlineto{\pgfqpoint{0.000000in}{0.000000in}}%
\pgfpathclose%
\pgfusepath{stroke,fill}%
\end{pgfscope}%
\begin{pgfscope}%
\pgfpathrectangle{\pgfqpoint{0.647939in}{0.492442in}}{\pgfqpoint{4.273799in}{2.331163in}}%
\pgfusepath{clip}%
\pgfsetroundcap%
\pgfsetroundjoin%
\pgfsetlinewidth{0.301125pt}%
\definecolor{currentstroke}{rgb}{0.500000,0.500000,0.500000}%
\pgfsetstrokecolor{currentstroke}%
\pgfsetstrokeopacity{0.300000}%
\pgfsetdash{}{0pt}%
\pgfpathmoveto{\pgfqpoint{3.093685in}{2.535682in}}%
\pgfusepath{stroke}%
\end{pgfscope}%
\begin{pgfscope}%
\pgfpathrectangle{\pgfqpoint{0.647939in}{0.492442in}}{\pgfqpoint{4.273799in}{2.331163in}}%
\pgfusepath{clip}%
\pgfsetroundcap%
\pgfsetroundjoin%
\definecolor{currentfill}{rgb}{0.500000,0.500000,0.500000}%
\pgfsetfillcolor{currentfill}%
\pgfsetfillopacity{0.300000}%
\pgfsetlinewidth{0.301125pt}%
\definecolor{currentstroke}{rgb}{0.500000,0.500000,0.500000}%
\pgfsetstrokecolor{currentstroke}%
\pgfsetstrokeopacity{0.300000}%
\pgfsetdash{}{0pt}%
\pgfpathmoveto{\pgfqpoint{0.000000in}{0.000000in}}%
\pgfpathlineto{\pgfqpoint{0.000000in}{0.000000in}}%
\pgfpathclose%
\pgfusepath{stroke,fill}%
\end{pgfscope}%
\begin{pgfscope}%
\pgfpathrectangle{\pgfqpoint{0.647939in}{0.492442in}}{\pgfqpoint{4.273799in}{2.331163in}}%
\pgfusepath{clip}%
\pgfsetroundcap%
\pgfsetroundjoin%
\pgfsetlinewidth{0.301125pt}%
\definecolor{currentstroke}{rgb}{0.500000,0.500000,0.500000}%
\pgfsetstrokecolor{currentstroke}%
\pgfsetstrokeopacity{0.300000}%
\pgfsetdash{}{0pt}%
\pgfpathmoveto{\pgfqpoint{2.804511in}{2.692485in}}%
\pgfusepath{stroke}%
\end{pgfscope}%
\begin{pgfscope}%
\pgfpathrectangle{\pgfqpoint{0.647939in}{0.492442in}}{\pgfqpoint{4.273799in}{2.331163in}}%
\pgfusepath{clip}%
\pgfsetroundcap%
\pgfsetroundjoin%
\definecolor{currentfill}{rgb}{0.500000,0.500000,0.500000}%
\pgfsetfillcolor{currentfill}%
\pgfsetfillopacity{0.300000}%
\pgfsetlinewidth{0.301125pt}%
\definecolor{currentstroke}{rgb}{0.500000,0.500000,0.500000}%
\pgfsetstrokecolor{currentstroke}%
\pgfsetstrokeopacity{0.300000}%
\pgfsetdash{}{0pt}%
\pgfpathmoveto{\pgfqpoint{0.000000in}{0.000000in}}%
\pgfpathlineto{\pgfqpoint{0.000000in}{0.000000in}}%
\pgfpathclose%
\pgfusepath{stroke,fill}%
\end{pgfscope}%
\begin{pgfscope}%
\pgfpathrectangle{\pgfqpoint{0.647939in}{0.492442in}}{\pgfqpoint{4.273799in}{2.331163in}}%
\pgfusepath{clip}%
\pgfsetroundcap%
\pgfsetroundjoin%
\pgfsetlinewidth{0.301125pt}%
\definecolor{currentstroke}{rgb}{0.500000,0.500000,0.500000}%
\pgfsetstrokecolor{currentstroke}%
\pgfsetstrokeopacity{0.300000}%
\pgfsetdash{}{0pt}%
\pgfpathmoveto{\pgfqpoint{2.659912in}{2.741485in}}%
\pgfusepath{stroke}%
\end{pgfscope}%
\begin{pgfscope}%
\pgfpathrectangle{\pgfqpoint{0.647939in}{0.492442in}}{\pgfqpoint{4.273799in}{2.331163in}}%
\pgfusepath{clip}%
\pgfsetroundcap%
\pgfsetroundjoin%
\definecolor{currentfill}{rgb}{0.500000,0.500000,0.500000}%
\pgfsetfillcolor{currentfill}%
\pgfsetfillopacity{0.300000}%
\pgfsetlinewidth{0.301125pt}%
\definecolor{currentstroke}{rgb}{0.500000,0.500000,0.500000}%
\pgfsetstrokecolor{currentstroke}%
\pgfsetstrokeopacity{0.300000}%
\pgfsetdash{}{0pt}%
\pgfpathmoveto{\pgfqpoint{0.000000in}{0.000000in}}%
\pgfpathlineto{\pgfqpoint{0.000000in}{0.000000in}}%
\pgfpathclose%
\pgfusepath{stroke,fill}%
\end{pgfscope}%
\begin{pgfscope}%
\pgfpathrectangle{\pgfqpoint{0.647939in}{0.492442in}}{\pgfqpoint{4.273799in}{2.331163in}}%
\pgfusepath{clip}%
\pgfsetroundcap%
\pgfsetroundjoin%
\pgfsetlinewidth{0.301125pt}%
\definecolor{currentstroke}{rgb}{0.500000,0.500000,0.500000}%
\pgfsetstrokecolor{currentstroke}%
\pgfsetstrokeopacity{0.300000}%
\pgfsetdash{}{0pt}%
\pgfpathmoveto{\pgfqpoint{2.989727in}{2.446780in}}%
\pgfusepath{stroke}%
\end{pgfscope}%
\begin{pgfscope}%
\pgfpathrectangle{\pgfqpoint{0.647939in}{0.492442in}}{\pgfqpoint{4.273799in}{2.331163in}}%
\pgfusepath{clip}%
\pgfsetroundcap%
\pgfsetroundjoin%
\definecolor{currentfill}{rgb}{0.500000,0.500000,0.500000}%
\pgfsetfillcolor{currentfill}%
\pgfsetfillopacity{0.300000}%
\pgfsetlinewidth{0.301125pt}%
\definecolor{currentstroke}{rgb}{0.500000,0.500000,0.500000}%
\pgfsetstrokecolor{currentstroke}%
\pgfsetstrokeopacity{0.300000}%
\pgfsetdash{}{0pt}%
\pgfpathmoveto{\pgfqpoint{0.000000in}{0.000000in}}%
\pgfpathlineto{\pgfqpoint{0.000000in}{0.000000in}}%
\pgfpathclose%
\pgfusepath{stroke,fill}%
\end{pgfscope}%
\begin{pgfscope}%
\pgfpathrectangle{\pgfqpoint{0.647939in}{0.492442in}}{\pgfqpoint{4.273799in}{2.331163in}}%
\pgfusepath{clip}%
\pgfsetroundcap%
\pgfsetroundjoin%
\pgfsetlinewidth{0.301125pt}%
\definecolor{currentstroke}{rgb}{0.500000,0.500000,0.500000}%
\pgfsetstrokecolor{currentstroke}%
\pgfsetstrokeopacity{0.300000}%
\pgfsetdash{}{0pt}%
\pgfpathmoveto{\pgfqpoint{2.369544in}{2.719367in}}%
\pgfusepath{stroke}%
\end{pgfscope}%
\begin{pgfscope}%
\pgfpathrectangle{\pgfqpoint{0.647939in}{0.492442in}}{\pgfqpoint{4.273799in}{2.331163in}}%
\pgfusepath{clip}%
\pgfsetroundcap%
\pgfsetroundjoin%
\definecolor{currentfill}{rgb}{0.500000,0.500000,0.500000}%
\pgfsetfillcolor{currentfill}%
\pgfsetfillopacity{0.300000}%
\pgfsetlinewidth{0.301125pt}%
\definecolor{currentstroke}{rgb}{0.500000,0.500000,0.500000}%
\pgfsetstrokecolor{currentstroke}%
\pgfsetstrokeopacity{0.300000}%
\pgfsetdash{}{0pt}%
\pgfpathmoveto{\pgfqpoint{0.000000in}{0.000000in}}%
\pgfpathlineto{\pgfqpoint{0.000000in}{0.000000in}}%
\pgfpathclose%
\pgfusepath{stroke,fill}%
\end{pgfscope}%
\begin{pgfscope}%
\pgfpathrectangle{\pgfqpoint{0.647939in}{0.492442in}}{\pgfqpoint{4.273799in}{2.331163in}}%
\pgfusepath{clip}%
\pgfsetroundcap%
\pgfsetroundjoin%
\pgfsetlinewidth{0.301125pt}%
\definecolor{currentstroke}{rgb}{0.500000,0.500000,0.500000}%
\pgfsetstrokecolor{currentstroke}%
\pgfsetstrokeopacity{0.300000}%
\pgfsetdash{}{0pt}%
\pgfpathmoveto{\pgfqpoint{2.150058in}{2.702001in}}%
\pgfusepath{stroke}%
\end{pgfscope}%
\begin{pgfscope}%
\pgfpathrectangle{\pgfqpoint{0.647939in}{0.492442in}}{\pgfqpoint{4.273799in}{2.331163in}}%
\pgfusepath{clip}%
\pgfsetroundcap%
\pgfsetroundjoin%
\definecolor{currentfill}{rgb}{0.500000,0.500000,0.500000}%
\pgfsetfillcolor{currentfill}%
\pgfsetfillopacity{0.300000}%
\pgfsetlinewidth{0.301125pt}%
\definecolor{currentstroke}{rgb}{0.500000,0.500000,0.500000}%
\pgfsetstrokecolor{currentstroke}%
\pgfsetstrokeopacity{0.300000}%
\pgfsetdash{}{0pt}%
\pgfpathmoveto{\pgfqpoint{0.000000in}{0.000000in}}%
\pgfpathlineto{\pgfqpoint{0.000000in}{0.000000in}}%
\pgfpathclose%
\pgfusepath{stroke,fill}%
\end{pgfscope}%
\begin{pgfscope}%
\pgfpathrectangle{\pgfqpoint{0.647939in}{0.492442in}}{\pgfqpoint{4.273799in}{2.331163in}}%
\pgfusepath{clip}%
\pgfsetroundcap%
\pgfsetroundjoin%
\pgfsetlinewidth{0.301125pt}%
\definecolor{currentstroke}{rgb}{0.500000,0.500000,0.500000}%
\pgfsetstrokecolor{currentstroke}%
\pgfsetstrokeopacity{0.300000}%
\pgfsetdash{}{0pt}%
\pgfpathmoveto{\pgfqpoint{1.863384in}{2.732964in}}%
\pgfusepath{stroke}%
\end{pgfscope}%
\begin{pgfscope}%
\pgfpathrectangle{\pgfqpoint{0.647939in}{0.492442in}}{\pgfqpoint{4.273799in}{2.331163in}}%
\pgfusepath{clip}%
\pgfsetroundcap%
\pgfsetroundjoin%
\definecolor{currentfill}{rgb}{0.500000,0.500000,0.500000}%
\pgfsetfillcolor{currentfill}%
\pgfsetfillopacity{0.300000}%
\pgfsetlinewidth{0.301125pt}%
\definecolor{currentstroke}{rgb}{0.500000,0.500000,0.500000}%
\pgfsetstrokecolor{currentstroke}%
\pgfsetstrokeopacity{0.300000}%
\pgfsetdash{}{0pt}%
\pgfpathmoveto{\pgfqpoint{0.000000in}{0.000000in}}%
\pgfpathlineto{\pgfqpoint{0.000000in}{0.000000in}}%
\pgfpathclose%
\pgfusepath{stroke,fill}%
\end{pgfscope}%
\begin{pgfscope}%
\pgfpathrectangle{\pgfqpoint{0.647939in}{0.492442in}}{\pgfqpoint{4.273799in}{2.331163in}}%
\pgfusepath{clip}%
\pgfsetroundcap%
\pgfsetroundjoin%
\pgfsetlinewidth{0.301125pt}%
\definecolor{currentstroke}{rgb}{0.500000,0.500000,0.500000}%
\pgfsetstrokecolor{currentstroke}%
\pgfsetstrokeopacity{0.300000}%
\pgfsetdash{}{0pt}%
\pgfpathmoveto{\pgfqpoint{2.319594in}{2.590270in}}%
\pgfusepath{stroke}%
\end{pgfscope}%
\begin{pgfscope}%
\pgfpathrectangle{\pgfqpoint{0.647939in}{0.492442in}}{\pgfqpoint{4.273799in}{2.331163in}}%
\pgfusepath{clip}%
\pgfsetroundcap%
\pgfsetroundjoin%
\definecolor{currentfill}{rgb}{0.500000,0.500000,0.500000}%
\pgfsetfillcolor{currentfill}%
\pgfsetfillopacity{0.300000}%
\pgfsetlinewidth{0.301125pt}%
\definecolor{currentstroke}{rgb}{0.500000,0.500000,0.500000}%
\pgfsetstrokecolor{currentstroke}%
\pgfsetstrokeopacity{0.300000}%
\pgfsetdash{}{0pt}%
\pgfpathmoveto{\pgfqpoint{0.000000in}{0.000000in}}%
\pgfpathlineto{\pgfqpoint{0.000000in}{0.000000in}}%
\pgfpathclose%
\pgfusepath{stroke,fill}%
\end{pgfscope}%
\begin{pgfscope}%
\pgfpathrectangle{\pgfqpoint{0.647939in}{0.492442in}}{\pgfqpoint{4.273799in}{2.331163in}}%
\pgfusepath{clip}%
\pgfsetroundcap%
\pgfsetroundjoin%
\pgfsetlinewidth{0.301125pt}%
\definecolor{currentstroke}{rgb}{0.500000,0.500000,0.500000}%
\pgfsetstrokecolor{currentstroke}%
\pgfsetstrokeopacity{0.300000}%
\pgfsetdash{}{0pt}%
\pgfpathmoveto{\pgfqpoint{1.923662in}{2.542146in}}%
\pgfusepath{stroke}%
\end{pgfscope}%
\begin{pgfscope}%
\pgfpathrectangle{\pgfqpoint{0.647939in}{0.492442in}}{\pgfqpoint{4.273799in}{2.331163in}}%
\pgfusepath{clip}%
\pgfsetroundcap%
\pgfsetroundjoin%
\definecolor{currentfill}{rgb}{0.500000,0.500000,0.500000}%
\pgfsetfillcolor{currentfill}%
\pgfsetfillopacity{0.300000}%
\pgfsetlinewidth{0.301125pt}%
\definecolor{currentstroke}{rgb}{0.500000,0.500000,0.500000}%
\pgfsetstrokecolor{currentstroke}%
\pgfsetstrokeopacity{0.300000}%
\pgfsetdash{}{0pt}%
\pgfpathmoveto{\pgfqpoint{0.000000in}{0.000000in}}%
\pgfpathlineto{\pgfqpoint{0.000000in}{0.000000in}}%
\pgfpathclose%
\pgfusepath{stroke,fill}%
\end{pgfscope}%
\begin{pgfscope}%
\pgfpathrectangle{\pgfqpoint{0.647939in}{0.492442in}}{\pgfqpoint{4.273799in}{2.331163in}}%
\pgfusepath{clip}%
\pgfsetroundcap%
\pgfsetroundjoin%
\pgfsetlinewidth{0.301125pt}%
\definecolor{currentstroke}{rgb}{0.500000,0.500000,0.500000}%
\pgfsetstrokecolor{currentstroke}%
\pgfsetstrokeopacity{0.300000}%
\pgfsetdash{}{0pt}%
\pgfpathmoveto{\pgfqpoint{1.966203in}{2.438323in}}%
\pgfusepath{stroke}%
\end{pgfscope}%
\begin{pgfscope}%
\pgfpathrectangle{\pgfqpoint{0.647939in}{0.492442in}}{\pgfqpoint{4.273799in}{2.331163in}}%
\pgfusepath{clip}%
\pgfsetroundcap%
\pgfsetroundjoin%
\definecolor{currentfill}{rgb}{0.500000,0.500000,0.500000}%
\pgfsetfillcolor{currentfill}%
\pgfsetfillopacity{0.300000}%
\pgfsetlinewidth{0.301125pt}%
\definecolor{currentstroke}{rgb}{0.500000,0.500000,0.500000}%
\pgfsetstrokecolor{currentstroke}%
\pgfsetstrokeopacity{0.300000}%
\pgfsetdash{}{0pt}%
\pgfpathmoveto{\pgfqpoint{0.000000in}{0.000000in}}%
\pgfpathlineto{\pgfqpoint{0.000000in}{0.000000in}}%
\pgfpathclose%
\pgfusepath{stroke,fill}%
\end{pgfscope}%
\begin{pgfscope}%
\pgfpathrectangle{\pgfqpoint{0.647939in}{0.492442in}}{\pgfqpoint{4.273799in}{2.331163in}}%
\pgfusepath{clip}%
\pgfsetroundcap%
\pgfsetroundjoin%
\pgfsetlinewidth{0.301125pt}%
\definecolor{currentstroke}{rgb}{0.500000,0.500000,0.500000}%
\pgfsetstrokecolor{currentstroke}%
\pgfsetstrokeopacity{0.300000}%
\pgfsetdash{}{0pt}%
\pgfpathmoveto{\pgfqpoint{1.566828in}{2.405285in}}%
\pgfusepath{stroke}%
\end{pgfscope}%
\begin{pgfscope}%
\pgfpathrectangle{\pgfqpoint{0.647939in}{0.492442in}}{\pgfqpoint{4.273799in}{2.331163in}}%
\pgfusepath{clip}%
\pgfsetroundcap%
\pgfsetroundjoin%
\definecolor{currentfill}{rgb}{0.500000,0.500000,0.500000}%
\pgfsetfillcolor{currentfill}%
\pgfsetfillopacity{0.300000}%
\pgfsetlinewidth{0.301125pt}%
\definecolor{currentstroke}{rgb}{0.500000,0.500000,0.500000}%
\pgfsetstrokecolor{currentstroke}%
\pgfsetstrokeopacity{0.300000}%
\pgfsetdash{}{0pt}%
\pgfpathmoveto{\pgfqpoint{0.000000in}{0.000000in}}%
\pgfpathlineto{\pgfqpoint{0.000000in}{0.000000in}}%
\pgfpathclose%
\pgfusepath{stroke,fill}%
\end{pgfscope}%
\begin{pgfscope}%
\pgfpathrectangle{\pgfqpoint{0.647939in}{0.492442in}}{\pgfqpoint{4.273799in}{2.331163in}}%
\pgfusepath{clip}%
\pgfsetroundcap%
\pgfsetroundjoin%
\pgfsetlinewidth{0.301125pt}%
\definecolor{currentstroke}{rgb}{0.500000,0.500000,0.500000}%
\pgfsetstrokecolor{currentstroke}%
\pgfsetstrokeopacity{0.300000}%
\pgfsetdash{}{0pt}%
\pgfpathmoveto{\pgfqpoint{1.429927in}{2.317127in}}%
\pgfusepath{stroke}%
\end{pgfscope}%
\begin{pgfscope}%
\pgfpathrectangle{\pgfqpoint{0.647939in}{0.492442in}}{\pgfqpoint{4.273799in}{2.331163in}}%
\pgfusepath{clip}%
\pgfsetroundcap%
\pgfsetroundjoin%
\definecolor{currentfill}{rgb}{0.500000,0.500000,0.500000}%
\pgfsetfillcolor{currentfill}%
\pgfsetfillopacity{0.300000}%
\pgfsetlinewidth{0.301125pt}%
\definecolor{currentstroke}{rgb}{0.500000,0.500000,0.500000}%
\pgfsetstrokecolor{currentstroke}%
\pgfsetstrokeopacity{0.300000}%
\pgfsetdash{}{0pt}%
\pgfpathmoveto{\pgfqpoint{0.000000in}{0.000000in}}%
\pgfpathlineto{\pgfqpoint{0.000000in}{0.000000in}}%
\pgfpathclose%
\pgfusepath{stroke,fill}%
\end{pgfscope}%
\begin{pgfscope}%
\pgfpathrectangle{\pgfqpoint{0.647939in}{0.492442in}}{\pgfqpoint{4.273799in}{2.331163in}}%
\pgfusepath{clip}%
\pgfsetroundcap%
\pgfsetroundjoin%
\pgfsetlinewidth{0.301125pt}%
\definecolor{currentstroke}{rgb}{0.500000,0.500000,0.500000}%
\pgfsetstrokecolor{currentstroke}%
\pgfsetstrokeopacity{0.300000}%
\pgfsetdash{}{0pt}%
\pgfpathmoveto{\pgfqpoint{1.292043in}{2.260496in}}%
\pgfusepath{stroke}%
\end{pgfscope}%
\begin{pgfscope}%
\pgfpathrectangle{\pgfqpoint{0.647939in}{0.492442in}}{\pgfqpoint{4.273799in}{2.331163in}}%
\pgfusepath{clip}%
\pgfsetroundcap%
\pgfsetroundjoin%
\definecolor{currentfill}{rgb}{0.500000,0.500000,0.500000}%
\pgfsetfillcolor{currentfill}%
\pgfsetfillopacity{0.300000}%
\pgfsetlinewidth{0.301125pt}%
\definecolor{currentstroke}{rgb}{0.500000,0.500000,0.500000}%
\pgfsetstrokecolor{currentstroke}%
\pgfsetstrokeopacity{0.300000}%
\pgfsetdash{}{0pt}%
\pgfpathmoveto{\pgfqpoint{0.000000in}{0.000000in}}%
\pgfpathlineto{\pgfqpoint{0.000000in}{0.000000in}}%
\pgfpathclose%
\pgfusepath{stroke,fill}%
\end{pgfscope}%
\begin{pgfscope}%
\pgfpathrectangle{\pgfqpoint{0.647939in}{0.492442in}}{\pgfqpoint{4.273799in}{2.331163in}}%
\pgfusepath{clip}%
\pgfsetroundcap%
\pgfsetroundjoin%
\pgfsetlinewidth{0.301125pt}%
\definecolor{currentstroke}{rgb}{0.500000,0.500000,0.500000}%
\pgfsetstrokecolor{currentstroke}%
\pgfsetstrokeopacity{0.300000}%
\pgfsetdash{}{0pt}%
\pgfpathmoveto{\pgfqpoint{1.166796in}{1.895537in}}%
\pgfusepath{stroke}%
\end{pgfscope}%
\begin{pgfscope}%
\pgfpathrectangle{\pgfqpoint{0.647939in}{0.492442in}}{\pgfqpoint{4.273799in}{2.331163in}}%
\pgfusepath{clip}%
\pgfsetroundcap%
\pgfsetroundjoin%
\definecolor{currentfill}{rgb}{0.500000,0.500000,0.500000}%
\pgfsetfillcolor{currentfill}%
\pgfsetfillopacity{0.300000}%
\pgfsetlinewidth{0.301125pt}%
\definecolor{currentstroke}{rgb}{0.500000,0.500000,0.500000}%
\pgfsetstrokecolor{currentstroke}%
\pgfsetstrokeopacity{0.300000}%
\pgfsetdash{}{0pt}%
\pgfpathmoveto{\pgfqpoint{0.000000in}{0.000000in}}%
\pgfpathlineto{\pgfqpoint{0.000000in}{0.000000in}}%
\pgfpathclose%
\pgfusepath{stroke,fill}%
\end{pgfscope}%
\begin{pgfscope}%
\pgfpathrectangle{\pgfqpoint{0.647939in}{0.492442in}}{\pgfqpoint{4.273799in}{2.331163in}}%
\pgfusepath{clip}%
\pgfsetroundcap%
\pgfsetroundjoin%
\pgfsetlinewidth{0.301125pt}%
\definecolor{currentstroke}{rgb}{0.500000,0.500000,0.500000}%
\pgfsetstrokecolor{currentstroke}%
\pgfsetstrokeopacity{0.300000}%
\pgfsetdash{}{0pt}%
\pgfpathmoveto{\pgfqpoint{1.047817in}{2.101134in}}%
\pgfusepath{stroke}%
\end{pgfscope}%
\begin{pgfscope}%
\pgfpathrectangle{\pgfqpoint{0.647939in}{0.492442in}}{\pgfqpoint{4.273799in}{2.331163in}}%
\pgfusepath{clip}%
\pgfsetroundcap%
\pgfsetroundjoin%
\definecolor{currentfill}{rgb}{0.500000,0.500000,0.500000}%
\pgfsetfillcolor{currentfill}%
\pgfsetfillopacity{0.300000}%
\pgfsetlinewidth{0.301125pt}%
\definecolor{currentstroke}{rgb}{0.500000,0.500000,0.500000}%
\pgfsetstrokecolor{currentstroke}%
\pgfsetstrokeopacity{0.300000}%
\pgfsetdash{}{0pt}%
\pgfpathmoveto{\pgfqpoint{0.000000in}{0.000000in}}%
\pgfpathlineto{\pgfqpoint{0.000000in}{0.000000in}}%
\pgfpathclose%
\pgfusepath{stroke,fill}%
\end{pgfscope}%
\begin{pgfscope}%
\pgfpathrectangle{\pgfqpoint{0.647939in}{0.492442in}}{\pgfqpoint{4.273799in}{2.331163in}}%
\pgfusepath{clip}%
\pgfsetroundcap%
\pgfsetroundjoin%
\pgfsetlinewidth{0.301125pt}%
\definecolor{currentstroke}{rgb}{0.500000,0.500000,0.500000}%
\pgfsetstrokecolor{currentstroke}%
\pgfsetstrokeopacity{0.300000}%
\pgfsetdash{}{0pt}%
\pgfpathmoveto{\pgfqpoint{0.919383in}{1.789542in}}%
\pgfusepath{stroke}%
\end{pgfscope}%
\begin{pgfscope}%
\pgfpathrectangle{\pgfqpoint{0.647939in}{0.492442in}}{\pgfqpoint{4.273799in}{2.331163in}}%
\pgfusepath{clip}%
\pgfsetroundcap%
\pgfsetroundjoin%
\definecolor{currentfill}{rgb}{0.500000,0.500000,0.500000}%
\pgfsetfillcolor{currentfill}%
\pgfsetfillopacity{0.300000}%
\pgfsetlinewidth{0.301125pt}%
\definecolor{currentstroke}{rgb}{0.500000,0.500000,0.500000}%
\pgfsetstrokecolor{currentstroke}%
\pgfsetstrokeopacity{0.300000}%
\pgfsetdash{}{0pt}%
\pgfpathmoveto{\pgfqpoint{0.000000in}{0.000000in}}%
\pgfpathlineto{\pgfqpoint{0.000000in}{0.000000in}}%
\pgfpathclose%
\pgfusepath{stroke,fill}%
\end{pgfscope}%
\begin{pgfscope}%
\pgfpathrectangle{\pgfqpoint{0.647939in}{0.492442in}}{\pgfqpoint{4.273799in}{2.331163in}}%
\pgfusepath{clip}%
\pgfsetroundcap%
\pgfsetroundjoin%
\pgfsetlinewidth{0.301125pt}%
\definecolor{currentstroke}{rgb}{0.500000,0.500000,0.500000}%
\pgfsetstrokecolor{currentstroke}%
\pgfsetstrokeopacity{0.300000}%
\pgfsetdash{}{0pt}%
\pgfpathmoveto{\pgfqpoint{0.819406in}{2.047875in}}%
\pgfusepath{stroke}%
\end{pgfscope}%
\begin{pgfscope}%
\pgfpathrectangle{\pgfqpoint{0.647939in}{0.492442in}}{\pgfqpoint{4.273799in}{2.331163in}}%
\pgfusepath{clip}%
\pgfsetroundcap%
\pgfsetroundjoin%
\definecolor{currentfill}{rgb}{0.500000,0.500000,0.500000}%
\pgfsetfillcolor{currentfill}%
\pgfsetfillopacity{0.300000}%
\pgfsetlinewidth{0.301125pt}%
\definecolor{currentstroke}{rgb}{0.500000,0.500000,0.500000}%
\pgfsetstrokecolor{currentstroke}%
\pgfsetstrokeopacity{0.300000}%
\pgfsetdash{}{0pt}%
\pgfpathmoveto{\pgfqpoint{0.000000in}{0.000000in}}%
\pgfpathlineto{\pgfqpoint{0.000000in}{0.000000in}}%
\pgfpathclose%
\pgfusepath{stroke,fill}%
\end{pgfscope}%
\begin{pgfscope}%
\pgfpathrectangle{\pgfqpoint{0.647939in}{0.492442in}}{\pgfqpoint{4.273799in}{2.331163in}}%
\pgfusepath{clip}%
\pgfsetroundcap%
\pgfsetroundjoin%
\pgfsetlinewidth{0.301125pt}%
\definecolor{currentstroke}{rgb}{0.500000,0.500000,0.500000}%
\pgfsetstrokecolor{currentstroke}%
\pgfsetstrokeopacity{0.300000}%
\pgfsetdash{}{0pt}%
\pgfpathmoveto{\pgfqpoint{0.710324in}{2.099309in}}%
\pgfusepath{stroke}%
\end{pgfscope}%
\begin{pgfscope}%
\pgfpathrectangle{\pgfqpoint{0.647939in}{0.492442in}}{\pgfqpoint{4.273799in}{2.331163in}}%
\pgfusepath{clip}%
\pgfsetroundcap%
\pgfsetroundjoin%
\definecolor{currentfill}{rgb}{0.500000,0.500000,0.500000}%
\pgfsetfillcolor{currentfill}%
\pgfsetfillopacity{0.300000}%
\pgfsetlinewidth{0.301125pt}%
\definecolor{currentstroke}{rgb}{0.500000,0.500000,0.500000}%
\pgfsetstrokecolor{currentstroke}%
\pgfsetstrokeopacity{0.300000}%
\pgfsetdash{}{0pt}%
\pgfpathmoveto{\pgfqpoint{0.000000in}{0.000000in}}%
\pgfpathlineto{\pgfqpoint{0.000000in}{0.000000in}}%
\pgfpathclose%
\pgfusepath{stroke,fill}%
\end{pgfscope}%
\begin{pgfscope}%
\pgfpathrectangle{\pgfqpoint{0.647939in}{0.492442in}}{\pgfqpoint{4.273799in}{2.331163in}}%
\pgfusepath{clip}%
\pgfsetroundcap%
\pgfsetroundjoin%
\pgfsetlinewidth{0.301125pt}%
\definecolor{currentstroke}{rgb}{0.500000,0.500000,0.500000}%
\pgfsetstrokecolor{currentstroke}%
\pgfsetstrokeopacity{0.300000}%
\pgfsetdash{}{0pt}%
\pgfpathmoveto{\pgfqpoint{0.657578in}{2.087821in}}%
\pgfusepath{stroke}%
\end{pgfscope}%
\begin{pgfscope}%
\pgfpathrectangle{\pgfqpoint{0.647939in}{0.492442in}}{\pgfqpoint{4.273799in}{2.331163in}}%
\pgfusepath{clip}%
\pgfsetroundcap%
\pgfsetroundjoin%
\definecolor{currentfill}{rgb}{0.500000,0.500000,0.500000}%
\pgfsetfillcolor{currentfill}%
\pgfsetfillopacity{0.300000}%
\pgfsetlinewidth{0.301125pt}%
\definecolor{currentstroke}{rgb}{0.500000,0.500000,0.500000}%
\pgfsetstrokecolor{currentstroke}%
\pgfsetstrokeopacity{0.300000}%
\pgfsetdash{}{0pt}%
\pgfpathmoveto{\pgfqpoint{0.000000in}{0.000000in}}%
\pgfpathlineto{\pgfqpoint{0.000000in}{0.000000in}}%
\pgfpathclose%
\pgfusepath{stroke,fill}%
\end{pgfscope}%
\begin{pgfscope}%
\pgfpathrectangle{\pgfqpoint{0.647939in}{0.492442in}}{\pgfqpoint{4.273799in}{2.331163in}}%
\pgfusepath{clip}%
\pgfsetroundcap%
\pgfsetroundjoin%
\pgfsetlinewidth{0.301125pt}%
\definecolor{currentstroke}{rgb}{0.500000,0.500000,0.500000}%
\pgfsetstrokecolor{currentstroke}%
\pgfsetstrokeopacity{0.300000}%
\pgfsetdash{}{0pt}%
\pgfpathmoveto{\pgfqpoint{1.381663in}{0.728448in}}%
\pgfusepath{stroke}%
\end{pgfscope}%
\begin{pgfscope}%
\pgfpathrectangle{\pgfqpoint{0.647939in}{0.492442in}}{\pgfqpoint{4.273799in}{2.331163in}}%
\pgfusepath{clip}%
\pgfsetroundcap%
\pgfsetroundjoin%
\definecolor{currentfill}{rgb}{0.500000,0.500000,0.500000}%
\pgfsetfillcolor{currentfill}%
\pgfsetfillopacity{0.300000}%
\pgfsetlinewidth{0.301125pt}%
\definecolor{currentstroke}{rgb}{0.500000,0.500000,0.500000}%
\pgfsetstrokecolor{currentstroke}%
\pgfsetstrokeopacity{0.300000}%
\pgfsetdash{}{0pt}%
\pgfpathmoveto{\pgfqpoint{0.000000in}{0.000000in}}%
\pgfpathlineto{\pgfqpoint{0.000000in}{0.000000in}}%
\pgfpathclose%
\pgfusepath{stroke,fill}%
\end{pgfscope}%
\begin{pgfscope}%
\pgfpathrectangle{\pgfqpoint{0.647939in}{0.492442in}}{\pgfqpoint{4.273799in}{2.331163in}}%
\pgfusepath{clip}%
\pgfsetroundcap%
\pgfsetroundjoin%
\pgfsetlinewidth{0.301125pt}%
\definecolor{currentstroke}{rgb}{0.500000,0.500000,0.500000}%
\pgfsetstrokecolor{currentstroke}%
\pgfsetstrokeopacity{0.300000}%
\pgfsetdash{}{0pt}%
\pgfpathmoveto{\pgfqpoint{1.301538in}{0.607340in}}%
\pgfusepath{stroke}%
\end{pgfscope}%
\begin{pgfscope}%
\pgfpathrectangle{\pgfqpoint{0.647939in}{0.492442in}}{\pgfqpoint{4.273799in}{2.331163in}}%
\pgfusepath{clip}%
\pgfsetroundcap%
\pgfsetroundjoin%
\definecolor{currentfill}{rgb}{0.500000,0.500000,0.500000}%
\pgfsetfillcolor{currentfill}%
\pgfsetfillopacity{0.300000}%
\pgfsetlinewidth{0.301125pt}%
\definecolor{currentstroke}{rgb}{0.500000,0.500000,0.500000}%
\pgfsetstrokecolor{currentstroke}%
\pgfsetstrokeopacity{0.300000}%
\pgfsetdash{}{0pt}%
\pgfpathmoveto{\pgfqpoint{0.000000in}{0.000000in}}%
\pgfpathlineto{\pgfqpoint{0.000000in}{0.000000in}}%
\pgfpathclose%
\pgfusepath{stroke,fill}%
\end{pgfscope}%
\begin{pgfscope}%
\pgfpathrectangle{\pgfqpoint{0.647939in}{0.492442in}}{\pgfqpoint{4.273799in}{2.331163in}}%
\pgfusepath{clip}%
\pgfsetroundcap%
\pgfsetroundjoin%
\pgfsetlinewidth{0.301125pt}%
\definecolor{currentstroke}{rgb}{0.500000,0.500000,0.500000}%
\pgfsetstrokecolor{currentstroke}%
\pgfsetstrokeopacity{0.300000}%
\pgfsetdash{}{0pt}%
\pgfpathmoveto{\pgfqpoint{4.710942in}{1.972090in}}%
\pgfusepath{stroke}%
\end{pgfscope}%
\begin{pgfscope}%
\pgfpathrectangle{\pgfqpoint{0.647939in}{0.492442in}}{\pgfqpoint{4.273799in}{2.331163in}}%
\pgfusepath{clip}%
\pgfsetroundcap%
\pgfsetroundjoin%
\definecolor{currentfill}{rgb}{0.500000,0.500000,0.500000}%
\pgfsetfillcolor{currentfill}%
\pgfsetfillopacity{0.300000}%
\pgfsetlinewidth{0.301125pt}%
\definecolor{currentstroke}{rgb}{0.500000,0.500000,0.500000}%
\pgfsetstrokecolor{currentstroke}%
\pgfsetstrokeopacity{0.300000}%
\pgfsetdash{}{0pt}%
\pgfpathmoveto{\pgfqpoint{0.000000in}{0.000000in}}%
\pgfpathlineto{\pgfqpoint{0.000000in}{0.000000in}}%
\pgfpathclose%
\pgfusepath{stroke,fill}%
\end{pgfscope}%
\begin{pgfscope}%
\pgfpathrectangle{\pgfqpoint{0.647939in}{0.492442in}}{\pgfqpoint{4.273799in}{2.331163in}}%
\pgfusepath{clip}%
\pgfsetroundcap%
\pgfsetroundjoin%
\pgfsetlinewidth{0.301125pt}%
\definecolor{currentstroke}{rgb}{0.500000,0.500000,0.500000}%
\pgfsetstrokecolor{currentstroke}%
\pgfsetstrokeopacity{0.300000}%
\pgfsetdash{}{0pt}%
\pgfpathmoveto{\pgfqpoint{3.741250in}{0.783654in}}%
\pgfusepath{stroke}%
\end{pgfscope}%
\begin{pgfscope}%
\pgfpathrectangle{\pgfqpoint{0.647939in}{0.492442in}}{\pgfqpoint{4.273799in}{2.331163in}}%
\pgfusepath{clip}%
\pgfsetroundcap%
\pgfsetroundjoin%
\definecolor{currentfill}{rgb}{0.500000,0.500000,0.500000}%
\pgfsetfillcolor{currentfill}%
\pgfsetfillopacity{0.300000}%
\pgfsetlinewidth{0.301125pt}%
\definecolor{currentstroke}{rgb}{0.500000,0.500000,0.500000}%
\pgfsetstrokecolor{currentstroke}%
\pgfsetstrokeopacity{0.300000}%
\pgfsetdash{}{0pt}%
\pgfpathmoveto{\pgfqpoint{0.000000in}{0.000000in}}%
\pgfpathlineto{\pgfqpoint{0.000000in}{0.000000in}}%
\pgfpathclose%
\pgfusepath{stroke,fill}%
\end{pgfscope}%
\begin{pgfscope}%
\pgfpathrectangle{\pgfqpoint{0.647939in}{0.492442in}}{\pgfqpoint{4.273799in}{2.331163in}}%
\pgfusepath{clip}%
\pgfsetroundcap%
\pgfsetroundjoin%
\pgfsetlinewidth{0.301125pt}%
\definecolor{currentstroke}{rgb}{0.500000,0.500000,0.500000}%
\pgfsetstrokecolor{currentstroke}%
\pgfsetstrokeopacity{0.300000}%
\pgfsetdash{}{0pt}%
\pgfpathmoveto{\pgfqpoint{4.567849in}{1.809409in}}%
\pgfusepath{stroke}%
\end{pgfscope}%
\begin{pgfscope}%
\pgfpathrectangle{\pgfqpoint{0.647939in}{0.492442in}}{\pgfqpoint{4.273799in}{2.331163in}}%
\pgfusepath{clip}%
\pgfsetroundcap%
\pgfsetroundjoin%
\definecolor{currentfill}{rgb}{0.500000,0.500000,0.500000}%
\pgfsetfillcolor{currentfill}%
\pgfsetfillopacity{0.300000}%
\pgfsetlinewidth{0.301125pt}%
\definecolor{currentstroke}{rgb}{0.500000,0.500000,0.500000}%
\pgfsetstrokecolor{currentstroke}%
\pgfsetstrokeopacity{0.300000}%
\pgfsetdash{}{0pt}%
\pgfpathmoveto{\pgfqpoint{0.000000in}{0.000000in}}%
\pgfpathlineto{\pgfqpoint{0.000000in}{0.000000in}}%
\pgfpathclose%
\pgfusepath{stroke,fill}%
\end{pgfscope}%
\begin{pgfscope}%
\pgfpathrectangle{\pgfqpoint{0.647939in}{0.492442in}}{\pgfqpoint{4.273799in}{2.331163in}}%
\pgfusepath{clip}%
\pgfsetroundcap%
\pgfsetroundjoin%
\pgfsetlinewidth{0.301125pt}%
\definecolor{currentstroke}{rgb}{0.500000,0.500000,0.500000}%
\pgfsetstrokecolor{currentstroke}%
\pgfsetstrokeopacity{0.300000}%
\pgfsetdash{}{0pt}%
\pgfpathmoveto{\pgfqpoint{1.379543in}{0.796859in}}%
\pgfusepath{stroke}%
\end{pgfscope}%
\begin{pgfscope}%
\pgfpathrectangle{\pgfqpoint{0.647939in}{0.492442in}}{\pgfqpoint{4.273799in}{2.331163in}}%
\pgfusepath{clip}%
\pgfsetroundcap%
\pgfsetroundjoin%
\definecolor{currentfill}{rgb}{0.500000,0.500000,0.500000}%
\pgfsetfillcolor{currentfill}%
\pgfsetfillopacity{0.300000}%
\pgfsetlinewidth{0.301125pt}%
\definecolor{currentstroke}{rgb}{0.500000,0.500000,0.500000}%
\pgfsetstrokecolor{currentstroke}%
\pgfsetstrokeopacity{0.300000}%
\pgfsetdash{}{0pt}%
\pgfpathmoveto{\pgfqpoint{0.000000in}{0.000000in}}%
\pgfpathlineto{\pgfqpoint{0.000000in}{0.000000in}}%
\pgfpathclose%
\pgfusepath{stroke,fill}%
\end{pgfscope}%
\begin{pgfscope}%
\pgfpathrectangle{\pgfqpoint{0.647939in}{0.492442in}}{\pgfqpoint{4.273799in}{2.331163in}}%
\pgfusepath{clip}%
\pgfsetroundcap%
\pgfsetroundjoin%
\pgfsetlinewidth{0.301125pt}%
\definecolor{currentstroke}{rgb}{0.500000,0.500000,0.500000}%
\pgfsetstrokecolor{currentstroke}%
\pgfsetstrokeopacity{0.300000}%
\pgfsetdash{}{0pt}%
\pgfpathmoveto{\pgfqpoint{3.635550in}{0.719277in}}%
\pgfusepath{stroke}%
\end{pgfscope}%
\begin{pgfscope}%
\pgfpathrectangle{\pgfqpoint{0.647939in}{0.492442in}}{\pgfqpoint{4.273799in}{2.331163in}}%
\pgfusepath{clip}%
\pgfsetroundcap%
\pgfsetroundjoin%
\definecolor{currentfill}{rgb}{0.500000,0.500000,0.500000}%
\pgfsetfillcolor{currentfill}%
\pgfsetfillopacity{0.300000}%
\pgfsetlinewidth{0.301125pt}%
\definecolor{currentstroke}{rgb}{0.500000,0.500000,0.500000}%
\pgfsetstrokecolor{currentstroke}%
\pgfsetstrokeopacity{0.300000}%
\pgfsetdash{}{0pt}%
\pgfpathmoveto{\pgfqpoint{0.000000in}{0.000000in}}%
\pgfpathlineto{\pgfqpoint{0.000000in}{0.000000in}}%
\pgfpathclose%
\pgfusepath{stroke,fill}%
\end{pgfscope}%
\begin{pgfscope}%
\pgfpathrectangle{\pgfqpoint{0.647939in}{0.492442in}}{\pgfqpoint{4.273799in}{2.331163in}}%
\pgfusepath{clip}%
\pgfsetroundcap%
\pgfsetroundjoin%
\pgfsetlinewidth{0.301125pt}%
\definecolor{currentstroke}{rgb}{0.500000,0.500000,0.500000}%
\pgfsetstrokecolor{currentstroke}%
\pgfsetstrokeopacity{0.300000}%
\pgfsetdash{}{0pt}%
\pgfpathmoveto{\pgfqpoint{3.871663in}{0.963865in}}%
\pgfusepath{stroke}%
\end{pgfscope}%
\begin{pgfscope}%
\pgfpathrectangle{\pgfqpoint{0.647939in}{0.492442in}}{\pgfqpoint{4.273799in}{2.331163in}}%
\pgfusepath{clip}%
\pgfsetroundcap%
\pgfsetroundjoin%
\definecolor{currentfill}{rgb}{0.500000,0.500000,0.500000}%
\pgfsetfillcolor{currentfill}%
\pgfsetfillopacity{0.300000}%
\pgfsetlinewidth{0.301125pt}%
\definecolor{currentstroke}{rgb}{0.500000,0.500000,0.500000}%
\pgfsetstrokecolor{currentstroke}%
\pgfsetstrokeopacity{0.300000}%
\pgfsetdash{}{0pt}%
\pgfpathmoveto{\pgfqpoint{0.000000in}{0.000000in}}%
\pgfpathlineto{\pgfqpoint{0.000000in}{0.000000in}}%
\pgfpathclose%
\pgfusepath{stroke,fill}%
\end{pgfscope}%
\begin{pgfscope}%
\pgfpathrectangle{\pgfqpoint{0.647939in}{0.492442in}}{\pgfqpoint{4.273799in}{2.331163in}}%
\pgfusepath{clip}%
\pgfsetroundcap%
\pgfsetroundjoin%
\pgfsetlinewidth{0.301125pt}%
\definecolor{currentstroke}{rgb}{0.500000,0.500000,0.500000}%
\pgfsetstrokecolor{currentstroke}%
\pgfsetstrokeopacity{0.300000}%
\pgfsetdash{}{0pt}%
\pgfpathmoveto{\pgfqpoint{4.117090in}{1.546841in}}%
\pgfusepath{stroke}%
\end{pgfscope}%
\begin{pgfscope}%
\pgfpathrectangle{\pgfqpoint{0.647939in}{0.492442in}}{\pgfqpoint{4.273799in}{2.331163in}}%
\pgfusepath{clip}%
\pgfsetroundcap%
\pgfsetroundjoin%
\definecolor{currentfill}{rgb}{0.500000,0.500000,0.500000}%
\pgfsetfillcolor{currentfill}%
\pgfsetfillopacity{0.300000}%
\pgfsetlinewidth{0.301125pt}%
\definecolor{currentstroke}{rgb}{0.500000,0.500000,0.500000}%
\pgfsetstrokecolor{currentstroke}%
\pgfsetstrokeopacity{0.300000}%
\pgfsetdash{}{0pt}%
\pgfpathmoveto{\pgfqpoint{0.000000in}{0.000000in}}%
\pgfpathlineto{\pgfqpoint{0.000000in}{0.000000in}}%
\pgfpathclose%
\pgfusepath{stroke,fill}%
\end{pgfscope}%
\begin{pgfscope}%
\pgfpathrectangle{\pgfqpoint{0.647939in}{0.492442in}}{\pgfqpoint{4.273799in}{2.331163in}}%
\pgfusepath{clip}%
\pgfsetroundcap%
\pgfsetroundjoin%
\pgfsetlinewidth{0.301125pt}%
\definecolor{currentstroke}{rgb}{0.500000,0.500000,0.500000}%
\pgfsetstrokecolor{currentstroke}%
\pgfsetstrokeopacity{0.300000}%
\pgfsetdash{}{0pt}%
\pgfpathmoveto{\pgfqpoint{4.405575in}{1.667188in}}%
\pgfusepath{stroke}%
\end{pgfscope}%
\begin{pgfscope}%
\pgfpathrectangle{\pgfqpoint{0.647939in}{0.492442in}}{\pgfqpoint{4.273799in}{2.331163in}}%
\pgfusepath{clip}%
\pgfsetroundcap%
\pgfsetroundjoin%
\definecolor{currentfill}{rgb}{0.500000,0.500000,0.500000}%
\pgfsetfillcolor{currentfill}%
\pgfsetfillopacity{0.300000}%
\pgfsetlinewidth{0.301125pt}%
\definecolor{currentstroke}{rgb}{0.500000,0.500000,0.500000}%
\pgfsetstrokecolor{currentstroke}%
\pgfsetstrokeopacity{0.300000}%
\pgfsetdash{}{0pt}%
\pgfpathmoveto{\pgfqpoint{0.000000in}{0.000000in}}%
\pgfpathlineto{\pgfqpoint{0.000000in}{0.000000in}}%
\pgfpathclose%
\pgfusepath{stroke,fill}%
\end{pgfscope}%
\begin{pgfscope}%
\pgfpathrectangle{\pgfqpoint{0.647939in}{0.492442in}}{\pgfqpoint{4.273799in}{2.331163in}}%
\pgfusepath{clip}%
\pgfsetroundcap%
\pgfsetroundjoin%
\pgfsetlinewidth{0.301125pt}%
\definecolor{currentstroke}{rgb}{0.500000,0.500000,0.500000}%
\pgfsetstrokecolor{currentstroke}%
\pgfsetstrokeopacity{0.300000}%
\pgfsetdash{}{0pt}%
\pgfpathmoveto{\pgfqpoint{1.871335in}{2.496025in}}%
\pgfusepath{stroke}%
\end{pgfscope}%
\begin{pgfscope}%
\pgfpathrectangle{\pgfqpoint{0.647939in}{0.492442in}}{\pgfqpoint{4.273799in}{2.331163in}}%
\pgfusepath{clip}%
\pgfsetroundcap%
\pgfsetroundjoin%
\definecolor{currentfill}{rgb}{0.500000,0.500000,0.500000}%
\pgfsetfillcolor{currentfill}%
\pgfsetfillopacity{0.300000}%
\pgfsetlinewidth{0.301125pt}%
\definecolor{currentstroke}{rgb}{0.500000,0.500000,0.500000}%
\pgfsetstrokecolor{currentstroke}%
\pgfsetstrokeopacity{0.300000}%
\pgfsetdash{}{0pt}%
\pgfpathmoveto{\pgfqpoint{0.000000in}{0.000000in}}%
\pgfpathlineto{\pgfqpoint{0.000000in}{0.000000in}}%
\pgfpathclose%
\pgfusepath{stroke,fill}%
\end{pgfscope}%
\begin{pgfscope}%
\pgfpathrectangle{\pgfqpoint{0.647939in}{0.492442in}}{\pgfqpoint{4.273799in}{2.331163in}}%
\pgfusepath{clip}%
\pgfsetroundcap%
\pgfsetroundjoin%
\pgfsetlinewidth{0.301125pt}%
\definecolor{currentstroke}{rgb}{0.500000,0.500000,0.500000}%
\pgfsetstrokecolor{currentstroke}%
\pgfsetstrokeopacity{0.300000}%
\pgfsetdash{}{0pt}%
\pgfpathmoveto{\pgfqpoint{1.779295in}{2.401823in}}%
\pgfusepath{stroke}%
\end{pgfscope}%
\begin{pgfscope}%
\pgfpathrectangle{\pgfqpoint{0.647939in}{0.492442in}}{\pgfqpoint{4.273799in}{2.331163in}}%
\pgfusepath{clip}%
\pgfsetroundcap%
\pgfsetroundjoin%
\definecolor{currentfill}{rgb}{0.500000,0.500000,0.500000}%
\pgfsetfillcolor{currentfill}%
\pgfsetfillopacity{0.300000}%
\pgfsetlinewidth{0.301125pt}%
\definecolor{currentstroke}{rgb}{0.500000,0.500000,0.500000}%
\pgfsetstrokecolor{currentstroke}%
\pgfsetstrokeopacity{0.300000}%
\pgfsetdash{}{0pt}%
\pgfpathmoveto{\pgfqpoint{0.000000in}{0.000000in}}%
\pgfpathlineto{\pgfqpoint{0.000000in}{0.000000in}}%
\pgfpathclose%
\pgfusepath{stroke,fill}%
\end{pgfscope}%
\begin{pgfscope}%
\pgfpathrectangle{\pgfqpoint{0.647939in}{0.492442in}}{\pgfqpoint{4.273799in}{2.331163in}}%
\pgfusepath{clip}%
\pgfsetroundcap%
\pgfsetroundjoin%
\pgfsetlinewidth{0.301125pt}%
\definecolor{currentstroke}{rgb}{0.500000,0.500000,0.500000}%
\pgfsetstrokecolor{currentstroke}%
\pgfsetstrokeopacity{0.300000}%
\pgfsetdash{}{0pt}%
\pgfpathmoveto{\pgfqpoint{1.540097in}{2.281887in}}%
\pgfusepath{stroke}%
\end{pgfscope}%
\begin{pgfscope}%
\pgfpathrectangle{\pgfqpoint{0.647939in}{0.492442in}}{\pgfqpoint{4.273799in}{2.331163in}}%
\pgfusepath{clip}%
\pgfsetroundcap%
\pgfsetroundjoin%
\definecolor{currentfill}{rgb}{0.500000,0.500000,0.500000}%
\pgfsetfillcolor{currentfill}%
\pgfsetfillopacity{0.300000}%
\pgfsetlinewidth{0.301125pt}%
\definecolor{currentstroke}{rgb}{0.500000,0.500000,0.500000}%
\pgfsetstrokecolor{currentstroke}%
\pgfsetstrokeopacity{0.300000}%
\pgfsetdash{}{0pt}%
\pgfpathmoveto{\pgfqpoint{0.000000in}{0.000000in}}%
\pgfpathlineto{\pgfqpoint{0.000000in}{0.000000in}}%
\pgfpathclose%
\pgfusepath{stroke,fill}%
\end{pgfscope}%
\begin{pgfscope}%
\pgfpathrectangle{\pgfqpoint{0.647939in}{0.492442in}}{\pgfqpoint{4.273799in}{2.331163in}}%
\pgfusepath{clip}%
\pgfsetroundcap%
\pgfsetroundjoin%
\pgfsetlinewidth{0.301125pt}%
\definecolor{currentstroke}{rgb}{0.500000,0.500000,0.500000}%
\pgfsetstrokecolor{currentstroke}%
\pgfsetstrokeopacity{0.300000}%
\pgfsetdash{}{0pt}%
\pgfpathmoveto{\pgfqpoint{1.251461in}{2.138819in}}%
\pgfusepath{stroke}%
\end{pgfscope}%
\begin{pgfscope}%
\pgfpathrectangle{\pgfqpoint{0.647939in}{0.492442in}}{\pgfqpoint{4.273799in}{2.331163in}}%
\pgfusepath{clip}%
\pgfsetroundcap%
\pgfsetroundjoin%
\definecolor{currentfill}{rgb}{0.500000,0.500000,0.500000}%
\pgfsetfillcolor{currentfill}%
\pgfsetfillopacity{0.300000}%
\pgfsetlinewidth{0.301125pt}%
\definecolor{currentstroke}{rgb}{0.500000,0.500000,0.500000}%
\pgfsetstrokecolor{currentstroke}%
\pgfsetstrokeopacity{0.300000}%
\pgfsetdash{}{0pt}%
\pgfpathmoveto{\pgfqpoint{0.000000in}{0.000000in}}%
\pgfpathlineto{\pgfqpoint{0.000000in}{0.000000in}}%
\pgfpathclose%
\pgfusepath{stroke,fill}%
\end{pgfscope}%
\begin{pgfscope}%
\pgfpathrectangle{\pgfqpoint{0.647939in}{0.492442in}}{\pgfqpoint{4.273799in}{2.331163in}}%
\pgfusepath{clip}%
\pgfsetroundcap%
\pgfsetroundjoin%
\pgfsetlinewidth{0.301125pt}%
\definecolor{currentstroke}{rgb}{0.500000,0.500000,0.500000}%
\pgfsetstrokecolor{currentstroke}%
\pgfsetstrokeopacity{0.300000}%
\pgfsetdash{}{0pt}%
\pgfpathmoveto{\pgfqpoint{1.525071in}{1.264439in}}%
\pgfusepath{stroke}%
\end{pgfscope}%
\begin{pgfscope}%
\pgfpathrectangle{\pgfqpoint{0.647939in}{0.492442in}}{\pgfqpoint{4.273799in}{2.331163in}}%
\pgfusepath{clip}%
\pgfsetroundcap%
\pgfsetroundjoin%
\definecolor{currentfill}{rgb}{0.500000,0.500000,0.500000}%
\pgfsetfillcolor{currentfill}%
\pgfsetfillopacity{0.300000}%
\pgfsetlinewidth{0.301125pt}%
\definecolor{currentstroke}{rgb}{0.500000,0.500000,0.500000}%
\pgfsetstrokecolor{currentstroke}%
\pgfsetstrokeopacity{0.300000}%
\pgfsetdash{}{0pt}%
\pgfpathmoveto{\pgfqpoint{0.000000in}{0.000000in}}%
\pgfpathlineto{\pgfqpoint{0.000000in}{0.000000in}}%
\pgfpathclose%
\pgfusepath{stroke,fill}%
\end{pgfscope}%
\begin{pgfscope}%
\pgfpathrectangle{\pgfqpoint{0.647939in}{0.492442in}}{\pgfqpoint{4.273799in}{2.331163in}}%
\pgfusepath{clip}%
\pgfsetroundcap%
\pgfsetroundjoin%
\pgfsetlinewidth{0.301125pt}%
\definecolor{currentstroke}{rgb}{0.500000,0.500000,0.500000}%
\pgfsetstrokecolor{currentstroke}%
\pgfsetstrokeopacity{0.300000}%
\pgfsetdash{}{0pt}%
\pgfpathmoveto{\pgfqpoint{2.217758in}{0.983589in}}%
\pgfusepath{stroke}%
\end{pgfscope}%
\begin{pgfscope}%
\pgfpathrectangle{\pgfqpoint{0.647939in}{0.492442in}}{\pgfqpoint{4.273799in}{2.331163in}}%
\pgfusepath{clip}%
\pgfsetroundcap%
\pgfsetroundjoin%
\definecolor{currentfill}{rgb}{0.500000,0.500000,0.500000}%
\pgfsetfillcolor{currentfill}%
\pgfsetfillopacity{0.300000}%
\pgfsetlinewidth{0.301125pt}%
\definecolor{currentstroke}{rgb}{0.500000,0.500000,0.500000}%
\pgfsetstrokecolor{currentstroke}%
\pgfsetstrokeopacity{0.300000}%
\pgfsetdash{}{0pt}%
\pgfpathmoveto{\pgfqpoint{0.000000in}{0.000000in}}%
\pgfpathlineto{\pgfqpoint{0.000000in}{0.000000in}}%
\pgfpathclose%
\pgfusepath{stroke,fill}%
\end{pgfscope}%
\begin{pgfscope}%
\pgfpathrectangle{\pgfqpoint{0.647939in}{0.492442in}}{\pgfqpoint{4.273799in}{2.331163in}}%
\pgfusepath{clip}%
\pgfsetroundcap%
\pgfsetroundjoin%
\pgfsetlinewidth{0.301125pt}%
\definecolor{currentstroke}{rgb}{0.500000,0.500000,0.500000}%
\pgfsetstrokecolor{currentstroke}%
\pgfsetstrokeopacity{0.300000}%
\pgfsetdash{}{0pt}%
\pgfpathmoveto{\pgfqpoint{4.032165in}{1.362389in}}%
\pgfusepath{stroke}%
\end{pgfscope}%
\begin{pgfscope}%
\pgfpathrectangle{\pgfqpoint{0.647939in}{0.492442in}}{\pgfqpoint{4.273799in}{2.331163in}}%
\pgfusepath{clip}%
\pgfsetroundcap%
\pgfsetroundjoin%
\definecolor{currentfill}{rgb}{0.500000,0.500000,0.500000}%
\pgfsetfillcolor{currentfill}%
\pgfsetfillopacity{0.300000}%
\pgfsetlinewidth{0.301125pt}%
\definecolor{currentstroke}{rgb}{0.500000,0.500000,0.500000}%
\pgfsetstrokecolor{currentstroke}%
\pgfsetstrokeopacity{0.300000}%
\pgfsetdash{}{0pt}%
\pgfpathmoveto{\pgfqpoint{0.000000in}{0.000000in}}%
\pgfpathlineto{\pgfqpoint{0.000000in}{0.000000in}}%
\pgfpathclose%
\pgfusepath{stroke,fill}%
\end{pgfscope}%
\begin{pgfscope}%
\pgfpathrectangle{\pgfqpoint{0.647939in}{0.492442in}}{\pgfqpoint{4.273799in}{2.331163in}}%
\pgfusepath{clip}%
\pgfsetroundcap%
\pgfsetroundjoin%
\pgfsetlinewidth{0.301125pt}%
\definecolor{currentstroke}{rgb}{0.500000,0.500000,0.500000}%
\pgfsetstrokecolor{currentstroke}%
\pgfsetstrokeopacity{0.300000}%
\pgfsetdash{}{0pt}%
\pgfpathmoveto{\pgfqpoint{4.091466in}{1.617872in}}%
\pgfusepath{stroke}%
\end{pgfscope}%
\begin{pgfscope}%
\pgfpathrectangle{\pgfqpoint{0.647939in}{0.492442in}}{\pgfqpoint{4.273799in}{2.331163in}}%
\pgfusepath{clip}%
\pgfsetroundcap%
\pgfsetroundjoin%
\definecolor{currentfill}{rgb}{0.500000,0.500000,0.500000}%
\pgfsetfillcolor{currentfill}%
\pgfsetfillopacity{0.300000}%
\pgfsetlinewidth{0.301125pt}%
\definecolor{currentstroke}{rgb}{0.500000,0.500000,0.500000}%
\pgfsetstrokecolor{currentstroke}%
\pgfsetstrokeopacity{0.300000}%
\pgfsetdash{}{0pt}%
\pgfpathmoveto{\pgfqpoint{0.000000in}{0.000000in}}%
\pgfpathlineto{\pgfqpoint{0.000000in}{0.000000in}}%
\pgfpathclose%
\pgfusepath{stroke,fill}%
\end{pgfscope}%
\begin{pgfscope}%
\pgfpathrectangle{\pgfqpoint{0.647939in}{0.492442in}}{\pgfqpoint{4.273799in}{2.331163in}}%
\pgfusepath{clip}%
\pgfsetroundcap%
\pgfsetroundjoin%
\pgfsetlinewidth{0.301125pt}%
\definecolor{currentstroke}{rgb}{0.500000,0.500000,0.500000}%
\pgfsetstrokecolor{currentstroke}%
\pgfsetstrokeopacity{0.300000}%
\pgfsetdash{}{0pt}%
\pgfpathmoveto{\pgfqpoint{4.241816in}{1.763986in}}%
\pgfusepath{stroke}%
\end{pgfscope}%
\begin{pgfscope}%
\pgfpathrectangle{\pgfqpoint{0.647939in}{0.492442in}}{\pgfqpoint{4.273799in}{2.331163in}}%
\pgfusepath{clip}%
\pgfsetroundcap%
\pgfsetroundjoin%
\definecolor{currentfill}{rgb}{0.500000,0.500000,0.500000}%
\pgfsetfillcolor{currentfill}%
\pgfsetfillopacity{0.300000}%
\pgfsetlinewidth{0.301125pt}%
\definecolor{currentstroke}{rgb}{0.500000,0.500000,0.500000}%
\pgfsetstrokecolor{currentstroke}%
\pgfsetstrokeopacity{0.300000}%
\pgfsetdash{}{0pt}%
\pgfpathmoveto{\pgfqpoint{0.000000in}{0.000000in}}%
\pgfpathlineto{\pgfqpoint{0.000000in}{0.000000in}}%
\pgfpathclose%
\pgfusepath{stroke,fill}%
\end{pgfscope}%
\begin{pgfscope}%
\pgfpathrectangle{\pgfqpoint{0.647939in}{0.492442in}}{\pgfqpoint{4.273799in}{2.331163in}}%
\pgfusepath{clip}%
\pgfsetroundcap%
\pgfsetroundjoin%
\pgfsetlinewidth{0.301125pt}%
\definecolor{currentstroke}{rgb}{0.500000,0.500000,0.500000}%
\pgfsetstrokecolor{currentstroke}%
\pgfsetstrokeopacity{0.300000}%
\pgfsetdash{}{0pt}%
\pgfpathmoveto{\pgfqpoint{3.417963in}{2.352852in}}%
\pgfusepath{stroke}%
\end{pgfscope}%
\begin{pgfscope}%
\pgfpathrectangle{\pgfqpoint{0.647939in}{0.492442in}}{\pgfqpoint{4.273799in}{2.331163in}}%
\pgfusepath{clip}%
\pgfsetroundcap%
\pgfsetroundjoin%
\definecolor{currentfill}{rgb}{0.500000,0.500000,0.500000}%
\pgfsetfillcolor{currentfill}%
\pgfsetfillopacity{0.300000}%
\pgfsetlinewidth{0.301125pt}%
\definecolor{currentstroke}{rgb}{0.500000,0.500000,0.500000}%
\pgfsetstrokecolor{currentstroke}%
\pgfsetstrokeopacity{0.300000}%
\pgfsetdash{}{0pt}%
\pgfpathmoveto{\pgfqpoint{0.000000in}{0.000000in}}%
\pgfpathlineto{\pgfqpoint{0.000000in}{0.000000in}}%
\pgfpathclose%
\pgfusepath{stroke,fill}%
\end{pgfscope}%
\begin{pgfscope}%
\pgfpathrectangle{\pgfqpoint{0.647939in}{0.492442in}}{\pgfqpoint{4.273799in}{2.331163in}}%
\pgfusepath{clip}%
\pgfsetroundcap%
\pgfsetroundjoin%
\pgfsetlinewidth{0.301125pt}%
\definecolor{currentstroke}{rgb}{0.500000,0.500000,0.500000}%
\pgfsetstrokecolor{currentstroke}%
\pgfsetstrokeopacity{0.300000}%
\pgfsetdash{}{0pt}%
\pgfpathmoveto{\pgfqpoint{1.541224in}{1.483515in}}%
\pgfusepath{stroke}%
\end{pgfscope}%
\begin{pgfscope}%
\pgfpathrectangle{\pgfqpoint{0.647939in}{0.492442in}}{\pgfqpoint{4.273799in}{2.331163in}}%
\pgfusepath{clip}%
\pgfsetroundcap%
\pgfsetroundjoin%
\definecolor{currentfill}{rgb}{0.500000,0.500000,0.500000}%
\pgfsetfillcolor{currentfill}%
\pgfsetfillopacity{0.300000}%
\pgfsetlinewidth{0.301125pt}%
\definecolor{currentstroke}{rgb}{0.500000,0.500000,0.500000}%
\pgfsetstrokecolor{currentstroke}%
\pgfsetstrokeopacity{0.300000}%
\pgfsetdash{}{0pt}%
\pgfpathmoveto{\pgfqpoint{0.000000in}{0.000000in}}%
\pgfpathlineto{\pgfqpoint{0.000000in}{0.000000in}}%
\pgfpathclose%
\pgfusepath{stroke,fill}%
\end{pgfscope}%
\begin{pgfscope}%
\pgfpathrectangle{\pgfqpoint{0.647939in}{0.492442in}}{\pgfqpoint{4.273799in}{2.331163in}}%
\pgfusepath{clip}%
\pgfsetroundcap%
\pgfsetroundjoin%
\pgfsetlinewidth{0.301125pt}%
\definecolor{currentstroke}{rgb}{0.500000,0.500000,0.500000}%
\pgfsetstrokecolor{currentstroke}%
\pgfsetstrokeopacity{0.300000}%
\pgfsetdash{}{0pt}%
\pgfpathmoveto{\pgfqpoint{3.828177in}{1.042089in}}%
\pgfusepath{stroke}%
\end{pgfscope}%
\begin{pgfscope}%
\pgfpathrectangle{\pgfqpoint{0.647939in}{0.492442in}}{\pgfqpoint{4.273799in}{2.331163in}}%
\pgfusepath{clip}%
\pgfsetroundcap%
\pgfsetroundjoin%
\definecolor{currentfill}{rgb}{0.500000,0.500000,0.500000}%
\pgfsetfillcolor{currentfill}%
\pgfsetfillopacity{0.300000}%
\pgfsetlinewidth{0.301125pt}%
\definecolor{currentstroke}{rgb}{0.500000,0.500000,0.500000}%
\pgfsetstrokecolor{currentstroke}%
\pgfsetstrokeopacity{0.300000}%
\pgfsetdash{}{0pt}%
\pgfpathmoveto{\pgfqpoint{0.000000in}{0.000000in}}%
\pgfpathlineto{\pgfqpoint{0.000000in}{0.000000in}}%
\pgfpathclose%
\pgfusepath{stroke,fill}%
\end{pgfscope}%
\begin{pgfscope}%
\pgfpathrectangle{\pgfqpoint{0.647939in}{0.492442in}}{\pgfqpoint{4.273799in}{2.331163in}}%
\pgfusepath{clip}%
\pgfsetroundcap%
\pgfsetroundjoin%
\pgfsetlinewidth{0.301125pt}%
\definecolor{currentstroke}{rgb}{0.500000,0.500000,0.500000}%
\pgfsetstrokecolor{currentstroke}%
\pgfsetstrokeopacity{0.300000}%
\pgfsetdash{}{0pt}%
\pgfpathmoveto{\pgfqpoint{3.843562in}{1.278907in}}%
\pgfusepath{stroke}%
\end{pgfscope}%
\begin{pgfscope}%
\pgfpathrectangle{\pgfqpoint{0.647939in}{0.492442in}}{\pgfqpoint{4.273799in}{2.331163in}}%
\pgfusepath{clip}%
\pgfsetroundcap%
\pgfsetroundjoin%
\definecolor{currentfill}{rgb}{0.500000,0.500000,0.500000}%
\pgfsetfillcolor{currentfill}%
\pgfsetfillopacity{0.300000}%
\pgfsetlinewidth{0.301125pt}%
\definecolor{currentstroke}{rgb}{0.500000,0.500000,0.500000}%
\pgfsetstrokecolor{currentstroke}%
\pgfsetstrokeopacity{0.300000}%
\pgfsetdash{}{0pt}%
\pgfpathmoveto{\pgfqpoint{0.000000in}{0.000000in}}%
\pgfpathlineto{\pgfqpoint{0.000000in}{0.000000in}}%
\pgfpathclose%
\pgfusepath{stroke,fill}%
\end{pgfscope}%
\begin{pgfscope}%
\pgfpathrectangle{\pgfqpoint{0.647939in}{0.492442in}}{\pgfqpoint{4.273799in}{2.331163in}}%
\pgfusepath{clip}%
\pgfsetroundcap%
\pgfsetroundjoin%
\pgfsetlinewidth{0.301125pt}%
\definecolor{currentstroke}{rgb}{0.500000,0.500000,0.500000}%
\pgfsetstrokecolor{currentstroke}%
\pgfsetstrokeopacity{0.300000}%
\pgfsetdash{}{0pt}%
\pgfpathmoveto{\pgfqpoint{2.160848in}{1.615142in}}%
\pgfusepath{stroke}%
\end{pgfscope}%
\begin{pgfscope}%
\pgfpathrectangle{\pgfqpoint{0.647939in}{0.492442in}}{\pgfqpoint{4.273799in}{2.331163in}}%
\pgfusepath{clip}%
\pgfsetroundcap%
\pgfsetroundjoin%
\definecolor{currentfill}{rgb}{0.500000,0.500000,0.500000}%
\pgfsetfillcolor{currentfill}%
\pgfsetfillopacity{0.300000}%
\pgfsetlinewidth{0.301125pt}%
\definecolor{currentstroke}{rgb}{0.500000,0.500000,0.500000}%
\pgfsetstrokecolor{currentstroke}%
\pgfsetstrokeopacity{0.300000}%
\pgfsetdash{}{0pt}%
\pgfpathmoveto{\pgfqpoint{0.000000in}{0.000000in}}%
\pgfpathlineto{\pgfqpoint{0.000000in}{0.000000in}}%
\pgfpathclose%
\pgfusepath{stroke,fill}%
\end{pgfscope}%
\begin{pgfscope}%
\pgfpathrectangle{\pgfqpoint{0.647939in}{0.492442in}}{\pgfqpoint{4.273799in}{2.331163in}}%
\pgfusepath{clip}%
\pgfsetroundcap%
\pgfsetroundjoin%
\pgfsetlinewidth{0.301125pt}%
\definecolor{currentstroke}{rgb}{0.500000,0.500000,0.500000}%
\pgfsetstrokecolor{currentstroke}%
\pgfsetstrokeopacity{0.300000}%
\pgfsetdash{}{0pt}%
\pgfpathmoveto{\pgfqpoint{3.958936in}{1.073497in}}%
\pgfusepath{stroke}%
\end{pgfscope}%
\begin{pgfscope}%
\pgfpathrectangle{\pgfqpoint{0.647939in}{0.492442in}}{\pgfqpoint{4.273799in}{2.331163in}}%
\pgfusepath{clip}%
\pgfsetroundcap%
\pgfsetroundjoin%
\definecolor{currentfill}{rgb}{0.500000,0.500000,0.500000}%
\pgfsetfillcolor{currentfill}%
\pgfsetfillopacity{0.300000}%
\pgfsetlinewidth{0.301125pt}%
\definecolor{currentstroke}{rgb}{0.500000,0.500000,0.500000}%
\pgfsetstrokecolor{currentstroke}%
\pgfsetstrokeopacity{0.300000}%
\pgfsetdash{}{0pt}%
\pgfpathmoveto{\pgfqpoint{0.000000in}{0.000000in}}%
\pgfpathlineto{\pgfqpoint{0.000000in}{0.000000in}}%
\pgfpathclose%
\pgfusepath{stroke,fill}%
\end{pgfscope}%
\begin{pgfscope}%
\pgfpathrectangle{\pgfqpoint{0.647939in}{0.492442in}}{\pgfqpoint{4.273799in}{2.331163in}}%
\pgfusepath{clip}%
\pgfsetroundcap%
\pgfsetroundjoin%
\pgfsetlinewidth{0.301125pt}%
\definecolor{currentstroke}{rgb}{0.500000,0.500000,0.500000}%
\pgfsetstrokecolor{currentstroke}%
\pgfsetstrokeopacity{0.300000}%
\pgfsetdash{}{0pt}%
\pgfpathmoveto{\pgfqpoint{2.726076in}{2.360951in}}%
\pgfusepath{stroke}%
\end{pgfscope}%
\begin{pgfscope}%
\pgfpathrectangle{\pgfqpoint{0.647939in}{0.492442in}}{\pgfqpoint{4.273799in}{2.331163in}}%
\pgfusepath{clip}%
\pgfsetroundcap%
\pgfsetroundjoin%
\definecolor{currentfill}{rgb}{0.500000,0.500000,0.500000}%
\pgfsetfillcolor{currentfill}%
\pgfsetfillopacity{0.300000}%
\pgfsetlinewidth{0.301125pt}%
\definecolor{currentstroke}{rgb}{0.500000,0.500000,0.500000}%
\pgfsetstrokecolor{currentstroke}%
\pgfsetstrokeopacity{0.300000}%
\pgfsetdash{}{0pt}%
\pgfpathmoveto{\pgfqpoint{0.000000in}{0.000000in}}%
\pgfpathlineto{\pgfqpoint{0.000000in}{0.000000in}}%
\pgfpathclose%
\pgfusepath{stroke,fill}%
\end{pgfscope}%
\begin{pgfscope}%
\pgfpathrectangle{\pgfqpoint{0.647939in}{0.492442in}}{\pgfqpoint{4.273799in}{2.331163in}}%
\pgfusepath{clip}%
\pgfsetroundcap%
\pgfsetroundjoin%
\pgfsetlinewidth{0.301125pt}%
\definecolor{currentstroke}{rgb}{0.500000,0.500000,0.500000}%
\pgfsetstrokecolor{currentstroke}%
\pgfsetstrokeopacity{0.300000}%
\pgfsetdash{}{0pt}%
\pgfpathmoveto{\pgfqpoint{3.772210in}{2.242742in}}%
\pgfusepath{stroke}%
\end{pgfscope}%
\begin{pgfscope}%
\pgfpathrectangle{\pgfqpoint{0.647939in}{0.492442in}}{\pgfqpoint{4.273799in}{2.331163in}}%
\pgfusepath{clip}%
\pgfsetroundcap%
\pgfsetroundjoin%
\definecolor{currentfill}{rgb}{0.500000,0.500000,0.500000}%
\pgfsetfillcolor{currentfill}%
\pgfsetfillopacity{0.300000}%
\pgfsetlinewidth{0.301125pt}%
\definecolor{currentstroke}{rgb}{0.500000,0.500000,0.500000}%
\pgfsetstrokecolor{currentstroke}%
\pgfsetstrokeopacity{0.300000}%
\pgfsetdash{}{0pt}%
\pgfpathmoveto{\pgfqpoint{0.000000in}{0.000000in}}%
\pgfpathlineto{\pgfqpoint{0.000000in}{0.000000in}}%
\pgfpathclose%
\pgfusepath{stroke,fill}%
\end{pgfscope}%
\begin{pgfscope}%
\pgfpathrectangle{\pgfqpoint{0.647939in}{0.492442in}}{\pgfqpoint{4.273799in}{2.331163in}}%
\pgfusepath{clip}%
\pgfsetroundcap%
\pgfsetroundjoin%
\pgfsetlinewidth{0.301125pt}%
\definecolor{currentstroke}{rgb}{0.500000,0.500000,0.500000}%
\pgfsetstrokecolor{currentstroke}%
\pgfsetstrokeopacity{0.300000}%
\pgfsetdash{}{0pt}%
\pgfpathmoveto{\pgfqpoint{2.449682in}{2.084725in}}%
\pgfusepath{stroke}%
\end{pgfscope}%
\begin{pgfscope}%
\pgfpathrectangle{\pgfqpoint{0.647939in}{0.492442in}}{\pgfqpoint{4.273799in}{2.331163in}}%
\pgfusepath{clip}%
\pgfsetroundcap%
\pgfsetroundjoin%
\definecolor{currentfill}{rgb}{0.500000,0.500000,0.500000}%
\pgfsetfillcolor{currentfill}%
\pgfsetfillopacity{0.300000}%
\pgfsetlinewidth{0.301125pt}%
\definecolor{currentstroke}{rgb}{0.500000,0.500000,0.500000}%
\pgfsetstrokecolor{currentstroke}%
\pgfsetstrokeopacity{0.300000}%
\pgfsetdash{}{0pt}%
\pgfpathmoveto{\pgfqpoint{0.000000in}{0.000000in}}%
\pgfpathlineto{\pgfqpoint{0.000000in}{0.000000in}}%
\pgfpathclose%
\pgfusepath{stroke,fill}%
\end{pgfscope}%
\begin{pgfscope}%
\pgfpathrectangle{\pgfqpoint{0.647939in}{0.492442in}}{\pgfqpoint{4.273799in}{2.331163in}}%
\pgfusepath{clip}%
\pgfsetroundcap%
\pgfsetroundjoin%
\pgfsetlinewidth{0.301125pt}%
\definecolor{currentstroke}{rgb}{0.500000,0.500000,0.500000}%
\pgfsetstrokecolor{currentstroke}%
\pgfsetstrokeopacity{0.300000}%
\pgfsetdash{}{0pt}%
\pgfpathmoveto{\pgfqpoint{1.835554in}{1.632442in}}%
\pgfusepath{stroke}%
\end{pgfscope}%
\begin{pgfscope}%
\pgfpathrectangle{\pgfqpoint{0.647939in}{0.492442in}}{\pgfqpoint{4.273799in}{2.331163in}}%
\pgfusepath{clip}%
\pgfsetroundcap%
\pgfsetroundjoin%
\definecolor{currentfill}{rgb}{0.500000,0.500000,0.500000}%
\pgfsetfillcolor{currentfill}%
\pgfsetfillopacity{0.300000}%
\pgfsetlinewidth{0.301125pt}%
\definecolor{currentstroke}{rgb}{0.500000,0.500000,0.500000}%
\pgfsetstrokecolor{currentstroke}%
\pgfsetstrokeopacity{0.300000}%
\pgfsetdash{}{0pt}%
\pgfpathmoveto{\pgfqpoint{0.000000in}{0.000000in}}%
\pgfpathlineto{\pgfqpoint{0.000000in}{0.000000in}}%
\pgfpathclose%
\pgfusepath{stroke,fill}%
\end{pgfscope}%
\begin{pgfscope}%
\pgfpathrectangle{\pgfqpoint{0.647939in}{0.492442in}}{\pgfqpoint{4.273799in}{2.331163in}}%
\pgfusepath{clip}%
\pgfsetroundcap%
\pgfsetroundjoin%
\pgfsetlinewidth{0.301125pt}%
\definecolor{currentstroke}{rgb}{0.500000,0.500000,0.500000}%
\pgfsetstrokecolor{currentstroke}%
\pgfsetstrokeopacity{0.300000}%
\pgfsetdash{}{0pt}%
\pgfpathmoveto{\pgfqpoint{1.978901in}{1.832120in}}%
\pgfusepath{stroke}%
\end{pgfscope}%
\begin{pgfscope}%
\pgfpathrectangle{\pgfqpoint{0.647939in}{0.492442in}}{\pgfqpoint{4.273799in}{2.331163in}}%
\pgfusepath{clip}%
\pgfsetroundcap%
\pgfsetroundjoin%
\definecolor{currentfill}{rgb}{0.500000,0.500000,0.500000}%
\pgfsetfillcolor{currentfill}%
\pgfsetfillopacity{0.300000}%
\pgfsetlinewidth{0.301125pt}%
\definecolor{currentstroke}{rgb}{0.500000,0.500000,0.500000}%
\pgfsetstrokecolor{currentstroke}%
\pgfsetstrokeopacity{0.300000}%
\pgfsetdash{}{0pt}%
\pgfpathmoveto{\pgfqpoint{0.000000in}{0.000000in}}%
\pgfpathlineto{\pgfqpoint{0.000000in}{0.000000in}}%
\pgfpathclose%
\pgfusepath{stroke,fill}%
\end{pgfscope}%
\begin{pgfscope}%
\pgfpathrectangle{\pgfqpoint{0.647939in}{0.492442in}}{\pgfqpoint{4.273799in}{2.331163in}}%
\pgfusepath{clip}%
\pgfsetroundcap%
\pgfsetroundjoin%
\pgfsetlinewidth{0.301125pt}%
\definecolor{currentstroke}{rgb}{0.500000,0.500000,0.500000}%
\pgfsetstrokecolor{currentstroke}%
\pgfsetstrokeopacity{0.300000}%
\pgfsetdash{}{0pt}%
\pgfpathmoveto{\pgfqpoint{3.474993in}{1.813017in}}%
\pgfusepath{stroke}%
\end{pgfscope}%
\begin{pgfscope}%
\pgfpathrectangle{\pgfqpoint{0.647939in}{0.492442in}}{\pgfqpoint{4.273799in}{2.331163in}}%
\pgfusepath{clip}%
\pgfsetroundcap%
\pgfsetroundjoin%
\definecolor{currentfill}{rgb}{0.500000,0.500000,0.500000}%
\pgfsetfillcolor{currentfill}%
\pgfsetfillopacity{0.300000}%
\pgfsetlinewidth{0.301125pt}%
\definecolor{currentstroke}{rgb}{0.500000,0.500000,0.500000}%
\pgfsetstrokecolor{currentstroke}%
\pgfsetstrokeopacity{0.300000}%
\pgfsetdash{}{0pt}%
\pgfpathmoveto{\pgfqpoint{0.000000in}{0.000000in}}%
\pgfpathlineto{\pgfqpoint{0.000000in}{0.000000in}}%
\pgfpathclose%
\pgfusepath{stroke,fill}%
\end{pgfscope}%
\begin{pgfscope}%
\pgfpathrectangle{\pgfqpoint{0.647939in}{0.492442in}}{\pgfqpoint{4.273799in}{2.331163in}}%
\pgfusepath{clip}%
\pgfsetroundcap%
\pgfsetroundjoin%
\pgfsetlinewidth{0.301125pt}%
\definecolor{currentstroke}{rgb}{0.500000,0.500000,0.500000}%
\pgfsetstrokecolor{currentstroke}%
\pgfsetstrokeopacity{0.300000}%
\pgfsetdash{}{0pt}%
\pgfpathmoveto{\pgfqpoint{3.276726in}{1.980199in}}%
\pgfusepath{stroke}%
\end{pgfscope}%
\begin{pgfscope}%
\pgfpathrectangle{\pgfqpoint{0.647939in}{0.492442in}}{\pgfqpoint{4.273799in}{2.331163in}}%
\pgfusepath{clip}%
\pgfsetroundcap%
\pgfsetroundjoin%
\definecolor{currentfill}{rgb}{0.500000,0.500000,0.500000}%
\pgfsetfillcolor{currentfill}%
\pgfsetfillopacity{0.300000}%
\pgfsetlinewidth{0.301125pt}%
\definecolor{currentstroke}{rgb}{0.500000,0.500000,0.500000}%
\pgfsetstrokecolor{currentstroke}%
\pgfsetstrokeopacity{0.300000}%
\pgfsetdash{}{0pt}%
\pgfpathmoveto{\pgfqpoint{0.000000in}{0.000000in}}%
\pgfpathlineto{\pgfqpoint{0.000000in}{0.000000in}}%
\pgfpathclose%
\pgfusepath{stroke,fill}%
\end{pgfscope}%
\begin{pgfscope}%
\pgfpathrectangle{\pgfqpoint{0.647939in}{0.492442in}}{\pgfqpoint{4.273799in}{2.331163in}}%
\pgfusepath{clip}%
\pgfsetroundcap%
\pgfsetroundjoin%
\pgfsetlinewidth{0.301125pt}%
\definecolor{currentstroke}{rgb}{0.500000,0.500000,0.500000}%
\pgfsetstrokecolor{currentstroke}%
\pgfsetstrokeopacity{0.300000}%
\pgfsetdash{}{0pt}%
\pgfpathmoveto{\pgfqpoint{2.682145in}{1.929263in}}%
\pgfusepath{stroke}%
\end{pgfscope}%
\begin{pgfscope}%
\pgfpathrectangle{\pgfqpoint{0.647939in}{0.492442in}}{\pgfqpoint{4.273799in}{2.331163in}}%
\pgfusepath{clip}%
\pgfsetroundcap%
\pgfsetroundjoin%
\definecolor{currentfill}{rgb}{0.500000,0.500000,0.500000}%
\pgfsetfillcolor{currentfill}%
\pgfsetfillopacity{0.300000}%
\pgfsetlinewidth{0.301125pt}%
\definecolor{currentstroke}{rgb}{0.500000,0.500000,0.500000}%
\pgfsetstrokecolor{currentstroke}%
\pgfsetstrokeopacity{0.300000}%
\pgfsetdash{}{0pt}%
\pgfpathmoveto{\pgfqpoint{0.000000in}{0.000000in}}%
\pgfpathlineto{\pgfqpoint{0.000000in}{0.000000in}}%
\pgfpathclose%
\pgfusepath{stroke,fill}%
\end{pgfscope}%
\begin{pgfscope}%
\pgfpathrectangle{\pgfqpoint{0.647939in}{0.492442in}}{\pgfqpoint{4.273799in}{2.331163in}}%
\pgfusepath{clip}%
\pgfsetroundcap%
\pgfsetroundjoin%
\pgfsetlinewidth{0.301125pt}%
\definecolor{currentstroke}{rgb}{0.500000,0.500000,0.500000}%
\pgfsetstrokecolor{currentstroke}%
\pgfsetstrokeopacity{0.300000}%
\pgfsetdash{}{0pt}%
\pgfpathmoveto{\pgfqpoint{3.047424in}{1.445056in}}%
\pgfusepath{stroke}%
\end{pgfscope}%
\begin{pgfscope}%
\pgfpathrectangle{\pgfqpoint{0.647939in}{0.492442in}}{\pgfqpoint{4.273799in}{2.331163in}}%
\pgfusepath{clip}%
\pgfsetroundcap%
\pgfsetroundjoin%
\definecolor{currentfill}{rgb}{0.500000,0.500000,0.500000}%
\pgfsetfillcolor{currentfill}%
\pgfsetfillopacity{0.300000}%
\pgfsetlinewidth{0.301125pt}%
\definecolor{currentstroke}{rgb}{0.500000,0.500000,0.500000}%
\pgfsetstrokecolor{currentstroke}%
\pgfsetstrokeopacity{0.300000}%
\pgfsetdash{}{0pt}%
\pgfpathmoveto{\pgfqpoint{0.000000in}{0.000000in}}%
\pgfpathlineto{\pgfqpoint{0.000000in}{0.000000in}}%
\pgfpathclose%
\pgfusepath{stroke,fill}%
\end{pgfscope}%
\begin{pgfscope}%
\pgfpathrectangle{\pgfqpoint{0.647939in}{0.492442in}}{\pgfqpoint{4.273799in}{2.331163in}}%
\pgfusepath{clip}%
\pgfsetbuttcap%
\pgfsetroundjoin%
\pgfsetlinewidth{0.301125pt}%
\definecolor{currentstroke}{rgb}{0.500000,0.500000,0.500000}%
\pgfsetstrokecolor{currentstroke}%
\pgfsetstrokeopacity{0.300000}%
\pgfsetdash{}{0pt}%
\pgfpathmoveto{\pgfqpoint{2.246901in}{0.492442in}}%
\pgfpathlineto{\pgfqpoint{2.223373in}{0.524915in}}%
\pgfpathlineto{\pgfqpoint{2.188608in}{0.573122in}}%
\pgfpathlineto{\pgfqpoint{2.154108in}{0.621387in}}%
\pgfpathlineto{\pgfqpoint{2.119841in}{0.669701in}}%
\pgfpathlineto{\pgfqpoint{2.085764in}{0.718055in}}%
\pgfpathlineto{\pgfqpoint{2.051834in}{0.766440in}}%
\pgfpathlineto{\pgfqpoint{2.018009in}{0.814847in}}%
\pgfpathlineto{\pgfqpoint{1.984241in}{0.863265in}}%
\pgfpathlineto{\pgfqpoint{1.950462in}{0.911681in}}%
\pgfpathlineto{\pgfqpoint{1.916590in}{0.960078in}}%
\pgfpathlineto{\pgfqpoint{1.882538in}{1.008437in}}%
\pgfpathlineto{\pgfqpoint{1.848196in}{1.056734in}}%
\pgfpathlineto{\pgfqpoint{1.813425in}{1.104941in}}%
\pgfpathlineto{\pgfqpoint{1.778025in}{1.153010in}}%
\pgfpathlineto{\pgfqpoint{1.741721in}{1.200876in}}%
\pgfpathlineto{\pgfqpoint{1.704136in}{1.248444in}}%
\pgfpathlineto{\pgfqpoint{1.664699in}{1.295558in}}%
\pgfpathlineto{\pgfqpoint{1.622473in}{1.341933in}}%
\pgfpathlineto{\pgfqpoint{1.575766in}{1.386968in}}%
\pgfpathlineto{\pgfqpoint{1.529892in}{1.423244in}}%
\pgfpathlineto{\pgfqpoint{1.489907in}{1.447222in}}%
\pgfpathlineto{\pgfqpoint{1.452689in}{1.462273in}}%
\pgfpathlineto{\pgfqpoint{1.411015in}{1.470094in}}%
\pgfpathlineto{\pgfqpoint{1.367529in}{1.467760in}}%
\pgfpathlineto{\pgfqpoint{1.367529in}{1.467760in}}%
\pgfpathlineto{\pgfqpoint{1.319870in}{1.453051in}}%
\pgfpathlineto{\pgfqpoint{1.319870in}{1.453051in}}%
\pgfpathlineto{\pgfqpoint{1.256366in}{1.415313in}}%
\pgfpathlineto{\pgfqpoint{1.204848in}{1.371896in}}%
\pgfpathlineto{\pgfqpoint{1.160128in}{1.326267in}}%
\pgfpathlineto{\pgfqpoint{1.119593in}{1.279457in}}%
\pgfpathlineto{\pgfqpoint{1.081937in}{1.231920in}}%
\pgfpathlineto{\pgfqpoint{1.046427in}{1.183892in}}%
\pgfpathlineto{\pgfqpoint{1.012596in}{1.135513in}}%
\pgfpathlineto{\pgfqpoint{0.980103in}{1.086857in}}%
\pgfpathlineto{\pgfqpoint{0.948704in}{1.037977in}}%
\pgfpathlineto{\pgfqpoint{0.918255in}{0.988917in}}%
\pgfpathlineto{\pgfqpoint{0.888643in}{0.939708in}}%
\pgfpathlineto{\pgfqpoint{0.859753in}{0.890368in}}%
\pgfpathlineto{\pgfqpoint{0.831511in}{0.840914in}}%
\pgfpathlineto{\pgfqpoint{0.803865in}{0.791363in}}%
\pgfpathlineto{\pgfqpoint{0.776748in}{0.741720in}}%
\pgfpathlineto{\pgfqpoint{0.750127in}{0.691998in}}%
\pgfpathlineto{\pgfqpoint{0.723969in}{0.642204in}}%
\pgfpathlineto{\pgfqpoint{0.698235in}{0.592343in}}%
\pgfpathlineto{\pgfqpoint{0.672901in}{0.542421in}}%
\pgfpathlineto{\pgfqpoint{0.647939in}{0.492442in}}%
\pgfpathlineto{\pgfqpoint{0.647939in}{0.492442in}}%
\pgfusepath{stroke}%
\end{pgfscope}%
\begin{pgfscope}%
\pgfpathrectangle{\pgfqpoint{0.647939in}{0.492442in}}{\pgfqpoint{4.273799in}{2.331163in}}%
\pgfusepath{clip}%
\pgfsetbuttcap%
\pgfsetroundjoin%
\pgfsetlinewidth{0.301125pt}%
\definecolor{currentstroke}{rgb}{0.500000,0.500000,0.500000}%
\pgfsetstrokecolor{currentstroke}%
\pgfsetstrokeopacity{0.300000}%
\pgfsetdash{}{0pt}%
\pgfpathmoveto{\pgfqpoint{1.792992in}{0.492442in}}%
\pgfpathlineto{\pgfqpoint{1.756555in}{0.535376in}}%
\pgfpathlineto{\pgfqpoint{1.716125in}{0.582248in}}%
\pgfpathlineto{\pgfqpoint{1.674448in}{0.628794in}}%
\pgfpathlineto{\pgfqpoint{1.631129in}{0.674888in}}%
\pgfpathlineto{\pgfqpoint{1.585604in}{0.720340in}}%
\pgfpathlineto{\pgfqpoint{1.537034in}{0.764831in}}%
\pgfpathlineto{\pgfqpoint{1.484101in}{0.807785in}}%
\pgfpathlineto{\pgfqpoint{1.425285in}{0.847615in}}%
\pgfpathlineto{\pgfqpoint{1.372692in}{0.875203in}}%
\pgfpathlineto{\pgfqpoint{1.323450in}{0.893354in}}%
\pgfpathlineto{\pgfqpoint{1.272722in}{0.903625in}}%
\pgfpathlineto{\pgfqpoint{1.212774in}{0.904093in}}%
\pgfpathlineto{\pgfqpoint{1.156258in}{0.892635in}}%
\pgfpathlineto{\pgfqpoint{1.156258in}{0.892635in}}%
\pgfpathlineto{\pgfqpoint{1.083189in}{0.860474in}}%
\pgfpathlineto{\pgfqpoint{1.023456in}{0.820402in}}%
\pgfpathlineto{\pgfqpoint{0.972436in}{0.776835in}}%
\pgfpathlineto{\pgfqpoint{0.927127in}{0.731380in}}%
\pgfpathlineto{\pgfqpoint{0.885855in}{0.684769in}}%
\pgfpathlineto{\pgfqpoint{0.847590in}{0.637388in}}%
\pgfpathlineto{\pgfqpoint{0.811659in}{0.589455in}}%
\pgfpathlineto{\pgfqpoint{0.777590in}{0.541108in}}%
\pgfpathlineto{\pgfqpoint{0.745071in}{0.492442in}}%
\pgfpathlineto{\pgfqpoint{0.745071in}{0.492442in}}%
\pgfusepath{stroke}%
\end{pgfscope}%
\begin{pgfscope}%
\pgfpathrectangle{\pgfqpoint{0.647939in}{0.492442in}}{\pgfqpoint{4.273799in}{2.331163in}}%
\pgfusepath{clip}%
\pgfsetbuttcap%
\pgfsetroundjoin%
\pgfsetlinewidth{0.301125pt}%
\definecolor{currentstroke}{rgb}{0.500000,0.500000,0.500000}%
\pgfsetstrokecolor{currentstroke}%
\pgfsetstrokeopacity{0.300000}%
\pgfsetdash{}{0pt}%
\pgfpathmoveto{\pgfqpoint{1.564086in}{0.492442in}}%
\pgfpathlineto{\pgfqpoint{1.513516in}{0.535498in}}%
\pgfpathlineto{\pgfqpoint{1.459988in}{0.578248in}}%
\pgfpathlineto{\pgfqpoint{1.400627in}{0.618573in}}%
\pgfpathlineto{\pgfqpoint{1.343410in}{0.649452in}}%
\pgfpathlineto{\pgfqpoint{1.290522in}{0.670230in}}%
\pgfpathlineto{\pgfqpoint{1.237887in}{0.682821in}}%
\pgfpathlineto{\pgfqpoint{1.179404in}{0.686687in}}%
\pgfpathlineto{\pgfqpoint{1.121664in}{0.679650in}}%
\pgfpathlineto{\pgfqpoint{1.121664in}{0.679650in}}%
\pgfpathlineto{\pgfqpoint{1.059915in}{0.659844in}}%
\pgfpathlineto{\pgfqpoint{1.059915in}{0.659844in}}%
\pgfpathlineto{\pgfqpoint{0.992482in}{0.623816in}}%
\pgfpathlineto{\pgfqpoint{0.935880in}{0.582368in}}%
\pgfpathlineto{\pgfqpoint{0.886533in}{0.538200in}}%
\pgfpathlineto{\pgfqpoint{0.842203in}{0.492442in}}%
\pgfpathlineto{\pgfqpoint{0.842203in}{0.492442in}}%
\pgfusepath{stroke}%
\end{pgfscope}%
\begin{pgfscope}%
\pgfpathrectangle{\pgfqpoint{0.647939in}{0.492442in}}{\pgfqpoint{4.273799in}{2.331163in}}%
\pgfusepath{clip}%
\pgfsetbuttcap%
\pgfsetroundjoin%
\pgfsetlinewidth{0.301125pt}%
\definecolor{currentstroke}{rgb}{0.500000,0.500000,0.500000}%
\pgfsetstrokecolor{currentstroke}%
\pgfsetstrokeopacity{0.300000}%
\pgfsetdash{}{0pt}%
\pgfpathmoveto{\pgfqpoint{1.396803in}{0.492442in}}%
\pgfpathlineto{\pgfqpoint{1.367474in}{0.509187in}}%
\pgfpathlineto{\pgfqpoint{1.309860in}{0.537906in}}%
\pgfpathlineto{\pgfqpoint{1.255932in}{0.557004in}}%
\pgfpathlineto{\pgfqpoint{1.201194in}{0.568031in}}%
\pgfpathlineto{\pgfqpoint{1.139287in}{0.569674in}}%
\pgfpathlineto{\pgfqpoint{1.079824in}{0.560031in}}%
\pgfpathlineto{\pgfqpoint{1.079824in}{0.560031in}}%
\pgfpathlineto{\pgfqpoint{1.002860in}{0.530682in}}%
\pgfpathlineto{\pgfqpoint{0.939334in}{0.492442in}}%
\pgfpathlineto{\pgfqpoint{0.939334in}{0.492442in}}%
\pgfusepath{stroke}%
\end{pgfscope}%
\begin{pgfscope}%
\pgfpathrectangle{\pgfqpoint{0.647939in}{0.492442in}}{\pgfqpoint{4.273799in}{2.331163in}}%
\pgfusepath{clip}%
\pgfsetbuttcap%
\pgfsetroundjoin%
\pgfsetlinewidth{0.301125pt}%
\definecolor{currentstroke}{rgb}{0.500000,0.500000,0.500000}%
\pgfsetstrokecolor{currentstroke}%
\pgfsetstrokeopacity{0.300000}%
\pgfsetdash{}{0pt}%
\pgfpathmoveto{\pgfqpoint{1.619257in}{0.492442in}}%
\pgfpathlineto{\pgfqpoint{1.619257in}{0.492442in}}%
\pgfpathlineto{\pgfqpoint{1.573122in}{0.537712in}}%
\pgfpathlineto{\pgfqpoint{1.524164in}{0.582082in}}%
\pgfpathlineto{\pgfqpoint{1.471281in}{0.625078in}}%
\pgfpathlineto{\pgfqpoint{1.412693in}{0.665767in}}%
\pgfpathlineto{\pgfqpoint{1.345437in}{0.702063in}}%
\pgfpathlineto{\pgfqpoint{1.345437in}{0.702063in}}%
\pgfpathlineto{\pgfqpoint{1.280599in}{0.725180in}}%
\pgfpathlineto{\pgfqpoint{1.280599in}{0.725180in}}%
\pgfpathlineto{\pgfqpoint{1.222906in}{0.734947in}}%
\pgfpathlineto{\pgfqpoint{1.161918in}{0.733485in}}%
\pgfusepath{stroke}%
\end{pgfscope}%
\begin{pgfscope}%
\pgfpathrectangle{\pgfqpoint{0.647939in}{0.492442in}}{\pgfqpoint{4.273799in}{2.331163in}}%
\pgfusepath{clip}%
\pgfsetbuttcap%
\pgfsetroundjoin%
\pgfsetlinewidth{0.301125pt}%
\definecolor{currentstroke}{rgb}{0.500000,0.500000,0.500000}%
\pgfsetstrokecolor{currentstroke}%
\pgfsetstrokeopacity{0.300000}%
\pgfsetdash{}{0pt}%
\pgfpathmoveto{\pgfqpoint{1.910652in}{0.492442in}}%
\pgfpathlineto{\pgfqpoint{1.910652in}{0.492442in}}%
\pgfpathlineto{\pgfqpoint{1.873536in}{0.540125in}}%
\pgfpathlineto{\pgfqpoint{1.835998in}{0.587710in}}%
\pgfpathlineto{\pgfqpoint{1.797894in}{0.635160in}}%
\pgfpathlineto{\pgfqpoint{1.759045in}{0.682430in}}%
\pgfpathlineto{\pgfqpoint{1.719214in}{0.729455in}}%
\pgfpathlineto{\pgfqpoint{1.678085in}{0.776145in}}%
\pgfpathlineto{\pgfqpoint{1.635215in}{0.822365in}}%
\pgfpathlineto{\pgfqpoint{1.589959in}{0.867897in}}%
\pgfpathlineto{\pgfqpoint{1.541326in}{0.912366in}}%
\pgfpathlineto{\pgfqpoint{1.487688in}{0.955045in}}%
\pgfpathlineto{\pgfqpoint{1.426168in}{0.994328in}}%
\pgfpathlineto{\pgfqpoint{1.351848in}{1.025646in}}%
\pgfpathlineto{\pgfqpoint{1.351848in}{1.025646in}}%
\pgfpathlineto{\pgfqpoint{1.298570in}{1.036031in}}%
\pgfpathlineto{\pgfqpoint{1.241124in}{1.035019in}}%
\pgfpathlineto{\pgfqpoint{1.194492in}{1.024787in}}%
\pgfpathlineto{\pgfqpoint{1.150122in}{1.007238in}}%
\pgfpathlineto{\pgfqpoint{1.103585in}{0.980848in}}%
\pgfpathlineto{\pgfqpoint{1.052804in}{0.943155in}}%
\pgfpathlineto{\pgfqpoint{1.004059in}{0.898817in}}%
\pgfpathlineto{\pgfqpoint{0.960438in}{0.852862in}}%
\pgfpathlineto{\pgfqpoint{0.920467in}{0.805910in}}%
\pgfpathlineto{\pgfqpoint{0.883232in}{0.758285in}}%
\pgfusepath{stroke}%
\end{pgfscope}%
\begin{pgfscope}%
\pgfpathrectangle{\pgfqpoint{0.647939in}{0.492442in}}{\pgfqpoint{4.273799in}{2.331163in}}%
\pgfusepath{clip}%
\pgfsetbuttcap%
\pgfsetroundjoin%
\pgfsetlinewidth{0.301125pt}%
\definecolor{currentstroke}{rgb}{0.500000,0.500000,0.500000}%
\pgfsetstrokecolor{currentstroke}%
\pgfsetstrokeopacity{0.300000}%
\pgfsetdash{}{0pt}%
\pgfpathmoveto{\pgfqpoint{2.007784in}{0.492442in}}%
\pgfpathlineto{\pgfqpoint{2.007784in}{0.492442in}}%
\pgfpathlineto{\pgfqpoint{1.971884in}{0.540402in}}%
\pgfpathlineto{\pgfqpoint{1.935853in}{0.588333in}}%
\pgfpathlineto{\pgfqpoint{1.899609in}{0.636215in}}%
\pgfpathlineto{\pgfqpoint{1.863041in}{0.684025in}}%
\pgfpathlineto{\pgfqpoint{1.826017in}{0.731729in}}%
\pgfpathlineto{\pgfqpoint{1.788378in}{0.779290in}}%
\pgfpathlineto{\pgfqpoint{1.749919in}{0.826653in}}%
\pgfpathlineto{\pgfqpoint{1.710364in}{0.873746in}}%
\pgfpathlineto{\pgfqpoint{1.669336in}{0.920460in}}%
\pgfpathlineto{\pgfqpoint{1.626286in}{0.966625in}}%
\pgfpathlineto{\pgfqpoint{1.580378in}{1.011952in}}%
\pgfpathlineto{\pgfqpoint{1.530253in}{1.055903in}}%
\pgfpathlineto{\pgfqpoint{1.473502in}{1.097317in}}%
\pgfpathlineto{\pgfqpoint{1.405620in}{1.133020in}}%
\pgfpathlineto{\pgfqpoint{1.405620in}{1.133020in}}%
\pgfpathlineto{\pgfqpoint{1.352338in}{1.149047in}}%
\pgfpathlineto{\pgfqpoint{1.352338in}{1.149047in}}%
\pgfpathlineto{\pgfqpoint{1.302888in}{1.153639in}}%
\pgfpathlineto{\pgfqpoint{1.253311in}{1.147969in}}%
\pgfpathlineto{\pgfqpoint{1.210214in}{1.134733in}}%
\pgfpathlineto{\pgfqpoint{1.166993in}{1.113837in}}%
\pgfpathlineto{\pgfqpoint{1.120272in}{1.083038in}}%
\pgfpathlineto{\pgfqpoint{1.068623in}{1.039714in}}%
\pgfusepath{stroke}%
\end{pgfscope}%
\begin{pgfscope}%
\pgfpathrectangle{\pgfqpoint{0.647939in}{0.492442in}}{\pgfqpoint{4.273799in}{2.331163in}}%
\pgfusepath{clip}%
\pgfsetbuttcap%
\pgfsetroundjoin%
\pgfsetlinewidth{0.301125pt}%
\definecolor{currentstroke}{rgb}{0.500000,0.500000,0.500000}%
\pgfsetstrokecolor{currentstroke}%
\pgfsetstrokeopacity{0.300000}%
\pgfsetdash{}{0pt}%
\pgfpathmoveto{\pgfqpoint{2.104916in}{0.492442in}}%
\pgfpathlineto{\pgfqpoint{2.104916in}{0.492442in}}%
\pgfpathlineto{\pgfqpoint{2.069708in}{0.540555in}}%
\pgfpathlineto{\pgfqpoint{2.034572in}{0.588683in}}%
\pgfpathlineto{\pgfqpoint{1.999447in}{0.636813in}}%
\pgfpathlineto{\pgfqpoint{1.964270in}{0.684932in}}%
\pgfpathlineto{\pgfqpoint{1.928970in}{0.733024in}}%
\pgfpathlineto{\pgfqpoint{1.893462in}{0.781071in}}%
\pgfpathlineto{\pgfqpoint{1.857640in}{0.829048in}}%
\pgfpathlineto{\pgfqpoint{1.821362in}{0.876922in}}%
\pgfpathlineto{\pgfqpoint{1.784448in}{0.924652in}}%
\pgfpathlineto{\pgfqpoint{1.746674in}{0.972178in}}%
\pgfpathlineto{\pgfqpoint{1.707729in}{1.019422in}}%
\pgfpathlineto{\pgfqpoint{1.667173in}{1.066256in}}%
\pgfpathlineto{\pgfqpoint{1.624345in}{1.112479in}}%
\pgfpathlineto{\pgfqpoint{1.578183in}{1.157719in}}%
\pgfpathlineto{\pgfqpoint{1.526837in}{1.201220in}}%
\pgfpathlineto{\pgfqpoint{1.466730in}{1.241111in}}%
\pgfpathlineto{\pgfqpoint{1.466730in}{1.241111in}}%
\pgfpathlineto{\pgfqpoint{1.410836in}{1.265678in}}%
\pgfpathlineto{\pgfqpoint{1.410836in}{1.265678in}}%
\pgfpathlineto{\pgfqpoint{1.362225in}{1.276002in}}%
\pgfpathlineto{\pgfqpoint{1.309391in}{1.274845in}}%
\pgfpathlineto{\pgfqpoint{1.266937in}{1.264674in}}%
\pgfpathlineto{\pgfqpoint{1.226352in}{1.247390in}}%
\pgfpathlineto{\pgfqpoint{1.182864in}{1.221059in}}%
\pgfpathlineto{\pgfqpoint{1.134357in}{1.182868in}}%
\pgfusepath{stroke}%
\end{pgfscope}%
\begin{pgfscope}%
\pgfpathrectangle{\pgfqpoint{0.647939in}{0.492442in}}{\pgfqpoint{4.273799in}{2.331163in}}%
\pgfusepath{clip}%
\pgfsetbuttcap%
\pgfsetroundjoin%
\pgfsetlinewidth{0.301125pt}%
\definecolor{currentstroke}{rgb}{0.500000,0.500000,0.500000}%
\pgfsetstrokecolor{currentstroke}%
\pgfsetstrokeopacity{0.300000}%
\pgfsetdash{}{0pt}%
\pgfpathmoveto{\pgfqpoint{2.396312in}{0.492442in}}%
\pgfpathlineto{\pgfqpoint{2.396312in}{0.492442in}}%
\pgfpathlineto{\pgfqpoint{2.360819in}{0.540492in}}%
\pgfpathlineto{\pgfqpoint{2.325787in}{0.588642in}}%
\pgfpathlineto{\pgfqpoint{2.291194in}{0.636887in}}%
\pgfpathlineto{\pgfqpoint{2.257013in}{0.685219in}}%
\pgfpathlineto{\pgfqpoint{2.223220in}{0.733633in}}%
\pgfpathlineto{\pgfqpoint{2.189794in}{0.782122in}}%
\pgfpathlineto{\pgfqpoint{2.156714in}{0.830681in}}%
\pgfpathlineto{\pgfqpoint{2.123951in}{0.879305in}}%
\pgfpathlineto{\pgfqpoint{2.091471in}{0.927985in}}%
\pgfpathlineto{\pgfqpoint{2.059246in}{0.976715in}}%
\pgfpathlineto{\pgfqpoint{2.027246in}{1.025489in}}%
\pgfpathlineto{\pgfqpoint{1.995431in}{1.074300in}}%
\pgfpathlineto{\pgfqpoint{1.963747in}{1.123135in}}%
\pgfpathlineto{\pgfqpoint{1.932145in}{1.171987in}}%
\pgfpathlineto{\pgfqpoint{1.900571in}{1.220843in}}%
\pgfpathlineto{\pgfqpoint{1.868935in}{1.269688in}}%
\pgfpathlineto{\pgfqpoint{1.837124in}{1.318498in}}%
\pgfpathlineto{\pgfqpoint{1.805011in}{1.367249in}}%
\pgfpathlineto{\pgfqpoint{1.772412in}{1.415904in}}%
\pgfpathlineto{\pgfqpoint{1.739037in}{1.464400in}}%
\pgfpathlineto{\pgfqpoint{1.704438in}{1.512635in}}%
\pgfpathlineto{\pgfqpoint{1.667894in}{1.560432in}}%
\pgfpathlineto{\pgfqpoint{1.628022in}{1.607407in}}%
\pgfpathlineto{\pgfqpoint{1.581669in}{1.652463in}}%
\pgfpathlineto{\pgfqpoint{1.581669in}{1.652463in}}%
\pgfpathlineto{\pgfqpoint{1.534376in}{1.684153in}}%
\pgfpathlineto{\pgfqpoint{1.534376in}{1.684153in}}%
\pgfpathlineto{\pgfqpoint{1.498820in}{1.696419in}}%
\pgfpathlineto{\pgfqpoint{1.498820in}{1.696419in}}%
\pgfpathlineto{\pgfqpoint{1.464412in}{1.697981in}}%
\pgfpathlineto{\pgfqpoint{1.432143in}{1.690600in}}%
\pgfpathlineto{\pgfqpoint{1.403204in}{1.677328in}}%
\pgfpathlineto{\pgfqpoint{1.369714in}{1.655147in}}%
\pgfpathlineto{\pgfqpoint{1.329248in}{1.620393in}}%
\pgfpathlineto{\pgfqpoint{1.285058in}{1.574699in}}%
\pgfpathlineto{\pgfqpoint{1.244932in}{1.527825in}}%
\pgfpathlineto{\pgfqpoint{1.207407in}{1.480270in}}%
\pgfpathlineto{\pgfqpoint{1.171761in}{1.432271in}}%
\pgfpathlineto{\pgfqpoint{1.137596in}{1.383967in}}%
\pgfusepath{stroke}%
\end{pgfscope}%
\begin{pgfscope}%
\pgfpathrectangle{\pgfqpoint{0.647939in}{0.492442in}}{\pgfqpoint{4.273799in}{2.331163in}}%
\pgfusepath{clip}%
\pgfsetbuttcap%
\pgfsetroundjoin%
\pgfsetlinewidth{0.301125pt}%
\definecolor{currentstroke}{rgb}{0.500000,0.500000,0.500000}%
\pgfsetstrokecolor{currentstroke}%
\pgfsetstrokeopacity{0.300000}%
\pgfsetdash{}{0pt}%
\pgfpathmoveto{\pgfqpoint{2.493443in}{0.492442in}}%
\pgfpathlineto{\pgfqpoint{2.493443in}{0.492442in}}%
\pgfpathlineto{\pgfqpoint{2.457225in}{0.540331in}}%
\pgfpathlineto{\pgfqpoint{2.421566in}{0.588344in}}%
\pgfpathlineto{\pgfqpoint{2.386444in}{0.636474in}}%
\pgfpathlineto{\pgfqpoint{2.351838in}{0.684717in}}%
\pgfpathlineto{\pgfqpoint{2.317734in}{0.733065in}}%
\pgfpathlineto{\pgfqpoint{2.284116in}{0.781514in}}%
\pgfpathlineto{\pgfqpoint{2.250965in}{0.830059in}}%
\pgfpathlineto{\pgfqpoint{2.218258in}{0.878694in}}%
\pgfpathlineto{\pgfqpoint{2.185977in}{0.927413in}}%
\pgfpathlineto{\pgfqpoint{2.154109in}{0.976213in}}%
\pgfpathlineto{\pgfqpoint{2.122636in}{1.025089in}}%
\pgfpathlineto{\pgfqpoint{2.091532in}{1.074036in}}%
\pgfpathlineto{\pgfqpoint{2.060779in}{1.123048in}}%
\pgfpathlineto{\pgfqpoint{2.030365in}{1.172124in}}%
\pgfpathlineto{\pgfqpoint{2.000260in}{1.221256in}}%
\pgfpathlineto{\pgfqpoint{1.970435in}{1.270438in}}%
\pgfpathlineto{\pgfqpoint{1.940873in}{1.319668in}}%
\pgfpathlineto{\pgfqpoint{1.911539in}{1.368938in}}%
\pgfpathlineto{\pgfqpoint{1.882384in}{1.418240in}}%
\pgfpathlineto{\pgfqpoint{1.853376in}{1.467567in}}%
\pgfpathlineto{\pgfqpoint{1.824448in}{1.516908in}}%
\pgfpathlineto{\pgfqpoint{1.795503in}{1.566245in}}%
\pgfpathlineto{\pgfqpoint{1.766436in}{1.615559in}}%
\pgfpathlineto{\pgfqpoint{1.737053in}{1.664818in}}%
\pgfpathlineto{\pgfqpoint{1.707015in}{1.713952in}}%
\pgfpathlineto{\pgfqpoint{1.675723in}{1.762844in}}%
\pgfpathlineto{\pgfqpoint{1.641772in}{1.811196in}}%
\pgfpathlineto{\pgfqpoint{1.600459in}{1.857624in}}%
\pgfpathlineto{\pgfqpoint{1.600459in}{1.857624in}}%
\pgfpathlineto{\pgfqpoint{1.576448in}{1.874646in}}%
\pgfpathlineto{\pgfqpoint{1.576448in}{1.874646in}}%
\pgfpathlineto{\pgfqpoint{1.550283in}{1.882065in}}%
\pgfpathlineto{\pgfqpoint{1.550283in}{1.882065in}}%
\pgfpathlineto{\pgfqpoint{1.527203in}{1.878661in}}%
\pgfpathlineto{\pgfqpoint{1.507743in}{1.869842in}}%
\pgfpathlineto{\pgfqpoint{1.483237in}{1.852875in}}%
\pgfpathlineto{\pgfqpoint{1.452273in}{1.824915in}}%
\pgfpathlineto{\pgfqpoint{1.409515in}{1.779107in}}%
\pgfpathlineto{\pgfqpoint{1.369903in}{1.732235in}}%
\pgfpathlineto{\pgfqpoint{1.332202in}{1.684794in}}%
\pgfusepath{stroke}%
\end{pgfscope}%
\begin{pgfscope}%
\pgfpathrectangle{\pgfqpoint{0.647939in}{0.492442in}}{\pgfqpoint{4.273799in}{2.331163in}}%
\pgfusepath{clip}%
\pgfsetbuttcap%
\pgfsetroundjoin%
\pgfsetlinewidth{0.301125pt}%
\definecolor{currentstroke}{rgb}{0.500000,0.500000,0.500000}%
\pgfsetstrokecolor{currentstroke}%
\pgfsetstrokeopacity{0.300000}%
\pgfsetdash{}{0pt}%
\pgfpathmoveto{\pgfqpoint{2.590575in}{0.492442in}}%
\pgfpathlineto{\pgfqpoint{2.590575in}{0.492442in}}%
\pgfpathlineto{\pgfqpoint{2.553350in}{0.540100in}}%
\pgfpathlineto{\pgfqpoint{2.516774in}{0.587908in}}%
\pgfpathlineto{\pgfqpoint{2.480831in}{0.635857in}}%
\pgfpathlineto{\pgfqpoint{2.445504in}{0.683944in}}%
\pgfpathlineto{\pgfqpoint{2.410781in}{0.732160in}}%
\pgfpathlineto{\pgfqpoint{2.376647in}{0.780502in}}%
\pgfusepath{stroke}%
\end{pgfscope}%
\begin{pgfscope}%
\pgfpathrectangle{\pgfqpoint{0.647939in}{0.492442in}}{\pgfqpoint{4.273799in}{2.331163in}}%
\pgfusepath{clip}%
\pgfsetbuttcap%
\pgfsetroundjoin%
\pgfsetlinewidth{0.301125pt}%
\definecolor{currentstroke}{rgb}{0.500000,0.500000,0.500000}%
\pgfsetstrokecolor{currentstroke}%
\pgfsetstrokeopacity{0.300000}%
\pgfsetdash{}{0pt}%
\pgfpathmoveto{\pgfqpoint{2.784839in}{0.492442in}}%
\pgfpathlineto{\pgfqpoint{2.784839in}{0.492442in}}%
\pgfpathlineto{\pgfqpoint{2.744800in}{0.539416in}}%
\pgfpathlineto{\pgfqpoint{2.705595in}{0.586599in}}%
\pgfpathlineto{\pgfqpoint{2.667210in}{0.633982in}}%
\pgfpathlineto{\pgfqpoint{2.629630in}{0.681557in}}%
\pgfpathlineto{\pgfqpoint{2.592842in}{0.729315in}}%
\pgfpathlineto{\pgfqpoint{2.556829in}{0.777250in}}%
\pgfpathlineto{\pgfqpoint{2.521578in}{0.825352in}}%
\pgfpathlineto{\pgfqpoint{2.487077in}{0.873616in}}%
\pgfpathlineto{\pgfqpoint{2.453319in}{0.922036in}}%
\pgfpathlineto{\pgfqpoint{2.420299in}{0.970607in}}%
\pgfpathlineto{\pgfqpoint{2.388006in}{1.019324in}}%
\pgfpathlineto{\pgfqpoint{2.356436in}{1.068181in}}%
\pgfpathlineto{\pgfqpoint{2.325593in}{1.117176in}}%
\pgfpathlineto{\pgfqpoint{2.295483in}{1.166307in}}%
\pgfpathlineto{\pgfqpoint{2.266110in}{1.215570in}}%
\pgfpathlineto{\pgfqpoint{2.237488in}{1.264965in}}%
\pgfpathlineto{\pgfqpoint{2.209641in}{1.314491in}}%
\pgfpathlineto{\pgfqpoint{2.182592in}{1.364149in}}%
\pgfpathlineto{\pgfqpoint{2.156382in}{1.413940in}}%
\pgfpathlineto{\pgfqpoint{2.131059in}{1.463867in}}%
\pgfpathlineto{\pgfqpoint{2.106683in}{1.513934in}}%
\pgfpathlineto{\pgfqpoint{2.083339in}{1.564148in}}%
\pgfpathlineto{\pgfqpoint{2.061132in}{1.614514in}}%
\pgfpathlineto{\pgfqpoint{2.040201in}{1.665042in}}%
\pgfpathlineto{\pgfqpoint{2.020727in}{1.715743in}}%
\pgfpathlineto{\pgfqpoint{2.002941in}{1.766627in}}%
\pgfpathlineto{\pgfqpoint{1.987162in}{1.817706in}}%
\pgfpathlineto{\pgfqpoint{1.973793in}{1.868988in}}%
\pgfpathlineto{\pgfqpoint{1.963370in}{1.920469in}}%
\pgfpathlineto{\pgfqpoint{1.956610in}{1.972128in}}%
\pgfpathlineto{\pgfqpoint{1.954426in}{2.023898in}}%
\pgfpathlineto{\pgfqpoint{1.957931in}{2.075634in}}%
\pgfpathlineto{\pgfqpoint{1.968380in}{2.127072in}}%
\pgfpathlineto{\pgfqpoint{1.986960in}{2.177799in}}%
\pgfpathlineto{\pgfqpoint{2.014594in}{2.227262in}}%
\pgfpathlineto{\pgfqpoint{2.051738in}{2.274817in}}%
\pgfpathlineto{\pgfqpoint{2.098402in}{2.319787in}}%
\pgfpathlineto{\pgfqpoint{2.154414in}{2.361432in}}%
\pgfpathlineto{\pgfqpoint{2.219726in}{2.398812in}}%
\pgfpathlineto{\pgfqpoint{2.294316in}{2.430523in}}%
\pgfpathlineto{\pgfqpoint{2.377766in}{2.454538in}}%
\pgfpathlineto{\pgfqpoint{2.466085in}{2.468162in}}%
\pgfpathlineto{\pgfqpoint{2.549631in}{2.470494in}}%
\pgfpathlineto{\pgfqpoint{2.628242in}{2.463354in}}%
\pgfpathlineto{\pgfqpoint{2.703118in}{2.447889in}}%
\pgfpathlineto{\pgfqpoint{2.776061in}{2.424253in}}%
\pgfpathlineto{\pgfqpoint{2.848016in}{2.391948in}}%
\pgfpathlineto{\pgfqpoint{2.912818in}{2.354277in}}%
\pgfpathlineto{\pgfqpoint{2.969991in}{2.313055in}}%
\pgfpathlineto{\pgfqpoint{3.020160in}{2.269157in}}%
\pgfpathlineto{\pgfqpoint{3.063652in}{2.223163in}}%
\pgfpathlineto{\pgfqpoint{3.100442in}{2.175462in}}%
\pgfpathlineto{\pgfqpoint{3.130025in}{2.126311in}}%
\pgfpathlineto{\pgfqpoint{3.151019in}{2.075884in}}%
\pgfpathlineto{\pgfqpoint{3.159983in}{2.024483in}}%
\pgfpathlineto{\pgfqpoint{3.159983in}{2.024483in}}%
\pgfpathlineto{\pgfqpoint{3.152970in}{1.985349in}}%
\pgfpathlineto{\pgfqpoint{3.152970in}{1.985349in}}%
\pgfpathlineto{\pgfqpoint{3.136629in}{1.963740in}}%
\pgfpathlineto{\pgfqpoint{3.136629in}{1.963740in}}%
\pgfpathlineto{\pgfqpoint{3.114589in}{1.953661in}}%
\pgfpathlineto{\pgfqpoint{3.086515in}{1.953978in}}%
\pgfpathlineto{\pgfqpoint{3.064404in}{1.961157in}}%
\pgfpathlineto{\pgfqpoint{3.041418in}{1.975335in}}%
\pgfpathlineto{\pgfqpoint{3.021485in}{1.996568in}}%
\pgfpathlineto{\pgfqpoint{3.013648in}{2.024866in}}%
\pgfpathlineto{\pgfqpoint{3.013648in}{2.024866in}}%
\pgfpathlineto{\pgfqpoint{3.020775in}{2.029828in}}%
\pgfpathlineto{\pgfqpoint{3.028632in}{2.027343in}}%
\pgfpathlineto{\pgfqpoint{3.033057in}{2.022150in}}%
\pgfpathlineto{\pgfqpoint{3.031201in}{2.019075in}}%
\pgfpathlineto{\pgfqpoint{3.030987in}{2.022484in}}%
\pgfpathlineto{\pgfqpoint{3.028254in}{2.016728in}}%
\pgfpathlineto{\pgfqpoint{3.028254in}{2.016728in}}%
\pgfpathlineto{\pgfqpoint{3.033741in}{2.025014in}}%
\pgfpathlineto{\pgfqpoint{3.034049in}{2.018534in}}%
\pgfpathlineto{\pgfqpoint{3.030265in}{2.020158in}}%
\pgfpathlineto{\pgfqpoint{3.032731in}{2.020965in}}%
\pgfusepath{stroke}%
\end{pgfscope}%
\begin{pgfscope}%
\pgfpathrectangle{\pgfqpoint{0.647939in}{0.492442in}}{\pgfqpoint{4.273799in}{2.331163in}}%
\pgfusepath{clip}%
\pgfsetbuttcap%
\pgfsetroundjoin%
\pgfsetlinewidth{0.301125pt}%
\definecolor{currentstroke}{rgb}{0.500000,0.500000,0.500000}%
\pgfsetstrokecolor{currentstroke}%
\pgfsetstrokeopacity{0.300000}%
\pgfsetdash{}{0pt}%
\pgfpathmoveto{\pgfqpoint{2.881971in}{0.492442in}}%
\pgfpathlineto{\pgfqpoint{2.881971in}{0.492442in}}%
\pgfpathlineto{\pgfqpoint{2.840157in}{0.538954in}}%
\pgfpathlineto{\pgfqpoint{2.799265in}{0.585708in}}%
\pgfpathlineto{\pgfqpoint{2.759283in}{0.632697in}}%
\pgfpathlineto{\pgfqpoint{2.720199in}{0.679909in}}%
\pgfpathlineto{\pgfqpoint{2.681997in}{0.727337in}}%
\pgfpathlineto{\pgfqpoint{2.644663in}{0.774969in}}%
\pgfpathlineto{\pgfqpoint{2.608185in}{0.822798in}}%
\pgfpathlineto{\pgfqpoint{2.572551in}{0.870816in}}%
\pgfusepath{stroke}%
\end{pgfscope}%
\begin{pgfscope}%
\pgfpathrectangle{\pgfqpoint{0.647939in}{0.492442in}}{\pgfqpoint{4.273799in}{2.331163in}}%
\pgfusepath{clip}%
\pgfsetbuttcap%
\pgfsetroundjoin%
\pgfsetlinewidth{0.301125pt}%
\definecolor{currentstroke}{rgb}{0.500000,0.500000,0.500000}%
\pgfsetstrokecolor{currentstroke}%
\pgfsetstrokeopacity{0.300000}%
\pgfsetdash{}{0pt}%
\pgfpathmoveto{\pgfqpoint{2.979102in}{0.492442in}}%
\pgfpathlineto{\pgfqpoint{2.979102in}{0.492442in}}%
\pgfpathlineto{\pgfqpoint{2.935294in}{0.538404in}}%
\pgfpathlineto{\pgfqpoint{2.892488in}{0.584647in}}%
\pgfpathlineto{\pgfqpoint{2.850678in}{0.631159in}}%
\pgfpathlineto{\pgfqpoint{2.809854in}{0.677932in}}%
\pgfpathlineto{\pgfqpoint{2.770004in}{0.724953in}}%
\pgfusepath{stroke}%
\end{pgfscope}%
\begin{pgfscope}%
\pgfpathrectangle{\pgfqpoint{0.647939in}{0.492442in}}{\pgfqpoint{4.273799in}{2.331163in}}%
\pgfusepath{clip}%
\pgfsetbuttcap%
\pgfsetroundjoin%
\pgfsetlinewidth{0.301125pt}%
\definecolor{currentstroke}{rgb}{0.500000,0.500000,0.500000}%
\pgfsetstrokecolor{currentstroke}%
\pgfsetstrokeopacity{0.300000}%
\pgfsetdash{}{0pt}%
\pgfpathmoveto{\pgfqpoint{3.173366in}{0.492442in}}%
\pgfpathlineto{\pgfqpoint{3.173366in}{0.492442in}}%
\pgfpathlineto{\pgfqpoint{3.125075in}{0.537047in}}%
\pgfpathlineto{\pgfqpoint{3.077891in}{0.582002in}}%
\pgfpathlineto{\pgfqpoint{3.031833in}{0.627304in}}%
\pgfpathlineto{\pgfqpoint{2.986913in}{0.672945in}}%
\pgfpathlineto{\pgfqpoint{2.943133in}{0.718914in}}%
\pgfpathlineto{\pgfqpoint{2.900491in}{0.765201in}}%
\pgfpathlineto{\pgfqpoint{2.858983in}{0.811793in}}%
\pgfpathlineto{\pgfqpoint{2.818598in}{0.858678in}}%
\pgfpathlineto{\pgfqpoint{2.779328in}{0.905844in}}%
\pgfpathlineto{\pgfqpoint{2.741163in}{0.953278in}}%
\pgfpathlineto{\pgfqpoint{2.704093in}{1.000971in}}%
\pgfpathlineto{\pgfqpoint{2.668113in}{1.048912in}}%
\pgfpathlineto{\pgfqpoint{2.633220in}{1.097091in}}%
\pgfpathlineto{\pgfqpoint{2.599412in}{1.145500in}}%
\pgfpathlineto{\pgfqpoint{2.566688in}{1.194130in}}%
\pgfpathlineto{\pgfqpoint{2.535061in}{1.242975in}}%
\pgfpathlineto{\pgfqpoint{2.504550in}{1.292032in}}%
\pgfpathlineto{\pgfqpoint{2.475174in}{1.341294in}}%
\pgfpathlineto{\pgfqpoint{2.446963in}{1.390758in}}%
\pgfpathlineto{\pgfqpoint{2.419965in}{1.440423in}}%
\pgfpathlineto{\pgfqpoint{2.394232in}{1.490287in}}%
\pgfpathlineto{\pgfqpoint{2.369835in}{1.540350in}}%
\pgfpathlineto{\pgfqpoint{2.346867in}{1.590614in}}%
\pgfpathlineto{\pgfqpoint{2.325438in}{1.641080in}}%
\pgfpathlineto{\pgfqpoint{2.305692in}{1.691749in}}%
\pgfpathlineto{\pgfqpoint{2.287809in}{1.742622in}}%
\pgfpathlineto{\pgfqpoint{2.272009in}{1.793700in}}%
\pgfpathlineto{\pgfqpoint{2.258569in}{1.844977in}}%
\pgfpathlineto{\pgfqpoint{2.247846in}{1.896442in}}%
\pgfpathlineto{\pgfqpoint{2.240282in}{1.948071in}}%
\pgfpathlineto{\pgfqpoint{2.236434in}{1.999820in}}%
\pgfpathlineto{\pgfqpoint{2.237011in}{2.051604in}}%
\pgfpathlineto{\pgfqpoint{2.242901in}{2.103280in}}%
\pgfpathlineto{\pgfqpoint{2.255223in}{2.154601in}}%
\pgfpathlineto{\pgfqpoint{2.275378in}{2.205157in}}%
\pgfpathlineto{\pgfqpoint{2.305067in}{2.254260in}}%
\pgfpathlineto{\pgfqpoint{2.346338in}{2.300737in}}%
\pgfpathlineto{\pgfqpoint{2.401463in}{2.342555in}}%
\pgfpathlineto{\pgfqpoint{2.471675in}{2.376142in}}%
\pgfpathlineto{\pgfqpoint{2.544260in}{2.395152in}}%
\pgfpathlineto{\pgfqpoint{2.613933in}{2.401620in}}%
\pgfpathlineto{\pgfqpoint{2.680297in}{2.398336in}}%
\pgfusepath{stroke}%
\end{pgfscope}%
\begin{pgfscope}%
\pgfpathrectangle{\pgfqpoint{0.647939in}{0.492442in}}{\pgfqpoint{4.273799in}{2.331163in}}%
\pgfusepath{clip}%
\pgfsetbuttcap%
\pgfsetroundjoin%
\pgfsetlinewidth{0.301125pt}%
\definecolor{currentstroke}{rgb}{0.500000,0.500000,0.500000}%
\pgfsetstrokecolor{currentstroke}%
\pgfsetstrokeopacity{0.300000}%
\pgfsetdash{}{0pt}%
\pgfpathmoveto{\pgfqpoint{3.367630in}{0.492442in}}%
\pgfpathlineto{\pgfqpoint{3.367630in}{0.492442in}}%
\pgfpathlineto{\pgfqpoint{3.314674in}{0.535443in}}%
\pgfpathlineto{\pgfqpoint{3.262763in}{0.578821in}}%
\pgfpathlineto{\pgfqpoint{3.211977in}{0.622593in}}%
\pgfpathlineto{\pgfqpoint{3.162375in}{0.666767in}}%
\pgfpathlineto{\pgfqpoint{3.114003in}{0.711345in}}%
\pgfpathlineto{\pgfqpoint{3.066892in}{0.756324in}}%
\pgfpathlineto{\pgfqpoint{3.021062in}{0.801695in}}%
\pgfpathlineto{\pgfqpoint{2.976522in}{0.847447in}}%
\pgfpathlineto{\pgfqpoint{2.933276in}{0.893566in}}%
\pgfpathlineto{\pgfqpoint{2.891320in}{0.940038in}}%
\pgfpathlineto{\pgfqpoint{2.850648in}{0.986850in}}%
\pgfpathlineto{\pgfqpoint{2.811254in}{1.033985in}}%
\pgfpathlineto{\pgfqpoint{2.773129in}{1.081429in}}%
\pgfpathlineto{\pgfqpoint{2.736268in}{1.129169in}}%
\pgfpathlineto{\pgfqpoint{2.700670in}{1.177194in}}%
\pgfpathlineto{\pgfqpoint{2.666339in}{1.225492in}}%
\pgfpathlineto{\pgfqpoint{2.633283in}{1.274055in}}%
\pgfpathlineto{\pgfqpoint{2.601517in}{1.322874in}}%
\pgfpathlineto{\pgfqpoint{2.571060in}{1.371939in}}%
\pgfpathlineto{\pgfqpoint{2.541950in}{1.421247in}}%
\pgfpathlineto{\pgfqpoint{2.514231in}{1.470793in}}%
\pgfpathlineto{\pgfqpoint{2.487961in}{1.520574in}}%
\pgfpathlineto{\pgfqpoint{2.463223in}{1.570586in}}%
\pgfpathlineto{\pgfqpoint{2.440110in}{1.620830in}}%
\pgfpathlineto{\pgfqpoint{2.418751in}{1.671304in}}%
\pgfpathlineto{\pgfqpoint{2.399302in}{1.722006in}}%
\pgfpathlineto{\pgfqpoint{2.381959in}{1.772934in}}%
\pgfpathlineto{\pgfqpoint{2.366976in}{1.824084in}}%
\pgfpathlineto{\pgfqpoint{2.354668in}{1.875443in}}%
\pgfpathlineto{\pgfqpoint{2.345438in}{1.926993in}}%
\pgfpathlineto{\pgfqpoint{2.339803in}{1.978694in}}%
\pgfpathlineto{\pgfqpoint{2.338430in}{2.030474in}}%
\pgfpathlineto{\pgfqpoint{2.342186in}{2.082210in}}%
\pgfpathlineto{\pgfqpoint{2.352219in}{2.133682in}}%
\pgfpathlineto{\pgfqpoint{2.370059in}{2.184500in}}%
\pgfpathlineto{\pgfqpoint{2.397780in}{2.233941in}}%
\pgfpathlineto{\pgfqpoint{2.438195in}{2.280612in}}%
\pgfpathlineto{\pgfqpoint{2.494763in}{2.321708in}}%
\pgfpathlineto{\pgfqpoint{2.494763in}{2.321708in}}%
\pgfpathlineto{\pgfqpoint{2.549966in}{2.346194in}}%
\pgfpathlineto{\pgfqpoint{2.615917in}{2.361322in}}%
\pgfpathlineto{\pgfqpoint{2.679067in}{2.364599in}}%
\pgfpathlineto{\pgfqpoint{2.739464in}{2.358775in}}%
\pgfpathlineto{\pgfqpoint{2.798610in}{2.345046in}}%
\pgfusepath{stroke}%
\end{pgfscope}%
\begin{pgfscope}%
\pgfpathrectangle{\pgfqpoint{0.647939in}{0.492442in}}{\pgfqpoint{4.273799in}{2.331163in}}%
\pgfusepath{clip}%
\pgfsetbuttcap%
\pgfsetroundjoin%
\pgfsetlinewidth{0.301125pt}%
\definecolor{currentstroke}{rgb}{0.500000,0.500000,0.500000}%
\pgfsetstrokecolor{currentstroke}%
\pgfsetstrokeopacity{0.300000}%
\pgfsetdash{}{0pt}%
\pgfpathmoveto{\pgfqpoint{3.561893in}{0.492442in}}%
\pgfpathlineto{\pgfqpoint{3.561893in}{0.492442in}}%
\pgfpathlineto{\pgfqpoint{3.505038in}{0.533937in}}%
\pgfpathlineto{\pgfqpoint{3.448832in}{0.575693in}}%
\pgfpathlineto{\pgfqpoint{3.393466in}{0.617781in}}%
\pgfpathlineto{\pgfqpoint{3.339099in}{0.660254in}}%
\pgfpathlineto{\pgfqpoint{3.285857in}{0.703149in}}%
\pgfpathlineto{\pgfqpoint{3.233848in}{0.746492in}}%
\pgfpathlineto{\pgfqpoint{3.183155in}{0.790296in}}%
\pgfpathlineto{\pgfqpoint{3.133839in}{0.834564in}}%
\pgfpathlineto{\pgfqpoint{3.085939in}{0.879291in}}%
\pgfpathlineto{\pgfqpoint{3.039479in}{0.924469in}}%
\pgfpathlineto{\pgfqpoint{2.994473in}{0.970083in}}%
\pgfpathlineto{\pgfqpoint{2.950927in}{1.016116in}}%
\pgfpathlineto{\pgfqpoint{2.908840in}{1.062551in}}%
\pgfpathlineto{\pgfqpoint{2.868208in}{1.109371in}}%
\pgfpathlineto{\pgfqpoint{2.829029in}{1.156558in}}%
\pgfpathlineto{\pgfqpoint{2.791298in}{1.204094in}}%
\pgfpathlineto{\pgfqpoint{2.755017in}{1.251966in}}%
\pgfpathlineto{\pgfqpoint{2.720195in}{1.300160in}}%
\pgfpathlineto{\pgfqpoint{2.686843in}{1.348662in}}%
\pgfpathlineto{\pgfqpoint{2.654978in}{1.397460in}}%
\pgfpathlineto{\pgfqpoint{2.624638in}{1.446546in}}%
\pgfpathlineto{\pgfqpoint{2.595870in}{1.495913in}}%
\pgfpathlineto{\pgfqpoint{2.568734in}{1.545554in}}%
\pgfpathlineto{\pgfqpoint{2.543313in}{1.595464in}}%
\pgfpathlineto{\pgfqpoint{2.519715in}{1.645640in}}%
\pgfpathlineto{\pgfqpoint{2.498075in}{1.696078in}}%
\pgfpathlineto{\pgfqpoint{2.478570in}{1.746772in}}%
\pgfpathlineto{\pgfqpoint{2.461423in}{1.797719in}}%
\pgfpathlineto{\pgfqpoint{2.446920in}{1.848908in}}%
\pgfpathlineto{\pgfqpoint{2.435422in}{1.900322in}}%
\pgfpathlineto{\pgfqpoint{2.427411in}{1.951930in}}%
\pgfpathlineto{\pgfqpoint{2.423514in}{2.003674in}}%
\pgfpathlineto{\pgfqpoint{2.424566in}{2.055450in}}%
\pgfpathlineto{\pgfqpoint{2.431715in}{2.107067in}}%
\pgfpathlineto{\pgfqpoint{2.446571in}{2.158168in}}%
\pgfpathlineto{\pgfqpoint{2.471456in}{2.208047in}}%
\pgfpathlineto{\pgfqpoint{2.509821in}{2.255197in}}%
\pgfpathlineto{\pgfqpoint{2.509821in}{2.255197in}}%
\pgfpathlineto{\pgfqpoint{2.556108in}{2.290508in}}%
\pgfpathlineto{\pgfqpoint{2.556108in}{2.290508in}}%
\pgfpathlineto{\pgfqpoint{2.606665in}{2.313803in}}%
\pgfusepath{stroke}%
\end{pgfscope}%
\begin{pgfscope}%
\pgfpathrectangle{\pgfqpoint{0.647939in}{0.492442in}}{\pgfqpoint{4.273799in}{2.331163in}}%
\pgfusepath{clip}%
\pgfsetbuttcap%
\pgfsetroundjoin%
\pgfsetlinewidth{0.301125pt}%
\definecolor{currentstroke}{rgb}{0.500000,0.500000,0.500000}%
\pgfsetstrokecolor{currentstroke}%
\pgfsetstrokeopacity{0.300000}%
\pgfsetdash{}{0pt}%
\pgfpathmoveto{\pgfqpoint{3.756157in}{0.492442in}}%
\pgfpathlineto{\pgfqpoint{3.756157in}{0.492442in}}%
\pgfpathlineto{\pgfqpoint{3.697383in}{0.533134in}}%
\pgfpathlineto{\pgfqpoint{3.638474in}{0.573768in}}%
\pgfpathlineto{\pgfqpoint{3.579700in}{0.614460in}}%
\pgfpathlineto{\pgfqpoint{3.521339in}{0.655328in}}%
\pgfpathlineto{\pgfqpoint{3.463647in}{0.696476in}}%
\pgfpathlineto{\pgfqpoint{3.406843in}{0.737990in}}%
\pgfpathlineto{\pgfqpoint{3.351127in}{0.779940in}}%
\pgfpathlineto{\pgfqpoint{3.296665in}{0.822375in}}%
\pgfpathlineto{\pgfqpoint{3.243579in}{0.865326in}}%
\pgfpathlineto{\pgfqpoint{3.191966in}{0.908809in}}%
\pgfpathlineto{\pgfqpoint{3.141902in}{0.952826in}}%
\pgfpathlineto{\pgfqpoint{3.093437in}{0.997371in}}%
\pgfpathlineto{\pgfqpoint{3.046597in}{1.042431in}}%
\pgfpathlineto{\pgfqpoint{3.001399in}{1.087987in}}%
\pgfpathlineto{\pgfqpoint{2.957850in}{1.134018in}}%
\pgfpathlineto{\pgfqpoint{2.915950in}{1.180502in}}%
\pgfpathlineto{\pgfqpoint{2.875698in}{1.227418in}}%
\pgfpathlineto{\pgfqpoint{2.837096in}{1.274745in}}%
\pgfpathlineto{\pgfqpoint{2.800146in}{1.322464in}}%
\pgfpathlineto{\pgfqpoint{2.764862in}{1.370557in}}%
\pgfusepath{stroke}%
\end{pgfscope}%
\begin{pgfscope}%
\pgfpathrectangle{\pgfqpoint{0.647939in}{0.492442in}}{\pgfqpoint{4.273799in}{2.331163in}}%
\pgfusepath{clip}%
\pgfsetbuttcap%
\pgfsetroundjoin%
\pgfsetlinewidth{0.301125pt}%
\definecolor{currentstroke}{rgb}{0.500000,0.500000,0.500000}%
\pgfsetstrokecolor{currentstroke}%
\pgfsetstrokeopacity{0.300000}%
\pgfsetdash{}{0pt}%
\pgfpathmoveto{\pgfqpoint{3.950420in}{0.492442in}}%
\pgfpathlineto{\pgfqpoint{3.950420in}{0.492442in}}%
\pgfpathlineto{\pgfqpoint{3.892909in}{0.533664in}}%
\pgfpathlineto{\pgfqpoint{3.834202in}{0.574384in}}%
\pgfpathlineto{\pgfqpoint{3.774577in}{0.614705in}}%
\pgfpathlineto{\pgfqpoint{3.714330in}{0.654751in}}%
\pgfpathlineto{\pgfqpoint{3.653795in}{0.694668in}}%
\pgfpathlineto{\pgfqpoint{3.593318in}{0.734610in}}%
\pgfpathlineto{\pgfqpoint{3.533222in}{0.774722in}}%
\pgfpathlineto{\pgfqpoint{3.473822in}{0.815141in}}%
\pgfpathlineto{\pgfqpoint{3.415394in}{0.855978in}}%
\pgfpathlineto{\pgfqpoint{3.358178in}{0.897321in}}%
\pgfpathlineto{\pgfqpoint{3.302378in}{0.939235in}}%
\pgfpathlineto{\pgfqpoint{3.248144in}{0.981755in}}%
\pgfpathlineto{\pgfqpoint{3.195588in}{1.024899in}}%
\pgfpathlineto{\pgfqpoint{3.144798in}{1.068667in}}%
\pgfpathlineto{\pgfqpoint{3.095832in}{1.113048in}}%
\pgfpathlineto{\pgfqpoint{3.048720in}{1.158022in}}%
\pgfpathlineto{\pgfqpoint{3.003477in}{1.203563in}}%
\pgfpathlineto{\pgfqpoint{2.960113in}{1.249644in}}%
\pgfusepath{stroke}%
\end{pgfscope}%
\begin{pgfscope}%
\pgfpathrectangle{\pgfqpoint{0.647939in}{0.492442in}}{\pgfqpoint{4.273799in}{2.331163in}}%
\pgfusepath{clip}%
\pgfsetbuttcap%
\pgfsetroundjoin%
\pgfsetlinewidth{0.301125pt}%
\definecolor{currentstroke}{rgb}{0.500000,0.500000,0.500000}%
\pgfsetstrokecolor{currentstroke}%
\pgfsetstrokeopacity{0.300000}%
\pgfsetdash{}{0pt}%
\pgfpathmoveto{\pgfqpoint{4.144684in}{0.492442in}}%
\pgfpathlineto{\pgfqpoint{4.144684in}{0.492442in}}%
\pgfpathlineto{\pgfqpoint{4.092118in}{0.535582in}}%
\pgfpathlineto{\pgfqpoint{4.037572in}{0.577983in}}%
\pgfpathlineto{\pgfqpoint{3.981101in}{0.619627in}}%
\pgfpathlineto{\pgfqpoint{3.922831in}{0.660530in}}%
\pgfpathlineto{\pgfqpoint{3.862989in}{0.700753in}}%
\pgfpathlineto{\pgfqpoint{3.801853in}{0.740394in}}%
\pgfpathlineto{\pgfqpoint{3.739772in}{0.779597in}}%
\pgfpathlineto{\pgfqpoint{3.677149in}{0.818543in}}%
\pgfpathlineto{\pgfqpoint{3.614397in}{0.857427in}}%
\pgfpathlineto{\pgfqpoint{3.551941in}{0.896452in}}%
\pgfpathlineto{\pgfqpoint{3.490174in}{0.935801in}}%
\pgfpathlineto{\pgfqpoint{3.429464in}{0.975633in}}%
\pgfpathlineto{\pgfqpoint{3.370116in}{1.016071in}}%
\pgfpathlineto{\pgfqpoint{3.312386in}{1.057199in}}%
\pgfpathlineto{\pgfqpoint{3.256481in}{1.099067in}}%
\pgfpathlineto{\pgfqpoint{3.202537in}{1.141694in}}%
\pgfpathlineto{\pgfqpoint{3.150650in}{1.185075in}}%
\pgfpathlineto{\pgfqpoint{3.100890in}{1.229192in}}%
\pgfpathlineto{\pgfqpoint{3.053295in}{1.274012in}}%
\pgfpathlineto{\pgfqpoint{3.007880in}{1.319500in}}%
\pgfpathlineto{\pgfqpoint{2.964655in}{1.365618in}}%
\pgfpathlineto{\pgfqpoint{2.923624in}{1.412330in}}%
\pgfpathlineto{\pgfqpoint{2.884799in}{1.459599in}}%
\pgfpathlineto{\pgfqpoint{2.848201in}{1.507396in}}%
\pgfpathlineto{\pgfqpoint{2.813868in}{1.555691in}}%
\pgfpathlineto{\pgfqpoint{2.781856in}{1.604456in}}%
\pgfpathlineto{\pgfqpoint{2.752255in}{1.653671in}}%
\pgfpathlineto{\pgfqpoint{2.725195in}{1.703320in}}%
\pgfpathlineto{\pgfqpoint{2.700850in}{1.753385in}}%
\pgfpathlineto{\pgfqpoint{2.679466in}{1.803849in}}%
\pgfpathlineto{\pgfqpoint{2.661377in}{1.854693in}}%
\pgfpathlineto{\pgfqpoint{2.647047in}{1.905890in}}%
\pgfpathlineto{\pgfqpoint{2.637138in}{1.957392in}}%
\pgfpathlineto{\pgfqpoint{2.632609in}{2.009107in}}%
\pgfpathlineto{\pgfqpoint{2.634930in}{2.060843in}}%
\pgfpathlineto{\pgfqpoint{2.646515in}{2.112162in}}%
\pgfpathlineto{\pgfqpoint{2.671731in}{2.161871in}}%
\pgfpathlineto{\pgfqpoint{2.671731in}{2.161871in}}%
\pgfpathlineto{\pgfqpoint{2.705432in}{2.196951in}}%
\pgfpathlineto{\pgfqpoint{2.705432in}{2.196951in}}%
\pgfpathlineto{\pgfqpoint{2.743925in}{2.218841in}}%
\pgfpathlineto{\pgfqpoint{2.743925in}{2.218841in}}%
\pgfpathlineto{\pgfqpoint{2.785795in}{2.229942in}}%
\pgfpathlineto{\pgfqpoint{2.832245in}{2.231361in}}%
\pgfpathlineto{\pgfqpoint{2.875368in}{2.224443in}}%
\pgfpathlineto{\pgfqpoint{2.918522in}{2.210205in}}%
\pgfpathlineto{\pgfqpoint{2.962882in}{2.187767in}}%
\pgfpathlineto{\pgfqpoint{3.007290in}{2.156080in}}%
\pgfpathlineto{\pgfqpoint{3.048548in}{2.114672in}}%
\pgfpathlineto{\pgfqpoint{3.079765in}{2.066105in}}%
\pgfpathlineto{\pgfqpoint{3.079765in}{2.066105in}}%
\pgfusepath{stroke}%
\end{pgfscope}%
\begin{pgfscope}%
\pgfpathrectangle{\pgfqpoint{0.647939in}{0.492442in}}{\pgfqpoint{4.273799in}{2.331163in}}%
\pgfusepath{clip}%
\pgfsetbuttcap%
\pgfsetroundjoin%
\pgfsetlinewidth{0.301125pt}%
\definecolor{currentstroke}{rgb}{0.500000,0.500000,0.500000}%
\pgfsetstrokecolor{currentstroke}%
\pgfsetstrokeopacity{0.300000}%
\pgfsetdash{}{0pt}%
\pgfpathmoveto{\pgfqpoint{4.338948in}{0.492442in}}%
\pgfpathlineto{\pgfqpoint{4.338948in}{0.492442in}}%
\pgfpathlineto{\pgfqpoint{4.294316in}{0.538162in}}%
\pgfpathlineto{\pgfqpoint{4.247658in}{0.583274in}}%
\pgfpathlineto{\pgfqpoint{4.198824in}{0.627696in}}%
\pgfpathlineto{\pgfqpoint{4.147677in}{0.671337in}}%
\pgfpathlineto{\pgfqpoint{4.094123in}{0.714110in}}%
\pgfpathlineto{\pgfqpoint{4.038142in}{0.755948in}}%
\pgfpathlineto{\pgfqpoint{3.979760in}{0.796797in}}%
\pgfpathlineto{\pgfqpoint{3.919104in}{0.836651in}}%
\pgfpathlineto{\pgfqpoint{3.856443in}{0.875572in}}%
\pgfpathlineto{\pgfqpoint{3.792134in}{0.913689in}}%
\pgfpathlineto{\pgfqpoint{3.726653in}{0.951209in}}%
\pgfpathlineto{\pgfqpoint{3.660540in}{0.988399in}}%
\pgfpathlineto{\pgfqpoint{3.594365in}{1.025556in}}%
\pgfpathlineto{\pgfqpoint{3.528700in}{1.062978in}}%
\pgfpathlineto{\pgfqpoint{3.464068in}{1.100929in}}%
\pgfpathlineto{\pgfqpoint{3.400931in}{1.139619in}}%
\pgfpathlineto{\pgfqpoint{3.339670in}{1.179191in}}%
\pgfpathlineto{\pgfqpoint{3.280569in}{1.219730in}}%
\pgfpathlineto{\pgfqpoint{3.223846in}{1.261266in}}%
\pgfpathlineto{\pgfqpoint{3.169650in}{1.303790in}}%
\pgfpathlineto{\pgfqpoint{3.118053in}{1.347267in}}%
\pgfpathlineto{\pgfqpoint{3.069099in}{1.391646in}}%
\pgfpathlineto{\pgfqpoint{3.022809in}{1.436871in}}%
\pgfpathlineto{\pgfqpoint{2.979196in}{1.482881in}}%
\pgfpathlineto{\pgfqpoint{2.938272in}{1.529621in}}%
\pgfpathlineto{\pgfqpoint{2.900060in}{1.577037in}}%
\pgfpathlineto{\pgfqpoint{2.864606in}{1.625085in}}%
\pgfusepath{stroke}%
\end{pgfscope}%
\begin{pgfscope}%
\pgfpathrectangle{\pgfqpoint{0.647939in}{0.492442in}}{\pgfqpoint{4.273799in}{2.331163in}}%
\pgfusepath{clip}%
\pgfsetbuttcap%
\pgfsetroundjoin%
\pgfsetlinewidth{0.301125pt}%
\definecolor{currentstroke}{rgb}{0.500000,0.500000,0.500000}%
\pgfsetstrokecolor{currentstroke}%
\pgfsetstrokeopacity{0.300000}%
\pgfsetdash{}{0pt}%
\pgfpathmoveto{\pgfqpoint{4.533211in}{0.492442in}}%
\pgfpathlineto{\pgfqpoint{4.533211in}{0.492442in}}%
\pgfpathlineto{\pgfqpoint{4.497502in}{0.540442in}}%
\pgfpathlineto{\pgfqpoint{4.460407in}{0.588126in}}%
\pgfpathlineto{\pgfqpoint{4.421764in}{0.635442in}}%
\pgfpathlineto{\pgfqpoint{4.381394in}{0.682325in}}%
\pgfpathlineto{\pgfqpoint{4.339090in}{0.728697in}}%
\pgfpathlineto{\pgfqpoint{4.294623in}{0.774461in}}%
\pgfpathlineto{\pgfqpoint{4.247739in}{0.819500in}}%
\pgfpathlineto{\pgfqpoint{4.198166in}{0.863672in}}%
\pgfpathlineto{\pgfqpoint{4.145620in}{0.906810in}}%
\pgfpathlineto{\pgfqpoint{4.089870in}{0.948733in}}%
\pgfpathlineto{\pgfqpoint{4.030775in}{0.989264in}}%
\pgfpathlineto{\pgfqpoint{3.968261in}{1.028241in}}%
\pgfpathlineto{\pgfqpoint{3.902510in}{1.065602in}}%
\pgfpathlineto{\pgfqpoint{3.833919in}{1.101416in}}%
\pgfpathlineto{\pgfqpoint{3.763109in}{1.135929in}}%
\pgfpathlineto{\pgfqpoint{3.690909in}{1.169580in}}%
\pgfpathlineto{\pgfqpoint{3.618226in}{1.202921in}}%
\pgfpathlineto{\pgfqpoint{3.545970in}{1.236534in}}%
\pgfpathlineto{\pgfqpoint{3.475019in}{1.270955in}}%
\pgfpathlineto{\pgfqpoint{3.406106in}{1.306578in}}%
\pgfpathlineto{\pgfqpoint{3.339817in}{1.343646in}}%
\pgfpathlineto{\pgfqpoint{3.276586in}{1.382268in}}%
\pgfpathlineto{\pgfqpoint{3.216688in}{1.422438in}}%
\pgfpathlineto{\pgfqpoint{3.160283in}{1.464084in}}%
\pgfpathlineto{\pgfqpoint{3.107428in}{1.507100in}}%
\pgfpathlineto{\pgfqpoint{3.058148in}{1.551363in}}%
\pgfpathlineto{\pgfqpoint{3.012445in}{1.596751in}}%
\pgfpathlineto{\pgfqpoint{2.970322in}{1.643159in}}%
\pgfpathlineto{\pgfqpoint{2.931815in}{1.690495in}}%
\pgfpathlineto{\pgfqpoint{2.897019in}{1.738681in}}%
\pgfpathlineto{\pgfqpoint{2.866115in}{1.787646in}}%
\pgfpathlineto{\pgfqpoint{2.839403in}{1.837337in}}%
\pgfpathlineto{\pgfqpoint{2.817375in}{1.887703in}}%
\pgfpathlineto{\pgfqpoint{2.800823in}{1.938680in}}%
\pgfpathlineto{\pgfqpoint{2.791085in}{1.990157in}}%
\pgfpathlineto{\pgfqpoint{2.790616in}{2.041861in}}%
\pgfpathlineto{\pgfqpoint{2.804701in}{2.092834in}}%
\pgfpathlineto{\pgfqpoint{2.804701in}{2.092834in}}%
\pgfpathlineto{\pgfqpoint{2.827804in}{2.124217in}}%
\pgfpathlineto{\pgfqpoint{2.827804in}{2.124217in}}%
\pgfpathlineto{\pgfqpoint{2.856898in}{2.142610in}}%
\pgfpathlineto{\pgfqpoint{2.856898in}{2.142610in}}%
\pgfpathlineto{\pgfqpoint{2.889052in}{2.150049in}}%
\pgfpathlineto{\pgfqpoint{2.923905in}{2.148290in}}%
\pgfusepath{stroke}%
\end{pgfscope}%
\begin{pgfscope}%
\pgfpathrectangle{\pgfqpoint{0.647939in}{0.492442in}}{\pgfqpoint{4.273799in}{2.331163in}}%
\pgfusepath{clip}%
\pgfsetbuttcap%
\pgfsetroundjoin%
\pgfsetlinewidth{0.301125pt}%
\definecolor{currentstroke}{rgb}{0.500000,0.500000,0.500000}%
\pgfsetstrokecolor{currentstroke}%
\pgfsetstrokeopacity{0.300000}%
\pgfsetdash{}{0pt}%
\pgfpathmoveto{\pgfqpoint{4.630343in}{0.492442in}}%
\pgfpathlineto{\pgfqpoint{4.630343in}{0.492442in}}%
\pgfpathlineto{\pgfqpoint{4.598887in}{0.541319in}}%
\pgfpathlineto{\pgfqpoint{4.566410in}{0.589996in}}%
\pgfpathlineto{\pgfqpoint{4.532791in}{0.638443in}}%
\pgfpathlineto{\pgfqpoint{4.497909in}{0.686623in}}%
\pgfpathlineto{\pgfqpoint{4.461617in}{0.734490in}}%
\pgfpathlineto{\pgfqpoint{4.423731in}{0.781987in}}%
\pgfpathlineto{\pgfqpoint{4.384031in}{0.829039in}}%
\pgfpathlineto{\pgfqpoint{4.342260in}{0.875553in}}%
\pgfpathlineto{\pgfqpoint{4.298115in}{0.921407in}}%
\pgfpathlineto{\pgfqpoint{4.251246in}{0.966447in}}%
\pgfpathlineto{\pgfqpoint{4.201248in}{1.010470in}}%
\pgfpathlineto{\pgfqpoint{4.147672in}{1.053221in}}%
\pgfpathlineto{\pgfqpoint{4.090094in}{1.094390in}}%
\pgfpathlineto{\pgfqpoint{4.028203in}{1.133643in}}%
\pgfpathlineto{\pgfqpoint{3.961834in}{1.170653in}}%
\pgfpathlineto{\pgfqpoint{3.891202in}{1.205239in}}%
\pgfpathlineto{\pgfqpoint{3.816957in}{1.237509in}}%
\pgfpathlineto{\pgfqpoint{3.740129in}{1.267946in}}%
\pgfpathlineto{\pgfqpoint{3.661949in}{1.297353in}}%
\pgfpathlineto{\pgfqpoint{3.583713in}{1.326712in}}%
\pgfpathlineto{\pgfqpoint{3.506676in}{1.356985in}}%
\pgfpathlineto{\pgfqpoint{3.431939in}{1.388903in}}%
\pgfpathlineto{\pgfqpoint{3.360341in}{1.422880in}}%
\pgfpathlineto{\pgfqpoint{3.292498in}{1.459075in}}%
\pgfpathlineto{\pgfqpoint{3.228814in}{1.497447in}}%
\pgfpathlineto{\pgfqpoint{3.169470in}{1.537838in}}%
\pgfpathlineto{\pgfqpoint{3.114514in}{1.580043in}}%
\pgfpathlineto{\pgfqpoint{3.063927in}{1.623853in}}%
\pgfusepath{stroke}%
\end{pgfscope}%
\begin{pgfscope}%
\pgfpathrectangle{\pgfqpoint{0.647939in}{0.492442in}}{\pgfqpoint{4.273799in}{2.331163in}}%
\pgfusepath{clip}%
\pgfsetbuttcap%
\pgfsetroundjoin%
\pgfsetlinewidth{0.301125pt}%
\definecolor{currentstroke}{rgb}{0.500000,0.500000,0.500000}%
\pgfsetstrokecolor{currentstroke}%
\pgfsetstrokeopacity{0.300000}%
\pgfsetdash{}{0pt}%
\pgfpathmoveto{\pgfqpoint{4.727475in}{0.492442in}}%
\pgfpathlineto{\pgfqpoint{4.727475in}{0.492442in}}%
\pgfpathlineto{\pgfqpoint{4.699903in}{0.542013in}}%
\pgfpathlineto{\pgfqpoint{4.671642in}{0.591468in}}%
\pgfpathlineto{\pgfqpoint{4.642631in}{0.640793in}}%
\pgfpathlineto{\pgfqpoint{4.612787in}{0.689970in}}%
\pgfpathlineto{\pgfqpoint{4.582015in}{0.738977in}}%
\pgfpathlineto{\pgfqpoint{4.550214in}{0.787786in}}%
\pgfpathlineto{\pgfqpoint{4.517254in}{0.836365in}}%
\pgfpathlineto{\pgfqpoint{4.482967in}{0.884670in}}%
\pgfpathlineto{\pgfqpoint{4.447155in}{0.932646in}}%
\pgfpathlineto{\pgfqpoint{4.409587in}{0.980220in}}%
\pgfpathlineto{\pgfqpoint{4.369978in}{1.027294in}}%
\pgfpathlineto{\pgfqpoint{4.327950in}{1.073737in}}%
\pgfpathlineto{\pgfqpoint{4.283028in}{1.119362in}}%
\pgfpathlineto{\pgfqpoint{4.234608in}{1.163905in}}%
\pgfpathlineto{\pgfqpoint{4.181952in}{1.206988in}}%
\pgfpathlineto{\pgfqpoint{4.124253in}{1.248082in}}%
\pgfpathlineto{\pgfqpoint{4.060618in}{1.286454in}}%
\pgfpathlineto{\pgfqpoint{3.990444in}{1.321246in}}%
\pgfpathlineto{\pgfqpoint{3.913884in}{1.351749in}}%
\pgfpathlineto{\pgfqpoint{3.832033in}{1.377887in}}%
\pgfpathlineto{\pgfqpoint{3.746779in}{1.400640in}}%
\pgfpathlineto{\pgfqpoint{3.660151in}{1.421835in}}%
\pgfpathlineto{\pgfqpoint{3.573956in}{1.443523in}}%
\pgfpathlineto{\pgfqpoint{3.489788in}{1.467404in}}%
\pgfpathlineto{\pgfqpoint{3.409042in}{1.494540in}}%
\pgfpathlineto{\pgfqpoint{3.332852in}{1.525337in}}%
\pgfpathlineto{\pgfqpoint{3.262001in}{1.559696in}}%
\pgfpathlineto{\pgfqpoint{3.196769in}{1.597244in}}%
\pgfpathlineto{\pgfqpoint{3.137227in}{1.637522in}}%
\pgfpathlineto{\pgfqpoint{3.083302in}{1.680085in}}%
\pgfpathlineto{\pgfqpoint{3.034845in}{1.724574in}}%
\pgfpathlineto{\pgfqpoint{2.991793in}{1.770695in}}%
\pgfpathlineto{\pgfqpoint{2.954223in}{1.818222in}}%
\pgfpathlineto{\pgfqpoint{2.922427in}{1.866984in}}%
\pgfpathlineto{\pgfqpoint{2.897066in}{1.916855in}}%
\pgfpathlineto{\pgfqpoint{2.879482in}{1.967690in}}%
\pgfpathlineto{\pgfqpoint{2.872555in}{2.019219in}}%
\pgfpathlineto{\pgfqpoint{2.884035in}{2.070164in}}%
\pgfpathlineto{\pgfqpoint{2.884035in}{2.070164in}}%
\pgfpathlineto{\pgfqpoint{2.903077in}{2.092785in}}%
\pgfpathlineto{\pgfqpoint{2.903077in}{2.092785in}}%
\pgfpathlineto{\pgfqpoint{2.926955in}{2.103707in}}%
\pgfpathlineto{\pgfqpoint{2.957394in}{2.104625in}}%
\pgfpathlineto{\pgfqpoint{2.982923in}{2.097852in}}%
\pgfpathlineto{\pgfqpoint{3.008857in}{2.084236in}}%
\pgfusepath{stroke}%
\end{pgfscope}%
\begin{pgfscope}%
\pgfpathrectangle{\pgfqpoint{0.647939in}{0.492442in}}{\pgfqpoint{4.273799in}{2.331163in}}%
\pgfusepath{clip}%
\pgfsetbuttcap%
\pgfsetroundjoin%
\pgfsetlinewidth{0.301125pt}%
\definecolor{currentstroke}{rgb}{0.500000,0.500000,0.500000}%
\pgfsetstrokecolor{currentstroke}%
\pgfsetstrokeopacity{0.300000}%
\pgfsetdash{}{0pt}%
\pgfpathmoveto{\pgfqpoint{4.824607in}{0.492442in}}%
\pgfpathlineto{\pgfqpoint{4.824607in}{0.492442in}}%
\pgfpathlineto{\pgfqpoint{4.800506in}{0.542549in}}%
\pgfpathlineto{\pgfqpoint{4.775987in}{0.592595in}}%
\pgfpathlineto{\pgfqpoint{4.751013in}{0.642574in}}%
\pgfpathlineto{\pgfqpoint{4.725552in}{0.692480in}}%
\pgfpathlineto{\pgfqpoint{4.699561in}{0.742304in}}%
\pgfpathlineto{\pgfqpoint{4.672986in}{0.792037in}}%
\pgfpathlineto{\pgfqpoint{4.645774in}{0.841666in}}%
\pgfpathlineto{\pgfqpoint{4.617846in}{0.891177in}}%
\pgfpathlineto{\pgfqpoint{4.589123in}{0.940552in}}%
\pgfpathlineto{\pgfqpoint{4.559511in}{0.989769in}}%
\pgfpathlineto{\pgfqpoint{4.528870in}{1.038799in}}%
\pgfpathlineto{\pgfqpoint{4.497044in}{1.087604in}}%
\pgfpathlineto{\pgfqpoint{4.463844in}{1.136133in}}%
\pgfpathlineto{\pgfqpoint{4.429007in}{1.184316in}}%
\pgfpathlineto{\pgfqpoint{4.392175in}{1.232054in}}%
\pgfpathlineto{\pgfqpoint{4.352867in}{1.279199in}}%
\pgfpathlineto{\pgfqpoint{4.310418in}{1.325521in}}%
\pgfpathlineto{\pgfqpoint{4.263879in}{1.370645in}}%
\pgfpathlineto{\pgfqpoint{4.211868in}{1.413920in}}%
\pgfpathlineto{\pgfqpoint{4.152373in}{1.454154in}}%
\pgfpathlineto{\pgfqpoint{4.082913in}{1.489161in}}%
\pgfpathlineto{\pgfqpoint{4.002140in}{1.515662in}}%
\pgfpathlineto{\pgfqpoint{3.919046in}{1.530688in}}%
\pgfpathlineto{\pgfqpoint{3.833022in}{1.537927in}}%
\pgfpathlineto{\pgfqpoint{3.738400in}{1.541602in}}%
\pgfpathlineto{\pgfqpoint{3.643822in}{1.545550in}}%
\pgfpathlineto{\pgfqpoint{3.550169in}{1.553386in}}%
\pgfpathlineto{\pgfqpoint{3.459084in}{1.567375in}}%
\pgfpathlineto{\pgfqpoint{3.372683in}{1.588353in}}%
\pgfpathlineto{\pgfqpoint{3.292754in}{1.615938in}}%
\pgfusepath{stroke}%
\end{pgfscope}%
\begin{pgfscope}%
\pgfpathrectangle{\pgfqpoint{0.647939in}{0.492442in}}{\pgfqpoint{4.273799in}{2.331163in}}%
\pgfusepath{clip}%
\pgfsetbuttcap%
\pgfsetroundjoin%
\pgfsetlinewidth{0.301125pt}%
\definecolor{currentstroke}{rgb}{0.500000,0.500000,0.500000}%
\pgfsetstrokecolor{currentstroke}%
\pgfsetstrokeopacity{0.300000}%
\pgfsetdash{}{0pt}%
\pgfpathmoveto{\pgfqpoint{4.921738in}{0.492442in}}%
\pgfpathlineto{\pgfqpoint{4.921738in}{0.492442in}}%
\pgfpathlineto{\pgfqpoint{4.900681in}{0.542956in}}%
\pgfpathlineto{\pgfqpoint{4.879405in}{0.593442in}}%
\pgfpathlineto{\pgfqpoint{4.857901in}{0.643899in}}%
\pgfpathlineto{\pgfqpoint{4.836157in}{0.694326in}}%
\pgfpathlineto{\pgfqpoint{4.814160in}{0.744720in}}%
\pgfpathlineto{\pgfqpoint{4.791897in}{0.795080in}}%
\pgfpathlineto{\pgfqpoint{4.769352in}{0.845401in}}%
\pgfpathlineto{\pgfqpoint{4.746507in}{0.895683in}}%
\pgfpathlineto{\pgfqpoint{4.723340in}{0.945920in}}%
\pgfpathlineto{\pgfqpoint{4.699830in}{0.996110in}}%
\pgfpathlineto{\pgfqpoint{4.675946in}{1.046247in}}%
\pgfpathlineto{\pgfqpoint{4.651656in}{1.096325in}}%
\pgfpathlineto{\pgfqpoint{4.626916in}{1.146338in}}%
\pgfpathlineto{\pgfqpoint{4.601681in}{1.196277in}}%
\pgfpathlineto{\pgfqpoint{4.575881in}{1.246130in}}%
\pgfpathlineto{\pgfqpoint{4.549445in}{1.295884in}}%
\pgfpathlineto{\pgfqpoint{4.522279in}{1.345518in}}%
\pgfpathlineto{\pgfqpoint{4.494233in}{1.395007in}}%
\pgfpathlineto{\pgfqpoint{4.465134in}{1.444312in}}%
\pgfpathlineto{\pgfqpoint{4.434721in}{1.493378in}}%
\pgfpathlineto{\pgfqpoint{4.402566in}{1.542114in}}%
\pgfpathlineto{\pgfqpoint{4.368047in}{1.590360in}}%
\pgfpathlineto{\pgfqpoint{4.330070in}{1.637803in}}%
\pgfpathlineto{\pgfqpoint{4.286425in}{1.683710in}}%
\pgfpathlineto{\pgfqpoint{4.231950in}{1.725789in}}%
\pgfpathlineto{\pgfqpoint{4.231950in}{1.725789in}}%
\pgfpathlineto{\pgfqpoint{4.187180in}{1.746648in}}%
\pgfpathlineto{\pgfqpoint{4.187180in}{1.746648in}}%
\pgfpathlineto{\pgfqpoint{4.143456in}{1.755505in}}%
\pgfpathlineto{\pgfqpoint{4.096566in}{1.754578in}}%
\pgfpathlineto{\pgfqpoint{4.052250in}{1.746650in}}%
\pgfpathlineto{\pgfqpoint{3.998832in}{1.731498in}}%
\pgfpathlineto{\pgfqpoint{3.923584in}{1.705778in}}%
\pgfpathlineto{\pgfqpoint{3.843319in}{1.678302in}}%
\pgfpathlineto{\pgfqpoint{3.760070in}{1.653664in}}%
\pgfpathlineto{\pgfqpoint{3.672410in}{1.634331in}}%
\pgfpathlineto{\pgfqpoint{3.580426in}{1.622857in}}%
\pgfpathlineto{\pgfqpoint{3.487544in}{1.621616in}}%
\pgfusepath{stroke}%
\end{pgfscope}%
\begin{pgfscope}%
\pgfpathrectangle{\pgfqpoint{0.647939in}{0.492442in}}{\pgfqpoint{4.273799in}{2.331163in}}%
\pgfusepath{clip}%
\pgfsetbuttcap%
\pgfsetroundjoin%
\pgfsetlinewidth{0.301125pt}%
\definecolor{currentstroke}{rgb}{0.500000,0.500000,0.500000}%
\pgfsetstrokecolor{currentstroke}%
\pgfsetstrokeopacity{0.300000}%
\pgfsetdash{}{0pt}%
\pgfpathmoveto{\pgfqpoint{4.921738in}{0.704366in}}%
\pgfpathlineto{\pgfqpoint{4.921738in}{0.704366in}}%
\pgfpathlineto{\pgfqpoint{4.902467in}{0.755091in}}%
\pgfpathlineto{\pgfqpoint{4.883105in}{0.805806in}}%
\pgfpathlineto{\pgfqpoint{4.863653in}{0.856511in}}%
\pgfpathlineto{\pgfqpoint{4.844116in}{0.907205in}}%
\pgfpathlineto{\pgfqpoint{4.824495in}{0.957890in}}%
\pgfpathlineto{\pgfqpoint{4.804799in}{1.008567in}}%
\pgfpathlineto{\pgfqpoint{4.785030in}{1.059234in}}%
\pgfpathlineto{\pgfqpoint{4.765199in}{1.109895in}}%
\pgfpathlineto{\pgfqpoint{4.745314in}{1.160549in}}%
\pgfpathlineto{\pgfqpoint{4.725388in}{1.211198in}}%
\pgfpathlineto{\pgfqpoint{4.705435in}{1.261845in}}%
\pgfpathlineto{\pgfqpoint{4.685475in}{1.312490in}}%
\pgfpathlineto{\pgfqpoint{4.665535in}{1.363137in}}%
\pgfpathlineto{\pgfqpoint{4.645643in}{1.413790in}}%
\pgfpathlineto{\pgfqpoint{4.625842in}{1.464453in}}%
\pgfpathlineto{\pgfqpoint{4.606180in}{1.515132in}}%
\pgfpathlineto{\pgfqpoint{4.586728in}{1.565835in}}%
\pgfpathlineto{\pgfqpoint{4.567573in}{1.616571in}}%
\pgfpathlineto{\pgfqpoint{4.548842in}{1.667353in}}%
\pgfpathlineto{\pgfqpoint{4.530700in}{1.718200in}}%
\pgfpathlineto{\pgfqpoint{4.513397in}{1.769132in}}%
\pgfpathlineto{\pgfqpoint{4.497295in}{1.820179in}}%
\pgfpathlineto{\pgfqpoint{4.482921in}{1.871378in}}%
\pgfpathlineto{\pgfqpoint{4.471103in}{1.922767in}}%
\pgfpathlineto{\pgfqpoint{4.463084in}{1.974367in}}%
\pgfpathlineto{\pgfqpoint{4.460629in}{2.026119in}}%
\pgfpathlineto{\pgfqpoint{4.465761in}{2.077780in}}%
\pgfpathlineto{\pgfqpoint{4.479637in}{2.128919in}}%
\pgfpathlineto{\pgfqpoint{4.501432in}{2.179238in}}%
\pgfpathlineto{\pgfqpoint{4.528831in}{2.228748in}}%
\pgfpathlineto{\pgfqpoint{4.559545in}{2.277687in}}%
\pgfpathlineto{\pgfqpoint{4.591953in}{2.326299in}}%
\pgfpathlineto{\pgfqpoint{4.625131in}{2.374790in}}%
\pgfpathlineto{\pgfqpoint{4.658502in}{2.423265in}}%
\pgfpathlineto{\pgfqpoint{4.691696in}{2.471768in}}%
\pgfpathlineto{\pgfqpoint{4.724535in}{2.520333in}}%
\pgfpathlineto{\pgfqpoint{4.756963in}{2.568992in}}%
\pgfpathlineto{\pgfqpoint{4.788937in}{2.617756in}}%
\pgfpathlineto{\pgfqpoint{4.820400in}{2.666615in}}%
\pgfpathlineto{\pgfqpoint{4.851340in}{2.715568in}}%
\pgfpathlineto{\pgfqpoint{4.881795in}{2.764621in}}%
\pgfpathlineto{\pgfqpoint{4.911762in}{2.813767in}}%
\pgfpathlineto{\pgfqpoint{4.917708in}{2.823605in}}%
\pgfusepath{stroke}%
\end{pgfscope}%
\begin{pgfscope}%
\pgfpathrectangle{\pgfqpoint{0.647939in}{0.492442in}}{\pgfqpoint{4.273799in}{2.331163in}}%
\pgfusepath{clip}%
\pgfsetbuttcap%
\pgfsetroundjoin%
\pgfsetlinewidth{0.301125pt}%
\definecolor{currentstroke}{rgb}{0.500000,0.500000,0.500000}%
\pgfsetstrokecolor{currentstroke}%
\pgfsetstrokeopacity{0.300000}%
\pgfsetdash{}{0pt}%
\pgfpathmoveto{\pgfqpoint{4.921738in}{0.969271in}}%
\pgfpathlineto{\pgfqpoint{4.921738in}{0.969271in}}%
\pgfpathlineto{\pgfqpoint{4.905075in}{1.020270in}}%
\pgfpathlineto{\pgfqpoint{4.888520in}{1.071280in}}%
\pgfpathlineto{\pgfqpoint{4.872092in}{1.122302in}}%
\pgfpathlineto{\pgfqpoint{4.855821in}{1.173339in}}%
\pgfpathlineto{\pgfqpoint{4.839737in}{1.224393in}}%
\pgfpathlineto{\pgfqpoint{4.823876in}{1.275469in}}%
\pgfpathlineto{\pgfqpoint{4.808282in}{1.326568in}}%
\pgfpathlineto{\pgfqpoint{4.792999in}{1.377696in}}%
\pgfpathlineto{\pgfqpoint{4.778087in}{1.428856in}}%
\pgfpathlineto{\pgfqpoint{4.763619in}{1.480054in}}%
\pgfpathlineto{\pgfqpoint{4.749665in}{1.531294in}}%
\pgfpathlineto{\pgfqpoint{4.736324in}{1.582583in}}%
\pgfpathlineto{\pgfqpoint{4.723711in}{1.633926in}}%
\pgfpathlineto{\pgfqpoint{4.711951in}{1.685330in}}%
\pgfpathlineto{\pgfqpoint{4.701198in}{1.736799in}}%
\pgfpathlineto{\pgfqpoint{4.691637in}{1.788337in}}%
\pgfpathlineto{\pgfqpoint{4.683480in}{1.839947in}}%
\pgfpathlineto{\pgfqpoint{4.676963in}{1.891625in}}%
\pgfpathlineto{\pgfqpoint{4.672352in}{1.943364in}}%
\pgfpathlineto{\pgfqpoint{4.669932in}{1.995146in}}%
\pgfpathlineto{\pgfqpoint{4.669988in}{2.046944in}}%
\pgfpathlineto{\pgfqpoint{4.672777in}{2.098719in}}%
\pgfpathlineto{\pgfqpoint{4.678490in}{2.150420in}}%
\pgfpathlineto{\pgfqpoint{4.687221in}{2.201994in}}%
\pgfpathlineto{\pgfqpoint{4.698943in}{2.253391in}}%
\pgfpathlineto{\pgfqpoint{4.713498in}{2.304573in}}%
\pgfpathlineto{\pgfqpoint{4.730601in}{2.355518in}}%
\pgfpathlineto{\pgfqpoint{4.749926in}{2.406229in}}%
\pgfusepath{stroke}%
\end{pgfscope}%
\begin{pgfscope}%
\pgfpathrectangle{\pgfqpoint{0.647939in}{0.492442in}}{\pgfqpoint{4.273799in}{2.331163in}}%
\pgfusepath{clip}%
\pgfsetbuttcap%
\pgfsetroundjoin%
\pgfsetlinewidth{0.301125pt}%
\definecolor{currentstroke}{rgb}{0.500000,0.500000,0.500000}%
\pgfsetstrokecolor{currentstroke}%
\pgfsetstrokeopacity{0.300000}%
\pgfsetdash{}{0pt}%
\pgfpathmoveto{\pgfqpoint{4.921738in}{1.287157in}}%
\pgfpathlineto{\pgfqpoint{4.921738in}{1.287157in}}%
\pgfpathlineto{\pgfqpoint{4.908908in}{1.338485in}}%
\pgfpathlineto{\pgfqpoint{4.896494in}{1.389843in}}%
\pgfpathlineto{\pgfqpoint{4.884545in}{1.441235in}}%
\pgfpathlineto{\pgfqpoint{4.873119in}{1.492661in}}%
\pgfpathlineto{\pgfqpoint{4.862286in}{1.544126in}}%
\pgfpathlineto{\pgfqpoint{4.852120in}{1.595631in}}%
\pgfpathlineto{\pgfqpoint{4.842702in}{1.647178in}}%
\pgfpathlineto{\pgfqpoint{4.834122in}{1.698769in}}%
\pgfpathlineto{\pgfqpoint{4.826483in}{1.750404in}}%
\pgfpathlineto{\pgfqpoint{4.819896in}{1.802082in}}%
\pgfpathlineto{\pgfqpoint{4.814477in}{1.853799in}}%
\pgfpathlineto{\pgfqpoint{4.810348in}{1.905553in}}%
\pgfpathlineto{\pgfqpoint{4.807635in}{1.957333in}}%
\pgfpathlineto{\pgfqpoint{4.806463in}{2.009131in}}%
\pgfpathlineto{\pgfqpoint{4.806949in}{2.060931in}}%
\pgfpathlineto{\pgfqpoint{4.809197in}{2.112717in}}%
\pgfpathlineto{\pgfqpoint{4.813289in}{2.164470in}}%
\pgfpathlineto{\pgfqpoint{4.819276in}{2.216167in}}%
\pgfpathlineto{\pgfqpoint{4.827177in}{2.267787in}}%
\pgfpathlineto{\pgfqpoint{4.836969in}{2.319311in}}%
\pgfpathlineto{\pgfqpoint{4.848583in}{2.370722in}}%
\pgfpathlineto{\pgfqpoint{4.861916in}{2.422008in}}%
\pgfpathlineto{\pgfqpoint{4.876847in}{2.473163in}}%
\pgfpathlineto{\pgfqpoint{4.893220in}{2.524187in}}%
\pgfpathlineto{\pgfqpoint{4.910867in}{2.575083in}}%
\pgfpathlineto{\pgfqpoint{4.921738in}{2.605404in}}%
\pgfusepath{stroke}%
\end{pgfscope}%
\begin{pgfscope}%
\pgfpathrectangle{\pgfqpoint{0.647939in}{0.492442in}}{\pgfqpoint{4.273799in}{2.331163in}}%
\pgfusepath{clip}%
\pgfsetbuttcap%
\pgfsetroundjoin%
\pgfsetlinewidth{0.301125pt}%
\definecolor{currentstroke}{rgb}{0.500000,0.500000,0.500000}%
\pgfsetstrokecolor{currentstroke}%
\pgfsetstrokeopacity{0.300000}%
\pgfsetdash{}{0pt}%
\pgfpathmoveto{\pgfqpoint{4.921738in}{1.499081in}}%
\pgfpathlineto{\pgfqpoint{4.921738in}{1.499081in}}%
\pgfpathlineto{\pgfqpoint{4.912017in}{1.550612in}}%
\pgfpathlineto{\pgfqpoint{4.902958in}{1.602178in}}%
\pgfpathlineto{\pgfqpoint{4.894629in}{1.653782in}}%
\pgfpathlineto{\pgfqpoint{4.887106in}{1.705422in}}%
\pgfpathlineto{\pgfqpoint{4.880470in}{1.757098in}}%
\pgfpathlineto{\pgfqpoint{4.874810in}{1.808808in}}%
\pgfusepath{stroke}%
\end{pgfscope}%
\begin{pgfscope}%
\pgfpathrectangle{\pgfqpoint{0.647939in}{0.492442in}}{\pgfqpoint{4.273799in}{2.331163in}}%
\pgfusepath{clip}%
\pgfsetbuttcap%
\pgfsetroundjoin%
\pgfsetlinewidth{0.301125pt}%
\definecolor{currentstroke}{rgb}{0.500000,0.500000,0.500000}%
\pgfsetstrokecolor{currentstroke}%
\pgfsetstrokeopacity{0.300000}%
\pgfsetdash{}{0pt}%
\pgfpathmoveto{\pgfqpoint{4.921738in}{1.869948in}}%
\pgfpathlineto{\pgfqpoint{4.921738in}{1.869948in}}%
\pgfpathlineto{\pgfqpoint{4.918903in}{1.921727in}}%
\pgfpathlineto{\pgfqpoint{4.917199in}{1.973521in}}%
\pgfpathlineto{\pgfqpoint{4.916696in}{2.025323in}}%
\pgfpathlineto{\pgfqpoint{4.917458in}{2.077124in}}%
\pgfpathlineto{\pgfqpoint{4.919541in}{2.128913in}}%
\pgfpathlineto{\pgfqpoint{4.921738in}{2.170261in}}%
\pgfusepath{stroke}%
\end{pgfscope}%
\begin{pgfscope}%
\pgfpathrectangle{\pgfqpoint{0.647939in}{0.492442in}}{\pgfqpoint{4.273799in}{2.331163in}}%
\pgfusepath{clip}%
\pgfsetbuttcap%
\pgfsetroundjoin%
\pgfsetlinewidth{0.301125pt}%
\definecolor{currentstroke}{rgb}{0.500000,0.500000,0.500000}%
\pgfsetstrokecolor{currentstroke}%
\pgfsetstrokeopacity{0.300000}%
\pgfsetdash{}{0pt}%
\pgfpathmoveto{\pgfqpoint{4.811517in}{2.465761in}}%
\pgfpathlineto{\pgfqpoint{4.831503in}{2.516397in}}%
\pgfpathlineto{\pgfqpoint{4.852746in}{2.566882in}}%
\pgfpathlineto{\pgfqpoint{4.875005in}{2.617235in}}%
\pgfpathlineto{\pgfqpoint{4.898063in}{2.667481in}}%
\pgfpathlineto{\pgfqpoint{4.921738in}{2.717643in}}%
\pgfpathlineto{\pgfqpoint{4.921738in}{2.717643in}}%
\pgfusepath{stroke}%
\end{pgfscope}%
\begin{pgfscope}%
\pgfpathrectangle{\pgfqpoint{0.647939in}{0.492442in}}{\pgfqpoint{4.273799in}{2.331163in}}%
\pgfusepath{clip}%
\pgfsetbuttcap%
\pgfsetroundjoin%
\pgfsetlinewidth{0.301125pt}%
\definecolor{currentstroke}{rgb}{0.500000,0.500000,0.500000}%
\pgfsetstrokecolor{currentstroke}%
\pgfsetstrokeopacity{0.300000}%
\pgfsetdash{}{0pt}%
\pgfpathmoveto{\pgfqpoint{4.282024in}{2.823605in}}%
\pgfpathlineto{\pgfqpoint{4.285284in}{2.820250in}}%
\pgfpathlineto{\pgfqpoint{4.331796in}{2.775121in}}%
\pgfpathlineto{\pgfqpoint{4.383893in}{2.731915in}}%
\pgfpathlineto{\pgfqpoint{4.430764in}{2.701002in}}%
\pgfpathlineto{\pgfqpoint{4.472908in}{2.680682in}}%
\pgfpathlineto{\pgfqpoint{4.514626in}{2.668316in}}%
\pgfpathlineto{\pgfqpoint{4.564685in}{2.664363in}}%
\pgfpathlineto{\pgfqpoint{4.613543in}{2.672129in}}%
\pgfpathlineto{\pgfqpoint{4.613543in}{2.672129in}}%
\pgfpathlineto{\pgfqpoint{4.669217in}{2.694680in}}%
\pgfpathlineto{\pgfqpoint{4.669217in}{2.694680in}}%
\pgfpathlineto{\pgfqpoint{4.729496in}{2.734256in}}%
\pgfpathlineto{\pgfqpoint{4.779985in}{2.777955in}}%
\pgfpathlineto{\pgfqpoint{4.824607in}{2.823605in}}%
\pgfpathlineto{\pgfqpoint{4.824607in}{2.823605in}}%
\pgfusepath{stroke}%
\end{pgfscope}%
\begin{pgfscope}%
\pgfpathrectangle{\pgfqpoint{0.647939in}{0.492442in}}{\pgfqpoint{4.273799in}{2.331163in}}%
\pgfusepath{clip}%
\pgfsetbuttcap%
\pgfsetroundjoin%
\pgfsetlinewidth{0.301125pt}%
\definecolor{currentstroke}{rgb}{0.500000,0.500000,0.500000}%
\pgfsetstrokecolor{currentstroke}%
\pgfsetstrokeopacity{0.300000}%
\pgfsetdash{}{0pt}%
\pgfpathmoveto{\pgfqpoint{4.144684in}{2.823605in}}%
\pgfpathlineto{\pgfqpoint{4.144684in}{2.823605in}}%
\pgfpathlineto{\pgfqpoint{4.182039in}{2.775979in}}%
\pgfpathlineto{\pgfqpoint{4.220818in}{2.728696in}}%
\pgfpathlineto{\pgfqpoint{4.261712in}{2.681954in}}%
\pgfpathlineto{\pgfqpoint{4.305891in}{2.636127in}}%
\pgfpathlineto{\pgfqpoint{4.355581in}{2.592058in}}%
\pgfpathlineto{\pgfqpoint{4.415670in}{2.552296in}}%
\pgfpathlineto{\pgfqpoint{4.415670in}{2.552296in}}%
\pgfpathlineto{\pgfqpoint{4.462172in}{2.533721in}}%
\pgfpathlineto{\pgfqpoint{4.462172in}{2.533721in}}%
\pgfpathlineto{\pgfqpoint{4.504792in}{2.527192in}}%
\pgfpathlineto{\pgfqpoint{4.548755in}{2.531310in}}%
\pgfpathlineto{\pgfqpoint{4.585433in}{2.542802in}}%
\pgfpathlineto{\pgfqpoint{4.623123in}{2.561794in}}%
\pgfpathlineto{\pgfqpoint{4.665066in}{2.590689in}}%
\pgfpathlineto{\pgfqpoint{4.713431in}{2.632903in}}%
\pgfpathlineto{\pgfqpoint{4.757727in}{2.678637in}}%
\pgfusepath{stroke}%
\end{pgfscope}%
\begin{pgfscope}%
\pgfpathrectangle{\pgfqpoint{0.647939in}{0.492442in}}{\pgfqpoint{4.273799in}{2.331163in}}%
\pgfusepath{clip}%
\pgfsetbuttcap%
\pgfsetroundjoin%
\pgfsetlinewidth{0.301125pt}%
\definecolor{currentstroke}{rgb}{0.500000,0.500000,0.500000}%
\pgfsetstrokecolor{currentstroke}%
\pgfsetstrokeopacity{0.300000}%
\pgfsetdash{}{0pt}%
\pgfpathmoveto{\pgfqpoint{4.047552in}{2.823605in}}%
\pgfpathlineto{\pgfqpoint{4.047552in}{2.823605in}}%
\pgfpathlineto{\pgfqpoint{4.081630in}{2.775252in}}%
\pgfpathlineto{\pgfqpoint{4.116134in}{2.726990in}}%
\pgfpathlineto{\pgfqpoint{4.151291in}{2.678869in}}%
\pgfpathlineto{\pgfqpoint{4.187440in}{2.630967in}}%
\pgfpathlineto{\pgfqpoint{4.225124in}{2.583422in}}%
\pgfpathlineto{\pgfqpoint{4.265251in}{2.536485in}}%
\pgfpathlineto{\pgfqpoint{4.309504in}{2.490696in}}%
\pgfpathlineto{\pgfqpoint{4.361654in}{2.447559in}}%
\pgfpathlineto{\pgfqpoint{4.361654in}{2.447559in}}%
\pgfpathlineto{\pgfqpoint{4.409327in}{2.421170in}}%
\pgfpathlineto{\pgfqpoint{4.409327in}{2.421170in}}%
\pgfpathlineto{\pgfqpoint{4.449194in}{2.410433in}}%
\pgfpathlineto{\pgfqpoint{4.449194in}{2.410433in}}%
\pgfpathlineto{\pgfqpoint{4.486634in}{2.410344in}}%
\pgfpathlineto{\pgfqpoint{4.521053in}{2.418513in}}%
\pgfpathlineto{\pgfqpoint{4.553757in}{2.433086in}}%
\pgfpathlineto{\pgfqpoint{4.590860in}{2.456749in}}%
\pgfpathlineto{\pgfqpoint{4.634377in}{2.492681in}}%
\pgfusepath{stroke}%
\end{pgfscope}%
\begin{pgfscope}%
\pgfpathrectangle{\pgfqpoint{0.647939in}{0.492442in}}{\pgfqpoint{4.273799in}{2.331163in}}%
\pgfusepath{clip}%
\pgfsetbuttcap%
\pgfsetroundjoin%
\pgfsetlinewidth{0.301125pt}%
\definecolor{currentstroke}{rgb}{0.500000,0.500000,0.500000}%
\pgfsetstrokecolor{currentstroke}%
\pgfsetstrokeopacity{0.300000}%
\pgfsetdash{}{0pt}%
\pgfpathmoveto{\pgfqpoint{3.950420in}{2.823605in}}%
\pgfpathlineto{\pgfqpoint{3.950420in}{2.823605in}}%
\pgfpathlineto{\pgfqpoint{3.982540in}{2.774855in}}%
\pgfpathlineto{\pgfqpoint{4.014553in}{2.726083in}}%
\pgfpathlineto{\pgfqpoint{4.046550in}{2.677308in}}%
\pgfpathlineto{\pgfqpoint{4.078644in}{2.628553in}}%
\pgfpathlineto{\pgfqpoint{4.110966in}{2.579844in}}%
\pgfpathlineto{\pgfqpoint{4.143702in}{2.531216in}}%
\pgfpathlineto{\pgfqpoint{4.177170in}{2.482739in}}%
\pgfpathlineto{\pgfqpoint{4.211875in}{2.434528in}}%
\pgfpathlineto{\pgfqpoint{4.248695in}{2.386800in}}%
\pgfpathlineto{\pgfqpoint{4.289483in}{2.340085in}}%
\pgfpathlineto{\pgfqpoint{4.339241in}{2.296339in}}%
\pgfpathlineto{\pgfqpoint{4.339241in}{2.296339in}}%
\pgfpathlineto{\pgfqpoint{4.374515in}{2.277164in}}%
\pgfpathlineto{\pgfqpoint{4.374515in}{2.277164in}}%
\pgfpathlineto{\pgfqpoint{4.407874in}{2.269881in}}%
\pgfpathlineto{\pgfqpoint{4.443403in}{2.274073in}}%
\pgfpathlineto{\pgfqpoint{4.470261in}{2.284393in}}%
\pgfusepath{stroke}%
\end{pgfscope}%
\begin{pgfscope}%
\pgfpathrectangle{\pgfqpoint{0.647939in}{0.492442in}}{\pgfqpoint{4.273799in}{2.331163in}}%
\pgfusepath{clip}%
\pgfsetbuttcap%
\pgfsetroundjoin%
\pgfsetlinewidth{0.301125pt}%
\definecolor{currentstroke}{rgb}{0.500000,0.500000,0.500000}%
\pgfsetstrokecolor{currentstroke}%
\pgfsetstrokeopacity{0.300000}%
\pgfsetdash{}{0pt}%
\pgfpathmoveto{\pgfqpoint{3.853289in}{2.823605in}}%
\pgfpathlineto{\pgfqpoint{3.853289in}{2.823605in}}%
\pgfpathlineto{\pgfqpoint{3.884345in}{2.774650in}}%
\pgfpathlineto{\pgfqpoint{3.914994in}{2.725618in}}%
\pgfpathlineto{\pgfqpoint{3.945259in}{2.676516in}}%
\pgfpathlineto{\pgfqpoint{3.975146in}{2.627344in}}%
\pgfpathlineto{\pgfqpoint{4.004681in}{2.578110in}}%
\pgfpathlineto{\pgfqpoint{4.033895in}{2.528819in}}%
\pgfpathlineto{\pgfqpoint{4.062804in}{2.479475in}}%
\pgfpathlineto{\pgfqpoint{4.091439in}{2.430083in}}%
\pgfpathlineto{\pgfqpoint{4.119856in}{2.380656in}}%
\pgfpathlineto{\pgfqpoint{4.148097in}{2.331198in}}%
\pgfpathlineto{\pgfqpoint{4.176259in}{2.281729in}}%
\pgfpathlineto{\pgfqpoint{4.204527in}{2.232285in}}%
\pgfpathlineto{\pgfqpoint{4.233214in}{2.182915in}}%
\pgfpathlineto{\pgfqpoint{4.263308in}{2.133831in}}%
\pgfpathlineto{\pgfqpoint{4.299559in}{2.087097in}}%
\pgfpathlineto{\pgfqpoint{4.299559in}{2.087097in}}%
\pgfpathlineto{\pgfqpoint{4.318864in}{2.075238in}}%
\pgfpathlineto{\pgfqpoint{4.318864in}{2.075238in}}%
\pgfpathlineto{\pgfqpoint{4.334141in}{2.076116in}}%
\pgfpathlineto{\pgfqpoint{4.345461in}{2.081178in}}%
\pgfpathlineto{\pgfqpoint{4.364291in}{2.094826in}}%
\pgfusepath{stroke}%
\end{pgfscope}%
\begin{pgfscope}%
\pgfpathrectangle{\pgfqpoint{0.647939in}{0.492442in}}{\pgfqpoint{4.273799in}{2.331163in}}%
\pgfusepath{clip}%
\pgfsetbuttcap%
\pgfsetroundjoin%
\pgfsetlinewidth{0.301125pt}%
\definecolor{currentstroke}{rgb}{0.500000,0.500000,0.500000}%
\pgfsetstrokecolor{currentstroke}%
\pgfsetstrokeopacity{0.300000}%
\pgfsetdash{}{0pt}%
\pgfpathmoveto{\pgfqpoint{3.756157in}{2.823605in}}%
\pgfpathlineto{\pgfqpoint{3.756157in}{2.823605in}}%
\pgfpathlineto{\pgfqpoint{3.786834in}{2.774579in}}%
\pgfpathlineto{\pgfqpoint{3.816895in}{2.725439in}}%
\pgfpathlineto{\pgfqpoint{3.846323in}{2.676186in}}%
\pgfpathlineto{\pgfqpoint{3.875105in}{2.626819in}}%
\pgfpathlineto{\pgfqpoint{3.903227in}{2.577339in}}%
\pgfpathlineto{\pgfqpoint{3.930644in}{2.527741in}}%
\pgfpathlineto{\pgfqpoint{3.957310in}{2.478023in}}%
\pgfpathlineto{\pgfqpoint{3.983163in}{2.428176in}}%
\pgfpathlineto{\pgfqpoint{4.008088in}{2.378190in}}%
\pgfpathlineto{\pgfqpoint{4.031946in}{2.328049in}}%
\pgfpathlineto{\pgfqpoint{4.054508in}{2.277731in}}%
\pgfpathlineto{\pgfqpoint{4.075443in}{2.227207in}}%
\pgfpathlineto{\pgfqpoint{4.094226in}{2.176434in}}%
\pgfpathlineto{\pgfqpoint{4.109993in}{2.125362in}}%
\pgfpathlineto{\pgfqpoint{4.121311in}{2.073953in}}%
\pgfpathlineto{\pgfqpoint{4.125777in}{2.022259in}}%
\pgfpathlineto{\pgfqpoint{4.119832in}{1.970688in}}%
\pgfpathlineto{\pgfqpoint{4.100116in}{1.920272in}}%
\pgfpathlineto{\pgfqpoint{4.066267in}{1.872198in}}%
\pgfpathlineto{\pgfqpoint{4.020851in}{1.827007in}}%
\pgfpathlineto{\pgfqpoint{3.966563in}{1.784716in}}%
\pgfpathlineto{\pgfqpoint{3.904762in}{1.745566in}}%
\pgfpathlineto{\pgfqpoint{3.836109in}{1.710020in}}%
\pgfusepath{stroke}%
\end{pgfscope}%
\begin{pgfscope}%
\pgfpathrectangle{\pgfqpoint{0.647939in}{0.492442in}}{\pgfqpoint{4.273799in}{2.331163in}}%
\pgfusepath{clip}%
\pgfsetbuttcap%
\pgfsetroundjoin%
\pgfsetlinewidth{0.301125pt}%
\definecolor{currentstroke}{rgb}{0.500000,0.500000,0.500000}%
\pgfsetstrokecolor{currentstroke}%
\pgfsetstrokeopacity{0.300000}%
\pgfsetdash{}{0pt}%
\pgfpathmoveto{\pgfqpoint{3.659025in}{2.823605in}}%
\pgfpathlineto{\pgfqpoint{3.659025in}{2.823605in}}%
\pgfpathlineto{\pgfqpoint{3.689882in}{2.774613in}}%
\pgfpathlineto{\pgfqpoint{3.719951in}{2.725475in}}%
\pgfpathlineto{\pgfqpoint{3.749211in}{2.676191in}}%
\pgfpathlineto{\pgfqpoint{3.777643in}{2.626764in}}%
\pgfpathlineto{\pgfqpoint{3.805204in}{2.577191in}}%
\pgfpathlineto{\pgfqpoint{3.831838in}{2.527467in}}%
\pgfpathlineto{\pgfqpoint{3.857479in}{2.477588in}}%
\pgfpathlineto{\pgfqpoint{3.882020in}{2.427545in}}%
\pgfpathlineto{\pgfqpoint{3.905335in}{2.377328in}}%
\pgfpathlineto{\pgfqpoint{3.927239in}{2.326923in}}%
\pgfpathlineto{\pgfqpoint{3.947488in}{2.276314in}}%
\pgfpathlineto{\pgfqpoint{3.965741in}{2.225481in}}%
\pgfpathlineto{\pgfqpoint{3.981511in}{2.174404in}}%
\pgfpathlineto{\pgfqpoint{3.994122in}{2.123070in}}%
\pgfpathlineto{\pgfqpoint{4.002602in}{2.071491in}}%
\pgfpathlineto{\pgfqpoint{4.005599in}{2.019743in}}%
\pgfpathlineto{\pgfqpoint{4.001346in}{1.968049in}}%
\pgfpathlineto{\pgfqpoint{3.987815in}{1.916875in}}%
\pgfpathlineto{\pgfqpoint{3.963300in}{1.866975in}}%
\pgfpathlineto{\pgfqpoint{3.926983in}{1.819301in}}%
\pgfpathlineto{\pgfqpoint{3.879175in}{1.774768in}}%
\pgfusepath{stroke}%
\end{pgfscope}%
\begin{pgfscope}%
\pgfpathrectangle{\pgfqpoint{0.647939in}{0.492442in}}{\pgfqpoint{4.273799in}{2.331163in}}%
\pgfusepath{clip}%
\pgfsetbuttcap%
\pgfsetroundjoin%
\pgfsetlinewidth{0.301125pt}%
\definecolor{currentstroke}{rgb}{0.500000,0.500000,0.500000}%
\pgfsetstrokecolor{currentstroke}%
\pgfsetstrokeopacity{0.300000}%
\pgfsetdash{}{0pt}%
\pgfpathmoveto{\pgfqpoint{3.561893in}{2.823605in}}%
\pgfpathlineto{\pgfqpoint{3.561893in}{2.823605in}}%
\pgfpathlineto{\pgfqpoint{3.593421in}{2.774740in}}%
\pgfpathlineto{\pgfqpoint{3.624013in}{2.725699in}}%
\pgfpathlineto{\pgfqpoint{3.653655in}{2.676484in}}%
\pgfpathlineto{\pgfqpoint{3.682317in}{2.627096in}}%
\pgfpathlineto{\pgfqpoint{3.709947in}{2.577535in}}%
\pgfpathlineto{\pgfqpoint{3.736492in}{2.527797in}}%
\pgfpathlineto{\pgfqpoint{3.761873in}{2.477879in}}%
\pgfpathlineto{\pgfqpoint{3.785984in}{2.427774in}}%
\pgfpathlineto{\pgfqpoint{3.808689in}{2.377474in}}%
\pgfpathlineto{\pgfqpoint{3.829808in}{2.326970in}}%
\pgfpathlineto{\pgfqpoint{3.849100in}{2.276250in}}%
\pgfpathlineto{\pgfqpoint{3.866249in}{2.225303in}}%
\pgfpathlineto{\pgfqpoint{3.880827in}{2.174120in}}%
\pgfpathlineto{\pgfqpoint{3.892263in}{2.122704in}}%
\pgfpathlineto{\pgfqpoint{3.899791in}{2.071078in}}%
\pgfpathlineto{\pgfqpoint{3.902387in}{2.019317in}}%
\pgfpathlineto{\pgfqpoint{3.898727in}{1.967593in}}%
\pgfpathlineto{\pgfqpoint{3.887216in}{1.916240in}}%
\pgfpathlineto{\pgfqpoint{3.866106in}{1.865840in}}%
\pgfpathlineto{\pgfqpoint{3.833813in}{1.817278in}}%
\pgfpathlineto{\pgfqpoint{3.789214in}{1.771767in}}%
\pgfpathlineto{\pgfqpoint{3.731728in}{1.730884in}}%
\pgfpathlineto{\pgfqpoint{3.661384in}{1.696589in}}%
\pgfpathlineto{\pgfqpoint{3.583018in}{1.672515in}}%
\pgfpathlineto{\pgfqpoint{3.504519in}{1.660371in}}%
\pgfpathlineto{\pgfqpoint{3.429496in}{1.658907in}}%
\pgfpathlineto{\pgfqpoint{3.358065in}{1.666467in}}%
\pgfpathlineto{\pgfqpoint{3.289226in}{1.682205in}}%
\pgfpathlineto{\pgfqpoint{3.222490in}{1.705956in}}%
\pgfpathlineto{\pgfqpoint{3.157291in}{1.738196in}}%
\pgfpathlineto{\pgfqpoint{3.096590in}{1.777780in}}%
\pgfpathlineto{\pgfqpoint{3.044779in}{1.821009in}}%
\pgfpathlineto{\pgfqpoint{3.001449in}{1.866956in}}%
\pgfpathlineto{\pgfqpoint{2.966783in}{1.915055in}}%
\pgfusepath{stroke}%
\end{pgfscope}%
\begin{pgfscope}%
\pgfpathrectangle{\pgfqpoint{0.647939in}{0.492442in}}{\pgfqpoint{4.273799in}{2.331163in}}%
\pgfusepath{clip}%
\pgfsetbuttcap%
\pgfsetroundjoin%
\pgfsetlinewidth{0.301125pt}%
\definecolor{currentstroke}{rgb}{0.500000,0.500000,0.500000}%
\pgfsetstrokecolor{currentstroke}%
\pgfsetstrokeopacity{0.300000}%
\pgfsetdash{}{0pt}%
\pgfpathmoveto{\pgfqpoint{3.464761in}{2.823605in}}%
\pgfpathlineto{\pgfqpoint{3.464761in}{2.823605in}}%
\pgfpathlineto{\pgfqpoint{3.497437in}{2.774965in}}%
\pgfpathlineto{\pgfqpoint{3.529041in}{2.726115in}}%
\pgfpathlineto{\pgfqpoint{3.559559in}{2.677060in}}%
\pgfpathlineto{\pgfqpoint{3.588954in}{2.627802in}}%
\pgfpathlineto{\pgfqpoint{3.617181in}{2.578340in}}%
\pgfpathlineto{\pgfqpoint{3.644187in}{2.528677in}}%
\pgfpathlineto{\pgfqpoint{3.669885in}{2.478807in}}%
\pgfpathlineto{\pgfqpoint{3.694178in}{2.428729in}}%
\pgfpathlineto{\pgfqpoint{3.716934in}{2.378437in}}%
\pgfpathlineto{\pgfqpoint{3.737974in}{2.327923in}}%
\pgfpathlineto{\pgfqpoint{3.757072in}{2.277181in}}%
\pgfpathlineto{\pgfqpoint{3.773931in}{2.226205in}}%
\pgfpathlineto{\pgfqpoint{3.788166in}{2.174994in}}%
\pgfpathlineto{\pgfqpoint{3.799254in}{2.123555in}}%
\pgfusepath{stroke}%
\end{pgfscope}%
\begin{pgfscope}%
\pgfpathrectangle{\pgfqpoint{0.647939in}{0.492442in}}{\pgfqpoint{4.273799in}{2.331163in}}%
\pgfusepath{clip}%
\pgfsetbuttcap%
\pgfsetroundjoin%
\pgfsetlinewidth{0.301125pt}%
\definecolor{currentstroke}{rgb}{0.500000,0.500000,0.500000}%
\pgfsetstrokecolor{currentstroke}%
\pgfsetstrokeopacity{0.300000}%
\pgfsetdash{}{0pt}%
\pgfpathmoveto{\pgfqpoint{3.367630in}{2.823605in}}%
\pgfpathlineto{\pgfqpoint{3.367630in}{2.823605in}}%
\pgfpathlineto{\pgfqpoint{3.401951in}{2.775304in}}%
\pgfpathlineto{\pgfqpoint{3.435057in}{2.726752in}}%
\pgfpathlineto{\pgfqpoint{3.466928in}{2.677954in}}%
\pgfpathlineto{\pgfqpoint{3.497531in}{2.628915in}}%
\pgfpathlineto{\pgfqpoint{3.526831in}{2.579640in}}%
\pgfpathlineto{\pgfqpoint{3.554778in}{2.530132in}}%
\pgfpathlineto{\pgfqpoint{3.581288in}{2.480389in}}%
\pgfpathlineto{\pgfqpoint{3.606267in}{2.430412in}}%
\pgfpathlineto{\pgfqpoint{3.629588in}{2.380197in}}%
\pgfpathlineto{\pgfqpoint{3.651080in}{2.329741in}}%
\pgfpathlineto{\pgfqpoint{3.670526in}{2.279038in}}%
\pgfpathlineto{\pgfqpoint{3.687643in}{2.228088in}}%
\pgfpathlineto{\pgfqpoint{3.702066in}{2.176893in}}%
\pgfusepath{stroke}%
\end{pgfscope}%
\begin{pgfscope}%
\pgfpathrectangle{\pgfqpoint{0.647939in}{0.492442in}}{\pgfqpoint{4.273799in}{2.331163in}}%
\pgfusepath{clip}%
\pgfsetbuttcap%
\pgfsetroundjoin%
\pgfsetlinewidth{0.301125pt}%
\definecolor{currentstroke}{rgb}{0.500000,0.500000,0.500000}%
\pgfsetstrokecolor{currentstroke}%
\pgfsetstrokeopacity{0.300000}%
\pgfsetdash{}{0pt}%
\pgfpathmoveto{\pgfqpoint{3.270498in}{2.823605in}}%
\pgfpathlineto{\pgfqpoint{3.270498in}{2.823605in}}%
\pgfpathlineto{\pgfqpoint{3.306999in}{2.775783in}}%
\pgfpathlineto{\pgfqpoint{3.342113in}{2.727653in}}%
\pgfpathlineto{\pgfqpoint{3.375828in}{2.679226in}}%
\pgfpathlineto{\pgfqpoint{3.408121in}{2.630511in}}%
\pgfpathlineto{\pgfqpoint{3.438965in}{2.581518in}}%
\pgfpathlineto{\pgfqpoint{3.468313in}{2.532252in}}%
\pgfpathlineto{\pgfqpoint{3.496087in}{2.482716in}}%
\pgfusepath{stroke}%
\end{pgfscope}%
\begin{pgfscope}%
\pgfpathrectangle{\pgfqpoint{0.647939in}{0.492442in}}{\pgfqpoint{4.273799in}{2.331163in}}%
\pgfusepath{clip}%
\pgfsetbuttcap%
\pgfsetroundjoin%
\pgfsetlinewidth{0.301125pt}%
\definecolor{currentstroke}{rgb}{0.500000,0.500000,0.500000}%
\pgfsetstrokecolor{currentstroke}%
\pgfsetstrokeopacity{0.300000}%
\pgfsetdash{}{0pt}%
\pgfpathmoveto{\pgfqpoint{3.076234in}{2.823605in}}%
\pgfpathlineto{\pgfqpoint{3.076234in}{2.823605in}}%
\pgfpathlineto{\pgfqpoint{3.118892in}{2.777327in}}%
\pgfpathlineto{\pgfqpoint{3.159724in}{2.730561in}}%
\pgfpathlineto{\pgfqpoint{3.198743in}{2.683335in}}%
\pgfpathlineto{\pgfqpoint{3.235950in}{2.635676in}}%
\pgfpathlineto{\pgfqpoint{3.271337in}{2.587606in}}%
\pgfpathlineto{\pgfqpoint{3.304882in}{2.539146in}}%
\pgfpathlineto{\pgfqpoint{3.336535in}{2.490309in}}%
\pgfpathlineto{\pgfqpoint{3.366220in}{2.441105in}}%
\pgfpathlineto{\pgfqpoint{3.393841in}{2.391545in}}%
\pgfpathlineto{\pgfqpoint{3.419258in}{2.341637in}}%
\pgfpathlineto{\pgfqpoint{3.442275in}{2.291384in}}%
\pgfpathlineto{\pgfqpoint{3.462636in}{2.240792in}}%
\pgfpathlineto{\pgfqpoint{3.479989in}{2.189869in}}%
\pgfpathlineto{\pgfqpoint{3.493858in}{2.138632in}}%
\pgfpathlineto{\pgfqpoint{3.503588in}{2.087118in}}%
\pgfpathlineto{\pgfqpoint{3.508258in}{2.035404in}}%
\pgfpathlineto{\pgfqpoint{3.506523in}{1.983654in}}%
\pgfpathlineto{\pgfqpoint{3.496356in}{1.932224in}}%
\pgfpathlineto{\pgfqpoint{3.474510in}{1.881964in}}%
\pgfpathlineto{\pgfqpoint{3.435637in}{1.835117in}}%
\pgfpathlineto{\pgfqpoint{3.435637in}{1.835117in}}%
\pgfpathlineto{\pgfqpoint{3.394419in}{1.807245in}}%
\pgfpathlineto{\pgfqpoint{3.394419in}{1.807245in}}%
\pgfpathlineto{\pgfqpoint{3.348764in}{1.790349in}}%
\pgfpathlineto{\pgfqpoint{3.295161in}{1.783160in}}%
\pgfpathlineto{\pgfqpoint{3.244968in}{1.785953in}}%
\pgfpathlineto{\pgfqpoint{3.196231in}{1.796506in}}%
\pgfpathlineto{\pgfqpoint{3.147417in}{1.814717in}}%
\pgfusepath{stroke}%
\end{pgfscope}%
\begin{pgfscope}%
\pgfpathrectangle{\pgfqpoint{0.647939in}{0.492442in}}{\pgfqpoint{4.273799in}{2.331163in}}%
\pgfusepath{clip}%
\pgfsetbuttcap%
\pgfsetroundjoin%
\pgfsetlinewidth{0.301125pt}%
\definecolor{currentstroke}{rgb}{0.500000,0.500000,0.500000}%
\pgfsetstrokecolor{currentstroke}%
\pgfsetstrokeopacity{0.300000}%
\pgfsetdash{}{0pt}%
\pgfpathmoveto{\pgfqpoint{2.881971in}{2.823605in}}%
\pgfpathlineto{\pgfqpoint{2.881971in}{2.823605in}}%
\pgfpathlineto{\pgfqpoint{2.933525in}{2.780109in}}%
\pgfpathlineto{\pgfqpoint{2.982652in}{2.735783in}}%
\pgfpathlineto{\pgfqpoint{3.029375in}{2.690691in}}%
\pgfpathlineto{\pgfqpoint{3.073727in}{2.644893in}}%
\pgfpathlineto{\pgfqpoint{3.115733in}{2.598440in}}%
\pgfpathlineto{\pgfqpoint{3.155406in}{2.551380in}}%
\pgfpathlineto{\pgfqpoint{3.192737in}{2.503752in}}%
\pgfpathlineto{\pgfqpoint{3.227701in}{2.455593in}}%
\pgfpathlineto{\pgfqpoint{3.260235in}{2.406932in}}%
\pgfpathlineto{\pgfqpoint{3.290233in}{2.357789in}}%
\pgfpathlineto{\pgfqpoint{3.317535in}{2.308180in}}%
\pgfpathlineto{\pgfqpoint{3.341920in}{2.258122in}}%
\pgfpathlineto{\pgfqpoint{3.363059in}{2.207629in}}%
\pgfpathlineto{\pgfqpoint{3.380494in}{2.156720in}}%
\pgfpathlineto{\pgfqpoint{3.393563in}{2.105429in}}%
\pgfpathlineto{\pgfqpoint{3.401279in}{2.053828in}}%
\pgfpathlineto{\pgfqpoint{3.402119in}{2.002075in}}%
\pgfpathlineto{\pgfqpoint{3.393551in}{1.950577in}}%
\pgfpathlineto{\pgfqpoint{3.370973in}{1.900518in}}%
\pgfpathlineto{\pgfqpoint{3.370973in}{1.900518in}}%
\pgfpathlineto{\pgfqpoint{3.339563in}{1.865716in}}%
\pgfpathlineto{\pgfqpoint{3.339563in}{1.865716in}}%
\pgfpathlineto{\pgfqpoint{3.302670in}{1.843779in}}%
\pgfpathlineto{\pgfqpoint{3.302670in}{1.843779in}}%
\pgfpathlineto{\pgfqpoint{3.262073in}{1.832525in}}%
\pgfusepath{stroke}%
\end{pgfscope}%
\begin{pgfscope}%
\pgfpathrectangle{\pgfqpoint{0.647939in}{0.492442in}}{\pgfqpoint{4.273799in}{2.331163in}}%
\pgfusepath{clip}%
\pgfsetbuttcap%
\pgfsetroundjoin%
\pgfsetlinewidth{0.301125pt}%
\definecolor{currentstroke}{rgb}{0.500000,0.500000,0.500000}%
\pgfsetstrokecolor{currentstroke}%
\pgfsetstrokeopacity{0.300000}%
\pgfsetdash{}{0pt}%
\pgfpathmoveto{\pgfqpoint{2.687707in}{2.823605in}}%
\pgfpathlineto{\pgfqpoint{2.687707in}{2.823605in}}%
\pgfpathlineto{\pgfqpoint{2.750624in}{2.784827in}}%
\pgfpathlineto{\pgfqpoint{2.810505in}{2.744642in}}%
\pgfpathlineto{\pgfqpoint{2.867295in}{2.703143in}}%
\pgfpathlineto{\pgfqpoint{2.920993in}{2.660438in}}%
\pgfpathlineto{\pgfqpoint{2.971648in}{2.616637in}}%
\pgfpathlineto{\pgfqpoint{3.019310in}{2.571843in}}%
\pgfpathlineto{\pgfqpoint{3.064026in}{2.526151in}}%
\pgfpathlineto{\pgfqpoint{3.105828in}{2.479645in}}%
\pgfpathlineto{\pgfqpoint{3.144720in}{2.432395in}}%
\pgfpathlineto{\pgfqpoint{3.180668in}{2.384458in}}%
\pgfpathlineto{\pgfqpoint{3.213587in}{2.335879in}}%
\pgfusepath{stroke}%
\end{pgfscope}%
\begin{pgfscope}%
\pgfpathrectangle{\pgfqpoint{0.647939in}{0.492442in}}{\pgfqpoint{4.273799in}{2.331163in}}%
\pgfusepath{clip}%
\pgfsetbuttcap%
\pgfsetroundjoin%
\pgfsetlinewidth{0.301125pt}%
\definecolor{currentstroke}{rgb}{0.500000,0.500000,0.500000}%
\pgfsetstrokecolor{currentstroke}%
\pgfsetstrokeopacity{0.300000}%
\pgfsetdash{}{0pt}%
\pgfpathmoveto{\pgfqpoint{2.493443in}{2.823605in}}%
\pgfpathlineto{\pgfqpoint{2.493443in}{2.823605in}}%
\pgfpathlineto{\pgfqpoint{2.567971in}{2.791536in}}%
\pgfpathlineto{\pgfqpoint{2.639569in}{2.757546in}}%
\pgfpathlineto{\pgfqpoint{2.707806in}{2.721559in}}%
\pgfpathlineto{\pgfqpoint{2.772416in}{2.683635in}}%
\pgfpathlineto{\pgfqpoint{2.833302in}{2.643917in}}%
\pgfpathlineto{\pgfqpoint{2.890468in}{2.602579in}}%
\pgfusepath{stroke}%
\end{pgfscope}%
\begin{pgfscope}%
\pgfpathrectangle{\pgfqpoint{0.647939in}{0.492442in}}{\pgfqpoint{4.273799in}{2.331163in}}%
\pgfusepath{clip}%
\pgfsetbuttcap%
\pgfsetroundjoin%
\pgfsetlinewidth{0.301125pt}%
\definecolor{currentstroke}{rgb}{0.500000,0.500000,0.500000}%
\pgfsetstrokecolor{currentstroke}%
\pgfsetstrokeopacity{0.300000}%
\pgfsetdash{}{0pt}%
\pgfpathmoveto{\pgfqpoint{2.299180in}{2.823605in}}%
\pgfpathlineto{\pgfqpoint{2.299180in}{2.823605in}}%
\pgfpathlineto{\pgfqpoint{2.381667in}{2.797940in}}%
\pgfpathlineto{\pgfqpoint{2.462836in}{2.771074in}}%
\pgfpathlineto{\pgfqpoint{2.541655in}{2.742223in}}%
\pgfpathlineto{\pgfqpoint{2.617274in}{2.710951in}}%
\pgfpathlineto{\pgfqpoint{2.689117in}{2.677145in}}%
\pgfusepath{stroke}%
\end{pgfscope}%
\begin{pgfscope}%
\pgfpathrectangle{\pgfqpoint{0.647939in}{0.492442in}}{\pgfqpoint{4.273799in}{2.331163in}}%
\pgfusepath{clip}%
\pgfsetbuttcap%
\pgfsetroundjoin%
\pgfsetlinewidth{0.301125pt}%
\definecolor{currentstroke}{rgb}{0.500000,0.500000,0.500000}%
\pgfsetstrokecolor{currentstroke}%
\pgfsetstrokeopacity{0.300000}%
\pgfsetdash{}{0pt}%
\pgfpathmoveto{\pgfqpoint{2.007784in}{2.823605in}}%
\pgfpathlineto{\pgfqpoint{2.007784in}{2.823605in}}%
\pgfpathlineto{\pgfqpoint{2.089500in}{2.797334in}}%
\pgfpathlineto{\pgfqpoint{2.174770in}{2.774598in}}%
\pgfpathlineto{\pgfqpoint{2.261834in}{2.753918in}}%
\pgfpathlineto{\pgfqpoint{2.349100in}{2.733498in}}%
\pgfpathlineto{\pgfqpoint{2.435192in}{2.711697in}}%
\pgfpathlineto{\pgfqpoint{2.518880in}{2.687322in}}%
\pgfpathlineto{\pgfqpoint{2.599112in}{2.659728in}}%
\pgfpathlineto{\pgfqpoint{2.675105in}{2.628756in}}%
\pgfpathlineto{\pgfqpoint{2.746330in}{2.594583in}}%
\pgfpathlineto{\pgfqpoint{2.812603in}{2.557564in}}%
\pgfpathlineto{\pgfqpoint{2.873976in}{2.518097in}}%
\pgfpathlineto{\pgfqpoint{2.930587in}{2.476555in}}%
\pgfpathlineto{\pgfqpoint{2.982624in}{2.433258in}}%
\pgfpathlineto{\pgfqpoint{3.030241in}{2.388468in}}%
\pgfpathlineto{\pgfqpoint{3.073542in}{2.342392in}}%
\pgfpathlineto{\pgfqpoint{3.112542in}{2.295184in}}%
\pgfpathlineto{\pgfqpoint{3.147131in}{2.246962in}}%
\pgfpathlineto{\pgfqpoint{3.177033in}{2.197821in}}%
\pgfusepath{stroke}%
\end{pgfscope}%
\begin{pgfscope}%
\pgfpathrectangle{\pgfqpoint{0.647939in}{0.492442in}}{\pgfqpoint{4.273799in}{2.331163in}}%
\pgfusepath{clip}%
\pgfsetbuttcap%
\pgfsetroundjoin%
\pgfsetlinewidth{0.301125pt}%
\definecolor{currentstroke}{rgb}{0.500000,0.500000,0.500000}%
\pgfsetstrokecolor{currentstroke}%
\pgfsetstrokeopacity{0.300000}%
\pgfsetdash{}{0pt}%
\pgfpathmoveto{\pgfqpoint{1.813521in}{2.823605in}}%
\pgfpathlineto{\pgfqpoint{1.813521in}{2.823605in}}%
\pgfpathlineto{\pgfqpoint{1.882676in}{2.788238in}}%
\pgfpathlineto{\pgfqpoint{1.959382in}{2.757922in}}%
\pgfpathlineto{\pgfqpoint{2.042805in}{2.733428in}}%
\pgfpathlineto{\pgfqpoint{2.130873in}{2.714246in}}%
\pgfpathlineto{\pgfqpoint{2.221300in}{2.698508in}}%
\pgfpathlineto{\pgfqpoint{2.312301in}{2.683723in}}%
\pgfpathlineto{\pgfqpoint{2.402466in}{2.667561in}}%
\pgfusepath{stroke}%
\end{pgfscope}%
\begin{pgfscope}%
\pgfpathrectangle{\pgfqpoint{0.647939in}{0.492442in}}{\pgfqpoint{4.273799in}{2.331163in}}%
\pgfusepath{clip}%
\pgfsetbuttcap%
\pgfsetroundjoin%
\pgfsetlinewidth{0.301125pt}%
\definecolor{currentstroke}{rgb}{0.500000,0.500000,0.500000}%
\pgfsetstrokecolor{currentstroke}%
\pgfsetstrokeopacity{0.300000}%
\pgfsetdash{}{0pt}%
\pgfpathmoveto{\pgfqpoint{1.619257in}{2.823605in}}%
\pgfpathlineto{\pgfqpoint{1.619257in}{2.823605in}}%
\pgfpathlineto{\pgfqpoint{1.668119in}{2.779231in}}%
\pgfpathlineto{\pgfqpoint{1.722810in}{2.736955in}}%
\pgfpathlineto{\pgfqpoint{1.785164in}{2.698022in}}%
\pgfpathlineto{\pgfqpoint{1.857078in}{2.664495in}}%
\pgfpathlineto{\pgfqpoint{1.939208in}{2.639063in}}%
\pgfpathlineto{\pgfqpoint{2.027104in}{2.623413in}}%
\pgfpathlineto{\pgfqpoint{2.119800in}{2.615232in}}%
\pgfpathlineto{\pgfqpoint{2.214400in}{2.610924in}}%
\pgfpathlineto{\pgfqpoint{2.309013in}{2.606683in}}%
\pgfpathlineto{\pgfqpoint{2.402890in}{2.599441in}}%
\pgfpathlineto{\pgfqpoint{2.494872in}{2.587118in}}%
\pgfpathlineto{\pgfqpoint{2.583326in}{2.568704in}}%
\pgfpathlineto{\pgfqpoint{2.666659in}{2.544200in}}%
\pgfpathlineto{\pgfqpoint{2.743857in}{2.514289in}}%
\pgfpathlineto{\pgfqpoint{2.814593in}{2.479912in}}%
\pgfpathlineto{\pgfqpoint{2.878991in}{2.441961in}}%
\pgfpathlineto{\pgfqpoint{2.937348in}{2.401179in}}%
\pgfpathlineto{\pgfqpoint{2.990017in}{2.358131in}}%
\pgfusepath{stroke}%
\end{pgfscope}%
\begin{pgfscope}%
\pgfpathrectangle{\pgfqpoint{0.647939in}{0.492442in}}{\pgfqpoint{4.273799in}{2.331163in}}%
\pgfusepath{clip}%
\pgfsetbuttcap%
\pgfsetroundjoin%
\pgfsetlinewidth{0.301125pt}%
\definecolor{currentstroke}{rgb}{0.500000,0.500000,0.500000}%
\pgfsetstrokecolor{currentstroke}%
\pgfsetstrokeopacity{0.300000}%
\pgfsetdash{}{0pt}%
\pgfpathmoveto{\pgfqpoint{1.522125in}{2.823605in}}%
\pgfpathlineto{\pgfqpoint{1.522125in}{2.823605in}}%
\pgfpathlineto{\pgfqpoint{1.561606in}{2.776506in}}%
\pgfpathlineto{\pgfqpoint{1.604589in}{2.730334in}}%
\pgfpathlineto{\pgfqpoint{1.652259in}{2.685575in}}%
\pgfpathlineto{\pgfqpoint{1.706410in}{2.643123in}}%
\pgfpathlineto{\pgfqpoint{1.769752in}{2.604758in}}%
\pgfpathlineto{\pgfqpoint{1.845218in}{2.573931in}}%
\pgfpathlineto{\pgfqpoint{1.926700in}{2.555956in}}%
\pgfpathlineto{\pgfqpoint{2.003200in}{2.549807in}}%
\pgfpathlineto{\pgfqpoint{2.085497in}{2.550504in}}%
\pgfpathlineto{\pgfqpoint{2.179913in}{2.555473in}}%
\pgfpathlineto{\pgfqpoint{2.274376in}{2.560382in}}%
\pgfpathlineto{\pgfqpoint{2.369154in}{2.561755in}}%
\pgfpathlineto{\pgfqpoint{2.463429in}{2.557090in}}%
\pgfusepath{stroke}%
\end{pgfscope}%
\begin{pgfscope}%
\pgfpathrectangle{\pgfqpoint{0.647939in}{0.492442in}}{\pgfqpoint{4.273799in}{2.331163in}}%
\pgfusepath{clip}%
\pgfsetbuttcap%
\pgfsetroundjoin%
\pgfsetlinewidth{0.301125pt}%
\definecolor{currentstroke}{rgb}{0.500000,0.500000,0.500000}%
\pgfsetstrokecolor{currentstroke}%
\pgfsetstrokeopacity{0.300000}%
\pgfsetdash{}{0pt}%
\pgfpathmoveto{\pgfqpoint{1.424993in}{2.823605in}}%
\pgfpathlineto{\pgfqpoint{1.424993in}{2.823605in}}%
\pgfpathlineto{\pgfqpoint{1.456714in}{2.774782in}}%
\pgfpathlineto{\pgfqpoint{1.490163in}{2.726309in}}%
\pgfpathlineto{\pgfqpoint{1.525833in}{2.678315in}}%
\pgfpathlineto{\pgfqpoint{1.564456in}{2.631012in}}%
\pgfpathlineto{\pgfqpoint{1.607178in}{2.584784in}}%
\pgfpathlineto{\pgfqpoint{1.655913in}{2.540410in}}%
\pgfpathlineto{\pgfqpoint{1.714049in}{2.499715in}}%
\pgfpathlineto{\pgfqpoint{1.714049in}{2.499715in}}%
\pgfpathlineto{\pgfqpoint{1.773944in}{2.471744in}}%
\pgfpathlineto{\pgfqpoint{1.773944in}{2.471744in}}%
\pgfpathlineto{\pgfqpoint{1.831643in}{2.457020in}}%
\pgfpathlineto{\pgfqpoint{1.895266in}{2.452263in}}%
\pgfpathlineto{\pgfqpoint{1.956603in}{2.455992in}}%
\pgfpathlineto{\pgfqpoint{2.027090in}{2.466578in}}%
\pgfpathlineto{\pgfqpoint{2.116302in}{2.484143in}}%
\pgfpathlineto{\pgfqpoint{2.205611in}{2.501560in}}%
\pgfpathlineto{\pgfqpoint{2.296859in}{2.515556in}}%
\pgfpathlineto{\pgfqpoint{2.390342in}{2.523461in}}%
\pgfpathlineto{\pgfqpoint{2.484730in}{2.523140in}}%
\pgfpathlineto{\pgfqpoint{2.577377in}{2.513470in}}%
\pgfpathlineto{\pgfqpoint{2.662800in}{2.495416in}}%
\pgfpathlineto{\pgfqpoint{2.743533in}{2.469380in}}%
\pgfpathlineto{\pgfqpoint{2.817333in}{2.437098in}}%
\pgfusepath{stroke}%
\end{pgfscope}%
\begin{pgfscope}%
\pgfpathrectangle{\pgfqpoint{0.647939in}{0.492442in}}{\pgfqpoint{4.273799in}{2.331163in}}%
\pgfusepath{clip}%
\pgfsetbuttcap%
\pgfsetroundjoin%
\pgfsetlinewidth{0.301125pt}%
\definecolor{currentstroke}{rgb}{0.500000,0.500000,0.500000}%
\pgfsetstrokecolor{currentstroke}%
\pgfsetstrokeopacity{0.300000}%
\pgfsetdash{}{0pt}%
\pgfpathmoveto{\pgfqpoint{1.327862in}{2.823605in}}%
\pgfpathlineto{\pgfqpoint{1.327862in}{2.823605in}}%
\pgfpathlineto{\pgfqpoint{1.353445in}{2.773721in}}%
\pgfpathlineto{\pgfqpoint{1.379724in}{2.723943in}}%
\pgfpathlineto{\pgfqpoint{1.406817in}{2.674296in}}%
\pgfpathlineto{\pgfqpoint{1.434889in}{2.624814in}}%
\pgfpathlineto{\pgfqpoint{1.464197in}{2.575546in}}%
\pgfpathlineto{\pgfqpoint{1.495090in}{2.526571in}}%
\pgfpathlineto{\pgfqpoint{1.528129in}{2.478025in}}%
\pgfpathlineto{\pgfqpoint{1.564328in}{2.430167in}}%
\pgfpathlineto{\pgfqpoint{1.605626in}{2.383584in}}%
\pgfpathlineto{\pgfqpoint{1.656369in}{2.340073in}}%
\pgfpathlineto{\pgfqpoint{1.656369in}{2.340073in}}%
\pgfpathlineto{\pgfqpoint{1.701522in}{2.315465in}}%
\pgfpathlineto{\pgfqpoint{1.701522in}{2.315465in}}%
\pgfpathlineto{\pgfqpoint{1.743217in}{2.304602in}}%
\pgfpathlineto{\pgfqpoint{1.790181in}{2.304284in}}%
\pgfpathlineto{\pgfqpoint{1.831513in}{2.311824in}}%
\pgfpathlineto{\pgfqpoint{1.878446in}{2.326404in}}%
\pgfpathlineto{\pgfqpoint{1.940243in}{2.351008in}}%
\pgfpathlineto{\pgfqpoint{2.014591in}{2.383128in}}%
\pgfpathlineto{\pgfqpoint{2.089948in}{2.414507in}}%
\pgfpathlineto{\pgfqpoint{2.168491in}{2.443479in}}%
\pgfusepath{stroke}%
\end{pgfscope}%
\begin{pgfscope}%
\pgfpathrectangle{\pgfqpoint{0.647939in}{0.492442in}}{\pgfqpoint{4.273799in}{2.331163in}}%
\pgfusepath{clip}%
\pgfsetbuttcap%
\pgfsetroundjoin%
\pgfsetlinewidth{0.301125pt}%
\definecolor{currentstroke}{rgb}{0.500000,0.500000,0.500000}%
\pgfsetstrokecolor{currentstroke}%
\pgfsetstrokeopacity{0.300000}%
\pgfsetdash{}{0pt}%
\pgfpathmoveto{\pgfqpoint{1.230730in}{2.823605in}}%
\pgfpathlineto{\pgfqpoint{1.230730in}{2.823605in}}%
\pgfpathlineto{\pgfqpoint{1.251561in}{2.773065in}}%
\pgfpathlineto{\pgfqpoint{1.272508in}{2.722538in}}%
\pgfpathlineto{\pgfqpoint{1.293564in}{2.672026in}}%
\pgfpathlineto{\pgfqpoint{1.314722in}{2.621526in}}%
\pgfpathlineto{\pgfqpoint{1.335967in}{2.571038in}}%
\pgfpathlineto{\pgfqpoint{1.357283in}{2.520559in}}%
\pgfpathlineto{\pgfqpoint{1.378646in}{2.470086in}}%
\pgfpathlineto{\pgfqpoint{1.400017in}{2.419615in}}%
\pgfpathlineto{\pgfqpoint{1.421344in}{2.369139in}}%
\pgfpathlineto{\pgfqpoint{1.442541in}{2.318649in}}%
\pgfpathlineto{\pgfqpoint{1.463476in}{2.268127in}}%
\pgfpathlineto{\pgfqpoint{1.483916in}{2.217548in}}%
\pgfpathlineto{\pgfqpoint{1.503428in}{2.166865in}}%
\pgfpathlineto{\pgfqpoint{1.521125in}{2.115991in}}%
\pgfpathlineto{\pgfqpoint{1.534839in}{2.064779in}}%
\pgfpathlineto{\pgfqpoint{1.538780in}{2.013204in}}%
\pgfpathlineto{\pgfqpoint{1.523828in}{1.962949in}}%
\pgfpathlineto{\pgfqpoint{1.500146in}{1.921042in}}%
\pgfpathlineto{\pgfqpoint{1.466145in}{1.872931in}}%
\pgfusepath{stroke}%
\end{pgfscope}%
\begin{pgfscope}%
\pgfpathrectangle{\pgfqpoint{0.647939in}{0.492442in}}{\pgfqpoint{4.273799in}{2.331163in}}%
\pgfusepath{clip}%
\pgfsetbuttcap%
\pgfsetroundjoin%
\pgfsetlinewidth{0.301125pt}%
\definecolor{currentstroke}{rgb}{0.500000,0.500000,0.500000}%
\pgfsetstrokecolor{currentstroke}%
\pgfsetstrokeopacity{0.300000}%
\pgfsetdash{}{0pt}%
\pgfpathmoveto{\pgfqpoint{1.133598in}{2.823605in}}%
\pgfpathlineto{\pgfqpoint{1.133598in}{2.823605in}}%
\pgfpathlineto{\pgfqpoint{1.150732in}{2.772653in}}%
\pgfpathlineto{\pgfqpoint{1.167692in}{2.721683in}}%
\pgfpathlineto{\pgfqpoint{1.184432in}{2.670692in}}%
\pgfpathlineto{\pgfqpoint{1.200897in}{2.619674in}}%
\pgfpathlineto{\pgfqpoint{1.217029in}{2.568624in}}%
\pgfpathlineto{\pgfqpoint{1.232746in}{2.517537in}}%
\pgfpathlineto{\pgfqpoint{1.247958in}{2.466404in}}%
\pgfpathlineto{\pgfqpoint{1.262549in}{2.415217in}}%
\pgfpathlineto{\pgfqpoint{1.276369in}{2.363967in}}%
\pgfpathlineto{\pgfqpoint{1.289244in}{2.312644in}}%
\pgfpathlineto{\pgfqpoint{1.300954in}{2.261238in}}%
\pgfpathlineto{\pgfqpoint{1.311215in}{2.209741in}}%
\pgfpathlineto{\pgfqpoint{1.319689in}{2.158148in}}%
\pgfpathlineto{\pgfqpoint{1.325973in}{2.106464in}}%
\pgfpathlineto{\pgfqpoint{1.329602in}{2.054705in}}%
\pgfpathlineto{\pgfqpoint{1.330078in}{2.002912in}}%
\pgfpathlineto{\pgfqpoint{1.326939in}{1.951148in}}%
\pgfpathlineto{\pgfqpoint{1.319860in}{1.899504in}}%
\pgfpathlineto{\pgfqpoint{1.308744in}{1.848077in}}%
\pgfpathlineto{\pgfqpoint{1.293766in}{1.796940in}}%
\pgfpathlineto{\pgfqpoint{1.275396in}{1.746134in}}%
\pgfpathlineto{\pgfqpoint{1.254220in}{1.695653in}}%
\pgfpathlineto{\pgfqpoint{1.230865in}{1.645457in}}%
\pgfpathlineto{\pgfqpoint{1.205902in}{1.595490in}}%
\pgfpathlineto{\pgfqpoint{1.179812in}{1.545695in}}%
\pgfpathlineto{\pgfqpoint{1.152982in}{1.496023in}}%
\pgfusepath{stroke}%
\end{pgfscope}%
\begin{pgfscope}%
\pgfpathrectangle{\pgfqpoint{0.647939in}{0.492442in}}{\pgfqpoint{4.273799in}{2.331163in}}%
\pgfusepath{clip}%
\pgfsetbuttcap%
\pgfsetroundjoin%
\pgfsetlinewidth{0.301125pt}%
\definecolor{currentstroke}{rgb}{0.500000,0.500000,0.500000}%
\pgfsetstrokecolor{currentstroke}%
\pgfsetstrokeopacity{0.300000}%
\pgfsetdash{}{0pt}%
\pgfpathmoveto{\pgfqpoint{1.036466in}{2.823605in}}%
\pgfpathlineto{\pgfqpoint{1.036466in}{2.823605in}}%
\pgfpathlineto{\pgfqpoint{1.050716in}{2.772389in}}%
\pgfpathlineto{\pgfqpoint{1.064648in}{2.721146in}}%
\pgfpathlineto{\pgfqpoint{1.078221in}{2.669875in}}%
\pgfpathlineto{\pgfqpoint{1.091380in}{2.618572in}}%
\pgfpathlineto{\pgfqpoint{1.104058in}{2.567233in}}%
\pgfpathlineto{\pgfqpoint{1.116190in}{2.515855in}}%
\pgfpathlineto{\pgfqpoint{1.127696in}{2.464434in}}%
\pgfpathlineto{\pgfqpoint{1.138481in}{2.412966in}}%
\pgfpathlineto{\pgfqpoint{1.148439in}{2.361450in}}%
\pgfpathlineto{\pgfqpoint{1.157456in}{2.309881in}}%
\pgfpathlineto{\pgfqpoint{1.165400in}{2.258260in}}%
\pgfpathlineto{\pgfqpoint{1.172121in}{2.206588in}}%
\pgfpathlineto{\pgfqpoint{1.177458in}{2.154868in}}%
\pgfpathlineto{\pgfqpoint{1.181242in}{2.103108in}}%
\pgfpathlineto{\pgfqpoint{1.183299in}{2.051320in}}%
\pgfpathlineto{\pgfqpoint{1.183461in}{1.999519in}}%
\pgfpathlineto{\pgfqpoint{1.181575in}{1.947729in}}%
\pgfpathlineto{\pgfqpoint{1.177518in}{1.895977in}}%
\pgfpathlineto{\pgfqpoint{1.171212in}{1.844293in}}%
\pgfpathlineto{\pgfqpoint{1.162635in}{1.792706in}}%
\pgfpathlineto{\pgfqpoint{1.151836in}{1.741244in}}%
\pgfpathlineto{\pgfqpoint{1.138924in}{1.689928in}}%
\pgfpathlineto{\pgfqpoint{1.124047in}{1.638770in}}%
\pgfpathlineto{\pgfqpoint{1.107409in}{1.587773in}}%
\pgfusepath{stroke}%
\end{pgfscope}%
\begin{pgfscope}%
\pgfpathrectangle{\pgfqpoint{0.647939in}{0.492442in}}{\pgfqpoint{4.273799in}{2.331163in}}%
\pgfusepath{clip}%
\pgfsetbuttcap%
\pgfsetroundjoin%
\pgfsetlinewidth{0.301125pt}%
\definecolor{currentstroke}{rgb}{0.500000,0.500000,0.500000}%
\pgfsetstrokecolor{currentstroke}%
\pgfsetstrokeopacity{0.300000}%
\pgfsetdash{}{0pt}%
\pgfpathmoveto{\pgfqpoint{0.939334in}{2.823605in}}%
\pgfpathlineto{\pgfqpoint{0.939334in}{2.823605in}}%
\pgfpathlineto{\pgfqpoint{0.951305in}{2.772215in}}%
\pgfpathlineto{\pgfqpoint{0.962906in}{2.720800in}}%
\pgfpathlineto{\pgfqpoint{0.974094in}{2.669358in}}%
\pgfpathlineto{\pgfqpoint{0.984827in}{2.617887in}}%
\pgfpathlineto{\pgfqpoint{0.995058in}{2.566385in}}%
\pgfpathlineto{\pgfqpoint{1.004729in}{2.514851in}}%
\pgfpathlineto{\pgfqpoint{1.013780in}{2.463284in}}%
\pgfpathlineto{\pgfqpoint{1.022148in}{2.411683in}}%
\pgfpathlineto{\pgfqpoint{1.029764in}{2.360047in}}%
\pgfpathlineto{\pgfqpoint{1.036555in}{2.308376in}}%
\pgfpathlineto{\pgfqpoint{1.042439in}{2.256673in}}%
\pgfpathlineto{\pgfqpoint{1.047330in}{2.204939in}}%
\pgfpathlineto{\pgfqpoint{1.051144in}{2.153178in}}%
\pgfpathlineto{\pgfqpoint{1.053794in}{2.101396in}}%
\pgfpathlineto{\pgfqpoint{1.055195in}{2.049600in}}%
\pgfpathlineto{\pgfqpoint{1.055266in}{1.997798in}}%
\pgfpathlineto{\pgfqpoint{1.053934in}{1.946001in}}%
\pgfpathlineto{\pgfqpoint{1.051139in}{1.894221in}}%
\pgfpathlineto{\pgfqpoint{1.046836in}{1.842473in}}%
\pgfpathlineto{\pgfqpoint{1.040998in}{1.790770in}}%
\pgfpathlineto{\pgfqpoint{1.033621in}{1.739125in}}%
\pgfpathlineto{\pgfqpoint{1.024728in}{1.687551in}}%
\pgfpathlineto{\pgfqpoint{1.014366in}{1.636060in}}%
\pgfpathlineto{\pgfqpoint{1.002594in}{1.584659in}}%
\pgfpathlineto{\pgfqpoint{0.989495in}{1.533353in}}%
\pgfpathlineto{\pgfqpoint{0.975170in}{1.482145in}}%
\pgfpathlineto{\pgfqpoint{0.959728in}{1.431034in}}%
\pgfpathlineto{\pgfqpoint{0.943277in}{1.380017in}}%
\pgfpathlineto{\pgfqpoint{0.925935in}{1.329088in}}%
\pgfpathlineto{\pgfqpoint{0.907807in}{1.278240in}}%
\pgfpathlineto{\pgfqpoint{0.889002in}{1.227465in}}%
\pgfpathlineto{\pgfqpoint{0.869624in}{1.176754in}}%
\pgfpathlineto{\pgfqpoint{0.849757in}{1.126100in}}%
\pgfpathlineto{\pgfqpoint{0.829487in}{1.075494in}}%
\pgfpathlineto{\pgfqpoint{0.808886in}{1.024927in}}%
\pgfpathlineto{\pgfqpoint{0.788018in}{0.974393in}}%
\pgfpathlineto{\pgfqpoint{0.766942in}{0.923885in}}%
\pgfpathlineto{\pgfqpoint{0.745707in}{0.873396in}}%
\pgfpathlineto{\pgfqpoint{0.724356in}{0.822922in}}%
\pgfusepath{stroke}%
\end{pgfscope}%
\begin{pgfscope}%
\pgfpathrectangle{\pgfqpoint{0.647939in}{0.492442in}}{\pgfqpoint{4.273799in}{2.331163in}}%
\pgfusepath{clip}%
\pgfsetbuttcap%
\pgfsetroundjoin%
\pgfsetlinewidth{0.301125pt}%
\definecolor{currentstroke}{rgb}{0.500000,0.500000,0.500000}%
\pgfsetstrokecolor{currentstroke}%
\pgfsetstrokeopacity{0.300000}%
\pgfsetdash{}{0pt}%
\pgfpathmoveto{\pgfqpoint{0.842203in}{2.823605in}}%
\pgfpathlineto{\pgfqpoint{0.842203in}{2.823605in}}%
\pgfpathlineto{\pgfqpoint{0.852352in}{2.772099in}}%
\pgfpathlineto{\pgfqpoint{0.862123in}{2.720570in}}%
\pgfpathlineto{\pgfqpoint{0.871479in}{2.669019in}}%
\pgfpathlineto{\pgfqpoint{0.880386in}{2.617444in}}%
\pgfpathlineto{\pgfqpoint{0.888805in}{2.565845in}}%
\pgfpathlineto{\pgfqpoint{0.896697in}{2.514221in}}%
\pgfpathlineto{\pgfqpoint{0.904023in}{2.462572in}}%
\pgfpathlineto{\pgfqpoint{0.910735in}{2.410899in}}%
\pgfpathlineto{\pgfqpoint{0.916787in}{2.359201in}}%
\pgfpathlineto{\pgfqpoint{0.922128in}{2.307480in}}%
\pgfpathlineto{\pgfqpoint{0.926710in}{2.255737in}}%
\pgfpathlineto{\pgfqpoint{0.930482in}{2.203975in}}%
\pgfpathlineto{\pgfqpoint{0.933393in}{2.152196in}}%
\pgfpathlineto{\pgfqpoint{0.935393in}{2.100405in}}%
\pgfpathlineto{\pgfqpoint{0.936433in}{2.048605in}}%
\pgfpathlineto{\pgfqpoint{0.936468in}{1.996802in}}%
\pgfpathlineto{\pgfqpoint{0.935458in}{1.945003in}}%
\pgfpathlineto{\pgfqpoint{0.933367in}{1.893213in}}%
\pgfpathlineto{\pgfqpoint{0.930169in}{1.841439in}}%
\pgfpathlineto{\pgfqpoint{0.925843in}{1.789691in}}%
\pgfpathlineto{\pgfqpoint{0.920381in}{1.737974in}}%
\pgfpathlineto{\pgfqpoint{0.913784in}{1.686297in}}%
\pgfpathlineto{\pgfqpoint{0.906061in}{1.634666in}}%
\pgfpathlineto{\pgfqpoint{0.897239in}{1.583088in}}%
\pgfpathlineto{\pgfqpoint{0.887354in}{1.531567in}}%
\pgfpathlineto{\pgfqpoint{0.876446in}{1.480108in}}%
\pgfpathlineto{\pgfqpoint{0.864561in}{1.428713in}}%
\pgfpathlineto{\pgfqpoint{0.851760in}{1.377384in}}%
\pgfpathlineto{\pgfqpoint{0.838110in}{1.326120in}}%
\pgfpathlineto{\pgfqpoint{0.823669in}{1.274921in}}%
\pgfusepath{stroke}%
\end{pgfscope}%
\begin{pgfscope}%
\pgfpathrectangle{\pgfqpoint{0.647939in}{0.492442in}}{\pgfqpoint{4.273799in}{2.331163in}}%
\pgfusepath{clip}%
\pgfsetbuttcap%
\pgfsetroundjoin%
\pgfsetlinewidth{0.301125pt}%
\definecolor{currentstroke}{rgb}{0.500000,0.500000,0.500000}%
\pgfsetstrokecolor{currentstroke}%
\pgfsetstrokeopacity{0.300000}%
\pgfsetdash{}{0pt}%
\pgfpathmoveto{\pgfqpoint{0.745071in}{2.823605in}}%
\pgfpathlineto{\pgfqpoint{0.745071in}{2.823605in}}%
\pgfpathlineto{\pgfqpoint{0.753754in}{2.772019in}}%
\pgfpathlineto{\pgfqpoint{0.762066in}{2.720415in}}%
\pgfpathlineto{\pgfqpoint{0.769983in}{2.668792in}}%
\pgfpathlineto{\pgfqpoint{0.777478in}{2.617150in}}%
\pgfpathlineto{\pgfqpoint{0.784522in}{2.565489in}}%
\pgfpathlineto{\pgfqpoint{0.791085in}{2.513809in}}%
\pgfpathlineto{\pgfqpoint{0.797137in}{2.462112in}}%
\pgfpathlineto{\pgfqpoint{0.802647in}{2.410396in}}%
\pgfpathlineto{\pgfqpoint{0.807583in}{2.358662in}}%
\pgfpathlineto{\pgfqpoint{0.811914in}{2.306913in}}%
\pgfpathlineto{\pgfqpoint{0.815605in}{2.255149in}}%
\pgfpathlineto{\pgfqpoint{0.818625in}{2.203372in}}%
\pgfpathlineto{\pgfqpoint{0.820940in}{2.151584in}}%
\pgfpathlineto{\pgfqpoint{0.822519in}{2.099788in}}%
\pgfpathlineto{\pgfqpoint{0.823332in}{2.047987in}}%
\pgfpathlineto{\pgfqpoint{0.823351in}{1.996184in}}%
\pgfpathlineto{\pgfqpoint{0.822551in}{1.944382in}}%
\pgfpathlineto{\pgfqpoint{0.820910in}{1.892587in}}%
\pgfpathlineto{\pgfqpoint{0.818410in}{1.840802in}}%
\pgfpathlineto{\pgfqpoint{0.815038in}{1.789032in}}%
\pgfpathlineto{\pgfqpoint{0.810786in}{1.737281in}}%
\pgfpathlineto{\pgfqpoint{0.805651in}{1.685554in}}%
\pgfpathlineto{\pgfqpoint{0.799635in}{1.633855in}}%
\pgfpathlineto{\pgfqpoint{0.792747in}{1.582189in}}%
\pgfpathlineto{\pgfqpoint{0.784999in}{1.530559in}}%
\pgfpathlineto{\pgfqpoint{0.776409in}{1.478968in}}%
\pgfpathlineto{\pgfqpoint{0.767005in}{1.427420in}}%
\pgfpathlineto{\pgfqpoint{0.756819in}{1.375916in}}%
\pgfpathlineto{\pgfqpoint{0.745881in}{1.324459in}}%
\pgfpathlineto{\pgfqpoint{0.734225in}{1.273048in}}%
\pgfpathlineto{\pgfqpoint{0.721893in}{1.221683in}}%
\pgfpathlineto{\pgfqpoint{0.708930in}{1.170366in}}%
\pgfpathlineto{\pgfqpoint{0.695374in}{1.119094in}}%
\pgfpathlineto{\pgfqpoint{0.681268in}{1.067866in}}%
\pgfpathlineto{\pgfqpoint{0.666659in}{1.016680in}}%
\pgfpathlineto{\pgfqpoint{0.651584in}{0.965535in}}%
\pgfpathlineto{\pgfqpoint{0.647939in}{0.953352in}}%
\pgfusepath{stroke}%
\end{pgfscope}%
\begin{pgfscope}%
\pgfpathrectangle{\pgfqpoint{0.647939in}{0.492442in}}{\pgfqpoint{4.273799in}{2.331163in}}%
\pgfusepath{clip}%
\pgfsetbuttcap%
\pgfsetroundjoin%
\pgfsetlinewidth{0.301125pt}%
\definecolor{currentstroke}{rgb}{0.500000,0.500000,0.500000}%
\pgfsetstrokecolor{currentstroke}%
\pgfsetstrokeopacity{0.300000}%
\pgfsetdash{}{0pt}%
\pgfpathmoveto{\pgfqpoint{0.647939in}{2.823605in}}%
\pgfpathlineto{\pgfqpoint{0.647939in}{2.823605in}}%
\pgfpathlineto{\pgfqpoint{0.655428in}{2.771963in}}%
\pgfpathlineto{\pgfqpoint{0.662566in}{2.720306in}}%
\pgfpathlineto{\pgfqpoint{0.669335in}{2.668635in}}%
\pgfpathlineto{\pgfqpoint{0.675714in}{2.616948in}}%
\pgfpathlineto{\pgfqpoint{0.681683in}{2.565247in}}%
\pgfpathlineto{\pgfqpoint{0.687221in}{2.513532in}}%
\pgfpathlineto{\pgfqpoint{0.692305in}{2.461803in}}%
\pgfpathlineto{\pgfqpoint{0.696914in}{2.410061in}}%
\pgfpathlineto{\pgfqpoint{0.701023in}{2.358306in}}%
\pgfpathlineto{\pgfqpoint{0.704611in}{2.306539in}}%
\pgfpathlineto{\pgfqpoint{0.707656in}{2.254763in}}%
\pgfpathlineto{\pgfqpoint{0.710134in}{2.202977in}}%
\pgfpathlineto{\pgfqpoint{0.712026in}{2.151184in}}%
\pgfpathlineto{\pgfqpoint{0.713310in}{2.099385in}}%
\pgfpathlineto{\pgfqpoint{0.713966in}{2.047583in}}%
\pgfpathlineto{\pgfqpoint{0.713977in}{1.995780in}}%
\pgfpathlineto{\pgfqpoint{0.713326in}{1.943978in}}%
\pgfpathlineto{\pgfqpoint{0.711998in}{1.892180in}}%
\pgfpathlineto{\pgfqpoint{0.709981in}{1.840388in}}%
\pgfpathlineto{\pgfqpoint{0.707266in}{1.788606in}}%
\pgfpathlineto{\pgfqpoint{0.703845in}{1.736837in}}%
\pgfpathlineto{\pgfqpoint{0.699714in}{1.685083in}}%
\pgfpathlineto{\pgfqpoint{0.694873in}{1.633347in}}%
\pgfpathlineto{\pgfqpoint{0.689323in}{1.581632in}}%
\pgfpathlineto{\pgfqpoint{0.683073in}{1.529941in}}%
\pgfpathlineto{\pgfqpoint{0.676131in}{1.478277in}}%
\pgfpathlineto{\pgfqpoint{0.668511in}{1.426641in}}%
\pgfpathlineto{\pgfqpoint{0.660227in}{1.375035in}}%
\pgfpathlineto{\pgfqpoint{0.651295in}{1.323462in}}%
\pgfpathlineto{\pgfqpoint{0.647939in}{1.304752in}}%
\pgfusepath{stroke}%
\end{pgfscope}%
\begin{pgfscope}%
\pgfpathrectangle{\pgfqpoint{0.647939in}{0.492442in}}{\pgfqpoint{4.273799in}{2.331163in}}%
\pgfusepath{clip}%
\pgfsetbuttcap%
\pgfsetroundjoin%
\pgfsetlinewidth{0.301125pt}%
\definecolor{currentstroke}{rgb}{0.500000,0.500000,0.500000}%
\pgfsetstrokecolor{currentstroke}%
\pgfsetstrokeopacity{0.300000}%
\pgfsetdash{}{0pt}%
\pgfpathmoveto{\pgfqpoint{0.647939in}{2.399758in}}%
\pgfpathlineto{\pgfqpoint{0.647939in}{2.399758in}}%
\pgfpathlineto{\pgfqpoint{0.651632in}{2.347993in}}%
\pgfpathlineto{\pgfqpoint{0.654835in}{2.296219in}}%
\pgfpathlineto{\pgfqpoint{0.657530in}{2.244437in}}%
\pgfpathlineto{\pgfqpoint{0.659697in}{2.192647in}}%
\pgfpathlineto{\pgfqpoint{0.661320in}{2.140851in}}%
\pgfpathlineto{\pgfqpoint{0.662382in}{2.089051in}}%
\pgfpathlineto{\pgfqpoint{0.662865in}{2.037248in}}%
\pgfpathlineto{\pgfqpoint{0.662756in}{1.985445in}}%
\pgfpathlineto{\pgfqpoint{0.662040in}{1.933643in}}%
\pgfpathlineto{\pgfqpoint{0.660707in}{1.881845in}}%
\pgfpathlineto{\pgfqpoint{0.658746in}{1.830053in}}%
\pgfpathlineto{\pgfqpoint{0.656148in}{1.778269in}}%
\pgfpathlineto{\pgfqpoint{0.652909in}{1.726496in}}%
\pgfusepath{stroke}%
\end{pgfscope}%
\begin{pgfscope}%
\pgfpathrectangle{\pgfqpoint{0.647939in}{0.492442in}}{\pgfqpoint{4.273799in}{2.331163in}}%
\pgfusepath{clip}%
\pgfsetbuttcap%
\pgfsetroundjoin%
\pgfsetlinewidth{0.301125pt}%
\definecolor{currentstroke}{rgb}{0.500000,0.500000,0.500000}%
\pgfsetstrokecolor{currentstroke}%
\pgfsetstrokeopacity{0.300000}%
\pgfsetdash{}{0pt}%
\pgfpathmoveto{\pgfqpoint{1.495449in}{0.492442in}}%
\pgfpathlineto{\pgfqpoint{1.481278in}{0.503751in}}%
\pgfpathlineto{\pgfqpoint{1.424993in}{0.545423in}}%
\pgfpathlineto{\pgfqpoint{1.361798in}{0.583946in}}%
\pgfpathlineto{\pgfqpoint{1.288136in}{0.616171in}}%
\pgfpathlineto{\pgfqpoint{1.288136in}{0.616171in}}%
\pgfpathlineto{\pgfqpoint{1.226973in}{0.631393in}}%
\pgfpathlineto{\pgfqpoint{1.226973in}{0.631393in}}%
\pgfpathlineto{\pgfqpoint{1.169917in}{0.635378in}}%
\pgfpathlineto{\pgfqpoint{1.113317in}{0.629127in}}%
\pgfusepath{stroke}%
\end{pgfscope}%
\begin{pgfscope}%
\pgfpathrectangle{\pgfqpoint{0.647939in}{0.492442in}}{\pgfqpoint{4.273799in}{2.331163in}}%
\pgfusepath{clip}%
\pgfsetbuttcap%
\pgfsetroundjoin%
\pgfsetlinewidth{0.301125pt}%
\definecolor{currentstroke}{rgb}{0.500000,0.500000,0.500000}%
\pgfsetstrokecolor{currentstroke}%
\pgfsetstrokeopacity{0.300000}%
\pgfsetdash{}{0pt}%
\pgfpathmoveto{\pgfqpoint{4.338948in}{0.598404in}}%
\pgfpathlineto{\pgfqpoint{4.294380in}{0.644141in}}%
\pgfpathlineto{\pgfqpoint{4.247628in}{0.689223in}}%
\pgfpathlineto{\pgfqpoint{4.198495in}{0.733545in}}%
\pgfpathlineto{\pgfqpoint{4.146792in}{0.776988in}}%
\pgfpathlineto{\pgfqpoint{4.092377in}{0.819435in}}%
\pgfpathlineto{\pgfqpoint{4.035189in}{0.860780in}}%
\pgfpathlineto{\pgfqpoint{3.975217in}{0.900934in}}%
\pgfpathlineto{\pgfqpoint{3.912606in}{0.939875in}}%
\pgfpathlineto{\pgfqpoint{3.847664in}{0.977665in}}%
\pgfpathlineto{\pgfqpoint{3.780839in}{1.014470in}}%
\pgfpathlineto{\pgfqpoint{3.712714in}{1.050561in}}%
\pgfpathlineto{\pgfqpoint{3.643969in}{1.086302in}}%
\pgfpathlineto{\pgfqpoint{3.575320in}{1.122097in}}%
\pgfpathlineto{\pgfqpoint{3.507434in}{1.158319in}}%
\pgfpathlineto{\pgfqpoint{3.440940in}{1.195295in}}%
\pgfpathlineto{\pgfqpoint{3.376352in}{1.233260in}}%
\pgfusepath{stroke}%
\end{pgfscope}%
\begin{pgfscope}%
\pgfpathrectangle{\pgfqpoint{0.647939in}{0.492442in}}{\pgfqpoint{4.273799in}{2.331163in}}%
\pgfusepath{clip}%
\pgfsetbuttcap%
\pgfsetroundjoin%
\pgfsetlinewidth{0.301125pt}%
\definecolor{currentstroke}{rgb}{0.500000,0.500000,0.500000}%
\pgfsetstrokecolor{currentstroke}%
\pgfsetstrokeopacity{0.300000}%
\pgfsetdash{}{0pt}%
\pgfpathmoveto{\pgfqpoint{4.727475in}{1.869948in}}%
\pgfpathlineto{\pgfqpoint{4.722779in}{1.921685in}}%
\pgfpathlineto{\pgfqpoint{4.719896in}{1.973462in}}%
\pgfpathlineto{\pgfqpoint{4.719034in}{2.025260in}}%
\pgfpathlineto{\pgfqpoint{4.720386in}{2.077054in}}%
\pgfpathlineto{\pgfqpoint{4.724114in}{2.128812in}}%
\pgfusepath{stroke}%
\end{pgfscope}%
\begin{pgfscope}%
\pgfpathrectangle{\pgfqpoint{0.647939in}{0.492442in}}{\pgfqpoint{4.273799in}{2.331163in}}%
\pgfusepath{clip}%
\pgfsetbuttcap%
\pgfsetroundjoin%
\pgfsetlinewidth{0.301125pt}%
\definecolor{currentstroke}{rgb}{0.500000,0.500000,0.500000}%
\pgfsetstrokecolor{currentstroke}%
\pgfsetstrokeopacity{0.300000}%
\pgfsetdash{}{0pt}%
\pgfpathmoveto{\pgfqpoint{1.571406in}{0.607125in}}%
\pgfpathlineto{\pgfqpoint{1.522125in}{0.651385in}}%
\pgfpathlineto{\pgfqpoint{1.468603in}{0.694141in}}%
\pgfpathlineto{\pgfqpoint{1.408788in}{0.734271in}}%
\pgfpathlineto{\pgfqpoint{1.339237in}{0.769179in}}%
\pgfpathlineto{\pgfqpoint{1.339237in}{0.769179in}}%
\pgfpathlineto{\pgfqpoint{1.278151in}{0.788155in}}%
\pgfpathlineto{\pgfqpoint{1.278151in}{0.788155in}}%
\pgfpathlineto{\pgfqpoint{1.222299in}{0.795015in}}%
\pgfpathlineto{\pgfqpoint{1.165006in}{0.791112in}}%
\pgfusepath{stroke}%
\end{pgfscope}%
\begin{pgfscope}%
\pgfpathrectangle{\pgfqpoint{0.647939in}{0.492442in}}{\pgfqpoint{4.273799in}{2.331163in}}%
\pgfusepath{clip}%
\pgfsetbuttcap%
\pgfsetroundjoin%
\pgfsetlinewidth{0.301125pt}%
\definecolor{currentstroke}{rgb}{0.500000,0.500000,0.500000}%
\pgfsetstrokecolor{currentstroke}%
\pgfsetstrokeopacity{0.300000}%
\pgfsetdash{}{0pt}%
\pgfpathmoveto{\pgfqpoint{4.238399in}{0.492442in}}%
\pgfpathlineto{\pgfqpoint{4.205352in}{0.522057in}}%
\pgfpathlineto{\pgfqpoint{4.154927in}{0.565947in}}%
\pgfpathlineto{\pgfqpoint{4.102329in}{0.609073in}}%
\pgfpathlineto{\pgfqpoint{4.047552in}{0.651385in}}%
\pgfpathlineto{\pgfqpoint{3.990650in}{0.692854in}}%
\pgfpathlineto{\pgfqpoint{3.931730in}{0.733476in}}%
\pgfpathlineto{\pgfqpoint{3.871006in}{0.773303in}}%
\pgfpathlineto{\pgfqpoint{3.808794in}{0.812441in}}%
\pgfpathlineto{\pgfqpoint{3.745465in}{0.851045in}}%
\pgfusepath{stroke}%
\end{pgfscope}%
\begin{pgfscope}%
\pgfpathrectangle{\pgfqpoint{0.647939in}{0.492442in}}{\pgfqpoint{4.273799in}{2.331163in}}%
\pgfusepath{clip}%
\pgfsetbuttcap%
\pgfsetroundjoin%
\pgfsetlinewidth{0.301125pt}%
\definecolor{currentstroke}{rgb}{0.500000,0.500000,0.500000}%
\pgfsetstrokecolor{currentstroke}%
\pgfsetstrokeopacity{0.300000}%
\pgfsetdash{}{0pt}%
\pgfpathmoveto{\pgfqpoint{4.645575in}{1.606893in}}%
\pgfpathlineto{\pgfqpoint{4.630343in}{1.658024in}}%
\pgfpathlineto{\pgfqpoint{4.615958in}{1.709227in}}%
\pgfpathlineto{\pgfqpoint{4.602647in}{1.760517in}}%
\pgfpathlineto{\pgfqpoint{4.590696in}{1.811905in}}%
\pgfpathlineto{\pgfqpoint{4.580458in}{1.863403in}}%
\pgfpathlineto{\pgfqpoint{4.572386in}{1.915013in}}%
\pgfpathlineto{\pgfqpoint{4.567036in}{1.966727in}}%
\pgfpathlineto{\pgfqpoint{4.565037in}{2.018507in}}%
\pgfpathlineto{\pgfqpoint{4.567031in}{2.070285in}}%
\pgfpathlineto{\pgfqpoint{4.573543in}{2.121948in}}%
\pgfpathlineto{\pgfqpoint{4.584799in}{2.173363in}}%
\pgfpathlineto{\pgfqpoint{4.600594in}{2.224414in}}%
\pgfpathlineto{\pgfqpoint{4.620388in}{2.275055in}}%
\pgfusepath{stroke}%
\end{pgfscope}%
\begin{pgfscope}%
\pgfpathrectangle{\pgfqpoint{0.647939in}{0.492442in}}{\pgfqpoint{4.273799in}{2.331163in}}%
\pgfusepath{clip}%
\pgfsetbuttcap%
\pgfsetroundjoin%
\pgfsetlinewidth{0.301125pt}%
\definecolor{currentstroke}{rgb}{0.500000,0.500000,0.500000}%
\pgfsetstrokecolor{currentstroke}%
\pgfsetstrokeopacity{0.300000}%
\pgfsetdash{}{0pt}%
\pgfpathmoveto{\pgfqpoint{3.817042in}{0.664609in}}%
\pgfpathlineto{\pgfqpoint{3.756157in}{0.704366in}}%
\pgfpathlineto{\pgfqpoint{3.694683in}{0.743853in}}%
\pgfpathlineto{\pgfqpoint{3.633004in}{0.783246in}}%
\pgfpathlineto{\pgfqpoint{3.571492in}{0.822714in}}%
\pgfpathlineto{\pgfqpoint{3.510518in}{0.862430in}}%
\pgfusepath{stroke}%
\end{pgfscope}%
\begin{pgfscope}%
\pgfpathrectangle{\pgfqpoint{0.647939in}{0.492442in}}{\pgfqpoint{4.273799in}{2.331163in}}%
\pgfusepath{clip}%
\pgfsetbuttcap%
\pgfsetroundjoin%
\pgfsetlinewidth{0.301125pt}%
\definecolor{currentstroke}{rgb}{0.500000,0.500000,0.500000}%
\pgfsetstrokecolor{currentstroke}%
\pgfsetstrokeopacity{0.300000}%
\pgfsetdash{}{0pt}%
\pgfpathmoveto{\pgfqpoint{4.556983in}{1.448929in}}%
\pgfpathlineto{\pgfqpoint{4.533211in}{1.499081in}}%
\pgfpathlineto{\pgfqpoint{4.509090in}{1.549182in}}%
\pgfpathlineto{\pgfqpoint{4.484580in}{1.599226in}}%
\pgfpathlineto{\pgfqpoint{4.459605in}{1.649200in}}%
\pgfpathlineto{\pgfqpoint{4.434076in}{1.699086in}}%
\pgfpathlineto{\pgfqpoint{4.407803in}{1.748857in}}%
\pgfpathlineto{\pgfqpoint{4.380492in}{1.798451in}}%
\pgfpathlineto{\pgfqpoint{4.351469in}{1.847732in}}%
\pgfpathlineto{\pgfqpoint{4.318655in}{1.896269in}}%
\pgfpathlineto{\pgfqpoint{4.318655in}{1.896269in}}%
\pgfpathlineto{\pgfqpoint{4.284930in}{1.931347in}}%
\pgfpathlineto{\pgfqpoint{4.284930in}{1.931347in}}%
\pgfpathlineto{\pgfqpoint{4.263811in}{1.941050in}}%
\pgfpathlineto{\pgfqpoint{4.263811in}{1.941050in}}%
\pgfpathlineto{\pgfqpoint{4.243827in}{1.940607in}}%
\pgfpathlineto{\pgfqpoint{4.224461in}{1.933801in}}%
\pgfpathlineto{\pgfqpoint{4.201060in}{1.920736in}}%
\pgfpathlineto{\pgfqpoint{4.165097in}{1.896333in}}%
\pgfpathlineto{\pgfqpoint{4.107986in}{1.855712in}}%
\pgfusepath{stroke}%
\end{pgfscope}%
\begin{pgfscope}%
\pgfpathrectangle{\pgfqpoint{0.647939in}{0.492442in}}{\pgfqpoint{4.273799in}{2.331163in}}%
\pgfusepath{clip}%
\pgfsetbuttcap%
\pgfsetroundjoin%
\pgfsetlinewidth{0.301125pt}%
\definecolor{currentstroke}{rgb}{0.500000,0.500000,0.500000}%
\pgfsetstrokecolor{currentstroke}%
\pgfsetstrokeopacity{0.300000}%
\pgfsetdash{}{0pt}%
\pgfpathmoveto{\pgfqpoint{1.079125in}{1.494067in}}%
\pgfpathlineto{\pgfqpoint{1.058247in}{1.443536in}}%
\pgfpathlineto{\pgfqpoint{1.036466in}{1.393119in}}%
\pgfpathlineto{\pgfqpoint{1.013973in}{1.342794in}}%
\pgfpathlineto{\pgfqpoint{0.990931in}{1.292544in}}%
\pgfpathlineto{\pgfqpoint{0.967477in}{1.242350in}}%
\pgfpathlineto{\pgfqpoint{0.943730in}{1.192199in}}%
\pgfpathlineto{\pgfqpoint{0.919781in}{1.142076in}}%
\pgfusepath{stroke}%
\end{pgfscope}%
\begin{pgfscope}%
\pgfpathrectangle{\pgfqpoint{0.647939in}{0.492442in}}{\pgfqpoint{4.273799in}{2.331163in}}%
\pgfusepath{clip}%
\pgfsetbuttcap%
\pgfsetroundjoin%
\pgfsetlinewidth{0.301125pt}%
\definecolor{currentstroke}{rgb}{0.500000,0.500000,0.500000}%
\pgfsetstrokecolor{currentstroke}%
\pgfsetstrokeopacity{0.300000}%
\pgfsetdash{}{0pt}%
\pgfpathmoveto{\pgfqpoint{2.784839in}{0.757347in}}%
\pgfpathlineto{\pgfqpoint{2.745808in}{0.804572in}}%
\pgfpathlineto{\pgfqpoint{2.707764in}{0.852037in}}%
\pgfpathlineto{\pgfqpoint{2.670699in}{0.899731in}}%
\pgfpathlineto{\pgfqpoint{2.634603in}{0.947646in}}%
\pgfpathlineto{\pgfqpoint{2.599465in}{0.995773in}}%
\pgfpathlineto{\pgfqpoint{2.565279in}{1.044103in}}%
\pgfpathlineto{\pgfqpoint{2.532042in}{1.092630in}}%
\pgfpathlineto{\pgfqpoint{2.499760in}{1.141348in}}%
\pgfpathlineto{\pgfqpoint{2.468439in}{1.190252in}}%
\pgfpathlineto{\pgfqpoint{2.438085in}{1.239338in}}%
\pgfpathlineto{\pgfqpoint{2.408718in}{1.288602in}}%
\pgfpathlineto{\pgfqpoint{2.380366in}{1.338042in}}%
\pgfpathlineto{\pgfqpoint{2.353060in}{1.387657in}}%
\pgfpathlineto{\pgfqpoint{2.326844in}{1.437447in}}%
\pgfpathlineto{\pgfqpoint{2.301779in}{1.487413in}}%
\pgfpathlineto{\pgfqpoint{2.277933in}{1.537555in}}%
\pgfusepath{stroke}%
\end{pgfscope}%
\begin{pgfscope}%
\pgfpathrectangle{\pgfqpoint{0.647939in}{0.492442in}}{\pgfqpoint{4.273799in}{2.331163in}}%
\pgfusepath{clip}%
\pgfsetbuttcap%
\pgfsetroundjoin%
\pgfsetlinewidth{0.301125pt}%
\definecolor{currentstroke}{rgb}{0.500000,0.500000,0.500000}%
\pgfsetstrokecolor{currentstroke}%
\pgfsetstrokeopacity{0.300000}%
\pgfsetdash{}{0pt}%
\pgfpathmoveto{\pgfqpoint{4.436079in}{1.022252in}}%
\pgfpathlineto{\pgfqpoint{4.398432in}{1.069804in}}%
\pgfpathlineto{\pgfqpoint{4.358546in}{1.116811in}}%
\pgfpathlineto{\pgfqpoint{4.315962in}{1.163106in}}%
\pgfpathlineto{\pgfqpoint{4.270083in}{1.208449in}}%
\pgfpathlineto{\pgfqpoint{4.220135in}{1.252480in}}%
\pgfpathlineto{\pgfqpoint{4.165117in}{1.294648in}}%
\pgfpathlineto{\pgfqpoint{4.103795in}{1.334113in}}%
\pgfpathlineto{\pgfqpoint{4.035038in}{1.369698in}}%
\pgfpathlineto{\pgfqpoint{3.958457in}{1.400094in}}%
\pgfpathlineto{\pgfqpoint{3.875037in}{1.424580in}}%
\pgfpathlineto{\pgfqpoint{3.787119in}{1.443975in}}%
\pgfusepath{stroke}%
\end{pgfscope}%
\begin{pgfscope}%
\pgfpathrectangle{\pgfqpoint{0.647939in}{0.492442in}}{\pgfqpoint{4.273799in}{2.331163in}}%
\pgfusepath{clip}%
\pgfsetbuttcap%
\pgfsetroundjoin%
\pgfsetlinewidth{0.301125pt}%
\definecolor{currentstroke}{rgb}{0.500000,0.500000,0.500000}%
\pgfsetstrokecolor{currentstroke}%
\pgfsetstrokeopacity{0.300000}%
\pgfsetdash{}{0pt}%
\pgfpathmoveto{\pgfqpoint{4.436079in}{1.287157in}}%
\pgfpathlineto{\pgfqpoint{4.400846in}{1.335256in}}%
\pgfpathlineto{\pgfqpoint{4.363172in}{1.382799in}}%
\pgfpathlineto{\pgfqpoint{4.322291in}{1.429538in}}%
\pgfpathlineto{\pgfqpoint{4.276991in}{1.475033in}}%
\pgfpathlineto{\pgfqpoint{4.225267in}{1.518406in}}%
\pgfpathlineto{\pgfqpoint{4.164003in}{1.557725in}}%
\pgfpathlineto{\pgfqpoint{4.088973in}{1.588617in}}%
\pgfpathlineto{\pgfqpoint{4.088973in}{1.588617in}}%
\pgfpathlineto{\pgfqpoint{4.024925in}{1.602162in}}%
\pgfpathlineto{\pgfqpoint{3.956169in}{1.606759in}}%
\pgfpathlineto{\pgfqpoint{3.883216in}{1.604399in}}%
\pgfpathlineto{\pgfqpoint{3.790876in}{1.596737in}}%
\pgfpathlineto{\pgfqpoint{3.697048in}{1.589178in}}%
\pgfusepath{stroke}%
\end{pgfscope}%
\begin{pgfscope}%
\pgfpathrectangle{\pgfqpoint{0.647939in}{0.492442in}}{\pgfqpoint{4.273799in}{2.331163in}}%
\pgfusepath{clip}%
\pgfsetbuttcap%
\pgfsetroundjoin%
\pgfsetlinewidth{0.301125pt}%
\definecolor{currentstroke}{rgb}{0.500000,0.500000,0.500000}%
\pgfsetstrokecolor{currentstroke}%
\pgfsetstrokeopacity{0.300000}%
\pgfsetdash{}{0pt}%
\pgfpathmoveto{\pgfqpoint{2.396312in}{0.810328in}}%
\pgfpathlineto{\pgfqpoint{2.362842in}{0.858808in}}%
\pgfpathlineto{\pgfqpoint{2.329971in}{0.907409in}}%
\pgfpathlineto{\pgfqpoint{2.297687in}{0.956128in}}%
\pgfpathlineto{\pgfqpoint{2.265975in}{1.004958in}}%
\pgfpathlineto{\pgfqpoint{2.234830in}{1.053897in}}%
\pgfpathlineto{\pgfqpoint{2.204254in}{1.102942in}}%
\pgfpathlineto{\pgfqpoint{2.174238in}{1.152090in}}%
\pgfpathlineto{\pgfqpoint{2.144777in}{1.201338in}}%
\pgfpathlineto{\pgfqpoint{2.115887in}{1.250686in}}%
\pgfpathlineto{\pgfqpoint{2.087568in}{1.300133in}}%
\pgfpathlineto{\pgfqpoint{2.059828in}{1.349677in}}%
\pgfpathlineto{\pgfqpoint{2.032696in}{1.399321in}}%
\pgfpathlineto{\pgfqpoint{2.006195in}{1.449067in}}%
\pgfpathlineto{\pgfqpoint{1.980362in}{1.498916in}}%
\pgfpathlineto{\pgfqpoint{1.955256in}{1.548876in}}%
\pgfpathlineto{\pgfqpoint{1.930946in}{1.598953in}}%
\pgfpathlineto{\pgfqpoint{1.907543in}{1.649158in}}%
\pgfpathlineto{\pgfqpoint{1.885188in}{1.699504in}}%
\pgfpathlineto{\pgfqpoint{1.864093in}{1.750011in}}%
\pgfpathlineto{\pgfqpoint{1.844564in}{1.800703in}}%
\pgfpathlineto{\pgfqpoint{1.827053in}{1.851612in}}%
\pgfpathlineto{\pgfqpoint{1.812269in}{1.902775in}}%
\pgfpathlineto{\pgfqpoint{1.801291in}{1.954215in}}%
\pgfpathlineto{\pgfqpoint{1.795770in}{2.005897in}}%
\pgfpathlineto{\pgfqpoint{1.798050in}{2.057613in}}%
\pgfpathlineto{\pgfqpoint{1.810656in}{2.108826in}}%
\pgfpathlineto{\pgfqpoint{1.834935in}{2.158733in}}%
\pgfpathlineto{\pgfqpoint{1.870114in}{2.206679in}}%
\pgfpathlineto{\pgfqpoint{1.914256in}{2.252393in}}%
\pgfusepath{stroke}%
\end{pgfscope}%
\begin{pgfscope}%
\pgfpathrectangle{\pgfqpoint{0.647939in}{0.492442in}}{\pgfqpoint{4.273799in}{2.331163in}}%
\pgfusepath{clip}%
\pgfsetbuttcap%
\pgfsetroundjoin%
\pgfsetlinewidth{0.301125pt}%
\definecolor{currentstroke}{rgb}{0.500000,0.500000,0.500000}%
\pgfsetstrokecolor{currentstroke}%
\pgfsetstrokeopacity{0.300000}%
\pgfsetdash{}{0pt}%
\pgfpathmoveto{\pgfqpoint{4.338948in}{1.499081in}}%
\pgfpathlineto{\pgfqpoint{4.296007in}{1.545233in}}%
\pgfpathlineto{\pgfqpoint{4.246527in}{1.589328in}}%
\pgfpathlineto{\pgfqpoint{4.186171in}{1.628991in}}%
\pgfpathlineto{\pgfqpoint{4.186171in}{1.628991in}}%
\pgfpathlineto{\pgfqpoint{4.130546in}{1.651919in}}%
\pgfpathlineto{\pgfqpoint{4.130546in}{1.651919in}}%
\pgfpathlineto{\pgfqpoint{4.075405in}{1.663141in}}%
\pgfpathlineto{\pgfqpoint{4.016015in}{1.664961in}}%
\pgfpathlineto{\pgfqpoint{3.956324in}{1.659509in}}%
\pgfusepath{stroke}%
\end{pgfscope}%
\begin{pgfscope}%
\pgfpathrectangle{\pgfqpoint{0.647939in}{0.492442in}}{\pgfqpoint{4.273799in}{2.331163in}}%
\pgfusepath{clip}%
\pgfsetbuttcap%
\pgfsetroundjoin%
\pgfsetlinewidth{0.301125pt}%
\definecolor{currentstroke}{rgb}{0.500000,0.500000,0.500000}%
\pgfsetstrokecolor{currentstroke}%
\pgfsetstrokeopacity{0.300000}%
\pgfsetdash{}{0pt}%
\pgfpathmoveto{\pgfqpoint{1.655463in}{1.213764in}}%
\pgfpathlineto{\pgfqpoint{1.612134in}{1.259837in}}%
\pgfpathlineto{\pgfqpoint{1.564377in}{1.304556in}}%
\pgfpathlineto{\pgfqpoint{1.515967in}{1.341935in}}%
\pgfpathlineto{\pgfqpoint{1.473640in}{1.366891in}}%
\pgfpathlineto{\pgfqpoint{1.434291in}{1.382748in}}%
\pgfpathlineto{\pgfqpoint{1.391521in}{1.391268in}}%
\pgfpathlineto{\pgfqpoint{1.345978in}{1.389994in}}%
\pgfpathlineto{\pgfqpoint{1.345978in}{1.389994in}}%
\pgfpathlineto{\pgfqpoint{1.296725in}{1.376629in}}%
\pgfpathlineto{\pgfqpoint{1.296725in}{1.376629in}}%
\pgfpathlineto{\pgfqpoint{1.230730in}{1.340138in}}%
\pgfusepath{stroke}%
\end{pgfscope}%
\begin{pgfscope}%
\pgfpathrectangle{\pgfqpoint{0.647939in}{0.492442in}}{\pgfqpoint{4.273799in}{2.331163in}}%
\pgfusepath{clip}%
\pgfsetbuttcap%
\pgfsetroundjoin%
\pgfsetlinewidth{0.301125pt}%
\definecolor{currentstroke}{rgb}{0.500000,0.500000,0.500000}%
\pgfsetstrokecolor{currentstroke}%
\pgfsetstrokeopacity{0.300000}%
\pgfsetdash{}{0pt}%
\pgfpathmoveto{\pgfqpoint{3.464761in}{2.452738in}}%
\pgfpathlineto{\pgfqpoint{3.490581in}{2.402889in}}%
\pgfpathlineto{\pgfqpoint{3.514448in}{2.352752in}}%
\pgfpathlineto{\pgfqpoint{3.536173in}{2.302326in}}%
\pgfpathlineto{\pgfqpoint{3.555513in}{2.251614in}}%
\pgfpathlineto{\pgfqpoint{3.572149in}{2.200618in}}%
\pgfpathlineto{\pgfqpoint{3.585649in}{2.149351in}}%
\pgfpathlineto{\pgfqpoint{3.595444in}{2.097837in}}%
\pgfpathlineto{\pgfqpoint{3.600763in}{2.046133in}}%
\pgfpathlineto{\pgfqpoint{3.600537in}{1.994360in}}%
\pgfpathlineto{\pgfqpoint{3.593264in}{1.942762in}}%
\pgfpathlineto{\pgfqpoint{3.576804in}{1.891841in}}%
\pgfpathlineto{\pgfqpoint{3.547985in}{1.842680in}}%
\pgfusepath{stroke}%
\end{pgfscope}%
\begin{pgfscope}%
\pgfpathrectangle{\pgfqpoint{0.647939in}{0.492442in}}{\pgfqpoint{4.273799in}{2.331163in}}%
\pgfusepath{clip}%
\pgfsetbuttcap%
\pgfsetroundjoin%
\pgfsetlinewidth{0.301125pt}%
\definecolor{currentstroke}{rgb}{0.500000,0.500000,0.500000}%
\pgfsetstrokecolor{currentstroke}%
\pgfsetstrokeopacity{0.300000}%
\pgfsetdash{}{0pt}%
\pgfpathmoveto{\pgfqpoint{1.537074in}{2.545944in}}%
\pgfpathlineto{\pgfqpoint{1.575525in}{2.498622in}}%
\pgfpathlineto{\pgfqpoint{1.619257in}{2.452738in}}%
\pgfpathlineto{\pgfqpoint{1.671955in}{2.409900in}}%
\pgfpathlineto{\pgfqpoint{1.671955in}{2.409900in}}%
\pgfpathlineto{\pgfqpoint{1.724186in}{2.381813in}}%
\pgfpathlineto{\pgfqpoint{1.724186in}{2.381813in}}%
\pgfpathlineto{\pgfqpoint{1.772505in}{2.368265in}}%
\pgfpathlineto{\pgfqpoint{1.827093in}{2.365297in}}%
\pgfusepath{stroke}%
\end{pgfscope}%
\begin{pgfscope}%
\pgfpathrectangle{\pgfqpoint{0.647939in}{0.492442in}}{\pgfqpoint{4.273799in}{2.331163in}}%
\pgfusepath{clip}%
\pgfsetbuttcap%
\pgfsetroundjoin%
\pgfsetlinewidth{0.301125pt}%
\definecolor{currentstroke}{rgb}{0.500000,0.500000,0.500000}%
\pgfsetstrokecolor{currentstroke}%
\pgfsetstrokeopacity{0.300000}%
\pgfsetdash{}{0pt}%
\pgfpathmoveto{\pgfqpoint{1.578243in}{1.445270in}}%
\pgfpathlineto{\pgfqpoint{1.534874in}{1.479336in}}%
\pgfpathlineto{\pgfqpoint{1.497074in}{1.501556in}}%
\pgfpathlineto{\pgfqpoint{1.461564in}{1.515262in}}%
\pgfpathlineto{\pgfqpoint{1.419384in}{1.521769in}}%
\pgfpathlineto{\pgfqpoint{1.376335in}{1.517249in}}%
\pgfpathlineto{\pgfqpoint{1.376335in}{1.517249in}}%
\pgfpathlineto{\pgfqpoint{1.327862in}{1.499081in}}%
\pgfpathlineto{\pgfqpoint{1.327862in}{1.499081in}}%
\pgfpathlineto{\pgfqpoint{1.267999in}{1.459773in}}%
\pgfpathlineto{\pgfqpoint{1.225268in}{1.422117in}}%
\pgfusepath{stroke}%
\end{pgfscope}%
\begin{pgfscope}%
\pgfpathrectangle{\pgfqpoint{0.647939in}{0.492442in}}{\pgfqpoint{4.273799in}{2.331163in}}%
\pgfusepath{clip}%
\pgfsetbuttcap%
\pgfsetroundjoin%
\pgfsetlinewidth{0.301125pt}%
\definecolor{currentstroke}{rgb}{0.500000,0.500000,0.500000}%
\pgfsetstrokecolor{currentstroke}%
\pgfsetstrokeopacity{0.300000}%
\pgfsetdash{}{0pt}%
\pgfpathmoveto{\pgfqpoint{4.159196in}{0.832526in}}%
\pgfpathlineto{\pgfqpoint{4.104893in}{0.875011in}}%
\pgfpathlineto{\pgfqpoint{4.047552in}{0.916290in}}%
\pgfpathlineto{\pgfqpoint{3.987137in}{0.956243in}}%
\pgfpathlineto{\pgfqpoint{3.923740in}{0.994798in}}%
\pgfpathlineto{\pgfqpoint{3.857665in}{1.031996in}}%
\pgfusepath{stroke}%
\end{pgfscope}%
\begin{pgfscope}%
\pgfpathrectangle{\pgfqpoint{0.647939in}{0.492442in}}{\pgfqpoint{4.273799in}{2.331163in}}%
\pgfusepath{clip}%
\pgfsetbuttcap%
\pgfsetroundjoin%
\pgfsetlinewidth{0.301125pt}%
\definecolor{currentstroke}{rgb}{0.500000,0.500000,0.500000}%
\pgfsetstrokecolor{currentstroke}%
\pgfsetstrokeopacity{0.300000}%
\pgfsetdash{}{0pt}%
\pgfpathmoveto{\pgfqpoint{4.199209in}{1.085829in}}%
\pgfpathlineto{\pgfqpoint{4.144684in}{1.128214in}}%
\pgfpathlineto{\pgfqpoint{4.085587in}{1.168729in}}%
\pgfpathlineto{\pgfqpoint{4.021487in}{1.206895in}}%
\pgfpathlineto{\pgfqpoint{3.952186in}{1.242250in}}%
\pgfpathlineto{\pgfqpoint{3.878031in}{1.274546in}}%
\pgfusepath{stroke}%
\end{pgfscope}%
\begin{pgfscope}%
\pgfpathrectangle{\pgfqpoint{0.647939in}{0.492442in}}{\pgfqpoint{4.273799in}{2.331163in}}%
\pgfusepath{clip}%
\pgfsetbuttcap%
\pgfsetroundjoin%
\pgfsetlinewidth{0.301125pt}%
\definecolor{currentstroke}{rgb}{0.500000,0.500000,0.500000}%
\pgfsetstrokecolor{currentstroke}%
\pgfsetstrokeopacity{0.300000}%
\pgfsetdash{}{0pt}%
\pgfpathmoveto{\pgfqpoint{4.255695in}{1.309344in}}%
\pgfpathlineto{\pgfqpoint{4.203506in}{1.352557in}}%
\pgfpathlineto{\pgfqpoint{4.144684in}{1.393119in}}%
\pgfpathlineto{\pgfqpoint{4.077442in}{1.429512in}}%
\pgfpathlineto{\pgfqpoint{4.000622in}{1.459575in}}%
\pgfpathlineto{\pgfqpoint{3.915116in}{1.481531in}}%
\pgfpathlineto{\pgfqpoint{3.824197in}{1.495998in}}%
\pgfpathlineto{\pgfqpoint{3.731142in}{1.506147in}}%
\pgfpathlineto{\pgfqpoint{3.637915in}{1.515879in}}%
\pgfusepath{stroke}%
\end{pgfscope}%
\begin{pgfscope}%
\pgfpathrectangle{\pgfqpoint{0.647939in}{0.492442in}}{\pgfqpoint{4.273799in}{2.331163in}}%
\pgfusepath{clip}%
\pgfsetbuttcap%
\pgfsetroundjoin%
\pgfsetlinewidth{0.301125pt}%
\definecolor{currentstroke}{rgb}{0.500000,0.500000,0.500000}%
\pgfsetstrokecolor{currentstroke}%
\pgfsetstrokeopacity{0.300000}%
\pgfsetdash{}{0pt}%
\pgfpathmoveto{\pgfqpoint{2.231806in}{1.618451in}}%
\pgfpathlineto{\pgfqpoint{2.211653in}{1.669073in}}%
\pgfpathlineto{\pgfqpoint{2.193172in}{1.719883in}}%
\pgfpathlineto{\pgfqpoint{2.176572in}{1.770886in}}%
\pgfpathlineto{\pgfqpoint{2.162123in}{1.822082in}}%
\pgfpathlineto{\pgfqpoint{2.150160in}{1.873467in}}%
\pgfpathlineto{\pgfqpoint{2.141102in}{1.925026in}}%
\pgfpathlineto{\pgfqpoint{2.135484in}{1.976728in}}%
\pgfpathlineto{\pgfqpoint{2.133977in}{2.028509in}}%
\pgfpathlineto{\pgfqpoint{2.137403in}{2.080254in}}%
\pgfpathlineto{\pgfqpoint{2.146753in}{2.131770in}}%
\pgfpathlineto{\pgfqpoint{2.163182in}{2.182742in}}%
\pgfpathlineto{\pgfqpoint{2.187990in}{2.232677in}}%
\pgfpathlineto{\pgfqpoint{2.222583in}{2.280825in}}%
\pgfpathlineto{\pgfqpoint{2.268333in}{2.326072in}}%
\pgfpathlineto{\pgfqpoint{2.325091in}{2.365949in}}%
\pgfpathlineto{\pgfqpoint{2.396312in}{2.399758in}}%
\pgfpathlineto{\pgfqpoint{2.396312in}{2.399758in}}%
\pgfusepath{stroke}%
\end{pgfscope}%
\begin{pgfscope}%
\pgfpathrectangle{\pgfqpoint{0.647939in}{0.492442in}}{\pgfqpoint{4.273799in}{2.331163in}}%
\pgfusepath{clip}%
\pgfsetbuttcap%
\pgfsetroundjoin%
\pgfsetlinewidth{0.301125pt}%
\definecolor{currentstroke}{rgb}{0.500000,0.500000,0.500000}%
\pgfsetstrokecolor{currentstroke}%
\pgfsetstrokeopacity{0.300000}%
\pgfsetdash{}{0pt}%
\pgfpathmoveto{\pgfqpoint{1.522125in}{2.399758in}}%
\pgfpathlineto{\pgfqpoint{1.555718in}{2.351322in}}%
\pgfpathlineto{\pgfqpoint{1.594093in}{2.303992in}}%
\pgfpathlineto{\pgfqpoint{1.642731in}{2.259783in}}%
\pgfpathlineto{\pgfqpoint{1.642731in}{2.259783in}}%
\pgfpathlineto{\pgfqpoint{1.678391in}{2.240621in}}%
\pgfpathlineto{\pgfqpoint{1.678391in}{2.240621in}}%
\pgfpathlineto{\pgfqpoint{1.712257in}{2.233236in}}%
\pgfpathlineto{\pgfqpoint{1.749435in}{2.235668in}}%
\pgfpathlineto{\pgfqpoint{1.782661in}{2.244737in}}%
\pgfpathlineto{\pgfqpoint{1.822508in}{2.261315in}}%
\pgfpathlineto{\pgfqpoint{1.877154in}{2.289285in}}%
\pgfusepath{stroke}%
\end{pgfscope}%
\begin{pgfscope}%
\pgfpathrectangle{\pgfqpoint{0.647939in}{0.492442in}}{\pgfqpoint{4.273799in}{2.331163in}}%
\pgfusepath{clip}%
\pgfsetbuttcap%
\pgfsetroundjoin%
\pgfsetlinewidth{0.301125pt}%
\definecolor{currentstroke}{rgb}{0.500000,0.500000,0.500000}%
\pgfsetstrokecolor{currentstroke}%
\pgfsetstrokeopacity{0.300000}%
\pgfsetdash{}{0pt}%
\pgfpathmoveto{\pgfqpoint{1.424993in}{2.240815in}}%
\pgfpathlineto{\pgfqpoint{1.440536in}{2.189715in}}%
\pgfpathlineto{\pgfqpoint{1.453958in}{2.138443in}}%
\pgfpathlineto{\pgfqpoint{1.464145in}{2.086954in}}%
\pgfpathlineto{\pgfqpoint{1.469327in}{2.035260in}}%
\pgfpathlineto{\pgfqpoint{1.467149in}{1.983535in}}%
\pgfpathlineto{\pgfqpoint{1.455756in}{1.932204in}}%
\pgfusepath{stroke}%
\end{pgfscope}%
\begin{pgfscope}%
\pgfpathrectangle{\pgfqpoint{0.647939in}{0.492442in}}{\pgfqpoint{4.273799in}{2.331163in}}%
\pgfusepath{clip}%
\pgfsetbuttcap%
\pgfsetroundjoin%
\pgfsetlinewidth{0.301125pt}%
\definecolor{currentstroke}{rgb}{0.500000,0.500000,0.500000}%
\pgfsetstrokecolor{currentstroke}%
\pgfsetstrokeopacity{0.300000}%
\pgfsetdash{}{0pt}%
\pgfpathmoveto{\pgfqpoint{1.542427in}{1.105398in}}%
\pgfpathlineto{\pgfqpoint{1.490915in}{1.144398in}}%
\pgfpathlineto{\pgfqpoint{1.424993in}{1.181195in}}%
\pgfpathlineto{\pgfqpoint{1.424993in}{1.181195in}}%
\pgfpathlineto{\pgfqpoint{1.372027in}{1.198974in}}%
\pgfpathlineto{\pgfqpoint{1.372027in}{1.198974in}}%
\pgfpathlineto{\pgfqpoint{1.323275in}{1.204904in}}%
\pgfpathlineto{\pgfqpoint{1.273841in}{1.200323in}}%
\pgfpathlineto{\pgfqpoint{1.231479in}{1.187996in}}%
\pgfpathlineto{\pgfqpoint{1.189082in}{1.168064in}}%
\pgfusepath{stroke}%
\end{pgfscope}%
\begin{pgfscope}%
\pgfpathrectangle{\pgfqpoint{0.647939in}{0.492442in}}{\pgfqpoint{4.273799in}{2.331163in}}%
\pgfusepath{clip}%
\pgfsetbuttcap%
\pgfsetroundjoin%
\pgfsetlinewidth{0.301125pt}%
\definecolor{currentstroke}{rgb}{0.500000,0.500000,0.500000}%
\pgfsetstrokecolor{currentstroke}%
\pgfsetstrokeopacity{0.300000}%
\pgfsetdash{}{0pt}%
\pgfpathmoveto{\pgfqpoint{3.690447in}{0.893000in}}%
\pgfpathlineto{\pgfqpoint{3.626058in}{0.931080in}}%
\pgfpathlineto{\pgfqpoint{3.561893in}{0.969271in}}%
\pgfpathlineto{\pgfqpoint{3.498426in}{1.007806in}}%
\pgfpathlineto{\pgfqpoint{3.436081in}{1.046880in}}%
\pgfpathlineto{\pgfqpoint{3.375231in}{1.086645in}}%
\pgfpathlineto{\pgfqpoint{3.316165in}{1.127202in}}%
\pgfpathlineto{\pgfqpoint{3.259117in}{1.168607in}}%
\pgfpathlineto{\pgfqpoint{3.204250in}{1.210879in}}%
\pgfpathlineto{\pgfqpoint{3.151667in}{1.254009in}}%
\pgfusepath{stroke}%
\end{pgfscope}%
\begin{pgfscope}%
\pgfpathrectangle{\pgfqpoint{0.647939in}{0.492442in}}{\pgfqpoint{4.273799in}{2.331163in}}%
\pgfusepath{clip}%
\pgfsetbuttcap%
\pgfsetroundjoin%
\pgfsetlinewidth{0.301125pt}%
\definecolor{currentstroke}{rgb}{0.500000,0.500000,0.500000}%
\pgfsetstrokecolor{currentstroke}%
\pgfsetstrokeopacity{0.300000}%
\pgfsetdash{}{0pt}%
\pgfpathmoveto{\pgfqpoint{4.047552in}{1.234176in}}%
\pgfpathlineto{\pgfqpoint{3.978721in}{1.269786in}}%
\pgfpathlineto{\pgfqpoint{3.904354in}{1.301913in}}%
\pgfpathlineto{\pgfqpoint{3.825358in}{1.330586in}}%
\pgfpathlineto{\pgfqpoint{3.743205in}{1.356532in}}%
\pgfpathlineto{\pgfqpoint{3.659619in}{1.381109in}}%
\pgfpathlineto{\pgfqpoint{3.576261in}{1.405904in}}%
\pgfusepath{stroke}%
\end{pgfscope}%
\begin{pgfscope}%
\pgfpathrectangle{\pgfqpoint{0.647939in}{0.492442in}}{\pgfqpoint{4.273799in}{2.331163in}}%
\pgfusepath{clip}%
\pgfsetbuttcap%
\pgfsetroundjoin%
\pgfsetlinewidth{0.301125pt}%
\definecolor{currentstroke}{rgb}{0.500000,0.500000,0.500000}%
\pgfsetstrokecolor{currentstroke}%
\pgfsetstrokeopacity{0.300000}%
\pgfsetdash{}{0pt}%
\pgfpathmoveto{\pgfqpoint{1.723622in}{1.795663in}}%
\pgfpathlineto{\pgfqpoint{1.698242in}{1.845570in}}%
\pgfpathlineto{\pgfqpoint{1.673254in}{1.895518in}}%
\pgfpathlineto{\pgfqpoint{1.648871in}{1.945534in}}%
\pgfpathlineto{\pgfqpoint{1.632599in}{1.980655in}}%
\pgfpathlineto{\pgfqpoint{1.620540in}{2.012564in}}%
\pgfpathlineto{\pgfqpoint{1.620540in}{2.012564in}}%
\pgfpathlineto{\pgfqpoint{1.619257in}{2.028891in}}%
\pgfpathlineto{\pgfqpoint{1.637885in}{2.046902in}}%
\pgfpathlineto{\pgfqpoint{1.676827in}{2.085547in}}%
\pgfpathlineto{\pgfqpoint{1.721671in}{2.128963in}}%
\pgfpathlineto{\pgfqpoint{1.770505in}{2.172835in}}%
\pgfusepath{stroke}%
\end{pgfscope}%
\begin{pgfscope}%
\pgfpathrectangle{\pgfqpoint{0.647939in}{0.492442in}}{\pgfqpoint{4.273799in}{2.331163in}}%
\pgfusepath{clip}%
\pgfsetbuttcap%
\pgfsetroundjoin%
\pgfsetlinewidth{0.301125pt}%
\definecolor{currentstroke}{rgb}{0.500000,0.500000,0.500000}%
\pgfsetstrokecolor{currentstroke}%
\pgfsetstrokeopacity{0.300000}%
\pgfsetdash{}{0pt}%
\pgfpathmoveto{\pgfqpoint{1.910652in}{1.075233in}}%
\pgfpathlineto{\pgfqpoint{1.877559in}{1.123789in}}%
\pgfpathlineto{\pgfqpoint{1.844214in}{1.172294in}}%
\pgfpathlineto{\pgfqpoint{1.810480in}{1.220719in}}%
\pgfpathlineto{\pgfqpoint{1.776158in}{1.269020in}}%
\pgfpathlineto{\pgfqpoint{1.740960in}{1.317131in}}%
\pgfpathlineto{\pgfqpoint{1.704477in}{1.364953in}}%
\pgfusepath{stroke}%
\end{pgfscope}%
\begin{pgfscope}%
\pgfpathrectangle{\pgfqpoint{0.647939in}{0.492442in}}{\pgfqpoint{4.273799in}{2.331163in}}%
\pgfusepath{clip}%
\pgfsetbuttcap%
\pgfsetroundjoin%
\pgfsetlinewidth{0.301125pt}%
\definecolor{currentstroke}{rgb}{0.500000,0.500000,0.500000}%
\pgfsetstrokecolor{currentstroke}%
\pgfsetstrokeopacity{0.300000}%
\pgfsetdash{}{0pt}%
\pgfpathmoveto{\pgfqpoint{3.785182in}{2.075576in}}%
\pgfpathlineto{\pgfqpoint{3.788094in}{2.023817in}}%
\pgfpathlineto{\pgfqpoint{3.785248in}{1.972068in}}%
\pgfpathlineto{\pgfqpoint{3.775194in}{1.920610in}}%
\pgfpathlineto{\pgfqpoint{3.756157in}{1.869948in}}%
\pgfpathlineto{\pgfqpoint{3.726010in}{1.820981in}}%
\pgfpathlineto{\pgfqpoint{3.682538in}{1.775197in}}%
\pgfusepath{stroke}%
\end{pgfscope}%
\begin{pgfscope}%
\pgfpathrectangle{\pgfqpoint{0.647939in}{0.492442in}}{\pgfqpoint{4.273799in}{2.331163in}}%
\pgfusepath{clip}%
\pgfsetbuttcap%
\pgfsetroundjoin%
\pgfsetlinewidth{0.301125pt}%
\definecolor{currentstroke}{rgb}{0.500000,0.500000,0.500000}%
\pgfsetstrokecolor{currentstroke}%
\pgfsetstrokeopacity{0.300000}%
\pgfsetdash{}{0pt}%
\pgfpathmoveto{\pgfqpoint{3.219515in}{2.287583in}}%
\pgfpathlineto{\pgfqpoint{3.246982in}{2.238008in}}%
\pgfpathlineto{\pgfqpoint{3.270498in}{2.187834in}}%
\pgfpathlineto{\pgfqpoint{3.289486in}{2.137097in}}%
\pgfpathlineto{\pgfqpoint{3.303020in}{2.085855in}}%
\pgfpathlineto{\pgfqpoint{3.309582in}{2.034231in}}%
\pgfpathlineto{\pgfqpoint{3.306449in}{1.982568in}}%
\pgfpathlineto{\pgfqpoint{3.287991in}{1.932064in}}%
\pgfpathlineto{\pgfqpoint{3.287991in}{1.932064in}}%
\pgfpathlineto{\pgfqpoint{3.261800in}{1.901525in}}%
\pgfpathlineto{\pgfqpoint{3.261800in}{1.901525in}}%
\pgfusepath{stroke}%
\end{pgfscope}%
\begin{pgfscope}%
\pgfpathrectangle{\pgfqpoint{0.647939in}{0.492442in}}{\pgfqpoint{4.273799in}{2.331163in}}%
\pgfusepath{clip}%
\pgfsetbuttcap%
\pgfsetroundjoin%
\pgfsetlinewidth{0.301125pt}%
\definecolor{currentstroke}{rgb}{0.500000,0.500000,0.500000}%
\pgfsetstrokecolor{currentstroke}%
\pgfsetstrokeopacity{0.300000}%
\pgfsetdash{}{0pt}%
\pgfpathmoveto{\pgfqpoint{3.681205in}{2.127108in}}%
\pgfpathlineto{\pgfqpoint{3.688812in}{2.075485in}}%
\pgfpathlineto{\pgfqpoint{3.691737in}{2.023727in}}%
\pgfpathlineto{\pgfqpoint{3.688848in}{1.971980in}}%
\pgfpathlineto{\pgfqpoint{3.678625in}{1.920534in}}%
\pgfpathlineto{\pgfqpoint{3.659025in}{1.869948in}}%
\pgfpathlineto{\pgfqpoint{3.627399in}{1.821293in}}%
\pgfpathlineto{\pgfqpoint{3.580459in}{1.776656in}}%
\pgfpathlineto{\pgfqpoint{3.580459in}{1.776656in}}%
\pgfpathlineto{\pgfqpoint{3.528983in}{1.745897in}}%
\pgfpathlineto{\pgfqpoint{3.463835in}{1.723507in}}%
\pgfpathlineto{\pgfqpoint{3.399128in}{1.714124in}}%
\pgfpathlineto{\pgfqpoint{3.337377in}{1.715019in}}%
\pgfusepath{stroke}%
\end{pgfscope}%
\begin{pgfscope}%
\pgfpathrectangle{\pgfqpoint{0.647939in}{0.492442in}}{\pgfqpoint{4.273799in}{2.331163in}}%
\pgfusepath{clip}%
\pgfsetbuttcap%
\pgfsetroundjoin%
\pgfsetlinewidth{0.301125pt}%
\definecolor{currentstroke}{rgb}{0.500000,0.500000,0.500000}%
\pgfsetstrokecolor{currentstroke}%
\pgfsetstrokeopacity{0.300000}%
\pgfsetdash{}{0pt}%
\pgfpathmoveto{\pgfqpoint{2.326913in}{1.290593in}}%
\pgfpathlineto{\pgfqpoint{2.299180in}{1.340138in}}%
\pgfpathlineto{\pgfqpoint{2.272395in}{1.389838in}}%
\pgfpathlineto{\pgfqpoint{2.246613in}{1.439695in}}%
\pgfpathlineto{\pgfqpoint{2.221886in}{1.489711in}}%
\pgfpathlineto{\pgfqpoint{2.198289in}{1.539889in}}%
\pgfusepath{stroke}%
\end{pgfscope}%
\begin{pgfscope}%
\pgfpathrectangle{\pgfqpoint{0.647939in}{0.492442in}}{\pgfqpoint{4.273799in}{2.331163in}}%
\pgfusepath{clip}%
\pgfsetbuttcap%
\pgfsetroundjoin%
\pgfsetlinewidth{0.301125pt}%
\definecolor{currentstroke}{rgb}{0.500000,0.500000,0.500000}%
\pgfsetstrokecolor{currentstroke}%
\pgfsetstrokeopacity{0.300000}%
\pgfsetdash{}{0pt}%
\pgfpathmoveto{\pgfqpoint{2.804587in}{1.420037in}}%
\pgfpathlineto{\pgfqpoint{2.770462in}{1.468377in}}%
\pgfpathlineto{\pgfqpoint{2.738258in}{1.517107in}}%
\pgfpathlineto{\pgfqpoint{2.708031in}{1.566212in}}%
\pgfpathlineto{\pgfqpoint{2.679865in}{1.615680in}}%
\pgfpathlineto{\pgfqpoint{2.653864in}{1.665500in}}%
\pgfpathlineto{\pgfqpoint{2.630176in}{1.715661in}}%
\pgfpathlineto{\pgfqpoint{2.608994in}{1.766154in}}%
\pgfpathlineto{\pgfqpoint{2.590575in}{1.816967in}}%
\pgfpathlineto{\pgfqpoint{2.575256in}{1.868082in}}%
\pgfpathlineto{\pgfqpoint{2.563492in}{1.919474in}}%
\pgfusepath{stroke}%
\end{pgfscope}%
\begin{pgfscope}%
\pgfpathrectangle{\pgfqpoint{0.647939in}{0.492442in}}{\pgfqpoint{4.273799in}{2.331163in}}%
\pgfusepath{clip}%
\pgfsetroundcap%
\pgfsetroundjoin%
\pgfsetlinewidth{0.301125pt}%
\definecolor{currentstroke}{rgb}{0.500000,0.500000,0.500000}%
\pgfsetstrokecolor{currentstroke}%
\pgfsetstrokeopacity{0.300000}%
\pgfsetdash{}{0pt}%
\pgfpathmoveto{\pgfqpoint{1.452689in}{1.462273in}}%
\pgfusepath{stroke}%
\end{pgfscope}%
\begin{pgfscope}%
\pgfpathrectangle{\pgfqpoint{0.647939in}{0.492442in}}{\pgfqpoint{4.273799in}{2.331163in}}%
\pgfusepath{clip}%
\pgfsetroundcap%
\pgfsetroundjoin%
\definecolor{currentfill}{rgb}{0.500000,0.500000,0.500000}%
\pgfsetfillcolor{currentfill}%
\pgfsetfillopacity{0.300000}%
\pgfsetlinewidth{0.301125pt}%
\definecolor{currentstroke}{rgb}{0.500000,0.500000,0.500000}%
\pgfsetstrokecolor{currentstroke}%
\pgfsetstrokeopacity{0.300000}%
\pgfsetdash{}{0pt}%
\pgfpathmoveto{\pgfqpoint{0.000000in}{0.000000in}}%
\pgfpathlineto{\pgfqpoint{0.000000in}{0.000000in}}%
\pgfpathclose%
\pgfusepath{stroke,fill}%
\end{pgfscope}%
\begin{pgfscope}%
\pgfpathrectangle{\pgfqpoint{0.647939in}{0.492442in}}{\pgfqpoint{4.273799in}{2.331163in}}%
\pgfusepath{clip}%
\pgfsetroundcap%
\pgfsetroundjoin%
\pgfsetlinewidth{0.301125pt}%
\definecolor{currentstroke}{rgb}{0.500000,0.500000,0.500000}%
\pgfsetstrokecolor{currentstroke}%
\pgfsetstrokeopacity{0.300000}%
\pgfsetdash{}{0pt}%
\pgfpathmoveto{\pgfqpoint{1.244943in}{0.903842in}}%
\pgfusepath{stroke}%
\end{pgfscope}%
\begin{pgfscope}%
\pgfpathrectangle{\pgfqpoint{0.647939in}{0.492442in}}{\pgfqpoint{4.273799in}{2.331163in}}%
\pgfusepath{clip}%
\pgfsetroundcap%
\pgfsetroundjoin%
\definecolor{currentfill}{rgb}{0.500000,0.500000,0.500000}%
\pgfsetfillcolor{currentfill}%
\pgfsetfillopacity{0.300000}%
\pgfsetlinewidth{0.301125pt}%
\definecolor{currentstroke}{rgb}{0.500000,0.500000,0.500000}%
\pgfsetstrokecolor{currentstroke}%
\pgfsetstrokeopacity{0.300000}%
\pgfsetdash{}{0pt}%
\pgfpathmoveto{\pgfqpoint{0.000000in}{0.000000in}}%
\pgfpathlineto{\pgfqpoint{0.000000in}{0.000000in}}%
\pgfpathclose%
\pgfusepath{stroke,fill}%
\end{pgfscope}%
\begin{pgfscope}%
\pgfpathrectangle{\pgfqpoint{0.647939in}{0.492442in}}{\pgfqpoint{4.273799in}{2.331163in}}%
\pgfusepath{clip}%
\pgfsetroundcap%
\pgfsetroundjoin%
\pgfsetlinewidth{0.301125pt}%
\definecolor{currentstroke}{rgb}{0.500000,0.500000,0.500000}%
\pgfsetstrokecolor{currentstroke}%
\pgfsetstrokeopacity{0.300000}%
\pgfsetdash{}{0pt}%
\pgfpathmoveto{\pgfqpoint{1.210159in}{0.684654in}}%
\pgfusepath{stroke}%
\end{pgfscope}%
\begin{pgfscope}%
\pgfpathrectangle{\pgfqpoint{0.647939in}{0.492442in}}{\pgfqpoint{4.273799in}{2.331163in}}%
\pgfusepath{clip}%
\pgfsetroundcap%
\pgfsetroundjoin%
\definecolor{currentfill}{rgb}{0.500000,0.500000,0.500000}%
\pgfsetfillcolor{currentfill}%
\pgfsetfillopacity{0.300000}%
\pgfsetlinewidth{0.301125pt}%
\definecolor{currentstroke}{rgb}{0.500000,0.500000,0.500000}%
\pgfsetstrokecolor{currentstroke}%
\pgfsetstrokeopacity{0.300000}%
\pgfsetdash{}{0pt}%
\pgfpathmoveto{\pgfqpoint{0.000000in}{0.000000in}}%
\pgfpathlineto{\pgfqpoint{0.000000in}{0.000000in}}%
\pgfpathclose%
\pgfusepath{stroke,fill}%
\end{pgfscope}%
\begin{pgfscope}%
\pgfpathrectangle{\pgfqpoint{0.647939in}{0.492442in}}{\pgfqpoint{4.273799in}{2.331163in}}%
\pgfusepath{clip}%
\pgfsetroundcap%
\pgfsetroundjoin%
\pgfsetlinewidth{0.301125pt}%
\definecolor{currentstroke}{rgb}{0.500000,0.500000,0.500000}%
\pgfsetstrokecolor{currentstroke}%
\pgfsetstrokeopacity{0.300000}%
\pgfsetdash{}{0pt}%
\pgfpathmoveto{\pgfqpoint{1.173414in}{0.568768in}}%
\pgfusepath{stroke}%
\end{pgfscope}%
\begin{pgfscope}%
\pgfpathrectangle{\pgfqpoint{0.647939in}{0.492442in}}{\pgfqpoint{4.273799in}{2.331163in}}%
\pgfusepath{clip}%
\pgfsetroundcap%
\pgfsetroundjoin%
\definecolor{currentfill}{rgb}{0.500000,0.500000,0.500000}%
\pgfsetfillcolor{currentfill}%
\pgfsetfillopacity{0.300000}%
\pgfsetlinewidth{0.301125pt}%
\definecolor{currentstroke}{rgb}{0.500000,0.500000,0.500000}%
\pgfsetstrokecolor{currentstroke}%
\pgfsetstrokeopacity{0.300000}%
\pgfsetdash{}{0pt}%
\pgfpathmoveto{\pgfqpoint{0.000000in}{0.000000in}}%
\pgfpathlineto{\pgfqpoint{0.000000in}{0.000000in}}%
\pgfpathclose%
\pgfusepath{stroke,fill}%
\end{pgfscope}%
\begin{pgfscope}%
\pgfpathrectangle{\pgfqpoint{0.647939in}{0.492442in}}{\pgfqpoint{4.273799in}{2.331163in}}%
\pgfusepath{clip}%
\pgfsetroundcap%
\pgfsetroundjoin%
\pgfsetlinewidth{0.301125pt}%
\definecolor{currentstroke}{rgb}{0.500000,0.500000,0.500000}%
\pgfsetstrokecolor{currentstroke}%
\pgfsetstrokeopacity{0.300000}%
\pgfsetdash{}{0pt}%
\pgfpathmoveto{\pgfqpoint{1.448481in}{0.640912in}}%
\pgfusepath{stroke}%
\end{pgfscope}%
\begin{pgfscope}%
\pgfpathrectangle{\pgfqpoint{0.647939in}{0.492442in}}{\pgfqpoint{4.273799in}{2.331163in}}%
\pgfusepath{clip}%
\pgfsetroundcap%
\pgfsetroundjoin%
\definecolor{currentfill}{rgb}{0.500000,0.500000,0.500000}%
\pgfsetfillcolor{currentfill}%
\pgfsetfillopacity{0.300000}%
\pgfsetlinewidth{0.301125pt}%
\definecolor{currentstroke}{rgb}{0.500000,0.500000,0.500000}%
\pgfsetstrokecolor{currentstroke}%
\pgfsetstrokeopacity{0.300000}%
\pgfsetdash{}{0pt}%
\pgfpathmoveto{\pgfqpoint{0.000000in}{0.000000in}}%
\pgfpathlineto{\pgfqpoint{0.000000in}{0.000000in}}%
\pgfpathclose%
\pgfusepath{stroke,fill}%
\end{pgfscope}%
\begin{pgfscope}%
\pgfpathrectangle{\pgfqpoint{0.647939in}{0.492442in}}{\pgfqpoint{4.273799in}{2.331163in}}%
\pgfusepath{clip}%
\pgfsetroundcap%
\pgfsetroundjoin%
\pgfsetlinewidth{0.301125pt}%
\definecolor{currentstroke}{rgb}{0.500000,0.500000,0.500000}%
\pgfsetstrokecolor{currentstroke}%
\pgfsetstrokeopacity{0.300000}%
\pgfsetdash{}{0pt}%
\pgfpathmoveto{\pgfqpoint{1.464288in}{0.969987in}}%
\pgfusepath{stroke}%
\end{pgfscope}%
\begin{pgfscope}%
\pgfpathrectangle{\pgfqpoint{0.647939in}{0.492442in}}{\pgfqpoint{4.273799in}{2.331163in}}%
\pgfusepath{clip}%
\pgfsetroundcap%
\pgfsetroundjoin%
\definecolor{currentfill}{rgb}{0.500000,0.500000,0.500000}%
\pgfsetfillcolor{currentfill}%
\pgfsetfillopacity{0.300000}%
\pgfsetlinewidth{0.301125pt}%
\definecolor{currentstroke}{rgb}{0.500000,0.500000,0.500000}%
\pgfsetstrokecolor{currentstroke}%
\pgfsetstrokeopacity{0.300000}%
\pgfsetdash{}{0pt}%
\pgfpathmoveto{\pgfqpoint{0.000000in}{0.000000in}}%
\pgfpathlineto{\pgfqpoint{0.000000in}{0.000000in}}%
\pgfpathclose%
\pgfusepath{stroke,fill}%
\end{pgfscope}%
\begin{pgfscope}%
\pgfpathrectangle{\pgfqpoint{0.647939in}{0.492442in}}{\pgfqpoint{4.273799in}{2.331163in}}%
\pgfusepath{clip}%
\pgfsetroundcap%
\pgfsetroundjoin%
\pgfsetlinewidth{0.301125pt}%
\definecolor{currentstroke}{rgb}{0.500000,0.500000,0.500000}%
\pgfsetstrokecolor{currentstroke}%
\pgfsetstrokeopacity{0.300000}%
\pgfsetdash{}{0pt}%
\pgfpathmoveto{\pgfqpoint{1.606537in}{0.986124in}}%
\pgfusepath{stroke}%
\end{pgfscope}%
\begin{pgfscope}%
\pgfpathrectangle{\pgfqpoint{0.647939in}{0.492442in}}{\pgfqpoint{4.273799in}{2.331163in}}%
\pgfusepath{clip}%
\pgfsetroundcap%
\pgfsetroundjoin%
\definecolor{currentfill}{rgb}{0.500000,0.500000,0.500000}%
\pgfsetfillcolor{currentfill}%
\pgfsetfillopacity{0.300000}%
\pgfsetlinewidth{0.301125pt}%
\definecolor{currentstroke}{rgb}{0.500000,0.500000,0.500000}%
\pgfsetstrokecolor{currentstroke}%
\pgfsetstrokeopacity{0.300000}%
\pgfsetdash{}{0pt}%
\pgfpathmoveto{\pgfqpoint{0.000000in}{0.000000in}}%
\pgfpathlineto{\pgfqpoint{0.000000in}{0.000000in}}%
\pgfpathclose%
\pgfusepath{stroke,fill}%
\end{pgfscope}%
\begin{pgfscope}%
\pgfpathrectangle{\pgfqpoint{0.647939in}{0.492442in}}{\pgfqpoint{4.273799in}{2.331163in}}%
\pgfusepath{clip}%
\pgfsetroundcap%
\pgfsetroundjoin%
\pgfsetlinewidth{0.301125pt}%
\definecolor{currentstroke}{rgb}{0.500000,0.500000,0.500000}%
\pgfsetstrokecolor{currentstroke}%
\pgfsetstrokeopacity{0.300000}%
\pgfsetdash{}{0pt}%
\pgfpathmoveto{\pgfqpoint{1.689530in}{1.040438in}}%
\pgfusepath{stroke}%
\end{pgfscope}%
\begin{pgfscope}%
\pgfpathrectangle{\pgfqpoint{0.647939in}{0.492442in}}{\pgfqpoint{4.273799in}{2.331163in}}%
\pgfusepath{clip}%
\pgfsetroundcap%
\pgfsetroundjoin%
\definecolor{currentfill}{rgb}{0.500000,0.500000,0.500000}%
\pgfsetfillcolor{currentfill}%
\pgfsetfillopacity{0.300000}%
\pgfsetlinewidth{0.301125pt}%
\definecolor{currentstroke}{rgb}{0.500000,0.500000,0.500000}%
\pgfsetstrokecolor{currentstroke}%
\pgfsetstrokeopacity{0.300000}%
\pgfsetdash{}{0pt}%
\pgfpathmoveto{\pgfqpoint{0.000000in}{0.000000in}}%
\pgfpathlineto{\pgfqpoint{0.000000in}{0.000000in}}%
\pgfpathclose%
\pgfusepath{stroke,fill}%
\end{pgfscope}%
\begin{pgfscope}%
\pgfpathrectangle{\pgfqpoint{0.647939in}{0.492442in}}{\pgfqpoint{4.273799in}{2.331163in}}%
\pgfusepath{clip}%
\pgfsetroundcap%
\pgfsetroundjoin%
\pgfsetlinewidth{0.301125pt}%
\definecolor{currentstroke}{rgb}{0.500000,0.500000,0.500000}%
\pgfsetstrokecolor{currentstroke}%
\pgfsetstrokeopacity{0.300000}%
\pgfsetdash{}{0pt}%
\pgfpathmoveto{\pgfqpoint{1.821836in}{1.341707in}}%
\pgfusepath{stroke}%
\end{pgfscope}%
\begin{pgfscope}%
\pgfpathrectangle{\pgfqpoint{0.647939in}{0.492442in}}{\pgfqpoint{4.273799in}{2.331163in}}%
\pgfusepath{clip}%
\pgfsetroundcap%
\pgfsetroundjoin%
\definecolor{currentfill}{rgb}{0.500000,0.500000,0.500000}%
\pgfsetfillcolor{currentfill}%
\pgfsetfillopacity{0.300000}%
\pgfsetlinewidth{0.301125pt}%
\definecolor{currentstroke}{rgb}{0.500000,0.500000,0.500000}%
\pgfsetstrokecolor{currentstroke}%
\pgfsetstrokeopacity{0.300000}%
\pgfsetdash{}{0pt}%
\pgfpathmoveto{\pgfqpoint{0.000000in}{0.000000in}}%
\pgfpathlineto{\pgfqpoint{0.000000in}{0.000000in}}%
\pgfpathclose%
\pgfusepath{stroke,fill}%
\end{pgfscope}%
\begin{pgfscope}%
\pgfpathrectangle{\pgfqpoint{0.647939in}{0.492442in}}{\pgfqpoint{4.273799in}{2.331163in}}%
\pgfusepath{clip}%
\pgfsetroundcap%
\pgfsetroundjoin%
\pgfsetlinewidth{0.301125pt}%
\definecolor{currentstroke}{rgb}{0.500000,0.500000,0.500000}%
\pgfsetstrokecolor{currentstroke}%
\pgfsetstrokeopacity{0.300000}%
\pgfsetdash{}{0pt}%
\pgfpathmoveto{\pgfqpoint{1.956130in}{1.294260in}}%
\pgfusepath{stroke}%
\end{pgfscope}%
\begin{pgfscope}%
\pgfpathrectangle{\pgfqpoint{0.647939in}{0.492442in}}{\pgfqpoint{4.273799in}{2.331163in}}%
\pgfusepath{clip}%
\pgfsetroundcap%
\pgfsetroundjoin%
\definecolor{currentfill}{rgb}{0.500000,0.500000,0.500000}%
\pgfsetfillcolor{currentfill}%
\pgfsetfillopacity{0.300000}%
\pgfsetlinewidth{0.301125pt}%
\definecolor{currentstroke}{rgb}{0.500000,0.500000,0.500000}%
\pgfsetstrokecolor{currentstroke}%
\pgfsetstrokeopacity{0.300000}%
\pgfsetdash{}{0pt}%
\pgfpathmoveto{\pgfqpoint{0.000000in}{0.000000in}}%
\pgfpathlineto{\pgfqpoint{0.000000in}{0.000000in}}%
\pgfpathclose%
\pgfusepath{stroke,fill}%
\end{pgfscope}%
\begin{pgfscope}%
\pgfpathrectangle{\pgfqpoint{0.647939in}{0.492442in}}{\pgfqpoint{4.273799in}{2.331163in}}%
\pgfusepath{clip}%
\pgfsetroundcap%
\pgfsetroundjoin%
\pgfsetlinewidth{0.301125pt}%
\definecolor{currentstroke}{rgb}{0.500000,0.500000,0.500000}%
\pgfsetstrokecolor{currentstroke}%
\pgfsetstrokeopacity{0.300000}%
\pgfsetdash{}{0pt}%
\pgfpathmoveto{\pgfqpoint{2.500119in}{0.610126in}}%
\pgfusepath{stroke}%
\end{pgfscope}%
\begin{pgfscope}%
\pgfpathrectangle{\pgfqpoint{0.647939in}{0.492442in}}{\pgfqpoint{4.273799in}{2.331163in}}%
\pgfusepath{clip}%
\pgfsetroundcap%
\pgfsetroundjoin%
\definecolor{currentfill}{rgb}{0.500000,0.500000,0.500000}%
\pgfsetfillcolor{currentfill}%
\pgfsetfillopacity{0.300000}%
\pgfsetlinewidth{0.301125pt}%
\definecolor{currentstroke}{rgb}{0.500000,0.500000,0.500000}%
\pgfsetstrokecolor{currentstroke}%
\pgfsetstrokeopacity{0.300000}%
\pgfsetdash{}{0pt}%
\pgfpathmoveto{\pgfqpoint{0.000000in}{0.000000in}}%
\pgfpathlineto{\pgfqpoint{0.000000in}{0.000000in}}%
\pgfpathclose%
\pgfusepath{stroke,fill}%
\end{pgfscope}%
\begin{pgfscope}%
\pgfpathrectangle{\pgfqpoint{0.647939in}{0.492442in}}{\pgfqpoint{4.273799in}{2.331163in}}%
\pgfusepath{clip}%
\pgfsetroundcap%
\pgfsetroundjoin%
\pgfsetlinewidth{0.301125pt}%
\definecolor{currentstroke}{rgb}{0.500000,0.500000,0.500000}%
\pgfsetstrokecolor{currentstroke}%
\pgfsetstrokeopacity{0.300000}%
\pgfsetdash{}{0pt}%
\pgfpathmoveto{\pgfqpoint{1.957931in}{2.075634in}}%
\pgfusepath{stroke}%
\end{pgfscope}%
\begin{pgfscope}%
\pgfpathrectangle{\pgfqpoint{0.647939in}{0.492442in}}{\pgfqpoint{4.273799in}{2.331163in}}%
\pgfusepath{clip}%
\pgfsetroundcap%
\pgfsetroundjoin%
\definecolor{currentfill}{rgb}{0.500000,0.500000,0.500000}%
\pgfsetfillcolor{currentfill}%
\pgfsetfillopacity{0.300000}%
\pgfsetlinewidth{0.301125pt}%
\definecolor{currentstroke}{rgb}{0.500000,0.500000,0.500000}%
\pgfsetstrokecolor{currentstroke}%
\pgfsetstrokeopacity{0.300000}%
\pgfsetdash{}{0pt}%
\pgfpathmoveto{\pgfqpoint{0.000000in}{0.000000in}}%
\pgfpathlineto{\pgfqpoint{0.000000in}{0.000000in}}%
\pgfpathclose%
\pgfusepath{stroke,fill}%
\end{pgfscope}%
\begin{pgfscope}%
\pgfpathrectangle{\pgfqpoint{0.647939in}{0.492442in}}{\pgfqpoint{4.273799in}{2.331163in}}%
\pgfusepath{clip}%
\pgfsetroundcap%
\pgfsetroundjoin%
\pgfsetlinewidth{0.301125pt}%
\definecolor{currentstroke}{rgb}{0.500000,0.500000,0.500000}%
\pgfsetstrokecolor{currentstroke}%
\pgfsetstrokeopacity{0.300000}%
\pgfsetdash{}{0pt}%
\pgfpathmoveto{\pgfqpoint{2.741554in}{0.654113in}}%
\pgfusepath{stroke}%
\end{pgfscope}%
\begin{pgfscope}%
\pgfpathrectangle{\pgfqpoint{0.647939in}{0.492442in}}{\pgfqpoint{4.273799in}{2.331163in}}%
\pgfusepath{clip}%
\pgfsetroundcap%
\pgfsetroundjoin%
\definecolor{currentfill}{rgb}{0.500000,0.500000,0.500000}%
\pgfsetfillcolor{currentfill}%
\pgfsetfillopacity{0.300000}%
\pgfsetlinewidth{0.301125pt}%
\definecolor{currentstroke}{rgb}{0.500000,0.500000,0.500000}%
\pgfsetstrokecolor{currentstroke}%
\pgfsetstrokeopacity{0.300000}%
\pgfsetdash{}{0pt}%
\pgfpathmoveto{\pgfqpoint{0.000000in}{0.000000in}}%
\pgfpathlineto{\pgfqpoint{0.000000in}{0.000000in}}%
\pgfpathclose%
\pgfusepath{stroke,fill}%
\end{pgfscope}%
\begin{pgfscope}%
\pgfpathrectangle{\pgfqpoint{0.647939in}{0.492442in}}{\pgfqpoint{4.273799in}{2.331163in}}%
\pgfusepath{clip}%
\pgfsetroundcap%
\pgfsetroundjoin%
\pgfsetlinewidth{0.301125pt}%
\definecolor{currentstroke}{rgb}{0.500000,0.500000,0.500000}%
\pgfsetstrokecolor{currentstroke}%
\pgfsetstrokeopacity{0.300000}%
\pgfsetdash{}{0pt}%
\pgfpathmoveto{\pgfqpoint{2.873931in}{0.605291in}}%
\pgfusepath{stroke}%
\end{pgfscope}%
\begin{pgfscope}%
\pgfpathrectangle{\pgfqpoint{0.647939in}{0.492442in}}{\pgfqpoint{4.273799in}{2.331163in}}%
\pgfusepath{clip}%
\pgfsetroundcap%
\pgfsetroundjoin%
\definecolor{currentfill}{rgb}{0.500000,0.500000,0.500000}%
\pgfsetfillcolor{currentfill}%
\pgfsetfillopacity{0.300000}%
\pgfsetlinewidth{0.301125pt}%
\definecolor{currentstroke}{rgb}{0.500000,0.500000,0.500000}%
\pgfsetstrokecolor{currentstroke}%
\pgfsetstrokeopacity{0.300000}%
\pgfsetdash{}{0pt}%
\pgfpathmoveto{\pgfqpoint{0.000000in}{0.000000in}}%
\pgfpathlineto{\pgfqpoint{0.000000in}{0.000000in}}%
\pgfpathclose%
\pgfusepath{stroke,fill}%
\end{pgfscope}%
\begin{pgfscope}%
\pgfpathrectangle{\pgfqpoint{0.647939in}{0.492442in}}{\pgfqpoint{4.273799in}{2.331163in}}%
\pgfusepath{clip}%
\pgfsetroundcap%
\pgfsetroundjoin%
\pgfsetlinewidth{0.301125pt}%
\definecolor{currentstroke}{rgb}{0.500000,0.500000,0.500000}%
\pgfsetstrokecolor{currentstroke}%
\pgfsetstrokeopacity{0.300000}%
\pgfsetdash{}{0pt}%
\pgfpathmoveto{\pgfqpoint{2.407236in}{1.465087in}}%
\pgfusepath{stroke}%
\end{pgfscope}%
\begin{pgfscope}%
\pgfpathrectangle{\pgfqpoint{0.647939in}{0.492442in}}{\pgfqpoint{4.273799in}{2.331163in}}%
\pgfusepath{clip}%
\pgfsetroundcap%
\pgfsetroundjoin%
\definecolor{currentfill}{rgb}{0.500000,0.500000,0.500000}%
\pgfsetfillcolor{currentfill}%
\pgfsetfillopacity{0.300000}%
\pgfsetlinewidth{0.301125pt}%
\definecolor{currentstroke}{rgb}{0.500000,0.500000,0.500000}%
\pgfsetstrokecolor{currentstroke}%
\pgfsetstrokeopacity{0.300000}%
\pgfsetdash{}{0pt}%
\pgfpathmoveto{\pgfqpoint{0.000000in}{0.000000in}}%
\pgfpathlineto{\pgfqpoint{0.000000in}{0.000000in}}%
\pgfpathclose%
\pgfusepath{stroke,fill}%
\end{pgfscope}%
\begin{pgfscope}%
\pgfpathrectangle{\pgfqpoint{0.647939in}{0.492442in}}{\pgfqpoint{4.273799in}{2.331163in}}%
\pgfusepath{clip}%
\pgfsetroundcap%
\pgfsetroundjoin%
\pgfsetlinewidth{0.301125pt}%
\definecolor{currentstroke}{rgb}{0.500000,0.500000,0.500000}%
\pgfsetstrokecolor{currentstroke}%
\pgfsetstrokeopacity{0.300000}%
\pgfsetdash{}{0pt}%
\pgfpathmoveto{\pgfqpoint{2.528375in}{1.445512in}}%
\pgfusepath{stroke}%
\end{pgfscope}%
\begin{pgfscope}%
\pgfpathrectangle{\pgfqpoint{0.647939in}{0.492442in}}{\pgfqpoint{4.273799in}{2.331163in}}%
\pgfusepath{clip}%
\pgfsetroundcap%
\pgfsetroundjoin%
\definecolor{currentfill}{rgb}{0.500000,0.500000,0.500000}%
\pgfsetfillcolor{currentfill}%
\pgfsetfillopacity{0.300000}%
\pgfsetlinewidth{0.301125pt}%
\definecolor{currentstroke}{rgb}{0.500000,0.500000,0.500000}%
\pgfsetstrokecolor{currentstroke}%
\pgfsetstrokeopacity{0.300000}%
\pgfsetdash{}{0pt}%
\pgfpathmoveto{\pgfqpoint{0.000000in}{0.000000in}}%
\pgfpathlineto{\pgfqpoint{0.000000in}{0.000000in}}%
\pgfpathclose%
\pgfusepath{stroke,fill}%
\end{pgfscope}%
\begin{pgfscope}%
\pgfpathrectangle{\pgfqpoint{0.647939in}{0.492442in}}{\pgfqpoint{4.273799in}{2.331163in}}%
\pgfusepath{clip}%
\pgfsetroundcap%
\pgfsetroundjoin%
\pgfsetlinewidth{0.301125pt}%
\definecolor{currentstroke}{rgb}{0.500000,0.500000,0.500000}%
\pgfsetstrokecolor{currentstroke}%
\pgfsetstrokeopacity{0.300000}%
\pgfsetdash{}{0pt}%
\pgfpathmoveto{\pgfqpoint{2.738746in}{1.274486in}}%
\pgfusepath{stroke}%
\end{pgfscope}%
\begin{pgfscope}%
\pgfpathrectangle{\pgfqpoint{0.647939in}{0.492442in}}{\pgfqpoint{4.273799in}{2.331163in}}%
\pgfusepath{clip}%
\pgfsetroundcap%
\pgfsetroundjoin%
\definecolor{currentfill}{rgb}{0.500000,0.500000,0.500000}%
\pgfsetfillcolor{currentfill}%
\pgfsetfillopacity{0.300000}%
\pgfsetlinewidth{0.301125pt}%
\definecolor{currentstroke}{rgb}{0.500000,0.500000,0.500000}%
\pgfsetstrokecolor{currentstroke}%
\pgfsetstrokeopacity{0.300000}%
\pgfsetdash{}{0pt}%
\pgfpathmoveto{\pgfqpoint{0.000000in}{0.000000in}}%
\pgfpathlineto{\pgfqpoint{0.000000in}{0.000000in}}%
\pgfpathclose%
\pgfusepath{stroke,fill}%
\end{pgfscope}%
\begin{pgfscope}%
\pgfpathrectangle{\pgfqpoint{0.647939in}{0.492442in}}{\pgfqpoint{4.273799in}{2.331163in}}%
\pgfusepath{clip}%
\pgfsetroundcap%
\pgfsetroundjoin%
\pgfsetlinewidth{0.301125pt}%
\definecolor{currentstroke}{rgb}{0.500000,0.500000,0.500000}%
\pgfsetstrokecolor{currentstroke}%
\pgfsetstrokeopacity{0.300000}%
\pgfsetdash{}{0pt}%
\pgfpathmoveto{\pgfqpoint{3.222334in}{0.883224in}}%
\pgfusepath{stroke}%
\end{pgfscope}%
\begin{pgfscope}%
\pgfpathrectangle{\pgfqpoint{0.647939in}{0.492442in}}{\pgfqpoint{4.273799in}{2.331163in}}%
\pgfusepath{clip}%
\pgfsetroundcap%
\pgfsetroundjoin%
\definecolor{currentfill}{rgb}{0.500000,0.500000,0.500000}%
\pgfsetfillcolor{currentfill}%
\pgfsetfillopacity{0.300000}%
\pgfsetlinewidth{0.301125pt}%
\definecolor{currentstroke}{rgb}{0.500000,0.500000,0.500000}%
\pgfsetstrokecolor{currentstroke}%
\pgfsetstrokeopacity{0.300000}%
\pgfsetdash{}{0pt}%
\pgfpathmoveto{\pgfqpoint{0.000000in}{0.000000in}}%
\pgfpathlineto{\pgfqpoint{0.000000in}{0.000000in}}%
\pgfpathclose%
\pgfusepath{stroke,fill}%
\end{pgfscope}%
\begin{pgfscope}%
\pgfpathrectangle{\pgfqpoint{0.647939in}{0.492442in}}{\pgfqpoint{4.273799in}{2.331163in}}%
\pgfusepath{clip}%
\pgfsetroundcap%
\pgfsetroundjoin%
\pgfsetlinewidth{0.301125pt}%
\definecolor{currentstroke}{rgb}{0.500000,0.500000,0.500000}%
\pgfsetstrokecolor{currentstroke}%
\pgfsetstrokeopacity{0.300000}%
\pgfsetdash{}{0pt}%
\pgfpathmoveto{\pgfqpoint{3.451027in}{0.831073in}}%
\pgfusepath{stroke}%
\end{pgfscope}%
\begin{pgfscope}%
\pgfpathrectangle{\pgfqpoint{0.647939in}{0.492442in}}{\pgfqpoint{4.273799in}{2.331163in}}%
\pgfusepath{clip}%
\pgfsetroundcap%
\pgfsetroundjoin%
\definecolor{currentfill}{rgb}{0.500000,0.500000,0.500000}%
\pgfsetfillcolor{currentfill}%
\pgfsetfillopacity{0.300000}%
\pgfsetlinewidth{0.301125pt}%
\definecolor{currentstroke}{rgb}{0.500000,0.500000,0.500000}%
\pgfsetstrokecolor{currentstroke}%
\pgfsetstrokeopacity{0.300000}%
\pgfsetdash{}{0pt}%
\pgfpathmoveto{\pgfqpoint{0.000000in}{0.000000in}}%
\pgfpathlineto{\pgfqpoint{0.000000in}{0.000000in}}%
\pgfpathclose%
\pgfusepath{stroke,fill}%
\end{pgfscope}%
\begin{pgfscope}%
\pgfpathrectangle{\pgfqpoint{0.647939in}{0.492442in}}{\pgfqpoint{4.273799in}{2.331163in}}%
\pgfusepath{clip}%
\pgfsetroundcap%
\pgfsetroundjoin%
\pgfsetlinewidth{0.301125pt}%
\definecolor{currentstroke}{rgb}{0.500000,0.500000,0.500000}%
\pgfsetstrokecolor{currentstroke}%
\pgfsetstrokeopacity{0.300000}%
\pgfsetdash{}{0pt}%
\pgfpathmoveto{\pgfqpoint{2.988864in}{1.339789in}}%
\pgfusepath{stroke}%
\end{pgfscope}%
\begin{pgfscope}%
\pgfpathrectangle{\pgfqpoint{0.647939in}{0.492442in}}{\pgfqpoint{4.273799in}{2.331163in}}%
\pgfusepath{clip}%
\pgfsetroundcap%
\pgfsetroundjoin%
\definecolor{currentfill}{rgb}{0.500000,0.500000,0.500000}%
\pgfsetfillcolor{currentfill}%
\pgfsetfillopacity{0.300000}%
\pgfsetlinewidth{0.301125pt}%
\definecolor{currentstroke}{rgb}{0.500000,0.500000,0.500000}%
\pgfsetstrokecolor{currentstroke}%
\pgfsetstrokeopacity{0.300000}%
\pgfsetdash{}{0pt}%
\pgfpathmoveto{\pgfqpoint{0.000000in}{0.000000in}}%
\pgfpathlineto{\pgfqpoint{0.000000in}{0.000000in}}%
\pgfpathclose%
\pgfusepath{stroke,fill}%
\end{pgfscope}%
\begin{pgfscope}%
\pgfpathrectangle{\pgfqpoint{0.647939in}{0.492442in}}{\pgfqpoint{4.273799in}{2.331163in}}%
\pgfusepath{clip}%
\pgfsetroundcap%
\pgfsetroundjoin%
\pgfsetlinewidth{0.301125pt}%
\definecolor{currentstroke}{rgb}{0.500000,0.500000,0.500000}%
\pgfsetstrokecolor{currentstroke}%
\pgfsetstrokeopacity{0.300000}%
\pgfsetdash{}{0pt}%
\pgfpathmoveto{\pgfqpoint{3.570222in}{1.039315in}}%
\pgfusepath{stroke}%
\end{pgfscope}%
\begin{pgfscope}%
\pgfpathrectangle{\pgfqpoint{0.647939in}{0.492442in}}{\pgfqpoint{4.273799in}{2.331163in}}%
\pgfusepath{clip}%
\pgfsetroundcap%
\pgfsetroundjoin%
\definecolor{currentfill}{rgb}{0.500000,0.500000,0.500000}%
\pgfsetfillcolor{currentfill}%
\pgfsetfillopacity{0.300000}%
\pgfsetlinewidth{0.301125pt}%
\definecolor{currentstroke}{rgb}{0.500000,0.500000,0.500000}%
\pgfsetstrokecolor{currentstroke}%
\pgfsetstrokeopacity{0.300000}%
\pgfsetdash{}{0pt}%
\pgfpathmoveto{\pgfqpoint{0.000000in}{0.000000in}}%
\pgfpathlineto{\pgfqpoint{0.000000in}{0.000000in}}%
\pgfpathclose%
\pgfusepath{stroke,fill}%
\end{pgfscope}%
\begin{pgfscope}%
\pgfpathrectangle{\pgfqpoint{0.647939in}{0.492442in}}{\pgfqpoint{4.273799in}{2.331163in}}%
\pgfusepath{clip}%
\pgfsetroundcap%
\pgfsetroundjoin%
\pgfsetlinewidth{0.301125pt}%
\definecolor{currentstroke}{rgb}{0.500000,0.500000,0.500000}%
\pgfsetstrokecolor{currentstroke}%
\pgfsetstrokeopacity{0.300000}%
\pgfsetdash{}{0pt}%
\pgfpathmoveto{\pgfqpoint{3.520992in}{1.248652in}}%
\pgfusepath{stroke}%
\end{pgfscope}%
\begin{pgfscope}%
\pgfpathrectangle{\pgfqpoint{0.647939in}{0.492442in}}{\pgfqpoint{4.273799in}{2.331163in}}%
\pgfusepath{clip}%
\pgfsetroundcap%
\pgfsetroundjoin%
\definecolor{currentfill}{rgb}{0.500000,0.500000,0.500000}%
\pgfsetfillcolor{currentfill}%
\pgfsetfillopacity{0.300000}%
\pgfsetlinewidth{0.301125pt}%
\definecolor{currentstroke}{rgb}{0.500000,0.500000,0.500000}%
\pgfsetstrokecolor{currentstroke}%
\pgfsetstrokeopacity{0.300000}%
\pgfsetdash{}{0pt}%
\pgfpathmoveto{\pgfqpoint{0.000000in}{0.000000in}}%
\pgfpathlineto{\pgfqpoint{0.000000in}{0.000000in}}%
\pgfpathclose%
\pgfusepath{stroke,fill}%
\end{pgfscope}%
\begin{pgfscope}%
\pgfpathrectangle{\pgfqpoint{0.647939in}{0.492442in}}{\pgfqpoint{4.273799in}{2.331163in}}%
\pgfusepath{clip}%
\pgfsetroundcap%
\pgfsetroundjoin%
\pgfsetlinewidth{0.301125pt}%
\definecolor{currentstroke}{rgb}{0.500000,0.500000,0.500000}%
\pgfsetstrokecolor{currentstroke}%
\pgfsetstrokeopacity{0.300000}%
\pgfsetdash{}{0pt}%
\pgfpathmoveto{\pgfqpoint{3.936899in}{1.182863in}}%
\pgfusepath{stroke}%
\end{pgfscope}%
\begin{pgfscope}%
\pgfpathrectangle{\pgfqpoint{0.647939in}{0.492442in}}{\pgfqpoint{4.273799in}{2.331163in}}%
\pgfusepath{clip}%
\pgfsetroundcap%
\pgfsetroundjoin%
\definecolor{currentfill}{rgb}{0.500000,0.500000,0.500000}%
\pgfsetfillcolor{currentfill}%
\pgfsetfillopacity{0.300000}%
\pgfsetlinewidth{0.301125pt}%
\definecolor{currentstroke}{rgb}{0.500000,0.500000,0.500000}%
\pgfsetstrokecolor{currentstroke}%
\pgfsetstrokeopacity{0.300000}%
\pgfsetdash{}{0pt}%
\pgfpathmoveto{\pgfqpoint{0.000000in}{0.000000in}}%
\pgfpathlineto{\pgfqpoint{0.000000in}{0.000000in}}%
\pgfpathclose%
\pgfusepath{stroke,fill}%
\end{pgfscope}%
\begin{pgfscope}%
\pgfpathrectangle{\pgfqpoint{0.647939in}{0.492442in}}{\pgfqpoint{4.273799in}{2.331163in}}%
\pgfusepath{clip}%
\pgfsetroundcap%
\pgfsetroundjoin%
\pgfsetlinewidth{0.301125pt}%
\definecolor{currentstroke}{rgb}{0.500000,0.500000,0.500000}%
\pgfsetstrokecolor{currentstroke}%
\pgfsetstrokeopacity{0.300000}%
\pgfsetdash{}{0pt}%
\pgfpathmoveto{\pgfqpoint{3.805183in}{1.385053in}}%
\pgfusepath{stroke}%
\end{pgfscope}%
\begin{pgfscope}%
\pgfpathrectangle{\pgfqpoint{0.647939in}{0.492442in}}{\pgfqpoint{4.273799in}{2.331163in}}%
\pgfusepath{clip}%
\pgfsetroundcap%
\pgfsetroundjoin%
\definecolor{currentfill}{rgb}{0.500000,0.500000,0.500000}%
\pgfsetfillcolor{currentfill}%
\pgfsetfillopacity{0.300000}%
\pgfsetlinewidth{0.301125pt}%
\definecolor{currentstroke}{rgb}{0.500000,0.500000,0.500000}%
\pgfsetstrokecolor{currentstroke}%
\pgfsetstrokeopacity{0.300000}%
\pgfsetdash{}{0pt}%
\pgfpathmoveto{\pgfqpoint{0.000000in}{0.000000in}}%
\pgfpathlineto{\pgfqpoint{0.000000in}{0.000000in}}%
\pgfpathclose%
\pgfusepath{stroke,fill}%
\end{pgfscope}%
\begin{pgfscope}%
\pgfpathrectangle{\pgfqpoint{0.647939in}{0.492442in}}{\pgfqpoint{4.273799in}{2.331163in}}%
\pgfusepath{clip}%
\pgfsetroundcap%
\pgfsetroundjoin%
\pgfsetlinewidth{0.301125pt}%
\definecolor{currentstroke}{rgb}{0.500000,0.500000,0.500000}%
\pgfsetstrokecolor{currentstroke}%
\pgfsetstrokeopacity{0.300000}%
\pgfsetdash{}{0pt}%
\pgfpathmoveto{\pgfqpoint{4.290489in}{1.344844in}}%
\pgfusepath{stroke}%
\end{pgfscope}%
\begin{pgfscope}%
\pgfpathrectangle{\pgfqpoint{0.647939in}{0.492442in}}{\pgfqpoint{4.273799in}{2.331163in}}%
\pgfusepath{clip}%
\pgfsetroundcap%
\pgfsetroundjoin%
\definecolor{currentfill}{rgb}{0.500000,0.500000,0.500000}%
\pgfsetfillcolor{currentfill}%
\pgfsetfillopacity{0.300000}%
\pgfsetlinewidth{0.301125pt}%
\definecolor{currentstroke}{rgb}{0.500000,0.500000,0.500000}%
\pgfsetstrokecolor{currentstroke}%
\pgfsetstrokeopacity{0.300000}%
\pgfsetdash{}{0pt}%
\pgfpathmoveto{\pgfqpoint{0.000000in}{0.000000in}}%
\pgfpathlineto{\pgfqpoint{0.000000in}{0.000000in}}%
\pgfpathclose%
\pgfusepath{stroke,fill}%
\end{pgfscope}%
\begin{pgfscope}%
\pgfpathrectangle{\pgfqpoint{0.647939in}{0.492442in}}{\pgfqpoint{4.273799in}{2.331163in}}%
\pgfusepath{clip}%
\pgfsetroundcap%
\pgfsetroundjoin%
\pgfsetlinewidth{0.301125pt}%
\definecolor{currentstroke}{rgb}{0.500000,0.500000,0.500000}%
\pgfsetstrokecolor{currentstroke}%
\pgfsetstrokeopacity{0.300000}%
\pgfsetdash{}{0pt}%
\pgfpathmoveto{\pgfqpoint{4.450507in}{1.467911in}}%
\pgfusepath{stroke}%
\end{pgfscope}%
\begin{pgfscope}%
\pgfpathrectangle{\pgfqpoint{0.647939in}{0.492442in}}{\pgfqpoint{4.273799in}{2.331163in}}%
\pgfusepath{clip}%
\pgfsetroundcap%
\pgfsetroundjoin%
\definecolor{currentfill}{rgb}{0.500000,0.500000,0.500000}%
\pgfsetfillcolor{currentfill}%
\pgfsetfillopacity{0.300000}%
\pgfsetlinewidth{0.301125pt}%
\definecolor{currentstroke}{rgb}{0.500000,0.500000,0.500000}%
\pgfsetstrokecolor{currentstroke}%
\pgfsetstrokeopacity{0.300000}%
\pgfsetdash{}{0pt}%
\pgfpathmoveto{\pgfqpoint{0.000000in}{0.000000in}}%
\pgfpathlineto{\pgfqpoint{0.000000in}{0.000000in}}%
\pgfpathclose%
\pgfusepath{stroke,fill}%
\end{pgfscope}%
\begin{pgfscope}%
\pgfpathrectangle{\pgfqpoint{0.647939in}{0.492442in}}{\pgfqpoint{4.273799in}{2.331163in}}%
\pgfusepath{clip}%
\pgfsetroundcap%
\pgfsetroundjoin%
\pgfsetlinewidth{0.301125pt}%
\definecolor{currentstroke}{rgb}{0.500000,0.500000,0.500000}%
\pgfsetstrokecolor{currentstroke}%
\pgfsetstrokeopacity{0.300000}%
\pgfsetdash{}{0pt}%
\pgfpathmoveto{\pgfqpoint{4.513397in}{1.769132in}}%
\pgfusepath{stroke}%
\end{pgfscope}%
\begin{pgfscope}%
\pgfpathrectangle{\pgfqpoint{0.647939in}{0.492442in}}{\pgfqpoint{4.273799in}{2.331163in}}%
\pgfusepath{clip}%
\pgfsetroundcap%
\pgfsetroundjoin%
\definecolor{currentfill}{rgb}{0.500000,0.500000,0.500000}%
\pgfsetfillcolor{currentfill}%
\pgfsetfillopacity{0.300000}%
\pgfsetlinewidth{0.301125pt}%
\definecolor{currentstroke}{rgb}{0.500000,0.500000,0.500000}%
\pgfsetstrokecolor{currentstroke}%
\pgfsetstrokeopacity{0.300000}%
\pgfsetdash{}{0pt}%
\pgfpathmoveto{\pgfqpoint{0.000000in}{0.000000in}}%
\pgfpathlineto{\pgfqpoint{0.000000in}{0.000000in}}%
\pgfpathclose%
\pgfusepath{stroke,fill}%
\end{pgfscope}%
\begin{pgfscope}%
\pgfpathrectangle{\pgfqpoint{0.647939in}{0.492442in}}{\pgfqpoint{4.273799in}{2.331163in}}%
\pgfusepath{clip}%
\pgfsetroundcap%
\pgfsetroundjoin%
\pgfsetlinewidth{0.301125pt}%
\definecolor{currentstroke}{rgb}{0.500000,0.500000,0.500000}%
\pgfsetstrokecolor{currentstroke}%
\pgfsetstrokeopacity{0.300000}%
\pgfsetdash{}{0pt}%
\pgfpathmoveto{\pgfqpoint{4.723711in}{1.633926in}}%
\pgfusepath{stroke}%
\end{pgfscope}%
\begin{pgfscope}%
\pgfpathrectangle{\pgfqpoint{0.647939in}{0.492442in}}{\pgfqpoint{4.273799in}{2.331163in}}%
\pgfusepath{clip}%
\pgfsetroundcap%
\pgfsetroundjoin%
\definecolor{currentfill}{rgb}{0.500000,0.500000,0.500000}%
\pgfsetfillcolor{currentfill}%
\pgfsetfillopacity{0.300000}%
\pgfsetlinewidth{0.301125pt}%
\definecolor{currentstroke}{rgb}{0.500000,0.500000,0.500000}%
\pgfsetstrokecolor{currentstroke}%
\pgfsetstrokeopacity{0.300000}%
\pgfsetdash{}{0pt}%
\pgfpathmoveto{\pgfqpoint{0.000000in}{0.000000in}}%
\pgfpathlineto{\pgfqpoint{0.000000in}{0.000000in}}%
\pgfpathclose%
\pgfusepath{stroke,fill}%
\end{pgfscope}%
\begin{pgfscope}%
\pgfpathrectangle{\pgfqpoint{0.647939in}{0.492442in}}{\pgfqpoint{4.273799in}{2.331163in}}%
\pgfusepath{clip}%
\pgfsetroundcap%
\pgfsetroundjoin%
\pgfsetlinewidth{0.301125pt}%
\definecolor{currentstroke}{rgb}{0.500000,0.500000,0.500000}%
\pgfsetstrokecolor{currentstroke}%
\pgfsetstrokeopacity{0.300000}%
\pgfsetdash{}{0pt}%
\pgfpathmoveto{\pgfqpoint{4.810348in}{1.905553in}}%
\pgfusepath{stroke}%
\end{pgfscope}%
\begin{pgfscope}%
\pgfpathrectangle{\pgfqpoint{0.647939in}{0.492442in}}{\pgfqpoint{4.273799in}{2.331163in}}%
\pgfusepath{clip}%
\pgfsetroundcap%
\pgfsetroundjoin%
\definecolor{currentfill}{rgb}{0.500000,0.500000,0.500000}%
\pgfsetfillcolor{currentfill}%
\pgfsetfillopacity{0.300000}%
\pgfsetlinewidth{0.301125pt}%
\definecolor{currentstroke}{rgb}{0.500000,0.500000,0.500000}%
\pgfsetstrokecolor{currentstroke}%
\pgfsetstrokeopacity{0.300000}%
\pgfsetdash{}{0pt}%
\pgfpathmoveto{\pgfqpoint{0.000000in}{0.000000in}}%
\pgfpathlineto{\pgfqpoint{0.000000in}{0.000000in}}%
\pgfpathclose%
\pgfusepath{stroke,fill}%
\end{pgfscope}%
\begin{pgfscope}%
\pgfpathrectangle{\pgfqpoint{0.647939in}{0.492442in}}{\pgfqpoint{4.273799in}{2.331163in}}%
\pgfusepath{clip}%
\pgfsetroundcap%
\pgfsetroundjoin%
\pgfsetlinewidth{0.301125pt}%
\definecolor{currentstroke}{rgb}{0.500000,0.500000,0.500000}%
\pgfsetstrokecolor{currentstroke}%
\pgfsetstrokeopacity{0.300000}%
\pgfsetdash{}{0pt}%
\pgfpathmoveto{\pgfqpoint{4.902958in}{1.602178in}}%
\pgfusepath{stroke}%
\end{pgfscope}%
\begin{pgfscope}%
\pgfpathrectangle{\pgfqpoint{0.647939in}{0.492442in}}{\pgfqpoint{4.273799in}{2.331163in}}%
\pgfusepath{clip}%
\pgfsetroundcap%
\pgfsetroundjoin%
\definecolor{currentfill}{rgb}{0.500000,0.500000,0.500000}%
\pgfsetfillcolor{currentfill}%
\pgfsetfillopacity{0.300000}%
\pgfsetlinewidth{0.301125pt}%
\definecolor{currentstroke}{rgb}{0.500000,0.500000,0.500000}%
\pgfsetstrokecolor{currentstroke}%
\pgfsetstrokeopacity{0.300000}%
\pgfsetdash{}{0pt}%
\pgfpathmoveto{\pgfqpoint{0.000000in}{0.000000in}}%
\pgfpathlineto{\pgfqpoint{0.000000in}{0.000000in}}%
\pgfpathclose%
\pgfusepath{stroke,fill}%
\end{pgfscope}%
\begin{pgfscope}%
\pgfpathrectangle{\pgfqpoint{0.647939in}{0.492442in}}{\pgfqpoint{4.273799in}{2.331163in}}%
\pgfusepath{clip}%
\pgfsetroundcap%
\pgfsetroundjoin%
\pgfsetlinewidth{0.301125pt}%
\definecolor{currentstroke}{rgb}{0.500000,0.500000,0.500000}%
\pgfsetstrokecolor{currentstroke}%
\pgfsetstrokeopacity{0.300000}%
\pgfsetdash{}{0pt}%
\pgfpathmoveto{\pgfqpoint{4.917199in}{1.973521in}}%
\pgfusepath{stroke}%
\end{pgfscope}%
\begin{pgfscope}%
\pgfpathrectangle{\pgfqpoint{0.647939in}{0.492442in}}{\pgfqpoint{4.273799in}{2.331163in}}%
\pgfusepath{clip}%
\pgfsetroundcap%
\pgfsetroundjoin%
\definecolor{currentfill}{rgb}{0.500000,0.500000,0.500000}%
\pgfsetfillcolor{currentfill}%
\pgfsetfillopacity{0.300000}%
\pgfsetlinewidth{0.301125pt}%
\definecolor{currentstroke}{rgb}{0.500000,0.500000,0.500000}%
\pgfsetstrokecolor{currentstroke}%
\pgfsetstrokeopacity{0.300000}%
\pgfsetdash{}{0pt}%
\pgfpathmoveto{\pgfqpoint{0.000000in}{0.000000in}}%
\pgfpathlineto{\pgfqpoint{0.000000in}{0.000000in}}%
\pgfpathclose%
\pgfusepath{stroke,fill}%
\end{pgfscope}%
\begin{pgfscope}%
\pgfpathrectangle{\pgfqpoint{0.647939in}{0.492442in}}{\pgfqpoint{4.273799in}{2.331163in}}%
\pgfusepath{clip}%
\pgfsetroundcap%
\pgfsetroundjoin%
\pgfsetlinewidth{0.301125pt}%
\definecolor{currentstroke}{rgb}{0.500000,0.500000,0.500000}%
\pgfsetstrokecolor{currentstroke}%
\pgfsetstrokeopacity{0.300000}%
\pgfsetdash{}{0pt}%
\pgfpathmoveto{\pgfqpoint{4.852746in}{2.566882in}}%
\pgfusepath{stroke}%
\end{pgfscope}%
\begin{pgfscope}%
\pgfpathrectangle{\pgfqpoint{0.647939in}{0.492442in}}{\pgfqpoint{4.273799in}{2.331163in}}%
\pgfusepath{clip}%
\pgfsetroundcap%
\pgfsetroundjoin%
\definecolor{currentfill}{rgb}{0.500000,0.500000,0.500000}%
\pgfsetfillcolor{currentfill}%
\pgfsetfillopacity{0.300000}%
\pgfsetlinewidth{0.301125pt}%
\definecolor{currentstroke}{rgb}{0.500000,0.500000,0.500000}%
\pgfsetstrokecolor{currentstroke}%
\pgfsetstrokeopacity{0.300000}%
\pgfsetdash{}{0pt}%
\pgfpathmoveto{\pgfqpoint{0.000000in}{0.000000in}}%
\pgfpathlineto{\pgfqpoint{0.000000in}{0.000000in}}%
\pgfpathclose%
\pgfusepath{stroke,fill}%
\end{pgfscope}%
\begin{pgfscope}%
\pgfpathrectangle{\pgfqpoint{0.647939in}{0.492442in}}{\pgfqpoint{4.273799in}{2.331163in}}%
\pgfusepath{clip}%
\pgfsetroundcap%
\pgfsetroundjoin%
\pgfsetlinewidth{0.301125pt}%
\definecolor{currentstroke}{rgb}{0.500000,0.500000,0.500000}%
\pgfsetstrokecolor{currentstroke}%
\pgfsetstrokeopacity{0.300000}%
\pgfsetdash{}{0pt}%
\pgfpathmoveto{\pgfqpoint{4.514626in}{2.668316in}}%
\pgfusepath{stroke}%
\end{pgfscope}%
\begin{pgfscope}%
\pgfpathrectangle{\pgfqpoint{0.647939in}{0.492442in}}{\pgfqpoint{4.273799in}{2.331163in}}%
\pgfusepath{clip}%
\pgfsetroundcap%
\pgfsetroundjoin%
\definecolor{currentfill}{rgb}{0.500000,0.500000,0.500000}%
\pgfsetfillcolor{currentfill}%
\pgfsetfillopacity{0.300000}%
\pgfsetlinewidth{0.301125pt}%
\definecolor{currentstroke}{rgb}{0.500000,0.500000,0.500000}%
\pgfsetstrokecolor{currentstroke}%
\pgfsetstrokeopacity{0.300000}%
\pgfsetdash{}{0pt}%
\pgfpathmoveto{\pgfqpoint{0.000000in}{0.000000in}}%
\pgfpathlineto{\pgfqpoint{0.000000in}{0.000000in}}%
\pgfpathclose%
\pgfusepath{stroke,fill}%
\end{pgfscope}%
\begin{pgfscope}%
\pgfpathrectangle{\pgfqpoint{0.647939in}{0.492442in}}{\pgfqpoint{4.273799in}{2.331163in}}%
\pgfusepath{clip}%
\pgfsetroundcap%
\pgfsetroundjoin%
\pgfsetlinewidth{0.301125pt}%
\definecolor{currentstroke}{rgb}{0.500000,0.500000,0.500000}%
\pgfsetstrokecolor{currentstroke}%
\pgfsetstrokeopacity{0.300000}%
\pgfsetdash{}{0pt}%
\pgfpathmoveto{\pgfqpoint{4.415670in}{2.552296in}}%
\pgfusepath{stroke}%
\end{pgfscope}%
\begin{pgfscope}%
\pgfpathrectangle{\pgfqpoint{0.647939in}{0.492442in}}{\pgfqpoint{4.273799in}{2.331163in}}%
\pgfusepath{clip}%
\pgfsetroundcap%
\pgfsetroundjoin%
\definecolor{currentfill}{rgb}{0.500000,0.500000,0.500000}%
\pgfsetfillcolor{currentfill}%
\pgfsetfillopacity{0.300000}%
\pgfsetlinewidth{0.301125pt}%
\definecolor{currentstroke}{rgb}{0.500000,0.500000,0.500000}%
\pgfsetstrokecolor{currentstroke}%
\pgfsetstrokeopacity{0.300000}%
\pgfsetdash{}{0pt}%
\pgfpathmoveto{\pgfqpoint{0.000000in}{0.000000in}}%
\pgfpathlineto{\pgfqpoint{0.000000in}{0.000000in}}%
\pgfpathclose%
\pgfusepath{stroke,fill}%
\end{pgfscope}%
\begin{pgfscope}%
\pgfpathrectangle{\pgfqpoint{0.647939in}{0.492442in}}{\pgfqpoint{4.273799in}{2.331163in}}%
\pgfusepath{clip}%
\pgfsetroundcap%
\pgfsetroundjoin%
\pgfsetlinewidth{0.301125pt}%
\definecolor{currentstroke}{rgb}{0.500000,0.500000,0.500000}%
\pgfsetstrokecolor{currentstroke}%
\pgfsetstrokeopacity{0.300000}%
\pgfsetdash{}{0pt}%
\pgfpathmoveto{\pgfqpoint{4.284546in}{2.516519in}}%
\pgfusepath{stroke}%
\end{pgfscope}%
\begin{pgfscope}%
\pgfpathrectangle{\pgfqpoint{0.647939in}{0.492442in}}{\pgfqpoint{4.273799in}{2.331163in}}%
\pgfusepath{clip}%
\pgfsetroundcap%
\pgfsetroundjoin%
\definecolor{currentfill}{rgb}{0.500000,0.500000,0.500000}%
\pgfsetfillcolor{currentfill}%
\pgfsetfillopacity{0.300000}%
\pgfsetlinewidth{0.301125pt}%
\definecolor{currentstroke}{rgb}{0.500000,0.500000,0.500000}%
\pgfsetstrokecolor{currentstroke}%
\pgfsetstrokeopacity{0.300000}%
\pgfsetdash{}{0pt}%
\pgfpathmoveto{\pgfqpoint{0.000000in}{0.000000in}}%
\pgfpathlineto{\pgfqpoint{0.000000in}{0.000000in}}%
\pgfpathclose%
\pgfusepath{stroke,fill}%
\end{pgfscope}%
\begin{pgfscope}%
\pgfpathrectangle{\pgfqpoint{0.647939in}{0.492442in}}{\pgfqpoint{4.273799in}{2.331163in}}%
\pgfusepath{clip}%
\pgfsetroundcap%
\pgfsetroundjoin%
\pgfsetlinewidth{0.301125pt}%
\definecolor{currentstroke}{rgb}{0.500000,0.500000,0.500000}%
\pgfsetstrokecolor{currentstroke}%
\pgfsetstrokeopacity{0.300000}%
\pgfsetdash{}{0pt}%
\pgfpathmoveto{\pgfqpoint{4.159472in}{2.508374in}}%
\pgfusepath{stroke}%
\end{pgfscope}%
\begin{pgfscope}%
\pgfpathrectangle{\pgfqpoint{0.647939in}{0.492442in}}{\pgfqpoint{4.273799in}{2.331163in}}%
\pgfusepath{clip}%
\pgfsetroundcap%
\pgfsetroundjoin%
\definecolor{currentfill}{rgb}{0.500000,0.500000,0.500000}%
\pgfsetfillcolor{currentfill}%
\pgfsetfillopacity{0.300000}%
\pgfsetlinewidth{0.301125pt}%
\definecolor{currentstroke}{rgb}{0.500000,0.500000,0.500000}%
\pgfsetstrokecolor{currentstroke}%
\pgfsetstrokeopacity{0.300000}%
\pgfsetdash{}{0pt}%
\pgfpathmoveto{\pgfqpoint{0.000000in}{0.000000in}}%
\pgfpathlineto{\pgfqpoint{0.000000in}{0.000000in}}%
\pgfpathclose%
\pgfusepath{stroke,fill}%
\end{pgfscope}%
\begin{pgfscope}%
\pgfpathrectangle{\pgfqpoint{0.647939in}{0.492442in}}{\pgfqpoint{4.273799in}{2.331163in}}%
\pgfusepath{clip}%
\pgfsetroundcap%
\pgfsetroundjoin%
\pgfsetlinewidth{0.301125pt}%
\definecolor{currentstroke}{rgb}{0.500000,0.500000,0.500000}%
\pgfsetstrokecolor{currentstroke}%
\pgfsetstrokeopacity{0.300000}%
\pgfsetdash{}{0pt}%
\pgfpathmoveto{\pgfqpoint{4.105273in}{2.406021in}}%
\pgfusepath{stroke}%
\end{pgfscope}%
\begin{pgfscope}%
\pgfpathrectangle{\pgfqpoint{0.647939in}{0.492442in}}{\pgfqpoint{4.273799in}{2.331163in}}%
\pgfusepath{clip}%
\pgfsetroundcap%
\pgfsetroundjoin%
\definecolor{currentfill}{rgb}{0.500000,0.500000,0.500000}%
\pgfsetfillcolor{currentfill}%
\pgfsetfillopacity{0.300000}%
\pgfsetlinewidth{0.301125pt}%
\definecolor{currentstroke}{rgb}{0.500000,0.500000,0.500000}%
\pgfsetstrokecolor{currentstroke}%
\pgfsetstrokeopacity{0.300000}%
\pgfsetdash{}{0pt}%
\pgfpathmoveto{\pgfqpoint{0.000000in}{0.000000in}}%
\pgfpathlineto{\pgfqpoint{0.000000in}{0.000000in}}%
\pgfpathclose%
\pgfusepath{stroke,fill}%
\end{pgfscope}%
\begin{pgfscope}%
\pgfpathrectangle{\pgfqpoint{0.647939in}{0.492442in}}{\pgfqpoint{4.273799in}{2.331163in}}%
\pgfusepath{clip}%
\pgfsetroundcap%
\pgfsetroundjoin%
\pgfsetlinewidth{0.301125pt}%
\definecolor{currentstroke}{rgb}{0.500000,0.500000,0.500000}%
\pgfsetstrokecolor{currentstroke}%
\pgfsetstrokeopacity{0.300000}%
\pgfsetdash{}{0pt}%
\pgfpathmoveto{\pgfqpoint{4.054508in}{2.277731in}}%
\pgfusepath{stroke}%
\end{pgfscope}%
\begin{pgfscope}%
\pgfpathrectangle{\pgfqpoint{0.647939in}{0.492442in}}{\pgfqpoint{4.273799in}{2.331163in}}%
\pgfusepath{clip}%
\pgfsetroundcap%
\pgfsetroundjoin%
\definecolor{currentfill}{rgb}{0.500000,0.500000,0.500000}%
\pgfsetfillcolor{currentfill}%
\pgfsetfillopacity{0.300000}%
\pgfsetlinewidth{0.301125pt}%
\definecolor{currentstroke}{rgb}{0.500000,0.500000,0.500000}%
\pgfsetstrokecolor{currentstroke}%
\pgfsetstrokeopacity{0.300000}%
\pgfsetdash{}{0pt}%
\pgfpathmoveto{\pgfqpoint{0.000000in}{0.000000in}}%
\pgfpathlineto{\pgfqpoint{0.000000in}{0.000000in}}%
\pgfpathclose%
\pgfusepath{stroke,fill}%
\end{pgfscope}%
\begin{pgfscope}%
\pgfpathrectangle{\pgfqpoint{0.647939in}{0.492442in}}{\pgfqpoint{4.273799in}{2.331163in}}%
\pgfusepath{clip}%
\pgfsetroundcap%
\pgfsetroundjoin%
\pgfsetlinewidth{0.301125pt}%
\definecolor{currentstroke}{rgb}{0.500000,0.500000,0.500000}%
\pgfsetstrokecolor{currentstroke}%
\pgfsetstrokeopacity{0.300000}%
\pgfsetdash{}{0pt}%
\pgfpathmoveto{\pgfqpoint{3.927239in}{2.326923in}}%
\pgfusepath{stroke}%
\end{pgfscope}%
\begin{pgfscope}%
\pgfpathrectangle{\pgfqpoint{0.647939in}{0.492442in}}{\pgfqpoint{4.273799in}{2.331163in}}%
\pgfusepath{clip}%
\pgfsetroundcap%
\pgfsetroundjoin%
\definecolor{currentfill}{rgb}{0.500000,0.500000,0.500000}%
\pgfsetfillcolor{currentfill}%
\pgfsetfillopacity{0.300000}%
\pgfsetlinewidth{0.301125pt}%
\definecolor{currentstroke}{rgb}{0.500000,0.500000,0.500000}%
\pgfsetstrokecolor{currentstroke}%
\pgfsetstrokeopacity{0.300000}%
\pgfsetdash{}{0pt}%
\pgfpathmoveto{\pgfqpoint{0.000000in}{0.000000in}}%
\pgfpathlineto{\pgfqpoint{0.000000in}{0.000000in}}%
\pgfpathclose%
\pgfusepath{stroke,fill}%
\end{pgfscope}%
\begin{pgfscope}%
\pgfpathrectangle{\pgfqpoint{0.647939in}{0.492442in}}{\pgfqpoint{4.273799in}{2.331163in}}%
\pgfusepath{clip}%
\pgfsetroundcap%
\pgfsetroundjoin%
\pgfsetlinewidth{0.301125pt}%
\definecolor{currentstroke}{rgb}{0.500000,0.500000,0.500000}%
\pgfsetstrokecolor{currentstroke}%
\pgfsetstrokeopacity{0.300000}%
\pgfsetdash{}{0pt}%
\pgfpathmoveto{\pgfqpoint{3.850732in}{1.842721in}}%
\pgfusepath{stroke}%
\end{pgfscope}%
\begin{pgfscope}%
\pgfpathrectangle{\pgfqpoint{0.647939in}{0.492442in}}{\pgfqpoint{4.273799in}{2.331163in}}%
\pgfusepath{clip}%
\pgfsetroundcap%
\pgfsetroundjoin%
\definecolor{currentfill}{rgb}{0.500000,0.500000,0.500000}%
\pgfsetfillcolor{currentfill}%
\pgfsetfillopacity{0.300000}%
\pgfsetlinewidth{0.301125pt}%
\definecolor{currentstroke}{rgb}{0.500000,0.500000,0.500000}%
\pgfsetstrokecolor{currentstroke}%
\pgfsetstrokeopacity{0.300000}%
\pgfsetdash{}{0pt}%
\pgfpathmoveto{\pgfqpoint{0.000000in}{0.000000in}}%
\pgfpathlineto{\pgfqpoint{0.000000in}{0.000000in}}%
\pgfpathclose%
\pgfusepath{stroke,fill}%
\end{pgfscope}%
\begin{pgfscope}%
\pgfpathrectangle{\pgfqpoint{0.647939in}{0.492442in}}{\pgfqpoint{4.273799in}{2.331163in}}%
\pgfusepath{clip}%
\pgfsetroundcap%
\pgfsetroundjoin%
\pgfsetlinewidth{0.301125pt}%
\definecolor{currentstroke}{rgb}{0.500000,0.500000,0.500000}%
\pgfsetstrokecolor{currentstroke}%
\pgfsetstrokeopacity{0.300000}%
\pgfsetdash{}{0pt}%
\pgfpathmoveto{\pgfqpoint{3.656923in}{2.503961in}}%
\pgfusepath{stroke}%
\end{pgfscope}%
\begin{pgfscope}%
\pgfpathrectangle{\pgfqpoint{0.647939in}{0.492442in}}{\pgfqpoint{4.273799in}{2.331163in}}%
\pgfusepath{clip}%
\pgfsetroundcap%
\pgfsetroundjoin%
\definecolor{currentfill}{rgb}{0.500000,0.500000,0.500000}%
\pgfsetfillcolor{currentfill}%
\pgfsetfillopacity{0.300000}%
\pgfsetlinewidth{0.301125pt}%
\definecolor{currentstroke}{rgb}{0.500000,0.500000,0.500000}%
\pgfsetstrokecolor{currentstroke}%
\pgfsetstrokeopacity{0.300000}%
\pgfsetdash{}{0pt}%
\pgfpathmoveto{\pgfqpoint{0.000000in}{0.000000in}}%
\pgfpathlineto{\pgfqpoint{0.000000in}{0.000000in}}%
\pgfpathclose%
\pgfusepath{stroke,fill}%
\end{pgfscope}%
\begin{pgfscope}%
\pgfpathrectangle{\pgfqpoint{0.647939in}{0.492442in}}{\pgfqpoint{4.273799in}{2.331163in}}%
\pgfusepath{clip}%
\pgfsetroundcap%
\pgfsetroundjoin%
\pgfsetlinewidth{0.301125pt}%
\definecolor{currentstroke}{rgb}{0.500000,0.500000,0.500000}%
\pgfsetstrokecolor{currentstroke}%
\pgfsetstrokeopacity{0.300000}%
\pgfsetdash{}{0pt}%
\pgfpathmoveto{\pgfqpoint{3.567838in}{2.505625in}}%
\pgfusepath{stroke}%
\end{pgfscope}%
\begin{pgfscope}%
\pgfpathrectangle{\pgfqpoint{0.647939in}{0.492442in}}{\pgfqpoint{4.273799in}{2.331163in}}%
\pgfusepath{clip}%
\pgfsetroundcap%
\pgfsetroundjoin%
\definecolor{currentfill}{rgb}{0.500000,0.500000,0.500000}%
\pgfsetfillcolor{currentfill}%
\pgfsetfillopacity{0.300000}%
\pgfsetlinewidth{0.301125pt}%
\definecolor{currentstroke}{rgb}{0.500000,0.500000,0.500000}%
\pgfsetstrokecolor{currentstroke}%
\pgfsetstrokeopacity{0.300000}%
\pgfsetdash{}{0pt}%
\pgfpathmoveto{\pgfqpoint{0.000000in}{0.000000in}}%
\pgfpathlineto{\pgfqpoint{0.000000in}{0.000000in}}%
\pgfpathclose%
\pgfusepath{stroke,fill}%
\end{pgfscope}%
\begin{pgfscope}%
\pgfpathrectangle{\pgfqpoint{0.647939in}{0.492442in}}{\pgfqpoint{4.273799in}{2.331163in}}%
\pgfusepath{clip}%
\pgfsetroundcap%
\pgfsetroundjoin%
\pgfsetlinewidth{0.301125pt}%
\definecolor{currentstroke}{rgb}{0.500000,0.500000,0.500000}%
\pgfsetstrokecolor{currentstroke}%
\pgfsetstrokeopacity{0.300000}%
\pgfsetdash{}{0pt}%
\pgfpathmoveto{\pgfqpoint{3.391170in}{2.656082in}}%
\pgfusepath{stroke}%
\end{pgfscope}%
\begin{pgfscope}%
\pgfpathrectangle{\pgfqpoint{0.647939in}{0.492442in}}{\pgfqpoint{4.273799in}{2.331163in}}%
\pgfusepath{clip}%
\pgfsetroundcap%
\pgfsetroundjoin%
\definecolor{currentfill}{rgb}{0.500000,0.500000,0.500000}%
\pgfsetfillcolor{currentfill}%
\pgfsetfillopacity{0.300000}%
\pgfsetlinewidth{0.301125pt}%
\definecolor{currentstroke}{rgb}{0.500000,0.500000,0.500000}%
\pgfsetstrokecolor{currentstroke}%
\pgfsetstrokeopacity{0.300000}%
\pgfsetdash{}{0pt}%
\pgfpathmoveto{\pgfqpoint{0.000000in}{0.000000in}}%
\pgfpathlineto{\pgfqpoint{0.000000in}{0.000000in}}%
\pgfpathclose%
\pgfusepath{stroke,fill}%
\end{pgfscope}%
\begin{pgfscope}%
\pgfpathrectangle{\pgfqpoint{0.647939in}{0.492442in}}{\pgfqpoint{4.273799in}{2.331163in}}%
\pgfusepath{clip}%
\pgfsetroundcap%
\pgfsetroundjoin%
\pgfsetlinewidth{0.301125pt}%
\definecolor{currentstroke}{rgb}{0.500000,0.500000,0.500000}%
\pgfsetstrokecolor{currentstroke}%
\pgfsetstrokeopacity{0.300000}%
\pgfsetdash{}{0pt}%
\pgfpathmoveto{\pgfqpoint{3.462636in}{2.240792in}}%
\pgfusepath{stroke}%
\end{pgfscope}%
\begin{pgfscope}%
\pgfpathrectangle{\pgfqpoint{0.647939in}{0.492442in}}{\pgfqpoint{4.273799in}{2.331163in}}%
\pgfusepath{clip}%
\pgfsetroundcap%
\pgfsetroundjoin%
\definecolor{currentfill}{rgb}{0.500000,0.500000,0.500000}%
\pgfsetfillcolor{currentfill}%
\pgfsetfillopacity{0.300000}%
\pgfsetlinewidth{0.301125pt}%
\definecolor{currentstroke}{rgb}{0.500000,0.500000,0.500000}%
\pgfsetstrokecolor{currentstroke}%
\pgfsetstrokeopacity{0.300000}%
\pgfsetdash{}{0pt}%
\pgfpathmoveto{\pgfqpoint{0.000000in}{0.000000in}}%
\pgfpathlineto{\pgfqpoint{0.000000in}{0.000000in}}%
\pgfpathclose%
\pgfusepath{stroke,fill}%
\end{pgfscope}%
\begin{pgfscope}%
\pgfpathrectangle{\pgfqpoint{0.647939in}{0.492442in}}{\pgfqpoint{4.273799in}{2.331163in}}%
\pgfusepath{clip}%
\pgfsetroundcap%
\pgfsetroundjoin%
\pgfsetlinewidth{0.301125pt}%
\definecolor{currentstroke}{rgb}{0.500000,0.500000,0.500000}%
\pgfsetstrokecolor{currentstroke}%
\pgfsetstrokeopacity{0.300000}%
\pgfsetdash{}{0pt}%
\pgfpathmoveto{\pgfqpoint{3.274721in}{2.383200in}}%
\pgfusepath{stroke}%
\end{pgfscope}%
\begin{pgfscope}%
\pgfpathrectangle{\pgfqpoint{0.647939in}{0.492442in}}{\pgfqpoint{4.273799in}{2.331163in}}%
\pgfusepath{clip}%
\pgfsetroundcap%
\pgfsetroundjoin%
\definecolor{currentfill}{rgb}{0.500000,0.500000,0.500000}%
\pgfsetfillcolor{currentfill}%
\pgfsetfillopacity{0.300000}%
\pgfsetlinewidth{0.301125pt}%
\definecolor{currentstroke}{rgb}{0.500000,0.500000,0.500000}%
\pgfsetstrokecolor{currentstroke}%
\pgfsetstrokeopacity{0.300000}%
\pgfsetdash{}{0pt}%
\pgfpathmoveto{\pgfqpoint{0.000000in}{0.000000in}}%
\pgfpathlineto{\pgfqpoint{0.000000in}{0.000000in}}%
\pgfpathclose%
\pgfusepath{stroke,fill}%
\end{pgfscope}%
\begin{pgfscope}%
\pgfpathrectangle{\pgfqpoint{0.647939in}{0.492442in}}{\pgfqpoint{4.273799in}{2.331163in}}%
\pgfusepath{clip}%
\pgfsetroundcap%
\pgfsetroundjoin%
\pgfsetlinewidth{0.301125pt}%
\definecolor{currentstroke}{rgb}{0.500000,0.500000,0.500000}%
\pgfsetstrokecolor{currentstroke}%
\pgfsetstrokeopacity{0.300000}%
\pgfsetdash{}{0pt}%
\pgfpathmoveto{\pgfqpoint{2.991872in}{2.597630in}}%
\pgfusepath{stroke}%
\end{pgfscope}%
\begin{pgfscope}%
\pgfpathrectangle{\pgfqpoint{0.647939in}{0.492442in}}{\pgfqpoint{4.273799in}{2.331163in}}%
\pgfusepath{clip}%
\pgfsetroundcap%
\pgfsetroundjoin%
\definecolor{currentfill}{rgb}{0.500000,0.500000,0.500000}%
\pgfsetfillcolor{currentfill}%
\pgfsetfillopacity{0.300000}%
\pgfsetlinewidth{0.301125pt}%
\definecolor{currentstroke}{rgb}{0.500000,0.500000,0.500000}%
\pgfsetstrokecolor{currentstroke}%
\pgfsetstrokeopacity{0.300000}%
\pgfsetdash{}{0pt}%
\pgfpathmoveto{\pgfqpoint{0.000000in}{0.000000in}}%
\pgfpathlineto{\pgfqpoint{0.000000in}{0.000000in}}%
\pgfpathclose%
\pgfusepath{stroke,fill}%
\end{pgfscope}%
\begin{pgfscope}%
\pgfpathrectangle{\pgfqpoint{0.647939in}{0.492442in}}{\pgfqpoint{4.273799in}{2.331163in}}%
\pgfusepath{clip}%
\pgfsetroundcap%
\pgfsetroundjoin%
\pgfsetlinewidth{0.301125pt}%
\definecolor{currentstroke}{rgb}{0.500000,0.500000,0.500000}%
\pgfsetstrokecolor{currentstroke}%
\pgfsetstrokeopacity{0.300000}%
\pgfsetdash{}{0pt}%
\pgfpathmoveto{\pgfqpoint{2.664125in}{2.744596in}}%
\pgfusepath{stroke}%
\end{pgfscope}%
\begin{pgfscope}%
\pgfpathrectangle{\pgfqpoint{0.647939in}{0.492442in}}{\pgfqpoint{4.273799in}{2.331163in}}%
\pgfusepath{clip}%
\pgfsetroundcap%
\pgfsetroundjoin%
\definecolor{currentfill}{rgb}{0.500000,0.500000,0.500000}%
\pgfsetfillcolor{currentfill}%
\pgfsetfillopacity{0.300000}%
\pgfsetlinewidth{0.301125pt}%
\definecolor{currentstroke}{rgb}{0.500000,0.500000,0.500000}%
\pgfsetstrokecolor{currentstroke}%
\pgfsetstrokeopacity{0.300000}%
\pgfsetdash{}{0pt}%
\pgfpathmoveto{\pgfqpoint{0.000000in}{0.000000in}}%
\pgfpathlineto{\pgfqpoint{0.000000in}{0.000000in}}%
\pgfpathclose%
\pgfusepath{stroke,fill}%
\end{pgfscope}%
\begin{pgfscope}%
\pgfpathrectangle{\pgfqpoint{0.647939in}{0.492442in}}{\pgfqpoint{4.273799in}{2.331163in}}%
\pgfusepath{clip}%
\pgfsetroundcap%
\pgfsetroundjoin%
\pgfsetlinewidth{0.301125pt}%
\definecolor{currentstroke}{rgb}{0.500000,0.500000,0.500000}%
\pgfsetstrokecolor{currentstroke}%
\pgfsetstrokeopacity{0.300000}%
\pgfsetdash{}{0pt}%
\pgfpathmoveto{\pgfqpoint{2.488891in}{2.761537in}}%
\pgfusepath{stroke}%
\end{pgfscope}%
\begin{pgfscope}%
\pgfpathrectangle{\pgfqpoint{0.647939in}{0.492442in}}{\pgfqpoint{4.273799in}{2.331163in}}%
\pgfusepath{clip}%
\pgfsetroundcap%
\pgfsetroundjoin%
\definecolor{currentfill}{rgb}{0.500000,0.500000,0.500000}%
\pgfsetfillcolor{currentfill}%
\pgfsetfillopacity{0.300000}%
\pgfsetlinewidth{0.301125pt}%
\definecolor{currentstroke}{rgb}{0.500000,0.500000,0.500000}%
\pgfsetstrokecolor{currentstroke}%
\pgfsetstrokeopacity{0.300000}%
\pgfsetdash{}{0pt}%
\pgfpathmoveto{\pgfqpoint{0.000000in}{0.000000in}}%
\pgfpathlineto{\pgfqpoint{0.000000in}{0.000000in}}%
\pgfpathclose%
\pgfusepath{stroke,fill}%
\end{pgfscope}%
\begin{pgfscope}%
\pgfpathrectangle{\pgfqpoint{0.647939in}{0.492442in}}{\pgfqpoint{4.273799in}{2.331163in}}%
\pgfusepath{clip}%
\pgfsetroundcap%
\pgfsetroundjoin%
\pgfsetlinewidth{0.301125pt}%
\definecolor{currentstroke}{rgb}{0.500000,0.500000,0.500000}%
\pgfsetstrokecolor{currentstroke}%
\pgfsetstrokeopacity{0.300000}%
\pgfsetdash{}{0pt}%
\pgfpathmoveto{\pgfqpoint{2.624827in}{2.649248in}}%
\pgfusepath{stroke}%
\end{pgfscope}%
\begin{pgfscope}%
\pgfpathrectangle{\pgfqpoint{0.647939in}{0.492442in}}{\pgfqpoint{4.273799in}{2.331163in}}%
\pgfusepath{clip}%
\pgfsetroundcap%
\pgfsetroundjoin%
\definecolor{currentfill}{rgb}{0.500000,0.500000,0.500000}%
\pgfsetfillcolor{currentfill}%
\pgfsetfillopacity{0.300000}%
\pgfsetlinewidth{0.301125pt}%
\definecolor{currentstroke}{rgb}{0.500000,0.500000,0.500000}%
\pgfsetstrokecolor{currentstroke}%
\pgfsetstrokeopacity{0.300000}%
\pgfsetdash{}{0pt}%
\pgfpathmoveto{\pgfqpoint{0.000000in}{0.000000in}}%
\pgfpathlineto{\pgfqpoint{0.000000in}{0.000000in}}%
\pgfpathclose%
\pgfusepath{stroke,fill}%
\end{pgfscope}%
\begin{pgfscope}%
\pgfpathrectangle{\pgfqpoint{0.647939in}{0.492442in}}{\pgfqpoint{4.273799in}{2.331163in}}%
\pgfusepath{clip}%
\pgfsetroundcap%
\pgfsetroundjoin%
\pgfsetlinewidth{0.301125pt}%
\definecolor{currentstroke}{rgb}{0.500000,0.500000,0.500000}%
\pgfsetstrokecolor{currentstroke}%
\pgfsetstrokeopacity{0.300000}%
\pgfsetdash{}{0pt}%
\pgfpathmoveto{\pgfqpoint{2.069939in}{2.727518in}}%
\pgfusepath{stroke}%
\end{pgfscope}%
\begin{pgfscope}%
\pgfpathrectangle{\pgfqpoint{0.647939in}{0.492442in}}{\pgfqpoint{4.273799in}{2.331163in}}%
\pgfusepath{clip}%
\pgfsetroundcap%
\pgfsetroundjoin%
\definecolor{currentfill}{rgb}{0.500000,0.500000,0.500000}%
\pgfsetfillcolor{currentfill}%
\pgfsetfillopacity{0.300000}%
\pgfsetlinewidth{0.301125pt}%
\definecolor{currentstroke}{rgb}{0.500000,0.500000,0.500000}%
\pgfsetstrokecolor{currentstroke}%
\pgfsetstrokeopacity{0.300000}%
\pgfsetdash{}{0pt}%
\pgfpathmoveto{\pgfqpoint{0.000000in}{0.000000in}}%
\pgfpathlineto{\pgfqpoint{0.000000in}{0.000000in}}%
\pgfpathclose%
\pgfusepath{stroke,fill}%
\end{pgfscope}%
\begin{pgfscope}%
\pgfpathrectangle{\pgfqpoint{0.647939in}{0.492442in}}{\pgfqpoint{4.273799in}{2.331163in}}%
\pgfusepath{clip}%
\pgfsetroundcap%
\pgfsetroundjoin%
\pgfsetlinewidth{0.301125pt}%
\definecolor{currentstroke}{rgb}{0.500000,0.500000,0.500000}%
\pgfsetstrokecolor{currentstroke}%
\pgfsetstrokeopacity{0.300000}%
\pgfsetdash{}{0pt}%
\pgfpathmoveto{\pgfqpoint{2.242165in}{2.609680in}}%
\pgfusepath{stroke}%
\end{pgfscope}%
\begin{pgfscope}%
\pgfpathrectangle{\pgfqpoint{0.647939in}{0.492442in}}{\pgfqpoint{4.273799in}{2.331163in}}%
\pgfusepath{clip}%
\pgfsetroundcap%
\pgfsetroundjoin%
\definecolor{currentfill}{rgb}{0.500000,0.500000,0.500000}%
\pgfsetfillcolor{currentfill}%
\pgfsetfillopacity{0.300000}%
\pgfsetlinewidth{0.301125pt}%
\definecolor{currentstroke}{rgb}{0.500000,0.500000,0.500000}%
\pgfsetstrokecolor{currentstroke}%
\pgfsetstrokeopacity{0.300000}%
\pgfsetdash{}{0pt}%
\pgfpathmoveto{\pgfqpoint{0.000000in}{0.000000in}}%
\pgfpathlineto{\pgfqpoint{0.000000in}{0.000000in}}%
\pgfpathclose%
\pgfusepath{stroke,fill}%
\end{pgfscope}%
\begin{pgfscope}%
\pgfpathrectangle{\pgfqpoint{0.647939in}{0.492442in}}{\pgfqpoint{4.273799in}{2.331163in}}%
\pgfusepath{clip}%
\pgfsetroundcap%
\pgfsetroundjoin%
\pgfsetlinewidth{0.301125pt}%
\definecolor{currentstroke}{rgb}{0.500000,0.500000,0.500000}%
\pgfsetstrokecolor{currentstroke}%
\pgfsetstrokeopacity{0.300000}%
\pgfsetdash{}{0pt}%
\pgfpathmoveto{\pgfqpoint{1.954379in}{2.553731in}}%
\pgfusepath{stroke}%
\end{pgfscope}%
\begin{pgfscope}%
\pgfpathrectangle{\pgfqpoint{0.647939in}{0.492442in}}{\pgfqpoint{4.273799in}{2.331163in}}%
\pgfusepath{clip}%
\pgfsetroundcap%
\pgfsetroundjoin%
\definecolor{currentfill}{rgb}{0.500000,0.500000,0.500000}%
\pgfsetfillcolor{currentfill}%
\pgfsetfillopacity{0.300000}%
\pgfsetlinewidth{0.301125pt}%
\definecolor{currentstroke}{rgb}{0.500000,0.500000,0.500000}%
\pgfsetstrokecolor{currentstroke}%
\pgfsetstrokeopacity{0.300000}%
\pgfsetdash{}{0pt}%
\pgfpathmoveto{\pgfqpoint{0.000000in}{0.000000in}}%
\pgfpathlineto{\pgfqpoint{0.000000in}{0.000000in}}%
\pgfpathclose%
\pgfusepath{stroke,fill}%
\end{pgfscope}%
\begin{pgfscope}%
\pgfpathrectangle{\pgfqpoint{0.647939in}{0.492442in}}{\pgfqpoint{4.273799in}{2.331163in}}%
\pgfusepath{clip}%
\pgfsetroundcap%
\pgfsetroundjoin%
\pgfsetlinewidth{0.301125pt}%
\definecolor{currentstroke}{rgb}{0.500000,0.500000,0.500000}%
\pgfsetstrokecolor{currentstroke}%
\pgfsetstrokeopacity{0.300000}%
\pgfsetdash{}{0pt}%
\pgfpathmoveto{\pgfqpoint{2.054316in}{2.471938in}}%
\pgfusepath{stroke}%
\end{pgfscope}%
\begin{pgfscope}%
\pgfpathrectangle{\pgfqpoint{0.647939in}{0.492442in}}{\pgfqpoint{4.273799in}{2.331163in}}%
\pgfusepath{clip}%
\pgfsetroundcap%
\pgfsetroundjoin%
\definecolor{currentfill}{rgb}{0.500000,0.500000,0.500000}%
\pgfsetfillcolor{currentfill}%
\pgfsetfillopacity{0.300000}%
\pgfsetlinewidth{0.301125pt}%
\definecolor{currentstroke}{rgb}{0.500000,0.500000,0.500000}%
\pgfsetstrokecolor{currentstroke}%
\pgfsetstrokeopacity{0.300000}%
\pgfsetdash{}{0pt}%
\pgfpathmoveto{\pgfqpoint{0.000000in}{0.000000in}}%
\pgfpathlineto{\pgfqpoint{0.000000in}{0.000000in}}%
\pgfpathclose%
\pgfusepath{stroke,fill}%
\end{pgfscope}%
\begin{pgfscope}%
\pgfpathrectangle{\pgfqpoint{0.647939in}{0.492442in}}{\pgfqpoint{4.273799in}{2.331163in}}%
\pgfusepath{clip}%
\pgfsetroundcap%
\pgfsetroundjoin%
\pgfsetlinewidth{0.301125pt}%
\definecolor{currentstroke}{rgb}{0.500000,0.500000,0.500000}%
\pgfsetstrokecolor{currentstroke}%
\pgfsetstrokeopacity{0.300000}%
\pgfsetdash{}{0pt}%
\pgfpathmoveto{\pgfqpoint{1.626711in}{2.365504in}}%
\pgfusepath{stroke}%
\end{pgfscope}%
\begin{pgfscope}%
\pgfpathrectangle{\pgfqpoint{0.647939in}{0.492442in}}{\pgfqpoint{4.273799in}{2.331163in}}%
\pgfusepath{clip}%
\pgfsetroundcap%
\pgfsetroundjoin%
\definecolor{currentfill}{rgb}{0.500000,0.500000,0.500000}%
\pgfsetfillcolor{currentfill}%
\pgfsetfillopacity{0.300000}%
\pgfsetlinewidth{0.301125pt}%
\definecolor{currentstroke}{rgb}{0.500000,0.500000,0.500000}%
\pgfsetstrokecolor{currentstroke}%
\pgfsetstrokeopacity{0.300000}%
\pgfsetdash{}{0pt}%
\pgfpathmoveto{\pgfqpoint{0.000000in}{0.000000in}}%
\pgfpathlineto{\pgfqpoint{0.000000in}{0.000000in}}%
\pgfpathclose%
\pgfusepath{stroke,fill}%
\end{pgfscope}%
\begin{pgfscope}%
\pgfpathrectangle{\pgfqpoint{0.647939in}{0.492442in}}{\pgfqpoint{4.273799in}{2.331163in}}%
\pgfusepath{clip}%
\pgfsetroundcap%
\pgfsetroundjoin%
\pgfsetlinewidth{0.301125pt}%
\definecolor{currentstroke}{rgb}{0.500000,0.500000,0.500000}%
\pgfsetstrokecolor{currentstroke}%
\pgfsetstrokeopacity{0.300000}%
\pgfsetdash{}{0pt}%
\pgfpathmoveto{\pgfqpoint{1.421344in}{2.369139in}}%
\pgfusepath{stroke}%
\end{pgfscope}%
\begin{pgfscope}%
\pgfpathrectangle{\pgfqpoint{0.647939in}{0.492442in}}{\pgfqpoint{4.273799in}{2.331163in}}%
\pgfusepath{clip}%
\pgfsetroundcap%
\pgfsetroundjoin%
\definecolor{currentfill}{rgb}{0.500000,0.500000,0.500000}%
\pgfsetfillcolor{currentfill}%
\pgfsetfillopacity{0.300000}%
\pgfsetlinewidth{0.301125pt}%
\definecolor{currentstroke}{rgb}{0.500000,0.500000,0.500000}%
\pgfsetstrokecolor{currentstroke}%
\pgfsetstrokeopacity{0.300000}%
\pgfsetdash{}{0pt}%
\pgfpathmoveto{\pgfqpoint{0.000000in}{0.000000in}}%
\pgfpathlineto{\pgfqpoint{0.000000in}{0.000000in}}%
\pgfpathclose%
\pgfusepath{stroke,fill}%
\end{pgfscope}%
\begin{pgfscope}%
\pgfpathrectangle{\pgfqpoint{0.647939in}{0.492442in}}{\pgfqpoint{4.273799in}{2.331163in}}%
\pgfusepath{clip}%
\pgfsetroundcap%
\pgfsetroundjoin%
\pgfsetlinewidth{0.301125pt}%
\definecolor{currentstroke}{rgb}{0.500000,0.500000,0.500000}%
\pgfsetstrokecolor{currentstroke}%
\pgfsetstrokeopacity{0.300000}%
\pgfsetdash{}{0pt}%
\pgfpathmoveto{\pgfqpoint{1.319689in}{2.158148in}}%
\pgfusepath{stroke}%
\end{pgfscope}%
\begin{pgfscope}%
\pgfpathrectangle{\pgfqpoint{0.647939in}{0.492442in}}{\pgfqpoint{4.273799in}{2.331163in}}%
\pgfusepath{clip}%
\pgfsetroundcap%
\pgfsetroundjoin%
\definecolor{currentfill}{rgb}{0.500000,0.500000,0.500000}%
\pgfsetfillcolor{currentfill}%
\pgfsetfillopacity{0.300000}%
\pgfsetlinewidth{0.301125pt}%
\definecolor{currentstroke}{rgb}{0.500000,0.500000,0.500000}%
\pgfsetstrokecolor{currentstroke}%
\pgfsetstrokeopacity{0.300000}%
\pgfsetdash{}{0pt}%
\pgfpathmoveto{\pgfqpoint{0.000000in}{0.000000in}}%
\pgfpathlineto{\pgfqpoint{0.000000in}{0.000000in}}%
\pgfpathclose%
\pgfusepath{stroke,fill}%
\end{pgfscope}%
\begin{pgfscope}%
\pgfpathrectangle{\pgfqpoint{0.647939in}{0.492442in}}{\pgfqpoint{4.273799in}{2.331163in}}%
\pgfusepath{clip}%
\pgfsetroundcap%
\pgfsetroundjoin%
\pgfsetlinewidth{0.301125pt}%
\definecolor{currentstroke}{rgb}{0.500000,0.500000,0.500000}%
\pgfsetstrokecolor{currentstroke}%
\pgfsetstrokeopacity{0.300000}%
\pgfsetdash{}{0pt}%
\pgfpathmoveto{\pgfqpoint{1.165400in}{2.258260in}}%
\pgfusepath{stroke}%
\end{pgfscope}%
\begin{pgfscope}%
\pgfpathrectangle{\pgfqpoint{0.647939in}{0.492442in}}{\pgfqpoint{4.273799in}{2.331163in}}%
\pgfusepath{clip}%
\pgfsetroundcap%
\pgfsetroundjoin%
\definecolor{currentfill}{rgb}{0.500000,0.500000,0.500000}%
\pgfsetfillcolor{currentfill}%
\pgfsetfillopacity{0.300000}%
\pgfsetlinewidth{0.301125pt}%
\definecolor{currentstroke}{rgb}{0.500000,0.500000,0.500000}%
\pgfsetstrokecolor{currentstroke}%
\pgfsetstrokeopacity{0.300000}%
\pgfsetdash{}{0pt}%
\pgfpathmoveto{\pgfqpoint{0.000000in}{0.000000in}}%
\pgfpathlineto{\pgfqpoint{0.000000in}{0.000000in}}%
\pgfpathclose%
\pgfusepath{stroke,fill}%
\end{pgfscope}%
\begin{pgfscope}%
\pgfpathrectangle{\pgfqpoint{0.647939in}{0.492442in}}{\pgfqpoint{4.273799in}{2.331163in}}%
\pgfusepath{clip}%
\pgfsetroundcap%
\pgfsetroundjoin%
\pgfsetlinewidth{0.301125pt}%
\definecolor{currentstroke}{rgb}{0.500000,0.500000,0.500000}%
\pgfsetstrokecolor{currentstroke}%
\pgfsetstrokeopacity{0.300000}%
\pgfsetdash{}{0pt}%
\pgfpathmoveto{\pgfqpoint{1.046836in}{1.842473in}}%
\pgfusepath{stroke}%
\end{pgfscope}%
\begin{pgfscope}%
\pgfpathrectangle{\pgfqpoint{0.647939in}{0.492442in}}{\pgfqpoint{4.273799in}{2.331163in}}%
\pgfusepath{clip}%
\pgfsetroundcap%
\pgfsetroundjoin%
\definecolor{currentfill}{rgb}{0.500000,0.500000,0.500000}%
\pgfsetfillcolor{currentfill}%
\pgfsetfillopacity{0.300000}%
\pgfsetlinewidth{0.301125pt}%
\definecolor{currentstroke}{rgb}{0.500000,0.500000,0.500000}%
\pgfsetstrokecolor{currentstroke}%
\pgfsetstrokeopacity{0.300000}%
\pgfsetdash{}{0pt}%
\pgfpathmoveto{\pgfqpoint{0.000000in}{0.000000in}}%
\pgfpathlineto{\pgfqpoint{0.000000in}{0.000000in}}%
\pgfpathclose%
\pgfusepath{stroke,fill}%
\end{pgfscope}%
\begin{pgfscope}%
\pgfpathrectangle{\pgfqpoint{0.647939in}{0.492442in}}{\pgfqpoint{4.273799in}{2.331163in}}%
\pgfusepath{clip}%
\pgfsetroundcap%
\pgfsetroundjoin%
\pgfsetlinewidth{0.301125pt}%
\definecolor{currentstroke}{rgb}{0.500000,0.500000,0.500000}%
\pgfsetstrokecolor{currentstroke}%
\pgfsetstrokeopacity{0.300000}%
\pgfsetdash{}{0pt}%
\pgfpathmoveto{\pgfqpoint{0.936433in}{2.048605in}}%
\pgfusepath{stroke}%
\end{pgfscope}%
\begin{pgfscope}%
\pgfpathrectangle{\pgfqpoint{0.647939in}{0.492442in}}{\pgfqpoint{4.273799in}{2.331163in}}%
\pgfusepath{clip}%
\pgfsetroundcap%
\pgfsetroundjoin%
\definecolor{currentfill}{rgb}{0.500000,0.500000,0.500000}%
\pgfsetfillcolor{currentfill}%
\pgfsetfillopacity{0.300000}%
\pgfsetlinewidth{0.301125pt}%
\definecolor{currentstroke}{rgb}{0.500000,0.500000,0.500000}%
\pgfsetstrokecolor{currentstroke}%
\pgfsetstrokeopacity{0.300000}%
\pgfsetdash{}{0pt}%
\pgfpathmoveto{\pgfqpoint{0.000000in}{0.000000in}}%
\pgfpathlineto{\pgfqpoint{0.000000in}{0.000000in}}%
\pgfpathclose%
\pgfusepath{stroke,fill}%
\end{pgfscope}%
\begin{pgfscope}%
\pgfpathrectangle{\pgfqpoint{0.647939in}{0.492442in}}{\pgfqpoint{4.273799in}{2.331163in}}%
\pgfusepath{clip}%
\pgfsetroundcap%
\pgfsetroundjoin%
\pgfsetlinewidth{0.301125pt}%
\definecolor{currentstroke}{rgb}{0.500000,0.500000,0.500000}%
\pgfsetstrokecolor{currentstroke}%
\pgfsetstrokeopacity{0.300000}%
\pgfsetdash{}{0pt}%
\pgfpathmoveto{\pgfqpoint{0.820910in}{1.892587in}}%
\pgfusepath{stroke}%
\end{pgfscope}%
\begin{pgfscope}%
\pgfpathrectangle{\pgfqpoint{0.647939in}{0.492442in}}{\pgfqpoint{4.273799in}{2.331163in}}%
\pgfusepath{clip}%
\pgfsetroundcap%
\pgfsetroundjoin%
\definecolor{currentfill}{rgb}{0.500000,0.500000,0.500000}%
\pgfsetfillcolor{currentfill}%
\pgfsetfillopacity{0.300000}%
\pgfsetlinewidth{0.301125pt}%
\definecolor{currentstroke}{rgb}{0.500000,0.500000,0.500000}%
\pgfsetstrokecolor{currentstroke}%
\pgfsetstrokeopacity{0.300000}%
\pgfsetdash{}{0pt}%
\pgfpathmoveto{\pgfqpoint{0.000000in}{0.000000in}}%
\pgfpathlineto{\pgfqpoint{0.000000in}{0.000000in}}%
\pgfpathclose%
\pgfusepath{stroke,fill}%
\end{pgfscope}%
\begin{pgfscope}%
\pgfpathrectangle{\pgfqpoint{0.647939in}{0.492442in}}{\pgfqpoint{4.273799in}{2.331163in}}%
\pgfusepath{clip}%
\pgfsetroundcap%
\pgfsetroundjoin%
\pgfsetlinewidth{0.301125pt}%
\definecolor{currentstroke}{rgb}{0.500000,0.500000,0.500000}%
\pgfsetstrokecolor{currentstroke}%
\pgfsetstrokeopacity{0.300000}%
\pgfsetdash{}{0pt}%
\pgfpathmoveto{\pgfqpoint{0.713310in}{2.099385in}}%
\pgfusepath{stroke}%
\end{pgfscope}%
\begin{pgfscope}%
\pgfpathrectangle{\pgfqpoint{0.647939in}{0.492442in}}{\pgfqpoint{4.273799in}{2.331163in}}%
\pgfusepath{clip}%
\pgfsetroundcap%
\pgfsetroundjoin%
\definecolor{currentfill}{rgb}{0.500000,0.500000,0.500000}%
\pgfsetfillcolor{currentfill}%
\pgfsetfillopacity{0.300000}%
\pgfsetlinewidth{0.301125pt}%
\definecolor{currentstroke}{rgb}{0.500000,0.500000,0.500000}%
\pgfsetstrokecolor{currentstroke}%
\pgfsetstrokeopacity{0.300000}%
\pgfsetdash{}{0pt}%
\pgfpathmoveto{\pgfqpoint{0.000000in}{0.000000in}}%
\pgfpathlineto{\pgfqpoint{0.000000in}{0.000000in}}%
\pgfpathclose%
\pgfusepath{stroke,fill}%
\end{pgfscope}%
\begin{pgfscope}%
\pgfpathrectangle{\pgfqpoint{0.647939in}{0.492442in}}{\pgfqpoint{4.273799in}{2.331163in}}%
\pgfusepath{clip}%
\pgfsetroundcap%
\pgfsetroundjoin%
\pgfsetlinewidth{0.301125pt}%
\definecolor{currentstroke}{rgb}{0.500000,0.500000,0.500000}%
\pgfsetstrokecolor{currentstroke}%
\pgfsetstrokeopacity{0.300000}%
\pgfsetdash{}{0pt}%
\pgfpathmoveto{\pgfqpoint{0.662382in}{2.089051in}}%
\pgfusepath{stroke}%
\end{pgfscope}%
\begin{pgfscope}%
\pgfpathrectangle{\pgfqpoint{0.647939in}{0.492442in}}{\pgfqpoint{4.273799in}{2.331163in}}%
\pgfusepath{clip}%
\pgfsetroundcap%
\pgfsetroundjoin%
\definecolor{currentfill}{rgb}{0.500000,0.500000,0.500000}%
\pgfsetfillcolor{currentfill}%
\pgfsetfillopacity{0.300000}%
\pgfsetlinewidth{0.301125pt}%
\definecolor{currentstroke}{rgb}{0.500000,0.500000,0.500000}%
\pgfsetstrokecolor{currentstroke}%
\pgfsetstrokeopacity{0.300000}%
\pgfsetdash{}{0pt}%
\pgfpathmoveto{\pgfqpoint{0.000000in}{0.000000in}}%
\pgfpathlineto{\pgfqpoint{0.000000in}{0.000000in}}%
\pgfpathclose%
\pgfusepath{stroke,fill}%
\end{pgfscope}%
\begin{pgfscope}%
\pgfpathrectangle{\pgfqpoint{0.647939in}{0.492442in}}{\pgfqpoint{4.273799in}{2.331163in}}%
\pgfusepath{clip}%
\pgfsetroundcap%
\pgfsetroundjoin%
\pgfsetlinewidth{0.301125pt}%
\definecolor{currentstroke}{rgb}{0.500000,0.500000,0.500000}%
\pgfsetstrokecolor{currentstroke}%
\pgfsetstrokeopacity{0.300000}%
\pgfsetdash{}{0pt}%
\pgfpathmoveto{\pgfqpoint{1.336369in}{0.595071in}}%
\pgfusepath{stroke}%
\end{pgfscope}%
\begin{pgfscope}%
\pgfpathrectangle{\pgfqpoint{0.647939in}{0.492442in}}{\pgfqpoint{4.273799in}{2.331163in}}%
\pgfusepath{clip}%
\pgfsetroundcap%
\pgfsetroundjoin%
\definecolor{currentfill}{rgb}{0.500000,0.500000,0.500000}%
\pgfsetfillcolor{currentfill}%
\pgfsetfillopacity{0.300000}%
\pgfsetlinewidth{0.301125pt}%
\definecolor{currentstroke}{rgb}{0.500000,0.500000,0.500000}%
\pgfsetstrokecolor{currentstroke}%
\pgfsetstrokeopacity{0.300000}%
\pgfsetdash{}{0pt}%
\pgfpathmoveto{\pgfqpoint{0.000000in}{0.000000in}}%
\pgfpathlineto{\pgfqpoint{0.000000in}{0.000000in}}%
\pgfpathclose%
\pgfusepath{stroke,fill}%
\end{pgfscope}%
\begin{pgfscope}%
\pgfpathrectangle{\pgfqpoint{0.647939in}{0.492442in}}{\pgfqpoint{4.273799in}{2.331163in}}%
\pgfusepath{clip}%
\pgfsetroundcap%
\pgfsetroundjoin%
\pgfsetlinewidth{0.301125pt}%
\definecolor{currentstroke}{rgb}{0.500000,0.500000,0.500000}%
\pgfsetstrokecolor{currentstroke}%
\pgfsetstrokeopacity{0.300000}%
\pgfsetdash{}{0pt}%
\pgfpathmoveto{\pgfqpoint{3.888601in}{0.953843in}}%
\pgfusepath{stroke}%
\end{pgfscope}%
\begin{pgfscope}%
\pgfpathrectangle{\pgfqpoint{0.647939in}{0.492442in}}{\pgfqpoint{4.273799in}{2.331163in}}%
\pgfusepath{clip}%
\pgfsetroundcap%
\pgfsetroundjoin%
\definecolor{currentfill}{rgb}{0.500000,0.500000,0.500000}%
\pgfsetfillcolor{currentfill}%
\pgfsetfillopacity{0.300000}%
\pgfsetlinewidth{0.301125pt}%
\definecolor{currentstroke}{rgb}{0.500000,0.500000,0.500000}%
\pgfsetstrokecolor{currentstroke}%
\pgfsetstrokeopacity{0.300000}%
\pgfsetdash{}{0pt}%
\pgfpathmoveto{\pgfqpoint{0.000000in}{0.000000in}}%
\pgfpathlineto{\pgfqpoint{0.000000in}{0.000000in}}%
\pgfpathclose%
\pgfusepath{stroke,fill}%
\end{pgfscope}%
\begin{pgfscope}%
\pgfpathrectangle{\pgfqpoint{0.647939in}{0.492442in}}{\pgfqpoint{4.273799in}{2.331163in}}%
\pgfusepath{clip}%
\pgfsetroundcap%
\pgfsetroundjoin%
\pgfsetlinewidth{0.301125pt}%
\definecolor{currentstroke}{rgb}{0.500000,0.500000,0.500000}%
\pgfsetstrokecolor{currentstroke}%
\pgfsetstrokeopacity{0.300000}%
\pgfsetdash{}{0pt}%
\pgfpathmoveto{\pgfqpoint{4.719896in}{1.973462in}}%
\pgfusepath{stroke}%
\end{pgfscope}%
\begin{pgfscope}%
\pgfpathrectangle{\pgfqpoint{0.647939in}{0.492442in}}{\pgfqpoint{4.273799in}{2.331163in}}%
\pgfusepath{clip}%
\pgfsetroundcap%
\pgfsetroundjoin%
\definecolor{currentfill}{rgb}{0.500000,0.500000,0.500000}%
\pgfsetfillcolor{currentfill}%
\pgfsetfillopacity{0.300000}%
\pgfsetlinewidth{0.301125pt}%
\definecolor{currentstroke}{rgb}{0.500000,0.500000,0.500000}%
\pgfsetstrokecolor{currentstroke}%
\pgfsetstrokeopacity{0.300000}%
\pgfsetdash{}{0pt}%
\pgfpathmoveto{\pgfqpoint{0.000000in}{0.000000in}}%
\pgfpathlineto{\pgfqpoint{0.000000in}{0.000000in}}%
\pgfpathclose%
\pgfusepath{stroke,fill}%
\end{pgfscope}%
\begin{pgfscope}%
\pgfpathrectangle{\pgfqpoint{0.647939in}{0.492442in}}{\pgfqpoint{4.273799in}{2.331163in}}%
\pgfusepath{clip}%
\pgfsetroundcap%
\pgfsetroundjoin%
\pgfsetlinewidth{0.301125pt}%
\definecolor{currentstroke}{rgb}{0.500000,0.500000,0.500000}%
\pgfsetstrokecolor{currentstroke}%
\pgfsetstrokeopacity{0.300000}%
\pgfsetdash{}{0pt}%
\pgfpathmoveto{\pgfqpoint{1.383963in}{0.746731in}}%
\pgfusepath{stroke}%
\end{pgfscope}%
\begin{pgfscope}%
\pgfpathrectangle{\pgfqpoint{0.647939in}{0.492442in}}{\pgfqpoint{4.273799in}{2.331163in}}%
\pgfusepath{clip}%
\pgfsetroundcap%
\pgfsetroundjoin%
\definecolor{currentfill}{rgb}{0.500000,0.500000,0.500000}%
\pgfsetfillcolor{currentfill}%
\pgfsetfillopacity{0.300000}%
\pgfsetlinewidth{0.301125pt}%
\definecolor{currentstroke}{rgb}{0.500000,0.500000,0.500000}%
\pgfsetstrokecolor{currentstroke}%
\pgfsetstrokeopacity{0.300000}%
\pgfsetdash{}{0pt}%
\pgfpathmoveto{\pgfqpoint{0.000000in}{0.000000in}}%
\pgfpathlineto{\pgfqpoint{0.000000in}{0.000000in}}%
\pgfpathclose%
\pgfusepath{stroke,fill}%
\end{pgfscope}%
\begin{pgfscope}%
\pgfpathrectangle{\pgfqpoint{0.647939in}{0.492442in}}{\pgfqpoint{4.273799in}{2.331163in}}%
\pgfusepath{clip}%
\pgfsetroundcap%
\pgfsetroundjoin%
\pgfsetlinewidth{0.301125pt}%
\definecolor{currentstroke}{rgb}{0.500000,0.500000,0.500000}%
\pgfsetstrokecolor{currentstroke}%
\pgfsetstrokeopacity{0.300000}%
\pgfsetdash{}{0pt}%
\pgfpathmoveto{\pgfqpoint{4.025130in}{0.667725in}}%
\pgfusepath{stroke}%
\end{pgfscope}%
\begin{pgfscope}%
\pgfpathrectangle{\pgfqpoint{0.647939in}{0.492442in}}{\pgfqpoint{4.273799in}{2.331163in}}%
\pgfusepath{clip}%
\pgfsetroundcap%
\pgfsetroundjoin%
\definecolor{currentfill}{rgb}{0.500000,0.500000,0.500000}%
\pgfsetfillcolor{currentfill}%
\pgfsetfillopacity{0.300000}%
\pgfsetlinewidth{0.301125pt}%
\definecolor{currentstroke}{rgb}{0.500000,0.500000,0.500000}%
\pgfsetstrokecolor{currentstroke}%
\pgfsetstrokeopacity{0.300000}%
\pgfsetdash{}{0pt}%
\pgfpathmoveto{\pgfqpoint{0.000000in}{0.000000in}}%
\pgfpathlineto{\pgfqpoint{0.000000in}{0.000000in}}%
\pgfpathclose%
\pgfusepath{stroke,fill}%
\end{pgfscope}%
\begin{pgfscope}%
\pgfpathrectangle{\pgfqpoint{0.647939in}{0.492442in}}{\pgfqpoint{4.273799in}{2.331163in}}%
\pgfusepath{clip}%
\pgfsetroundcap%
\pgfsetroundjoin%
\pgfsetlinewidth{0.301125pt}%
\definecolor{currentstroke}{rgb}{0.500000,0.500000,0.500000}%
\pgfsetstrokecolor{currentstroke}%
\pgfsetstrokeopacity{0.300000}%
\pgfsetdash{}{0pt}%
\pgfpathmoveto{\pgfqpoint{4.572386in}{1.915013in}}%
\pgfusepath{stroke}%
\end{pgfscope}%
\begin{pgfscope}%
\pgfpathrectangle{\pgfqpoint{0.647939in}{0.492442in}}{\pgfqpoint{4.273799in}{2.331163in}}%
\pgfusepath{clip}%
\pgfsetroundcap%
\pgfsetroundjoin%
\definecolor{currentfill}{rgb}{0.500000,0.500000,0.500000}%
\pgfsetfillcolor{currentfill}%
\pgfsetfillopacity{0.300000}%
\pgfsetlinewidth{0.301125pt}%
\definecolor{currentstroke}{rgb}{0.500000,0.500000,0.500000}%
\pgfsetstrokecolor{currentstroke}%
\pgfsetstrokeopacity{0.300000}%
\pgfsetdash{}{0pt}%
\pgfpathmoveto{\pgfqpoint{0.000000in}{0.000000in}}%
\pgfpathlineto{\pgfqpoint{0.000000in}{0.000000in}}%
\pgfpathclose%
\pgfusepath{stroke,fill}%
\end{pgfscope}%
\begin{pgfscope}%
\pgfpathrectangle{\pgfqpoint{0.647939in}{0.492442in}}{\pgfqpoint{4.273799in}{2.331163in}}%
\pgfusepath{clip}%
\pgfsetroundcap%
\pgfsetroundjoin%
\pgfsetlinewidth{0.301125pt}%
\definecolor{currentstroke}{rgb}{0.500000,0.500000,0.500000}%
\pgfsetstrokecolor{currentstroke}%
\pgfsetstrokeopacity{0.300000}%
\pgfsetdash{}{0pt}%
\pgfpathmoveto{\pgfqpoint{3.671283in}{0.758799in}}%
\pgfusepath{stroke}%
\end{pgfscope}%
\begin{pgfscope}%
\pgfpathrectangle{\pgfqpoint{0.647939in}{0.492442in}}{\pgfqpoint{4.273799in}{2.331163in}}%
\pgfusepath{clip}%
\pgfsetroundcap%
\pgfsetroundjoin%
\definecolor{currentfill}{rgb}{0.500000,0.500000,0.500000}%
\pgfsetfillcolor{currentfill}%
\pgfsetfillopacity{0.300000}%
\pgfsetlinewidth{0.301125pt}%
\definecolor{currentstroke}{rgb}{0.500000,0.500000,0.500000}%
\pgfsetstrokecolor{currentstroke}%
\pgfsetstrokeopacity{0.300000}%
\pgfsetdash{}{0pt}%
\pgfpathmoveto{\pgfqpoint{0.000000in}{0.000000in}}%
\pgfpathlineto{\pgfqpoint{0.000000in}{0.000000in}}%
\pgfpathclose%
\pgfusepath{stroke,fill}%
\end{pgfscope}%
\begin{pgfscope}%
\pgfpathrectangle{\pgfqpoint{0.647939in}{0.492442in}}{\pgfqpoint{4.273799in}{2.331163in}}%
\pgfusepath{clip}%
\pgfsetroundcap%
\pgfsetroundjoin%
\pgfsetlinewidth{0.301125pt}%
\definecolor{currentstroke}{rgb}{0.500000,0.500000,0.500000}%
\pgfsetstrokecolor{currentstroke}%
\pgfsetstrokeopacity{0.300000}%
\pgfsetdash{}{0pt}%
\pgfpathmoveto{\pgfqpoint{4.394401in}{1.773194in}}%
\pgfusepath{stroke}%
\end{pgfscope}%
\begin{pgfscope}%
\pgfpathrectangle{\pgfqpoint{0.647939in}{0.492442in}}{\pgfqpoint{4.273799in}{2.331163in}}%
\pgfusepath{clip}%
\pgfsetroundcap%
\pgfsetroundjoin%
\definecolor{currentfill}{rgb}{0.500000,0.500000,0.500000}%
\pgfsetfillcolor{currentfill}%
\pgfsetfillopacity{0.300000}%
\pgfsetlinewidth{0.301125pt}%
\definecolor{currentstroke}{rgb}{0.500000,0.500000,0.500000}%
\pgfsetstrokecolor{currentstroke}%
\pgfsetstrokeopacity{0.300000}%
\pgfsetdash{}{0pt}%
\pgfpathmoveto{\pgfqpoint{0.000000in}{0.000000in}}%
\pgfpathlineto{\pgfqpoint{0.000000in}{0.000000in}}%
\pgfpathclose%
\pgfusepath{stroke,fill}%
\end{pgfscope}%
\begin{pgfscope}%
\pgfpathrectangle{\pgfqpoint{0.647939in}{0.492442in}}{\pgfqpoint{4.273799in}{2.331163in}}%
\pgfusepath{clip}%
\pgfsetroundcap%
\pgfsetroundjoin%
\pgfsetlinewidth{0.301125pt}%
\definecolor{currentstroke}{rgb}{0.500000,0.500000,0.500000}%
\pgfsetstrokecolor{currentstroke}%
\pgfsetstrokeopacity{0.300000}%
\pgfsetdash{}{0pt}%
\pgfpathmoveto{\pgfqpoint{1.013973in}{1.342794in}}%
\pgfusepath{stroke}%
\end{pgfscope}%
\begin{pgfscope}%
\pgfpathrectangle{\pgfqpoint{0.647939in}{0.492442in}}{\pgfqpoint{4.273799in}{2.331163in}}%
\pgfusepath{clip}%
\pgfsetroundcap%
\pgfsetroundjoin%
\definecolor{currentfill}{rgb}{0.500000,0.500000,0.500000}%
\pgfsetfillcolor{currentfill}%
\pgfsetfillopacity{0.300000}%
\pgfsetlinewidth{0.301125pt}%
\definecolor{currentstroke}{rgb}{0.500000,0.500000,0.500000}%
\pgfsetstrokecolor{currentstroke}%
\pgfsetstrokeopacity{0.300000}%
\pgfsetdash{}{0pt}%
\pgfpathmoveto{\pgfqpoint{0.000000in}{0.000000in}}%
\pgfpathlineto{\pgfqpoint{0.000000in}{0.000000in}}%
\pgfpathclose%
\pgfusepath{stroke,fill}%
\end{pgfscope}%
\begin{pgfscope}%
\pgfpathrectangle{\pgfqpoint{0.647939in}{0.492442in}}{\pgfqpoint{4.273799in}{2.331163in}}%
\pgfusepath{clip}%
\pgfsetroundcap%
\pgfsetroundjoin%
\pgfsetlinewidth{0.301125pt}%
\definecolor{currentstroke}{rgb}{0.500000,0.500000,0.500000}%
\pgfsetstrokecolor{currentstroke}%
\pgfsetstrokeopacity{0.300000}%
\pgfsetdash{}{0pt}%
\pgfpathmoveto{\pgfqpoint{2.516705in}{1.115775in}}%
\pgfusepath{stroke}%
\end{pgfscope}%
\begin{pgfscope}%
\pgfpathrectangle{\pgfqpoint{0.647939in}{0.492442in}}{\pgfqpoint{4.273799in}{2.331163in}}%
\pgfusepath{clip}%
\pgfsetroundcap%
\pgfsetroundjoin%
\definecolor{currentfill}{rgb}{0.500000,0.500000,0.500000}%
\pgfsetfillcolor{currentfill}%
\pgfsetfillopacity{0.300000}%
\pgfsetlinewidth{0.301125pt}%
\definecolor{currentstroke}{rgb}{0.500000,0.500000,0.500000}%
\pgfsetstrokecolor{currentstroke}%
\pgfsetstrokeopacity{0.300000}%
\pgfsetdash{}{0pt}%
\pgfpathmoveto{\pgfqpoint{0.000000in}{0.000000in}}%
\pgfpathlineto{\pgfqpoint{0.000000in}{0.000000in}}%
\pgfpathclose%
\pgfusepath{stroke,fill}%
\end{pgfscope}%
\begin{pgfscope}%
\pgfpathrectangle{\pgfqpoint{0.647939in}{0.492442in}}{\pgfqpoint{4.273799in}{2.331163in}}%
\pgfusepath{clip}%
\pgfsetroundcap%
\pgfsetroundjoin%
\pgfsetlinewidth{0.301125pt}%
\definecolor{currentstroke}{rgb}{0.500000,0.500000,0.500000}%
\pgfsetstrokecolor{currentstroke}%
\pgfsetstrokeopacity{0.300000}%
\pgfsetdash{}{0pt}%
\pgfpathmoveto{\pgfqpoint{4.141732in}{1.309698in}}%
\pgfusepath{stroke}%
\end{pgfscope}%
\begin{pgfscope}%
\pgfpathrectangle{\pgfqpoint{0.647939in}{0.492442in}}{\pgfqpoint{4.273799in}{2.331163in}}%
\pgfusepath{clip}%
\pgfsetroundcap%
\pgfsetroundjoin%
\definecolor{currentfill}{rgb}{0.500000,0.500000,0.500000}%
\pgfsetfillcolor{currentfill}%
\pgfsetfillopacity{0.300000}%
\pgfsetlinewidth{0.301125pt}%
\definecolor{currentstroke}{rgb}{0.500000,0.500000,0.500000}%
\pgfsetstrokecolor{currentstroke}%
\pgfsetstrokeopacity{0.300000}%
\pgfsetdash{}{0pt}%
\pgfpathmoveto{\pgfqpoint{0.000000in}{0.000000in}}%
\pgfpathlineto{\pgfqpoint{0.000000in}{0.000000in}}%
\pgfpathclose%
\pgfusepath{stroke,fill}%
\end{pgfscope}%
\begin{pgfscope}%
\pgfpathrectangle{\pgfqpoint{0.647939in}{0.492442in}}{\pgfqpoint{4.273799in}{2.331163in}}%
\pgfusepath{clip}%
\pgfsetroundcap%
\pgfsetroundjoin%
\pgfsetlinewidth{0.301125pt}%
\definecolor{currentstroke}{rgb}{0.500000,0.500000,0.500000}%
\pgfsetstrokecolor{currentstroke}%
\pgfsetstrokeopacity{0.300000}%
\pgfsetdash{}{0pt}%
\pgfpathmoveto{\pgfqpoint{4.138321in}{1.568299in}}%
\pgfusepath{stroke}%
\end{pgfscope}%
\begin{pgfscope}%
\pgfpathrectangle{\pgfqpoint{0.647939in}{0.492442in}}{\pgfqpoint{4.273799in}{2.331163in}}%
\pgfusepath{clip}%
\pgfsetroundcap%
\pgfsetroundjoin%
\definecolor{currentfill}{rgb}{0.500000,0.500000,0.500000}%
\pgfsetfillcolor{currentfill}%
\pgfsetfillopacity{0.300000}%
\pgfsetlinewidth{0.301125pt}%
\definecolor{currentstroke}{rgb}{0.500000,0.500000,0.500000}%
\pgfsetstrokecolor{currentstroke}%
\pgfsetstrokeopacity{0.300000}%
\pgfsetdash{}{0pt}%
\pgfpathmoveto{\pgfqpoint{0.000000in}{0.000000in}}%
\pgfpathlineto{\pgfqpoint{0.000000in}{0.000000in}}%
\pgfpathclose%
\pgfusepath{stroke,fill}%
\end{pgfscope}%
\begin{pgfscope}%
\pgfpathrectangle{\pgfqpoint{0.647939in}{0.492442in}}{\pgfqpoint{4.273799in}{2.331163in}}%
\pgfusepath{clip}%
\pgfsetroundcap%
\pgfsetroundjoin%
\pgfsetlinewidth{0.301125pt}%
\definecolor{currentstroke}{rgb}{0.500000,0.500000,0.500000}%
\pgfsetstrokecolor{currentstroke}%
\pgfsetstrokeopacity{0.300000}%
\pgfsetdash{}{0pt}%
\pgfpathmoveto{\pgfqpoint{1.967895in}{1.523726in}}%
\pgfusepath{stroke}%
\end{pgfscope}%
\begin{pgfscope}%
\pgfpathrectangle{\pgfqpoint{0.647939in}{0.492442in}}{\pgfqpoint{4.273799in}{2.331163in}}%
\pgfusepath{clip}%
\pgfsetroundcap%
\pgfsetroundjoin%
\definecolor{currentfill}{rgb}{0.500000,0.500000,0.500000}%
\pgfsetfillcolor{currentfill}%
\pgfsetfillopacity{0.300000}%
\pgfsetlinewidth{0.301125pt}%
\definecolor{currentstroke}{rgb}{0.500000,0.500000,0.500000}%
\pgfsetstrokecolor{currentstroke}%
\pgfsetstrokeopacity{0.300000}%
\pgfsetdash{}{0pt}%
\pgfpathmoveto{\pgfqpoint{0.000000in}{0.000000in}}%
\pgfpathlineto{\pgfqpoint{0.000000in}{0.000000in}}%
\pgfpathclose%
\pgfusepath{stroke,fill}%
\end{pgfscope}%
\begin{pgfscope}%
\pgfpathrectangle{\pgfqpoint{0.647939in}{0.492442in}}{\pgfqpoint{4.273799in}{2.331163in}}%
\pgfusepath{clip}%
\pgfsetroundcap%
\pgfsetroundjoin%
\pgfsetlinewidth{0.301125pt}%
\definecolor{currentstroke}{rgb}{0.500000,0.500000,0.500000}%
\pgfsetstrokecolor{currentstroke}%
\pgfsetstrokeopacity{0.300000}%
\pgfsetdash{}{0pt}%
\pgfpathmoveto{\pgfqpoint{4.160504in}{1.639571in}}%
\pgfusepath{stroke}%
\end{pgfscope}%
\begin{pgfscope}%
\pgfpathrectangle{\pgfqpoint{0.647939in}{0.492442in}}{\pgfqpoint{4.273799in}{2.331163in}}%
\pgfusepath{clip}%
\pgfsetroundcap%
\pgfsetroundjoin%
\definecolor{currentfill}{rgb}{0.500000,0.500000,0.500000}%
\pgfsetfillcolor{currentfill}%
\pgfsetfillopacity{0.300000}%
\pgfsetlinewidth{0.301125pt}%
\definecolor{currentstroke}{rgb}{0.500000,0.500000,0.500000}%
\pgfsetstrokecolor{currentstroke}%
\pgfsetstrokeopacity{0.300000}%
\pgfsetdash{}{0pt}%
\pgfpathmoveto{\pgfqpoint{0.000000in}{0.000000in}}%
\pgfpathlineto{\pgfqpoint{0.000000in}{0.000000in}}%
\pgfpathclose%
\pgfusepath{stroke,fill}%
\end{pgfscope}%
\begin{pgfscope}%
\pgfpathrectangle{\pgfqpoint{0.647939in}{0.492442in}}{\pgfqpoint{4.273799in}{2.331163in}}%
\pgfusepath{clip}%
\pgfsetroundcap%
\pgfsetroundjoin%
\pgfsetlinewidth{0.301125pt}%
\definecolor{currentstroke}{rgb}{0.500000,0.500000,0.500000}%
\pgfsetstrokecolor{currentstroke}%
\pgfsetstrokeopacity{0.300000}%
\pgfsetdash{}{0pt}%
\pgfpathmoveto{\pgfqpoint{1.473640in}{1.366891in}}%
\pgfusepath{stroke}%
\end{pgfscope}%
\begin{pgfscope}%
\pgfpathrectangle{\pgfqpoint{0.647939in}{0.492442in}}{\pgfqpoint{4.273799in}{2.331163in}}%
\pgfusepath{clip}%
\pgfsetroundcap%
\pgfsetroundjoin%
\definecolor{currentfill}{rgb}{0.500000,0.500000,0.500000}%
\pgfsetfillcolor{currentfill}%
\pgfsetfillopacity{0.300000}%
\pgfsetlinewidth{0.301125pt}%
\definecolor{currentstroke}{rgb}{0.500000,0.500000,0.500000}%
\pgfsetstrokecolor{currentstroke}%
\pgfsetstrokeopacity{0.300000}%
\pgfsetdash{}{0pt}%
\pgfpathmoveto{\pgfqpoint{0.000000in}{0.000000in}}%
\pgfpathlineto{\pgfqpoint{0.000000in}{0.000000in}}%
\pgfpathclose%
\pgfusepath{stroke,fill}%
\end{pgfscope}%
\begin{pgfscope}%
\pgfpathrectangle{\pgfqpoint{0.647939in}{0.492442in}}{\pgfqpoint{4.273799in}{2.331163in}}%
\pgfusepath{clip}%
\pgfsetroundcap%
\pgfsetroundjoin%
\pgfsetlinewidth{0.301125pt}%
\definecolor{currentstroke}{rgb}{0.500000,0.500000,0.500000}%
\pgfsetstrokecolor{currentstroke}%
\pgfsetstrokeopacity{0.300000}%
\pgfsetdash{}{0pt}%
\pgfpathmoveto{\pgfqpoint{3.572149in}{2.200618in}}%
\pgfusepath{stroke}%
\end{pgfscope}%
\begin{pgfscope}%
\pgfpathrectangle{\pgfqpoint{0.647939in}{0.492442in}}{\pgfqpoint{4.273799in}{2.331163in}}%
\pgfusepath{clip}%
\pgfsetroundcap%
\pgfsetroundjoin%
\definecolor{currentfill}{rgb}{0.500000,0.500000,0.500000}%
\pgfsetfillcolor{currentfill}%
\pgfsetfillopacity{0.300000}%
\pgfsetlinewidth{0.301125pt}%
\definecolor{currentstroke}{rgb}{0.500000,0.500000,0.500000}%
\pgfsetstrokecolor{currentstroke}%
\pgfsetstrokeopacity{0.300000}%
\pgfsetdash{}{0pt}%
\pgfpathmoveto{\pgfqpoint{0.000000in}{0.000000in}}%
\pgfpathlineto{\pgfqpoint{0.000000in}{0.000000in}}%
\pgfpathclose%
\pgfusepath{stroke,fill}%
\end{pgfscope}%
\begin{pgfscope}%
\pgfpathrectangle{\pgfqpoint{0.647939in}{0.492442in}}{\pgfqpoint{4.273799in}{2.331163in}}%
\pgfusepath{clip}%
\pgfsetroundcap%
\pgfsetroundjoin%
\pgfsetlinewidth{0.301125pt}%
\definecolor{currentstroke}{rgb}{0.500000,0.500000,0.500000}%
\pgfsetstrokecolor{currentstroke}%
\pgfsetstrokeopacity{0.300000}%
\pgfsetdash{}{0pt}%
\pgfpathmoveto{\pgfqpoint{1.640794in}{2.435231in}}%
\pgfusepath{stroke}%
\end{pgfscope}%
\begin{pgfscope}%
\pgfpathrectangle{\pgfqpoint{0.647939in}{0.492442in}}{\pgfqpoint{4.273799in}{2.331163in}}%
\pgfusepath{clip}%
\pgfsetroundcap%
\pgfsetroundjoin%
\definecolor{currentfill}{rgb}{0.500000,0.500000,0.500000}%
\pgfsetfillcolor{currentfill}%
\pgfsetfillopacity{0.300000}%
\pgfsetlinewidth{0.301125pt}%
\definecolor{currentstroke}{rgb}{0.500000,0.500000,0.500000}%
\pgfsetstrokecolor{currentstroke}%
\pgfsetstrokeopacity{0.300000}%
\pgfsetdash{}{0pt}%
\pgfpathmoveto{\pgfqpoint{0.000000in}{0.000000in}}%
\pgfpathlineto{\pgfqpoint{0.000000in}{0.000000in}}%
\pgfpathclose%
\pgfusepath{stroke,fill}%
\end{pgfscope}%
\begin{pgfscope}%
\pgfpathrectangle{\pgfqpoint{0.647939in}{0.492442in}}{\pgfqpoint{4.273799in}{2.331163in}}%
\pgfusepath{clip}%
\pgfsetroundcap%
\pgfsetroundjoin%
\pgfsetlinewidth{0.301125pt}%
\definecolor{currentstroke}{rgb}{0.500000,0.500000,0.500000}%
\pgfsetstrokecolor{currentstroke}%
\pgfsetstrokeopacity{0.300000}%
\pgfsetdash{}{0pt}%
\pgfpathmoveto{\pgfqpoint{1.419384in}{1.521769in}}%
\pgfusepath{stroke}%
\end{pgfscope}%
\begin{pgfscope}%
\pgfpathrectangle{\pgfqpoint{0.647939in}{0.492442in}}{\pgfqpoint{4.273799in}{2.331163in}}%
\pgfusepath{clip}%
\pgfsetroundcap%
\pgfsetroundjoin%
\definecolor{currentfill}{rgb}{0.500000,0.500000,0.500000}%
\pgfsetfillcolor{currentfill}%
\pgfsetfillopacity{0.300000}%
\pgfsetlinewidth{0.301125pt}%
\definecolor{currentstroke}{rgb}{0.500000,0.500000,0.500000}%
\pgfsetstrokecolor{currentstroke}%
\pgfsetstrokeopacity{0.300000}%
\pgfsetdash{}{0pt}%
\pgfpathmoveto{\pgfqpoint{0.000000in}{0.000000in}}%
\pgfpathlineto{\pgfqpoint{0.000000in}{0.000000in}}%
\pgfpathclose%
\pgfusepath{stroke,fill}%
\end{pgfscope}%
\begin{pgfscope}%
\pgfpathrectangle{\pgfqpoint{0.647939in}{0.492442in}}{\pgfqpoint{4.273799in}{2.331163in}}%
\pgfusepath{clip}%
\pgfsetroundcap%
\pgfsetroundjoin%
\pgfsetlinewidth{0.301125pt}%
\definecolor{currentstroke}{rgb}{0.500000,0.500000,0.500000}%
\pgfsetstrokecolor{currentstroke}%
\pgfsetstrokeopacity{0.300000}%
\pgfsetdash{}{0pt}%
\pgfpathmoveto{\pgfqpoint{4.024395in}{0.931604in}}%
\pgfusepath{stroke}%
\end{pgfscope}%
\begin{pgfscope}%
\pgfpathrectangle{\pgfqpoint{0.647939in}{0.492442in}}{\pgfqpoint{4.273799in}{2.331163in}}%
\pgfusepath{clip}%
\pgfsetroundcap%
\pgfsetroundjoin%
\definecolor{currentfill}{rgb}{0.500000,0.500000,0.500000}%
\pgfsetfillcolor{currentfill}%
\pgfsetfillopacity{0.300000}%
\pgfsetlinewidth{0.301125pt}%
\definecolor{currentstroke}{rgb}{0.500000,0.500000,0.500000}%
\pgfsetstrokecolor{currentstroke}%
\pgfsetstrokeopacity{0.300000}%
\pgfsetdash{}{0pt}%
\pgfpathmoveto{\pgfqpoint{0.000000in}{0.000000in}}%
\pgfpathlineto{\pgfqpoint{0.000000in}{0.000000in}}%
\pgfpathclose%
\pgfusepath{stroke,fill}%
\end{pgfscope}%
\begin{pgfscope}%
\pgfpathrectangle{\pgfqpoint{0.647939in}{0.492442in}}{\pgfqpoint{4.273799in}{2.331163in}}%
\pgfusepath{clip}%
\pgfsetroundcap%
\pgfsetroundjoin%
\pgfsetlinewidth{0.301125pt}%
\definecolor{currentstroke}{rgb}{0.500000,0.500000,0.500000}%
\pgfsetstrokecolor{currentstroke}%
\pgfsetstrokeopacity{0.300000}%
\pgfsetdash{}{0pt}%
\pgfpathmoveto{\pgfqpoint{4.061706in}{1.182948in}}%
\pgfusepath{stroke}%
\end{pgfscope}%
\begin{pgfscope}%
\pgfpathrectangle{\pgfqpoint{0.647939in}{0.492442in}}{\pgfqpoint{4.273799in}{2.331163in}}%
\pgfusepath{clip}%
\pgfsetroundcap%
\pgfsetroundjoin%
\definecolor{currentfill}{rgb}{0.500000,0.500000,0.500000}%
\pgfsetfillcolor{currentfill}%
\pgfsetfillopacity{0.300000}%
\pgfsetlinewidth{0.301125pt}%
\definecolor{currentstroke}{rgb}{0.500000,0.500000,0.500000}%
\pgfsetstrokecolor{currentstroke}%
\pgfsetstrokeopacity{0.300000}%
\pgfsetdash{}{0pt}%
\pgfpathmoveto{\pgfqpoint{0.000000in}{0.000000in}}%
\pgfpathlineto{\pgfqpoint{0.000000in}{0.000000in}}%
\pgfpathclose%
\pgfusepath{stroke,fill}%
\end{pgfscope}%
\begin{pgfscope}%
\pgfpathrectangle{\pgfqpoint{0.647939in}{0.492442in}}{\pgfqpoint{4.273799in}{2.331163in}}%
\pgfusepath{clip}%
\pgfsetroundcap%
\pgfsetroundjoin%
\pgfsetlinewidth{0.301125pt}%
\definecolor{currentstroke}{rgb}{0.500000,0.500000,0.500000}%
\pgfsetstrokecolor{currentstroke}%
\pgfsetstrokeopacity{0.300000}%
\pgfsetdash{}{0pt}%
\pgfpathmoveto{\pgfqpoint{3.973692in}{1.466490in}}%
\pgfusepath{stroke}%
\end{pgfscope}%
\begin{pgfscope}%
\pgfpathrectangle{\pgfqpoint{0.647939in}{0.492442in}}{\pgfqpoint{4.273799in}{2.331163in}}%
\pgfusepath{clip}%
\pgfsetroundcap%
\pgfsetroundjoin%
\definecolor{currentfill}{rgb}{0.500000,0.500000,0.500000}%
\pgfsetfillcolor{currentfill}%
\pgfsetfillopacity{0.300000}%
\pgfsetlinewidth{0.301125pt}%
\definecolor{currentstroke}{rgb}{0.500000,0.500000,0.500000}%
\pgfsetstrokecolor{currentstroke}%
\pgfsetstrokeopacity{0.300000}%
\pgfsetdash{}{0pt}%
\pgfpathmoveto{\pgfqpoint{0.000000in}{0.000000in}}%
\pgfpathlineto{\pgfqpoint{0.000000in}{0.000000in}}%
\pgfpathclose%
\pgfusepath{stroke,fill}%
\end{pgfscope}%
\begin{pgfscope}%
\pgfpathrectangle{\pgfqpoint{0.647939in}{0.492442in}}{\pgfqpoint{4.273799in}{2.331163in}}%
\pgfusepath{clip}%
\pgfsetroundcap%
\pgfsetroundjoin%
\pgfsetlinewidth{0.301125pt}%
\definecolor{currentstroke}{rgb}{0.500000,0.500000,0.500000}%
\pgfsetstrokecolor{currentstroke}%
\pgfsetstrokeopacity{0.300000}%
\pgfsetdash{}{0pt}%
\pgfpathmoveto{\pgfqpoint{2.133977in}{2.028509in}}%
\pgfusepath{stroke}%
\end{pgfscope}%
\begin{pgfscope}%
\pgfpathrectangle{\pgfqpoint{0.647939in}{0.492442in}}{\pgfqpoint{4.273799in}{2.331163in}}%
\pgfusepath{clip}%
\pgfsetroundcap%
\pgfsetroundjoin%
\definecolor{currentfill}{rgb}{0.500000,0.500000,0.500000}%
\pgfsetfillcolor{currentfill}%
\pgfsetfillopacity{0.300000}%
\pgfsetlinewidth{0.301125pt}%
\definecolor{currentstroke}{rgb}{0.500000,0.500000,0.500000}%
\pgfsetstrokecolor{currentstroke}%
\pgfsetstrokeopacity{0.300000}%
\pgfsetdash{}{0pt}%
\pgfpathmoveto{\pgfqpoint{0.000000in}{0.000000in}}%
\pgfpathlineto{\pgfqpoint{0.000000in}{0.000000in}}%
\pgfpathclose%
\pgfusepath{stroke,fill}%
\end{pgfscope}%
\begin{pgfscope}%
\pgfpathrectangle{\pgfqpoint{0.647939in}{0.492442in}}{\pgfqpoint{4.273799in}{2.331163in}}%
\pgfusepath{clip}%
\pgfsetroundcap%
\pgfsetroundjoin%
\pgfsetlinewidth{0.301125pt}%
\definecolor{currentstroke}{rgb}{0.500000,0.500000,0.500000}%
\pgfsetstrokecolor{currentstroke}%
\pgfsetstrokeopacity{0.300000}%
\pgfsetdash{}{0pt}%
\pgfpathmoveto{\pgfqpoint{1.642731in}{2.259783in}}%
\pgfusepath{stroke}%
\end{pgfscope}%
\begin{pgfscope}%
\pgfpathrectangle{\pgfqpoint{0.647939in}{0.492442in}}{\pgfqpoint{4.273799in}{2.331163in}}%
\pgfusepath{clip}%
\pgfsetroundcap%
\pgfsetroundjoin%
\definecolor{currentfill}{rgb}{0.500000,0.500000,0.500000}%
\pgfsetfillcolor{currentfill}%
\pgfsetfillopacity{0.300000}%
\pgfsetlinewidth{0.301125pt}%
\definecolor{currentstroke}{rgb}{0.500000,0.500000,0.500000}%
\pgfsetstrokecolor{currentstroke}%
\pgfsetstrokeopacity{0.300000}%
\pgfsetdash{}{0pt}%
\pgfpathmoveto{\pgfqpoint{0.000000in}{0.000000in}}%
\pgfpathlineto{\pgfqpoint{0.000000in}{0.000000in}}%
\pgfpathclose%
\pgfusepath{stroke,fill}%
\end{pgfscope}%
\begin{pgfscope}%
\pgfpathrectangle{\pgfqpoint{0.647939in}{0.492442in}}{\pgfqpoint{4.273799in}{2.331163in}}%
\pgfusepath{clip}%
\pgfsetroundcap%
\pgfsetroundjoin%
\pgfsetlinewidth{0.301125pt}%
\definecolor{currentstroke}{rgb}{0.500000,0.500000,0.500000}%
\pgfsetstrokecolor{currentstroke}%
\pgfsetstrokeopacity{0.300000}%
\pgfsetdash{}{0pt}%
\pgfpathmoveto{\pgfqpoint{1.453958in}{2.138443in}}%
\pgfusepath{stroke}%
\end{pgfscope}%
\begin{pgfscope}%
\pgfpathrectangle{\pgfqpoint{0.647939in}{0.492442in}}{\pgfqpoint{4.273799in}{2.331163in}}%
\pgfusepath{clip}%
\pgfsetroundcap%
\pgfsetroundjoin%
\definecolor{currentfill}{rgb}{0.500000,0.500000,0.500000}%
\pgfsetfillcolor{currentfill}%
\pgfsetfillopacity{0.300000}%
\pgfsetlinewidth{0.301125pt}%
\definecolor{currentstroke}{rgb}{0.500000,0.500000,0.500000}%
\pgfsetstrokecolor{currentstroke}%
\pgfsetstrokeopacity{0.300000}%
\pgfsetdash{}{0pt}%
\pgfpathmoveto{\pgfqpoint{0.000000in}{0.000000in}}%
\pgfpathlineto{\pgfqpoint{0.000000in}{0.000000in}}%
\pgfpathclose%
\pgfusepath{stroke,fill}%
\end{pgfscope}%
\begin{pgfscope}%
\pgfpathrectangle{\pgfqpoint{0.647939in}{0.492442in}}{\pgfqpoint{4.273799in}{2.331163in}}%
\pgfusepath{clip}%
\pgfsetroundcap%
\pgfsetroundjoin%
\pgfsetlinewidth{0.301125pt}%
\definecolor{currentstroke}{rgb}{0.500000,0.500000,0.500000}%
\pgfsetstrokecolor{currentstroke}%
\pgfsetstrokeopacity{0.300000}%
\pgfsetdash{}{0pt}%
\pgfpathmoveto{\pgfqpoint{1.398639in}{1.190041in}}%
\pgfusepath{stroke}%
\end{pgfscope}%
\begin{pgfscope}%
\pgfpathrectangle{\pgfqpoint{0.647939in}{0.492442in}}{\pgfqpoint{4.273799in}{2.331163in}}%
\pgfusepath{clip}%
\pgfsetroundcap%
\pgfsetroundjoin%
\definecolor{currentfill}{rgb}{0.500000,0.500000,0.500000}%
\pgfsetfillcolor{currentfill}%
\pgfsetfillopacity{0.300000}%
\pgfsetlinewidth{0.301125pt}%
\definecolor{currentstroke}{rgb}{0.500000,0.500000,0.500000}%
\pgfsetstrokecolor{currentstroke}%
\pgfsetstrokeopacity{0.300000}%
\pgfsetdash{}{0pt}%
\pgfpathmoveto{\pgfqpoint{0.000000in}{0.000000in}}%
\pgfpathlineto{\pgfqpoint{0.000000in}{0.000000in}}%
\pgfpathclose%
\pgfusepath{stroke,fill}%
\end{pgfscope}%
\begin{pgfscope}%
\pgfpathrectangle{\pgfqpoint{0.647939in}{0.492442in}}{\pgfqpoint{4.273799in}{2.331163in}}%
\pgfusepath{clip}%
\pgfsetroundcap%
\pgfsetroundjoin%
\pgfsetlinewidth{0.301125pt}%
\definecolor{currentstroke}{rgb}{0.500000,0.500000,0.500000}%
\pgfsetstrokecolor{currentstroke}%
\pgfsetstrokeopacity{0.300000}%
\pgfsetdash{}{0pt}%
\pgfpathmoveto{\pgfqpoint{3.412816in}{1.062083in}}%
\pgfusepath{stroke}%
\end{pgfscope}%
\begin{pgfscope}%
\pgfpathrectangle{\pgfqpoint{0.647939in}{0.492442in}}{\pgfqpoint{4.273799in}{2.331163in}}%
\pgfusepath{clip}%
\pgfsetroundcap%
\pgfsetroundjoin%
\definecolor{currentfill}{rgb}{0.500000,0.500000,0.500000}%
\pgfsetfillcolor{currentfill}%
\pgfsetfillopacity{0.300000}%
\pgfsetlinewidth{0.301125pt}%
\definecolor{currentstroke}{rgb}{0.500000,0.500000,0.500000}%
\pgfsetstrokecolor{currentstroke}%
\pgfsetstrokeopacity{0.300000}%
\pgfsetdash{}{0pt}%
\pgfpathmoveto{\pgfqpoint{0.000000in}{0.000000in}}%
\pgfpathlineto{\pgfqpoint{0.000000in}{0.000000in}}%
\pgfpathclose%
\pgfusepath{stroke,fill}%
\end{pgfscope}%
\begin{pgfscope}%
\pgfpathrectangle{\pgfqpoint{0.647939in}{0.492442in}}{\pgfqpoint{4.273799in}{2.331163in}}%
\pgfusepath{clip}%
\pgfsetroundcap%
\pgfsetroundjoin%
\pgfsetlinewidth{0.301125pt}%
\definecolor{currentstroke}{rgb}{0.500000,0.500000,0.500000}%
\pgfsetstrokecolor{currentstroke}%
\pgfsetstrokeopacity{0.300000}%
\pgfsetdash{}{0pt}%
\pgfpathmoveto{\pgfqpoint{3.798843in}{1.338960in}}%
\pgfusepath{stroke}%
\end{pgfscope}%
\begin{pgfscope}%
\pgfpathrectangle{\pgfqpoint{0.647939in}{0.492442in}}{\pgfqpoint{4.273799in}{2.331163in}}%
\pgfusepath{clip}%
\pgfsetroundcap%
\pgfsetroundjoin%
\definecolor{currentfill}{rgb}{0.500000,0.500000,0.500000}%
\pgfsetfillcolor{currentfill}%
\pgfsetfillopacity{0.300000}%
\pgfsetlinewidth{0.301125pt}%
\definecolor{currentstroke}{rgb}{0.500000,0.500000,0.500000}%
\pgfsetstrokecolor{currentstroke}%
\pgfsetstrokeopacity{0.300000}%
\pgfsetdash{}{0pt}%
\pgfpathmoveto{\pgfqpoint{0.000000in}{0.000000in}}%
\pgfpathlineto{\pgfqpoint{0.000000in}{0.000000in}}%
\pgfpathclose%
\pgfusepath{stroke,fill}%
\end{pgfscope}%
\begin{pgfscope}%
\pgfpathrectangle{\pgfqpoint{0.647939in}{0.492442in}}{\pgfqpoint{4.273799in}{2.331163in}}%
\pgfusepath{clip}%
\pgfsetroundcap%
\pgfsetroundjoin%
\pgfsetlinewidth{0.301125pt}%
\definecolor{currentstroke}{rgb}{0.500000,0.500000,0.500000}%
\pgfsetstrokecolor{currentstroke}%
\pgfsetstrokeopacity{0.300000}%
\pgfsetdash{}{0pt}%
\pgfpathmoveto{\pgfqpoint{1.632599in}{1.980655in}}%
\pgfusepath{stroke}%
\end{pgfscope}%
\begin{pgfscope}%
\pgfpathrectangle{\pgfqpoint{0.647939in}{0.492442in}}{\pgfqpoint{4.273799in}{2.331163in}}%
\pgfusepath{clip}%
\pgfsetroundcap%
\pgfsetroundjoin%
\definecolor{currentfill}{rgb}{0.500000,0.500000,0.500000}%
\pgfsetfillcolor{currentfill}%
\pgfsetfillopacity{0.300000}%
\pgfsetlinewidth{0.301125pt}%
\definecolor{currentstroke}{rgb}{0.500000,0.500000,0.500000}%
\pgfsetstrokecolor{currentstroke}%
\pgfsetstrokeopacity{0.300000}%
\pgfsetdash{}{0pt}%
\pgfpathmoveto{\pgfqpoint{0.000000in}{0.000000in}}%
\pgfpathlineto{\pgfqpoint{0.000000in}{0.000000in}}%
\pgfpathclose%
\pgfusepath{stroke,fill}%
\end{pgfscope}%
\begin{pgfscope}%
\pgfpathrectangle{\pgfqpoint{0.647939in}{0.492442in}}{\pgfqpoint{4.273799in}{2.331163in}}%
\pgfusepath{clip}%
\pgfsetroundcap%
\pgfsetroundjoin%
\pgfsetlinewidth{0.301125pt}%
\definecolor{currentstroke}{rgb}{0.500000,0.500000,0.500000}%
\pgfsetstrokecolor{currentstroke}%
\pgfsetstrokeopacity{0.300000}%
\pgfsetdash{}{0pt}%
\pgfpathmoveto{\pgfqpoint{1.794375in}{1.243383in}}%
\pgfusepath{stroke}%
\end{pgfscope}%
\begin{pgfscope}%
\pgfpathrectangle{\pgfqpoint{0.647939in}{0.492442in}}{\pgfqpoint{4.273799in}{2.331163in}}%
\pgfusepath{clip}%
\pgfsetroundcap%
\pgfsetroundjoin%
\definecolor{currentfill}{rgb}{0.500000,0.500000,0.500000}%
\pgfsetfillcolor{currentfill}%
\pgfsetfillopacity{0.300000}%
\pgfsetlinewidth{0.301125pt}%
\definecolor{currentstroke}{rgb}{0.500000,0.500000,0.500000}%
\pgfsetstrokecolor{currentstroke}%
\pgfsetstrokeopacity{0.300000}%
\pgfsetdash{}{0pt}%
\pgfpathmoveto{\pgfqpoint{0.000000in}{0.000000in}}%
\pgfpathlineto{\pgfqpoint{0.000000in}{0.000000in}}%
\pgfpathclose%
\pgfusepath{stroke,fill}%
\end{pgfscope}%
\begin{pgfscope}%
\pgfpathrectangle{\pgfqpoint{0.647939in}{0.492442in}}{\pgfqpoint{4.273799in}{2.331163in}}%
\pgfusepath{clip}%
\pgfsetroundcap%
\pgfsetroundjoin%
\pgfsetlinewidth{0.301125pt}%
\definecolor{currentstroke}{rgb}{0.500000,0.500000,0.500000}%
\pgfsetstrokecolor{currentstroke}%
\pgfsetstrokeopacity{0.300000}%
\pgfsetdash{}{0pt}%
\pgfpathmoveto{\pgfqpoint{3.775194in}{1.920610in}}%
\pgfusepath{stroke}%
\end{pgfscope}%
\begin{pgfscope}%
\pgfpathrectangle{\pgfqpoint{0.647939in}{0.492442in}}{\pgfqpoint{4.273799in}{2.331163in}}%
\pgfusepath{clip}%
\pgfsetroundcap%
\pgfsetroundjoin%
\definecolor{currentfill}{rgb}{0.500000,0.500000,0.500000}%
\pgfsetfillcolor{currentfill}%
\pgfsetfillopacity{0.300000}%
\pgfsetlinewidth{0.301125pt}%
\definecolor{currentstroke}{rgb}{0.500000,0.500000,0.500000}%
\pgfsetstrokecolor{currentstroke}%
\pgfsetstrokeopacity{0.300000}%
\pgfsetdash{}{0pt}%
\pgfpathmoveto{\pgfqpoint{0.000000in}{0.000000in}}%
\pgfpathlineto{\pgfqpoint{0.000000in}{0.000000in}}%
\pgfpathclose%
\pgfusepath{stroke,fill}%
\end{pgfscope}%
\begin{pgfscope}%
\pgfpathrectangle{\pgfqpoint{0.647939in}{0.492442in}}{\pgfqpoint{4.273799in}{2.331163in}}%
\pgfusepath{clip}%
\pgfsetroundcap%
\pgfsetroundjoin%
\pgfsetlinewidth{0.301125pt}%
\definecolor{currentstroke}{rgb}{0.500000,0.500000,0.500000}%
\pgfsetstrokecolor{currentstroke}%
\pgfsetstrokeopacity{0.300000}%
\pgfsetdash{}{0pt}%
\pgfpathmoveto{\pgfqpoint{3.289486in}{2.137097in}}%
\pgfusepath{stroke}%
\end{pgfscope}%
\begin{pgfscope}%
\pgfpathrectangle{\pgfqpoint{0.647939in}{0.492442in}}{\pgfqpoint{4.273799in}{2.331163in}}%
\pgfusepath{clip}%
\pgfsetroundcap%
\pgfsetroundjoin%
\definecolor{currentfill}{rgb}{0.500000,0.500000,0.500000}%
\pgfsetfillcolor{currentfill}%
\pgfsetfillopacity{0.300000}%
\pgfsetlinewidth{0.301125pt}%
\definecolor{currentstroke}{rgb}{0.500000,0.500000,0.500000}%
\pgfsetstrokecolor{currentstroke}%
\pgfsetstrokeopacity{0.300000}%
\pgfsetdash{}{0pt}%
\pgfpathmoveto{\pgfqpoint{0.000000in}{0.000000in}}%
\pgfpathlineto{\pgfqpoint{0.000000in}{0.000000in}}%
\pgfpathclose%
\pgfusepath{stroke,fill}%
\end{pgfscope}%
\begin{pgfscope}%
\pgfpathrectangle{\pgfqpoint{0.647939in}{0.492442in}}{\pgfqpoint{4.273799in}{2.331163in}}%
\pgfusepath{clip}%
\pgfsetroundcap%
\pgfsetroundjoin%
\pgfsetlinewidth{0.301125pt}%
\definecolor{currentstroke}{rgb}{0.500000,0.500000,0.500000}%
\pgfsetstrokecolor{currentstroke}%
\pgfsetstrokeopacity{0.300000}%
\pgfsetdash{}{0pt}%
\pgfpathmoveto{\pgfqpoint{3.607252in}{1.802135in}}%
\pgfusepath{stroke}%
\end{pgfscope}%
\begin{pgfscope}%
\pgfpathrectangle{\pgfqpoint{0.647939in}{0.492442in}}{\pgfqpoint{4.273799in}{2.331163in}}%
\pgfusepath{clip}%
\pgfsetroundcap%
\pgfsetroundjoin%
\definecolor{currentfill}{rgb}{0.500000,0.500000,0.500000}%
\pgfsetfillcolor{currentfill}%
\pgfsetfillopacity{0.300000}%
\pgfsetlinewidth{0.301125pt}%
\definecolor{currentstroke}{rgb}{0.500000,0.500000,0.500000}%
\pgfsetstrokecolor{currentstroke}%
\pgfsetstrokeopacity{0.300000}%
\pgfsetdash{}{0pt}%
\pgfpathmoveto{\pgfqpoint{0.000000in}{0.000000in}}%
\pgfpathlineto{\pgfqpoint{0.000000in}{0.000000in}}%
\pgfpathclose%
\pgfusepath{stroke,fill}%
\end{pgfscope}%
\begin{pgfscope}%
\pgfpathrectangle{\pgfqpoint{0.647939in}{0.492442in}}{\pgfqpoint{4.273799in}{2.331163in}}%
\pgfusepath{clip}%
\pgfsetroundcap%
\pgfsetroundjoin%
\pgfsetlinewidth{0.301125pt}%
\definecolor{currentstroke}{rgb}{0.500000,0.500000,0.500000}%
\pgfsetstrokecolor{currentstroke}%
\pgfsetstrokeopacity{0.300000}%
\pgfsetdash{}{0pt}%
\pgfpathmoveto{\pgfqpoint{2.259643in}{1.414499in}}%
\pgfusepath{stroke}%
\end{pgfscope}%
\begin{pgfscope}%
\pgfpathrectangle{\pgfqpoint{0.647939in}{0.492442in}}{\pgfqpoint{4.273799in}{2.331163in}}%
\pgfusepath{clip}%
\pgfsetroundcap%
\pgfsetroundjoin%
\definecolor{currentfill}{rgb}{0.500000,0.500000,0.500000}%
\pgfsetfillcolor{currentfill}%
\pgfsetfillopacity{0.300000}%
\pgfsetlinewidth{0.301125pt}%
\definecolor{currentstroke}{rgb}{0.500000,0.500000,0.500000}%
\pgfsetstrokecolor{currentstroke}%
\pgfsetstrokeopacity{0.300000}%
\pgfsetdash{}{0pt}%
\pgfpathmoveto{\pgfqpoint{0.000000in}{0.000000in}}%
\pgfpathlineto{\pgfqpoint{0.000000in}{0.000000in}}%
\pgfpathclose%
\pgfusepath{stroke,fill}%
\end{pgfscope}%
\begin{pgfscope}%
\pgfpathrectangle{\pgfqpoint{0.647939in}{0.492442in}}{\pgfqpoint{4.273799in}{2.331163in}}%
\pgfusepath{clip}%
\pgfsetroundcap%
\pgfsetroundjoin%
\pgfsetlinewidth{0.301125pt}%
\definecolor{currentstroke}{rgb}{0.500000,0.500000,0.500000}%
\pgfsetstrokecolor{currentstroke}%
\pgfsetstrokeopacity{0.300000}%
\pgfsetdash{}{0pt}%
\pgfpathmoveto{\pgfqpoint{2.667004in}{1.640323in}}%
\pgfusepath{stroke}%
\end{pgfscope}%
\begin{pgfscope}%
\pgfpathrectangle{\pgfqpoint{0.647939in}{0.492442in}}{\pgfqpoint{4.273799in}{2.331163in}}%
\pgfusepath{clip}%
\pgfsetroundcap%
\pgfsetroundjoin%
\definecolor{currentfill}{rgb}{0.500000,0.500000,0.500000}%
\pgfsetfillcolor{currentfill}%
\pgfsetfillopacity{0.300000}%
\pgfsetlinewidth{0.301125pt}%
\definecolor{currentstroke}{rgb}{0.500000,0.500000,0.500000}%
\pgfsetstrokecolor{currentstroke}%
\pgfsetstrokeopacity{0.300000}%
\pgfsetdash{}{0pt}%
\pgfpathmoveto{\pgfqpoint{0.000000in}{0.000000in}}%
\pgfpathlineto{\pgfqpoint{0.000000in}{0.000000in}}%
\pgfpathclose%
\pgfusepath{stroke,fill}%
\end{pgfscope}%
\begin{pgfscope}%
\pgfpathrectangle{\pgfqpoint{0.647939in}{0.492442in}}{\pgfqpoint{4.273799in}{2.331163in}}%
\pgfusepath{clip}%
\pgfsetbuttcap%
\pgfsetroundjoin%
\pgfsetlinewidth{0.301125pt}%
\definecolor{currentstroke}{rgb}{0.500000,0.500000,0.500000}%
\pgfsetstrokecolor{currentstroke}%
\pgfsetstrokeopacity{0.300000}%
\pgfsetdash{}{0pt}%
\pgfpathmoveto{\pgfqpoint{2.237947in}{0.492442in}}%
\pgfpathlineto{\pgfqpoint{2.226745in}{0.508161in}}%
\pgfpathlineto{\pgfqpoint{2.192486in}{0.556477in}}%
\pgfpathlineto{\pgfqpoint{2.158510in}{0.604852in}}%
\pgfpathlineto{\pgfqpoint{2.124783in}{0.653279in}}%
\pgfpathlineto{\pgfqpoint{2.091269in}{0.701750in}}%
\pgfpathlineto{\pgfqpoint{2.057932in}{0.750258in}}%
\pgfpathlineto{\pgfqpoint{2.024739in}{0.798794in}}%
\pgfpathlineto{\pgfqpoint{1.991641in}{0.847350in}}%
\pgfpathlineto{\pgfqpoint{1.958580in}{0.895914in}}%
\pgfpathlineto{\pgfqpoint{1.925492in}{0.944471in}}%
\pgfpathlineto{\pgfqpoint{1.892307in}{0.993009in}}%
\pgfpathlineto{\pgfqpoint{1.858937in}{1.041510in}}%
\pgfpathlineto{\pgfqpoint{1.825256in}{1.089945in}}%
\pgfpathlineto{\pgfqpoint{1.791104in}{1.138282in}}%
\pgfpathlineto{\pgfqpoint{1.756285in}{1.186476in}}%
\pgfpathlineto{\pgfqpoint{1.720519in}{1.234462in}}%
\pgfpathlineto{\pgfqpoint{1.683396in}{1.282140in}}%
\pgfpathlineto{\pgfqpoint{1.644254in}{1.329332in}}%
\pgfpathlineto{\pgfqpoint{1.601978in}{1.375697in}}%
\pgfpathlineto{\pgfqpoint{1.554458in}{1.420486in}}%
\pgfpathlineto{\pgfqpoint{1.512353in}{1.452195in}}%
\pgfpathlineto{\pgfqpoint{1.475240in}{1.472763in}}%
\pgfpathlineto{\pgfqpoint{1.439009in}{1.485359in}}%
\pgfpathlineto{\pgfqpoint{1.393915in}{1.489999in}}%
\pgfpathlineto{\pgfqpoint{1.349860in}{1.482665in}}%
\pgfpathlineto{\pgfqpoint{1.349860in}{1.482665in}}%
\pgfpathlineto{\pgfqpoint{1.298503in}{1.460008in}}%
\pgfpathlineto{\pgfqpoint{1.298503in}{1.460008in}}%
\pgfpathlineto{\pgfqpoint{1.241242in}{1.418988in}}%
\pgfpathlineto{\pgfqpoint{1.193600in}{1.374347in}}%
\pgfpathlineto{\pgfqpoint{1.151324in}{1.328054in}}%
\pgfpathlineto{\pgfqpoint{1.112507in}{1.280831in}}%
\pgfpathlineto{\pgfqpoint{1.076125in}{1.233007in}}%
\pgfpathlineto{\pgfqpoint{1.041592in}{1.184765in}}%
\pgfpathlineto{\pgfqpoint{1.008538in}{1.136220in}}%
\pgfpathlineto{\pgfqpoint{0.976694in}{1.087437in}}%
\pgfpathlineto{\pgfqpoint{0.945844in}{1.038456in}}%
\pgfpathlineto{\pgfqpoint{0.915857in}{0.989311in}}%
\pgfpathlineto{\pgfqpoint{0.886639in}{0.940031in}}%
\pgfpathlineto{\pgfqpoint{0.858093in}{0.890632in}}%
\pgfpathlineto{\pgfqpoint{0.830152in}{0.841128in}}%
\pgfpathlineto{\pgfqpoint{0.802770in}{0.791532in}}%
\pgfpathlineto{\pgfqpoint{0.775890in}{0.741852in}}%
\pgfpathlineto{\pgfqpoint{0.749478in}{0.692097in}}%
\pgfpathlineto{\pgfqpoint{0.723508in}{0.642274in}}%
\pgfpathlineto{\pgfqpoint{0.697943in}{0.592387in}}%
\pgfpathlineto{\pgfqpoint{0.672763in}{0.542442in}}%
\pgfpathlineto{\pgfqpoint{0.647939in}{0.492442in}}%
\pgfpathlineto{\pgfqpoint{0.647939in}{0.492442in}}%
\pgfusepath{stroke}%
\end{pgfscope}%
\begin{pgfscope}%
\pgfpathrectangle{\pgfqpoint{0.647939in}{0.492442in}}{\pgfqpoint{4.273799in}{2.331163in}}%
\pgfusepath{clip}%
\pgfsetbuttcap%
\pgfsetroundjoin%
\pgfsetlinewidth{0.301125pt}%
\definecolor{currentstroke}{rgb}{0.500000,0.500000,0.500000}%
\pgfsetstrokecolor{currentstroke}%
\pgfsetstrokeopacity{0.300000}%
\pgfsetdash{}{0pt}%
\pgfpathmoveto{\pgfqpoint{1.781061in}{0.492442in}}%
\pgfpathlineto{\pgfqpoint{1.748636in}{0.531198in}}%
\pgfpathlineto{\pgfqpoint{1.708682in}{0.578191in}}%
\pgfpathlineto{\pgfqpoint{1.667501in}{0.624867in}}%
\pgfpathlineto{\pgfqpoint{1.624707in}{0.671106in}}%
\pgfpathlineto{\pgfqpoint{1.579746in}{0.716724in}}%
\pgfpathlineto{\pgfqpoint{1.531798in}{0.761416in}}%
\pgfpathlineto{\pgfqpoint{1.479568in}{0.804629in}}%
\pgfpathlineto{\pgfqpoint{1.421145in}{0.845052in}}%
\pgfpathlineto{\pgfqpoint{1.368928in}{0.873130in}}%
\pgfpathlineto{\pgfqpoint{1.320224in}{0.891637in}}%
\pgfpathlineto{\pgfqpoint{1.270391in}{0.902256in}}%
\pgfpathlineto{\pgfqpoint{1.211530in}{0.903331in}}%
\pgfpathlineto{\pgfqpoint{1.155740in}{0.892550in}}%
\pgfpathlineto{\pgfqpoint{1.155740in}{0.892550in}}%
\pgfpathlineto{\pgfqpoint{1.082352in}{0.860656in}}%
\pgfpathlineto{\pgfqpoint{1.022595in}{0.820611in}}%
\pgfpathlineto{\pgfqpoint{0.971662in}{0.777018in}}%
\pgfpathlineto{\pgfqpoint{0.926480in}{0.731526in}}%
\pgfpathlineto{\pgfqpoint{0.885345in}{0.684880in}}%
\pgfpathlineto{\pgfqpoint{0.847219in}{0.637466in}}%
\pgfpathlineto{\pgfqpoint{0.811419in}{0.589504in}}%
\pgfpathlineto{\pgfqpoint{0.777474in}{0.541130in}}%
\pgfpathlineto{\pgfqpoint{0.745071in}{0.492442in}}%
\pgfpathlineto{\pgfqpoint{0.745071in}{0.492442in}}%
\pgfusepath{stroke}%
\end{pgfscope}%
\begin{pgfscope}%
\pgfpathrectangle{\pgfqpoint{0.647939in}{0.492442in}}{\pgfqpoint{4.273799in}{2.331163in}}%
\pgfusepath{clip}%
\pgfsetbuttcap%
\pgfsetroundjoin%
\pgfsetlinewidth{0.301125pt}%
\definecolor{currentstroke}{rgb}{0.500000,0.500000,0.500000}%
\pgfsetstrokecolor{currentstroke}%
\pgfsetstrokeopacity{0.300000}%
\pgfsetdash{}{0pt}%
\pgfpathmoveto{\pgfqpoint{1.553259in}{0.492442in}}%
\pgfpathlineto{\pgfqpoint{1.508986in}{0.530768in}}%
\pgfpathlineto{\pgfqpoint{1.456104in}{0.573759in}}%
\pgfpathlineto{\pgfqpoint{1.397484in}{0.614409in}}%
\pgfpathlineto{\pgfqpoint{1.340554in}{0.645906in}}%
\pgfpathlineto{\pgfqpoint{1.288032in}{0.667182in}}%
\pgfpathlineto{\pgfqpoint{1.236090in}{0.680199in}}%
\pgfpathlineto{\pgfqpoint{1.178652in}{0.684662in}}%
\pgfpathlineto{\pgfqpoint{1.121521in}{0.678351in}}%
\pgfpathlineto{\pgfqpoint{1.121521in}{0.678351in}}%
\pgfpathlineto{\pgfqpoint{1.060593in}{0.659460in}}%
\pgfpathlineto{\pgfqpoint{1.060593in}{0.659460in}}%
\pgfpathlineto{\pgfqpoint{0.992703in}{0.623719in}}%
\pgfpathlineto{\pgfqpoint{0.935914in}{0.582352in}}%
\pgfpathlineto{\pgfqpoint{0.886518in}{0.538203in}}%
\pgfpathlineto{\pgfqpoint{0.842203in}{0.492442in}}%
\pgfpathlineto{\pgfqpoint{0.842203in}{0.492442in}}%
\pgfusepath{stroke}%
\end{pgfscope}%
\begin{pgfscope}%
\pgfpathrectangle{\pgfqpoint{0.647939in}{0.492442in}}{\pgfqpoint{4.273799in}{2.331163in}}%
\pgfusepath{clip}%
\pgfsetbuttcap%
\pgfsetroundjoin%
\pgfsetlinewidth{0.301125pt}%
\definecolor{currentstroke}{rgb}{0.500000,0.500000,0.500000}%
\pgfsetstrokecolor{currentstroke}%
\pgfsetstrokeopacity{0.300000}%
\pgfsetdash{}{0pt}%
\pgfpathmoveto{\pgfqpoint{1.387912in}{0.492442in}}%
\pgfpathlineto{\pgfqpoint{1.354997in}{0.510897in}}%
\pgfpathlineto{\pgfqpoint{1.298503in}{0.538451in}}%
\pgfpathlineto{\pgfqpoint{1.245293in}{0.556628in}}%
\pgfpathlineto{\pgfqpoint{1.190601in}{0.566795in}}%
\pgfpathlineto{\pgfqpoint{1.127733in}{0.567158in}}%
\pgfpathlineto{\pgfqpoint{1.068038in}{0.555895in}}%
\pgfpathlineto{\pgfqpoint{1.068038in}{0.555895in}}%
\pgfpathlineto{\pgfqpoint{1.003146in}{0.530526in}}%
\pgfpathlineto{\pgfqpoint{1.003146in}{0.530526in}}%
\pgfpathlineto{\pgfqpoint{0.939334in}{0.492442in}}%
\pgfpathlineto{\pgfqpoint{0.939334in}{0.492442in}}%
\pgfusepath{stroke}%
\end{pgfscope}%
\begin{pgfscope}%
\pgfpathrectangle{\pgfqpoint{0.647939in}{0.492442in}}{\pgfqpoint{4.273799in}{2.331163in}}%
\pgfusepath{clip}%
\pgfsetbuttcap%
\pgfsetroundjoin%
\pgfsetlinewidth{0.301125pt}%
\definecolor{currentstroke}{rgb}{0.500000,0.500000,0.500000}%
\pgfsetstrokecolor{currentstroke}%
\pgfsetstrokeopacity{0.300000}%
\pgfsetdash{}{0pt}%
\pgfpathmoveto{\pgfqpoint{1.619257in}{0.492442in}}%
\pgfpathlineto{\pgfqpoint{1.619257in}{0.492442in}}%
\pgfpathlineto{\pgfqpoint{1.573957in}{0.537963in}}%
\pgfpathlineto{\pgfqpoint{1.525973in}{0.582650in}}%
\pgfpathlineto{\pgfqpoint{1.474268in}{0.626072in}}%
\pgfpathlineto{\pgfqpoint{1.417176in}{0.667402in}}%
\pgfpathlineto{\pgfqpoint{1.351895in}{0.704787in}}%
\pgfpathlineto{\pgfqpoint{1.274231in}{0.733555in}}%
\pgfpathlineto{\pgfqpoint{1.274231in}{0.733555in}}%
\pgfpathlineto{\pgfqpoint{1.217520in}{0.742724in}}%
\pgfpathlineto{\pgfqpoint{1.157761in}{0.740757in}}%
\pgfusepath{stroke}%
\end{pgfscope}%
\begin{pgfscope}%
\pgfpathrectangle{\pgfqpoint{0.647939in}{0.492442in}}{\pgfqpoint{4.273799in}{2.331163in}}%
\pgfusepath{clip}%
\pgfsetbuttcap%
\pgfsetroundjoin%
\pgfsetlinewidth{0.301125pt}%
\definecolor{currentstroke}{rgb}{0.500000,0.500000,0.500000}%
\pgfsetstrokecolor{currentstroke}%
\pgfsetstrokeopacity{0.300000}%
\pgfsetdash{}{0pt}%
\pgfpathmoveto{\pgfqpoint{1.910652in}{0.492442in}}%
\pgfpathlineto{\pgfqpoint{1.910652in}{0.492442in}}%
\pgfpathlineto{\pgfqpoint{1.874181in}{0.540273in}}%
\pgfpathlineto{\pgfqpoint{1.837331in}{0.588018in}}%
\pgfpathlineto{\pgfqpoint{1.799971in}{0.635644in}}%
\pgfpathlineto{\pgfqpoint{1.761937in}{0.683110in}}%
\pgfpathlineto{\pgfqpoint{1.723013in}{0.730362in}}%
\pgfpathlineto{\pgfqpoint{1.682914in}{0.777319in}}%
\pgfpathlineto{\pgfqpoint{1.641240in}{0.823866in}}%
\pgfpathlineto{\pgfqpoint{1.597420in}{0.869819in}}%
\pgfpathlineto{\pgfqpoint{1.550584in}{0.914871in}}%
\pgfpathlineto{\pgfqpoint{1.499314in}{0.958431in}}%
\pgfpathlineto{\pgfqpoint{1.441138in}{0.999227in}}%
\pgfpathlineto{\pgfqpoint{1.371514in}{1.033936in}}%
\pgfpathlineto{\pgfqpoint{1.371514in}{1.033936in}}%
\pgfpathlineto{\pgfqpoint{1.317522in}{1.048901in}}%
\pgfpathlineto{\pgfqpoint{1.317522in}{1.048901in}}%
\pgfpathlineto{\pgfqpoint{1.267192in}{1.052638in}}%
\pgfpathlineto{\pgfqpoint{1.217384in}{1.046333in}}%
\pgfpathlineto{\pgfqpoint{1.173320in}{1.032483in}}%
\pgfpathlineto{\pgfqpoint{1.128862in}{1.010758in}}%
\pgfpathlineto{\pgfqpoint{1.080975in}{0.978980in}}%
\pgfpathlineto{\pgfqpoint{1.029400in}{0.935611in}}%
\pgfpathlineto{\pgfqpoint{0.983996in}{0.890182in}}%
\pgfpathlineto{\pgfqpoint{0.942795in}{0.843548in}}%
\pgfpathlineto{\pgfqpoint{0.904646in}{0.796128in}}%
\pgfpathlineto{\pgfqpoint{0.868837in}{0.748159in}}%
\pgfusepath{stroke}%
\end{pgfscope}%
\begin{pgfscope}%
\pgfpathrectangle{\pgfqpoint{0.647939in}{0.492442in}}{\pgfqpoint{4.273799in}{2.331163in}}%
\pgfusepath{clip}%
\pgfsetbuttcap%
\pgfsetroundjoin%
\pgfsetlinewidth{0.301125pt}%
\definecolor{currentstroke}{rgb}{0.500000,0.500000,0.500000}%
\pgfsetstrokecolor{currentstroke}%
\pgfsetstrokeopacity{0.300000}%
\pgfsetdash{}{0pt}%
\pgfpathmoveto{\pgfqpoint{2.007784in}{0.492442in}}%
\pgfpathlineto{\pgfqpoint{2.007784in}{0.492442in}}%
\pgfpathlineto{\pgfqpoint{1.972503in}{0.540538in}}%
\pgfpathlineto{\pgfqpoint{1.937123in}{0.588613in}}%
\pgfpathlineto{\pgfqpoint{1.901567in}{0.636649in}}%
\pgfpathlineto{\pgfqpoint{1.865735in}{0.684624in}}%
\pgfpathlineto{\pgfqpoint{1.829507in}{0.732510in}}%
\pgfpathlineto{\pgfqpoint{1.792741in}{0.780273in}}%
\pgfpathlineto{\pgfqpoint{1.755255in}{0.827869in}}%
\pgfpathlineto{\pgfqpoint{1.716806in}{0.875234in}}%
\pgfpathlineto{\pgfqpoint{1.677062in}{0.922280in}}%
\pgfpathlineto{\pgfqpoint{1.635551in}{0.968866in}}%
\pgfpathlineto{\pgfqpoint{1.591558in}{1.014765in}}%
\pgfpathlineto{\pgfqpoint{1.543941in}{1.059556in}}%
\pgfpathlineto{\pgfqpoint{1.490719in}{1.102369in}}%
\pgfpathlineto{\pgfqpoint{1.428132in}{1.141037in}}%
\pgfpathlineto{\pgfqpoint{1.428132in}{1.141037in}}%
\pgfpathlineto{\pgfqpoint{1.372241in}{1.163292in}}%
\pgfpathlineto{\pgfqpoint{1.372241in}{1.163292in}}%
\pgfpathlineto{\pgfqpoint{1.322611in}{1.172223in}}%
\pgfpathlineto{\pgfqpoint{1.270079in}{1.169961in}}%
\pgfpathlineto{\pgfqpoint{1.226843in}{1.159120in}}%
\pgfpathlineto{\pgfqpoint{1.184922in}{1.140980in}}%
\pgfpathlineto{\pgfqpoint{1.140160in}{1.113658in}}%
\pgfpathlineto{\pgfqpoint{1.090488in}{1.074359in}}%
\pgfpathlineto{\pgfqpoint{1.043643in}{1.029395in}}%
\pgfusepath{stroke}%
\end{pgfscope}%
\begin{pgfscope}%
\pgfpathrectangle{\pgfqpoint{0.647939in}{0.492442in}}{\pgfqpoint{4.273799in}{2.331163in}}%
\pgfusepath{clip}%
\pgfsetbuttcap%
\pgfsetroundjoin%
\pgfsetlinewidth{0.301125pt}%
\definecolor{currentstroke}{rgb}{0.500000,0.500000,0.500000}%
\pgfsetstrokecolor{currentstroke}%
\pgfsetstrokeopacity{0.300000}%
\pgfsetdash{}{0pt}%
\pgfpathmoveto{\pgfqpoint{2.104916in}{0.492442in}}%
\pgfpathlineto{\pgfqpoint{2.104916in}{0.492442in}}%
\pgfpathlineto{\pgfqpoint{2.070312in}{0.540685in}}%
\pgfpathlineto{\pgfqpoint{2.035802in}{0.588947in}}%
\pgfpathlineto{\pgfqpoint{2.001332in}{0.637218in}}%
\pgfpathlineto{\pgfqpoint{1.966845in}{0.685486in}}%
\pgfpathlineto{\pgfqpoint{1.932280in}{0.733736in}}%
\pgfpathlineto{\pgfqpoint{1.897560in}{0.781954in}}%
\pgfpathlineto{\pgfqpoint{1.862587in}{0.830117in}}%
\pgfpathlineto{\pgfqpoint{1.827234in}{0.878197in}}%
\pgfpathlineto{\pgfqpoint{1.791352in}{0.926160in}}%
\pgfpathlineto{\pgfqpoint{1.754746in}{0.973959in}}%
\pgfpathlineto{\pgfqpoint{1.717157in}{1.021529in}}%
\pgfpathlineto{\pgfqpoint{1.678215in}{1.068773in}}%
\pgfpathlineto{\pgfqpoint{1.637379in}{1.115536in}}%
\pgfpathlineto{\pgfqpoint{1.593792in}{1.161548in}}%
\pgfpathlineto{\pgfqpoint{1.546006in}{1.206278in}}%
\pgfpathlineto{\pgfqpoint{1.491305in}{1.248476in}}%
\pgfpathlineto{\pgfqpoint{1.424085in}{1.284434in}}%
\pgfpathlineto{\pgfqpoint{1.424085in}{1.284434in}}%
\pgfpathlineto{\pgfqpoint{1.375925in}{1.297961in}}%
\pgfpathlineto{\pgfqpoint{1.375925in}{1.297961in}}%
\pgfpathlineto{\pgfqpoint{1.330741in}{1.300481in}}%
\pgfpathlineto{\pgfqpoint{1.287073in}{1.293444in}}%
\pgfpathlineto{\pgfqpoint{1.247815in}{1.279374in}}%
\pgfpathlineto{\pgfqpoint{1.206807in}{1.257180in}}%
\pgfpathlineto{\pgfqpoint{1.161564in}{1.224410in}}%
\pgfpathlineto{\pgfqpoint{1.112667in}{1.180136in}}%
\pgfusepath{stroke}%
\end{pgfscope}%
\begin{pgfscope}%
\pgfpathrectangle{\pgfqpoint{0.647939in}{0.492442in}}{\pgfqpoint{4.273799in}{2.331163in}}%
\pgfusepath{clip}%
\pgfsetbuttcap%
\pgfsetroundjoin%
\pgfsetlinewidth{0.301125pt}%
\definecolor{currentstroke}{rgb}{0.500000,0.500000,0.500000}%
\pgfsetstrokecolor{currentstroke}%
\pgfsetstrokeopacity{0.300000}%
\pgfsetdash{}{0pt}%
\pgfpathmoveto{\pgfqpoint{2.396312in}{0.492442in}}%
\pgfpathlineto{\pgfqpoint{2.396312in}{0.492442in}}%
\pgfpathlineto{\pgfqpoint{2.361425in}{0.540624in}}%
\pgfpathlineto{\pgfqpoint{2.327007in}{0.588906in}}%
\pgfpathlineto{\pgfqpoint{2.293039in}{0.637283in}}%
\pgfpathlineto{\pgfqpoint{2.259497in}{0.685748in}}%
\pgfpathlineto{\pgfqpoint{2.226361in}{0.734296in}}%
\pgfpathlineto{\pgfqpoint{2.193614in}{0.782923in}}%
\pgfpathlineto{\pgfqpoint{2.161239in}{0.831623in}}%
\pgfpathlineto{\pgfqpoint{2.129208in}{0.880392in}}%
\pgfpathlineto{\pgfqpoint{2.097496in}{0.929222in}}%
\pgfpathlineto{\pgfqpoint{2.066085in}{0.978110in}}%
\pgfpathlineto{\pgfqpoint{2.034949in}{1.027051in}}%
\pgfpathlineto{\pgfqpoint{2.004054in}{1.076036in}}%
\pgfpathlineto{\pgfqpoint{1.973366in}{1.125061in}}%
\pgfpathlineto{\pgfqpoint{1.942857in}{1.174119in}}%
\pgfpathlineto{\pgfqpoint{1.912477in}{1.223200in}}%
\pgfpathlineto{\pgfqpoint{1.882163in}{1.272294in}}%
\pgfpathlineto{\pgfqpoint{1.851860in}{1.321389in}}%
\pgfpathlineto{\pgfqpoint{1.821480in}{1.370470in}}%
\pgfpathlineto{\pgfqpoint{1.790896in}{1.419512in}}%
\pgfpathlineto{\pgfqpoint{1.759948in}{1.468485in}}%
\pgfpathlineto{\pgfqpoint{1.728410in}{1.517346in}}%
\pgfpathlineto{\pgfqpoint{1.695884in}{1.566012in}}%
\pgfpathlineto{\pgfqpoint{1.661691in}{1.614324in}}%
\pgfpathlineto{\pgfqpoint{1.624534in}{1.661955in}}%
\pgfpathlineto{\pgfqpoint{1.581240in}{1.707894in}}%
\pgfpathlineto{\pgfqpoint{1.581240in}{1.707894in}}%
\pgfpathlineto{\pgfqpoint{1.538614in}{1.738818in}}%
\pgfpathlineto{\pgfqpoint{1.538614in}{1.738818in}}%
\pgfpathlineto{\pgfqpoint{1.506401in}{1.750350in}}%
\pgfpathlineto{\pgfqpoint{1.506401in}{1.750350in}}%
\pgfpathlineto{\pgfqpoint{1.476536in}{1.750955in}}%
\pgfpathlineto{\pgfqpoint{1.449358in}{1.743667in}}%
\pgfpathlineto{\pgfqpoint{1.422729in}{1.730168in}}%
\pgfpathlineto{\pgfqpoint{1.390845in}{1.707207in}}%
\pgfpathlineto{\pgfqpoint{1.352143in}{1.671547in}}%
\pgfpathlineto{\pgfqpoint{1.309739in}{1.625408in}}%
\pgfpathlineto{\pgfqpoint{1.270791in}{1.578286in}}%
\pgfpathlineto{\pgfqpoint{1.234068in}{1.530597in}}%
\pgfpathlineto{\pgfqpoint{1.198939in}{1.482520in}}%
\pgfpathlineto{\pgfqpoint{1.165034in}{1.434151in}}%
\pgfpathlineto{\pgfqpoint{1.132167in}{1.385563in}}%
\pgfusepath{stroke}%
\end{pgfscope}%
\begin{pgfscope}%
\pgfpathrectangle{\pgfqpoint{0.647939in}{0.492442in}}{\pgfqpoint{4.273799in}{2.331163in}}%
\pgfusepath{clip}%
\pgfsetbuttcap%
\pgfsetroundjoin%
\pgfsetlinewidth{0.301125pt}%
\definecolor{currentstroke}{rgb}{0.500000,0.500000,0.500000}%
\pgfsetstrokecolor{currentstroke}%
\pgfsetstrokeopacity{0.300000}%
\pgfsetdash{}{0pt}%
\pgfpathmoveto{\pgfqpoint{2.493443in}{0.492442in}}%
\pgfpathlineto{\pgfqpoint{2.493443in}{0.492442in}}%
\pgfpathlineto{\pgfqpoint{2.457845in}{0.540469in}}%
\pgfpathlineto{\pgfqpoint{2.422809in}{0.588618in}}%
\pgfpathlineto{\pgfqpoint{2.388318in}{0.636885in}}%
\pgfpathlineto{\pgfqpoint{2.354354in}{0.685262in}}%
\pgfpathlineto{\pgfqpoint{2.320905in}{0.733746in}}%
\pgfpathlineto{\pgfqpoint{2.287958in}{0.782333in}}%
\pgfpathlineto{\pgfqpoint{2.255496in}{0.831016in}}%
\pgfpathlineto{\pgfqpoint{2.223500in}{0.879791in}}%
\pgfpathlineto{\pgfqpoint{2.191959in}{0.928654in}}%
\pgfpathlineto{\pgfqpoint{2.160866in}{0.977603in}}%
\pgfpathlineto{\pgfqpoint{2.130203in}{1.026632in}}%
\pgfpathlineto{\pgfqpoint{2.099953in}{1.075737in}}%
\pgfpathlineto{\pgfqpoint{2.070113in}{1.124917in}}%
\pgfpathlineto{\pgfqpoint{2.040671in}{1.174169in}}%
\pgfpathlineto{\pgfqpoint{2.011608in}{1.223487in}}%
\pgfpathlineto{\pgfqpoint{1.982924in}{1.272871in}}%
\pgfpathlineto{\pgfqpoint{1.954609in}{1.322319in}}%
\pgfpathlineto{\pgfqpoint{1.926648in}{1.371826in}}%
\pgfpathlineto{\pgfqpoint{1.899044in}{1.421392in}}%
\pgfpathlineto{\pgfqpoint{1.871788in}{1.471016in}}%
\pgfpathlineto{\pgfqpoint{1.844867in}{1.520693in}}%
\pgfpathlineto{\pgfqpoint{1.818293in}{1.570427in}}%
\pgfpathlineto{\pgfqpoint{1.792060in}{1.620213in}}%
\pgfpathlineto{\pgfqpoint{1.766179in}{1.670053in}}%
\pgfpathlineto{\pgfqpoint{1.740669in}{1.719950in}}%
\pgfpathlineto{\pgfqpoint{1.715556in}{1.769902in}}%
\pgfpathlineto{\pgfqpoint{1.690930in}{1.819924in}}%
\pgfpathlineto{\pgfqpoint{1.666948in}{1.870026in}}%
\pgfpathlineto{\pgfqpoint{1.644101in}{1.920280in}}%
\pgfpathlineto{\pgfqpoint{1.624459in}{1.970795in}}%
\pgfpathlineto{\pgfqpoint{1.624459in}{1.970795in}}%
\pgfpathlineto{\pgfqpoint{1.618682in}{1.998105in}}%
\pgfpathlineto{\pgfqpoint{1.618682in}{1.998105in}}%
\pgfpathlineto{\pgfqpoint{1.621582in}{2.017290in}}%
\pgfpathlineto{\pgfqpoint{1.633146in}{2.037389in}}%
\pgfpathlineto{\pgfqpoint{1.654158in}{2.062640in}}%
\pgfpathlineto{\pgfqpoint{1.689610in}{2.099169in}}%
\pgfpathlineto{\pgfqpoint{1.736824in}{2.143705in}}%
\pgfpathlineto{\pgfqpoint{1.786698in}{2.187531in}}%
\pgfpathlineto{\pgfqpoint{1.839202in}{2.230492in}}%
\pgfpathlineto{\pgfqpoint{1.894542in}{2.272414in}}%
\pgfpathlineto{\pgfqpoint{1.953159in}{2.313043in}}%
\pgfpathlineto{\pgfqpoint{2.015608in}{2.351973in}}%
\pgfpathlineto{\pgfqpoint{2.082394in}{2.388695in}}%
\pgfpathlineto{\pgfqpoint{2.154209in}{2.422456in}}%
\pgfpathlineto{\pgfqpoint{2.231746in}{2.452157in}}%
\pgfpathlineto{\pgfqpoint{2.315424in}{2.476232in}}%
\pgfpathlineto{\pgfqpoint{2.404927in}{2.492678in}}%
\pgfpathlineto{\pgfqpoint{2.498387in}{2.499354in}}%
\pgfpathlineto{\pgfqpoint{2.585660in}{2.495668in}}%
\pgfpathlineto{\pgfqpoint{2.667228in}{2.483136in}}%
\pgfpathlineto{\pgfqpoint{2.745254in}{2.462437in}}%
\pgfpathlineto{\pgfqpoint{2.821495in}{2.433364in}}%
\pgfpathlineto{\pgfqpoint{2.891195in}{2.398410in}}%
\pgfpathlineto{\pgfqpoint{2.953530in}{2.359466in}}%
\pgfpathlineto{\pgfqpoint{3.009075in}{2.317559in}}%
\pgfpathlineto{\pgfqpoint{3.058352in}{2.273351in}}%
\pgfpathlineto{\pgfqpoint{3.101632in}{2.227295in}}%
\pgfpathlineto{\pgfqpoint{3.138910in}{2.179704in}}%
\pgfpathlineto{\pgfqpoint{3.169807in}{2.130780in}}%
\pgfpathlineto{\pgfqpoint{3.193328in}{2.080661in}}%
\pgfpathlineto{\pgfqpoint{3.207245in}{2.029530in}}%
\pgfpathlineto{\pgfqpoint{3.205853in}{1.978039in}}%
\pgfpathlineto{\pgfqpoint{3.205853in}{1.978039in}}%
\pgfpathlineto{\pgfqpoint{3.192047in}{1.949555in}}%
\pgfpathlineto{\pgfqpoint{3.192047in}{1.949555in}}%
\pgfpathlineto{\pgfqpoint{3.171142in}{1.933773in}}%
\pgfpathlineto{\pgfqpoint{3.171142in}{1.933773in}}%
\pgfpathlineto{\pgfqpoint{3.145549in}{1.928481in}}%
\pgfpathlineto{\pgfqpoint{3.119150in}{1.932118in}}%
\pgfpathlineto{\pgfqpoint{3.096060in}{1.941501in}}%
\pgfpathlineto{\pgfqpoint{3.071655in}{1.958700in}}%
\pgfpathlineto{\pgfqpoint{3.051559in}{1.983591in}}%
\pgfpathlineto{\pgfqpoint{3.051559in}{1.983591in}}%
\pgfpathlineto{\pgfqpoint{3.046207in}{2.005259in}}%
\pgfpathlineto{\pgfqpoint{3.046207in}{2.005259in}}%
\pgfpathlineto{\pgfqpoint{3.051704in}{2.012763in}}%
\pgfpathlineto{\pgfqpoint{3.060699in}{2.011997in}}%
\pgfpathlineto{\pgfqpoint{3.068084in}{2.005002in}}%
\pgfusepath{stroke}%
\end{pgfscope}%
\begin{pgfscope}%
\pgfpathrectangle{\pgfqpoint{0.647939in}{0.492442in}}{\pgfqpoint{4.273799in}{2.331163in}}%
\pgfusepath{clip}%
\pgfsetbuttcap%
\pgfsetroundjoin%
\pgfsetlinewidth{0.301125pt}%
\definecolor{currentstroke}{rgb}{0.500000,0.500000,0.500000}%
\pgfsetstrokecolor{currentstroke}%
\pgfsetstrokeopacity{0.300000}%
\pgfsetdash{}{0pt}%
\pgfpathmoveto{\pgfqpoint{2.590575in}{0.492442in}}%
\pgfpathlineto{\pgfqpoint{2.590575in}{0.492442in}}%
\pgfpathlineto{\pgfqpoint{2.553988in}{0.540247in}}%
\pgfpathlineto{\pgfqpoint{2.518052in}{0.588198in}}%
\pgfpathlineto{\pgfqpoint{2.482752in}{0.636290in}}%
\pgfpathlineto{\pgfqpoint{2.448076in}{0.684517in}}%
\pgfpathlineto{\pgfqpoint{2.414013in}{0.732874in}}%
\pgfpathlineto{\pgfqpoint{2.380548in}{0.781355in}}%
\pgfusepath{stroke}%
\end{pgfscope}%
\begin{pgfscope}%
\pgfpathrectangle{\pgfqpoint{0.647939in}{0.492442in}}{\pgfqpoint{4.273799in}{2.331163in}}%
\pgfusepath{clip}%
\pgfsetbuttcap%
\pgfsetroundjoin%
\pgfsetlinewidth{0.301125pt}%
\definecolor{currentstroke}{rgb}{0.500000,0.500000,0.500000}%
\pgfsetstrokecolor{currentstroke}%
\pgfsetstrokeopacity{0.300000}%
\pgfsetdash{}{0pt}%
\pgfpathmoveto{\pgfqpoint{2.784839in}{0.492442in}}%
\pgfpathlineto{\pgfqpoint{2.784839in}{0.492442in}}%
\pgfpathlineto{\pgfqpoint{2.745492in}{0.539590in}}%
\pgfpathlineto{\pgfqpoint{2.706974in}{0.586940in}}%
\pgfpathlineto{\pgfqpoint{2.669273in}{0.634487in}}%
\pgfpathlineto{\pgfqpoint{2.632377in}{0.682221in}}%
\pgfpathlineto{\pgfqpoint{2.596273in}{0.730135in}}%
\pgfpathlineto{\pgfqpoint{2.560948in}{0.778221in}}%
\pgfpathlineto{\pgfqpoint{2.526391in}{0.826473in}}%
\pgfpathlineto{\pgfqpoint{2.492595in}{0.874885in}}%
\pgfpathlineto{\pgfqpoint{2.459557in}{0.923452in}}%
\pgfpathlineto{\pgfqpoint{2.427268in}{0.972169in}}%
\pgfpathlineto{\pgfqpoint{2.395724in}{1.021032in}}%
\pgfpathlineto{\pgfqpoint{2.364927in}{1.070035in}}%
\pgfpathlineto{\pgfqpoint{2.334887in}{1.119179in}}%
\pgfpathlineto{\pgfqpoint{2.305606in}{1.168458in}}%
\pgfpathlineto{\pgfqpoint{2.277099in}{1.217873in}}%
\pgfpathlineto{\pgfqpoint{2.249388in}{1.267422in}}%
\pgfpathlineto{\pgfqpoint{2.222497in}{1.317105in}}%
\pgfpathlineto{\pgfqpoint{2.196461in}{1.366923in}}%
\pgfpathlineto{\pgfqpoint{2.171326in}{1.416879in}}%
\pgfpathlineto{\pgfqpoint{2.147147in}{1.466975in}}%
\pgfpathlineto{\pgfqpoint{2.124000in}{1.517215in}}%
\pgfpathlineto{\pgfqpoint{2.101978in}{1.567605in}}%
\pgfpathlineto{\pgfqpoint{2.081199in}{1.618152in}}%
\pgfpathlineto{\pgfqpoint{2.061815in}{1.668864in}}%
\pgfpathlineto{\pgfqpoint{2.044015in}{1.719747in}}%
\pgfpathlineto{\pgfqpoint{2.028051in}{1.770810in}}%
\pgfpathlineto{\pgfqpoint{2.014236in}{1.822058in}}%
\pgfpathlineto{\pgfqpoint{2.002969in}{1.873489in}}%
\pgfpathlineto{\pgfqpoint{1.994775in}{1.925090in}}%
\pgfpathlineto{\pgfqpoint{1.990306in}{1.976823in}}%
\pgfpathlineto{\pgfqpoint{1.990364in}{2.028607in}}%
\pgfpathlineto{\pgfqpoint{1.995896in}{2.080292in}}%
\pgfpathlineto{\pgfqpoint{2.007945in}{2.131632in}}%
\pgfpathlineto{\pgfqpoint{2.027517in}{2.182262in}}%
\pgfpathlineto{\pgfqpoint{2.055469in}{2.231694in}}%
\pgfpathlineto{\pgfqpoint{2.092405in}{2.279326in}}%
\pgfpathlineto{\pgfqpoint{2.138644in}{2.324454in}}%
\pgfpathlineto{\pgfqpoint{2.194367in}{2.366229in}}%
\pgfpathlineto{\pgfqpoint{2.259750in}{2.403542in}}%
\pgfusepath{stroke}%
\end{pgfscope}%
\begin{pgfscope}%
\pgfpathrectangle{\pgfqpoint{0.647939in}{0.492442in}}{\pgfqpoint{4.273799in}{2.331163in}}%
\pgfusepath{clip}%
\pgfsetbuttcap%
\pgfsetroundjoin%
\pgfsetlinewidth{0.301125pt}%
\definecolor{currentstroke}{rgb}{0.500000,0.500000,0.500000}%
\pgfsetstrokecolor{currentstroke}%
\pgfsetstrokeopacity{0.300000}%
\pgfsetdash{}{0pt}%
\pgfpathmoveto{\pgfqpoint{2.881971in}{0.492442in}}%
\pgfpathlineto{\pgfqpoint{2.881971in}{0.492442in}}%
\pgfpathlineto{\pgfqpoint{2.840882in}{0.539145in}}%
\pgfpathlineto{\pgfqpoint{2.800707in}{0.586085in}}%
\pgfpathlineto{\pgfqpoint{2.761437in}{0.633251in}}%
\pgfpathlineto{\pgfqpoint{2.723059in}{0.680636in}}%
\pgfpathlineto{\pgfqpoint{2.685561in}{0.728230in}}%
\pgfpathlineto{\pgfqpoint{2.648932in}{0.776025in}}%
\pgfpathlineto{\pgfqpoint{2.613162in}{0.824013in}}%
\pgfpathlineto{\pgfqpoint{2.578242in}{0.872187in}}%
\pgfpathlineto{\pgfqpoint{2.544167in}{0.920541in}}%
\pgfpathlineto{\pgfqpoint{2.510932in}{0.969068in}}%
\pgfpathlineto{\pgfqpoint{2.478530in}{1.017763in}}%
\pgfpathlineto{\pgfqpoint{2.446963in}{1.066620in}}%
\pgfpathlineto{\pgfqpoint{2.416239in}{1.115638in}}%
\pgfpathlineto{\pgfqpoint{2.386365in}{1.164811in}}%
\pgfpathlineto{\pgfqpoint{2.357350in}{1.214137in}}%
\pgfpathlineto{\pgfqpoint{2.329219in}{1.263615in}}%
\pgfpathlineto{\pgfqpoint{2.301998in}{1.313245in}}%
\pgfpathlineto{\pgfqpoint{2.275718in}{1.363025in}}%
\pgfusepath{stroke}%
\end{pgfscope}%
\begin{pgfscope}%
\pgfpathrectangle{\pgfqpoint{0.647939in}{0.492442in}}{\pgfqpoint{4.273799in}{2.331163in}}%
\pgfusepath{clip}%
\pgfsetbuttcap%
\pgfsetroundjoin%
\pgfsetlinewidth{0.301125pt}%
\definecolor{currentstroke}{rgb}{0.500000,0.500000,0.500000}%
\pgfsetstrokecolor{currentstroke}%
\pgfsetstrokeopacity{0.300000}%
\pgfsetdash{}{0pt}%
\pgfpathmoveto{\pgfqpoint{2.979102in}{0.492442in}}%
\pgfpathlineto{\pgfqpoint{2.979102in}{0.492442in}}%
\pgfpathlineto{\pgfqpoint{2.936056in}{0.538618in}}%
\pgfpathlineto{\pgfqpoint{2.894001in}{0.585065in}}%
\pgfpathlineto{\pgfqpoint{2.852933in}{0.631774in}}%
\pgfpathlineto{\pgfqpoint{2.812842in}{0.678735in}}%
\pgfpathlineto{\pgfqpoint{2.773721in}{0.725938in}}%
\pgfusepath{stroke}%
\end{pgfscope}%
\begin{pgfscope}%
\pgfpathrectangle{\pgfqpoint{0.647939in}{0.492442in}}{\pgfqpoint{4.273799in}{2.331163in}}%
\pgfusepath{clip}%
\pgfsetbuttcap%
\pgfsetroundjoin%
\pgfsetlinewidth{0.301125pt}%
\definecolor{currentstroke}{rgb}{0.500000,0.500000,0.500000}%
\pgfsetstrokecolor{currentstroke}%
\pgfsetstrokeopacity{0.300000}%
\pgfsetdash{}{0pt}%
\pgfpathmoveto{\pgfqpoint{3.173366in}{0.492442in}}%
\pgfpathlineto{\pgfqpoint{3.173366in}{0.492442in}}%
\pgfpathlineto{\pgfqpoint{3.125917in}{0.537315in}}%
\pgfpathlineto{\pgfqpoint{3.079559in}{0.582526in}}%
\pgfpathlineto{\pgfqpoint{3.034314in}{0.628070in}}%
\pgfpathlineto{\pgfqpoint{2.990192in}{0.673942in}}%
\pgfpathlineto{\pgfqpoint{2.947199in}{0.720132in}}%
\pgfpathlineto{\pgfqpoint{2.905334in}{0.766629in}}%
\pgfpathlineto{\pgfqpoint{2.864594in}{0.813422in}}%
\pgfpathlineto{\pgfqpoint{2.824973in}{0.860500in}}%
\pgfpathlineto{\pgfqpoint{2.786464in}{0.907852in}}%
\pgfpathlineto{\pgfqpoint{2.749061in}{0.955467in}}%
\pgfpathlineto{\pgfqpoint{2.712758in}{1.003335in}}%
\pgfpathlineto{\pgfqpoint{2.677555in}{1.051447in}}%
\pgfpathlineto{\pgfqpoint{2.643446in}{1.099794in}}%
\pgfpathlineto{\pgfqpoint{2.610435in}{1.148366in}}%
\pgfpathlineto{\pgfqpoint{2.578529in}{1.197157in}}%
\pgfpathlineto{\pgfqpoint{2.547746in}{1.246163in}}%
\pgfpathlineto{\pgfqpoint{2.518103in}{1.295377in}}%
\pgfpathlineto{\pgfqpoint{2.489624in}{1.344795in}}%
\pgfpathlineto{\pgfqpoint{2.462353in}{1.394416in}}%
\pgfpathlineto{\pgfqpoint{2.436335in}{1.444236in}}%
\pgfpathlineto{\pgfqpoint{2.411629in}{1.494255in}}%
\pgfpathlineto{\pgfqpoint{2.388318in}{1.544472in}}%
\pgfpathlineto{\pgfqpoint{2.366495in}{1.594887in}}%
\pgfpathlineto{\pgfqpoint{2.346282in}{1.645501in}}%
\pgfpathlineto{\pgfqpoint{2.327832in}{1.696315in}}%
\pgfpathlineto{\pgfqpoint{2.311328in}{1.747327in}}%
\pgfpathlineto{\pgfqpoint{2.297000in}{1.798533in}}%
\pgfpathlineto{\pgfqpoint{2.285139in}{1.849925in}}%
\pgfpathlineto{\pgfqpoint{2.276101in}{1.901486in}}%
\pgfpathlineto{\pgfqpoint{2.270332in}{1.953185in}}%
\pgfpathlineto{\pgfqpoint{2.268395in}{2.004964in}}%
\pgfpathlineto{\pgfqpoint{2.270998in}{2.056727in}}%
\pgfpathlineto{\pgfqpoint{2.279025in}{2.108315in}}%
\pgfpathlineto{\pgfqpoint{2.293585in}{2.159462in}}%
\pgfpathlineto{\pgfqpoint{2.316099in}{2.209724in}}%
\pgfpathlineto{\pgfqpoint{2.348326in}{2.258350in}}%
\pgfpathlineto{\pgfqpoint{2.392443in}{2.304040in}}%
\pgfpathlineto{\pgfqpoint{2.450873in}{2.344466in}}%
\pgfpathlineto{\pgfqpoint{2.522117in}{2.374753in}}%
\pgfpathlineto{\pgfqpoint{2.593901in}{2.390420in}}%
\pgfpathlineto{\pgfqpoint{2.662293in}{2.394143in}}%
\pgfpathlineto{\pgfqpoint{2.727426in}{2.388616in}}%
\pgfpathlineto{\pgfqpoint{2.791499in}{2.374937in}}%
\pgfpathlineto{\pgfqpoint{2.855432in}{2.353103in}}%
\pgfpathlineto{\pgfqpoint{2.919889in}{2.322366in}}%
\pgfusepath{stroke}%
\end{pgfscope}%
\begin{pgfscope}%
\pgfpathrectangle{\pgfqpoint{0.647939in}{0.492442in}}{\pgfqpoint{4.273799in}{2.331163in}}%
\pgfusepath{clip}%
\pgfsetbuttcap%
\pgfsetroundjoin%
\pgfsetlinewidth{0.301125pt}%
\definecolor{currentstroke}{rgb}{0.500000,0.500000,0.500000}%
\pgfsetstrokecolor{currentstroke}%
\pgfsetstrokeopacity{0.300000}%
\pgfsetdash{}{0pt}%
\pgfpathmoveto{\pgfqpoint{3.367630in}{0.492442in}}%
\pgfpathlineto{\pgfqpoint{3.367630in}{0.492442in}}%
\pgfpathlineto{\pgfqpoint{3.315593in}{0.535776in}}%
\pgfpathlineto{\pgfqpoint{3.264587in}{0.579472in}}%
\pgfpathlineto{\pgfqpoint{3.214688in}{0.623547in}}%
\pgfpathlineto{\pgfqpoint{3.165957in}{0.668008in}}%
\pgfpathlineto{\pgfqpoint{3.118439in}{0.712859in}}%
\pgfpathlineto{\pgfqpoint{3.072166in}{0.758095in}}%
\pgfpathlineto{\pgfqpoint{3.027158in}{0.803710in}}%
\pgfpathlineto{\pgfqpoint{2.983427in}{0.849693in}}%
\pgfpathlineto{\pgfqpoint{2.940980in}{0.896032in}}%
\pgfpathlineto{\pgfqpoint{2.899816in}{0.942715in}}%
\pgfpathlineto{\pgfqpoint{2.859932in}{0.989727in}}%
\pgfpathlineto{\pgfqpoint{2.821326in}{1.037055in}}%
\pgfpathlineto{\pgfqpoint{2.783993in}{1.084685in}}%
\pgfpathlineto{\pgfqpoint{2.747934in}{1.132608in}}%
\pgfpathlineto{\pgfqpoint{2.713151in}{1.180810in}}%
\pgfpathlineto{\pgfqpoint{2.679653in}{1.229282in}}%
\pgfpathlineto{\pgfqpoint{2.647450in}{1.278015in}}%
\pgfpathlineto{\pgfqpoint{2.616560in}{1.327000in}}%
\pgfpathlineto{\pgfqpoint{2.587015in}{1.376231in}}%
\pgfpathlineto{\pgfqpoint{2.558854in}{1.425703in}}%
\pgfpathlineto{\pgfqpoint{2.532125in}{1.475410in}}%
\pgfpathlineto{\pgfqpoint{2.506900in}{1.525351in}}%
\pgfpathlineto{\pgfqpoint{2.483260in}{1.575522in}}%
\pgfpathlineto{\pgfqpoint{2.461313in}{1.625921in}}%
\pgfpathlineto{\pgfqpoint{2.441194in}{1.676545in}}%
\pgfpathlineto{\pgfqpoint{2.423068in}{1.727393in}}%
\pgfpathlineto{\pgfqpoint{2.407147in}{1.778459in}}%
\pgfpathlineto{\pgfqpoint{2.393691in}{1.829734in}}%
\pgfpathlineto{\pgfqpoint{2.383035in}{1.881203in}}%
\pgfpathlineto{\pgfqpoint{2.375599in}{1.932838in}}%
\pgfpathlineto{\pgfqpoint{2.371917in}{1.984589in}}%
\pgfpathlineto{\pgfqpoint{2.372680in}{2.036371in}}%
\pgfpathlineto{\pgfqpoint{2.378790in}{2.088038in}}%
\pgfpathlineto{\pgfqpoint{2.391457in}{2.139337in}}%
\pgfpathlineto{\pgfqpoint{2.412309in}{2.189808in}}%
\pgfusepath{stroke}%
\end{pgfscope}%
\begin{pgfscope}%
\pgfpathrectangle{\pgfqpoint{0.647939in}{0.492442in}}{\pgfqpoint{4.273799in}{2.331163in}}%
\pgfusepath{clip}%
\pgfsetbuttcap%
\pgfsetroundjoin%
\pgfsetlinewidth{0.301125pt}%
\definecolor{currentstroke}{rgb}{0.500000,0.500000,0.500000}%
\pgfsetstrokecolor{currentstroke}%
\pgfsetstrokeopacity{0.300000}%
\pgfsetdash{}{0pt}%
\pgfpathmoveto{\pgfqpoint{3.561893in}{0.492442in}}%
\pgfpathlineto{\pgfqpoint{3.561893in}{0.492442in}}%
\pgfpathlineto{\pgfqpoint{3.506018in}{0.534331in}}%
\pgfpathlineto{\pgfqpoint{3.450785in}{0.576472in}}%
\pgfpathlineto{\pgfqpoint{3.396380in}{0.618931in}}%
\pgfpathlineto{\pgfqpoint{3.342958in}{0.661760in}}%
\pgfpathlineto{\pgfqpoint{3.290644in}{0.704993in}}%
\pgfpathlineto{\pgfqpoint{3.239538in}{0.748655in}}%
\pgfpathlineto{\pgfqpoint{3.189725in}{0.792758in}}%
\pgfpathlineto{\pgfqpoint{3.141269in}{0.837309in}}%
\pgfpathlineto{\pgfqpoint{3.094211in}{0.882302in}}%
\pgfpathlineto{\pgfqpoint{3.048579in}{0.927730in}}%
\pgfpathlineto{\pgfqpoint{3.004389in}{0.973581in}}%
\pgfpathlineto{\pgfqpoint{2.961651in}{1.019839in}}%
\pgfpathlineto{\pgfqpoint{2.920369in}{1.066489in}}%
\pgfpathlineto{\pgfqpoint{2.880544in}{1.113514in}}%
\pgfpathlineto{\pgfqpoint{2.842176in}{1.160899in}}%
\pgfpathlineto{\pgfqpoint{2.805270in}{1.208629in}}%
\pgfpathlineto{\pgfqpoint{2.769830in}{1.256689in}}%
\pgfusepath{stroke}%
\end{pgfscope}%
\begin{pgfscope}%
\pgfpathrectangle{\pgfqpoint{0.647939in}{0.492442in}}{\pgfqpoint{4.273799in}{2.331163in}}%
\pgfusepath{clip}%
\pgfsetbuttcap%
\pgfsetroundjoin%
\pgfsetlinewidth{0.301125pt}%
\definecolor{currentstroke}{rgb}{0.500000,0.500000,0.500000}%
\pgfsetstrokecolor{currentstroke}%
\pgfsetstrokeopacity{0.300000}%
\pgfsetdash{}{0pt}%
\pgfpathmoveto{\pgfqpoint{3.756157in}{0.492442in}}%
\pgfpathlineto{\pgfqpoint{3.756157in}{0.492442in}}%
\pgfpathlineto{\pgfqpoint{3.698393in}{0.533562in}}%
\pgfpathlineto{\pgfqpoint{3.640510in}{0.574633in}}%
\pgfpathlineto{\pgfqpoint{3.582763in}{0.615760in}}%
\pgfpathlineto{\pgfqpoint{3.525420in}{0.657054in}}%
\pgfpathlineto{\pgfqpoint{3.468735in}{0.698617in}}%
\pgfpathlineto{\pgfqpoint{3.412913in}{0.740526in}}%
\pgfpathlineto{\pgfqpoint{3.358152in}{0.782848in}}%
\pgfpathlineto{\pgfqpoint{3.304618in}{0.825634in}}%
\pgfpathlineto{\pgfqpoint{3.252434in}{0.868912in}}%
\pgfpathlineto{\pgfqpoint{3.201699in}{0.912700in}}%
\pgfpathlineto{\pgfqpoint{3.152485in}{0.957003in}}%
\pgfpathlineto{\pgfqpoint{3.104849in}{1.001815in}}%
\pgfpathlineto{\pgfqpoint{3.058826in}{1.047126in}}%
\pgfpathlineto{\pgfqpoint{3.014437in}{1.092919in}}%
\pgfpathlineto{\pgfqpoint{2.971694in}{1.139175in}}%
\pgfpathlineto{\pgfqpoint{2.930605in}{1.185876in}}%
\pgfpathlineto{\pgfqpoint{2.891174in}{1.233000in}}%
\pgfpathlineto{\pgfqpoint{2.853408in}{1.280529in}}%
\pgfpathlineto{\pgfqpoint{2.817317in}{1.328443in}}%
\pgfpathlineto{\pgfqpoint{2.782918in}{1.376725in}}%
\pgfpathlineto{\pgfqpoint{2.750236in}{1.425361in}}%
\pgfpathlineto{\pgfqpoint{2.719314in}{1.474338in}}%
\pgfpathlineto{\pgfqpoint{2.690204in}{1.523645in}}%
\pgfpathlineto{\pgfqpoint{2.662976in}{1.573270in}}%
\pgfpathlineto{\pgfqpoint{2.637732in}{1.623206in}}%
\pgfpathlineto{\pgfqpoint{2.614598in}{1.673445in}}%
\pgfpathlineto{\pgfqpoint{2.593739in}{1.723978in}}%
\pgfpathlineto{\pgfqpoint{2.575367in}{1.774797in}}%
\pgfpathlineto{\pgfqpoint{2.559757in}{1.825888in}}%
\pgfpathlineto{\pgfqpoint{2.547272in}{1.877233in}}%
\pgfpathlineto{\pgfqpoint{2.538388in}{1.928798in}}%
\pgfpathlineto{\pgfqpoint{2.533741in}{1.980523in}}%
\pgfpathlineto{\pgfqpoint{2.534215in}{2.032299in}}%
\pgfpathlineto{\pgfqpoint{2.541065in}{2.083924in}}%
\pgfpathlineto{\pgfqpoint{2.556158in}{2.134993in}}%
\pgfpathlineto{\pgfqpoint{2.582449in}{2.184622in}}%
\pgfpathlineto{\pgfqpoint{2.624724in}{2.230616in}}%
\pgfpathlineto{\pgfqpoint{2.624724in}{2.230616in}}%
\pgfpathlineto{\pgfqpoint{2.669027in}{2.258748in}}%
\pgfpathlineto{\pgfqpoint{2.669027in}{2.258748in}}%
\pgfpathlineto{\pgfqpoint{2.716949in}{2.275577in}}%
\pgfpathlineto{\pgfqpoint{2.772658in}{2.282568in}}%
\pgfpathlineto{\pgfqpoint{2.824437in}{2.279502in}}%
\pgfpathlineto{\pgfqpoint{2.874750in}{2.268698in}}%
\pgfpathlineto{\pgfqpoint{2.926056in}{2.250012in}}%
\pgfpathlineto{\pgfqpoint{2.978524in}{2.222540in}}%
\pgfpathlineto{\pgfqpoint{3.030925in}{2.185333in}}%
\pgfpathlineto{\pgfqpoint{3.078069in}{2.140593in}}%
\pgfpathlineto{\pgfqpoint{3.114745in}{2.092989in}}%
\pgfusepath{stroke}%
\end{pgfscope}%
\begin{pgfscope}%
\pgfpathrectangle{\pgfqpoint{0.647939in}{0.492442in}}{\pgfqpoint{4.273799in}{2.331163in}}%
\pgfusepath{clip}%
\pgfsetbuttcap%
\pgfsetroundjoin%
\pgfsetlinewidth{0.301125pt}%
\definecolor{currentstroke}{rgb}{0.500000,0.500000,0.500000}%
\pgfsetstrokecolor{currentstroke}%
\pgfsetstrokeopacity{0.300000}%
\pgfsetdash{}{0pt}%
\pgfpathmoveto{\pgfqpoint{3.950420in}{0.492442in}}%
\pgfpathlineto{\pgfqpoint{3.950420in}{0.492442in}}%
\pgfpathlineto{\pgfqpoint{3.893912in}{0.534076in}}%
\pgfpathlineto{\pgfqpoint{3.836255in}{0.575239in}}%
\pgfpathlineto{\pgfqpoint{3.777713in}{0.616029in}}%
\pgfpathlineto{\pgfqpoint{3.718569in}{0.656562in}}%
\pgfpathlineto{\pgfqpoint{3.659135in}{0.696968in}}%
\pgfpathlineto{\pgfqpoint{3.599745in}{0.737392in}}%
\pgfpathlineto{\pgfqpoint{3.540707in}{0.777970in}}%
\pgfpathlineto{\pgfqpoint{3.482329in}{0.818830in}}%
\pgfpathlineto{\pgfqpoint{3.424895in}{0.860084in}}%
\pgfpathlineto{\pgfqpoint{3.368627in}{0.901813in}}%
\pgfpathlineto{\pgfqpoint{3.313730in}{0.944082in}}%
\pgfpathlineto{\pgfqpoint{3.260371in}{0.986930in}}%
\pgfpathlineto{\pgfqpoint{3.208665in}{1.030377in}}%
\pgfpathlineto{\pgfqpoint{3.158699in}{1.074425in}}%
\pgfpathlineto{\pgfqpoint{3.110533in}{1.119068in}}%
\pgfpathlineto{\pgfqpoint{3.064210in}{1.164287in}}%
\pgfpathlineto{\pgfqpoint{3.019758in}{1.210061in}}%
\pgfpathlineto{\pgfqpoint{2.977191in}{1.256366in}}%
\pgfpathlineto{\pgfqpoint{2.936521in}{1.303175in}}%
\pgfpathlineto{\pgfqpoint{2.897759in}{1.350464in}}%
\pgfpathlineto{\pgfqpoint{2.860920in}{1.398207in}}%
\pgfusepath{stroke}%
\end{pgfscope}%
\begin{pgfscope}%
\pgfpathrectangle{\pgfqpoint{0.647939in}{0.492442in}}{\pgfqpoint{4.273799in}{2.331163in}}%
\pgfusepath{clip}%
\pgfsetbuttcap%
\pgfsetroundjoin%
\pgfsetlinewidth{0.301125pt}%
\definecolor{currentstroke}{rgb}{0.500000,0.500000,0.500000}%
\pgfsetstrokecolor{currentstroke}%
\pgfsetstrokeopacity{0.300000}%
\pgfsetdash{}{0pt}%
\pgfpathmoveto{\pgfqpoint{4.144684in}{0.492442in}}%
\pgfpathlineto{\pgfqpoint{4.144684in}{0.492442in}}%
\pgfpathlineto{\pgfqpoint{4.093056in}{0.535918in}}%
\pgfpathlineto{\pgfqpoint{4.039532in}{0.578705in}}%
\pgfpathlineto{\pgfqpoint{3.984161in}{0.620788in}}%
\pgfpathlineto{\pgfqpoint{3.927051in}{0.662174in}}%
\pgfpathlineto{\pgfqpoint{3.868410in}{0.702921in}}%
\pgfpathlineto{\pgfqpoint{3.808507in}{0.743118in}}%
\pgfpathlineto{\pgfqpoint{3.747655in}{0.782890in}}%
\pgfpathlineto{\pgfqpoint{3.686236in}{0.822403in}}%
\pgfpathlineto{\pgfqpoint{3.624642in}{0.861834in}}%
\pgfpathlineto{\pgfqpoint{3.563287in}{0.901375in}}%
\pgfpathlineto{\pgfqpoint{3.502564in}{0.941205in}}%
\pgfpathlineto{\pgfqpoint{3.442829in}{0.981475in}}%
\pgfpathlineto{\pgfqpoint{3.384401in}{1.022309in}}%
\pgfpathlineto{\pgfqpoint{3.327533in}{1.063792in}}%
\pgfpathlineto{\pgfqpoint{3.272434in}{1.105980in}}%
\pgfpathlineto{\pgfqpoint{3.219268in}{1.148897in}}%
\pgfpathlineto{\pgfqpoint{3.168144in}{1.192544in}}%
\pgfpathlineto{\pgfqpoint{3.119132in}{1.236907in}}%
\pgfpathlineto{\pgfqpoint{3.072282in}{1.281961in}}%
\pgfpathlineto{\pgfqpoint{3.027625in}{1.327673in}}%
\pgfpathlineto{\pgfqpoint{2.985181in}{1.374008in}}%
\pgfpathlineto{\pgfqpoint{2.944970in}{1.420932in}}%
\pgfpathlineto{\pgfqpoint{2.907012in}{1.468411in}}%
\pgfpathlineto{\pgfqpoint{2.871342in}{1.516414in}}%
\pgfpathlineto{\pgfqpoint{2.838008in}{1.564914in}}%
\pgfpathlineto{\pgfqpoint{2.807088in}{1.613888in}}%
\pgfpathlineto{\pgfqpoint{2.778688in}{1.663315in}}%
\pgfpathlineto{\pgfqpoint{2.752957in}{1.713172in}}%
\pgfpathlineto{\pgfqpoint{2.730104in}{1.763445in}}%
\pgfpathlineto{\pgfqpoint{2.710413in}{1.814113in}}%
\pgfpathlineto{\pgfqpoint{2.694277in}{1.865151in}}%
\pgfpathlineto{\pgfqpoint{2.682248in}{1.916519in}}%
\pgfpathlineto{\pgfqpoint{2.675118in}{1.968152in}}%
\pgfpathlineto{\pgfqpoint{2.674077in}{2.019915in}}%
\pgfpathlineto{\pgfqpoint{2.681021in}{2.071515in}}%
\pgfpathlineto{\pgfqpoint{2.699265in}{2.122198in}}%
\pgfpathlineto{\pgfqpoint{2.735167in}{2.169656in}}%
\pgfpathlineto{\pgfqpoint{2.735167in}{2.169656in}}%
\pgfpathlineto{\pgfqpoint{2.771851in}{2.194916in}}%
\pgfpathlineto{\pgfqpoint{2.771851in}{2.194916in}}%
\pgfpathlineto{\pgfqpoint{2.812543in}{2.208912in}}%
\pgfpathlineto{\pgfqpoint{2.859976in}{2.212825in}}%
\pgfpathlineto{\pgfqpoint{2.902915in}{2.207562in}}%
\pgfpathlineto{\pgfqpoint{2.945237in}{2.195069in}}%
\pgfpathlineto{\pgfqpoint{2.988881in}{2.174632in}}%
\pgfusepath{stroke}%
\end{pgfscope}%
\begin{pgfscope}%
\pgfpathrectangle{\pgfqpoint{0.647939in}{0.492442in}}{\pgfqpoint{4.273799in}{2.331163in}}%
\pgfusepath{clip}%
\pgfsetbuttcap%
\pgfsetroundjoin%
\pgfsetlinewidth{0.301125pt}%
\definecolor{currentstroke}{rgb}{0.500000,0.500000,0.500000}%
\pgfsetstrokecolor{currentstroke}%
\pgfsetstrokeopacity{0.300000}%
\pgfsetdash{}{0pt}%
\pgfpathmoveto{\pgfqpoint{4.338948in}{0.492442in}}%
\pgfpathlineto{\pgfqpoint{4.338948in}{0.492442in}}%
\pgfpathlineto{\pgfqpoint{4.295118in}{0.538392in}}%
\pgfpathlineto{\pgfqpoint{4.249363in}{0.583780in}}%
\pgfpathlineto{\pgfqpoint{4.201540in}{0.628528in}}%
\pgfpathlineto{\pgfqpoint{4.151520in}{0.672556in}}%
\pgfpathlineto{\pgfqpoint{4.099197in}{0.715781in}}%
\pgfpathlineto{\pgfqpoint{4.044540in}{0.758138in}}%
\pgfpathlineto{\pgfqpoint{3.987577in}{0.799580in}}%
\pgfpathlineto{\pgfqpoint{3.928389in}{0.840085in}}%
\pgfpathlineto{\pgfqpoint{3.867206in}{0.879701in}}%
\pgfpathlineto{\pgfqpoint{3.804364in}{0.918538in}}%
\pgfpathlineto{\pgfqpoint{3.740282in}{0.956770in}}%
\pgfpathlineto{\pgfqpoint{3.675472in}{0.994637in}}%
\pgfpathlineto{\pgfqpoint{3.610486in}{1.032413in}}%
\pgfpathlineto{\pgfqpoint{3.545880in}{1.070381in}}%
\pgfpathlineto{\pgfqpoint{3.482187in}{1.108803in}}%
\pgfpathlineto{\pgfqpoint{3.419881in}{1.147892in}}%
\pgfpathlineto{\pgfqpoint{3.359356in}{1.187802in}}%
\pgfpathlineto{\pgfqpoint{3.300933in}{1.228631in}}%
\pgfpathlineto{\pgfqpoint{3.244844in}{1.270421in}}%
\pgfpathlineto{\pgfqpoint{3.191249in}{1.313175in}}%
\pgfpathlineto{\pgfqpoint{3.140261in}{1.356867in}}%
\pgfpathlineto{\pgfqpoint{3.091948in}{1.401451in}}%
\pgfpathlineto{\pgfqpoint{3.046343in}{1.446876in}}%
\pgfpathlineto{\pgfqpoint{3.003474in}{1.493088in}}%
\pgfpathlineto{\pgfqpoint{2.963368in}{1.540032in}}%
\pgfpathlineto{\pgfqpoint{2.926068in}{1.587661in}}%
\pgfpathlineto{\pgfqpoint{2.891644in}{1.635931in}}%
\pgfpathlineto{\pgfqpoint{2.860201in}{1.684803in}}%
\pgfusepath{stroke}%
\end{pgfscope}%
\begin{pgfscope}%
\pgfpathrectangle{\pgfqpoint{0.647939in}{0.492442in}}{\pgfqpoint{4.273799in}{2.331163in}}%
\pgfusepath{clip}%
\pgfsetbuttcap%
\pgfsetroundjoin%
\pgfsetlinewidth{0.301125pt}%
\definecolor{currentstroke}{rgb}{0.500000,0.500000,0.500000}%
\pgfsetstrokecolor{currentstroke}%
\pgfsetstrokeopacity{0.300000}%
\pgfsetdash{}{0pt}%
\pgfpathmoveto{\pgfqpoint{4.436079in}{0.492442in}}%
\pgfpathlineto{\pgfqpoint{4.436079in}{0.492442in}}%
\pgfpathlineto{\pgfqpoint{4.396629in}{0.539561in}}%
\pgfpathlineto{\pgfqpoint{4.355509in}{0.586253in}}%
\pgfpathlineto{\pgfqpoint{4.312557in}{0.632451in}}%
\pgfpathlineto{\pgfqpoint{4.267599in}{0.678076in}}%
\pgfpathlineto{\pgfqpoint{4.220455in}{0.723039in}}%
\pgfpathlineto{\pgfqpoint{4.170947in}{0.767237in}}%
\pgfpathlineto{\pgfqpoint{4.118909in}{0.810560in}}%
\pgfpathlineto{\pgfqpoint{4.064207in}{0.852894in}}%
\pgfpathlineto{\pgfqpoint{4.006784in}{0.894145in}}%
\pgfpathlineto{\pgfqpoint{3.946717in}{0.934259in}}%
\pgfpathlineto{\pgfqpoint{3.884169in}{0.973230in}}%
\pgfpathlineto{\pgfqpoint{3.819498in}{1.011161in}}%
\pgfpathlineto{\pgfqpoint{3.753194in}{1.048247in}}%
\pgfpathlineto{\pgfqpoint{3.685867in}{1.084782in}}%
\pgfpathlineto{\pgfqpoint{3.618204in}{1.121132in}}%
\pgfpathlineto{\pgfqpoint{3.550894in}{1.157675in}}%
\pgfpathlineto{\pgfqpoint{3.484606in}{1.194765in}}%
\pgfpathlineto{\pgfqpoint{3.419925in}{1.232684in}}%
\pgfpathlineto{\pgfqpoint{3.357340in}{1.271633in}}%
\pgfusepath{stroke}%
\end{pgfscope}%
\begin{pgfscope}%
\pgfpathrectangle{\pgfqpoint{0.647939in}{0.492442in}}{\pgfqpoint{4.273799in}{2.331163in}}%
\pgfusepath{clip}%
\pgfsetbuttcap%
\pgfsetroundjoin%
\pgfsetlinewidth{0.301125pt}%
\definecolor{currentstroke}{rgb}{0.500000,0.500000,0.500000}%
\pgfsetstrokecolor{currentstroke}%
\pgfsetstrokeopacity{0.300000}%
\pgfsetdash{}{0pt}%
\pgfpathmoveto{\pgfqpoint{4.533211in}{0.492442in}}%
\pgfpathlineto{\pgfqpoint{4.533211in}{0.492442in}}%
\pgfpathlineto{\pgfqpoint{4.498128in}{0.540579in}}%
\pgfpathlineto{\pgfqpoint{4.461737in}{0.588425in}}%
\pgfpathlineto{\pgfqpoint{4.423892in}{0.635932in}}%
\pgfpathlineto{\pgfqpoint{4.384424in}{0.683044in}}%
\pgfpathlineto{\pgfqpoint{4.343144in}{0.729691in}}%
\pgfpathlineto{\pgfqpoint{4.299840in}{0.775787in}}%
\pgfpathlineto{\pgfqpoint{4.254278in}{0.821229in}}%
\pgfpathlineto{\pgfqpoint{4.206202in}{0.865892in}}%
\pgfpathlineto{\pgfqpoint{4.155346in}{0.909627in}}%
\pgfpathlineto{\pgfqpoint{4.101446in}{0.952265in}}%
\pgfpathlineto{\pgfqpoint{4.044321in}{0.993633in}}%
\pgfpathlineto{\pgfqpoint{3.983904in}{1.033582in}}%
\pgfpathlineto{\pgfqpoint{3.920238in}{1.072002in}}%
\pgfpathlineto{\pgfqpoint{3.853626in}{1.108910in}}%
\pgfpathlineto{\pgfqpoint{3.784601in}{1.144481in}}%
\pgfpathlineto{\pgfqpoint{3.713886in}{1.179056in}}%
\pgfpathlineto{\pgfqpoint{3.642375in}{1.213143in}}%
\pgfpathlineto{\pgfqpoint{3.571013in}{1.247320in}}%
\pgfpathlineto{\pgfqpoint{3.500666in}{1.282113in}}%
\pgfpathlineto{\pgfqpoint{3.432153in}{1.317967in}}%
\pgfpathlineto{\pgfqpoint{3.366125in}{1.355178in}}%
\pgfpathlineto{\pgfqpoint{3.303078in}{1.393891in}}%
\pgfpathlineto{\pgfqpoint{3.243359in}{1.434140in}}%
\pgfpathlineto{\pgfqpoint{3.187167in}{1.475874in}}%
\pgfpathlineto{\pgfqpoint{3.134623in}{1.518997in}}%
\pgfpathlineto{\pgfqpoint{3.085775in}{1.563392in}}%
\pgfpathlineto{\pgfqpoint{3.040646in}{1.608944in}}%
\pgfpathlineto{\pgfqpoint{2.999270in}{1.655547in}}%
\pgfpathlineto{\pgfqpoint{2.961722in}{1.703104in}}%
\pgfpathlineto{\pgfqpoint{2.928141in}{1.751538in}}%
\pgfpathlineto{\pgfqpoint{2.898772in}{1.800781in}}%
\pgfpathlineto{\pgfqpoint{2.874020in}{1.850769in}}%
\pgfpathlineto{\pgfqpoint{2.854549in}{1.901441in}}%
\pgfpathlineto{\pgfqpoint{2.841469in}{1.952710in}}%
\pgfpathlineto{\pgfqpoint{2.836786in}{2.004374in}}%
\pgfpathlineto{\pgfqpoint{2.844643in}{2.055791in}}%
\pgfpathlineto{\pgfqpoint{2.844643in}{2.055791in}}%
\pgfpathlineto{\pgfqpoint{2.864815in}{2.093169in}}%
\pgfpathlineto{\pgfqpoint{2.864815in}{2.093169in}}%
\pgfpathlineto{\pgfqpoint{2.891256in}{2.114908in}}%
\pgfpathlineto{\pgfqpoint{2.891256in}{2.114908in}}%
\pgfpathlineto{\pgfqpoint{2.921319in}{2.125136in}}%
\pgfpathlineto{\pgfqpoint{2.956270in}{2.125734in}}%
\pgfpathlineto{\pgfqpoint{2.987680in}{2.118609in}}%
\pgfpathlineto{\pgfqpoint{3.019710in}{2.104388in}}%
\pgfusepath{stroke}%
\end{pgfscope}%
\begin{pgfscope}%
\pgfpathrectangle{\pgfqpoint{0.647939in}{0.492442in}}{\pgfqpoint{4.273799in}{2.331163in}}%
\pgfusepath{clip}%
\pgfsetbuttcap%
\pgfsetroundjoin%
\pgfsetlinewidth{0.301125pt}%
\definecolor{currentstroke}{rgb}{0.500000,0.500000,0.500000}%
\pgfsetstrokecolor{currentstroke}%
\pgfsetstrokeopacity{0.300000}%
\pgfsetdash{}{0pt}%
\pgfpathmoveto{\pgfqpoint{4.630343in}{0.492442in}}%
\pgfpathlineto{\pgfqpoint{4.630343in}{0.492442in}}%
\pgfpathlineto{\pgfqpoint{4.599424in}{0.541421in}}%
\pgfpathlineto{\pgfqpoint{4.567547in}{0.590216in}}%
\pgfpathlineto{\pgfqpoint{4.534605in}{0.638801in}}%
\pgfpathlineto{\pgfqpoint{4.500486in}{0.687144in}}%
\pgfpathlineto{\pgfqpoint{4.465059in}{0.735203in}}%
\pgfpathlineto{\pgfqpoint{4.428154in}{0.782930in}}%
\pgfpathlineto{\pgfqpoint{4.389571in}{0.830259in}}%
\pgfpathlineto{\pgfqpoint{4.349077in}{0.877109in}}%
\pgfpathlineto{\pgfqpoint{4.306399in}{0.923378in}}%
\pgfpathlineto{\pgfqpoint{4.261217in}{0.968932in}}%
\pgfpathlineto{\pgfqpoint{4.213162in}{1.013598in}}%
\pgfpathlineto{\pgfqpoint{4.161817in}{1.057155in}}%
\pgfpathlineto{\pgfqpoint{4.106728in}{1.099326in}}%
\pgfpathlineto{\pgfqpoint{4.047494in}{1.139788in}}%
\pgfpathlineto{\pgfqpoint{3.983885in}{1.178214in}}%
\pgfpathlineto{\pgfqpoint{3.915898in}{1.214342in}}%
\pgfpathlineto{\pgfqpoint{3.843944in}{1.248114in}}%
\pgfpathlineto{\pgfqpoint{3.768889in}{1.279833in}}%
\pgfpathlineto{\pgfqpoint{3.691947in}{1.310189in}}%
\pgfpathlineto{\pgfqpoint{3.614423in}{1.340108in}}%
\pgfpathlineto{\pgfqpoint{3.537622in}{1.370567in}}%
\pgfpathlineto{\pgfqpoint{3.462759in}{1.402408in}}%
\pgfpathlineto{\pgfqpoint{3.390860in}{1.436196in}}%
\pgfpathlineto{\pgfqpoint{3.322642in}{1.472180in}}%
\pgfpathlineto{\pgfqpoint{3.258609in}{1.510380in}}%
\pgfpathlineto{\pgfqpoint{3.199041in}{1.550674in}}%
\pgfpathlineto{\pgfqpoint{3.144046in}{1.592864in}}%
\pgfpathlineto{\pgfqpoint{3.093660in}{1.636730in}}%
\pgfpathlineto{\pgfqpoint{3.047881in}{1.682074in}}%
\pgfusepath{stroke}%
\end{pgfscope}%
\begin{pgfscope}%
\pgfpathrectangle{\pgfqpoint{0.647939in}{0.492442in}}{\pgfqpoint{4.273799in}{2.331163in}}%
\pgfusepath{clip}%
\pgfsetbuttcap%
\pgfsetroundjoin%
\pgfsetlinewidth{0.301125pt}%
\definecolor{currentstroke}{rgb}{0.500000,0.500000,0.500000}%
\pgfsetstrokecolor{currentstroke}%
\pgfsetstrokeopacity{0.300000}%
\pgfsetdash{}{0pt}%
\pgfpathmoveto{\pgfqpoint{4.727475in}{0.492442in}}%
\pgfpathlineto{\pgfqpoint{4.727475in}{0.492442in}}%
\pgfpathlineto{\pgfqpoint{4.700361in}{0.542088in}}%
\pgfpathlineto{\pgfqpoint{4.672607in}{0.591629in}}%
\pgfpathlineto{\pgfqpoint{4.644158in}{0.641051in}}%
\pgfpathlineto{\pgfqpoint{4.614939in}{0.690340in}}%
\pgfpathlineto{\pgfqpoint{4.584868in}{0.739476in}}%
\pgfpathlineto{\pgfqpoint{4.553858in}{0.788437in}}%
\pgfpathlineto{\pgfqpoint{4.521788in}{0.837194in}}%
\pgfpathlineto{\pgfqpoint{4.488509in}{0.885710in}}%
\pgfpathlineto{\pgfqpoint{4.453861in}{0.933939in}}%
\pgfpathlineto{\pgfqpoint{4.417647in}{0.981823in}}%
\pgfpathlineto{\pgfqpoint{4.379608in}{1.029280in}}%
\pgfpathlineto{\pgfqpoint{4.339414in}{1.076204in}}%
\pgfpathlineto{\pgfqpoint{4.296656in}{1.122444in}}%
\pgfpathlineto{\pgfqpoint{4.250809in}{1.167790in}}%
\pgfpathlineto{\pgfqpoint{4.201210in}{1.211936in}}%
\pgfpathlineto{\pgfqpoint{4.147024in}{1.254440in}}%
\pgfpathlineto{\pgfqpoint{4.087332in}{1.294672in}}%
\pgfpathlineto{\pgfqpoint{4.021302in}{1.331804in}}%
\pgfpathlineto{\pgfqpoint{3.948493in}{1.364926in}}%
\pgfpathlineto{\pgfqpoint{3.869466in}{1.393497in}}%
\pgfpathlineto{\pgfqpoint{3.785816in}{1.417908in}}%
\pgfpathlineto{\pgfqpoint{3.699695in}{1.439696in}}%
\pgfpathlineto{\pgfqpoint{3.613177in}{1.461024in}}%
\pgfpathlineto{\pgfqpoint{3.528086in}{1.483940in}}%
\pgfpathlineto{\pgfqpoint{3.446046in}{1.509889in}}%
\pgfpathlineto{\pgfqpoint{3.368395in}{1.539567in}}%
\pgfpathlineto{\pgfqpoint{3.296135in}{1.573043in}}%
\pgfpathlineto{\pgfqpoint{3.229808in}{1.609982in}}%
\pgfpathlineto{\pgfqpoint{3.169526in}{1.649905in}}%
\pgfpathlineto{\pgfqpoint{3.115255in}{1.692333in}}%
\pgfpathlineto{\pgfqpoint{3.066896in}{1.736860in}}%
\pgfpathlineto{\pgfqpoint{3.024438in}{1.783155in}}%
\pgfpathlineto{\pgfqpoint{2.988048in}{1.830957in}}%
\pgfpathlineto{\pgfqpoint{2.958186in}{1.880073in}}%
\pgfpathlineto{\pgfqpoint{2.935866in}{1.930358in}}%
\pgfusepath{stroke}%
\end{pgfscope}%
\begin{pgfscope}%
\pgfpathrectangle{\pgfqpoint{0.647939in}{0.492442in}}{\pgfqpoint{4.273799in}{2.331163in}}%
\pgfusepath{clip}%
\pgfsetbuttcap%
\pgfsetroundjoin%
\pgfsetlinewidth{0.301125pt}%
\definecolor{currentstroke}{rgb}{0.500000,0.500000,0.500000}%
\pgfsetstrokecolor{currentstroke}%
\pgfsetstrokeopacity{0.300000}%
\pgfsetdash{}{0pt}%
\pgfpathmoveto{\pgfqpoint{4.824607in}{0.492442in}}%
\pgfpathlineto{\pgfqpoint{4.824607in}{0.492442in}}%
\pgfpathlineto{\pgfqpoint{4.800894in}{0.542604in}}%
\pgfpathlineto{\pgfqpoint{4.776800in}{0.592711in}}%
\pgfpathlineto{\pgfqpoint{4.752292in}{0.642759in}}%
\pgfpathlineto{\pgfqpoint{4.727341in}{0.692742in}}%
\pgfpathlineto{\pgfqpoint{4.701912in}{0.742652in}}%
\pgfpathlineto{\pgfqpoint{4.675959in}{0.792483in}}%
\pgfpathlineto{\pgfqpoint{4.649439in}{0.842223in}}%
\pgfpathlineto{\pgfqpoint{4.622285in}{0.891862in}}%
\pgfpathlineto{\pgfqpoint{4.594432in}{0.941386in}}%
\pgfpathlineto{\pgfqpoint{4.565799in}{0.990776in}}%
\pgfpathlineto{\pgfqpoint{4.536273in}{1.040009in}}%
\pgfpathlineto{\pgfqpoint{4.505733in}{1.089058in}}%
\pgfpathlineto{\pgfqpoint{4.474023in}{1.137883in}}%
\pgfpathlineto{\pgfqpoint{4.440920in}{1.186431in}}%
\pgfpathlineto{\pgfqpoint{4.406142in}{1.234631in}}%
\pgfpathlineto{\pgfqpoint{4.369328in}{1.282376in}}%
\pgfpathlineto{\pgfqpoint{4.329969in}{1.329505in}}%
\pgfpathlineto{\pgfqpoint{4.287309in}{1.375761in}}%
\pgfpathlineto{\pgfqpoint{4.240226in}{1.420705in}}%
\pgfpathlineto{\pgfqpoint{4.186989in}{1.463530in}}%
\pgfpathlineto{\pgfqpoint{4.125150in}{1.502663in}}%
\pgfpathlineto{\pgfqpoint{4.051814in}{1.535084in}}%
\pgfpathlineto{\pgfqpoint{3.969155in}{1.556263in}}%
\pgfpathlineto{\pgfqpoint{3.890586in}{1.565618in}}%
\pgfpathlineto{\pgfqpoint{3.804175in}{1.568995in}}%
\pgfpathlineto{\pgfqpoint{3.709313in}{1.570121in}}%
\pgfpathlineto{\pgfqpoint{3.614709in}{1.573368in}}%
\pgfpathlineto{\pgfqpoint{3.521484in}{1.582168in}}%
\pgfpathlineto{\pgfqpoint{3.431805in}{1.598294in}}%
\pgfpathlineto{\pgfqpoint{3.347946in}{1.621974in}}%
\pgfpathlineto{\pgfqpoint{3.271523in}{1.652303in}}%
\pgfpathlineto{\pgfqpoint{3.203104in}{1.687981in}}%
\pgfpathlineto{\pgfqpoint{3.142706in}{1.727773in}}%
\pgfpathlineto{\pgfqpoint{3.090002in}{1.770727in}}%
\pgfpathlineto{\pgfqpoint{3.044777in}{1.816159in}}%
\pgfusepath{stroke}%
\end{pgfscope}%
\begin{pgfscope}%
\pgfpathrectangle{\pgfqpoint{0.647939in}{0.492442in}}{\pgfqpoint{4.273799in}{2.331163in}}%
\pgfusepath{clip}%
\pgfsetbuttcap%
\pgfsetroundjoin%
\pgfsetlinewidth{0.301125pt}%
\definecolor{currentstroke}{rgb}{0.500000,0.500000,0.500000}%
\pgfsetstrokecolor{currentstroke}%
\pgfsetstrokeopacity{0.300000}%
\pgfsetdash{}{0pt}%
\pgfpathmoveto{\pgfqpoint{4.921738in}{0.492442in}}%
\pgfpathlineto{\pgfqpoint{4.921738in}{0.492442in}}%
\pgfpathlineto{\pgfqpoint{4.901010in}{0.542996in}}%
\pgfpathlineto{\pgfqpoint{4.880090in}{0.593527in}}%
\pgfpathlineto{\pgfqpoint{4.858971in}{0.644032in}}%
\pgfpathlineto{\pgfqpoint{4.837643in}{0.694512in}}%
\pgfpathlineto{\pgfqpoint{4.816099in}{0.744964in}}%
\pgfpathlineto{\pgfqpoint{4.794327in}{0.795388in}}%
\pgfpathlineto{\pgfqpoint{4.772319in}{0.845780in}}%
\pgfpathlineto{\pgfqpoint{4.750059in}{0.896140in}}%
\pgfpathlineto{\pgfqpoint{4.727536in}{0.946464in}}%
\pgfpathlineto{\pgfqpoint{4.704733in}{0.996751in}}%
\pgfpathlineto{\pgfqpoint{4.681632in}{1.046998in}}%
\pgfpathlineto{\pgfqpoint{4.658213in}{1.097200in}}%
\pgfpathlineto{\pgfqpoint{4.634450in}{1.147354in}}%
\pgfpathlineto{\pgfqpoint{4.610314in}{1.197454in}}%
\pgfpathlineto{\pgfqpoint{4.585764in}{1.247495in}}%
\pgfpathlineto{\pgfqpoint{4.560760in}{1.297467in}}%
\pgfpathlineto{\pgfqpoint{4.535234in}{1.347362in}}%
\pgfpathlineto{\pgfqpoint{4.509117in}{1.397165in}}%
\pgfpathlineto{\pgfqpoint{4.482314in}{1.446857in}}%
\pgfpathlineto{\pgfqpoint{4.454673in}{1.496413in}}%
\pgfpathlineto{\pgfqpoint{4.426004in}{1.545791in}}%
\pgfpathlineto{\pgfqpoint{4.395989in}{1.594927in}}%
\pgfpathlineto{\pgfqpoint{4.364100in}{1.643710in}}%
\pgfpathlineto{\pgfqpoint{4.329431in}{1.691910in}}%
\pgfpathlineto{\pgfqpoint{4.290051in}{1.738960in}}%
\pgfpathlineto{\pgfqpoint{4.240754in}{1.782846in}}%
\pgfpathlineto{\pgfqpoint{4.240754in}{1.782846in}}%
\pgfpathlineto{\pgfqpoint{4.204099in}{1.802616in}}%
\pgfpathlineto{\pgfqpoint{4.204099in}{1.802616in}}%
\pgfpathlineto{\pgfqpoint{4.168373in}{1.810876in}}%
\pgfpathlineto{\pgfqpoint{4.128773in}{1.809162in}}%
\pgfpathlineto{\pgfqpoint{4.093194in}{1.800775in}}%
\pgfpathlineto{\pgfqpoint{4.049808in}{1.785154in}}%
\pgfpathlineto{\pgfqpoint{3.987547in}{1.758307in}}%
\pgfpathlineto{\pgfqpoint{3.913775in}{1.725980in}}%
\pgfpathlineto{\pgfqpoint{3.837151in}{1.695577in}}%
\pgfpathlineto{\pgfqpoint{3.755802in}{1.669131in}}%
\pgfpathlineto{\pgfqpoint{3.668831in}{1.648859in}}%
\pgfpathlineto{\pgfqpoint{3.576958in}{1.637433in}}%
\pgfpathlineto{\pgfqpoint{3.489275in}{1.636766in}}%
\pgfpathlineto{\pgfqpoint{3.409483in}{1.645488in}}%
\pgfpathlineto{\pgfqpoint{3.335198in}{1.662379in}}%
\pgfpathlineto{\pgfqpoint{3.265060in}{1.687014in}}%
\pgfusepath{stroke}%
\end{pgfscope}%
\begin{pgfscope}%
\pgfpathrectangle{\pgfqpoint{0.647939in}{0.492442in}}{\pgfqpoint{4.273799in}{2.331163in}}%
\pgfusepath{clip}%
\pgfsetbuttcap%
\pgfsetroundjoin%
\pgfsetlinewidth{0.301125pt}%
\definecolor{currentstroke}{rgb}{0.500000,0.500000,0.500000}%
\pgfsetstrokecolor{currentstroke}%
\pgfsetstrokeopacity{0.300000}%
\pgfsetdash{}{0pt}%
\pgfpathmoveto{\pgfqpoint{4.921738in}{0.704366in}}%
\pgfpathlineto{\pgfqpoint{4.921738in}{0.704366in}}%
\pgfpathlineto{\pgfqpoint{4.902812in}{0.755130in}}%
\pgfpathlineto{\pgfqpoint{4.883821in}{0.805887in}}%
\pgfpathlineto{\pgfqpoint{4.864769in}{0.856636in}}%
\pgfpathlineto{\pgfqpoint{4.845665in}{0.907380in}}%
\pgfpathlineto{\pgfqpoint{4.826513in}{0.958118in}}%
\pgfpathlineto{\pgfqpoint{4.807324in}{1.008853in}}%
\pgfpathlineto{\pgfqpoint{4.788106in}{1.059584in}}%
\pgfpathlineto{\pgfqpoint{4.768876in}{1.110313in}}%
\pgfpathlineto{\pgfqpoint{4.749647in}{1.161043in}}%
\pgfpathlineto{\pgfqpoint{4.730439in}{1.211775in}}%
\pgfpathlineto{\pgfqpoint{4.711280in}{1.262512in}}%
\pgfpathlineto{\pgfqpoint{4.692194in}{1.313258in}}%
\pgfpathlineto{\pgfqpoint{4.673221in}{1.364015in}}%
\pgfpathlineto{\pgfqpoint{4.654402in}{1.414790in}}%
\pgfpathlineto{\pgfqpoint{4.635796in}{1.465588in}}%
\pgfpathlineto{\pgfqpoint{4.617471in}{1.516416in}}%
\pgfpathlineto{\pgfqpoint{4.599520in}{1.567284in}}%
\pgfpathlineto{\pgfqpoint{4.582065in}{1.618202in}}%
\pgfpathlineto{\pgfqpoint{4.565260in}{1.669186in}}%
\pgfpathlineto{\pgfqpoint{4.549319in}{1.720251in}}%
\pgfpathlineto{\pgfqpoint{4.534515in}{1.771418in}}%
\pgfpathlineto{\pgfqpoint{4.521242in}{1.822709in}}%
\pgfpathlineto{\pgfqpoint{4.510039in}{1.874143in}}%
\pgfpathlineto{\pgfqpoint{4.501621in}{1.925733in}}%
\pgfpathlineto{\pgfqpoint{4.496930in}{1.977459in}}%
\pgfpathlineto{\pgfqpoint{4.497088in}{2.029238in}}%
\pgfpathlineto{\pgfqpoint{4.503148in}{2.080898in}}%
\pgfpathlineto{\pgfqpoint{4.515634in}{2.132199in}}%
\pgfpathlineto{\pgfqpoint{4.534219in}{2.182953in}}%
\pgfpathlineto{\pgfqpoint{4.557733in}{2.233099in}}%
\pgfpathlineto{\pgfqpoint{4.584761in}{2.282718in}}%
\pgfpathlineto{\pgfqpoint{4.614067in}{2.331960in}}%
\pgfpathlineto{\pgfqpoint{4.644680in}{2.380959in}}%
\pgfpathlineto{\pgfqpoint{4.675962in}{2.429828in}}%
\pgfpathlineto{\pgfqpoint{4.707537in}{2.478662in}}%
\pgfpathlineto{\pgfqpoint{4.739102in}{2.527502in}}%
\pgfpathlineto{\pgfqpoint{4.770459in}{2.576371in}}%
\pgfpathlineto{\pgfqpoint{4.801532in}{2.625298in}}%
\pgfpathlineto{\pgfqpoint{4.832275in}{2.674299in}}%
\pgfpathlineto{\pgfqpoint{4.862620in}{2.723372in}}%
\pgfpathlineto{\pgfqpoint{4.892542in}{2.772519in}}%
\pgfpathlineto{\pgfqpoint{4.921738in}{2.820822in}}%
\pgfusepath{stroke}%
\end{pgfscope}%
\begin{pgfscope}%
\pgfpathrectangle{\pgfqpoint{0.647939in}{0.492442in}}{\pgfqpoint{4.273799in}{2.331163in}}%
\pgfusepath{clip}%
\pgfsetbuttcap%
\pgfsetroundjoin%
\pgfsetlinewidth{0.301125pt}%
\definecolor{currentstroke}{rgb}{0.500000,0.500000,0.500000}%
\pgfsetstrokecolor{currentstroke}%
\pgfsetstrokeopacity{0.300000}%
\pgfsetdash{}{0pt}%
\pgfpathmoveto{\pgfqpoint{4.921738in}{0.969271in}}%
\pgfpathlineto{\pgfqpoint{4.921738in}{0.969271in}}%
\pgfpathlineto{\pgfqpoint{4.905439in}{1.020305in}}%
\pgfpathlineto{\pgfqpoint{4.889274in}{1.071352in}}%
\pgfpathlineto{\pgfqpoint{4.873265in}{1.122414in}}%
\pgfpathlineto{\pgfqpoint{4.857444in}{1.173493in}}%
\pgfpathlineto{\pgfqpoint{4.841844in}{1.224592in}}%
\pgfpathlineto{\pgfqpoint{4.826502in}{1.275715in}}%
\pgfpathlineto{\pgfqpoint{4.811467in}{1.326864in}}%
\pgfpathlineto{\pgfqpoint{4.796787in}{1.378044in}}%
\pgfpathlineto{\pgfqpoint{4.782523in}{1.429259in}}%
\pgfpathlineto{\pgfqpoint{4.768750in}{1.480514in}}%
\pgfpathlineto{\pgfqpoint{4.755545in}{1.531813in}}%
\pgfpathlineto{\pgfqpoint{4.743005in}{1.583162in}}%
\pgfpathlineto{\pgfqpoint{4.731244in}{1.634566in}}%
\pgfpathlineto{\pgfqpoint{4.720394in}{1.686028in}}%
\pgfpathlineto{\pgfqpoint{4.710601in}{1.737554in}}%
\pgfpathlineto{\pgfqpoint{4.702039in}{1.789145in}}%
\pgfpathlineto{\pgfqpoint{4.694910in}{1.840801in}}%
\pgfpathlineto{\pgfqpoint{4.689432in}{1.892515in}}%
\pgfpathlineto{\pgfqpoint{4.685841in}{1.944278in}}%
\pgfpathlineto{\pgfqpoint{4.684380in}{1.996071in}}%
\pgfpathlineto{\pgfqpoint{4.685285in}{2.047868in}}%
\pgfpathlineto{\pgfqpoint{4.688758in}{2.099631in}}%
\pgfpathlineto{\pgfqpoint{4.694946in}{2.151317in}}%
\pgfpathlineto{\pgfqpoint{4.703912in}{2.202881in}}%
\pgfpathlineto{\pgfqpoint{4.715612in}{2.254282in}}%
\pgfpathlineto{\pgfqpoint{4.729893in}{2.305487in}}%
\pgfpathlineto{\pgfqpoint{4.746537in}{2.356481in}}%
\pgfpathlineto{\pgfqpoint{4.765248in}{2.407260in}}%
\pgfpathlineto{\pgfqpoint{4.785705in}{2.457837in}}%
\pgfpathlineto{\pgfqpoint{4.807596in}{2.508235in}}%
\pgfpathlineto{\pgfqpoint{4.830631in}{2.558482in}}%
\pgfusepath{stroke}%
\end{pgfscope}%
\begin{pgfscope}%
\pgfpathrectangle{\pgfqpoint{0.647939in}{0.492442in}}{\pgfqpoint{4.273799in}{2.331163in}}%
\pgfusepath{clip}%
\pgfsetbuttcap%
\pgfsetroundjoin%
\pgfsetlinewidth{0.301125pt}%
\definecolor{currentstroke}{rgb}{0.500000,0.500000,0.500000}%
\pgfsetstrokecolor{currentstroke}%
\pgfsetstrokeopacity{0.300000}%
\pgfsetdash{}{0pt}%
\pgfpathmoveto{\pgfqpoint{4.921738in}{1.287157in}}%
\pgfpathlineto{\pgfqpoint{4.921738in}{1.287157in}}%
\pgfpathlineto{\pgfqpoint{4.909295in}{1.338513in}}%
\pgfpathlineto{\pgfqpoint{4.897291in}{1.389901in}}%
\pgfpathlineto{\pgfqpoint{4.885777in}{1.441322in}}%
\pgfpathlineto{\pgfqpoint{4.874810in}{1.492778in}}%
\pgfpathlineto{\pgfqpoint{4.864462in}{1.544273in}}%
\pgfpathlineto{\pgfqpoint{4.854806in}{1.595807in}}%
\pgfpathlineto{\pgfqpoint{4.845923in}{1.647382in}}%
\pgfpathlineto{\pgfqpoint{4.837900in}{1.699000in}}%
\pgfpathlineto{\pgfqpoint{4.830837in}{1.750659in}}%
\pgfpathlineto{\pgfqpoint{4.824841in}{1.802358in}}%
\pgfpathlineto{\pgfqpoint{4.820022in}{1.854094in}}%
\pgfpathlineto{\pgfqpoint{4.816495in}{1.905860in}}%
\pgfpathlineto{\pgfqpoint{4.814377in}{1.957649in}}%
\pgfpathlineto{\pgfqpoint{4.813780in}{2.009450in}}%
\pgfpathlineto{\pgfqpoint{4.814808in}{2.061248in}}%
\pgfpathlineto{\pgfqpoint{4.817552in}{2.113027in}}%
\pgfpathlineto{\pgfqpoint{4.822080in}{2.164769in}}%
\pgfpathlineto{\pgfqpoint{4.828431in}{2.216453in}}%
\pgfpathlineto{\pgfqpoint{4.836617in}{2.268061in}}%
\pgfpathlineto{\pgfqpoint{4.846603in}{2.319574in}}%
\pgfpathlineto{\pgfqpoint{4.858320in}{2.370977in}}%
\pgfpathlineto{\pgfqpoint{4.871680in}{2.422262in}}%
\pgfpathlineto{\pgfqpoint{4.886561in}{2.473422in}}%
\pgfpathlineto{\pgfqpoint{4.902816in}{2.524456in}}%
\pgfpathlineto{\pgfqpoint{4.920302in}{2.575370in}}%
\pgfpathlineto{\pgfqpoint{4.921738in}{2.579417in}}%
\pgfusepath{stroke}%
\end{pgfscope}%
\begin{pgfscope}%
\pgfpathrectangle{\pgfqpoint{0.647939in}{0.492442in}}{\pgfqpoint{4.273799in}{2.331163in}}%
\pgfusepath{clip}%
\pgfsetbuttcap%
\pgfsetroundjoin%
\pgfsetlinewidth{0.301125pt}%
\definecolor{currentstroke}{rgb}{0.500000,0.500000,0.500000}%
\pgfsetstrokecolor{currentstroke}%
\pgfsetstrokeopacity{0.300000}%
\pgfsetdash{}{0pt}%
\pgfpathmoveto{\pgfqpoint{4.921738in}{1.552062in}}%
\pgfpathlineto{\pgfqpoint{4.921738in}{1.552062in}}%
\pgfpathlineto{\pgfqpoint{4.913281in}{1.603659in}}%
\pgfpathlineto{\pgfqpoint{4.905568in}{1.655291in}}%
\pgfpathlineto{\pgfqpoint{4.898672in}{1.706957in}}%
\pgfpathlineto{\pgfqpoint{4.892669in}{1.758656in}}%
\pgfpathlineto{\pgfqpoint{4.887640in}{1.810386in}}%
\pgfpathlineto{\pgfqpoint{4.883667in}{1.862143in}}%
\pgfpathlineto{\pgfqpoint{4.880837in}{1.913923in}}%
\pgfpathlineto{\pgfqpoint{4.879232in}{1.965717in}}%
\pgfpathlineto{\pgfqpoint{4.878935in}{2.017519in}}%
\pgfpathlineto{\pgfqpoint{4.880017in}{2.069318in}}%
\pgfpathlineto{\pgfqpoint{4.882541in}{2.121102in}}%
\pgfpathlineto{\pgfqpoint{4.886554in}{2.172857in}}%
\pgfpathlineto{\pgfqpoint{4.892086in}{2.224570in}}%
\pgfpathlineto{\pgfqpoint{4.899147in}{2.276228in}}%
\pgfpathlineto{\pgfqpoint{4.907724in}{2.327817in}}%
\pgfpathlineto{\pgfqpoint{4.917786in}{2.379327in}}%
\pgfpathlineto{\pgfqpoint{4.921738in}{2.398194in}}%
\pgfusepath{stroke}%
\end{pgfscope}%
\begin{pgfscope}%
\pgfpathrectangle{\pgfqpoint{0.647939in}{0.492442in}}{\pgfqpoint{4.273799in}{2.331163in}}%
\pgfusepath{clip}%
\pgfsetbuttcap%
\pgfsetroundjoin%
\pgfsetlinewidth{0.301125pt}%
\definecolor{currentstroke}{rgb}{0.500000,0.500000,0.500000}%
\pgfsetstrokecolor{currentstroke}%
\pgfsetstrokeopacity{0.300000}%
\pgfsetdash{}{0pt}%
\pgfpathmoveto{\pgfqpoint{4.267321in}{2.823605in}}%
\pgfpathlineto{\pgfqpoint{4.277250in}{2.813629in}}%
\pgfpathlineto{\pgfqpoint{4.324675in}{2.768778in}}%
\pgfpathlineto{\pgfqpoint{4.377880in}{2.725979in}}%
\pgfpathlineto{\pgfqpoint{4.425126in}{2.695934in}}%
\pgfpathlineto{\pgfqpoint{4.467852in}{2.676227in}}%
\pgfpathlineto{\pgfqpoint{4.510718in}{2.664351in}}%
\pgfpathlineto{\pgfqpoint{4.562689in}{2.661285in}}%
\pgfpathlineto{\pgfqpoint{4.612557in}{2.670171in}}%
\pgfpathlineto{\pgfqpoint{4.612557in}{2.670171in}}%
\pgfpathlineto{\pgfqpoint{4.669854in}{2.694352in}}%
\pgfpathlineto{\pgfqpoint{4.669854in}{2.694352in}}%
\pgfpathlineto{\pgfqpoint{4.729600in}{2.734205in}}%
\pgfpathlineto{\pgfqpoint{4.779962in}{2.777959in}}%
\pgfpathlineto{\pgfqpoint{4.824607in}{2.823605in}}%
\pgfpathlineto{\pgfqpoint{4.824607in}{2.823605in}}%
\pgfusepath{stroke}%
\end{pgfscope}%
\begin{pgfscope}%
\pgfpathrectangle{\pgfqpoint{0.647939in}{0.492442in}}{\pgfqpoint{4.273799in}{2.331163in}}%
\pgfusepath{clip}%
\pgfsetbuttcap%
\pgfsetroundjoin%
\pgfsetlinewidth{0.301125pt}%
\definecolor{currentstroke}{rgb}{0.500000,0.500000,0.500000}%
\pgfsetstrokecolor{currentstroke}%
\pgfsetstrokeopacity{0.300000}%
\pgfsetdash{}{0pt}%
\pgfpathmoveto{\pgfqpoint{4.144684in}{2.823605in}}%
\pgfpathlineto{\pgfqpoint{4.144684in}{2.823605in}}%
\pgfpathlineto{\pgfqpoint{4.183329in}{2.776287in}}%
\pgfpathlineto{\pgfqpoint{4.223626in}{2.729384in}}%
\pgfpathlineto{\pgfqpoint{4.266386in}{2.683142in}}%
\pgfpathlineto{\pgfqpoint{4.313009in}{2.638044in}}%
\pgfpathlineto{\pgfqpoint{4.366217in}{2.595229in}}%
\pgfpathlineto{\pgfqpoint{4.431914in}{2.558554in}}%
\pgfpathlineto{\pgfqpoint{4.431914in}{2.558554in}}%
\pgfpathlineto{\pgfqpoint{4.477591in}{2.545283in}}%
\pgfpathlineto{\pgfqpoint{4.477591in}{2.545283in}}%
\pgfpathlineto{\pgfqpoint{4.520530in}{2.542834in}}%
\pgfpathlineto{\pgfqpoint{4.562094in}{2.549767in}}%
\pgfpathlineto{\pgfqpoint{4.599548in}{2.563546in}}%
\pgfpathlineto{\pgfqpoint{4.639318in}{2.585505in}}%
\pgfpathlineto{\pgfqpoint{4.684038in}{2.618336in}}%
\pgfpathlineto{\pgfqpoint{4.732710in}{2.662619in}}%
\pgfusepath{stroke}%
\end{pgfscope}%
\begin{pgfscope}%
\pgfpathrectangle{\pgfqpoint{0.647939in}{0.492442in}}{\pgfqpoint{4.273799in}{2.331163in}}%
\pgfusepath{clip}%
\pgfsetbuttcap%
\pgfsetroundjoin%
\pgfsetlinewidth{0.301125pt}%
\definecolor{currentstroke}{rgb}{0.500000,0.500000,0.500000}%
\pgfsetstrokecolor{currentstroke}%
\pgfsetstrokeopacity{0.300000}%
\pgfsetdash{}{0pt}%
\pgfpathmoveto{\pgfqpoint{4.047552in}{2.823605in}}%
\pgfpathlineto{\pgfqpoint{4.047552in}{2.823605in}}%
\pgfpathlineto{\pgfqpoint{4.082780in}{2.775498in}}%
\pgfpathlineto{\pgfqpoint{4.118575in}{2.727517in}}%
\pgfpathlineto{\pgfqpoint{4.155221in}{2.679729in}}%
\pgfpathlineto{\pgfqpoint{4.193139in}{2.632238in}}%
\pgfpathlineto{\pgfqpoint{4.233002in}{2.585226in}}%
\pgfpathlineto{\pgfqpoint{4.275982in}{2.539046in}}%
\pgfpathlineto{\pgfqpoint{4.324384in}{2.494541in}}%
\pgfpathlineto{\pgfqpoint{4.383496in}{2.454467in}}%
\pgfpathlineto{\pgfqpoint{4.383496in}{2.454467in}}%
\pgfpathlineto{\pgfqpoint{4.426840in}{2.437147in}}%
\pgfpathlineto{\pgfqpoint{4.426840in}{2.437147in}}%
\pgfpathlineto{\pgfqpoint{4.467076in}{2.431386in}}%
\pgfpathlineto{\pgfqpoint{4.508185in}{2.435968in}}%
\pgfpathlineto{\pgfqpoint{4.542921in}{2.447595in}}%
\pgfpathlineto{\pgfqpoint{4.579396in}{2.466851in}}%
\pgfpathlineto{\pgfqpoint{4.620657in}{2.496325in}}%
\pgfusepath{stroke}%
\end{pgfscope}%
\begin{pgfscope}%
\pgfpathrectangle{\pgfqpoint{0.647939in}{0.492442in}}{\pgfqpoint{4.273799in}{2.331163in}}%
\pgfusepath{clip}%
\pgfsetbuttcap%
\pgfsetroundjoin%
\pgfsetlinewidth{0.301125pt}%
\definecolor{currentstroke}{rgb}{0.500000,0.500000,0.500000}%
\pgfsetstrokecolor{currentstroke}%
\pgfsetstrokeopacity{0.300000}%
\pgfsetdash{}{0pt}%
\pgfpathmoveto{\pgfqpoint{3.950420in}{2.823605in}}%
\pgfpathlineto{\pgfqpoint{3.950420in}{2.823605in}}%
\pgfpathlineto{\pgfqpoint{3.983607in}{2.775068in}}%
\pgfpathlineto{\pgfqpoint{4.016781in}{2.726527in}}%
\pgfpathlineto{\pgfqpoint{4.050058in}{2.678008in}}%
\pgfpathlineto{\pgfqpoint{4.083591in}{2.629542in}}%
\pgfpathlineto{\pgfqpoint{4.117573in}{2.581171in}}%
\pgfpathlineto{\pgfqpoint{4.152281in}{2.532954in}}%
\pgfpathlineto{\pgfqpoint{4.188151in}{2.484991in}}%
\pgfpathlineto{\pgfqpoint{4.225952in}{2.437477in}}%
\pgfpathlineto{\pgfqpoint{4.267144in}{2.390831in}}%
\pgfpathlineto{\pgfqpoint{4.315040in}{2.346221in}}%
\pgfpathlineto{\pgfqpoint{4.315040in}{2.346221in}}%
\pgfpathlineto{\pgfqpoint{4.361993in}{2.316150in}}%
\pgfpathlineto{\pgfqpoint{4.361993in}{2.316150in}}%
\pgfpathlineto{\pgfqpoint{4.398106in}{2.304370in}}%
\pgfpathlineto{\pgfqpoint{4.398106in}{2.304370in}}%
\pgfpathlineto{\pgfqpoint{4.432082in}{2.303126in}}%
\pgfpathlineto{\pgfqpoint{4.463811in}{2.310186in}}%
\pgfpathlineto{\pgfqpoint{4.494327in}{2.323698in}}%
\pgfpathlineto{\pgfqpoint{4.529148in}{2.345999in}}%
\pgfpathlineto{\pgfqpoint{4.570408in}{2.380222in}}%
\pgfusepath{stroke}%
\end{pgfscope}%
\begin{pgfscope}%
\pgfpathrectangle{\pgfqpoint{0.647939in}{0.492442in}}{\pgfqpoint{4.273799in}{2.331163in}}%
\pgfusepath{clip}%
\pgfsetbuttcap%
\pgfsetroundjoin%
\pgfsetlinewidth{0.301125pt}%
\definecolor{currentstroke}{rgb}{0.500000,0.500000,0.500000}%
\pgfsetstrokecolor{currentstroke}%
\pgfsetstrokeopacity{0.300000}%
\pgfsetdash{}{0pt}%
\pgfpathmoveto{\pgfqpoint{3.853289in}{2.823605in}}%
\pgfpathlineto{\pgfqpoint{3.853289in}{2.823605in}}%
\pgfpathlineto{\pgfqpoint{3.885365in}{2.774846in}}%
\pgfpathlineto{\pgfqpoint{3.917098in}{2.726020in}}%
\pgfpathlineto{\pgfqpoint{3.948533in}{2.677136in}}%
\pgfpathlineto{\pgfqpoint{3.979698in}{2.628202in}}%
\pgfpathlineto{\pgfqpoint{4.010639in}{2.579225in}}%
\pgfpathlineto{\pgfqpoint{4.041428in}{2.530220in}}%
\pgfpathlineto{\pgfqpoint{4.072147in}{2.481202in}}%
\pgfpathlineto{\pgfqpoint{4.102889in}{2.432189in}}%
\pgfpathlineto{\pgfqpoint{4.133830in}{2.383214in}}%
\pgfpathlineto{\pgfqpoint{4.165254in}{2.334335in}}%
\pgfpathlineto{\pgfqpoint{4.197616in}{2.285644in}}%
\pgfpathlineto{\pgfqpoint{4.231862in}{2.237338in}}%
\pgfpathlineto{\pgfqpoint{4.270584in}{2.190104in}}%
\pgfpathlineto{\pgfqpoint{4.270584in}{2.190104in}}%
\pgfpathlineto{\pgfqpoint{4.310283in}{2.156086in}}%
\pgfpathlineto{\pgfqpoint{4.310283in}{2.156086in}}%
\pgfpathlineto{\pgfqpoint{4.335030in}{2.145604in}}%
\pgfpathlineto{\pgfqpoint{4.335030in}{2.145604in}}%
\pgfpathlineto{\pgfqpoint{4.361234in}{2.144861in}}%
\pgfpathlineto{\pgfqpoint{4.384868in}{2.152159in}}%
\pgfpathlineto{\pgfqpoint{4.406683in}{2.164362in}}%
\pgfpathlineto{\pgfqpoint{4.437177in}{2.187605in}}%
\pgfpathlineto{\pgfqpoint{4.476364in}{2.224485in}}%
\pgfpathlineto{\pgfqpoint{4.519277in}{2.270365in}}%
\pgfpathlineto{\pgfqpoint{4.559634in}{2.317060in}}%
\pgfusepath{stroke}%
\end{pgfscope}%
\begin{pgfscope}%
\pgfpathrectangle{\pgfqpoint{0.647939in}{0.492442in}}{\pgfqpoint{4.273799in}{2.331163in}}%
\pgfusepath{clip}%
\pgfsetbuttcap%
\pgfsetroundjoin%
\pgfsetlinewidth{0.301125pt}%
\definecolor{currentstroke}{rgb}{0.500000,0.500000,0.500000}%
\pgfsetstrokecolor{currentstroke}%
\pgfsetstrokeopacity{0.300000}%
\pgfsetdash{}{0pt}%
\pgfpathmoveto{\pgfqpoint{3.756157in}{2.823605in}}%
\pgfpathlineto{\pgfqpoint{3.756157in}{2.823605in}}%
\pgfpathlineto{\pgfqpoint{3.787835in}{2.774769in}}%
\pgfpathlineto{\pgfqpoint{3.818947in}{2.725824in}}%
\pgfpathlineto{\pgfqpoint{3.849488in}{2.676772in}}%
\pgfpathlineto{\pgfqpoint{3.879455in}{2.627615in}}%
\pgfpathlineto{\pgfqpoint{3.908852in}{2.578356in}}%
\pgfpathlineto{\pgfqpoint{3.937662in}{2.528994in}}%
\pgfpathlineto{\pgfqpoint{3.965858in}{2.479527in}}%
\pgfpathlineto{\pgfqpoint{3.993413in}{2.429953in}}%
\pgfpathlineto{\pgfqpoint{4.020266in}{2.380264in}}%
\pgfpathlineto{\pgfqpoint{4.046330in}{2.330451in}}%
\pgfpathlineto{\pgfqpoint{4.071477in}{2.280499in}}%
\pgfpathlineto{\pgfqpoint{4.095476in}{2.230380in}}%
\pgfpathlineto{\pgfqpoint{4.117961in}{2.180056in}}%
\pgfpathlineto{\pgfqpoint{4.138248in}{2.129459in}}%
\pgfpathlineto{\pgfqpoint{4.155004in}{2.078492in}}%
\pgfpathlineto{\pgfqpoint{4.165375in}{2.027060in}}%
\pgfpathlineto{\pgfqpoint{4.163316in}{1.975475in}}%
\pgfpathlineto{\pgfqpoint{4.141743in}{1.925662in}}%
\pgfpathlineto{\pgfqpoint{4.109447in}{1.885147in}}%
\pgfpathlineto{\pgfqpoint{4.064406in}{1.843932in}}%
\pgfusepath{stroke}%
\end{pgfscope}%
\begin{pgfscope}%
\pgfpathrectangle{\pgfqpoint{0.647939in}{0.492442in}}{\pgfqpoint{4.273799in}{2.331163in}}%
\pgfusepath{clip}%
\pgfsetbuttcap%
\pgfsetroundjoin%
\pgfsetlinewidth{0.301125pt}%
\definecolor{currentstroke}{rgb}{0.500000,0.500000,0.500000}%
\pgfsetstrokecolor{currentstroke}%
\pgfsetstrokeopacity{0.300000}%
\pgfsetdash{}{0pt}%
\pgfpathmoveto{\pgfqpoint{3.659025in}{2.823605in}}%
\pgfpathlineto{\pgfqpoint{3.659025in}{2.823605in}}%
\pgfpathlineto{\pgfqpoint{3.690888in}{2.774805in}}%
\pgfpathlineto{\pgfqpoint{3.721999in}{2.725860in}}%
\pgfpathlineto{\pgfqpoint{3.752346in}{2.676773in}}%
\pgfpathlineto{\pgfqpoint{3.781918in}{2.627545in}}%
\pgfpathlineto{\pgfqpoint{3.810690in}{2.578176in}}%
\pgfpathlineto{\pgfqpoint{3.838614in}{2.528663in}}%
\pgfpathlineto{\pgfqpoint{3.865641in}{2.479002in}}%
\pgfpathlineto{\pgfqpoint{3.891693in}{2.429187in}}%
\pgfpathlineto{\pgfqpoint{3.916658in}{2.379207in}}%
\pgfpathlineto{\pgfqpoint{3.940391in}{2.329048in}}%
\pgfpathlineto{\pgfqpoint{3.962671in}{2.278693in}}%
\pgfpathlineto{\pgfqpoint{3.983195in}{2.228117in}}%
\pgfpathlineto{\pgfqpoint{4.001516in}{2.177293in}}%
\pgfpathlineto{\pgfqpoint{4.016961in}{2.126189in}}%
\pgfpathlineto{\pgfqpoint{4.028527in}{2.074789in}}%
\pgfpathlineto{\pgfqpoint{4.034706in}{2.023125in}}%
\pgfpathlineto{\pgfqpoint{4.033304in}{1.971388in}}%
\pgfpathlineto{\pgfqpoint{4.021622in}{1.920105in}}%
\pgfpathlineto{\pgfqpoint{3.997368in}{1.870227in}}%
\pgfpathlineto{\pgfqpoint{3.959931in}{1.822881in}}%
\pgfusepath{stroke}%
\end{pgfscope}%
\begin{pgfscope}%
\pgfpathrectangle{\pgfqpoint{0.647939in}{0.492442in}}{\pgfqpoint{4.273799in}{2.331163in}}%
\pgfusepath{clip}%
\pgfsetbuttcap%
\pgfsetroundjoin%
\pgfsetlinewidth{0.301125pt}%
\definecolor{currentstroke}{rgb}{0.500000,0.500000,0.500000}%
\pgfsetstrokecolor{currentstroke}%
\pgfsetstrokeopacity{0.300000}%
\pgfsetdash{}{0pt}%
\pgfpathmoveto{\pgfqpoint{3.561893in}{2.823605in}}%
\pgfpathlineto{\pgfqpoint{3.561893in}{2.823605in}}%
\pgfpathlineto{\pgfqpoint{3.594450in}{2.774942in}}%
\pgfpathlineto{\pgfqpoint{3.626097in}{2.726099in}}%
\pgfpathlineto{\pgfqpoint{3.656826in}{2.677083in}}%
\pgfpathlineto{\pgfqpoint{3.686619in}{2.627895in}}%
\pgfpathlineto{\pgfqpoint{3.715433in}{2.578534in}}%
\pgfpathlineto{\pgfqpoint{3.743221in}{2.528999in}}%
\pgfpathlineto{\pgfqpoint{3.769924in}{2.479286in}}%
\pgfpathlineto{\pgfqpoint{3.795441in}{2.429389in}}%
\pgfpathlineto{\pgfqpoint{3.819659in}{2.379299in}}%
\pgfpathlineto{\pgfqpoint{3.842411in}{2.329006in}}%
\pgfpathlineto{\pgfqpoint{3.863479in}{2.278496in}}%
\pgfpathlineto{\pgfqpoint{3.882567in}{2.227754in}}%
\pgfpathlineto{\pgfqpoint{3.899262in}{2.176763in}}%
\pgfpathlineto{\pgfqpoint{3.913003in}{2.125514in}}%
\pgfpathlineto{\pgfqpoint{3.922990in}{2.074013in}}%
\pgfpathlineto{\pgfqpoint{3.928128in}{2.022308in}}%
\pgfpathlineto{\pgfqpoint{3.926941in}{1.970549in}}%
\pgfpathlineto{\pgfqpoint{3.917558in}{1.919074in}}%
\pgfpathlineto{\pgfqpoint{3.897892in}{1.868520in}}%
\pgfpathlineto{\pgfqpoint{3.866175in}{1.819882in}}%
\pgfpathlineto{\pgfqpoint{3.821416in}{1.774464in}}%
\pgfpathlineto{\pgfqpoint{3.763525in}{1.733782in}}%
\pgfpathlineto{\pgfqpoint{3.692941in}{1.699653in}}%
\pgfusepath{stroke}%
\end{pgfscope}%
\begin{pgfscope}%
\pgfpathrectangle{\pgfqpoint{0.647939in}{0.492442in}}{\pgfqpoint{4.273799in}{2.331163in}}%
\pgfusepath{clip}%
\pgfsetbuttcap%
\pgfsetroundjoin%
\pgfsetlinewidth{0.301125pt}%
\definecolor{currentstroke}{rgb}{0.500000,0.500000,0.500000}%
\pgfsetstrokecolor{currentstroke}%
\pgfsetstrokeopacity{0.300000}%
\pgfsetdash{}{0pt}%
\pgfpathmoveto{\pgfqpoint{3.464761in}{2.823605in}}%
\pgfpathlineto{\pgfqpoint{3.464761in}{2.823605in}}%
\pgfpathlineto{\pgfqpoint{3.498508in}{2.775184in}}%
\pgfpathlineto{\pgfqpoint{3.531200in}{2.726547in}}%
\pgfpathlineto{\pgfqpoint{3.562830in}{2.677702in}}%
\pgfpathlineto{\pgfqpoint{3.593370in}{2.628651in}}%
\pgfpathlineto{\pgfqpoint{3.622779in}{2.579395in}}%
\pgfpathlineto{\pgfqpoint{3.651016in}{2.529935in}}%
\pgfpathlineto{\pgfqpoint{3.678012in}{2.480270in}}%
\pgfpathlineto{\pgfqpoint{3.703671in}{2.430395in}}%
\pgfpathlineto{\pgfqpoint{3.727878in}{2.380304in}}%
\pgfpathlineto{\pgfqpoint{3.750466in}{2.329989in}}%
\pgfpathlineto{\pgfqpoint{3.771227in}{2.279442in}}%
\pgfpathlineto{\pgfqpoint{3.789879in}{2.228651in}}%
\pgfusepath{stroke}%
\end{pgfscope}%
\begin{pgfscope}%
\pgfpathrectangle{\pgfqpoint{0.647939in}{0.492442in}}{\pgfqpoint{4.273799in}{2.331163in}}%
\pgfusepath{clip}%
\pgfsetbuttcap%
\pgfsetroundjoin%
\pgfsetlinewidth{0.301125pt}%
\definecolor{currentstroke}{rgb}{0.500000,0.500000,0.500000}%
\pgfsetstrokecolor{currentstroke}%
\pgfsetstrokeopacity{0.300000}%
\pgfsetdash{}{0pt}%
\pgfpathmoveto{\pgfqpoint{3.367630in}{2.823605in}}%
\pgfpathlineto{\pgfqpoint{3.367630in}{2.823605in}}%
\pgfpathlineto{\pgfqpoint{3.403082in}{2.775548in}}%
\pgfpathlineto{\pgfqpoint{3.437327in}{2.727231in}}%
\pgfpathlineto{\pgfqpoint{3.470352in}{2.678662in}}%
\pgfpathlineto{\pgfqpoint{3.502133in}{2.629847in}}%
\pgfpathlineto{\pgfqpoint{3.532640in}{2.580790in}}%
\pgfpathlineto{\pgfqpoint{3.561832in}{2.531496in}}%
\pgfpathlineto{\pgfqpoint{3.589641in}{2.481965in}}%
\pgfusepath{stroke}%
\end{pgfscope}%
\begin{pgfscope}%
\pgfpathrectangle{\pgfqpoint{0.647939in}{0.492442in}}{\pgfqpoint{4.273799in}{2.331163in}}%
\pgfusepath{clip}%
\pgfsetbuttcap%
\pgfsetroundjoin%
\pgfsetlinewidth{0.301125pt}%
\definecolor{currentstroke}{rgb}{0.500000,0.500000,0.500000}%
\pgfsetstrokecolor{currentstroke}%
\pgfsetstrokeopacity{0.300000}%
\pgfsetdash{}{0pt}%
\pgfpathmoveto{\pgfqpoint{3.270498in}{2.823605in}}%
\pgfpathlineto{\pgfqpoint{3.270498in}{2.823605in}}%
\pgfpathlineto{\pgfqpoint{3.308211in}{2.776064in}}%
\pgfpathlineto{\pgfqpoint{3.344533in}{2.728201in}}%
\pgfpathlineto{\pgfqpoint{3.379462in}{2.680031in}}%
\pgfpathlineto{\pgfqpoint{3.412982in}{2.631564in}}%
\pgfpathlineto{\pgfqpoint{3.445072in}{2.582809in}}%
\pgfpathlineto{\pgfqpoint{3.475697in}{2.533775in}}%
\pgfpathlineto{\pgfqpoint{3.504794in}{2.484465in}}%
\pgfpathlineto{\pgfqpoint{3.532281in}{2.434881in}}%
\pgfpathlineto{\pgfqpoint{3.558051in}{2.385024in}}%
\pgfpathlineto{\pgfqpoint{3.581952in}{2.334891in}}%
\pgfpathlineto{\pgfqpoint{3.603792in}{2.284480in}}%
\pgfpathlineto{\pgfqpoint{3.623313in}{2.233788in}}%
\pgfpathlineto{\pgfqpoint{3.640172in}{2.182814in}}%
\pgfpathlineto{\pgfqpoint{3.653915in}{2.131565in}}%
\pgfpathlineto{\pgfqpoint{3.663919in}{2.080064in}}%
\pgfpathlineto{\pgfqpoint{3.669345in}{2.028367in}}%
\pgfpathlineto{\pgfqpoint{3.669040in}{1.976595in}}%
\pgfpathlineto{\pgfqpoint{3.661398in}{1.925013in}}%
\pgfpathlineto{\pgfqpoint{3.644160in}{1.874176in}}%
\pgfpathlineto{\pgfqpoint{3.614246in}{1.825223in}}%
\pgfpathlineto{\pgfqpoint{3.567634in}{1.780571in}}%
\pgfpathlineto{\pgfqpoint{3.567634in}{1.780571in}}%
\pgfpathlineto{\pgfqpoint{3.518573in}{1.752149in}}%
\pgfpathlineto{\pgfqpoint{3.456340in}{1.732223in}}%
\pgfpathlineto{\pgfqpoint{3.394779in}{1.724934in}}%
\pgfpathlineto{\pgfqpoint{3.336563in}{1.727426in}}%
\pgfpathlineto{\pgfqpoint{3.279663in}{1.737994in}}%
\pgfpathlineto{\pgfqpoint{3.223343in}{1.756413in}}%
\pgfpathlineto{\pgfqpoint{3.167856in}{1.782951in}}%
\pgfusepath{stroke}%
\end{pgfscope}%
\begin{pgfscope}%
\pgfpathrectangle{\pgfqpoint{0.647939in}{0.492442in}}{\pgfqpoint{4.273799in}{2.331163in}}%
\pgfusepath{clip}%
\pgfsetbuttcap%
\pgfsetroundjoin%
\pgfsetlinewidth{0.301125pt}%
\definecolor{currentstroke}{rgb}{0.500000,0.500000,0.500000}%
\pgfsetstrokecolor{currentstroke}%
\pgfsetstrokeopacity{0.300000}%
\pgfsetdash{}{0pt}%
\pgfpathmoveto{\pgfqpoint{3.076234in}{2.823605in}}%
\pgfpathlineto{\pgfqpoint{3.076234in}{2.823605in}}%
\pgfpathlineto{\pgfqpoint{3.120323in}{2.777728in}}%
\pgfpathlineto{\pgfqpoint{3.162555in}{2.731334in}}%
\pgfpathlineto{\pgfqpoint{3.202952in}{2.684455in}}%
\pgfpathlineto{\pgfqpoint{3.241526in}{2.637121in}}%
\pgfpathlineto{\pgfqpoint{3.278279in}{2.589357in}}%
\pgfpathlineto{\pgfqpoint{3.313199in}{2.541186in}}%
\pgfpathlineto{\pgfqpoint{3.346255in}{2.492626in}}%
\pgfpathlineto{\pgfqpoint{3.377384in}{2.443689in}}%
\pgfpathlineto{\pgfqpoint{3.406497in}{2.394384in}}%
\pgfpathlineto{\pgfqpoint{3.433474in}{2.344719in}}%
\pgfpathlineto{\pgfqpoint{3.458139in}{2.294699in}}%
\pgfpathlineto{\pgfqpoint{3.480256in}{2.244326in}}%
\pgfpathlineto{\pgfqpoint{3.499501in}{2.193606in}}%
\pgfpathlineto{\pgfqpoint{3.515431in}{2.142547in}}%
\pgfpathlineto{\pgfqpoint{3.527437in}{2.091173in}}%
\pgfpathlineto{\pgfqpoint{3.534648in}{2.039543in}}%
\pgfpathlineto{\pgfqpoint{3.535810in}{1.987783in}}%
\pgfpathlineto{\pgfqpoint{3.529033in}{1.936182in}}%
\pgfpathlineto{\pgfqpoint{3.511324in}{1.885426in}}%
\pgfpathlineto{\pgfqpoint{3.477795in}{1.837290in}}%
\pgfpathlineto{\pgfqpoint{3.477795in}{1.837290in}}%
\pgfpathlineto{\pgfqpoint{3.438780in}{1.805971in}}%
\pgfpathlineto{\pgfqpoint{3.438780in}{1.805971in}}%
\pgfpathlineto{\pgfqpoint{3.394570in}{1.785935in}}%
\pgfpathlineto{\pgfqpoint{3.340144in}{1.775432in}}%
\pgfusepath{stroke}%
\end{pgfscope}%
\begin{pgfscope}%
\pgfpathrectangle{\pgfqpoint{0.647939in}{0.492442in}}{\pgfqpoint{4.273799in}{2.331163in}}%
\pgfusepath{clip}%
\pgfsetbuttcap%
\pgfsetroundjoin%
\pgfsetlinewidth{0.301125pt}%
\definecolor{currentstroke}{rgb}{0.500000,0.500000,0.500000}%
\pgfsetstrokecolor{currentstroke}%
\pgfsetstrokeopacity{0.300000}%
\pgfsetdash{}{0pt}%
\pgfpathmoveto{\pgfqpoint{2.881971in}{2.823605in}}%
\pgfpathlineto{\pgfqpoint{2.881971in}{2.823605in}}%
\pgfpathlineto{\pgfqpoint{2.935236in}{2.780728in}}%
\pgfpathlineto{\pgfqpoint{2.986001in}{2.736956in}}%
\pgfpathlineto{\pgfqpoint{3.034302in}{2.692363in}}%
\pgfpathlineto{\pgfqpoint{3.080186in}{2.647017in}}%
\pgfpathlineto{\pgfqpoint{3.123694in}{2.600980in}}%
\pgfpathlineto{\pgfqpoint{3.164857in}{2.554304in}}%
\pgfpathlineto{\pgfqpoint{3.203683in}{2.507036in}}%
\pgfpathlineto{\pgfqpoint{3.240159in}{2.459212in}}%
\pgfpathlineto{\pgfqpoint{3.274242in}{2.410866in}}%
\pgfpathlineto{\pgfqpoint{3.305854in}{2.362024in}}%
\pgfpathlineto{\pgfqpoint{3.334862in}{2.312704in}}%
\pgfpathlineto{\pgfqpoint{3.361063in}{2.262920in}}%
\pgfpathlineto{\pgfqpoint{3.384177in}{2.212684in}}%
\pgfpathlineto{\pgfqpoint{3.403792in}{2.162009in}}%
\pgfpathlineto{\pgfqpoint{3.419318in}{2.110920in}}%
\pgfpathlineto{\pgfqpoint{3.429885in}{2.059464in}}%
\pgfpathlineto{\pgfqpoint{3.434159in}{2.007757in}}%
\pgfpathlineto{\pgfqpoint{3.429975in}{1.956082in}}%
\pgfpathlineto{\pgfqpoint{3.413521in}{1.905234in}}%
\pgfpathlineto{\pgfqpoint{3.413521in}{1.905234in}}%
\pgfpathlineto{\pgfqpoint{3.385067in}{1.864808in}}%
\pgfpathlineto{\pgfqpoint{3.385067in}{1.864808in}}%
\pgfpathlineto{\pgfqpoint{3.350172in}{1.838563in}}%
\pgfpathlineto{\pgfqpoint{3.350172in}{1.838563in}}%
\pgfpathlineto{\pgfqpoint{3.310900in}{1.823593in}}%
\pgfpathlineto{\pgfqpoint{3.263890in}{1.818422in}}%
\pgfpathlineto{\pgfqpoint{3.221058in}{1.822694in}}%
\pgfpathlineto{\pgfqpoint{3.179215in}{1.834239in}}%
\pgfpathlineto{\pgfqpoint{3.136743in}{1.853495in}}%
\pgfpathlineto{\pgfqpoint{3.094547in}{1.881343in}}%
\pgfusepath{stroke}%
\end{pgfscope}%
\begin{pgfscope}%
\pgfpathrectangle{\pgfqpoint{0.647939in}{0.492442in}}{\pgfqpoint{4.273799in}{2.331163in}}%
\pgfusepath{clip}%
\pgfsetbuttcap%
\pgfsetroundjoin%
\pgfsetlinewidth{0.301125pt}%
\definecolor{currentstroke}{rgb}{0.500000,0.500000,0.500000}%
\pgfsetstrokecolor{currentstroke}%
\pgfsetstrokeopacity{0.300000}%
\pgfsetdash{}{0pt}%
\pgfpathmoveto{\pgfqpoint{2.687707in}{2.823605in}}%
\pgfpathlineto{\pgfqpoint{2.687707in}{2.823605in}}%
\pgfpathlineto{\pgfqpoint{2.752573in}{2.785797in}}%
\pgfpathlineto{\pgfqpoint{2.814296in}{2.746454in}}%
\pgfpathlineto{\pgfqpoint{2.872828in}{2.705685in}}%
\pgfpathlineto{\pgfqpoint{2.928191in}{2.663617in}}%
\pgfpathlineto{\pgfqpoint{2.980445in}{2.620377in}}%
\pgfpathlineto{\pgfqpoint{3.029661in}{2.576086in}}%
\pgfpathlineto{\pgfqpoint{3.075906in}{2.530854in}}%
\pgfpathlineto{\pgfqpoint{3.119239in}{2.484771in}}%
\pgfpathlineto{\pgfqpoint{3.159689in}{2.437914in}}%
\pgfpathlineto{\pgfqpoint{3.197244in}{2.390343in}}%
\pgfusepath{stroke}%
\end{pgfscope}%
\begin{pgfscope}%
\pgfpathrectangle{\pgfqpoint{0.647939in}{0.492442in}}{\pgfqpoint{4.273799in}{2.331163in}}%
\pgfusepath{clip}%
\pgfsetbuttcap%
\pgfsetroundjoin%
\pgfsetlinewidth{0.301125pt}%
\definecolor{currentstroke}{rgb}{0.500000,0.500000,0.500000}%
\pgfsetstrokecolor{currentstroke}%
\pgfsetstrokeopacity{0.300000}%
\pgfsetdash{}{0pt}%
\pgfpathmoveto{\pgfqpoint{2.396312in}{2.823605in}}%
\pgfpathlineto{\pgfqpoint{2.396312in}{2.823605in}}%
\pgfpathlineto{\pgfqpoint{2.477339in}{2.796620in}}%
\pgfpathlineto{\pgfqpoint{2.555990in}{2.767642in}}%
\pgfpathlineto{\pgfqpoint{2.631536in}{2.736303in}}%
\pgfpathlineto{\pgfqpoint{2.703417in}{2.702506in}}%
\pgfpathlineto{\pgfqpoint{2.771330in}{2.666355in}}%
\pgfpathlineto{\pgfqpoint{2.835180in}{2.628057in}}%
\pgfpathlineto{\pgfqpoint{2.894986in}{2.587856in}}%
\pgfpathlineto{\pgfqpoint{2.950847in}{2.545994in}}%
\pgfpathlineto{\pgfqpoint{3.002885in}{2.502692in}}%
\pgfpathlineto{\pgfqpoint{3.051234in}{2.458133in}}%
\pgfpathlineto{\pgfqpoint{3.095995in}{2.412468in}}%
\pgfpathlineto{\pgfqpoint{3.137218in}{2.365820in}}%
\pgfpathlineto{\pgfqpoint{3.174891in}{2.318288in}}%
\pgfpathlineto{\pgfqpoint{3.208915in}{2.269943in}}%
\pgfpathlineto{\pgfqpoint{3.239074in}{2.220839in}}%
\pgfpathlineto{\pgfqpoint{3.264999in}{2.171023in}}%
\pgfpathlineto{\pgfqpoint{3.286078in}{2.120538in}}%
\pgfpathlineto{\pgfqpoint{3.301286in}{2.069438in}}%
\pgfpathlineto{\pgfqpoint{3.308850in}{2.017863in}}%
\pgfpathlineto{\pgfqpoint{3.305342in}{1.966251in}}%
\pgfpathlineto{\pgfqpoint{3.282877in}{1.916506in}}%
\pgfpathlineto{\pgfqpoint{3.282877in}{1.916506in}}%
\pgfpathlineto{\pgfqpoint{3.256702in}{1.893054in}}%
\pgfpathlineto{\pgfqpoint{3.256702in}{1.893054in}}%
\pgfpathlineto{\pgfqpoint{3.225364in}{1.880630in}}%
\pgfusepath{stroke}%
\end{pgfscope}%
\begin{pgfscope}%
\pgfpathrectangle{\pgfqpoint{0.647939in}{0.492442in}}{\pgfqpoint{4.273799in}{2.331163in}}%
\pgfusepath{clip}%
\pgfsetbuttcap%
\pgfsetroundjoin%
\pgfsetlinewidth{0.301125pt}%
\definecolor{currentstroke}{rgb}{0.500000,0.500000,0.500000}%
\pgfsetstrokecolor{currentstroke}%
\pgfsetstrokeopacity{0.300000}%
\pgfsetdash{}{0pt}%
\pgfpathmoveto{\pgfqpoint{2.104916in}{2.823605in}}%
\pgfpathlineto{\pgfqpoint{2.104916in}{2.823605in}}%
\pgfpathlineto{\pgfqpoint{2.190585in}{2.801292in}}%
\pgfpathlineto{\pgfqpoint{2.277655in}{2.780616in}}%
\pgfpathlineto{\pgfqpoint{2.364683in}{2.759901in}}%
\pgfpathlineto{\pgfqpoint{2.450405in}{2.737666in}}%
\pgfpathlineto{\pgfqpoint{2.533683in}{2.712864in}}%
\pgfpathlineto{\pgfqpoint{2.613549in}{2.684952in}}%
\pgfpathlineto{\pgfqpoint{2.689300in}{2.653818in}}%
\pgfpathlineto{\pgfqpoint{2.760541in}{2.619641in}}%
\pgfpathlineto{\pgfqpoint{2.827082in}{2.582751in}}%
\pgfpathlineto{\pgfqpoint{2.888974in}{2.543515in}}%
\pgfusepath{stroke}%
\end{pgfscope}%
\begin{pgfscope}%
\pgfpathrectangle{\pgfqpoint{0.647939in}{0.492442in}}{\pgfqpoint{4.273799in}{2.331163in}}%
\pgfusepath{clip}%
\pgfsetbuttcap%
\pgfsetroundjoin%
\pgfsetlinewidth{0.301125pt}%
\definecolor{currentstroke}{rgb}{0.500000,0.500000,0.500000}%
\pgfsetstrokecolor{currentstroke}%
\pgfsetstrokeopacity{0.300000}%
\pgfsetdash{}{0pt}%
\pgfpathmoveto{\pgfqpoint{1.910652in}{2.823605in}}%
\pgfpathlineto{\pgfqpoint{1.910652in}{2.823605in}}%
\pgfpathlineto{\pgfqpoint{1.989532in}{2.794945in}}%
\pgfpathlineto{\pgfqpoint{2.074022in}{2.771477in}}%
\pgfpathlineto{\pgfqpoint{2.162150in}{2.752281in}}%
\pgfpathlineto{\pgfqpoint{2.251968in}{2.735478in}}%
\pgfpathlineto{\pgfqpoint{2.341916in}{2.718900in}}%
\pgfpathlineto{\pgfqpoint{2.430713in}{2.700632in}}%
\pgfpathlineto{\pgfqpoint{2.517148in}{2.679320in}}%
\pgfusepath{stroke}%
\end{pgfscope}%
\begin{pgfscope}%
\pgfpathrectangle{\pgfqpoint{0.647939in}{0.492442in}}{\pgfqpoint{4.273799in}{2.331163in}}%
\pgfusepath{clip}%
\pgfsetbuttcap%
\pgfsetroundjoin%
\pgfsetlinewidth{0.301125pt}%
\definecolor{currentstroke}{rgb}{0.500000,0.500000,0.500000}%
\pgfsetstrokecolor{currentstroke}%
\pgfsetstrokeopacity{0.300000}%
\pgfsetdash{}{0pt}%
\pgfpathmoveto{\pgfqpoint{1.716389in}{2.823605in}}%
\pgfpathlineto{\pgfqpoint{1.716389in}{2.823605in}}%
\pgfpathlineto{\pgfqpoint{1.777728in}{2.784154in}}%
\pgfpathlineto{\pgfqpoint{1.847083in}{2.748958in}}%
\pgfpathlineto{\pgfqpoint{1.925143in}{2.719792in}}%
\pgfpathlineto{\pgfqpoint{2.010836in}{2.697945in}}%
\pgfpathlineto{\pgfqpoint{2.101496in}{2.682943in}}%
\pgfpathlineto{\pgfqpoint{2.194414in}{2.672424in}}%
\pgfpathlineto{\pgfqpoint{2.287869in}{2.663294in}}%
\pgfpathlineto{\pgfqpoint{2.380764in}{2.652698in}}%
\pgfusepath{stroke}%
\end{pgfscope}%
\begin{pgfscope}%
\pgfpathrectangle{\pgfqpoint{0.647939in}{0.492442in}}{\pgfqpoint{4.273799in}{2.331163in}}%
\pgfusepath{clip}%
\pgfsetbuttcap%
\pgfsetroundjoin%
\pgfsetlinewidth{0.301125pt}%
\definecolor{currentstroke}{rgb}{0.500000,0.500000,0.500000}%
\pgfsetstrokecolor{currentstroke}%
\pgfsetstrokeopacity{0.300000}%
\pgfsetdash{}{0pt}%
\pgfpathmoveto{\pgfqpoint{1.619257in}{2.823605in}}%
\pgfpathlineto{\pgfqpoint{1.619257in}{2.823605in}}%
\pgfpathlineto{\pgfqpoint{1.669880in}{2.779830in}}%
\pgfpathlineto{\pgfqpoint{1.726853in}{2.738469in}}%
\pgfpathlineto{\pgfqpoint{1.792048in}{2.700974in}}%
\pgfpathlineto{\pgfqpoint{1.867122in}{2.669634in}}%
\pgfpathlineto{\pgfqpoint{1.951921in}{2.647022in}}%
\pgfpathlineto{\pgfqpoint{2.040104in}{2.634109in}}%
\pgfpathlineto{\pgfqpoint{2.134140in}{2.627626in}}%
\pgfpathlineto{\pgfqpoint{2.228854in}{2.624287in}}%
\pgfpathlineto{\pgfqpoint{2.323491in}{2.620459in}}%
\pgfpathlineto{\pgfqpoint{2.417416in}{2.613328in}}%
\pgfpathlineto{\pgfqpoint{2.509414in}{2.600982in}}%
\pgfpathlineto{\pgfqpoint{2.597859in}{2.582563in}}%
\pgfpathlineto{\pgfqpoint{2.681310in}{2.558163in}}%
\pgfpathlineto{\pgfqpoint{2.758866in}{2.528469in}}%
\pgfpathlineto{\pgfqpoint{2.830146in}{2.494401in}}%
\pgfpathlineto{\pgfqpoint{2.895301in}{2.456829in}}%
\pgfpathlineto{\pgfqpoint{2.954662in}{2.416472in}}%
\pgfpathlineto{\pgfqpoint{3.008565in}{2.373891in}}%
\pgfpathlineto{\pgfqpoint{3.057324in}{2.329488in}}%
\pgfusepath{stroke}%
\end{pgfscope}%
\begin{pgfscope}%
\pgfpathrectangle{\pgfqpoint{0.647939in}{0.492442in}}{\pgfqpoint{4.273799in}{2.331163in}}%
\pgfusepath{clip}%
\pgfsetbuttcap%
\pgfsetroundjoin%
\pgfsetlinewidth{0.301125pt}%
\definecolor{currentstroke}{rgb}{0.500000,0.500000,0.500000}%
\pgfsetstrokecolor{currentstroke}%
\pgfsetstrokeopacity{0.300000}%
\pgfsetdash{}{0pt}%
\pgfpathmoveto{\pgfqpoint{1.522125in}{2.823605in}}%
\pgfpathlineto{\pgfqpoint{1.522125in}{2.823605in}}%
\pgfpathlineto{\pgfqpoint{1.563005in}{2.776866in}}%
\pgfpathlineto{\pgfqpoint{1.607775in}{2.731210in}}%
\pgfpathlineto{\pgfqpoint{1.657779in}{2.687232in}}%
\pgfpathlineto{\pgfqpoint{1.715043in}{2.646044in}}%
\pgfpathlineto{\pgfqpoint{1.782399in}{2.609832in}}%
\pgfpathlineto{\pgfqpoint{1.862118in}{2.582484in}}%
\pgfpathlineto{\pgfqpoint{1.942040in}{2.568766in}}%
\pgfpathlineto{\pgfqpoint{2.020046in}{2.564818in}}%
\pgfpathlineto{\pgfqpoint{2.108563in}{2.567018in}}%
\pgfpathlineto{\pgfqpoint{2.202966in}{2.572397in}}%
\pgfpathlineto{\pgfqpoint{2.297484in}{2.576804in}}%
\pgfpathlineto{\pgfqpoint{2.392220in}{2.577137in}}%
\pgfpathlineto{\pgfqpoint{2.486205in}{2.571259in}}%
\pgfusepath{stroke}%
\end{pgfscope}%
\begin{pgfscope}%
\pgfpathrectangle{\pgfqpoint{0.647939in}{0.492442in}}{\pgfqpoint{4.273799in}{2.331163in}}%
\pgfusepath{clip}%
\pgfsetbuttcap%
\pgfsetroundjoin%
\pgfsetlinewidth{0.301125pt}%
\definecolor{currentstroke}{rgb}{0.500000,0.500000,0.500000}%
\pgfsetstrokecolor{currentstroke}%
\pgfsetstrokeopacity{0.300000}%
\pgfsetdash{}{0pt}%
\pgfpathmoveto{\pgfqpoint{1.424993in}{2.823605in}}%
\pgfpathlineto{\pgfqpoint{1.424993in}{2.823605in}}%
\pgfpathlineto{\pgfqpoint{1.457794in}{2.774995in}}%
\pgfpathlineto{\pgfqpoint{1.492568in}{2.726803in}}%
\pgfpathlineto{\pgfqpoint{1.529901in}{2.679190in}}%
\pgfpathlineto{\pgfqpoint{1.570669in}{2.632436in}}%
\pgfpathlineto{\pgfqpoint{1.616266in}{2.587055in}}%
\pgfpathlineto{\pgfqpoint{1.669040in}{2.544128in}}%
\pgfpathlineto{\pgfqpoint{1.733089in}{2.506226in}}%
\pgfpathlineto{\pgfqpoint{1.733089in}{2.506226in}}%
\pgfpathlineto{\pgfqpoint{1.792830in}{2.484171in}}%
\pgfpathlineto{\pgfqpoint{1.863096in}{2.472581in}}%
\pgfpathlineto{\pgfqpoint{1.926762in}{2.472294in}}%
\pgfpathlineto{\pgfqpoint{1.994106in}{2.479162in}}%
\pgfpathlineto{\pgfqpoint{2.076948in}{2.493126in}}%
\pgfpathlineto{\pgfqpoint{2.166616in}{2.510109in}}%
\pgfpathlineto{\pgfqpoint{2.257273in}{2.525274in}}%
\pgfusepath{stroke}%
\end{pgfscope}%
\begin{pgfscope}%
\pgfpathrectangle{\pgfqpoint{0.647939in}{0.492442in}}{\pgfqpoint{4.273799in}{2.331163in}}%
\pgfusepath{clip}%
\pgfsetbuttcap%
\pgfsetroundjoin%
\pgfsetlinewidth{0.301125pt}%
\definecolor{currentstroke}{rgb}{0.500000,0.500000,0.500000}%
\pgfsetstrokecolor{currentstroke}%
\pgfsetstrokeopacity{0.300000}%
\pgfsetdash{}{0pt}%
\pgfpathmoveto{\pgfqpoint{1.327862in}{2.823605in}}%
\pgfpathlineto{\pgfqpoint{1.327862in}{2.823605in}}%
\pgfpathlineto{\pgfqpoint{1.354272in}{2.773849in}}%
\pgfpathlineto{\pgfqpoint{1.381522in}{2.724227in}}%
\pgfpathlineto{\pgfqpoint{1.409771in}{2.674773in}}%
\pgfpathlineto{\pgfqpoint{1.439232in}{2.625534in}}%
\pgfpathlineto{\pgfqpoint{1.470265in}{2.576582in}}%
\pgfpathlineto{\pgfqpoint{1.503363in}{2.528037in}}%
\pgfpathlineto{\pgfqpoint{1.539325in}{2.480119in}}%
\pgfpathlineto{\pgfqpoint{1.579619in}{2.433272in}}%
\pgfpathlineto{\pgfqpoint{1.627209in}{2.388612in}}%
\pgfpathlineto{\pgfqpoint{1.688994in}{2.349926in}}%
\pgfpathlineto{\pgfqpoint{1.688994in}{2.349926in}}%
\pgfpathlineto{\pgfqpoint{1.734278in}{2.335508in}}%
\pgfpathlineto{\pgfqpoint{1.787395in}{2.331969in}}%
\pgfpathlineto{\pgfqpoint{1.832889in}{2.337745in}}%
\pgfpathlineto{\pgfqpoint{1.881971in}{2.350383in}}%
\pgfpathlineto{\pgfqpoint{1.944045in}{2.372010in}}%
\pgfpathlineto{\pgfqpoint{2.021084in}{2.402222in}}%
\pgfpathlineto{\pgfqpoint{2.098531in}{2.432098in}}%
\pgfusepath{stroke}%
\end{pgfscope}%
\begin{pgfscope}%
\pgfpathrectangle{\pgfqpoint{0.647939in}{0.492442in}}{\pgfqpoint{4.273799in}{2.331163in}}%
\pgfusepath{clip}%
\pgfsetbuttcap%
\pgfsetroundjoin%
\pgfsetlinewidth{0.301125pt}%
\definecolor{currentstroke}{rgb}{0.500000,0.500000,0.500000}%
\pgfsetstrokecolor{currentstroke}%
\pgfsetstrokeopacity{0.300000}%
\pgfsetdash{}{0pt}%
\pgfpathmoveto{\pgfqpoint{1.230730in}{2.823605in}}%
\pgfpathlineto{\pgfqpoint{1.230730in}{2.823605in}}%
\pgfpathlineto{\pgfqpoint{1.252202in}{2.773145in}}%
\pgfpathlineto{\pgfqpoint{1.273872in}{2.722709in}}%
\pgfpathlineto{\pgfqpoint{1.295750in}{2.672301in}}%
\pgfpathlineto{\pgfqpoint{1.317846in}{2.621921in}}%
\pgfpathlineto{\pgfqpoint{1.340172in}{2.571571in}}%
\pgfpathlineto{\pgfqpoint{1.362746in}{2.521256in}}%
\pgfpathlineto{\pgfqpoint{1.385588in}{2.470977in}}%
\pgfpathlineto{\pgfqpoint{1.408728in}{2.420739in}}%
\pgfpathlineto{\pgfqpoint{1.432203in}{2.370550in}}%
\pgfpathlineto{\pgfqpoint{1.456081in}{2.320420in}}%
\pgfpathlineto{\pgfqpoint{1.480455in}{2.270366in}}%
\pgfpathlineto{\pgfqpoint{1.505526in}{2.220417in}}%
\pgfpathlineto{\pgfqpoint{1.531672in}{2.170655in}}%
\pgfpathlineto{\pgfqpoint{1.560034in}{2.121297in}}%
\pgfusepath{stroke}%
\end{pgfscope}%
\begin{pgfscope}%
\pgfpathrectangle{\pgfqpoint{0.647939in}{0.492442in}}{\pgfqpoint{4.273799in}{2.331163in}}%
\pgfusepath{clip}%
\pgfsetbuttcap%
\pgfsetroundjoin%
\pgfsetlinewidth{0.301125pt}%
\definecolor{currentstroke}{rgb}{0.500000,0.500000,0.500000}%
\pgfsetstrokecolor{currentstroke}%
\pgfsetstrokeopacity{0.300000}%
\pgfsetdash{}{0pt}%
\pgfpathmoveto{\pgfqpoint{1.133598in}{2.823605in}}%
\pgfpathlineto{\pgfqpoint{1.133598in}{2.823605in}}%
\pgfpathlineto{\pgfqpoint{1.151236in}{2.772704in}}%
\pgfpathlineto{\pgfqpoint{1.168748in}{2.721790in}}%
\pgfpathlineto{\pgfqpoint{1.186093in}{2.670859in}}%
\pgfpathlineto{\pgfqpoint{1.203224in}{2.619907in}}%
\pgfpathlineto{\pgfqpoint{1.220090in}{2.568928in}}%
\pgfpathlineto{\pgfqpoint{1.236615in}{2.517916in}}%
\pgfpathlineto{\pgfqpoint{1.252723in}{2.466865in}}%
\pgfpathlineto{\pgfqpoint{1.268302in}{2.415764in}}%
\pgfpathlineto{\pgfqpoint{1.283210in}{2.364606in}}%
\pgfpathlineto{\pgfqpoint{1.297288in}{2.313376in}}%
\pgfpathlineto{\pgfqpoint{1.310309in}{2.262065in}}%
\pgfpathlineto{\pgfqpoint{1.321989in}{2.210658in}}%
\pgfpathlineto{\pgfqpoint{1.331978in}{2.159146in}}%
\pgfpathlineto{\pgfqpoint{1.339829in}{2.107525in}}%
\pgfpathlineto{\pgfqpoint{1.344993in}{2.055806in}}%
\pgfpathlineto{\pgfqpoint{1.346848in}{2.004023in}}%
\pgfpathlineto{\pgfqpoint{1.344776in}{1.952246in}}%
\pgfpathlineto{\pgfqpoint{1.338288in}{1.900581in}}%
\pgfpathlineto{\pgfqpoint{1.327205in}{1.849154in}}%
\pgfpathlineto{\pgfqpoint{1.311764in}{1.798069in}}%
\pgfpathlineto{\pgfqpoint{1.292509in}{1.747365in}}%
\pgfpathlineto{\pgfqpoint{1.270234in}{1.697031in}}%
\pgfpathlineto{\pgfqpoint{1.245709in}{1.647008in}}%
\pgfpathlineto{\pgfqpoint{1.219604in}{1.597220in}}%
\pgfusepath{stroke}%
\end{pgfscope}%
\begin{pgfscope}%
\pgfpathrectangle{\pgfqpoint{0.647939in}{0.492442in}}{\pgfqpoint{4.273799in}{2.331163in}}%
\pgfusepath{clip}%
\pgfsetbuttcap%
\pgfsetroundjoin%
\pgfsetlinewidth{0.301125pt}%
\definecolor{currentstroke}{rgb}{0.500000,0.500000,0.500000}%
\pgfsetstrokecolor{currentstroke}%
\pgfsetstrokeopacity{0.300000}%
\pgfsetdash{}{0pt}%
\pgfpathmoveto{\pgfqpoint{1.036466in}{2.823605in}}%
\pgfpathlineto{\pgfqpoint{1.036466in}{2.823605in}}%
\pgfpathlineto{\pgfqpoint{1.051119in}{2.772423in}}%
\pgfpathlineto{\pgfqpoint{1.065484in}{2.721216in}}%
\pgfpathlineto{\pgfqpoint{1.079521in}{2.669982in}}%
\pgfpathlineto{\pgfqpoint{1.093176in}{2.618717in}}%
\pgfpathlineto{\pgfqpoint{1.106385in}{2.567418in}}%
\pgfpathlineto{\pgfqpoint{1.119085in}{2.516081in}}%
\pgfpathlineto{\pgfqpoint{1.131196in}{2.464701in}}%
\pgfpathlineto{\pgfqpoint{1.142623in}{2.413275in}}%
\pgfpathlineto{\pgfqpoint{1.153262in}{2.361799in}}%
\pgfpathlineto{\pgfqpoint{1.162996in}{2.310269in}}%
\pgfpathlineto{\pgfqpoint{1.171687in}{2.258684in}}%
\pgfpathlineto{\pgfqpoint{1.179178in}{2.207044in}}%
\pgfpathlineto{\pgfqpoint{1.185299in}{2.155350in}}%
\pgfpathlineto{\pgfqpoint{1.189865in}{2.103609in}}%
\pgfpathlineto{\pgfqpoint{1.192684in}{2.051831in}}%
\pgfpathlineto{\pgfqpoint{1.193563in}{2.000033in}}%
\pgfpathlineto{\pgfqpoint{1.192322in}{1.948237in}}%
\pgfpathlineto{\pgfqpoint{1.188810in}{1.896474in}}%
\pgfpathlineto{\pgfqpoint{1.182926in}{1.844776in}}%
\pgfpathlineto{\pgfqpoint{1.174631in}{1.793177in}}%
\pgfpathlineto{\pgfqpoint{1.163957in}{1.741707in}}%
\pgfpathlineto{\pgfqpoint{1.151024in}{1.690392in}}%
\pgfpathlineto{\pgfqpoint{1.136009in}{1.639248in}}%
\pgfpathlineto{\pgfqpoint{1.119122in}{1.588276in}}%
\pgfpathlineto{\pgfqpoint{1.100618in}{1.537473in}}%
\pgfpathlineto{\pgfqpoint{1.080743in}{1.486824in}}%
\pgfpathlineto{\pgfqpoint{1.059741in}{1.436309in}}%
\pgfpathlineto{\pgfqpoint{1.037830in}{1.385909in}}%
\pgfpathlineto{\pgfqpoint{1.015207in}{1.335602in}}%
\pgfpathlineto{\pgfqpoint{0.992036in}{1.285370in}}%
\pgfpathlineto{\pgfqpoint{0.968461in}{1.235193in}}%
\pgfpathlineto{\pgfqpoint{0.944602in}{1.185058in}}%
\pgfpathlineto{\pgfqpoint{0.920550in}{1.134950in}}%
\pgfusepath{stroke}%
\end{pgfscope}%
\begin{pgfscope}%
\pgfpathrectangle{\pgfqpoint{0.647939in}{0.492442in}}{\pgfqpoint{4.273799in}{2.331163in}}%
\pgfusepath{clip}%
\pgfsetbuttcap%
\pgfsetroundjoin%
\pgfsetlinewidth{0.301125pt}%
\definecolor{currentstroke}{rgb}{0.500000,0.500000,0.500000}%
\pgfsetstrokecolor{currentstroke}%
\pgfsetstrokeopacity{0.300000}%
\pgfsetdash{}{0pt}%
\pgfpathmoveto{\pgfqpoint{0.939334in}{2.823605in}}%
\pgfpathlineto{\pgfqpoint{0.939334in}{2.823605in}}%
\pgfpathlineto{\pgfqpoint{0.951633in}{2.772238in}}%
\pgfpathlineto{\pgfqpoint{0.963579in}{2.720847in}}%
\pgfpathlineto{\pgfqpoint{0.975132in}{2.669428in}}%
\pgfpathlineto{\pgfqpoint{0.986249in}{2.617982in}}%
\pgfpathlineto{\pgfqpoint{0.996882in}{2.566504in}}%
\pgfpathlineto{\pgfqpoint{1.006975in}{2.514994in}}%
\pgfpathlineto{\pgfqpoint{1.016467in}{2.463451in}}%
\pgfpathlineto{\pgfqpoint{1.025294in}{2.411872in}}%
\pgfpathlineto{\pgfqpoint{1.033387in}{2.360258in}}%
\pgfpathlineto{\pgfqpoint{1.040669in}{2.308608in}}%
\pgfpathlineto{\pgfqpoint{1.047053in}{2.256922in}}%
\pgfpathlineto{\pgfqpoint{1.052453in}{2.205203in}}%
\pgfpathlineto{\pgfqpoint{1.056779in}{2.153455in}}%
\pgfpathlineto{\pgfqpoint{1.059937in}{2.101681in}}%
\pgfpathlineto{\pgfqpoint{1.061836in}{2.049889in}}%
\pgfpathlineto{\pgfqpoint{1.062385in}{1.998088in}}%
\pgfpathlineto{\pgfqpoint{1.061504in}{1.946288in}}%
\pgfpathlineto{\pgfqpoint{1.059122in}{1.894503in}}%
\pgfpathlineto{\pgfqpoint{1.055186in}{1.842746in}}%
\pgfpathlineto{\pgfqpoint{1.049663in}{1.791033in}}%
\pgfpathlineto{\pgfqpoint{1.042541in}{1.739377in}}%
\pgfpathlineto{\pgfqpoint{1.033837in}{1.687794in}}%
\pgfpathlineto{\pgfqpoint{1.023596in}{1.636295in}}%
\pgfpathlineto{\pgfqpoint{1.011888in}{1.584891in}}%
\pgfpathlineto{\pgfqpoint{0.998794in}{1.533585in}}%
\pgfpathlineto{\pgfqpoint{0.984419in}{1.482381in}}%
\pgfpathlineto{\pgfqpoint{0.968883in}{1.431279in}}%
\pgfpathlineto{\pgfqpoint{0.952299in}{1.380275in}}%
\pgfpathlineto{\pgfqpoint{0.934792in}{1.329362in}}%
\pgfpathlineto{\pgfqpoint{0.916485in}{1.278534in}}%
\pgfusepath{stroke}%
\end{pgfscope}%
\begin{pgfscope}%
\pgfpathrectangle{\pgfqpoint{0.647939in}{0.492442in}}{\pgfqpoint{4.273799in}{2.331163in}}%
\pgfusepath{clip}%
\pgfsetbuttcap%
\pgfsetroundjoin%
\pgfsetlinewidth{0.301125pt}%
\definecolor{currentstroke}{rgb}{0.500000,0.500000,0.500000}%
\pgfsetstrokecolor{currentstroke}%
\pgfsetstrokeopacity{0.300000}%
\pgfsetdash{}{0pt}%
\pgfpathmoveto{\pgfqpoint{0.842203in}{2.823605in}}%
\pgfpathlineto{\pgfqpoint{0.842203in}{2.823605in}}%
\pgfpathlineto{\pgfqpoint{0.852623in}{2.772115in}}%
\pgfpathlineto{\pgfqpoint{0.862675in}{2.720603in}}%
\pgfpathlineto{\pgfqpoint{0.872326in}{2.669067in}}%
\pgfpathlineto{\pgfqpoint{0.881537in}{2.617509in}}%
\pgfpathlineto{\pgfqpoint{0.890272in}{2.565925in}}%
\pgfpathlineto{\pgfqpoint{0.898493in}{2.514316in}}%
\pgfpathlineto{\pgfqpoint{0.906157in}{2.462682in}}%
\pgfpathlineto{\pgfqpoint{0.913217in}{2.411022in}}%
\pgfpathlineto{\pgfqpoint{0.919624in}{2.359337in}}%
\pgfpathlineto{\pgfqpoint{0.925328in}{2.307627in}}%
\pgfpathlineto{\pgfqpoint{0.930278in}{2.255895in}}%
\pgfpathlineto{\pgfqpoint{0.934421in}{2.204141in}}%
\pgfpathlineto{\pgfqpoint{0.937703in}{2.152369in}}%
\pgfpathlineto{\pgfqpoint{0.940072in}{2.100583in}}%
\pgfpathlineto{\pgfqpoint{0.941475in}{2.048785in}}%
\pgfpathlineto{\pgfqpoint{0.941863in}{1.996983in}}%
\pgfpathlineto{\pgfqpoint{0.941191in}{1.945182in}}%
\pgfpathlineto{\pgfqpoint{0.939422in}{1.893388in}}%
\pgfpathlineto{\pgfqpoint{0.936523in}{1.841610in}}%
\pgfpathlineto{\pgfqpoint{0.932472in}{1.789855in}}%
\pgfpathlineto{\pgfqpoint{0.927257in}{1.738130in}}%
\pgfpathlineto{\pgfqpoint{0.920875in}{1.686445in}}%
\pgfpathlineto{\pgfqpoint{0.913336in}{1.634807in}}%
\pgfpathlineto{\pgfqpoint{0.904662in}{1.583221in}}%
\pgfpathlineto{\pgfqpoint{0.894887in}{1.531694in}}%
\pgfpathlineto{\pgfqpoint{0.884058in}{1.480230in}}%
\pgfpathlineto{\pgfqpoint{0.872224in}{1.428832in}}%
\pgfpathlineto{\pgfqpoint{0.859440in}{1.377502in}}%
\pgfpathlineto{\pgfqpoint{0.845778in}{1.326239in}}%
\pgfpathlineto{\pgfqpoint{0.831304in}{1.275043in}}%
\pgfpathlineto{\pgfqpoint{0.816084in}{1.223910in}}%
\pgfpathlineto{\pgfqpoint{0.800192in}{1.172839in}}%
\pgfpathlineto{\pgfqpoint{0.783696in}{1.121825in}}%
\pgfpathlineto{\pgfqpoint{0.766660in}{1.070864in}}%
\pgfpathlineto{\pgfqpoint{0.749151in}{1.019951in}}%
\pgfpathlineto{\pgfqpoint{0.731222in}{0.969080in}}%
\pgfpathlineto{\pgfqpoint{0.712933in}{0.918248in}}%
\pgfpathlineto{\pgfqpoint{0.694333in}{0.867450in}}%
\pgfpathlineto{\pgfqpoint{0.675469in}{0.816680in}}%
\pgfpathlineto{\pgfqpoint{0.656384in}{0.765936in}}%
\pgfpathlineto{\pgfqpoint{0.647939in}{0.743603in}}%
\pgfusepath{stroke}%
\end{pgfscope}%
\begin{pgfscope}%
\pgfpathrectangle{\pgfqpoint{0.647939in}{0.492442in}}{\pgfqpoint{4.273799in}{2.331163in}}%
\pgfusepath{clip}%
\pgfsetbuttcap%
\pgfsetroundjoin%
\pgfsetlinewidth{0.301125pt}%
\definecolor{currentstroke}{rgb}{0.500000,0.500000,0.500000}%
\pgfsetstrokecolor{currentstroke}%
\pgfsetstrokeopacity{0.300000}%
\pgfsetdash{}{0pt}%
\pgfpathmoveto{\pgfqpoint{0.745071in}{2.823605in}}%
\pgfpathlineto{\pgfqpoint{0.745071in}{2.823605in}}%
\pgfpathlineto{\pgfqpoint{0.753980in}{2.772031in}}%
\pgfpathlineto{\pgfqpoint{0.762525in}{2.720437in}}%
\pgfpathlineto{\pgfqpoint{0.770683in}{2.668826in}}%
\pgfpathlineto{\pgfqpoint{0.778427in}{2.617195in}}%
\pgfpathlineto{\pgfqpoint{0.785726in}{2.565544in}}%
\pgfpathlineto{\pgfqpoint{0.792551in}{2.513875in}}%
\pgfpathlineto{\pgfqpoint{0.798871in}{2.462187in}}%
\pgfpathlineto{\pgfqpoint{0.804655in}{2.410480in}}%
\pgfpathlineto{\pgfqpoint{0.809870in}{2.358755in}}%
\pgfpathlineto{\pgfqpoint{0.814483in}{2.307012in}}%
\pgfpathlineto{\pgfqpoint{0.818459in}{2.255255in}}%
\pgfpathlineto{\pgfqpoint{0.821765in}{2.203483in}}%
\pgfpathlineto{\pgfqpoint{0.824365in}{2.151699in}}%
\pgfpathlineto{\pgfqpoint{0.826228in}{2.099906in}}%
\pgfpathlineto{\pgfqpoint{0.827321in}{2.048106in}}%
\pgfpathlineto{\pgfqpoint{0.827615in}{1.996303in}}%
\pgfpathlineto{\pgfqpoint{0.827082in}{1.944501in}}%
\pgfpathlineto{\pgfqpoint{0.825700in}{1.892703in}}%
\pgfpathlineto{\pgfqpoint{0.823446in}{1.840915in}}%
\pgfpathlineto{\pgfqpoint{0.820307in}{1.789140in}}%
\pgfpathlineto{\pgfqpoint{0.816272in}{1.737384in}}%
\pgfpathlineto{\pgfqpoint{0.811336in}{1.685651in}}%
\pgfpathlineto{\pgfqpoint{0.805502in}{1.633946in}}%
\pgfpathlineto{\pgfqpoint{0.798776in}{1.582274in}}%
\pgfpathlineto{\pgfqpoint{0.791171in}{1.530638in}}%
\pgfpathlineto{\pgfqpoint{0.782705in}{1.479041in}}%
\pgfpathlineto{\pgfqpoint{0.773402in}{1.427488in}}%
\pgfpathlineto{\pgfqpoint{0.763295in}{1.375979in}}%
\pgfpathlineto{\pgfqpoint{0.752418in}{1.324518in}}%
\pgfpathlineto{\pgfqpoint{0.740806in}{1.273104in}}%
\pgfpathlineto{\pgfqpoint{0.728499in}{1.221738in}}%
\pgfpathlineto{\pgfqpoint{0.715543in}{1.170420in}}%
\pgfusepath{stroke}%
\end{pgfscope}%
\begin{pgfscope}%
\pgfpathrectangle{\pgfqpoint{0.647939in}{0.492442in}}{\pgfqpoint{4.273799in}{2.331163in}}%
\pgfusepath{clip}%
\pgfsetbuttcap%
\pgfsetroundjoin%
\pgfsetlinewidth{0.301125pt}%
\definecolor{currentstroke}{rgb}{0.500000,0.500000,0.500000}%
\pgfsetstrokecolor{currentstroke}%
\pgfsetstrokeopacity{0.300000}%
\pgfsetdash{}{0pt}%
\pgfpathmoveto{\pgfqpoint{0.647939in}{2.823605in}}%
\pgfpathlineto{\pgfqpoint{0.647939in}{2.823605in}}%
\pgfpathlineto{\pgfqpoint{0.655619in}{2.771972in}}%
\pgfpathlineto{\pgfqpoint{0.662954in}{2.720323in}}%
\pgfpathlineto{\pgfqpoint{0.669923in}{2.668659in}}%
\pgfpathlineto{\pgfqpoint{0.676508in}{2.616981in}}%
\pgfpathlineto{\pgfqpoint{0.682688in}{2.565287in}}%
\pgfpathlineto{\pgfqpoint{0.688441in}{2.513579in}}%
\pgfpathlineto{\pgfqpoint{0.693744in}{2.461856in}}%
\pgfpathlineto{\pgfqpoint{0.698574in}{2.410120in}}%
\pgfpathlineto{\pgfqpoint{0.702908in}{2.358370in}}%
\pgfpathlineto{\pgfqpoint{0.706722in}{2.306609in}}%
\pgfpathlineto{\pgfqpoint{0.709995in}{2.254836in}}%
\pgfpathlineto{\pgfqpoint{0.712702in}{2.203054in}}%
\pgfpathlineto{\pgfqpoint{0.714822in}{2.151263in}}%
\pgfpathlineto{\pgfqpoint{0.716333in}{2.099467in}}%
\pgfpathlineto{\pgfqpoint{0.717214in}{2.047666in}}%
\pgfpathlineto{\pgfqpoint{0.717447in}{1.995862in}}%
\pgfpathlineto{\pgfqpoint{0.717012in}{1.944060in}}%
\pgfpathlineto{\pgfqpoint{0.715896in}{1.892260in}}%
\pgfpathlineto{\pgfqpoint{0.714084in}{1.840466in}}%
\pgfpathlineto{\pgfqpoint{0.711565in}{1.788681in}}%
\pgfpathlineto{\pgfqpoint{0.708332in}{1.736908in}}%
\pgfpathlineto{\pgfqpoint{0.704380in}{1.685150in}}%
\pgfpathlineto{\pgfqpoint{0.699706in}{1.633410in}}%
\pgfpathlineto{\pgfqpoint{0.694313in}{1.581690in}}%
\pgfpathlineto{\pgfqpoint{0.688205in}{1.529994in}}%
\pgfpathlineto{\pgfqpoint{0.681393in}{1.478325in}}%
\pgfpathlineto{\pgfqpoint{0.673889in}{1.426684in}}%
\pgfpathlineto{\pgfqpoint{0.665709in}{1.375073in}}%
\pgfpathlineto{\pgfqpoint{0.656868in}{1.323495in}}%
\pgfpathlineto{\pgfqpoint{0.647939in}{1.273244in}}%
\pgfusepath{stroke}%
\end{pgfscope}%
\begin{pgfscope}%
\pgfpathrectangle{\pgfqpoint{0.647939in}{0.492442in}}{\pgfqpoint{4.273799in}{2.331163in}}%
\pgfusepath{clip}%
\pgfsetbuttcap%
\pgfsetroundjoin%
\pgfsetlinewidth{0.301125pt}%
\definecolor{currentstroke}{rgb}{0.500000,0.500000,0.500000}%
\pgfsetstrokecolor{currentstroke}%
\pgfsetstrokeopacity{0.300000}%
\pgfsetdash{}{0pt}%
\pgfpathmoveto{\pgfqpoint{0.647939in}{2.399758in}}%
\pgfpathlineto{\pgfqpoint{0.647939in}{2.399758in}}%
\pgfpathlineto{\pgfqpoint{0.651827in}{2.347997in}}%
\pgfpathlineto{\pgfqpoint{0.655227in}{2.296227in}}%
\pgfpathlineto{\pgfqpoint{0.658120in}{2.244448in}}%
\pgfpathlineto{\pgfqpoint{0.660489in}{2.192661in}}%
\pgfpathlineto{\pgfqpoint{0.662313in}{2.140867in}}%
\pgfpathlineto{\pgfqpoint{0.663576in}{2.089068in}}%
\pgfpathlineto{\pgfqpoint{0.664261in}{2.037266in}}%
\pgfpathlineto{\pgfqpoint{0.664352in}{1.985463in}}%
\pgfpathlineto{\pgfqpoint{0.663835in}{1.933660in}}%
\pgfpathlineto{\pgfqpoint{0.662697in}{1.881861in}}%
\pgfpathlineto{\pgfqpoint{0.660928in}{1.830066in}}%
\pgfpathlineto{\pgfqpoint{0.658518in}{1.778280in}}%
\pgfpathlineto{\pgfqpoint{0.655461in}{1.726503in}}%
\pgfpathlineto{\pgfqpoint{0.651753in}{1.674740in}}%
\pgfpathlineto{\pgfqpoint{0.647939in}{1.625797in}}%
\pgfusepath{stroke}%
\end{pgfscope}%
\begin{pgfscope}%
\pgfpathrectangle{\pgfqpoint{0.647939in}{0.492442in}}{\pgfqpoint{4.273799in}{2.331163in}}%
\pgfusepath{clip}%
\pgfsetbuttcap%
\pgfsetroundjoin%
\pgfsetlinewidth{0.301125pt}%
\definecolor{currentstroke}{rgb}{0.500000,0.500000,0.500000}%
\pgfsetstrokecolor{currentstroke}%
\pgfsetstrokeopacity{0.300000}%
\pgfsetdash{}{0pt}%
\pgfpathmoveto{\pgfqpoint{1.493707in}{0.492442in}}%
\pgfpathlineto{\pgfqpoint{1.480331in}{0.503371in}}%
\pgfpathlineto{\pgfqpoint{1.424993in}{0.545423in}}%
\pgfpathlineto{\pgfqpoint{1.362961in}{0.584517in}}%
\pgfpathlineto{\pgfqpoint{1.290709in}{0.617725in}}%
\pgfpathlineto{\pgfqpoint{1.290709in}{0.617725in}}%
\pgfpathlineto{\pgfqpoint{1.229464in}{0.634298in}}%
\pgfpathlineto{\pgfqpoint{1.229464in}{0.634298in}}%
\pgfpathlineto{\pgfqpoint{1.172659in}{0.639358in}}%
\pgfpathlineto{\pgfqpoint{1.115664in}{0.633978in}}%
\pgfusepath{stroke}%
\end{pgfscope}%
\begin{pgfscope}%
\pgfpathrectangle{\pgfqpoint{0.647939in}{0.492442in}}{\pgfqpoint{4.273799in}{2.331163in}}%
\pgfusepath{clip}%
\pgfsetbuttcap%
\pgfsetroundjoin%
\pgfsetlinewidth{0.301125pt}%
\definecolor{currentstroke}{rgb}{0.500000,0.500000,0.500000}%
\pgfsetstrokecolor{currentstroke}%
\pgfsetstrokeopacity{0.300000}%
\pgfsetdash{}{0pt}%
\pgfpathmoveto{\pgfqpoint{4.260698in}{0.492442in}}%
\pgfpathlineto{\pgfqpoint{4.242703in}{0.509681in}}%
\pgfpathlineto{\pgfqpoint{4.194726in}{0.554382in}}%
\pgfpathlineto{\pgfqpoint{4.144684in}{0.598404in}}%
\pgfpathlineto{\pgfqpoint{4.092520in}{0.641688in}}%
\pgfpathlineto{\pgfqpoint{4.038237in}{0.684188in}}%
\pgfpathlineto{\pgfqpoint{3.981866in}{0.725872in}}%
\pgfpathlineto{\pgfqpoint{3.923518in}{0.766740in}}%
\pgfpathlineto{\pgfqpoint{3.863422in}{0.806850in}}%
\pgfpathlineto{\pgfqpoint{3.801873in}{0.846299in}}%
\pgfpathlineto{\pgfqpoint{3.739254in}{0.885246in}}%
\pgfpathlineto{\pgfqpoint{3.676017in}{0.923896in}}%
\pgfusepath{stroke}%
\end{pgfscope}%
\begin{pgfscope}%
\pgfpathrectangle{\pgfqpoint{0.647939in}{0.492442in}}{\pgfqpoint{4.273799in}{2.331163in}}%
\pgfusepath{clip}%
\pgfsetbuttcap%
\pgfsetroundjoin%
\pgfsetlinewidth{0.301125pt}%
\definecolor{currentstroke}{rgb}{0.500000,0.500000,0.500000}%
\pgfsetstrokecolor{currentstroke}%
\pgfsetstrokeopacity{0.300000}%
\pgfsetdash{}{0pt}%
\pgfpathmoveto{\pgfqpoint{4.643259in}{1.659685in}}%
\pgfpathlineto{\pgfqpoint{4.630343in}{1.711005in}}%
\pgfpathlineto{\pgfqpoint{4.618615in}{1.762410in}}%
\pgfpathlineto{\pgfqpoint{4.608353in}{1.813907in}}%
\pgfpathlineto{\pgfqpoint{4.599894in}{1.865501in}}%
\pgfpathlineto{\pgfqpoint{4.593628in}{1.917186in}}%
\pgfpathlineto{\pgfqpoint{4.590005in}{1.968944in}}%
\pgfpathlineto{\pgfqpoint{4.589508in}{2.020738in}}%
\pgfpathlineto{\pgfqpoint{4.592589in}{2.072503in}}%
\pgfpathlineto{\pgfqpoint{4.599584in}{2.124150in}}%
\pgfpathlineto{\pgfqpoint{4.610622in}{2.175583in}}%
\pgfpathlineto{\pgfqpoint{4.625563in}{2.226723in}}%
\pgfusepath{stroke}%
\end{pgfscope}%
\begin{pgfscope}%
\pgfpathrectangle{\pgfqpoint{0.647939in}{0.492442in}}{\pgfqpoint{4.273799in}{2.331163in}}%
\pgfusepath{clip}%
\pgfsetbuttcap%
\pgfsetroundjoin%
\pgfsetlinewidth{0.301125pt}%
\definecolor{currentstroke}{rgb}{0.500000,0.500000,0.500000}%
\pgfsetstrokecolor{currentstroke}%
\pgfsetstrokeopacity{0.300000}%
\pgfsetdash{}{0pt}%
\pgfpathmoveto{\pgfqpoint{4.561257in}{1.131704in}}%
\pgfpathlineto{\pgfqpoint{4.533211in}{1.181195in}}%
\pgfpathlineto{\pgfqpoint{4.504272in}{1.230532in}}%
\pgfpathlineto{\pgfqpoint{4.474296in}{1.279681in}}%
\pgfpathlineto{\pgfqpoint{4.443053in}{1.328596in}}%
\pgfpathlineto{\pgfqpoint{4.410258in}{1.377206in}}%
\pgfpathlineto{\pgfqpoint{4.375516in}{1.425407in}}%
\pgfpathlineto{\pgfqpoint{4.338205in}{1.473022in}}%
\pgfpathlineto{\pgfqpoint{4.297331in}{1.519744in}}%
\pgfpathlineto{\pgfqpoint{4.251199in}{1.564947in}}%
\pgfpathlineto{\pgfqpoint{4.196695in}{1.607168in}}%
\pgfpathlineto{\pgfqpoint{4.128223in}{1.642338in}}%
\pgfpathlineto{\pgfqpoint{4.128223in}{1.642338in}}%
\pgfpathlineto{\pgfqpoint{4.073439in}{1.657294in}}%
\pgfpathlineto{\pgfqpoint{4.011541in}{1.662537in}}%
\pgfpathlineto{\pgfqpoint{3.952191in}{1.659542in}}%
\pgfpathlineto{\pgfqpoint{3.881888in}{1.650186in}}%
\pgfusepath{stroke}%
\end{pgfscope}%
\begin{pgfscope}%
\pgfpathrectangle{\pgfqpoint{0.647939in}{0.492442in}}{\pgfqpoint{4.273799in}{2.331163in}}%
\pgfusepath{clip}%
\pgfsetbuttcap%
\pgfsetroundjoin%
\pgfsetlinewidth{0.301125pt}%
\definecolor{currentstroke}{rgb}{0.500000,0.500000,0.500000}%
\pgfsetstrokecolor{currentstroke}%
\pgfsetstrokeopacity{0.300000}%
\pgfsetdash{}{0pt}%
\pgfpathmoveto{\pgfqpoint{4.555985in}{1.448791in}}%
\pgfpathlineto{\pgfqpoint{4.533211in}{1.499081in}}%
\pgfpathlineto{\pgfqpoint{4.510279in}{1.549349in}}%
\pgfpathlineto{\pgfqpoint{4.487201in}{1.599596in}}%
\pgfpathlineto{\pgfqpoint{4.463995in}{1.649824in}}%
\pgfpathlineto{\pgfqpoint{4.440696in}{1.700037in}}%
\pgfpathlineto{\pgfqpoint{4.417355in}{1.750243in}}%
\pgfpathlineto{\pgfqpoint{4.394077in}{1.800452in}}%
\pgfpathlineto{\pgfqpoint{4.371059in}{1.850692in}}%
\pgfpathlineto{\pgfqpoint{4.348792in}{1.901015in}}%
\pgfpathlineto{\pgfqpoint{4.328770in}{1.951596in}}%
\pgfpathlineto{\pgfqpoint{4.319389in}{2.001730in}}%
\pgfpathlineto{\pgfqpoint{4.319389in}{2.001730in}}%
\pgfpathlineto{\pgfqpoint{4.324685in}{2.026335in}}%
\pgfpathlineto{\pgfqpoint{4.342563in}{2.054444in}}%
\pgfpathlineto{\pgfqpoint{4.366935in}{2.084121in}}%
\pgfpathlineto{\pgfqpoint{4.408026in}{2.130337in}}%
\pgfusepath{stroke}%
\end{pgfscope}%
\begin{pgfscope}%
\pgfpathrectangle{\pgfqpoint{0.647939in}{0.492442in}}{\pgfqpoint{4.273799in}{2.331163in}}%
\pgfusepath{clip}%
\pgfsetbuttcap%
\pgfsetroundjoin%
\pgfsetlinewidth{0.301125pt}%
\definecolor{currentstroke}{rgb}{0.500000,0.500000,0.500000}%
\pgfsetstrokecolor{currentstroke}%
\pgfsetstrokeopacity{0.300000}%
\pgfsetdash{}{0pt}%
\pgfpathmoveto{\pgfqpoint{1.487361in}{0.694496in}}%
\pgfpathlineto{\pgfqpoint{1.430988in}{0.736103in}}%
\pgfpathlineto{\pgfqpoint{1.374835in}{0.769585in}}%
\pgfpathlineto{\pgfqpoint{1.323483in}{0.792395in}}%
\pgfpathlineto{\pgfqpoint{1.273639in}{0.806689in}}%
\pgfpathlineto{\pgfqpoint{1.219351in}{0.812852in}}%
\pgfpathlineto{\pgfqpoint{1.164139in}{0.808639in}}%
\pgfpathlineto{\pgfqpoint{1.164139in}{0.808639in}}%
\pgfpathlineto{\pgfqpoint{1.105584in}{0.792284in}}%
\pgfpathlineto{\pgfqpoint{1.105584in}{0.792284in}}%
\pgfpathlineto{\pgfqpoint{1.036466in}{0.757347in}}%
\pgfpathlineto{\pgfqpoint{0.979228in}{0.716225in}}%
\pgfpathlineto{\pgfqpoint{0.929741in}{0.672132in}}%
\pgfusepath{stroke}%
\end{pgfscope}%
\begin{pgfscope}%
\pgfpathrectangle{\pgfqpoint{0.647939in}{0.492442in}}{\pgfqpoint{4.273799in}{2.331163in}}%
\pgfusepath{clip}%
\pgfsetbuttcap%
\pgfsetroundjoin%
\pgfsetlinewidth{0.301125pt}%
\definecolor{currentstroke}{rgb}{0.500000,0.500000,0.500000}%
\pgfsetstrokecolor{currentstroke}%
\pgfsetstrokeopacity{0.300000}%
\pgfsetdash{}{0pt}%
\pgfpathmoveto{\pgfqpoint{2.784839in}{0.757347in}}%
\pgfpathlineto{\pgfqpoint{2.746621in}{0.804770in}}%
\pgfpathlineto{\pgfqpoint{2.709386in}{0.852425in}}%
\pgfpathlineto{\pgfqpoint{2.673126in}{0.900303in}}%
\pgfpathlineto{\pgfqpoint{2.637834in}{0.948396in}}%
\pgfpathlineto{\pgfqpoint{2.603503in}{0.996695in}}%
\pgfpathlineto{\pgfqpoint{2.570129in}{1.045194in}}%
\pgfusepath{stroke}%
\end{pgfscope}%
\begin{pgfscope}%
\pgfpathrectangle{\pgfqpoint{0.647939in}{0.492442in}}{\pgfqpoint{4.273799in}{2.331163in}}%
\pgfusepath{clip}%
\pgfsetbuttcap%
\pgfsetroundjoin%
\pgfsetlinewidth{0.301125pt}%
\definecolor{currentstroke}{rgb}{0.500000,0.500000,0.500000}%
\pgfsetstrokecolor{currentstroke}%
\pgfsetstrokeopacity{0.300000}%
\pgfsetdash{}{0pt}%
\pgfpathmoveto{\pgfqpoint{1.133598in}{2.134853in}}%
\pgfpathlineto{\pgfqpoint{1.136924in}{2.083083in}}%
\pgfpathlineto{\pgfqpoint{1.138681in}{2.031290in}}%
\pgfpathlineto{\pgfqpoint{1.138733in}{1.979489in}}%
\pgfpathlineto{\pgfqpoint{1.136961in}{1.927697in}}%
\pgfpathlineto{\pgfqpoint{1.133267in}{1.875936in}}%
\pgfusepath{stroke}%
\end{pgfscope}%
\begin{pgfscope}%
\pgfpathrectangle{\pgfqpoint{0.647939in}{0.492442in}}{\pgfqpoint{4.273799in}{2.331163in}}%
\pgfusepath{clip}%
\pgfsetbuttcap%
\pgfsetroundjoin%
\pgfsetlinewidth{0.301125pt}%
\definecolor{currentstroke}{rgb}{0.500000,0.500000,0.500000}%
\pgfsetstrokecolor{currentstroke}%
\pgfsetstrokeopacity{0.300000}%
\pgfsetdash{}{0pt}%
\pgfpathmoveto{\pgfqpoint{2.396312in}{0.810328in}}%
\pgfpathlineto{\pgfqpoint{2.363558in}{0.858953in}}%
\pgfpathlineto{\pgfqpoint{2.331416in}{0.907699in}}%
\pgfpathlineto{\pgfqpoint{2.299876in}{0.956563in}}%
\pgfpathlineto{\pgfqpoint{2.268927in}{1.005538in}}%
\pgfpathlineto{\pgfqpoint{2.238571in}{1.054624in}}%
\pgfpathlineto{\pgfqpoint{2.208811in}{1.103818in}}%
\pgfpathlineto{\pgfqpoint{2.179642in}{1.153118in}}%
\pgfpathlineto{\pgfqpoint{2.151074in}{1.202522in}}%
\pgfpathlineto{\pgfqpoint{2.123120in}{1.252031in}}%
\pgfpathlineto{\pgfqpoint{2.095792in}{1.301643in}}%
\pgfpathlineto{\pgfqpoint{2.069115in}{1.351360in}}%
\pgfpathlineto{\pgfqpoint{2.043121in}{1.401185in}}%
\pgfpathlineto{\pgfqpoint{2.017848in}{1.451120in}}%
\pgfpathlineto{\pgfqpoint{1.993356in}{1.501171in}}%
\pgfpathlineto{\pgfqpoint{1.969718in}{1.551343in}}%
\pgfpathlineto{\pgfqpoint{1.947036in}{1.601647in}}%
\pgfpathlineto{\pgfqpoint{1.925442in}{1.652092in}}%
\pgfpathlineto{\pgfqpoint{1.905119in}{1.702693in}}%
\pgfpathlineto{\pgfqpoint{1.886318in}{1.753469in}}%
\pgfpathlineto{\pgfqpoint{1.869385in}{1.804437in}}%
\pgfpathlineto{\pgfqpoint{1.854818in}{1.855622in}}%
\pgfpathlineto{\pgfqpoint{1.843310in}{1.907034in}}%
\pgfpathlineto{\pgfqpoint{1.835833in}{1.958659in}}%
\pgfpathlineto{\pgfqpoint{1.833718in}{2.010418in}}%
\pgfpathlineto{\pgfqpoint{1.838596in}{2.062099in}}%
\pgfpathlineto{\pgfqpoint{1.852076in}{2.113297in}}%
\pgfpathlineto{\pgfqpoint{1.875169in}{2.163448in}}%
\pgfpathlineto{\pgfqpoint{1.907833in}{2.211989in}}%
\pgfpathlineto{\pgfqpoint{1.949224in}{2.258522in}}%
\pgfpathlineto{\pgfqpoint{1.998365in}{2.302763in}}%
\pgfusepath{stroke}%
\end{pgfscope}%
\begin{pgfscope}%
\pgfpathrectangle{\pgfqpoint{0.647939in}{0.492442in}}{\pgfqpoint{4.273799in}{2.331163in}}%
\pgfusepath{clip}%
\pgfsetbuttcap%
\pgfsetroundjoin%
\pgfsetlinewidth{0.301125pt}%
\definecolor{currentstroke}{rgb}{0.500000,0.500000,0.500000}%
\pgfsetstrokecolor{currentstroke}%
\pgfsetstrokeopacity{0.300000}%
\pgfsetdash{}{0pt}%
\pgfpathmoveto{\pgfqpoint{4.338948in}{1.181195in}}%
\pgfpathlineto{\pgfqpoint{4.296498in}{1.227516in}}%
\pgfpathlineto{\pgfqpoint{4.250580in}{1.272833in}}%
\pgfpathlineto{\pgfqpoint{4.200259in}{1.316724in}}%
\pgfpathlineto{\pgfqpoint{4.144273in}{1.358505in}}%
\pgfpathlineto{\pgfqpoint{4.081165in}{1.397100in}}%
\pgfpathlineto{\pgfqpoint{4.009603in}{1.430937in}}%
\pgfpathlineto{\pgfqpoint{3.929426in}{1.458361in}}%
\pgfpathlineto{\pgfqpoint{3.842531in}{1.478934in}}%
\pgfpathlineto{\pgfqpoint{3.752055in}{1.494497in}}%
\pgfpathlineto{\pgfqpoint{3.660600in}{1.508371in}}%
\pgfusepath{stroke}%
\end{pgfscope}%
\begin{pgfscope}%
\pgfpathrectangle{\pgfqpoint{0.647939in}{0.492442in}}{\pgfqpoint{4.273799in}{2.331163in}}%
\pgfusepath{clip}%
\pgfsetbuttcap%
\pgfsetroundjoin%
\pgfsetlinewidth{0.301125pt}%
\definecolor{currentstroke}{rgb}{0.500000,0.500000,0.500000}%
\pgfsetstrokecolor{currentstroke}%
\pgfsetstrokeopacity{0.300000}%
\pgfsetdash{}{0pt}%
\pgfpathmoveto{\pgfqpoint{4.338948in}{1.552062in}}%
\pgfpathlineto{\pgfqpoint{4.298980in}{1.599007in}}%
\pgfpathlineto{\pgfqpoint{4.252978in}{1.644215in}}%
\pgfpathlineto{\pgfqpoint{4.196125in}{1.685312in}}%
\pgfpathlineto{\pgfqpoint{4.196125in}{1.685312in}}%
\pgfpathlineto{\pgfqpoint{4.147381in}{1.706907in}}%
\pgfpathlineto{\pgfqpoint{4.147381in}{1.706907in}}%
\pgfpathlineto{\pgfqpoint{4.099223in}{1.716630in}}%
\pgfpathlineto{\pgfqpoint{4.047728in}{1.716813in}}%
\pgfusepath{stroke}%
\end{pgfscope}%
\begin{pgfscope}%
\pgfpathrectangle{\pgfqpoint{0.647939in}{0.492442in}}{\pgfqpoint{4.273799in}{2.331163in}}%
\pgfusepath{clip}%
\pgfsetbuttcap%
\pgfsetroundjoin%
\pgfsetlinewidth{0.301125pt}%
\definecolor{currentstroke}{rgb}{0.500000,0.500000,0.500000}%
\pgfsetstrokecolor{currentstroke}%
\pgfsetstrokeopacity{0.300000}%
\pgfsetdash{}{0pt}%
\pgfpathmoveto{\pgfqpoint{1.573215in}{2.550926in}}%
\pgfpathlineto{\pgfqpoint{1.619257in}{2.505719in}}%
\pgfpathlineto{\pgfqpoint{1.674210in}{2.463707in}}%
\pgfpathlineto{\pgfqpoint{1.674210in}{2.463707in}}%
\pgfpathlineto{\pgfqpoint{1.731697in}{2.434152in}}%
\pgfpathlineto{\pgfqpoint{1.731697in}{2.434152in}}%
\pgfpathlineto{\pgfqpoint{1.785267in}{2.419071in}}%
\pgfpathlineto{\pgfqpoint{1.845549in}{2.414362in}}%
\pgfpathlineto{\pgfqpoint{1.901278in}{2.418781in}}%
\pgfusepath{stroke}%
\end{pgfscope}%
\begin{pgfscope}%
\pgfpathrectangle{\pgfqpoint{0.647939in}{0.492442in}}{\pgfqpoint{4.273799in}{2.331163in}}%
\pgfusepath{clip}%
\pgfsetbuttcap%
\pgfsetroundjoin%
\pgfsetlinewidth{0.301125pt}%
\definecolor{currentstroke}{rgb}{0.500000,0.500000,0.500000}%
\pgfsetstrokecolor{currentstroke}%
\pgfsetstrokeopacity{0.300000}%
\pgfsetdash{}{0pt}%
\pgfpathmoveto{\pgfqpoint{1.606976in}{1.251853in}}%
\pgfpathlineto{\pgfqpoint{1.560448in}{1.296967in}}%
\pgfpathlineto{\pgfqpoint{1.512007in}{1.335974in}}%
\pgfpathlineto{\pgfqpoint{1.469959in}{1.362037in}}%
\pgfpathlineto{\pgfqpoint{1.431314in}{1.378686in}}%
\pgfpathlineto{\pgfqpoint{1.390138in}{1.388029in}}%
\pgfpathlineto{\pgfqpoint{1.345558in}{1.387986in}}%
\pgfpathlineto{\pgfqpoint{1.345558in}{1.387986in}}%
\pgfpathlineto{\pgfqpoint{1.297522in}{1.376110in}}%
\pgfpathlineto{\pgfqpoint{1.297522in}{1.376110in}}%
\pgfpathlineto{\pgfqpoint{1.230730in}{1.340138in}}%
\pgfusepath{stroke}%
\end{pgfscope}%
\begin{pgfscope}%
\pgfpathrectangle{\pgfqpoint{0.647939in}{0.492442in}}{\pgfqpoint{4.273799in}{2.331163in}}%
\pgfusepath{clip}%
\pgfsetbuttcap%
\pgfsetroundjoin%
\pgfsetlinewidth{0.301125pt}%
\definecolor{currentstroke}{rgb}{0.500000,0.500000,0.500000}%
\pgfsetstrokecolor{currentstroke}%
\pgfsetstrokeopacity{0.300000}%
\pgfsetdash{}{0pt}%
\pgfpathmoveto{\pgfqpoint{1.489803in}{0.871515in}}%
\pgfpathlineto{\pgfqpoint{1.432353in}{0.911771in}}%
\pgfpathlineto{\pgfqpoint{1.381315in}{0.939515in}}%
\pgfpathlineto{\pgfqpoint{1.333818in}{0.957730in}}%
\pgfpathlineto{\pgfqpoint{1.284757in}{0.968178in}}%
\pgfpathlineto{\pgfqpoint{1.230730in}{0.969271in}}%
\pgfpathlineto{\pgfqpoint{1.230730in}{0.969271in}}%
\pgfpathlineto{\pgfqpoint{1.230730in}{0.969271in}}%
\pgfpathlineto{\pgfqpoint{1.178090in}{0.959559in}}%
\pgfpathlineto{\pgfqpoint{1.133454in}{0.943063in}}%
\pgfusepath{stroke}%
\end{pgfscope}%
\begin{pgfscope}%
\pgfpathrectangle{\pgfqpoint{0.647939in}{0.492442in}}{\pgfqpoint{4.273799in}{2.331163in}}%
\pgfusepath{clip}%
\pgfsetbuttcap%
\pgfsetroundjoin%
\pgfsetlinewidth{0.301125pt}%
\definecolor{currentstroke}{rgb}{0.500000,0.500000,0.500000}%
\pgfsetstrokecolor{currentstroke}%
\pgfsetstrokeopacity{0.300000}%
\pgfsetdash{}{0pt}%
\pgfpathmoveto{\pgfqpoint{4.241816in}{1.075233in}}%
\pgfpathlineto{\pgfqpoint{4.192022in}{1.119324in}}%
\pgfpathlineto{\pgfqpoint{4.138240in}{1.161995in}}%
\pgfpathlineto{\pgfqpoint{4.079900in}{1.202830in}}%
\pgfpathlineto{\pgfqpoint{4.016444in}{1.241315in}}%
\pgfpathlineto{\pgfqpoint{3.947629in}{1.276950in}}%
\pgfpathlineto{\pgfqpoint{3.873777in}{1.309445in}}%
\pgfpathlineto{\pgfqpoint{3.795791in}{1.338957in}}%
\pgfpathlineto{\pgfqpoint{3.715153in}{1.366302in}}%
\pgfpathlineto{\pgfqpoint{3.633548in}{1.392792in}}%
\pgfpathlineto{\pgfqpoint{3.552605in}{1.419857in}}%
\pgfpathlineto{\pgfqpoint{3.473780in}{1.448679in}}%
\pgfpathlineto{\pgfqpoint{3.398282in}{1.480009in}}%
\pgfpathlineto{\pgfqpoint{3.326994in}{1.514157in}}%
\pgfusepath{stroke}%
\end{pgfscope}%
\begin{pgfscope}%
\pgfpathrectangle{\pgfqpoint{0.647939in}{0.492442in}}{\pgfqpoint{4.273799in}{2.331163in}}%
\pgfusepath{clip}%
\pgfsetbuttcap%
\pgfsetroundjoin%
\pgfsetlinewidth{0.301125pt}%
\definecolor{currentstroke}{rgb}{0.500000,0.500000,0.500000}%
\pgfsetstrokecolor{currentstroke}%
\pgfsetstrokeopacity{0.300000}%
\pgfsetdash{}{0pt}%
\pgfpathmoveto{\pgfqpoint{4.241816in}{1.499081in}}%
\pgfpathlineto{\pgfqpoint{4.187266in}{1.541366in}}%
\pgfpathlineto{\pgfqpoint{4.121532in}{1.578368in}}%
\pgfpathlineto{\pgfqpoint{4.121532in}{1.578368in}}%
\pgfpathlineto{\pgfqpoint{4.059279in}{1.600297in}}%
\pgfpathlineto{\pgfqpoint{3.987605in}{1.612361in}}%
\pgfpathlineto{\pgfqpoint{3.918861in}{1.614609in}}%
\pgfpathlineto{\pgfqpoint{3.841553in}{1.610856in}}%
\pgfpathlineto{\pgfqpoint{3.747645in}{1.603618in}}%
\pgfpathlineto{\pgfqpoint{3.653310in}{1.598624in}}%
\pgfusepath{stroke}%
\end{pgfscope}%
\begin{pgfscope}%
\pgfpathrectangle{\pgfqpoint{0.647939in}{0.492442in}}{\pgfqpoint{4.273799in}{2.331163in}}%
\pgfusepath{clip}%
\pgfsetbuttcap%
\pgfsetroundjoin%
\pgfsetlinewidth{0.301125pt}%
\definecolor{currentstroke}{rgb}{0.500000,0.500000,0.500000}%
\pgfsetstrokecolor{currentstroke}%
\pgfsetstrokeopacity{0.300000}%
\pgfsetdash{}{0pt}%
\pgfpathmoveto{\pgfqpoint{1.522125in}{2.452738in}}%
\pgfpathlineto{\pgfqpoint{1.559413in}{2.405127in}}%
\pgfpathlineto{\pgfqpoint{1.602789in}{2.359114in}}%
\pgfpathlineto{\pgfqpoint{1.658057in}{2.317466in}}%
\pgfpathlineto{\pgfqpoint{1.658057in}{2.317466in}}%
\pgfpathlineto{\pgfqpoint{1.699477in}{2.299640in}}%
\pgfpathlineto{\pgfqpoint{1.699477in}{2.299640in}}%
\pgfpathlineto{\pgfqpoint{1.740940in}{2.292881in}}%
\pgfpathlineto{\pgfqpoint{1.784201in}{2.295697in}}%
\pgfpathlineto{\pgfqpoint{1.825554in}{2.305374in}}%
\pgfusepath{stroke}%
\end{pgfscope}%
\begin{pgfscope}%
\pgfpathrectangle{\pgfqpoint{0.647939in}{0.492442in}}{\pgfqpoint{4.273799in}{2.331163in}}%
\pgfusepath{clip}%
\pgfsetbuttcap%
\pgfsetroundjoin%
\pgfsetlinewidth{0.301125pt}%
\definecolor{currentstroke}{rgb}{0.500000,0.500000,0.500000}%
\pgfsetstrokecolor{currentstroke}%
\pgfsetstrokeopacity{0.300000}%
\pgfsetdash{}{0pt}%
\pgfpathmoveto{\pgfqpoint{1.723781in}{1.342183in}}%
\pgfpathlineto{\pgfqpoint{1.688045in}{1.390176in}}%
\pgfpathlineto{\pgfqpoint{1.650314in}{1.437704in}}%
\pgfpathlineto{\pgfqpoint{1.609358in}{1.484408in}}%
\pgfpathlineto{\pgfqpoint{1.564159in}{1.528117in}}%
\pgfpathlineto{\pgfqpoint{1.526362in}{1.556760in}}%
\pgfpathlineto{\pgfqpoint{1.493095in}{1.574842in}}%
\pgfpathlineto{\pgfqpoint{1.459666in}{1.585512in}}%
\pgfpathlineto{\pgfqpoint{1.421550in}{1.588021in}}%
\pgfpathlineto{\pgfqpoint{1.421550in}{1.588021in}}%
\pgfpathlineto{\pgfqpoint{1.379673in}{1.578969in}}%
\pgfpathlineto{\pgfqpoint{1.379673in}{1.578969in}}%
\pgfpathlineto{\pgfqpoint{1.327862in}{1.552062in}}%
\pgfpathlineto{\pgfqpoint{1.327862in}{1.552062in}}%
\pgfusepath{stroke}%
\end{pgfscope}%
\begin{pgfscope}%
\pgfpathrectangle{\pgfqpoint{0.647939in}{0.492442in}}{\pgfqpoint{4.273799in}{2.331163in}}%
\pgfusepath{clip}%
\pgfsetbuttcap%
\pgfsetroundjoin%
\pgfsetlinewidth{0.301125pt}%
\definecolor{currentstroke}{rgb}{0.500000,0.500000,0.500000}%
\pgfsetstrokecolor{currentstroke}%
\pgfsetstrokeopacity{0.300000}%
\pgfsetdash{}{0pt}%
\pgfpathmoveto{\pgfqpoint{2.007784in}{0.916290in}}%
\pgfpathlineto{\pgfqpoint{1.975504in}{0.965009in}}%
\pgfpathlineto{\pgfqpoint{1.943291in}{1.013742in}}%
\pgfpathlineto{\pgfqpoint{1.911075in}{1.062474in}}%
\pgfpathlineto{\pgfqpoint{1.878774in}{1.111189in}}%
\pgfpathlineto{\pgfqpoint{1.846300in}{1.159869in}}%
\pgfpathlineto{\pgfqpoint{1.813541in}{1.208493in}}%
\pgfpathlineto{\pgfqpoint{1.780327in}{1.257024in}}%
\pgfusepath{stroke}%
\end{pgfscope}%
\begin{pgfscope}%
\pgfpathrectangle{\pgfqpoint{0.647939in}{0.492442in}}{\pgfqpoint{4.273799in}{2.331163in}}%
\pgfusepath{clip}%
\pgfsetbuttcap%
\pgfsetroundjoin%
\pgfsetlinewidth{0.301125pt}%
\definecolor{currentstroke}{rgb}{0.500000,0.500000,0.500000}%
\pgfsetstrokecolor{currentstroke}%
\pgfsetstrokeopacity{0.300000}%
\pgfsetdash{}{0pt}%
\pgfpathmoveto{\pgfqpoint{2.310038in}{1.414198in}}%
\pgfpathlineto{\pgfqpoint{2.285635in}{1.464262in}}%
\pgfpathlineto{\pgfqpoint{2.262419in}{1.514492in}}%
\pgfpathlineto{\pgfqpoint{2.240481in}{1.564893in}}%
\pgfpathlineto{\pgfqpoint{2.219932in}{1.615468in}}%
\pgfpathlineto{\pgfqpoint{2.200915in}{1.666220in}}%
\pgfpathlineto{\pgfqpoint{2.183601in}{1.717153in}}%
\pgfpathlineto{\pgfqpoint{2.168203in}{1.768268in}}%
\pgfpathlineto{\pgfqpoint{2.154997in}{1.819563in}}%
\pgfpathlineto{\pgfqpoint{2.144316in}{1.871032in}}%
\pgfpathlineto{\pgfqpoint{2.136576in}{1.922655in}}%
\pgfpathlineto{\pgfqpoint{2.132301in}{1.974394in}}%
\pgfpathlineto{\pgfqpoint{2.132129in}{2.026183in}}%
\pgfpathlineto{\pgfqpoint{2.136836in}{2.077899in}}%
\pgfpathlineto{\pgfqpoint{2.147338in}{2.129351in}}%
\pgfpathlineto{\pgfqpoint{2.164684in}{2.180234in}}%
\pgfpathlineto{\pgfqpoint{2.190038in}{2.230090in}}%
\pgfpathlineto{\pgfqpoint{2.224665in}{2.278234in}}%
\pgfpathlineto{\pgfqpoint{2.269816in}{2.323669in}}%
\pgfpathlineto{\pgfqpoint{2.326695in}{2.364920in}}%
\pgfpathlineto{\pgfqpoint{2.396312in}{2.399758in}}%
\pgfusepath{stroke}%
\end{pgfscope}%
\begin{pgfscope}%
\pgfpathrectangle{\pgfqpoint{0.647939in}{0.492442in}}{\pgfqpoint{4.273799in}{2.331163in}}%
\pgfusepath{clip}%
\pgfsetbuttcap%
\pgfsetroundjoin%
\pgfsetlinewidth{0.301125pt}%
\definecolor{currentstroke}{rgb}{0.500000,0.500000,0.500000}%
\pgfsetstrokecolor{currentstroke}%
\pgfsetstrokeopacity{0.300000}%
\pgfsetdash{}{0pt}%
\pgfpathmoveto{\pgfqpoint{1.406542in}{2.291626in}}%
\pgfpathlineto{\pgfqpoint{1.424993in}{2.240815in}}%
\pgfpathlineto{\pgfqpoint{1.442417in}{2.189897in}}%
\pgfpathlineto{\pgfqpoint{1.458190in}{2.138824in}}%
\pgfpathlineto{\pgfqpoint{1.471268in}{2.087530in}}%
\pgfpathlineto{\pgfqpoint{1.479730in}{2.035965in}}%
\pgfpathlineto{\pgfqpoint{1.480418in}{1.984246in}}%
\pgfpathlineto{\pgfqpoint{1.470036in}{1.932927in}}%
\pgfpathlineto{\pgfqpoint{1.448868in}{1.882599in}}%
\pgfpathlineto{\pgfqpoint{1.420646in}{1.833242in}}%
\pgfusepath{stroke}%
\end{pgfscope}%
\begin{pgfscope}%
\pgfpathrectangle{\pgfqpoint{0.647939in}{0.492442in}}{\pgfqpoint{4.273799in}{2.331163in}}%
\pgfusepath{clip}%
\pgfsetbuttcap%
\pgfsetroundjoin%
\pgfsetlinewidth{0.301125pt}%
\definecolor{currentstroke}{rgb}{0.500000,0.500000,0.500000}%
\pgfsetstrokecolor{currentstroke}%
\pgfsetstrokeopacity{0.300000}%
\pgfsetdash{}{0pt}%
\pgfpathmoveto{\pgfqpoint{4.105671in}{1.034288in}}%
\pgfpathlineto{\pgfqpoint{4.047552in}{1.075233in}}%
\pgfpathlineto{\pgfqpoint{3.985578in}{1.114457in}}%
\pgfpathlineto{\pgfqpoint{3.919776in}{1.151781in}}%
\pgfpathlineto{\pgfqpoint{3.850489in}{1.187184in}}%
\pgfpathlineto{\pgfqpoint{3.778394in}{1.220890in}}%
\pgfusepath{stroke}%
\end{pgfscope}%
\begin{pgfscope}%
\pgfpathrectangle{\pgfqpoint{0.647939in}{0.492442in}}{\pgfqpoint{4.273799in}{2.331163in}}%
\pgfusepath{clip}%
\pgfsetbuttcap%
\pgfsetroundjoin%
\pgfsetlinewidth{0.301125pt}%
\definecolor{currentstroke}{rgb}{0.500000,0.500000,0.500000}%
\pgfsetstrokecolor{currentstroke}%
\pgfsetstrokeopacity{0.300000}%
\pgfsetdash{}{0pt}%
\pgfpathmoveto{\pgfqpoint{2.445536in}{2.249398in}}%
\pgfpathlineto{\pgfqpoint{2.493443in}{2.293796in}}%
\pgfpathlineto{\pgfqpoint{2.493443in}{2.293796in}}%
\pgfpathlineto{\pgfqpoint{2.546348in}{2.325224in}}%
\pgfpathlineto{\pgfqpoint{2.613324in}{2.347872in}}%
\pgfpathlineto{\pgfqpoint{2.678451in}{2.356587in}}%
\pgfpathlineto{\pgfqpoint{2.739898in}{2.354886in}}%
\pgfpathlineto{\pgfqpoint{2.799177in}{2.344944in}}%
\pgfusepath{stroke}%
\end{pgfscope}%
\begin{pgfscope}%
\pgfpathrectangle{\pgfqpoint{0.647939in}{0.492442in}}{\pgfqpoint{4.273799in}{2.331163in}}%
\pgfusepath{clip}%
\pgfsetbuttcap%
\pgfsetroundjoin%
\pgfsetlinewidth{0.301125pt}%
\definecolor{currentstroke}{rgb}{0.500000,0.500000,0.500000}%
\pgfsetstrokecolor{currentstroke}%
\pgfsetstrokeopacity{0.300000}%
\pgfsetdash{}{0pt}%
\pgfpathmoveto{\pgfqpoint{3.788215in}{2.174530in}}%
\pgfpathlineto{\pgfqpoint{3.801125in}{2.123217in}}%
\pgfpathlineto{\pgfqpoint{3.810307in}{2.071670in}}%
\pgfpathlineto{\pgfqpoint{3.814840in}{2.019946in}}%
\pgfpathlineto{\pgfqpoint{3.813483in}{1.968184in}}%
\pgfpathlineto{\pgfqpoint{3.804627in}{1.916668in}}%
\pgfpathlineto{\pgfqpoint{3.786264in}{1.865941in}}%
\pgfpathlineto{\pgfqpoint{3.756157in}{1.816967in}}%
\pgfpathlineto{\pgfqpoint{3.712023in}{1.771390in}}%
\pgfusepath{stroke}%
\end{pgfscope}%
\begin{pgfscope}%
\pgfpathrectangle{\pgfqpoint{0.647939in}{0.492442in}}{\pgfqpoint{4.273799in}{2.331163in}}%
\pgfusepath{clip}%
\pgfsetbuttcap%
\pgfsetroundjoin%
\pgfsetlinewidth{0.301125pt}%
\definecolor{currentstroke}{rgb}{0.500000,0.500000,0.500000}%
\pgfsetstrokecolor{currentstroke}%
\pgfsetstrokeopacity{0.300000}%
\pgfsetdash{}{0pt}%
\pgfpathmoveto{\pgfqpoint{1.813521in}{1.711005in}}%
\pgfpathlineto{\pgfqpoint{1.792166in}{1.761479in}}%
\pgfpathlineto{\pgfqpoint{1.772355in}{1.812139in}}%
\pgfpathlineto{\pgfqpoint{1.754677in}{1.863029in}}%
\pgfpathlineto{\pgfqpoint{1.740133in}{1.914205in}}%
\pgfpathlineto{\pgfqpoint{1.730509in}{1.965708in}}%
\pgfpathlineto{\pgfqpoint{1.728777in}{2.017425in}}%
\pgfusepath{stroke}%
\end{pgfscope}%
\begin{pgfscope}%
\pgfpathrectangle{\pgfqpoint{0.647939in}{0.492442in}}{\pgfqpoint{4.273799in}{2.331163in}}%
\pgfusepath{clip}%
\pgfsetbuttcap%
\pgfsetroundjoin%
\pgfsetlinewidth{0.301125pt}%
\definecolor{currentstroke}{rgb}{0.500000,0.500000,0.500000}%
\pgfsetstrokecolor{currentstroke}%
\pgfsetstrokeopacity{0.300000}%
\pgfsetdash{}{0pt}%
\pgfpathmoveto{\pgfqpoint{2.556081in}{1.083387in}}%
\pgfpathlineto{\pgfqpoint{2.524269in}{1.132197in}}%
\pgfpathlineto{\pgfqpoint{2.493443in}{1.181195in}}%
\pgfpathlineto{\pgfqpoint{2.463621in}{1.230377in}}%
\pgfpathlineto{\pgfqpoint{2.434817in}{1.279740in}}%
\pgfpathlineto{\pgfqpoint{2.407060in}{1.329281in}}%
\pgfpathlineto{\pgfqpoint{2.380391in}{1.378999in}}%
\pgfusepath{stroke}%
\end{pgfscope}%
\begin{pgfscope}%
\pgfpathrectangle{\pgfqpoint{0.647939in}{0.492442in}}{\pgfqpoint{4.273799in}{2.331163in}}%
\pgfusepath{clip}%
\pgfsetbuttcap%
\pgfsetroundjoin%
\pgfsetlinewidth{0.301125pt}%
\definecolor{currentstroke}{rgb}{0.500000,0.500000,0.500000}%
\pgfsetstrokecolor{currentstroke}%
\pgfsetstrokeopacity{0.300000}%
\pgfsetdash{}{0pt}%
\pgfpathmoveto{\pgfqpoint{2.794539in}{1.785558in}}%
\pgfpathlineto{\pgfqpoint{2.773919in}{1.836113in}}%
\pgfpathlineto{\pgfqpoint{2.757475in}{1.887117in}}%
\pgfpathlineto{\pgfqpoint{2.745953in}{1.938513in}}%
\pgfpathlineto{\pgfqpoint{2.740514in}{1.990192in}}%
\pgfpathlineto{\pgfqpoint{2.743098in}{2.041905in}}%
\pgfpathlineto{\pgfqpoint{2.754447in}{2.086338in}}%
\pgfpathlineto{\pgfqpoint{2.784839in}{2.134853in}}%
\pgfpathlineto{\pgfqpoint{2.784839in}{2.134853in}}%
\pgfpathlineto{\pgfqpoint{2.816362in}{2.159865in}}%
\pgfpathlineto{\pgfqpoint{2.816362in}{2.159865in}}%
\pgfusepath{stroke}%
\end{pgfscope}%
\begin{pgfscope}%
\pgfpathrectangle{\pgfqpoint{0.647939in}{0.492442in}}{\pgfqpoint{4.273799in}{2.331163in}}%
\pgfusepath{clip}%
\pgfsetbuttcap%
\pgfsetroundjoin%
\pgfsetlinewidth{0.301125pt}%
\definecolor{currentstroke}{rgb}{0.500000,0.500000,0.500000}%
\pgfsetstrokecolor{currentstroke}%
\pgfsetstrokeopacity{0.300000}%
\pgfsetdash{}{0pt}%
\pgfpathmoveto{\pgfqpoint{2.196610in}{1.876559in}}%
\pgfpathlineto{\pgfqpoint{2.189269in}{1.928200in}}%
\pgfpathlineto{\pgfqpoint{2.185436in}{1.979950in}}%
\pgfpathlineto{\pgfqpoint{2.185745in}{2.031737in}}%
\pgfpathlineto{\pgfqpoint{2.190971in}{2.083437in}}%
\pgfpathlineto{\pgfqpoint{2.202048in}{2.134853in}}%
\pgfusepath{stroke}%
\end{pgfscope}%
\begin{pgfscope}%
\pgfpathrectangle{\pgfqpoint{0.647939in}{0.492442in}}{\pgfqpoint{4.273799in}{2.331163in}}%
\pgfusepath{clip}%
\pgfsetbuttcap%
\pgfsetroundjoin%
\pgfsetlinewidth{0.301125pt}%
\definecolor{currentstroke}{rgb}{0.500000,0.500000,0.500000}%
\pgfsetstrokecolor{currentstroke}%
\pgfsetstrokeopacity{0.300000}%
\pgfsetdash{}{0pt}%
\pgfpathmoveto{\pgfqpoint{1.901944in}{1.877095in}}%
\pgfpathlineto{\pgfqpoint{1.892929in}{1.928654in}}%
\pgfpathlineto{\pgfqpoint{1.888099in}{1.980373in}}%
\pgfpathlineto{\pgfqpoint{1.888583in}{2.032147in}}%
\pgfpathlineto{\pgfqpoint{1.895677in}{2.083760in}}%
\pgfpathlineto{\pgfqpoint{1.910652in}{2.134853in}}%
\pgfusepath{stroke}%
\end{pgfscope}%
\begin{pgfscope}%
\pgfpathrectangle{\pgfqpoint{0.647939in}{0.492442in}}{\pgfqpoint{4.273799in}{2.331163in}}%
\pgfusepath{clip}%
\pgfsetbuttcap%
\pgfsetroundjoin%
\pgfsetlinewidth{0.301125pt}%
\definecolor{currentstroke}{rgb}{0.500000,0.500000,0.500000}%
\pgfsetstrokecolor{currentstroke}%
\pgfsetstrokeopacity{0.300000}%
\pgfsetdash{}{0pt}%
\pgfpathmoveto{\pgfqpoint{2.752188in}{1.295673in}}%
\pgfpathlineto{\pgfqpoint{2.719180in}{1.344245in}}%
\pgfpathlineto{\pgfqpoint{2.687707in}{1.393119in}}%
\pgfpathlineto{\pgfqpoint{2.657802in}{1.442284in}}%
\pgfpathlineto{\pgfqpoint{2.629519in}{1.491734in}}%
\pgfpathlineto{\pgfqpoint{2.602925in}{1.541463in}}%
\pgfusepath{stroke}%
\end{pgfscope}%
\begin{pgfscope}%
\pgfpathrectangle{\pgfqpoint{0.647939in}{0.492442in}}{\pgfqpoint{4.273799in}{2.331163in}}%
\pgfusepath{clip}%
\pgfsetroundcap%
\pgfsetroundjoin%
\pgfsetlinewidth{0.301125pt}%
\definecolor{currentstroke}{rgb}{0.500000,0.500000,0.500000}%
\pgfsetstrokecolor{currentstroke}%
\pgfsetstrokeopacity{0.300000}%
\pgfsetdash{}{0pt}%
\pgfpathmoveto{\pgfqpoint{1.439009in}{1.485359in}}%
\pgfusepath{stroke}%
\end{pgfscope}%
\begin{pgfscope}%
\pgfpathrectangle{\pgfqpoint{0.647939in}{0.492442in}}{\pgfqpoint{4.273799in}{2.331163in}}%
\pgfusepath{clip}%
\pgfsetroundcap%
\pgfsetroundjoin%
\definecolor{currentfill}{rgb}{0.500000,0.500000,0.500000}%
\pgfsetfillcolor{currentfill}%
\pgfsetfillopacity{0.300000}%
\pgfsetlinewidth{0.301125pt}%
\definecolor{currentstroke}{rgb}{0.500000,0.500000,0.500000}%
\pgfsetstrokecolor{currentstroke}%
\pgfsetstrokeopacity{0.300000}%
\pgfsetdash{}{0pt}%
\pgfpathmoveto{\pgfqpoint{0.000000in}{0.000000in}}%
\pgfpathlineto{\pgfqpoint{0.000000in}{0.000000in}}%
\pgfpathclose%
\pgfusepath{stroke,fill}%
\end{pgfscope}%
\begin{pgfscope}%
\pgfpathrectangle{\pgfqpoint{0.647939in}{0.492442in}}{\pgfqpoint{4.273799in}{2.331163in}}%
\pgfusepath{clip}%
\pgfsetroundcap%
\pgfsetroundjoin%
\pgfsetlinewidth{0.301125pt}%
\definecolor{currentstroke}{rgb}{0.500000,0.500000,0.500000}%
\pgfsetstrokecolor{currentstroke}%
\pgfsetstrokeopacity{0.300000}%
\pgfsetdash{}{0pt}%
\pgfpathmoveto{\pgfqpoint{1.242599in}{0.902764in}}%
\pgfusepath{stroke}%
\end{pgfscope}%
\begin{pgfscope}%
\pgfpathrectangle{\pgfqpoint{0.647939in}{0.492442in}}{\pgfqpoint{4.273799in}{2.331163in}}%
\pgfusepath{clip}%
\pgfsetroundcap%
\pgfsetroundjoin%
\definecolor{currentfill}{rgb}{0.500000,0.500000,0.500000}%
\pgfsetfillcolor{currentfill}%
\pgfsetfillopacity{0.300000}%
\pgfsetlinewidth{0.301125pt}%
\definecolor{currentstroke}{rgb}{0.500000,0.500000,0.500000}%
\pgfsetstrokecolor{currentstroke}%
\pgfsetstrokeopacity{0.300000}%
\pgfsetdash{}{0pt}%
\pgfpathmoveto{\pgfqpoint{0.000000in}{0.000000in}}%
\pgfpathlineto{\pgfqpoint{0.000000in}{0.000000in}}%
\pgfpathclose%
\pgfusepath{stroke,fill}%
\end{pgfscope}%
\begin{pgfscope}%
\pgfpathrectangle{\pgfqpoint{0.647939in}{0.492442in}}{\pgfqpoint{4.273799in}{2.331163in}}%
\pgfusepath{clip}%
\pgfsetroundcap%
\pgfsetroundjoin%
\pgfsetlinewidth{0.301125pt}%
\definecolor{currentstroke}{rgb}{0.500000,0.500000,0.500000}%
\pgfsetstrokecolor{currentstroke}%
\pgfsetstrokeopacity{0.300000}%
\pgfsetdash{}{0pt}%
\pgfpathmoveto{\pgfqpoint{1.208409in}{0.682350in}}%
\pgfusepath{stroke}%
\end{pgfscope}%
\begin{pgfscope}%
\pgfpathrectangle{\pgfqpoint{0.647939in}{0.492442in}}{\pgfqpoint{4.273799in}{2.331163in}}%
\pgfusepath{clip}%
\pgfsetroundcap%
\pgfsetroundjoin%
\definecolor{currentfill}{rgb}{0.500000,0.500000,0.500000}%
\pgfsetfillcolor{currentfill}%
\pgfsetfillopacity{0.300000}%
\pgfsetlinewidth{0.301125pt}%
\definecolor{currentstroke}{rgb}{0.500000,0.500000,0.500000}%
\pgfsetstrokecolor{currentstroke}%
\pgfsetstrokeopacity{0.300000}%
\pgfsetdash{}{0pt}%
\pgfpathmoveto{\pgfqpoint{0.000000in}{0.000000in}}%
\pgfpathlineto{\pgfqpoint{0.000000in}{0.000000in}}%
\pgfpathclose%
\pgfusepath{stroke,fill}%
\end{pgfscope}%
\begin{pgfscope}%
\pgfpathrectangle{\pgfqpoint{0.647939in}{0.492442in}}{\pgfqpoint{4.273799in}{2.331163in}}%
\pgfusepath{clip}%
\pgfsetroundcap%
\pgfsetroundjoin%
\pgfsetlinewidth{0.301125pt}%
\definecolor{currentstroke}{rgb}{0.500000,0.500000,0.500000}%
\pgfsetstrokecolor{currentstroke}%
\pgfsetstrokeopacity{0.300000}%
\pgfsetdash{}{0pt}%
\pgfpathmoveto{\pgfqpoint{1.162820in}{0.566955in}}%
\pgfusepath{stroke}%
\end{pgfscope}%
\begin{pgfscope}%
\pgfpathrectangle{\pgfqpoint{0.647939in}{0.492442in}}{\pgfqpoint{4.273799in}{2.331163in}}%
\pgfusepath{clip}%
\pgfsetroundcap%
\pgfsetroundjoin%
\definecolor{currentfill}{rgb}{0.500000,0.500000,0.500000}%
\pgfsetfillcolor{currentfill}%
\pgfsetfillopacity{0.300000}%
\pgfsetlinewidth{0.301125pt}%
\definecolor{currentstroke}{rgb}{0.500000,0.500000,0.500000}%
\pgfsetstrokecolor{currentstroke}%
\pgfsetstrokeopacity{0.300000}%
\pgfsetdash{}{0pt}%
\pgfpathmoveto{\pgfqpoint{0.000000in}{0.000000in}}%
\pgfpathlineto{\pgfqpoint{0.000000in}{0.000000in}}%
\pgfpathclose%
\pgfusepath{stroke,fill}%
\end{pgfscope}%
\begin{pgfscope}%
\pgfpathrectangle{\pgfqpoint{0.647939in}{0.492442in}}{\pgfqpoint{4.273799in}{2.331163in}}%
\pgfusepath{clip}%
\pgfsetroundcap%
\pgfsetroundjoin%
\pgfsetlinewidth{0.301125pt}%
\definecolor{currentstroke}{rgb}{0.500000,0.500000,0.500000}%
\pgfsetstrokecolor{currentstroke}%
\pgfsetstrokeopacity{0.300000}%
\pgfsetdash{}{0pt}%
\pgfpathmoveto{\pgfqpoint{1.393047in}{0.681221in}}%
\pgfusepath{stroke}%
\end{pgfscope}%
\begin{pgfscope}%
\pgfpathrectangle{\pgfqpoint{0.647939in}{0.492442in}}{\pgfqpoint{4.273799in}{2.331163in}}%
\pgfusepath{clip}%
\pgfsetroundcap%
\pgfsetroundjoin%
\definecolor{currentfill}{rgb}{0.500000,0.500000,0.500000}%
\pgfsetfillcolor{currentfill}%
\pgfsetfillopacity{0.300000}%
\pgfsetlinewidth{0.301125pt}%
\definecolor{currentstroke}{rgb}{0.500000,0.500000,0.500000}%
\pgfsetstrokecolor{currentstroke}%
\pgfsetstrokeopacity{0.300000}%
\pgfsetdash{}{0pt}%
\pgfpathmoveto{\pgfqpoint{0.000000in}{0.000000in}}%
\pgfpathlineto{\pgfqpoint{0.000000in}{0.000000in}}%
\pgfpathclose%
\pgfusepath{stroke,fill}%
\end{pgfscope}%
\begin{pgfscope}%
\pgfpathrectangle{\pgfqpoint{0.647939in}{0.492442in}}{\pgfqpoint{4.273799in}{2.331163in}}%
\pgfusepath{clip}%
\pgfsetroundcap%
\pgfsetroundjoin%
\pgfsetlinewidth{0.301125pt}%
\definecolor{currentstroke}{rgb}{0.500000,0.500000,0.500000}%
\pgfsetstrokecolor{currentstroke}%
\pgfsetstrokeopacity{0.300000}%
\pgfsetdash{}{0pt}%
\pgfpathmoveto{\pgfqpoint{1.416287in}{1.011616in}}%
\pgfusepath{stroke}%
\end{pgfscope}%
\begin{pgfscope}%
\pgfpathrectangle{\pgfqpoint{0.647939in}{0.492442in}}{\pgfqpoint{4.273799in}{2.331163in}}%
\pgfusepath{clip}%
\pgfsetroundcap%
\pgfsetroundjoin%
\definecolor{currentfill}{rgb}{0.500000,0.500000,0.500000}%
\pgfsetfillcolor{currentfill}%
\pgfsetfillopacity{0.300000}%
\pgfsetlinewidth{0.301125pt}%
\definecolor{currentstroke}{rgb}{0.500000,0.500000,0.500000}%
\pgfsetstrokecolor{currentstroke}%
\pgfsetstrokeopacity{0.300000}%
\pgfsetdash{}{0pt}%
\pgfpathmoveto{\pgfqpoint{0.000000in}{0.000000in}}%
\pgfpathlineto{\pgfqpoint{0.000000in}{0.000000in}}%
\pgfpathclose%
\pgfusepath{stroke,fill}%
\end{pgfscope}%
\begin{pgfscope}%
\pgfpathrectangle{\pgfqpoint{0.647939in}{0.492442in}}{\pgfqpoint{4.273799in}{2.331163in}}%
\pgfusepath{clip}%
\pgfsetroundcap%
\pgfsetroundjoin%
\pgfsetlinewidth{0.301125pt}%
\definecolor{currentstroke}{rgb}{0.500000,0.500000,0.500000}%
\pgfsetstrokecolor{currentstroke}%
\pgfsetstrokeopacity{0.300000}%
\pgfsetdash{}{0pt}%
\pgfpathmoveto{\pgfqpoint{1.616326in}{0.988924in}}%
\pgfusepath{stroke}%
\end{pgfscope}%
\begin{pgfscope}%
\pgfpathrectangle{\pgfqpoint{0.647939in}{0.492442in}}{\pgfqpoint{4.273799in}{2.331163in}}%
\pgfusepath{clip}%
\pgfsetroundcap%
\pgfsetroundjoin%
\definecolor{currentfill}{rgb}{0.500000,0.500000,0.500000}%
\pgfsetfillcolor{currentfill}%
\pgfsetfillopacity{0.300000}%
\pgfsetlinewidth{0.301125pt}%
\definecolor{currentstroke}{rgb}{0.500000,0.500000,0.500000}%
\pgfsetstrokecolor{currentstroke}%
\pgfsetstrokeopacity{0.300000}%
\pgfsetdash{}{0pt}%
\pgfpathmoveto{\pgfqpoint{0.000000in}{0.000000in}}%
\pgfpathlineto{\pgfqpoint{0.000000in}{0.000000in}}%
\pgfpathclose%
\pgfusepath{stroke,fill}%
\end{pgfscope}%
\begin{pgfscope}%
\pgfpathrectangle{\pgfqpoint{0.647939in}{0.492442in}}{\pgfqpoint{4.273799in}{2.331163in}}%
\pgfusepath{clip}%
\pgfsetroundcap%
\pgfsetroundjoin%
\pgfsetlinewidth{0.301125pt}%
\definecolor{currentstroke}{rgb}{0.500000,0.500000,0.500000}%
\pgfsetstrokecolor{currentstroke}%
\pgfsetstrokeopacity{0.300000}%
\pgfsetdash{}{0pt}%
\pgfpathmoveto{\pgfqpoint{1.699492in}{1.042959in}}%
\pgfusepath{stroke}%
\end{pgfscope}%
\begin{pgfscope}%
\pgfpathrectangle{\pgfqpoint{0.647939in}{0.492442in}}{\pgfqpoint{4.273799in}{2.331163in}}%
\pgfusepath{clip}%
\pgfsetroundcap%
\pgfsetroundjoin%
\definecolor{currentfill}{rgb}{0.500000,0.500000,0.500000}%
\pgfsetfillcolor{currentfill}%
\pgfsetfillopacity{0.300000}%
\pgfsetlinewidth{0.301125pt}%
\definecolor{currentstroke}{rgb}{0.500000,0.500000,0.500000}%
\pgfsetstrokecolor{currentstroke}%
\pgfsetstrokeopacity{0.300000}%
\pgfsetdash{}{0pt}%
\pgfpathmoveto{\pgfqpoint{0.000000in}{0.000000in}}%
\pgfpathlineto{\pgfqpoint{0.000000in}{0.000000in}}%
\pgfpathclose%
\pgfusepath{stroke,fill}%
\end{pgfscope}%
\begin{pgfscope}%
\pgfpathrectangle{\pgfqpoint{0.647939in}{0.492442in}}{\pgfqpoint{4.273799in}{2.331163in}}%
\pgfusepath{clip}%
\pgfsetroundcap%
\pgfsetroundjoin%
\pgfsetlinewidth{0.301125pt}%
\definecolor{currentstroke}{rgb}{0.500000,0.500000,0.500000}%
\pgfsetstrokecolor{currentstroke}%
\pgfsetstrokeopacity{0.300000}%
\pgfsetdash{}{0pt}%
\pgfpathmoveto{\pgfqpoint{1.837249in}{1.344995in}}%
\pgfusepath{stroke}%
\end{pgfscope}%
\begin{pgfscope}%
\pgfpathrectangle{\pgfqpoint{0.647939in}{0.492442in}}{\pgfqpoint{4.273799in}{2.331163in}}%
\pgfusepath{clip}%
\pgfsetroundcap%
\pgfsetroundjoin%
\definecolor{currentfill}{rgb}{0.500000,0.500000,0.500000}%
\pgfsetfillcolor{currentfill}%
\pgfsetfillopacity{0.300000}%
\pgfsetlinewidth{0.301125pt}%
\definecolor{currentstroke}{rgb}{0.500000,0.500000,0.500000}%
\pgfsetstrokecolor{currentstroke}%
\pgfsetstrokeopacity{0.300000}%
\pgfsetdash{}{0pt}%
\pgfpathmoveto{\pgfqpoint{0.000000in}{0.000000in}}%
\pgfpathlineto{\pgfqpoint{0.000000in}{0.000000in}}%
\pgfpathclose%
\pgfusepath{stroke,fill}%
\end{pgfscope}%
\begin{pgfscope}%
\pgfpathrectangle{\pgfqpoint{0.647939in}{0.492442in}}{\pgfqpoint{4.273799in}{2.331163in}}%
\pgfusepath{clip}%
\pgfsetroundcap%
\pgfsetroundjoin%
\pgfsetlinewidth{0.301125pt}%
\definecolor{currentstroke}{rgb}{0.500000,0.500000,0.500000}%
\pgfsetstrokecolor{currentstroke}%
\pgfsetstrokeopacity{0.300000}%
\pgfsetdash{}{0pt}%
\pgfpathmoveto{\pgfqpoint{1.757694in}{2.162044in}}%
\pgfusepath{stroke}%
\end{pgfscope}%
\begin{pgfscope}%
\pgfpathrectangle{\pgfqpoint{0.647939in}{0.492442in}}{\pgfqpoint{4.273799in}{2.331163in}}%
\pgfusepath{clip}%
\pgfsetroundcap%
\pgfsetroundjoin%
\definecolor{currentfill}{rgb}{0.500000,0.500000,0.500000}%
\pgfsetfillcolor{currentfill}%
\pgfsetfillopacity{0.300000}%
\pgfsetlinewidth{0.301125pt}%
\definecolor{currentstroke}{rgb}{0.500000,0.500000,0.500000}%
\pgfsetstrokecolor{currentstroke}%
\pgfsetstrokeopacity{0.300000}%
\pgfsetdash{}{0pt}%
\pgfpathmoveto{\pgfqpoint{0.000000in}{0.000000in}}%
\pgfpathlineto{\pgfqpoint{0.000000in}{0.000000in}}%
\pgfpathclose%
\pgfusepath{stroke,fill}%
\end{pgfscope}%
\begin{pgfscope}%
\pgfpathrectangle{\pgfqpoint{0.647939in}{0.492442in}}{\pgfqpoint{4.273799in}{2.331163in}}%
\pgfusepath{clip}%
\pgfsetroundcap%
\pgfsetroundjoin%
\pgfsetlinewidth{0.301125pt}%
\definecolor{currentstroke}{rgb}{0.500000,0.500000,0.500000}%
\pgfsetstrokecolor{currentstroke}%
\pgfsetstrokeopacity{0.300000}%
\pgfsetdash{}{0pt}%
\pgfpathmoveto{\pgfqpoint{2.501626in}{0.610577in}}%
\pgfusepath{stroke}%
\end{pgfscope}%
\begin{pgfscope}%
\pgfpathrectangle{\pgfqpoint{0.647939in}{0.492442in}}{\pgfqpoint{4.273799in}{2.331163in}}%
\pgfusepath{clip}%
\pgfsetroundcap%
\pgfsetroundjoin%
\definecolor{currentfill}{rgb}{0.500000,0.500000,0.500000}%
\pgfsetfillcolor{currentfill}%
\pgfsetfillopacity{0.300000}%
\pgfsetlinewidth{0.301125pt}%
\definecolor{currentstroke}{rgb}{0.500000,0.500000,0.500000}%
\pgfsetstrokecolor{currentstroke}%
\pgfsetstrokeopacity{0.300000}%
\pgfsetdash{}{0pt}%
\pgfpathmoveto{\pgfqpoint{0.000000in}{0.000000in}}%
\pgfpathlineto{\pgfqpoint{0.000000in}{0.000000in}}%
\pgfpathclose%
\pgfusepath{stroke,fill}%
\end{pgfscope}%
\begin{pgfscope}%
\pgfpathrectangle{\pgfqpoint{0.647939in}{0.492442in}}{\pgfqpoint{4.273799in}{2.331163in}}%
\pgfusepath{clip}%
\pgfsetroundcap%
\pgfsetroundjoin%
\pgfsetlinewidth{0.301125pt}%
\definecolor{currentstroke}{rgb}{0.500000,0.500000,0.500000}%
\pgfsetstrokecolor{currentstroke}%
\pgfsetstrokeopacity{0.300000}%
\pgfsetdash{}{0pt}%
\pgfpathmoveto{\pgfqpoint{2.159248in}{1.441902in}}%
\pgfusepath{stroke}%
\end{pgfscope}%
\begin{pgfscope}%
\pgfpathrectangle{\pgfqpoint{0.647939in}{0.492442in}}{\pgfqpoint{4.273799in}{2.331163in}}%
\pgfusepath{clip}%
\pgfsetroundcap%
\pgfsetroundjoin%
\definecolor{currentfill}{rgb}{0.500000,0.500000,0.500000}%
\pgfsetfillcolor{currentfill}%
\pgfsetfillopacity{0.300000}%
\pgfsetlinewidth{0.301125pt}%
\definecolor{currentstroke}{rgb}{0.500000,0.500000,0.500000}%
\pgfsetstrokecolor{currentstroke}%
\pgfsetstrokeopacity{0.300000}%
\pgfsetdash{}{0pt}%
\pgfpathmoveto{\pgfqpoint{0.000000in}{0.000000in}}%
\pgfpathlineto{\pgfqpoint{0.000000in}{0.000000in}}%
\pgfpathclose%
\pgfusepath{stroke,fill}%
\end{pgfscope}%
\begin{pgfscope}%
\pgfpathrectangle{\pgfqpoint{0.647939in}{0.492442in}}{\pgfqpoint{4.273799in}{2.331163in}}%
\pgfusepath{clip}%
\pgfsetroundcap%
\pgfsetroundjoin%
\pgfsetlinewidth{0.301125pt}%
\definecolor{currentstroke}{rgb}{0.500000,0.500000,0.500000}%
\pgfsetstrokecolor{currentstroke}%
\pgfsetstrokeopacity{0.300000}%
\pgfsetdash{}{0pt}%
\pgfpathmoveto{\pgfqpoint{2.562253in}{0.894876in}}%
\pgfusepath{stroke}%
\end{pgfscope}%
\begin{pgfscope}%
\pgfpathrectangle{\pgfqpoint{0.647939in}{0.492442in}}{\pgfqpoint{4.273799in}{2.331163in}}%
\pgfusepath{clip}%
\pgfsetroundcap%
\pgfsetroundjoin%
\definecolor{currentfill}{rgb}{0.500000,0.500000,0.500000}%
\pgfsetfillcolor{currentfill}%
\pgfsetfillopacity{0.300000}%
\pgfsetlinewidth{0.301125pt}%
\definecolor{currentstroke}{rgb}{0.500000,0.500000,0.500000}%
\pgfsetstrokecolor{currentstroke}%
\pgfsetstrokeopacity{0.300000}%
\pgfsetdash{}{0pt}%
\pgfpathmoveto{\pgfqpoint{0.000000in}{0.000000in}}%
\pgfpathlineto{\pgfqpoint{0.000000in}{0.000000in}}%
\pgfpathclose%
\pgfusepath{stroke,fill}%
\end{pgfscope}%
\begin{pgfscope}%
\pgfpathrectangle{\pgfqpoint{0.647939in}{0.492442in}}{\pgfqpoint{4.273799in}{2.331163in}}%
\pgfusepath{clip}%
\pgfsetroundcap%
\pgfsetroundjoin%
\pgfsetlinewidth{0.301125pt}%
\definecolor{currentstroke}{rgb}{0.500000,0.500000,0.500000}%
\pgfsetstrokecolor{currentstroke}%
\pgfsetstrokeopacity{0.300000}%
\pgfsetdash{}{0pt}%
\pgfpathmoveto{\pgfqpoint{2.875653in}{0.605933in}}%
\pgfusepath{stroke}%
\end{pgfscope}%
\begin{pgfscope}%
\pgfpathrectangle{\pgfqpoint{0.647939in}{0.492442in}}{\pgfqpoint{4.273799in}{2.331163in}}%
\pgfusepath{clip}%
\pgfsetroundcap%
\pgfsetroundjoin%
\definecolor{currentfill}{rgb}{0.500000,0.500000,0.500000}%
\pgfsetfillcolor{currentfill}%
\pgfsetfillopacity{0.300000}%
\pgfsetlinewidth{0.301125pt}%
\definecolor{currentstroke}{rgb}{0.500000,0.500000,0.500000}%
\pgfsetstrokecolor{currentstroke}%
\pgfsetstrokeopacity{0.300000}%
\pgfsetdash{}{0pt}%
\pgfpathmoveto{\pgfqpoint{0.000000in}{0.000000in}}%
\pgfpathlineto{\pgfqpoint{0.000000in}{0.000000in}}%
\pgfpathclose%
\pgfusepath{stroke,fill}%
\end{pgfscope}%
\begin{pgfscope}%
\pgfpathrectangle{\pgfqpoint{0.647939in}{0.492442in}}{\pgfqpoint{4.273799in}{2.331163in}}%
\pgfusepath{clip}%
\pgfsetroundcap%
\pgfsetroundjoin%
\pgfsetlinewidth{0.301125pt}%
\definecolor{currentstroke}{rgb}{0.500000,0.500000,0.500000}%
\pgfsetstrokecolor{currentstroke}%
\pgfsetstrokeopacity{0.300000}%
\pgfsetdash{}{0pt}%
\pgfpathmoveto{\pgfqpoint{2.388318in}{1.544472in}}%
\pgfusepath{stroke}%
\end{pgfscope}%
\begin{pgfscope}%
\pgfpathrectangle{\pgfqpoint{0.647939in}{0.492442in}}{\pgfqpoint{4.273799in}{2.331163in}}%
\pgfusepath{clip}%
\pgfsetroundcap%
\pgfsetroundjoin%
\definecolor{currentfill}{rgb}{0.500000,0.500000,0.500000}%
\pgfsetfillcolor{currentfill}%
\pgfsetfillopacity{0.300000}%
\pgfsetlinewidth{0.301125pt}%
\definecolor{currentstroke}{rgb}{0.500000,0.500000,0.500000}%
\pgfsetstrokecolor{currentstroke}%
\pgfsetstrokeopacity{0.300000}%
\pgfsetdash{}{0pt}%
\pgfpathmoveto{\pgfqpoint{0.000000in}{0.000000in}}%
\pgfpathlineto{\pgfqpoint{0.000000in}{0.000000in}}%
\pgfpathclose%
\pgfusepath{stroke,fill}%
\end{pgfscope}%
\begin{pgfscope}%
\pgfpathrectangle{\pgfqpoint{0.647939in}{0.492442in}}{\pgfqpoint{4.273799in}{2.331163in}}%
\pgfusepath{clip}%
\pgfsetroundcap%
\pgfsetroundjoin%
\pgfsetlinewidth{0.301125pt}%
\definecolor{currentstroke}{rgb}{0.500000,0.500000,0.500000}%
\pgfsetstrokecolor{currentstroke}%
\pgfsetstrokeopacity{0.300000}%
\pgfsetdash{}{0pt}%
\pgfpathmoveto{\pgfqpoint{2.664353in}{1.252435in}}%
\pgfusepath{stroke}%
\end{pgfscope}%
\begin{pgfscope}%
\pgfpathrectangle{\pgfqpoint{0.647939in}{0.492442in}}{\pgfqpoint{4.273799in}{2.331163in}}%
\pgfusepath{clip}%
\pgfsetroundcap%
\pgfsetroundjoin%
\definecolor{currentfill}{rgb}{0.500000,0.500000,0.500000}%
\pgfsetfillcolor{currentfill}%
\pgfsetfillopacity{0.300000}%
\pgfsetlinewidth{0.301125pt}%
\definecolor{currentstroke}{rgb}{0.500000,0.500000,0.500000}%
\pgfsetstrokecolor{currentstroke}%
\pgfsetstrokeopacity{0.300000}%
\pgfsetdash{}{0pt}%
\pgfpathmoveto{\pgfqpoint{0.000000in}{0.000000in}}%
\pgfpathlineto{\pgfqpoint{0.000000in}{0.000000in}}%
\pgfpathclose%
\pgfusepath{stroke,fill}%
\end{pgfscope}%
\begin{pgfscope}%
\pgfpathrectangle{\pgfqpoint{0.647939in}{0.492442in}}{\pgfqpoint{4.273799in}{2.331163in}}%
\pgfusepath{clip}%
\pgfsetroundcap%
\pgfsetroundjoin%
\pgfsetlinewidth{0.301125pt}%
\definecolor{currentstroke}{rgb}{0.500000,0.500000,0.500000}%
\pgfsetstrokecolor{currentstroke}%
\pgfsetstrokeopacity{0.300000}%
\pgfsetdash{}{0pt}%
\pgfpathmoveto{\pgfqpoint{3.121210in}{0.856488in}}%
\pgfusepath{stroke}%
\end{pgfscope}%
\begin{pgfscope}%
\pgfpathrectangle{\pgfqpoint{0.647939in}{0.492442in}}{\pgfqpoint{4.273799in}{2.331163in}}%
\pgfusepath{clip}%
\pgfsetroundcap%
\pgfsetroundjoin%
\definecolor{currentfill}{rgb}{0.500000,0.500000,0.500000}%
\pgfsetfillcolor{currentfill}%
\pgfsetfillopacity{0.300000}%
\pgfsetlinewidth{0.301125pt}%
\definecolor{currentstroke}{rgb}{0.500000,0.500000,0.500000}%
\pgfsetstrokecolor{currentstroke}%
\pgfsetstrokeopacity{0.300000}%
\pgfsetdash{}{0pt}%
\pgfpathmoveto{\pgfqpoint{0.000000in}{0.000000in}}%
\pgfpathlineto{\pgfqpoint{0.000000in}{0.000000in}}%
\pgfpathclose%
\pgfusepath{stroke,fill}%
\end{pgfscope}%
\begin{pgfscope}%
\pgfpathrectangle{\pgfqpoint{0.647939in}{0.492442in}}{\pgfqpoint{4.273799in}{2.331163in}}%
\pgfusepath{clip}%
\pgfsetroundcap%
\pgfsetroundjoin%
\pgfsetlinewidth{0.301125pt}%
\definecolor{currentstroke}{rgb}{0.500000,0.500000,0.500000}%
\pgfsetstrokecolor{currentstroke}%
\pgfsetstrokeopacity{0.300000}%
\pgfsetdash{}{0pt}%
\pgfpathmoveto{\pgfqpoint{2.735394in}{1.448869in}}%
\pgfusepath{stroke}%
\end{pgfscope}%
\begin{pgfscope}%
\pgfpathrectangle{\pgfqpoint{0.647939in}{0.492442in}}{\pgfqpoint{4.273799in}{2.331163in}}%
\pgfusepath{clip}%
\pgfsetroundcap%
\pgfsetroundjoin%
\definecolor{currentfill}{rgb}{0.500000,0.500000,0.500000}%
\pgfsetfillcolor{currentfill}%
\pgfsetfillopacity{0.300000}%
\pgfsetlinewidth{0.301125pt}%
\definecolor{currentstroke}{rgb}{0.500000,0.500000,0.500000}%
\pgfsetstrokecolor{currentstroke}%
\pgfsetstrokeopacity{0.300000}%
\pgfsetdash{}{0pt}%
\pgfpathmoveto{\pgfqpoint{0.000000in}{0.000000in}}%
\pgfpathlineto{\pgfqpoint{0.000000in}{0.000000in}}%
\pgfpathclose%
\pgfusepath{stroke,fill}%
\end{pgfscope}%
\begin{pgfscope}%
\pgfpathrectangle{\pgfqpoint{0.647939in}{0.492442in}}{\pgfqpoint{4.273799in}{2.331163in}}%
\pgfusepath{clip}%
\pgfsetroundcap%
\pgfsetroundjoin%
\pgfsetlinewidth{0.301125pt}%
\definecolor{currentstroke}{rgb}{0.500000,0.500000,0.500000}%
\pgfsetstrokecolor{currentstroke}%
\pgfsetstrokeopacity{0.300000}%
\pgfsetdash{}{0pt}%
\pgfpathmoveto{\pgfqpoint{3.346620in}{0.918758in}}%
\pgfusepath{stroke}%
\end{pgfscope}%
\begin{pgfscope}%
\pgfpathrectangle{\pgfqpoint{0.647939in}{0.492442in}}{\pgfqpoint{4.273799in}{2.331163in}}%
\pgfusepath{clip}%
\pgfsetroundcap%
\pgfsetroundjoin%
\definecolor{currentfill}{rgb}{0.500000,0.500000,0.500000}%
\pgfsetfillcolor{currentfill}%
\pgfsetfillopacity{0.300000}%
\pgfsetlinewidth{0.301125pt}%
\definecolor{currentstroke}{rgb}{0.500000,0.500000,0.500000}%
\pgfsetstrokecolor{currentstroke}%
\pgfsetstrokeopacity{0.300000}%
\pgfsetdash{}{0pt}%
\pgfpathmoveto{\pgfqpoint{0.000000in}{0.000000in}}%
\pgfpathlineto{\pgfqpoint{0.000000in}{0.000000in}}%
\pgfpathclose%
\pgfusepath{stroke,fill}%
\end{pgfscope}%
\begin{pgfscope}%
\pgfpathrectangle{\pgfqpoint{0.647939in}{0.492442in}}{\pgfqpoint{4.273799in}{2.331163in}}%
\pgfusepath{clip}%
\pgfsetroundcap%
\pgfsetroundjoin%
\pgfsetlinewidth{0.301125pt}%
\definecolor{currentstroke}{rgb}{0.500000,0.500000,0.500000}%
\pgfsetstrokecolor{currentstroke}%
\pgfsetstrokeopacity{0.300000}%
\pgfsetdash{}{0pt}%
\pgfpathmoveto{\pgfqpoint{3.099116in}{1.256156in}}%
\pgfusepath{stroke}%
\end{pgfscope}%
\begin{pgfscope}%
\pgfpathrectangle{\pgfqpoint{0.647939in}{0.492442in}}{\pgfqpoint{4.273799in}{2.331163in}}%
\pgfusepath{clip}%
\pgfsetroundcap%
\pgfsetroundjoin%
\definecolor{currentfill}{rgb}{0.500000,0.500000,0.500000}%
\pgfsetfillcolor{currentfill}%
\pgfsetfillopacity{0.300000}%
\pgfsetlinewidth{0.301125pt}%
\definecolor{currentstroke}{rgb}{0.500000,0.500000,0.500000}%
\pgfsetstrokecolor{currentstroke}%
\pgfsetstrokeopacity{0.300000}%
\pgfsetdash{}{0pt}%
\pgfpathmoveto{\pgfqpoint{0.000000in}{0.000000in}}%
\pgfpathlineto{\pgfqpoint{0.000000in}{0.000000in}}%
\pgfpathclose%
\pgfusepath{stroke,fill}%
\end{pgfscope}%
\begin{pgfscope}%
\pgfpathrectangle{\pgfqpoint{0.647939in}{0.492442in}}{\pgfqpoint{4.273799in}{2.331163in}}%
\pgfusepath{clip}%
\pgfsetroundcap%
\pgfsetroundjoin%
\pgfsetlinewidth{0.301125pt}%
\definecolor{currentstroke}{rgb}{0.500000,0.500000,0.500000}%
\pgfsetstrokecolor{currentstroke}%
\pgfsetstrokeopacity{0.300000}%
\pgfsetdash{}{0pt}%
\pgfpathmoveto{\pgfqpoint{3.586543in}{1.046484in}}%
\pgfusepath{stroke}%
\end{pgfscope}%
\begin{pgfscope}%
\pgfpathrectangle{\pgfqpoint{0.647939in}{0.492442in}}{\pgfqpoint{4.273799in}{2.331163in}}%
\pgfusepath{clip}%
\pgfsetroundcap%
\pgfsetroundjoin%
\definecolor{currentfill}{rgb}{0.500000,0.500000,0.500000}%
\pgfsetfillcolor{currentfill}%
\pgfsetfillopacity{0.300000}%
\pgfsetlinewidth{0.301125pt}%
\definecolor{currentstroke}{rgb}{0.500000,0.500000,0.500000}%
\pgfsetstrokecolor{currentstroke}%
\pgfsetstrokeopacity{0.300000}%
\pgfsetdash{}{0pt}%
\pgfpathmoveto{\pgfqpoint{0.000000in}{0.000000in}}%
\pgfpathlineto{\pgfqpoint{0.000000in}{0.000000in}}%
\pgfpathclose%
\pgfusepath{stroke,fill}%
\end{pgfscope}%
\begin{pgfscope}%
\pgfpathrectangle{\pgfqpoint{0.647939in}{0.492442in}}{\pgfqpoint{4.273799in}{2.331163in}}%
\pgfusepath{clip}%
\pgfsetroundcap%
\pgfsetroundjoin%
\pgfsetlinewidth{0.301125pt}%
\definecolor{currentstroke}{rgb}{0.500000,0.500000,0.500000}%
\pgfsetstrokecolor{currentstroke}%
\pgfsetstrokeopacity{0.300000}%
\pgfsetdash{}{0pt}%
\pgfpathmoveto{\pgfqpoint{3.923170in}{0.948930in}}%
\pgfusepath{stroke}%
\end{pgfscope}%
\begin{pgfscope}%
\pgfpathrectangle{\pgfqpoint{0.647939in}{0.492442in}}{\pgfqpoint{4.273799in}{2.331163in}}%
\pgfusepath{clip}%
\pgfsetroundcap%
\pgfsetroundjoin%
\definecolor{currentfill}{rgb}{0.500000,0.500000,0.500000}%
\pgfsetfillcolor{currentfill}%
\pgfsetfillopacity{0.300000}%
\pgfsetlinewidth{0.301125pt}%
\definecolor{currentstroke}{rgb}{0.500000,0.500000,0.500000}%
\pgfsetstrokecolor{currentstroke}%
\pgfsetstrokeopacity{0.300000}%
\pgfsetdash{}{0pt}%
\pgfpathmoveto{\pgfqpoint{0.000000in}{0.000000in}}%
\pgfpathlineto{\pgfqpoint{0.000000in}{0.000000in}}%
\pgfpathclose%
\pgfusepath{stroke,fill}%
\end{pgfscope}%
\begin{pgfscope}%
\pgfpathrectangle{\pgfqpoint{0.647939in}{0.492442in}}{\pgfqpoint{4.273799in}{2.331163in}}%
\pgfusepath{clip}%
\pgfsetroundcap%
\pgfsetroundjoin%
\pgfsetlinewidth{0.301125pt}%
\definecolor{currentstroke}{rgb}{0.500000,0.500000,0.500000}%
\pgfsetstrokecolor{currentstroke}%
\pgfsetstrokeopacity{0.300000}%
\pgfsetdash{}{0pt}%
\pgfpathmoveto{\pgfqpoint{3.546110in}{1.259637in}}%
\pgfusepath{stroke}%
\end{pgfscope}%
\begin{pgfscope}%
\pgfpathrectangle{\pgfqpoint{0.647939in}{0.492442in}}{\pgfqpoint{4.273799in}{2.331163in}}%
\pgfusepath{clip}%
\pgfsetroundcap%
\pgfsetroundjoin%
\definecolor{currentfill}{rgb}{0.500000,0.500000,0.500000}%
\pgfsetfillcolor{currentfill}%
\pgfsetfillopacity{0.300000}%
\pgfsetlinewidth{0.301125pt}%
\definecolor{currentstroke}{rgb}{0.500000,0.500000,0.500000}%
\pgfsetstrokecolor{currentstroke}%
\pgfsetstrokeopacity{0.300000}%
\pgfsetdash{}{0pt}%
\pgfpathmoveto{\pgfqpoint{0.000000in}{0.000000in}}%
\pgfpathlineto{\pgfqpoint{0.000000in}{0.000000in}}%
\pgfpathclose%
\pgfusepath{stroke,fill}%
\end{pgfscope}%
\begin{pgfscope}%
\pgfpathrectangle{\pgfqpoint{0.647939in}{0.492442in}}{\pgfqpoint{4.273799in}{2.331163in}}%
\pgfusepath{clip}%
\pgfsetroundcap%
\pgfsetroundjoin%
\pgfsetlinewidth{0.301125pt}%
\definecolor{currentstroke}{rgb}{0.500000,0.500000,0.500000}%
\pgfsetstrokecolor{currentstroke}%
\pgfsetstrokeopacity{0.300000}%
\pgfsetdash{}{0pt}%
\pgfpathmoveto{\pgfqpoint{3.959353in}{1.191250in}}%
\pgfusepath{stroke}%
\end{pgfscope}%
\begin{pgfscope}%
\pgfpathrectangle{\pgfqpoint{0.647939in}{0.492442in}}{\pgfqpoint{4.273799in}{2.331163in}}%
\pgfusepath{clip}%
\pgfsetroundcap%
\pgfsetroundjoin%
\definecolor{currentfill}{rgb}{0.500000,0.500000,0.500000}%
\pgfsetfillcolor{currentfill}%
\pgfsetfillopacity{0.300000}%
\pgfsetlinewidth{0.301125pt}%
\definecolor{currentstroke}{rgb}{0.500000,0.500000,0.500000}%
\pgfsetstrokecolor{currentstroke}%
\pgfsetstrokeopacity{0.300000}%
\pgfsetdash{}{0pt}%
\pgfpathmoveto{\pgfqpoint{0.000000in}{0.000000in}}%
\pgfpathlineto{\pgfqpoint{0.000000in}{0.000000in}}%
\pgfpathclose%
\pgfusepath{stroke,fill}%
\end{pgfscope}%
\begin{pgfscope}%
\pgfpathrectangle{\pgfqpoint{0.647939in}{0.492442in}}{\pgfqpoint{4.273799in}{2.331163in}}%
\pgfusepath{clip}%
\pgfsetroundcap%
\pgfsetroundjoin%
\pgfsetlinewidth{0.301125pt}%
\definecolor{currentstroke}{rgb}{0.500000,0.500000,0.500000}%
\pgfsetstrokecolor{currentstroke}%
\pgfsetstrokeopacity{0.300000}%
\pgfsetdash{}{0pt}%
\pgfpathmoveto{\pgfqpoint{3.922369in}{1.374371in}}%
\pgfusepath{stroke}%
\end{pgfscope}%
\begin{pgfscope}%
\pgfpathrectangle{\pgfqpoint{0.647939in}{0.492442in}}{\pgfqpoint{4.273799in}{2.331163in}}%
\pgfusepath{clip}%
\pgfsetroundcap%
\pgfsetroundjoin%
\definecolor{currentfill}{rgb}{0.500000,0.500000,0.500000}%
\pgfsetfillcolor{currentfill}%
\pgfsetfillopacity{0.300000}%
\pgfsetlinewidth{0.301125pt}%
\definecolor{currentstroke}{rgb}{0.500000,0.500000,0.500000}%
\pgfsetstrokecolor{currentstroke}%
\pgfsetstrokeopacity{0.300000}%
\pgfsetdash{}{0pt}%
\pgfpathmoveto{\pgfqpoint{0.000000in}{0.000000in}}%
\pgfpathlineto{\pgfqpoint{0.000000in}{0.000000in}}%
\pgfpathclose%
\pgfusepath{stroke,fill}%
\end{pgfscope}%
\begin{pgfscope}%
\pgfpathrectangle{\pgfqpoint{0.647939in}{0.492442in}}{\pgfqpoint{4.273799in}{2.331163in}}%
\pgfusepath{clip}%
\pgfsetroundcap%
\pgfsetroundjoin%
\pgfsetlinewidth{0.301125pt}%
\definecolor{currentstroke}{rgb}{0.500000,0.500000,0.500000}%
\pgfsetstrokecolor{currentstroke}%
\pgfsetstrokeopacity{0.300000}%
\pgfsetdash{}{0pt}%
\pgfpathmoveto{\pgfqpoint{4.163528in}{1.478377in}}%
\pgfusepath{stroke}%
\end{pgfscope}%
\begin{pgfscope}%
\pgfpathrectangle{\pgfqpoint{0.647939in}{0.492442in}}{\pgfqpoint{4.273799in}{2.331163in}}%
\pgfusepath{clip}%
\pgfsetroundcap%
\pgfsetroundjoin%
\definecolor{currentfill}{rgb}{0.500000,0.500000,0.500000}%
\pgfsetfillcolor{currentfill}%
\pgfsetfillopacity{0.300000}%
\pgfsetlinewidth{0.301125pt}%
\definecolor{currentstroke}{rgb}{0.500000,0.500000,0.500000}%
\pgfsetstrokecolor{currentstroke}%
\pgfsetstrokeopacity{0.300000}%
\pgfsetdash{}{0pt}%
\pgfpathmoveto{\pgfqpoint{0.000000in}{0.000000in}}%
\pgfpathlineto{\pgfqpoint{0.000000in}{0.000000in}}%
\pgfpathclose%
\pgfusepath{stroke,fill}%
\end{pgfscope}%
\begin{pgfscope}%
\pgfpathrectangle{\pgfqpoint{0.647939in}{0.492442in}}{\pgfqpoint{4.273799in}{2.331163in}}%
\pgfusepath{clip}%
\pgfsetroundcap%
\pgfsetroundjoin%
\pgfsetlinewidth{0.301125pt}%
\definecolor{currentstroke}{rgb}{0.500000,0.500000,0.500000}%
\pgfsetstrokecolor{currentstroke}%
\pgfsetstrokeopacity{0.300000}%
\pgfsetdash{}{0pt}%
\pgfpathmoveto{\pgfqpoint{4.380776in}{1.618199in}}%
\pgfusepath{stroke}%
\end{pgfscope}%
\begin{pgfscope}%
\pgfpathrectangle{\pgfqpoint{0.647939in}{0.492442in}}{\pgfqpoint{4.273799in}{2.331163in}}%
\pgfusepath{clip}%
\pgfsetroundcap%
\pgfsetroundjoin%
\definecolor{currentfill}{rgb}{0.500000,0.500000,0.500000}%
\pgfsetfillcolor{currentfill}%
\pgfsetfillopacity{0.300000}%
\pgfsetlinewidth{0.301125pt}%
\definecolor{currentstroke}{rgb}{0.500000,0.500000,0.500000}%
\pgfsetstrokecolor{currentstroke}%
\pgfsetstrokeopacity{0.300000}%
\pgfsetdash{}{0pt}%
\pgfpathmoveto{\pgfqpoint{0.000000in}{0.000000in}}%
\pgfpathlineto{\pgfqpoint{0.000000in}{0.000000in}}%
\pgfpathclose%
\pgfusepath{stroke,fill}%
\end{pgfscope}%
\begin{pgfscope}%
\pgfpathrectangle{\pgfqpoint{0.647939in}{0.492442in}}{\pgfqpoint{4.273799in}{2.331163in}}%
\pgfusepath{clip}%
\pgfsetroundcap%
\pgfsetroundjoin%
\pgfsetlinewidth{0.301125pt}%
\definecolor{currentstroke}{rgb}{0.500000,0.500000,0.500000}%
\pgfsetstrokecolor{currentstroke}%
\pgfsetstrokeopacity{0.300000}%
\pgfsetdash{}{0pt}%
\pgfpathmoveto{\pgfqpoint{4.534515in}{1.771418in}}%
\pgfusepath{stroke}%
\end{pgfscope}%
\begin{pgfscope}%
\pgfpathrectangle{\pgfqpoint{0.647939in}{0.492442in}}{\pgfqpoint{4.273799in}{2.331163in}}%
\pgfusepath{clip}%
\pgfsetroundcap%
\pgfsetroundjoin%
\definecolor{currentfill}{rgb}{0.500000,0.500000,0.500000}%
\pgfsetfillcolor{currentfill}%
\pgfsetfillopacity{0.300000}%
\pgfsetlinewidth{0.301125pt}%
\definecolor{currentstroke}{rgb}{0.500000,0.500000,0.500000}%
\pgfsetstrokecolor{currentstroke}%
\pgfsetstrokeopacity{0.300000}%
\pgfsetdash{}{0pt}%
\pgfpathmoveto{\pgfqpoint{0.000000in}{0.000000in}}%
\pgfpathlineto{\pgfqpoint{0.000000in}{0.000000in}}%
\pgfpathclose%
\pgfusepath{stroke,fill}%
\end{pgfscope}%
\begin{pgfscope}%
\pgfpathrectangle{\pgfqpoint{0.647939in}{0.492442in}}{\pgfqpoint{4.273799in}{2.331163in}}%
\pgfusepath{clip}%
\pgfsetroundcap%
\pgfsetroundjoin%
\pgfsetlinewidth{0.301125pt}%
\definecolor{currentstroke}{rgb}{0.500000,0.500000,0.500000}%
\pgfsetstrokecolor{currentstroke}%
\pgfsetstrokeopacity{0.300000}%
\pgfsetdash{}{0pt}%
\pgfpathmoveto{\pgfqpoint{4.710601in}{1.737554in}}%
\pgfusepath{stroke}%
\end{pgfscope}%
\begin{pgfscope}%
\pgfpathrectangle{\pgfqpoint{0.647939in}{0.492442in}}{\pgfqpoint{4.273799in}{2.331163in}}%
\pgfusepath{clip}%
\pgfsetroundcap%
\pgfsetroundjoin%
\definecolor{currentfill}{rgb}{0.500000,0.500000,0.500000}%
\pgfsetfillcolor{currentfill}%
\pgfsetfillopacity{0.300000}%
\pgfsetlinewidth{0.301125pt}%
\definecolor{currentstroke}{rgb}{0.500000,0.500000,0.500000}%
\pgfsetstrokecolor{currentstroke}%
\pgfsetstrokeopacity{0.300000}%
\pgfsetdash{}{0pt}%
\pgfpathmoveto{\pgfqpoint{0.000000in}{0.000000in}}%
\pgfpathlineto{\pgfqpoint{0.000000in}{0.000000in}}%
\pgfpathclose%
\pgfusepath{stroke,fill}%
\end{pgfscope}%
\begin{pgfscope}%
\pgfpathrectangle{\pgfqpoint{0.647939in}{0.492442in}}{\pgfqpoint{4.273799in}{2.331163in}}%
\pgfusepath{clip}%
\pgfsetroundcap%
\pgfsetroundjoin%
\pgfsetlinewidth{0.301125pt}%
\definecolor{currentstroke}{rgb}{0.500000,0.500000,0.500000}%
\pgfsetstrokecolor{currentstroke}%
\pgfsetstrokeopacity{0.300000}%
\pgfsetdash{}{0pt}%
\pgfpathmoveto{\pgfqpoint{4.816495in}{1.905860in}}%
\pgfusepath{stroke}%
\end{pgfscope}%
\begin{pgfscope}%
\pgfpathrectangle{\pgfqpoint{0.647939in}{0.492442in}}{\pgfqpoint{4.273799in}{2.331163in}}%
\pgfusepath{clip}%
\pgfsetroundcap%
\pgfsetroundjoin%
\definecolor{currentfill}{rgb}{0.500000,0.500000,0.500000}%
\pgfsetfillcolor{currentfill}%
\pgfsetfillopacity{0.300000}%
\pgfsetlinewidth{0.301125pt}%
\definecolor{currentstroke}{rgb}{0.500000,0.500000,0.500000}%
\pgfsetstrokecolor{currentstroke}%
\pgfsetstrokeopacity{0.300000}%
\pgfsetdash{}{0pt}%
\pgfpathmoveto{\pgfqpoint{0.000000in}{0.000000in}}%
\pgfpathlineto{\pgfqpoint{0.000000in}{0.000000in}}%
\pgfpathclose%
\pgfusepath{stroke,fill}%
\end{pgfscope}%
\begin{pgfscope}%
\pgfpathrectangle{\pgfqpoint{0.647939in}{0.492442in}}{\pgfqpoint{4.273799in}{2.331163in}}%
\pgfusepath{clip}%
\pgfsetroundcap%
\pgfsetroundjoin%
\pgfsetlinewidth{0.301125pt}%
\definecolor{currentstroke}{rgb}{0.500000,0.500000,0.500000}%
\pgfsetstrokecolor{currentstroke}%
\pgfsetstrokeopacity{0.300000}%
\pgfsetdash{}{0pt}%
\pgfpathmoveto{\pgfqpoint{4.879232in}{1.965717in}}%
\pgfusepath{stroke}%
\end{pgfscope}%
\begin{pgfscope}%
\pgfpathrectangle{\pgfqpoint{0.647939in}{0.492442in}}{\pgfqpoint{4.273799in}{2.331163in}}%
\pgfusepath{clip}%
\pgfsetroundcap%
\pgfsetroundjoin%
\definecolor{currentfill}{rgb}{0.500000,0.500000,0.500000}%
\pgfsetfillcolor{currentfill}%
\pgfsetfillopacity{0.300000}%
\pgfsetlinewidth{0.301125pt}%
\definecolor{currentstroke}{rgb}{0.500000,0.500000,0.500000}%
\pgfsetstrokecolor{currentstroke}%
\pgfsetstrokeopacity{0.300000}%
\pgfsetdash{}{0pt}%
\pgfpathmoveto{\pgfqpoint{0.000000in}{0.000000in}}%
\pgfpathlineto{\pgfqpoint{0.000000in}{0.000000in}}%
\pgfpathclose%
\pgfusepath{stroke,fill}%
\end{pgfscope}%
\begin{pgfscope}%
\pgfpathrectangle{\pgfqpoint{0.647939in}{0.492442in}}{\pgfqpoint{4.273799in}{2.331163in}}%
\pgfusepath{clip}%
\pgfsetroundcap%
\pgfsetroundjoin%
\pgfsetlinewidth{0.301125pt}%
\definecolor{currentstroke}{rgb}{0.500000,0.500000,0.500000}%
\pgfsetstrokecolor{currentstroke}%
\pgfsetstrokeopacity{0.300000}%
\pgfsetdash{}{0pt}%
\pgfpathmoveto{\pgfqpoint{4.510718in}{2.664351in}}%
\pgfusepath{stroke}%
\end{pgfscope}%
\begin{pgfscope}%
\pgfpathrectangle{\pgfqpoint{0.647939in}{0.492442in}}{\pgfqpoint{4.273799in}{2.331163in}}%
\pgfusepath{clip}%
\pgfsetroundcap%
\pgfsetroundjoin%
\definecolor{currentfill}{rgb}{0.500000,0.500000,0.500000}%
\pgfsetfillcolor{currentfill}%
\pgfsetfillopacity{0.300000}%
\pgfsetlinewidth{0.301125pt}%
\definecolor{currentstroke}{rgb}{0.500000,0.500000,0.500000}%
\pgfsetstrokecolor{currentstroke}%
\pgfsetstrokeopacity{0.300000}%
\pgfsetdash{}{0pt}%
\pgfpathmoveto{\pgfqpoint{0.000000in}{0.000000in}}%
\pgfpathlineto{\pgfqpoint{0.000000in}{0.000000in}}%
\pgfpathclose%
\pgfusepath{stroke,fill}%
\end{pgfscope}%
\begin{pgfscope}%
\pgfpathrectangle{\pgfqpoint{0.647939in}{0.492442in}}{\pgfqpoint{4.273799in}{2.331163in}}%
\pgfusepath{clip}%
\pgfsetroundcap%
\pgfsetroundjoin%
\pgfsetlinewidth{0.301125pt}%
\definecolor{currentstroke}{rgb}{0.500000,0.500000,0.500000}%
\pgfsetstrokecolor{currentstroke}%
\pgfsetstrokeopacity{0.300000}%
\pgfsetdash{}{0pt}%
\pgfpathmoveto{\pgfqpoint{4.390500in}{2.581673in}}%
\pgfusepath{stroke}%
\end{pgfscope}%
\begin{pgfscope}%
\pgfpathrectangle{\pgfqpoint{0.647939in}{0.492442in}}{\pgfqpoint{4.273799in}{2.331163in}}%
\pgfusepath{clip}%
\pgfsetroundcap%
\pgfsetroundjoin%
\definecolor{currentfill}{rgb}{0.500000,0.500000,0.500000}%
\pgfsetfillcolor{currentfill}%
\pgfsetfillopacity{0.300000}%
\pgfsetlinewidth{0.301125pt}%
\definecolor{currentstroke}{rgb}{0.500000,0.500000,0.500000}%
\pgfsetstrokecolor{currentstroke}%
\pgfsetstrokeopacity{0.300000}%
\pgfsetdash{}{0pt}%
\pgfpathmoveto{\pgfqpoint{0.000000in}{0.000000in}}%
\pgfpathlineto{\pgfqpoint{0.000000in}{0.000000in}}%
\pgfpathclose%
\pgfusepath{stroke,fill}%
\end{pgfscope}%
\begin{pgfscope}%
\pgfpathrectangle{\pgfqpoint{0.647939in}{0.492442in}}{\pgfqpoint{4.273799in}{2.331163in}}%
\pgfusepath{clip}%
\pgfsetroundcap%
\pgfsetroundjoin%
\pgfsetlinewidth{0.301125pt}%
\definecolor{currentstroke}{rgb}{0.500000,0.500000,0.500000}%
\pgfsetstrokecolor{currentstroke}%
\pgfsetstrokeopacity{0.300000}%
\pgfsetdash{}{0pt}%
\pgfpathmoveto{\pgfqpoint{4.296425in}{2.520249in}}%
\pgfusepath{stroke}%
\end{pgfscope}%
\begin{pgfscope}%
\pgfpathrectangle{\pgfqpoint{0.647939in}{0.492442in}}{\pgfqpoint{4.273799in}{2.331163in}}%
\pgfusepath{clip}%
\pgfsetroundcap%
\pgfsetroundjoin%
\definecolor{currentfill}{rgb}{0.500000,0.500000,0.500000}%
\pgfsetfillcolor{currentfill}%
\pgfsetfillopacity{0.300000}%
\pgfsetlinewidth{0.301125pt}%
\definecolor{currentstroke}{rgb}{0.500000,0.500000,0.500000}%
\pgfsetstrokecolor{currentstroke}%
\pgfsetstrokeopacity{0.300000}%
\pgfsetdash{}{0pt}%
\pgfpathmoveto{\pgfqpoint{0.000000in}{0.000000in}}%
\pgfpathlineto{\pgfqpoint{0.000000in}{0.000000in}}%
\pgfpathclose%
\pgfusepath{stroke,fill}%
\end{pgfscope}%
\begin{pgfscope}%
\pgfpathrectangle{\pgfqpoint{0.647939in}{0.492442in}}{\pgfqpoint{4.273799in}{2.331163in}}%
\pgfusepath{clip}%
\pgfsetroundcap%
\pgfsetroundjoin%
\pgfsetlinewidth{0.301125pt}%
\definecolor{currentstroke}{rgb}{0.500000,0.500000,0.500000}%
\pgfsetstrokecolor{currentstroke}%
\pgfsetstrokeopacity{0.300000}%
\pgfsetdash{}{0pt}%
\pgfpathmoveto{\pgfqpoint{4.205445in}{2.463252in}}%
\pgfusepath{stroke}%
\end{pgfscope}%
\begin{pgfscope}%
\pgfpathrectangle{\pgfqpoint{0.647939in}{0.492442in}}{\pgfqpoint{4.273799in}{2.331163in}}%
\pgfusepath{clip}%
\pgfsetroundcap%
\pgfsetroundjoin%
\definecolor{currentfill}{rgb}{0.500000,0.500000,0.500000}%
\pgfsetfillcolor{currentfill}%
\pgfsetfillopacity{0.300000}%
\pgfsetlinewidth{0.301125pt}%
\definecolor{currentstroke}{rgb}{0.500000,0.500000,0.500000}%
\pgfsetstrokecolor{currentstroke}%
\pgfsetstrokeopacity{0.300000}%
\pgfsetdash{}{0pt}%
\pgfpathmoveto{\pgfqpoint{0.000000in}{0.000000in}}%
\pgfpathlineto{\pgfqpoint{0.000000in}{0.000000in}}%
\pgfpathclose%
\pgfusepath{stroke,fill}%
\end{pgfscope}%
\begin{pgfscope}%
\pgfpathrectangle{\pgfqpoint{0.647939in}{0.492442in}}{\pgfqpoint{4.273799in}{2.331163in}}%
\pgfusepath{clip}%
\pgfsetroundcap%
\pgfsetroundjoin%
\pgfsetlinewidth{0.301125pt}%
\definecolor{currentstroke}{rgb}{0.500000,0.500000,0.500000}%
\pgfsetstrokecolor{currentstroke}%
\pgfsetstrokeopacity{0.300000}%
\pgfsetdash{}{0pt}%
\pgfpathmoveto{\pgfqpoint{4.148852in}{2.359849in}}%
\pgfusepath{stroke}%
\end{pgfscope}%
\begin{pgfscope}%
\pgfpathrectangle{\pgfqpoint{0.647939in}{0.492442in}}{\pgfqpoint{4.273799in}{2.331163in}}%
\pgfusepath{clip}%
\pgfsetroundcap%
\pgfsetroundjoin%
\definecolor{currentfill}{rgb}{0.500000,0.500000,0.500000}%
\pgfsetfillcolor{currentfill}%
\pgfsetfillopacity{0.300000}%
\pgfsetlinewidth{0.301125pt}%
\definecolor{currentstroke}{rgb}{0.500000,0.500000,0.500000}%
\pgfsetstrokecolor{currentstroke}%
\pgfsetstrokeopacity{0.300000}%
\pgfsetdash{}{0pt}%
\pgfpathmoveto{\pgfqpoint{0.000000in}{0.000000in}}%
\pgfpathlineto{\pgfqpoint{0.000000in}{0.000000in}}%
\pgfpathclose%
\pgfusepath{stroke,fill}%
\end{pgfscope}%
\begin{pgfscope}%
\pgfpathrectangle{\pgfqpoint{0.647939in}{0.492442in}}{\pgfqpoint{4.273799in}{2.331163in}}%
\pgfusepath{clip}%
\pgfsetroundcap%
\pgfsetroundjoin%
\pgfsetlinewidth{0.301125pt}%
\definecolor{currentstroke}{rgb}{0.500000,0.500000,0.500000}%
\pgfsetstrokecolor{currentstroke}%
\pgfsetstrokeopacity{0.300000}%
\pgfsetdash{}{0pt}%
\pgfpathmoveto{\pgfqpoint{4.033132in}{2.355674in}}%
\pgfusepath{stroke}%
\end{pgfscope}%
\begin{pgfscope}%
\pgfpathrectangle{\pgfqpoint{0.647939in}{0.492442in}}{\pgfqpoint{4.273799in}{2.331163in}}%
\pgfusepath{clip}%
\pgfsetroundcap%
\pgfsetroundjoin%
\definecolor{currentfill}{rgb}{0.500000,0.500000,0.500000}%
\pgfsetfillcolor{currentfill}%
\pgfsetfillopacity{0.300000}%
\pgfsetlinewidth{0.301125pt}%
\definecolor{currentstroke}{rgb}{0.500000,0.500000,0.500000}%
\pgfsetstrokecolor{currentstroke}%
\pgfsetstrokeopacity{0.300000}%
\pgfsetdash{}{0pt}%
\pgfpathmoveto{\pgfqpoint{0.000000in}{0.000000in}}%
\pgfpathlineto{\pgfqpoint{0.000000in}{0.000000in}}%
\pgfpathclose%
\pgfusepath{stroke,fill}%
\end{pgfscope}%
\begin{pgfscope}%
\pgfpathrectangle{\pgfqpoint{0.647939in}{0.492442in}}{\pgfqpoint{4.273799in}{2.331163in}}%
\pgfusepath{clip}%
\pgfsetroundcap%
\pgfsetroundjoin%
\pgfsetlinewidth{0.301125pt}%
\definecolor{currentstroke}{rgb}{0.500000,0.500000,0.500000}%
\pgfsetstrokecolor{currentstroke}%
\pgfsetstrokeopacity{0.300000}%
\pgfsetdash{}{0pt}%
\pgfpathmoveto{\pgfqpoint{3.916658in}{2.379207in}}%
\pgfusepath{stroke}%
\end{pgfscope}%
\begin{pgfscope}%
\pgfpathrectangle{\pgfqpoint{0.647939in}{0.492442in}}{\pgfqpoint{4.273799in}{2.331163in}}%
\pgfusepath{clip}%
\pgfsetroundcap%
\pgfsetroundjoin%
\definecolor{currentfill}{rgb}{0.500000,0.500000,0.500000}%
\pgfsetfillcolor{currentfill}%
\pgfsetfillopacity{0.300000}%
\pgfsetlinewidth{0.301125pt}%
\definecolor{currentstroke}{rgb}{0.500000,0.500000,0.500000}%
\pgfsetstrokecolor{currentstroke}%
\pgfsetstrokeopacity{0.300000}%
\pgfsetdash{}{0pt}%
\pgfpathmoveto{\pgfqpoint{0.000000in}{0.000000in}}%
\pgfpathlineto{\pgfqpoint{0.000000in}{0.000000in}}%
\pgfpathclose%
\pgfusepath{stroke,fill}%
\end{pgfscope}%
\begin{pgfscope}%
\pgfpathrectangle{\pgfqpoint{0.647939in}{0.492442in}}{\pgfqpoint{4.273799in}{2.331163in}}%
\pgfusepath{clip}%
\pgfsetroundcap%
\pgfsetroundjoin%
\pgfsetlinewidth{0.301125pt}%
\definecolor{currentstroke}{rgb}{0.500000,0.500000,0.500000}%
\pgfsetstrokecolor{currentstroke}%
\pgfsetstrokeopacity{0.300000}%
\pgfsetdash{}{0pt}%
\pgfpathmoveto{\pgfqpoint{3.863479in}{2.278496in}}%
\pgfusepath{stroke}%
\end{pgfscope}%
\begin{pgfscope}%
\pgfpathrectangle{\pgfqpoint{0.647939in}{0.492442in}}{\pgfqpoint{4.273799in}{2.331163in}}%
\pgfusepath{clip}%
\pgfsetroundcap%
\pgfsetroundjoin%
\definecolor{currentfill}{rgb}{0.500000,0.500000,0.500000}%
\pgfsetfillcolor{currentfill}%
\pgfsetfillopacity{0.300000}%
\pgfsetlinewidth{0.301125pt}%
\definecolor{currentstroke}{rgb}{0.500000,0.500000,0.500000}%
\pgfsetstrokecolor{currentstroke}%
\pgfsetstrokeopacity{0.300000}%
\pgfsetdash{}{0pt}%
\pgfpathmoveto{\pgfqpoint{0.000000in}{0.000000in}}%
\pgfpathlineto{\pgfqpoint{0.000000in}{0.000000in}}%
\pgfpathclose%
\pgfusepath{stroke,fill}%
\end{pgfscope}%
\begin{pgfscope}%
\pgfpathrectangle{\pgfqpoint{0.647939in}{0.492442in}}{\pgfqpoint{4.273799in}{2.331163in}}%
\pgfusepath{clip}%
\pgfsetroundcap%
\pgfsetroundjoin%
\pgfsetlinewidth{0.301125pt}%
\definecolor{currentstroke}{rgb}{0.500000,0.500000,0.500000}%
\pgfsetstrokecolor{currentstroke}%
\pgfsetstrokeopacity{0.300000}%
\pgfsetdash{}{0pt}%
\pgfpathmoveto{\pgfqpoint{3.636553in}{2.555269in}}%
\pgfusepath{stroke}%
\end{pgfscope}%
\begin{pgfscope}%
\pgfpathrectangle{\pgfqpoint{0.647939in}{0.492442in}}{\pgfqpoint{4.273799in}{2.331163in}}%
\pgfusepath{clip}%
\pgfsetroundcap%
\pgfsetroundjoin%
\definecolor{currentfill}{rgb}{0.500000,0.500000,0.500000}%
\pgfsetfillcolor{currentfill}%
\pgfsetfillopacity{0.300000}%
\pgfsetlinewidth{0.301125pt}%
\definecolor{currentstroke}{rgb}{0.500000,0.500000,0.500000}%
\pgfsetstrokecolor{currentstroke}%
\pgfsetstrokeopacity{0.300000}%
\pgfsetdash{}{0pt}%
\pgfpathmoveto{\pgfqpoint{0.000000in}{0.000000in}}%
\pgfpathlineto{\pgfqpoint{0.000000in}{0.000000in}}%
\pgfpathclose%
\pgfusepath{stroke,fill}%
\end{pgfscope}%
\begin{pgfscope}%
\pgfpathrectangle{\pgfqpoint{0.647939in}{0.492442in}}{\pgfqpoint{4.273799in}{2.331163in}}%
\pgfusepath{clip}%
\pgfsetroundcap%
\pgfsetroundjoin%
\pgfsetlinewidth{0.301125pt}%
\definecolor{currentstroke}{rgb}{0.500000,0.500000,0.500000}%
\pgfsetstrokecolor{currentstroke}%
\pgfsetstrokeopacity{0.300000}%
\pgfsetdash{}{0pt}%
\pgfpathmoveto{\pgfqpoint{3.485513in}{2.655374in}}%
\pgfusepath{stroke}%
\end{pgfscope}%
\begin{pgfscope}%
\pgfpathrectangle{\pgfqpoint{0.647939in}{0.492442in}}{\pgfqpoint{4.273799in}{2.331163in}}%
\pgfusepath{clip}%
\pgfsetroundcap%
\pgfsetroundjoin%
\definecolor{currentfill}{rgb}{0.500000,0.500000,0.500000}%
\pgfsetfillcolor{currentfill}%
\pgfsetfillopacity{0.300000}%
\pgfsetlinewidth{0.301125pt}%
\definecolor{currentstroke}{rgb}{0.500000,0.500000,0.500000}%
\pgfsetstrokecolor{currentstroke}%
\pgfsetstrokeopacity{0.300000}%
\pgfsetdash{}{0pt}%
\pgfpathmoveto{\pgfqpoint{0.000000in}{0.000000in}}%
\pgfpathlineto{\pgfqpoint{0.000000in}{0.000000in}}%
\pgfpathclose%
\pgfusepath{stroke,fill}%
\end{pgfscope}%
\begin{pgfscope}%
\pgfpathrectangle{\pgfqpoint{0.647939in}{0.492442in}}{\pgfqpoint{4.273799in}{2.331163in}}%
\pgfusepath{clip}%
\pgfsetroundcap%
\pgfsetroundjoin%
\pgfsetlinewidth{0.301125pt}%
\definecolor{currentstroke}{rgb}{0.500000,0.500000,0.500000}%
\pgfsetstrokecolor{currentstroke}%
\pgfsetstrokeopacity{0.300000}%
\pgfsetdash{}{0pt}%
\pgfpathmoveto{\pgfqpoint{3.653915in}{2.131565in}}%
\pgfusepath{stroke}%
\end{pgfscope}%
\begin{pgfscope}%
\pgfpathrectangle{\pgfqpoint{0.647939in}{0.492442in}}{\pgfqpoint{4.273799in}{2.331163in}}%
\pgfusepath{clip}%
\pgfsetroundcap%
\pgfsetroundjoin%
\definecolor{currentfill}{rgb}{0.500000,0.500000,0.500000}%
\pgfsetfillcolor{currentfill}%
\pgfsetfillopacity{0.300000}%
\pgfsetlinewidth{0.301125pt}%
\definecolor{currentstroke}{rgb}{0.500000,0.500000,0.500000}%
\pgfsetstrokecolor{currentstroke}%
\pgfsetstrokeopacity{0.300000}%
\pgfsetdash{}{0pt}%
\pgfpathmoveto{\pgfqpoint{0.000000in}{0.000000in}}%
\pgfpathlineto{\pgfqpoint{0.000000in}{0.000000in}}%
\pgfpathclose%
\pgfusepath{stroke,fill}%
\end{pgfscope}%
\begin{pgfscope}%
\pgfpathrectangle{\pgfqpoint{0.647939in}{0.492442in}}{\pgfqpoint{4.273799in}{2.331163in}}%
\pgfusepath{clip}%
\pgfsetroundcap%
\pgfsetroundjoin%
\pgfsetlinewidth{0.301125pt}%
\definecolor{currentstroke}{rgb}{0.500000,0.500000,0.500000}%
\pgfsetstrokecolor{currentstroke}%
\pgfsetstrokeopacity{0.300000}%
\pgfsetdash{}{0pt}%
\pgfpathmoveto{\pgfqpoint{3.445770in}{2.319782in}}%
\pgfusepath{stroke}%
\end{pgfscope}%
\begin{pgfscope}%
\pgfpathrectangle{\pgfqpoint{0.647939in}{0.492442in}}{\pgfqpoint{4.273799in}{2.331163in}}%
\pgfusepath{clip}%
\pgfsetroundcap%
\pgfsetroundjoin%
\definecolor{currentfill}{rgb}{0.500000,0.500000,0.500000}%
\pgfsetfillcolor{currentfill}%
\pgfsetfillopacity{0.300000}%
\pgfsetlinewidth{0.301125pt}%
\definecolor{currentstroke}{rgb}{0.500000,0.500000,0.500000}%
\pgfsetstrokecolor{currentstroke}%
\pgfsetstrokeopacity{0.300000}%
\pgfsetdash{}{0pt}%
\pgfpathmoveto{\pgfqpoint{0.000000in}{0.000000in}}%
\pgfpathlineto{\pgfqpoint{0.000000in}{0.000000in}}%
\pgfpathclose%
\pgfusepath{stroke,fill}%
\end{pgfscope}%
\begin{pgfscope}%
\pgfpathrectangle{\pgfqpoint{0.647939in}{0.492442in}}{\pgfqpoint{4.273799in}{2.331163in}}%
\pgfusepath{clip}%
\pgfsetroundcap%
\pgfsetroundjoin%
\pgfsetlinewidth{0.301125pt}%
\definecolor{currentstroke}{rgb}{0.500000,0.500000,0.500000}%
\pgfsetstrokecolor{currentstroke}%
\pgfsetstrokeopacity{0.300000}%
\pgfsetdash{}{0pt}%
\pgfpathmoveto{\pgfqpoint{3.347796in}{2.288128in}}%
\pgfusepath{stroke}%
\end{pgfscope}%
\begin{pgfscope}%
\pgfpathrectangle{\pgfqpoint{0.647939in}{0.492442in}}{\pgfqpoint{4.273799in}{2.331163in}}%
\pgfusepath{clip}%
\pgfsetroundcap%
\pgfsetroundjoin%
\definecolor{currentfill}{rgb}{0.500000,0.500000,0.500000}%
\pgfsetfillcolor{currentfill}%
\pgfsetfillopacity{0.300000}%
\pgfsetlinewidth{0.301125pt}%
\definecolor{currentstroke}{rgb}{0.500000,0.500000,0.500000}%
\pgfsetstrokecolor{currentstroke}%
\pgfsetstrokeopacity{0.300000}%
\pgfsetdash{}{0pt}%
\pgfpathmoveto{\pgfqpoint{0.000000in}{0.000000in}}%
\pgfpathlineto{\pgfqpoint{0.000000in}{0.000000in}}%
\pgfpathclose%
\pgfusepath{stroke,fill}%
\end{pgfscope}%
\begin{pgfscope}%
\pgfpathrectangle{\pgfqpoint{0.647939in}{0.492442in}}{\pgfqpoint{4.273799in}{2.331163in}}%
\pgfusepath{clip}%
\pgfsetroundcap%
\pgfsetroundjoin%
\pgfsetlinewidth{0.301125pt}%
\definecolor{currentstroke}{rgb}{0.500000,0.500000,0.500000}%
\pgfsetstrokecolor{currentstroke}%
\pgfsetstrokeopacity{0.300000}%
\pgfsetdash{}{0pt}%
\pgfpathmoveto{\pgfqpoint{2.949598in}{2.645903in}}%
\pgfusepath{stroke}%
\end{pgfscope}%
\begin{pgfscope}%
\pgfpathrectangle{\pgfqpoint{0.647939in}{0.492442in}}{\pgfqpoint{4.273799in}{2.331163in}}%
\pgfusepath{clip}%
\pgfsetroundcap%
\pgfsetroundjoin%
\definecolor{currentfill}{rgb}{0.500000,0.500000,0.500000}%
\pgfsetfillcolor{currentfill}%
\pgfsetfillopacity{0.300000}%
\pgfsetlinewidth{0.301125pt}%
\definecolor{currentstroke}{rgb}{0.500000,0.500000,0.500000}%
\pgfsetstrokecolor{currentstroke}%
\pgfsetstrokeopacity{0.300000}%
\pgfsetdash{}{0pt}%
\pgfpathmoveto{\pgfqpoint{0.000000in}{0.000000in}}%
\pgfpathlineto{\pgfqpoint{0.000000in}{0.000000in}}%
\pgfpathclose%
\pgfusepath{stroke,fill}%
\end{pgfscope}%
\begin{pgfscope}%
\pgfpathrectangle{\pgfqpoint{0.647939in}{0.492442in}}{\pgfqpoint{4.273799in}{2.331163in}}%
\pgfusepath{clip}%
\pgfsetroundcap%
\pgfsetroundjoin%
\pgfsetlinewidth{0.301125pt}%
\definecolor{currentstroke}{rgb}{0.500000,0.500000,0.500000}%
\pgfsetstrokecolor{currentstroke}%
\pgfsetstrokeopacity{0.300000}%
\pgfsetdash{}{0pt}%
\pgfpathmoveto{\pgfqpoint{3.023306in}{2.483872in}}%
\pgfusepath{stroke}%
\end{pgfscope}%
\begin{pgfscope}%
\pgfpathrectangle{\pgfqpoint{0.647939in}{0.492442in}}{\pgfqpoint{4.273799in}{2.331163in}}%
\pgfusepath{clip}%
\pgfsetroundcap%
\pgfsetroundjoin%
\definecolor{currentfill}{rgb}{0.500000,0.500000,0.500000}%
\pgfsetfillcolor{currentfill}%
\pgfsetfillopacity{0.300000}%
\pgfsetlinewidth{0.301125pt}%
\definecolor{currentstroke}{rgb}{0.500000,0.500000,0.500000}%
\pgfsetstrokecolor{currentstroke}%
\pgfsetstrokeopacity{0.300000}%
\pgfsetdash{}{0pt}%
\pgfpathmoveto{\pgfqpoint{0.000000in}{0.000000in}}%
\pgfpathlineto{\pgfqpoint{0.000000in}{0.000000in}}%
\pgfpathclose%
\pgfusepath{stroke,fill}%
\end{pgfscope}%
\begin{pgfscope}%
\pgfpathrectangle{\pgfqpoint{0.647939in}{0.492442in}}{\pgfqpoint{4.273799in}{2.331163in}}%
\pgfusepath{clip}%
\pgfsetroundcap%
\pgfsetroundjoin%
\pgfsetlinewidth{0.301125pt}%
\definecolor{currentstroke}{rgb}{0.500000,0.500000,0.500000}%
\pgfsetstrokecolor{currentstroke}%
\pgfsetstrokeopacity{0.300000}%
\pgfsetdash{}{0pt}%
\pgfpathmoveto{\pgfqpoint{2.477039in}{2.729734in}}%
\pgfusepath{stroke}%
\end{pgfscope}%
\begin{pgfscope}%
\pgfpathrectangle{\pgfqpoint{0.647939in}{0.492442in}}{\pgfqpoint{4.273799in}{2.331163in}}%
\pgfusepath{clip}%
\pgfsetroundcap%
\pgfsetroundjoin%
\definecolor{currentfill}{rgb}{0.500000,0.500000,0.500000}%
\pgfsetfillcolor{currentfill}%
\pgfsetfillopacity{0.300000}%
\pgfsetlinewidth{0.301125pt}%
\definecolor{currentstroke}{rgb}{0.500000,0.500000,0.500000}%
\pgfsetstrokecolor{currentstroke}%
\pgfsetstrokeopacity{0.300000}%
\pgfsetdash{}{0pt}%
\pgfpathmoveto{\pgfqpoint{0.000000in}{0.000000in}}%
\pgfpathlineto{\pgfqpoint{0.000000in}{0.000000in}}%
\pgfpathclose%
\pgfusepath{stroke,fill}%
\end{pgfscope}%
\begin{pgfscope}%
\pgfpathrectangle{\pgfqpoint{0.647939in}{0.492442in}}{\pgfqpoint{4.273799in}{2.331163in}}%
\pgfusepath{clip}%
\pgfsetroundcap%
\pgfsetroundjoin%
\pgfsetlinewidth{0.301125pt}%
\definecolor{currentstroke}{rgb}{0.500000,0.500000,0.500000}%
\pgfsetstrokecolor{currentstroke}%
\pgfsetstrokeopacity{0.300000}%
\pgfsetdash{}{0pt}%
\pgfpathmoveto{\pgfqpoint{2.189473in}{2.747169in}}%
\pgfusepath{stroke}%
\end{pgfscope}%
\begin{pgfscope}%
\pgfpathrectangle{\pgfqpoint{0.647939in}{0.492442in}}{\pgfqpoint{4.273799in}{2.331163in}}%
\pgfusepath{clip}%
\pgfsetroundcap%
\pgfsetroundjoin%
\definecolor{currentfill}{rgb}{0.500000,0.500000,0.500000}%
\pgfsetfillcolor{currentfill}%
\pgfsetfillopacity{0.300000}%
\pgfsetlinewidth{0.301125pt}%
\definecolor{currentstroke}{rgb}{0.500000,0.500000,0.500000}%
\pgfsetstrokecolor{currentstroke}%
\pgfsetstrokeopacity{0.300000}%
\pgfsetdash{}{0pt}%
\pgfpathmoveto{\pgfqpoint{0.000000in}{0.000000in}}%
\pgfpathlineto{\pgfqpoint{0.000000in}{0.000000in}}%
\pgfpathclose%
\pgfusepath{stroke,fill}%
\end{pgfscope}%
\begin{pgfscope}%
\pgfpathrectangle{\pgfqpoint{0.647939in}{0.492442in}}{\pgfqpoint{4.273799in}{2.331163in}}%
\pgfusepath{clip}%
\pgfsetroundcap%
\pgfsetroundjoin%
\pgfsetlinewidth{0.301125pt}%
\definecolor{currentstroke}{rgb}{0.500000,0.500000,0.500000}%
\pgfsetstrokecolor{currentstroke}%
\pgfsetstrokeopacity{0.300000}%
\pgfsetdash{}{0pt}%
\pgfpathmoveto{\pgfqpoint{2.038237in}{2.693411in}}%
\pgfusepath{stroke}%
\end{pgfscope}%
\begin{pgfscope}%
\pgfpathrectangle{\pgfqpoint{0.647939in}{0.492442in}}{\pgfqpoint{4.273799in}{2.331163in}}%
\pgfusepath{clip}%
\pgfsetroundcap%
\pgfsetroundjoin%
\definecolor{currentfill}{rgb}{0.500000,0.500000,0.500000}%
\pgfsetfillcolor{currentfill}%
\pgfsetfillopacity{0.300000}%
\pgfsetlinewidth{0.301125pt}%
\definecolor{currentstroke}{rgb}{0.500000,0.500000,0.500000}%
\pgfsetstrokecolor{currentstroke}%
\pgfsetstrokeopacity{0.300000}%
\pgfsetdash{}{0pt}%
\pgfpathmoveto{\pgfqpoint{0.000000in}{0.000000in}}%
\pgfpathlineto{\pgfqpoint{0.000000in}{0.000000in}}%
\pgfpathclose%
\pgfusepath{stroke,fill}%
\end{pgfscope}%
\begin{pgfscope}%
\pgfpathrectangle{\pgfqpoint{0.647939in}{0.492442in}}{\pgfqpoint{4.273799in}{2.331163in}}%
\pgfusepath{clip}%
\pgfsetroundcap%
\pgfsetroundjoin%
\pgfsetlinewidth{0.301125pt}%
\definecolor{currentstroke}{rgb}{0.500000,0.500000,0.500000}%
\pgfsetstrokecolor{currentstroke}%
\pgfsetstrokeopacity{0.300000}%
\pgfsetdash{}{0pt}%
\pgfpathmoveto{\pgfqpoint{2.351145in}{2.618359in}}%
\pgfusepath{stroke}%
\end{pgfscope}%
\begin{pgfscope}%
\pgfpathrectangle{\pgfqpoint{0.647939in}{0.492442in}}{\pgfqpoint{4.273799in}{2.331163in}}%
\pgfusepath{clip}%
\pgfsetroundcap%
\pgfsetroundjoin%
\definecolor{currentfill}{rgb}{0.500000,0.500000,0.500000}%
\pgfsetfillcolor{currentfill}%
\pgfsetfillopacity{0.300000}%
\pgfsetlinewidth{0.301125pt}%
\definecolor{currentstroke}{rgb}{0.500000,0.500000,0.500000}%
\pgfsetstrokecolor{currentstroke}%
\pgfsetstrokeopacity{0.300000}%
\pgfsetdash{}{0pt}%
\pgfpathmoveto{\pgfqpoint{0.000000in}{0.000000in}}%
\pgfpathlineto{\pgfqpoint{0.000000in}{0.000000in}}%
\pgfpathclose%
\pgfusepath{stroke,fill}%
\end{pgfscope}%
\begin{pgfscope}%
\pgfpathrectangle{\pgfqpoint{0.647939in}{0.492442in}}{\pgfqpoint{4.273799in}{2.331163in}}%
\pgfusepath{clip}%
\pgfsetroundcap%
\pgfsetroundjoin%
\pgfsetlinewidth{0.301125pt}%
\definecolor{currentstroke}{rgb}{0.500000,0.500000,0.500000}%
\pgfsetstrokecolor{currentstroke}%
\pgfsetstrokeopacity{0.300000}%
\pgfsetdash{}{0pt}%
\pgfpathmoveto{\pgfqpoint{1.969807in}{2.567361in}}%
\pgfusepath{stroke}%
\end{pgfscope}%
\begin{pgfscope}%
\pgfpathrectangle{\pgfqpoint{0.647939in}{0.492442in}}{\pgfqpoint{4.273799in}{2.331163in}}%
\pgfusepath{clip}%
\pgfsetroundcap%
\pgfsetroundjoin%
\definecolor{currentfill}{rgb}{0.500000,0.500000,0.500000}%
\pgfsetfillcolor{currentfill}%
\pgfsetfillopacity{0.300000}%
\pgfsetlinewidth{0.301125pt}%
\definecolor{currentstroke}{rgb}{0.500000,0.500000,0.500000}%
\pgfsetstrokecolor{currentstroke}%
\pgfsetstrokeopacity{0.300000}%
\pgfsetdash{}{0pt}%
\pgfpathmoveto{\pgfqpoint{0.000000in}{0.000000in}}%
\pgfpathlineto{\pgfqpoint{0.000000in}{0.000000in}}%
\pgfpathclose%
\pgfusepath{stroke,fill}%
\end{pgfscope}%
\begin{pgfscope}%
\pgfpathrectangle{\pgfqpoint{0.647939in}{0.492442in}}{\pgfqpoint{4.273799in}{2.331163in}}%
\pgfusepath{clip}%
\pgfsetroundcap%
\pgfsetroundjoin%
\pgfsetlinewidth{0.301125pt}%
\definecolor{currentstroke}{rgb}{0.500000,0.500000,0.500000}%
\pgfsetstrokecolor{currentstroke}%
\pgfsetstrokeopacity{0.300000}%
\pgfsetdash{}{0pt}%
\pgfpathmoveto{\pgfqpoint{1.759138in}{2.496609in}}%
\pgfusepath{stroke}%
\end{pgfscope}%
\begin{pgfscope}%
\pgfpathrectangle{\pgfqpoint{0.647939in}{0.492442in}}{\pgfqpoint{4.273799in}{2.331163in}}%
\pgfusepath{clip}%
\pgfsetroundcap%
\pgfsetroundjoin%
\definecolor{currentfill}{rgb}{0.500000,0.500000,0.500000}%
\pgfsetfillcolor{currentfill}%
\pgfsetfillopacity{0.300000}%
\pgfsetlinewidth{0.301125pt}%
\definecolor{currentstroke}{rgb}{0.500000,0.500000,0.500000}%
\pgfsetstrokecolor{currentstroke}%
\pgfsetstrokeopacity{0.300000}%
\pgfsetdash{}{0pt}%
\pgfpathmoveto{\pgfqpoint{0.000000in}{0.000000in}}%
\pgfpathlineto{\pgfqpoint{0.000000in}{0.000000in}}%
\pgfpathclose%
\pgfusepath{stroke,fill}%
\end{pgfscope}%
\begin{pgfscope}%
\pgfpathrectangle{\pgfqpoint{0.647939in}{0.492442in}}{\pgfqpoint{4.273799in}{2.331163in}}%
\pgfusepath{clip}%
\pgfsetroundcap%
\pgfsetroundjoin%
\pgfsetlinewidth{0.301125pt}%
\definecolor{currentstroke}{rgb}{0.500000,0.500000,0.500000}%
\pgfsetstrokecolor{currentstroke}%
\pgfsetstrokeopacity{0.300000}%
\pgfsetdash{}{0pt}%
\pgfpathmoveto{\pgfqpoint{1.599858in}{2.414279in}}%
\pgfusepath{stroke}%
\end{pgfscope}%
\begin{pgfscope}%
\pgfpathrectangle{\pgfqpoint{0.647939in}{0.492442in}}{\pgfqpoint{4.273799in}{2.331163in}}%
\pgfusepath{clip}%
\pgfsetroundcap%
\pgfsetroundjoin%
\definecolor{currentfill}{rgb}{0.500000,0.500000,0.500000}%
\pgfsetfillcolor{currentfill}%
\pgfsetfillopacity{0.300000}%
\pgfsetlinewidth{0.301125pt}%
\definecolor{currentstroke}{rgb}{0.500000,0.500000,0.500000}%
\pgfsetstrokecolor{currentstroke}%
\pgfsetstrokeopacity{0.300000}%
\pgfsetdash{}{0pt}%
\pgfpathmoveto{\pgfqpoint{0.000000in}{0.000000in}}%
\pgfpathlineto{\pgfqpoint{0.000000in}{0.000000in}}%
\pgfpathclose%
\pgfusepath{stroke,fill}%
\end{pgfscope}%
\begin{pgfscope}%
\pgfpathrectangle{\pgfqpoint{0.647939in}{0.492442in}}{\pgfqpoint{4.273799in}{2.331163in}}%
\pgfusepath{clip}%
\pgfsetroundcap%
\pgfsetroundjoin%
\pgfsetlinewidth{0.301125pt}%
\definecolor{currentstroke}{rgb}{0.500000,0.500000,0.500000}%
\pgfsetstrokecolor{currentstroke}%
\pgfsetstrokeopacity{0.300000}%
\pgfsetdash{}{0pt}%
\pgfpathmoveto{\pgfqpoint{1.385588in}{2.470977in}}%
\pgfusepath{stroke}%
\end{pgfscope}%
\begin{pgfscope}%
\pgfpathrectangle{\pgfqpoint{0.647939in}{0.492442in}}{\pgfqpoint{4.273799in}{2.331163in}}%
\pgfusepath{clip}%
\pgfsetroundcap%
\pgfsetroundjoin%
\definecolor{currentfill}{rgb}{0.500000,0.500000,0.500000}%
\pgfsetfillcolor{currentfill}%
\pgfsetfillopacity{0.300000}%
\pgfsetlinewidth{0.301125pt}%
\definecolor{currentstroke}{rgb}{0.500000,0.500000,0.500000}%
\pgfsetstrokecolor{currentstroke}%
\pgfsetstrokeopacity{0.300000}%
\pgfsetdash{}{0pt}%
\pgfpathmoveto{\pgfqpoint{0.000000in}{0.000000in}}%
\pgfpathlineto{\pgfqpoint{0.000000in}{0.000000in}}%
\pgfpathclose%
\pgfusepath{stroke,fill}%
\end{pgfscope}%
\begin{pgfscope}%
\pgfpathrectangle{\pgfqpoint{0.647939in}{0.492442in}}{\pgfqpoint{4.273799in}{2.331163in}}%
\pgfusepath{clip}%
\pgfsetroundcap%
\pgfsetroundjoin%
\pgfsetlinewidth{0.301125pt}%
\definecolor{currentstroke}{rgb}{0.500000,0.500000,0.500000}%
\pgfsetstrokecolor{currentstroke}%
\pgfsetstrokeopacity{0.300000}%
\pgfsetdash{}{0pt}%
\pgfpathmoveto{\pgfqpoint{1.310309in}{2.262065in}}%
\pgfusepath{stroke}%
\end{pgfscope}%
\begin{pgfscope}%
\pgfpathrectangle{\pgfqpoint{0.647939in}{0.492442in}}{\pgfqpoint{4.273799in}{2.331163in}}%
\pgfusepath{clip}%
\pgfsetroundcap%
\pgfsetroundjoin%
\definecolor{currentfill}{rgb}{0.500000,0.500000,0.500000}%
\pgfsetfillcolor{currentfill}%
\pgfsetfillopacity{0.300000}%
\pgfsetlinewidth{0.301125pt}%
\definecolor{currentstroke}{rgb}{0.500000,0.500000,0.500000}%
\pgfsetstrokecolor{currentstroke}%
\pgfsetstrokeopacity{0.300000}%
\pgfsetdash{}{0pt}%
\pgfpathmoveto{\pgfqpoint{0.000000in}{0.000000in}}%
\pgfpathlineto{\pgfqpoint{0.000000in}{0.000000in}}%
\pgfpathclose%
\pgfusepath{stroke,fill}%
\end{pgfscope}%
\begin{pgfscope}%
\pgfpathrectangle{\pgfqpoint{0.647939in}{0.492442in}}{\pgfqpoint{4.273799in}{2.331163in}}%
\pgfusepath{clip}%
\pgfsetroundcap%
\pgfsetroundjoin%
\pgfsetlinewidth{0.301125pt}%
\definecolor{currentstroke}{rgb}{0.500000,0.500000,0.500000}%
\pgfsetstrokecolor{currentstroke}%
\pgfsetstrokeopacity{0.300000}%
\pgfsetdash{}{0pt}%
\pgfpathmoveto{\pgfqpoint{1.193563in}{2.000033in}}%
\pgfusepath{stroke}%
\end{pgfscope}%
\begin{pgfscope}%
\pgfpathrectangle{\pgfqpoint{0.647939in}{0.492442in}}{\pgfqpoint{4.273799in}{2.331163in}}%
\pgfusepath{clip}%
\pgfsetroundcap%
\pgfsetroundjoin%
\definecolor{currentfill}{rgb}{0.500000,0.500000,0.500000}%
\pgfsetfillcolor{currentfill}%
\pgfsetfillopacity{0.300000}%
\pgfsetlinewidth{0.301125pt}%
\definecolor{currentstroke}{rgb}{0.500000,0.500000,0.500000}%
\pgfsetstrokecolor{currentstroke}%
\pgfsetstrokeopacity{0.300000}%
\pgfsetdash{}{0pt}%
\pgfpathmoveto{\pgfqpoint{0.000000in}{0.000000in}}%
\pgfpathlineto{\pgfqpoint{0.000000in}{0.000000in}}%
\pgfpathclose%
\pgfusepath{stroke,fill}%
\end{pgfscope}%
\begin{pgfscope}%
\pgfpathrectangle{\pgfqpoint{0.647939in}{0.492442in}}{\pgfqpoint{4.273799in}{2.331163in}}%
\pgfusepath{clip}%
\pgfsetroundcap%
\pgfsetroundjoin%
\pgfsetlinewidth{0.301125pt}%
\definecolor{currentstroke}{rgb}{0.500000,0.500000,0.500000}%
\pgfsetstrokecolor{currentstroke}%
\pgfsetstrokeopacity{0.300000}%
\pgfsetdash{}{0pt}%
\pgfpathmoveto{\pgfqpoint{1.061836in}{2.049889in}}%
\pgfusepath{stroke}%
\end{pgfscope}%
\begin{pgfscope}%
\pgfpathrectangle{\pgfqpoint{0.647939in}{0.492442in}}{\pgfqpoint{4.273799in}{2.331163in}}%
\pgfusepath{clip}%
\pgfsetroundcap%
\pgfsetroundjoin%
\definecolor{currentfill}{rgb}{0.500000,0.500000,0.500000}%
\pgfsetfillcolor{currentfill}%
\pgfsetfillopacity{0.300000}%
\pgfsetlinewidth{0.301125pt}%
\definecolor{currentstroke}{rgb}{0.500000,0.500000,0.500000}%
\pgfsetstrokecolor{currentstroke}%
\pgfsetstrokeopacity{0.300000}%
\pgfsetdash{}{0pt}%
\pgfpathmoveto{\pgfqpoint{0.000000in}{0.000000in}}%
\pgfpathlineto{\pgfqpoint{0.000000in}{0.000000in}}%
\pgfpathclose%
\pgfusepath{stroke,fill}%
\end{pgfscope}%
\begin{pgfscope}%
\pgfpathrectangle{\pgfqpoint{0.647939in}{0.492442in}}{\pgfqpoint{4.273799in}{2.331163in}}%
\pgfusepath{clip}%
\pgfsetroundcap%
\pgfsetroundjoin%
\pgfsetlinewidth{0.301125pt}%
\definecolor{currentstroke}{rgb}{0.500000,0.500000,0.500000}%
\pgfsetstrokecolor{currentstroke}%
\pgfsetstrokeopacity{0.300000}%
\pgfsetdash{}{0pt}%
\pgfpathmoveto{\pgfqpoint{0.932472in}{1.789855in}}%
\pgfusepath{stroke}%
\end{pgfscope}%
\begin{pgfscope}%
\pgfpathrectangle{\pgfqpoint{0.647939in}{0.492442in}}{\pgfqpoint{4.273799in}{2.331163in}}%
\pgfusepath{clip}%
\pgfsetroundcap%
\pgfsetroundjoin%
\definecolor{currentfill}{rgb}{0.500000,0.500000,0.500000}%
\pgfsetfillcolor{currentfill}%
\pgfsetfillopacity{0.300000}%
\pgfsetlinewidth{0.301125pt}%
\definecolor{currentstroke}{rgb}{0.500000,0.500000,0.500000}%
\pgfsetstrokecolor{currentstroke}%
\pgfsetstrokeopacity{0.300000}%
\pgfsetdash{}{0pt}%
\pgfpathmoveto{\pgfqpoint{0.000000in}{0.000000in}}%
\pgfpathlineto{\pgfqpoint{0.000000in}{0.000000in}}%
\pgfpathclose%
\pgfusepath{stroke,fill}%
\end{pgfscope}%
\begin{pgfscope}%
\pgfpathrectangle{\pgfqpoint{0.647939in}{0.492442in}}{\pgfqpoint{4.273799in}{2.331163in}}%
\pgfusepath{clip}%
\pgfsetroundcap%
\pgfsetroundjoin%
\pgfsetlinewidth{0.301125pt}%
\definecolor{currentstroke}{rgb}{0.500000,0.500000,0.500000}%
\pgfsetstrokecolor{currentstroke}%
\pgfsetstrokeopacity{0.300000}%
\pgfsetdash{}{0pt}%
\pgfpathmoveto{\pgfqpoint{0.827615in}{1.996303in}}%
\pgfusepath{stroke}%
\end{pgfscope}%
\begin{pgfscope}%
\pgfpathrectangle{\pgfqpoint{0.647939in}{0.492442in}}{\pgfqpoint{4.273799in}{2.331163in}}%
\pgfusepath{clip}%
\pgfsetroundcap%
\pgfsetroundjoin%
\definecolor{currentfill}{rgb}{0.500000,0.500000,0.500000}%
\pgfsetfillcolor{currentfill}%
\pgfsetfillopacity{0.300000}%
\pgfsetlinewidth{0.301125pt}%
\definecolor{currentstroke}{rgb}{0.500000,0.500000,0.500000}%
\pgfsetstrokecolor{currentstroke}%
\pgfsetstrokeopacity{0.300000}%
\pgfsetdash{}{0pt}%
\pgfpathmoveto{\pgfqpoint{0.000000in}{0.000000in}}%
\pgfpathlineto{\pgfqpoint{0.000000in}{0.000000in}}%
\pgfpathclose%
\pgfusepath{stroke,fill}%
\end{pgfscope}%
\begin{pgfscope}%
\pgfpathrectangle{\pgfqpoint{0.647939in}{0.492442in}}{\pgfqpoint{4.273799in}{2.331163in}}%
\pgfusepath{clip}%
\pgfsetroundcap%
\pgfsetroundjoin%
\pgfsetlinewidth{0.301125pt}%
\definecolor{currentstroke}{rgb}{0.500000,0.500000,0.500000}%
\pgfsetstrokecolor{currentstroke}%
\pgfsetstrokeopacity{0.300000}%
\pgfsetdash{}{0pt}%
\pgfpathmoveto{\pgfqpoint{0.716333in}{2.099467in}}%
\pgfusepath{stroke}%
\end{pgfscope}%
\begin{pgfscope}%
\pgfpathrectangle{\pgfqpoint{0.647939in}{0.492442in}}{\pgfqpoint{4.273799in}{2.331163in}}%
\pgfusepath{clip}%
\pgfsetroundcap%
\pgfsetroundjoin%
\definecolor{currentfill}{rgb}{0.500000,0.500000,0.500000}%
\pgfsetfillcolor{currentfill}%
\pgfsetfillopacity{0.300000}%
\pgfsetlinewidth{0.301125pt}%
\definecolor{currentstroke}{rgb}{0.500000,0.500000,0.500000}%
\pgfsetstrokecolor{currentstroke}%
\pgfsetstrokeopacity{0.300000}%
\pgfsetdash{}{0pt}%
\pgfpathmoveto{\pgfqpoint{0.000000in}{0.000000in}}%
\pgfpathlineto{\pgfqpoint{0.000000in}{0.000000in}}%
\pgfpathclose%
\pgfusepath{stroke,fill}%
\end{pgfscope}%
\begin{pgfscope}%
\pgfpathrectangle{\pgfqpoint{0.647939in}{0.492442in}}{\pgfqpoint{4.273799in}{2.331163in}}%
\pgfusepath{clip}%
\pgfsetroundcap%
\pgfsetroundjoin%
\pgfsetlinewidth{0.301125pt}%
\definecolor{currentstroke}{rgb}{0.500000,0.500000,0.500000}%
\pgfsetstrokecolor{currentstroke}%
\pgfsetstrokeopacity{0.300000}%
\pgfsetdash{}{0pt}%
\pgfpathmoveto{\pgfqpoint{0.664261in}{2.037266in}}%
\pgfusepath{stroke}%
\end{pgfscope}%
\begin{pgfscope}%
\pgfpathrectangle{\pgfqpoint{0.647939in}{0.492442in}}{\pgfqpoint{4.273799in}{2.331163in}}%
\pgfusepath{clip}%
\pgfsetroundcap%
\pgfsetroundjoin%
\definecolor{currentfill}{rgb}{0.500000,0.500000,0.500000}%
\pgfsetfillcolor{currentfill}%
\pgfsetfillopacity{0.300000}%
\pgfsetlinewidth{0.301125pt}%
\definecolor{currentstroke}{rgb}{0.500000,0.500000,0.500000}%
\pgfsetstrokecolor{currentstroke}%
\pgfsetstrokeopacity{0.300000}%
\pgfsetdash{}{0pt}%
\pgfpathmoveto{\pgfqpoint{0.000000in}{0.000000in}}%
\pgfpathlineto{\pgfqpoint{0.000000in}{0.000000in}}%
\pgfpathclose%
\pgfusepath{stroke,fill}%
\end{pgfscope}%
\begin{pgfscope}%
\pgfpathrectangle{\pgfqpoint{0.647939in}{0.492442in}}{\pgfqpoint{4.273799in}{2.331163in}}%
\pgfusepath{clip}%
\pgfsetroundcap%
\pgfsetroundjoin%
\pgfsetlinewidth{0.301125pt}%
\definecolor{currentstroke}{rgb}{0.500000,0.500000,0.500000}%
\pgfsetstrokecolor{currentstroke}%
\pgfsetstrokeopacity{0.300000}%
\pgfsetdash{}{0pt}%
\pgfpathmoveto{\pgfqpoint{1.337736in}{0.596111in}}%
\pgfusepath{stroke}%
\end{pgfscope}%
\begin{pgfscope}%
\pgfpathrectangle{\pgfqpoint{0.647939in}{0.492442in}}{\pgfqpoint{4.273799in}{2.331163in}}%
\pgfusepath{clip}%
\pgfsetroundcap%
\pgfsetroundjoin%
\definecolor{currentfill}{rgb}{0.500000,0.500000,0.500000}%
\pgfsetfillcolor{currentfill}%
\pgfsetfillopacity{0.300000}%
\pgfsetlinewidth{0.301125pt}%
\definecolor{currentstroke}{rgb}{0.500000,0.500000,0.500000}%
\pgfsetstrokecolor{currentstroke}%
\pgfsetstrokeopacity{0.300000}%
\pgfsetdash{}{0pt}%
\pgfpathmoveto{\pgfqpoint{0.000000in}{0.000000in}}%
\pgfpathlineto{\pgfqpoint{0.000000in}{0.000000in}}%
\pgfpathclose%
\pgfusepath{stroke,fill}%
\end{pgfscope}%
\begin{pgfscope}%
\pgfpathrectangle{\pgfqpoint{0.647939in}{0.492442in}}{\pgfqpoint{4.273799in}{2.331163in}}%
\pgfusepath{clip}%
\pgfsetroundcap%
\pgfsetroundjoin%
\pgfsetlinewidth{0.301125pt}%
\definecolor{currentstroke}{rgb}{0.500000,0.500000,0.500000}%
\pgfsetstrokecolor{currentstroke}%
\pgfsetstrokeopacity{0.300000}%
\pgfsetdash{}{0pt}%
\pgfpathmoveto{\pgfqpoint{3.959103in}{0.741816in}}%
\pgfusepath{stroke}%
\end{pgfscope}%
\begin{pgfscope}%
\pgfpathrectangle{\pgfqpoint{0.647939in}{0.492442in}}{\pgfqpoint{4.273799in}{2.331163in}}%
\pgfusepath{clip}%
\pgfsetroundcap%
\pgfsetroundjoin%
\definecolor{currentfill}{rgb}{0.500000,0.500000,0.500000}%
\pgfsetfillcolor{currentfill}%
\pgfsetfillopacity{0.300000}%
\pgfsetlinewidth{0.301125pt}%
\definecolor{currentstroke}{rgb}{0.500000,0.500000,0.500000}%
\pgfsetstrokecolor{currentstroke}%
\pgfsetstrokeopacity{0.300000}%
\pgfsetdash{}{0pt}%
\pgfpathmoveto{\pgfqpoint{0.000000in}{0.000000in}}%
\pgfpathlineto{\pgfqpoint{0.000000in}{0.000000in}}%
\pgfpathclose%
\pgfusepath{stroke,fill}%
\end{pgfscope}%
\begin{pgfscope}%
\pgfpathrectangle{\pgfqpoint{0.647939in}{0.492442in}}{\pgfqpoint{4.273799in}{2.331163in}}%
\pgfusepath{clip}%
\pgfsetroundcap%
\pgfsetroundjoin%
\pgfsetlinewidth{0.301125pt}%
\definecolor{currentstroke}{rgb}{0.500000,0.500000,0.500000}%
\pgfsetstrokecolor{currentstroke}%
\pgfsetstrokeopacity{0.300000}%
\pgfsetdash{}{0pt}%
\pgfpathmoveto{\pgfqpoint{4.593628in}{1.917186in}}%
\pgfusepath{stroke}%
\end{pgfscope}%
\begin{pgfscope}%
\pgfpathrectangle{\pgfqpoint{0.647939in}{0.492442in}}{\pgfqpoint{4.273799in}{2.331163in}}%
\pgfusepath{clip}%
\pgfsetroundcap%
\pgfsetroundjoin%
\definecolor{currentfill}{rgb}{0.500000,0.500000,0.500000}%
\pgfsetfillcolor{currentfill}%
\pgfsetfillopacity{0.300000}%
\pgfsetlinewidth{0.301125pt}%
\definecolor{currentstroke}{rgb}{0.500000,0.500000,0.500000}%
\pgfsetstrokecolor{currentstroke}%
\pgfsetstrokeopacity{0.300000}%
\pgfsetdash{}{0pt}%
\pgfpathmoveto{\pgfqpoint{0.000000in}{0.000000in}}%
\pgfpathlineto{\pgfqpoint{0.000000in}{0.000000in}}%
\pgfpathclose%
\pgfusepath{stroke,fill}%
\end{pgfscope}%
\begin{pgfscope}%
\pgfpathrectangle{\pgfqpoint{0.647939in}{0.492442in}}{\pgfqpoint{4.273799in}{2.331163in}}%
\pgfusepath{clip}%
\pgfsetroundcap%
\pgfsetroundjoin%
\pgfsetlinewidth{0.301125pt}%
\definecolor{currentstroke}{rgb}{0.500000,0.500000,0.500000}%
\pgfsetstrokecolor{currentstroke}%
\pgfsetstrokeopacity{0.300000}%
\pgfsetdash{}{0pt}%
\pgfpathmoveto{\pgfqpoint{4.319904in}{1.493942in}}%
\pgfusepath{stroke}%
\end{pgfscope}%
\begin{pgfscope}%
\pgfpathrectangle{\pgfqpoint{0.647939in}{0.492442in}}{\pgfqpoint{4.273799in}{2.331163in}}%
\pgfusepath{clip}%
\pgfsetroundcap%
\pgfsetroundjoin%
\definecolor{currentfill}{rgb}{0.500000,0.500000,0.500000}%
\pgfsetfillcolor{currentfill}%
\pgfsetfillopacity{0.300000}%
\pgfsetlinewidth{0.301125pt}%
\definecolor{currentstroke}{rgb}{0.500000,0.500000,0.500000}%
\pgfsetstrokecolor{currentstroke}%
\pgfsetstrokeopacity{0.300000}%
\pgfsetdash{}{0pt}%
\pgfpathmoveto{\pgfqpoint{0.000000in}{0.000000in}}%
\pgfpathlineto{\pgfqpoint{0.000000in}{0.000000in}}%
\pgfpathclose%
\pgfusepath{stroke,fill}%
\end{pgfscope}%
\begin{pgfscope}%
\pgfpathrectangle{\pgfqpoint{0.647939in}{0.492442in}}{\pgfqpoint{4.273799in}{2.331163in}}%
\pgfusepath{clip}%
\pgfsetroundcap%
\pgfsetroundjoin%
\pgfsetlinewidth{0.301125pt}%
\definecolor{currentstroke}{rgb}{0.500000,0.500000,0.500000}%
\pgfsetstrokecolor{currentstroke}%
\pgfsetstrokeopacity{0.300000}%
\pgfsetdash{}{0pt}%
\pgfpathmoveto{\pgfqpoint{4.417355in}{1.750243in}}%
\pgfusepath{stroke}%
\end{pgfscope}%
\begin{pgfscope}%
\pgfpathrectangle{\pgfqpoint{0.647939in}{0.492442in}}{\pgfqpoint{4.273799in}{2.331163in}}%
\pgfusepath{clip}%
\pgfsetroundcap%
\pgfsetroundjoin%
\definecolor{currentfill}{rgb}{0.500000,0.500000,0.500000}%
\pgfsetfillcolor{currentfill}%
\pgfsetfillopacity{0.300000}%
\pgfsetlinewidth{0.301125pt}%
\definecolor{currentstroke}{rgb}{0.500000,0.500000,0.500000}%
\pgfsetstrokecolor{currentstroke}%
\pgfsetstrokeopacity{0.300000}%
\pgfsetdash{}{0pt}%
\pgfpathmoveto{\pgfqpoint{0.000000in}{0.000000in}}%
\pgfpathlineto{\pgfqpoint{0.000000in}{0.000000in}}%
\pgfpathclose%
\pgfusepath{stroke,fill}%
\end{pgfscope}%
\begin{pgfscope}%
\pgfpathrectangle{\pgfqpoint{0.647939in}{0.492442in}}{\pgfqpoint{4.273799in}{2.331163in}}%
\pgfusepath{clip}%
\pgfsetroundcap%
\pgfsetroundjoin%
\pgfsetlinewidth{0.301125pt}%
\definecolor{currentstroke}{rgb}{0.500000,0.500000,0.500000}%
\pgfsetstrokecolor{currentstroke}%
\pgfsetstrokeopacity{0.300000}%
\pgfsetdash{}{0pt}%
\pgfpathmoveto{\pgfqpoint{1.219351in}{0.812852in}}%
\pgfusepath{stroke}%
\end{pgfscope}%
\begin{pgfscope}%
\pgfpathrectangle{\pgfqpoint{0.647939in}{0.492442in}}{\pgfqpoint{4.273799in}{2.331163in}}%
\pgfusepath{clip}%
\pgfsetroundcap%
\pgfsetroundjoin%
\definecolor{currentfill}{rgb}{0.500000,0.500000,0.500000}%
\pgfsetfillcolor{currentfill}%
\pgfsetfillopacity{0.300000}%
\pgfsetlinewidth{0.301125pt}%
\definecolor{currentstroke}{rgb}{0.500000,0.500000,0.500000}%
\pgfsetstrokecolor{currentstroke}%
\pgfsetstrokeopacity{0.300000}%
\pgfsetdash{}{0pt}%
\pgfpathmoveto{\pgfqpoint{0.000000in}{0.000000in}}%
\pgfpathlineto{\pgfqpoint{0.000000in}{0.000000in}}%
\pgfpathclose%
\pgfusepath{stroke,fill}%
\end{pgfscope}%
\begin{pgfscope}%
\pgfpathrectangle{\pgfqpoint{0.647939in}{0.492442in}}{\pgfqpoint{4.273799in}{2.331163in}}%
\pgfusepath{clip}%
\pgfsetroundcap%
\pgfsetroundjoin%
\pgfsetlinewidth{0.301125pt}%
\definecolor{currentstroke}{rgb}{0.500000,0.500000,0.500000}%
\pgfsetstrokecolor{currentstroke}%
\pgfsetstrokeopacity{0.300000}%
\pgfsetdash{}{0pt}%
\pgfpathmoveto{\pgfqpoint{2.692620in}{0.874564in}}%
\pgfusepath{stroke}%
\end{pgfscope}%
\begin{pgfscope}%
\pgfpathrectangle{\pgfqpoint{0.647939in}{0.492442in}}{\pgfqpoint{4.273799in}{2.331163in}}%
\pgfusepath{clip}%
\pgfsetroundcap%
\pgfsetroundjoin%
\definecolor{currentfill}{rgb}{0.500000,0.500000,0.500000}%
\pgfsetfillcolor{currentfill}%
\pgfsetfillopacity{0.300000}%
\pgfsetlinewidth{0.301125pt}%
\definecolor{currentstroke}{rgb}{0.500000,0.500000,0.500000}%
\pgfsetstrokecolor{currentstroke}%
\pgfsetstrokeopacity{0.300000}%
\pgfsetdash{}{0pt}%
\pgfpathmoveto{\pgfqpoint{0.000000in}{0.000000in}}%
\pgfpathlineto{\pgfqpoint{0.000000in}{0.000000in}}%
\pgfpathclose%
\pgfusepath{stroke,fill}%
\end{pgfscope}%
\begin{pgfscope}%
\pgfpathrectangle{\pgfqpoint{0.647939in}{0.492442in}}{\pgfqpoint{4.273799in}{2.331163in}}%
\pgfusepath{clip}%
\pgfsetroundcap%
\pgfsetroundjoin%
\pgfsetlinewidth{0.301125pt}%
\definecolor{currentstroke}{rgb}{0.500000,0.500000,0.500000}%
\pgfsetstrokecolor{currentstroke}%
\pgfsetstrokeopacity{0.300000}%
\pgfsetdash{}{0pt}%
\pgfpathmoveto{\pgfqpoint{1.138681in}{2.031290in}}%
\pgfusepath{stroke}%
\end{pgfscope}%
\begin{pgfscope}%
\pgfpathrectangle{\pgfqpoint{0.647939in}{0.492442in}}{\pgfqpoint{4.273799in}{2.331163in}}%
\pgfusepath{clip}%
\pgfsetroundcap%
\pgfsetroundjoin%
\definecolor{currentfill}{rgb}{0.500000,0.500000,0.500000}%
\pgfsetfillcolor{currentfill}%
\pgfsetfillopacity{0.300000}%
\pgfsetlinewidth{0.301125pt}%
\definecolor{currentstroke}{rgb}{0.500000,0.500000,0.500000}%
\pgfsetstrokecolor{currentstroke}%
\pgfsetstrokeopacity{0.300000}%
\pgfsetdash{}{0pt}%
\pgfpathmoveto{\pgfqpoint{0.000000in}{0.000000in}}%
\pgfpathlineto{\pgfqpoint{0.000000in}{0.000000in}}%
\pgfpathclose%
\pgfusepath{stroke,fill}%
\end{pgfscope}%
\begin{pgfscope}%
\pgfpathrectangle{\pgfqpoint{0.647939in}{0.492442in}}{\pgfqpoint{4.273799in}{2.331163in}}%
\pgfusepath{clip}%
\pgfsetroundcap%
\pgfsetroundjoin%
\pgfsetlinewidth{0.301125pt}%
\definecolor{currentstroke}{rgb}{0.500000,0.500000,0.500000}%
\pgfsetstrokecolor{currentstroke}%
\pgfsetstrokeopacity{0.300000}%
\pgfsetdash{}{0pt}%
\pgfpathmoveto{\pgfqpoint{1.993356in}{1.501171in}}%
\pgfusepath{stroke}%
\end{pgfscope}%
\begin{pgfscope}%
\pgfpathrectangle{\pgfqpoint{0.647939in}{0.492442in}}{\pgfqpoint{4.273799in}{2.331163in}}%
\pgfusepath{clip}%
\pgfsetroundcap%
\pgfsetroundjoin%
\definecolor{currentfill}{rgb}{0.500000,0.500000,0.500000}%
\pgfsetfillcolor{currentfill}%
\pgfsetfillopacity{0.300000}%
\pgfsetlinewidth{0.301125pt}%
\definecolor{currentstroke}{rgb}{0.500000,0.500000,0.500000}%
\pgfsetstrokecolor{currentstroke}%
\pgfsetstrokeopacity{0.300000}%
\pgfsetdash{}{0pt}%
\pgfpathmoveto{\pgfqpoint{0.000000in}{0.000000in}}%
\pgfpathlineto{\pgfqpoint{0.000000in}{0.000000in}}%
\pgfpathclose%
\pgfusepath{stroke,fill}%
\end{pgfscope}%
\begin{pgfscope}%
\pgfpathrectangle{\pgfqpoint{0.647939in}{0.492442in}}{\pgfqpoint{4.273799in}{2.331163in}}%
\pgfusepath{clip}%
\pgfsetroundcap%
\pgfsetroundjoin%
\pgfsetlinewidth{0.301125pt}%
\definecolor{currentstroke}{rgb}{0.500000,0.500000,0.500000}%
\pgfsetstrokecolor{currentstroke}%
\pgfsetstrokeopacity{0.300000}%
\pgfsetdash{}{0pt}%
\pgfpathmoveto{\pgfqpoint{4.056041in}{1.408979in}}%
\pgfusepath{stroke}%
\end{pgfscope}%
\begin{pgfscope}%
\pgfpathrectangle{\pgfqpoint{0.647939in}{0.492442in}}{\pgfqpoint{4.273799in}{2.331163in}}%
\pgfusepath{clip}%
\pgfsetroundcap%
\pgfsetroundjoin%
\definecolor{currentfill}{rgb}{0.500000,0.500000,0.500000}%
\pgfsetfillcolor{currentfill}%
\pgfsetfillopacity{0.300000}%
\pgfsetlinewidth{0.301125pt}%
\definecolor{currentstroke}{rgb}{0.500000,0.500000,0.500000}%
\pgfsetstrokecolor{currentstroke}%
\pgfsetstrokeopacity{0.300000}%
\pgfsetdash{}{0pt}%
\pgfpathmoveto{\pgfqpoint{0.000000in}{0.000000in}}%
\pgfpathlineto{\pgfqpoint{0.000000in}{0.000000in}}%
\pgfpathclose%
\pgfusepath{stroke,fill}%
\end{pgfscope}%
\begin{pgfscope}%
\pgfpathrectangle{\pgfqpoint{0.647939in}{0.492442in}}{\pgfqpoint{4.273799in}{2.331163in}}%
\pgfusepath{clip}%
\pgfsetroundcap%
\pgfsetroundjoin%
\pgfsetlinewidth{0.301125pt}%
\definecolor{currentstroke}{rgb}{0.500000,0.500000,0.500000}%
\pgfsetstrokecolor{currentstroke}%
\pgfsetstrokeopacity{0.300000}%
\pgfsetdash{}{0pt}%
\pgfpathmoveto{\pgfqpoint{4.230465in}{1.660489in}}%
\pgfusepath{stroke}%
\end{pgfscope}%
\begin{pgfscope}%
\pgfpathrectangle{\pgfqpoint{0.647939in}{0.492442in}}{\pgfqpoint{4.273799in}{2.331163in}}%
\pgfusepath{clip}%
\pgfsetroundcap%
\pgfsetroundjoin%
\definecolor{currentfill}{rgb}{0.500000,0.500000,0.500000}%
\pgfsetfillcolor{currentfill}%
\pgfsetfillopacity{0.300000}%
\pgfsetlinewidth{0.301125pt}%
\definecolor{currentstroke}{rgb}{0.500000,0.500000,0.500000}%
\pgfsetstrokecolor{currentstroke}%
\pgfsetstrokeopacity{0.300000}%
\pgfsetdash{}{0pt}%
\pgfpathmoveto{\pgfqpoint{0.000000in}{0.000000in}}%
\pgfpathlineto{\pgfqpoint{0.000000in}{0.000000in}}%
\pgfpathclose%
\pgfusepath{stroke,fill}%
\end{pgfscope}%
\begin{pgfscope}%
\pgfpathrectangle{\pgfqpoint{0.647939in}{0.492442in}}{\pgfqpoint{4.273799in}{2.331163in}}%
\pgfusepath{clip}%
\pgfsetroundcap%
\pgfsetroundjoin%
\pgfsetlinewidth{0.301125pt}%
\definecolor{currentstroke}{rgb}{0.500000,0.500000,0.500000}%
\pgfsetstrokecolor{currentstroke}%
\pgfsetstrokeopacity{0.300000}%
\pgfsetdash{}{0pt}%
\pgfpathmoveto{\pgfqpoint{1.698939in}{2.450993in}}%
\pgfusepath{stroke}%
\end{pgfscope}%
\begin{pgfscope}%
\pgfpathrectangle{\pgfqpoint{0.647939in}{0.492442in}}{\pgfqpoint{4.273799in}{2.331163in}}%
\pgfusepath{clip}%
\pgfsetroundcap%
\pgfsetroundjoin%
\definecolor{currentfill}{rgb}{0.500000,0.500000,0.500000}%
\pgfsetfillcolor{currentfill}%
\pgfsetfillopacity{0.300000}%
\pgfsetlinewidth{0.301125pt}%
\definecolor{currentstroke}{rgb}{0.500000,0.500000,0.500000}%
\pgfsetstrokecolor{currentstroke}%
\pgfsetstrokeopacity{0.300000}%
\pgfsetdash{}{0pt}%
\pgfpathmoveto{\pgfqpoint{0.000000in}{0.000000in}}%
\pgfpathlineto{\pgfqpoint{0.000000in}{0.000000in}}%
\pgfpathclose%
\pgfusepath{stroke,fill}%
\end{pgfscope}%
\begin{pgfscope}%
\pgfpathrectangle{\pgfqpoint{0.647939in}{0.492442in}}{\pgfqpoint{4.273799in}{2.331163in}}%
\pgfusepath{clip}%
\pgfsetroundcap%
\pgfsetroundjoin%
\pgfsetlinewidth{0.301125pt}%
\definecolor{currentstroke}{rgb}{0.500000,0.500000,0.500000}%
\pgfsetstrokecolor{currentstroke}%
\pgfsetstrokeopacity{0.300000}%
\pgfsetdash{}{0pt}%
\pgfpathmoveto{\pgfqpoint{1.469959in}{1.362037in}}%
\pgfusepath{stroke}%
\end{pgfscope}%
\begin{pgfscope}%
\pgfpathrectangle{\pgfqpoint{0.647939in}{0.492442in}}{\pgfqpoint{4.273799in}{2.331163in}}%
\pgfusepath{clip}%
\pgfsetroundcap%
\pgfsetroundjoin%
\definecolor{currentfill}{rgb}{0.500000,0.500000,0.500000}%
\pgfsetfillcolor{currentfill}%
\pgfsetfillopacity{0.300000}%
\pgfsetlinewidth{0.301125pt}%
\definecolor{currentstroke}{rgb}{0.500000,0.500000,0.500000}%
\pgfsetstrokecolor{currentstroke}%
\pgfsetstrokeopacity{0.300000}%
\pgfsetdash{}{0pt}%
\pgfpathmoveto{\pgfqpoint{0.000000in}{0.000000in}}%
\pgfpathlineto{\pgfqpoint{0.000000in}{0.000000in}}%
\pgfpathclose%
\pgfusepath{stroke,fill}%
\end{pgfscope}%
\begin{pgfscope}%
\pgfpathrectangle{\pgfqpoint{0.647939in}{0.492442in}}{\pgfqpoint{4.273799in}{2.331163in}}%
\pgfusepath{clip}%
\pgfsetroundcap%
\pgfsetroundjoin%
\pgfsetlinewidth{0.301125pt}%
\definecolor{currentstroke}{rgb}{0.500000,0.500000,0.500000}%
\pgfsetstrokecolor{currentstroke}%
\pgfsetstrokeopacity{0.300000}%
\pgfsetdash{}{0pt}%
\pgfpathmoveto{\pgfqpoint{1.333818in}{0.957730in}}%
\pgfusepath{stroke}%
\end{pgfscope}%
\begin{pgfscope}%
\pgfpathrectangle{\pgfqpoint{0.647939in}{0.492442in}}{\pgfqpoint{4.273799in}{2.331163in}}%
\pgfusepath{clip}%
\pgfsetroundcap%
\pgfsetroundjoin%
\definecolor{currentfill}{rgb}{0.500000,0.500000,0.500000}%
\pgfsetfillcolor{currentfill}%
\pgfsetfillopacity{0.300000}%
\pgfsetlinewidth{0.301125pt}%
\definecolor{currentstroke}{rgb}{0.500000,0.500000,0.500000}%
\pgfsetstrokecolor{currentstroke}%
\pgfsetstrokeopacity{0.300000}%
\pgfsetdash{}{0pt}%
\pgfpathmoveto{\pgfqpoint{0.000000in}{0.000000in}}%
\pgfpathlineto{\pgfqpoint{0.000000in}{0.000000in}}%
\pgfpathclose%
\pgfusepath{stroke,fill}%
\end{pgfscope}%
\begin{pgfscope}%
\pgfpathrectangle{\pgfqpoint{0.647939in}{0.492442in}}{\pgfqpoint{4.273799in}{2.331163in}}%
\pgfusepath{clip}%
\pgfsetroundcap%
\pgfsetroundjoin%
\pgfsetlinewidth{0.301125pt}%
\definecolor{currentstroke}{rgb}{0.500000,0.500000,0.500000}%
\pgfsetstrokecolor{currentstroke}%
\pgfsetstrokeopacity{0.300000}%
\pgfsetdash{}{0pt}%
\pgfpathmoveto{\pgfqpoint{3.847769in}{1.319287in}}%
\pgfusepath{stroke}%
\end{pgfscope}%
\begin{pgfscope}%
\pgfpathrectangle{\pgfqpoint{0.647939in}{0.492442in}}{\pgfqpoint{4.273799in}{2.331163in}}%
\pgfusepath{clip}%
\pgfsetroundcap%
\pgfsetroundjoin%
\definecolor{currentfill}{rgb}{0.500000,0.500000,0.500000}%
\pgfsetfillcolor{currentfill}%
\pgfsetfillopacity{0.300000}%
\pgfsetlinewidth{0.301125pt}%
\definecolor{currentstroke}{rgb}{0.500000,0.500000,0.500000}%
\pgfsetstrokecolor{currentstroke}%
\pgfsetstrokeopacity{0.300000}%
\pgfsetdash{}{0pt}%
\pgfpathmoveto{\pgfqpoint{0.000000in}{0.000000in}}%
\pgfpathlineto{\pgfqpoint{0.000000in}{0.000000in}}%
\pgfpathclose%
\pgfusepath{stroke,fill}%
\end{pgfscope}%
\begin{pgfscope}%
\pgfpathrectangle{\pgfqpoint{0.647939in}{0.492442in}}{\pgfqpoint{4.273799in}{2.331163in}}%
\pgfusepath{clip}%
\pgfsetroundcap%
\pgfsetroundjoin%
\pgfsetlinewidth{0.301125pt}%
\definecolor{currentstroke}{rgb}{0.500000,0.500000,0.500000}%
\pgfsetstrokecolor{currentstroke}%
\pgfsetstrokeopacity{0.300000}%
\pgfsetdash{}{0pt}%
\pgfpathmoveto{\pgfqpoint{3.959846in}{1.613269in}}%
\pgfusepath{stroke}%
\end{pgfscope}%
\begin{pgfscope}%
\pgfpathrectangle{\pgfqpoint{0.647939in}{0.492442in}}{\pgfqpoint{4.273799in}{2.331163in}}%
\pgfusepath{clip}%
\pgfsetroundcap%
\pgfsetroundjoin%
\definecolor{currentfill}{rgb}{0.500000,0.500000,0.500000}%
\pgfsetfillcolor{currentfill}%
\pgfsetfillopacity{0.300000}%
\pgfsetlinewidth{0.301125pt}%
\definecolor{currentstroke}{rgb}{0.500000,0.500000,0.500000}%
\pgfsetstrokecolor{currentstroke}%
\pgfsetstrokeopacity{0.300000}%
\pgfsetdash{}{0pt}%
\pgfpathmoveto{\pgfqpoint{0.000000in}{0.000000in}}%
\pgfpathlineto{\pgfqpoint{0.000000in}{0.000000in}}%
\pgfpathclose%
\pgfusepath{stroke,fill}%
\end{pgfscope}%
\begin{pgfscope}%
\pgfpathrectangle{\pgfqpoint{0.647939in}{0.492442in}}{\pgfqpoint{4.273799in}{2.331163in}}%
\pgfusepath{clip}%
\pgfsetroundcap%
\pgfsetroundjoin%
\pgfsetlinewidth{0.301125pt}%
\definecolor{currentstroke}{rgb}{0.500000,0.500000,0.500000}%
\pgfsetstrokecolor{currentstroke}%
\pgfsetstrokeopacity{0.300000}%
\pgfsetdash{}{0pt}%
\pgfpathmoveto{\pgfqpoint{1.624999in}{2.342378in}}%
\pgfusepath{stroke}%
\end{pgfscope}%
\begin{pgfscope}%
\pgfpathrectangle{\pgfqpoint{0.647939in}{0.492442in}}{\pgfqpoint{4.273799in}{2.331163in}}%
\pgfusepath{clip}%
\pgfsetroundcap%
\pgfsetroundjoin%
\definecolor{currentfill}{rgb}{0.500000,0.500000,0.500000}%
\pgfsetfillcolor{currentfill}%
\pgfsetfillopacity{0.300000}%
\pgfsetlinewidth{0.301125pt}%
\definecolor{currentstroke}{rgb}{0.500000,0.500000,0.500000}%
\pgfsetstrokecolor{currentstroke}%
\pgfsetstrokeopacity{0.300000}%
\pgfsetdash{}{0pt}%
\pgfpathmoveto{\pgfqpoint{0.000000in}{0.000000in}}%
\pgfpathlineto{\pgfqpoint{0.000000in}{0.000000in}}%
\pgfpathclose%
\pgfusepath{stroke,fill}%
\end{pgfscope}%
\begin{pgfscope}%
\pgfpathrectangle{\pgfqpoint{0.647939in}{0.492442in}}{\pgfqpoint{4.273799in}{2.331163in}}%
\pgfusepath{clip}%
\pgfsetroundcap%
\pgfsetroundjoin%
\pgfsetlinewidth{0.301125pt}%
\definecolor{currentstroke}{rgb}{0.500000,0.500000,0.500000}%
\pgfsetstrokecolor{currentstroke}%
\pgfsetstrokeopacity{0.300000}%
\pgfsetdash{}{0pt}%
\pgfpathmoveto{\pgfqpoint{1.564159in}{1.528117in}}%
\pgfusepath{stroke}%
\end{pgfscope}%
\begin{pgfscope}%
\pgfpathrectangle{\pgfqpoint{0.647939in}{0.492442in}}{\pgfqpoint{4.273799in}{2.331163in}}%
\pgfusepath{clip}%
\pgfsetroundcap%
\pgfsetroundjoin%
\definecolor{currentfill}{rgb}{0.500000,0.500000,0.500000}%
\pgfsetfillcolor{currentfill}%
\pgfsetfillopacity{0.300000}%
\pgfsetlinewidth{0.301125pt}%
\definecolor{currentstroke}{rgb}{0.500000,0.500000,0.500000}%
\pgfsetstrokecolor{currentstroke}%
\pgfsetstrokeopacity{0.300000}%
\pgfsetdash{}{0pt}%
\pgfpathmoveto{\pgfqpoint{0.000000in}{0.000000in}}%
\pgfpathlineto{\pgfqpoint{0.000000in}{0.000000in}}%
\pgfpathclose%
\pgfusepath{stroke,fill}%
\end{pgfscope}%
\begin{pgfscope}%
\pgfpathrectangle{\pgfqpoint{0.647939in}{0.492442in}}{\pgfqpoint{4.273799in}{2.331163in}}%
\pgfusepath{clip}%
\pgfsetroundcap%
\pgfsetroundjoin%
\pgfsetlinewidth{0.301125pt}%
\definecolor{currentstroke}{rgb}{0.500000,0.500000,0.500000}%
\pgfsetstrokecolor{currentstroke}%
\pgfsetstrokeopacity{0.300000}%
\pgfsetdash{}{0pt}%
\pgfpathmoveto{\pgfqpoint{1.895729in}{1.085618in}}%
\pgfusepath{stroke}%
\end{pgfscope}%
\begin{pgfscope}%
\pgfpathrectangle{\pgfqpoint{0.647939in}{0.492442in}}{\pgfqpoint{4.273799in}{2.331163in}}%
\pgfusepath{clip}%
\pgfsetroundcap%
\pgfsetroundjoin%
\definecolor{currentfill}{rgb}{0.500000,0.500000,0.500000}%
\pgfsetfillcolor{currentfill}%
\pgfsetfillopacity{0.300000}%
\pgfsetlinewidth{0.301125pt}%
\definecolor{currentstroke}{rgb}{0.500000,0.500000,0.500000}%
\pgfsetstrokecolor{currentstroke}%
\pgfsetstrokeopacity{0.300000}%
\pgfsetdash{}{0pt}%
\pgfpathmoveto{\pgfqpoint{0.000000in}{0.000000in}}%
\pgfpathlineto{\pgfqpoint{0.000000in}{0.000000in}}%
\pgfpathclose%
\pgfusepath{stroke,fill}%
\end{pgfscope}%
\begin{pgfscope}%
\pgfpathrectangle{\pgfqpoint{0.647939in}{0.492442in}}{\pgfqpoint{4.273799in}{2.331163in}}%
\pgfusepath{clip}%
\pgfsetroundcap%
\pgfsetroundjoin%
\pgfsetlinewidth{0.301125pt}%
\definecolor{currentstroke}{rgb}{0.500000,0.500000,0.500000}%
\pgfsetstrokecolor{currentstroke}%
\pgfsetstrokeopacity{0.300000}%
\pgfsetdash{}{0pt}%
\pgfpathmoveto{\pgfqpoint{2.136576in}{1.922655in}}%
\pgfusepath{stroke}%
\end{pgfscope}%
\begin{pgfscope}%
\pgfpathrectangle{\pgfqpoint{0.647939in}{0.492442in}}{\pgfqpoint{4.273799in}{2.331163in}}%
\pgfusepath{clip}%
\pgfsetroundcap%
\pgfsetroundjoin%
\definecolor{currentfill}{rgb}{0.500000,0.500000,0.500000}%
\pgfsetfillcolor{currentfill}%
\pgfsetfillopacity{0.300000}%
\pgfsetlinewidth{0.301125pt}%
\definecolor{currentstroke}{rgb}{0.500000,0.500000,0.500000}%
\pgfsetstrokecolor{currentstroke}%
\pgfsetstrokeopacity{0.300000}%
\pgfsetdash{}{0pt}%
\pgfpathmoveto{\pgfqpoint{0.000000in}{0.000000in}}%
\pgfpathlineto{\pgfqpoint{0.000000in}{0.000000in}}%
\pgfpathclose%
\pgfusepath{stroke,fill}%
\end{pgfscope}%
\begin{pgfscope}%
\pgfpathrectangle{\pgfqpoint{0.647939in}{0.492442in}}{\pgfqpoint{4.273799in}{2.331163in}}%
\pgfusepath{clip}%
\pgfsetroundcap%
\pgfsetroundjoin%
\pgfsetlinewidth{0.301125pt}%
\definecolor{currentstroke}{rgb}{0.500000,0.500000,0.500000}%
\pgfsetstrokecolor{currentstroke}%
\pgfsetstrokeopacity{0.300000}%
\pgfsetdash{}{0pt}%
\pgfpathmoveto{\pgfqpoint{1.471268in}{2.087530in}}%
\pgfusepath{stroke}%
\end{pgfscope}%
\begin{pgfscope}%
\pgfpathrectangle{\pgfqpoint{0.647939in}{0.492442in}}{\pgfqpoint{4.273799in}{2.331163in}}%
\pgfusepath{clip}%
\pgfsetroundcap%
\pgfsetroundjoin%
\definecolor{currentfill}{rgb}{0.500000,0.500000,0.500000}%
\pgfsetfillcolor{currentfill}%
\pgfsetfillopacity{0.300000}%
\pgfsetlinewidth{0.301125pt}%
\definecolor{currentstroke}{rgb}{0.500000,0.500000,0.500000}%
\pgfsetstrokecolor{currentstroke}%
\pgfsetstrokeopacity{0.300000}%
\pgfsetdash{}{0pt}%
\pgfpathmoveto{\pgfqpoint{0.000000in}{0.000000in}}%
\pgfpathlineto{\pgfqpoint{0.000000in}{0.000000in}}%
\pgfpathclose%
\pgfusepath{stroke,fill}%
\end{pgfscope}%
\begin{pgfscope}%
\pgfpathrectangle{\pgfqpoint{0.647939in}{0.492442in}}{\pgfqpoint{4.273799in}{2.331163in}}%
\pgfusepath{clip}%
\pgfsetroundcap%
\pgfsetroundjoin%
\pgfsetlinewidth{0.301125pt}%
\definecolor{currentstroke}{rgb}{0.500000,0.500000,0.500000}%
\pgfsetstrokecolor{currentstroke}%
\pgfsetstrokeopacity{0.300000}%
\pgfsetdash{}{0pt}%
\pgfpathmoveto{\pgfqpoint{3.961449in}{1.128143in}}%
\pgfusepath{stroke}%
\end{pgfscope}%
\begin{pgfscope}%
\pgfpathrectangle{\pgfqpoint{0.647939in}{0.492442in}}{\pgfqpoint{4.273799in}{2.331163in}}%
\pgfusepath{clip}%
\pgfsetroundcap%
\pgfsetroundjoin%
\definecolor{currentfill}{rgb}{0.500000,0.500000,0.500000}%
\pgfsetfillcolor{currentfill}%
\pgfsetfillopacity{0.300000}%
\pgfsetlinewidth{0.301125pt}%
\definecolor{currentstroke}{rgb}{0.500000,0.500000,0.500000}%
\pgfsetstrokecolor{currentstroke}%
\pgfsetstrokeopacity{0.300000}%
\pgfsetdash{}{0pt}%
\pgfpathmoveto{\pgfqpoint{0.000000in}{0.000000in}}%
\pgfpathlineto{\pgfqpoint{0.000000in}{0.000000in}}%
\pgfpathclose%
\pgfusepath{stroke,fill}%
\end{pgfscope}%
\begin{pgfscope}%
\pgfpathrectangle{\pgfqpoint{0.647939in}{0.492442in}}{\pgfqpoint{4.273799in}{2.331163in}}%
\pgfusepath{clip}%
\pgfsetroundcap%
\pgfsetroundjoin%
\pgfsetlinewidth{0.301125pt}%
\definecolor{currentstroke}{rgb}{0.500000,0.500000,0.500000}%
\pgfsetstrokecolor{currentstroke}%
\pgfsetstrokeopacity{0.300000}%
\pgfsetdash{}{0pt}%
\pgfpathmoveto{\pgfqpoint{2.572674in}{2.334126in}}%
\pgfusepath{stroke}%
\end{pgfscope}%
\begin{pgfscope}%
\pgfpathrectangle{\pgfqpoint{0.647939in}{0.492442in}}{\pgfqpoint{4.273799in}{2.331163in}}%
\pgfusepath{clip}%
\pgfsetroundcap%
\pgfsetroundjoin%
\definecolor{currentfill}{rgb}{0.500000,0.500000,0.500000}%
\pgfsetfillcolor{currentfill}%
\pgfsetfillopacity{0.300000}%
\pgfsetlinewidth{0.301125pt}%
\definecolor{currentstroke}{rgb}{0.500000,0.500000,0.500000}%
\pgfsetstrokecolor{currentstroke}%
\pgfsetstrokeopacity{0.300000}%
\pgfsetdash{}{0pt}%
\pgfpathmoveto{\pgfqpoint{0.000000in}{0.000000in}}%
\pgfpathlineto{\pgfqpoint{0.000000in}{0.000000in}}%
\pgfpathclose%
\pgfusepath{stroke,fill}%
\end{pgfscope}%
\begin{pgfscope}%
\pgfpathrectangle{\pgfqpoint{0.647939in}{0.492442in}}{\pgfqpoint{4.273799in}{2.331163in}}%
\pgfusepath{clip}%
\pgfsetroundcap%
\pgfsetroundjoin%
\pgfsetlinewidth{0.301125pt}%
\definecolor{currentstroke}{rgb}{0.500000,0.500000,0.500000}%
\pgfsetstrokecolor{currentstroke}%
\pgfsetstrokeopacity{0.300000}%
\pgfsetdash{}{0pt}%
\pgfpathmoveto{\pgfqpoint{3.813483in}{1.968184in}}%
\pgfusepath{stroke}%
\end{pgfscope}%
\begin{pgfscope}%
\pgfpathrectangle{\pgfqpoint{0.647939in}{0.492442in}}{\pgfqpoint{4.273799in}{2.331163in}}%
\pgfusepath{clip}%
\pgfsetroundcap%
\pgfsetroundjoin%
\definecolor{currentfill}{rgb}{0.500000,0.500000,0.500000}%
\pgfsetfillcolor{currentfill}%
\pgfsetfillopacity{0.300000}%
\pgfsetlinewidth{0.301125pt}%
\definecolor{currentstroke}{rgb}{0.500000,0.500000,0.500000}%
\pgfsetstrokecolor{currentstroke}%
\pgfsetstrokeopacity{0.300000}%
\pgfsetdash{}{0pt}%
\pgfpathmoveto{\pgfqpoint{0.000000in}{0.000000in}}%
\pgfpathlineto{\pgfqpoint{0.000000in}{0.000000in}}%
\pgfpathclose%
\pgfusepath{stroke,fill}%
\end{pgfscope}%
\begin{pgfscope}%
\pgfpathrectangle{\pgfqpoint{0.647939in}{0.492442in}}{\pgfqpoint{4.273799in}{2.331163in}}%
\pgfusepath{clip}%
\pgfsetroundcap%
\pgfsetroundjoin%
\pgfsetlinewidth{0.301125pt}%
\definecolor{currentstroke}{rgb}{0.500000,0.500000,0.500000}%
\pgfsetstrokecolor{currentstroke}%
\pgfsetstrokeopacity{0.300000}%
\pgfsetdash{}{0pt}%
\pgfpathmoveto{\pgfqpoint{1.772355in}{1.812139in}}%
\pgfusepath{stroke}%
\end{pgfscope}%
\begin{pgfscope}%
\pgfpathrectangle{\pgfqpoint{0.647939in}{0.492442in}}{\pgfqpoint{4.273799in}{2.331163in}}%
\pgfusepath{clip}%
\pgfsetroundcap%
\pgfsetroundjoin%
\definecolor{currentfill}{rgb}{0.500000,0.500000,0.500000}%
\pgfsetfillcolor{currentfill}%
\pgfsetfillopacity{0.300000}%
\pgfsetlinewidth{0.301125pt}%
\definecolor{currentstroke}{rgb}{0.500000,0.500000,0.500000}%
\pgfsetstrokecolor{currentstroke}%
\pgfsetstrokeopacity{0.300000}%
\pgfsetdash{}{0pt}%
\pgfpathmoveto{\pgfqpoint{0.000000in}{0.000000in}}%
\pgfpathlineto{\pgfqpoint{0.000000in}{0.000000in}}%
\pgfpathclose%
\pgfusepath{stroke,fill}%
\end{pgfscope}%
\begin{pgfscope}%
\pgfpathrectangle{\pgfqpoint{0.647939in}{0.492442in}}{\pgfqpoint{4.273799in}{2.331163in}}%
\pgfusepath{clip}%
\pgfsetroundcap%
\pgfsetroundjoin%
\pgfsetlinewidth{0.301125pt}%
\definecolor{currentstroke}{rgb}{0.500000,0.500000,0.500000}%
\pgfsetstrokecolor{currentstroke}%
\pgfsetstrokeopacity{0.300000}%
\pgfsetdash{}{0pt}%
\pgfpathmoveto{\pgfqpoint{2.479042in}{1.204946in}}%
\pgfusepath{stroke}%
\end{pgfscope}%
\begin{pgfscope}%
\pgfpathrectangle{\pgfqpoint{0.647939in}{0.492442in}}{\pgfqpoint{4.273799in}{2.331163in}}%
\pgfusepath{clip}%
\pgfsetroundcap%
\pgfsetroundjoin%
\definecolor{currentfill}{rgb}{0.500000,0.500000,0.500000}%
\pgfsetfillcolor{currentfill}%
\pgfsetfillopacity{0.300000}%
\pgfsetlinewidth{0.301125pt}%
\definecolor{currentstroke}{rgb}{0.500000,0.500000,0.500000}%
\pgfsetstrokecolor{currentstroke}%
\pgfsetstrokeopacity{0.300000}%
\pgfsetdash{}{0pt}%
\pgfpathmoveto{\pgfqpoint{0.000000in}{0.000000in}}%
\pgfpathlineto{\pgfqpoint{0.000000in}{0.000000in}}%
\pgfpathclose%
\pgfusepath{stroke,fill}%
\end{pgfscope}%
\begin{pgfscope}%
\pgfpathrectangle{\pgfqpoint{0.647939in}{0.492442in}}{\pgfqpoint{4.273799in}{2.331163in}}%
\pgfusepath{clip}%
\pgfsetroundcap%
\pgfsetroundjoin%
\pgfsetlinewidth{0.301125pt}%
\definecolor{currentstroke}{rgb}{0.500000,0.500000,0.500000}%
\pgfsetstrokecolor{currentstroke}%
\pgfsetstrokeopacity{0.300000}%
\pgfsetdash{}{0pt}%
\pgfpathmoveto{\pgfqpoint{2.745953in}{1.938513in}}%
\pgfusepath{stroke}%
\end{pgfscope}%
\begin{pgfscope}%
\pgfpathrectangle{\pgfqpoint{0.647939in}{0.492442in}}{\pgfqpoint{4.273799in}{2.331163in}}%
\pgfusepath{clip}%
\pgfsetroundcap%
\pgfsetroundjoin%
\definecolor{currentfill}{rgb}{0.500000,0.500000,0.500000}%
\pgfsetfillcolor{currentfill}%
\pgfsetfillopacity{0.300000}%
\pgfsetlinewidth{0.301125pt}%
\definecolor{currentstroke}{rgb}{0.500000,0.500000,0.500000}%
\pgfsetstrokecolor{currentstroke}%
\pgfsetstrokeopacity{0.300000}%
\pgfsetdash{}{0pt}%
\pgfpathmoveto{\pgfqpoint{0.000000in}{0.000000in}}%
\pgfpathlineto{\pgfqpoint{0.000000in}{0.000000in}}%
\pgfpathclose%
\pgfusepath{stroke,fill}%
\end{pgfscope}%
\begin{pgfscope}%
\pgfpathrectangle{\pgfqpoint{0.647939in}{0.492442in}}{\pgfqpoint{4.273799in}{2.331163in}}%
\pgfusepath{clip}%
\pgfsetroundcap%
\pgfsetroundjoin%
\pgfsetlinewidth{0.301125pt}%
\definecolor{currentstroke}{rgb}{0.500000,0.500000,0.500000}%
\pgfsetstrokecolor{currentstroke}%
\pgfsetstrokeopacity{0.300000}%
\pgfsetdash{}{0pt}%
\pgfpathmoveto{\pgfqpoint{2.185436in}{1.979950in}}%
\pgfusepath{stroke}%
\end{pgfscope}%
\begin{pgfscope}%
\pgfpathrectangle{\pgfqpoint{0.647939in}{0.492442in}}{\pgfqpoint{4.273799in}{2.331163in}}%
\pgfusepath{clip}%
\pgfsetroundcap%
\pgfsetroundjoin%
\definecolor{currentfill}{rgb}{0.500000,0.500000,0.500000}%
\pgfsetfillcolor{currentfill}%
\pgfsetfillopacity{0.300000}%
\pgfsetlinewidth{0.301125pt}%
\definecolor{currentstroke}{rgb}{0.500000,0.500000,0.500000}%
\pgfsetstrokecolor{currentstroke}%
\pgfsetstrokeopacity{0.300000}%
\pgfsetdash{}{0pt}%
\pgfpathmoveto{\pgfqpoint{0.000000in}{0.000000in}}%
\pgfpathlineto{\pgfqpoint{0.000000in}{0.000000in}}%
\pgfpathclose%
\pgfusepath{stroke,fill}%
\end{pgfscope}%
\begin{pgfscope}%
\pgfpathrectangle{\pgfqpoint{0.647939in}{0.492442in}}{\pgfqpoint{4.273799in}{2.331163in}}%
\pgfusepath{clip}%
\pgfsetroundcap%
\pgfsetroundjoin%
\pgfsetlinewidth{0.301125pt}%
\definecolor{currentstroke}{rgb}{0.500000,0.500000,0.500000}%
\pgfsetstrokecolor{currentstroke}%
\pgfsetstrokeopacity{0.300000}%
\pgfsetdash{}{0pt}%
\pgfpathmoveto{\pgfqpoint{1.888099in}{1.980373in}}%
\pgfusepath{stroke}%
\end{pgfscope}%
\begin{pgfscope}%
\pgfpathrectangle{\pgfqpoint{0.647939in}{0.492442in}}{\pgfqpoint{4.273799in}{2.331163in}}%
\pgfusepath{clip}%
\pgfsetroundcap%
\pgfsetroundjoin%
\definecolor{currentfill}{rgb}{0.500000,0.500000,0.500000}%
\pgfsetfillcolor{currentfill}%
\pgfsetfillopacity{0.300000}%
\pgfsetlinewidth{0.301125pt}%
\definecolor{currentstroke}{rgb}{0.500000,0.500000,0.500000}%
\pgfsetstrokecolor{currentstroke}%
\pgfsetstrokeopacity{0.300000}%
\pgfsetdash{}{0pt}%
\pgfpathmoveto{\pgfqpoint{0.000000in}{0.000000in}}%
\pgfpathlineto{\pgfqpoint{0.000000in}{0.000000in}}%
\pgfpathclose%
\pgfusepath{stroke,fill}%
\end{pgfscope}%
\begin{pgfscope}%
\pgfpathrectangle{\pgfqpoint{0.647939in}{0.492442in}}{\pgfqpoint{4.273799in}{2.331163in}}%
\pgfusepath{clip}%
\pgfsetroundcap%
\pgfsetroundjoin%
\pgfsetlinewidth{0.301125pt}%
\definecolor{currentstroke}{rgb}{0.500000,0.500000,0.500000}%
\pgfsetstrokecolor{currentstroke}%
\pgfsetstrokeopacity{0.300000}%
\pgfsetdash{}{0pt}%
\pgfpathmoveto{\pgfqpoint{2.673266in}{1.416861in}}%
\pgfusepath{stroke}%
\end{pgfscope}%
\begin{pgfscope}%
\pgfpathrectangle{\pgfqpoint{0.647939in}{0.492442in}}{\pgfqpoint{4.273799in}{2.331163in}}%
\pgfusepath{clip}%
\pgfsetroundcap%
\pgfsetroundjoin%
\definecolor{currentfill}{rgb}{0.500000,0.500000,0.500000}%
\pgfsetfillcolor{currentfill}%
\pgfsetfillopacity{0.300000}%
\pgfsetlinewidth{0.301125pt}%
\definecolor{currentstroke}{rgb}{0.500000,0.500000,0.500000}%
\pgfsetstrokecolor{currentstroke}%
\pgfsetstrokeopacity{0.300000}%
\pgfsetdash{}{0pt}%
\pgfpathmoveto{\pgfqpoint{0.000000in}{0.000000in}}%
\pgfpathlineto{\pgfqpoint{0.000000in}{0.000000in}}%
\pgfpathclose%
\pgfusepath{stroke,fill}%
\end{pgfscope}%
\begin{pgfscope}%
\pgfpathrectangle{\pgfqpoint{0.647939in}{0.492442in}}{\pgfqpoint{4.273799in}{2.331163in}}%
\pgfusepath{clip}%
\pgfsetbuttcap%
\pgfsetroundjoin%
\pgfsetlinewidth{0.301125pt}%
\definecolor{currentstroke}{rgb}{0.500000,0.500000,0.500000}%
\pgfsetstrokecolor{currentstroke}%
\pgfsetstrokeopacity{0.300000}%
\pgfsetdash{}{0pt}%
\pgfpathmoveto{\pgfqpoint{2.230900in}{0.492442in}}%
\pgfpathlineto{\pgfqpoint{2.228012in}{0.496567in}}%
\pgfpathlineto{\pgfqpoint{2.194277in}{0.544992in}}%
\pgfpathlineto{\pgfqpoint{2.160839in}{0.593479in}}%
\pgfpathlineto{\pgfqpoint{2.127666in}{0.642019in}}%
\pgfpathlineto{\pgfqpoint{2.094726in}{0.690607in}}%
\pgfpathlineto{\pgfqpoint{2.061992in}{0.739237in}}%
\pgfpathlineto{\pgfqpoint{2.029433in}{0.787901in}}%
\pgfpathlineto{\pgfqpoint{1.997005in}{0.836591in}}%
\pgfpathlineto{\pgfqpoint{1.964655in}{0.885297in}}%
\pgfpathlineto{\pgfqpoint{1.932337in}{0.934008in}}%
\pgfpathlineto{\pgfqpoint{1.899995in}{0.982716in}}%
\pgfpathlineto{\pgfqpoint{1.867545in}{1.031401in}}%
\pgfpathlineto{\pgfqpoint{1.834884in}{1.080045in}}%
\pgfpathlineto{\pgfqpoint{1.801899in}{1.128622in}}%
\pgfpathlineto{\pgfqpoint{1.768438in}{1.177103in}}%
\pgfpathlineto{\pgfqpoint{1.734278in}{1.225438in}}%
\pgfpathlineto{\pgfqpoint{1.699096in}{1.273552in}}%
\pgfpathlineto{\pgfqpoint{1.662424in}{1.321328in}}%
\pgfpathlineto{\pgfqpoint{1.623514in}{1.368565in}}%
\pgfpathlineto{\pgfqpoint{1.581020in}{1.414845in}}%
\pgfpathlineto{\pgfqpoint{1.535495in}{1.456486in}}%
\pgfpathlineto{\pgfqpoint{1.496842in}{1.483961in}}%
\pgfpathlineto{\pgfqpoint{1.462672in}{1.501223in}}%
\pgfpathlineto{\pgfqpoint{1.426892in}{1.511477in}}%
\pgfpathlineto{\pgfqpoint{1.386659in}{1.512964in}}%
\pgfpathlineto{\pgfqpoint{1.386659in}{1.512964in}}%
\pgfpathlineto{\pgfqpoint{1.342856in}{1.502578in}}%
\pgfpathlineto{\pgfqpoint{1.342856in}{1.502578in}}%
\pgfpathlineto{\pgfqpoint{1.277727in}{1.465773in}}%
\pgfpathlineto{\pgfqpoint{1.226598in}{1.422442in}}%
\pgfpathlineto{\pgfqpoint{1.182447in}{1.376719in}}%
\pgfpathlineto{\pgfqpoint{1.142487in}{1.329809in}}%
\pgfpathlineto{\pgfqpoint{1.105344in}{1.282188in}}%
\pgfpathlineto{\pgfqpoint{1.070239in}{1.234089in}}%
\pgfpathlineto{\pgfqpoint{1.036692in}{1.185640in}}%
\pgfpathlineto{\pgfqpoint{1.004423in}{1.136930in}}%
\pgfpathlineto{\pgfqpoint{0.973233in}{1.088019in}}%
\pgfpathlineto{\pgfqpoint{0.942939in}{1.038937in}}%
\pgfpathlineto{\pgfqpoint{0.913421in}{0.989708in}}%
\pgfpathlineto{\pgfqpoint{0.884603in}{0.940357in}}%
\pgfpathlineto{\pgfqpoint{0.856408in}{0.890898in}}%
\pgfpathlineto{\pgfqpoint{0.828773in}{0.841343in}}%
\pgfpathlineto{\pgfqpoint{0.801660in}{0.791703in}}%
\pgfpathlineto{\pgfqpoint{0.775019in}{0.741985in}}%
\pgfpathlineto{\pgfqpoint{0.748820in}{0.692196in}}%
\pgfpathlineto{\pgfqpoint{0.723041in}{0.642344in}}%
\pgfpathlineto{\pgfqpoint{0.697648in}{0.592431in}}%
\pgfpathlineto{\pgfqpoint{0.672622in}{0.542463in}}%
\pgfpathlineto{\pgfqpoint{0.647939in}{0.492442in}}%
\pgfpathlineto{\pgfqpoint{0.647939in}{0.492442in}}%
\pgfusepath{stroke}%
\end{pgfscope}%
\begin{pgfscope}%
\pgfpathrectangle{\pgfqpoint{0.647939in}{0.492442in}}{\pgfqpoint{4.273799in}{2.331163in}}%
\pgfusepath{clip}%
\pgfsetbuttcap%
\pgfsetroundjoin%
\pgfsetlinewidth{0.301125pt}%
\definecolor{currentstroke}{rgb}{0.500000,0.500000,0.500000}%
\pgfsetstrokecolor{currentstroke}%
\pgfsetstrokeopacity{0.300000}%
\pgfsetdash{}{0pt}%
\pgfpathmoveto{\pgfqpoint{1.769281in}{0.492442in}}%
\pgfpathlineto{\pgfqpoint{1.740641in}{0.527165in}}%
\pgfpathlineto{\pgfqpoint{1.701158in}{0.574278in}}%
\pgfpathlineto{\pgfqpoint{1.660470in}{0.621082in}}%
\pgfpathlineto{\pgfqpoint{1.618195in}{0.667463in}}%
\pgfpathlineto{\pgfqpoint{1.573792in}{0.713244in}}%
\pgfpathlineto{\pgfqpoint{1.526458in}{0.758131in}}%
\pgfpathlineto{\pgfqpoint{1.474920in}{0.801597in}}%
\pgfpathlineto{\pgfqpoint{1.417015in}{0.842525in}}%
\pgfpathlineto{\pgfqpoint{1.365171in}{0.871085in}}%
\pgfpathlineto{\pgfqpoint{1.316995in}{0.889952in}}%
\pgfpathlineto{\pgfqpoint{1.268040in}{0.900917in}}%
\pgfpathlineto{\pgfqpoint{1.210230in}{0.902585in}}%
\pgfpathlineto{\pgfqpoint{1.155180in}{0.892473in}}%
\pgfpathlineto{\pgfqpoint{1.155180in}{0.892473in}}%
\pgfpathlineto{\pgfqpoint{1.081476in}{0.860849in}}%
\pgfpathlineto{\pgfqpoint{1.021703in}{0.820828in}}%
\pgfpathlineto{\pgfqpoint{0.970865in}{0.777206in}}%
\pgfpathlineto{\pgfqpoint{0.925816in}{0.731675in}}%
\pgfpathlineto{\pgfqpoint{0.884824in}{0.684992in}}%
\pgfpathlineto{\pgfqpoint{0.846839in}{0.637545in}}%
\pgfpathlineto{\pgfqpoint{0.811174in}{0.589553in}}%
\pgfpathlineto{\pgfqpoint{0.777356in}{0.541154in}}%
\pgfpathlineto{\pgfqpoint{0.745071in}{0.492442in}}%
\pgfpathlineto{\pgfqpoint{0.745071in}{0.492442in}}%
\pgfusepath{stroke}%
\end{pgfscope}%
\begin{pgfscope}%
\pgfpathrectangle{\pgfqpoint{0.647939in}{0.492442in}}{\pgfqpoint{4.273799in}{2.331163in}}%
\pgfusepath{clip}%
\pgfsetbuttcap%
\pgfsetroundjoin%
\pgfsetlinewidth{0.301125pt}%
\definecolor{currentstroke}{rgb}{0.500000,0.500000,0.500000}%
\pgfsetstrokecolor{currentstroke}%
\pgfsetstrokeopacity{0.300000}%
\pgfsetdash{}{0pt}%
\pgfpathmoveto{\pgfqpoint{1.542628in}{0.492442in}}%
\pgfpathlineto{\pgfqpoint{1.504541in}{0.525961in}}%
\pgfpathlineto{\pgfqpoint{1.452302in}{0.569183in}}%
\pgfpathlineto{\pgfqpoint{1.394431in}{0.610156in}}%
\pgfpathlineto{\pgfqpoint{1.337774in}{0.642296in}}%
\pgfpathlineto{\pgfqpoint{1.285603in}{0.664092in}}%
\pgfpathlineto{\pgfqpoint{1.234332in}{0.677545in}}%
\pgfpathlineto{\pgfqpoint{1.177931in}{0.682605in}}%
\pgfpathlineto{\pgfqpoint{1.121408in}{0.677028in}}%
\pgfpathlineto{\pgfqpoint{1.121408in}{0.677028in}}%
\pgfpathlineto{\pgfqpoint{1.061295in}{0.659057in}}%
\pgfpathlineto{\pgfqpoint{1.061295in}{0.659057in}}%
\pgfpathlineto{\pgfqpoint{0.992931in}{0.623618in}}%
\pgfpathlineto{\pgfqpoint{0.935950in}{0.582335in}}%
\pgfpathlineto{\pgfqpoint{0.886502in}{0.538206in}}%
\pgfpathlineto{\pgfqpoint{0.842203in}{0.492442in}}%
\pgfpathlineto{\pgfqpoint{0.842203in}{0.492442in}}%
\pgfusepath{stroke}%
\end{pgfscope}%
\begin{pgfscope}%
\pgfpathrectangle{\pgfqpoint{0.647939in}{0.492442in}}{\pgfqpoint{4.273799in}{2.331163in}}%
\pgfusepath{clip}%
\pgfsetbuttcap%
\pgfsetroundjoin%
\pgfsetlinewidth{0.301125pt}%
\definecolor{currentstroke}{rgb}{0.500000,0.500000,0.500000}%
\pgfsetstrokecolor{currentstroke}%
\pgfsetstrokeopacity{0.300000}%
\pgfsetdash{}{0pt}%
\pgfpathmoveto{\pgfqpoint{1.378156in}{0.492442in}}%
\pgfpathlineto{\pgfqpoint{1.351429in}{0.507724in}}%
\pgfpathlineto{\pgfqpoint{1.295311in}{0.535688in}}%
\pgfpathlineto{\pgfqpoint{1.242532in}{0.554211in}}%
\pgfpathlineto{\pgfqpoint{1.188586in}{0.564707in}}%
\pgfpathlineto{\pgfqpoint{1.126658in}{0.565605in}}%
\pgfpathlineto{\pgfqpoint{1.067532in}{0.554937in}}%
\pgfpathlineto{\pgfqpoint{1.067532in}{0.554937in}}%
\pgfpathlineto{\pgfqpoint{1.003440in}{0.530363in}}%
\pgfpathlineto{\pgfqpoint{1.003440in}{0.530363in}}%
\pgfpathlineto{\pgfqpoint{0.939334in}{0.492442in}}%
\pgfpathlineto{\pgfqpoint{0.939334in}{0.492442in}}%
\pgfusepath{stroke}%
\end{pgfscope}%
\begin{pgfscope}%
\pgfpathrectangle{\pgfqpoint{0.647939in}{0.492442in}}{\pgfqpoint{4.273799in}{2.331163in}}%
\pgfusepath{clip}%
\pgfsetbuttcap%
\pgfsetroundjoin%
\pgfsetlinewidth{0.301125pt}%
\definecolor{currentstroke}{rgb}{0.500000,0.500000,0.500000}%
\pgfsetstrokecolor{currentstroke}%
\pgfsetstrokeopacity{0.300000}%
\pgfsetdash{}{0pt}%
\pgfpathmoveto{\pgfqpoint{1.619257in}{0.492442in}}%
\pgfpathlineto{\pgfqpoint{1.619257in}{0.492442in}}%
\pgfpathlineto{\pgfqpoint{1.574786in}{0.538206in}}%
\pgfpathlineto{\pgfqpoint{1.527767in}{0.583199in}}%
\pgfpathlineto{\pgfqpoint{1.477226in}{0.627028in}}%
\pgfpathlineto{\pgfqpoint{1.421610in}{0.668968in}}%
\pgfpathlineto{\pgfqpoint{1.358286in}{0.707378in}}%
\pgfpathlineto{\pgfqpoint{1.283011in}{0.738197in}}%
\pgfpathlineto{\pgfqpoint{1.283011in}{0.738197in}}%
\pgfpathlineto{\pgfqpoint{1.226164in}{0.749660in}}%
\pgfpathlineto{\pgfqpoint{1.164489in}{0.749690in}}%
\pgfpathlineto{\pgfqpoint{1.113664in}{0.739788in}}%
\pgfpathlineto{\pgfqpoint{1.066858in}{0.722627in}}%
\pgfpathlineto{\pgfqpoint{1.018362in}{0.696792in}}%
\pgfpathlineto{\pgfqpoint{0.966221in}{0.660122in}}%
\pgfusepath{stroke}%
\end{pgfscope}%
\begin{pgfscope}%
\pgfpathrectangle{\pgfqpoint{0.647939in}{0.492442in}}{\pgfqpoint{4.273799in}{2.331163in}}%
\pgfusepath{clip}%
\pgfsetbuttcap%
\pgfsetroundjoin%
\pgfsetlinewidth{0.301125pt}%
\definecolor{currentstroke}{rgb}{0.500000,0.500000,0.500000}%
\pgfsetstrokecolor{currentstroke}%
\pgfsetstrokeopacity{0.300000}%
\pgfsetdash{}{0pt}%
\pgfpathmoveto{\pgfqpoint{1.910652in}{0.492442in}}%
\pgfpathlineto{\pgfqpoint{1.910652in}{0.492442in}}%
\pgfpathlineto{\pgfqpoint{1.874820in}{0.540417in}}%
\pgfpathlineto{\pgfqpoint{1.838652in}{0.588316in}}%
\pgfpathlineto{\pgfqpoint{1.802028in}{0.636112in}}%
\pgfpathlineto{\pgfqpoint{1.764798in}{0.683768in}}%
\pgfpathlineto{\pgfqpoint{1.726768in}{0.731236in}}%
\pgfpathlineto{\pgfqpoint{1.687672in}{0.778445in}}%
\pgfpathlineto{\pgfqpoint{1.647149in}{0.825293in}}%
\pgfpathlineto{\pgfqpoint{1.604696in}{0.871625in}}%
\pgfpathlineto{\pgfqpoint{1.559555in}{0.917188in}}%
\pgfpathlineto{\pgfqpoint{1.510517in}{0.961525in}}%
\pgfpathlineto{\pgfqpoint{1.455469in}{1.003649in}}%
\pgfpathlineto{\pgfqpoint{1.390483in}{1.041046in}}%
\pgfpathlineto{\pgfqpoint{1.390483in}{1.041046in}}%
\pgfpathlineto{\pgfqpoint{1.334514in}{1.061139in}}%
\pgfpathlineto{\pgfqpoint{1.334514in}{1.061139in}}%
\pgfpathlineto{\pgfqpoint{1.284076in}{1.068623in}}%
\pgfpathlineto{\pgfqpoint{1.231976in}{1.065243in}}%
\pgfpathlineto{\pgfqpoint{1.188064in}{1.053653in}}%
\pgfpathlineto{\pgfqpoint{1.144942in}{1.034585in}}%
\pgfpathlineto{\pgfqpoint{1.098970in}{1.006170in}}%
\pgfpathlineto{\pgfqpoint{1.048166in}{0.965627in}}%
\pgfpathlineto{\pgfqpoint{1.001506in}{0.920601in}}%
\pgfpathlineto{\pgfqpoint{0.959492in}{0.874191in}}%
\pgfpathlineto{\pgfqpoint{0.920803in}{0.826917in}}%
\pgfpathlineto{\pgfqpoint{0.884618in}{0.779050in}}%
\pgfusepath{stroke}%
\end{pgfscope}%
\begin{pgfscope}%
\pgfpathrectangle{\pgfqpoint{0.647939in}{0.492442in}}{\pgfqpoint{4.273799in}{2.331163in}}%
\pgfusepath{clip}%
\pgfsetbuttcap%
\pgfsetroundjoin%
\pgfsetlinewidth{0.301125pt}%
\definecolor{currentstroke}{rgb}{0.500000,0.500000,0.500000}%
\pgfsetstrokecolor{currentstroke}%
\pgfsetstrokeopacity{0.300000}%
\pgfsetdash{}{0pt}%
\pgfpathmoveto{\pgfqpoint{2.007784in}{0.492442in}}%
\pgfpathlineto{\pgfqpoint{2.007784in}{0.492442in}}%
\pgfpathlineto{\pgfqpoint{1.973117in}{0.540671in}}%
\pgfpathlineto{\pgfqpoint{1.938381in}{0.588885in}}%
\pgfpathlineto{\pgfqpoint{1.903507in}{0.637070in}}%
\pgfpathlineto{\pgfqpoint{1.868402in}{0.685204in}}%
\pgfpathlineto{\pgfqpoint{1.832959in}{0.733265in}}%
\pgfpathlineto{\pgfqpoint{1.797053in}{0.781222in}}%
\pgfpathlineto{\pgfqpoint{1.760521in}{0.829039in}}%
\pgfpathlineto{\pgfqpoint{1.723153in}{0.876662in}}%
\pgfpathlineto{\pgfqpoint{1.684658in}{0.924017in}}%
\pgfpathlineto{\pgfqpoint{1.644621in}{0.970989in}}%
\pgfpathlineto{\pgfqpoint{1.602434in}{1.017392in}}%
\pgfpathlineto{\pgfqpoint{1.557155in}{1.062912in}}%
\pgfpathlineto{\pgfqpoint{1.507178in}{1.106923in}}%
\pgfpathlineto{\pgfqpoint{1.449479in}{1.147884in}}%
\pgfpathlineto{\pgfqpoint{1.378111in}{1.181205in}}%
\pgfpathlineto{\pgfqpoint{1.378111in}{1.181205in}}%
\pgfpathlineto{\pgfqpoint{1.329350in}{1.191632in}}%
\pgfpathlineto{\pgfqpoint{1.276229in}{1.190623in}}%
\pgfpathlineto{\pgfqpoint{1.233600in}{1.180570in}}%
\pgfpathlineto{\pgfqpoint{1.192736in}{1.163336in}}%
\pgfpathlineto{\pgfqpoint{1.149175in}{1.137126in}}%
\pgfpathlineto{\pgfqpoint{1.100812in}{1.099190in}}%
\pgfpathlineto{\pgfqpoint{1.053875in}{1.054292in}}%
\pgfpathlineto{\pgfqpoint{1.011785in}{1.007927in}}%
\pgfusepath{stroke}%
\end{pgfscope}%
\begin{pgfscope}%
\pgfpathrectangle{\pgfqpoint{0.647939in}{0.492442in}}{\pgfqpoint{4.273799in}{2.331163in}}%
\pgfusepath{clip}%
\pgfsetbuttcap%
\pgfsetroundjoin%
\pgfsetlinewidth{0.301125pt}%
\definecolor{currentstroke}{rgb}{0.500000,0.500000,0.500000}%
\pgfsetstrokecolor{currentstroke}%
\pgfsetstrokeopacity{0.300000}%
\pgfsetdash{}{0pt}%
\pgfpathmoveto{\pgfqpoint{2.104916in}{0.492442in}}%
\pgfpathlineto{\pgfqpoint{2.104916in}{0.492442in}}%
\pgfpathlineto{\pgfqpoint{2.070910in}{0.540811in}}%
\pgfpathlineto{\pgfqpoint{2.037021in}{0.589204in}}%
\pgfpathlineto{\pgfqpoint{2.003199in}{0.637612in}}%
\pgfpathlineto{\pgfqpoint{1.969396in}{0.686023in}}%
\pgfpathlineto{\pgfqpoint{1.935555in}{0.734426in}}%
\pgfpathlineto{\pgfqpoint{1.901611in}{0.782808in}}%
\pgfpathlineto{\pgfqpoint{1.867471in}{0.831148in}}%
\pgfpathlineto{\pgfqpoint{1.833028in}{0.879425in}}%
\pgfpathlineto{\pgfqpoint{1.798155in}{0.927608in}}%
\pgfpathlineto{\pgfqpoint{1.762689in}{0.975663in}}%
\pgfpathlineto{\pgfqpoint{1.726413in}{1.023537in}}%
\pgfpathlineto{\pgfqpoint{1.689010in}{1.071152in}}%
\pgfpathlineto{\pgfqpoint{1.650026in}{1.118384in}}%
\pgfpathlineto{\pgfqpoint{1.608782in}{1.165035in}}%
\pgfpathlineto{\pgfqpoint{1.564165in}{1.210736in}}%
\pgfpathlineto{\pgfqpoint{1.514148in}{1.254706in}}%
\pgfpathlineto{\pgfqpoint{1.454581in}{1.294764in}}%
\pgfpathlineto{\pgfqpoint{1.454581in}{1.294764in}}%
\pgfpathlineto{\pgfqpoint{1.403433in}{1.316592in}}%
\pgfpathlineto{\pgfqpoint{1.403433in}{1.316592in}}%
\pgfpathlineto{\pgfqpoint{1.358086in}{1.325129in}}%
\pgfpathlineto{\pgfqpoint{1.309762in}{1.322488in}}%
\pgfpathlineto{\pgfqpoint{1.270575in}{1.311659in}}%
\pgfpathlineto{\pgfqpoint{1.231969in}{1.293622in}}%
\pgfpathlineto{\pgfqpoint{1.189784in}{1.266087in}}%
\pgfpathlineto{\pgfqpoint{1.142035in}{1.225974in}}%
\pgfpathlineto{\pgfqpoint{1.097023in}{1.180486in}}%
\pgfusepath{stroke}%
\end{pgfscope}%
\begin{pgfscope}%
\pgfpathrectangle{\pgfqpoint{0.647939in}{0.492442in}}{\pgfqpoint{4.273799in}{2.331163in}}%
\pgfusepath{clip}%
\pgfsetbuttcap%
\pgfsetroundjoin%
\pgfsetlinewidth{0.301125pt}%
\definecolor{currentstroke}{rgb}{0.500000,0.500000,0.500000}%
\pgfsetstrokecolor{currentstroke}%
\pgfsetstrokeopacity{0.300000}%
\pgfsetdash{}{0pt}%
\pgfpathmoveto{\pgfqpoint{2.396312in}{0.492442in}}%
\pgfpathlineto{\pgfqpoint{2.396312in}{0.492442in}}%
\pgfpathlineto{\pgfqpoint{2.362026in}{0.540752in}}%
\pgfpathlineto{\pgfqpoint{2.328217in}{0.589162in}}%
\pgfpathlineto{\pgfqpoint{2.294869in}{0.637667in}}%
\pgfpathlineto{\pgfqpoint{2.261960in}{0.686261in}}%
\pgfpathlineto{\pgfqpoint{2.229475in}{0.734940in}}%
\pgfpathlineto{\pgfqpoint{2.197400in}{0.783700in}}%
\pgfpathlineto{\pgfqpoint{2.165720in}{0.832536in}}%
\pgfpathlineto{\pgfqpoint{2.134411in}{0.881444in}}%
\pgfpathlineto{\pgfqpoint{2.103458in}{0.930419in}}%
\pgfpathlineto{\pgfqpoint{2.072850in}{0.979458in}}%
\pgfpathlineto{\pgfqpoint{2.042561in}{1.028556in}}%
\pgfpathlineto{\pgfqpoint{2.012567in}{1.077708in}}%
\pgfpathlineto{\pgfqpoint{1.982854in}{1.126912in}}%
\pgfpathlineto{\pgfqpoint{1.953398in}{1.176160in}}%
\pgfpathlineto{\pgfqpoint{1.924164in}{1.225448in}}%
\pgfpathlineto{\pgfqpoint{1.895127in}{1.274771in}}%
\pgfpathlineto{\pgfqpoint{1.866252in}{1.324122in}}%
\pgfpathlineto{\pgfqpoint{1.837483in}{1.373490in}}%
\pgfpathlineto{\pgfqpoint{1.808769in}{1.422868in}}%
\pgfpathlineto{\pgfqpoint{1.780031in}{1.472242in}}%
\pgfpathlineto{\pgfqpoint{1.751146in}{1.521589in}}%
\pgfpathlineto{\pgfqpoint{1.721962in}{1.570882in}}%
\pgfpathlineto{\pgfqpoint{1.692217in}{1.620075in}}%
\pgfpathlineto{\pgfqpoint{1.661433in}{1.669068in}}%
\pgfpathlineto{\pgfqpoint{1.628697in}{1.717665in}}%
\pgfpathlineto{\pgfqpoint{1.591721in}{1.765306in}}%
\pgfpathlineto{\pgfqpoint{1.541601in}{1.808415in}}%
\pgfpathlineto{\pgfqpoint{1.541601in}{1.808415in}}%
\pgfpathlineto{\pgfqpoint{1.514551in}{1.817028in}}%
\pgfpathlineto{\pgfqpoint{1.514551in}{1.817028in}}%
\pgfpathlineto{\pgfqpoint{1.490206in}{1.815049in}}%
\pgfpathlineto{\pgfqpoint{1.469295in}{1.806948in}}%
\pgfpathlineto{\pgfqpoint{1.445071in}{1.791586in}}%
\pgfpathlineto{\pgfqpoint{1.415653in}{1.766423in}}%
\pgfpathlineto{\pgfqpoint{1.374450in}{1.723359in}}%
\pgfpathlineto{\pgfqpoint{1.334724in}{1.676429in}}%
\pgfpathlineto{\pgfqpoint{1.297307in}{1.628857in}}%
\pgfpathlineto{\pgfqpoint{1.261537in}{1.580915in}}%
\pgfpathlineto{\pgfqpoint{1.227035in}{1.532715in}}%
\pgfpathlineto{\pgfqpoint{1.193536in}{1.484295in}}%
\pgfpathlineto{\pgfqpoint{1.160870in}{1.435682in}}%
\pgfpathlineto{\pgfqpoint{1.128952in}{1.386908in}}%
\pgfusepath{stroke}%
\end{pgfscope}%
\begin{pgfscope}%
\pgfpathrectangle{\pgfqpoint{0.647939in}{0.492442in}}{\pgfqpoint{4.273799in}{2.331163in}}%
\pgfusepath{clip}%
\pgfsetbuttcap%
\pgfsetroundjoin%
\pgfsetlinewidth{0.301125pt}%
\definecolor{currentstroke}{rgb}{0.500000,0.500000,0.500000}%
\pgfsetstrokecolor{currentstroke}%
\pgfsetstrokeopacity{0.300000}%
\pgfsetdash{}{0pt}%
\pgfpathmoveto{\pgfqpoint{2.493443in}{0.492442in}}%
\pgfpathlineto{\pgfqpoint{2.493443in}{0.492442in}}%
\pgfpathlineto{\pgfqpoint{2.458459in}{0.540603in}}%
\pgfpathlineto{\pgfqpoint{2.424042in}{0.588885in}}%
\pgfpathlineto{\pgfqpoint{2.390177in}{0.637283in}}%
\pgfpathlineto{\pgfqpoint{2.356849in}{0.685792in}}%
\pgfpathlineto{\pgfqpoint{2.324049in}{0.734408in}}%
\pgfpathlineto{\pgfqpoint{2.291766in}{0.783126in}}%
\pgfpathlineto{\pgfqpoint{2.259984in}{0.831943in}}%
\pgfpathlineto{\pgfqpoint{2.228691in}{0.880854in}}%
\pgfpathlineto{\pgfqpoint{2.197883in}{0.929856in}}%
\pgfpathlineto{\pgfqpoint{2.167551in}{0.978946in}}%
\pgfpathlineto{\pgfqpoint{2.137682in}{1.028121in}}%
\pgfpathlineto{\pgfqpoint{2.108273in}{1.077378in}}%
\pgfpathlineto{\pgfqpoint{2.079325in}{1.126716in}}%
\pgfpathlineto{\pgfqpoint{2.050827in}{1.176133in}}%
\pgfpathlineto{\pgfqpoint{2.022784in}{1.225626in}}%
\pgfpathlineto{\pgfqpoint{1.995201in}{1.275197in}}%
\pgfpathlineto{\pgfqpoint{1.968079in}{1.324843in}}%
\pgfpathlineto{\pgfqpoint{1.941437in}{1.374566in}}%
\pgfpathlineto{\pgfqpoint{1.915290in}{1.424367in}}%
\pgfpathlineto{\pgfqpoint{1.889659in}{1.474247in}}%
\pgfpathlineto{\pgfqpoint{1.864591in}{1.524213in}}%
\pgfpathlineto{\pgfqpoint{1.840136in}{1.574268in}}%
\pgfpathlineto{\pgfqpoint{1.816383in}{1.624424in}}%
\pgfpathlineto{\pgfqpoint{1.793446in}{1.674692in}}%
\pgfpathlineto{\pgfqpoint{1.771515in}{1.725093in}}%
\pgfpathlineto{\pgfqpoint{1.750873in}{1.775653in}}%
\pgfpathlineto{\pgfqpoint{1.731993in}{1.826415in}}%
\pgfpathlineto{\pgfqpoint{1.715700in}{1.877439in}}%
\pgfpathlineto{\pgfqpoint{1.703473in}{1.928788in}}%
\pgfpathlineto{\pgfqpoint{1.698034in}{1.980449in}}%
\pgfpathlineto{\pgfqpoint{1.703591in}{2.031998in}}%
\pgfpathlineto{\pgfqpoint{1.723314in}{2.082353in}}%
\pgfpathlineto{\pgfqpoint{1.755601in}{2.130760in}}%
\pgfpathlineto{\pgfqpoint{1.796652in}{2.177209in}}%
\pgfpathlineto{\pgfqpoint{1.843712in}{2.221995in}}%
\pgfpathlineto{\pgfqpoint{1.895434in}{2.265292in}}%
\pgfpathlineto{\pgfqpoint{1.951389in}{2.307060in}}%
\pgfpathlineto{\pgfqpoint{2.011646in}{2.347017in}}%
\pgfpathlineto{\pgfqpoint{2.076417in}{2.384789in}}%
\pgfpathlineto{\pgfqpoint{2.146311in}{2.419739in}}%
\pgfpathlineto{\pgfqpoint{2.221913in}{2.450904in}}%
\pgfpathlineto{\pgfqpoint{2.303673in}{2.476923in}}%
\pgfpathlineto{\pgfqpoint{2.391506in}{2.495936in}}%
\pgfpathlineto{\pgfqpoint{2.484061in}{2.505808in}}%
\pgfpathlineto{\pgfqpoint{2.573873in}{2.505254in}}%
\pgfpathlineto{\pgfqpoint{2.657722in}{2.495435in}}%
\pgfpathlineto{\pgfqpoint{2.737724in}{2.477282in}}%
\pgfpathlineto{\pgfqpoint{2.815677in}{2.450835in}}%
\pgfpathlineto{\pgfqpoint{2.888404in}{2.417759in}}%
\pgfpathlineto{\pgfqpoint{2.953793in}{2.380354in}}%
\pgfpathlineto{\pgfqpoint{3.012467in}{2.339734in}}%
\pgfpathlineto{\pgfqpoint{3.064998in}{2.296648in}}%
\pgfpathlineto{\pgfqpoint{3.111772in}{2.251609in}}%
\pgfpathlineto{\pgfqpoint{3.152946in}{2.204968in}}%
\pgfpathlineto{\pgfqpoint{3.188416in}{2.156951in}}%
\pgfpathlineto{\pgfqpoint{3.217698in}{2.107724in}}%
\pgfpathlineto{\pgfqpoint{3.239663in}{2.057395in}}%
\pgfpathlineto{\pgfqpoint{3.251852in}{2.006138in}}%
\pgfpathlineto{\pgfqpoint{3.248002in}{1.954722in}}%
\pgfpathlineto{\pgfqpoint{3.248002in}{1.954722in}}%
\pgfpathlineto{\pgfqpoint{3.232682in}{1.927519in}}%
\pgfpathlineto{\pgfqpoint{3.232682in}{1.927519in}}%
\pgfpathlineto{\pgfqpoint{3.210752in}{1.912687in}}%
\pgfpathlineto{\pgfqpoint{3.210752in}{1.912687in}}%
\pgfpathlineto{\pgfqpoint{3.184481in}{1.907989in}}%
\pgfpathlineto{\pgfqpoint{3.157599in}{1.911863in}}%
\pgfpathlineto{\pgfqpoint{3.133001in}{1.921822in}}%
\pgfpathlineto{\pgfqpoint{3.108066in}{1.939172in}}%
\pgfpathlineto{\pgfqpoint{3.086752in}{1.964959in}}%
\pgfpathlineto{\pgfqpoint{3.086752in}{1.964959in}}%
\pgfpathlineto{\pgfqpoint{3.080774in}{1.987971in}}%
\pgfpathlineto{\pgfqpoint{3.080774in}{1.987971in}}%
\pgfpathlineto{\pgfqpoint{3.086900in}{1.995870in}}%
\pgfpathlineto{\pgfqpoint{3.096114in}{1.994724in}}%
\pgfpathlineto{\pgfqpoint{3.102798in}{1.988697in}}%
\pgfpathlineto{\pgfqpoint{3.100048in}{1.983554in}}%
\pgfpathlineto{\pgfqpoint{3.100509in}{1.987970in}}%
\pgfpathlineto{\pgfqpoint{3.101297in}{1.983733in}}%
\pgfpathlineto{\pgfqpoint{3.102880in}{1.990591in}}%
\pgfpathlineto{\pgfqpoint{3.102880in}{1.990591in}}%
\pgfpathlineto{\pgfqpoint{3.099908in}{1.981486in}}%
\pgfpathlineto{\pgfqpoint{3.099007in}{1.988486in}}%
\pgfpathlineto{\pgfqpoint{3.102277in}{1.985444in}}%
\pgfpathlineto{\pgfqpoint{3.098890in}{1.987818in}}%
\pgfpathlineto{\pgfqpoint{3.104608in}{1.981943in}}%
\pgfpathlineto{\pgfqpoint{3.104608in}{1.981943in}}%
\pgfpathlineto{\pgfqpoint{3.097959in}{1.988834in}}%
\pgfpathlineto{\pgfqpoint{3.102298in}{1.986034in}}%
\pgfpathlineto{\pgfqpoint{3.099370in}{1.986439in}}%
\pgfpathlineto{\pgfqpoint{3.103720in}{1.985028in}}%
\pgfpathlineto{\pgfqpoint{3.103720in}{1.985028in}}%
\pgfpathlineto{\pgfqpoint{3.098751in}{1.986888in}}%
\pgfpathlineto{\pgfqpoint{3.101730in}{1.986467in}}%
\pgfpathlineto{\pgfqpoint{3.099609in}{1.985958in}}%
\pgfpathlineto{\pgfqpoint{3.105226in}{1.984835in}}%
\pgfpathlineto{\pgfqpoint{3.105226in}{1.984835in}}%
\pgfpathlineto{\pgfqpoint{3.099952in}{1.985277in}}%
\pgfpathlineto{\pgfqpoint{3.100274in}{1.986761in}}%
\pgfpathlineto{\pgfqpoint{3.102282in}{1.984517in}}%
\pgfpathlineto{\pgfqpoint{3.099123in}{1.991011in}}%
\pgfpathlineto{\pgfqpoint{3.099123in}{1.991011in}}%
\pgfpathlineto{\pgfqpoint{3.103003in}{1.981200in}}%
\pgfpathlineto{\pgfqpoint{3.098752in}{1.989993in}}%
\pgfpathlineto{\pgfqpoint{3.103290in}{1.982215in}}%
\pgfpathlineto{\pgfqpoint{3.098679in}{1.990119in}}%
\pgfpathlineto{\pgfqpoint{3.103432in}{1.981358in}}%
\pgfpathlineto{\pgfqpoint{3.098438in}{1.990598in}}%
\pgfpathlineto{\pgfqpoint{3.103600in}{1.981789in}}%
\pgfpathlineto{\pgfqpoint{3.098422in}{1.989814in}}%
\pgfpathlineto{\pgfqpoint{3.103453in}{1.983020in}}%
\pgfpathlineto{\pgfqpoint{3.098508in}{1.989487in}}%
\pgfpathlineto{\pgfqpoint{3.103447in}{1.982980in}}%
\pgfpathlineto{\pgfqpoint{3.098443in}{1.989773in}}%
\pgfpathlineto{\pgfqpoint{3.103556in}{1.982568in}}%
\pgfpathlineto{\pgfqpoint{3.098408in}{1.989868in}}%
\pgfpathlineto{\pgfqpoint{3.103532in}{1.982728in}}%
\pgfpathlineto{\pgfqpoint{3.098451in}{1.989666in}}%
\pgfpathlineto{\pgfqpoint{3.103473in}{1.982941in}}%
\pgfpathlineto{\pgfqpoint{3.098460in}{1.989655in}}%
\pgfusepath{stroke}%
\end{pgfscope}%
\begin{pgfscope}%
\pgfpathrectangle{\pgfqpoint{0.647939in}{0.492442in}}{\pgfqpoint{4.273799in}{2.331163in}}%
\pgfusepath{clip}%
\pgfsetbuttcap%
\pgfsetroundjoin%
\pgfsetlinewidth{0.301125pt}%
\definecolor{currentstroke}{rgb}{0.500000,0.500000,0.500000}%
\pgfsetstrokecolor{currentstroke}%
\pgfsetstrokeopacity{0.300000}%
\pgfsetdash{}{0pt}%
\pgfpathmoveto{\pgfqpoint{2.590575in}{0.492442in}}%
\pgfpathlineto{\pgfqpoint{2.590575in}{0.492442in}}%
\pgfpathlineto{\pgfqpoint{2.554621in}{0.540390in}}%
\pgfpathlineto{\pgfqpoint{2.519319in}{0.588481in}}%
\pgfpathlineto{\pgfqpoint{2.484658in}{0.636711in}}%
\pgfpathlineto{\pgfqpoint{2.450627in}{0.685074in}}%
\pgfpathlineto{\pgfqpoint{2.417218in}{0.733567in}}%
\pgfpathlineto{\pgfqpoint{2.384418in}{0.782182in}}%
\pgfpathlineto{\pgfqpoint{2.352214in}{0.830916in}}%
\pgfpathlineto{\pgfqpoint{2.320601in}{0.879766in}}%
\pgfpathlineto{\pgfqpoint{2.289579in}{0.928728in}}%
\pgfpathlineto{\pgfqpoint{2.259139in}{0.977798in}}%
\pgfpathlineto{\pgfqpoint{2.229275in}{1.026973in}}%
\pgfpathlineto{\pgfqpoint{2.199996in}{1.076254in}}%
\pgfpathlineto{\pgfqpoint{2.171302in}{1.125636in}}%
\pgfpathlineto{\pgfqpoint{2.143198in}{1.175119in}}%
\pgfpathlineto{\pgfqpoint{2.115703in}{1.224704in}}%
\pgfpathlineto{\pgfqpoint{2.088830in}{1.274390in}}%
\pgfpathlineto{\pgfqpoint{2.062597in}{1.324178in}}%
\pgfpathlineto{\pgfqpoint{2.037045in}{1.374071in}}%
\pgfpathlineto{\pgfqpoint{2.012211in}{1.424072in}}%
\pgfpathlineto{\pgfqpoint{1.988157in}{1.474185in}}%
\pgfpathlineto{\pgfqpoint{1.964951in}{1.524418in}}%
\pgfpathlineto{\pgfqpoint{1.942693in}{1.574778in}}%
\pgfpathlineto{\pgfqpoint{1.921512in}{1.625275in}}%
\pgfpathlineto{\pgfqpoint{1.901579in}{1.675923in}}%
\pgfpathlineto{\pgfqpoint{1.883131in}{1.726737in}}%
\pgfpathlineto{\pgfqpoint{1.866482in}{1.777734in}}%
\pgfpathlineto{\pgfqpoint{1.852084in}{1.828933in}}%
\pgfpathlineto{\pgfqpoint{1.840544in}{1.880344in}}%
\pgfpathlineto{\pgfqpoint{1.832704in}{1.931957in}}%
\pgfpathlineto{\pgfqpoint{1.829692in}{1.983710in}}%
\pgfpathlineto{\pgfqpoint{1.832894in}{2.035441in}}%
\pgfpathlineto{\pgfqpoint{1.843757in}{2.086840in}}%
\pgfpathlineto{\pgfqpoint{1.863341in}{2.137449in}}%
\pgfpathlineto{\pgfqpoint{1.891992in}{2.186748in}}%
\pgfusepath{stroke}%
\end{pgfscope}%
\begin{pgfscope}%
\pgfpathrectangle{\pgfqpoint{0.647939in}{0.492442in}}{\pgfqpoint{4.273799in}{2.331163in}}%
\pgfusepath{clip}%
\pgfsetbuttcap%
\pgfsetroundjoin%
\pgfsetlinewidth{0.301125pt}%
\definecolor{currentstroke}{rgb}{0.500000,0.500000,0.500000}%
\pgfsetstrokecolor{currentstroke}%
\pgfsetstrokeopacity{0.300000}%
\pgfsetdash{}{0pt}%
\pgfpathmoveto{\pgfqpoint{2.687707in}{0.492442in}}%
\pgfpathlineto{\pgfqpoint{2.687707in}{0.492442in}}%
\pgfpathlineto{\pgfqpoint{2.650522in}{0.540109in}}%
\pgfpathlineto{\pgfqpoint{2.614076in}{0.587946in}}%
\pgfpathlineto{\pgfqpoint{2.578359in}{0.635946in}}%
\pgfpathlineto{\pgfqpoint{2.543361in}{0.684103in}}%
\pgfpathlineto{\pgfqpoint{2.509071in}{0.732412in}}%
\pgfpathlineto{\pgfqpoint{2.475477in}{0.780866in}}%
\pgfusepath{stroke}%
\end{pgfscope}%
\begin{pgfscope}%
\pgfpathrectangle{\pgfqpoint{0.647939in}{0.492442in}}{\pgfqpoint{4.273799in}{2.331163in}}%
\pgfusepath{clip}%
\pgfsetbuttcap%
\pgfsetroundjoin%
\pgfsetlinewidth{0.301125pt}%
\definecolor{currentstroke}{rgb}{0.500000,0.500000,0.500000}%
\pgfsetstrokecolor{currentstroke}%
\pgfsetstrokeopacity{0.300000}%
\pgfsetdash{}{0pt}%
\pgfpathmoveto{\pgfqpoint{2.881971in}{0.492442in}}%
\pgfpathlineto{\pgfqpoint{2.881971in}{0.492442in}}%
\pgfpathlineto{\pgfqpoint{2.841601in}{0.539332in}}%
\pgfpathlineto{\pgfqpoint{2.802139in}{0.586451in}}%
\pgfpathlineto{\pgfqpoint{2.763575in}{0.633790in}}%
\pgfpathlineto{\pgfqpoint{2.725899in}{0.681343in}}%
\pgfpathlineto{\pgfqpoint{2.689102in}{0.729099in}}%
\pgfpathlineto{\pgfqpoint{2.653173in}{0.777052in}}%
\pgfpathlineto{\pgfqpoint{2.618107in}{0.825194in}}%
\pgfpathlineto{\pgfqpoint{2.583898in}{0.873519in}}%
\pgfpathlineto{\pgfqpoint{2.550542in}{0.922022in}}%
\pgfpathlineto{\pgfqpoint{2.518034in}{0.970696in}}%
\pgfpathlineto{\pgfqpoint{2.486374in}{1.019535in}}%
\pgfpathlineto{\pgfqpoint{2.455568in}{1.068537in}}%
\pgfpathlineto{\pgfqpoint{2.425626in}{1.117698in}}%
\pgfpathlineto{\pgfqpoint{2.396554in}{1.167014in}}%
\pgfpathlineto{\pgfqpoint{2.368374in}{1.216484in}}%
\pgfpathlineto{\pgfqpoint{2.341111in}{1.266107in}}%
\pgfpathlineto{\pgfqpoint{2.314793in}{1.315881in}}%
\pgfpathlineto{\pgfqpoint{2.289464in}{1.365807in}}%
\pgfpathlineto{\pgfqpoint{2.265173in}{1.415887in}}%
\pgfpathlineto{\pgfqpoint{2.241983in}{1.466121in}}%
\pgfpathlineto{\pgfqpoint{2.219973in}{1.516513in}}%
\pgfpathlineto{\pgfqpoint{2.199242in}{1.567066in}}%
\pgfpathlineto{\pgfqpoint{2.179911in}{1.617783in}}%
\pgfpathlineto{\pgfqpoint{2.162130in}{1.668669in}}%
\pgfpathlineto{\pgfqpoint{2.146084in}{1.719725in}}%
\pgfpathlineto{\pgfqpoint{2.132000in}{1.770952in}}%
\pgfpathlineto{\pgfqpoint{2.120167in}{1.822347in}}%
\pgfpathlineto{\pgfqpoint{2.110935in}{1.873898in}}%
\pgfpathlineto{\pgfqpoint{2.104733in}{1.925583in}}%
\pgfpathlineto{\pgfqpoint{2.102092in}{1.977355in}}%
\pgfpathlineto{\pgfqpoint{2.103649in}{2.029135in}}%
\pgfpathlineto{\pgfqpoint{2.110155in}{2.080792in}}%
\pgfpathlineto{\pgfqpoint{2.122469in}{2.132123in}}%
\pgfpathlineto{\pgfqpoint{2.141552in}{2.182823in}}%
\pgfpathlineto{\pgfqpoint{2.168393in}{2.232452in}}%
\pgfpathlineto{\pgfqpoint{2.204000in}{2.280392in}}%
\pgfpathlineto{\pgfqpoint{2.249364in}{2.325779in}}%
\pgfpathlineto{\pgfqpoint{2.305488in}{2.367356in}}%
\pgfpathlineto{\pgfqpoint{2.373162in}{2.403321in}}%
\pgfpathlineto{\pgfqpoint{2.452464in}{2.431068in}}%
\pgfpathlineto{\pgfqpoint{2.533701in}{2.446540in}}%
\pgfpathlineto{\pgfqpoint{2.611788in}{2.450568in}}%
\pgfpathlineto{\pgfqpoint{2.686286in}{2.445100in}}%
\pgfusepath{stroke}%
\end{pgfscope}%
\begin{pgfscope}%
\pgfpathrectangle{\pgfqpoint{0.647939in}{0.492442in}}{\pgfqpoint{4.273799in}{2.331163in}}%
\pgfusepath{clip}%
\pgfsetbuttcap%
\pgfsetroundjoin%
\pgfsetlinewidth{0.301125pt}%
\definecolor{currentstroke}{rgb}{0.500000,0.500000,0.500000}%
\pgfsetstrokecolor{currentstroke}%
\pgfsetstrokeopacity{0.300000}%
\pgfsetdash{}{0pt}%
\pgfpathmoveto{\pgfqpoint{2.979102in}{0.492442in}}%
\pgfpathlineto{\pgfqpoint{2.979102in}{0.492442in}}%
\pgfpathlineto{\pgfqpoint{2.936812in}{0.538826in}}%
\pgfpathlineto{\pgfqpoint{2.895504in}{0.585471in}}%
\pgfpathlineto{\pgfqpoint{2.855172in}{0.632371in}}%
\pgfpathlineto{\pgfqpoint{2.815812in}{0.679515in}}%
\pgfpathlineto{\pgfqpoint{2.777415in}{0.726894in}}%
\pgfpathlineto{\pgfqpoint{2.739974in}{0.774501in}}%
\pgfpathlineto{\pgfqpoint{2.703482in}{0.822327in}}%
\pgfpathlineto{\pgfqpoint{2.667932in}{0.870363in}}%
\pgfusepath{stroke}%
\end{pgfscope}%
\begin{pgfscope}%
\pgfpathrectangle{\pgfqpoint{0.647939in}{0.492442in}}{\pgfqpoint{4.273799in}{2.331163in}}%
\pgfusepath{clip}%
\pgfsetbuttcap%
\pgfsetroundjoin%
\pgfsetlinewidth{0.301125pt}%
\definecolor{currentstroke}{rgb}{0.500000,0.500000,0.500000}%
\pgfsetstrokecolor{currentstroke}%
\pgfsetstrokeopacity{0.300000}%
\pgfsetdash{}{0pt}%
\pgfpathmoveto{\pgfqpoint{3.173366in}{0.492442in}}%
\pgfpathlineto{\pgfqpoint{3.173366in}{0.492442in}}%
\pgfpathlineto{\pgfqpoint{3.126754in}{0.537575in}}%
\pgfpathlineto{\pgfqpoint{3.081219in}{0.583034in}}%
\pgfpathlineto{\pgfqpoint{3.036782in}{0.628815in}}%
\pgfpathlineto{\pgfqpoint{2.993455in}{0.674912in}}%
\pgfpathlineto{\pgfqpoint{2.951246in}{0.721316in}}%
\pgfpathlineto{\pgfqpoint{2.910156in}{0.768018in}}%
\pgfpathlineto{\pgfqpoint{2.870183in}{0.815007in}}%
\pgfpathlineto{\pgfqpoint{2.831324in}{0.862274in}}%
\pgfpathlineto{\pgfqpoint{2.793575in}{0.909808in}}%
\pgfpathlineto{\pgfqpoint{2.756934in}{0.957599in}}%
\pgfpathlineto{\pgfqpoint{2.721398in}{1.005639in}}%
\pgfpathlineto{\pgfqpoint{2.686965in}{1.053916in}}%
\pgfpathlineto{\pgfqpoint{2.653637in}{1.102424in}}%
\pgfpathlineto{\pgfqpoint{2.621422in}{1.151155in}}%
\pgfpathlineto{\pgfqpoint{2.590333in}{1.200103in}}%
\pgfpathlineto{\pgfqpoint{2.560387in}{1.249263in}}%
\pgfpathlineto{\pgfqpoint{2.531603in}{1.298628in}}%
\pgfpathlineto{\pgfqpoint{2.504020in}{1.348198in}}%
\pgfpathlineto{\pgfqpoint{2.477678in}{1.397967in}}%
\pgfpathlineto{\pgfqpoint{2.452627in}{1.447934in}}%
\pgfpathlineto{\pgfqpoint{2.428938in}{1.498099in}}%
\pgfpathlineto{\pgfqpoint{2.406693in}{1.548459in}}%
\pgfpathlineto{\pgfqpoint{2.385996in}{1.599016in}}%
\pgfpathlineto{\pgfqpoint{2.366976in}{1.649767in}}%
\pgfpathlineto{\pgfqpoint{2.349788in}{1.700713in}}%
\pgfpathlineto{\pgfqpoint{2.334629in}{1.751848in}}%
\pgfpathlineto{\pgfqpoint{2.321735in}{1.803167in}}%
\pgfpathlineto{\pgfqpoint{2.311397in}{1.854657in}}%
\pgfpathlineto{\pgfqpoint{2.303983in}{1.906295in}}%
\pgfpathlineto{\pgfqpoint{2.299947in}{1.958041in}}%
\pgfpathlineto{\pgfqpoint{2.299854in}{2.009829in}}%
\pgfpathlineto{\pgfqpoint{2.304414in}{2.061551in}}%
\pgfpathlineto{\pgfqpoint{2.314520in}{2.113030in}}%
\pgfpathlineto{\pgfqpoint{2.331299in}{2.163976in}}%
\pgfpathlineto{\pgfqpoint{2.356197in}{2.213899in}}%
\pgfpathlineto{\pgfqpoint{2.391068in}{2.261963in}}%
\pgfpathlineto{\pgfqpoint{2.438226in}{2.306699in}}%
\pgfpathlineto{\pgfqpoint{2.500247in}{2.345440in}}%
\pgfpathlineto{\pgfqpoint{2.572178in}{2.372105in}}%
\pgfpathlineto{\pgfqpoint{2.642785in}{2.384361in}}%
\pgfpathlineto{\pgfqpoint{2.709752in}{2.385442in}}%
\pgfpathlineto{\pgfqpoint{2.774123in}{2.377721in}}%
\pgfpathlineto{\pgfqpoint{2.837886in}{2.361935in}}%
\pgfpathlineto{\pgfqpoint{2.902059in}{2.337803in}}%
\pgfpathlineto{\pgfqpoint{2.966993in}{2.304463in}}%
\pgfpathlineto{\pgfqpoint{3.027155in}{2.264590in}}%
\pgfusepath{stroke}%
\end{pgfscope}%
\begin{pgfscope}%
\pgfpathrectangle{\pgfqpoint{0.647939in}{0.492442in}}{\pgfqpoint{4.273799in}{2.331163in}}%
\pgfusepath{clip}%
\pgfsetbuttcap%
\pgfsetroundjoin%
\pgfsetlinewidth{0.301125pt}%
\definecolor{currentstroke}{rgb}{0.500000,0.500000,0.500000}%
\pgfsetstrokecolor{currentstroke}%
\pgfsetstrokeopacity{0.300000}%
\pgfsetdash{}{0pt}%
\pgfpathmoveto{\pgfqpoint{3.367630in}{0.492442in}}%
\pgfpathlineto{\pgfqpoint{3.367630in}{0.492442in}}%
\pgfpathlineto{\pgfqpoint{3.316508in}{0.536099in}}%
\pgfpathlineto{\pgfqpoint{3.266403in}{0.580105in}}%
\pgfpathlineto{\pgfqpoint{3.217390in}{0.624474in}}%
\pgfpathlineto{\pgfqpoint{3.169529in}{0.669216in}}%
\pgfpathlineto{\pgfqpoint{3.122863in}{0.714332in}}%
\pgfpathlineto{\pgfqpoint{3.077427in}{0.759819in}}%
\pgfpathlineto{\pgfqpoint{3.033241in}{0.805672in}}%
\pgfpathlineto{\pgfqpoint{2.990321in}{0.851881in}}%
\pgfpathlineto{\pgfqpoint{2.948674in}{0.898436in}}%
\pgfpathlineto{\pgfqpoint{2.908304in}{0.945324in}}%
\pgfpathlineto{\pgfqpoint{2.869210in}{0.992532in}}%
\pgfpathlineto{\pgfqpoint{2.831394in}{1.040050in}}%
\pgfpathlineto{\pgfqpoint{2.794857in}{1.087864in}}%
\pgfpathlineto{\pgfqpoint{2.759602in}{1.135964in}}%
\pgfpathlineto{\pgfqpoint{2.725636in}{1.184339in}}%
\pgfpathlineto{\pgfqpoint{2.692968in}{1.232980in}}%
\pgfpathlineto{\pgfqpoint{2.661615in}{1.281877in}}%
\pgfpathlineto{\pgfqpoint{2.631604in}{1.331024in}}%
\pgfpathlineto{\pgfqpoint{2.602969in}{1.380415in}}%
\pgfpathlineto{\pgfqpoint{2.575751in}{1.430044in}}%
\pgfpathlineto{\pgfqpoint{2.550011in}{1.479906in}}%
\pgfpathlineto{\pgfqpoint{2.525821in}{1.529999in}}%
\pgfpathlineto{\pgfqpoint{2.503272in}{1.580319in}}%
\pgfpathlineto{\pgfqpoint{2.482480in}{1.630863in}}%
\pgfpathlineto{\pgfqpoint{2.463587in}{1.681627in}}%
\pgfpathlineto{\pgfqpoint{2.446770in}{1.732608in}}%
\pgfpathlineto{\pgfqpoint{2.432248in}{1.783797in}}%
\pgfpathlineto{\pgfqpoint{2.420302in}{1.835184in}}%
\pgfpathlineto{\pgfqpoint{2.411275in}{1.886745in}}%
\pgfpathlineto{\pgfqpoint{2.405605in}{1.938445in}}%
\pgfpathlineto{\pgfqpoint{2.403850in}{1.990225in}}%
\pgfpathlineto{\pgfqpoint{2.406736in}{2.041983in}}%
\pgfpathlineto{\pgfqpoint{2.415213in}{2.093551in}}%
\pgfpathlineto{\pgfqpoint{2.430559in}{2.144628in}}%
\pgfpathlineto{\pgfqpoint{2.454536in}{2.194668in}}%
\pgfpathlineto{\pgfqpoint{2.489598in}{2.242645in}}%
\pgfpathlineto{\pgfqpoint{2.539128in}{2.286474in}}%
\pgfpathlineto{\pgfqpoint{2.539128in}{2.286474in}}%
\pgfpathlineto{\pgfqpoint{2.592025in}{2.316073in}}%
\pgfpathlineto{\pgfqpoint{2.658443in}{2.336682in}}%
\pgfpathlineto{\pgfqpoint{2.722519in}{2.343695in}}%
\pgfpathlineto{\pgfqpoint{2.782818in}{2.340678in}}%
\pgfusepath{stroke}%
\end{pgfscope}%
\begin{pgfscope}%
\pgfpathrectangle{\pgfqpoint{0.647939in}{0.492442in}}{\pgfqpoint{4.273799in}{2.331163in}}%
\pgfusepath{clip}%
\pgfsetbuttcap%
\pgfsetroundjoin%
\pgfsetlinewidth{0.301125pt}%
\definecolor{currentstroke}{rgb}{0.500000,0.500000,0.500000}%
\pgfsetstrokecolor{currentstroke}%
\pgfsetstrokeopacity{0.300000}%
\pgfsetdash{}{0pt}%
\pgfpathmoveto{\pgfqpoint{3.561893in}{0.492442in}}%
\pgfpathlineto{\pgfqpoint{3.561893in}{0.492442in}}%
\pgfpathlineto{\pgfqpoint{3.506995in}{0.534713in}}%
\pgfpathlineto{\pgfqpoint{3.452734in}{0.577229in}}%
\pgfpathlineto{\pgfqpoint{3.399289in}{0.620050in}}%
\pgfpathlineto{\pgfqpoint{3.346813in}{0.663225in}}%
\pgfpathlineto{\pgfqpoint{3.295427in}{0.706788in}}%
\pgfpathlineto{\pgfqpoint{3.245231in}{0.750762in}}%
\pgfpathlineto{\pgfqpoint{3.196305in}{0.795160in}}%
\pgfpathlineto{\pgfqpoint{3.148712in}{0.839987in}}%
\pgfpathlineto{\pgfqpoint{3.102500in}{0.885242in}}%
\pgfpathlineto{\pgfqpoint{3.057700in}{0.930917in}}%
\pgfpathlineto{\pgfqpoint{3.014331in}{0.977000in}}%
\pgfpathlineto{\pgfqpoint{2.972406in}{1.023480in}}%
\pgfpathlineto{\pgfqpoint{2.931934in}{1.070342in}}%
\pgfpathlineto{\pgfqpoint{2.892921in}{1.117571in}}%
\pgfpathlineto{\pgfqpoint{2.855371in}{1.165151in}}%
\pgfpathlineto{\pgfqpoint{2.819292in}{1.213068in}}%
\pgfpathlineto{\pgfqpoint{2.784693in}{1.261309in}}%
\pgfpathlineto{\pgfqpoint{2.751593in}{1.309861in}}%
\pgfpathlineto{\pgfqpoint{2.720018in}{1.358715in}}%
\pgfpathlineto{\pgfqpoint{2.690000in}{1.407861in}}%
\pgfpathlineto{\pgfqpoint{2.661582in}{1.457288in}}%
\pgfusepath{stroke}%
\end{pgfscope}%
\begin{pgfscope}%
\pgfpathrectangle{\pgfqpoint{0.647939in}{0.492442in}}{\pgfqpoint{4.273799in}{2.331163in}}%
\pgfusepath{clip}%
\pgfsetbuttcap%
\pgfsetroundjoin%
\pgfsetlinewidth{0.301125pt}%
\definecolor{currentstroke}{rgb}{0.500000,0.500000,0.500000}%
\pgfsetstrokecolor{currentstroke}%
\pgfsetstrokeopacity{0.300000}%
\pgfsetdash{}{0pt}%
\pgfpathmoveto{\pgfqpoint{3.756157in}{0.492442in}}%
\pgfpathlineto{\pgfqpoint{3.756157in}{0.492442in}}%
\pgfpathlineto{\pgfqpoint{3.699401in}{0.533978in}}%
\pgfpathlineto{\pgfqpoint{3.642544in}{0.575473in}}%
\pgfpathlineto{\pgfqpoint{3.585829in}{0.617025in}}%
\pgfpathlineto{\pgfqpoint{3.529511in}{0.658737in}}%
\pgfpathlineto{\pgfqpoint{3.473835in}{0.700704in}}%
\pgfpathlineto{\pgfqpoint{3.419007in}{0.743001in}}%
\pgfpathlineto{\pgfqpoint{3.365211in}{0.785690in}}%
\pgfpathlineto{\pgfqpoint{3.312609in}{0.828819in}}%
\pgfpathlineto{\pgfqpoint{3.261336in}{0.872420in}}%
\pgfpathlineto{\pgfqpoint{3.211487in}{0.916510in}}%
\pgfpathlineto{\pgfqpoint{3.163140in}{0.961094in}}%
\pgfpathlineto{\pgfqpoint{3.116349in}{1.006169in}}%
\pgfpathlineto{\pgfqpoint{3.071156in}{1.051728in}}%
\pgfusepath{stroke}%
\end{pgfscope}%
\begin{pgfscope}%
\pgfpathrectangle{\pgfqpoint{0.647939in}{0.492442in}}{\pgfqpoint{4.273799in}{2.331163in}}%
\pgfusepath{clip}%
\pgfsetbuttcap%
\pgfsetroundjoin%
\pgfsetlinewidth{0.301125pt}%
\definecolor{currentstroke}{rgb}{0.500000,0.500000,0.500000}%
\pgfsetstrokecolor{currentstroke}%
\pgfsetstrokeopacity{0.300000}%
\pgfsetdash{}{0pt}%
\pgfpathmoveto{\pgfqpoint{3.950420in}{0.492442in}}%
\pgfpathlineto{\pgfqpoint{3.950420in}{0.492442in}}%
\pgfpathlineto{\pgfqpoint{3.894914in}{0.534475in}}%
\pgfpathlineto{\pgfqpoint{3.838306in}{0.576069in}}%
\pgfpathlineto{\pgfqpoint{3.780845in}{0.617315in}}%
\pgfpathlineto{\pgfqpoint{3.722808in}{0.658320in}}%
\pgfpathlineto{\pgfqpoint{3.664481in}{0.699203in}}%
\pgfpathlineto{\pgfqpoint{3.606185in}{0.740099in}}%
\pgfpathlineto{\pgfqpoint{3.548227in}{0.781137in}}%
\pgfpathlineto{\pgfqpoint{3.490894in}{0.822435in}}%
\pgfpathlineto{\pgfqpoint{3.434463in}{0.864100in}}%
\pgfpathlineto{\pgfqpoint{3.379168in}{0.906214in}}%
\pgfpathlineto{\pgfqpoint{3.325199in}{0.948837in}}%
\pgfpathlineto{\pgfqpoint{3.272726in}{0.992011in}}%
\pgfpathlineto{\pgfqpoint{3.221880in}{1.035759in}}%
\pgfpathlineto{\pgfqpoint{3.172753in}{1.080087in}}%
\pgfpathlineto{\pgfqpoint{3.125412in}{1.124989in}}%
\pgfpathlineto{\pgfqpoint{3.079904in}{1.170451in}}%
\pgfpathlineto{\pgfqpoint{3.036262in}{1.216455in}}%
\pgfpathlineto{\pgfqpoint{2.994512in}{1.262979in}}%
\pgfpathlineto{\pgfqpoint{2.954672in}{1.309998in}}%
\pgfpathlineto{\pgfqpoint{2.916762in}{1.357489in}}%
\pgfpathlineto{\pgfqpoint{2.880806in}{1.405431in}}%
\pgfpathlineto{\pgfqpoint{2.846839in}{1.453803in}}%
\pgfpathlineto{\pgfqpoint{2.814909in}{1.502586in}}%
\pgfpathlineto{\pgfqpoint{2.785077in}{1.551763in}}%
\pgfpathlineto{\pgfqpoint{2.757434in}{1.601318in}}%
\pgfpathlineto{\pgfqpoint{2.732098in}{1.651239in}}%
\pgfpathlineto{\pgfqpoint{2.709225in}{1.701511in}}%
\pgfpathlineto{\pgfqpoint{2.689023in}{1.752121in}}%
\pgfpathlineto{\pgfqpoint{2.671768in}{1.803054in}}%
\pgfpathlineto{\pgfqpoint{2.657825in}{1.854285in}}%
\pgfpathlineto{\pgfqpoint{2.647690in}{1.905780in}}%
\pgfpathlineto{\pgfqpoint{2.642053in}{1.957473in}}%
\pgfpathlineto{\pgfqpoint{2.641898in}{2.009245in}}%
\pgfpathlineto{\pgfqpoint{2.648708in}{2.060862in}}%
\pgfpathlineto{\pgfqpoint{2.664856in}{2.111809in}}%
\pgfpathlineto{\pgfqpoint{2.694454in}{2.160795in}}%
\pgfpathlineto{\pgfqpoint{2.694454in}{2.160795in}}%
\pgfpathlineto{\pgfqpoint{2.732115in}{2.195975in}}%
\pgfpathlineto{\pgfqpoint{2.732115in}{2.195975in}}%
\pgfpathlineto{\pgfqpoint{2.774078in}{2.218073in}}%
\pgfpathlineto{\pgfqpoint{2.774078in}{2.218073in}}%
\pgfpathlineto{\pgfqpoint{2.819154in}{2.229382in}}%
\pgfpathlineto{\pgfqpoint{2.868616in}{2.230995in}}%
\pgfpathlineto{\pgfqpoint{2.914954in}{2.224156in}}%
\pgfpathlineto{\pgfqpoint{2.961487in}{2.209846in}}%
\pgfpathlineto{\pgfqpoint{3.009496in}{2.187269in}}%
\pgfpathlineto{\pgfqpoint{3.058278in}{2.155313in}}%
\pgfpathlineto{\pgfqpoint{3.105306in}{2.113276in}}%
\pgfusepath{stroke}%
\end{pgfscope}%
\begin{pgfscope}%
\pgfpathrectangle{\pgfqpoint{0.647939in}{0.492442in}}{\pgfqpoint{4.273799in}{2.331163in}}%
\pgfusepath{clip}%
\pgfsetbuttcap%
\pgfsetroundjoin%
\pgfsetlinewidth{0.301125pt}%
\definecolor{currentstroke}{rgb}{0.500000,0.500000,0.500000}%
\pgfsetstrokecolor{currentstroke}%
\pgfsetstrokeopacity{0.300000}%
\pgfsetdash{}{0pt}%
\pgfpathmoveto{\pgfqpoint{4.144684in}{0.492442in}}%
\pgfpathlineto{\pgfqpoint{4.144684in}{0.492442in}}%
\pgfpathlineto{\pgfqpoint{4.093989in}{0.536244in}}%
\pgfpathlineto{\pgfqpoint{4.041482in}{0.579405in}}%
\pgfpathlineto{\pgfqpoint{3.987209in}{0.621911in}}%
\pgfpathlineto{\pgfqpoint{3.931264in}{0.663769in}}%
\pgfpathlineto{\pgfqpoint{3.873832in}{0.705025in}}%
\pgfpathlineto{\pgfqpoint{3.815167in}{0.745762in}}%
\pgfpathlineto{\pgfqpoint{3.755553in}{0.786088in}}%
\pgfpathlineto{\pgfqpoint{3.695351in}{0.826154in}}%
\pgfpathlineto{\pgfqpoint{3.634951in}{0.866131in}}%
\pgfpathlineto{\pgfqpoint{3.574732in}{0.906189in}}%
\pgfpathlineto{\pgfqpoint{3.515087in}{0.946500in}}%
\pgfpathlineto{\pgfqpoint{3.456361in}{0.987210in}}%
\pgfpathlineto{\pgfqpoint{3.398874in}{1.028441in}}%
\pgfpathlineto{\pgfqpoint{3.342902in}{1.070285in}}%
\pgfpathlineto{\pgfqpoint{3.288650in}{1.112797in}}%
\pgfpathlineto{\pgfqpoint{3.236283in}{1.156007in}}%
\pgfpathlineto{\pgfqpoint{3.185932in}{1.199924in}}%
\pgfpathlineto{\pgfqpoint{3.137687in}{1.244536in}}%
\pgfpathlineto{\pgfqpoint{3.091606in}{1.289824in}}%
\pgfpathlineto{\pgfqpoint{3.047730in}{1.335759in}}%
\pgfpathlineto{\pgfqpoint{3.006092in}{1.382309in}}%
\pgfpathlineto{\pgfqpoint{2.966721in}{1.429443in}}%
\pgfpathlineto{\pgfqpoint{2.929651in}{1.477130in}}%
\pgfpathlineto{\pgfqpoint{2.894931in}{1.525341in}}%
\pgfpathlineto{\pgfqpoint{2.862626in}{1.574049in}}%
\pgfusepath{stroke}%
\end{pgfscope}%
\begin{pgfscope}%
\pgfpathrectangle{\pgfqpoint{0.647939in}{0.492442in}}{\pgfqpoint{4.273799in}{2.331163in}}%
\pgfusepath{clip}%
\pgfsetbuttcap%
\pgfsetroundjoin%
\pgfsetlinewidth{0.301125pt}%
\definecolor{currentstroke}{rgb}{0.500000,0.500000,0.500000}%
\pgfsetstrokecolor{currentstroke}%
\pgfsetstrokeopacity{0.300000}%
\pgfsetdash{}{0pt}%
\pgfpathmoveto{\pgfqpoint{4.241816in}{0.492442in}}%
\pgfpathlineto{\pgfqpoint{4.241816in}{0.492442in}}%
\pgfpathlineto{\pgfqpoint{4.194704in}{0.537417in}}%
\pgfpathlineto{\pgfqpoint{4.145656in}{0.581772in}}%
\pgfpathlineto{\pgfqpoint{4.094619in}{0.625455in}}%
\pgfpathlineto{\pgfqpoint{4.041594in}{0.668426in}}%
\pgfpathlineto{\pgfqpoint{3.986613in}{0.710660in}}%
\pgfpathlineto{\pgfqpoint{3.929765in}{0.752152in}}%
\pgfpathlineto{\pgfqpoint{3.871246in}{0.792950in}}%
\pgfpathlineto{\pgfqpoint{3.811338in}{0.833145in}}%
\pgfpathlineto{\pgfqpoint{3.750365in}{0.872861in}}%
\pgfpathlineto{\pgfqpoint{3.688744in}{0.912280in}}%
\pgfpathlineto{\pgfqpoint{3.626907in}{0.951598in}}%
\pgfusepath{stroke}%
\end{pgfscope}%
\begin{pgfscope}%
\pgfpathrectangle{\pgfqpoint{0.647939in}{0.492442in}}{\pgfqpoint{4.273799in}{2.331163in}}%
\pgfusepath{clip}%
\pgfsetbuttcap%
\pgfsetroundjoin%
\pgfsetlinewidth{0.301125pt}%
\definecolor{currentstroke}{rgb}{0.500000,0.500000,0.500000}%
\pgfsetstrokecolor{currentstroke}%
\pgfsetstrokeopacity{0.300000}%
\pgfsetdash{}{0pt}%
\pgfpathmoveto{\pgfqpoint{4.338948in}{0.492442in}}%
\pgfpathlineto{\pgfqpoint{4.338948in}{0.492442in}}%
\pgfpathlineto{\pgfqpoint{4.295914in}{0.538616in}}%
\pgfpathlineto{\pgfqpoint{4.251052in}{0.584269in}}%
\pgfpathlineto{\pgfqpoint{4.204230in}{0.629332in}}%
\pgfpathlineto{\pgfqpoint{4.155326in}{0.673731in}}%
\pgfpathlineto{\pgfqpoint{4.104233in}{0.717392in}}%
\pgfpathlineto{\pgfqpoint{4.050901in}{0.760250in}}%
\pgfpathlineto{\pgfqpoint{3.995352in}{0.802260in}}%
\pgfpathlineto{\pgfqpoint{3.937660in}{0.843402in}}%
\pgfpathlineto{\pgfqpoint{3.877994in}{0.883700in}}%
\pgfpathlineto{\pgfqpoint{3.816656in}{0.923247in}}%
\pgfpathlineto{\pgfqpoint{3.754023in}{0.962186in}}%
\pgfpathlineto{\pgfqpoint{3.690575in}{1.000733in}}%
\pgfpathlineto{\pgfqpoint{3.626836in}{1.039136in}}%
\pgfpathlineto{\pgfqpoint{3.563354in}{1.077665in}}%
\pgfpathlineto{\pgfqpoint{3.500661in}{1.116575in}}%
\pgfpathlineto{\pgfqpoint{3.439238in}{1.156078in}}%
\pgfpathlineto{\pgfqpoint{3.379506in}{1.196342in}}%
\pgfpathlineto{\pgfqpoint{3.321791in}{1.237472in}}%
\pgfpathlineto{\pgfqpoint{3.266361in}{1.279525in}}%
\pgfpathlineto{\pgfqpoint{3.213412in}{1.322514in}}%
\pgfpathlineto{\pgfqpoint{3.163065in}{1.366424in}}%
\pgfpathlineto{\pgfqpoint{3.115406in}{1.411219in}}%
\pgfpathlineto{\pgfqpoint{3.070495in}{1.456853in}}%
\pgfpathlineto{\pgfqpoint{3.028382in}{1.503275in}}%
\pgfpathlineto{\pgfqpoint{2.989117in}{1.550432in}}%
\pgfpathlineto{\pgfqpoint{2.952763in}{1.598277in}}%
\pgfpathlineto{\pgfqpoint{2.919412in}{1.646768in}}%
\pgfpathlineto{\pgfqpoint{2.889200in}{1.695868in}}%
\pgfpathlineto{\pgfqpoint{2.862329in}{1.745543in}}%
\pgfpathlineto{\pgfqpoint{2.839091in}{1.795756in}}%
\pgfpathlineto{\pgfqpoint{2.819914in}{1.846474in}}%
\pgfpathlineto{\pgfqpoint{2.805436in}{1.897649in}}%
\pgfpathlineto{\pgfqpoint{2.796636in}{1.949197in}}%
\pgfpathlineto{\pgfqpoint{2.795121in}{2.000941in}}%
\pgfpathlineto{\pgfqpoint{2.803798in}{2.052407in}}%
\pgfpathlineto{\pgfqpoint{2.828734in}{2.101984in}}%
\pgfpathlineto{\pgfqpoint{2.828734in}{2.101984in}}%
\pgfpathlineto{\pgfqpoint{2.858522in}{2.129824in}}%
\pgfpathlineto{\pgfqpoint{2.858522in}{2.129824in}}%
\pgfpathlineto{\pgfqpoint{2.892865in}{2.145519in}}%
\pgfpathlineto{\pgfqpoint{2.892865in}{2.145519in}}%
\pgfpathlineto{\pgfqpoint{2.929785in}{2.150971in}}%
\pgfpathlineto{\pgfqpoint{2.967624in}{2.147737in}}%
\pgfpathlineto{\pgfqpoint{3.004061in}{2.137502in}}%
\pgfusepath{stroke}%
\end{pgfscope}%
\begin{pgfscope}%
\pgfpathrectangle{\pgfqpoint{0.647939in}{0.492442in}}{\pgfqpoint{4.273799in}{2.331163in}}%
\pgfusepath{clip}%
\pgfsetbuttcap%
\pgfsetroundjoin%
\pgfsetlinewidth{0.301125pt}%
\definecolor{currentstroke}{rgb}{0.500000,0.500000,0.500000}%
\pgfsetstrokecolor{currentstroke}%
\pgfsetstrokeopacity{0.300000}%
\pgfsetdash{}{0pt}%
\pgfpathmoveto{\pgfqpoint{4.436079in}{0.492442in}}%
\pgfpathlineto{\pgfqpoint{4.436079in}{0.492442in}}%
\pgfpathlineto{\pgfqpoint{4.397337in}{0.539736in}}%
\pgfpathlineto{\pgfqpoint{4.357015in}{0.586635in}}%
\pgfpathlineto{\pgfqpoint{4.314962in}{0.633079in}}%
\pgfpathlineto{\pgfqpoint{4.271015in}{0.678999in}}%
\pgfpathlineto{\pgfqpoint{4.225007in}{0.724312in}}%
\pgfpathlineto{\pgfqpoint{4.176768in}{0.768927in}}%
\pgfpathlineto{\pgfqpoint{4.126142in}{0.812747in}}%
\pgfpathlineto{\pgfqpoint{4.072992in}{0.855666in}}%
\pgfpathlineto{\pgfqpoint{4.017231in}{0.897590in}}%
\pgfpathlineto{\pgfqpoint{3.958885in}{0.938454in}}%
\pgfpathlineto{\pgfqpoint{3.898110in}{0.978252in}}%
\pgfpathlineto{\pgfqpoint{3.835167in}{1.017037in}}%
\pgfpathlineto{\pgfqpoint{3.770494in}{1.054969in}}%
\pgfpathlineto{\pgfqpoint{3.704647in}{1.092299in}}%
\pgfpathlineto{\pgfqpoint{3.638274in}{1.129350in}}%
\pgfpathlineto{\pgfqpoint{3.572063in}{1.166486in}}%
\pgfusepath{stroke}%
\end{pgfscope}%
\begin{pgfscope}%
\pgfpathrectangle{\pgfqpoint{0.647939in}{0.492442in}}{\pgfqpoint{4.273799in}{2.331163in}}%
\pgfusepath{clip}%
\pgfsetbuttcap%
\pgfsetroundjoin%
\pgfsetlinewidth{0.301125pt}%
\definecolor{currentstroke}{rgb}{0.500000,0.500000,0.500000}%
\pgfsetstrokecolor{currentstroke}%
\pgfsetstrokeopacity{0.300000}%
\pgfsetdash{}{0pt}%
\pgfpathmoveto{\pgfqpoint{4.533211in}{0.492442in}}%
\pgfpathlineto{\pgfqpoint{4.533211in}{0.492442in}}%
\pgfpathlineto{\pgfqpoint{4.498747in}{0.540712in}}%
\pgfpathlineto{\pgfqpoint{4.463053in}{0.588714in}}%
\pgfpathlineto{\pgfqpoint{4.425994in}{0.636406in}}%
\pgfpathlineto{\pgfqpoint{4.387415in}{0.683737in}}%
\pgfpathlineto{\pgfqpoint{4.347141in}{0.730645in}}%
\pgfpathlineto{\pgfqpoint{4.304977in}{0.777057in}}%
\pgfpathlineto{\pgfqpoint{4.260707in}{0.822879in}}%
\pgfpathlineto{\pgfqpoint{4.214096in}{0.868004in}}%
\pgfpathlineto{\pgfqpoint{4.164892in}{0.912299in}}%
\pgfpathlineto{\pgfqpoint{4.112844in}{0.955615in}}%
\pgfpathlineto{\pgfqpoint{4.057717in}{0.997783in}}%
\pgfpathlineto{\pgfqpoint{3.999397in}{1.038652in}}%
\pgfpathlineto{\pgfqpoint{3.937899in}{1.078109in}}%
\pgfpathlineto{\pgfqpoint{3.873389in}{1.116111in}}%
\pgfpathlineto{\pgfqpoint{3.806293in}{1.152763in}}%
\pgfpathlineto{\pgfqpoint{3.737263in}{1.188335in}}%
\pgfpathlineto{\pgfqpoint{3.667113in}{1.223254in}}%
\pgfpathlineto{\pgfqpoint{3.596768in}{1.258054in}}%
\pgfpathlineto{\pgfqpoint{3.527166in}{1.293291in}}%
\pgfpathlineto{\pgfqpoint{3.459136in}{1.329423in}}%
\pgfpathlineto{\pgfqpoint{3.393422in}{1.366797in}}%
\pgfpathlineto{\pgfqpoint{3.330588in}{1.405614in}}%
\pgfpathlineto{\pgfqpoint{3.271046in}{1.445940in}}%
\pgfpathlineto{\pgfqpoint{3.215073in}{1.487756in}}%
\pgfpathlineto{\pgfqpoint{3.162822in}{1.530981in}}%
\pgfpathlineto{\pgfqpoint{3.114379in}{1.575512in}}%
\pgfpathlineto{\pgfqpoint{3.069815in}{1.621236in}}%
\pgfpathlineto{\pgfqpoint{3.029207in}{1.668044in}}%
\pgfpathlineto{\pgfqpoint{2.992674in}{1.715837in}}%
\pgfpathlineto{\pgfqpoint{2.960412in}{1.764533in}}%
\pgfpathlineto{\pgfqpoint{2.932753in}{1.814065in}}%
\pgfpathlineto{\pgfqpoint{2.910251in}{1.864363in}}%
\pgfpathlineto{\pgfqpoint{2.893835in}{1.915344in}}%
\pgfpathlineto{\pgfqpoint{2.885157in}{1.966862in}}%
\pgfpathlineto{\pgfqpoint{2.887526in}{2.018502in}}%
\pgfpathlineto{\pgfqpoint{2.909215in}{2.068418in}}%
\pgfpathlineto{\pgfqpoint{2.909215in}{2.068418in}}%
\pgfpathlineto{\pgfqpoint{2.934053in}{2.090182in}}%
\pgfpathlineto{\pgfqpoint{2.934053in}{2.090182in}}%
\pgfpathlineto{\pgfqpoint{2.962998in}{2.100651in}}%
\pgfpathlineto{\pgfqpoint{2.997198in}{2.101395in}}%
\pgfpathlineto{\pgfqpoint{3.027721in}{2.094349in}}%
\pgfpathlineto{\pgfqpoint{3.058684in}{2.080315in}}%
\pgfpathlineto{\pgfqpoint{3.089380in}{2.058622in}}%
\pgfpathlineto{\pgfqpoint{3.117560in}{2.027592in}}%
\pgfpathlineto{\pgfqpoint{3.132791in}{1.989793in}}%
\pgfpathlineto{\pgfqpoint{3.132791in}{1.989793in}}%
\pgfpathlineto{\pgfqpoint{3.128745in}{1.974345in}}%
\pgfpathlineto{\pgfqpoint{3.128745in}{1.974345in}}%
\pgfusepath{stroke}%
\end{pgfscope}%
\begin{pgfscope}%
\pgfpathrectangle{\pgfqpoint{0.647939in}{0.492442in}}{\pgfqpoint{4.273799in}{2.331163in}}%
\pgfusepath{clip}%
\pgfsetbuttcap%
\pgfsetroundjoin%
\pgfsetlinewidth{0.301125pt}%
\definecolor{currentstroke}{rgb}{0.500000,0.500000,0.500000}%
\pgfsetstrokecolor{currentstroke}%
\pgfsetstrokeopacity{0.300000}%
\pgfsetdash{}{0pt}%
\pgfpathmoveto{\pgfqpoint{4.630343in}{0.492442in}}%
\pgfpathlineto{\pgfqpoint{4.630343in}{0.492442in}}%
\pgfpathlineto{\pgfqpoint{4.599956in}{0.541520in}}%
\pgfpathlineto{\pgfqpoint{4.568673in}{0.590430in}}%
\pgfpathlineto{\pgfqpoint{4.536398in}{0.639148in}}%
\pgfpathlineto{\pgfqpoint{4.503030in}{0.687646in}}%
\pgfpathlineto{\pgfqpoint{4.468451in}{0.735889in}}%
\pgfpathlineto{\pgfqpoint{4.432505in}{0.783834in}}%
\pgfpathlineto{\pgfqpoint{4.395011in}{0.831424in}}%
\pgfpathlineto{\pgfqpoint{4.355760in}{0.878591in}}%
\pgfpathlineto{\pgfqpoint{4.314503in}{0.925245in}}%
\pgfpathlineto{\pgfqpoint{4.270952in}{0.971274in}}%
\pgfpathlineto{\pgfqpoint{4.224772in}{1.016531in}}%
\pgfpathlineto{\pgfqpoint{4.175577in}{1.060827in}}%
\pgfpathlineto{\pgfqpoint{4.122945in}{1.103924in}}%
\pgfpathlineto{\pgfqpoint{4.066431in}{1.145528in}}%
\pgfpathlineto{\pgfqpoint{4.005667in}{1.185306in}}%
\pgfpathlineto{\pgfqpoint{3.940528in}{1.222963in}}%
\pgfpathlineto{\pgfqpoint{3.871196in}{1.258326in}}%
\pgfpathlineto{\pgfqpoint{3.798282in}{1.291489in}}%
\pgfpathlineto{\pgfqpoint{3.722860in}{1.322957in}}%
\pgfpathlineto{\pgfqpoint{3.646277in}{1.353585in}}%
\pgfpathlineto{\pgfqpoint{3.569899in}{1.384357in}}%
\pgfpathlineto{\pgfqpoint{3.495017in}{1.416193in}}%
\pgfpathlineto{\pgfqpoint{3.422799in}{1.449789in}}%
\pgfpathlineto{\pgfqpoint{3.354172in}{1.485532in}}%
\pgfpathlineto{\pgfqpoint{3.289730in}{1.523522in}}%
\pgfpathlineto{\pgfqpoint{3.229865in}{1.563677in}}%
\pgfpathlineto{\pgfqpoint{3.174774in}{1.605815in}}%
\pgfusepath{stroke}%
\end{pgfscope}%
\begin{pgfscope}%
\pgfpathrectangle{\pgfqpoint{0.647939in}{0.492442in}}{\pgfqpoint{4.273799in}{2.331163in}}%
\pgfusepath{clip}%
\pgfsetbuttcap%
\pgfsetroundjoin%
\pgfsetlinewidth{0.301125pt}%
\definecolor{currentstroke}{rgb}{0.500000,0.500000,0.500000}%
\pgfsetstrokecolor{currentstroke}%
\pgfsetstrokeopacity{0.300000}%
\pgfsetdash{}{0pt}%
\pgfpathmoveto{\pgfqpoint{4.727475in}{0.492442in}}%
\pgfpathlineto{\pgfqpoint{4.727475in}{0.492442in}}%
\pgfpathlineto{\pgfqpoint{4.700814in}{0.542161in}}%
\pgfpathlineto{\pgfqpoint{4.673561in}{0.591785in}}%
\pgfpathlineto{\pgfqpoint{4.645668in}{0.641301in}}%
\pgfpathlineto{\pgfqpoint{4.617064in}{0.690698in}}%
\pgfpathlineto{\pgfqpoint{4.587682in}{0.739958in}}%
\pgfpathlineto{\pgfqpoint{4.557445in}{0.789063in}}%
\pgfpathlineto{\pgfqpoint{4.526243in}{0.837987in}}%
\pgfpathlineto{\pgfqpoint{4.493954in}{0.886702in}}%
\pgfpathlineto{\pgfqpoint{4.460440in}{0.935169in}}%
\pgfpathlineto{\pgfqpoint{4.425528in}{0.983339in}}%
\pgfpathlineto{\pgfqpoint{4.388990in}{1.031148in}}%
\pgfpathlineto{\pgfqpoint{4.350542in}{1.078508in}}%
\pgfpathlineto{\pgfqpoint{4.309827in}{1.125299in}}%
\pgfpathlineto{\pgfqpoint{4.266398in}{1.171356in}}%
\pgfpathlineto{\pgfqpoint{4.219678in}{1.216437in}}%
\pgfpathlineto{\pgfqpoint{4.168938in}{1.260194in}}%
\pgfpathlineto{\pgfqpoint{4.113263in}{1.302108in}}%
\pgfpathlineto{\pgfqpoint{4.051601in}{1.341431in}}%
\pgfpathlineto{\pgfqpoint{3.983158in}{1.377223in}}%
\pgfpathlineto{\pgfqpoint{3.907779in}{1.408572in}}%
\pgfpathlineto{\pgfqpoint{3.826502in}{1.435212in}}%
\pgfpathlineto{\pgfqpoint{3.741388in}{1.458121in}}%
\pgfpathlineto{\pgfqpoint{3.654791in}{1.479375in}}%
\pgfpathlineto{\pgfqpoint{3.568847in}{1.501346in}}%
\pgfpathlineto{\pgfqpoint{3.485422in}{1.525946in}}%
\pgfpathlineto{\pgfqpoint{3.406122in}{1.554276in}}%
\pgfpathlineto{\pgfqpoint{3.332200in}{1.586640in}}%
\pgfpathlineto{\pgfqpoint{3.264478in}{1.622795in}}%
\pgfpathlineto{\pgfqpoint{3.203205in}{1.662249in}}%
\pgfpathlineto{\pgfqpoint{3.148427in}{1.704471in}}%
\pgfpathlineto{\pgfqpoint{3.100106in}{1.748992in}}%
\pgfpathlineto{\pgfqpoint{3.058296in}{1.795433in}}%
\pgfpathlineto{\pgfqpoint{3.023261in}{1.843522in}}%
\pgfpathlineto{\pgfqpoint{2.995755in}{1.893033in}}%
\pgfpathlineto{\pgfqpoint{2.977480in}{1.943756in}}%
\pgfpathlineto{\pgfqpoint{2.972713in}{1.995259in}}%
\pgfpathlineto{\pgfqpoint{2.972713in}{1.995259in}}%
\pgfpathlineto{\pgfqpoint{2.982840in}{2.028839in}}%
\pgfpathlineto{\pgfqpoint{2.982840in}{2.028839in}}%
\pgfpathlineto{\pgfqpoint{3.000922in}{2.047201in}}%
\pgfpathlineto{\pgfqpoint{3.000922in}{2.047201in}}%
\pgfpathlineto{\pgfqpoint{3.024077in}{2.054465in}}%
\pgfusepath{stroke}%
\end{pgfscope}%
\begin{pgfscope}%
\pgfpathrectangle{\pgfqpoint{0.647939in}{0.492442in}}{\pgfqpoint{4.273799in}{2.331163in}}%
\pgfusepath{clip}%
\pgfsetbuttcap%
\pgfsetroundjoin%
\pgfsetlinewidth{0.301125pt}%
\definecolor{currentstroke}{rgb}{0.500000,0.500000,0.500000}%
\pgfsetstrokecolor{currentstroke}%
\pgfsetstrokeopacity{0.300000}%
\pgfsetdash{}{0pt}%
\pgfpathmoveto{\pgfqpoint{4.824607in}{0.492442in}}%
\pgfpathlineto{\pgfqpoint{4.824607in}{0.492442in}}%
\pgfpathlineto{\pgfqpoint{4.801280in}{0.542658in}}%
\pgfpathlineto{\pgfqpoint{4.777605in}{0.592825in}}%
\pgfpathlineto{\pgfqpoint{4.753557in}{0.642939in}}%
\pgfpathlineto{\pgfqpoint{4.729110in}{0.692995in}}%
\pgfpathlineto{\pgfqpoint{4.704233in}{0.742989in}}%
\pgfpathlineto{\pgfqpoint{4.678891in}{0.792913in}}%
\pgfpathlineto{\pgfqpoint{4.653047in}{0.842760in}}%
\pgfpathlineto{\pgfqpoint{4.626648in}{0.892520in}}%
\pgfpathlineto{\pgfqpoint{4.599641in}{0.942182in}}%
\pgfpathlineto{\pgfqpoint{4.571955in}{0.991733in}}%
\pgfpathlineto{\pgfqpoint{4.543506in}{1.041156in}}%
\pgfpathlineto{\pgfqpoint{4.514201in}{1.090427in}}%
\pgfpathlineto{\pgfqpoint{4.483902in}{1.139520in}}%
\pgfpathlineto{\pgfqpoint{4.452443in}{1.188395in}}%
\pgfpathlineto{\pgfqpoint{4.419623in}{1.237001in}}%
\pgfpathlineto{\pgfqpoint{4.385160in}{1.285263in}}%
\pgfpathlineto{\pgfqpoint{4.348648in}{1.333072in}}%
\pgfpathlineto{\pgfqpoint{4.309517in}{1.380258in}}%
\pgfpathlineto{\pgfqpoint{4.266928in}{1.426537in}}%
\pgfpathlineto{\pgfqpoint{4.219599in}{1.471402in}}%
\pgfpathlineto{\pgfqpoint{4.165495in}{1.513860in}}%
\pgfpathlineto{\pgfqpoint{4.101390in}{1.551793in}}%
\pgfpathlineto{\pgfqpoint{4.023947in}{1.580910in}}%
\pgfpathlineto{\pgfqpoint{3.946721in}{1.595350in}}%
\pgfpathlineto{\pgfqpoint{3.872711in}{1.599981in}}%
\pgfpathlineto{\pgfqpoint{3.786457in}{1.599621in}}%
\pgfpathlineto{\pgfqpoint{3.691562in}{1.598192in}}%
\pgfpathlineto{\pgfqpoint{3.596920in}{1.600501in}}%
\pgfpathlineto{\pgfqpoint{3.503981in}{1.609788in}}%
\pgfusepath{stroke}%
\end{pgfscope}%
\begin{pgfscope}%
\pgfpathrectangle{\pgfqpoint{0.647939in}{0.492442in}}{\pgfqpoint{4.273799in}{2.331163in}}%
\pgfusepath{clip}%
\pgfsetbuttcap%
\pgfsetroundjoin%
\pgfsetlinewidth{0.301125pt}%
\definecolor{currentstroke}{rgb}{0.500000,0.500000,0.500000}%
\pgfsetstrokecolor{currentstroke}%
\pgfsetstrokeopacity{0.300000}%
\pgfsetdash{}{0pt}%
\pgfpathmoveto{\pgfqpoint{4.921738in}{0.492442in}}%
\pgfpathlineto{\pgfqpoint{4.921738in}{0.492442in}}%
\pgfpathlineto{\pgfqpoint{4.901337in}{0.543036in}}%
\pgfpathlineto{\pgfqpoint{4.880770in}{0.593609in}}%
\pgfpathlineto{\pgfqpoint{4.860030in}{0.644162in}}%
\pgfpathlineto{\pgfqpoint{4.839114in}{0.694693in}}%
\pgfpathlineto{\pgfqpoint{4.818015in}{0.745201in}}%
\pgfpathlineto{\pgfqpoint{4.796728in}{0.795685in}}%
\pgfpathlineto{\pgfqpoint{4.775245in}{0.846145in}}%
\pgfpathlineto{\pgfqpoint{4.753561in}{0.896580in}}%
\pgfpathlineto{\pgfqpoint{4.731668in}{0.946987in}}%
\pgfpathlineto{\pgfqpoint{4.709557in}{0.997366in}}%
\pgfpathlineto{\pgfqpoint{4.687218in}{1.047714in}}%
\pgfpathlineto{\pgfqpoint{4.664641in}{1.098032in}}%
\pgfpathlineto{\pgfqpoint{4.641814in}{1.148315in}}%
\pgfpathlineto{\pgfqpoint{4.618722in}{1.198563in}}%
\pgfpathlineto{\pgfqpoint{4.595351in}{1.248771in}}%
\pgfpathlineto{\pgfqpoint{4.571683in}{1.298938in}}%
\pgfpathlineto{\pgfqpoint{4.547692in}{1.349059in}}%
\pgfpathlineto{\pgfqpoint{4.523352in}{1.399129in}}%
\pgfpathlineto{\pgfqpoint{4.498621in}{1.449141in}}%
\pgfpathlineto{\pgfqpoint{4.473454in}{1.499088in}}%
\pgfpathlineto{\pgfqpoint{4.447771in}{1.548958in}}%
\pgfpathlineto{\pgfqpoint{4.421484in}{1.598732in}}%
\pgfpathlineto{\pgfqpoint{4.394452in}{1.648382in}}%
\pgfpathlineto{\pgfqpoint{4.366406in}{1.697864in}}%
\pgfpathlineto{\pgfqpoint{4.336911in}{1.747083in}}%
\pgfpathlineto{\pgfqpoint{4.304982in}{1.795818in}}%
\pgfpathlineto{\pgfqpoint{4.267663in}{1.843334in}}%
\pgfpathlineto{\pgfqpoint{4.267663in}{1.843334in}}%
\pgfpathlineto{\pgfqpoint{4.234553in}{1.872132in}}%
\pgfpathlineto{\pgfqpoint{4.234553in}{1.872132in}}%
\pgfpathlineto{\pgfqpoint{4.211051in}{1.882195in}}%
\pgfpathlineto{\pgfqpoint{4.211051in}{1.882195in}}%
\pgfpathlineto{\pgfqpoint{4.185198in}{1.882916in}}%
\pgfpathlineto{\pgfqpoint{4.160912in}{1.876556in}}%
\pgfpathlineto{\pgfqpoint{4.134628in}{1.864886in}}%
\pgfpathlineto{\pgfqpoint{4.094508in}{1.842559in}}%
\pgfpathlineto{\pgfqpoint{4.030494in}{1.804842in}}%
\pgfpathlineto{\pgfqpoint{3.964419in}{1.767965in}}%
\pgfpathlineto{\pgfqpoint{3.894293in}{1.733314in}}%
\pgfpathlineto{\pgfqpoint{3.818895in}{1.702170in}}%
\pgfpathlineto{\pgfqpoint{3.737291in}{1.676153in}}%
\pgfpathlineto{\pgfqpoint{3.649390in}{1.657551in}}%
\pgfpathlineto{\pgfqpoint{3.558276in}{1.649097in}}%
\pgfpathlineto{\pgfqpoint{3.475677in}{1.651395in}}%
\pgfpathlineto{\pgfqpoint{3.400215in}{1.662476in}}%
\pgfpathlineto{\pgfqpoint{3.329334in}{1.681488in}}%
\pgfpathlineto{\pgfqpoint{3.261957in}{1.708368in}}%
\pgfpathlineto{\pgfqpoint{3.197793in}{1.743426in}}%
\pgfpathlineto{\pgfqpoint{3.140611in}{1.784553in}}%
\pgfusepath{stroke}%
\end{pgfscope}%
\begin{pgfscope}%
\pgfpathrectangle{\pgfqpoint{0.647939in}{0.492442in}}{\pgfqpoint{4.273799in}{2.331163in}}%
\pgfusepath{clip}%
\pgfsetbuttcap%
\pgfsetroundjoin%
\pgfsetlinewidth{0.301125pt}%
\definecolor{currentstroke}{rgb}{0.500000,0.500000,0.500000}%
\pgfsetstrokecolor{currentstroke}%
\pgfsetstrokeopacity{0.300000}%
\pgfsetdash{}{0pt}%
\pgfpathmoveto{\pgfqpoint{4.921738in}{0.704366in}}%
\pgfpathlineto{\pgfqpoint{4.921738in}{0.704366in}}%
\pgfpathlineto{\pgfqpoint{4.903154in}{0.755168in}}%
\pgfpathlineto{\pgfqpoint{4.884530in}{0.805965in}}%
\pgfpathlineto{\pgfqpoint{4.865874in}{0.856758in}}%
\pgfpathlineto{\pgfqpoint{4.847196in}{0.907549in}}%
\pgfpathlineto{\pgfqpoint{4.828505in}{0.958339in}}%
\pgfpathlineto{\pgfqpoint{4.809814in}{1.009128in}}%
\pgfpathlineto{\pgfqpoint{4.791136in}{1.059920in}}%
\pgfpathlineto{\pgfqpoint{4.772492in}{1.110714in}}%
\pgfpathlineto{\pgfqpoint{4.753903in}{1.161515in}}%
\pgfpathlineto{\pgfqpoint{4.735395in}{1.212324in}}%
\pgfpathlineto{\pgfqpoint{4.716999in}{1.263145in}}%
\pgfpathlineto{\pgfqpoint{4.698749in}{1.313982in}}%
\pgfpathlineto{\pgfqpoint{4.680695in}{1.364840in}}%
\pgfpathlineto{\pgfqpoint{4.662887in}{1.415723in}}%
\pgfpathlineto{\pgfqpoint{4.645396in}{1.466640in}}%
\pgfpathlineto{\pgfqpoint{4.628315in}{1.517596in}}%
\pgfpathlineto{\pgfqpoint{4.611743in}{1.568603in}}%
\pgfpathlineto{\pgfqpoint{4.595823in}{1.619671in}}%
\pgfpathlineto{\pgfqpoint{4.580725in}{1.670814in}}%
\pgfpathlineto{\pgfqpoint{4.566675in}{1.722045in}}%
\pgfpathlineto{\pgfqpoint{4.553970in}{1.773379in}}%
\pgfpathlineto{\pgfqpoint{4.542984in}{1.824830in}}%
\pgfpathlineto{\pgfqpoint{4.534194in}{1.876405in}}%
\pgfpathlineto{\pgfqpoint{4.528209in}{1.928098in}}%
\pgfpathlineto{\pgfqpoint{4.525747in}{1.979873in}}%
\pgfpathlineto{\pgfqpoint{4.527555in}{2.031648in}}%
\pgfpathlineto{\pgfqpoint{4.534266in}{2.083298in}}%
\pgfpathlineto{\pgfqpoint{4.546153in}{2.134665in}}%
\pgfpathlineto{\pgfqpoint{4.562982in}{2.185614in}}%
\pgfpathlineto{\pgfqpoint{4.584092in}{2.236093in}}%
\pgfpathlineto{\pgfqpoint{4.608530in}{2.286122in}}%
\pgfpathlineto{\pgfqpoint{4.635358in}{2.335781in}}%
\pgfpathlineto{\pgfqpoint{4.663819in}{2.385183in}}%
\pgfpathlineto{\pgfqpoint{4.693273in}{2.434412in}}%
\pgfpathlineto{\pgfqpoint{4.723257in}{2.483535in}}%
\pgfpathlineto{\pgfqpoint{4.753504in}{2.532620in}}%
\pgfpathlineto{\pgfqpoint{4.783791in}{2.581703in}}%
\pgfpathlineto{\pgfqpoint{4.813950in}{2.630805in}}%
\pgfpathlineto{\pgfqpoint{4.843900in}{2.679947in}}%
\pgfpathlineto{\pgfqpoint{4.873590in}{2.729145in}}%
\pgfpathlineto{\pgfqpoint{4.902957in}{2.778397in}}%
\pgfpathlineto{\pgfqpoint{4.921738in}{2.810094in}}%
\pgfusepath{stroke}%
\end{pgfscope}%
\begin{pgfscope}%
\pgfpathrectangle{\pgfqpoint{0.647939in}{0.492442in}}{\pgfqpoint{4.273799in}{2.331163in}}%
\pgfusepath{clip}%
\pgfsetbuttcap%
\pgfsetroundjoin%
\pgfsetlinewidth{0.301125pt}%
\definecolor{currentstroke}{rgb}{0.500000,0.500000,0.500000}%
\pgfsetstrokecolor{currentstroke}%
\pgfsetstrokeopacity{0.300000}%
\pgfsetdash{}{0pt}%
\pgfpathmoveto{\pgfqpoint{4.921738in}{0.969271in}}%
\pgfpathlineto{\pgfqpoint{4.921738in}{0.969271in}}%
\pgfpathlineto{\pgfqpoint{4.905799in}{1.020339in}}%
\pgfpathlineto{\pgfqpoint{4.890020in}{1.071422in}}%
\pgfpathlineto{\pgfqpoint{4.874425in}{1.122522in}}%
\pgfpathlineto{\pgfqpoint{4.859046in}{1.173641in}}%
\pgfpathlineto{\pgfqpoint{4.843921in}{1.224783in}}%
\pgfpathlineto{\pgfqpoint{4.829088in}{1.275950in}}%
\pgfpathlineto{\pgfqpoint{4.814597in}{1.327146in}}%
\pgfpathlineto{\pgfqpoint{4.800503in}{1.378375in}}%
\pgfpathlineto{\pgfqpoint{4.786865in}{1.429641in}}%
\pgfpathlineto{\pgfqpoint{4.773760in}{1.480948in}}%
\pgfpathlineto{\pgfqpoint{4.761273in}{1.532301in}}%
\pgfpathlineto{\pgfqpoint{4.749493in}{1.583703in}}%
\pgfpathlineto{\pgfqpoint{4.738538in}{1.635160in}}%
\pgfpathlineto{\pgfqpoint{4.728539in}{1.686674in}}%
\pgfpathlineto{\pgfqpoint{4.719640in}{1.738248in}}%
\pgfpathlineto{\pgfqpoint{4.712003in}{1.789882in}}%
\pgfpathlineto{\pgfqpoint{4.705812in}{1.841574in}}%
\pgfpathlineto{\pgfqpoint{4.701269in}{1.893315in}}%
\pgfpathlineto{\pgfqpoint{4.698585in}{1.945095in}}%
\pgfpathlineto{\pgfqpoint{4.697969in}{1.996894in}}%
\pgfpathlineto{\pgfqpoint{4.699617in}{2.048685in}}%
\pgfpathlineto{\pgfqpoint{4.703693in}{2.100436in}}%
\pgfpathlineto{\pgfqpoint{4.710304in}{2.152108in}}%
\pgfpathlineto{\pgfqpoint{4.719481in}{2.203662in}}%
\pgfpathlineto{\pgfqpoint{4.731176in}{2.255063in}}%
\pgfpathlineto{\pgfqpoint{4.745265in}{2.306286in}}%
\pgfpathlineto{\pgfqpoint{4.761545in}{2.357316in}}%
\pgfpathlineto{\pgfqpoint{4.779757in}{2.408149in}}%
\pgfpathlineto{\pgfqpoint{4.799638in}{2.458796in}}%
\pgfpathlineto{\pgfqpoint{4.820902in}{2.509275in}}%
\pgfusepath{stroke}%
\end{pgfscope}%
\begin{pgfscope}%
\pgfpathrectangle{\pgfqpoint{0.647939in}{0.492442in}}{\pgfqpoint{4.273799in}{2.331163in}}%
\pgfusepath{clip}%
\pgfsetbuttcap%
\pgfsetroundjoin%
\pgfsetlinewidth{0.301125pt}%
\definecolor{currentstroke}{rgb}{0.500000,0.500000,0.500000}%
\pgfsetstrokecolor{currentstroke}%
\pgfsetstrokeopacity{0.300000}%
\pgfsetdash{}{0pt}%
\pgfpathmoveto{\pgfqpoint{4.921738in}{1.234176in}}%
\pgfpathlineto{\pgfqpoint{4.921738in}{1.234176in}}%
\pgfpathlineto{\pgfqpoint{4.908968in}{1.285509in}}%
\pgfpathlineto{\pgfqpoint{4.896605in}{1.336871in}}%
\pgfpathlineto{\pgfqpoint{4.884694in}{1.388265in}}%
\pgfpathlineto{\pgfqpoint{4.873291in}{1.439693in}}%
\pgfpathlineto{\pgfqpoint{4.862461in}{1.491158in}}%
\pgfpathlineto{\pgfqpoint{4.852274in}{1.542662in}}%
\pgfpathlineto{\pgfqpoint{4.842804in}{1.594206in}}%
\pgfpathlineto{\pgfqpoint{4.834133in}{1.645793in}}%
\pgfpathlineto{\pgfqpoint{4.826357in}{1.697422in}}%
\pgfpathlineto{\pgfqpoint{4.819579in}{1.749092in}}%
\pgfpathlineto{\pgfqpoint{4.813905in}{1.800802in}}%
\pgfpathlineto{\pgfqpoint{4.809447in}{1.852547in}}%
\pgfpathlineto{\pgfqpoint{4.806321in}{1.904321in}}%
\pgfpathlineto{\pgfqpoint{4.804645in}{1.956114in}}%
\pgfpathlineto{\pgfqpoint{4.804530in}{2.007916in}}%
\pgfpathlineto{\pgfqpoint{4.806079in}{2.059710in}}%
\pgfpathlineto{\pgfqpoint{4.809376in}{2.111479in}}%
\pgfpathlineto{\pgfqpoint{4.814482in}{2.163205in}}%
\pgfpathlineto{\pgfqpoint{4.821428in}{2.214867in}}%
\pgfpathlineto{\pgfqpoint{4.830214in}{2.266445in}}%
\pgfpathlineto{\pgfqpoint{4.840799in}{2.317922in}}%
\pgfpathlineto{\pgfqpoint{4.853104in}{2.369285in}}%
\pgfusepath{stroke}%
\end{pgfscope}%
\begin{pgfscope}%
\pgfpathrectangle{\pgfqpoint{0.647939in}{0.492442in}}{\pgfqpoint{4.273799in}{2.331163in}}%
\pgfusepath{clip}%
\pgfsetbuttcap%
\pgfsetroundjoin%
\pgfsetlinewidth{0.301125pt}%
\definecolor{currentstroke}{rgb}{0.500000,0.500000,0.500000}%
\pgfsetstrokecolor{currentstroke}%
\pgfsetstrokeopacity{0.300000}%
\pgfsetdash{}{0pt}%
\pgfpathmoveto{\pgfqpoint{4.921738in}{1.499081in}}%
\pgfpathlineto{\pgfqpoint{4.921738in}{1.499081in}}%
\pgfpathlineto{\pgfqpoint{4.912812in}{1.550655in}}%
\pgfpathlineto{\pgfqpoint{4.904584in}{1.602263in}}%
\pgfpathlineto{\pgfqpoint{4.897123in}{1.653905in}}%
\pgfpathlineto{\pgfqpoint{4.890501in}{1.705582in}}%
\pgfpathlineto{\pgfqpoint{4.884795in}{1.757291in}}%
\pgfpathlineto{\pgfqpoint{4.880087in}{1.809030in}}%
\pgfpathlineto{\pgfqpoint{4.876462in}{1.860795in}}%
\pgfpathlineto{\pgfqpoint{4.874004in}{1.912580in}}%
\pgfusepath{stroke}%
\end{pgfscope}%
\begin{pgfscope}%
\pgfpathrectangle{\pgfqpoint{0.647939in}{0.492442in}}{\pgfqpoint{4.273799in}{2.331163in}}%
\pgfusepath{clip}%
\pgfsetbuttcap%
\pgfsetroundjoin%
\pgfsetlinewidth{0.301125pt}%
\definecolor{currentstroke}{rgb}{0.500000,0.500000,0.500000}%
\pgfsetstrokecolor{currentstroke}%
\pgfsetstrokeopacity{0.300000}%
\pgfsetdash{}{0pt}%
\pgfpathmoveto{\pgfqpoint{4.905606in}{1.982007in}}%
\pgfpathlineto{\pgfqpoint{4.906145in}{2.033808in}}%
\pgfpathlineto{\pgfqpoint{4.907988in}{2.085601in}}%
\pgfpathlineto{\pgfqpoint{4.911181in}{2.137374in}}%
\pgfpathlineto{\pgfqpoint{4.915758in}{2.189115in}}%
\pgfpathlineto{\pgfqpoint{4.921738in}{2.240815in}}%
\pgfpathlineto{\pgfqpoint{4.921738in}{2.240815in}}%
\pgfusepath{stroke}%
\end{pgfscope}%
\begin{pgfscope}%
\pgfpathrectangle{\pgfqpoint{0.647939in}{0.492442in}}{\pgfqpoint{4.273799in}{2.331163in}}%
\pgfusepath{clip}%
\pgfsetbuttcap%
\pgfsetroundjoin%
\pgfsetlinewidth{0.301125pt}%
\definecolor{currentstroke}{rgb}{0.500000,0.500000,0.500000}%
\pgfsetstrokecolor{currentstroke}%
\pgfsetstrokeopacity{0.300000}%
\pgfsetdash{}{0pt}%
\pgfpathmoveto{\pgfqpoint{4.252235in}{2.823605in}}%
\pgfpathlineto{\pgfqpoint{4.269039in}{2.807134in}}%
\pgfpathlineto{\pgfqpoint{4.317396in}{2.762576in}}%
\pgfpathlineto{\pgfqpoint{4.371709in}{2.720194in}}%
\pgfpathlineto{\pgfqpoint{4.419360in}{2.690955in}}%
\pgfpathlineto{\pgfqpoint{4.462689in}{2.671823in}}%
\pgfpathlineto{\pgfqpoint{4.506767in}{2.660418in}}%
\pgfpathlineto{\pgfqpoint{4.560660in}{2.658260in}}%
\pgfpathlineto{\pgfqpoint{4.611538in}{2.668249in}}%
\pgfpathlineto{\pgfqpoint{4.611538in}{2.668249in}}%
\pgfpathlineto{\pgfqpoint{4.670450in}{2.694050in}}%
\pgfpathlineto{\pgfqpoint{4.670450in}{2.694050in}}%
\pgfpathlineto{\pgfqpoint{4.729700in}{2.734157in}}%
\pgfpathlineto{\pgfqpoint{4.779940in}{2.777962in}}%
\pgfpathlineto{\pgfqpoint{4.824607in}{2.823605in}}%
\pgfpathlineto{\pgfqpoint{4.824607in}{2.823605in}}%
\pgfusepath{stroke}%
\end{pgfscope}%
\begin{pgfscope}%
\pgfpathrectangle{\pgfqpoint{0.647939in}{0.492442in}}{\pgfqpoint{4.273799in}{2.331163in}}%
\pgfusepath{clip}%
\pgfsetbuttcap%
\pgfsetroundjoin%
\pgfsetlinewidth{0.301125pt}%
\definecolor{currentstroke}{rgb}{0.500000,0.500000,0.500000}%
\pgfsetstrokecolor{currentstroke}%
\pgfsetstrokeopacity{0.300000}%
\pgfsetdash{}{0pt}%
\pgfpathmoveto{\pgfqpoint{4.144684in}{2.823605in}}%
\pgfpathlineto{\pgfqpoint{4.144684in}{2.823605in}}%
\pgfpathlineto{\pgfqpoint{4.184641in}{2.776613in}}%
\pgfpathlineto{\pgfqpoint{4.226490in}{2.730116in}}%
\pgfpathlineto{\pgfqpoint{4.271172in}{2.684419in}}%
\pgfpathlineto{\pgfqpoint{4.320339in}{2.640134in}}%
\pgfpathlineto{\pgfqpoint{4.377280in}{2.598863in}}%
\pgfpathlineto{\pgfqpoint{4.377280in}{2.598863in}}%
\pgfpathlineto{\pgfqpoint{4.433722in}{2.570885in}}%
\pgfpathlineto{\pgfqpoint{4.433722in}{2.570885in}}%
\pgfpathlineto{\pgfqpoint{4.480695in}{2.558828in}}%
\pgfpathlineto{\pgfqpoint{4.480695in}{2.558828in}}%
\pgfpathlineto{\pgfqpoint{4.524754in}{2.557487in}}%
\pgfpathlineto{\pgfqpoint{4.566847in}{2.565251in}}%
\pgfpathlineto{\pgfqpoint{4.605473in}{2.579882in}}%
\pgfpathlineto{\pgfqpoint{4.646551in}{2.602860in}}%
\pgfpathlineto{\pgfqpoint{4.692721in}{2.636964in}}%
\pgfpathlineto{\pgfqpoint{4.741428in}{2.681285in}}%
\pgfpathlineto{\pgfqpoint{4.785018in}{2.727245in}}%
\pgfusepath{stroke}%
\end{pgfscope}%
\begin{pgfscope}%
\pgfpathrectangle{\pgfqpoint{0.647939in}{0.492442in}}{\pgfqpoint{4.273799in}{2.331163in}}%
\pgfusepath{clip}%
\pgfsetbuttcap%
\pgfsetroundjoin%
\pgfsetlinewidth{0.301125pt}%
\definecolor{currentstroke}{rgb}{0.500000,0.500000,0.500000}%
\pgfsetstrokecolor{currentstroke}%
\pgfsetstrokeopacity{0.300000}%
\pgfsetdash{}{0pt}%
\pgfpathmoveto{\pgfqpoint{4.047552in}{2.823605in}}%
\pgfpathlineto{\pgfqpoint{4.047552in}{2.823605in}}%
\pgfpathlineto{\pgfqpoint{4.083950in}{2.775759in}}%
\pgfpathlineto{\pgfqpoint{4.121064in}{2.728077in}}%
\pgfpathlineto{\pgfqpoint{4.159237in}{2.680647in}}%
\pgfpathlineto{\pgfqpoint{4.198988in}{2.633607in}}%
\pgfpathlineto{\pgfqpoint{4.241159in}{2.587208in}}%
\pgfpathlineto{\pgfqpoint{4.287252in}{2.541959in}}%
\pgfpathlineto{\pgfqpoint{4.340304in}{2.499156in}}%
\pgfpathlineto{\pgfqpoint{4.407384in}{2.463327in}}%
\pgfpathlineto{\pgfqpoint{4.407384in}{2.463327in}}%
\pgfpathlineto{\pgfqpoint{4.450495in}{2.452830in}}%
\pgfpathlineto{\pgfqpoint{4.450495in}{2.452830in}}%
\pgfpathlineto{\pgfqpoint{4.491546in}{2.452796in}}%
\pgfpathlineto{\pgfqpoint{4.529838in}{2.461327in}}%
\pgfpathlineto{\pgfqpoint{4.565739in}{2.476445in}}%
\pgfpathlineto{\pgfqpoint{4.605153in}{2.500305in}}%
\pgfusepath{stroke}%
\end{pgfscope}%
\begin{pgfscope}%
\pgfpathrectangle{\pgfqpoint{0.647939in}{0.492442in}}{\pgfqpoint{4.273799in}{2.331163in}}%
\pgfusepath{clip}%
\pgfsetbuttcap%
\pgfsetroundjoin%
\pgfsetlinewidth{0.301125pt}%
\definecolor{currentstroke}{rgb}{0.500000,0.500000,0.500000}%
\pgfsetstrokecolor{currentstroke}%
\pgfsetstrokeopacity{0.300000}%
\pgfsetdash{}{0pt}%
\pgfpathmoveto{\pgfqpoint{3.950420in}{2.823605in}}%
\pgfpathlineto{\pgfqpoint{3.950420in}{2.823605in}}%
\pgfpathlineto{\pgfqpoint{3.984692in}{2.775293in}}%
\pgfpathlineto{\pgfqpoint{4.019051in}{2.726998in}}%
\pgfpathlineto{\pgfqpoint{4.053638in}{2.678753in}}%
\pgfpathlineto{\pgfqpoint{4.088650in}{2.630599in}}%
\pgfpathlineto{\pgfqpoint{4.124351in}{2.582598in}}%
\pgfpathlineto{\pgfqpoint{4.161127in}{2.534842in}}%
\pgfpathlineto{\pgfqpoint{4.199597in}{2.487490in}}%
\pgfpathlineto{\pgfqpoint{4.240860in}{2.440856in}}%
\pgfpathlineto{\pgfqpoint{4.287150in}{2.395703in}}%
\pgfpathlineto{\pgfqpoint{4.343939in}{2.354658in}}%
\pgfpathlineto{\pgfqpoint{4.343939in}{2.354658in}}%
\pgfpathlineto{\pgfqpoint{4.384149in}{2.337513in}}%
\pgfpathlineto{\pgfqpoint{4.384149in}{2.337513in}}%
\pgfpathlineto{\pgfqpoint{4.421756in}{2.331757in}}%
\pgfpathlineto{\pgfqpoint{4.460349in}{2.336311in}}%
\pgfpathlineto{\pgfqpoint{4.492865in}{2.347684in}}%
\pgfpathlineto{\pgfqpoint{4.527646in}{2.366700in}}%
\pgfpathlineto{\pgfqpoint{4.567399in}{2.395921in}}%
\pgfpathlineto{\pgfqpoint{4.615624in}{2.440028in}}%
\pgfusepath{stroke}%
\end{pgfscope}%
\begin{pgfscope}%
\pgfpathrectangle{\pgfqpoint{0.647939in}{0.492442in}}{\pgfqpoint{4.273799in}{2.331163in}}%
\pgfusepath{clip}%
\pgfsetbuttcap%
\pgfsetroundjoin%
\pgfsetlinewidth{0.301125pt}%
\definecolor{currentstroke}{rgb}{0.500000,0.500000,0.500000}%
\pgfsetstrokecolor{currentstroke}%
\pgfsetstrokeopacity{0.300000}%
\pgfsetdash{}{0pt}%
\pgfpathmoveto{\pgfqpoint{3.853289in}{2.823605in}}%
\pgfpathlineto{\pgfqpoint{3.853289in}{2.823605in}}%
\pgfpathlineto{\pgfqpoint{3.886402in}{2.775053in}}%
\pgfpathlineto{\pgfqpoint{3.919242in}{2.726445in}}%
\pgfpathlineto{\pgfqpoint{3.951870in}{2.677794in}}%
\pgfpathlineto{\pgfqpoint{3.984345in}{2.629114in}}%
\pgfpathlineto{\pgfqpoint{4.016736in}{2.580417in}}%
\pgfpathlineto{\pgfqpoint{4.049157in}{2.531726in}}%
\pgfpathlineto{\pgfqpoint{4.081768in}{2.483074in}}%
\pgfpathlineto{\pgfqpoint{4.114773in}{2.434502in}}%
\pgfpathlineto{\pgfqpoint{4.148486in}{2.386076in}}%
\pgfpathlineto{\pgfqpoint{4.183471in}{2.337922in}}%
\pgfpathlineto{\pgfqpoint{4.220835in}{2.290318in}}%
\pgfpathlineto{\pgfqpoint{4.263137in}{2.244033in}}%
\pgfpathlineto{\pgfqpoint{4.318795in}{2.202827in}}%
\pgfpathlineto{\pgfqpoint{4.318795in}{2.202827in}}%
\pgfpathlineto{\pgfqpoint{4.349345in}{2.192861in}}%
\pgfpathlineto{\pgfqpoint{4.349345in}{2.192861in}}%
\pgfpathlineto{\pgfqpoint{4.378953in}{2.192900in}}%
\pgfpathlineto{\pgfqpoint{4.406128in}{2.200496in}}%
\pgfpathlineto{\pgfqpoint{4.433907in}{2.214594in}}%
\pgfpathlineto{\pgfqpoint{4.466023in}{2.237358in}}%
\pgfpathlineto{\pgfqpoint{4.507871in}{2.274569in}}%
\pgfpathlineto{\pgfqpoint{4.551987in}{2.320187in}}%
\pgfusepath{stroke}%
\end{pgfscope}%
\begin{pgfscope}%
\pgfpathrectangle{\pgfqpoint{0.647939in}{0.492442in}}{\pgfqpoint{4.273799in}{2.331163in}}%
\pgfusepath{clip}%
\pgfsetbuttcap%
\pgfsetroundjoin%
\pgfsetlinewidth{0.301125pt}%
\definecolor{currentstroke}{rgb}{0.500000,0.500000,0.500000}%
\pgfsetstrokecolor{currentstroke}%
\pgfsetstrokeopacity{0.300000}%
\pgfsetdash{}{0pt}%
\pgfpathmoveto{\pgfqpoint{3.756157in}{2.823605in}}%
\pgfpathlineto{\pgfqpoint{3.756157in}{2.823605in}}%
\pgfpathlineto{\pgfqpoint{3.788853in}{2.774969in}}%
\pgfpathlineto{\pgfqpoint{3.821034in}{2.726230in}}%
\pgfpathlineto{\pgfqpoint{3.852709in}{2.677393in}}%
\pgfpathlineto{\pgfqpoint{3.883886in}{2.628460in}}%
\pgfpathlineto{\pgfqpoint{3.914588in}{2.579438in}}%
\pgfpathlineto{\pgfqpoint{3.944833in}{2.530332in}}%
\pgfpathlineto{\pgfqpoint{3.974617in}{2.481143in}}%
\pgfpathlineto{\pgfqpoint{4.003951in}{2.431873in}}%
\pgfpathlineto{\pgfqpoint{4.032853in}{2.382528in}}%
\pgfpathlineto{\pgfqpoint{4.061304in}{2.333105in}}%
\pgfpathlineto{\pgfqpoint{4.089299in}{2.283605in}}%
\pgfpathlineto{\pgfqpoint{4.116829in}{2.234030in}}%
\pgfpathlineto{\pgfqpoint{4.143808in}{2.184365in}}%
\pgfpathlineto{\pgfqpoint{4.170130in}{2.134602in}}%
\pgfpathlineto{\pgfqpoint{4.195482in}{2.084709in}}%
\pgfpathlineto{\pgfqpoint{4.218642in}{2.034541in}}%
\pgfpathlineto{\pgfqpoint{4.218642in}{2.034541in}}%
\pgfpathlineto{\pgfqpoint{4.230326in}{1.997341in}}%
\pgfpathlineto{\pgfqpoint{4.230326in}{1.997341in}}%
\pgfpathlineto{\pgfqpoint{4.230206in}{1.979309in}}%
\pgfpathlineto{\pgfqpoint{4.230206in}{1.979309in}}%
\pgfpathlineto{\pgfqpoint{4.219162in}{1.959989in}}%
\pgfpathlineto{\pgfqpoint{4.202857in}{1.941675in}}%
\pgfpathlineto{\pgfqpoint{4.173838in}{1.915601in}}%
\pgfusepath{stroke}%
\end{pgfscope}%
\begin{pgfscope}%
\pgfpathrectangle{\pgfqpoint{0.647939in}{0.492442in}}{\pgfqpoint{4.273799in}{2.331163in}}%
\pgfusepath{clip}%
\pgfsetbuttcap%
\pgfsetroundjoin%
\pgfsetlinewidth{0.301125pt}%
\definecolor{currentstroke}{rgb}{0.500000,0.500000,0.500000}%
\pgfsetstrokecolor{currentstroke}%
\pgfsetstrokeopacity{0.300000}%
\pgfsetdash{}{0pt}%
\pgfpathmoveto{\pgfqpoint{3.659025in}{2.823605in}}%
\pgfpathlineto{\pgfqpoint{3.659025in}{2.823605in}}%
\pgfpathlineto{\pgfqpoint{3.691910in}{2.775007in}}%
\pgfpathlineto{\pgfqpoint{3.724082in}{2.726266in}}%
\pgfpathlineto{\pgfqpoint{3.755534in}{2.677386in}}%
\pgfpathlineto{\pgfqpoint{3.786268in}{2.628371in}}%
\pgfpathlineto{\pgfqpoint{3.816278in}{2.579222in}}%
\pgfpathlineto{\pgfqpoint{3.845527in}{2.529937in}}%
\pgfpathlineto{\pgfqpoint{3.873984in}{2.480514in}}%
\pgfpathlineto{\pgfqpoint{3.901604in}{2.430951in}}%
\pgfpathlineto{\pgfqpoint{3.928291in}{2.381235in}}%
\pgfpathlineto{\pgfqpoint{3.953944in}{2.331359in}}%
\pgfpathlineto{\pgfqpoint{3.978401in}{2.281304in}}%
\pgfpathlineto{\pgfqpoint{4.001405in}{2.231046in}}%
\pgfpathlineto{\pgfqpoint{4.022581in}{2.180552in}}%
\pgfpathlineto{\pgfqpoint{4.041325in}{2.129776in}}%
\pgfpathlineto{\pgfqpoint{4.056646in}{2.078667in}}%
\pgfpathlineto{\pgfqpoint{4.066892in}{2.027197in}}%
\pgfpathlineto{\pgfqpoint{4.069316in}{1.975478in}}%
\pgfpathlineto{\pgfqpoint{4.059948in}{1.924096in}}%
\pgfpathlineto{\pgfqpoint{4.035333in}{1.874390in}}%
\pgfpathlineto{\pgfqpoint{3.995526in}{1.827754in}}%
\pgfusepath{stroke}%
\end{pgfscope}%
\begin{pgfscope}%
\pgfpathrectangle{\pgfqpoint{0.647939in}{0.492442in}}{\pgfqpoint{4.273799in}{2.331163in}}%
\pgfusepath{clip}%
\pgfsetbuttcap%
\pgfsetroundjoin%
\pgfsetlinewidth{0.301125pt}%
\definecolor{currentstroke}{rgb}{0.500000,0.500000,0.500000}%
\pgfsetstrokecolor{currentstroke}%
\pgfsetstrokeopacity{0.300000}%
\pgfsetdash{}{0pt}%
\pgfpathmoveto{\pgfqpoint{3.561893in}{2.823605in}}%
\pgfpathlineto{\pgfqpoint{3.561893in}{2.823605in}}%
\pgfpathlineto{\pgfqpoint{3.595496in}{2.775154in}}%
\pgfpathlineto{\pgfqpoint{3.628217in}{2.726522in}}%
\pgfpathlineto{\pgfqpoint{3.660052in}{2.677716in}}%
\pgfpathlineto{\pgfqpoint{3.690996in}{2.628740in}}%
\pgfpathlineto{\pgfqpoint{3.721019in}{2.579594in}}%
\pgfpathlineto{\pgfqpoint{3.750080in}{2.530276in}}%
\pgfpathlineto{\pgfqpoint{3.778135in}{2.480785in}}%
\pgfusepath{stroke}%
\end{pgfscope}%
\begin{pgfscope}%
\pgfpathrectangle{\pgfqpoint{0.647939in}{0.492442in}}{\pgfqpoint{4.273799in}{2.331163in}}%
\pgfusepath{clip}%
\pgfsetbuttcap%
\pgfsetroundjoin%
\pgfsetlinewidth{0.301125pt}%
\definecolor{currentstroke}{rgb}{0.500000,0.500000,0.500000}%
\pgfsetstrokecolor{currentstroke}%
\pgfsetstrokeopacity{0.300000}%
\pgfsetdash{}{0pt}%
\pgfpathmoveto{\pgfqpoint{3.464761in}{2.823605in}}%
\pgfpathlineto{\pgfqpoint{3.464761in}{2.823605in}}%
\pgfpathlineto{\pgfqpoint{3.499597in}{2.775413in}}%
\pgfpathlineto{\pgfqpoint{3.533393in}{2.727002in}}%
\pgfpathlineto{\pgfqpoint{3.566151in}{2.678379in}}%
\pgfpathlineto{\pgfqpoint{3.597854in}{2.629548in}}%
\pgfpathlineto{\pgfqpoint{3.628469in}{2.580510in}}%
\pgfpathlineto{\pgfqpoint{3.657961in}{2.531269in}}%
\pgfpathlineto{\pgfqpoint{3.686279in}{2.481823in}}%
\pgfpathlineto{\pgfqpoint{3.713335in}{2.432168in}}%
\pgfpathlineto{\pgfqpoint{3.739029in}{2.382298in}}%
\pgfpathlineto{\pgfqpoint{3.763221in}{2.332205in}}%
\pgfpathlineto{\pgfqpoint{3.785712in}{2.281878in}}%
\pgfpathlineto{\pgfqpoint{3.806243in}{2.231303in}}%
\pgfpathlineto{\pgfqpoint{3.824453in}{2.180466in}}%
\pgfpathlineto{\pgfqpoint{3.839846in}{2.129356in}}%
\pgfpathlineto{\pgfqpoint{3.851739in}{2.077974in}}%
\pgfpathlineto{\pgfqpoint{3.859180in}{2.026350in}}%
\pgfpathlineto{\pgfqpoint{3.860838in}{1.974587in}}%
\pgfpathlineto{\pgfqpoint{3.854920in}{1.922944in}}%
\pgfpathlineto{\pgfqpoint{3.839202in}{1.871967in}}%
\pgfpathlineto{\pgfqpoint{3.811206in}{1.822662in}}%
\pgfpathlineto{\pgfqpoint{3.768831in}{1.776624in}}%
\pgfpathlineto{\pgfqpoint{3.710970in}{1.736067in}}%
\pgfpathlineto{\pgfqpoint{3.643252in}{1.705722in}}%
\pgfpathlineto{\pgfqpoint{3.571376in}{1.686888in}}%
\pgfusepath{stroke}%
\end{pgfscope}%
\begin{pgfscope}%
\pgfpathrectangle{\pgfqpoint{0.647939in}{0.492442in}}{\pgfqpoint{4.273799in}{2.331163in}}%
\pgfusepath{clip}%
\pgfsetbuttcap%
\pgfsetroundjoin%
\pgfsetlinewidth{0.301125pt}%
\definecolor{currentstroke}{rgb}{0.500000,0.500000,0.500000}%
\pgfsetstrokecolor{currentstroke}%
\pgfsetstrokeopacity{0.300000}%
\pgfsetdash{}{0pt}%
\pgfpathmoveto{\pgfqpoint{3.367630in}{2.823605in}}%
\pgfpathlineto{\pgfqpoint{3.367630in}{2.823605in}}%
\pgfpathlineto{\pgfqpoint{3.404231in}{2.775805in}}%
\pgfpathlineto{\pgfqpoint{3.439631in}{2.727736in}}%
\pgfpathlineto{\pgfqpoint{3.473826in}{2.679409in}}%
\pgfpathlineto{\pgfqpoint{3.506803in}{2.630830in}}%
\pgfpathlineto{\pgfqpoint{3.538534in}{2.582005in}}%
\pgfpathlineto{\pgfqpoint{3.568990in}{2.532939in}}%
\pgfpathlineto{\pgfqpoint{3.598120in}{2.483634in}}%
\pgfpathlineto{\pgfqpoint{3.625836in}{2.434087in}}%
\pgfpathlineto{\pgfqpoint{3.652043in}{2.384297in}}%
\pgfpathlineto{\pgfqpoint{3.676605in}{2.334259in}}%
\pgfpathlineto{\pgfqpoint{3.699327in}{2.283963in}}%
\pgfpathlineto{\pgfqpoint{3.719963in}{2.233401in}}%
\pgfpathlineto{\pgfqpoint{3.738171in}{2.182564in}}%
\pgfpathlineto{\pgfqpoint{3.753492in}{2.131448in}}%
\pgfpathlineto{\pgfqpoint{3.765299in}{2.080059in}}%
\pgfpathlineto{\pgfqpoint{3.772734in}{2.028435in}}%
\pgfpathlineto{\pgfqpoint{3.774600in}{1.976671in}}%
\pgfpathlineto{\pgfqpoint{3.769256in}{1.925002in}}%
\pgfpathlineto{\pgfqpoint{3.754531in}{1.873924in}}%
\pgfpathlineto{\pgfqpoint{3.727674in}{1.824425in}}%
\pgfpathlineto{\pgfqpoint{3.685639in}{1.778335in}}%
\pgfpathlineto{\pgfqpoint{3.627032in}{1.739330in}}%
\pgfusepath{stroke}%
\end{pgfscope}%
\begin{pgfscope}%
\pgfpathrectangle{\pgfqpoint{0.647939in}{0.492442in}}{\pgfqpoint{4.273799in}{2.331163in}}%
\pgfusepath{clip}%
\pgfsetbuttcap%
\pgfsetroundjoin%
\pgfsetlinewidth{0.301125pt}%
\definecolor{currentstroke}{rgb}{0.500000,0.500000,0.500000}%
\pgfsetstrokecolor{currentstroke}%
\pgfsetstrokeopacity{0.300000}%
\pgfsetdash{}{0pt}%
\pgfpathmoveto{\pgfqpoint{3.173366in}{2.823605in}}%
\pgfpathlineto{\pgfqpoint{3.173366in}{2.823605in}}%
\pgfpathlineto{\pgfqpoint{3.215267in}{2.777121in}}%
\pgfpathlineto{\pgfqpoint{3.255547in}{2.730212in}}%
\pgfpathlineto{\pgfqpoint{3.294223in}{2.682902in}}%
\pgfpathlineto{\pgfqpoint{3.331305in}{2.635214in}}%
\pgfpathlineto{\pgfqpoint{3.366784in}{2.587163in}}%
\pgfpathlineto{\pgfqpoint{3.400649in}{2.538768in}}%
\pgfpathlineto{\pgfqpoint{3.432867in}{2.490039in}}%
\pgfpathlineto{\pgfqpoint{3.463374in}{2.440985in}}%
\pgfpathlineto{\pgfqpoint{3.492079in}{2.391607in}}%
\pgfpathlineto{\pgfqpoint{3.518867in}{2.341911in}}%
\pgfpathlineto{\pgfqpoint{3.543566in}{2.291895in}}%
\pgfpathlineto{\pgfqpoint{3.565948in}{2.241556in}}%
\pgfpathlineto{\pgfqpoint{3.585704in}{2.190892in}}%
\pgfpathlineto{\pgfqpoint{3.602410in}{2.139905in}}%
\pgfpathlineto{\pgfqpoint{3.615488in}{2.088608in}}%
\pgfpathlineto{\pgfqpoint{3.624119in}{2.037039in}}%
\pgfpathlineto{\pgfqpoint{3.627150in}{1.985296in}}%
\pgfpathlineto{\pgfqpoint{3.622908in}{1.933601in}}%
\pgfpathlineto{\pgfqpoint{3.608885in}{1.882465in}}%
\pgfpathlineto{\pgfqpoint{3.581336in}{1.833108in}}%
\pgfpathlineto{\pgfqpoint{3.534916in}{1.788490in}}%
\pgfpathlineto{\pgfqpoint{3.534916in}{1.788490in}}%
\pgfpathlineto{\pgfqpoint{3.489254in}{1.763565in}}%
\pgfpathlineto{\pgfqpoint{3.431071in}{1.747485in}}%
\pgfpathlineto{\pgfqpoint{3.375007in}{1.743519in}}%
\pgfpathlineto{\pgfqpoint{3.321326in}{1.748511in}}%
\pgfpathlineto{\pgfqpoint{3.268993in}{1.761173in}}%
\pgfusepath{stroke}%
\end{pgfscope}%
\begin{pgfscope}%
\pgfpathrectangle{\pgfqpoint{0.647939in}{0.492442in}}{\pgfqpoint{4.273799in}{2.331163in}}%
\pgfusepath{clip}%
\pgfsetbuttcap%
\pgfsetroundjoin%
\pgfsetlinewidth{0.301125pt}%
\definecolor{currentstroke}{rgb}{0.500000,0.500000,0.500000}%
\pgfsetstrokecolor{currentstroke}%
\pgfsetstrokeopacity{0.300000}%
\pgfsetdash{}{0pt}%
\pgfpathmoveto{\pgfqpoint{2.979102in}{2.823605in}}%
\pgfpathlineto{\pgfqpoint{2.979102in}{2.823605in}}%
\pgfpathlineto{\pgfqpoint{3.029003in}{2.779536in}}%
\pgfpathlineto{\pgfqpoint{3.076688in}{2.734744in}}%
\pgfpathlineto{\pgfqpoint{3.122207in}{2.689287in}}%
\pgfpathlineto{\pgfqpoint{3.165604in}{2.643216in}}%
\pgfpathlineto{\pgfqpoint{3.206912in}{2.596576in}}%
\pgfpathlineto{\pgfqpoint{3.246150in}{2.549407in}}%
\pgfpathlineto{\pgfqpoint{3.283317in}{2.501740in}}%
\pgfpathlineto{\pgfqpoint{3.318391in}{2.453604in}}%
\pgfpathlineto{\pgfqpoint{3.351326in}{2.405020in}}%
\pgfpathlineto{\pgfqpoint{3.382034in}{2.356006in}}%
\pgfpathlineto{\pgfqpoint{3.410371in}{2.306568in}}%
\pgfpathlineto{\pgfqpoint{3.436150in}{2.256715in}}%
\pgfpathlineto{\pgfqpoint{3.459096in}{2.206455in}}%
\pgfpathlineto{\pgfqpoint{3.478815in}{2.155791in}}%
\pgfpathlineto{\pgfqpoint{3.494753in}{2.104738in}}%
\pgfpathlineto{\pgfqpoint{3.506111in}{2.053328in}}%
\pgfpathlineto{\pgfqpoint{3.511702in}{2.001648in}}%
\pgfpathlineto{\pgfqpoint{3.509675in}{1.949916in}}%
\pgfpathlineto{\pgfqpoint{3.497000in}{1.898709in}}%
\pgfpathlineto{\pgfqpoint{3.468370in}{1.849693in}}%
\pgfpathlineto{\pgfqpoint{3.468370in}{1.849693in}}%
\pgfpathlineto{\pgfqpoint{3.433493in}{1.818438in}}%
\pgfpathlineto{\pgfqpoint{3.433493in}{1.818438in}}%
\pgfpathlineto{\pgfqpoint{3.393119in}{1.798655in}}%
\pgfusepath{stroke}%
\end{pgfscope}%
\begin{pgfscope}%
\pgfpathrectangle{\pgfqpoint{0.647939in}{0.492442in}}{\pgfqpoint{4.273799in}{2.331163in}}%
\pgfusepath{clip}%
\pgfsetbuttcap%
\pgfsetroundjoin%
\pgfsetlinewidth{0.301125pt}%
\definecolor{currentstroke}{rgb}{0.500000,0.500000,0.500000}%
\pgfsetstrokecolor{currentstroke}%
\pgfsetstrokeopacity{0.300000}%
\pgfsetdash{}{0pt}%
\pgfpathmoveto{\pgfqpoint{2.784839in}{2.823605in}}%
\pgfpathlineto{\pgfqpoint{2.784839in}{2.823605in}}%
\pgfpathlineto{\pgfqpoint{2.845535in}{2.783784in}}%
\pgfpathlineto{\pgfqpoint{2.903282in}{2.742676in}}%
\pgfpathlineto{\pgfqpoint{2.958108in}{2.700397in}}%
\pgfpathlineto{\pgfqpoint{3.010081in}{2.657057in}}%
\pgfpathlineto{\pgfqpoint{3.059279in}{2.612760in}}%
\pgfpathlineto{\pgfqpoint{3.105773in}{2.567599in}}%
\pgfpathlineto{\pgfqpoint{3.149624in}{2.521656in}}%
\pgfpathlineto{\pgfqpoint{3.190872in}{2.475002in}}%
\pgfusepath{stroke}%
\end{pgfscope}%
\begin{pgfscope}%
\pgfpathrectangle{\pgfqpoint{0.647939in}{0.492442in}}{\pgfqpoint{4.273799in}{2.331163in}}%
\pgfusepath{clip}%
\pgfsetbuttcap%
\pgfsetroundjoin%
\pgfsetlinewidth{0.301125pt}%
\definecolor{currentstroke}{rgb}{0.500000,0.500000,0.500000}%
\pgfsetstrokecolor{currentstroke}%
\pgfsetstrokeopacity{0.300000}%
\pgfsetdash{}{0pt}%
\pgfpathmoveto{\pgfqpoint{2.590575in}{2.823605in}}%
\pgfpathlineto{\pgfqpoint{2.590575in}{2.823605in}}%
\pgfpathlineto{\pgfqpoint{2.663444in}{2.790422in}}%
\pgfpathlineto{\pgfqpoint{2.732959in}{2.755175in}}%
\pgfpathlineto{\pgfqpoint{2.798945in}{2.717958in}}%
\pgfpathlineto{\pgfqpoint{2.861324in}{2.678933in}}%
\pgfpathlineto{\pgfqpoint{2.920130in}{2.638288in}}%
\pgfpathlineto{\pgfqpoint{2.975457in}{2.596208in}}%
\pgfpathlineto{\pgfqpoint{3.027418in}{2.552865in}}%
\pgfpathlineto{\pgfqpoint{3.076125in}{2.508412in}}%
\pgfpathlineto{\pgfqpoint{3.121674in}{2.462973in}}%
\pgfpathlineto{\pgfqpoint{3.164132in}{2.416651in}}%
\pgfpathlineto{\pgfqpoint{3.203518in}{2.369527in}}%
\pgfpathlineto{\pgfqpoint{3.239798in}{2.321664in}}%
\pgfpathlineto{\pgfqpoint{3.272866in}{2.273114in}}%
\pgfpathlineto{\pgfqpoint{3.302530in}{2.223917in}}%
\pgfpathlineto{\pgfqpoint{3.328456in}{2.174098in}}%
\pgfpathlineto{\pgfqpoint{3.350123in}{2.123681in}}%
\pgfpathlineto{\pgfqpoint{3.366712in}{2.072701in}}%
\pgfpathlineto{\pgfqpoint{3.376883in}{2.021237in}}%
\pgfpathlineto{\pgfqpoint{3.378297in}{1.969526in}}%
\pgfpathlineto{\pgfqpoint{3.366353in}{1.918375in}}%
\pgfpathlineto{\pgfqpoint{3.366353in}{1.918375in}}%
\pgfpathlineto{\pgfqpoint{3.343109in}{1.882246in}}%
\pgfpathlineto{\pgfqpoint{3.343109in}{1.882246in}}%
\pgfpathlineto{\pgfqpoint{3.313155in}{1.859876in}}%
\pgfpathlineto{\pgfqpoint{3.313155in}{1.859876in}}%
\pgfpathlineto{\pgfqpoint{3.278850in}{1.848518in}}%
\pgfpathlineto{\pgfqpoint{3.238736in}{1.846915in}}%
\pgfpathlineto{\pgfqpoint{3.202823in}{1.853471in}}%
\pgfpathlineto{\pgfqpoint{3.167590in}{1.866746in}}%
\pgfpathlineto{\pgfqpoint{3.131581in}{1.887984in}}%
\pgfusepath{stroke}%
\end{pgfscope}%
\begin{pgfscope}%
\pgfpathrectangle{\pgfqpoint{0.647939in}{0.492442in}}{\pgfqpoint{4.273799in}{2.331163in}}%
\pgfusepath{clip}%
\pgfsetbuttcap%
\pgfsetroundjoin%
\pgfsetlinewidth{0.301125pt}%
\definecolor{currentstroke}{rgb}{0.500000,0.500000,0.500000}%
\pgfsetstrokecolor{currentstroke}%
\pgfsetstrokeopacity{0.300000}%
\pgfsetdash{}{0pt}%
\pgfpathmoveto{\pgfqpoint{2.396312in}{2.823605in}}%
\pgfpathlineto{\pgfqpoint{2.396312in}{2.823605in}}%
\pgfpathlineto{\pgfqpoint{2.479076in}{2.798242in}}%
\pgfpathlineto{\pgfqpoint{2.559422in}{2.770689in}}%
\pgfpathlineto{\pgfqpoint{2.636602in}{2.740572in}}%
\pgfpathlineto{\pgfqpoint{2.710072in}{2.707809in}}%
\pgfpathlineto{\pgfqpoint{2.779503in}{2.672531in}}%
\pgfpathlineto{\pgfqpoint{2.844815in}{2.634974in}}%
\pgfpathlineto{\pgfqpoint{2.906039in}{2.595414in}}%
\pgfusepath{stroke}%
\end{pgfscope}%
\begin{pgfscope}%
\pgfpathrectangle{\pgfqpoint{0.647939in}{0.492442in}}{\pgfqpoint{4.273799in}{2.331163in}}%
\pgfusepath{clip}%
\pgfsetbuttcap%
\pgfsetroundjoin%
\pgfsetlinewidth{0.301125pt}%
\definecolor{currentstroke}{rgb}{0.500000,0.500000,0.500000}%
\pgfsetstrokecolor{currentstroke}%
\pgfsetstrokeopacity{0.300000}%
\pgfsetdash{}{0pt}%
\pgfpathmoveto{\pgfqpoint{2.104916in}{2.823605in}}%
\pgfpathlineto{\pgfqpoint{2.104916in}{2.823605in}}%
\pgfpathlineto{\pgfqpoint{2.192105in}{2.803120in}}%
\pgfpathlineto{\pgfqpoint{2.280513in}{2.784214in}}%
\pgfpathlineto{\pgfqpoint{2.368768in}{2.765113in}}%
\pgfpathlineto{\pgfqpoint{2.455675in}{2.744294in}}%
\pgfpathlineto{\pgfqpoint{2.540132in}{2.720704in}}%
\pgfusepath{stroke}%
\end{pgfscope}%
\begin{pgfscope}%
\pgfpathrectangle{\pgfqpoint{0.647939in}{0.492442in}}{\pgfqpoint{4.273799in}{2.331163in}}%
\pgfusepath{clip}%
\pgfsetbuttcap%
\pgfsetroundjoin%
\pgfsetlinewidth{0.301125pt}%
\definecolor{currentstroke}{rgb}{0.500000,0.500000,0.500000}%
\pgfsetstrokecolor{currentstroke}%
\pgfsetstrokeopacity{0.300000}%
\pgfsetdash{}{0pt}%
\pgfpathmoveto{\pgfqpoint{1.910652in}{2.823605in}}%
\pgfpathlineto{\pgfqpoint{1.910652in}{2.823605in}}%
\pgfpathlineto{\pgfqpoint{1.991418in}{2.796570in}}%
\pgfpathlineto{\pgfqpoint{2.077541in}{2.774942in}}%
\pgfpathlineto{\pgfqpoint{2.166931in}{2.757554in}}%
\pgfpathlineto{\pgfqpoint{2.257706in}{2.742371in}}%
\pgfpathlineto{\pgfqpoint{2.348469in}{2.727170in}}%
\pgfpathlineto{\pgfqpoint{2.438032in}{2.710054in}}%
\pgfpathlineto{\pgfqpoint{2.525234in}{2.689711in}}%
\pgfpathlineto{\pgfqpoint{2.608983in}{2.665476in}}%
\pgfpathlineto{\pgfqpoint{2.688413in}{2.637245in}}%
\pgfpathlineto{\pgfqpoint{2.763009in}{2.605298in}}%
\pgfpathlineto{\pgfqpoint{2.832543in}{2.570102in}}%
\pgfpathlineto{\pgfqpoint{2.897055in}{2.532162in}}%
\pgfpathlineto{\pgfqpoint{2.956762in}{2.491933in}}%
\pgfpathlineto{\pgfqpoint{3.011901in}{2.449804in}}%
\pgfpathlineto{\pgfqpoint{3.062716in}{2.406072in}}%
\pgfpathlineto{\pgfqpoint{3.109387in}{2.360978in}}%
\pgfpathlineto{\pgfqpoint{3.152021in}{2.314707in}}%
\pgfpathlineto{\pgfqpoint{3.190629in}{2.267395in}}%
\pgfusepath{stroke}%
\end{pgfscope}%
\begin{pgfscope}%
\pgfpathrectangle{\pgfqpoint{0.647939in}{0.492442in}}{\pgfqpoint{4.273799in}{2.331163in}}%
\pgfusepath{clip}%
\pgfsetbuttcap%
\pgfsetroundjoin%
\pgfsetlinewidth{0.301125pt}%
\definecolor{currentstroke}{rgb}{0.500000,0.500000,0.500000}%
\pgfsetstrokecolor{currentstroke}%
\pgfsetstrokeopacity{0.300000}%
\pgfsetdash{}{0pt}%
\pgfpathmoveto{\pgfqpoint{1.716389in}{2.823605in}}%
\pgfpathlineto{\pgfqpoint{1.716389in}{2.823605in}}%
\pgfpathlineto{\pgfqpoint{1.779763in}{2.785142in}}%
\pgfpathlineto{\pgfqpoint{1.851495in}{2.751418in}}%
\pgfpathlineto{\pgfqpoint{1.931928in}{2.724256in}}%
\pgfpathlineto{\pgfqpoint{2.019458in}{2.704668in}}%
\pgfpathlineto{\pgfqpoint{2.111214in}{2.691717in}}%
\pgfpathlineto{\pgfqpoint{2.204703in}{2.682757in}}%
\pgfpathlineto{\pgfqpoint{2.298484in}{2.674655in}}%
\pgfpathlineto{\pgfqpoint{2.391604in}{2.664729in}}%
\pgfusepath{stroke}%
\end{pgfscope}%
\begin{pgfscope}%
\pgfpathrectangle{\pgfqpoint{0.647939in}{0.492442in}}{\pgfqpoint{4.273799in}{2.331163in}}%
\pgfusepath{clip}%
\pgfsetbuttcap%
\pgfsetroundjoin%
\pgfsetlinewidth{0.301125pt}%
\definecolor{currentstroke}{rgb}{0.500000,0.500000,0.500000}%
\pgfsetstrokecolor{currentstroke}%
\pgfsetstrokeopacity{0.300000}%
\pgfsetdash{}{0pt}%
\pgfpathmoveto{\pgfqpoint{1.619257in}{2.823605in}}%
\pgfpathlineto{\pgfqpoint{1.619257in}{2.823605in}}%
\pgfpathlineto{\pgfqpoint{1.671663in}{2.780467in}}%
\pgfpathlineto{\pgfqpoint{1.730940in}{2.740090in}}%
\pgfpathlineto{\pgfqpoint{1.798915in}{2.704136in}}%
\pgfpathlineto{\pgfqpoint{1.876913in}{2.675065in}}%
\pgfpathlineto{\pgfqpoint{1.963936in}{2.655172in}}%
\pgfpathlineto{\pgfqpoint{2.053408in}{2.644579in}}%
\pgfpathlineto{\pgfqpoint{2.147822in}{2.639752in}}%
\pgfpathlineto{\pgfqpoint{2.242666in}{2.637335in}}%
\pgfpathlineto{\pgfqpoint{2.337386in}{2.633896in}}%
\pgfpathlineto{\pgfqpoint{2.431311in}{2.626827in}}%
\pgfpathlineto{\pgfqpoint{2.523258in}{2.614458in}}%
\pgfpathlineto{\pgfqpoint{2.611715in}{2.596078in}}%
\pgfpathlineto{\pgfqpoint{2.695284in}{2.571828in}}%
\pgfusepath{stroke}%
\end{pgfscope}%
\begin{pgfscope}%
\pgfpathrectangle{\pgfqpoint{0.647939in}{0.492442in}}{\pgfqpoint{4.273799in}{2.331163in}}%
\pgfusepath{clip}%
\pgfsetbuttcap%
\pgfsetroundjoin%
\pgfsetlinewidth{0.301125pt}%
\definecolor{currentstroke}{rgb}{0.500000,0.500000,0.500000}%
\pgfsetstrokecolor{currentstroke}%
\pgfsetstrokeopacity{0.300000}%
\pgfsetdash{}{0pt}%
\pgfpathmoveto{\pgfqpoint{1.522125in}{2.823605in}}%
\pgfpathlineto{\pgfqpoint{1.522125in}{2.823605in}}%
\pgfpathlineto{\pgfqpoint{1.564430in}{2.777248in}}%
\pgfpathlineto{\pgfqpoint{1.611029in}{2.732150in}}%
\pgfpathlineto{\pgfqpoint{1.663434in}{2.689030in}}%
\pgfpathlineto{\pgfqpoint{1.723895in}{2.649249in}}%
\pgfpathlineto{\pgfqpoint{1.795154in}{2.615412in}}%
\pgfpathlineto{\pgfqpoint{1.878399in}{2.591574in}}%
\pgfpathlineto{\pgfqpoint{1.957990in}{2.581210in}}%
\pgfpathlineto{\pgfqpoint{2.038661in}{2.579231in}}%
\pgfpathlineto{\pgfqpoint{2.133226in}{2.582894in}}%
\pgfusepath{stroke}%
\end{pgfscope}%
\begin{pgfscope}%
\pgfpathrectangle{\pgfqpoint{0.647939in}{0.492442in}}{\pgfqpoint{4.273799in}{2.331163in}}%
\pgfusepath{clip}%
\pgfsetbuttcap%
\pgfsetroundjoin%
\pgfsetlinewidth{0.301125pt}%
\definecolor{currentstroke}{rgb}{0.500000,0.500000,0.500000}%
\pgfsetstrokecolor{currentstroke}%
\pgfsetstrokeopacity{0.300000}%
\pgfsetdash{}{0pt}%
\pgfpathmoveto{\pgfqpoint{1.424993in}{2.823605in}}%
\pgfpathlineto{\pgfqpoint{1.424993in}{2.823605in}}%
\pgfpathlineto{\pgfqpoint{1.458895in}{2.775221in}}%
\pgfpathlineto{\pgfqpoint{1.495030in}{2.727329in}}%
\pgfpathlineto{\pgfqpoint{1.534085in}{2.680134in}}%
\pgfpathlineto{\pgfqpoint{1.577102in}{2.633993in}}%
\pgfpathlineto{\pgfqpoint{1.625755in}{2.589587in}}%
\pgfpathlineto{\pgfqpoint{1.682876in}{2.548380in}}%
\pgfpathlineto{\pgfqpoint{1.752692in}{2.513854in}}%
\pgfpathlineto{\pgfqpoint{1.752692in}{2.513854in}}%
\pgfpathlineto{\pgfqpoint{1.813985in}{2.496656in}}%
\pgfpathlineto{\pgfqpoint{1.882105in}{2.489509in}}%
\pgfpathlineto{\pgfqpoint{1.948012in}{2.491478in}}%
\pgfpathlineto{\pgfqpoint{2.021171in}{2.500297in}}%
\pgfpathlineto{\pgfqpoint{2.111673in}{2.515777in}}%
\pgfpathlineto{\pgfqpoint{2.201935in}{2.531780in}}%
\pgfpathlineto{\pgfqpoint{2.293552in}{2.545089in}}%
\pgfpathlineto{\pgfqpoint{2.387077in}{2.553146in}}%
\pgfpathlineto{\pgfqpoint{2.481577in}{2.553896in}}%
\pgfpathlineto{\pgfqpoint{2.574900in}{2.546109in}}%
\pgfusepath{stroke}%
\end{pgfscope}%
\begin{pgfscope}%
\pgfpathrectangle{\pgfqpoint{0.647939in}{0.492442in}}{\pgfqpoint{4.273799in}{2.331163in}}%
\pgfusepath{clip}%
\pgfsetbuttcap%
\pgfsetroundjoin%
\pgfsetlinewidth{0.301125pt}%
\definecolor{currentstroke}{rgb}{0.500000,0.500000,0.500000}%
\pgfsetstrokecolor{currentstroke}%
\pgfsetstrokeopacity{0.300000}%
\pgfsetdash{}{0pt}%
\pgfpathmoveto{\pgfqpoint{1.327862in}{2.823605in}}%
\pgfpathlineto{\pgfqpoint{1.327862in}{2.823605in}}%
\pgfpathlineto{\pgfqpoint{1.355113in}{2.773985in}}%
\pgfpathlineto{\pgfqpoint{1.383356in}{2.724529in}}%
\pgfpathlineto{\pgfqpoint{1.412793in}{2.675282in}}%
\pgfpathlineto{\pgfqpoint{1.443703in}{2.626310in}}%
\pgfpathlineto{\pgfqpoint{1.476538in}{2.577715in}}%
\pgfpathlineto{\pgfqpoint{1.511993in}{2.529670in}}%
\pgfpathlineto{\pgfqpoint{1.551190in}{2.482509in}}%
\pgfpathlineto{\pgfqpoint{1.596209in}{2.436963in}}%
\pgfpathlineto{\pgfqpoint{1.651382in}{2.395044in}}%
\pgfpathlineto{\pgfqpoint{1.651382in}{2.395044in}}%
\pgfpathlineto{\pgfqpoint{1.702110in}{2.370499in}}%
\pgfpathlineto{\pgfqpoint{1.702110in}{2.370499in}}%
\pgfpathlineto{\pgfqpoint{1.750066in}{2.359103in}}%
\pgfpathlineto{\pgfqpoint{1.803117in}{2.357971in}}%
\pgfpathlineto{\pgfqpoint{1.851918in}{2.365019in}}%
\pgfpathlineto{\pgfqpoint{1.906716in}{2.379198in}}%
\pgfpathlineto{\pgfqpoint{1.977803in}{2.403042in}}%
\pgfpathlineto{\pgfqpoint{2.056662in}{2.431791in}}%
\pgfpathlineto{\pgfqpoint{2.136508in}{2.459708in}}%
\pgfusepath{stroke}%
\end{pgfscope}%
\begin{pgfscope}%
\pgfpathrectangle{\pgfqpoint{0.647939in}{0.492442in}}{\pgfqpoint{4.273799in}{2.331163in}}%
\pgfusepath{clip}%
\pgfsetbuttcap%
\pgfsetroundjoin%
\pgfsetlinewidth{0.301125pt}%
\definecolor{currentstroke}{rgb}{0.500000,0.500000,0.500000}%
\pgfsetstrokecolor{currentstroke}%
\pgfsetstrokeopacity{0.300000}%
\pgfsetdash{}{0pt}%
\pgfpathmoveto{\pgfqpoint{1.230730in}{2.823605in}}%
\pgfpathlineto{\pgfqpoint{1.230730in}{2.823605in}}%
\pgfpathlineto{\pgfqpoint{1.252853in}{2.773228in}}%
\pgfpathlineto{\pgfqpoint{1.275260in}{2.722889in}}%
\pgfpathlineto{\pgfqpoint{1.297977in}{2.672592in}}%
\pgfpathlineto{\pgfqpoint{1.321038in}{2.622341in}}%
\pgfpathlineto{\pgfqpoint{1.344485in}{2.572145in}}%
\pgfpathlineto{\pgfqpoint{1.368378in}{2.522012in}}%
\pgfpathlineto{\pgfqpoint{1.392791in}{2.471955in}}%
\pgfpathlineto{\pgfqpoint{1.417849in}{2.421994in}}%
\pgfpathlineto{\pgfqpoint{1.443717in}{2.372159in}}%
\pgfpathlineto{\pgfqpoint{1.470696in}{2.322500in}}%
\pgfpathlineto{\pgfqpoint{1.499291in}{2.273120in}}%
\pgfpathlineto{\pgfqpoint{1.530538in}{2.224257in}}%
\pgfpathlineto{\pgfqpoint{1.567317in}{2.176671in}}%
\pgfpathlineto{\pgfqpoint{1.567317in}{2.176671in}}%
\pgfpathlineto{\pgfqpoint{1.604690in}{2.144307in}}%
\pgfpathlineto{\pgfqpoint{1.604690in}{2.144307in}}%
\pgfpathlineto{\pgfqpoint{1.630909in}{2.133669in}}%
\pgfpathlineto{\pgfqpoint{1.630909in}{2.133669in}}%
\pgfpathlineto{\pgfqpoint{1.657351in}{2.133140in}}%
\pgfusepath{stroke}%
\end{pgfscope}%
\begin{pgfscope}%
\pgfpathrectangle{\pgfqpoint{0.647939in}{0.492442in}}{\pgfqpoint{4.273799in}{2.331163in}}%
\pgfusepath{clip}%
\pgfsetbuttcap%
\pgfsetroundjoin%
\pgfsetlinewidth{0.301125pt}%
\definecolor{currentstroke}{rgb}{0.500000,0.500000,0.500000}%
\pgfsetstrokecolor{currentstroke}%
\pgfsetstrokeopacity{0.300000}%
\pgfsetdash{}{0pt}%
\pgfpathmoveto{\pgfqpoint{1.133598in}{2.823605in}}%
\pgfpathlineto{\pgfqpoint{1.133598in}{2.823605in}}%
\pgfpathlineto{\pgfqpoint{1.151747in}{2.772758in}}%
\pgfpathlineto{\pgfqpoint{1.169819in}{2.721902in}}%
\pgfpathlineto{\pgfqpoint{1.187782in}{2.671035in}}%
\pgfpathlineto{\pgfqpoint{1.205598in}{2.620152in}}%
\pgfpathlineto{\pgfqpoint{1.223219in}{2.569250in}}%
\pgfpathlineto{\pgfqpoint{1.240584in}{2.518321in}}%
\pgfpathlineto{\pgfqpoint{1.257623in}{2.467360in}}%
\pgfpathlineto{\pgfqpoint{1.274238in}{2.416358in}}%
\pgfpathlineto{\pgfqpoint{1.290310in}{2.365304in}}%
\pgfpathlineto{\pgfqpoint{1.305681in}{2.314186in}}%
\pgfpathlineto{\pgfqpoint{1.320137in}{2.262990in}}%
\pgfpathlineto{\pgfqpoint{1.333411in}{2.211698in}}%
\pgfpathlineto{\pgfqpoint{1.345137in}{2.160295in}}%
\pgfpathlineto{\pgfqpoint{1.354823in}{2.108767in}}%
\pgfpathlineto{\pgfqpoint{1.361839in}{2.057115in}}%
\pgfpathlineto{\pgfqpoint{1.365416in}{2.005361in}}%
\pgfpathlineto{\pgfqpoint{1.364702in}{1.953576in}}%
\pgfpathlineto{\pgfqpoint{1.358961in}{1.901893in}}%
\pgfpathlineto{\pgfqpoint{1.347851in}{1.850480in}}%
\pgfpathlineto{\pgfqpoint{1.331629in}{1.799472in}}%
\pgfpathlineto{\pgfqpoint{1.311131in}{1.748930in}}%
\pgfusepath{stroke}%
\end{pgfscope}%
\begin{pgfscope}%
\pgfpathrectangle{\pgfqpoint{0.647939in}{0.492442in}}{\pgfqpoint{4.273799in}{2.331163in}}%
\pgfusepath{clip}%
\pgfsetbuttcap%
\pgfsetroundjoin%
\pgfsetlinewidth{0.301125pt}%
\definecolor{currentstroke}{rgb}{0.500000,0.500000,0.500000}%
\pgfsetstrokecolor{currentstroke}%
\pgfsetstrokeopacity{0.300000}%
\pgfsetdash{}{0pt}%
\pgfpathmoveto{\pgfqpoint{1.036466in}{2.823605in}}%
\pgfpathlineto{\pgfqpoint{1.036466in}{2.823605in}}%
\pgfpathlineto{\pgfqpoint{1.051527in}{2.772458in}}%
\pgfpathlineto{\pgfqpoint{1.066330in}{2.721288in}}%
\pgfpathlineto{\pgfqpoint{1.080838in}{2.670093in}}%
\pgfpathlineto{\pgfqpoint{1.094998in}{2.618870in}}%
\pgfpathlineto{\pgfqpoint{1.108750in}{2.567613in}}%
\pgfpathlineto{\pgfqpoint{1.122035in}{2.516320in}}%
\pgfpathlineto{\pgfqpoint{1.134771in}{2.464985in}}%
\pgfpathlineto{\pgfqpoint{1.146863in}{2.413604in}}%
\pgfpathlineto{\pgfqpoint{1.158213in}{2.362173in}}%
\pgfpathlineto{\pgfqpoint{1.168699in}{2.310688in}}%
\pgfpathlineto{\pgfqpoint{1.178179in}{2.259144in}}%
\pgfpathlineto{\pgfqpoint{1.186490in}{2.207541in}}%
\pgfpathlineto{\pgfqpoint{1.193452in}{2.155879in}}%
\pgfpathlineto{\pgfqpoint{1.198868in}{2.104162in}}%
\pgfpathlineto{\pgfqpoint{1.202521in}{2.052400in}}%
\pgfpathlineto{\pgfqpoint{1.204190in}{2.000608in}}%
\pgfpathlineto{\pgfqpoint{1.203663in}{1.948809in}}%
\pgfpathlineto{\pgfqpoint{1.200754in}{1.897035in}}%
\pgfpathlineto{\pgfqpoint{1.195332in}{1.845322in}}%
\pgfpathlineto{\pgfqpoint{1.187339in}{1.793710in}}%
\pgfpathlineto{\pgfqpoint{1.176799in}{1.742234in}}%
\pgfpathlineto{\pgfqpoint{1.163821in}{1.690923in}}%
\pgfpathlineto{\pgfqpoint{1.148613in}{1.639795in}}%
\pgfpathlineto{\pgfqpoint{1.131425in}{1.588857in}}%
\pgfpathlineto{\pgfqpoint{1.112528in}{1.538098in}}%
\pgfpathlineto{\pgfqpoint{1.092211in}{1.487503in}}%
\pgfpathlineto{\pgfqpoint{1.070736in}{1.437050in}}%
\pgfpathlineto{\pgfqpoint{1.048346in}{1.386715in}}%
\pgfpathlineto{\pgfqpoint{1.025251in}{1.336475in}}%
\pgfpathlineto{\pgfqpoint{1.001626in}{1.286308in}}%
\pgfpathlineto{\pgfqpoint{0.977622in}{1.236195in}}%
\pgfpathlineto{\pgfqpoint{0.953353in}{1.186118in}}%
\pgfpathlineto{\pgfqpoint{0.928922in}{1.136065in}}%
\pgfusepath{stroke}%
\end{pgfscope}%
\begin{pgfscope}%
\pgfpathrectangle{\pgfqpoint{0.647939in}{0.492442in}}{\pgfqpoint{4.273799in}{2.331163in}}%
\pgfusepath{clip}%
\pgfsetbuttcap%
\pgfsetroundjoin%
\pgfsetlinewidth{0.301125pt}%
\definecolor{currentstroke}{rgb}{0.500000,0.500000,0.500000}%
\pgfsetstrokecolor{currentstroke}%
\pgfsetstrokeopacity{0.300000}%
\pgfsetdash{}{0pt}%
\pgfpathmoveto{\pgfqpoint{0.939334in}{2.823605in}}%
\pgfpathlineto{\pgfqpoint{0.939334in}{2.823605in}}%
\pgfpathlineto{\pgfqpoint{0.951964in}{2.772262in}}%
\pgfpathlineto{\pgfqpoint{0.964260in}{2.720895in}}%
\pgfpathlineto{\pgfqpoint{0.976182in}{2.669502in}}%
\pgfpathlineto{\pgfqpoint{0.987689in}{2.618081in}}%
\pgfpathlineto{\pgfqpoint{0.998732in}{2.566629in}}%
\pgfpathlineto{\pgfqpoint{1.009256in}{2.515145in}}%
\pgfpathlineto{\pgfqpoint{1.019199in}{2.463627in}}%
\pgfpathlineto{\pgfqpoint{1.028498in}{2.412073in}}%
\pgfpathlineto{\pgfqpoint{1.037082in}{2.360482in}}%
\pgfpathlineto{\pgfqpoint{1.044871in}{2.308854in}}%
\pgfpathlineto{\pgfqpoint{1.051775in}{2.257188in}}%
\pgfpathlineto{\pgfqpoint{1.057706in}{2.205487in}}%
\pgfpathlineto{\pgfqpoint{1.062567in}{2.153752in}}%
\pgfpathlineto{\pgfqpoint{1.066260in}{2.101989in}}%
\pgfpathlineto{\pgfqpoint{1.068684in}{2.050204in}}%
\pgfpathlineto{\pgfqpoint{1.069740in}{1.998405in}}%
\pgfpathlineto{\pgfqpoint{1.069337in}{1.946604in}}%
\pgfpathlineto{\pgfqpoint{1.067394in}{1.894813in}}%
\pgfpathlineto{\pgfqpoint{1.063849in}{1.843048in}}%
\pgfpathlineto{\pgfqpoint{1.058659in}{1.791324in}}%
\pgfpathlineto{\pgfqpoint{1.051807in}{1.739659in}}%
\pgfpathlineto{\pgfqpoint{1.043304in}{1.688066in}}%
\pgfpathlineto{\pgfqpoint{1.033190in}{1.636560in}}%
\pgfpathlineto{\pgfqpoint{1.021537in}{1.585151in}}%
\pgfpathlineto{\pgfqpoint{1.008441in}{1.533846in}}%
\pgfpathlineto{\pgfqpoint{0.994002in}{1.482648in}}%
\pgfpathlineto{\pgfqpoint{0.978350in}{1.431555in}}%
\pgfpathlineto{\pgfqpoint{0.961618in}{1.380567in}}%
\pgfpathlineto{\pgfqpoint{0.943928in}{1.329673in}}%
\pgfpathlineto{\pgfqpoint{0.925414in}{1.278868in}}%
\pgfusepath{stroke}%
\end{pgfscope}%
\begin{pgfscope}%
\pgfpathrectangle{\pgfqpoint{0.647939in}{0.492442in}}{\pgfqpoint{4.273799in}{2.331163in}}%
\pgfusepath{clip}%
\pgfsetbuttcap%
\pgfsetroundjoin%
\pgfsetlinewidth{0.301125pt}%
\definecolor{currentstroke}{rgb}{0.500000,0.500000,0.500000}%
\pgfsetstrokecolor{currentstroke}%
\pgfsetstrokeopacity{0.300000}%
\pgfsetdash{}{0pt}%
\pgfpathmoveto{\pgfqpoint{0.842203in}{2.823605in}}%
\pgfpathlineto{\pgfqpoint{0.842203in}{2.823605in}}%
\pgfpathlineto{\pgfqpoint{0.852895in}{2.772132in}}%
\pgfpathlineto{\pgfqpoint{0.863233in}{2.720636in}}%
\pgfpathlineto{\pgfqpoint{0.873180in}{2.669118in}}%
\pgfpathlineto{\pgfqpoint{0.882700in}{2.617575in}}%
\pgfpathlineto{\pgfqpoint{0.891757in}{2.566008in}}%
\pgfpathlineto{\pgfqpoint{0.900311in}{2.514416in}}%
\pgfpathlineto{\pgfqpoint{0.908319in}{2.462797in}}%
\pgfpathlineto{\pgfqpoint{0.915734in}{2.411152in}}%
\pgfpathlineto{\pgfqpoint{0.922504in}{2.359481in}}%
\pgfpathlineto{\pgfqpoint{0.928579in}{2.307784in}}%
\pgfpathlineto{\pgfqpoint{0.933907in}{2.256062in}}%
\pgfpathlineto{\pgfqpoint{0.938432in}{2.204318in}}%
\pgfpathlineto{\pgfqpoint{0.942098in}{2.152554in}}%
\pgfpathlineto{\pgfqpoint{0.944849in}{2.100773in}}%
\pgfpathlineto{\pgfqpoint{0.946628in}{2.048979in}}%
\pgfpathlineto{\pgfqpoint{0.947383in}{1.997178in}}%
\pgfpathlineto{\pgfqpoint{0.947065in}{1.945376in}}%
\pgfpathlineto{\pgfqpoint{0.945630in}{1.893579in}}%
\pgfpathlineto{\pgfqpoint{0.943044in}{1.841796in}}%
\pgfpathlineto{\pgfqpoint{0.939279in}{1.790034in}}%
\pgfpathlineto{\pgfqpoint{0.934321in}{1.738303in}}%
\pgfpathlineto{\pgfqpoint{0.928164in}{1.686609in}}%
\pgfpathlineto{\pgfqpoint{0.920815in}{1.634963in}}%
\pgfpathlineto{\pgfqpoint{0.912295in}{1.583370in}}%
\pgfpathlineto{\pgfqpoint{0.902636in}{1.531836in}}%
\pgfpathlineto{\pgfqpoint{0.891885in}{1.480367in}}%
\pgfpathlineto{\pgfqpoint{0.880097in}{1.428966in}}%
\pgfpathlineto{\pgfqpoint{0.867329in}{1.377635in}}%
\pgfpathlineto{\pgfqpoint{0.853649in}{1.326373in}}%
\pgfpathlineto{\pgfqpoint{0.839133in}{1.275181in}}%
\pgfpathlineto{\pgfqpoint{0.823850in}{1.224054in}}%
\pgfpathlineto{\pgfqpoint{0.807874in}{1.172991in}}%
\pgfpathlineto{\pgfqpoint{0.791281in}{1.121986in}}%
\pgfpathlineto{\pgfqpoint{0.774133in}{1.071036in}}%
\pgfpathlineto{\pgfqpoint{0.756501in}{1.020135in}}%
\pgfpathlineto{\pgfqpoint{0.738447in}{0.969279in}}%
\pgfpathlineto{\pgfqpoint{0.720026in}{0.918461in}}%
\pgfpathlineto{\pgfqpoint{0.701293in}{0.867677in}}%
\pgfpathlineto{\pgfqpoint{0.682294in}{0.816923in}}%
\pgfpathlineto{\pgfqpoint{0.663074in}{0.766194in}}%
\pgfpathlineto{\pgfqpoint{0.647939in}{0.726457in}}%
\pgfusepath{stroke}%
\end{pgfscope}%
\begin{pgfscope}%
\pgfpathrectangle{\pgfqpoint{0.647939in}{0.492442in}}{\pgfqpoint{4.273799in}{2.331163in}}%
\pgfusepath{clip}%
\pgfsetbuttcap%
\pgfsetroundjoin%
\pgfsetlinewidth{0.301125pt}%
\definecolor{currentstroke}{rgb}{0.500000,0.500000,0.500000}%
\pgfsetstrokecolor{currentstroke}%
\pgfsetstrokeopacity{0.300000}%
\pgfsetdash{}{0pt}%
\pgfpathmoveto{\pgfqpoint{0.745071in}{2.823605in}}%
\pgfpathlineto{\pgfqpoint{0.745071in}{2.823605in}}%
\pgfpathlineto{\pgfqpoint{0.754207in}{2.772042in}}%
\pgfpathlineto{\pgfqpoint{0.762989in}{2.720461in}}%
\pgfpathlineto{\pgfqpoint{0.771390in}{2.668861in}}%
\pgfpathlineto{\pgfqpoint{0.779385in}{2.617241in}}%
\pgfpathlineto{\pgfqpoint{0.786942in}{2.565602in}}%
\pgfpathlineto{\pgfqpoint{0.794033in}{2.513943in}}%
\pgfpathlineto{\pgfqpoint{0.800625in}{2.462265in}}%
\pgfpathlineto{\pgfqpoint{0.806687in}{2.410568in}}%
\pgfpathlineto{\pgfqpoint{0.812185in}{2.358851in}}%
\pgfpathlineto{\pgfqpoint{0.817085in}{2.307117in}}%
\pgfpathlineto{\pgfqpoint{0.821352in}{2.255366in}}%
\pgfpathlineto{\pgfqpoint{0.824950in}{2.203600in}}%
\pgfpathlineto{\pgfqpoint{0.827843in}{2.151821in}}%
\pgfpathlineto{\pgfqpoint{0.829997in}{2.100031in}}%
\pgfpathlineto{\pgfqpoint{0.831377in}{2.048233in}}%
\pgfpathlineto{\pgfqpoint{0.831954in}{1.996431in}}%
\pgfpathlineto{\pgfqpoint{0.831696in}{1.944628in}}%
\pgfpathlineto{\pgfqpoint{0.830580in}{1.892829in}}%
\pgfpathlineto{\pgfqpoint{0.828581in}{1.841038in}}%
\pgfpathlineto{\pgfqpoint{0.825682in}{1.789259in}}%
\pgfpathlineto{\pgfqpoint{0.821872in}{1.737497in}}%
\pgfpathlineto{\pgfqpoint{0.817143in}{1.685759in}}%
\pgfpathlineto{\pgfqpoint{0.811495in}{1.634048in}}%
\pgfpathlineto{\pgfqpoint{0.804937in}{1.582369in}}%
\pgfpathlineto{\pgfqpoint{0.797479in}{1.530726in}}%
\pgfpathlineto{\pgfqpoint{0.789139in}{1.479124in}}%
\pgfpathlineto{\pgfqpoint{0.779940in}{1.427565in}}%
\pgfpathlineto{\pgfqpoint{0.769913in}{1.376051in}}%
\pgfpathlineto{\pgfqpoint{0.759097in}{1.324586in}}%
\pgfpathlineto{\pgfqpoint{0.747528in}{1.273170in}}%
\pgfpathlineto{\pgfqpoint{0.735244in}{1.221802in}}%
\pgfpathlineto{\pgfqpoint{0.722292in}{1.170484in}}%
\pgfusepath{stroke}%
\end{pgfscope}%
\begin{pgfscope}%
\pgfpathrectangle{\pgfqpoint{0.647939in}{0.492442in}}{\pgfqpoint{4.273799in}{2.331163in}}%
\pgfusepath{clip}%
\pgfsetbuttcap%
\pgfsetroundjoin%
\pgfsetlinewidth{0.301125pt}%
\definecolor{currentstroke}{rgb}{0.500000,0.500000,0.500000}%
\pgfsetstrokecolor{currentstroke}%
\pgfsetstrokeopacity{0.300000}%
\pgfsetdash{}{0pt}%
\pgfpathmoveto{\pgfqpoint{0.647939in}{2.823605in}}%
\pgfpathlineto{\pgfqpoint{0.647939in}{2.823605in}}%
\pgfpathlineto{\pgfqpoint{0.655811in}{2.771980in}}%
\pgfpathlineto{\pgfqpoint{0.663343in}{2.720340in}}%
\pgfpathlineto{\pgfqpoint{0.670516in}{2.668685in}}%
\pgfpathlineto{\pgfqpoint{0.677309in}{2.617014in}}%
\pgfpathlineto{\pgfqpoint{0.683701in}{2.565328in}}%
\pgfpathlineto{\pgfqpoint{0.689671in}{2.513627in}}%
\pgfpathlineto{\pgfqpoint{0.695195in}{2.461911in}}%
\pgfpathlineto{\pgfqpoint{0.700250in}{2.410181in}}%
\pgfpathlineto{\pgfqpoint{0.704811in}{2.358438in}}%
\pgfpathlineto{\pgfqpoint{0.708856in}{2.306681in}}%
\pgfpathlineto{\pgfqpoint{0.712360in}{2.254913in}}%
\pgfpathlineto{\pgfqpoint{0.715300in}{2.203135in}}%
\pgfpathlineto{\pgfqpoint{0.717652in}{2.151347in}}%
\pgfpathlineto{\pgfqpoint{0.719395in}{2.099553in}}%
\pgfpathlineto{\pgfqpoint{0.720506in}{2.047753in}}%
\pgfpathlineto{\pgfqpoint{0.720964in}{1.995950in}}%
\pgfpathlineto{\pgfqpoint{0.720752in}{1.944147in}}%
\pgfpathlineto{\pgfqpoint{0.719852in}{1.892346in}}%
\pgfpathlineto{\pgfqpoint{0.718249in}{1.840551in}}%
\pgfpathlineto{\pgfqpoint{0.715932in}{1.788763in}}%
\pgfpathlineto{\pgfqpoint{0.712892in}{1.736986in}}%
\pgfpathlineto{\pgfqpoint{0.709123in}{1.685224in}}%
\pgfpathlineto{\pgfqpoint{0.704621in}{1.633479in}}%
\pgfpathlineto{\pgfqpoint{0.699388in}{1.581755in}}%
\pgfpathlineto{\pgfqpoint{0.693427in}{1.530054in}}%
\pgfpathlineto{\pgfqpoint{0.686748in}{1.478379in}}%
\pgfpathlineto{\pgfqpoint{0.679363in}{1.426733in}}%
\pgfpathlineto{\pgfqpoint{0.671288in}{1.375117in}}%
\pgfpathlineto{\pgfqpoint{0.662540in}{1.323535in}}%
\pgfpathlineto{\pgfqpoint{0.653140in}{1.271986in}}%
\pgfpathlineto{\pgfqpoint{0.647939in}{1.244410in}}%
\pgfusepath{stroke}%
\end{pgfscope}%
\begin{pgfscope}%
\pgfpathrectangle{\pgfqpoint{0.647939in}{0.492442in}}{\pgfqpoint{4.273799in}{2.331163in}}%
\pgfusepath{clip}%
\pgfsetbuttcap%
\pgfsetroundjoin%
\pgfsetlinewidth{0.301125pt}%
\definecolor{currentstroke}{rgb}{0.500000,0.500000,0.500000}%
\pgfsetstrokecolor{currentstroke}%
\pgfsetstrokeopacity{0.300000}%
\pgfsetdash{}{0pt}%
\pgfpathmoveto{\pgfqpoint{0.647939in}{2.399758in}}%
\pgfpathlineto{\pgfqpoint{0.647939in}{2.399758in}}%
\pgfpathlineto{\pgfqpoint{0.652022in}{2.348002in}}%
\pgfpathlineto{\pgfqpoint{0.655621in}{2.296236in}}%
\pgfpathlineto{\pgfqpoint{0.658716in}{2.244460in}}%
\pgfpathlineto{\pgfqpoint{0.661287in}{2.192676in}}%
\pgfpathlineto{\pgfqpoint{0.663315in}{2.140884in}}%
\pgfpathlineto{\pgfqpoint{0.664782in}{2.089087in}}%
\pgfpathlineto{\pgfqpoint{0.665671in}{2.037286in}}%
\pgfpathlineto{\pgfqpoint{0.665965in}{1.985483in}}%
\pgfpathlineto{\pgfqpoint{0.665649in}{1.933680in}}%
\pgfpathlineto{\pgfqpoint{0.664709in}{1.881879in}}%
\pgfpathlineto{\pgfqpoint{0.663135in}{1.830083in}}%
\pgfpathlineto{\pgfqpoint{0.660915in}{1.778294in}}%
\pgfpathlineto{\pgfqpoint{0.658044in}{1.726514in}}%
\pgfpathlineto{\pgfqpoint{0.654515in}{1.674747in}}%
\pgfpathlineto{\pgfqpoint{0.650326in}{1.622994in}}%
\pgfpathlineto{\pgfqpoint{0.647939in}{1.595645in}}%
\pgfusepath{stroke}%
\end{pgfscope}%
\begin{pgfscope}%
\pgfpathrectangle{\pgfqpoint{0.647939in}{0.492442in}}{\pgfqpoint{4.273799in}{2.331163in}}%
\pgfusepath{clip}%
\pgfsetbuttcap%
\pgfsetroundjoin%
\pgfsetlinewidth{0.301125pt}%
\definecolor{currentstroke}{rgb}{0.500000,0.500000,0.500000}%
\pgfsetstrokecolor{currentstroke}%
\pgfsetstrokeopacity{0.300000}%
\pgfsetdash{}{0pt}%
\pgfpathmoveto{\pgfqpoint{4.630343in}{1.658024in}}%
\pgfpathlineto{\pgfqpoint{4.617795in}{1.709371in}}%
\pgfpathlineto{\pgfqpoint{4.606523in}{1.760806in}}%
\pgfpathlineto{\pgfqpoint{4.596820in}{1.812335in}}%
\pgfpathlineto{\pgfqpoint{4.589027in}{1.863960in}}%
\pgfpathlineto{\pgfqpoint{4.583549in}{1.915671in}}%
\pgfusepath{stroke}%
\end{pgfscope}%
\begin{pgfscope}%
\pgfpathrectangle{\pgfqpoint{0.647939in}{0.492442in}}{\pgfqpoint{4.273799in}{2.331163in}}%
\pgfusepath{clip}%
\pgfsetbuttcap%
\pgfsetroundjoin%
\pgfsetlinewidth{0.301125pt}%
\definecolor{currentstroke}{rgb}{0.500000,0.500000,0.500000}%
\pgfsetstrokecolor{currentstroke}%
\pgfsetstrokeopacity{0.300000}%
\pgfsetdash{}{0pt}%
\pgfpathmoveto{\pgfqpoint{4.560466in}{1.131572in}}%
\pgfpathlineto{\pgfqpoint{4.533211in}{1.181195in}}%
\pgfpathlineto{\pgfqpoint{4.505194in}{1.230690in}}%
\pgfpathlineto{\pgfqpoint{4.476301in}{1.280033in}}%
\pgfpathlineto{\pgfqpoint{4.446347in}{1.329189in}}%
\pgfpathlineto{\pgfqpoint{4.415122in}{1.378107in}}%
\pgfpathlineto{\pgfqpoint{4.382332in}{1.426714in}}%
\pgfpathlineto{\pgfqpoint{4.347497in}{1.474891in}}%
\pgfpathlineto{\pgfqpoint{4.309876in}{1.522435in}}%
\pgfpathlineto{\pgfqpoint{4.268232in}{1.568960in}}%
\pgfpathlineto{\pgfqpoint{4.220324in}{1.613599in}}%
\pgfpathlineto{\pgfqpoint{4.161712in}{1.653980in}}%
\pgfpathlineto{\pgfqpoint{4.161712in}{1.653980in}}%
\pgfpathlineto{\pgfqpoint{4.109636in}{1.676411in}}%
\pgfpathlineto{\pgfqpoint{4.109636in}{1.676411in}}%
\pgfpathlineto{\pgfqpoint{4.057660in}{1.687184in}}%
\pgfpathlineto{\pgfqpoint{4.001782in}{1.688626in}}%
\pgfpathlineto{\pgfqpoint{3.945398in}{1.683002in}}%
\pgfpathlineto{\pgfqpoint{3.875645in}{1.670859in}}%
\pgfusepath{stroke}%
\end{pgfscope}%
\begin{pgfscope}%
\pgfpathrectangle{\pgfqpoint{0.647939in}{0.492442in}}{\pgfqpoint{4.273799in}{2.331163in}}%
\pgfusepath{clip}%
\pgfsetbuttcap%
\pgfsetroundjoin%
\pgfsetlinewidth{0.301125pt}%
\definecolor{currentstroke}{rgb}{0.500000,0.500000,0.500000}%
\pgfsetstrokecolor{currentstroke}%
\pgfsetstrokeopacity{0.300000}%
\pgfsetdash{}{0pt}%
\pgfpathmoveto{\pgfqpoint{1.481085in}{0.715620in}}%
\pgfpathlineto{\pgfqpoint{1.424993in}{0.757347in}}%
\pgfpathlineto{\pgfqpoint{1.360460in}{0.795110in}}%
\pgfpathlineto{\pgfqpoint{1.360460in}{0.795110in}}%
\pgfpathlineto{\pgfqpoint{1.296907in}{0.820041in}}%
\pgfpathlineto{\pgfqpoint{1.296907in}{0.820041in}}%
\pgfpathlineto{\pgfqpoint{1.241441in}{0.830861in}}%
\pgfpathlineto{\pgfqpoint{1.182054in}{0.830288in}}%
\pgfpathlineto{\pgfqpoint{1.132895in}{0.820176in}}%
\pgfpathlineto{\pgfqpoint{1.086646in}{0.802645in}}%
\pgfpathlineto{\pgfqpoint{1.038755in}{0.776408in}}%
\pgfusepath{stroke}%
\end{pgfscope}%
\begin{pgfscope}%
\pgfpathrectangle{\pgfqpoint{0.647939in}{0.492442in}}{\pgfqpoint{4.273799in}{2.331163in}}%
\pgfusepath{clip}%
\pgfsetbuttcap%
\pgfsetroundjoin%
\pgfsetlinewidth{0.301125pt}%
\definecolor{currentstroke}{rgb}{0.500000,0.500000,0.500000}%
\pgfsetstrokecolor{currentstroke}%
\pgfsetstrokeopacity{0.300000}%
\pgfsetdash{}{0pt}%
\pgfpathmoveto{\pgfqpoint{4.436079in}{1.711005in}}%
\pgfpathlineto{\pgfqpoint{4.414921in}{1.761498in}}%
\pgfpathlineto{\pgfqpoint{4.394587in}{1.812090in}}%
\pgfpathlineto{\pgfqpoint{4.375746in}{1.862848in}}%
\pgfpathlineto{\pgfqpoint{4.359771in}{1.913889in}}%
\pgfpathlineto{\pgfqpoint{4.349876in}{1.965342in}}%
\pgfpathlineto{\pgfqpoint{4.353174in}{2.016803in}}%
\pgfpathlineto{\pgfqpoint{4.375029in}{2.066709in}}%
\pgfpathlineto{\pgfqpoint{4.403889in}{2.109940in}}%
\pgfpathlineto{\pgfqpoint{4.440933in}{2.157456in}}%
\pgfusepath{stroke}%
\end{pgfscope}%
\begin{pgfscope}%
\pgfpathrectangle{\pgfqpoint{0.647939in}{0.492442in}}{\pgfqpoint{4.273799in}{2.331163in}}%
\pgfusepath{clip}%
\pgfsetbuttcap%
\pgfsetroundjoin%
\pgfsetlinewidth{0.301125pt}%
\definecolor{currentstroke}{rgb}{0.500000,0.500000,0.500000}%
\pgfsetstrokecolor{currentstroke}%
\pgfsetstrokeopacity{0.300000}%
\pgfsetdash{}{0pt}%
\pgfpathmoveto{\pgfqpoint{1.424993in}{2.558700in}}%
\pgfpathlineto{\pgfqpoint{1.454886in}{2.509538in}}%
\pgfpathlineto{\pgfqpoint{1.486716in}{2.460750in}}%
\pgfpathlineto{\pgfqpoint{1.521367in}{2.412548in}}%
\pgfpathlineto{\pgfqpoint{1.560491in}{2.365394in}}%
\pgfpathlineto{\pgfqpoint{1.607706in}{2.320615in}}%
\pgfpathlineto{\pgfqpoint{1.607706in}{2.320615in}}%
\pgfpathlineto{\pgfqpoint{1.655008in}{2.290852in}}%
\pgfpathlineto{\pgfqpoint{1.655008in}{2.290852in}}%
\pgfpathlineto{\pgfqpoint{1.695648in}{2.277801in}}%
\pgfpathlineto{\pgfqpoint{1.695648in}{2.277801in}}%
\pgfpathlineto{\pgfqpoint{1.736497in}{2.275352in}}%
\pgfpathlineto{\pgfqpoint{1.776527in}{2.281250in}}%
\pgfpathlineto{\pgfqpoint{1.818399in}{2.293797in}}%
\pgfusepath{stroke}%
\end{pgfscope}%
\begin{pgfscope}%
\pgfpathrectangle{\pgfqpoint{0.647939in}{0.492442in}}{\pgfqpoint{4.273799in}{2.331163in}}%
\pgfusepath{clip}%
\pgfsetbuttcap%
\pgfsetroundjoin%
\pgfsetlinewidth{0.301125pt}%
\definecolor{currentstroke}{rgb}{0.500000,0.500000,0.500000}%
\pgfsetstrokecolor{currentstroke}%
\pgfsetstrokeopacity{0.300000}%
\pgfsetdash{}{0pt}%
\pgfpathmoveto{\pgfqpoint{1.133598in}{2.187834in}}%
\pgfpathlineto{\pgfqpoint{1.139040in}{2.136117in}}%
\pgfpathlineto{\pgfqpoint{1.143055in}{2.084361in}}%
\pgfpathlineto{\pgfqpoint{1.145492in}{2.032577in}}%
\pgfpathlineto{\pgfqpoint{1.146200in}{1.980778in}}%
\pgfpathlineto{\pgfqpoint{1.145040in}{1.928981in}}%
\pgfpathlineto{\pgfqpoint{1.141897in}{1.877209in}}%
\pgfpathlineto{\pgfqpoint{1.136687in}{1.825488in}}%
\pgfusepath{stroke}%
\end{pgfscope}%
\begin{pgfscope}%
\pgfpathrectangle{\pgfqpoint{0.647939in}{0.492442in}}{\pgfqpoint{4.273799in}{2.331163in}}%
\pgfusepath{clip}%
\pgfsetbuttcap%
\pgfsetroundjoin%
\pgfsetlinewidth{0.301125pt}%
\definecolor{currentstroke}{rgb}{0.500000,0.500000,0.500000}%
\pgfsetstrokecolor{currentstroke}%
\pgfsetstrokeopacity{0.300000}%
\pgfsetdash{}{0pt}%
\pgfpathmoveto{\pgfqpoint{2.493443in}{0.810328in}}%
\pgfpathlineto{\pgfqpoint{2.460606in}{0.858936in}}%
\pgfpathlineto{\pgfqpoint{2.428486in}{0.907687in}}%
\pgfpathlineto{\pgfqpoint{2.397078in}{0.956575in}}%
\pgfpathlineto{\pgfqpoint{2.366386in}{1.005598in}}%
\pgfpathlineto{\pgfqpoint{2.336415in}{1.054754in}}%
\pgfpathlineto{\pgfqpoint{2.307167in}{1.104040in}}%
\pgfpathlineto{\pgfqpoint{2.278652in}{1.153453in}}%
\pgfpathlineto{\pgfqpoint{2.250891in}{1.202994in}}%
\pgfpathlineto{\pgfqpoint{2.223900in}{1.252661in}}%
\pgfpathlineto{\pgfqpoint{2.197710in}{1.302455in}}%
\pgfpathlineto{\pgfqpoint{2.172358in}{1.352378in}}%
\pgfpathlineto{\pgfqpoint{2.147887in}{1.402432in}}%
\pgfpathlineto{\pgfqpoint{2.124361in}{1.452620in}}%
\pgfpathlineto{\pgfqpoint{2.101853in}{1.502947in}}%
\pgfpathlineto{\pgfqpoint{2.080459in}{1.553418in}}%
\pgfpathlineto{\pgfqpoint{2.060299in}{1.604039in}}%
\pgfpathlineto{\pgfqpoint{2.041522in}{1.654818in}}%
\pgfpathlineto{\pgfqpoint{2.024325in}{1.705763in}}%
\pgfpathlineto{\pgfqpoint{2.008948in}{1.756879in}}%
\pgfpathlineto{\pgfqpoint{1.995702in}{1.808172in}}%
\pgfpathlineto{\pgfqpoint{1.984988in}{1.859639in}}%
\pgfpathlineto{\pgfqpoint{1.977302in}{1.911264in}}%
\pgfpathlineto{\pgfqpoint{1.973265in}{1.963007in}}%
\pgfpathlineto{\pgfqpoint{1.973637in}{2.014791in}}%
\pgfpathlineto{\pgfqpoint{1.979292in}{2.066475in}}%
\pgfpathlineto{\pgfqpoint{1.991165in}{2.117836in}}%
\pgfpathlineto{\pgfqpoint{2.010180in}{2.168540in}}%
\pgfpathlineto{\pgfqpoint{2.037060in}{2.218155in}}%
\pgfpathlineto{\pgfqpoint{2.072308in}{2.266169in}}%
\pgfpathlineto{\pgfqpoint{2.116189in}{2.311999in}}%
\pgfusepath{stroke}%
\end{pgfscope}%
\begin{pgfscope}%
\pgfpathrectangle{\pgfqpoint{0.647939in}{0.492442in}}{\pgfqpoint{4.273799in}{2.331163in}}%
\pgfusepath{clip}%
\pgfsetbuttcap%
\pgfsetroundjoin%
\pgfsetlinewidth{0.301125pt}%
\definecolor{currentstroke}{rgb}{0.500000,0.500000,0.500000}%
\pgfsetstrokecolor{currentstroke}%
\pgfsetstrokeopacity{0.300000}%
\pgfsetdash{}{0pt}%
\pgfpathmoveto{\pgfqpoint{4.338948in}{1.181195in}}%
\pgfpathlineto{\pgfqpoint{4.297947in}{1.227906in}}%
\pgfpathlineto{\pgfqpoint{4.253828in}{1.273756in}}%
\pgfpathlineto{\pgfqpoint{4.205768in}{1.318401in}}%
\pgfpathlineto{\pgfqpoint{4.152647in}{1.361286in}}%
\pgfpathlineto{\pgfqpoint{4.093003in}{1.401509in}}%
\pgfpathlineto{\pgfqpoint{4.025361in}{1.437690in}}%
\pgfpathlineto{\pgfqpoint{3.948863in}{1.468071in}}%
\pgfpathlineto{\pgfqpoint{3.864593in}{1.491544in}}%
\pgfpathlineto{\pgfqpoint{3.775453in}{1.509126in}}%
\pgfusepath{stroke}%
\end{pgfscope}%
\begin{pgfscope}%
\pgfpathrectangle{\pgfqpoint{0.647939in}{0.492442in}}{\pgfqpoint{4.273799in}{2.331163in}}%
\pgfusepath{clip}%
\pgfsetbuttcap%
\pgfsetroundjoin%
\pgfsetlinewidth{0.301125pt}%
\definecolor{currentstroke}{rgb}{0.500000,0.500000,0.500000}%
\pgfsetstrokecolor{currentstroke}%
\pgfsetstrokeopacity{0.300000}%
\pgfsetdash{}{0pt}%
\pgfpathmoveto{\pgfqpoint{4.372256in}{1.556553in}}%
\pgfpathlineto{\pgfqpoint{4.338948in}{1.605043in}}%
\pgfpathlineto{\pgfqpoint{4.302604in}{1.652864in}}%
\pgfpathlineto{\pgfqpoint{4.261198in}{1.699400in}}%
\pgfpathlineto{\pgfqpoint{4.209828in}{1.742604in}}%
\pgfpathlineto{\pgfqpoint{4.209828in}{1.742604in}}%
\pgfpathlineto{\pgfqpoint{4.168744in}{1.763927in}}%
\pgfpathlineto{\pgfqpoint{4.168744in}{1.763927in}}%
\pgfpathlineto{\pgfqpoint{4.129065in}{1.773213in}}%
\pgfpathlineto{\pgfqpoint{4.085187in}{1.772595in}}%
\pgfpathlineto{\pgfqpoint{4.044808in}{1.764998in}}%
\pgfusepath{stroke}%
\end{pgfscope}%
\begin{pgfscope}%
\pgfpathrectangle{\pgfqpoint{0.647939in}{0.492442in}}{\pgfqpoint{4.273799in}{2.331163in}}%
\pgfusepath{clip}%
\pgfsetbuttcap%
\pgfsetroundjoin%
\pgfsetlinewidth{0.301125pt}%
\definecolor{currentstroke}{rgb}{0.500000,0.500000,0.500000}%
\pgfsetstrokecolor{currentstroke}%
\pgfsetstrokeopacity{0.300000}%
\pgfsetdash{}{0pt}%
\pgfpathmoveto{\pgfqpoint{1.529022in}{2.596624in}}%
\pgfpathlineto{\pgfqpoint{1.571032in}{2.550220in}}%
\pgfpathlineto{\pgfqpoint{1.619257in}{2.505719in}}%
\pgfpathlineto{\pgfqpoint{1.677506in}{2.465112in}}%
\pgfpathlineto{\pgfqpoint{1.677506in}{2.465112in}}%
\pgfpathlineto{\pgfqpoint{1.734357in}{2.439484in}}%
\pgfpathlineto{\pgfqpoint{1.734357in}{2.439484in}}%
\pgfpathlineto{\pgfqpoint{1.789071in}{2.426766in}}%
\pgfpathlineto{\pgfqpoint{1.848770in}{2.424088in}}%
\pgfpathlineto{\pgfqpoint{1.905931in}{2.429717in}}%
\pgfpathlineto{\pgfqpoint{1.970731in}{2.442439in}}%
\pgfpathlineto{\pgfqpoint{2.055795in}{2.464187in}}%
\pgfusepath{stroke}%
\end{pgfscope}%
\begin{pgfscope}%
\pgfpathrectangle{\pgfqpoint{0.647939in}{0.492442in}}{\pgfqpoint{4.273799in}{2.331163in}}%
\pgfusepath{clip}%
\pgfsetbuttcap%
\pgfsetroundjoin%
\pgfsetlinewidth{0.301125pt}%
\definecolor{currentstroke}{rgb}{0.500000,0.500000,0.500000}%
\pgfsetstrokecolor{currentstroke}%
\pgfsetstrokeopacity{0.300000}%
\pgfsetdash{}{0pt}%
\pgfpathmoveto{\pgfqpoint{1.487270in}{0.867389in}}%
\pgfpathlineto{\pgfqpoint{1.430155in}{0.908653in}}%
\pgfpathlineto{\pgfqpoint{1.379378in}{0.937277in}}%
\pgfpathlineto{\pgfqpoint{1.332356in}{0.956187in}}%
\pgfpathlineto{\pgfqpoint{1.284333in}{0.967280in}}%
\pgfpathlineto{\pgfqpoint{1.230730in}{0.969271in}}%
\pgfpathlineto{\pgfqpoint{1.230730in}{0.969271in}}%
\pgfpathlineto{\pgfqpoint{1.230730in}{0.969271in}}%
\pgfpathlineto{\pgfqpoint{1.178732in}{0.960436in}}%
\pgfpathlineto{\pgfqpoint{1.134548in}{0.944666in}}%
\pgfusepath{stroke}%
\end{pgfscope}%
\begin{pgfscope}%
\pgfpathrectangle{\pgfqpoint{0.647939in}{0.492442in}}{\pgfqpoint{4.273799in}{2.331163in}}%
\pgfusepath{clip}%
\pgfsetbuttcap%
\pgfsetroundjoin%
\pgfsetlinewidth{0.301125pt}%
\definecolor{currentstroke}{rgb}{0.500000,0.500000,0.500000}%
\pgfsetstrokecolor{currentstroke}%
\pgfsetstrokeopacity{0.300000}%
\pgfsetdash{}{0pt}%
\pgfpathmoveto{\pgfqpoint{1.706023in}{1.131475in}}%
\pgfpathlineto{\pgfqpoint{1.668421in}{1.179038in}}%
\pgfpathlineto{\pgfqpoint{1.628845in}{1.226117in}}%
\pgfpathlineto{\pgfqpoint{1.586301in}{1.272404in}}%
\pgfpathlineto{\pgfqpoint{1.538989in}{1.317245in}}%
\pgfpathlineto{\pgfqpoint{1.493614in}{1.352214in}}%
\pgfpathlineto{\pgfqpoint{1.453807in}{1.375289in}}%
\pgfpathlineto{\pgfqpoint{1.416427in}{1.389688in}}%
\pgfpathlineto{\pgfqpoint{1.373371in}{1.396845in}}%
\pgfpathlineto{\pgfqpoint{1.327862in}{1.393119in}}%
\pgfpathlineto{\pgfqpoint{1.327862in}{1.393119in}}%
\pgfpathlineto{\pgfqpoint{1.327862in}{1.393119in}}%
\pgfpathlineto{\pgfqpoint{1.280889in}{1.377344in}}%
\pgfpathlineto{\pgfqpoint{1.244067in}{1.357132in}}%
\pgfpathlineto{\pgfqpoint{1.200929in}{1.325422in}}%
\pgfusepath{stroke}%
\end{pgfscope}%
\begin{pgfscope}%
\pgfpathrectangle{\pgfqpoint{0.647939in}{0.492442in}}{\pgfqpoint{4.273799in}{2.331163in}}%
\pgfusepath{clip}%
\pgfsetbuttcap%
\pgfsetroundjoin%
\pgfsetlinewidth{0.301125pt}%
\definecolor{currentstroke}{rgb}{0.500000,0.500000,0.500000}%
\pgfsetstrokecolor{currentstroke}%
\pgfsetstrokeopacity{0.300000}%
\pgfsetdash{}{0pt}%
\pgfpathmoveto{\pgfqpoint{4.244731in}{1.093206in}}%
\pgfpathlineto{\pgfqpoint{4.196597in}{1.137841in}}%
\pgfpathlineto{\pgfqpoint{4.144684in}{1.181195in}}%
\pgfpathlineto{\pgfqpoint{4.088369in}{1.222875in}}%
\pgfpathlineto{\pgfqpoint{4.027077in}{1.262392in}}%
\pgfpathlineto{\pgfqpoint{3.960357in}{1.299188in}}%
\pgfpathlineto{\pgfqpoint{3.888276in}{1.332840in}}%
\pgfpathlineto{\pgfqpoint{3.811599in}{1.363346in}}%
\pgfpathlineto{\pgfqpoint{3.731769in}{1.391376in}}%
\pgfpathlineto{\pgfqpoint{3.650515in}{1.418177in}}%
\pgfpathlineto{\pgfqpoint{3.569596in}{1.445258in}}%
\pgfpathlineto{\pgfqpoint{3.490617in}{1.473958in}}%
\pgfusepath{stroke}%
\end{pgfscope}%
\begin{pgfscope}%
\pgfpathrectangle{\pgfqpoint{0.647939in}{0.492442in}}{\pgfqpoint{4.273799in}{2.331163in}}%
\pgfusepath{clip}%
\pgfsetbuttcap%
\pgfsetroundjoin%
\pgfsetlinewidth{0.301125pt}%
\definecolor{currentstroke}{rgb}{0.500000,0.500000,0.500000}%
\pgfsetstrokecolor{currentstroke}%
\pgfsetstrokeopacity{0.300000}%
\pgfsetdash{}{0pt}%
\pgfpathmoveto{\pgfqpoint{4.199982in}{1.404078in}}%
\pgfpathlineto{\pgfqpoint{4.144684in}{1.446100in}}%
\pgfpathlineto{\pgfqpoint{4.080911in}{1.484307in}}%
\pgfpathlineto{\pgfqpoint{4.006546in}{1.516093in}}%
\pgfpathlineto{\pgfqpoint{3.921745in}{1.538591in}}%
\pgfpathlineto{\pgfqpoint{3.836744in}{1.551166in}}%
\pgfpathlineto{\pgfqpoint{3.743079in}{1.559325in}}%
\pgfpathlineto{\pgfqpoint{3.649233in}{1.566853in}}%
\pgfpathlineto{\pgfqpoint{3.556650in}{1.577808in}}%
\pgfusepath{stroke}%
\end{pgfscope}%
\begin{pgfscope}%
\pgfpathrectangle{\pgfqpoint{0.647939in}{0.492442in}}{\pgfqpoint{4.273799in}{2.331163in}}%
\pgfusepath{clip}%
\pgfsetbuttcap%
\pgfsetroundjoin%
\pgfsetlinewidth{0.301125pt}%
\definecolor{currentstroke}{rgb}{0.500000,0.500000,0.500000}%
\pgfsetstrokecolor{currentstroke}%
\pgfsetstrokeopacity{0.300000}%
\pgfsetdash{}{0pt}%
\pgfpathmoveto{\pgfqpoint{2.652711in}{0.934086in}}%
\pgfpathlineto{\pgfqpoint{2.618862in}{0.982486in}}%
\pgfpathlineto{\pgfqpoint{2.585980in}{1.031085in}}%
\pgfpathlineto{\pgfqpoint{2.554069in}{1.079876in}}%
\pgfpathlineto{\pgfqpoint{2.523133in}{1.128853in}}%
\pgfpathlineto{\pgfqpoint{2.493184in}{1.178013in}}%
\pgfpathlineto{\pgfqpoint{2.464244in}{1.227351in}}%
\pgfpathlineto{\pgfqpoint{2.436333in}{1.276867in}}%
\pgfpathlineto{\pgfqpoint{2.409482in}{1.326556in}}%
\pgfpathlineto{\pgfqpoint{2.383735in}{1.376418in}}%
\pgfpathlineto{\pgfqpoint{2.359141in}{1.426453in}}%
\pgfpathlineto{\pgfqpoint{2.335766in}{1.476662in}}%
\pgfpathlineto{\pgfqpoint{2.313690in}{1.527045in}}%
\pgfpathlineto{\pgfqpoint{2.293010in}{1.577604in}}%
\pgfpathlineto{\pgfqpoint{2.273849in}{1.628340in}}%
\pgfpathlineto{\pgfqpoint{2.256352in}{1.679254in}}%
\pgfpathlineto{\pgfqpoint{2.240708in}{1.730347in}}%
\pgfpathlineto{\pgfqpoint{2.227138in}{1.781615in}}%
\pgfpathlineto{\pgfqpoint{2.215918in}{1.833051in}}%
\pgfpathlineto{\pgfqpoint{2.207392in}{1.884639in}}%
\pgfpathlineto{\pgfqpoint{2.201979in}{1.936349in}}%
\pgfpathlineto{\pgfqpoint{2.200190in}{1.988130in}}%
\pgfpathlineto{\pgfqpoint{2.202651in}{2.039898in}}%
\pgfpathlineto{\pgfqpoint{2.210127in}{2.091518in}}%
\pgfpathlineto{\pgfqpoint{2.223536in}{2.142770in}}%
\pgfpathlineto{\pgfqpoint{2.243951in}{2.193313in}}%
\pgfpathlineto{\pgfqpoint{2.272660in}{2.242616in}}%
\pgfpathlineto{\pgfqpoint{2.311136in}{2.289856in}}%
\pgfpathlineto{\pgfqpoint{2.361021in}{2.333740in}}%
\pgfpathlineto{\pgfqpoint{2.417748in}{2.369117in}}%
\pgfpathlineto{\pgfqpoint{2.493443in}{2.399758in}}%
\pgfpathlineto{\pgfqpoint{2.493443in}{2.399758in}}%
\pgfpathlineto{\pgfqpoint{2.561197in}{2.415308in}}%
\pgfpathlineto{\pgfqpoint{2.634586in}{2.421364in}}%
\pgfusepath{stroke}%
\end{pgfscope}%
\begin{pgfscope}%
\pgfpathrectangle{\pgfqpoint{0.647939in}{0.492442in}}{\pgfqpoint{4.273799in}{2.331163in}}%
\pgfusepath{clip}%
\pgfsetbuttcap%
\pgfsetroundjoin%
\pgfsetlinewidth{0.301125pt}%
\definecolor{currentstroke}{rgb}{0.500000,0.500000,0.500000}%
\pgfsetstrokecolor{currentstroke}%
\pgfsetstrokeopacity{0.300000}%
\pgfsetdash{}{0pt}%
\pgfpathmoveto{\pgfqpoint{1.424993in}{2.293796in}}%
\pgfpathlineto{\pgfqpoint{1.446963in}{2.243405in}}%
\pgfpathlineto{\pgfqpoint{1.468857in}{2.193006in}}%
\pgfpathlineto{\pgfqpoint{1.490485in}{2.142581in}}%
\pgfpathlineto{\pgfqpoint{1.511420in}{2.092071in}}%
\pgfpathlineto{\pgfqpoint{1.530313in}{2.041369in}}%
\pgfpathlineto{\pgfqpoint{1.540363in}{1.990165in}}%
\pgfpathlineto{\pgfqpoint{1.540363in}{1.990165in}}%
\pgfpathlineto{\pgfqpoint{1.535404in}{1.962626in}}%
\pgfpathlineto{\pgfqpoint{1.518458in}{1.932242in}}%
\pgfpathlineto{\pgfqpoint{1.494492in}{1.899072in}}%
\pgfpathlineto{\pgfqpoint{1.457596in}{1.852040in}}%
\pgfusepath{stroke}%
\end{pgfscope}%
\begin{pgfscope}%
\pgfpathrectangle{\pgfqpoint{0.647939in}{0.492442in}}{\pgfqpoint{4.273799in}{2.331163in}}%
\pgfusepath{clip}%
\pgfsetbuttcap%
\pgfsetroundjoin%
\pgfsetlinewidth{0.301125pt}%
\definecolor{currentstroke}{rgb}{0.500000,0.500000,0.500000}%
\pgfsetstrokecolor{currentstroke}%
\pgfsetstrokeopacity{0.300000}%
\pgfsetdash{}{0pt}%
\pgfpathmoveto{\pgfqpoint{1.720296in}{1.347371in}}%
\pgfpathlineto{\pgfqpoint{1.685910in}{1.395656in}}%
\pgfpathlineto{\pgfqpoint{1.649817in}{1.443564in}}%
\pgfpathlineto{\pgfqpoint{1.610964in}{1.490807in}}%
\pgfpathlineto{\pgfqpoint{1.567241in}{1.536705in}}%
\pgfpathlineto{\pgfqpoint{1.529471in}{1.568391in}}%
\pgfpathlineto{\pgfqpoint{1.497118in}{1.588365in}}%
\pgfpathlineto{\pgfqpoint{1.466445in}{1.600377in}}%
\pgfpathlineto{\pgfqpoint{1.424993in}{1.605043in}}%
\pgfpathlineto{\pgfqpoint{1.424993in}{1.605043in}}%
\pgfpathlineto{\pgfqpoint{1.424993in}{1.605043in}}%
\pgfpathlineto{\pgfqpoint{1.384742in}{1.596912in}}%
\pgfusepath{stroke}%
\end{pgfscope}%
\begin{pgfscope}%
\pgfpathrectangle{\pgfqpoint{0.647939in}{0.492442in}}{\pgfqpoint{4.273799in}{2.331163in}}%
\pgfusepath{clip}%
\pgfsetbuttcap%
\pgfsetroundjoin%
\pgfsetlinewidth{0.301125pt}%
\definecolor{currentstroke}{rgb}{0.500000,0.500000,0.500000}%
\pgfsetstrokecolor{currentstroke}%
\pgfsetstrokeopacity{0.300000}%
\pgfsetdash{}{0pt}%
\pgfpathmoveto{\pgfqpoint{1.537836in}{1.156762in}}%
\pgfpathlineto{\pgfqpoint{1.488788in}{1.196296in}}%
\pgfpathlineto{\pgfqpoint{1.424993in}{1.234176in}}%
\pgfpathlineto{\pgfqpoint{1.424993in}{1.234176in}}%
\pgfpathlineto{\pgfqpoint{1.374721in}{1.252099in}}%
\pgfpathlineto{\pgfqpoint{1.374721in}{1.252099in}}%
\pgfpathlineto{\pgfqpoint{1.328428in}{1.258097in}}%
\pgfpathlineto{\pgfqpoint{1.281439in}{1.253541in}}%
\pgfpathlineto{\pgfqpoint{1.241549in}{1.241425in}}%
\pgfpathlineto{\pgfqpoint{1.201194in}{1.221710in}}%
\pgfusepath{stroke}%
\end{pgfscope}%
\begin{pgfscope}%
\pgfpathrectangle{\pgfqpoint{0.647939in}{0.492442in}}{\pgfqpoint{4.273799in}{2.331163in}}%
\pgfusepath{clip}%
\pgfsetbuttcap%
\pgfsetroundjoin%
\pgfsetlinewidth{0.301125pt}%
\definecolor{currentstroke}{rgb}{0.500000,0.500000,0.500000}%
\pgfsetstrokecolor{currentstroke}%
\pgfsetstrokeopacity{0.300000}%
\pgfsetdash{}{0pt}%
\pgfpathmoveto{\pgfqpoint{4.104118in}{1.033642in}}%
\pgfpathlineto{\pgfqpoint{4.047552in}{1.075233in}}%
\pgfpathlineto{\pgfqpoint{3.987353in}{1.115274in}}%
\pgfpathlineto{\pgfqpoint{3.923492in}{1.153592in}}%
\pgfpathlineto{\pgfqpoint{3.856244in}{1.190150in}}%
\pgfpathlineto{\pgfqpoint{3.786183in}{1.225107in}}%
\pgfpathlineto{\pgfqpoint{3.714135in}{1.258851in}}%
\pgfpathlineto{\pgfqpoint{3.641164in}{1.292004in}}%
\pgfpathlineto{\pgfqpoint{3.568403in}{1.325288in}}%
\pgfusepath{stroke}%
\end{pgfscope}%
\begin{pgfscope}%
\pgfpathrectangle{\pgfqpoint{0.647939in}{0.492442in}}{\pgfqpoint{4.273799in}{2.331163in}}%
\pgfusepath{clip}%
\pgfsetbuttcap%
\pgfsetroundjoin%
\pgfsetlinewidth{0.301125pt}%
\definecolor{currentstroke}{rgb}{0.500000,0.500000,0.500000}%
\pgfsetstrokecolor{currentstroke}%
\pgfsetstrokeopacity{0.300000}%
\pgfsetdash{}{0pt}%
\pgfpathmoveto{\pgfqpoint{1.650278in}{1.528077in}}%
\pgfpathlineto{\pgfqpoint{1.612639in}{1.575605in}}%
\pgfpathlineto{\pgfqpoint{1.575618in}{1.615963in}}%
\pgfpathlineto{\pgfqpoint{1.522125in}{1.658024in}}%
\pgfpathlineto{\pgfqpoint{1.522125in}{1.658024in}}%
\pgfpathlineto{\pgfqpoint{1.487897in}{1.673346in}}%
\pgfpathlineto{\pgfqpoint{1.487897in}{1.673346in}}%
\pgfpathlineto{\pgfqpoint{1.454750in}{1.677868in}}%
\pgfpathlineto{\pgfqpoint{1.421486in}{1.672346in}}%
\pgfpathlineto{\pgfqpoint{1.393199in}{1.660522in}}%
\pgfusepath{stroke}%
\end{pgfscope}%
\begin{pgfscope}%
\pgfpathrectangle{\pgfqpoint{0.647939in}{0.492442in}}{\pgfqpoint{4.273799in}{2.331163in}}%
\pgfusepath{clip}%
\pgfsetbuttcap%
\pgfsetroundjoin%
\pgfsetlinewidth{0.301125pt}%
\definecolor{currentstroke}{rgb}{0.500000,0.500000,0.500000}%
\pgfsetstrokecolor{currentstroke}%
\pgfsetstrokeopacity{0.300000}%
\pgfsetdash{}{0pt}%
\pgfpathmoveto{\pgfqpoint{4.010383in}{0.982094in}}%
\pgfpathlineto{\pgfqpoint{3.950420in}{1.022252in}}%
\pgfpathlineto{\pgfqpoint{3.887668in}{1.061122in}}%
\pgfpathlineto{\pgfqpoint{3.822470in}{1.098781in}}%
\pgfpathlineto{\pgfqpoint{3.755361in}{1.135431in}}%
\pgfpathlineto{\pgfqpoint{3.687042in}{1.171413in}}%
\pgfpathlineto{\pgfqpoint{3.618319in}{1.207167in}}%
\pgfusepath{stroke}%
\end{pgfscope}%
\begin{pgfscope}%
\pgfpathrectangle{\pgfqpoint{0.647939in}{0.492442in}}{\pgfqpoint{4.273799in}{2.331163in}}%
\pgfusepath{clip}%
\pgfsetbuttcap%
\pgfsetroundjoin%
\pgfsetlinewidth{0.301125pt}%
\definecolor{currentstroke}{rgb}{0.500000,0.500000,0.500000}%
\pgfsetstrokecolor{currentstroke}%
\pgfsetstrokeopacity{0.300000}%
\pgfsetdash{}{0pt}%
\pgfpathmoveto{\pgfqpoint{3.849970in}{2.379237in}}%
\pgfpathlineto{\pgfqpoint{3.874524in}{2.329196in}}%
\pgfpathlineto{\pgfqpoint{3.897599in}{2.278947in}}%
\pgfpathlineto{\pgfqpoint{3.918920in}{2.228469in}}%
\pgfpathlineto{\pgfqpoint{3.938094in}{2.177738in}}%
\pgfpathlineto{\pgfqpoint{3.954559in}{2.126728in}}%
\pgfpathlineto{\pgfqpoint{3.967469in}{2.075420in}}%
\pgfpathlineto{\pgfqpoint{3.975579in}{2.023830in}}%
\pgfpathlineto{\pgfqpoint{3.977103in}{1.972080in}}%
\pgfpathlineto{\pgfqpoint{3.969618in}{1.920527in}}%
\pgfpathlineto{\pgfqpoint{3.950420in}{1.869948in}}%
\pgfusepath{stroke}%
\end{pgfscope}%
\begin{pgfscope}%
\pgfpathrectangle{\pgfqpoint{0.647939in}{0.492442in}}{\pgfqpoint{4.273799in}{2.331163in}}%
\pgfusepath{clip}%
\pgfsetbuttcap%
\pgfsetroundjoin%
\pgfsetlinewidth{0.301125pt}%
\definecolor{currentstroke}{rgb}{0.500000,0.500000,0.500000}%
\pgfsetstrokecolor{currentstroke}%
\pgfsetstrokeopacity{0.300000}%
\pgfsetdash{}{0pt}%
\pgfpathmoveto{\pgfqpoint{2.657857in}{1.511827in}}%
\pgfpathlineto{\pgfqpoint{2.632493in}{1.561746in}}%
\pgfpathlineto{\pgfqpoint{2.609025in}{1.611939in}}%
\pgfpathlineto{\pgfqpoint{2.587583in}{1.662401in}}%
\pgfpathlineto{\pgfqpoint{2.568331in}{1.713125in}}%
\pgfpathlineto{\pgfqpoint{2.551478in}{1.764101in}}%
\pgfpathlineto{\pgfqpoint{2.537293in}{1.815317in}}%
\pgfpathlineto{\pgfqpoint{2.526112in}{1.866752in}}%
\pgfpathlineto{\pgfqpoint{2.518381in}{1.918372in}}%
\pgfpathlineto{\pgfqpoint{2.514684in}{1.970120in}}%
\pgfpathlineto{\pgfqpoint{2.515806in}{2.021899in}}%
\pgfpathlineto{\pgfqpoint{2.522834in}{2.073527in}}%
\pgfpathlineto{\pgfqpoint{2.537326in}{2.124662in}}%
\pgfpathlineto{\pgfqpoint{2.561627in}{2.174618in}}%
\pgfpathlineto{\pgfqpoint{2.593674in}{2.216275in}}%
\pgfpathlineto{\pgfqpoint{2.631191in}{2.248316in}}%
\pgfpathlineto{\pgfqpoint{2.674196in}{2.271966in}}%
\pgfpathlineto{\pgfqpoint{2.725362in}{2.287824in}}%
\pgfpathlineto{\pgfqpoint{2.784839in}{2.293796in}}%
\pgfpathlineto{\pgfqpoint{2.784839in}{2.293796in}}%
\pgfpathlineto{\pgfqpoint{2.784839in}{2.293796in}}%
\pgfpathlineto{\pgfqpoint{2.842742in}{2.289172in}}%
\pgfpathlineto{\pgfqpoint{2.895829in}{2.276842in}}%
\pgfusepath{stroke}%
\end{pgfscope}%
\begin{pgfscope}%
\pgfpathrectangle{\pgfqpoint{0.647939in}{0.492442in}}{\pgfqpoint{4.273799in}{2.331163in}}%
\pgfusepath{clip}%
\pgfsetbuttcap%
\pgfsetroundjoin%
\pgfsetlinewidth{0.301125pt}%
\definecolor{currentstroke}{rgb}{0.500000,0.500000,0.500000}%
\pgfsetstrokecolor{currentstroke}%
\pgfsetstrokeopacity{0.300000}%
\pgfsetdash{}{0pt}%
\pgfpathmoveto{\pgfqpoint{1.660321in}{1.825349in}}%
\pgfpathlineto{\pgfqpoint{1.636538in}{1.875473in}}%
\pgfpathlineto{\pgfqpoint{1.613985in}{1.925713in}}%
\pgfpathlineto{\pgfqpoint{1.600135in}{1.962300in}}%
\pgfpathlineto{\pgfqpoint{1.595834in}{1.984854in}}%
\pgfpathlineto{\pgfqpoint{1.601040in}{2.005102in}}%
\pgfpathlineto{\pgfqpoint{1.601040in}{2.005102in}}%
\pgfpathlineto{\pgfqpoint{1.619257in}{2.028891in}}%
\pgfpathlineto{\pgfqpoint{1.619257in}{2.028891in}}%
\pgfpathlineto{\pgfqpoint{1.660675in}{2.071535in}}%
\pgfusepath{stroke}%
\end{pgfscope}%
\begin{pgfscope}%
\pgfpathrectangle{\pgfqpoint{0.647939in}{0.492442in}}{\pgfqpoint{4.273799in}{2.331163in}}%
\pgfusepath{clip}%
\pgfsetbuttcap%
\pgfsetroundjoin%
\pgfsetlinewidth{0.301125pt}%
\definecolor{currentstroke}{rgb}{0.500000,0.500000,0.500000}%
\pgfsetstrokecolor{currentstroke}%
\pgfsetstrokeopacity{0.300000}%
\pgfsetdash{}{0pt}%
\pgfpathmoveto{\pgfqpoint{2.979102in}{1.128214in}}%
\pgfpathlineto{\pgfqpoint{2.938963in}{1.175161in}}%
\pgfpathlineto{\pgfqpoint{2.900451in}{1.222511in}}%
\pgfpathlineto{\pgfqpoint{2.863580in}{1.270247in}}%
\pgfpathlineto{\pgfqpoint{2.828362in}{1.318353in}}%
\pgfpathlineto{\pgfqpoint{2.794822in}{1.366815in}}%
\pgfpathlineto{\pgfqpoint{2.762991in}{1.415619in}}%
\pgfusepath{stroke}%
\end{pgfscope}%
\begin{pgfscope}%
\pgfpathrectangle{\pgfqpoint{0.647939in}{0.492442in}}{\pgfqpoint{4.273799in}{2.331163in}}%
\pgfusepath{clip}%
\pgfsetbuttcap%
\pgfsetroundjoin%
\pgfsetlinewidth{0.301125pt}%
\definecolor{currentstroke}{rgb}{0.500000,0.500000,0.500000}%
\pgfsetstrokecolor{currentstroke}%
\pgfsetstrokeopacity{0.300000}%
\pgfsetdash{}{0pt}%
\pgfpathmoveto{\pgfqpoint{2.582650in}{1.825940in}}%
\pgfpathlineto{\pgfqpoint{2.571590in}{1.877381in}}%
\pgfpathlineto{\pgfqpoint{2.564253in}{1.929018in}}%
\pgfpathlineto{\pgfqpoint{2.561316in}{1.980779in}}%
\pgfpathlineto{\pgfqpoint{2.563712in}{2.032536in}}%
\pgfpathlineto{\pgfqpoint{2.572789in}{2.084053in}}%
\pgfpathlineto{\pgfqpoint{2.590575in}{2.134853in}}%
\pgfusepath{stroke}%
\end{pgfscope}%
\begin{pgfscope}%
\pgfpathrectangle{\pgfqpoint{0.647939in}{0.492442in}}{\pgfqpoint{4.273799in}{2.331163in}}%
\pgfusepath{clip}%
\pgfsetbuttcap%
\pgfsetroundjoin%
\pgfsetlinewidth{0.301125pt}%
\definecolor{currentstroke}{rgb}{0.500000,0.500000,0.500000}%
\pgfsetstrokecolor{currentstroke}%
\pgfsetstrokeopacity{0.300000}%
\pgfsetdash{}{0pt}%
\pgfpathmoveto{\pgfqpoint{1.921246in}{1.823678in}}%
\pgfpathlineto{\pgfqpoint{1.910589in}{1.875146in}}%
\pgfpathlineto{\pgfqpoint{1.903308in}{1.926785in}}%
\pgfpathlineto{\pgfqpoint{1.900217in}{1.978545in}}%
\pgfpathlineto{\pgfqpoint{1.902298in}{2.030309in}}%
\pgfpathlineto{\pgfqpoint{1.910652in}{2.081872in}}%
\pgfusepath{stroke}%
\end{pgfscope}%
\begin{pgfscope}%
\pgfpathrectangle{\pgfqpoint{0.647939in}{0.492442in}}{\pgfqpoint{4.273799in}{2.331163in}}%
\pgfusepath{clip}%
\pgfsetbuttcap%
\pgfsetroundjoin%
\pgfsetlinewidth{0.301125pt}%
\definecolor{currentstroke}{rgb}{0.500000,0.500000,0.500000}%
\pgfsetstrokecolor{currentstroke}%
\pgfsetstrokeopacity{0.300000}%
\pgfsetdash{}{0pt}%
\pgfpathmoveto{\pgfqpoint{3.464761in}{1.234176in}}%
\pgfpathlineto{\pgfqpoint{3.401912in}{1.273000in}}%
\pgfpathlineto{\pgfqpoint{3.341346in}{1.312883in}}%
\pgfpathlineto{\pgfqpoint{3.283402in}{1.353911in}}%
\pgfpathlineto{\pgfqpoint{3.228340in}{1.396101in}}%
\pgfpathlineto{\pgfqpoint{3.176333in}{1.439425in}}%
\pgfusepath{stroke}%
\end{pgfscope}%
\begin{pgfscope}%
\pgfpathrectangle{\pgfqpoint{0.647939in}{0.492442in}}{\pgfqpoint{4.273799in}{2.331163in}}%
\pgfusepath{clip}%
\pgfsetroundcap%
\pgfsetroundjoin%
\pgfsetlinewidth{0.301125pt}%
\definecolor{currentstroke}{rgb}{0.500000,0.500000,0.500000}%
\pgfsetstrokecolor{currentstroke}%
\pgfsetstrokeopacity{0.300000}%
\pgfsetdash{}{0pt}%
\pgfpathmoveto{\pgfqpoint{1.426892in}{1.511477in}}%
\pgfusepath{stroke}%
\end{pgfscope}%
\begin{pgfscope}%
\pgfpathrectangle{\pgfqpoint{0.647939in}{0.492442in}}{\pgfqpoint{4.273799in}{2.331163in}}%
\pgfusepath{clip}%
\pgfsetroundcap%
\pgfsetroundjoin%
\definecolor{currentfill}{rgb}{0.500000,0.500000,0.500000}%
\pgfsetfillcolor{currentfill}%
\pgfsetfillopacity{0.300000}%
\pgfsetlinewidth{0.301125pt}%
\definecolor{currentstroke}{rgb}{0.500000,0.500000,0.500000}%
\pgfsetstrokecolor{currentstroke}%
\pgfsetstrokeopacity{0.300000}%
\pgfsetdash{}{0pt}%
\pgfpathmoveto{\pgfqpoint{0.000000in}{0.000000in}}%
\pgfpathlineto{\pgfqpoint{0.000000in}{0.000000in}}%
\pgfpathclose%
\pgfusepath{stroke,fill}%
\end{pgfscope}%
\begin{pgfscope}%
\pgfpathrectangle{\pgfqpoint{0.647939in}{0.492442in}}{\pgfqpoint{4.273799in}{2.331163in}}%
\pgfusepath{clip}%
\pgfsetroundcap%
\pgfsetroundjoin%
\pgfsetlinewidth{0.301125pt}%
\definecolor{currentstroke}{rgb}{0.500000,0.500000,0.500000}%
\pgfsetstrokecolor{currentstroke}%
\pgfsetstrokeopacity{0.300000}%
\pgfsetdash{}{0pt}%
\pgfpathmoveto{\pgfqpoint{1.240292in}{0.901718in}}%
\pgfusepath{stroke}%
\end{pgfscope}%
\begin{pgfscope}%
\pgfpathrectangle{\pgfqpoint{0.647939in}{0.492442in}}{\pgfqpoint{4.273799in}{2.331163in}}%
\pgfusepath{clip}%
\pgfsetroundcap%
\pgfsetroundjoin%
\definecolor{currentfill}{rgb}{0.500000,0.500000,0.500000}%
\pgfsetfillcolor{currentfill}%
\pgfsetfillopacity{0.300000}%
\pgfsetlinewidth{0.301125pt}%
\definecolor{currentstroke}{rgb}{0.500000,0.500000,0.500000}%
\pgfsetstrokecolor{currentstroke}%
\pgfsetstrokeopacity{0.300000}%
\pgfsetdash{}{0pt}%
\pgfpathmoveto{\pgfqpoint{0.000000in}{0.000000in}}%
\pgfpathlineto{\pgfqpoint{0.000000in}{0.000000in}}%
\pgfpathclose%
\pgfusepath{stroke,fill}%
\end{pgfscope}%
\begin{pgfscope}%
\pgfpathrectangle{\pgfqpoint{0.647939in}{0.492442in}}{\pgfqpoint{4.273799in}{2.331163in}}%
\pgfusepath{clip}%
\pgfsetroundcap%
\pgfsetroundjoin%
\pgfsetlinewidth{0.301125pt}%
\definecolor{currentstroke}{rgb}{0.500000,0.500000,0.500000}%
\pgfsetstrokecolor{currentstroke}%
\pgfsetstrokeopacity{0.300000}%
\pgfsetdash{}{0pt}%
\pgfpathmoveto{\pgfqpoint{1.206655in}{0.680028in}}%
\pgfusepath{stroke}%
\end{pgfscope}%
\begin{pgfscope}%
\pgfpathrectangle{\pgfqpoint{0.647939in}{0.492442in}}{\pgfqpoint{4.273799in}{2.331163in}}%
\pgfusepath{clip}%
\pgfsetroundcap%
\pgfsetroundjoin%
\definecolor{currentfill}{rgb}{0.500000,0.500000,0.500000}%
\pgfsetfillcolor{currentfill}%
\pgfsetfillopacity{0.300000}%
\pgfsetlinewidth{0.301125pt}%
\definecolor{currentstroke}{rgb}{0.500000,0.500000,0.500000}%
\pgfsetstrokecolor{currentstroke}%
\pgfsetstrokeopacity{0.300000}%
\pgfsetdash{}{0pt}%
\pgfpathmoveto{\pgfqpoint{0.000000in}{0.000000in}}%
\pgfpathlineto{\pgfqpoint{0.000000in}{0.000000in}}%
\pgfpathclose%
\pgfusepath{stroke,fill}%
\end{pgfscope}%
\begin{pgfscope}%
\pgfpathrectangle{\pgfqpoint{0.647939in}{0.492442in}}{\pgfqpoint{4.273799in}{2.331163in}}%
\pgfusepath{clip}%
\pgfsetroundcap%
\pgfsetroundjoin%
\pgfsetlinewidth{0.301125pt}%
\definecolor{currentstroke}{rgb}{0.500000,0.500000,0.500000}%
\pgfsetstrokecolor{currentstroke}%
\pgfsetstrokeopacity{0.300000}%
\pgfsetdash{}{0pt}%
\pgfpathmoveto{\pgfqpoint{1.160797in}{0.565110in}}%
\pgfusepath{stroke}%
\end{pgfscope}%
\begin{pgfscope}%
\pgfpathrectangle{\pgfqpoint{0.647939in}{0.492442in}}{\pgfqpoint{4.273799in}{2.331163in}}%
\pgfusepath{clip}%
\pgfsetroundcap%
\pgfsetroundjoin%
\definecolor{currentfill}{rgb}{0.500000,0.500000,0.500000}%
\pgfsetfillcolor{currentfill}%
\pgfsetfillopacity{0.300000}%
\pgfsetlinewidth{0.301125pt}%
\definecolor{currentstroke}{rgb}{0.500000,0.500000,0.500000}%
\pgfsetstrokecolor{currentstroke}%
\pgfsetstrokeopacity{0.300000}%
\pgfsetdash{}{0pt}%
\pgfpathmoveto{\pgfqpoint{0.000000in}{0.000000in}}%
\pgfpathlineto{\pgfqpoint{0.000000in}{0.000000in}}%
\pgfpathclose%
\pgfusepath{stroke,fill}%
\end{pgfscope}%
\begin{pgfscope}%
\pgfpathrectangle{\pgfqpoint{0.647939in}{0.492442in}}{\pgfqpoint{4.273799in}{2.331163in}}%
\pgfusepath{clip}%
\pgfsetroundcap%
\pgfsetroundjoin%
\pgfsetlinewidth{0.301125pt}%
\definecolor{currentstroke}{rgb}{0.500000,0.500000,0.500000}%
\pgfsetstrokecolor{currentstroke}%
\pgfsetstrokeopacity{0.300000}%
\pgfsetdash{}{0pt}%
\pgfpathmoveto{\pgfqpoint{1.332594in}{0.717897in}}%
\pgfusepath{stroke}%
\end{pgfscope}%
\begin{pgfscope}%
\pgfpathrectangle{\pgfqpoint{0.647939in}{0.492442in}}{\pgfqpoint{4.273799in}{2.331163in}}%
\pgfusepath{clip}%
\pgfsetroundcap%
\pgfsetroundjoin%
\definecolor{currentfill}{rgb}{0.500000,0.500000,0.500000}%
\pgfsetfillcolor{currentfill}%
\pgfsetfillopacity{0.300000}%
\pgfsetlinewidth{0.301125pt}%
\definecolor{currentstroke}{rgb}{0.500000,0.500000,0.500000}%
\pgfsetstrokecolor{currentstroke}%
\pgfsetstrokeopacity{0.300000}%
\pgfsetdash{}{0pt}%
\pgfpathmoveto{\pgfqpoint{0.000000in}{0.000000in}}%
\pgfpathlineto{\pgfqpoint{0.000000in}{0.000000in}}%
\pgfpathclose%
\pgfusepath{stroke,fill}%
\end{pgfscope}%
\begin{pgfscope}%
\pgfpathrectangle{\pgfqpoint{0.647939in}{0.492442in}}{\pgfqpoint{4.273799in}{2.331163in}}%
\pgfusepath{clip}%
\pgfsetroundcap%
\pgfsetroundjoin%
\pgfsetlinewidth{0.301125pt}%
\definecolor{currentstroke}{rgb}{0.500000,0.500000,0.500000}%
\pgfsetstrokecolor{currentstroke}%
\pgfsetstrokeopacity{0.300000}%
\pgfsetdash{}{0pt}%
\pgfpathmoveto{\pgfqpoint{1.431385in}{1.017509in}}%
\pgfusepath{stroke}%
\end{pgfscope}%
\begin{pgfscope}%
\pgfpathrectangle{\pgfqpoint{0.647939in}{0.492442in}}{\pgfqpoint{4.273799in}{2.331163in}}%
\pgfusepath{clip}%
\pgfsetroundcap%
\pgfsetroundjoin%
\definecolor{currentfill}{rgb}{0.500000,0.500000,0.500000}%
\pgfsetfillcolor{currentfill}%
\pgfsetfillopacity{0.300000}%
\pgfsetlinewidth{0.301125pt}%
\definecolor{currentstroke}{rgb}{0.500000,0.500000,0.500000}%
\pgfsetstrokecolor{currentstroke}%
\pgfsetstrokeopacity{0.300000}%
\pgfsetdash{}{0pt}%
\pgfpathmoveto{\pgfqpoint{0.000000in}{0.000000in}}%
\pgfpathlineto{\pgfqpoint{0.000000in}{0.000000in}}%
\pgfpathclose%
\pgfusepath{stroke,fill}%
\end{pgfscope}%
\begin{pgfscope}%
\pgfpathrectangle{\pgfqpoint{0.647939in}{0.492442in}}{\pgfqpoint{4.273799in}{2.331163in}}%
\pgfusepath{clip}%
\pgfsetroundcap%
\pgfsetroundjoin%
\pgfsetlinewidth{0.301125pt}%
\definecolor{currentstroke}{rgb}{0.500000,0.500000,0.500000}%
\pgfsetstrokecolor{currentstroke}%
\pgfsetstrokeopacity{0.300000}%
\pgfsetdash{}{0pt}%
\pgfpathmoveto{\pgfqpoint{1.582823in}{1.037107in}}%
\pgfusepath{stroke}%
\end{pgfscope}%
\begin{pgfscope}%
\pgfpathrectangle{\pgfqpoint{0.647939in}{0.492442in}}{\pgfqpoint{4.273799in}{2.331163in}}%
\pgfusepath{clip}%
\pgfsetroundcap%
\pgfsetroundjoin%
\definecolor{currentfill}{rgb}{0.500000,0.500000,0.500000}%
\pgfsetfillcolor{currentfill}%
\pgfsetfillopacity{0.300000}%
\pgfsetlinewidth{0.301125pt}%
\definecolor{currentstroke}{rgb}{0.500000,0.500000,0.500000}%
\pgfsetstrokecolor{currentstroke}%
\pgfsetstrokeopacity{0.300000}%
\pgfsetdash{}{0pt}%
\pgfpathmoveto{\pgfqpoint{0.000000in}{0.000000in}}%
\pgfpathlineto{\pgfqpoint{0.000000in}{0.000000in}}%
\pgfpathclose%
\pgfusepath{stroke,fill}%
\end{pgfscope}%
\begin{pgfscope}%
\pgfpathrectangle{\pgfqpoint{0.647939in}{0.492442in}}{\pgfqpoint{4.273799in}{2.331163in}}%
\pgfusepath{clip}%
\pgfsetroundcap%
\pgfsetroundjoin%
\pgfsetlinewidth{0.301125pt}%
\definecolor{currentstroke}{rgb}{0.500000,0.500000,0.500000}%
\pgfsetstrokecolor{currentstroke}%
\pgfsetstrokeopacity{0.300000}%
\pgfsetdash{}{0pt}%
\pgfpathmoveto{\pgfqpoint{1.671326in}{1.092577in}}%
\pgfusepath{stroke}%
\end{pgfscope}%
\begin{pgfscope}%
\pgfpathrectangle{\pgfqpoint{0.647939in}{0.492442in}}{\pgfqpoint{4.273799in}{2.331163in}}%
\pgfusepath{clip}%
\pgfsetroundcap%
\pgfsetroundjoin%
\definecolor{currentfill}{rgb}{0.500000,0.500000,0.500000}%
\pgfsetfillcolor{currentfill}%
\pgfsetfillopacity{0.300000}%
\pgfsetlinewidth{0.301125pt}%
\definecolor{currentstroke}{rgb}{0.500000,0.500000,0.500000}%
\pgfsetstrokecolor{currentstroke}%
\pgfsetstrokeopacity{0.300000}%
\pgfsetdash{}{0pt}%
\pgfpathmoveto{\pgfqpoint{0.000000in}{0.000000in}}%
\pgfpathlineto{\pgfqpoint{0.000000in}{0.000000in}}%
\pgfpathclose%
\pgfusepath{stroke,fill}%
\end{pgfscope}%
\begin{pgfscope}%
\pgfpathrectangle{\pgfqpoint{0.647939in}{0.492442in}}{\pgfqpoint{4.273799in}{2.331163in}}%
\pgfusepath{clip}%
\pgfsetroundcap%
\pgfsetroundjoin%
\pgfsetlinewidth{0.301125pt}%
\definecolor{currentstroke}{rgb}{0.500000,0.500000,0.500000}%
\pgfsetstrokecolor{currentstroke}%
\pgfsetstrokeopacity{0.300000}%
\pgfsetdash{}{0pt}%
\pgfpathmoveto{\pgfqpoint{1.823532in}{1.397480in}}%
\pgfusepath{stroke}%
\end{pgfscope}%
\begin{pgfscope}%
\pgfpathrectangle{\pgfqpoint{0.647939in}{0.492442in}}{\pgfqpoint{4.273799in}{2.331163in}}%
\pgfusepath{clip}%
\pgfsetroundcap%
\pgfsetroundjoin%
\definecolor{currentfill}{rgb}{0.500000,0.500000,0.500000}%
\pgfsetfillcolor{currentfill}%
\pgfsetfillopacity{0.300000}%
\pgfsetlinewidth{0.301125pt}%
\definecolor{currentstroke}{rgb}{0.500000,0.500000,0.500000}%
\pgfsetstrokecolor{currentstroke}%
\pgfsetstrokeopacity{0.300000}%
\pgfsetdash{}{0pt}%
\pgfpathmoveto{\pgfqpoint{0.000000in}{0.000000in}}%
\pgfpathlineto{\pgfqpoint{0.000000in}{0.000000in}}%
\pgfpathclose%
\pgfusepath{stroke,fill}%
\end{pgfscope}%
\begin{pgfscope}%
\pgfpathrectangle{\pgfqpoint{0.647939in}{0.492442in}}{\pgfqpoint{4.273799in}{2.331163in}}%
\pgfusepath{clip}%
\pgfsetroundcap%
\pgfsetroundjoin%
\pgfsetlinewidth{0.301125pt}%
\definecolor{currentstroke}{rgb}{0.500000,0.500000,0.500000}%
\pgfsetstrokecolor{currentstroke}%
\pgfsetstrokeopacity{0.300000}%
\pgfsetdash{}{0pt}%
\pgfpathmoveto{\pgfqpoint{1.974544in}{2.322415in}}%
\pgfusepath{stroke}%
\end{pgfscope}%
\begin{pgfscope}%
\pgfpathrectangle{\pgfqpoint{0.647939in}{0.492442in}}{\pgfqpoint{4.273799in}{2.331163in}}%
\pgfusepath{clip}%
\pgfsetroundcap%
\pgfsetroundjoin%
\definecolor{currentfill}{rgb}{0.500000,0.500000,0.500000}%
\pgfsetfillcolor{currentfill}%
\pgfsetfillopacity{0.300000}%
\pgfsetlinewidth{0.301125pt}%
\definecolor{currentstroke}{rgb}{0.500000,0.500000,0.500000}%
\pgfsetstrokecolor{currentstroke}%
\pgfsetstrokeopacity{0.300000}%
\pgfsetdash{}{0pt}%
\pgfpathmoveto{\pgfqpoint{0.000000in}{0.000000in}}%
\pgfpathlineto{\pgfqpoint{0.000000in}{0.000000in}}%
\pgfpathclose%
\pgfusepath{stroke,fill}%
\end{pgfscope}%
\begin{pgfscope}%
\pgfpathrectangle{\pgfqpoint{0.647939in}{0.492442in}}{\pgfqpoint{4.273799in}{2.331163in}}%
\pgfusepath{clip}%
\pgfsetroundcap%
\pgfsetroundjoin%
\pgfsetlinewidth{0.301125pt}%
\definecolor{currentstroke}{rgb}{0.500000,0.500000,0.500000}%
\pgfsetstrokecolor{currentstroke}%
\pgfsetstrokeopacity{0.300000}%
\pgfsetdash{}{0pt}%
\pgfpathmoveto{\pgfqpoint{2.075880in}{1.298968in}}%
\pgfusepath{stroke}%
\end{pgfscope}%
\begin{pgfscope}%
\pgfpathrectangle{\pgfqpoint{0.647939in}{0.492442in}}{\pgfqpoint{4.273799in}{2.331163in}}%
\pgfusepath{clip}%
\pgfsetroundcap%
\pgfsetroundjoin%
\definecolor{currentfill}{rgb}{0.500000,0.500000,0.500000}%
\pgfsetfillcolor{currentfill}%
\pgfsetfillopacity{0.300000}%
\pgfsetlinewidth{0.301125pt}%
\definecolor{currentstroke}{rgb}{0.500000,0.500000,0.500000}%
\pgfsetstrokecolor{currentstroke}%
\pgfsetstrokeopacity{0.300000}%
\pgfsetdash{}{0pt}%
\pgfpathmoveto{\pgfqpoint{0.000000in}{0.000000in}}%
\pgfpathlineto{\pgfqpoint{0.000000in}{0.000000in}}%
\pgfpathclose%
\pgfusepath{stroke,fill}%
\end{pgfscope}%
\begin{pgfscope}%
\pgfpathrectangle{\pgfqpoint{0.647939in}{0.492442in}}{\pgfqpoint{4.273799in}{2.331163in}}%
\pgfusepath{clip}%
\pgfsetroundcap%
\pgfsetroundjoin%
\pgfsetlinewidth{0.301125pt}%
\definecolor{currentstroke}{rgb}{0.500000,0.500000,0.500000}%
\pgfsetstrokecolor{currentstroke}%
\pgfsetstrokeopacity{0.300000}%
\pgfsetdash{}{0pt}%
\pgfpathmoveto{\pgfqpoint{2.597491in}{0.610235in}}%
\pgfusepath{stroke}%
\end{pgfscope}%
\begin{pgfscope}%
\pgfpathrectangle{\pgfqpoint{0.647939in}{0.492442in}}{\pgfqpoint{4.273799in}{2.331163in}}%
\pgfusepath{clip}%
\pgfsetroundcap%
\pgfsetroundjoin%
\definecolor{currentfill}{rgb}{0.500000,0.500000,0.500000}%
\pgfsetfillcolor{currentfill}%
\pgfsetfillopacity{0.300000}%
\pgfsetlinewidth{0.301125pt}%
\definecolor{currentstroke}{rgb}{0.500000,0.500000,0.500000}%
\pgfsetstrokecolor{currentstroke}%
\pgfsetstrokeopacity{0.300000}%
\pgfsetdash{}{0pt}%
\pgfpathmoveto{\pgfqpoint{0.000000in}{0.000000in}}%
\pgfpathlineto{\pgfqpoint{0.000000in}{0.000000in}}%
\pgfpathclose%
\pgfusepath{stroke,fill}%
\end{pgfscope}%
\begin{pgfscope}%
\pgfpathrectangle{\pgfqpoint{0.647939in}{0.492442in}}{\pgfqpoint{4.273799in}{2.331163in}}%
\pgfusepath{clip}%
\pgfsetroundcap%
\pgfsetroundjoin%
\pgfsetlinewidth{0.301125pt}%
\definecolor{currentstroke}{rgb}{0.500000,0.500000,0.500000}%
\pgfsetstrokecolor{currentstroke}%
\pgfsetstrokeopacity{0.300000}%
\pgfsetdash{}{0pt}%
\pgfpathmoveto{\pgfqpoint{2.199242in}{1.567066in}}%
\pgfusepath{stroke}%
\end{pgfscope}%
\begin{pgfscope}%
\pgfpathrectangle{\pgfqpoint{0.647939in}{0.492442in}}{\pgfqpoint{4.273799in}{2.331163in}}%
\pgfusepath{clip}%
\pgfsetroundcap%
\pgfsetroundjoin%
\definecolor{currentfill}{rgb}{0.500000,0.500000,0.500000}%
\pgfsetfillcolor{currentfill}%
\pgfsetfillopacity{0.300000}%
\pgfsetlinewidth{0.301125pt}%
\definecolor{currentstroke}{rgb}{0.500000,0.500000,0.500000}%
\pgfsetstrokecolor{currentstroke}%
\pgfsetstrokeopacity{0.300000}%
\pgfsetdash{}{0pt}%
\pgfpathmoveto{\pgfqpoint{0.000000in}{0.000000in}}%
\pgfpathlineto{\pgfqpoint{0.000000in}{0.000000in}}%
\pgfpathclose%
\pgfusepath{stroke,fill}%
\end{pgfscope}%
\begin{pgfscope}%
\pgfpathrectangle{\pgfqpoint{0.647939in}{0.492442in}}{\pgfqpoint{4.273799in}{2.331163in}}%
\pgfusepath{clip}%
\pgfsetroundcap%
\pgfsetroundjoin%
\pgfsetlinewidth{0.301125pt}%
\definecolor{currentstroke}{rgb}{0.500000,0.500000,0.500000}%
\pgfsetstrokecolor{currentstroke}%
\pgfsetstrokeopacity{0.300000}%
\pgfsetdash{}{0pt}%
\pgfpathmoveto{\pgfqpoint{2.837356in}{0.653710in}}%
\pgfusepath{stroke}%
\end{pgfscope}%
\begin{pgfscope}%
\pgfpathrectangle{\pgfqpoint{0.647939in}{0.492442in}}{\pgfqpoint{4.273799in}{2.331163in}}%
\pgfusepath{clip}%
\pgfsetroundcap%
\pgfsetroundjoin%
\definecolor{currentfill}{rgb}{0.500000,0.500000,0.500000}%
\pgfsetfillcolor{currentfill}%
\pgfsetfillopacity{0.300000}%
\pgfsetlinewidth{0.301125pt}%
\definecolor{currentstroke}{rgb}{0.500000,0.500000,0.500000}%
\pgfsetstrokecolor{currentstroke}%
\pgfsetstrokeopacity{0.300000}%
\pgfsetdash{}{0pt}%
\pgfpathmoveto{\pgfqpoint{0.000000in}{0.000000in}}%
\pgfpathlineto{\pgfqpoint{0.000000in}{0.000000in}}%
\pgfpathclose%
\pgfusepath{stroke,fill}%
\end{pgfscope}%
\begin{pgfscope}%
\pgfpathrectangle{\pgfqpoint{0.647939in}{0.492442in}}{\pgfqpoint{4.273799in}{2.331163in}}%
\pgfusepath{clip}%
\pgfsetroundcap%
\pgfsetroundjoin%
\pgfsetlinewidth{0.301125pt}%
\definecolor{currentstroke}{rgb}{0.500000,0.500000,0.500000}%
\pgfsetstrokecolor{currentstroke}%
\pgfsetstrokeopacity{0.300000}%
\pgfsetdash{}{0pt}%
\pgfpathmoveto{\pgfqpoint{2.385996in}{1.599016in}}%
\pgfusepath{stroke}%
\end{pgfscope}%
\begin{pgfscope}%
\pgfpathrectangle{\pgfqpoint{0.647939in}{0.492442in}}{\pgfqpoint{4.273799in}{2.331163in}}%
\pgfusepath{clip}%
\pgfsetroundcap%
\pgfsetroundjoin%
\definecolor{currentfill}{rgb}{0.500000,0.500000,0.500000}%
\pgfsetfillcolor{currentfill}%
\pgfsetfillopacity{0.300000}%
\pgfsetlinewidth{0.301125pt}%
\definecolor{currentstroke}{rgb}{0.500000,0.500000,0.500000}%
\pgfsetstrokecolor{currentstroke}%
\pgfsetstrokeopacity{0.300000}%
\pgfsetdash{}{0pt}%
\pgfpathmoveto{\pgfqpoint{0.000000in}{0.000000in}}%
\pgfpathlineto{\pgfqpoint{0.000000in}{0.000000in}}%
\pgfpathclose%
\pgfusepath{stroke,fill}%
\end{pgfscope}%
\begin{pgfscope}%
\pgfpathrectangle{\pgfqpoint{0.647939in}{0.492442in}}{\pgfqpoint{4.273799in}{2.331163in}}%
\pgfusepath{clip}%
\pgfsetroundcap%
\pgfsetroundjoin%
\pgfsetlinewidth{0.301125pt}%
\definecolor{currentstroke}{rgb}{0.500000,0.500000,0.500000}%
\pgfsetstrokecolor{currentstroke}%
\pgfsetstrokeopacity{0.300000}%
\pgfsetdash{}{0pt}%
\pgfpathmoveto{\pgfqpoint{2.589612in}{1.404769in}}%
\pgfusepath{stroke}%
\end{pgfscope}%
\begin{pgfscope}%
\pgfpathrectangle{\pgfqpoint{0.647939in}{0.492442in}}{\pgfqpoint{4.273799in}{2.331163in}}%
\pgfusepath{clip}%
\pgfsetroundcap%
\pgfsetroundjoin%
\definecolor{currentfill}{rgb}{0.500000,0.500000,0.500000}%
\pgfsetfillcolor{currentfill}%
\pgfsetfillopacity{0.300000}%
\pgfsetlinewidth{0.301125pt}%
\definecolor{currentstroke}{rgb}{0.500000,0.500000,0.500000}%
\pgfsetstrokecolor{currentstroke}%
\pgfsetstrokeopacity{0.300000}%
\pgfsetdash{}{0pt}%
\pgfpathmoveto{\pgfqpoint{0.000000in}{0.000000in}}%
\pgfpathlineto{\pgfqpoint{0.000000in}{0.000000in}}%
\pgfpathclose%
\pgfusepath{stroke,fill}%
\end{pgfscope}%
\begin{pgfscope}%
\pgfpathrectangle{\pgfqpoint{0.647939in}{0.492442in}}{\pgfqpoint{4.273799in}{2.331163in}}%
\pgfusepath{clip}%
\pgfsetroundcap%
\pgfsetroundjoin%
\pgfsetlinewidth{0.301125pt}%
\definecolor{currentstroke}{rgb}{0.500000,0.500000,0.500000}%
\pgfsetstrokecolor{currentstroke}%
\pgfsetstrokeopacity{0.300000}%
\pgfsetdash{}{0pt}%
\pgfpathmoveto{\pgfqpoint{3.083053in}{0.905069in}}%
\pgfusepath{stroke}%
\end{pgfscope}%
\begin{pgfscope}%
\pgfpathrectangle{\pgfqpoint{0.647939in}{0.492442in}}{\pgfqpoint{4.273799in}{2.331163in}}%
\pgfusepath{clip}%
\pgfsetroundcap%
\pgfsetroundjoin%
\definecolor{currentfill}{rgb}{0.500000,0.500000,0.500000}%
\pgfsetfillcolor{currentfill}%
\pgfsetfillopacity{0.300000}%
\pgfsetlinewidth{0.301125pt}%
\definecolor{currentstroke}{rgb}{0.500000,0.500000,0.500000}%
\pgfsetstrokecolor{currentstroke}%
\pgfsetstrokeopacity{0.300000}%
\pgfsetdash{}{0pt}%
\pgfpathmoveto{\pgfqpoint{0.000000in}{0.000000in}}%
\pgfpathlineto{\pgfqpoint{0.000000in}{0.000000in}}%
\pgfpathclose%
\pgfusepath{stroke,fill}%
\end{pgfscope}%
\begin{pgfscope}%
\pgfpathrectangle{\pgfqpoint{0.647939in}{0.492442in}}{\pgfqpoint{4.273799in}{2.331163in}}%
\pgfusepath{clip}%
\pgfsetroundcap%
\pgfsetroundjoin%
\pgfsetlinewidth{0.301125pt}%
\definecolor{currentstroke}{rgb}{0.500000,0.500000,0.500000}%
\pgfsetstrokecolor{currentstroke}%
\pgfsetstrokeopacity{0.300000}%
\pgfsetdash{}{0pt}%
\pgfpathmoveto{\pgfqpoint{3.397231in}{0.760281in}}%
\pgfusepath{stroke}%
\end{pgfscope}%
\begin{pgfscope}%
\pgfpathrectangle{\pgfqpoint{0.647939in}{0.492442in}}{\pgfqpoint{4.273799in}{2.331163in}}%
\pgfusepath{clip}%
\pgfsetroundcap%
\pgfsetroundjoin%
\definecolor{currentfill}{rgb}{0.500000,0.500000,0.500000}%
\pgfsetfillcolor{currentfill}%
\pgfsetfillopacity{0.300000}%
\pgfsetlinewidth{0.301125pt}%
\definecolor{currentstroke}{rgb}{0.500000,0.500000,0.500000}%
\pgfsetstrokecolor{currentstroke}%
\pgfsetstrokeopacity{0.300000}%
\pgfsetdash{}{0pt}%
\pgfpathmoveto{\pgfqpoint{0.000000in}{0.000000in}}%
\pgfpathlineto{\pgfqpoint{0.000000in}{0.000000in}}%
\pgfpathclose%
\pgfusepath{stroke,fill}%
\end{pgfscope}%
\begin{pgfscope}%
\pgfpathrectangle{\pgfqpoint{0.647939in}{0.492442in}}{\pgfqpoint{4.273799in}{2.331163in}}%
\pgfusepath{clip}%
\pgfsetroundcap%
\pgfsetroundjoin%
\pgfsetlinewidth{0.301125pt}%
\definecolor{currentstroke}{rgb}{0.500000,0.500000,0.500000}%
\pgfsetstrokecolor{currentstroke}%
\pgfsetstrokeopacity{0.300000}%
\pgfsetdash{}{0pt}%
\pgfpathmoveto{\pgfqpoint{2.937327in}{1.331726in}}%
\pgfusepath{stroke}%
\end{pgfscope}%
\begin{pgfscope}%
\pgfpathrectangle{\pgfqpoint{0.647939in}{0.492442in}}{\pgfqpoint{4.273799in}{2.331163in}}%
\pgfusepath{clip}%
\pgfsetroundcap%
\pgfsetroundjoin%
\definecolor{currentfill}{rgb}{0.500000,0.500000,0.500000}%
\pgfsetfillcolor{currentfill}%
\pgfsetfillopacity{0.300000}%
\pgfsetlinewidth{0.301125pt}%
\definecolor{currentstroke}{rgb}{0.500000,0.500000,0.500000}%
\pgfsetstrokecolor{currentstroke}%
\pgfsetstrokeopacity{0.300000}%
\pgfsetdash{}{0pt}%
\pgfpathmoveto{\pgfqpoint{0.000000in}{0.000000in}}%
\pgfpathlineto{\pgfqpoint{0.000000in}{0.000000in}}%
\pgfpathclose%
\pgfusepath{stroke,fill}%
\end{pgfscope}%
\begin{pgfscope}%
\pgfpathrectangle{\pgfqpoint{0.647939in}{0.492442in}}{\pgfqpoint{4.273799in}{2.331163in}}%
\pgfusepath{clip}%
\pgfsetroundcap%
\pgfsetroundjoin%
\pgfsetlinewidth{0.301125pt}%
\definecolor{currentstroke}{rgb}{0.500000,0.500000,0.500000}%
\pgfsetstrokecolor{currentstroke}%
\pgfsetstrokeopacity{0.300000}%
\pgfsetdash{}{0pt}%
\pgfpathmoveto{\pgfqpoint{3.492233in}{0.962343in}}%
\pgfusepath{stroke}%
\end{pgfscope}%
\begin{pgfscope}%
\pgfpathrectangle{\pgfqpoint{0.647939in}{0.492442in}}{\pgfqpoint{4.273799in}{2.331163in}}%
\pgfusepath{clip}%
\pgfsetroundcap%
\pgfsetroundjoin%
\definecolor{currentfill}{rgb}{0.500000,0.500000,0.500000}%
\pgfsetfillcolor{currentfill}%
\pgfsetfillopacity{0.300000}%
\pgfsetlinewidth{0.301125pt}%
\definecolor{currentstroke}{rgb}{0.500000,0.500000,0.500000}%
\pgfsetstrokecolor{currentstroke}%
\pgfsetstrokeopacity{0.300000}%
\pgfsetdash{}{0pt}%
\pgfpathmoveto{\pgfqpoint{0.000000in}{0.000000in}}%
\pgfpathlineto{\pgfqpoint{0.000000in}{0.000000in}}%
\pgfpathclose%
\pgfusepath{stroke,fill}%
\end{pgfscope}%
\begin{pgfscope}%
\pgfpathrectangle{\pgfqpoint{0.647939in}{0.492442in}}{\pgfqpoint{4.273799in}{2.331163in}}%
\pgfusepath{clip}%
\pgfsetroundcap%
\pgfsetroundjoin%
\pgfsetlinewidth{0.301125pt}%
\definecolor{currentstroke}{rgb}{0.500000,0.500000,0.500000}%
\pgfsetstrokecolor{currentstroke}%
\pgfsetstrokeopacity{0.300000}%
\pgfsetdash{}{0pt}%
\pgfpathmoveto{\pgfqpoint{3.964157in}{0.727050in}}%
\pgfusepath{stroke}%
\end{pgfscope}%
\begin{pgfscope}%
\pgfpathrectangle{\pgfqpoint{0.647939in}{0.492442in}}{\pgfqpoint{4.273799in}{2.331163in}}%
\pgfusepath{clip}%
\pgfsetroundcap%
\pgfsetroundjoin%
\definecolor{currentfill}{rgb}{0.500000,0.500000,0.500000}%
\pgfsetfillcolor{currentfill}%
\pgfsetfillopacity{0.300000}%
\pgfsetlinewidth{0.301125pt}%
\definecolor{currentstroke}{rgb}{0.500000,0.500000,0.500000}%
\pgfsetstrokecolor{currentstroke}%
\pgfsetstrokeopacity{0.300000}%
\pgfsetdash{}{0pt}%
\pgfpathmoveto{\pgfqpoint{0.000000in}{0.000000in}}%
\pgfpathlineto{\pgfqpoint{0.000000in}{0.000000in}}%
\pgfpathclose%
\pgfusepath{stroke,fill}%
\end{pgfscope}%
\begin{pgfscope}%
\pgfpathrectangle{\pgfqpoint{0.647939in}{0.492442in}}{\pgfqpoint{4.273799in}{2.331163in}}%
\pgfusepath{clip}%
\pgfsetroundcap%
\pgfsetroundjoin%
\pgfsetlinewidth{0.301125pt}%
\definecolor{currentstroke}{rgb}{0.500000,0.500000,0.500000}%
\pgfsetstrokecolor{currentstroke}%
\pgfsetstrokeopacity{0.300000}%
\pgfsetdash{}{0pt}%
\pgfpathmoveto{\pgfqpoint{3.356876in}{1.212469in}}%
\pgfusepath{stroke}%
\end{pgfscope}%
\begin{pgfscope}%
\pgfpathrectangle{\pgfqpoint{0.647939in}{0.492442in}}{\pgfqpoint{4.273799in}{2.331163in}}%
\pgfusepath{clip}%
\pgfsetroundcap%
\pgfsetroundjoin%
\definecolor{currentfill}{rgb}{0.500000,0.500000,0.500000}%
\pgfsetfillcolor{currentfill}%
\pgfsetfillopacity{0.300000}%
\pgfsetlinewidth{0.301125pt}%
\definecolor{currentstroke}{rgb}{0.500000,0.500000,0.500000}%
\pgfsetstrokecolor{currentstroke}%
\pgfsetstrokeopacity{0.300000}%
\pgfsetdash{}{0pt}%
\pgfpathmoveto{\pgfqpoint{0.000000in}{0.000000in}}%
\pgfpathlineto{\pgfqpoint{0.000000in}{0.000000in}}%
\pgfpathclose%
\pgfusepath{stroke,fill}%
\end{pgfscope}%
\begin{pgfscope}%
\pgfpathrectangle{\pgfqpoint{0.647939in}{0.492442in}}{\pgfqpoint{4.273799in}{2.331163in}}%
\pgfusepath{clip}%
\pgfsetroundcap%
\pgfsetroundjoin%
\pgfsetlinewidth{0.301125pt}%
\definecolor{currentstroke}{rgb}{0.500000,0.500000,0.500000}%
\pgfsetstrokecolor{currentstroke}%
\pgfsetstrokeopacity{0.300000}%
\pgfsetdash{}{0pt}%
\pgfpathmoveto{\pgfqpoint{4.050802in}{0.872350in}}%
\pgfusepath{stroke}%
\end{pgfscope}%
\begin{pgfscope}%
\pgfpathrectangle{\pgfqpoint{0.647939in}{0.492442in}}{\pgfqpoint{4.273799in}{2.331163in}}%
\pgfusepath{clip}%
\pgfsetroundcap%
\pgfsetroundjoin%
\definecolor{currentfill}{rgb}{0.500000,0.500000,0.500000}%
\pgfsetfillcolor{currentfill}%
\pgfsetfillopacity{0.300000}%
\pgfsetlinewidth{0.301125pt}%
\definecolor{currentstroke}{rgb}{0.500000,0.500000,0.500000}%
\pgfsetstrokecolor{currentstroke}%
\pgfsetstrokeopacity{0.300000}%
\pgfsetdash{}{0pt}%
\pgfpathmoveto{\pgfqpoint{0.000000in}{0.000000in}}%
\pgfpathlineto{\pgfqpoint{0.000000in}{0.000000in}}%
\pgfpathclose%
\pgfusepath{stroke,fill}%
\end{pgfscope}%
\begin{pgfscope}%
\pgfpathrectangle{\pgfqpoint{0.647939in}{0.492442in}}{\pgfqpoint{4.273799in}{2.331163in}}%
\pgfusepath{clip}%
\pgfsetroundcap%
\pgfsetroundjoin%
\pgfsetlinewidth{0.301125pt}%
\definecolor{currentstroke}{rgb}{0.500000,0.500000,0.500000}%
\pgfsetstrokecolor{currentstroke}%
\pgfsetstrokeopacity{0.300000}%
\pgfsetdash{}{0pt}%
\pgfpathmoveto{\pgfqpoint{3.502618in}{1.306329in}}%
\pgfusepath{stroke}%
\end{pgfscope}%
\begin{pgfscope}%
\pgfpathrectangle{\pgfqpoint{0.647939in}{0.492442in}}{\pgfqpoint{4.273799in}{2.331163in}}%
\pgfusepath{clip}%
\pgfsetroundcap%
\pgfsetroundjoin%
\definecolor{currentfill}{rgb}{0.500000,0.500000,0.500000}%
\pgfsetfillcolor{currentfill}%
\pgfsetfillopacity{0.300000}%
\pgfsetlinewidth{0.301125pt}%
\definecolor{currentstroke}{rgb}{0.500000,0.500000,0.500000}%
\pgfsetstrokecolor{currentstroke}%
\pgfsetstrokeopacity{0.300000}%
\pgfsetdash{}{0pt}%
\pgfpathmoveto{\pgfqpoint{0.000000in}{0.000000in}}%
\pgfpathlineto{\pgfqpoint{0.000000in}{0.000000in}}%
\pgfpathclose%
\pgfusepath{stroke,fill}%
\end{pgfscope}%
\begin{pgfscope}%
\pgfpathrectangle{\pgfqpoint{0.647939in}{0.492442in}}{\pgfqpoint{4.273799in}{2.331163in}}%
\pgfusepath{clip}%
\pgfsetroundcap%
\pgfsetroundjoin%
\pgfsetlinewidth{0.301125pt}%
\definecolor{currentstroke}{rgb}{0.500000,0.500000,0.500000}%
\pgfsetstrokecolor{currentstroke}%
\pgfsetstrokeopacity{0.300000}%
\pgfsetdash{}{0pt}%
\pgfpathmoveto{\pgfqpoint{4.043200in}{1.160736in}}%
\pgfusepath{stroke}%
\end{pgfscope}%
\begin{pgfscope}%
\pgfpathrectangle{\pgfqpoint{0.647939in}{0.492442in}}{\pgfqpoint{4.273799in}{2.331163in}}%
\pgfusepath{clip}%
\pgfsetroundcap%
\pgfsetroundjoin%
\definecolor{currentfill}{rgb}{0.500000,0.500000,0.500000}%
\pgfsetfillcolor{currentfill}%
\pgfsetfillopacity{0.300000}%
\pgfsetlinewidth{0.301125pt}%
\definecolor{currentstroke}{rgb}{0.500000,0.500000,0.500000}%
\pgfsetstrokecolor{currentstroke}%
\pgfsetstrokeopacity{0.300000}%
\pgfsetdash{}{0pt}%
\pgfpathmoveto{\pgfqpoint{0.000000in}{0.000000in}}%
\pgfpathlineto{\pgfqpoint{0.000000in}{0.000000in}}%
\pgfpathclose%
\pgfusepath{stroke,fill}%
\end{pgfscope}%
\begin{pgfscope}%
\pgfpathrectangle{\pgfqpoint{0.647939in}{0.492442in}}{\pgfqpoint{4.273799in}{2.331163in}}%
\pgfusepath{clip}%
\pgfsetroundcap%
\pgfsetroundjoin%
\pgfsetlinewidth{0.301125pt}%
\definecolor{currentstroke}{rgb}{0.500000,0.500000,0.500000}%
\pgfsetstrokecolor{currentstroke}%
\pgfsetstrokeopacity{0.300000}%
\pgfsetdash{}{0pt}%
\pgfpathmoveto{\pgfqpoint{3.881388in}{1.417222in}}%
\pgfusepath{stroke}%
\end{pgfscope}%
\begin{pgfscope}%
\pgfpathrectangle{\pgfqpoint{0.647939in}{0.492442in}}{\pgfqpoint{4.273799in}{2.331163in}}%
\pgfusepath{clip}%
\pgfsetroundcap%
\pgfsetroundjoin%
\definecolor{currentfill}{rgb}{0.500000,0.500000,0.500000}%
\pgfsetfillcolor{currentfill}%
\pgfsetfillopacity{0.300000}%
\pgfsetlinewidth{0.301125pt}%
\definecolor{currentstroke}{rgb}{0.500000,0.500000,0.500000}%
\pgfsetstrokecolor{currentstroke}%
\pgfsetstrokeopacity{0.300000}%
\pgfsetdash{}{0pt}%
\pgfpathmoveto{\pgfqpoint{0.000000in}{0.000000in}}%
\pgfpathlineto{\pgfqpoint{0.000000in}{0.000000in}}%
\pgfpathclose%
\pgfusepath{stroke,fill}%
\end{pgfscope}%
\begin{pgfscope}%
\pgfpathrectangle{\pgfqpoint{0.647939in}{0.492442in}}{\pgfqpoint{4.273799in}{2.331163in}}%
\pgfusepath{clip}%
\pgfsetroundcap%
\pgfsetroundjoin%
\pgfsetlinewidth{0.301125pt}%
\definecolor{currentstroke}{rgb}{0.500000,0.500000,0.500000}%
\pgfsetstrokecolor{currentstroke}%
\pgfsetstrokeopacity{0.300000}%
\pgfsetdash{}{0pt}%
\pgfpathmoveto{\pgfqpoint{4.368313in}{1.307323in}}%
\pgfusepath{stroke}%
\end{pgfscope}%
\begin{pgfscope}%
\pgfpathrectangle{\pgfqpoint{0.647939in}{0.492442in}}{\pgfqpoint{4.273799in}{2.331163in}}%
\pgfusepath{clip}%
\pgfsetroundcap%
\pgfsetroundjoin%
\definecolor{currentfill}{rgb}{0.500000,0.500000,0.500000}%
\pgfsetfillcolor{currentfill}%
\pgfsetfillopacity{0.300000}%
\pgfsetlinewidth{0.301125pt}%
\definecolor{currentstroke}{rgb}{0.500000,0.500000,0.500000}%
\pgfsetstrokecolor{currentstroke}%
\pgfsetstrokeopacity{0.300000}%
\pgfsetdash{}{0pt}%
\pgfpathmoveto{\pgfqpoint{0.000000in}{0.000000in}}%
\pgfpathlineto{\pgfqpoint{0.000000in}{0.000000in}}%
\pgfpathclose%
\pgfusepath{stroke,fill}%
\end{pgfscope}%
\begin{pgfscope}%
\pgfpathrectangle{\pgfqpoint{0.647939in}{0.492442in}}{\pgfqpoint{4.273799in}{2.331163in}}%
\pgfusepath{clip}%
\pgfsetroundcap%
\pgfsetroundjoin%
\pgfsetlinewidth{0.301125pt}%
\definecolor{currentstroke}{rgb}{0.500000,0.500000,0.500000}%
\pgfsetstrokecolor{currentstroke}%
\pgfsetstrokeopacity{0.300000}%
\pgfsetdash{}{0pt}%
\pgfpathmoveto{\pgfqpoint{4.352134in}{1.721680in}}%
\pgfusepath{stroke}%
\end{pgfscope}%
\begin{pgfscope}%
\pgfpathrectangle{\pgfqpoint{0.647939in}{0.492442in}}{\pgfqpoint{4.273799in}{2.331163in}}%
\pgfusepath{clip}%
\pgfsetroundcap%
\pgfsetroundjoin%
\definecolor{currentfill}{rgb}{0.500000,0.500000,0.500000}%
\pgfsetfillcolor{currentfill}%
\pgfsetfillopacity{0.300000}%
\pgfsetlinewidth{0.301125pt}%
\definecolor{currentstroke}{rgb}{0.500000,0.500000,0.500000}%
\pgfsetstrokecolor{currentstroke}%
\pgfsetstrokeopacity{0.300000}%
\pgfsetdash{}{0pt}%
\pgfpathmoveto{\pgfqpoint{0.000000in}{0.000000in}}%
\pgfpathlineto{\pgfqpoint{0.000000in}{0.000000in}}%
\pgfpathclose%
\pgfusepath{stroke,fill}%
\end{pgfscope}%
\begin{pgfscope}%
\pgfpathrectangle{\pgfqpoint{0.647939in}{0.492442in}}{\pgfqpoint{4.273799in}{2.331163in}}%
\pgfusepath{clip}%
\pgfsetroundcap%
\pgfsetroundjoin%
\pgfsetlinewidth{0.301125pt}%
\definecolor{currentstroke}{rgb}{0.500000,0.500000,0.500000}%
\pgfsetstrokecolor{currentstroke}%
\pgfsetstrokeopacity{0.300000}%
\pgfsetdash{}{0pt}%
\pgfpathmoveto{\pgfqpoint{4.553970in}{1.773379in}}%
\pgfusepath{stroke}%
\end{pgfscope}%
\begin{pgfscope}%
\pgfpathrectangle{\pgfqpoint{0.647939in}{0.492442in}}{\pgfqpoint{4.273799in}{2.331163in}}%
\pgfusepath{clip}%
\pgfsetroundcap%
\pgfsetroundjoin%
\definecolor{currentfill}{rgb}{0.500000,0.500000,0.500000}%
\pgfsetfillcolor{currentfill}%
\pgfsetfillopacity{0.300000}%
\pgfsetlinewidth{0.301125pt}%
\definecolor{currentstroke}{rgb}{0.500000,0.500000,0.500000}%
\pgfsetstrokecolor{currentstroke}%
\pgfsetstrokeopacity{0.300000}%
\pgfsetdash{}{0pt}%
\pgfpathmoveto{\pgfqpoint{0.000000in}{0.000000in}}%
\pgfpathlineto{\pgfqpoint{0.000000in}{0.000000in}}%
\pgfpathclose%
\pgfusepath{stroke,fill}%
\end{pgfscope}%
\begin{pgfscope}%
\pgfpathrectangle{\pgfqpoint{0.647939in}{0.492442in}}{\pgfqpoint{4.273799in}{2.331163in}}%
\pgfusepath{clip}%
\pgfsetroundcap%
\pgfsetroundjoin%
\pgfsetlinewidth{0.301125pt}%
\definecolor{currentstroke}{rgb}{0.500000,0.500000,0.500000}%
\pgfsetstrokecolor{currentstroke}%
\pgfsetstrokeopacity{0.300000}%
\pgfsetdash{}{0pt}%
\pgfpathmoveto{\pgfqpoint{4.728539in}{1.686674in}}%
\pgfusepath{stroke}%
\end{pgfscope}%
\begin{pgfscope}%
\pgfpathrectangle{\pgfqpoint{0.647939in}{0.492442in}}{\pgfqpoint{4.273799in}{2.331163in}}%
\pgfusepath{clip}%
\pgfsetroundcap%
\pgfsetroundjoin%
\definecolor{currentfill}{rgb}{0.500000,0.500000,0.500000}%
\pgfsetfillcolor{currentfill}%
\pgfsetfillopacity{0.300000}%
\pgfsetlinewidth{0.301125pt}%
\definecolor{currentstroke}{rgb}{0.500000,0.500000,0.500000}%
\pgfsetstrokecolor{currentstroke}%
\pgfsetstrokeopacity{0.300000}%
\pgfsetdash{}{0pt}%
\pgfpathmoveto{\pgfqpoint{0.000000in}{0.000000in}}%
\pgfpathlineto{\pgfqpoint{0.000000in}{0.000000in}}%
\pgfpathclose%
\pgfusepath{stroke,fill}%
\end{pgfscope}%
\begin{pgfscope}%
\pgfpathrectangle{\pgfqpoint{0.647939in}{0.492442in}}{\pgfqpoint{4.273799in}{2.331163in}}%
\pgfusepath{clip}%
\pgfsetroundcap%
\pgfsetroundjoin%
\pgfsetlinewidth{0.301125pt}%
\definecolor{currentstroke}{rgb}{0.500000,0.500000,0.500000}%
\pgfsetstrokecolor{currentstroke}%
\pgfsetstrokeopacity{0.300000}%
\pgfsetdash{}{0pt}%
\pgfpathmoveto{\pgfqpoint{4.819579in}{1.749092in}}%
\pgfusepath{stroke}%
\end{pgfscope}%
\begin{pgfscope}%
\pgfpathrectangle{\pgfqpoint{0.647939in}{0.492442in}}{\pgfqpoint{4.273799in}{2.331163in}}%
\pgfusepath{clip}%
\pgfsetroundcap%
\pgfsetroundjoin%
\definecolor{currentfill}{rgb}{0.500000,0.500000,0.500000}%
\pgfsetfillcolor{currentfill}%
\pgfsetfillopacity{0.300000}%
\pgfsetlinewidth{0.301125pt}%
\definecolor{currentstroke}{rgb}{0.500000,0.500000,0.500000}%
\pgfsetstrokecolor{currentstroke}%
\pgfsetstrokeopacity{0.300000}%
\pgfsetdash{}{0pt}%
\pgfpathmoveto{\pgfqpoint{0.000000in}{0.000000in}}%
\pgfpathlineto{\pgfqpoint{0.000000in}{0.000000in}}%
\pgfpathclose%
\pgfusepath{stroke,fill}%
\end{pgfscope}%
\begin{pgfscope}%
\pgfpathrectangle{\pgfqpoint{0.647939in}{0.492442in}}{\pgfqpoint{4.273799in}{2.331163in}}%
\pgfusepath{clip}%
\pgfsetroundcap%
\pgfsetroundjoin%
\pgfsetlinewidth{0.301125pt}%
\definecolor{currentstroke}{rgb}{0.500000,0.500000,0.500000}%
\pgfsetstrokecolor{currentstroke}%
\pgfsetstrokeopacity{0.300000}%
\pgfsetdash{}{0pt}%
\pgfpathmoveto{\pgfqpoint{4.897123in}{1.653905in}}%
\pgfusepath{stroke}%
\end{pgfscope}%
\begin{pgfscope}%
\pgfpathrectangle{\pgfqpoint{0.647939in}{0.492442in}}{\pgfqpoint{4.273799in}{2.331163in}}%
\pgfusepath{clip}%
\pgfsetroundcap%
\pgfsetroundjoin%
\definecolor{currentfill}{rgb}{0.500000,0.500000,0.500000}%
\pgfsetfillcolor{currentfill}%
\pgfsetfillopacity{0.300000}%
\pgfsetlinewidth{0.301125pt}%
\definecolor{currentstroke}{rgb}{0.500000,0.500000,0.500000}%
\pgfsetstrokecolor{currentstroke}%
\pgfsetstrokeopacity{0.300000}%
\pgfsetdash{}{0pt}%
\pgfpathmoveto{\pgfqpoint{0.000000in}{0.000000in}}%
\pgfpathlineto{\pgfqpoint{0.000000in}{0.000000in}}%
\pgfpathclose%
\pgfusepath{stroke,fill}%
\end{pgfscope}%
\begin{pgfscope}%
\pgfpathrectangle{\pgfqpoint{0.647939in}{0.492442in}}{\pgfqpoint{4.273799in}{2.331163in}}%
\pgfusepath{clip}%
\pgfsetroundcap%
\pgfsetroundjoin%
\pgfsetlinewidth{0.301125pt}%
\definecolor{currentstroke}{rgb}{0.500000,0.500000,0.500000}%
\pgfsetstrokecolor{currentstroke}%
\pgfsetstrokeopacity{0.300000}%
\pgfsetdash{}{0pt}%
\pgfpathmoveto{\pgfqpoint{4.907988in}{2.085601in}}%
\pgfusepath{stroke}%
\end{pgfscope}%
\begin{pgfscope}%
\pgfpathrectangle{\pgfqpoint{0.647939in}{0.492442in}}{\pgfqpoint{4.273799in}{2.331163in}}%
\pgfusepath{clip}%
\pgfsetroundcap%
\pgfsetroundjoin%
\definecolor{currentfill}{rgb}{0.500000,0.500000,0.500000}%
\pgfsetfillcolor{currentfill}%
\pgfsetfillopacity{0.300000}%
\pgfsetlinewidth{0.301125pt}%
\definecolor{currentstroke}{rgb}{0.500000,0.500000,0.500000}%
\pgfsetstrokecolor{currentstroke}%
\pgfsetstrokeopacity{0.300000}%
\pgfsetdash{}{0pt}%
\pgfpathmoveto{\pgfqpoint{0.000000in}{0.000000in}}%
\pgfpathlineto{\pgfqpoint{0.000000in}{0.000000in}}%
\pgfpathclose%
\pgfusepath{stroke,fill}%
\end{pgfscope}%
\begin{pgfscope}%
\pgfpathrectangle{\pgfqpoint{0.647939in}{0.492442in}}{\pgfqpoint{4.273799in}{2.331163in}}%
\pgfusepath{clip}%
\pgfsetroundcap%
\pgfsetroundjoin%
\pgfsetlinewidth{0.301125pt}%
\definecolor{currentstroke}{rgb}{0.500000,0.500000,0.500000}%
\pgfsetstrokecolor{currentstroke}%
\pgfsetstrokeopacity{0.300000}%
\pgfsetdash{}{0pt}%
\pgfpathmoveto{\pgfqpoint{4.506767in}{2.660418in}}%
\pgfusepath{stroke}%
\end{pgfscope}%
\begin{pgfscope}%
\pgfpathrectangle{\pgfqpoint{0.647939in}{0.492442in}}{\pgfqpoint{4.273799in}{2.331163in}}%
\pgfusepath{clip}%
\pgfsetroundcap%
\pgfsetroundjoin%
\definecolor{currentfill}{rgb}{0.500000,0.500000,0.500000}%
\pgfsetfillcolor{currentfill}%
\pgfsetfillopacity{0.300000}%
\pgfsetlinewidth{0.301125pt}%
\definecolor{currentstroke}{rgb}{0.500000,0.500000,0.500000}%
\pgfsetstrokecolor{currentstroke}%
\pgfsetstrokeopacity{0.300000}%
\pgfsetdash{}{0pt}%
\pgfpathmoveto{\pgfqpoint{0.000000in}{0.000000in}}%
\pgfpathlineto{\pgfqpoint{0.000000in}{0.000000in}}%
\pgfpathclose%
\pgfusepath{stroke,fill}%
\end{pgfscope}%
\begin{pgfscope}%
\pgfpathrectangle{\pgfqpoint{0.647939in}{0.492442in}}{\pgfqpoint{4.273799in}{2.331163in}}%
\pgfusepath{clip}%
\pgfsetroundcap%
\pgfsetroundjoin%
\pgfsetlinewidth{0.301125pt}%
\definecolor{currentstroke}{rgb}{0.500000,0.500000,0.500000}%
\pgfsetstrokecolor{currentstroke}%
\pgfsetstrokeopacity{0.300000}%
\pgfsetdash{}{0pt}%
\pgfpathmoveto{\pgfqpoint{4.433722in}{2.570885in}}%
\pgfusepath{stroke}%
\end{pgfscope}%
\begin{pgfscope}%
\pgfpathrectangle{\pgfqpoint{0.647939in}{0.492442in}}{\pgfqpoint{4.273799in}{2.331163in}}%
\pgfusepath{clip}%
\pgfsetroundcap%
\pgfsetroundjoin%
\definecolor{currentfill}{rgb}{0.500000,0.500000,0.500000}%
\pgfsetfillcolor{currentfill}%
\pgfsetfillopacity{0.300000}%
\pgfsetlinewidth{0.301125pt}%
\definecolor{currentstroke}{rgb}{0.500000,0.500000,0.500000}%
\pgfsetstrokecolor{currentstroke}%
\pgfsetstrokeopacity{0.300000}%
\pgfsetdash{}{0pt}%
\pgfpathmoveto{\pgfqpoint{0.000000in}{0.000000in}}%
\pgfpathlineto{\pgfqpoint{0.000000in}{0.000000in}}%
\pgfpathclose%
\pgfusepath{stroke,fill}%
\end{pgfscope}%
\begin{pgfscope}%
\pgfpathrectangle{\pgfqpoint{0.647939in}{0.492442in}}{\pgfqpoint{4.273799in}{2.331163in}}%
\pgfusepath{clip}%
\pgfsetroundcap%
\pgfsetroundjoin%
\pgfsetlinewidth{0.301125pt}%
\definecolor{currentstroke}{rgb}{0.500000,0.500000,0.500000}%
\pgfsetstrokecolor{currentstroke}%
\pgfsetstrokeopacity{0.300000}%
\pgfsetdash{}{0pt}%
\pgfpathmoveto{\pgfqpoint{4.260987in}{2.567743in}}%
\pgfusepath{stroke}%
\end{pgfscope}%
\begin{pgfscope}%
\pgfpathrectangle{\pgfqpoint{0.647939in}{0.492442in}}{\pgfqpoint{4.273799in}{2.331163in}}%
\pgfusepath{clip}%
\pgfsetroundcap%
\pgfsetroundjoin%
\definecolor{currentfill}{rgb}{0.500000,0.500000,0.500000}%
\pgfsetfillcolor{currentfill}%
\pgfsetfillopacity{0.300000}%
\pgfsetlinewidth{0.301125pt}%
\definecolor{currentstroke}{rgb}{0.500000,0.500000,0.500000}%
\pgfsetstrokecolor{currentstroke}%
\pgfsetstrokeopacity{0.300000}%
\pgfsetdash{}{0pt}%
\pgfpathmoveto{\pgfqpoint{0.000000in}{0.000000in}}%
\pgfpathlineto{\pgfqpoint{0.000000in}{0.000000in}}%
\pgfpathclose%
\pgfusepath{stroke,fill}%
\end{pgfscope}%
\begin{pgfscope}%
\pgfpathrectangle{\pgfqpoint{0.647939in}{0.492442in}}{\pgfqpoint{4.273799in}{2.331163in}}%
\pgfusepath{clip}%
\pgfsetroundcap%
\pgfsetroundjoin%
\pgfsetlinewidth{0.301125pt}%
\definecolor{currentstroke}{rgb}{0.500000,0.500000,0.500000}%
\pgfsetstrokecolor{currentstroke}%
\pgfsetstrokeopacity{0.300000}%
\pgfsetdash{}{0pt}%
\pgfpathmoveto{\pgfqpoint{4.217992in}{2.466701in}}%
\pgfusepath{stroke}%
\end{pgfscope}%
\begin{pgfscope}%
\pgfpathrectangle{\pgfqpoint{0.647939in}{0.492442in}}{\pgfqpoint{4.273799in}{2.331163in}}%
\pgfusepath{clip}%
\pgfsetroundcap%
\pgfsetroundjoin%
\definecolor{currentfill}{rgb}{0.500000,0.500000,0.500000}%
\pgfsetfillcolor{currentfill}%
\pgfsetfillopacity{0.300000}%
\pgfsetlinewidth{0.301125pt}%
\definecolor{currentstroke}{rgb}{0.500000,0.500000,0.500000}%
\pgfsetstrokecolor{currentstroke}%
\pgfsetstrokeopacity{0.300000}%
\pgfsetdash{}{0pt}%
\pgfpathmoveto{\pgfqpoint{0.000000in}{0.000000in}}%
\pgfpathlineto{\pgfqpoint{0.000000in}{0.000000in}}%
\pgfpathclose%
\pgfusepath{stroke,fill}%
\end{pgfscope}%
\begin{pgfscope}%
\pgfpathrectangle{\pgfqpoint{0.647939in}{0.492442in}}{\pgfqpoint{4.273799in}{2.331163in}}%
\pgfusepath{clip}%
\pgfsetroundcap%
\pgfsetroundjoin%
\pgfsetlinewidth{0.301125pt}%
\definecolor{currentstroke}{rgb}{0.500000,0.500000,0.500000}%
\pgfsetstrokecolor{currentstroke}%
\pgfsetstrokeopacity{0.300000}%
\pgfsetdash{}{0pt}%
\pgfpathmoveto{\pgfqpoint{4.164800in}{2.363622in}}%
\pgfusepath{stroke}%
\end{pgfscope}%
\begin{pgfscope}%
\pgfpathrectangle{\pgfqpoint{0.647939in}{0.492442in}}{\pgfqpoint{4.273799in}{2.331163in}}%
\pgfusepath{clip}%
\pgfsetroundcap%
\pgfsetroundjoin%
\definecolor{currentfill}{rgb}{0.500000,0.500000,0.500000}%
\pgfsetfillcolor{currentfill}%
\pgfsetfillopacity{0.300000}%
\pgfsetlinewidth{0.301125pt}%
\definecolor{currentstroke}{rgb}{0.500000,0.500000,0.500000}%
\pgfsetstrokecolor{currentstroke}%
\pgfsetstrokeopacity{0.300000}%
\pgfsetdash{}{0pt}%
\pgfpathmoveto{\pgfqpoint{0.000000in}{0.000000in}}%
\pgfpathlineto{\pgfqpoint{0.000000in}{0.000000in}}%
\pgfpathclose%
\pgfusepath{stroke,fill}%
\end{pgfscope}%
\begin{pgfscope}%
\pgfpathrectangle{\pgfqpoint{0.647939in}{0.492442in}}{\pgfqpoint{4.273799in}{2.331163in}}%
\pgfusepath{clip}%
\pgfsetroundcap%
\pgfsetroundjoin%
\pgfsetlinewidth{0.301125pt}%
\definecolor{currentstroke}{rgb}{0.500000,0.500000,0.500000}%
\pgfsetstrokecolor{currentstroke}%
\pgfsetstrokeopacity{0.300000}%
\pgfsetdash{}{0pt}%
\pgfpathmoveto{\pgfqpoint{4.046704in}{2.358468in}}%
\pgfusepath{stroke}%
\end{pgfscope}%
\begin{pgfscope}%
\pgfpathrectangle{\pgfqpoint{0.647939in}{0.492442in}}{\pgfqpoint{4.273799in}{2.331163in}}%
\pgfusepath{clip}%
\pgfsetroundcap%
\pgfsetroundjoin%
\definecolor{currentfill}{rgb}{0.500000,0.500000,0.500000}%
\pgfsetfillcolor{currentfill}%
\pgfsetfillopacity{0.300000}%
\pgfsetlinewidth{0.301125pt}%
\definecolor{currentstroke}{rgb}{0.500000,0.500000,0.500000}%
\pgfsetstrokecolor{currentstroke}%
\pgfsetstrokeopacity{0.300000}%
\pgfsetdash{}{0pt}%
\pgfpathmoveto{\pgfqpoint{0.000000in}{0.000000in}}%
\pgfpathlineto{\pgfqpoint{0.000000in}{0.000000in}}%
\pgfpathclose%
\pgfusepath{stroke,fill}%
\end{pgfscope}%
\begin{pgfscope}%
\pgfpathrectangle{\pgfqpoint{0.647939in}{0.492442in}}{\pgfqpoint{4.273799in}{2.331163in}}%
\pgfusepath{clip}%
\pgfsetroundcap%
\pgfsetroundjoin%
\pgfsetlinewidth{0.301125pt}%
\definecolor{currentstroke}{rgb}{0.500000,0.500000,0.500000}%
\pgfsetstrokecolor{currentstroke}%
\pgfsetstrokeopacity{0.300000}%
\pgfsetdash{}{0pt}%
\pgfpathmoveto{\pgfqpoint{3.941005in}{2.356516in}}%
\pgfusepath{stroke}%
\end{pgfscope}%
\begin{pgfscope}%
\pgfpathrectangle{\pgfqpoint{0.647939in}{0.492442in}}{\pgfqpoint{4.273799in}{2.331163in}}%
\pgfusepath{clip}%
\pgfsetroundcap%
\pgfsetroundjoin%
\definecolor{currentfill}{rgb}{0.500000,0.500000,0.500000}%
\pgfsetfillcolor{currentfill}%
\pgfsetfillopacity{0.300000}%
\pgfsetlinewidth{0.301125pt}%
\definecolor{currentstroke}{rgb}{0.500000,0.500000,0.500000}%
\pgfsetstrokecolor{currentstroke}%
\pgfsetstrokeopacity{0.300000}%
\pgfsetdash{}{0pt}%
\pgfpathmoveto{\pgfqpoint{0.000000in}{0.000000in}}%
\pgfpathlineto{\pgfqpoint{0.000000in}{0.000000in}}%
\pgfpathclose%
\pgfusepath{stroke,fill}%
\end{pgfscope}%
\begin{pgfscope}%
\pgfpathrectangle{\pgfqpoint{0.647939in}{0.492442in}}{\pgfqpoint{4.273799in}{2.331163in}}%
\pgfusepath{clip}%
\pgfsetroundcap%
\pgfsetroundjoin%
\pgfsetlinewidth{0.301125pt}%
\definecolor{currentstroke}{rgb}{0.500000,0.500000,0.500000}%
\pgfsetstrokecolor{currentstroke}%
\pgfsetstrokeopacity{0.300000}%
\pgfsetdash{}{0pt}%
\pgfpathmoveto{\pgfqpoint{3.674875in}{2.654257in}}%
\pgfusepath{stroke}%
\end{pgfscope}%
\begin{pgfscope}%
\pgfpathrectangle{\pgfqpoint{0.647939in}{0.492442in}}{\pgfqpoint{4.273799in}{2.331163in}}%
\pgfusepath{clip}%
\pgfsetroundcap%
\pgfsetroundjoin%
\definecolor{currentfill}{rgb}{0.500000,0.500000,0.500000}%
\pgfsetfillcolor{currentfill}%
\pgfsetfillopacity{0.300000}%
\pgfsetlinewidth{0.301125pt}%
\definecolor{currentstroke}{rgb}{0.500000,0.500000,0.500000}%
\pgfsetstrokecolor{currentstroke}%
\pgfsetstrokeopacity{0.300000}%
\pgfsetdash{}{0pt}%
\pgfpathmoveto{\pgfqpoint{0.000000in}{0.000000in}}%
\pgfpathlineto{\pgfqpoint{0.000000in}{0.000000in}}%
\pgfpathclose%
\pgfusepath{stroke,fill}%
\end{pgfscope}%
\begin{pgfscope}%
\pgfpathrectangle{\pgfqpoint{0.647939in}{0.492442in}}{\pgfqpoint{4.273799in}{2.331163in}}%
\pgfusepath{clip}%
\pgfsetroundcap%
\pgfsetroundjoin%
\pgfsetlinewidth{0.301125pt}%
\definecolor{currentstroke}{rgb}{0.500000,0.500000,0.500000}%
\pgfsetstrokecolor{currentstroke}%
\pgfsetstrokeopacity{0.300000}%
\pgfsetdash{}{0pt}%
\pgfpathmoveto{\pgfqpoint{3.806243in}{2.231303in}}%
\pgfusepath{stroke}%
\end{pgfscope}%
\begin{pgfscope}%
\pgfpathrectangle{\pgfqpoint{0.647939in}{0.492442in}}{\pgfqpoint{4.273799in}{2.331163in}}%
\pgfusepath{clip}%
\pgfsetroundcap%
\pgfsetroundjoin%
\definecolor{currentfill}{rgb}{0.500000,0.500000,0.500000}%
\pgfsetfillcolor{currentfill}%
\pgfsetfillopacity{0.300000}%
\pgfsetlinewidth{0.301125pt}%
\definecolor{currentstroke}{rgb}{0.500000,0.500000,0.500000}%
\pgfsetstrokecolor{currentstroke}%
\pgfsetstrokeopacity{0.300000}%
\pgfsetdash{}{0pt}%
\pgfpathmoveto{\pgfqpoint{0.000000in}{0.000000in}}%
\pgfpathlineto{\pgfqpoint{0.000000in}{0.000000in}}%
\pgfpathclose%
\pgfusepath{stroke,fill}%
\end{pgfscope}%
\begin{pgfscope}%
\pgfpathrectangle{\pgfqpoint{0.647939in}{0.492442in}}{\pgfqpoint{4.273799in}{2.331163in}}%
\pgfusepath{clip}%
\pgfsetroundcap%
\pgfsetroundjoin%
\pgfsetlinewidth{0.301125pt}%
\definecolor{currentstroke}{rgb}{0.500000,0.500000,0.500000}%
\pgfsetstrokecolor{currentstroke}%
\pgfsetstrokeopacity{0.300000}%
\pgfsetdash{}{0pt}%
\pgfpathmoveto{\pgfqpoint{3.676605in}{2.334259in}}%
\pgfusepath{stroke}%
\end{pgfscope}%
\begin{pgfscope}%
\pgfpathrectangle{\pgfqpoint{0.647939in}{0.492442in}}{\pgfqpoint{4.273799in}{2.331163in}}%
\pgfusepath{clip}%
\pgfsetroundcap%
\pgfsetroundjoin%
\definecolor{currentfill}{rgb}{0.500000,0.500000,0.500000}%
\pgfsetfillcolor{currentfill}%
\pgfsetfillopacity{0.300000}%
\pgfsetlinewidth{0.301125pt}%
\definecolor{currentstroke}{rgb}{0.500000,0.500000,0.500000}%
\pgfsetstrokecolor{currentstroke}%
\pgfsetstrokeopacity{0.300000}%
\pgfsetdash{}{0pt}%
\pgfpathmoveto{\pgfqpoint{0.000000in}{0.000000in}}%
\pgfpathlineto{\pgfqpoint{0.000000in}{0.000000in}}%
\pgfpathclose%
\pgfusepath{stroke,fill}%
\end{pgfscope}%
\begin{pgfscope}%
\pgfpathrectangle{\pgfqpoint{0.647939in}{0.492442in}}{\pgfqpoint{4.273799in}{2.331163in}}%
\pgfusepath{clip}%
\pgfsetroundcap%
\pgfsetroundjoin%
\pgfsetlinewidth{0.301125pt}%
\definecolor{currentstroke}{rgb}{0.500000,0.500000,0.500000}%
\pgfsetstrokecolor{currentstroke}%
\pgfsetstrokeopacity{0.300000}%
\pgfsetdash{}{0pt}%
\pgfpathmoveto{\pgfqpoint{3.565948in}{2.241556in}}%
\pgfusepath{stroke}%
\end{pgfscope}%
\begin{pgfscope}%
\pgfpathrectangle{\pgfqpoint{0.647939in}{0.492442in}}{\pgfqpoint{4.273799in}{2.331163in}}%
\pgfusepath{clip}%
\pgfsetroundcap%
\pgfsetroundjoin%
\definecolor{currentfill}{rgb}{0.500000,0.500000,0.500000}%
\pgfsetfillcolor{currentfill}%
\pgfsetfillopacity{0.300000}%
\pgfsetlinewidth{0.301125pt}%
\definecolor{currentstroke}{rgb}{0.500000,0.500000,0.500000}%
\pgfsetstrokecolor{currentstroke}%
\pgfsetstrokeopacity{0.300000}%
\pgfsetdash{}{0pt}%
\pgfpathmoveto{\pgfqpoint{0.000000in}{0.000000in}}%
\pgfpathlineto{\pgfqpoint{0.000000in}{0.000000in}}%
\pgfpathclose%
\pgfusepath{stroke,fill}%
\end{pgfscope}%
\begin{pgfscope}%
\pgfpathrectangle{\pgfqpoint{0.647939in}{0.492442in}}{\pgfqpoint{4.273799in}{2.331163in}}%
\pgfusepath{clip}%
\pgfsetroundcap%
\pgfsetroundjoin%
\pgfsetlinewidth{0.301125pt}%
\definecolor{currentstroke}{rgb}{0.500000,0.500000,0.500000}%
\pgfsetstrokecolor{currentstroke}%
\pgfsetstrokeopacity{0.300000}%
\pgfsetdash{}{0pt}%
\pgfpathmoveto{\pgfqpoint{3.395857in}{2.331890in}}%
\pgfusepath{stroke}%
\end{pgfscope}%
\begin{pgfscope}%
\pgfpathrectangle{\pgfqpoint{0.647939in}{0.492442in}}{\pgfqpoint{4.273799in}{2.331163in}}%
\pgfusepath{clip}%
\pgfsetroundcap%
\pgfsetroundjoin%
\definecolor{currentfill}{rgb}{0.500000,0.500000,0.500000}%
\pgfsetfillcolor{currentfill}%
\pgfsetfillopacity{0.300000}%
\pgfsetlinewidth{0.301125pt}%
\definecolor{currentstroke}{rgb}{0.500000,0.500000,0.500000}%
\pgfsetstrokecolor{currentstroke}%
\pgfsetstrokeopacity{0.300000}%
\pgfsetdash{}{0pt}%
\pgfpathmoveto{\pgfqpoint{0.000000in}{0.000000in}}%
\pgfpathlineto{\pgfqpoint{0.000000in}{0.000000in}}%
\pgfpathclose%
\pgfusepath{stroke,fill}%
\end{pgfscope}%
\begin{pgfscope}%
\pgfpathrectangle{\pgfqpoint{0.647939in}{0.492442in}}{\pgfqpoint{4.273799in}{2.331163in}}%
\pgfusepath{clip}%
\pgfsetroundcap%
\pgfsetroundjoin%
\pgfsetlinewidth{0.301125pt}%
\definecolor{currentstroke}{rgb}{0.500000,0.500000,0.500000}%
\pgfsetstrokecolor{currentstroke}%
\pgfsetstrokeopacity{0.300000}%
\pgfsetdash{}{0pt}%
\pgfpathmoveto{\pgfqpoint{2.979450in}{2.682599in}}%
\pgfusepath{stroke}%
\end{pgfscope}%
\begin{pgfscope}%
\pgfpathrectangle{\pgfqpoint{0.647939in}{0.492442in}}{\pgfqpoint{4.273799in}{2.331163in}}%
\pgfusepath{clip}%
\pgfsetroundcap%
\pgfsetroundjoin%
\definecolor{currentfill}{rgb}{0.500000,0.500000,0.500000}%
\pgfsetfillcolor{currentfill}%
\pgfsetfillopacity{0.300000}%
\pgfsetlinewidth{0.301125pt}%
\definecolor{currentstroke}{rgb}{0.500000,0.500000,0.500000}%
\pgfsetstrokecolor{currentstroke}%
\pgfsetstrokeopacity{0.300000}%
\pgfsetdash{}{0pt}%
\pgfpathmoveto{\pgfqpoint{0.000000in}{0.000000in}}%
\pgfpathlineto{\pgfqpoint{0.000000in}{0.000000in}}%
\pgfpathclose%
\pgfusepath{stroke,fill}%
\end{pgfscope}%
\begin{pgfscope}%
\pgfpathrectangle{\pgfqpoint{0.647939in}{0.492442in}}{\pgfqpoint{4.273799in}{2.331163in}}%
\pgfusepath{clip}%
\pgfsetroundcap%
\pgfsetroundjoin%
\pgfsetlinewidth{0.301125pt}%
\definecolor{currentstroke}{rgb}{0.500000,0.500000,0.500000}%
\pgfsetstrokecolor{currentstroke}%
\pgfsetstrokeopacity{0.300000}%
\pgfsetdash{}{0pt}%
\pgfpathmoveto{\pgfqpoint{3.181960in}{2.395321in}}%
\pgfusepath{stroke}%
\end{pgfscope}%
\begin{pgfscope}%
\pgfpathrectangle{\pgfqpoint{0.647939in}{0.492442in}}{\pgfqpoint{4.273799in}{2.331163in}}%
\pgfusepath{clip}%
\pgfsetroundcap%
\pgfsetroundjoin%
\definecolor{currentfill}{rgb}{0.500000,0.500000,0.500000}%
\pgfsetfillcolor{currentfill}%
\pgfsetfillopacity{0.300000}%
\pgfsetlinewidth{0.301125pt}%
\definecolor{currentstroke}{rgb}{0.500000,0.500000,0.500000}%
\pgfsetstrokecolor{currentstroke}%
\pgfsetstrokeopacity{0.300000}%
\pgfsetdash{}{0pt}%
\pgfpathmoveto{\pgfqpoint{0.000000in}{0.000000in}}%
\pgfpathlineto{\pgfqpoint{0.000000in}{0.000000in}}%
\pgfpathclose%
\pgfusepath{stroke,fill}%
\end{pgfscope}%
\begin{pgfscope}%
\pgfpathrectangle{\pgfqpoint{0.647939in}{0.492442in}}{\pgfqpoint{4.273799in}{2.331163in}}%
\pgfusepath{clip}%
\pgfsetroundcap%
\pgfsetroundjoin%
\pgfsetlinewidth{0.301125pt}%
\definecolor{currentstroke}{rgb}{0.500000,0.500000,0.500000}%
\pgfsetstrokecolor{currentstroke}%
\pgfsetstrokeopacity{0.300000}%
\pgfsetdash{}{0pt}%
\pgfpathmoveto{\pgfqpoint{2.661965in}{2.729262in}}%
\pgfusepath{stroke}%
\end{pgfscope}%
\begin{pgfscope}%
\pgfpathrectangle{\pgfqpoint{0.647939in}{0.492442in}}{\pgfqpoint{4.273799in}{2.331163in}}%
\pgfusepath{clip}%
\pgfsetroundcap%
\pgfsetroundjoin%
\definecolor{currentfill}{rgb}{0.500000,0.500000,0.500000}%
\pgfsetfillcolor{currentfill}%
\pgfsetfillopacity{0.300000}%
\pgfsetlinewidth{0.301125pt}%
\definecolor{currentstroke}{rgb}{0.500000,0.500000,0.500000}%
\pgfsetstrokecolor{currentstroke}%
\pgfsetstrokeopacity{0.300000}%
\pgfsetdash{}{0pt}%
\pgfpathmoveto{\pgfqpoint{0.000000in}{0.000000in}}%
\pgfpathlineto{\pgfqpoint{0.000000in}{0.000000in}}%
\pgfpathclose%
\pgfusepath{stroke,fill}%
\end{pgfscope}%
\begin{pgfscope}%
\pgfpathrectangle{\pgfqpoint{0.647939in}{0.492442in}}{\pgfqpoint{4.273799in}{2.331163in}}%
\pgfusepath{clip}%
\pgfsetroundcap%
\pgfsetroundjoin%
\pgfsetlinewidth{0.301125pt}%
\definecolor{currentstroke}{rgb}{0.500000,0.500000,0.500000}%
\pgfsetstrokecolor{currentstroke}%
\pgfsetstrokeopacity{0.300000}%
\pgfsetdash{}{0pt}%
\pgfpathmoveto{\pgfqpoint{2.307705in}{2.778329in}}%
\pgfusepath{stroke}%
\end{pgfscope}%
\begin{pgfscope}%
\pgfpathrectangle{\pgfqpoint{0.647939in}{0.492442in}}{\pgfqpoint{4.273799in}{2.331163in}}%
\pgfusepath{clip}%
\pgfsetroundcap%
\pgfsetroundjoin%
\definecolor{currentfill}{rgb}{0.500000,0.500000,0.500000}%
\pgfsetfillcolor{currentfill}%
\pgfsetfillopacity{0.300000}%
\pgfsetlinewidth{0.301125pt}%
\definecolor{currentstroke}{rgb}{0.500000,0.500000,0.500000}%
\pgfsetstrokecolor{currentstroke}%
\pgfsetstrokeopacity{0.300000}%
\pgfsetdash{}{0pt}%
\pgfpathmoveto{\pgfqpoint{0.000000in}{0.000000in}}%
\pgfpathlineto{\pgfqpoint{0.000000in}{0.000000in}}%
\pgfpathclose%
\pgfusepath{stroke,fill}%
\end{pgfscope}%
\begin{pgfscope}%
\pgfpathrectangle{\pgfqpoint{0.647939in}{0.492442in}}{\pgfqpoint{4.273799in}{2.331163in}}%
\pgfusepath{clip}%
\pgfsetroundcap%
\pgfsetroundjoin%
\pgfsetlinewidth{0.301125pt}%
\definecolor{currentstroke}{rgb}{0.500000,0.500000,0.500000}%
\pgfsetstrokecolor{currentstroke}%
\pgfsetstrokeopacity{0.300000}%
\pgfsetdash{}{0pt}%
\pgfpathmoveto{\pgfqpoint{2.551937in}{2.681984in}}%
\pgfusepath{stroke}%
\end{pgfscope}%
\begin{pgfscope}%
\pgfpathrectangle{\pgfqpoint{0.647939in}{0.492442in}}{\pgfqpoint{4.273799in}{2.331163in}}%
\pgfusepath{clip}%
\pgfsetroundcap%
\pgfsetroundjoin%
\definecolor{currentfill}{rgb}{0.500000,0.500000,0.500000}%
\pgfsetfillcolor{currentfill}%
\pgfsetfillopacity{0.300000}%
\pgfsetlinewidth{0.301125pt}%
\definecolor{currentstroke}{rgb}{0.500000,0.500000,0.500000}%
\pgfsetstrokecolor{currentstroke}%
\pgfsetstrokeopacity{0.300000}%
\pgfsetdash{}{0pt}%
\pgfpathmoveto{\pgfqpoint{0.000000in}{0.000000in}}%
\pgfpathlineto{\pgfqpoint{0.000000in}{0.000000in}}%
\pgfpathclose%
\pgfusepath{stroke,fill}%
\end{pgfscope}%
\begin{pgfscope}%
\pgfpathrectangle{\pgfqpoint{0.647939in}{0.492442in}}{\pgfqpoint{4.273799in}{2.331163in}}%
\pgfusepath{clip}%
\pgfsetroundcap%
\pgfsetroundjoin%
\pgfsetlinewidth{0.301125pt}%
\definecolor{currentstroke}{rgb}{0.500000,0.500000,0.500000}%
\pgfsetstrokecolor{currentstroke}%
\pgfsetstrokeopacity{0.300000}%
\pgfsetdash{}{0pt}%
\pgfpathmoveto{\pgfqpoint{2.046922in}{2.700791in}}%
\pgfusepath{stroke}%
\end{pgfscope}%
\begin{pgfscope}%
\pgfpathrectangle{\pgfqpoint{0.647939in}{0.492442in}}{\pgfqpoint{4.273799in}{2.331163in}}%
\pgfusepath{clip}%
\pgfsetroundcap%
\pgfsetroundjoin%
\definecolor{currentfill}{rgb}{0.500000,0.500000,0.500000}%
\pgfsetfillcolor{currentfill}%
\pgfsetfillopacity{0.300000}%
\pgfsetlinewidth{0.301125pt}%
\definecolor{currentstroke}{rgb}{0.500000,0.500000,0.500000}%
\pgfsetstrokecolor{currentstroke}%
\pgfsetstrokeopacity{0.300000}%
\pgfsetdash{}{0pt}%
\pgfpathmoveto{\pgfqpoint{0.000000in}{0.000000in}}%
\pgfpathlineto{\pgfqpoint{0.000000in}{0.000000in}}%
\pgfpathclose%
\pgfusepath{stroke,fill}%
\end{pgfscope}%
\begin{pgfscope}%
\pgfpathrectangle{\pgfqpoint{0.647939in}{0.492442in}}{\pgfqpoint{4.273799in}{2.331163in}}%
\pgfusepath{clip}%
\pgfsetroundcap%
\pgfsetroundjoin%
\pgfsetlinewidth{0.301125pt}%
\definecolor{currentstroke}{rgb}{0.500000,0.500000,0.500000}%
\pgfsetstrokecolor{currentstroke}%
\pgfsetstrokeopacity{0.300000}%
\pgfsetdash{}{0pt}%
\pgfpathmoveto{\pgfqpoint{2.081115in}{2.643162in}}%
\pgfusepath{stroke}%
\end{pgfscope}%
\begin{pgfscope}%
\pgfpathrectangle{\pgfqpoint{0.647939in}{0.492442in}}{\pgfqpoint{4.273799in}{2.331163in}}%
\pgfusepath{clip}%
\pgfsetroundcap%
\pgfsetroundjoin%
\definecolor{currentfill}{rgb}{0.500000,0.500000,0.500000}%
\pgfsetfillcolor{currentfill}%
\pgfsetfillopacity{0.300000}%
\pgfsetlinewidth{0.301125pt}%
\definecolor{currentstroke}{rgb}{0.500000,0.500000,0.500000}%
\pgfsetstrokecolor{currentstroke}%
\pgfsetstrokeopacity{0.300000}%
\pgfsetdash{}{0pt}%
\pgfpathmoveto{\pgfqpoint{0.000000in}{0.000000in}}%
\pgfpathlineto{\pgfqpoint{0.000000in}{0.000000in}}%
\pgfpathclose%
\pgfusepath{stroke,fill}%
\end{pgfscope}%
\begin{pgfscope}%
\pgfpathrectangle{\pgfqpoint{0.647939in}{0.492442in}}{\pgfqpoint{4.273799in}{2.331163in}}%
\pgfusepath{clip}%
\pgfsetroundcap%
\pgfsetroundjoin%
\pgfsetlinewidth{0.301125pt}%
\definecolor{currentstroke}{rgb}{0.500000,0.500000,0.500000}%
\pgfsetstrokecolor{currentstroke}%
\pgfsetstrokeopacity{0.300000}%
\pgfsetdash{}{0pt}%
\pgfpathmoveto{\pgfqpoint{1.748982in}{2.637336in}}%
\pgfusepath{stroke}%
\end{pgfscope}%
\begin{pgfscope}%
\pgfpathrectangle{\pgfqpoint{0.647939in}{0.492442in}}{\pgfqpoint{4.273799in}{2.331163in}}%
\pgfusepath{clip}%
\pgfsetroundcap%
\pgfsetroundjoin%
\definecolor{currentfill}{rgb}{0.500000,0.500000,0.500000}%
\pgfsetfillcolor{currentfill}%
\pgfsetfillopacity{0.300000}%
\pgfsetlinewidth{0.301125pt}%
\definecolor{currentstroke}{rgb}{0.500000,0.500000,0.500000}%
\pgfsetstrokecolor{currentstroke}%
\pgfsetstrokeopacity{0.300000}%
\pgfsetdash{}{0pt}%
\pgfpathmoveto{\pgfqpoint{0.000000in}{0.000000in}}%
\pgfpathlineto{\pgfqpoint{0.000000in}{0.000000in}}%
\pgfpathclose%
\pgfusepath{stroke,fill}%
\end{pgfscope}%
\begin{pgfscope}%
\pgfpathrectangle{\pgfqpoint{0.647939in}{0.492442in}}{\pgfqpoint{4.273799in}{2.331163in}}%
\pgfusepath{clip}%
\pgfsetroundcap%
\pgfsetroundjoin%
\pgfsetlinewidth{0.301125pt}%
\definecolor{currentstroke}{rgb}{0.500000,0.500000,0.500000}%
\pgfsetstrokecolor{currentstroke}%
\pgfsetstrokeopacity{0.300000}%
\pgfsetdash{}{0pt}%
\pgfpathmoveto{\pgfqpoint{1.909877in}{2.490339in}}%
\pgfusepath{stroke}%
\end{pgfscope}%
\begin{pgfscope}%
\pgfpathrectangle{\pgfqpoint{0.647939in}{0.492442in}}{\pgfqpoint{4.273799in}{2.331163in}}%
\pgfusepath{clip}%
\pgfsetroundcap%
\pgfsetroundjoin%
\definecolor{currentfill}{rgb}{0.500000,0.500000,0.500000}%
\pgfsetfillcolor{currentfill}%
\pgfsetfillopacity{0.300000}%
\pgfsetlinewidth{0.301125pt}%
\definecolor{currentstroke}{rgb}{0.500000,0.500000,0.500000}%
\pgfsetstrokecolor{currentstroke}%
\pgfsetstrokeopacity{0.300000}%
\pgfsetdash{}{0pt}%
\pgfpathmoveto{\pgfqpoint{0.000000in}{0.000000in}}%
\pgfpathlineto{\pgfqpoint{0.000000in}{0.000000in}}%
\pgfpathclose%
\pgfusepath{stroke,fill}%
\end{pgfscope}%
\begin{pgfscope}%
\pgfpathrectangle{\pgfqpoint{0.647939in}{0.492442in}}{\pgfqpoint{4.273799in}{2.331163in}}%
\pgfusepath{clip}%
\pgfsetroundcap%
\pgfsetroundjoin%
\pgfsetlinewidth{0.301125pt}%
\definecolor{currentstroke}{rgb}{0.500000,0.500000,0.500000}%
\pgfsetstrokecolor{currentstroke}%
\pgfsetstrokeopacity{0.300000}%
\pgfsetdash{}{0pt}%
\pgfpathmoveto{\pgfqpoint{1.618326in}{2.420159in}}%
\pgfusepath{stroke}%
\end{pgfscope}%
\begin{pgfscope}%
\pgfpathrectangle{\pgfqpoint{0.647939in}{0.492442in}}{\pgfqpoint{4.273799in}{2.331163in}}%
\pgfusepath{clip}%
\pgfsetroundcap%
\pgfsetroundjoin%
\definecolor{currentfill}{rgb}{0.500000,0.500000,0.500000}%
\pgfsetfillcolor{currentfill}%
\pgfsetfillopacity{0.300000}%
\pgfsetlinewidth{0.301125pt}%
\definecolor{currentstroke}{rgb}{0.500000,0.500000,0.500000}%
\pgfsetstrokecolor{currentstroke}%
\pgfsetstrokeopacity{0.300000}%
\pgfsetdash{}{0pt}%
\pgfpathmoveto{\pgfqpoint{0.000000in}{0.000000in}}%
\pgfpathlineto{\pgfqpoint{0.000000in}{0.000000in}}%
\pgfpathclose%
\pgfusepath{stroke,fill}%
\end{pgfscope}%
\begin{pgfscope}%
\pgfpathrectangle{\pgfqpoint{0.647939in}{0.492442in}}{\pgfqpoint{4.273799in}{2.331163in}}%
\pgfusepath{clip}%
\pgfsetroundcap%
\pgfsetroundjoin%
\pgfsetlinewidth{0.301125pt}%
\definecolor{currentstroke}{rgb}{0.500000,0.500000,0.500000}%
\pgfsetstrokecolor{currentstroke}%
\pgfsetstrokeopacity{0.300000}%
\pgfsetdash{}{0pt}%
\pgfpathmoveto{\pgfqpoint{1.405235in}{2.447145in}}%
\pgfusepath{stroke}%
\end{pgfscope}%
\begin{pgfscope}%
\pgfpathrectangle{\pgfqpoint{0.647939in}{0.492442in}}{\pgfqpoint{4.273799in}{2.331163in}}%
\pgfusepath{clip}%
\pgfsetroundcap%
\pgfsetroundjoin%
\definecolor{currentfill}{rgb}{0.500000,0.500000,0.500000}%
\pgfsetfillcolor{currentfill}%
\pgfsetfillopacity{0.300000}%
\pgfsetlinewidth{0.301125pt}%
\definecolor{currentstroke}{rgb}{0.500000,0.500000,0.500000}%
\pgfsetstrokecolor{currentstroke}%
\pgfsetstrokeopacity{0.300000}%
\pgfsetdash{}{0pt}%
\pgfpathmoveto{\pgfqpoint{0.000000in}{0.000000in}}%
\pgfpathlineto{\pgfqpoint{0.000000in}{0.000000in}}%
\pgfpathclose%
\pgfusepath{stroke,fill}%
\end{pgfscope}%
\begin{pgfscope}%
\pgfpathrectangle{\pgfqpoint{0.647939in}{0.492442in}}{\pgfqpoint{4.273799in}{2.331163in}}%
\pgfusepath{clip}%
\pgfsetroundcap%
\pgfsetroundjoin%
\pgfsetlinewidth{0.301125pt}%
\definecolor{currentstroke}{rgb}{0.500000,0.500000,0.500000}%
\pgfsetstrokecolor{currentstroke}%
\pgfsetstrokeopacity{0.300000}%
\pgfsetdash{}{0pt}%
\pgfpathmoveto{\pgfqpoint{1.305681in}{2.314186in}}%
\pgfusepath{stroke}%
\end{pgfscope}%
\begin{pgfscope}%
\pgfpathrectangle{\pgfqpoint{0.647939in}{0.492442in}}{\pgfqpoint{4.273799in}{2.331163in}}%
\pgfusepath{clip}%
\pgfsetroundcap%
\pgfsetroundjoin%
\definecolor{currentfill}{rgb}{0.500000,0.500000,0.500000}%
\pgfsetfillcolor{currentfill}%
\pgfsetfillopacity{0.300000}%
\pgfsetlinewidth{0.301125pt}%
\definecolor{currentstroke}{rgb}{0.500000,0.500000,0.500000}%
\pgfsetstrokecolor{currentstroke}%
\pgfsetstrokeopacity{0.300000}%
\pgfsetdash{}{0pt}%
\pgfpathmoveto{\pgfqpoint{0.000000in}{0.000000in}}%
\pgfpathlineto{\pgfqpoint{0.000000in}{0.000000in}}%
\pgfpathclose%
\pgfusepath{stroke,fill}%
\end{pgfscope}%
\begin{pgfscope}%
\pgfpathrectangle{\pgfqpoint{0.647939in}{0.492442in}}{\pgfqpoint{4.273799in}{2.331163in}}%
\pgfusepath{clip}%
\pgfsetroundcap%
\pgfsetroundjoin%
\pgfsetlinewidth{0.301125pt}%
\definecolor{currentstroke}{rgb}{0.500000,0.500000,0.500000}%
\pgfsetstrokecolor{currentstroke}%
\pgfsetstrokeopacity{0.300000}%
\pgfsetdash{}{0pt}%
\pgfpathmoveto{\pgfqpoint{1.204190in}{2.000608in}}%
\pgfusepath{stroke}%
\end{pgfscope}%
\begin{pgfscope}%
\pgfpathrectangle{\pgfqpoint{0.647939in}{0.492442in}}{\pgfqpoint{4.273799in}{2.331163in}}%
\pgfusepath{clip}%
\pgfsetroundcap%
\pgfsetroundjoin%
\definecolor{currentfill}{rgb}{0.500000,0.500000,0.500000}%
\pgfsetfillcolor{currentfill}%
\pgfsetfillopacity{0.300000}%
\pgfsetlinewidth{0.301125pt}%
\definecolor{currentstroke}{rgb}{0.500000,0.500000,0.500000}%
\pgfsetstrokecolor{currentstroke}%
\pgfsetstrokeopacity{0.300000}%
\pgfsetdash{}{0pt}%
\pgfpathmoveto{\pgfqpoint{0.000000in}{0.000000in}}%
\pgfpathlineto{\pgfqpoint{0.000000in}{0.000000in}}%
\pgfpathclose%
\pgfusepath{stroke,fill}%
\end{pgfscope}%
\begin{pgfscope}%
\pgfpathrectangle{\pgfqpoint{0.647939in}{0.492442in}}{\pgfqpoint{4.273799in}{2.331163in}}%
\pgfusepath{clip}%
\pgfsetroundcap%
\pgfsetroundjoin%
\pgfsetlinewidth{0.301125pt}%
\definecolor{currentstroke}{rgb}{0.500000,0.500000,0.500000}%
\pgfsetstrokecolor{currentstroke}%
\pgfsetstrokeopacity{0.300000}%
\pgfsetdash{}{0pt}%
\pgfpathmoveto{\pgfqpoint{1.068684in}{2.050204in}}%
\pgfusepath{stroke}%
\end{pgfscope}%
\begin{pgfscope}%
\pgfpathrectangle{\pgfqpoint{0.647939in}{0.492442in}}{\pgfqpoint{4.273799in}{2.331163in}}%
\pgfusepath{clip}%
\pgfsetroundcap%
\pgfsetroundjoin%
\definecolor{currentfill}{rgb}{0.500000,0.500000,0.500000}%
\pgfsetfillcolor{currentfill}%
\pgfsetfillopacity{0.300000}%
\pgfsetlinewidth{0.301125pt}%
\definecolor{currentstroke}{rgb}{0.500000,0.500000,0.500000}%
\pgfsetstrokecolor{currentstroke}%
\pgfsetstrokeopacity{0.300000}%
\pgfsetdash{}{0pt}%
\pgfpathmoveto{\pgfqpoint{0.000000in}{0.000000in}}%
\pgfpathlineto{\pgfqpoint{0.000000in}{0.000000in}}%
\pgfpathclose%
\pgfusepath{stroke,fill}%
\end{pgfscope}%
\begin{pgfscope}%
\pgfpathrectangle{\pgfqpoint{0.647939in}{0.492442in}}{\pgfqpoint{4.273799in}{2.331163in}}%
\pgfusepath{clip}%
\pgfsetroundcap%
\pgfsetroundjoin%
\pgfsetlinewidth{0.301125pt}%
\definecolor{currentstroke}{rgb}{0.500000,0.500000,0.500000}%
\pgfsetstrokecolor{currentstroke}%
\pgfsetstrokeopacity{0.300000}%
\pgfsetdash{}{0pt}%
\pgfpathmoveto{\pgfqpoint{0.939279in}{1.790034in}}%
\pgfusepath{stroke}%
\end{pgfscope}%
\begin{pgfscope}%
\pgfpathrectangle{\pgfqpoint{0.647939in}{0.492442in}}{\pgfqpoint{4.273799in}{2.331163in}}%
\pgfusepath{clip}%
\pgfsetroundcap%
\pgfsetroundjoin%
\definecolor{currentfill}{rgb}{0.500000,0.500000,0.500000}%
\pgfsetfillcolor{currentfill}%
\pgfsetfillopacity{0.300000}%
\pgfsetlinewidth{0.301125pt}%
\definecolor{currentstroke}{rgb}{0.500000,0.500000,0.500000}%
\pgfsetstrokecolor{currentstroke}%
\pgfsetstrokeopacity{0.300000}%
\pgfsetdash{}{0pt}%
\pgfpathmoveto{\pgfqpoint{0.000000in}{0.000000in}}%
\pgfpathlineto{\pgfqpoint{0.000000in}{0.000000in}}%
\pgfpathclose%
\pgfusepath{stroke,fill}%
\end{pgfscope}%
\begin{pgfscope}%
\pgfpathrectangle{\pgfqpoint{0.647939in}{0.492442in}}{\pgfqpoint{4.273799in}{2.331163in}}%
\pgfusepath{clip}%
\pgfsetroundcap%
\pgfsetroundjoin%
\pgfsetlinewidth{0.301125pt}%
\definecolor{currentstroke}{rgb}{0.500000,0.500000,0.500000}%
\pgfsetstrokecolor{currentstroke}%
\pgfsetstrokeopacity{0.300000}%
\pgfsetdash{}{0pt}%
\pgfpathmoveto{\pgfqpoint{0.831954in}{1.996431in}}%
\pgfusepath{stroke}%
\end{pgfscope}%
\begin{pgfscope}%
\pgfpathrectangle{\pgfqpoint{0.647939in}{0.492442in}}{\pgfqpoint{4.273799in}{2.331163in}}%
\pgfusepath{clip}%
\pgfsetroundcap%
\pgfsetroundjoin%
\definecolor{currentfill}{rgb}{0.500000,0.500000,0.500000}%
\pgfsetfillcolor{currentfill}%
\pgfsetfillopacity{0.300000}%
\pgfsetlinewidth{0.301125pt}%
\definecolor{currentstroke}{rgb}{0.500000,0.500000,0.500000}%
\pgfsetstrokecolor{currentstroke}%
\pgfsetstrokeopacity{0.300000}%
\pgfsetdash{}{0pt}%
\pgfpathmoveto{\pgfqpoint{0.000000in}{0.000000in}}%
\pgfpathlineto{\pgfqpoint{0.000000in}{0.000000in}}%
\pgfpathclose%
\pgfusepath{stroke,fill}%
\end{pgfscope}%
\begin{pgfscope}%
\pgfpathrectangle{\pgfqpoint{0.647939in}{0.492442in}}{\pgfqpoint{4.273799in}{2.331163in}}%
\pgfusepath{clip}%
\pgfsetroundcap%
\pgfsetroundjoin%
\pgfsetlinewidth{0.301125pt}%
\definecolor{currentstroke}{rgb}{0.500000,0.500000,0.500000}%
\pgfsetstrokecolor{currentstroke}%
\pgfsetstrokeopacity{0.300000}%
\pgfsetdash{}{0pt}%
\pgfpathmoveto{\pgfqpoint{0.720506in}{2.047753in}}%
\pgfusepath{stroke}%
\end{pgfscope}%
\begin{pgfscope}%
\pgfpathrectangle{\pgfqpoint{0.647939in}{0.492442in}}{\pgfqpoint{4.273799in}{2.331163in}}%
\pgfusepath{clip}%
\pgfsetroundcap%
\pgfsetroundjoin%
\definecolor{currentfill}{rgb}{0.500000,0.500000,0.500000}%
\pgfsetfillcolor{currentfill}%
\pgfsetfillopacity{0.300000}%
\pgfsetlinewidth{0.301125pt}%
\definecolor{currentstroke}{rgb}{0.500000,0.500000,0.500000}%
\pgfsetstrokecolor{currentstroke}%
\pgfsetstrokeopacity{0.300000}%
\pgfsetdash{}{0pt}%
\pgfpathmoveto{\pgfqpoint{0.000000in}{0.000000in}}%
\pgfpathlineto{\pgfqpoint{0.000000in}{0.000000in}}%
\pgfpathclose%
\pgfusepath{stroke,fill}%
\end{pgfscope}%
\begin{pgfscope}%
\pgfpathrectangle{\pgfqpoint{0.647939in}{0.492442in}}{\pgfqpoint{4.273799in}{2.331163in}}%
\pgfusepath{clip}%
\pgfsetroundcap%
\pgfsetroundjoin%
\pgfsetlinewidth{0.301125pt}%
\definecolor{currentstroke}{rgb}{0.500000,0.500000,0.500000}%
\pgfsetstrokecolor{currentstroke}%
\pgfsetstrokeopacity{0.300000}%
\pgfsetdash{}{0pt}%
\pgfpathmoveto{\pgfqpoint{0.665671in}{2.037286in}}%
\pgfusepath{stroke}%
\end{pgfscope}%
\begin{pgfscope}%
\pgfpathrectangle{\pgfqpoint{0.647939in}{0.492442in}}{\pgfqpoint{4.273799in}{2.331163in}}%
\pgfusepath{clip}%
\pgfsetroundcap%
\pgfsetroundjoin%
\definecolor{currentfill}{rgb}{0.500000,0.500000,0.500000}%
\pgfsetfillcolor{currentfill}%
\pgfsetfillopacity{0.300000}%
\pgfsetlinewidth{0.301125pt}%
\definecolor{currentstroke}{rgb}{0.500000,0.500000,0.500000}%
\pgfsetstrokecolor{currentstroke}%
\pgfsetstrokeopacity{0.300000}%
\pgfsetdash{}{0pt}%
\pgfpathmoveto{\pgfqpoint{0.000000in}{0.000000in}}%
\pgfpathlineto{\pgfqpoint{0.000000in}{0.000000in}}%
\pgfpathclose%
\pgfusepath{stroke,fill}%
\end{pgfscope}%
\begin{pgfscope}%
\pgfpathrectangle{\pgfqpoint{0.647939in}{0.492442in}}{\pgfqpoint{4.273799in}{2.331163in}}%
\pgfusepath{clip}%
\pgfsetroundcap%
\pgfsetroundjoin%
\pgfsetlinewidth{0.301125pt}%
\definecolor{currentstroke}{rgb}{0.500000,0.500000,0.500000}%
\pgfsetstrokecolor{currentstroke}%
\pgfsetstrokeopacity{0.300000}%
\pgfsetdash{}{0pt}%
\pgfpathmoveto{\pgfqpoint{4.606523in}{1.760806in}}%
\pgfusepath{stroke}%
\end{pgfscope}%
\begin{pgfscope}%
\pgfpathrectangle{\pgfqpoint{0.647939in}{0.492442in}}{\pgfqpoint{4.273799in}{2.331163in}}%
\pgfusepath{clip}%
\pgfsetroundcap%
\pgfsetroundjoin%
\definecolor{currentfill}{rgb}{0.500000,0.500000,0.500000}%
\pgfsetfillcolor{currentfill}%
\pgfsetfillopacity{0.300000}%
\pgfsetlinewidth{0.301125pt}%
\definecolor{currentstroke}{rgb}{0.500000,0.500000,0.500000}%
\pgfsetstrokecolor{currentstroke}%
\pgfsetstrokeopacity{0.300000}%
\pgfsetdash{}{0pt}%
\pgfpathmoveto{\pgfqpoint{0.000000in}{0.000000in}}%
\pgfpathlineto{\pgfqpoint{0.000000in}{0.000000in}}%
\pgfpathclose%
\pgfusepath{stroke,fill}%
\end{pgfscope}%
\begin{pgfscope}%
\pgfpathrectangle{\pgfqpoint{0.647939in}{0.492442in}}{\pgfqpoint{4.273799in}{2.331163in}}%
\pgfusepath{clip}%
\pgfsetroundcap%
\pgfsetroundjoin%
\pgfsetlinewidth{0.301125pt}%
\definecolor{currentstroke}{rgb}{0.500000,0.500000,0.500000}%
\pgfsetstrokecolor{currentstroke}%
\pgfsetstrokeopacity{0.300000}%
\pgfsetdash{}{0pt}%
\pgfpathmoveto{\pgfqpoint{4.291352in}{1.543131in}}%
\pgfusepath{stroke}%
\end{pgfscope}%
\begin{pgfscope}%
\pgfpathrectangle{\pgfqpoint{0.647939in}{0.492442in}}{\pgfqpoint{4.273799in}{2.331163in}}%
\pgfusepath{clip}%
\pgfsetroundcap%
\pgfsetroundjoin%
\definecolor{currentfill}{rgb}{0.500000,0.500000,0.500000}%
\pgfsetfillcolor{currentfill}%
\pgfsetfillopacity{0.300000}%
\pgfsetlinewidth{0.301125pt}%
\definecolor{currentstroke}{rgb}{0.500000,0.500000,0.500000}%
\pgfsetstrokecolor{currentstroke}%
\pgfsetstrokeopacity{0.300000}%
\pgfsetdash{}{0pt}%
\pgfpathmoveto{\pgfqpoint{0.000000in}{0.000000in}}%
\pgfpathlineto{\pgfqpoint{0.000000in}{0.000000in}}%
\pgfpathclose%
\pgfusepath{stroke,fill}%
\end{pgfscope}%
\begin{pgfscope}%
\pgfpathrectangle{\pgfqpoint{0.647939in}{0.492442in}}{\pgfqpoint{4.273799in}{2.331163in}}%
\pgfusepath{clip}%
\pgfsetroundcap%
\pgfsetroundjoin%
\pgfsetlinewidth{0.301125pt}%
\definecolor{currentstroke}{rgb}{0.500000,0.500000,0.500000}%
\pgfsetstrokecolor{currentstroke}%
\pgfsetstrokeopacity{0.300000}%
\pgfsetdash{}{0pt}%
\pgfpathmoveto{\pgfqpoint{1.269635in}{0.825361in}}%
\pgfusepath{stroke}%
\end{pgfscope}%
\begin{pgfscope}%
\pgfpathrectangle{\pgfqpoint{0.647939in}{0.492442in}}{\pgfqpoint{4.273799in}{2.331163in}}%
\pgfusepath{clip}%
\pgfsetroundcap%
\pgfsetroundjoin%
\definecolor{currentfill}{rgb}{0.500000,0.500000,0.500000}%
\pgfsetfillcolor{currentfill}%
\pgfsetfillopacity{0.300000}%
\pgfsetlinewidth{0.301125pt}%
\definecolor{currentstroke}{rgb}{0.500000,0.500000,0.500000}%
\pgfsetstrokecolor{currentstroke}%
\pgfsetstrokeopacity{0.300000}%
\pgfsetdash{}{0pt}%
\pgfpathmoveto{\pgfqpoint{0.000000in}{0.000000in}}%
\pgfpathlineto{\pgfqpoint{0.000000in}{0.000000in}}%
\pgfpathclose%
\pgfusepath{stroke,fill}%
\end{pgfscope}%
\begin{pgfscope}%
\pgfpathrectangle{\pgfqpoint{0.647939in}{0.492442in}}{\pgfqpoint{4.273799in}{2.331163in}}%
\pgfusepath{clip}%
\pgfsetroundcap%
\pgfsetroundjoin%
\pgfsetlinewidth{0.301125pt}%
\definecolor{currentstroke}{rgb}{0.500000,0.500000,0.500000}%
\pgfsetstrokecolor{currentstroke}%
\pgfsetstrokeopacity{0.300000}%
\pgfsetdash{}{0pt}%
\pgfpathmoveto{\pgfqpoint{4.359771in}{1.913889in}}%
\pgfusepath{stroke}%
\end{pgfscope}%
\begin{pgfscope}%
\pgfpathrectangle{\pgfqpoint{0.647939in}{0.492442in}}{\pgfqpoint{4.273799in}{2.331163in}}%
\pgfusepath{clip}%
\pgfsetroundcap%
\pgfsetroundjoin%
\definecolor{currentfill}{rgb}{0.500000,0.500000,0.500000}%
\pgfsetfillcolor{currentfill}%
\pgfsetfillopacity{0.300000}%
\pgfsetlinewidth{0.301125pt}%
\definecolor{currentstroke}{rgb}{0.500000,0.500000,0.500000}%
\pgfsetstrokecolor{currentstroke}%
\pgfsetstrokeopacity{0.300000}%
\pgfsetdash{}{0pt}%
\pgfpathmoveto{\pgfqpoint{0.000000in}{0.000000in}}%
\pgfpathlineto{\pgfqpoint{0.000000in}{0.000000in}}%
\pgfpathclose%
\pgfusepath{stroke,fill}%
\end{pgfscope}%
\begin{pgfscope}%
\pgfpathrectangle{\pgfqpoint{0.647939in}{0.492442in}}{\pgfqpoint{4.273799in}{2.331163in}}%
\pgfusepath{clip}%
\pgfsetroundcap%
\pgfsetroundjoin%
\pgfsetlinewidth{0.301125pt}%
\definecolor{currentstroke}{rgb}{0.500000,0.500000,0.500000}%
\pgfsetstrokecolor{currentstroke}%
\pgfsetstrokeopacity{0.300000}%
\pgfsetdash{}{0pt}%
\pgfpathmoveto{\pgfqpoint{1.580663in}{2.346262in}}%
\pgfusepath{stroke}%
\end{pgfscope}%
\begin{pgfscope}%
\pgfpathrectangle{\pgfqpoint{0.647939in}{0.492442in}}{\pgfqpoint{4.273799in}{2.331163in}}%
\pgfusepath{clip}%
\pgfsetroundcap%
\pgfsetroundjoin%
\definecolor{currentfill}{rgb}{0.500000,0.500000,0.500000}%
\pgfsetfillcolor{currentfill}%
\pgfsetfillopacity{0.300000}%
\pgfsetlinewidth{0.301125pt}%
\definecolor{currentstroke}{rgb}{0.500000,0.500000,0.500000}%
\pgfsetstrokecolor{currentstroke}%
\pgfsetstrokeopacity{0.300000}%
\pgfsetdash{}{0pt}%
\pgfpathmoveto{\pgfqpoint{0.000000in}{0.000000in}}%
\pgfpathlineto{\pgfqpoint{0.000000in}{0.000000in}}%
\pgfpathclose%
\pgfusepath{stroke,fill}%
\end{pgfscope}%
\begin{pgfscope}%
\pgfpathrectangle{\pgfqpoint{0.647939in}{0.492442in}}{\pgfqpoint{4.273799in}{2.331163in}}%
\pgfusepath{clip}%
\pgfsetroundcap%
\pgfsetroundjoin%
\pgfsetlinewidth{0.301125pt}%
\definecolor{currentstroke}{rgb}{0.500000,0.500000,0.500000}%
\pgfsetstrokecolor{currentstroke}%
\pgfsetstrokeopacity{0.300000}%
\pgfsetdash{}{0pt}%
\pgfpathmoveto{\pgfqpoint{1.145492in}{2.032577in}}%
\pgfusepath{stroke}%
\end{pgfscope}%
\begin{pgfscope}%
\pgfpathrectangle{\pgfqpoint{0.647939in}{0.492442in}}{\pgfqpoint{4.273799in}{2.331163in}}%
\pgfusepath{clip}%
\pgfsetroundcap%
\pgfsetroundjoin%
\definecolor{currentfill}{rgb}{0.500000,0.500000,0.500000}%
\pgfsetfillcolor{currentfill}%
\pgfsetfillopacity{0.300000}%
\pgfsetlinewidth{0.301125pt}%
\definecolor{currentstroke}{rgb}{0.500000,0.500000,0.500000}%
\pgfsetstrokecolor{currentstroke}%
\pgfsetstrokeopacity{0.300000}%
\pgfsetdash{}{0pt}%
\pgfpathmoveto{\pgfqpoint{0.000000in}{0.000000in}}%
\pgfpathlineto{\pgfqpoint{0.000000in}{0.000000in}}%
\pgfpathclose%
\pgfusepath{stroke,fill}%
\end{pgfscope}%
\begin{pgfscope}%
\pgfpathrectangle{\pgfqpoint{0.647939in}{0.492442in}}{\pgfqpoint{4.273799in}{2.331163in}}%
\pgfusepath{clip}%
\pgfsetroundcap%
\pgfsetroundjoin%
\pgfsetlinewidth{0.301125pt}%
\definecolor{currentstroke}{rgb}{0.500000,0.500000,0.500000}%
\pgfsetstrokecolor{currentstroke}%
\pgfsetstrokeopacity{0.300000}%
\pgfsetdash{}{0pt}%
\pgfpathmoveto{\pgfqpoint{2.101853in}{1.502947in}}%
\pgfusepath{stroke}%
\end{pgfscope}%
\begin{pgfscope}%
\pgfpathrectangle{\pgfqpoint{0.647939in}{0.492442in}}{\pgfqpoint{4.273799in}{2.331163in}}%
\pgfusepath{clip}%
\pgfsetroundcap%
\pgfsetroundjoin%
\definecolor{currentfill}{rgb}{0.500000,0.500000,0.500000}%
\pgfsetfillcolor{currentfill}%
\pgfsetfillopacity{0.300000}%
\pgfsetlinewidth{0.301125pt}%
\definecolor{currentstroke}{rgb}{0.500000,0.500000,0.500000}%
\pgfsetstrokecolor{currentstroke}%
\pgfsetstrokeopacity{0.300000}%
\pgfsetdash{}{0pt}%
\pgfpathmoveto{\pgfqpoint{0.000000in}{0.000000in}}%
\pgfpathlineto{\pgfqpoint{0.000000in}{0.000000in}}%
\pgfpathclose%
\pgfusepath{stroke,fill}%
\end{pgfscope}%
\begin{pgfscope}%
\pgfpathrectangle{\pgfqpoint{0.647939in}{0.492442in}}{\pgfqpoint{4.273799in}{2.331163in}}%
\pgfusepath{clip}%
\pgfsetroundcap%
\pgfsetroundjoin%
\pgfsetlinewidth{0.301125pt}%
\definecolor{currentstroke}{rgb}{0.500000,0.500000,0.500000}%
\pgfsetstrokecolor{currentstroke}%
\pgfsetstrokeopacity{0.300000}%
\pgfsetdash{}{0pt}%
\pgfpathmoveto{\pgfqpoint{4.068529in}{1.414600in}}%
\pgfusepath{stroke}%
\end{pgfscope}%
\begin{pgfscope}%
\pgfpathrectangle{\pgfqpoint{0.647939in}{0.492442in}}{\pgfqpoint{4.273799in}{2.331163in}}%
\pgfusepath{clip}%
\pgfsetroundcap%
\pgfsetroundjoin%
\definecolor{currentfill}{rgb}{0.500000,0.500000,0.500000}%
\pgfsetfillcolor{currentfill}%
\pgfsetfillopacity{0.300000}%
\pgfsetlinewidth{0.301125pt}%
\definecolor{currentstroke}{rgb}{0.500000,0.500000,0.500000}%
\pgfsetstrokecolor{currentstroke}%
\pgfsetstrokeopacity{0.300000}%
\pgfsetdash{}{0pt}%
\pgfpathmoveto{\pgfqpoint{0.000000in}{0.000000in}}%
\pgfpathlineto{\pgfqpoint{0.000000in}{0.000000in}}%
\pgfpathclose%
\pgfusepath{stroke,fill}%
\end{pgfscope}%
\begin{pgfscope}%
\pgfpathrectangle{\pgfqpoint{0.647939in}{0.492442in}}{\pgfqpoint{4.273799in}{2.331163in}}%
\pgfusepath{clip}%
\pgfsetroundcap%
\pgfsetroundjoin%
\pgfsetlinewidth{0.301125pt}%
\definecolor{currentstroke}{rgb}{0.500000,0.500000,0.500000}%
\pgfsetstrokecolor{currentstroke}%
\pgfsetstrokeopacity{0.300000}%
\pgfsetdash{}{0pt}%
\pgfpathmoveto{\pgfqpoint{4.239953in}{1.717268in}}%
\pgfusepath{stroke}%
\end{pgfscope}%
\begin{pgfscope}%
\pgfpathrectangle{\pgfqpoint{0.647939in}{0.492442in}}{\pgfqpoint{4.273799in}{2.331163in}}%
\pgfusepath{clip}%
\pgfsetroundcap%
\pgfsetroundjoin%
\definecolor{currentfill}{rgb}{0.500000,0.500000,0.500000}%
\pgfsetfillcolor{currentfill}%
\pgfsetfillopacity{0.300000}%
\pgfsetlinewidth{0.301125pt}%
\definecolor{currentstroke}{rgb}{0.500000,0.500000,0.500000}%
\pgfsetstrokecolor{currentstroke}%
\pgfsetstrokeopacity{0.300000}%
\pgfsetdash{}{0pt}%
\pgfpathmoveto{\pgfqpoint{0.000000in}{0.000000in}}%
\pgfpathlineto{\pgfqpoint{0.000000in}{0.000000in}}%
\pgfpathclose%
\pgfusepath{stroke,fill}%
\end{pgfscope}%
\begin{pgfscope}%
\pgfpathrectangle{\pgfqpoint{0.647939in}{0.492442in}}{\pgfqpoint{4.273799in}{2.331163in}}%
\pgfusepath{clip}%
\pgfsetroundcap%
\pgfsetroundjoin%
\pgfsetlinewidth{0.301125pt}%
\definecolor{currentstroke}{rgb}{0.500000,0.500000,0.500000}%
\pgfsetstrokecolor{currentstroke}%
\pgfsetstrokeopacity{0.300000}%
\pgfsetdash{}{0pt}%
\pgfpathmoveto{\pgfqpoint{1.761420in}{2.433193in}}%
\pgfusepath{stroke}%
\end{pgfscope}%
\begin{pgfscope}%
\pgfpathrectangle{\pgfqpoint{0.647939in}{0.492442in}}{\pgfqpoint{4.273799in}{2.331163in}}%
\pgfusepath{clip}%
\pgfsetroundcap%
\pgfsetroundjoin%
\definecolor{currentfill}{rgb}{0.500000,0.500000,0.500000}%
\pgfsetfillcolor{currentfill}%
\pgfsetfillopacity{0.300000}%
\pgfsetlinewidth{0.301125pt}%
\definecolor{currentstroke}{rgb}{0.500000,0.500000,0.500000}%
\pgfsetstrokecolor{currentstroke}%
\pgfsetstrokeopacity{0.300000}%
\pgfsetdash{}{0pt}%
\pgfpathmoveto{\pgfqpoint{0.000000in}{0.000000in}}%
\pgfpathlineto{\pgfqpoint{0.000000in}{0.000000in}}%
\pgfpathclose%
\pgfusepath{stroke,fill}%
\end{pgfscope}%
\begin{pgfscope}%
\pgfpathrectangle{\pgfqpoint{0.647939in}{0.492442in}}{\pgfqpoint{4.273799in}{2.331163in}}%
\pgfusepath{clip}%
\pgfsetroundcap%
\pgfsetroundjoin%
\pgfsetlinewidth{0.301125pt}%
\definecolor{currentstroke}{rgb}{0.500000,0.500000,0.500000}%
\pgfsetstrokecolor{currentstroke}%
\pgfsetstrokeopacity{0.300000}%
\pgfsetdash{}{0pt}%
\pgfpathmoveto{\pgfqpoint{1.332356in}{0.956187in}}%
\pgfusepath{stroke}%
\end{pgfscope}%
\begin{pgfscope}%
\pgfpathrectangle{\pgfqpoint{0.647939in}{0.492442in}}{\pgfqpoint{4.273799in}{2.331163in}}%
\pgfusepath{clip}%
\pgfsetroundcap%
\pgfsetroundjoin%
\definecolor{currentfill}{rgb}{0.500000,0.500000,0.500000}%
\pgfsetfillcolor{currentfill}%
\pgfsetfillopacity{0.300000}%
\pgfsetlinewidth{0.301125pt}%
\definecolor{currentstroke}{rgb}{0.500000,0.500000,0.500000}%
\pgfsetstrokecolor{currentstroke}%
\pgfsetstrokeopacity{0.300000}%
\pgfsetdash{}{0pt}%
\pgfpathmoveto{\pgfqpoint{0.000000in}{0.000000in}}%
\pgfpathlineto{\pgfqpoint{0.000000in}{0.000000in}}%
\pgfpathclose%
\pgfusepath{stroke,fill}%
\end{pgfscope}%
\begin{pgfscope}%
\pgfpathrectangle{\pgfqpoint{0.647939in}{0.492442in}}{\pgfqpoint{4.273799in}{2.331163in}}%
\pgfusepath{clip}%
\pgfsetroundcap%
\pgfsetroundjoin%
\pgfsetlinewidth{0.301125pt}%
\definecolor{currentstroke}{rgb}{0.500000,0.500000,0.500000}%
\pgfsetstrokecolor{currentstroke}%
\pgfsetstrokeopacity{0.300000}%
\pgfsetdash{}{0pt}%
\pgfpathmoveto{\pgfqpoint{1.493614in}{1.352214in}}%
\pgfusepath{stroke}%
\end{pgfscope}%
\begin{pgfscope}%
\pgfpathrectangle{\pgfqpoint{0.647939in}{0.492442in}}{\pgfqpoint{4.273799in}{2.331163in}}%
\pgfusepath{clip}%
\pgfsetroundcap%
\pgfsetroundjoin%
\definecolor{currentfill}{rgb}{0.500000,0.500000,0.500000}%
\pgfsetfillcolor{currentfill}%
\pgfsetfillopacity{0.300000}%
\pgfsetlinewidth{0.301125pt}%
\definecolor{currentstroke}{rgb}{0.500000,0.500000,0.500000}%
\pgfsetstrokecolor{currentstroke}%
\pgfsetstrokeopacity{0.300000}%
\pgfsetdash{}{0pt}%
\pgfpathmoveto{\pgfqpoint{0.000000in}{0.000000in}}%
\pgfpathlineto{\pgfqpoint{0.000000in}{0.000000in}}%
\pgfpathclose%
\pgfusepath{stroke,fill}%
\end{pgfscope}%
\begin{pgfscope}%
\pgfpathrectangle{\pgfqpoint{0.647939in}{0.492442in}}{\pgfqpoint{4.273799in}{2.331163in}}%
\pgfusepath{clip}%
\pgfsetroundcap%
\pgfsetroundjoin%
\pgfsetlinewidth{0.301125pt}%
\definecolor{currentstroke}{rgb}{0.500000,0.500000,0.500000}%
\pgfsetstrokecolor{currentstroke}%
\pgfsetstrokeopacity{0.300000}%
\pgfsetdash{}{0pt}%
\pgfpathmoveto{\pgfqpoint{3.935192in}{1.310937in}}%
\pgfusepath{stroke}%
\end{pgfscope}%
\begin{pgfscope}%
\pgfpathrectangle{\pgfqpoint{0.647939in}{0.492442in}}{\pgfqpoint{4.273799in}{2.331163in}}%
\pgfusepath{clip}%
\pgfsetroundcap%
\pgfsetroundjoin%
\definecolor{currentfill}{rgb}{0.500000,0.500000,0.500000}%
\pgfsetfillcolor{currentfill}%
\pgfsetfillopacity{0.300000}%
\pgfsetlinewidth{0.301125pt}%
\definecolor{currentstroke}{rgb}{0.500000,0.500000,0.500000}%
\pgfsetstrokecolor{currentstroke}%
\pgfsetstrokeopacity{0.300000}%
\pgfsetdash{}{0pt}%
\pgfpathmoveto{\pgfqpoint{0.000000in}{0.000000in}}%
\pgfpathlineto{\pgfqpoint{0.000000in}{0.000000in}}%
\pgfpathclose%
\pgfusepath{stroke,fill}%
\end{pgfscope}%
\begin{pgfscope}%
\pgfpathrectangle{\pgfqpoint{0.647939in}{0.492442in}}{\pgfqpoint{4.273799in}{2.331163in}}%
\pgfusepath{clip}%
\pgfsetroundcap%
\pgfsetroundjoin%
\pgfsetlinewidth{0.301125pt}%
\definecolor{currentstroke}{rgb}{0.500000,0.500000,0.500000}%
\pgfsetstrokecolor{currentstroke}%
\pgfsetstrokeopacity{0.300000}%
\pgfsetdash{}{0pt}%
\pgfpathmoveto{\pgfqpoint{3.894228in}{1.542662in}}%
\pgfusepath{stroke}%
\end{pgfscope}%
\begin{pgfscope}%
\pgfpathrectangle{\pgfqpoint{0.647939in}{0.492442in}}{\pgfqpoint{4.273799in}{2.331163in}}%
\pgfusepath{clip}%
\pgfsetroundcap%
\pgfsetroundjoin%
\definecolor{currentfill}{rgb}{0.500000,0.500000,0.500000}%
\pgfsetfillcolor{currentfill}%
\pgfsetfillopacity{0.300000}%
\pgfsetlinewidth{0.301125pt}%
\definecolor{currentstroke}{rgb}{0.500000,0.500000,0.500000}%
\pgfsetstrokecolor{currentstroke}%
\pgfsetstrokeopacity{0.300000}%
\pgfsetdash{}{0pt}%
\pgfpathmoveto{\pgfqpoint{0.000000in}{0.000000in}}%
\pgfpathlineto{\pgfqpoint{0.000000in}{0.000000in}}%
\pgfpathclose%
\pgfusepath{stroke,fill}%
\end{pgfscope}%
\begin{pgfscope}%
\pgfpathrectangle{\pgfqpoint{0.647939in}{0.492442in}}{\pgfqpoint{4.273799in}{2.331163in}}%
\pgfusepath{clip}%
\pgfsetroundcap%
\pgfsetroundjoin%
\pgfsetlinewidth{0.301125pt}%
\definecolor{currentstroke}{rgb}{0.500000,0.500000,0.500000}%
\pgfsetstrokecolor{currentstroke}%
\pgfsetstrokeopacity{0.300000}%
\pgfsetdash{}{0pt}%
\pgfpathmoveto{\pgfqpoint{2.240708in}{1.730347in}}%
\pgfusepath{stroke}%
\end{pgfscope}%
\begin{pgfscope}%
\pgfpathrectangle{\pgfqpoint{0.647939in}{0.492442in}}{\pgfqpoint{4.273799in}{2.331163in}}%
\pgfusepath{clip}%
\pgfsetroundcap%
\pgfsetroundjoin%
\definecolor{currentfill}{rgb}{0.500000,0.500000,0.500000}%
\pgfsetfillcolor{currentfill}%
\pgfsetfillopacity{0.300000}%
\pgfsetlinewidth{0.301125pt}%
\definecolor{currentstroke}{rgb}{0.500000,0.500000,0.500000}%
\pgfsetstrokecolor{currentstroke}%
\pgfsetstrokeopacity{0.300000}%
\pgfsetdash{}{0pt}%
\pgfpathmoveto{\pgfqpoint{0.000000in}{0.000000in}}%
\pgfpathlineto{\pgfqpoint{0.000000in}{0.000000in}}%
\pgfpathclose%
\pgfusepath{stroke,fill}%
\end{pgfscope}%
\begin{pgfscope}%
\pgfpathrectangle{\pgfqpoint{0.647939in}{0.492442in}}{\pgfqpoint{4.273799in}{2.331163in}}%
\pgfusepath{clip}%
\pgfsetroundcap%
\pgfsetroundjoin%
\pgfsetlinewidth{0.301125pt}%
\definecolor{currentstroke}{rgb}{0.500000,0.500000,0.500000}%
\pgfsetstrokecolor{currentstroke}%
\pgfsetstrokeopacity{0.300000}%
\pgfsetdash{}{0pt}%
\pgfpathmoveto{\pgfqpoint{1.511420in}{2.092071in}}%
\pgfusepath{stroke}%
\end{pgfscope}%
\begin{pgfscope}%
\pgfpathrectangle{\pgfqpoint{0.647939in}{0.492442in}}{\pgfqpoint{4.273799in}{2.331163in}}%
\pgfusepath{clip}%
\pgfsetroundcap%
\pgfsetroundjoin%
\definecolor{currentfill}{rgb}{0.500000,0.500000,0.500000}%
\pgfsetfillcolor{currentfill}%
\pgfsetfillopacity{0.300000}%
\pgfsetlinewidth{0.301125pt}%
\definecolor{currentstroke}{rgb}{0.500000,0.500000,0.500000}%
\pgfsetstrokecolor{currentstroke}%
\pgfsetstrokeopacity{0.300000}%
\pgfsetdash{}{0pt}%
\pgfpathmoveto{\pgfqpoint{0.000000in}{0.000000in}}%
\pgfpathlineto{\pgfqpoint{0.000000in}{0.000000in}}%
\pgfpathclose%
\pgfusepath{stroke,fill}%
\end{pgfscope}%
\begin{pgfscope}%
\pgfpathrectangle{\pgfqpoint{0.647939in}{0.492442in}}{\pgfqpoint{4.273799in}{2.331163in}}%
\pgfusepath{clip}%
\pgfsetroundcap%
\pgfsetroundjoin%
\pgfsetlinewidth{0.301125pt}%
\definecolor{currentstroke}{rgb}{0.500000,0.500000,0.500000}%
\pgfsetstrokecolor{currentstroke}%
\pgfsetstrokeopacity{0.300000}%
\pgfsetdash{}{0pt}%
\pgfpathmoveto{\pgfqpoint{1.591814in}{1.510910in}}%
\pgfusepath{stroke}%
\end{pgfscope}%
\begin{pgfscope}%
\pgfpathrectangle{\pgfqpoint{0.647939in}{0.492442in}}{\pgfqpoint{4.273799in}{2.331163in}}%
\pgfusepath{clip}%
\pgfsetroundcap%
\pgfsetroundjoin%
\definecolor{currentfill}{rgb}{0.500000,0.500000,0.500000}%
\pgfsetfillcolor{currentfill}%
\pgfsetfillopacity{0.300000}%
\pgfsetlinewidth{0.301125pt}%
\definecolor{currentstroke}{rgb}{0.500000,0.500000,0.500000}%
\pgfsetstrokecolor{currentstroke}%
\pgfsetstrokeopacity{0.300000}%
\pgfsetdash{}{0pt}%
\pgfpathmoveto{\pgfqpoint{0.000000in}{0.000000in}}%
\pgfpathlineto{\pgfqpoint{0.000000in}{0.000000in}}%
\pgfpathclose%
\pgfusepath{stroke,fill}%
\end{pgfscope}%
\begin{pgfscope}%
\pgfpathrectangle{\pgfqpoint{0.647939in}{0.492442in}}{\pgfqpoint{4.273799in}{2.331163in}}%
\pgfusepath{clip}%
\pgfsetroundcap%
\pgfsetroundjoin%
\pgfsetlinewidth{0.301125pt}%
\definecolor{currentstroke}{rgb}{0.500000,0.500000,0.500000}%
\pgfsetstrokecolor{currentstroke}%
\pgfsetstrokeopacity{0.300000}%
\pgfsetdash{}{0pt}%
\pgfpathmoveto{\pgfqpoint{1.424993in}{1.234176in}}%
\pgfusepath{stroke}%
\end{pgfscope}%
\begin{pgfscope}%
\pgfpathrectangle{\pgfqpoint{0.647939in}{0.492442in}}{\pgfqpoint{4.273799in}{2.331163in}}%
\pgfusepath{clip}%
\pgfsetroundcap%
\pgfsetroundjoin%
\definecolor{currentfill}{rgb}{0.500000,0.500000,0.500000}%
\pgfsetfillcolor{currentfill}%
\pgfsetfillopacity{0.300000}%
\pgfsetlinewidth{0.301125pt}%
\definecolor{currentstroke}{rgb}{0.500000,0.500000,0.500000}%
\pgfsetstrokecolor{currentstroke}%
\pgfsetstrokeopacity{0.300000}%
\pgfsetdash{}{0pt}%
\pgfpathmoveto{\pgfqpoint{0.000000in}{0.000000in}}%
\pgfpathlineto{\pgfqpoint{0.000000in}{0.000000in}}%
\pgfpathclose%
\pgfusepath{stroke,fill}%
\end{pgfscope}%
\begin{pgfscope}%
\pgfpathrectangle{\pgfqpoint{0.647939in}{0.492442in}}{\pgfqpoint{4.273799in}{2.331163in}}%
\pgfusepath{clip}%
\pgfsetroundcap%
\pgfsetroundjoin%
\pgfsetlinewidth{0.301125pt}%
\definecolor{currentstroke}{rgb}{0.500000,0.500000,0.500000}%
\pgfsetstrokecolor{currentstroke}%
\pgfsetstrokeopacity{0.300000}%
\pgfsetdash{}{0pt}%
\pgfpathmoveto{\pgfqpoint{3.831374in}{1.202559in}}%
\pgfusepath{stroke}%
\end{pgfscope}%
\begin{pgfscope}%
\pgfpathrectangle{\pgfqpoint{0.647939in}{0.492442in}}{\pgfqpoint{4.273799in}{2.331163in}}%
\pgfusepath{clip}%
\pgfsetroundcap%
\pgfsetroundjoin%
\definecolor{currentfill}{rgb}{0.500000,0.500000,0.500000}%
\pgfsetfillcolor{currentfill}%
\pgfsetfillopacity{0.300000}%
\pgfsetlinewidth{0.301125pt}%
\definecolor{currentstroke}{rgb}{0.500000,0.500000,0.500000}%
\pgfsetstrokecolor{currentstroke}%
\pgfsetstrokeopacity{0.300000}%
\pgfsetdash{}{0pt}%
\pgfpathmoveto{\pgfqpoint{0.000000in}{0.000000in}}%
\pgfpathlineto{\pgfqpoint{0.000000in}{0.000000in}}%
\pgfpathclose%
\pgfusepath{stroke,fill}%
\end{pgfscope}%
\begin{pgfscope}%
\pgfpathrectangle{\pgfqpoint{0.647939in}{0.492442in}}{\pgfqpoint{4.273799in}{2.331163in}}%
\pgfusepath{clip}%
\pgfsetroundcap%
\pgfsetroundjoin%
\pgfsetlinewidth{0.301125pt}%
\definecolor{currentstroke}{rgb}{0.500000,0.500000,0.500000}%
\pgfsetstrokecolor{currentstroke}%
\pgfsetstrokeopacity{0.300000}%
\pgfsetdash{}{0pt}%
\pgfpathmoveto{\pgfqpoint{1.553756in}{1.633153in}}%
\pgfusepath{stroke}%
\end{pgfscope}%
\begin{pgfscope}%
\pgfpathrectangle{\pgfqpoint{0.647939in}{0.492442in}}{\pgfqpoint{4.273799in}{2.331163in}}%
\pgfusepath{clip}%
\pgfsetroundcap%
\pgfsetroundjoin%
\definecolor{currentfill}{rgb}{0.500000,0.500000,0.500000}%
\pgfsetfillcolor{currentfill}%
\pgfsetfillopacity{0.300000}%
\pgfsetlinewidth{0.301125pt}%
\definecolor{currentstroke}{rgb}{0.500000,0.500000,0.500000}%
\pgfsetstrokecolor{currentstroke}%
\pgfsetstrokeopacity{0.300000}%
\pgfsetdash{}{0pt}%
\pgfpathmoveto{\pgfqpoint{0.000000in}{0.000000in}}%
\pgfpathlineto{\pgfqpoint{0.000000in}{0.000000in}}%
\pgfpathclose%
\pgfusepath{stroke,fill}%
\end{pgfscope}%
\begin{pgfscope}%
\pgfpathrectangle{\pgfqpoint{0.647939in}{0.492442in}}{\pgfqpoint{4.273799in}{2.331163in}}%
\pgfusepath{clip}%
\pgfsetroundcap%
\pgfsetroundjoin%
\pgfsetlinewidth{0.301125pt}%
\definecolor{currentstroke}{rgb}{0.500000,0.500000,0.500000}%
\pgfsetstrokecolor{currentstroke}%
\pgfsetstrokeopacity{0.300000}%
\pgfsetdash{}{0pt}%
\pgfpathmoveto{\pgfqpoint{3.798123in}{1.112077in}}%
\pgfusepath{stroke}%
\end{pgfscope}%
\begin{pgfscope}%
\pgfpathrectangle{\pgfqpoint{0.647939in}{0.492442in}}{\pgfqpoint{4.273799in}{2.331163in}}%
\pgfusepath{clip}%
\pgfsetroundcap%
\pgfsetroundjoin%
\definecolor{currentfill}{rgb}{0.500000,0.500000,0.500000}%
\pgfsetfillcolor{currentfill}%
\pgfsetfillopacity{0.300000}%
\pgfsetlinewidth{0.301125pt}%
\definecolor{currentstroke}{rgb}{0.500000,0.500000,0.500000}%
\pgfsetstrokecolor{currentstroke}%
\pgfsetstrokeopacity{0.300000}%
\pgfsetdash{}{0pt}%
\pgfpathmoveto{\pgfqpoint{0.000000in}{0.000000in}}%
\pgfpathlineto{\pgfqpoint{0.000000in}{0.000000in}}%
\pgfpathclose%
\pgfusepath{stroke,fill}%
\end{pgfscope}%
\begin{pgfscope}%
\pgfpathrectangle{\pgfqpoint{0.647939in}{0.492442in}}{\pgfqpoint{4.273799in}{2.331163in}}%
\pgfusepath{clip}%
\pgfsetroundcap%
\pgfsetroundjoin%
\pgfsetlinewidth{0.301125pt}%
\definecolor{currentstroke}{rgb}{0.500000,0.500000,0.500000}%
\pgfsetstrokecolor{currentstroke}%
\pgfsetstrokeopacity{0.300000}%
\pgfsetdash{}{0pt}%
\pgfpathmoveto{\pgfqpoint{3.938094in}{2.177738in}}%
\pgfusepath{stroke}%
\end{pgfscope}%
\begin{pgfscope}%
\pgfpathrectangle{\pgfqpoint{0.647939in}{0.492442in}}{\pgfqpoint{4.273799in}{2.331163in}}%
\pgfusepath{clip}%
\pgfsetroundcap%
\pgfsetroundjoin%
\definecolor{currentfill}{rgb}{0.500000,0.500000,0.500000}%
\pgfsetfillcolor{currentfill}%
\pgfsetfillopacity{0.300000}%
\pgfsetlinewidth{0.301125pt}%
\definecolor{currentstroke}{rgb}{0.500000,0.500000,0.500000}%
\pgfsetstrokecolor{currentstroke}%
\pgfsetstrokeopacity{0.300000}%
\pgfsetdash{}{0pt}%
\pgfpathmoveto{\pgfqpoint{0.000000in}{0.000000in}}%
\pgfpathlineto{\pgfqpoint{0.000000in}{0.000000in}}%
\pgfpathclose%
\pgfusepath{stroke,fill}%
\end{pgfscope}%
\begin{pgfscope}%
\pgfpathrectangle{\pgfqpoint{0.647939in}{0.492442in}}{\pgfqpoint{4.273799in}{2.331163in}}%
\pgfusepath{clip}%
\pgfsetroundcap%
\pgfsetroundjoin%
\pgfsetlinewidth{0.301125pt}%
\definecolor{currentstroke}{rgb}{0.500000,0.500000,0.500000}%
\pgfsetstrokecolor{currentstroke}%
\pgfsetstrokeopacity{0.300000}%
\pgfsetdash{}{0pt}%
\pgfpathmoveto{\pgfqpoint{2.515806in}{2.021899in}}%
\pgfusepath{stroke}%
\end{pgfscope}%
\begin{pgfscope}%
\pgfpathrectangle{\pgfqpoint{0.647939in}{0.492442in}}{\pgfqpoint{4.273799in}{2.331163in}}%
\pgfusepath{clip}%
\pgfsetroundcap%
\pgfsetroundjoin%
\definecolor{currentfill}{rgb}{0.500000,0.500000,0.500000}%
\pgfsetfillcolor{currentfill}%
\pgfsetfillopacity{0.300000}%
\pgfsetlinewidth{0.301125pt}%
\definecolor{currentstroke}{rgb}{0.500000,0.500000,0.500000}%
\pgfsetstrokecolor{currentstroke}%
\pgfsetstrokeopacity{0.300000}%
\pgfsetdash{}{0pt}%
\pgfpathmoveto{\pgfqpoint{0.000000in}{0.000000in}}%
\pgfpathlineto{\pgfqpoint{0.000000in}{0.000000in}}%
\pgfpathclose%
\pgfusepath{stroke,fill}%
\end{pgfscope}%
\begin{pgfscope}%
\pgfpathrectangle{\pgfqpoint{0.647939in}{0.492442in}}{\pgfqpoint{4.273799in}{2.331163in}}%
\pgfusepath{clip}%
\pgfsetroundcap%
\pgfsetroundjoin%
\pgfsetlinewidth{0.301125pt}%
\definecolor{currentstroke}{rgb}{0.500000,0.500000,0.500000}%
\pgfsetstrokecolor{currentstroke}%
\pgfsetstrokeopacity{0.300000}%
\pgfsetdash{}{0pt}%
\pgfpathmoveto{\pgfqpoint{1.613985in}{1.925713in}}%
\pgfusepath{stroke}%
\end{pgfscope}%
\begin{pgfscope}%
\pgfpathrectangle{\pgfqpoint{0.647939in}{0.492442in}}{\pgfqpoint{4.273799in}{2.331163in}}%
\pgfusepath{clip}%
\pgfsetroundcap%
\pgfsetroundjoin%
\definecolor{currentfill}{rgb}{0.500000,0.500000,0.500000}%
\pgfsetfillcolor{currentfill}%
\pgfsetfillopacity{0.300000}%
\pgfsetlinewidth{0.301125pt}%
\definecolor{currentstroke}{rgb}{0.500000,0.500000,0.500000}%
\pgfsetstrokecolor{currentstroke}%
\pgfsetstrokeopacity{0.300000}%
\pgfsetdash{}{0pt}%
\pgfpathmoveto{\pgfqpoint{0.000000in}{0.000000in}}%
\pgfpathlineto{\pgfqpoint{0.000000in}{0.000000in}}%
\pgfpathclose%
\pgfusepath{stroke,fill}%
\end{pgfscope}%
\begin{pgfscope}%
\pgfpathrectangle{\pgfqpoint{0.647939in}{0.492442in}}{\pgfqpoint{4.273799in}{2.331163in}}%
\pgfusepath{clip}%
\pgfsetroundcap%
\pgfsetroundjoin%
\pgfsetlinewidth{0.301125pt}%
\definecolor{currentstroke}{rgb}{0.500000,0.500000,0.500000}%
\pgfsetstrokecolor{currentstroke}%
\pgfsetstrokeopacity{0.300000}%
\pgfsetdash{}{0pt}%
\pgfpathmoveto{\pgfqpoint{2.883474in}{1.244491in}}%
\pgfusepath{stroke}%
\end{pgfscope}%
\begin{pgfscope}%
\pgfpathrectangle{\pgfqpoint{0.647939in}{0.492442in}}{\pgfqpoint{4.273799in}{2.331163in}}%
\pgfusepath{clip}%
\pgfsetroundcap%
\pgfsetroundjoin%
\definecolor{currentfill}{rgb}{0.500000,0.500000,0.500000}%
\pgfsetfillcolor{currentfill}%
\pgfsetfillopacity{0.300000}%
\pgfsetlinewidth{0.301125pt}%
\definecolor{currentstroke}{rgb}{0.500000,0.500000,0.500000}%
\pgfsetstrokecolor{currentstroke}%
\pgfsetstrokeopacity{0.300000}%
\pgfsetdash{}{0pt}%
\pgfpathmoveto{\pgfqpoint{0.000000in}{0.000000in}}%
\pgfpathlineto{\pgfqpoint{0.000000in}{0.000000in}}%
\pgfpathclose%
\pgfusepath{stroke,fill}%
\end{pgfscope}%
\begin{pgfscope}%
\pgfpathrectangle{\pgfqpoint{0.647939in}{0.492442in}}{\pgfqpoint{4.273799in}{2.331163in}}%
\pgfusepath{clip}%
\pgfsetroundcap%
\pgfsetroundjoin%
\pgfsetlinewidth{0.301125pt}%
\definecolor{currentstroke}{rgb}{0.500000,0.500000,0.500000}%
\pgfsetstrokecolor{currentstroke}%
\pgfsetstrokeopacity{0.300000}%
\pgfsetdash{}{0pt}%
\pgfpathmoveto{\pgfqpoint{2.561316in}{1.980779in}}%
\pgfusepath{stroke}%
\end{pgfscope}%
\begin{pgfscope}%
\pgfpathrectangle{\pgfqpoint{0.647939in}{0.492442in}}{\pgfqpoint{4.273799in}{2.331163in}}%
\pgfusepath{clip}%
\pgfsetroundcap%
\pgfsetroundjoin%
\definecolor{currentfill}{rgb}{0.500000,0.500000,0.500000}%
\pgfsetfillcolor{currentfill}%
\pgfsetfillopacity{0.300000}%
\pgfsetlinewidth{0.301125pt}%
\definecolor{currentstroke}{rgb}{0.500000,0.500000,0.500000}%
\pgfsetstrokecolor{currentstroke}%
\pgfsetstrokeopacity{0.300000}%
\pgfsetdash{}{0pt}%
\pgfpathmoveto{\pgfqpoint{0.000000in}{0.000000in}}%
\pgfpathlineto{\pgfqpoint{0.000000in}{0.000000in}}%
\pgfpathclose%
\pgfusepath{stroke,fill}%
\end{pgfscope}%
\begin{pgfscope}%
\pgfpathrectangle{\pgfqpoint{0.647939in}{0.492442in}}{\pgfqpoint{4.273799in}{2.331163in}}%
\pgfusepath{clip}%
\pgfsetroundcap%
\pgfsetroundjoin%
\pgfsetlinewidth{0.301125pt}%
\definecolor{currentstroke}{rgb}{0.500000,0.500000,0.500000}%
\pgfsetstrokecolor{currentstroke}%
\pgfsetstrokeopacity{0.300000}%
\pgfsetdash{}{0pt}%
\pgfpathmoveto{\pgfqpoint{1.903308in}{1.926785in}}%
\pgfusepath{stroke}%
\end{pgfscope}%
\begin{pgfscope}%
\pgfpathrectangle{\pgfqpoint{0.647939in}{0.492442in}}{\pgfqpoint{4.273799in}{2.331163in}}%
\pgfusepath{clip}%
\pgfsetroundcap%
\pgfsetroundjoin%
\definecolor{currentfill}{rgb}{0.500000,0.500000,0.500000}%
\pgfsetfillcolor{currentfill}%
\pgfsetfillopacity{0.300000}%
\pgfsetlinewidth{0.301125pt}%
\definecolor{currentstroke}{rgb}{0.500000,0.500000,0.500000}%
\pgfsetstrokecolor{currentstroke}%
\pgfsetstrokeopacity{0.300000}%
\pgfsetdash{}{0pt}%
\pgfpathmoveto{\pgfqpoint{0.000000in}{0.000000in}}%
\pgfpathlineto{\pgfqpoint{0.000000in}{0.000000in}}%
\pgfpathclose%
\pgfusepath{stroke,fill}%
\end{pgfscope}%
\begin{pgfscope}%
\pgfpathrectangle{\pgfqpoint{0.647939in}{0.492442in}}{\pgfqpoint{4.273799in}{2.331163in}}%
\pgfusepath{clip}%
\pgfsetroundcap%
\pgfsetroundjoin%
\pgfsetlinewidth{0.301125pt}%
\definecolor{currentstroke}{rgb}{0.500000,0.500000,0.500000}%
\pgfsetstrokecolor{currentstroke}%
\pgfsetstrokeopacity{0.300000}%
\pgfsetdash{}{0pt}%
\pgfpathmoveto{\pgfqpoint{3.318683in}{1.328930in}}%
\pgfusepath{stroke}%
\end{pgfscope}%
\begin{pgfscope}%
\pgfpathrectangle{\pgfqpoint{0.647939in}{0.492442in}}{\pgfqpoint{4.273799in}{2.331163in}}%
\pgfusepath{clip}%
\pgfsetroundcap%
\pgfsetroundjoin%
\definecolor{currentfill}{rgb}{0.500000,0.500000,0.500000}%
\pgfsetfillcolor{currentfill}%
\pgfsetfillopacity{0.300000}%
\pgfsetlinewidth{0.301125pt}%
\definecolor{currentstroke}{rgb}{0.500000,0.500000,0.500000}%
\pgfsetstrokecolor{currentstroke}%
\pgfsetstrokeopacity{0.300000}%
\pgfsetdash{}{0pt}%
\pgfpathmoveto{\pgfqpoint{0.000000in}{0.000000in}}%
\pgfpathlineto{\pgfqpoint{0.000000in}{0.000000in}}%
\pgfpathclose%
\pgfusepath{stroke,fill}%
\end{pgfscope}%
\begin{pgfscope}%
\pgfpathrectangle{\pgfqpoint{0.647939in}{0.492442in}}{\pgfqpoint{4.273799in}{2.331163in}}%
\pgfusepath{clip}%
\pgfsetbuttcap%
\pgfsetroundjoin%
\pgfsetlinewidth{0.301125pt}%
\definecolor{currentstroke}{rgb}{0.500000,0.500000,0.500000}%
\pgfsetstrokecolor{currentstroke}%
\pgfsetstrokeopacity{0.300000}%
\pgfsetdash{}{0pt}%
\pgfpathmoveto{\pgfqpoint{2.225345in}{0.492442in}}%
\pgfpathlineto{\pgfqpoint{2.218924in}{0.501807in}}%
\pgfpathlineto{\pgfqpoint{2.185810in}{0.550360in}}%
\pgfpathlineto{\pgfqpoint{2.152998in}{0.598973in}}%
\pgfpathlineto{\pgfqpoint{2.120463in}{0.647642in}}%
\pgfpathlineto{\pgfqpoint{2.088175in}{0.696360in}}%
\pgfpathlineto{\pgfqpoint{2.056105in}{0.745121in}}%
\pgfpathlineto{\pgfqpoint{2.024227in}{0.793919in}}%
\pgfpathlineto{\pgfqpoint{1.992506in}{0.842748in}}%
\pgfpathlineto{\pgfqpoint{1.960895in}{0.891598in}}%
\pgfpathlineto{\pgfqpoint{1.929351in}{0.940460in}}%
\pgfpathlineto{\pgfqpoint{1.897826in}{0.989326in}}%
\pgfpathlineto{\pgfqpoint{1.866250in}{1.038183in}}%
\pgfpathlineto{\pgfqpoint{1.834535in}{1.087012in}}%
\pgfpathlineto{\pgfqpoint{1.802582in}{1.135794in}}%
\pgfpathlineto{\pgfqpoint{1.770270in}{1.184507in}}%
\pgfpathlineto{\pgfqpoint{1.737407in}{1.233109in}}%
\pgfpathlineto{\pgfqpoint{1.703726in}{1.281541in}}%
\pgfpathlineto{\pgfqpoint{1.668857in}{1.329720in}}%
\pgfpathlineto{\pgfqpoint{1.632213in}{1.377499in}}%
\pgfpathlineto{\pgfqpoint{1.592765in}{1.424600in}}%
\pgfpathlineto{\pgfqpoint{1.548471in}{1.470354in}}%
\pgfpathlineto{\pgfqpoint{1.509338in}{1.502836in}}%
\pgfpathlineto{\pgfqpoint{1.475411in}{1.523687in}}%
\pgfpathlineto{\pgfqpoint{1.442778in}{1.536442in}}%
\pgfpathlineto{\pgfqpoint{1.401336in}{1.541576in}}%
\pgfpathlineto{\pgfqpoint{1.361068in}{1.534742in}}%
\pgfpathlineto{\pgfqpoint{1.361068in}{1.534742in}}%
\pgfpathlineto{\pgfqpoint{1.313474in}{1.512599in}}%
\pgfpathlineto{\pgfqpoint{1.313474in}{1.512599in}}%
\pgfpathlineto{\pgfqpoint{1.258355in}{1.470915in}}%
\pgfpathlineto{\pgfqpoint{1.212325in}{1.425702in}}%
\pgfpathlineto{\pgfqpoint{1.171347in}{1.379013in}}%
\pgfpathlineto{\pgfqpoint{1.133590in}{1.331524in}}%
\pgfpathlineto{\pgfqpoint{1.098080in}{1.283523in}}%
\pgfpathlineto{\pgfqpoint{1.064243in}{1.235155in}}%
\pgfpathlineto{\pgfqpoint{1.031707in}{1.186508in}}%
\pgfpathlineto{\pgfqpoint{1.000249in}{1.137642in}}%
\pgfpathlineto{\pgfqpoint{0.969721in}{1.088604in}}%
\pgfpathlineto{\pgfqpoint{0.939990in}{1.039420in}}%
\pgfpathlineto{\pgfqpoint{0.910947in}{0.990107in}}%
\pgfpathlineto{\pgfqpoint{0.882536in}{0.940684in}}%
\pgfpathlineto{\pgfqpoint{0.854698in}{0.891166in}}%
\pgfpathlineto{\pgfqpoint{0.827373in}{0.841559in}}%
\pgfpathlineto{\pgfqpoint{0.800533in}{0.791875in}}%
\pgfpathlineto{\pgfqpoint{0.774135in}{0.742119in}}%
\pgfpathlineto{\pgfqpoint{0.748153in}{0.692296in}}%
\pgfpathlineto{\pgfqpoint{0.722568in}{0.642414in}}%
\pgfpathlineto{\pgfqpoint{0.697348in}{0.592475in}}%
\pgfpathlineto{\pgfqpoint{0.672479in}{0.542484in}}%
\pgfpathlineto{\pgfqpoint{0.647939in}{0.492442in}}%
\pgfpathlineto{\pgfqpoint{0.647939in}{0.492442in}}%
\pgfusepath{stroke}%
\end{pgfscope}%
\begin{pgfscope}%
\pgfpathrectangle{\pgfqpoint{0.647939in}{0.492442in}}{\pgfqpoint{4.273799in}{2.331163in}}%
\pgfusepath{clip}%
\pgfsetbuttcap%
\pgfsetroundjoin%
\pgfsetlinewidth{0.301125pt}%
\definecolor{currentstroke}{rgb}{0.500000,0.500000,0.500000}%
\pgfsetstrokecolor{currentstroke}%
\pgfsetstrokeopacity{0.300000}%
\pgfsetdash{}{0pt}%
\pgfpathmoveto{\pgfqpoint{1.757627in}{0.492442in}}%
\pgfpathlineto{\pgfqpoint{1.732475in}{0.523374in}}%
\pgfpathlineto{\pgfqpoint{1.693458in}{0.570602in}}%
\pgfpathlineto{\pgfqpoint{1.653255in}{0.617532in}}%
\pgfpathlineto{\pgfqpoint{1.611490in}{0.664051in}}%
\pgfpathlineto{\pgfqpoint{1.567634in}{0.709989in}}%
\pgfpathlineto{\pgfqpoint{1.520898in}{0.755065in}}%
\pgfpathlineto{\pgfqpoint{1.470030in}{0.798773in}}%
\pgfpathlineto{\pgfqpoint{1.412886in}{0.840019in}}%
\pgfpathlineto{\pgfqpoint{1.361411in}{0.869070in}}%
\pgfpathlineto{\pgfqpoint{1.313755in}{0.888303in}}%
\pgfpathlineto{\pgfqpoint{1.265648in}{0.899611in}}%
\pgfpathlineto{\pgfqpoint{1.208889in}{0.901857in}}%
\pgfpathlineto{\pgfqpoint{1.154576in}{0.892405in}}%
\pgfpathlineto{\pgfqpoint{1.154576in}{0.892405in}}%
\pgfpathlineto{\pgfqpoint{1.080558in}{0.861054in}}%
\pgfpathlineto{\pgfqpoint{1.020780in}{0.821053in}}%
\pgfpathlineto{\pgfqpoint{0.970044in}{0.777398in}}%
\pgfpathlineto{\pgfqpoint{0.925134in}{0.731828in}}%
\pgfpathlineto{\pgfqpoint{0.884291in}{0.685107in}}%
\pgfpathlineto{\pgfqpoint{0.846451in}{0.637625in}}%
\pgfpathlineto{\pgfqpoint{0.810924in}{0.589603in}}%
\pgfpathlineto{\pgfqpoint{0.777236in}{0.541177in}}%
\pgfpathlineto{\pgfqpoint{0.745071in}{0.492442in}}%
\pgfpathlineto{\pgfqpoint{0.745071in}{0.492442in}}%
\pgfusepath{stroke}%
\end{pgfscope}%
\begin{pgfscope}%
\pgfpathrectangle{\pgfqpoint{0.647939in}{0.492442in}}{\pgfqpoint{4.273799in}{2.331163in}}%
\pgfusepath{clip}%
\pgfsetbuttcap%
\pgfsetroundjoin%
\pgfsetlinewidth{0.301125pt}%
\definecolor{currentstroke}{rgb}{0.500000,0.500000,0.500000}%
\pgfsetstrokecolor{currentstroke}%
\pgfsetstrokeopacity{0.300000}%
\pgfsetdash{}{0pt}%
\pgfpathmoveto{\pgfqpoint{1.532161in}{0.492442in}}%
\pgfpathlineto{\pgfqpoint{1.500195in}{0.521047in}}%
\pgfpathlineto{\pgfqpoint{1.448605in}{0.564500in}}%
\pgfpathlineto{\pgfqpoint{1.391497in}{0.605797in}}%
\pgfpathlineto{\pgfqpoint{1.335096in}{0.638603in}}%
\pgfpathlineto{\pgfqpoint{1.283258in}{0.660946in}}%
\pgfpathlineto{\pgfqpoint{1.232633in}{0.674853in}}%
\pgfpathlineto{\pgfqpoint{1.177269in}{0.680511in}}%
\pgfpathlineto{\pgfqpoint{1.121350in}{0.675683in}}%
\pgfpathlineto{\pgfqpoint{1.121350in}{0.675683in}}%
\pgfpathlineto{\pgfqpoint{1.062023in}{0.658632in}}%
\pgfpathlineto{\pgfqpoint{1.062023in}{0.658632in}}%
\pgfpathlineto{\pgfqpoint{0.993166in}{0.623512in}}%
\pgfpathlineto{\pgfqpoint{0.935986in}{0.582318in}}%
\pgfpathlineto{\pgfqpoint{0.886487in}{0.538209in}}%
\pgfpathlineto{\pgfqpoint{0.842203in}{0.492442in}}%
\pgfpathlineto{\pgfqpoint{0.842203in}{0.492442in}}%
\pgfusepath{stroke}%
\end{pgfscope}%
\begin{pgfscope}%
\pgfpathrectangle{\pgfqpoint{0.647939in}{0.492442in}}{\pgfqpoint{4.273799in}{2.331163in}}%
\pgfusepath{clip}%
\pgfsetbuttcap%
\pgfsetroundjoin%
\pgfsetlinewidth{0.301125pt}%
\definecolor{currentstroke}{rgb}{0.500000,0.500000,0.500000}%
\pgfsetstrokecolor{currentstroke}%
\pgfsetstrokeopacity{0.300000}%
\pgfsetdash{}{0pt}%
\pgfpathmoveto{\pgfqpoint{1.368633in}{0.492442in}}%
\pgfpathlineto{\pgfqpoint{1.347880in}{0.504545in}}%
\pgfpathlineto{\pgfqpoint{1.292147in}{0.532906in}}%
\pgfpathlineto{\pgfqpoint{1.239793in}{0.551784in}}%
\pgfpathlineto{\pgfqpoint{1.186579in}{0.562610in}}%
\pgfpathlineto{\pgfqpoint{1.125583in}{0.564039in}}%
\pgfpathlineto{\pgfqpoint{1.067031in}{0.553970in}}%
\pgfpathlineto{\pgfqpoint{1.067031in}{0.553970in}}%
\pgfpathlineto{\pgfqpoint{1.003743in}{0.530193in}}%
\pgfpathlineto{\pgfqpoint{1.003743in}{0.530193in}}%
\pgfpathlineto{\pgfqpoint{0.939334in}{0.492442in}}%
\pgfpathlineto{\pgfqpoint{0.939334in}{0.492442in}}%
\pgfusepath{stroke}%
\end{pgfscope}%
\begin{pgfscope}%
\pgfpathrectangle{\pgfqpoint{0.647939in}{0.492442in}}{\pgfqpoint{4.273799in}{2.331163in}}%
\pgfusepath{clip}%
\pgfsetbuttcap%
\pgfsetroundjoin%
\pgfsetlinewidth{0.301125pt}%
\definecolor{currentstroke}{rgb}{0.500000,0.500000,0.500000}%
\pgfsetstrokecolor{currentstroke}%
\pgfsetstrokeopacity{0.300000}%
\pgfsetdash{}{0pt}%
\pgfpathmoveto{\pgfqpoint{1.619257in}{0.492442in}}%
\pgfpathlineto{\pgfqpoint{1.619257in}{0.492442in}}%
\pgfpathlineto{\pgfqpoint{1.575607in}{0.538441in}}%
\pgfpathlineto{\pgfqpoint{1.529543in}{0.583729in}}%
\pgfpathlineto{\pgfqpoint{1.480153in}{0.627948in}}%
\pgfpathlineto{\pgfqpoint{1.425988in}{0.670451in}}%
\pgfpathlineto{\pgfqpoint{1.364593in}{0.709820in}}%
\pgfpathlineto{\pgfqpoint{1.291802in}{0.742530in}}%
\pgfpathlineto{\pgfqpoint{1.291802in}{0.742530in}}%
\pgfpathlineto{\pgfqpoint{1.234719in}{0.756411in}}%
\pgfpathlineto{\pgfqpoint{1.234719in}{0.756411in}}%
\pgfpathlineto{\pgfqpoint{1.181271in}{0.759256in}}%
\pgfpathlineto{\pgfqpoint{1.128839in}{0.752193in}}%
\pgfpathlineto{\pgfqpoint{1.081734in}{0.737459in}}%
\pgfpathlineto{\pgfqpoint{1.034513in}{0.714736in}}%
\pgfpathlineto{\pgfqpoint{0.984343in}{0.682060in}}%
\pgfpathlineto{\pgfqpoint{0.932023in}{0.638983in}}%
\pgfusepath{stroke}%
\end{pgfscope}%
\begin{pgfscope}%
\pgfpathrectangle{\pgfqpoint{0.647939in}{0.492442in}}{\pgfqpoint{4.273799in}{2.331163in}}%
\pgfusepath{clip}%
\pgfsetbuttcap%
\pgfsetroundjoin%
\pgfsetlinewidth{0.301125pt}%
\definecolor{currentstroke}{rgb}{0.500000,0.500000,0.500000}%
\pgfsetstrokecolor{currentstroke}%
\pgfsetstrokeopacity{0.300000}%
\pgfsetdash{}{0pt}%
\pgfpathmoveto{\pgfqpoint{1.813521in}{0.492442in}}%
\pgfpathlineto{\pgfqpoint{1.813521in}{0.492442in}}%
\pgfpathlineto{\pgfqpoint{1.776545in}{0.540158in}}%
\pgfpathlineto{\pgfqpoint{1.738889in}{0.587714in}}%
\pgfpathlineto{\pgfqpoint{1.700346in}{0.635058in}}%
\pgfpathlineto{\pgfqpoint{1.660647in}{0.682114in}}%
\pgfpathlineto{\pgfqpoint{1.619418in}{0.728775in}}%
\pgfpathlineto{\pgfqpoint{1.576127in}{0.774872in}}%
\pgfpathlineto{\pgfqpoint{1.529977in}{0.820126in}}%
\pgfpathlineto{\pgfqpoint{1.479693in}{0.864029in}}%
\pgfpathlineto{\pgfqpoint{1.423080in}{0.905531in}}%
\pgfpathlineto{\pgfqpoint{1.356150in}{0.941856in}}%
\pgfpathlineto{\pgfqpoint{1.356150in}{0.941856in}}%
\pgfpathlineto{\pgfqpoint{1.299979in}{0.960314in}}%
\pgfpathlineto{\pgfqpoint{1.299979in}{0.960314in}}%
\pgfpathlineto{\pgfqpoint{1.248561in}{0.966728in}}%
\pgfpathlineto{\pgfqpoint{1.196281in}{0.962446in}}%
\pgfpathlineto{\pgfqpoint{1.151818in}{0.950221in}}%
\pgfusepath{stroke}%
\end{pgfscope}%
\begin{pgfscope}%
\pgfpathrectangle{\pgfqpoint{0.647939in}{0.492442in}}{\pgfqpoint{4.273799in}{2.331163in}}%
\pgfusepath{clip}%
\pgfsetbuttcap%
\pgfsetroundjoin%
\pgfsetlinewidth{0.301125pt}%
\definecolor{currentstroke}{rgb}{0.500000,0.500000,0.500000}%
\pgfsetstrokecolor{currentstroke}%
\pgfsetstrokeopacity{0.300000}%
\pgfsetdash{}{0pt}%
\pgfpathmoveto{\pgfqpoint{1.910652in}{0.492442in}}%
\pgfpathlineto{\pgfqpoint{1.910652in}{0.492442in}}%
\pgfpathlineto{\pgfqpoint{1.875454in}{0.540556in}}%
\pgfpathlineto{\pgfqpoint{1.839961in}{0.588606in}}%
\pgfpathlineto{\pgfqpoint{1.804064in}{0.636566in}}%
\pgfpathlineto{\pgfqpoint{1.767627in}{0.684404in}}%
\pgfpathlineto{\pgfqpoint{1.730470in}{0.732078in}}%
\pgfpathlineto{\pgfqpoint{1.692353in}{0.779524in}}%
\pgfpathlineto{\pgfqpoint{1.652954in}{0.826655in}}%
\pgfpathlineto{\pgfqpoint{1.611830in}{0.873343in}}%
\pgfpathlineto{\pgfqpoint{1.568317in}{0.919375in}}%
\pgfpathlineto{\pgfqpoint{1.521376in}{0.964379in}}%
\pgfpathlineto{\pgfqpoint{1.469231in}{1.007616in}}%
\pgfpathlineto{\pgfqpoint{1.408573in}{1.047251in}}%
\pgfpathlineto{\pgfqpoint{1.408573in}{1.047251in}}%
\pgfpathlineto{\pgfqpoint{1.349781in}{1.072976in}}%
\pgfpathlineto{\pgfqpoint{1.349781in}{1.072976in}}%
\pgfpathlineto{\pgfqpoint{1.298938in}{1.084070in}}%
\pgfpathlineto{\pgfqpoint{1.243347in}{1.083516in}}%
\pgfpathlineto{\pgfqpoint{1.199164in}{1.073608in}}%
\pgfpathlineto{\pgfqpoint{1.157240in}{1.056526in}}%
\pgfpathlineto{\pgfqpoint{1.112894in}{1.030594in}}%
\pgfpathlineto{\pgfqpoint{1.064096in}{0.993235in}}%
\pgfpathlineto{\pgfqpoint{1.016466in}{0.948552in}}%
\pgfpathlineto{\pgfqpoint{0.973873in}{0.902326in}}%
\pgfpathlineto{\pgfqpoint{0.934802in}{0.855156in}}%
\pgfpathlineto{\pgfqpoint{0.898340in}{0.807350in}}%
\pgfpathlineto{\pgfqpoint{0.863892in}{0.759087in}}%
\pgfusepath{stroke}%
\end{pgfscope}%
\begin{pgfscope}%
\pgfpathrectangle{\pgfqpoint{0.647939in}{0.492442in}}{\pgfqpoint{4.273799in}{2.331163in}}%
\pgfusepath{clip}%
\pgfsetbuttcap%
\pgfsetroundjoin%
\pgfsetlinewidth{0.301125pt}%
\definecolor{currentstroke}{rgb}{0.500000,0.500000,0.500000}%
\pgfsetstrokecolor{currentstroke}%
\pgfsetstrokeopacity{0.300000}%
\pgfsetdash{}{0pt}%
\pgfpathmoveto{\pgfqpoint{2.007784in}{0.492442in}}%
\pgfpathlineto{\pgfqpoint{2.007784in}{0.492442in}}%
\pgfpathlineto{\pgfqpoint{1.973724in}{0.540800in}}%
\pgfpathlineto{\pgfqpoint{1.939628in}{0.589150in}}%
\pgfpathlineto{\pgfqpoint{1.905428in}{0.637478in}}%
\pgfpathlineto{\pgfqpoint{1.871041in}{0.685766in}}%
\pgfpathlineto{\pgfqpoint{1.836373in}{0.733995in}}%
\pgfpathlineto{\pgfqpoint{1.801313in}{0.782138in}}%
\pgfpathlineto{\pgfqpoint{1.765720in}{0.830165in}}%
\pgfpathlineto{\pgfqpoint{1.729404in}{0.878031in}}%
\pgfpathlineto{\pgfqpoint{1.692106in}{0.925670in}}%
\pgfpathlineto{\pgfqpoint{1.653468in}{0.972988in}}%
\pgfpathlineto{\pgfqpoint{1.612984in}{1.019840in}}%
\pgfpathlineto{\pgfqpoint{1.569870in}{1.065979in}}%
\pgfpathlineto{\pgfqpoint{1.522827in}{1.110939in}}%
\pgfpathlineto{\pgfqpoint{1.469474in}{1.153700in}}%
\pgfpathlineto{\pgfqpoint{1.405037in}{1.191277in}}%
\pgfpathlineto{\pgfqpoint{1.405037in}{1.191277in}}%
\pgfpathlineto{\pgfqpoint{1.354940in}{1.208290in}}%
\pgfpathlineto{\pgfqpoint{1.354940in}{1.208290in}}%
\pgfpathlineto{\pgfqpoint{1.308636in}{1.213640in}}%
\pgfpathlineto{\pgfqpoint{1.261773in}{1.208595in}}%
\pgfpathlineto{\pgfqpoint{1.221685in}{1.196065in}}%
\pgfpathlineto{\pgfqpoint{1.181159in}{1.175920in}}%
\pgfpathlineto{\pgfqpoint{1.136816in}{1.145765in}}%
\pgfpathlineto{\pgfqpoint{1.086755in}{1.102400in}}%
\pgfpathlineto{\pgfqpoint{1.042449in}{1.056648in}}%
\pgfpathlineto{\pgfqpoint{1.002251in}{1.009753in}}%
\pgfusepath{stroke}%
\end{pgfscope}%
\begin{pgfscope}%
\pgfpathrectangle{\pgfqpoint{0.647939in}{0.492442in}}{\pgfqpoint{4.273799in}{2.331163in}}%
\pgfusepath{clip}%
\pgfsetbuttcap%
\pgfsetroundjoin%
\pgfsetlinewidth{0.301125pt}%
\definecolor{currentstroke}{rgb}{0.500000,0.500000,0.500000}%
\pgfsetstrokecolor{currentstroke}%
\pgfsetstrokeopacity{0.300000}%
\pgfsetdash{}{0pt}%
\pgfpathmoveto{\pgfqpoint{2.104916in}{0.492442in}}%
\pgfpathlineto{\pgfqpoint{2.104916in}{0.492442in}}%
\pgfpathlineto{\pgfqpoint{2.071503in}{0.540934in}}%
\pgfpathlineto{\pgfqpoint{2.038229in}{0.589454in}}%
\pgfpathlineto{\pgfqpoint{2.005049in}{0.637993in}}%
\pgfpathlineto{\pgfqpoint{1.971921in}{0.686543in}}%
\pgfpathlineto{\pgfqpoint{1.938797in}{0.735094in}}%
\pgfpathlineto{\pgfqpoint{1.905614in}{0.783632in}}%
\pgfpathlineto{\pgfqpoint{1.872293in}{0.832143in}}%
\pgfpathlineto{\pgfqpoint{1.838743in}{0.880605in}}%
\pgfpathlineto{\pgfqpoint{1.804858in}{0.928999in}}%
\pgfpathlineto{\pgfqpoint{1.770502in}{0.977294in}}%
\pgfpathlineto{\pgfqpoint{1.735481in}{1.025445in}}%
\pgfpathlineto{\pgfqpoint{1.699528in}{1.073390in}}%
\pgfpathlineto{\pgfqpoint{1.662287in}{1.121039in}}%
\pgfpathlineto{\pgfqpoint{1.623221in}{1.168246in}}%
\pgfpathlineto{\pgfqpoint{1.581473in}{1.214753in}}%
\pgfpathlineto{\pgfqpoint{1.535541in}{1.260042in}}%
\pgfpathlineto{\pgfqpoint{1.482466in}{1.302833in}}%
\pgfpathlineto{\pgfqpoint{1.415599in}{1.338933in}}%
\pgfpathlineto{\pgfqpoint{1.415599in}{1.338933in}}%
\pgfpathlineto{\pgfqpoint{1.371119in}{1.350319in}}%
\pgfpathlineto{\pgfqpoint{1.371119in}{1.350319in}}%
\pgfpathlineto{\pgfqpoint{1.328930in}{1.351069in}}%
\pgfpathlineto{\pgfqpoint{1.289159in}{1.342864in}}%
\pgfpathlineto{\pgfqpoint{1.252237in}{1.327807in}}%
\pgfpathlineto{\pgfqpoint{1.212679in}{1.304210in}}%
\pgfusepath{stroke}%
\end{pgfscope}%
\begin{pgfscope}%
\pgfpathrectangle{\pgfqpoint{0.647939in}{0.492442in}}{\pgfqpoint{4.273799in}{2.331163in}}%
\pgfusepath{clip}%
\pgfsetbuttcap%
\pgfsetroundjoin%
\pgfsetlinewidth{0.301125pt}%
\definecolor{currentstroke}{rgb}{0.500000,0.500000,0.500000}%
\pgfsetstrokecolor{currentstroke}%
\pgfsetstrokeopacity{0.300000}%
\pgfsetdash{}{0pt}%
\pgfpathmoveto{\pgfqpoint{2.299180in}{0.492442in}}%
\pgfpathlineto{\pgfqpoint{2.299180in}{0.492442in}}%
\pgfpathlineto{\pgfqpoint{2.265897in}{0.540960in}}%
\pgfpathlineto{\pgfqpoint{2.233002in}{0.589557in}}%
\pgfpathlineto{\pgfqpoint{2.200478in}{0.638228in}}%
\pgfpathlineto{\pgfqpoint{2.168304in}{0.686969in}}%
\pgfpathlineto{\pgfqpoint{2.136456in}{0.735773in}}%
\pgfpathlineto{\pgfqpoint{2.104915in}{0.784636in}}%
\pgfpathlineto{\pgfqpoint{2.073666in}{0.833555in}}%
\pgfusepath{stroke}%
\end{pgfscope}%
\begin{pgfscope}%
\pgfpathrectangle{\pgfqpoint{0.647939in}{0.492442in}}{\pgfqpoint{4.273799in}{2.331163in}}%
\pgfusepath{clip}%
\pgfsetbuttcap%
\pgfsetroundjoin%
\pgfsetlinewidth{0.301125pt}%
\definecolor{currentstroke}{rgb}{0.500000,0.500000,0.500000}%
\pgfsetstrokecolor{currentstroke}%
\pgfsetstrokeopacity{0.300000}%
\pgfsetdash{}{0pt}%
\pgfpathmoveto{\pgfqpoint{2.396312in}{0.492442in}}%
\pgfpathlineto{\pgfqpoint{2.396312in}{0.492442in}}%
\pgfpathlineto{\pgfqpoint{2.362621in}{0.540876in}}%
\pgfpathlineto{\pgfqpoint{2.329417in}{0.589411in}}%
\pgfpathlineto{\pgfqpoint{2.296683in}{0.638040in}}%
\pgfpathlineto{\pgfqpoint{2.264401in}{0.686759in}}%
\pgfpathlineto{\pgfqpoint{2.232561in}{0.735564in}}%
\pgfpathlineto{\pgfqpoint{2.201152in}{0.784453in}}%
\pgfpathlineto{\pgfqpoint{2.170159in}{0.833420in}}%
\pgfpathlineto{\pgfqpoint{2.139564in}{0.882462in}}%
\pgfpathlineto{\pgfqpoint{2.109361in}{0.931576in}}%
\pgfpathlineto{\pgfqpoint{2.079537in}{0.980759in}}%
\pgfpathlineto{\pgfqpoint{2.050075in}{1.030007in}}%
\pgfpathlineto{\pgfqpoint{2.020967in}{1.079317in}}%
\pgfpathlineto{\pgfqpoint{1.992203in}{1.128688in}}%
\pgfpathlineto{\pgfqpoint{1.963765in}{1.178114in}}%
\pgfpathlineto{\pgfqpoint{1.935644in}{1.227595in}}%
\pgfpathlineto{\pgfqpoint{1.907831in}{1.277127in}}%
\pgfpathlineto{\pgfqpoint{1.880306in}{1.326706in}}%
\pgfpathlineto{\pgfqpoint{1.853062in}{1.376332in}}%
\pgfpathlineto{\pgfqpoint{1.826081in}{1.426000in}}%
\pgfpathlineto{\pgfqpoint{1.799338in}{1.475706in}}%
\pgfpathlineto{\pgfqpoint{1.772826in}{1.525449in}}%
\pgfpathlineto{\pgfqpoint{1.746507in}{1.575222in}}%
\pgfpathlineto{\pgfqpoint{1.720342in}{1.625017in}}%
\pgfpathlineto{\pgfqpoint{1.694282in}{1.674828in}}%
\pgfpathlineto{\pgfqpoint{1.668222in}{1.724633in}}%
\pgfpathlineto{\pgfqpoint{1.642005in}{1.774414in}}%
\pgfpathlineto{\pgfqpoint{1.615209in}{1.824082in}}%
\pgfpathlineto{\pgfqpoint{1.586495in}{1.873289in}}%
\pgfpathlineto{\pgfqpoint{1.586495in}{1.873289in}}%
\pgfpathlineto{\pgfqpoint{1.557189in}{1.910177in}}%
\pgfpathlineto{\pgfqpoint{1.557189in}{1.910177in}}%
\pgfpathlineto{\pgfqpoint{1.557189in}{1.910177in}}%
\pgfpathlineto{\pgfqpoint{1.544172in}{1.916413in}}%
\pgfpathlineto{\pgfqpoint{1.532025in}{1.915076in}}%
\pgfpathlineto{\pgfqpoint{1.519171in}{1.907965in}}%
\pgfpathlineto{\pgfqpoint{1.501879in}{1.893106in}}%
\pgfpathlineto{\pgfqpoint{1.477796in}{1.867860in}}%
\pgfpathlineto{\pgfqpoint{1.438611in}{1.821967in}}%
\pgfpathlineto{\pgfqpoint{1.400705in}{1.774993in}}%
\pgfpathlineto{\pgfqpoint{1.363884in}{1.727522in}}%
\pgfpathlineto{\pgfqpoint{1.327997in}{1.679713in}}%
\pgfpathlineto{\pgfqpoint{1.292942in}{1.631634in}}%
\pgfpathlineto{\pgfqpoint{1.258704in}{1.583353in}}%
\pgfpathlineto{\pgfqpoint{1.225277in}{1.534920in}}%
\pgfpathlineto{\pgfqpoint{1.192576in}{1.486338in}}%
\pgfpathlineto{\pgfqpoint{1.160518in}{1.437607in}}%
\pgfusepath{stroke}%
\end{pgfscope}%
\begin{pgfscope}%
\pgfpathrectangle{\pgfqpoint{0.647939in}{0.492442in}}{\pgfqpoint{4.273799in}{2.331163in}}%
\pgfusepath{clip}%
\pgfsetbuttcap%
\pgfsetroundjoin%
\pgfsetlinewidth{0.301125pt}%
\definecolor{currentstroke}{rgb}{0.500000,0.500000,0.500000}%
\pgfsetstrokecolor{currentstroke}%
\pgfsetstrokeopacity{0.300000}%
\pgfsetdash{}{0pt}%
\pgfpathmoveto{\pgfqpoint{2.493443in}{0.492442in}}%
\pgfpathlineto{\pgfqpoint{2.493443in}{0.492442in}}%
\pgfpathlineto{\pgfqpoint{2.459068in}{0.540733in}}%
\pgfpathlineto{\pgfqpoint{2.425265in}{0.589144in}}%
\pgfpathlineto{\pgfqpoint{2.392020in}{0.637670in}}%
\pgfpathlineto{\pgfqpoint{2.359323in}{0.686306in}}%
\pgfpathlineto{\pgfqpoint{2.327166in}{0.735050in}}%
\pgfpathlineto{\pgfqpoint{2.295540in}{0.783896in}}%
\pgfpathlineto{\pgfqpoint{2.264431in}{0.832842in}}%
\pgfpathlineto{\pgfqpoint{2.233835in}{0.881884in}}%
\pgfpathlineto{\pgfqpoint{2.203751in}{0.931019in}}%
\pgfpathlineto{\pgfqpoint{2.174169in}{0.980246in}}%
\pgfpathlineto{\pgfqpoint{2.145084in}{1.029560in}}%
\pgfpathlineto{\pgfqpoint{2.116505in}{1.078962in}}%
\pgfpathlineto{\pgfqpoint{2.088428in}{1.128450in}}%
\pgfpathlineto{\pgfqpoint{2.060856in}{1.178022in}}%
\pgfpathlineto{\pgfqpoint{2.033805in}{1.227680in}}%
\pgfpathlineto{\pgfqpoint{2.007285in}{1.277423in}}%
\pgfpathlineto{\pgfqpoint{1.981317in}{1.327252in}}%
\pgfpathlineto{\pgfqpoint{1.955931in}{1.377170in}}%
\pgfpathlineto{\pgfqpoint{1.931159in}{1.427179in}}%
\pgfpathlineto{\pgfqpoint{1.907059in}{1.477287in}}%
\pgfpathlineto{\pgfqpoint{1.883696in}{1.527497in}}%
\pgfpathlineto{\pgfqpoint{1.861171in}{1.577822in}}%
\pgfpathlineto{\pgfqpoint{1.839610in}{1.628271in}}%
\pgfpathlineto{\pgfqpoint{1.819199in}{1.678861in}}%
\pgfpathlineto{\pgfqpoint{1.800201in}{1.729614in}}%
\pgfpathlineto{\pgfqpoint{1.782995in}{1.780555in}}%
\pgfpathlineto{\pgfqpoint{1.768158in}{1.831716in}}%
\pgfpathlineto{\pgfqpoint{1.756537in}{1.883119in}}%
\pgfpathlineto{\pgfqpoint{1.749392in}{1.934753in}}%
\pgfpathlineto{\pgfqpoint{1.748512in}{1.986508in}}%
\pgfpathlineto{\pgfqpoint{1.755990in}{2.038068in}}%
\pgfpathlineto{\pgfqpoint{1.773388in}{2.088883in}}%
\pgfpathlineto{\pgfqpoint{1.800847in}{2.138347in}}%
\pgfpathlineto{\pgfqpoint{1.837071in}{2.186116in}}%
\pgfpathlineto{\pgfqpoint{1.880410in}{2.232107in}}%
\pgfpathlineto{\pgfqpoint{1.929619in}{2.276313in}}%
\pgfpathlineto{\pgfqpoint{1.984015in}{2.318678in}}%
\pgfpathlineto{\pgfqpoint{2.043453in}{2.358982in}}%
\pgfpathlineto{\pgfqpoint{2.108151in}{2.396810in}}%
\pgfpathlineto{\pgfqpoint{2.178448in}{2.431503in}}%
\pgfpathlineto{\pgfqpoint{2.254843in}{2.462062in}}%
\pgfpathlineto{\pgfqpoint{2.337663in}{2.487046in}}%
\pgfpathlineto{\pgfqpoint{2.426542in}{2.504500in}}%
\pgfpathlineto{\pgfqpoint{2.519719in}{2.512356in}}%
\pgfpathlineto{\pgfqpoint{2.608306in}{2.509838in}}%
\pgfpathlineto{\pgfqpoint{2.691654in}{2.498330in}}%
\pgfpathlineto{\pgfqpoint{2.771929in}{2.478514in}}%
\pgfpathlineto{\pgfqpoint{2.850714in}{2.450259in}}%
\pgfpathlineto{\pgfqpoint{2.922352in}{2.416453in}}%
\pgfpathlineto{\pgfqpoint{2.986984in}{2.378650in}}%
\pgfpathlineto{\pgfqpoint{3.045268in}{2.337848in}}%
\pgfpathlineto{\pgfqpoint{3.097731in}{2.294742in}}%
\pgfpathlineto{\pgfqpoint{3.144759in}{2.249788in}}%
\pgfpathlineto{\pgfqpoint{3.186524in}{2.203305in}}%
\pgfpathlineto{\pgfqpoint{3.222956in}{2.155508in}}%
\pgfpathlineto{\pgfqpoint{3.253661in}{2.106536in}}%
\pgfpathlineto{\pgfqpoint{3.277710in}{2.056476in}}%
\pgfpathlineto{\pgfqpoint{3.293189in}{2.005455in}}%
\pgfpathlineto{\pgfqpoint{3.295644in}{1.953877in}}%
\pgfpathlineto{\pgfqpoint{3.295644in}{1.953877in}}%
\pgfpathlineto{\pgfqpoint{3.283718in}{1.918728in}}%
\pgfpathlineto{\pgfqpoint{3.283718in}{1.918728in}}%
\pgfpathlineto{\pgfqpoint{3.263809in}{1.898548in}}%
\pgfpathlineto{\pgfqpoint{3.263809in}{1.898548in}}%
\pgfpathlineto{\pgfqpoint{3.238378in}{1.889149in}}%
\pgfpathlineto{\pgfqpoint{3.208017in}{1.889662in}}%
\pgfpathlineto{\pgfqpoint{3.183104in}{1.896829in}}%
\pgfpathlineto{\pgfqpoint{3.156502in}{1.911292in}}%
\pgfpathlineto{\pgfqpoint{3.132161in}{1.933194in}}%
\pgfpathlineto{\pgfqpoint{3.116416in}{1.962669in}}%
\pgfpathlineto{\pgfqpoint{3.116416in}{1.962669in}}%
\pgfpathlineto{\pgfqpoint{3.119001in}{1.976857in}}%
\pgfpathlineto{\pgfqpoint{3.119001in}{1.976857in}}%
\pgfpathlineto{\pgfqpoint{3.127557in}{1.978973in}}%
\pgfpathlineto{\pgfqpoint{3.134592in}{1.975818in}}%
\pgfpathlineto{\pgfqpoint{3.139430in}{1.968824in}}%
\pgfpathlineto{\pgfqpoint{3.135355in}{1.968380in}}%
\pgfpathlineto{\pgfqpoint{3.137451in}{1.969448in}}%
\pgfpathlineto{\pgfqpoint{3.135150in}{1.968401in}}%
\pgfpathlineto{\pgfqpoint{3.139861in}{1.968676in}}%
\pgfpathlineto{\pgfqpoint{3.139861in}{1.968676in}}%
\pgfpathlineto{\pgfqpoint{3.135316in}{1.968304in}}%
\pgfpathlineto{\pgfqpoint{3.137091in}{1.969576in}}%
\pgfpathlineto{\pgfqpoint{3.135798in}{1.968031in}}%
\pgfpathlineto{\pgfqpoint{3.138903in}{1.969444in}}%
\pgfpathlineto{\pgfqpoint{3.138903in}{1.969444in}}%
\pgfpathlineto{\pgfqpoint{3.134773in}{1.968309in}}%
\pgfpathlineto{\pgfqpoint{3.138274in}{1.969458in}}%
\pgfpathlineto{\pgfqpoint{3.134798in}{1.968445in}}%
\pgfpathlineto{\pgfqpoint{3.139812in}{1.968743in}}%
\pgfpathlineto{\pgfqpoint{3.139812in}{1.968743in}}%
\pgfpathlineto{\pgfqpoint{3.135298in}{1.968294in}}%
\pgfpathlineto{\pgfqpoint{3.137119in}{1.969583in}}%
\pgfpathlineto{\pgfqpoint{3.135767in}{1.968039in}}%
\pgfpathlineto{\pgfqpoint{3.138951in}{1.969420in}}%
\pgfpathlineto{\pgfqpoint{3.138951in}{1.969420in}}%
\pgfpathlineto{\pgfqpoint{3.134815in}{1.968303in}}%
\pgfpathlineto{\pgfqpoint{3.138170in}{1.969481in}}%
\pgfpathlineto{\pgfqpoint{3.134864in}{1.968419in}}%
\pgfpathlineto{\pgfqpoint{3.140066in}{1.968650in}}%
\pgfpathlineto{\pgfqpoint{3.140066in}{1.968650in}}%
\pgfpathlineto{\pgfqpoint{3.135495in}{1.968148in}}%
\pgfpathlineto{\pgfqpoint{3.136854in}{1.969627in}}%
\pgfpathlineto{\pgfqpoint{3.136219in}{1.967836in}}%
\pgfpathlineto{\pgfqpoint{3.138257in}{1.969887in}}%
\pgfpathlineto{\pgfqpoint{3.133001in}{1.968332in}}%
\pgfpathlineto{\pgfqpoint{3.133001in}{1.968332in}}%
\pgfpathlineto{\pgfqpoint{3.137214in}{1.970370in}}%
\pgfpathlineto{\pgfqpoint{3.137148in}{1.967853in}}%
\pgfpathlineto{\pgfqpoint{3.136621in}{1.970301in}}%
\pgfpathlineto{\pgfqpoint{3.136246in}{1.966772in}}%
\pgfpathlineto{\pgfqpoint{3.138704in}{1.971264in}}%
\pgfpathlineto{\pgfqpoint{3.138704in}{1.971264in}}%
\pgfusepath{stroke}%
\end{pgfscope}%
\begin{pgfscope}%
\pgfpathrectangle{\pgfqpoint{0.647939in}{0.492442in}}{\pgfqpoint{4.273799in}{2.331163in}}%
\pgfusepath{clip}%
\pgfsetbuttcap%
\pgfsetroundjoin%
\pgfsetlinewidth{0.301125pt}%
\definecolor{currentstroke}{rgb}{0.500000,0.500000,0.500000}%
\pgfsetstrokecolor{currentstroke}%
\pgfsetstrokeopacity{0.300000}%
\pgfsetdash{}{0pt}%
\pgfpathmoveto{\pgfqpoint{2.590575in}{0.492442in}}%
\pgfpathlineto{\pgfqpoint{2.590575in}{0.492442in}}%
\pgfpathlineto{\pgfqpoint{2.555248in}{0.540528in}}%
\pgfpathlineto{\pgfqpoint{2.520576in}{0.588756in}}%
\pgfpathlineto{\pgfqpoint{2.486548in}{0.637120in}}%
\pgfpathlineto{\pgfqpoint{2.453157in}{0.685616in}}%
\pgfpathlineto{\pgfqpoint{2.420396in}{0.734239in}}%
\pgfpathlineto{\pgfqpoint{2.388253in}{0.782985in}}%
\pgfpathlineto{\pgfqpoint{2.356720in}{0.831850in}}%
\pgfpathlineto{\pgfqpoint{2.325797in}{0.880830in}}%
\pgfpathlineto{\pgfqpoint{2.295482in}{0.929923in}}%
\pgfpathlineto{\pgfqpoint{2.265770in}{0.979126in}}%
\pgfpathlineto{\pgfqpoint{2.236664in}{1.028437in}}%
\pgfpathlineto{\pgfqpoint{2.208174in}{1.077855in}}%
\pgfpathlineto{\pgfqpoint{2.180304in}{1.127377in}}%
\pgfusepath{stroke}%
\end{pgfscope}%
\begin{pgfscope}%
\pgfpathrectangle{\pgfqpoint{0.647939in}{0.492442in}}{\pgfqpoint{4.273799in}{2.331163in}}%
\pgfusepath{clip}%
\pgfsetbuttcap%
\pgfsetroundjoin%
\pgfsetlinewidth{0.301125pt}%
\definecolor{currentstroke}{rgb}{0.500000,0.500000,0.500000}%
\pgfsetstrokecolor{currentstroke}%
\pgfsetstrokeopacity{0.300000}%
\pgfsetdash{}{0pt}%
\pgfpathmoveto{\pgfqpoint{2.687707in}{0.492442in}}%
\pgfpathlineto{\pgfqpoint{2.687707in}{0.492442in}}%
\pgfpathlineto{\pgfqpoint{2.651173in}{0.540259in}}%
\pgfpathlineto{\pgfqpoint{2.615378in}{0.588241in}}%
\pgfpathlineto{\pgfqpoint{2.580313in}{0.636384in}}%
\pgfpathlineto{\pgfqpoint{2.545970in}{0.684682in}}%
\pgfpathlineto{\pgfqpoint{2.512338in}{0.733128in}}%
\pgfpathlineto{\pgfqpoint{2.479409in}{0.781718in}}%
\pgfpathlineto{\pgfqpoint{2.447178in}{0.830446in}}%
\pgfpathlineto{\pgfqpoint{2.415645in}{0.879311in}}%
\pgfpathlineto{\pgfqpoint{2.384806in}{0.928307in}}%
\pgfpathlineto{\pgfqpoint{2.354659in}{0.977431in}}%
\pgfpathlineto{\pgfqpoint{2.325211in}{1.026681in}}%
\pgfpathlineto{\pgfqpoint{2.296472in}{1.076055in}}%
\pgfpathlineto{\pgfqpoint{2.268448in}{1.125552in}}%
\pgfpathlineto{\pgfqpoint{2.241158in}{1.175170in}}%
\pgfpathlineto{\pgfqpoint{2.214625in}{1.224911in}}%
\pgfpathlineto{\pgfqpoint{2.188872in}{1.274773in}}%
\pgfpathlineto{\pgfqpoint{2.163942in}{1.324759in}}%
\pgfpathlineto{\pgfqpoint{2.139876in}{1.374871in}}%
\pgfpathlineto{\pgfqpoint{2.116736in}{1.425113in}}%
\pgfpathlineto{\pgfqpoint{2.094592in}{1.475488in}}%
\pgfpathlineto{\pgfqpoint{2.073535in}{1.526001in}}%
\pgfpathlineto{\pgfqpoint{2.053678in}{1.576658in}}%
\pgfpathlineto{\pgfqpoint{2.035162in}{1.627466in}}%
\pgfpathlineto{\pgfqpoint{2.018169in}{1.678431in}}%
\pgfpathlineto{\pgfqpoint{2.002922in}{1.729560in}}%
\pgfpathlineto{\pgfqpoint{1.989710in}{1.780855in}}%
\pgfpathlineto{\pgfqpoint{1.978890in}{1.832316in}}%
\pgfpathlineto{\pgfqpoint{1.970906in}{1.883929in}}%
\pgfpathlineto{\pgfqpoint{1.966314in}{1.935661in}}%
\pgfpathlineto{\pgfqpoint{1.965784in}{1.987447in}}%
\pgfpathlineto{\pgfqpoint{1.970090in}{2.039173in}}%
\pgfpathlineto{\pgfqpoint{1.980069in}{2.090659in}}%
\pgfpathlineto{\pgfqpoint{1.996549in}{2.141637in}}%
\pgfpathlineto{\pgfqpoint{2.020266in}{2.191746in}}%
\pgfpathlineto{\pgfqpoint{2.051740in}{2.240550in}}%
\pgfpathlineto{\pgfqpoint{2.091301in}{2.287557in}}%
\pgfpathlineto{\pgfqpoint{2.139147in}{2.332200in}}%
\pgfpathlineto{\pgfqpoint{2.195471in}{2.373789in}}%
\pgfpathlineto{\pgfqpoint{2.260490in}{2.411351in}}%
\pgfusepath{stroke}%
\end{pgfscope}%
\begin{pgfscope}%
\pgfpathrectangle{\pgfqpoint{0.647939in}{0.492442in}}{\pgfqpoint{4.273799in}{2.331163in}}%
\pgfusepath{clip}%
\pgfsetbuttcap%
\pgfsetroundjoin%
\pgfsetlinewidth{0.301125pt}%
\definecolor{currentstroke}{rgb}{0.500000,0.500000,0.500000}%
\pgfsetstrokecolor{currentstroke}%
\pgfsetstrokeopacity{0.300000}%
\pgfsetdash{}{0pt}%
\pgfpathmoveto{\pgfqpoint{2.881971in}{0.492442in}}%
\pgfpathlineto{\pgfqpoint{2.881971in}{0.492442in}}%
\pgfpathlineto{\pgfqpoint{2.842315in}{0.539513in}}%
\pgfpathlineto{\pgfqpoint{2.803559in}{0.586806in}}%
\pgfpathlineto{\pgfqpoint{2.765696in}{0.634314in}}%
\pgfpathlineto{\pgfqpoint{2.728718in}{0.682029in}}%
\pgfpathlineto{\pgfqpoint{2.692616in}{0.729942in}}%
\pgfpathlineto{\pgfqpoint{2.657385in}{0.778049in}}%
\pgfpathlineto{\pgfqpoint{2.623019in}{0.826341in}}%
\pgfpathlineto{\pgfqpoint{2.589516in}{0.874814in}}%
\pgfpathlineto{\pgfqpoint{2.556874in}{0.923460in}}%
\pgfpathlineto{\pgfqpoint{2.525088in}{0.972276in}}%
\pgfpathlineto{\pgfqpoint{2.494165in}{1.021256in}}%
\pgfpathlineto{\pgfqpoint{2.464115in}{1.070397in}}%
\pgfpathlineto{\pgfqpoint{2.434945in}{1.119696in}}%
\pgfpathlineto{\pgfqpoint{2.406670in}{1.169150in}}%
\pgfpathlineto{\pgfqpoint{2.379316in}{1.218757in}}%
\pgfpathlineto{\pgfqpoint{2.352907in}{1.268517in}}%
\pgfpathlineto{\pgfqpoint{2.327480in}{1.318429in}}%
\pgfpathlineto{\pgfqpoint{2.303081in}{1.368493in}}%
\pgfpathlineto{\pgfqpoint{2.279764in}{1.418710in}}%
\pgfpathlineto{\pgfqpoint{2.257599in}{1.469081in}}%
\pgfpathlineto{\pgfqpoint{2.236671in}{1.519610in}}%
\pgfpathlineto{\pgfqpoint{2.217083in}{1.570298in}}%
\pgfpathlineto{\pgfqpoint{2.198964in}{1.621148in}}%
\pgfpathlineto{\pgfqpoint{2.182468in}{1.672162in}}%
\pgfpathlineto{\pgfqpoint{2.167782in}{1.723340in}}%
\pgfpathlineto{\pgfqpoint{2.155143in}{1.774678in}}%
\pgfpathlineto{\pgfqpoint{2.144832in}{1.826170in}}%
\pgfpathlineto{\pgfqpoint{2.137191in}{1.877799in}}%
\pgfpathlineto{\pgfqpoint{2.132643in}{1.929535in}}%
\pgfpathlineto{\pgfqpoint{2.131696in}{1.981324in}}%
\pgfpathlineto{\pgfqpoint{2.134950in}{2.033080in}}%
\pgfpathlineto{\pgfqpoint{2.143112in}{2.084668in}}%
\pgfpathlineto{\pgfqpoint{2.156985in}{2.135883in}}%
\pgfpathlineto{\pgfqpoint{2.177473in}{2.186424in}}%
\pgfpathlineto{\pgfqpoint{2.205550in}{2.235854in}}%
\pgfpathlineto{\pgfqpoint{2.242265in}{2.283545in}}%
\pgfpathlineto{\pgfqpoint{2.288716in}{2.328604in}}%
\pgfpathlineto{\pgfqpoint{2.346032in}{2.369697in}}%
\pgfpathlineto{\pgfqpoint{2.415159in}{2.404798in}}%
\pgfpathlineto{\pgfqpoint{2.496075in}{2.431032in}}%
\pgfpathlineto{\pgfqpoint{2.576326in}{2.444246in}}%
\pgfpathlineto{\pgfqpoint{2.653024in}{2.446342in}}%
\pgfpathlineto{\pgfqpoint{2.726599in}{2.439253in}}%
\pgfpathlineto{\pgfqpoint{2.798563in}{2.423850in}}%
\pgfusepath{stroke}%
\end{pgfscope}%
\begin{pgfscope}%
\pgfpathrectangle{\pgfqpoint{0.647939in}{0.492442in}}{\pgfqpoint{4.273799in}{2.331163in}}%
\pgfusepath{clip}%
\pgfsetbuttcap%
\pgfsetroundjoin%
\pgfsetlinewidth{0.301125pt}%
\definecolor{currentstroke}{rgb}{0.500000,0.500000,0.500000}%
\pgfsetstrokecolor{currentstroke}%
\pgfsetstrokeopacity{0.300000}%
\pgfsetdash{}{0pt}%
\pgfpathmoveto{\pgfqpoint{2.979102in}{0.492442in}}%
\pgfpathlineto{\pgfqpoint{2.979102in}{0.492442in}}%
\pgfpathlineto{\pgfqpoint{2.937563in}{0.539027in}}%
\pgfpathlineto{\pgfqpoint{2.896995in}{0.585866in}}%
\pgfpathlineto{\pgfqpoint{2.857396in}{0.632950in}}%
\pgfpathlineto{\pgfqpoint{2.818761in}{0.680272in}}%
\pgfpathlineto{\pgfqpoint{2.781086in}{0.727824in}}%
\pgfpathlineto{\pgfqpoint{2.744363in}{0.775597in}}%
\pgfpathlineto{\pgfqpoint{2.708590in}{0.823584in}}%
\pgfpathlineto{\pgfqpoint{2.673762in}{0.871778in}}%
\pgfpathlineto{\pgfqpoint{2.639877in}{0.920171in}}%
\pgfpathlineto{\pgfqpoint{2.606931in}{0.968757in}}%
\pgfpathlineto{\pgfqpoint{2.574927in}{1.017530in}}%
\pgfpathlineto{\pgfqpoint{2.543875in}{1.066485in}}%
\pgfpathlineto{\pgfqpoint{2.513781in}{1.115618in}}%
\pgfpathlineto{\pgfqpoint{2.484659in}{1.164925in}}%
\pgfpathlineto{\pgfqpoint{2.456529in}{1.214403in}}%
\pgfusepath{stroke}%
\end{pgfscope}%
\begin{pgfscope}%
\pgfpathrectangle{\pgfqpoint{0.647939in}{0.492442in}}{\pgfqpoint{4.273799in}{2.331163in}}%
\pgfusepath{clip}%
\pgfsetbuttcap%
\pgfsetroundjoin%
\pgfsetlinewidth{0.301125pt}%
\definecolor{currentstroke}{rgb}{0.500000,0.500000,0.500000}%
\pgfsetstrokecolor{currentstroke}%
\pgfsetstrokeopacity{0.300000}%
\pgfsetdash{}{0pt}%
\pgfpathmoveto{\pgfqpoint{3.076234in}{0.492442in}}%
\pgfpathlineto{\pgfqpoint{3.076234in}{0.492442in}}%
\pgfpathlineto{\pgfqpoint{3.032636in}{0.538463in}}%
\pgfpathlineto{\pgfqpoint{2.990067in}{0.584770in}}%
\pgfpathlineto{\pgfqpoint{2.948535in}{0.631356in}}%
\pgfpathlineto{\pgfqpoint{2.908040in}{0.678212in}}%
\pgfpathlineto{\pgfqpoint{2.868582in}{0.725332in}}%
\pgfusepath{stroke}%
\end{pgfscope}%
\begin{pgfscope}%
\pgfpathrectangle{\pgfqpoint{0.647939in}{0.492442in}}{\pgfqpoint{4.273799in}{2.331163in}}%
\pgfusepath{clip}%
\pgfsetbuttcap%
\pgfsetroundjoin%
\pgfsetlinewidth{0.301125pt}%
\definecolor{currentstroke}{rgb}{0.500000,0.500000,0.500000}%
\pgfsetstrokecolor{currentstroke}%
\pgfsetstrokeopacity{0.300000}%
\pgfsetdash{}{0pt}%
\pgfpathmoveto{\pgfqpoint{3.270498in}{0.492442in}}%
\pgfpathlineto{\pgfqpoint{3.270498in}{0.492442in}}%
\pgfpathlineto{\pgfqpoint{3.222481in}{0.537135in}}%
\pgfpathlineto{\pgfqpoint{3.175524in}{0.582161in}}%
\pgfpathlineto{\pgfqpoint{3.129672in}{0.627525in}}%
\pgfpathlineto{\pgfqpoint{3.084954in}{0.673224in}}%
\pgfpathlineto{\pgfqpoint{3.041394in}{0.719255in}}%
\pgfpathlineto{\pgfqpoint{2.999007in}{0.765611in}}%
\pgfpathlineto{\pgfqpoint{2.957802in}{0.812282in}}%
\pgfpathlineto{\pgfqpoint{2.917783in}{0.859260in}}%
\pgfpathlineto{\pgfqpoint{2.878954in}{0.906534in}}%
\pgfpathlineto{\pgfqpoint{2.841316in}{0.954094in}}%
\pgfpathlineto{\pgfqpoint{2.804870in}{1.001929in}}%
\pgfpathlineto{\pgfqpoint{2.769618in}{1.050031in}}%
\pgfpathlineto{\pgfqpoint{2.735564in}{1.098388in}}%
\pgfpathlineto{\pgfqpoint{2.702716in}{1.146992in}}%
\pgfpathlineto{\pgfqpoint{2.671088in}{1.195837in}}%
\pgfpathlineto{\pgfqpoint{2.640698in}{1.244915in}}%
\pgfpathlineto{\pgfqpoint{2.611569in}{1.294221in}}%
\pgfpathlineto{\pgfqpoint{2.583735in}{1.343748in}}%
\pgfpathlineto{\pgfqpoint{2.557241in}{1.393493in}}%
\pgfpathlineto{\pgfqpoint{2.532137in}{1.443452in}}%
\pgfpathlineto{\pgfqpoint{2.508495in}{1.493623in}}%
\pgfpathlineto{\pgfqpoint{2.486398in}{1.544002in}}%
\pgfpathlineto{\pgfqpoint{2.465949in}{1.594588in}}%
\pgfpathlineto{\pgfqpoint{2.447278in}{1.645378in}}%
\pgfpathlineto{\pgfqpoint{2.430540in}{1.696367in}}%
\pgfpathlineto{\pgfqpoint{2.415932in}{1.747550in}}%
\pgfpathlineto{\pgfqpoint{2.403691in}{1.798916in}}%
\pgfpathlineto{\pgfqpoint{2.394108in}{1.850449in}}%
\pgfpathlineto{\pgfqpoint{2.387554in}{1.902120in}}%
\pgfpathlineto{\pgfqpoint{2.384487in}{1.953886in}}%
\pgfpathlineto{\pgfqpoint{2.385487in}{2.005671in}}%
\pgfpathlineto{\pgfqpoint{2.391292in}{2.057355in}}%
\pgfpathlineto{\pgfqpoint{2.402866in}{2.108737in}}%
\pgfpathlineto{\pgfqpoint{2.421465in}{2.159486in}}%
\pgfpathlineto{\pgfqpoint{2.448813in}{2.209011in}}%
\pgfpathlineto{\pgfqpoint{2.487244in}{2.256228in}}%
\pgfpathlineto{\pgfqpoint{2.539811in}{2.299021in}}%
\pgfpathlineto{\pgfqpoint{2.539811in}{2.299021in}}%
\pgfpathlineto{\pgfqpoint{2.595564in}{2.328118in}}%
\pgfpathlineto{\pgfqpoint{2.664210in}{2.348095in}}%
\pgfpathlineto{\pgfqpoint{2.730171in}{2.354809in}}%
\pgfpathlineto{\pgfqpoint{2.792635in}{2.351618in}}%
\pgfpathlineto{\pgfqpoint{2.853714in}{2.340256in}}%
\pgfpathlineto{\pgfqpoint{2.915301in}{2.320793in}}%
\pgfpathlineto{\pgfqpoint{2.977931in}{2.292585in}}%
\pgfpathlineto{\pgfqpoint{3.041550in}{2.254526in}}%
\pgfpathlineto{\pgfqpoint{3.096797in}{2.212562in}}%
\pgfusepath{stroke}%
\end{pgfscope}%
\begin{pgfscope}%
\pgfpathrectangle{\pgfqpoint{0.647939in}{0.492442in}}{\pgfqpoint{4.273799in}{2.331163in}}%
\pgfusepath{clip}%
\pgfsetbuttcap%
\pgfsetroundjoin%
\pgfsetlinewidth{0.301125pt}%
\definecolor{currentstroke}{rgb}{0.500000,0.500000,0.500000}%
\pgfsetstrokecolor{currentstroke}%
\pgfsetstrokeopacity{0.300000}%
\pgfsetdash{}{0pt}%
\pgfpathmoveto{\pgfqpoint{3.464761in}{0.492442in}}%
\pgfpathlineto{\pgfqpoint{3.464761in}{0.492442in}}%
\pgfpathlineto{\pgfqpoint{3.412533in}{0.535708in}}%
\pgfpathlineto{\pgfqpoint{3.361170in}{0.579281in}}%
\pgfpathlineto{\pgfqpoint{3.310792in}{0.623194in}}%
\pgfpathlineto{\pgfqpoint{3.261496in}{0.667470in}}%
\pgfpathlineto{\pgfqpoint{3.213359in}{0.712123in}}%
\pgfpathlineto{\pgfqpoint{3.166441in}{0.757161in}}%
\pgfpathlineto{\pgfqpoint{3.120789in}{0.802584in}}%
\pgfpathlineto{\pgfqpoint{3.076436in}{0.848389in}}%
\pgfpathlineto{\pgfqpoint{3.033409in}{0.894568in}}%
\pgfpathlineto{\pgfqpoint{2.991723in}{0.941112in}}%
\pgfpathlineto{\pgfqpoint{2.951390in}{0.988009in}}%
\pgfpathlineto{\pgfqpoint{2.912417in}{1.035246in}}%
\pgfpathlineto{\pgfqpoint{2.874810in}{1.082813in}}%
\pgfpathlineto{\pgfqpoint{2.838577in}{1.130695in}}%
\pgfpathlineto{\pgfqpoint{2.803728in}{1.178883in}}%
\pgfpathlineto{\pgfqpoint{2.770276in}{1.227365in}}%
\pgfpathlineto{\pgfqpoint{2.738237in}{1.276129in}}%
\pgfpathlineto{\pgfqpoint{2.707638in}{1.325168in}}%
\pgfpathlineto{\pgfqpoint{2.678515in}{1.374473in}}%
\pgfpathlineto{\pgfqpoint{2.650912in}{1.424038in}}%
\pgfpathlineto{\pgfqpoint{2.624884in}{1.473856in}}%
\pgfpathlineto{\pgfqpoint{2.600509in}{1.523921in}}%
\pgfpathlineto{\pgfqpoint{2.577876in}{1.574229in}}%
\pgfpathlineto{\pgfqpoint{2.557105in}{1.624776in}}%
\pgfusepath{stroke}%
\end{pgfscope}%
\begin{pgfscope}%
\pgfpathrectangle{\pgfqpoint{0.647939in}{0.492442in}}{\pgfqpoint{4.273799in}{2.331163in}}%
\pgfusepath{clip}%
\pgfsetbuttcap%
\pgfsetroundjoin%
\pgfsetlinewidth{0.301125pt}%
\definecolor{currentstroke}{rgb}{0.500000,0.500000,0.500000}%
\pgfsetstrokecolor{currentstroke}%
\pgfsetstrokeopacity{0.300000}%
\pgfsetdash{}{0pt}%
\pgfpathmoveto{\pgfqpoint{3.659025in}{0.492442in}}%
\pgfpathlineto{\pgfqpoint{3.659025in}{0.492442in}}%
\pgfpathlineto{\pgfqpoint{3.603876in}{0.534617in}}%
\pgfpathlineto{\pgfqpoint{3.549045in}{0.576914in}}%
\pgfpathlineto{\pgfqpoint{3.494738in}{0.619412in}}%
\pgfpathlineto{\pgfqpoint{3.441158in}{0.662183in}}%
\pgfpathlineto{\pgfqpoint{3.388482in}{0.705286in}}%
\pgfpathlineto{\pgfqpoint{3.336857in}{0.748765in}}%
\pgfpathlineto{\pgfqpoint{3.286404in}{0.792652in}}%
\pgfpathlineto{\pgfqpoint{3.237228in}{0.836968in}}%
\pgfpathlineto{\pgfqpoint{3.189412in}{0.881724in}}%
\pgfpathlineto{\pgfqpoint{3.143019in}{0.926923in}}%
\pgfpathlineto{\pgfqpoint{3.098095in}{0.972561in}}%
\pgfpathlineto{\pgfqpoint{3.054672in}{1.018629in}}%
\pgfpathlineto{\pgfqpoint{3.012773in}{1.065115in}}%
\pgfpathlineto{\pgfqpoint{2.972415in}{1.112005in}}%
\pgfpathlineto{\pgfqpoint{2.933613in}{1.159285in}}%
\pgfpathlineto{\pgfqpoint{2.896380in}{1.206939in}}%
\pgfpathlineto{\pgfqpoint{2.860731in}{1.254951in}}%
\pgfusepath{stroke}%
\end{pgfscope}%
\begin{pgfscope}%
\pgfpathrectangle{\pgfqpoint{0.647939in}{0.492442in}}{\pgfqpoint{4.273799in}{2.331163in}}%
\pgfusepath{clip}%
\pgfsetbuttcap%
\pgfsetroundjoin%
\pgfsetlinewidth{0.301125pt}%
\definecolor{currentstroke}{rgb}{0.500000,0.500000,0.500000}%
\pgfsetstrokecolor{currentstroke}%
\pgfsetstrokeopacity{0.300000}%
\pgfsetdash{}{0pt}%
\pgfpathmoveto{\pgfqpoint{3.853289in}{0.492442in}}%
\pgfpathlineto{\pgfqpoint{3.853289in}{0.492442in}}%
\pgfpathlineto{\pgfqpoint{3.797720in}{0.534452in}}%
\pgfpathlineto{\pgfqpoint{3.741590in}{0.576240in}}%
\pgfpathlineto{\pgfqpoint{3.685159in}{0.617907in}}%
\pgfpathlineto{\pgfqpoint{3.628680in}{0.659555in}}%
\pgfpathlineto{\pgfqpoint{3.572429in}{0.701295in}}%
\pgfpathlineto{\pgfqpoint{3.516675in}{0.743231in}}%
\pgfpathlineto{\pgfqpoint{3.461653in}{0.785453in}}%
\pgfpathlineto{\pgfqpoint{3.407580in}{0.828039in}}%
\pgfpathlineto{\pgfqpoint{3.354653in}{0.871049in}}%
\pgfpathlineto{\pgfqpoint{3.303034in}{0.914529in}}%
\pgfpathlineto{\pgfqpoint{3.252850in}{0.958505in}}%
\pgfpathlineto{\pgfqpoint{3.204201in}{1.002991in}}%
\pgfpathlineto{\pgfqpoint{3.157165in}{1.047991in}}%
\pgfpathlineto{\pgfqpoint{3.111800in}{1.093498in}}%
\pgfpathlineto{\pgfqpoint{3.068151in}{1.139503in}}%
\pgfusepath{stroke}%
\end{pgfscope}%
\begin{pgfscope}%
\pgfpathrectangle{\pgfqpoint{0.647939in}{0.492442in}}{\pgfqpoint{4.273799in}{2.331163in}}%
\pgfusepath{clip}%
\pgfsetbuttcap%
\pgfsetroundjoin%
\pgfsetlinewidth{0.301125pt}%
\definecolor{currentstroke}{rgb}{0.500000,0.500000,0.500000}%
\pgfsetstrokecolor{currentstroke}%
\pgfsetstrokeopacity{0.300000}%
\pgfsetdash{}{0pt}%
\pgfpathmoveto{\pgfqpoint{4.047552in}{0.492442in}}%
\pgfpathlineto{\pgfqpoint{4.047552in}{0.492442in}}%
\pgfpathlineto{\pgfqpoint{3.994997in}{0.535587in}}%
\pgfpathlineto{\pgfqpoint{3.940985in}{0.578194in}}%
\pgfpathlineto{\pgfqpoint{3.885650in}{0.620294in}}%
\pgfpathlineto{\pgfqpoint{3.829207in}{0.661955in}}%
\pgfpathlineto{\pgfqpoint{3.771919in}{0.703273in}}%
\pgfpathlineto{\pgfqpoint{3.714055in}{0.744351in}}%
\pgfpathlineto{\pgfqpoint{3.655943in}{0.785325in}}%
\pgfpathlineto{\pgfqpoint{3.597923in}{0.826337in}}%
\pgfpathlineto{\pgfqpoint{3.540308in}{0.867519in}}%
\pgfpathlineto{\pgfqpoint{3.483417in}{0.908998in}}%
\pgfpathlineto{\pgfqpoint{3.427547in}{0.950886in}}%
\pgfpathlineto{\pgfqpoint{3.372933in}{0.993263in}}%
\pgfpathlineto{\pgfqpoint{3.319785in}{1.036191in}}%
\pgfpathlineto{\pgfqpoint{3.268280in}{1.079709in}}%
\pgfpathlineto{\pgfqpoint{3.218550in}{1.123836in}}%
\pgfpathlineto{\pgfqpoint{3.170694in}{1.168574in}}%
\pgfpathlineto{\pgfqpoint{3.124782in}{1.213915in}}%
\pgfpathlineto{\pgfqpoint{3.080872in}{1.259841in}}%
\pgfpathlineto{\pgfqpoint{3.039003in}{1.306331in}}%
\pgfpathlineto{\pgfqpoint{2.999212in}{1.353361in}}%
\pgfpathlineto{\pgfqpoint{2.961533in}{1.400906in}}%
\pgfpathlineto{\pgfqpoint{2.926007in}{1.448942in}}%
\pgfpathlineto{\pgfqpoint{2.892685in}{1.497447in}}%
\pgfpathlineto{\pgfqpoint{2.861637in}{1.546398in}}%
\pgfpathlineto{\pgfqpoint{2.832952in}{1.595775in}}%
\pgfpathlineto{\pgfqpoint{2.806758in}{1.645563in}}%
\pgfpathlineto{\pgfqpoint{2.783220in}{1.695744in}}%
\pgfpathlineto{\pgfqpoint{2.762558in}{1.746298in}}%
\pgfpathlineto{\pgfqpoint{2.745072in}{1.797206in}}%
\pgfpathlineto{\pgfqpoint{2.731167in}{1.848437in}}%
\pgfpathlineto{\pgfqpoint{2.721407in}{1.899950in}}%
\pgfpathlineto{\pgfqpoint{2.716598in}{1.951664in}}%
\pgfpathlineto{\pgfqpoint{2.717938in}{2.003425in}}%
\pgfpathlineto{\pgfqpoint{2.727334in}{2.054903in}}%
\pgfpathlineto{\pgfqpoint{2.748094in}{2.105283in}}%
\pgfpathlineto{\pgfqpoint{2.786459in}{2.152134in}}%
\pgfpathlineto{\pgfqpoint{2.786459in}{2.152134in}}%
\pgfpathlineto{\pgfqpoint{2.824748in}{2.176960in}}%
\pgfpathlineto{\pgfqpoint{2.824748in}{2.176960in}}%
\pgfpathlineto{\pgfqpoint{2.866655in}{2.190565in}}%
\pgfpathlineto{\pgfqpoint{2.914974in}{2.194219in}}%
\pgfpathlineto{\pgfqpoint{2.959194in}{2.188810in}}%
\pgfpathlineto{\pgfqpoint{3.003170in}{2.176052in}}%
\pgfpathlineto{\pgfqpoint{3.048611in}{2.155286in}}%
\pgfpathlineto{\pgfqpoint{3.095195in}{2.125249in}}%
\pgfpathlineto{\pgfqpoint{3.140349in}{2.085157in}}%
\pgfpathlineto{\pgfqpoint{3.177487in}{2.037777in}}%
\pgfusepath{stroke}%
\end{pgfscope}%
\begin{pgfscope}%
\pgfpathrectangle{\pgfqpoint{0.647939in}{0.492442in}}{\pgfqpoint{4.273799in}{2.331163in}}%
\pgfusepath{clip}%
\pgfsetbuttcap%
\pgfsetroundjoin%
\pgfsetlinewidth{0.301125pt}%
\definecolor{currentstroke}{rgb}{0.500000,0.500000,0.500000}%
\pgfsetstrokecolor{currentstroke}%
\pgfsetstrokeopacity{0.300000}%
\pgfsetdash{}{0pt}%
\pgfpathmoveto{\pgfqpoint{4.241816in}{0.492442in}}%
\pgfpathlineto{\pgfqpoint{4.241816in}{0.492442in}}%
\pgfpathlineto{\pgfqpoint{4.195571in}{0.537684in}}%
\pgfpathlineto{\pgfqpoint{4.147487in}{0.582353in}}%
\pgfpathlineto{\pgfqpoint{4.097509in}{0.626398in}}%
\pgfpathlineto{\pgfqpoint{4.045630in}{0.669785in}}%
\pgfpathlineto{\pgfqpoint{3.991885in}{0.712490in}}%
\pgfpathlineto{\pgfqpoint{3.936350in}{0.754509in}}%
\pgfpathlineto{\pgfqpoint{3.879183in}{0.795873in}}%
\pgfpathlineto{\pgfqpoint{3.820643in}{0.836663in}}%
\pgfpathlineto{\pgfqpoint{3.761034in}{0.876991in}}%
\pgfpathlineto{\pgfqpoint{3.700733in}{0.917012in}}%
\pgfpathlineto{\pgfqpoint{3.640167in}{0.956915in}}%
\pgfpathlineto{\pgfqpoint{3.579763in}{0.996889in}}%
\pgfpathlineto{\pgfqpoint{3.519961in}{1.037131in}}%
\pgfpathlineto{\pgfqpoint{3.461167in}{1.077810in}}%
\pgfpathlineto{\pgfqpoint{3.403734in}{1.119062in}}%
\pgfpathlineto{\pgfqpoint{3.347976in}{1.160990in}}%
\pgfpathlineto{\pgfqpoint{3.294138in}{1.203656in}}%
\pgfpathlineto{\pgfqpoint{3.242403in}{1.247089in}}%
\pgfpathlineto{\pgfqpoint{3.192917in}{1.291297in}}%
\pgfpathlineto{\pgfqpoint{3.145789in}{1.336264in}}%
\pgfpathlineto{\pgfqpoint{3.101095in}{1.381963in}}%
\pgfpathlineto{\pgfqpoint{3.058894in}{1.428361in}}%
\pgfpathlineto{\pgfqpoint{3.019238in}{1.475424in}}%
\pgfpathlineto{\pgfqpoint{2.982185in}{1.523115in}}%
\pgfpathlineto{\pgfqpoint{2.947811in}{1.571397in}}%
\pgfpathlineto{\pgfqpoint{2.916215in}{1.620239in}}%
\pgfpathlineto{\pgfqpoint{2.887534in}{1.669614in}}%
\pgfpathlineto{\pgfqpoint{2.861961in}{1.719494in}}%
\pgfpathlineto{\pgfqpoint{2.839767in}{1.769850in}}%
\pgfusepath{stroke}%
\end{pgfscope}%
\begin{pgfscope}%
\pgfpathrectangle{\pgfqpoint{0.647939in}{0.492442in}}{\pgfqpoint{4.273799in}{2.331163in}}%
\pgfusepath{clip}%
\pgfsetbuttcap%
\pgfsetroundjoin%
\pgfsetlinewidth{0.301125pt}%
\definecolor{currentstroke}{rgb}{0.500000,0.500000,0.500000}%
\pgfsetstrokecolor{currentstroke}%
\pgfsetstrokeopacity{0.300000}%
\pgfsetdash{}{0pt}%
\pgfpathmoveto{\pgfqpoint{4.436079in}{0.492442in}}%
\pgfpathlineto{\pgfqpoint{4.436079in}{0.492442in}}%
\pgfpathlineto{\pgfqpoint{4.398039in}{0.539905in}}%
\pgfpathlineto{\pgfqpoint{4.358506in}{0.587004in}}%
\pgfpathlineto{\pgfqpoint{4.317343in}{0.633686in}}%
\pgfpathlineto{\pgfqpoint{4.274396in}{0.679887in}}%
\pgfpathlineto{\pgfqpoint{4.229508in}{0.725535in}}%
\pgfpathlineto{\pgfqpoint{4.182520in}{0.770548in}}%
\pgfpathlineto{\pgfqpoint{4.133283in}{0.814839in}}%
\pgfpathlineto{\pgfqpoint{4.081667in}{0.858313in}}%
\pgfpathlineto{\pgfqpoint{4.027576in}{0.900883in}}%
\pgfpathlineto{\pgfqpoint{3.970977in}{0.942472in}}%
\pgfpathlineto{\pgfqpoint{3.911985in}{0.983062in}}%
\pgfpathlineto{\pgfqpoint{3.850833in}{1.022690in}}%
\pgfpathlineto{\pgfqpoint{3.787869in}{1.061468in}}%
\pgfpathlineto{\pgfqpoint{3.723602in}{1.099607in}}%
\pgfpathlineto{\pgfqpoint{3.658634in}{1.137392in}}%
\pgfpathlineto{\pgfqpoint{3.593629in}{1.175158in}}%
\pgfpathlineto{\pgfqpoint{3.529261in}{1.213244in}}%
\pgfpathlineto{\pgfqpoint{3.466164in}{1.251952in}}%
\pgfpathlineto{\pgfqpoint{3.404889in}{1.291519in}}%
\pgfpathlineto{\pgfqpoint{3.345902in}{1.332102in}}%
\pgfpathlineto{\pgfqpoint{3.289554in}{1.373783in}}%
\pgfpathlineto{\pgfqpoint{3.236097in}{1.416583in}}%
\pgfpathlineto{\pgfqpoint{3.185718in}{1.460480in}}%
\pgfpathlineto{\pgfqpoint{3.138551in}{1.505423in}}%
\pgfpathlineto{\pgfqpoint{3.094682in}{1.551348in}}%
\pgfpathlineto{\pgfqpoint{3.054198in}{1.598189in}}%
\pgfpathlineto{\pgfqpoint{3.017204in}{1.645884in}}%
\pgfpathlineto{\pgfqpoint{2.983849in}{1.694372in}}%
\pgfpathlineto{\pgfqpoint{2.954360in}{1.743596in}}%
\pgfpathlineto{\pgfqpoint{2.929070in}{1.793508in}}%
\pgfpathlineto{\pgfqpoint{2.908500in}{1.844056in}}%
\pgfpathlineto{\pgfqpoint{2.893469in}{1.895173in}}%
\pgfpathlineto{\pgfqpoint{2.885350in}{1.946734in}}%
\pgfpathlineto{\pgfqpoint{2.886667in}{1.998438in}}%
\pgfpathlineto{\pgfqpoint{2.902897in}{2.049196in}}%
\pgfpathlineto{\pgfqpoint{2.902897in}{2.049196in}}%
\pgfpathlineto{\pgfqpoint{2.927146in}{2.079481in}}%
\pgfpathlineto{\pgfqpoint{2.927146in}{2.079481in}}%
\pgfpathlineto{\pgfqpoint{2.956768in}{2.096905in}}%
\pgfpathlineto{\pgfqpoint{2.956768in}{2.096905in}}%
\pgfpathlineto{\pgfqpoint{2.989850in}{2.103767in}}%
\pgfpathlineto{\pgfqpoint{3.025153in}{2.101445in}}%
\pgfpathlineto{\pgfqpoint{3.058167in}{2.092042in}}%
\pgfpathlineto{\pgfqpoint{3.091542in}{2.075665in}}%
\pgfusepath{stroke}%
\end{pgfscope}%
\begin{pgfscope}%
\pgfpathrectangle{\pgfqpoint{0.647939in}{0.492442in}}{\pgfqpoint{4.273799in}{2.331163in}}%
\pgfusepath{clip}%
\pgfsetbuttcap%
\pgfsetroundjoin%
\pgfsetlinewidth{0.301125pt}%
\definecolor{currentstroke}{rgb}{0.500000,0.500000,0.500000}%
\pgfsetstrokecolor{currentstroke}%
\pgfsetstrokeopacity{0.300000}%
\pgfsetdash{}{0pt}%
\pgfpathmoveto{\pgfqpoint{4.533211in}{0.492442in}}%
\pgfpathlineto{\pgfqpoint{4.533211in}{0.492442in}}%
\pgfpathlineto{\pgfqpoint{4.499360in}{0.540841in}}%
\pgfpathlineto{\pgfqpoint{4.464355in}{0.588994in}}%
\pgfpathlineto{\pgfqpoint{4.428072in}{0.636863in}}%
\pgfpathlineto{\pgfqpoint{4.390367in}{0.684404in}}%
\pgfpathlineto{\pgfqpoint{4.351080in}{0.731563in}}%
\pgfpathlineto{\pgfqpoint{4.310034in}{0.778272in}}%
\pgfpathlineto{\pgfqpoint{4.267028in}{0.824454in}}%
\pgfpathlineto{\pgfqpoint{4.221847in}{0.870012in}}%
\pgfpathlineto{\pgfqpoint{4.174256in}{0.914832in}}%
\pgfpathlineto{\pgfqpoint{4.124018in}{0.958781in}}%
\pgfpathlineto{\pgfqpoint{4.070904in}{1.001710in}}%
\pgfpathlineto{\pgfqpoint{4.014722in}{1.043461in}}%
\pgfpathlineto{\pgfqpoint{3.955427in}{1.083912in}}%
\pgfpathlineto{\pgfqpoint{3.893134in}{1.122999in}}%
\pgfpathlineto{\pgfqpoint{3.828129in}{1.160754in}}%
\pgfpathlineto{\pgfqpoint{3.760956in}{1.197367in}}%
\pgfpathlineto{\pgfqpoint{3.692364in}{1.233194in}}%
\pgfpathlineto{\pgfqpoint{3.623233in}{1.268713in}}%
\pgfpathlineto{\pgfqpoint{3.554499in}{1.304455in}}%
\pgfpathlineto{\pgfqpoint{3.487074in}{1.340924in}}%
\pgfpathlineto{\pgfqpoint{3.421740in}{1.378499in}}%
\pgfpathlineto{\pgfqpoint{3.359154in}{1.417434in}}%
\pgfpathlineto{\pgfqpoint{3.299807in}{1.457844in}}%
\pgfpathlineto{\pgfqpoint{3.244034in}{1.499741in}}%
\pgfpathlineto{\pgfqpoint{3.192054in}{1.543069in}}%
\pgfpathlineto{\pgfqpoint{3.144023in}{1.587732in}}%
\pgfpathlineto{\pgfqpoint{3.100054in}{1.633621in}}%
\pgfpathlineto{\pgfqpoint{3.060262in}{1.680632in}}%
\pgfusepath{stroke}%
\end{pgfscope}%
\begin{pgfscope}%
\pgfpathrectangle{\pgfqpoint{0.647939in}{0.492442in}}{\pgfqpoint{4.273799in}{2.331163in}}%
\pgfusepath{clip}%
\pgfsetbuttcap%
\pgfsetroundjoin%
\pgfsetlinewidth{0.301125pt}%
\definecolor{currentstroke}{rgb}{0.500000,0.500000,0.500000}%
\pgfsetstrokecolor{currentstroke}%
\pgfsetstrokeopacity{0.300000}%
\pgfsetdash{}{0pt}%
\pgfpathmoveto{\pgfqpoint{4.630343in}{0.492442in}}%
\pgfpathlineto{\pgfqpoint{4.630343in}{0.492442in}}%
\pgfpathlineto{\pgfqpoint{4.600483in}{0.541616in}}%
\pgfpathlineto{\pgfqpoint{4.569787in}{0.590637in}}%
\pgfpathlineto{\pgfqpoint{4.538169in}{0.639483in}}%
\pgfpathlineto{\pgfqpoint{4.505541in}{0.688131in}}%
\pgfpathlineto{\pgfqpoint{4.471793in}{0.736549in}}%
\pgfpathlineto{\pgfqpoint{4.436785in}{0.784701in}}%
\pgfpathlineto{\pgfqpoint{4.400354in}{0.832537in}}%
\pgfpathlineto{\pgfqpoint{4.362317in}{0.880001in}}%
\pgfpathlineto{\pgfqpoint{4.322462in}{0.927017in}}%
\pgfpathlineto{\pgfqpoint{4.280530in}{0.973491in}}%
\pgfpathlineto{\pgfqpoint{4.236208in}{1.019297in}}%
\pgfpathlineto{\pgfqpoint{4.189126in}{1.064276in}}%
\pgfpathlineto{\pgfqpoint{4.138878in}{1.108220in}}%
\pgfpathlineto{\pgfqpoint{4.085059in}{1.150880in}}%
\pgfpathlineto{\pgfqpoint{4.027251in}{1.191949in}}%
\pgfpathlineto{\pgfqpoint{3.965131in}{1.231098in}}%
\pgfpathlineto{\pgfqpoint{3.898698in}{1.268080in}}%
\pgfpathlineto{\pgfqpoint{3.828334in}{1.302836in}}%
\pgfpathlineto{\pgfqpoint{3.754850in}{1.335630in}}%
\pgfpathlineto{\pgfqpoint{3.679502in}{1.367157in}}%
\pgfpathlineto{\pgfqpoint{3.603744in}{1.398386in}}%
\pgfpathlineto{\pgfqpoint{3.528982in}{1.430304in}}%
\pgfpathlineto{\pgfqpoint{3.456479in}{1.463728in}}%
\pgfpathlineto{\pgfqpoint{3.387362in}{1.499196in}}%
\pgfpathlineto{\pgfqpoint{3.322412in}{1.536924in}}%
\pgfpathlineto{\pgfqpoint{3.262134in}{1.576892in}}%
\pgfusepath{stroke}%
\end{pgfscope}%
\begin{pgfscope}%
\pgfpathrectangle{\pgfqpoint{0.647939in}{0.492442in}}{\pgfqpoint{4.273799in}{2.331163in}}%
\pgfusepath{clip}%
\pgfsetbuttcap%
\pgfsetroundjoin%
\pgfsetlinewidth{0.301125pt}%
\definecolor{currentstroke}{rgb}{0.500000,0.500000,0.500000}%
\pgfsetstrokecolor{currentstroke}%
\pgfsetstrokeopacity{0.300000}%
\pgfsetdash{}{0pt}%
\pgfpathmoveto{\pgfqpoint{4.727475in}{0.492442in}}%
\pgfpathlineto{\pgfqpoint{4.727475in}{0.492442in}}%
\pgfpathlineto{\pgfqpoint{4.701264in}{0.542232in}}%
\pgfpathlineto{\pgfqpoint{4.674507in}{0.591936in}}%
\pgfpathlineto{\pgfqpoint{4.647161in}{0.641544in}}%
\pgfpathlineto{\pgfqpoint{4.619163in}{0.691044in}}%
\pgfpathlineto{\pgfqpoint{4.590457in}{0.740423in}}%
\pgfpathlineto{\pgfqpoint{4.560977in}{0.789664in}}%
\pgfpathlineto{\pgfqpoint{4.530622in}{0.838748in}}%
\pgfpathlineto{\pgfqpoint{4.499295in}{0.887650in}}%
\pgfpathlineto{\pgfqpoint{4.466881in}{0.936339in}}%
\pgfpathlineto{\pgfqpoint{4.433222in}{0.984774in}}%
\pgfpathlineto{\pgfqpoint{4.398119in}{1.032904in}}%
\pgfpathlineto{\pgfqpoint{4.361335in}{1.080660in}}%
\pgfpathlineto{\pgfqpoint{4.322580in}{1.127948in}}%
\pgfpathlineto{\pgfqpoint{4.281480in}{1.174639in}}%
\pgfpathlineto{\pgfqpoint{4.237532in}{1.220547in}}%
\pgfpathlineto{\pgfqpoint{4.190075in}{1.265399in}}%
\pgfpathlineto{\pgfqpoint{4.138270in}{1.308786in}}%
\pgfpathlineto{\pgfqpoint{4.081140in}{1.350108in}}%
\pgfpathlineto{\pgfqpoint{4.017548in}{1.388484in}}%
\pgfpathlineto{\pgfqpoint{3.946688in}{1.422828in}}%
\pgfpathlineto{\pgfqpoint{3.868814in}{1.452285in}}%
\pgfpathlineto{\pgfqpoint{3.785506in}{1.477009in}}%
\pgfpathlineto{\pgfqpoint{3.699280in}{1.498659in}}%
\pgfpathlineto{\pgfqpoint{3.612639in}{1.519805in}}%
\pgfpathlineto{\pgfqpoint{3.527770in}{1.542891in}}%
\pgfpathlineto{\pgfqpoint{3.446599in}{1.569562in}}%
\pgfpathlineto{\pgfqpoint{3.370672in}{1.600500in}}%
\pgfpathlineto{\pgfqpoint{3.301176in}{1.635622in}}%
\pgfpathlineto{\pgfqpoint{3.238600in}{1.674436in}}%
\pgfpathlineto{\pgfqpoint{3.183079in}{1.716353in}}%
\pgfpathlineto{\pgfqpoint{3.134698in}{1.760816in}}%
\pgfpathlineto{\pgfqpoint{3.093560in}{1.807404in}}%
\pgfpathlineto{\pgfqpoint{3.060141in}{1.855802in}}%
\pgfpathlineto{\pgfqpoint{3.035688in}{1.905760in}}%
\pgfpathlineto{\pgfqpoint{3.023376in}{1.956938in}}%
\pgfpathlineto{\pgfqpoint{3.023376in}{1.956938in}}%
\pgfusepath{stroke}%
\end{pgfscope}%
\begin{pgfscope}%
\pgfpathrectangle{\pgfqpoint{0.647939in}{0.492442in}}{\pgfqpoint{4.273799in}{2.331163in}}%
\pgfusepath{clip}%
\pgfsetbuttcap%
\pgfsetroundjoin%
\pgfsetlinewidth{0.301125pt}%
\definecolor{currentstroke}{rgb}{0.500000,0.500000,0.500000}%
\pgfsetstrokecolor{currentstroke}%
\pgfsetstrokeopacity{0.300000}%
\pgfsetdash{}{0pt}%
\pgfpathmoveto{\pgfqpoint{4.824607in}{0.492442in}}%
\pgfpathlineto{\pgfqpoint{4.824607in}{0.492442in}}%
\pgfpathlineto{\pgfqpoint{4.801662in}{0.542710in}}%
\pgfpathlineto{\pgfqpoint{4.778403in}{0.592935in}}%
\pgfpathlineto{\pgfqpoint{4.754809in}{0.643113in}}%
\pgfpathlineto{\pgfqpoint{4.730859in}{0.693241in}}%
\pgfpathlineto{\pgfqpoint{4.706526in}{0.743315in}}%
\pgfpathlineto{\pgfqpoint{4.681783in}{0.793328in}}%
\pgfpathlineto{\pgfqpoint{4.656601in}{0.843276in}}%
\pgfpathlineto{\pgfqpoint{4.630938in}{0.893151in}}%
\pgfpathlineto{\pgfqpoint{4.604752in}{0.942944in}}%
\pgfpathlineto{\pgfqpoint{4.577983in}{0.992645in}}%
\pgfpathlineto{\pgfqpoint{4.550571in}{1.042242in}}%
\pgfpathlineto{\pgfqpoint{4.522444in}{1.091718in}}%
\pgfpathlineto{\pgfqpoint{4.493491in}{1.141053in}}%
\pgfpathlineto{\pgfqpoint{4.463598in}{1.190220in}}%
\pgfpathlineto{\pgfqpoint{4.432612in}{1.239182in}}%
\pgfpathlineto{\pgfqpoint{4.400304in}{1.287891in}}%
\pgfpathlineto{\pgfqpoint{4.366390in}{1.336273in}}%
\pgfpathlineto{\pgfqpoint{4.330485in}{1.384220in}}%
\pgfpathlineto{\pgfqpoint{4.291999in}{1.431559in}}%
\pgfpathlineto{\pgfqpoint{4.250011in}{1.477989in}}%
\pgfpathlineto{\pgfqpoint{4.203030in}{1.522943in}}%
\pgfpathlineto{\pgfqpoint{4.148477in}{1.565205in}}%
\pgfpathlineto{\pgfqpoint{4.082165in}{1.601842in}}%
\pgfpathlineto{\pgfqpoint{4.082165in}{1.601842in}}%
\pgfpathlineto{\pgfqpoint{4.021768in}{1.622190in}}%
\pgfpathlineto{\pgfqpoint{3.952261in}{1.633002in}}%
\pgfpathlineto{\pgfqpoint{3.884814in}{1.634768in}}%
\pgfpathlineto{\pgfqpoint{3.806366in}{1.631320in}}%
\pgfpathlineto{\pgfqpoint{3.712045in}{1.626041in}}%
\pgfpathlineto{\pgfqpoint{3.617418in}{1.624816in}}%
\pgfpathlineto{\pgfqpoint{3.523865in}{1.631379in}}%
\pgfpathlineto{\pgfqpoint{3.435833in}{1.646980in}}%
\pgfpathlineto{\pgfqpoint{3.358888in}{1.669751in}}%
\pgfpathlineto{\pgfqpoint{3.288117in}{1.699895in}}%
\pgfpathlineto{\pgfqpoint{3.223142in}{1.737325in}}%
\pgfusepath{stroke}%
\end{pgfscope}%
\begin{pgfscope}%
\pgfpathrectangle{\pgfqpoint{0.647939in}{0.492442in}}{\pgfqpoint{4.273799in}{2.331163in}}%
\pgfusepath{clip}%
\pgfsetbuttcap%
\pgfsetroundjoin%
\pgfsetlinewidth{0.301125pt}%
\definecolor{currentstroke}{rgb}{0.500000,0.500000,0.500000}%
\pgfsetstrokecolor{currentstroke}%
\pgfsetstrokeopacity{0.300000}%
\pgfsetdash{}{0pt}%
\pgfpathmoveto{\pgfqpoint{4.921738in}{0.492442in}}%
\pgfpathlineto{\pgfqpoint{4.921738in}{0.492442in}}%
\pgfpathlineto{\pgfqpoint{4.901662in}{0.543074in}}%
\pgfpathlineto{\pgfqpoint{4.881444in}{0.593690in}}%
\pgfpathlineto{\pgfqpoint{4.861080in}{0.644288in}}%
\pgfpathlineto{\pgfqpoint{4.840570in}{0.694868in}}%
\pgfpathlineto{\pgfqpoint{4.819910in}{0.745430in}}%
\pgfpathlineto{\pgfqpoint{4.799098in}{0.795974in}}%
\pgfpathlineto{\pgfqpoint{4.778132in}{0.846498in}}%
\pgfpathlineto{\pgfqpoint{4.757011in}{0.897004in}}%
\pgfpathlineto{\pgfqpoint{4.735732in}{0.947489in}}%
\pgfpathlineto{\pgfqpoint{4.714293in}{0.997954in}}%
\pgfpathlineto{\pgfqpoint{4.692693in}{1.048399in}}%
\pgfpathlineto{\pgfqpoint{4.670931in}{1.098822in}}%
\pgfpathlineto{\pgfqpoint{4.649007in}{1.149225in}}%
\pgfpathlineto{\pgfqpoint{4.626920in}{1.199607in}}%
\pgfpathlineto{\pgfqpoint{4.604673in}{1.249967in}}%
\pgfpathlineto{\pgfqpoint{4.582267in}{1.300306in}}%
\pgfpathlineto{\pgfqpoint{4.559706in}{1.350624in}}%
\pgfpathlineto{\pgfqpoint{4.536996in}{1.400923in}}%
\pgfpathlineto{\pgfqpoint{4.514149in}{1.451202in}}%
\pgfpathlineto{\pgfqpoint{4.491177in}{1.501465in}}%
\pgfpathlineto{\pgfqpoint{4.468106in}{1.551713in}}%
\pgfpathlineto{\pgfqpoint{4.444971in}{1.601953in}}%
\pgfpathlineto{\pgfqpoint{4.421827in}{1.652189in}}%
\pgfpathlineto{\pgfqpoint{4.398767in}{1.702436in}}%
\pgfpathlineto{\pgfqpoint{4.375941in}{1.752711in}}%
\pgfpathlineto{\pgfqpoint{4.353637in}{1.803051in}}%
\pgfpathlineto{\pgfqpoint{4.332442in}{1.853527in}}%
\pgfpathlineto{\pgfqpoint{4.313819in}{1.904291in}}%
\pgfpathlineto{\pgfqpoint{4.302498in}{1.955609in}}%
\pgfpathlineto{\pgfqpoint{4.302498in}{1.955609in}}%
\pgfpathlineto{\pgfqpoint{4.306112in}{1.992897in}}%
\pgfpathlineto{\pgfqpoint{4.327637in}{2.031876in}}%
\pgfpathlineto{\pgfqpoint{4.353247in}{2.065478in}}%
\pgfpathlineto{\pgfqpoint{4.393466in}{2.111760in}}%
\pgfpathlineto{\pgfqpoint{4.434525in}{2.158063in}}%
\pgfpathlineto{\pgfqpoint{4.475121in}{2.204619in}}%
\pgfpathlineto{\pgfqpoint{4.514838in}{2.251479in}}%
\pgfpathlineto{\pgfqpoint{4.553569in}{2.298636in}}%
\pgfpathlineto{\pgfqpoint{4.591319in}{2.346071in}}%
\pgfpathlineto{\pgfqpoint{4.628133in}{2.393760in}}%
\pgfpathlineto{\pgfqpoint{4.664067in}{2.441684in}}%
\pgfpathlineto{\pgfqpoint{4.699137in}{2.489807in}}%
\pgfpathlineto{\pgfqpoint{4.733357in}{2.538098in}}%
\pgfpathlineto{\pgfqpoint{4.766804in}{2.586549in}}%
\pgfpathlineto{\pgfqpoint{4.799548in}{2.635154in}}%
\pgfpathlineto{\pgfqpoint{4.831635in}{2.683900in}}%
\pgfpathlineto{\pgfqpoint{4.863074in}{2.732768in}}%
\pgfpathlineto{\pgfqpoint{4.893898in}{2.781748in}}%
\pgfpathlineto{\pgfqpoint{4.919977in}{2.823605in}}%
\pgfusepath{stroke}%
\end{pgfscope}%
\begin{pgfscope}%
\pgfpathrectangle{\pgfqpoint{0.647939in}{0.492442in}}{\pgfqpoint{4.273799in}{2.331163in}}%
\pgfusepath{clip}%
\pgfsetbuttcap%
\pgfsetroundjoin%
\pgfsetlinewidth{0.301125pt}%
\definecolor{currentstroke}{rgb}{0.500000,0.500000,0.500000}%
\pgfsetstrokecolor{currentstroke}%
\pgfsetstrokeopacity{0.300000}%
\pgfsetdash{}{0pt}%
\pgfpathmoveto{\pgfqpoint{4.921738in}{0.757347in}}%
\pgfpathlineto{\pgfqpoint{4.921738in}{0.757347in}}%
\pgfpathlineto{\pgfqpoint{4.903989in}{0.808237in}}%
\pgfpathlineto{\pgfqpoint{4.886263in}{0.859130in}}%
\pgfpathlineto{\pgfqpoint{4.868570in}{0.910026in}}%
\pgfpathlineto{\pgfqpoint{4.850928in}{0.960927in}}%
\pgfpathlineto{\pgfqpoint{4.833352in}{1.011835in}}%
\pgfpathlineto{\pgfqpoint{4.815866in}{1.062753in}}%
\pgfpathlineto{\pgfqpoint{4.798495in}{1.113681in}}%
\pgfpathlineto{\pgfqpoint{4.781267in}{1.164625in}}%
\pgfpathlineto{\pgfqpoint{4.764216in}{1.215585in}}%
\pgfpathlineto{\pgfqpoint{4.747380in}{1.266568in}}%
\pgfpathlineto{\pgfqpoint{4.730807in}{1.317576in}}%
\pgfpathlineto{\pgfqpoint{4.714556in}{1.368614in}}%
\pgfpathlineto{\pgfqpoint{4.698690in}{1.419688in}}%
\pgfpathlineto{\pgfqpoint{4.683293in}{1.470805in}}%
\pgfpathlineto{\pgfqpoint{4.668458in}{1.521971in}}%
\pgfpathlineto{\pgfqpoint{4.654301in}{1.573195in}}%
\pgfpathlineto{\pgfqpoint{4.640968in}{1.624483in}}%
\pgfpathlineto{\pgfqpoint{4.628627in}{1.675846in}}%
\pgfpathlineto{\pgfqpoint{4.617482in}{1.727289in}}%
\pgfpathlineto{\pgfqpoint{4.607787in}{1.778819in}}%
\pgfpathlineto{\pgfqpoint{4.599842in}{1.830437in}}%
\pgfpathlineto{\pgfqpoint{4.593989in}{1.882138in}}%
\pgfpathlineto{\pgfqpoint{4.590614in}{1.933903in}}%
\pgfpathlineto{\pgfqpoint{4.590122in}{1.985698in}}%
\pgfpathlineto{\pgfqpoint{4.592896in}{2.037469in}}%
\pgfpathlineto{\pgfqpoint{4.599232in}{2.089145in}}%
\pgfpathlineto{\pgfqpoint{4.609270in}{2.140644in}}%
\pgfpathlineto{\pgfqpoint{4.622955in}{2.191894in}}%
\pgfpathlineto{\pgfqpoint{4.639990in}{2.242841in}}%
\pgfpathlineto{\pgfqpoint{4.659946in}{2.293471in}}%
\pgfpathlineto{\pgfqpoint{4.682302in}{2.343802in}}%
\pgfpathlineto{\pgfqpoint{4.706543in}{2.393876in}}%
\pgfpathlineto{\pgfqpoint{4.732183in}{2.443744in}}%
\pgfusepath{stroke}%
\end{pgfscope}%
\begin{pgfscope}%
\pgfpathrectangle{\pgfqpoint{0.647939in}{0.492442in}}{\pgfqpoint{4.273799in}{2.331163in}}%
\pgfusepath{clip}%
\pgfsetbuttcap%
\pgfsetroundjoin%
\pgfsetlinewidth{0.301125pt}%
\definecolor{currentstroke}{rgb}{0.500000,0.500000,0.500000}%
\pgfsetstrokecolor{currentstroke}%
\pgfsetstrokeopacity{0.300000}%
\pgfsetdash{}{0pt}%
\pgfpathmoveto{\pgfqpoint{4.921738in}{1.022252in}}%
\pgfpathlineto{\pgfqpoint{4.921738in}{1.022252in}}%
\pgfpathlineto{\pgfqpoint{4.906748in}{1.073406in}}%
\pgfpathlineto{\pgfqpoint{4.891987in}{1.124579in}}%
\pgfpathlineto{\pgfqpoint{4.877484in}{1.175775in}}%
\pgfpathlineto{\pgfqpoint{4.863279in}{1.226995in}}%
\pgfpathlineto{\pgfqpoint{4.849419in}{1.278243in}}%
\pgfpathlineto{\pgfqpoint{4.835946in}{1.329522in}}%
\pgfpathlineto{\pgfqpoint{4.822919in}{1.380836in}}%
\pgfpathlineto{\pgfqpoint{4.810404in}{1.432186in}}%
\pgfpathlineto{\pgfqpoint{4.798470in}{1.483579in}}%
\pgfpathlineto{\pgfqpoint{4.787195in}{1.535015in}}%
\pgfpathlineto{\pgfqpoint{4.776677in}{1.586499in}}%
\pgfpathlineto{\pgfqpoint{4.767019in}{1.638033in}}%
\pgfpathlineto{\pgfqpoint{4.758335in}{1.689618in}}%
\pgfpathlineto{\pgfqpoint{4.750752in}{1.741255in}}%
\pgfpathlineto{\pgfqpoint{4.744414in}{1.792942in}}%
\pgfpathlineto{\pgfqpoint{4.739475in}{1.844673in}}%
\pgfpathlineto{\pgfqpoint{4.736096in}{1.896442in}}%
\pgfpathlineto{\pgfqpoint{4.734437in}{1.948235in}}%
\pgfpathlineto{\pgfqpoint{4.734655in}{2.000035in}}%
\pgfpathlineto{\pgfqpoint{4.736890in}{2.051821in}}%
\pgfpathlineto{\pgfqpoint{4.741256in}{2.103565in}}%
\pgfpathlineto{\pgfqpoint{4.747827in}{2.155240in}}%
\pgfpathlineto{\pgfqpoint{4.756621in}{2.206816in}}%
\pgfpathlineto{\pgfqpoint{4.767596in}{2.258268in}}%
\pgfpathlineto{\pgfqpoint{4.780661in}{2.309573in}}%
\pgfpathlineto{\pgfqpoint{4.795679in}{2.360719in}}%
\pgfpathlineto{\pgfqpoint{4.812458in}{2.411702in}}%
\pgfpathlineto{\pgfqpoint{4.830795in}{2.462524in}}%
\pgfpathlineto{\pgfqpoint{4.850476in}{2.513197in}}%
\pgfpathlineto{\pgfqpoint{4.871287in}{2.563735in}}%
\pgfpathlineto{\pgfqpoint{4.893031in}{2.614156in}}%
\pgfpathlineto{\pgfqpoint{4.915528in}{2.664479in}}%
\pgfpathlineto{\pgfqpoint{4.921738in}{2.678161in}}%
\pgfusepath{stroke}%
\end{pgfscope}%
\begin{pgfscope}%
\pgfpathrectangle{\pgfqpoint{0.647939in}{0.492442in}}{\pgfqpoint{4.273799in}{2.331163in}}%
\pgfusepath{clip}%
\pgfsetbuttcap%
\pgfsetroundjoin%
\pgfsetlinewidth{0.301125pt}%
\definecolor{currentstroke}{rgb}{0.500000,0.500000,0.500000}%
\pgfsetstrokecolor{currentstroke}%
\pgfsetstrokeopacity{0.300000}%
\pgfsetdash{}{0pt}%
\pgfpathmoveto{\pgfqpoint{4.921738in}{1.393119in}}%
\pgfpathlineto{\pgfqpoint{4.921738in}{1.393119in}}%
\pgfpathlineto{\pgfqpoint{4.911566in}{1.444624in}}%
\pgfpathlineto{\pgfqpoint{4.901990in}{1.496163in}}%
\pgfpathlineto{\pgfqpoint{4.893071in}{1.547737in}}%
\pgfpathlineto{\pgfqpoint{4.884874in}{1.599346in}}%
\pgfpathlineto{\pgfqpoint{4.877472in}{1.650992in}}%
\pgfpathlineto{\pgfqpoint{4.870945in}{1.702672in}}%
\pgfpathlineto{\pgfqpoint{4.865376in}{1.754386in}}%
\pgfpathlineto{\pgfqpoint{4.860851in}{1.806129in}}%
\pgfpathlineto{\pgfqpoint{4.857456in}{1.857899in}}%
\pgfpathlineto{\pgfqpoint{4.855281in}{1.909687in}}%
\pgfpathlineto{\pgfqpoint{4.854410in}{1.961487in}}%
\pgfpathlineto{\pgfqpoint{4.854925in}{2.013288in}}%
\pgfpathlineto{\pgfqpoint{4.856895in}{2.065079in}}%
\pgfpathlineto{\pgfqpoint{4.860379in}{2.116846in}}%
\pgfpathlineto{\pgfqpoint{4.865417in}{2.168574in}}%
\pgfpathlineto{\pgfqpoint{4.872030in}{2.220250in}}%
\pgfpathlineto{\pgfqpoint{4.880216in}{2.271858in}}%
\pgfpathlineto{\pgfqpoint{4.889944in}{2.323387in}}%
\pgfpathlineto{\pgfqpoint{4.901160in}{2.374825in}}%
\pgfpathlineto{\pgfqpoint{4.913796in}{2.426165in}}%
\pgfpathlineto{\pgfqpoint{4.921738in}{2.456767in}}%
\pgfusepath{stroke}%
\end{pgfscope}%
\begin{pgfscope}%
\pgfpathrectangle{\pgfqpoint{0.647939in}{0.492442in}}{\pgfqpoint{4.273799in}{2.331163in}}%
\pgfusepath{clip}%
\pgfsetbuttcap%
\pgfsetroundjoin%
\pgfsetlinewidth{0.301125pt}%
\definecolor{currentstroke}{rgb}{0.500000,0.500000,0.500000}%
\pgfsetstrokecolor{currentstroke}%
\pgfsetstrokeopacity{0.300000}%
\pgfsetdash{}{0pt}%
\pgfpathmoveto{\pgfqpoint{4.921738in}{1.711005in}}%
\pgfpathlineto{\pgfqpoint{4.921738in}{1.711005in}}%
\pgfpathlineto{\pgfqpoint{4.916903in}{1.762740in}}%
\pgfpathlineto{\pgfqpoint{4.913031in}{1.814500in}}%
\pgfpathlineto{\pgfqpoint{4.910191in}{1.866280in}}%
\pgfpathlineto{\pgfqpoint{4.908451in}{1.918073in}}%
\pgfpathlineto{\pgfqpoint{4.907876in}{1.969875in}}%
\pgfpathlineto{\pgfqpoint{4.908529in}{2.021676in}}%
\pgfpathlineto{\pgfqpoint{4.910462in}{2.073468in}}%
\pgfpathlineto{\pgfqpoint{4.913719in}{2.125239in}}%
\pgfpathlineto{\pgfqpoint{4.918332in}{2.176980in}}%
\pgfpathlineto{\pgfqpoint{4.921738in}{2.210272in}}%
\pgfusepath{stroke}%
\end{pgfscope}%
\begin{pgfscope}%
\pgfpathrectangle{\pgfqpoint{0.647939in}{0.492442in}}{\pgfqpoint{4.273799in}{2.331163in}}%
\pgfusepath{clip}%
\pgfsetbuttcap%
\pgfsetroundjoin%
\pgfsetlinewidth{0.301125pt}%
\definecolor{currentstroke}{rgb}{0.500000,0.500000,0.500000}%
\pgfsetstrokecolor{currentstroke}%
\pgfsetstrokeopacity{0.300000}%
\pgfsetdash{}{0pt}%
\pgfpathmoveto{\pgfqpoint{4.236729in}{2.823605in}}%
\pgfpathlineto{\pgfqpoint{4.260938in}{2.800477in}}%
\pgfpathlineto{\pgfqpoint{4.310272in}{2.756249in}}%
\pgfpathlineto{\pgfqpoint{4.365386in}{2.714548in}}%
\pgfpathlineto{\pgfqpoint{4.413467in}{2.686059in}}%
\pgfpathlineto{\pgfqpoint{4.457421in}{2.667464in}}%
\pgfpathlineto{\pgfqpoint{4.502781in}{2.656512in}}%
\pgfpathlineto{\pgfqpoint{4.558602in}{2.655282in}}%
\pgfpathlineto{\pgfqpoint{4.610493in}{2.666362in}}%
\pgfpathlineto{\pgfqpoint{4.610493in}{2.666362in}}%
\pgfpathlineto{\pgfqpoint{4.671009in}{2.693773in}}%
\pgfpathlineto{\pgfqpoint{4.671009in}{2.693773in}}%
\pgfpathlineto{\pgfqpoint{4.729794in}{2.734112in}}%
\pgfpathlineto{\pgfqpoint{4.779919in}{2.777965in}}%
\pgfpathlineto{\pgfqpoint{4.824607in}{2.823605in}}%
\pgfpathlineto{\pgfqpoint{4.824607in}{2.823605in}}%
\pgfusepath{stroke}%
\end{pgfscope}%
\begin{pgfscope}%
\pgfpathrectangle{\pgfqpoint{0.647939in}{0.492442in}}{\pgfqpoint{4.273799in}{2.331163in}}%
\pgfusepath{clip}%
\pgfsetbuttcap%
\pgfsetroundjoin%
\pgfsetlinewidth{0.301125pt}%
\definecolor{currentstroke}{rgb}{0.500000,0.500000,0.500000}%
\pgfsetstrokecolor{currentstroke}%
\pgfsetstrokeopacity{0.300000}%
\pgfsetdash{}{0pt}%
\pgfpathmoveto{\pgfqpoint{4.144684in}{2.823605in}}%
\pgfpathlineto{\pgfqpoint{4.144684in}{2.823605in}}%
\pgfpathlineto{\pgfqpoint{4.185976in}{2.776958in}}%
\pgfpathlineto{\pgfqpoint{4.229409in}{2.730896in}}%
\pgfpathlineto{\pgfqpoint{4.276066in}{2.685791in}}%
\pgfpathlineto{\pgfqpoint{4.327904in}{2.642450in}}%
\pgfpathlineto{\pgfqpoint{4.388775in}{2.602988in}}%
\pgfpathlineto{\pgfqpoint{4.388775in}{2.602988in}}%
\pgfpathlineto{\pgfqpoint{4.442789in}{2.580250in}}%
\pgfpathlineto{\pgfqpoint{4.442789in}{2.580250in}}%
\pgfpathlineto{\pgfqpoint{4.490279in}{2.571022in}}%
\pgfpathlineto{\pgfqpoint{4.541221in}{2.573085in}}%
\pgfpathlineto{\pgfqpoint{4.582029in}{2.583478in}}%
\pgfpathlineto{\pgfqpoint{4.622346in}{2.601112in}}%
\pgfpathlineto{\pgfqpoint{4.665921in}{2.627875in}}%
\pgfpathlineto{\pgfqpoint{4.714938in}{2.666666in}}%
\pgfusepath{stroke}%
\end{pgfscope}%
\begin{pgfscope}%
\pgfpathrectangle{\pgfqpoint{0.647939in}{0.492442in}}{\pgfqpoint{4.273799in}{2.331163in}}%
\pgfusepath{clip}%
\pgfsetbuttcap%
\pgfsetroundjoin%
\pgfsetlinewidth{0.301125pt}%
\definecolor{currentstroke}{rgb}{0.500000,0.500000,0.500000}%
\pgfsetstrokecolor{currentstroke}%
\pgfsetstrokeopacity{0.300000}%
\pgfsetdash{}{0pt}%
\pgfpathmoveto{\pgfqpoint{4.047552in}{2.823605in}}%
\pgfpathlineto{\pgfqpoint{4.047552in}{2.823605in}}%
\pgfpathlineto{\pgfqpoint{4.085140in}{2.776033in}}%
\pgfpathlineto{\pgfqpoint{4.123599in}{2.728670in}}%
\pgfpathlineto{\pgfqpoint{4.163338in}{2.681626in}}%
\pgfpathlineto{\pgfqpoint{4.204984in}{2.635080in}}%
\pgfpathlineto{\pgfqpoint{4.249569in}{2.589367in}}%
\pgfpathlineto{\pgfqpoint{4.298976in}{2.545197in}}%
\pgfpathlineto{\pgfqpoint{4.357114in}{2.504479in}}%
\pgfpathlineto{\pgfqpoint{4.357114in}{2.504479in}}%
\pgfpathlineto{\pgfqpoint{4.408695in}{2.480744in}}%
\pgfpathlineto{\pgfqpoint{4.408695in}{2.480744in}}%
\pgfpathlineto{\pgfqpoint{4.453487in}{2.471057in}}%
\pgfpathlineto{\pgfqpoint{4.502153in}{2.472810in}}%
\pgfpathlineto{\pgfqpoint{4.540685in}{2.482899in}}%
\pgfpathlineto{\pgfqpoint{4.578872in}{2.500109in}}%
\pgfpathlineto{\pgfqpoint{4.620752in}{2.526541in}}%
\pgfusepath{stroke}%
\end{pgfscope}%
\begin{pgfscope}%
\pgfpathrectangle{\pgfqpoint{0.647939in}{0.492442in}}{\pgfqpoint{4.273799in}{2.331163in}}%
\pgfusepath{clip}%
\pgfsetbuttcap%
\pgfsetroundjoin%
\pgfsetlinewidth{0.301125pt}%
\definecolor{currentstroke}{rgb}{0.500000,0.500000,0.500000}%
\pgfsetstrokecolor{currentstroke}%
\pgfsetstrokeopacity{0.300000}%
\pgfsetdash{}{0pt}%
\pgfpathmoveto{\pgfqpoint{3.950420in}{2.823605in}}%
\pgfpathlineto{\pgfqpoint{3.950420in}{2.823605in}}%
\pgfpathlineto{\pgfqpoint{3.985794in}{2.775530in}}%
\pgfpathlineto{\pgfqpoint{4.021362in}{2.727497in}}%
\pgfpathlineto{\pgfqpoint{4.057289in}{2.679543in}}%
\pgfpathlineto{\pgfqpoint{4.093820in}{2.631726in}}%
\pgfpathlineto{\pgfqpoint{4.131301in}{2.584131in}}%
\pgfpathlineto{\pgfqpoint{4.170240in}{2.536890in}}%
\pgfpathlineto{\pgfqpoint{4.211477in}{2.490243in}}%
\pgfpathlineto{\pgfqpoint{4.256558in}{2.444696in}}%
\pgfpathlineto{\pgfqpoint{4.308772in}{2.401589in}}%
\pgfpathlineto{\pgfqpoint{4.308772in}{2.401589in}}%
\pgfpathlineto{\pgfqpoint{4.360757in}{2.371963in}}%
\pgfpathlineto{\pgfqpoint{4.360757in}{2.371963in}}%
\pgfpathlineto{\pgfqpoint{4.402123in}{2.359676in}}%
\pgfpathlineto{\pgfqpoint{4.402123in}{2.359676in}}%
\pgfpathlineto{\pgfqpoint{4.440999in}{2.358071in}}%
\pgfpathlineto{\pgfqpoint{4.477743in}{2.365245in}}%
\pgfpathlineto{\pgfqpoint{4.511731in}{2.378893in}}%
\pgfpathlineto{\pgfqpoint{4.549356in}{2.401057in}}%
\pgfusepath{stroke}%
\end{pgfscope}%
\begin{pgfscope}%
\pgfpathrectangle{\pgfqpoint{0.647939in}{0.492442in}}{\pgfqpoint{4.273799in}{2.331163in}}%
\pgfusepath{clip}%
\pgfsetbuttcap%
\pgfsetroundjoin%
\pgfsetlinewidth{0.301125pt}%
\definecolor{currentstroke}{rgb}{0.500000,0.500000,0.500000}%
\pgfsetstrokecolor{currentstroke}%
\pgfsetstrokeopacity{0.300000}%
\pgfsetdash{}{0pt}%
\pgfpathmoveto{\pgfqpoint{3.853289in}{2.823605in}}%
\pgfpathlineto{\pgfqpoint{3.853289in}{2.823605in}}%
\pgfpathlineto{\pgfqpoint{3.887457in}{2.775271in}}%
\pgfpathlineto{\pgfqpoint{3.921423in}{2.726893in}}%
\pgfpathlineto{\pgfqpoint{3.955270in}{2.678491in}}%
\pgfpathlineto{\pgfqpoint{3.989085in}{2.630083in}}%
\pgfpathlineto{\pgfqpoint{4.022978in}{2.581692in}}%
\pgfpathlineto{\pgfqpoint{4.057100in}{2.533348in}}%
\pgfpathlineto{\pgfqpoint{4.091688in}{2.485103in}}%
\pgfpathlineto{\pgfqpoint{4.127092in}{2.437038in}}%
\pgfpathlineto{\pgfqpoint{4.163853in}{2.389283in}}%
\pgfpathlineto{\pgfqpoint{4.202952in}{2.342098in}}%
\pgfpathlineto{\pgfqpoint{4.246459in}{2.296125in}}%
\pgfpathlineto{\pgfqpoint{4.299868in}{2.253741in}}%
\pgfpathlineto{\pgfqpoint{4.299868in}{2.253741in}}%
\pgfpathlineto{\pgfqpoint{4.336071in}{2.236564in}}%
\pgfpathlineto{\pgfqpoint{4.336071in}{2.236564in}}%
\pgfpathlineto{\pgfqpoint{4.370651in}{2.230480in}}%
\pgfpathlineto{\pgfqpoint{4.406719in}{2.234963in}}%
\pgfpathlineto{\pgfqpoint{4.436252in}{2.245894in}}%
\pgfpathlineto{\pgfqpoint{4.467884in}{2.263984in}}%
\pgfusepath{stroke}%
\end{pgfscope}%
\begin{pgfscope}%
\pgfpathrectangle{\pgfqpoint{0.647939in}{0.492442in}}{\pgfqpoint{4.273799in}{2.331163in}}%
\pgfusepath{clip}%
\pgfsetbuttcap%
\pgfsetroundjoin%
\pgfsetlinewidth{0.301125pt}%
\definecolor{currentstroke}{rgb}{0.500000,0.500000,0.500000}%
\pgfsetstrokecolor{currentstroke}%
\pgfsetstrokeopacity{0.300000}%
\pgfsetdash{}{0pt}%
\pgfpathmoveto{\pgfqpoint{3.756157in}{2.823605in}}%
\pgfpathlineto{\pgfqpoint{3.756157in}{2.823605in}}%
\pgfpathlineto{\pgfqpoint{3.789887in}{2.775179in}}%
\pgfpathlineto{\pgfqpoint{3.823155in}{2.726658in}}%
\pgfpathlineto{\pgfqpoint{3.855986in}{2.678048in}}%
\pgfpathlineto{\pgfqpoint{3.888404in}{2.629356in}}%
\pgfpathlineto{\pgfqpoint{3.920447in}{2.580590in}}%
\pgfpathlineto{\pgfqpoint{3.952168in}{2.531762in}}%
\pgfpathlineto{\pgfqpoint{3.983608in}{2.482880in}}%
\pgfpathlineto{\pgfqpoint{4.014813in}{2.433953in}}%
\pgfpathlineto{\pgfqpoint{4.045877in}{2.385000in}}%
\pgfpathlineto{\pgfqpoint{4.076928in}{2.336047in}}%
\pgfpathlineto{\pgfqpoint{4.108123in}{2.287121in}}%
\pgfpathlineto{\pgfqpoint{4.139750in}{2.238279in}}%
\pgfpathlineto{\pgfqpoint{4.172415in}{2.189651in}}%
\pgfpathlineto{\pgfqpoint{4.207527in}{2.141577in}}%
\pgfpathlineto{\pgfqpoint{4.250006in}{2.095709in}}%
\pgfpathlineto{\pgfqpoint{4.250006in}{2.095709in}}%
\pgfpathlineto{\pgfqpoint{4.273613in}{2.080084in}}%
\pgfpathlineto{\pgfqpoint{4.273613in}{2.080084in}}%
\pgfusepath{stroke}%
\end{pgfscope}%
\begin{pgfscope}%
\pgfpathrectangle{\pgfqpoint{0.647939in}{0.492442in}}{\pgfqpoint{4.273799in}{2.331163in}}%
\pgfusepath{clip}%
\pgfsetbuttcap%
\pgfsetroundjoin%
\pgfsetlinewidth{0.301125pt}%
\definecolor{currentstroke}{rgb}{0.500000,0.500000,0.500000}%
\pgfsetstrokecolor{currentstroke}%
\pgfsetstrokeopacity{0.300000}%
\pgfsetdash{}{0pt}%
\pgfpathmoveto{\pgfqpoint{3.659025in}{2.823605in}}%
\pgfpathlineto{\pgfqpoint{3.659025in}{2.823605in}}%
\pgfpathlineto{\pgfqpoint{3.692948in}{2.775219in}}%
\pgfpathlineto{\pgfqpoint{3.726198in}{2.726695in}}%
\pgfpathlineto{\pgfqpoint{3.758778in}{2.678035in}}%
\pgfpathlineto{\pgfqpoint{3.790700in}{2.629246in}}%
\pgfpathlineto{\pgfqpoint{3.821973in}{2.580332in}}%
\pgfpathlineto{\pgfqpoint{3.852587in}{2.531294in}}%
\pgfpathlineto{\pgfqpoint{3.882520in}{2.482131in}}%
\pgfpathlineto{\pgfqpoint{3.911758in}{2.432844in}}%
\pgfpathlineto{\pgfqpoint{3.940265in}{2.383430in}}%
\pgfpathlineto{\pgfqpoint{3.967961in}{2.333879in}}%
\pgfpathlineto{\pgfqpoint{3.994758in}{2.284182in}}%
\pgfpathlineto{\pgfqpoint{4.020497in}{2.234320in}}%
\pgfpathlineto{\pgfqpoint{4.044912in}{2.184261in}}%
\pgfpathlineto{\pgfqpoint{4.067571in}{2.133960in}}%
\pgfpathlineto{\pgfqpoint{4.087658in}{2.083342in}}%
\pgfpathlineto{\pgfqpoint{4.103569in}{2.032299in}}%
\pgfpathlineto{\pgfqpoint{4.111887in}{1.980777in}}%
\pgfpathlineto{\pgfqpoint{4.105757in}{1.929364in}}%
\pgfpathlineto{\pgfqpoint{4.105757in}{1.929364in}}%
\pgfpathlineto{\pgfqpoint{4.085366in}{1.888317in}}%
\pgfpathlineto{\pgfqpoint{4.051106in}{1.849321in}}%
\pgfpathlineto{\pgfqpoint{4.003827in}{1.810748in}}%
\pgfpathlineto{\pgfqpoint{3.942540in}{1.771517in}}%
\pgfpathlineto{\pgfqpoint{3.874033in}{1.736050in}}%
\pgfpathlineto{\pgfqpoint{3.798379in}{1.705184in}}%
\pgfpathlineto{\pgfqpoint{3.715513in}{1.680631in}}%
\pgfpathlineto{\pgfqpoint{3.626023in}{1.664874in}}%
\pgfpathlineto{\pgfqpoint{3.540840in}{1.660375in}}%
\pgfusepath{stroke}%
\end{pgfscope}%
\begin{pgfscope}%
\pgfpathrectangle{\pgfqpoint{0.647939in}{0.492442in}}{\pgfqpoint{4.273799in}{2.331163in}}%
\pgfusepath{clip}%
\pgfsetbuttcap%
\pgfsetroundjoin%
\pgfsetlinewidth{0.301125pt}%
\definecolor{currentstroke}{rgb}{0.500000,0.500000,0.500000}%
\pgfsetstrokecolor{currentstroke}%
\pgfsetstrokeopacity{0.300000}%
\pgfsetdash{}{0pt}%
\pgfpathmoveto{\pgfqpoint{3.561893in}{2.823605in}}%
\pgfpathlineto{\pgfqpoint{3.561893in}{2.823605in}}%
\pgfpathlineto{\pgfqpoint{3.596559in}{2.775377in}}%
\pgfpathlineto{\pgfqpoint{3.630369in}{2.726968in}}%
\pgfpathlineto{\pgfqpoint{3.663330in}{2.678385in}}%
\pgfpathlineto{\pgfqpoint{3.695445in}{2.629633in}}%
\pgfpathlineto{\pgfqpoint{3.726699in}{2.580716in}}%
\pgfpathlineto{\pgfqpoint{3.757061in}{2.531631in}}%
\pgfpathlineto{\pgfqpoint{3.786500in}{2.482380in}}%
\pgfpathlineto{\pgfqpoint{3.814972in}{2.432960in}}%
\pgfpathlineto{\pgfqpoint{3.842382in}{2.383363in}}%
\pgfpathlineto{\pgfqpoint{3.868629in}{2.333578in}}%
\pgfpathlineto{\pgfqpoint{3.893560in}{2.283593in}}%
\pgfpathlineto{\pgfqpoint{3.916936in}{2.233386in}}%
\pgfpathlineto{\pgfqpoint{3.938429in}{2.182930in}}%
\pgfpathlineto{\pgfqpoint{3.957532in}{2.132193in}}%
\pgfpathlineto{\pgfqpoint{3.973474in}{2.081137in}}%
\pgfpathlineto{\pgfqpoint{3.985051in}{2.029742in}}%
\pgfpathlineto{\pgfqpoint{3.990383in}{1.978063in}}%
\pgfpathlineto{\pgfqpoint{3.986632in}{1.926391in}}%
\pgfpathlineto{\pgfqpoint{3.970320in}{1.875551in}}%
\pgfpathlineto{\pgfqpoint{3.938801in}{1.827018in}}%
\pgfpathlineto{\pgfqpoint{3.892024in}{1.782359in}}%
\pgfpathlineto{\pgfqpoint{3.832719in}{1.743201in}}%
\pgfusepath{stroke}%
\end{pgfscope}%
\begin{pgfscope}%
\pgfpathrectangle{\pgfqpoint{0.647939in}{0.492442in}}{\pgfqpoint{4.273799in}{2.331163in}}%
\pgfusepath{clip}%
\pgfsetbuttcap%
\pgfsetroundjoin%
\pgfsetlinewidth{0.301125pt}%
\definecolor{currentstroke}{rgb}{0.500000,0.500000,0.500000}%
\pgfsetstrokecolor{currentstroke}%
\pgfsetstrokeopacity{0.300000}%
\pgfsetdash{}{0pt}%
\pgfpathmoveto{\pgfqpoint{3.464761in}{2.823605in}}%
\pgfpathlineto{\pgfqpoint{3.464761in}{2.823605in}}%
\pgfpathlineto{\pgfqpoint{3.500702in}{2.775656in}}%
\pgfpathlineto{\pgfqpoint{3.535619in}{2.727481in}}%
\pgfpathlineto{\pgfqpoint{3.569523in}{2.679092in}}%
\pgfpathlineto{\pgfqpoint{3.602407in}{2.630494in}}%
\pgfpathlineto{\pgfqpoint{3.634251in}{2.581690in}}%
\pgfpathlineto{\pgfqpoint{3.665023in}{2.532682in}}%
\pgfpathlineto{\pgfqpoint{3.694688in}{2.483472in}}%
\pgfpathlineto{\pgfqpoint{3.723185in}{2.434056in}}%
\pgfpathlineto{\pgfqpoint{3.750415in}{2.384429in}}%
\pgfpathlineto{\pgfqpoint{3.776265in}{2.334584in}}%
\pgfpathlineto{\pgfqpoint{3.800559in}{2.284506in}}%
\pgfpathlineto{\pgfqpoint{3.823063in}{2.234181in}}%
\pgfpathlineto{\pgfqpoint{3.843448in}{2.183589in}}%
\pgfpathlineto{\pgfqpoint{3.861247in}{2.132710in}}%
\pgfpathlineto{\pgfqpoint{3.875793in}{2.081530in}}%
\pgfpathlineto{\pgfqpoint{3.886102in}{2.030053in}}%
\pgfpathlineto{\pgfqpoint{3.890761in}{1.978345in}}%
\pgfpathlineto{\pgfqpoint{3.887764in}{1.926628in}}%
\pgfpathlineto{\pgfqpoint{3.874478in}{1.875454in}}%
\pgfpathlineto{\pgfqpoint{3.847986in}{1.825934in}}%
\pgfusepath{stroke}%
\end{pgfscope}%
\begin{pgfscope}%
\pgfpathrectangle{\pgfqpoint{0.647939in}{0.492442in}}{\pgfqpoint{4.273799in}{2.331163in}}%
\pgfusepath{clip}%
\pgfsetbuttcap%
\pgfsetroundjoin%
\pgfsetlinewidth{0.301125pt}%
\definecolor{currentstroke}{rgb}{0.500000,0.500000,0.500000}%
\pgfsetstrokecolor{currentstroke}%
\pgfsetstrokeopacity{0.300000}%
\pgfsetdash{}{0pt}%
\pgfpathmoveto{\pgfqpoint{3.367630in}{2.823605in}}%
\pgfpathlineto{\pgfqpoint{3.367630in}{2.823605in}}%
\pgfpathlineto{\pgfqpoint{3.405397in}{2.776075in}}%
\pgfpathlineto{\pgfqpoint{3.441968in}{2.728268in}}%
\pgfpathlineto{\pgfqpoint{3.477350in}{2.680196in}}%
\pgfpathlineto{\pgfqpoint{3.511539in}{2.631867in}}%
\pgfpathlineto{\pgfqpoint{3.544516in}{2.583288in}}%
\pgfpathlineto{\pgfqpoint{3.576257in}{2.534465in}}%
\pgfpathlineto{\pgfqpoint{3.606728in}{2.485401in}}%
\pgfpathlineto{\pgfqpoint{3.635862in}{2.436097in}}%
\pgfpathlineto{\pgfqpoint{3.663565in}{2.386549in}}%
\pgfpathlineto{\pgfqpoint{3.689723in}{2.336751in}}%
\pgfusepath{stroke}%
\end{pgfscope}%
\begin{pgfscope}%
\pgfpathrectangle{\pgfqpoint{0.647939in}{0.492442in}}{\pgfqpoint{4.273799in}{2.331163in}}%
\pgfusepath{clip}%
\pgfsetbuttcap%
\pgfsetroundjoin%
\pgfsetlinewidth{0.301125pt}%
\definecolor{currentstroke}{rgb}{0.500000,0.500000,0.500000}%
\pgfsetstrokecolor{currentstroke}%
\pgfsetstrokeopacity{0.300000}%
\pgfsetdash{}{0pt}%
\pgfpathmoveto{\pgfqpoint{3.173366in}{2.823605in}}%
\pgfpathlineto{\pgfqpoint{3.173366in}{2.823605in}}%
\pgfpathlineto{\pgfqpoint{3.216614in}{2.777490in}}%
\pgfpathlineto{\pgfqpoint{3.258221in}{2.730927in}}%
\pgfpathlineto{\pgfqpoint{3.298214in}{2.683945in}}%
\pgfpathlineto{\pgfqpoint{3.336613in}{2.636568in}}%
\pgfpathlineto{\pgfqpoint{3.373423in}{2.588817in}}%
\pgfpathlineto{\pgfqpoint{3.408636in}{2.540708in}}%
\pgfpathlineto{\pgfqpoint{3.442233in}{2.492257in}}%
\pgfpathlineto{\pgfqpoint{3.474172in}{2.443474in}}%
\pgfpathlineto{\pgfqpoint{3.504376in}{2.394363in}}%
\pgfpathlineto{\pgfqpoint{3.532736in}{2.344927in}}%
\pgfpathlineto{\pgfqpoint{3.559111in}{2.295165in}}%
\pgfpathlineto{\pgfqpoint{3.583286in}{2.245073in}}%
\pgfpathlineto{\pgfqpoint{3.604982in}{2.194646in}}%
\pgfpathlineto{\pgfqpoint{3.623806in}{2.143878in}}%
\pgfpathlineto{\pgfqpoint{3.639209in}{2.092771in}}%
\pgfpathlineto{\pgfqpoint{3.650421in}{2.041347in}}%
\pgfpathlineto{\pgfqpoint{3.656321in}{1.989674in}}%
\pgfpathlineto{\pgfqpoint{3.655272in}{1.937928in}}%
\pgfpathlineto{\pgfqpoint{3.644823in}{1.886539in}}%
\pgfpathlineto{\pgfqpoint{3.621307in}{1.836545in}}%
\pgfpathlineto{\pgfqpoint{3.579407in}{1.790535in}}%
\pgfpathlineto{\pgfqpoint{3.579407in}{1.790535in}}%
\pgfpathlineto{\pgfqpoint{3.535564in}{1.763111in}}%
\pgfpathlineto{\pgfqpoint{3.477945in}{1.744044in}}%
\pgfpathlineto{\pgfqpoint{3.421325in}{1.737601in}}%
\pgfpathlineto{\pgfqpoint{3.366924in}{1.740548in}}%
\pgfpathlineto{\pgfqpoint{3.314011in}{1.751352in}}%
\pgfpathlineto{\pgfqpoint{3.262207in}{1.769647in}}%
\pgfpathlineto{\pgfqpoint{3.210681in}{1.796289in}}%
\pgfpathlineto{\pgfqpoint{3.161375in}{1.831752in}}%
\pgfusepath{stroke}%
\end{pgfscope}%
\begin{pgfscope}%
\pgfpathrectangle{\pgfqpoint{0.647939in}{0.492442in}}{\pgfqpoint{4.273799in}{2.331163in}}%
\pgfusepath{clip}%
\pgfsetbuttcap%
\pgfsetroundjoin%
\pgfsetlinewidth{0.301125pt}%
\definecolor{currentstroke}{rgb}{0.500000,0.500000,0.500000}%
\pgfsetstrokecolor{currentstroke}%
\pgfsetstrokeopacity{0.300000}%
\pgfsetdash{}{0pt}%
\pgfpathmoveto{\pgfqpoint{2.979102in}{2.823605in}}%
\pgfpathlineto{\pgfqpoint{2.979102in}{2.823605in}}%
\pgfpathlineto{\pgfqpoint{3.030598in}{2.780086in}}%
\pgfpathlineto{\pgfqpoint{3.079819in}{2.735793in}}%
\pgfpathlineto{\pgfqpoint{3.126827in}{2.690790in}}%
\pgfpathlineto{\pgfqpoint{3.171682in}{2.645137in}}%
\pgfpathlineto{\pgfqpoint{3.214431in}{2.598886in}}%
\pgfpathlineto{\pgfqpoint{3.255108in}{2.552082in}}%
\pgfpathlineto{\pgfqpoint{3.293730in}{2.504762in}}%
\pgfpathlineto{\pgfqpoint{3.330287in}{2.456956in}}%
\pgfpathlineto{\pgfqpoint{3.364745in}{2.408687in}}%
\pgfpathlineto{\pgfqpoint{3.397042in}{2.359976in}}%
\pgfpathlineto{\pgfqpoint{3.427069in}{2.310836in}}%
\pgfpathlineto{\pgfqpoint{3.454648in}{2.261271in}}%
\pgfpathlineto{\pgfqpoint{3.479540in}{2.211286in}}%
\pgfpathlineto{\pgfqpoint{3.501399in}{2.160882in}}%
\pgfpathlineto{\pgfqpoint{3.519721in}{2.110065in}}%
\pgfpathlineto{\pgfqpoint{3.533782in}{2.058852in}}%
\pgfpathlineto{\pgfqpoint{3.542499in}{2.007298in}}%
\pgfpathlineto{\pgfqpoint{3.544209in}{1.955555in}}%
\pgfpathlineto{\pgfqpoint{3.536210in}{1.904036in}}%
\pgfpathlineto{\pgfqpoint{3.513814in}{1.853962in}}%
\pgfpathlineto{\pgfqpoint{3.513814in}{1.853962in}}%
\pgfpathlineto{\pgfqpoint{3.481995in}{1.818413in}}%
\pgfpathlineto{\pgfqpoint{3.481995in}{1.818413in}}%
\pgfpathlineto{\pgfqpoint{3.443550in}{1.794782in}}%
\pgfpathlineto{\pgfqpoint{3.443550in}{1.794782in}}%
\pgfusepath{stroke}%
\end{pgfscope}%
\begin{pgfscope}%
\pgfpathrectangle{\pgfqpoint{0.647939in}{0.492442in}}{\pgfqpoint{4.273799in}{2.331163in}}%
\pgfusepath{clip}%
\pgfsetbuttcap%
\pgfsetroundjoin%
\pgfsetlinewidth{0.301125pt}%
\definecolor{currentstroke}{rgb}{0.500000,0.500000,0.500000}%
\pgfsetstrokecolor{currentstroke}%
\pgfsetstrokeopacity{0.300000}%
\pgfsetdash{}{0pt}%
\pgfpathmoveto{\pgfqpoint{2.784839in}{2.823605in}}%
\pgfpathlineto{\pgfqpoint{2.784839in}{2.823605in}}%
\pgfpathlineto{\pgfqpoint{2.847373in}{2.784641in}}%
\pgfpathlineto{\pgfqpoint{2.906858in}{2.744279in}}%
\pgfpathlineto{\pgfqpoint{2.963335in}{2.702653in}}%
\pgfpathlineto{\pgfqpoint{3.016891in}{2.659889in}}%
\pgfpathlineto{\pgfqpoint{3.067620in}{2.616108in}}%
\pgfpathlineto{\pgfqpoint{3.115614in}{2.571417in}}%
\pgfpathlineto{\pgfqpoint{3.160952in}{2.525909in}}%
\pgfpathlineto{\pgfqpoint{3.203697in}{2.479660in}}%
\pgfpathlineto{\pgfqpoint{3.243881in}{2.432731in}}%
\pgfpathlineto{\pgfqpoint{3.281499in}{2.385172in}}%
\pgfpathlineto{\pgfqpoint{3.316508in}{2.337024in}}%
\pgfpathlineto{\pgfqpoint{3.348813in}{2.288319in}}%
\pgfpathlineto{\pgfqpoint{3.378238in}{2.239075in}}%
\pgfpathlineto{\pgfqpoint{3.404505in}{2.189303in}}%
\pgfusepath{stroke}%
\end{pgfscope}%
\begin{pgfscope}%
\pgfpathrectangle{\pgfqpoint{0.647939in}{0.492442in}}{\pgfqpoint{4.273799in}{2.331163in}}%
\pgfusepath{clip}%
\pgfsetbuttcap%
\pgfsetroundjoin%
\pgfsetlinewidth{0.301125pt}%
\definecolor{currentstroke}{rgb}{0.500000,0.500000,0.500000}%
\pgfsetstrokecolor{currentstroke}%
\pgfsetstrokeopacity{0.300000}%
\pgfsetdash{}{0pt}%
\pgfpathmoveto{\pgfqpoint{2.590575in}{2.823605in}}%
\pgfpathlineto{\pgfqpoint{2.590575in}{2.823605in}}%
\pgfpathlineto{\pgfqpoint{2.665331in}{2.791699in}}%
\pgfpathlineto{\pgfqpoint{2.736646in}{2.757544in}}%
\pgfpathlineto{\pgfqpoint{2.804344in}{2.721254in}}%
\pgfpathlineto{\pgfqpoint{2.868347in}{2.683020in}}%
\pgfpathlineto{\pgfqpoint{2.928700in}{2.643056in}}%
\pgfpathlineto{\pgfqpoint{2.985518in}{2.601573in}}%
\pgfusepath{stroke}%
\end{pgfscope}%
\begin{pgfscope}%
\pgfpathrectangle{\pgfqpoint{0.647939in}{0.492442in}}{\pgfqpoint{4.273799in}{2.331163in}}%
\pgfusepath{clip}%
\pgfsetbuttcap%
\pgfsetroundjoin%
\pgfsetlinewidth{0.301125pt}%
\definecolor{currentstroke}{rgb}{0.500000,0.500000,0.500000}%
\pgfsetstrokecolor{currentstroke}%
\pgfsetstrokeopacity{0.300000}%
\pgfsetdash{}{0pt}%
\pgfpathmoveto{\pgfqpoint{2.299180in}{2.823605in}}%
\pgfpathlineto{\pgfqpoint{2.299180in}{2.823605in}}%
\pgfpathlineto{\pgfqpoint{2.386431in}{2.803170in}}%
\pgfpathlineto{\pgfqpoint{2.472226in}{2.781019in}}%
\pgfpathlineto{\pgfqpoint{2.555578in}{2.756281in}}%
\pgfpathlineto{\pgfqpoint{2.635664in}{2.728524in}}%
\pgfpathlineto{\pgfqpoint{2.711862in}{2.697681in}}%
\pgfpathlineto{\pgfqpoint{2.783792in}{2.663939in}}%
\pgfpathlineto{\pgfqpoint{2.851372in}{2.627607in}}%
\pgfpathlineto{\pgfqpoint{2.914656in}{2.589029in}}%
\pgfpathlineto{\pgfqpoint{2.973786in}{2.548532in}}%
\pgfpathlineto{\pgfqpoint{3.028964in}{2.506400in}}%
\pgfpathlineto{\pgfqpoint{3.080378in}{2.462873in}}%
\pgfpathlineto{\pgfqpoint{3.128195in}{2.418141in}}%
\pgfpathlineto{\pgfqpoint{3.172534in}{2.372354in}}%
\pgfpathlineto{\pgfqpoint{3.213455in}{2.325626in}}%
\pgfpathlineto{\pgfqpoint{3.250939in}{2.278047in}}%
\pgfpathlineto{\pgfqpoint{3.284870in}{2.229679in}}%
\pgfpathlineto{\pgfqpoint{3.315010in}{2.180570in}}%
\pgfpathlineto{\pgfqpoint{3.340948in}{2.130758in}}%
\pgfpathlineto{\pgfqpoint{3.361978in}{2.080269in}}%
\pgfpathlineto{\pgfqpoint{3.376935in}{2.029153in}}%
\pgfpathlineto{\pgfqpoint{3.383750in}{1.977555in}}%
\pgfpathlineto{\pgfqpoint{3.378301in}{1.926018in}}%
\pgfpathlineto{\pgfqpoint{3.378301in}{1.926018in}}%
\pgfpathlineto{\pgfqpoint{3.359615in}{1.886786in}}%
\pgfpathlineto{\pgfqpoint{3.359615in}{1.886786in}}%
\pgfpathlineto{\pgfqpoint{3.333344in}{1.862684in}}%
\pgfpathlineto{\pgfqpoint{3.333344in}{1.862684in}}%
\pgfusepath{stroke}%
\end{pgfscope}%
\begin{pgfscope}%
\pgfpathrectangle{\pgfqpoint{0.647939in}{0.492442in}}{\pgfqpoint{4.273799in}{2.331163in}}%
\pgfusepath{clip}%
\pgfsetbuttcap%
\pgfsetroundjoin%
\pgfsetlinewidth{0.301125pt}%
\definecolor{currentstroke}{rgb}{0.500000,0.500000,0.500000}%
\pgfsetstrokecolor{currentstroke}%
\pgfsetstrokeopacity{0.300000}%
\pgfsetdash{}{0pt}%
\pgfpathmoveto{\pgfqpoint{2.007784in}{2.823605in}}%
\pgfpathlineto{\pgfqpoint{2.007784in}{2.823605in}}%
\pgfpathlineto{\pgfqpoint{2.094496in}{2.802646in}}%
\pgfpathlineto{\pgfqpoint{2.183964in}{2.785357in}}%
\pgfpathlineto{\pgfqpoint{2.274521in}{2.769783in}}%
\pgfpathlineto{\pgfqpoint{2.364889in}{2.753889in}}%
\pgfpathlineto{\pgfqpoint{2.453945in}{2.735985in}}%
\pgfpathlineto{\pgfqpoint{2.540596in}{2.714918in}}%
\pgfpathlineto{\pgfqpoint{2.623835in}{2.690126in}}%
\pgfpathlineto{\pgfqpoint{2.702876in}{2.661549in}}%
\pgfpathlineto{\pgfqpoint{2.777262in}{2.629464in}}%
\pgfusepath{stroke}%
\end{pgfscope}%
\begin{pgfscope}%
\pgfpathrectangle{\pgfqpoint{0.647939in}{0.492442in}}{\pgfqpoint{4.273799in}{2.331163in}}%
\pgfusepath{clip}%
\pgfsetbuttcap%
\pgfsetroundjoin%
\pgfsetlinewidth{0.301125pt}%
\definecolor{currentstroke}{rgb}{0.500000,0.500000,0.500000}%
\pgfsetstrokecolor{currentstroke}%
\pgfsetstrokeopacity{0.300000}%
\pgfsetdash{}{0pt}%
\pgfpathmoveto{\pgfqpoint{1.813521in}{2.823605in}}%
\pgfpathlineto{\pgfqpoint{1.813521in}{2.823605in}}%
\pgfpathlineto{\pgfqpoint{1.888889in}{2.792307in}}%
\pgfpathlineto{\pgfqpoint{1.971716in}{2.767303in}}%
\pgfpathlineto{\pgfqpoint{2.060131in}{2.748747in}}%
\pgfpathlineto{\pgfqpoint{2.151693in}{2.735201in}}%
\pgfpathlineto{\pgfqpoint{2.244478in}{2.724220in}}%
\pgfpathlineto{\pgfqpoint{2.337241in}{2.713195in}}%
\pgfpathlineto{\pgfqpoint{2.428974in}{2.699960in}}%
\pgfpathlineto{\pgfqpoint{2.518585in}{2.683044in}}%
\pgfusepath{stroke}%
\end{pgfscope}%
\begin{pgfscope}%
\pgfpathrectangle{\pgfqpoint{0.647939in}{0.492442in}}{\pgfqpoint{4.273799in}{2.331163in}}%
\pgfusepath{clip}%
\pgfsetbuttcap%
\pgfsetroundjoin%
\pgfsetlinewidth{0.301125pt}%
\definecolor{currentstroke}{rgb}{0.500000,0.500000,0.500000}%
\pgfsetstrokecolor{currentstroke}%
\pgfsetstrokeopacity{0.300000}%
\pgfsetdash{}{0pt}%
\pgfpathmoveto{\pgfqpoint{1.619257in}{2.823605in}}%
\pgfpathlineto{\pgfqpoint{1.619257in}{2.823605in}}%
\pgfpathlineto{\pgfqpoint{1.673463in}{2.781143in}}%
\pgfpathlineto{\pgfqpoint{1.735017in}{2.741817in}}%
\pgfpathlineto{\pgfqpoint{1.805715in}{2.707502in}}%
\pgfpathlineto{\pgfqpoint{1.886440in}{2.680772in}}%
\pgfpathlineto{\pgfqpoint{1.975249in}{2.663545in}}%
\pgfpathlineto{\pgfqpoint{2.066363in}{2.655067in}}%
\pgfpathlineto{\pgfqpoint{2.161034in}{2.651752in}}%
\pgfpathlineto{\pgfqpoint{2.255917in}{2.650106in}}%
\pgfpathlineto{\pgfqpoint{2.350622in}{2.646964in}}%
\pgfpathlineto{\pgfqpoint{2.444546in}{2.639989in}}%
\pgfpathlineto{\pgfqpoint{2.536539in}{2.627669in}}%
\pgfpathlineto{\pgfqpoint{2.625103in}{2.609379in}}%
\pgfpathlineto{\pgfqpoint{2.708896in}{2.585294in}}%
\pgfpathlineto{\pgfqpoint{2.787078in}{2.556113in}}%
\pgfpathlineto{\pgfqpoint{2.859412in}{2.522710in}}%
\pgfpathlineto{\pgfqpoint{2.926049in}{2.485903in}}%
\pgfpathlineto{\pgfqpoint{2.987284in}{2.446383in}}%
\pgfpathlineto{\pgfqpoint{3.043487in}{2.404684in}}%
\pgfusepath{stroke}%
\end{pgfscope}%
\begin{pgfscope}%
\pgfpathrectangle{\pgfqpoint{0.647939in}{0.492442in}}{\pgfqpoint{4.273799in}{2.331163in}}%
\pgfusepath{clip}%
\pgfsetbuttcap%
\pgfsetroundjoin%
\pgfsetlinewidth{0.301125pt}%
\definecolor{currentstroke}{rgb}{0.500000,0.500000,0.500000}%
\pgfsetstrokecolor{currentstroke}%
\pgfsetstrokeopacity{0.300000}%
\pgfsetdash{}{0pt}%
\pgfpathmoveto{\pgfqpoint{1.522125in}{2.823605in}}%
\pgfpathlineto{\pgfqpoint{1.522125in}{2.823605in}}%
\pgfpathlineto{\pgfqpoint{1.565880in}{2.777655in}}%
\pgfpathlineto{\pgfqpoint{1.614349in}{2.733157in}}%
\pgfpathlineto{\pgfqpoint{1.669214in}{2.690976in}}%
\pgfpathlineto{\pgfqpoint{1.732925in}{2.652753in}}%
\pgfpathlineto{\pgfqpoint{1.807907in}{2.621487in}}%
\pgfpathlineto{\pgfqpoint{1.892166in}{2.601498in}}%
\pgfpathlineto{\pgfqpoint{1.971952in}{2.593791in}}%
\pgfpathlineto{\pgfqpoint{2.056222in}{2.593487in}}%
\pgfpathlineto{\pgfqpoint{2.150700in}{2.598092in}}%
\pgfpathlineto{\pgfqpoint{2.245040in}{2.603775in}}%
\pgfpathlineto{\pgfqpoint{2.339692in}{2.607204in}}%
\pgfpathlineto{\pgfqpoint{2.434410in}{2.605891in}}%
\pgfusepath{stroke}%
\end{pgfscope}%
\begin{pgfscope}%
\pgfpathrectangle{\pgfqpoint{0.647939in}{0.492442in}}{\pgfqpoint{4.273799in}{2.331163in}}%
\pgfusepath{clip}%
\pgfsetbuttcap%
\pgfsetroundjoin%
\pgfsetlinewidth{0.301125pt}%
\definecolor{currentstroke}{rgb}{0.500000,0.500000,0.500000}%
\pgfsetstrokecolor{currentstroke}%
\pgfsetstrokeopacity{0.300000}%
\pgfsetdash{}{0pt}%
\pgfpathmoveto{\pgfqpoint{1.424993in}{2.823605in}}%
\pgfpathlineto{\pgfqpoint{1.424993in}{2.823605in}}%
\pgfpathlineto{\pgfqpoint{1.460017in}{2.775461in}}%
\pgfpathlineto{\pgfqpoint{1.497549in}{2.727891in}}%
\pgfpathlineto{\pgfqpoint{1.538389in}{2.681151in}}%
\pgfpathlineto{\pgfqpoint{1.583762in}{2.635695in}}%
\pgfpathlineto{\pgfqpoint{1.635657in}{2.592410in}}%
\pgfpathlineto{\pgfqpoint{1.697275in}{2.553238in}}%
\pgfpathlineto{\pgfqpoint{1.772639in}{2.522584in}}%
\pgfpathlineto{\pgfqpoint{1.772639in}{2.522584in}}%
\pgfpathlineto{\pgfqpoint{1.836359in}{2.509618in}}%
\pgfpathlineto{\pgfqpoint{1.904397in}{2.506178in}}%
\pgfpathlineto{\pgfqpoint{1.973435in}{2.510529in}}%
\pgfpathlineto{\pgfqpoint{2.055419in}{2.521780in}}%
\pgfpathlineto{\pgfqpoint{2.145986in}{2.537239in}}%
\pgfpathlineto{\pgfqpoint{2.236908in}{2.552045in}}%
\pgfpathlineto{\pgfqpoint{2.329398in}{2.563424in}}%
\pgfpathlineto{\pgfqpoint{2.423522in}{2.569054in}}%
\pgfpathlineto{\pgfqpoint{2.517986in}{2.567122in}}%
\pgfpathlineto{\pgfqpoint{2.610525in}{2.556774in}}%
\pgfpathlineto{\pgfqpoint{2.698722in}{2.538297in}}%
\pgfpathlineto{\pgfqpoint{2.780946in}{2.512808in}}%
\pgfusepath{stroke}%
\end{pgfscope}%
\begin{pgfscope}%
\pgfpathrectangle{\pgfqpoint{0.647939in}{0.492442in}}{\pgfqpoint{4.273799in}{2.331163in}}%
\pgfusepath{clip}%
\pgfsetbuttcap%
\pgfsetroundjoin%
\pgfsetlinewidth{0.301125pt}%
\definecolor{currentstroke}{rgb}{0.500000,0.500000,0.500000}%
\pgfsetstrokecolor{currentstroke}%
\pgfsetstrokeopacity{0.300000}%
\pgfsetdash{}{0pt}%
\pgfpathmoveto{\pgfqpoint{1.327862in}{2.823605in}}%
\pgfpathlineto{\pgfqpoint{1.327862in}{2.823605in}}%
\pgfpathlineto{\pgfqpoint{1.355969in}{2.774128in}}%
\pgfpathlineto{\pgfqpoint{1.385226in}{2.724849in}}%
\pgfpathlineto{\pgfqpoint{1.415888in}{2.675826in}}%
\pgfpathlineto{\pgfqpoint{1.448309in}{2.627147in}}%
\pgfpathlineto{\pgfqpoint{1.483049in}{2.578953in}}%
\pgfpathlineto{\pgfqpoint{1.520998in}{2.531494in}}%
\pgfpathlineto{\pgfqpoint{1.563661in}{2.485264in}}%
\pgfpathlineto{\pgfqpoint{1.613825in}{2.441429in}}%
\pgfpathlineto{\pgfqpoint{1.677028in}{2.403398in}}%
\pgfpathlineto{\pgfqpoint{1.677028in}{2.403398in}}%
\pgfpathlineto{\pgfqpoint{1.727916in}{2.386506in}}%
\pgfpathlineto{\pgfqpoint{1.787643in}{2.380479in}}%
\pgfpathlineto{\pgfqpoint{1.839749in}{2.384671in}}%
\pgfpathlineto{\pgfqpoint{1.894635in}{2.395942in}}%
\pgfpathlineto{\pgfqpoint{1.961696in}{2.415490in}}%
\pgfpathlineto{\pgfqpoint{2.042443in}{2.442601in}}%
\pgfpathlineto{\pgfqpoint{2.123279in}{2.469743in}}%
\pgfpathlineto{\pgfqpoint{2.206342in}{2.494767in}}%
\pgfusepath{stroke}%
\end{pgfscope}%
\begin{pgfscope}%
\pgfpathrectangle{\pgfqpoint{0.647939in}{0.492442in}}{\pgfqpoint{4.273799in}{2.331163in}}%
\pgfusepath{clip}%
\pgfsetbuttcap%
\pgfsetroundjoin%
\pgfsetlinewidth{0.301125pt}%
\definecolor{currentstroke}{rgb}{0.500000,0.500000,0.500000}%
\pgfsetstrokecolor{currentstroke}%
\pgfsetstrokeopacity{0.300000}%
\pgfsetdash{}{0pt}%
\pgfpathmoveto{\pgfqpoint{1.230730in}{2.823605in}}%
\pgfpathlineto{\pgfqpoint{1.230730in}{2.823605in}}%
\pgfpathlineto{\pgfqpoint{1.253513in}{2.773316in}}%
\pgfpathlineto{\pgfqpoint{1.276672in}{2.723079in}}%
\pgfpathlineto{\pgfqpoint{1.300250in}{2.672900in}}%
\pgfpathlineto{\pgfqpoint{1.324309in}{2.622790in}}%
\pgfpathlineto{\pgfqpoint{1.348924in}{2.572761in}}%
\pgfpathlineto{\pgfqpoint{1.374209in}{2.522832in}}%
\pgfpathlineto{\pgfqpoint{1.400308in}{2.473031in}}%
\pgfpathlineto{\pgfqpoint{1.427462in}{2.423397in}}%
\pgfpathlineto{\pgfqpoint{1.456025in}{2.374003in}}%
\pgfpathlineto{\pgfqpoint{1.486594in}{2.324984in}}%
\pgfpathlineto{\pgfqpoint{1.520415in}{2.276624in}}%
\pgfpathlineto{\pgfqpoint{1.560448in}{2.229748in}}%
\pgfpathlineto{\pgfqpoint{1.560448in}{2.229748in}}%
\pgfpathlineto{\pgfqpoint{1.604587in}{2.194736in}}%
\pgfpathlineto{\pgfqpoint{1.604587in}{2.194736in}}%
\pgfpathlineto{\pgfqpoint{1.637070in}{2.181741in}}%
\pgfpathlineto{\pgfqpoint{1.637070in}{2.181741in}}%
\pgfpathlineto{\pgfqpoint{1.668929in}{2.179362in}}%
\pgfpathlineto{\pgfqpoint{1.699840in}{2.185041in}}%
\pgfpathlineto{\pgfqpoint{1.731891in}{2.197105in}}%
\pgfpathlineto{\pgfqpoint{1.772167in}{2.217999in}}%
\pgfpathlineto{\pgfqpoint{1.829816in}{2.253213in}}%
\pgfusepath{stroke}%
\end{pgfscope}%
\begin{pgfscope}%
\pgfpathrectangle{\pgfqpoint{0.647939in}{0.492442in}}{\pgfqpoint{4.273799in}{2.331163in}}%
\pgfusepath{clip}%
\pgfsetbuttcap%
\pgfsetroundjoin%
\pgfsetlinewidth{0.301125pt}%
\definecolor{currentstroke}{rgb}{0.500000,0.500000,0.500000}%
\pgfsetstrokecolor{currentstroke}%
\pgfsetstrokeopacity{0.300000}%
\pgfsetdash{}{0pt}%
\pgfpathmoveto{\pgfqpoint{1.133598in}{2.823605in}}%
\pgfpathlineto{\pgfqpoint{1.133598in}{2.823605in}}%
\pgfpathlineto{\pgfqpoint{1.152264in}{2.772814in}}%
\pgfpathlineto{\pgfqpoint{1.170907in}{2.722019in}}%
\pgfpathlineto{\pgfqpoint{1.189501in}{2.671220in}}%
\pgfpathlineto{\pgfqpoint{1.208018in}{2.620412in}}%
\pgfpathlineto{\pgfqpoint{1.226420in}{2.569591in}}%
\pgfpathlineto{\pgfqpoint{1.244656in}{2.518754in}}%
\pgfpathlineto{\pgfqpoint{1.262671in}{2.467893in}}%
\pgfpathlineto{\pgfqpoint{1.280383in}{2.417001in}}%
\pgfpathlineto{\pgfqpoint{1.297697in}{2.366068in}}%
\pgfpathlineto{\pgfqpoint{1.314472in}{2.315082in}}%
\pgfpathlineto{\pgfqpoint{1.330520in}{2.264027in}}%
\pgfpathlineto{\pgfqpoint{1.345593in}{2.212884in}}%
\pgfpathlineto{\pgfqpoint{1.359329in}{2.161630in}}%
\pgfpathlineto{\pgfqpoint{1.371225in}{2.110242in}}%
\pgfpathlineto{\pgfqpoint{1.380584in}{2.058699in}}%
\pgfpathlineto{\pgfqpoint{1.386431in}{2.007009in}}%
\pgfpathlineto{\pgfqpoint{1.387558in}{1.955232in}}%
\pgfpathlineto{\pgfqpoint{1.382767in}{1.903529in}}%
\pgfpathlineto{\pgfqpoint{1.371416in}{1.852142in}}%
\pgfpathlineto{\pgfqpoint{1.353924in}{1.801289in}}%
\pgfpathlineto{\pgfqpoint{1.331509in}{1.751002in}}%
\pgfusepath{stroke}%
\end{pgfscope}%
\begin{pgfscope}%
\pgfpathrectangle{\pgfqpoint{0.647939in}{0.492442in}}{\pgfqpoint{4.273799in}{2.331163in}}%
\pgfusepath{clip}%
\pgfsetbuttcap%
\pgfsetroundjoin%
\pgfsetlinewidth{0.301125pt}%
\definecolor{currentstroke}{rgb}{0.500000,0.500000,0.500000}%
\pgfsetstrokecolor{currentstroke}%
\pgfsetstrokeopacity{0.300000}%
\pgfsetdash{}{0pt}%
\pgfpathmoveto{\pgfqpoint{1.036466in}{2.823605in}}%
\pgfpathlineto{\pgfqpoint{1.036466in}{2.823605in}}%
\pgfpathlineto{\pgfqpoint{1.051940in}{2.772495in}}%
\pgfpathlineto{\pgfqpoint{1.067187in}{2.721364in}}%
\pgfpathlineto{\pgfqpoint{1.082173in}{2.670210in}}%
\pgfpathlineto{\pgfqpoint{1.096849in}{2.619030in}}%
\pgfpathlineto{\pgfqpoint{1.111157in}{2.567819in}}%
\pgfpathlineto{\pgfqpoint{1.125041in}{2.516572in}}%
\pgfpathlineto{\pgfqpoint{1.138420in}{2.465286in}}%
\pgfpathlineto{\pgfqpoint{1.151205in}{2.413956in}}%
\pgfpathlineto{\pgfqpoint{1.163297in}{2.362575in}}%
\pgfpathlineto{\pgfqpoint{1.174574in}{2.311139in}}%
\pgfpathlineto{\pgfqpoint{1.184887in}{2.259644in}}%
\pgfpathlineto{\pgfqpoint{1.194075in}{2.208085in}}%
\pgfpathlineto{\pgfqpoint{1.201947in}{2.156462in}}%
\pgfpathlineto{\pgfqpoint{1.208289in}{2.104776in}}%
\pgfpathlineto{\pgfqpoint{1.212860in}{2.053036in}}%
\pgfpathlineto{\pgfqpoint{1.215407in}{2.001255in}}%
\pgfpathlineto{\pgfqpoint{1.215679in}{1.949456in}}%
\pgfpathlineto{\pgfqpoint{1.213449in}{1.897672in}}%
\pgfpathlineto{\pgfqpoint{1.208542in}{1.845944in}}%
\pgfpathlineto{\pgfqpoint{1.200870in}{1.794318in}}%
\pgfpathlineto{\pgfqpoint{1.190452in}{1.742835in}}%
\pgfpathlineto{\pgfqpoint{1.177409in}{1.691532in}}%
\pgfpathlineto{\pgfqpoint{1.161951in}{1.640427in}}%
\pgfpathlineto{\pgfqpoint{1.144379in}{1.589528in}}%
\pgfpathlineto{\pgfqpoint{1.125006in}{1.538824in}}%
\pgfpathlineto{\pgfqpoint{1.104155in}{1.488293in}}%
\pgfpathlineto{\pgfqpoint{1.082125in}{1.437912in}}%
\pgfpathlineto{\pgfqpoint{1.059181in}{1.387651in}}%
\pgfpathlineto{\pgfqpoint{1.035545in}{1.337484in}}%
\pgfpathlineto{\pgfqpoint{1.011410in}{1.287388in}}%
\pgfpathlineto{\pgfqpoint{0.986932in}{1.237343in}}%
\pgfpathlineto{\pgfqpoint{0.962226in}{1.187331in}}%
\pgfpathlineto{\pgfqpoint{0.937396in}{1.137339in}}%
\pgfusepath{stroke}%
\end{pgfscope}%
\begin{pgfscope}%
\pgfpathrectangle{\pgfqpoint{0.647939in}{0.492442in}}{\pgfqpoint{4.273799in}{2.331163in}}%
\pgfusepath{clip}%
\pgfsetbuttcap%
\pgfsetroundjoin%
\pgfsetlinewidth{0.301125pt}%
\definecolor{currentstroke}{rgb}{0.500000,0.500000,0.500000}%
\pgfsetstrokecolor{currentstroke}%
\pgfsetstrokeopacity{0.300000}%
\pgfsetdash{}{0pt}%
\pgfpathmoveto{\pgfqpoint{0.939334in}{2.823605in}}%
\pgfpathlineto{\pgfqpoint{0.939334in}{2.823605in}}%
\pgfpathlineto{\pgfqpoint{0.952298in}{2.772287in}}%
\pgfpathlineto{\pgfqpoint{0.964948in}{2.720945in}}%
\pgfpathlineto{\pgfqpoint{0.977244in}{2.669579in}}%
\pgfpathlineto{\pgfqpoint{0.989147in}{2.618184in}}%
\pgfpathlineto{\pgfqpoint{1.000609in}{2.566760in}}%
\pgfpathlineto{\pgfqpoint{1.011572in}{2.515303in}}%
\pgfpathlineto{\pgfqpoint{1.021977in}{2.463812in}}%
\pgfpathlineto{\pgfqpoint{1.031762in}{2.412285in}}%
\pgfpathlineto{\pgfqpoint{1.040851in}{2.360720in}}%
\pgfpathlineto{\pgfqpoint{1.049163in}{2.309116in}}%
\pgfpathlineto{\pgfqpoint{1.056608in}{2.257473in}}%
\pgfpathlineto{\pgfqpoint{1.063092in}{2.205791in}}%
\pgfpathlineto{\pgfqpoint{1.068515in}{2.154073in}}%
\pgfpathlineto{\pgfqpoint{1.072771in}{2.102323in}}%
\pgfpathlineto{\pgfqpoint{1.075751in}{2.050546in}}%
\pgfpathlineto{\pgfqpoint{1.077345in}{1.998752in}}%
\pgfpathlineto{\pgfqpoint{1.077452in}{1.946950in}}%
\pgfpathlineto{\pgfqpoint{1.075979in}{1.895155in}}%
\pgfpathlineto{\pgfqpoint{1.072850in}{1.843382in}}%
\pgfpathlineto{\pgfqpoint{1.068015in}{1.791648in}}%
\pgfpathlineto{\pgfqpoint{1.061448in}{1.739971in}}%
\pgfpathlineto{\pgfqpoint{1.053157in}{1.688369in}}%
\pgfpathlineto{\pgfqpoint{1.043177in}{1.636856in}}%
\pgfpathlineto{\pgfqpoint{1.031577in}{1.585443in}}%
\pgfpathlineto{\pgfqpoint{1.018462in}{1.534140in}}%
\pgfpathlineto{\pgfqpoint{1.003950in}{1.482949in}}%
\pgfusepath{stroke}%
\end{pgfscope}%
\begin{pgfscope}%
\pgfpathrectangle{\pgfqpoint{0.647939in}{0.492442in}}{\pgfqpoint{4.273799in}{2.331163in}}%
\pgfusepath{clip}%
\pgfsetbuttcap%
\pgfsetroundjoin%
\pgfsetlinewidth{0.301125pt}%
\definecolor{currentstroke}{rgb}{0.500000,0.500000,0.500000}%
\pgfsetstrokecolor{currentstroke}%
\pgfsetstrokeopacity{0.300000}%
\pgfsetdash{}{0pt}%
\pgfpathmoveto{\pgfqpoint{0.842203in}{2.823605in}}%
\pgfpathlineto{\pgfqpoint{0.842203in}{2.823605in}}%
\pgfpathlineto{\pgfqpoint{0.853170in}{2.772149in}}%
\pgfpathlineto{\pgfqpoint{0.863795in}{2.720671in}}%
\pgfpathlineto{\pgfqpoint{0.874043in}{2.669170in}}%
\pgfpathlineto{\pgfqpoint{0.883876in}{2.617645in}}%
\pgfpathlineto{\pgfqpoint{0.893259in}{2.566095in}}%
\pgfpathlineto{\pgfqpoint{0.902152in}{2.514520in}}%
\pgfpathlineto{\pgfqpoint{0.910510in}{2.462918in}}%
\pgfpathlineto{\pgfqpoint{0.918286in}{2.411288in}}%
\pgfpathlineto{\pgfqpoint{0.925428in}{2.359632in}}%
\pgfpathlineto{\pgfqpoint{0.931885in}{2.307949in}}%
\pgfpathlineto{\pgfqpoint{0.937602in}{2.256240in}}%
\pgfpathlineto{\pgfqpoint{0.942520in}{2.204507in}}%
\pgfpathlineto{\pgfqpoint{0.946583in}{2.152751in}}%
\pgfpathlineto{\pgfqpoint{0.949729in}{2.100977in}}%
\pgfpathlineto{\pgfqpoint{0.951899in}{2.049187in}}%
\pgfpathlineto{\pgfqpoint{0.953035in}{1.997388in}}%
\pgfpathlineto{\pgfqpoint{0.953086in}{1.945586in}}%
\pgfpathlineto{\pgfqpoint{0.952002in}{1.893787in}}%
\pgfpathlineto{\pgfqpoint{0.949743in}{1.841999in}}%
\pgfpathlineto{\pgfqpoint{0.946278in}{1.790231in}}%
\pgfpathlineto{\pgfqpoint{0.941587in}{1.738492in}}%
\pgfpathlineto{\pgfqpoint{0.935663in}{1.686791in}}%
\pgfpathlineto{\pgfqpoint{0.928513in}{1.635136in}}%
\pgfpathlineto{\pgfqpoint{0.920153in}{1.583535in}}%
\pgfpathlineto{\pgfqpoint{0.910614in}{1.531995in}}%
\pgfpathlineto{\pgfqpoint{0.899942in}{1.480521in}}%
\pgfpathlineto{\pgfqpoint{0.888196in}{1.429117in}}%
\pgfpathlineto{\pgfqpoint{0.875440in}{1.377786in}}%
\pgfpathlineto{\pgfqpoint{0.861737in}{1.326526in}}%
\pgfpathlineto{\pgfqpoint{0.847170in}{1.275337in}}%
\pgfpathlineto{\pgfqpoint{0.831816in}{1.224218in}}%
\pgfpathlineto{\pgfqpoint{0.815747in}{1.173163in}}%
\pgfpathlineto{\pgfqpoint{0.799044in}{1.122169in}}%
\pgfpathlineto{\pgfqpoint{0.781776in}{1.071231in}}%
\pgfpathlineto{\pgfqpoint{0.764013in}{1.020344in}}%
\pgfpathlineto{\pgfqpoint{0.745823in}{0.969502in}}%
\pgfpathlineto{\pgfqpoint{0.727260in}{0.918700in}}%
\pgfpathlineto{\pgfqpoint{0.708382in}{0.867933in}}%
\pgfpathlineto{\pgfqpoint{0.689236in}{0.817194in}}%
\pgfpathlineto{\pgfqpoint{0.669870in}{0.766481in}}%
\pgfpathlineto{\pgfqpoint{0.650322in}{0.715789in}}%
\pgfpathlineto{\pgfqpoint{0.647939in}{0.709635in}}%
\pgfusepath{stroke}%
\end{pgfscope}%
\begin{pgfscope}%
\pgfpathrectangle{\pgfqpoint{0.647939in}{0.492442in}}{\pgfqpoint{4.273799in}{2.331163in}}%
\pgfusepath{clip}%
\pgfsetbuttcap%
\pgfsetroundjoin%
\pgfsetlinewidth{0.301125pt}%
\definecolor{currentstroke}{rgb}{0.500000,0.500000,0.500000}%
\pgfsetstrokecolor{currentstroke}%
\pgfsetstrokeopacity{0.300000}%
\pgfsetdash{}{0pt}%
\pgfpathmoveto{\pgfqpoint{0.745071in}{2.823605in}}%
\pgfpathlineto{\pgfqpoint{0.745071in}{2.823605in}}%
\pgfpathlineto{\pgfqpoint{0.754437in}{2.772055in}}%
\pgfpathlineto{\pgfqpoint{0.763455in}{2.720485in}}%
\pgfpathlineto{\pgfqpoint{0.772103in}{2.668897in}}%
\pgfpathlineto{\pgfqpoint{0.780351in}{2.617290in}}%
\pgfpathlineto{\pgfqpoint{0.788170in}{2.565662in}}%
\pgfpathlineto{\pgfqpoint{0.795529in}{2.514015in}}%
\pgfpathlineto{\pgfqpoint{0.802397in}{2.462347in}}%
\pgfpathlineto{\pgfqpoint{0.808742in}{2.410660in}}%
\pgfpathlineto{\pgfqpoint{0.814529in}{2.358953in}}%
\pgfpathlineto{\pgfqpoint{0.819722in}{2.307227in}}%
\pgfpathlineto{\pgfqpoint{0.824286in}{2.255483in}}%
\pgfpathlineto{\pgfqpoint{0.828182in}{2.203724in}}%
\pgfpathlineto{\pgfqpoint{0.831375in}{2.151950in}}%
\pgfpathlineto{\pgfqpoint{0.833826in}{2.100164in}}%
\pgfpathlineto{\pgfqpoint{0.835503in}{2.048369in}}%
\pgfpathlineto{\pgfqpoint{0.836370in}{1.996568in}}%
\pgfpathlineto{\pgfqpoint{0.836396in}{1.944765in}}%
\pgfpathlineto{\pgfqpoint{0.835553in}{1.892964in}}%
\pgfpathlineto{\pgfqpoint{0.833817in}{1.841170in}}%
\pgfpathlineto{\pgfqpoint{0.831168in}{1.789387in}}%
\pgfpathlineto{\pgfqpoint{0.827590in}{1.737621in}}%
\pgfpathlineto{\pgfqpoint{0.823076in}{1.685877in}}%
\pgfpathlineto{\pgfqpoint{0.817623in}{1.634159in}}%
\pgfpathlineto{\pgfqpoint{0.811236in}{1.582474in}}%
\pgfpathlineto{\pgfqpoint{0.803929in}{1.530825in}}%
\pgfpathlineto{\pgfqpoint{0.795718in}{1.479216in}}%
\pgfpathlineto{\pgfqpoint{0.786626in}{1.427652in}}%
\pgfpathlineto{\pgfqpoint{0.776683in}{1.376134in}}%
\pgfpathlineto{\pgfqpoint{0.765926in}{1.324665in}}%
\pgfpathlineto{\pgfqpoint{0.754397in}{1.273246in}}%
\pgfpathlineto{\pgfqpoint{0.742137in}{1.221877in}}%
\pgfpathlineto{\pgfqpoint{0.729188in}{1.170558in}}%
\pgfpathlineto{\pgfqpoint{0.715602in}{1.119288in}}%
\pgfusepath{stroke}%
\end{pgfscope}%
\begin{pgfscope}%
\pgfpathrectangle{\pgfqpoint{0.647939in}{0.492442in}}{\pgfqpoint{4.273799in}{2.331163in}}%
\pgfusepath{clip}%
\pgfsetbuttcap%
\pgfsetroundjoin%
\pgfsetlinewidth{0.301125pt}%
\definecolor{currentstroke}{rgb}{0.500000,0.500000,0.500000}%
\pgfsetstrokecolor{currentstroke}%
\pgfsetstrokeopacity{0.300000}%
\pgfsetdash{}{0pt}%
\pgfpathmoveto{\pgfqpoint{0.647939in}{2.823605in}}%
\pgfpathlineto{\pgfqpoint{0.647939in}{2.823605in}}%
\pgfpathlineto{\pgfqpoint{0.656005in}{2.771989in}}%
\pgfpathlineto{\pgfqpoint{0.663736in}{2.720358in}}%
\pgfpathlineto{\pgfqpoint{0.671113in}{2.668711in}}%
\pgfpathlineto{\pgfqpoint{0.678116in}{2.617048in}}%
\pgfpathlineto{\pgfqpoint{0.684723in}{2.565371in}}%
\pgfpathlineto{\pgfqpoint{0.690912in}{2.513677in}}%
\pgfpathlineto{\pgfqpoint{0.696660in}{2.461969in}}%
\pgfpathlineto{\pgfqpoint{0.701942in}{2.410246in}}%
\pgfpathlineto{\pgfqpoint{0.706734in}{2.358508in}}%
\pgfpathlineto{\pgfqpoint{0.711012in}{2.306757in}}%
\pgfpathlineto{\pgfqpoint{0.714752in}{2.254994in}}%
\pgfpathlineto{\pgfqpoint{0.717928in}{2.203220in}}%
\pgfpathlineto{\pgfqpoint{0.720518in}{2.151436in}}%
\pgfpathlineto{\pgfqpoint{0.722496in}{2.099644in}}%
\pgfpathlineto{\pgfqpoint{0.723841in}{2.047846in}}%
\pgfpathlineto{\pgfqpoint{0.724531in}{1.996044in}}%
\pgfpathlineto{\pgfqpoint{0.724546in}{1.944241in}}%
\pgfpathlineto{\pgfqpoint{0.723868in}{1.892439in}}%
\pgfpathlineto{\pgfqpoint{0.722480in}{1.840641in}}%
\pgfpathlineto{\pgfqpoint{0.720370in}{1.788851in}}%
\pgfpathlineto{\pgfqpoint{0.717527in}{1.737071in}}%
\pgfpathlineto{\pgfqpoint{0.713945in}{1.685304in}}%
\pgfpathlineto{\pgfqpoint{0.709619in}{1.633555in}}%
\pgfpathlineto{\pgfqpoint{0.704550in}{1.581826in}}%
\pgfpathlineto{\pgfqpoint{0.698741in}{1.530120in}}%
\pgfpathlineto{\pgfqpoint{0.692199in}{1.478440in}}%
\pgfpathlineto{\pgfqpoint{0.684936in}{1.426789in}}%
\pgfpathlineto{\pgfqpoint{0.676969in}{1.375168in}}%
\pgfpathlineto{\pgfqpoint{0.668315in}{1.323581in}}%
\pgfpathlineto{\pgfqpoint{0.658995in}{1.272028in}}%
\pgfpathlineto{\pgfqpoint{0.649031in}{1.220511in}}%
\pgfpathlineto{\pgfqpoint{0.647939in}{1.215040in}}%
\pgfusepath{stroke}%
\end{pgfscope}%
\begin{pgfscope}%
\pgfpathrectangle{\pgfqpoint{0.647939in}{0.492442in}}{\pgfqpoint{4.273799in}{2.331163in}}%
\pgfusepath{clip}%
\pgfsetbuttcap%
\pgfsetroundjoin%
\pgfsetlinewidth{0.301125pt}%
\definecolor{currentstroke}{rgb}{0.500000,0.500000,0.500000}%
\pgfsetstrokecolor{currentstroke}%
\pgfsetstrokeopacity{0.300000}%
\pgfsetdash{}{0pt}%
\pgfpathmoveto{\pgfqpoint{0.647939in}{2.452738in}}%
\pgfpathlineto{\pgfqpoint{0.647939in}{2.452738in}}%
\pgfpathlineto{\pgfqpoint{0.652729in}{2.401001in}}%
\pgfpathlineto{\pgfqpoint{0.657055in}{2.349251in}}%
\pgfpathlineto{\pgfqpoint{0.660896in}{2.297490in}}%
\pgfpathlineto{\pgfqpoint{0.664233in}{2.245719in}}%
\pgfpathlineto{\pgfqpoint{0.667045in}{2.193938in}}%
\pgfpathlineto{\pgfqpoint{0.669312in}{2.142150in}}%
\pgfpathlineto{\pgfqpoint{0.671015in}{2.090355in}}%
\pgfpathlineto{\pgfqpoint{0.672134in}{2.038555in}}%
\pgfpathlineto{\pgfqpoint{0.672653in}{1.986752in}}%
\pgfpathlineto{\pgfqpoint{0.672554in}{1.934949in}}%
\pgfpathlineto{\pgfqpoint{0.671825in}{1.883147in}}%
\pgfpathlineto{\pgfqpoint{0.670451in}{1.831349in}}%
\pgfpathlineto{\pgfqpoint{0.668422in}{1.779558in}}%
\pgfpathlineto{\pgfqpoint{0.665729in}{1.727776in}}%
\pgfpathlineto{\pgfqpoint{0.662368in}{1.676005in}}%
\pgfpathlineto{\pgfqpoint{0.658333in}{1.624249in}}%
\pgfpathlineto{\pgfqpoint{0.653625in}{1.572509in}}%
\pgfusepath{stroke}%
\end{pgfscope}%
\begin{pgfscope}%
\pgfpathrectangle{\pgfqpoint{0.647939in}{0.492442in}}{\pgfqpoint{4.273799in}{2.331163in}}%
\pgfusepath{clip}%
\pgfsetbuttcap%
\pgfsetroundjoin%
\pgfsetlinewidth{0.301125pt}%
\definecolor{currentstroke}{rgb}{0.500000,0.500000,0.500000}%
\pgfsetstrokecolor{currentstroke}%
\pgfsetstrokeopacity{0.300000}%
\pgfsetdash{}{0pt}%
\pgfpathmoveto{\pgfqpoint{4.389140in}{2.823605in}}%
\pgfpathlineto{\pgfqpoint{4.430537in}{2.801742in}}%
\pgfpathlineto{\pgfqpoint{4.477597in}{2.780812in}}%
\pgfpathlineto{\pgfqpoint{4.523831in}{2.768051in}}%
\pgfpathlineto{\pgfqpoint{4.577364in}{2.763645in}}%
\pgfpathlineto{\pgfqpoint{4.630343in}{2.770624in}}%
\pgfpathlineto{\pgfqpoint{4.630343in}{2.770624in}}%
\pgfpathlineto{\pgfqpoint{4.630343in}{2.770624in}}%
\pgfpathlineto{\pgfqpoint{4.683958in}{2.789169in}}%
\pgfpathlineto{\pgfqpoint{4.727954in}{2.812625in}}%
\pgfpathlineto{\pgfqpoint{4.745876in}{2.823605in}}%
\pgfusepath{stroke}%
\end{pgfscope}%
\begin{pgfscope}%
\pgfpathrectangle{\pgfqpoint{0.647939in}{0.492442in}}{\pgfqpoint{4.273799in}{2.331163in}}%
\pgfusepath{clip}%
\pgfsetbuttcap%
\pgfsetroundjoin%
\pgfsetlinewidth{0.301125pt}%
\definecolor{currentstroke}{rgb}{0.500000,0.500000,0.500000}%
\pgfsetstrokecolor{currentstroke}%
\pgfsetstrokeopacity{0.300000}%
\pgfsetdash{}{0pt}%
\pgfpathmoveto{\pgfqpoint{4.142377in}{0.492442in}}%
\pgfpathlineto{\pgfqpoint{4.108514in}{0.522133in}}%
\pgfpathlineto{\pgfqpoint{4.057490in}{0.565821in}}%
\pgfpathlineto{\pgfqpoint{4.004759in}{0.608902in}}%
\pgfpathlineto{\pgfqpoint{3.950420in}{0.651385in}}%
\pgfpathlineto{\pgfqpoint{3.894603in}{0.693294in}}%
\pgfpathlineto{\pgfqpoint{3.837504in}{0.734688in}}%
\pgfusepath{stroke}%
\end{pgfscope}%
\begin{pgfscope}%
\pgfpathrectangle{\pgfqpoint{0.647939in}{0.492442in}}{\pgfqpoint{4.273799in}{2.331163in}}%
\pgfusepath{clip}%
\pgfsetbuttcap%
\pgfsetroundjoin%
\pgfsetlinewidth{0.301125pt}%
\definecolor{currentstroke}{rgb}{0.500000,0.500000,0.500000}%
\pgfsetstrokecolor{currentstroke}%
\pgfsetstrokeopacity{0.300000}%
\pgfsetdash{}{0pt}%
\pgfpathmoveto{\pgfqpoint{1.619257in}{2.664662in}}%
\pgfpathlineto{\pgfqpoint{1.676072in}{2.623298in}}%
\pgfpathlineto{\pgfqpoint{1.743564in}{2.587215in}}%
\pgfpathlineto{\pgfqpoint{1.743564in}{2.587215in}}%
\pgfpathlineto{\pgfqpoint{1.810797in}{2.564201in}}%
\pgfpathlineto{\pgfqpoint{1.887186in}{2.551262in}}%
\pgfpathlineto{\pgfqpoint{1.960520in}{2.548746in}}%
\pgfpathlineto{\pgfqpoint{2.040003in}{2.553216in}}%
\pgfusepath{stroke}%
\end{pgfscope}%
\begin{pgfscope}%
\pgfpathrectangle{\pgfqpoint{0.647939in}{0.492442in}}{\pgfqpoint{4.273799in}{2.331163in}}%
\pgfusepath{clip}%
\pgfsetbuttcap%
\pgfsetroundjoin%
\pgfsetlinewidth{0.301125pt}%
\definecolor{currentstroke}{rgb}{0.500000,0.500000,0.500000}%
\pgfsetstrokecolor{currentstroke}%
\pgfsetstrokeopacity{0.300000}%
\pgfsetdash{}{0pt}%
\pgfpathmoveto{\pgfqpoint{4.144684in}{0.704366in}}%
\pgfpathlineto{\pgfqpoint{4.094102in}{0.748205in}}%
\pgfpathlineto{\pgfqpoint{4.041361in}{0.791279in}}%
\pgfpathlineto{\pgfqpoint{3.986457in}{0.833541in}}%
\pgfpathlineto{\pgfqpoint{3.929452in}{0.874968in}}%
\pgfpathlineto{\pgfqpoint{3.870538in}{0.915595in}}%
\pgfpathlineto{\pgfqpoint{3.810015in}{0.955514in}}%
\pgfpathlineto{\pgfqpoint{3.748252in}{0.994865in}}%
\pgfpathlineto{\pgfqpoint{3.685729in}{1.033858in}}%
\pgfpathlineto{\pgfqpoint{3.622964in}{1.072736in}}%
\pgfpathlineto{\pgfqpoint{3.560512in}{1.111762in}}%
\pgfpathlineto{\pgfqpoint{3.498900in}{1.151182in}}%
\pgfpathlineto{\pgfqpoint{3.438619in}{1.191206in}}%
\pgfpathlineto{\pgfqpoint{3.380100in}{1.231996in}}%
\pgfusepath{stroke}%
\end{pgfscope}%
\begin{pgfscope}%
\pgfpathrectangle{\pgfqpoint{0.647939in}{0.492442in}}{\pgfqpoint{4.273799in}{2.331163in}}%
\pgfusepath{clip}%
\pgfsetbuttcap%
\pgfsetroundjoin%
\pgfsetlinewidth{0.301125pt}%
\definecolor{currentstroke}{rgb}{0.500000,0.500000,0.500000}%
\pgfsetstrokecolor{currentstroke}%
\pgfsetstrokeopacity{0.300000}%
\pgfsetdash{}{0pt}%
\pgfpathmoveto{\pgfqpoint{4.558930in}{1.184310in}}%
\pgfpathlineto{\pgfqpoint{4.533211in}{1.234176in}}%
\pgfpathlineto{\pgfqpoint{4.506941in}{1.283956in}}%
\pgfpathlineto{\pgfqpoint{4.480051in}{1.333635in}}%
\pgfpathlineto{\pgfqpoint{4.452426in}{1.383195in}}%
\pgfpathlineto{\pgfqpoint{4.423938in}{1.432607in}}%
\pgfpathlineto{\pgfqpoint{4.394391in}{1.481832in}}%
\pgfpathlineto{\pgfqpoint{4.363479in}{1.530807in}}%
\pgfpathlineto{\pgfqpoint{4.330763in}{1.579427in}}%
\pgfpathlineto{\pgfqpoint{4.295485in}{1.627493in}}%
\pgfpathlineto{\pgfqpoint{4.256130in}{1.674575in}}%
\pgfpathlineto{\pgfqpoint{4.209292in}{1.719438in}}%
\pgfpathlineto{\pgfqpoint{4.209292in}{1.719438in}}%
\pgfpathlineto{\pgfqpoint{4.162298in}{1.749834in}}%
\pgfpathlineto{\pgfqpoint{4.162298in}{1.749834in}}%
\pgfpathlineto{\pgfqpoint{4.122997in}{1.763267in}}%
\pgfpathlineto{\pgfqpoint{4.122997in}{1.763267in}}%
\pgfpathlineto{\pgfqpoint{4.082334in}{1.766729in}}%
\pgfpathlineto{\pgfqpoint{4.041397in}{1.762193in}}%
\pgfusepath{stroke}%
\end{pgfscope}%
\begin{pgfscope}%
\pgfpathrectangle{\pgfqpoint{0.647939in}{0.492442in}}{\pgfqpoint{4.273799in}{2.331163in}}%
\pgfusepath{clip}%
\pgfsetbuttcap%
\pgfsetroundjoin%
\pgfsetlinewidth{0.301125pt}%
\definecolor{currentstroke}{rgb}{0.500000,0.500000,0.500000}%
\pgfsetstrokecolor{currentstroke}%
\pgfsetstrokeopacity{0.300000}%
\pgfsetdash{}{0pt}%
\pgfpathmoveto{\pgfqpoint{4.551347in}{1.554194in}}%
\pgfpathlineto{\pgfqpoint{4.533211in}{1.605043in}}%
\pgfpathlineto{\pgfqpoint{4.515759in}{1.655962in}}%
\pgfpathlineto{\pgfqpoint{4.499240in}{1.706971in}}%
\pgfpathlineto{\pgfqpoint{4.483975in}{1.758097in}}%
\pgfpathlineto{\pgfqpoint{4.470446in}{1.809366in}}%
\pgfpathlineto{\pgfqpoint{4.459324in}{1.860804in}}%
\pgfpathlineto{\pgfqpoint{4.451541in}{1.912420in}}%
\pgfpathlineto{\pgfqpoint{4.448355in}{1.964170in}}%
\pgfpathlineto{\pgfqpoint{4.451214in}{2.015911in}}%
\pgfpathlineto{\pgfqpoint{4.461250in}{2.067369in}}%
\pgfusepath{stroke}%
\end{pgfscope}%
\begin{pgfscope}%
\pgfpathrectangle{\pgfqpoint{0.647939in}{0.492442in}}{\pgfqpoint{4.273799in}{2.331163in}}%
\pgfusepath{clip}%
\pgfsetbuttcap%
\pgfsetroundjoin%
\pgfsetlinewidth{0.301125pt}%
\definecolor{currentstroke}{rgb}{0.500000,0.500000,0.500000}%
\pgfsetstrokecolor{currentstroke}%
\pgfsetstrokeopacity{0.300000}%
\pgfsetdash{}{0pt}%
\pgfpathmoveto{\pgfqpoint{4.533211in}{1.763986in}}%
\pgfpathlineto{\pgfqpoint{4.522229in}{1.815438in}}%
\pgfpathlineto{\pgfqpoint{4.513578in}{1.867019in}}%
\pgfpathlineto{\pgfqpoint{4.507908in}{1.918720in}}%
\pgfpathlineto{\pgfqpoint{4.505986in}{1.970498in}}%
\pgfpathlineto{\pgfqpoint{4.508610in}{2.022262in}}%
\pgfpathlineto{\pgfqpoint{4.516415in}{2.073864in}}%
\pgfpathlineto{\pgfqpoint{4.529628in}{2.125129in}}%
\pgfpathlineto{\pgfqpoint{4.547898in}{2.175929in}}%
\pgfusepath{stroke}%
\end{pgfscope}%
\begin{pgfscope}%
\pgfpathrectangle{\pgfqpoint{0.647939in}{0.492442in}}{\pgfqpoint{4.273799in}{2.331163in}}%
\pgfusepath{clip}%
\pgfsetbuttcap%
\pgfsetroundjoin%
\pgfsetlinewidth{0.301125pt}%
\definecolor{currentstroke}{rgb}{0.500000,0.500000,0.500000}%
\pgfsetstrokecolor{currentstroke}%
\pgfsetstrokeopacity{0.300000}%
\pgfsetdash{}{0pt}%
\pgfpathmoveto{\pgfqpoint{2.881971in}{0.757347in}}%
\pgfpathlineto{\pgfqpoint{2.843427in}{0.804691in}}%
\pgfpathlineto{\pgfqpoint{2.805952in}{0.852290in}}%
\pgfpathlineto{\pgfqpoint{2.769542in}{0.900134in}}%
\pgfpathlineto{\pgfqpoint{2.734195in}{0.948215in}}%
\pgfpathlineto{\pgfqpoint{2.699911in}{0.996524in}}%
\pgfpathlineto{\pgfqpoint{2.666694in}{1.045054in}}%
\pgfusepath{stroke}%
\end{pgfscope}%
\begin{pgfscope}%
\pgfpathrectangle{\pgfqpoint{0.647939in}{0.492442in}}{\pgfqpoint{4.273799in}{2.331163in}}%
\pgfusepath{clip}%
\pgfsetbuttcap%
\pgfsetroundjoin%
\pgfsetlinewidth{0.301125pt}%
\definecolor{currentstroke}{rgb}{0.500000,0.500000,0.500000}%
\pgfsetstrokecolor{currentstroke}%
\pgfsetstrokeopacity{0.300000}%
\pgfsetdash{}{0pt}%
\pgfpathmoveto{\pgfqpoint{4.436079in}{1.075233in}}%
\pgfpathlineto{\pgfqpoint{4.402003in}{1.123583in}}%
\pgfpathlineto{\pgfqpoint{4.366278in}{1.171577in}}%
\pgfpathlineto{\pgfqpoint{4.328586in}{1.219117in}}%
\pgfpathlineto{\pgfqpoint{4.288483in}{1.266060in}}%
\pgfpathlineto{\pgfqpoint{4.245359in}{1.312193in}}%
\pgfpathlineto{\pgfqpoint{4.198363in}{1.357176in}}%
\pgfpathlineto{\pgfqpoint{4.146282in}{1.400445in}}%
\pgfpathlineto{\pgfqpoint{4.087519in}{1.441048in}}%
\pgfpathlineto{\pgfqpoint{4.020262in}{1.477413in}}%
\pgfpathlineto{\pgfqpoint{3.943304in}{1.507378in}}%
\pgfpathlineto{\pgfqpoint{3.857927in}{1.529524in}}%
\pgfpathlineto{\pgfqpoint{3.767594in}{1.545182in}}%
\pgfpathlineto{\pgfqpoint{3.675713in}{1.558135in}}%
\pgfusepath{stroke}%
\end{pgfscope}%
\begin{pgfscope}%
\pgfpathrectangle{\pgfqpoint{0.647939in}{0.492442in}}{\pgfqpoint{4.273799in}{2.331163in}}%
\pgfusepath{clip}%
\pgfsetbuttcap%
\pgfsetroundjoin%
\pgfsetlinewidth{0.301125pt}%
\definecolor{currentstroke}{rgb}{0.500000,0.500000,0.500000}%
\pgfsetstrokecolor{currentstroke}%
\pgfsetstrokeopacity{0.300000}%
\pgfsetdash{}{0pt}%
\pgfpathmoveto{\pgfqpoint{1.569264in}{2.602623in}}%
\pgfpathlineto{\pgfqpoint{1.619257in}{2.558700in}}%
\pgfpathlineto{\pgfqpoint{1.678862in}{2.518639in}}%
\pgfpathlineto{\pgfqpoint{1.678862in}{2.518639in}}%
\pgfpathlineto{\pgfqpoint{1.740435in}{2.491115in}}%
\pgfpathlineto{\pgfqpoint{1.740435in}{2.491115in}}%
\pgfpathlineto{\pgfqpoint{1.799789in}{2.476628in}}%
\pgfpathlineto{\pgfqpoint{1.864967in}{2.472070in}}%
\pgfpathlineto{\pgfqpoint{1.928193in}{2.476063in}}%
\pgfusepath{stroke}%
\end{pgfscope}%
\begin{pgfscope}%
\pgfpathrectangle{\pgfqpoint{0.647939in}{0.492442in}}{\pgfqpoint{4.273799in}{2.331163in}}%
\pgfusepath{clip}%
\pgfsetbuttcap%
\pgfsetroundjoin%
\pgfsetlinewidth{0.301125pt}%
\definecolor{currentstroke}{rgb}{0.500000,0.500000,0.500000}%
\pgfsetstrokecolor{currentstroke}%
\pgfsetstrokeopacity{0.300000}%
\pgfsetdash{}{0pt}%
\pgfpathmoveto{\pgfqpoint{1.133598in}{2.240815in}}%
\pgfpathlineto{\pgfqpoint{1.141070in}{2.189173in}}%
\pgfpathlineto{\pgfqpoint{1.147279in}{2.137482in}}%
\pgfpathlineto{\pgfqpoint{1.152064in}{2.085746in}}%
\pgfpathlineto{\pgfqpoint{1.155255in}{2.033975in}}%
\pgfpathlineto{\pgfqpoint{1.156681in}{1.982180in}}%
\pgfpathlineto{\pgfqpoint{1.156179in}{1.930380in}}%
\pgfpathlineto{\pgfqpoint{1.153608in}{1.878599in}}%
\pgfpathlineto{\pgfqpoint{1.148865in}{1.826864in}}%
\pgfpathlineto{\pgfqpoint{1.141895in}{1.775206in}}%
\pgfusepath{stroke}%
\end{pgfscope}%
\begin{pgfscope}%
\pgfpathrectangle{\pgfqpoint{0.647939in}{0.492442in}}{\pgfqpoint{4.273799in}{2.331163in}}%
\pgfusepath{clip}%
\pgfsetbuttcap%
\pgfsetroundjoin%
\pgfsetlinewidth{0.301125pt}%
\definecolor{currentstroke}{rgb}{0.500000,0.500000,0.500000}%
\pgfsetstrokecolor{currentstroke}%
\pgfsetstrokeopacity{0.300000}%
\pgfsetdash{}{0pt}%
\pgfpathmoveto{\pgfqpoint{4.338948in}{1.658024in}}%
\pgfpathlineto{\pgfqpoint{4.307075in}{1.706796in}}%
\pgfpathlineto{\pgfqpoint{4.271820in}{1.754867in}}%
\pgfpathlineto{\pgfqpoint{4.229349in}{1.801073in}}%
\pgfpathlineto{\pgfqpoint{4.229349in}{1.801073in}}%
\pgfpathlineto{\pgfqpoint{4.194906in}{1.825688in}}%
\pgfpathlineto{\pgfqpoint{4.194906in}{1.825688in}}%
\pgfpathlineto{\pgfqpoint{4.166117in}{1.835693in}}%
\pgfpathlineto{\pgfqpoint{4.166117in}{1.835693in}}%
\pgfpathlineto{\pgfqpoint{4.135487in}{1.836308in}}%
\pgfpathlineto{\pgfqpoint{4.105859in}{1.829762in}}%
\pgfusepath{stroke}%
\end{pgfscope}%
\begin{pgfscope}%
\pgfpathrectangle{\pgfqpoint{0.647939in}{0.492442in}}{\pgfqpoint{4.273799in}{2.331163in}}%
\pgfusepath{clip}%
\pgfsetbuttcap%
\pgfsetroundjoin%
\pgfsetlinewidth{0.301125pt}%
\definecolor{currentstroke}{rgb}{0.500000,0.500000,0.500000}%
\pgfsetstrokecolor{currentstroke}%
\pgfsetstrokeopacity{0.300000}%
\pgfsetdash{}{0pt}%
\pgfpathmoveto{\pgfqpoint{1.622961in}{1.291659in}}%
\pgfpathlineto{\pgfqpoint{1.581843in}{1.338324in}}%
\pgfpathlineto{\pgfqpoint{1.535808in}{1.383545in}}%
\pgfpathlineto{\pgfqpoint{1.493665in}{1.416892in}}%
\pgfpathlineto{\pgfqpoint{1.456934in}{1.438525in}}%
\pgfpathlineto{\pgfqpoint{1.422401in}{1.451780in}}%
\pgfpathlineto{\pgfqpoint{1.380321in}{1.457924in}}%
\pgfpathlineto{\pgfqpoint{1.337626in}{1.452730in}}%
\pgfpathlineto{\pgfqpoint{1.337626in}{1.452730in}}%
\pgfpathlineto{\pgfqpoint{1.289190in}{1.433375in}}%
\pgfpathlineto{\pgfqpoint{1.289190in}{1.433375in}}%
\pgfpathlineto{\pgfqpoint{1.230730in}{1.393119in}}%
\pgfusepath{stroke}%
\end{pgfscope}%
\begin{pgfscope}%
\pgfpathrectangle{\pgfqpoint{0.647939in}{0.492442in}}{\pgfqpoint{4.273799in}{2.331163in}}%
\pgfusepath{clip}%
\pgfsetbuttcap%
\pgfsetroundjoin%
\pgfsetlinewidth{0.301125pt}%
\definecolor{currentstroke}{rgb}{0.500000,0.500000,0.500000}%
\pgfsetstrokecolor{currentstroke}%
\pgfsetstrokeopacity{0.300000}%
\pgfsetdash{}{0pt}%
\pgfpathmoveto{\pgfqpoint{4.321666in}{1.511158in}}%
\pgfpathlineto{\pgfqpoint{4.283912in}{1.558667in}}%
\pgfpathlineto{\pgfqpoint{4.241816in}{1.605043in}}%
\pgfpathlineto{\pgfqpoint{4.192514in}{1.649183in}}%
\pgfpathlineto{\pgfqpoint{4.129732in}{1.687459in}}%
\pgfpathlineto{\pgfqpoint{4.129732in}{1.687459in}}%
\pgfpathlineto{\pgfqpoint{4.081974in}{1.703296in}}%
\pgfpathlineto{\pgfqpoint{4.024922in}{1.709060in}}%
\pgfpathlineto{\pgfqpoint{3.973855in}{1.705629in}}%
\pgfpathlineto{\pgfqpoint{3.916100in}{1.696037in}}%
\pgfusepath{stroke}%
\end{pgfscope}%
\begin{pgfscope}%
\pgfpathrectangle{\pgfqpoint{0.647939in}{0.492442in}}{\pgfqpoint{4.273799in}{2.331163in}}%
\pgfusepath{clip}%
\pgfsetbuttcap%
\pgfsetroundjoin%
\pgfsetlinewidth{0.301125pt}%
\definecolor{currentstroke}{rgb}{0.500000,0.500000,0.500000}%
\pgfsetstrokecolor{currentstroke}%
\pgfsetstrokeopacity{0.300000}%
\pgfsetdash{}{0pt}%
\pgfpathmoveto{\pgfqpoint{4.151657in}{2.130434in}}%
\pgfpathlineto{\pgfqpoint{4.179301in}{2.080895in}}%
\pgfpathlineto{\pgfqpoint{4.206797in}{2.031428in}}%
\pgfpathlineto{\pgfqpoint{4.223202in}{2.002295in}}%
\pgfpathlineto{\pgfqpoint{4.234712in}{1.983345in}}%
\pgfpathlineto{\pgfqpoint{4.241816in}{1.975910in}}%
\pgfpathlineto{\pgfqpoint{4.241816in}{1.975910in}}%
\pgfpathlineto{\pgfqpoint{4.241816in}{1.975910in}}%
\pgfpathlineto{\pgfqpoint{4.247277in}{1.979573in}}%
\pgfpathlineto{\pgfqpoint{4.255794in}{1.985717in}}%
\pgfpathlineto{\pgfqpoint{4.273082in}{2.000461in}}%
\pgfusepath{stroke}%
\end{pgfscope}%
\begin{pgfscope}%
\pgfpathrectangle{\pgfqpoint{0.647939in}{0.492442in}}{\pgfqpoint{4.273799in}{2.331163in}}%
\pgfusepath{clip}%
\pgfsetbuttcap%
\pgfsetroundjoin%
\pgfsetlinewidth{0.301125pt}%
\definecolor{currentstroke}{rgb}{0.500000,0.500000,0.500000}%
\pgfsetstrokecolor{currentstroke}%
\pgfsetstrokeopacity{0.300000}%
\pgfsetdash{}{0pt}%
\pgfpathmoveto{\pgfqpoint{1.522125in}{2.452738in}}%
\pgfpathlineto{\pgfqpoint{1.564502in}{2.406448in}}%
\pgfpathlineto{\pgfqpoint{1.615960in}{2.363165in}}%
\pgfpathlineto{\pgfqpoint{1.615960in}{2.363165in}}%
\pgfpathlineto{\pgfqpoint{1.665602in}{2.335854in}}%
\pgfpathlineto{\pgfqpoint{1.665602in}{2.335854in}}%
\pgfpathlineto{\pgfqpoint{1.710482in}{2.323239in}}%
\pgfpathlineto{\pgfqpoint{1.762189in}{2.321296in}}%
\pgfpathlineto{\pgfqpoint{1.807318in}{2.328116in}}%
\pgfpathlineto{\pgfqpoint{1.856550in}{2.341953in}}%
\pgfpathlineto{\pgfqpoint{1.918595in}{2.365072in}}%
\pgfusepath{stroke}%
\end{pgfscope}%
\begin{pgfscope}%
\pgfpathrectangle{\pgfqpoint{0.647939in}{0.492442in}}{\pgfqpoint{4.273799in}{2.331163in}}%
\pgfusepath{clip}%
\pgfsetbuttcap%
\pgfsetroundjoin%
\pgfsetlinewidth{0.301125pt}%
\definecolor{currentstroke}{rgb}{0.500000,0.500000,0.500000}%
\pgfsetstrokecolor{currentstroke}%
\pgfsetstrokeopacity{0.300000}%
\pgfsetdash{}{0pt}%
\pgfpathmoveto{\pgfqpoint{1.725065in}{1.381407in}}%
\pgfpathlineto{\pgfqpoint{1.692998in}{1.430165in}}%
\pgfpathlineto{\pgfqpoint{1.659721in}{1.478676in}}%
\pgfpathlineto{\pgfqpoint{1.624498in}{1.526763in}}%
\pgfpathlineto{\pgfqpoint{1.585887in}{1.574040in}}%
\pgfpathlineto{\pgfqpoint{1.547732in}{1.613038in}}%
\pgfpathlineto{\pgfqpoint{1.516332in}{1.637686in}}%
\pgfpathlineto{\pgfqpoint{1.488562in}{1.652665in}}%
\pgfpathlineto{\pgfqpoint{1.458204in}{1.660906in}}%
\pgfpathlineto{\pgfqpoint{1.424148in}{1.659839in}}%
\pgfpathlineto{\pgfqpoint{1.424148in}{1.659839in}}%
\pgfpathlineto{\pgfqpoint{1.385075in}{1.645826in}}%
\pgfpathlineto{\pgfqpoint{1.385075in}{1.645826in}}%
\pgfpathlineto{\pgfqpoint{1.327862in}{1.605043in}}%
\pgfusepath{stroke}%
\end{pgfscope}%
\begin{pgfscope}%
\pgfpathrectangle{\pgfqpoint{0.647939in}{0.492442in}}{\pgfqpoint{4.273799in}{2.331163in}}%
\pgfusepath{clip}%
\pgfsetbuttcap%
\pgfsetroundjoin%
\pgfsetlinewidth{0.301125pt}%
\definecolor{currentstroke}{rgb}{0.500000,0.500000,0.500000}%
\pgfsetstrokecolor{currentstroke}%
\pgfsetstrokeopacity{0.300000}%
\pgfsetdash{}{0pt}%
\pgfpathmoveto{\pgfqpoint{4.241819in}{1.092227in}}%
\pgfpathlineto{\pgfqpoint{4.194994in}{1.137278in}}%
\pgfpathlineto{\pgfqpoint{4.144684in}{1.181195in}}%
\pgfpathlineto{\pgfqpoint{4.090307in}{1.223640in}}%
\pgfpathlineto{\pgfqpoint{4.031305in}{1.264188in}}%
\pgfpathlineto{\pgfqpoint{3.967146in}{1.302326in}}%
\pgfpathlineto{\pgfqpoint{3.897752in}{1.337629in}}%
\pgfpathlineto{\pgfqpoint{3.823647in}{1.369964in}}%
\pgfpathlineto{\pgfqpoint{3.745998in}{1.399751in}}%
\pgfpathlineto{\pgfqpoint{3.666448in}{1.428032in}}%
\pgfusepath{stroke}%
\end{pgfscope}%
\begin{pgfscope}%
\pgfpathrectangle{\pgfqpoint{0.647939in}{0.492442in}}{\pgfqpoint{4.273799in}{2.331163in}}%
\pgfusepath{clip}%
\pgfsetbuttcap%
\pgfsetroundjoin%
\pgfsetlinewidth{0.301125pt}%
\definecolor{currentstroke}{rgb}{0.500000,0.500000,0.500000}%
\pgfsetstrokecolor{currentstroke}%
\pgfsetstrokeopacity{0.300000}%
\pgfsetdash{}{0pt}%
\pgfpathmoveto{\pgfqpoint{4.245687in}{1.411590in}}%
\pgfpathlineto{\pgfqpoint{4.198406in}{1.456462in}}%
\pgfpathlineto{\pgfqpoint{4.144684in}{1.499081in}}%
\pgfpathlineto{\pgfqpoint{4.081772in}{1.537669in}}%
\pgfpathlineto{\pgfqpoint{4.006618in}{1.568712in}}%
\pgfpathlineto{\pgfqpoint{3.926570in}{1.587026in}}%
\pgfusepath{stroke}%
\end{pgfscope}%
\begin{pgfscope}%
\pgfpathrectangle{\pgfqpoint{0.647939in}{0.492442in}}{\pgfqpoint{4.273799in}{2.331163in}}%
\pgfusepath{clip}%
\pgfsetbuttcap%
\pgfsetroundjoin%
\pgfsetlinewidth{0.301125pt}%
\definecolor{currentstroke}{rgb}{0.500000,0.500000,0.500000}%
\pgfsetstrokecolor{currentstroke}%
\pgfsetstrokeopacity{0.300000}%
\pgfsetdash{}{0pt}%
\pgfpathmoveto{\pgfqpoint{2.502786in}{1.264964in}}%
\pgfpathlineto{\pgfqpoint{2.475994in}{1.314662in}}%
\pgfpathlineto{\pgfqpoint{2.450370in}{1.364544in}}%
\pgfpathlineto{\pgfqpoint{2.425970in}{1.414607in}}%
\pgfpathlineto{\pgfqpoint{2.402855in}{1.464851in}}%
\pgfpathlineto{\pgfqpoint{2.381105in}{1.515276in}}%
\pgfpathlineto{\pgfqpoint{2.360817in}{1.565882in}}%
\pgfpathlineto{\pgfqpoint{2.342106in}{1.616668in}}%
\pgfpathlineto{\pgfqpoint{2.325119in}{1.667634in}}%
\pgfpathlineto{\pgfqpoint{2.310027in}{1.718776in}}%
\pgfpathlineto{\pgfqpoint{2.297044in}{1.770089in}}%
\pgfpathlineto{\pgfqpoint{2.286434in}{1.821563in}}%
\pgfpathlineto{\pgfqpoint{2.278511in}{1.873179in}}%
\pgfpathlineto{\pgfqpoint{2.273663in}{1.924906in}}%
\pgfpathlineto{\pgfqpoint{2.272371in}{1.976694in}}%
\pgfpathlineto{\pgfqpoint{2.275222in}{2.028458in}}%
\pgfpathlineto{\pgfqpoint{2.282937in}{2.080067in}}%
\pgfpathlineto{\pgfqpoint{2.296407in}{2.131312in}}%
\pgfpathlineto{\pgfqpoint{2.316736in}{2.181865in}}%
\pgfpathlineto{\pgfqpoint{2.345299in}{2.231196in}}%
\pgfpathlineto{\pgfqpoint{2.383780in}{2.278437in}}%
\pgfpathlineto{\pgfqpoint{2.432532in}{2.320890in}}%
\pgfpathlineto{\pgfqpoint{2.487103in}{2.354127in}}%
\pgfpathlineto{\pgfqpoint{2.547184in}{2.378571in}}%
\pgfpathlineto{\pgfqpoint{2.614427in}{2.394271in}}%
\pgfpathlineto{\pgfqpoint{2.687707in}{2.399758in}}%
\pgfpathlineto{\pgfqpoint{2.687707in}{2.399758in}}%
\pgfpathlineto{\pgfqpoint{2.687707in}{2.399758in}}%
\pgfusepath{stroke}%
\end{pgfscope}%
\begin{pgfscope}%
\pgfpathrectangle{\pgfqpoint{0.647939in}{0.492442in}}{\pgfqpoint{4.273799in}{2.331163in}}%
\pgfusepath{clip}%
\pgfsetbuttcap%
\pgfsetroundjoin%
\pgfsetlinewidth{0.301125pt}%
\definecolor{currentstroke}{rgb}{0.500000,0.500000,0.500000}%
\pgfsetstrokecolor{currentstroke}%
\pgfsetstrokeopacity{0.300000}%
\pgfsetdash{}{0pt}%
\pgfpathmoveto{\pgfqpoint{1.401533in}{2.343988in}}%
\pgfpathlineto{\pgfqpoint{1.424993in}{2.293796in}}%
\pgfpathlineto{\pgfqpoint{1.448941in}{2.243676in}}%
\pgfpathlineto{\pgfqpoint{1.473546in}{2.193659in}}%
\pgfpathlineto{\pgfqpoint{1.499197in}{2.143812in}}%
\pgfpathlineto{\pgfqpoint{1.526869in}{2.094363in}}%
\pgfpathlineto{\pgfqpoint{1.561547in}{2.046668in}}%
\pgfpathlineto{\pgfqpoint{1.561547in}{2.046668in}}%
\pgfpathlineto{\pgfqpoint{1.577159in}{2.035330in}}%
\pgfpathlineto{\pgfqpoint{1.577159in}{2.035330in}}%
\pgfpathlineto{\pgfqpoint{1.593015in}{2.033070in}}%
\pgfpathlineto{\pgfqpoint{1.610404in}{2.039317in}}%
\pgfpathlineto{\pgfqpoint{1.626781in}{2.050067in}}%
\pgfpathlineto{\pgfqpoint{1.653936in}{2.072659in}}%
\pgfpathlineto{\pgfqpoint{1.699426in}{2.113754in}}%
\pgfpathlineto{\pgfqpoint{1.748499in}{2.157716in}}%
\pgfusepath{stroke}%
\end{pgfscope}%
\begin{pgfscope}%
\pgfpathrectangle{\pgfqpoint{0.647939in}{0.492442in}}{\pgfqpoint{4.273799in}{2.331163in}}%
\pgfusepath{clip}%
\pgfsetbuttcap%
\pgfsetroundjoin%
\pgfsetlinewidth{0.301125pt}%
\definecolor{currentstroke}{rgb}{0.500000,0.500000,0.500000}%
\pgfsetstrokecolor{currentstroke}%
\pgfsetstrokeopacity{0.300000}%
\pgfsetdash{}{0pt}%
\pgfpathmoveto{\pgfqpoint{4.008829in}{0.981417in}}%
\pgfpathlineto{\pgfqpoint{3.950420in}{1.022252in}}%
\pgfpathlineto{\pgfqpoint{3.889374in}{1.061924in}}%
\pgfpathlineto{\pgfqpoint{3.825987in}{1.100493in}}%
\pgfpathlineto{\pgfqpoint{3.760745in}{1.138132in}}%
\pgfpathlineto{\pgfqpoint{3.694277in}{1.175133in}}%
\pgfusepath{stroke}%
\end{pgfscope}%
\begin{pgfscope}%
\pgfpathrectangle{\pgfqpoint{0.647939in}{0.492442in}}{\pgfqpoint{4.273799in}{2.331163in}}%
\pgfusepath{clip}%
\pgfsetbuttcap%
\pgfsetroundjoin%
\pgfsetlinewidth{0.301125pt}%
\definecolor{currentstroke}{rgb}{0.500000,0.500000,0.500000}%
\pgfsetstrokecolor{currentstroke}%
\pgfsetstrokeopacity{0.300000}%
\pgfsetdash{}{0pt}%
\pgfpathmoveto{\pgfqpoint{3.950420in}{1.181195in}}%
\pgfpathlineto{\pgfqpoint{3.885166in}{1.218807in}}%
\pgfpathlineto{\pgfqpoint{3.816613in}{1.254638in}}%
\pgfpathlineto{\pgfqpoint{3.745531in}{1.288980in}}%
\pgfpathlineto{\pgfqpoint{3.672920in}{1.322366in}}%
\pgfpathlineto{\pgfqpoint{3.599996in}{1.355546in}}%
\pgfusepath{stroke}%
\end{pgfscope}%
\begin{pgfscope}%
\pgfpathrectangle{\pgfqpoint{0.647939in}{0.492442in}}{\pgfqpoint{4.273799in}{2.331163in}}%
\pgfusepath{clip}%
\pgfsetbuttcap%
\pgfsetroundjoin%
\pgfsetlinewidth{0.301125pt}%
\definecolor{currentstroke}{rgb}{0.500000,0.500000,0.500000}%
\pgfsetstrokecolor{currentstroke}%
\pgfsetstrokeopacity{0.300000}%
\pgfsetdash{}{0pt}%
\pgfpathmoveto{\pgfqpoint{3.659025in}{2.293796in}}%
\pgfpathlineto{\pgfqpoint{3.682063in}{2.243543in}}%
\pgfpathlineto{\pgfqpoint{3.702800in}{2.192995in}}%
\pgfpathlineto{\pgfqpoint{3.720831in}{2.142141in}}%
\pgfpathlineto{\pgfqpoint{3.735592in}{2.090977in}}%
\pgfpathlineto{\pgfqpoint{3.746290in}{2.039520in}}%
\pgfpathlineto{\pgfqpoint{3.751798in}{1.987833in}}%
\pgfpathlineto{\pgfqpoint{3.750515in}{1.936086in}}%
\pgfpathlineto{\pgfqpoint{3.740169in}{1.884679in}}%
\pgfpathlineto{\pgfqpoint{3.717666in}{1.834513in}}%
\pgfpathlineto{\pgfqpoint{3.679229in}{1.787479in}}%
\pgfpathlineto{\pgfqpoint{3.624091in}{1.748539in}}%
\pgfpathlineto{\pgfqpoint{3.562160in}{1.723766in}}%
\pgfpathlineto{\pgfqpoint{3.498947in}{1.711630in}}%
\pgfpathlineto{\pgfqpoint{3.436931in}{1.709849in}}%
\pgfpathlineto{\pgfqpoint{3.377532in}{1.716636in}}%
\pgfpathlineto{\pgfqpoint{3.319191in}{1.731317in}}%
\pgfusepath{stroke}%
\end{pgfscope}%
\begin{pgfscope}%
\pgfpathrectangle{\pgfqpoint{0.647939in}{0.492442in}}{\pgfqpoint{4.273799in}{2.331163in}}%
\pgfusepath{clip}%
\pgfsetbuttcap%
\pgfsetroundjoin%
\pgfsetlinewidth{0.301125pt}%
\definecolor{currentstroke}{rgb}{0.500000,0.500000,0.500000}%
\pgfsetstrokecolor{currentstroke}%
\pgfsetstrokeopacity{0.300000}%
\pgfsetdash{}{0pt}%
\pgfpathmoveto{\pgfqpoint{2.570346in}{1.632362in}}%
\pgfpathlineto{\pgfqpoint{2.551660in}{1.683149in}}%
\pgfpathlineto{\pgfqpoint{2.535218in}{1.734166in}}%
\pgfpathlineto{\pgfqpoint{2.521250in}{1.785401in}}%
\pgfpathlineto{\pgfqpoint{2.510058in}{1.836836in}}%
\pgfpathlineto{\pgfqpoint{2.502017in}{1.888445in}}%
\pgfpathlineto{\pgfqpoint{2.497609in}{1.940179in}}%
\pgfpathlineto{\pgfqpoint{2.497467in}{1.991965in}}%
\pgfpathlineto{\pgfqpoint{2.502435in}{2.043672in}}%
\pgfpathlineto{\pgfqpoint{2.513678in}{2.095073in}}%
\pgfpathlineto{\pgfqpoint{2.532848in}{2.145742in}}%
\pgfpathlineto{\pgfqpoint{2.562388in}{2.194835in}}%
\pgfpathlineto{\pgfqpoint{2.598578in}{2.234385in}}%
\pgfpathlineto{\pgfqpoint{2.639975in}{2.264643in}}%
\pgfpathlineto{\pgfqpoint{2.687244in}{2.286734in}}%
\pgfpathlineto{\pgfqpoint{2.743892in}{2.300759in}}%
\pgfpathlineto{\pgfqpoint{2.813684in}{2.303669in}}%
\pgfpathlineto{\pgfqpoint{2.881971in}{2.293796in}}%
\pgfpathlineto{\pgfqpoint{2.881971in}{2.293796in}}%
\pgfpathlineto{\pgfqpoint{2.881971in}{2.293796in}}%
\pgfusepath{stroke}%
\end{pgfscope}%
\begin{pgfscope}%
\pgfpathrectangle{\pgfqpoint{0.647939in}{0.492442in}}{\pgfqpoint{4.273799in}{2.331163in}}%
\pgfusepath{clip}%
\pgfsetbuttcap%
\pgfsetroundjoin%
\pgfsetlinewidth{0.301125pt}%
\definecolor{currentstroke}{rgb}{0.500000,0.500000,0.500000}%
\pgfsetstrokecolor{currentstroke}%
\pgfsetstrokeopacity{0.300000}%
\pgfsetdash{}{0pt}%
\pgfpathmoveto{\pgfqpoint{2.953887in}{1.240877in}}%
\pgfpathlineto{\pgfqpoint{2.916749in}{1.288551in}}%
\pgfpathlineto{\pgfqpoint{2.881382in}{1.336625in}}%
\pgfpathlineto{\pgfqpoint{2.847817in}{1.385082in}}%
\pgfpathlineto{\pgfqpoint{2.816092in}{1.433906in}}%
\pgfpathlineto{\pgfqpoint{2.786260in}{1.483083in}}%
\pgfpathlineto{\pgfqpoint{2.758390in}{1.532602in}}%
\pgfpathlineto{\pgfqpoint{2.732575in}{1.582451in}}%
\pgfpathlineto{\pgfqpoint{2.708931in}{1.632618in}}%
\pgfpathlineto{\pgfqpoint{2.687612in}{1.683094in}}%
\pgfpathlineto{\pgfqpoint{2.668816in}{1.733867in}}%
\pgfpathlineto{\pgfqpoint{2.652797in}{1.784921in}}%
\pgfpathlineto{\pgfqpoint{2.639892in}{1.836236in}}%
\pgfpathlineto{\pgfqpoint{2.630544in}{1.887776in}}%
\pgfpathlineto{\pgfqpoint{2.625346in}{1.939485in}}%
\pgfpathlineto{\pgfqpoint{2.625128in}{1.991262in}}%
\pgfpathlineto{\pgfqpoint{2.631086in}{2.042925in}}%
\pgfpathlineto{\pgfqpoint{2.645039in}{2.094101in}}%
\pgfpathlineto{\pgfqpoint{2.668787in}{2.142223in}}%
\pgfpathlineto{\pgfqpoint{2.698617in}{2.179457in}}%
\pgfpathlineto{\pgfqpoint{2.733451in}{2.207394in}}%
\pgfpathlineto{\pgfqpoint{2.774205in}{2.227487in}}%
\pgfpathlineto{\pgfqpoint{2.825054in}{2.239695in}}%
\pgfpathlineto{\pgfqpoint{2.881971in}{2.240815in}}%
\pgfpathlineto{\pgfqpoint{2.881971in}{2.240815in}}%
\pgfpathlineto{\pgfqpoint{2.881971in}{2.240815in}}%
\pgfpathlineto{\pgfqpoint{2.938042in}{2.231285in}}%
\pgfpathlineto{\pgfqpoint{2.987052in}{2.215071in}}%
\pgfusepath{stroke}%
\end{pgfscope}%
\begin{pgfscope}%
\pgfpathrectangle{\pgfqpoint{0.647939in}{0.492442in}}{\pgfqpoint{4.273799in}{2.331163in}}%
\pgfusepath{clip}%
\pgfsetbuttcap%
\pgfsetroundjoin%
\pgfsetlinewidth{0.301125pt}%
\definecolor{currentstroke}{rgb}{0.500000,0.500000,0.500000}%
\pgfsetstrokecolor{currentstroke}%
\pgfsetstrokeopacity{0.300000}%
\pgfsetdash{}{0pt}%
\pgfpathmoveto{\pgfqpoint{1.940457in}{1.079027in}}%
\pgfpathlineto{\pgfqpoint{1.910652in}{1.128214in}}%
\pgfpathlineto{\pgfqpoint{1.880950in}{1.177418in}}%
\pgfpathlineto{\pgfqpoint{1.851308in}{1.226634in}}%
\pgfpathlineto{\pgfqpoint{1.821660in}{1.275848in}}%
\pgfpathlineto{\pgfqpoint{1.791915in}{1.325044in}}%
\pgfusepath{stroke}%
\end{pgfscope}%
\begin{pgfscope}%
\pgfpathrectangle{\pgfqpoint{0.647939in}{0.492442in}}{\pgfqpoint{4.273799in}{2.331163in}}%
\pgfusepath{clip}%
\pgfsetbuttcap%
\pgfsetroundjoin%
\pgfsetlinewidth{0.301125pt}%
\definecolor{currentstroke}{rgb}{0.500000,0.500000,0.500000}%
\pgfsetstrokecolor{currentstroke}%
\pgfsetstrokeopacity{0.300000}%
\pgfsetdash{}{0pt}%
\pgfpathmoveto{\pgfqpoint{1.851697in}{1.881376in}}%
\pgfpathlineto{\pgfqpoint{1.845998in}{1.933072in}}%
\pgfpathlineto{\pgfqpoint{1.845094in}{1.984849in}}%
\pgfpathlineto{\pgfqpoint{1.850133in}{2.036542in}}%
\pgfpathlineto{\pgfqpoint{1.862253in}{2.087873in}}%
\pgfpathlineto{\pgfqpoint{1.882316in}{2.138448in}}%
\pgfpathlineto{\pgfqpoint{1.910652in}{2.187834in}}%
\pgfusepath{stroke}%
\end{pgfscope}%
\begin{pgfscope}%
\pgfpathrectangle{\pgfqpoint{0.647939in}{0.492442in}}{\pgfqpoint{4.273799in}{2.331163in}}%
\pgfusepath{clip}%
\pgfsetbuttcap%
\pgfsetroundjoin%
\pgfsetlinewidth{0.301125pt}%
\definecolor{currentstroke}{rgb}{0.500000,0.500000,0.500000}%
\pgfsetstrokecolor{currentstroke}%
\pgfsetstrokeopacity{0.300000}%
\pgfsetdash{}{0pt}%
\pgfpathmoveto{\pgfqpoint{1.949672in}{1.719519in}}%
\pgfpathlineto{\pgfqpoint{1.935392in}{1.770730in}}%
\pgfpathlineto{\pgfqpoint{1.923436in}{1.822115in}}%
\pgfpathlineto{\pgfqpoint{1.914295in}{1.873669in}}%
\pgfpathlineto{\pgfqpoint{1.908591in}{1.925368in}}%
\pgfpathlineto{\pgfqpoint{1.907081in}{1.977148in}}%
\pgfpathlineto{\pgfqpoint{1.910652in}{2.028891in}}%
\pgfusepath{stroke}%
\end{pgfscope}%
\begin{pgfscope}%
\pgfpathrectangle{\pgfqpoint{0.647939in}{0.492442in}}{\pgfqpoint{4.273799in}{2.331163in}}%
\pgfusepath{clip}%
\pgfsetbuttcap%
\pgfsetroundjoin%
\pgfsetlinewidth{0.301125pt}%
\definecolor{currentstroke}{rgb}{0.500000,0.500000,0.500000}%
\pgfsetstrokecolor{currentstroke}%
\pgfsetstrokeopacity{0.300000}%
\pgfsetdash{}{0pt}%
\pgfpathmoveto{\pgfqpoint{2.204859in}{1.823359in}}%
\pgfpathlineto{\pgfqpoint{2.197115in}{1.874984in}}%
\pgfpathlineto{\pgfqpoint{2.192425in}{1.926715in}}%
\pgfpathlineto{\pgfqpoint{2.191276in}{1.978504in}}%
\pgfpathlineto{\pgfqpoint{2.194253in}{2.030266in}}%
\pgfpathlineto{\pgfqpoint{2.202048in}{2.081872in}}%
\pgfusepath{stroke}%
\end{pgfscope}%
\begin{pgfscope}%
\pgfpathrectangle{\pgfqpoint{0.647939in}{0.492442in}}{\pgfqpoint{4.273799in}{2.331163in}}%
\pgfusepath{clip}%
\pgfsetbuttcap%
\pgfsetroundjoin%
\pgfsetlinewidth{0.301125pt}%
\definecolor{currentstroke}{rgb}{0.500000,0.500000,0.500000}%
\pgfsetstrokecolor{currentstroke}%
\pgfsetstrokeopacity{0.300000}%
\pgfsetdash{}{0pt}%
\pgfpathmoveto{\pgfqpoint{2.127507in}{1.248455in}}%
\pgfpathlineto{\pgfqpoint{2.101961in}{1.298349in}}%
\pgfpathlineto{\pgfqpoint{2.077163in}{1.348355in}}%
\pgfpathlineto{\pgfqpoint{2.053156in}{1.398476in}}%
\pgfpathlineto{\pgfqpoint{2.030005in}{1.448716in}}%
\pgfpathlineto{\pgfqpoint{2.007784in}{1.499081in}}%
\pgfusepath{stroke}%
\end{pgfscope}%
\begin{pgfscope}%
\pgfpathrectangle{\pgfqpoint{0.647939in}{0.492442in}}{\pgfqpoint{4.273799in}{2.331163in}}%
\pgfusepath{clip}%
\pgfsetbuttcap%
\pgfsetroundjoin%
\pgfsetlinewidth{0.301125pt}%
\definecolor{currentstroke}{rgb}{0.500000,0.500000,0.500000}%
\pgfsetstrokecolor{currentstroke}%
\pgfsetstrokeopacity{0.300000}%
\pgfsetdash{}{0pt}%
\pgfpathmoveto{\pgfqpoint{3.447066in}{2.078916in}}%
\pgfpathlineto{\pgfqpoint{3.459422in}{2.027583in}}%
\pgfpathlineto{\pgfqpoint{3.464761in}{1.975910in}}%
\pgfpathlineto{\pgfqpoint{3.460262in}{1.924273in}}%
\pgfpathlineto{\pgfqpoint{3.440414in}{1.873943in}}%
\pgfpathlineto{\pgfqpoint{3.440414in}{1.873943in}}%
\pgfpathlineto{\pgfqpoint{3.412188in}{1.841814in}}%
\pgfpathlineto{\pgfqpoint{3.412188in}{1.841814in}}%
\pgfpathlineto{\pgfqpoint{3.377931in}{1.821763in}}%
\pgfpathlineto{\pgfqpoint{3.377931in}{1.821763in}}%
\pgfpathlineto{\pgfqpoint{3.339497in}{1.811908in}}%
\pgfpathlineto{\pgfqpoint{3.296980in}{1.811333in}}%
\pgfpathlineto{\pgfqpoint{3.257335in}{1.818560in}}%
\pgfusepath{stroke}%
\end{pgfscope}%
\begin{pgfscope}%
\pgfpathrectangle{\pgfqpoint{0.647939in}{0.492442in}}{\pgfqpoint{4.273799in}{2.331163in}}%
\pgfusepath{clip}%
\pgfsetbuttcap%
\pgfsetroundjoin%
\pgfsetlinewidth{0.301125pt}%
\definecolor{currentstroke}{rgb}{0.500000,0.500000,0.500000}%
\pgfsetstrokecolor{currentstroke}%
\pgfsetstrokeopacity{0.300000}%
\pgfsetdash{}{0pt}%
\pgfpathmoveto{\pgfqpoint{3.226696in}{1.590221in}}%
\pgfpathlineto{\pgfqpoint{3.174859in}{1.633575in}}%
\pgfpathlineto{\pgfqpoint{3.128015in}{1.678588in}}%
\pgfpathlineto{\pgfqpoint{3.086290in}{1.725078in}}%
\pgfpathlineto{\pgfqpoint{3.049907in}{1.772886in}}%
\pgfpathlineto{\pgfqpoint{3.019275in}{1.821882in}}%
\pgfpathlineto{\pgfqpoint{2.995174in}{1.871941in}}%
\pgfpathlineto{\pgfqpoint{2.979102in}{1.922929in}}%
\pgfusepath{stroke}%
\end{pgfscope}%
\begin{pgfscope}%
\pgfpathrectangle{\pgfqpoint{0.647939in}{0.492442in}}{\pgfqpoint{4.273799in}{2.331163in}}%
\pgfusepath{clip}%
\pgfsetroundcap%
\pgfsetroundjoin%
\pgfsetlinewidth{0.301125pt}%
\definecolor{currentstroke}{rgb}{0.500000,0.500000,0.500000}%
\pgfsetstrokecolor{currentstroke}%
\pgfsetstrokeopacity{0.300000}%
\pgfsetdash{}{0pt}%
\pgfpathmoveto{\pgfqpoint{1.442778in}{1.536442in}}%
\pgfusepath{stroke}%
\end{pgfscope}%
\begin{pgfscope}%
\pgfpathrectangle{\pgfqpoint{0.647939in}{0.492442in}}{\pgfqpoint{4.273799in}{2.331163in}}%
\pgfusepath{clip}%
\pgfsetroundcap%
\pgfsetroundjoin%
\definecolor{currentfill}{rgb}{0.500000,0.500000,0.500000}%
\pgfsetfillcolor{currentfill}%
\pgfsetfillopacity{0.300000}%
\pgfsetlinewidth{0.301125pt}%
\definecolor{currentstroke}{rgb}{0.500000,0.500000,0.500000}%
\pgfsetstrokecolor{currentstroke}%
\pgfsetstrokeopacity{0.300000}%
\pgfsetdash{}{0pt}%
\pgfpathmoveto{\pgfqpoint{0.000000in}{0.000000in}}%
\pgfpathlineto{\pgfqpoint{0.000000in}{0.000000in}}%
\pgfpathclose%
\pgfusepath{stroke,fill}%
\end{pgfscope}%
\begin{pgfscope}%
\pgfpathrectangle{\pgfqpoint{0.647939in}{0.492442in}}{\pgfqpoint{4.273799in}{2.331163in}}%
\pgfusepath{clip}%
\pgfsetroundcap%
\pgfsetroundjoin%
\pgfsetlinewidth{0.301125pt}%
\definecolor{currentstroke}{rgb}{0.500000,0.500000,0.500000}%
\pgfsetstrokecolor{currentstroke}%
\pgfsetstrokeopacity{0.300000}%
\pgfsetdash{}{0pt}%
\pgfpathmoveto{\pgfqpoint{1.237906in}{0.900709in}}%
\pgfusepath{stroke}%
\end{pgfscope}%
\begin{pgfscope}%
\pgfpathrectangle{\pgfqpoint{0.647939in}{0.492442in}}{\pgfqpoint{4.273799in}{2.331163in}}%
\pgfusepath{clip}%
\pgfsetroundcap%
\pgfsetroundjoin%
\definecolor{currentfill}{rgb}{0.500000,0.500000,0.500000}%
\pgfsetfillcolor{currentfill}%
\pgfsetfillopacity{0.300000}%
\pgfsetlinewidth{0.301125pt}%
\definecolor{currentstroke}{rgb}{0.500000,0.500000,0.500000}%
\pgfsetstrokecolor{currentstroke}%
\pgfsetstrokeopacity{0.300000}%
\pgfsetdash{}{0pt}%
\pgfpathmoveto{\pgfqpoint{0.000000in}{0.000000in}}%
\pgfpathlineto{\pgfqpoint{0.000000in}{0.000000in}}%
\pgfpathclose%
\pgfusepath{stroke,fill}%
\end{pgfscope}%
\begin{pgfscope}%
\pgfpathrectangle{\pgfqpoint{0.647939in}{0.492442in}}{\pgfqpoint{4.273799in}{2.331163in}}%
\pgfusepath{clip}%
\pgfsetroundcap%
\pgfsetroundjoin%
\pgfsetlinewidth{0.301125pt}%
\definecolor{currentstroke}{rgb}{0.500000,0.500000,0.500000}%
\pgfsetstrokecolor{currentstroke}%
\pgfsetstrokeopacity{0.300000}%
\pgfsetdash{}{0pt}%
\pgfpathmoveto{\pgfqpoint{1.204978in}{0.677679in}}%
\pgfusepath{stroke}%
\end{pgfscope}%
\begin{pgfscope}%
\pgfpathrectangle{\pgfqpoint{0.647939in}{0.492442in}}{\pgfqpoint{4.273799in}{2.331163in}}%
\pgfusepath{clip}%
\pgfsetroundcap%
\pgfsetroundjoin%
\definecolor{currentfill}{rgb}{0.500000,0.500000,0.500000}%
\pgfsetfillcolor{currentfill}%
\pgfsetfillopacity{0.300000}%
\pgfsetlinewidth{0.301125pt}%
\definecolor{currentstroke}{rgb}{0.500000,0.500000,0.500000}%
\pgfsetstrokecolor{currentstroke}%
\pgfsetstrokeopacity{0.300000}%
\pgfsetdash{}{0pt}%
\pgfpathmoveto{\pgfqpoint{0.000000in}{0.000000in}}%
\pgfpathlineto{\pgfqpoint{0.000000in}{0.000000in}}%
\pgfpathclose%
\pgfusepath{stroke,fill}%
\end{pgfscope}%
\begin{pgfscope}%
\pgfpathrectangle{\pgfqpoint{0.647939in}{0.492442in}}{\pgfqpoint{4.273799in}{2.331163in}}%
\pgfusepath{clip}%
\pgfsetroundcap%
\pgfsetroundjoin%
\pgfsetlinewidth{0.301125pt}%
\definecolor{currentstroke}{rgb}{0.500000,0.500000,0.500000}%
\pgfsetstrokecolor{currentstroke}%
\pgfsetstrokeopacity{0.300000}%
\pgfsetdash{}{0pt}%
\pgfpathmoveto{\pgfqpoint{1.158791in}{0.563261in}}%
\pgfusepath{stroke}%
\end{pgfscope}%
\begin{pgfscope}%
\pgfpathrectangle{\pgfqpoint{0.647939in}{0.492442in}}{\pgfqpoint{4.273799in}{2.331163in}}%
\pgfusepath{clip}%
\pgfsetroundcap%
\pgfsetroundjoin%
\definecolor{currentfill}{rgb}{0.500000,0.500000,0.500000}%
\pgfsetfillcolor{currentfill}%
\pgfsetfillopacity{0.300000}%
\pgfsetlinewidth{0.301125pt}%
\definecolor{currentstroke}{rgb}{0.500000,0.500000,0.500000}%
\pgfsetstrokecolor{currentstroke}%
\pgfsetstrokeopacity{0.300000}%
\pgfsetdash{}{0pt}%
\pgfpathmoveto{\pgfqpoint{0.000000in}{0.000000in}}%
\pgfpathlineto{\pgfqpoint{0.000000in}{0.000000in}}%
\pgfpathclose%
\pgfusepath{stroke,fill}%
\end{pgfscope}%
\begin{pgfscope}%
\pgfpathrectangle{\pgfqpoint{0.647939in}{0.492442in}}{\pgfqpoint{4.273799in}{2.331163in}}%
\pgfusepath{clip}%
\pgfsetroundcap%
\pgfsetroundjoin%
\pgfsetlinewidth{0.301125pt}%
\definecolor{currentstroke}{rgb}{0.500000,0.500000,0.500000}%
\pgfsetstrokecolor{currentstroke}%
\pgfsetstrokeopacity{0.300000}%
\pgfsetdash{}{0pt}%
\pgfpathmoveto{\pgfqpoint{1.339251in}{0.721208in}}%
\pgfusepath{stroke}%
\end{pgfscope}%
\begin{pgfscope}%
\pgfpathrectangle{\pgfqpoint{0.647939in}{0.492442in}}{\pgfqpoint{4.273799in}{2.331163in}}%
\pgfusepath{clip}%
\pgfsetroundcap%
\pgfsetroundjoin%
\definecolor{currentfill}{rgb}{0.500000,0.500000,0.500000}%
\pgfsetfillcolor{currentfill}%
\pgfsetfillopacity{0.300000}%
\pgfsetlinewidth{0.301125pt}%
\definecolor{currentstroke}{rgb}{0.500000,0.500000,0.500000}%
\pgfsetstrokecolor{currentstroke}%
\pgfsetstrokeopacity{0.300000}%
\pgfsetdash{}{0pt}%
\pgfpathmoveto{\pgfqpoint{0.000000in}{0.000000in}}%
\pgfpathlineto{\pgfqpoint{0.000000in}{0.000000in}}%
\pgfpathclose%
\pgfusepath{stroke,fill}%
\end{pgfscope}%
\begin{pgfscope}%
\pgfpathrectangle{\pgfqpoint{0.647939in}{0.492442in}}{\pgfqpoint{4.273799in}{2.331163in}}%
\pgfusepath{clip}%
\pgfsetroundcap%
\pgfsetroundjoin%
\pgfsetlinewidth{0.301125pt}%
\definecolor{currentstroke}{rgb}{0.500000,0.500000,0.500000}%
\pgfsetstrokecolor{currentstroke}%
\pgfsetstrokeopacity{0.300000}%
\pgfsetdash{}{0pt}%
\pgfpathmoveto{\pgfqpoint{1.556274in}{0.794340in}}%
\pgfusepath{stroke}%
\end{pgfscope}%
\begin{pgfscope}%
\pgfpathrectangle{\pgfqpoint{0.647939in}{0.492442in}}{\pgfqpoint{4.273799in}{2.331163in}}%
\pgfusepath{clip}%
\pgfsetroundcap%
\pgfsetroundjoin%
\definecolor{currentfill}{rgb}{0.500000,0.500000,0.500000}%
\pgfsetfillcolor{currentfill}%
\pgfsetfillopacity{0.300000}%
\pgfsetlinewidth{0.301125pt}%
\definecolor{currentstroke}{rgb}{0.500000,0.500000,0.500000}%
\pgfsetstrokecolor{currentstroke}%
\pgfsetstrokeopacity{0.300000}%
\pgfsetdash{}{0pt}%
\pgfpathmoveto{\pgfqpoint{0.000000in}{0.000000in}}%
\pgfpathlineto{\pgfqpoint{0.000000in}{0.000000in}}%
\pgfpathclose%
\pgfusepath{stroke,fill}%
\end{pgfscope}%
\begin{pgfscope}%
\pgfpathrectangle{\pgfqpoint{0.647939in}{0.492442in}}{\pgfqpoint{4.273799in}{2.331163in}}%
\pgfusepath{clip}%
\pgfsetroundcap%
\pgfsetroundjoin%
\pgfsetlinewidth{0.301125pt}%
\definecolor{currentstroke}{rgb}{0.500000,0.500000,0.500000}%
\pgfsetstrokecolor{currentstroke}%
\pgfsetstrokeopacity{0.300000}%
\pgfsetdash{}{0pt}%
\pgfpathmoveto{\pgfqpoint{1.445981in}{1.022808in}}%
\pgfusepath{stroke}%
\end{pgfscope}%
\begin{pgfscope}%
\pgfpathrectangle{\pgfqpoint{0.647939in}{0.492442in}}{\pgfqpoint{4.273799in}{2.331163in}}%
\pgfusepath{clip}%
\pgfsetroundcap%
\pgfsetroundjoin%
\definecolor{currentfill}{rgb}{0.500000,0.500000,0.500000}%
\pgfsetfillcolor{currentfill}%
\pgfsetfillopacity{0.300000}%
\pgfsetlinewidth{0.301125pt}%
\definecolor{currentstroke}{rgb}{0.500000,0.500000,0.500000}%
\pgfsetstrokecolor{currentstroke}%
\pgfsetstrokeopacity{0.300000}%
\pgfsetdash{}{0pt}%
\pgfpathmoveto{\pgfqpoint{0.000000in}{0.000000in}}%
\pgfpathlineto{\pgfqpoint{0.000000in}{0.000000in}}%
\pgfpathclose%
\pgfusepath{stroke,fill}%
\end{pgfscope}%
\begin{pgfscope}%
\pgfpathrectangle{\pgfqpoint{0.647939in}{0.492442in}}{\pgfqpoint{4.273799in}{2.331163in}}%
\pgfusepath{clip}%
\pgfsetroundcap%
\pgfsetroundjoin%
\pgfsetlinewidth{0.301125pt}%
\definecolor{currentstroke}{rgb}{0.500000,0.500000,0.500000}%
\pgfsetstrokecolor{currentstroke}%
\pgfsetstrokeopacity{0.300000}%
\pgfsetdash{}{0pt}%
\pgfpathmoveto{\pgfqpoint{1.594017in}{1.040138in}}%
\pgfusepath{stroke}%
\end{pgfscope}%
\begin{pgfscope}%
\pgfpathrectangle{\pgfqpoint{0.647939in}{0.492442in}}{\pgfqpoint{4.273799in}{2.331163in}}%
\pgfusepath{clip}%
\pgfsetroundcap%
\pgfsetroundjoin%
\definecolor{currentfill}{rgb}{0.500000,0.500000,0.500000}%
\pgfsetfillcolor{currentfill}%
\pgfsetfillopacity{0.300000}%
\pgfsetlinewidth{0.301125pt}%
\definecolor{currentstroke}{rgb}{0.500000,0.500000,0.500000}%
\pgfsetstrokecolor{currentstroke}%
\pgfsetstrokeopacity{0.300000}%
\pgfsetdash{}{0pt}%
\pgfpathmoveto{\pgfqpoint{0.000000in}{0.000000in}}%
\pgfpathlineto{\pgfqpoint{0.000000in}{0.000000in}}%
\pgfpathclose%
\pgfusepath{stroke,fill}%
\end{pgfscope}%
\begin{pgfscope}%
\pgfpathrectangle{\pgfqpoint{0.647939in}{0.492442in}}{\pgfqpoint{4.273799in}{2.331163in}}%
\pgfusepath{clip}%
\pgfsetroundcap%
\pgfsetroundjoin%
\pgfsetlinewidth{0.301125pt}%
\definecolor{currentstroke}{rgb}{0.500000,0.500000,0.500000}%
\pgfsetstrokecolor{currentstroke}%
\pgfsetstrokeopacity{0.300000}%
\pgfsetdash{}{0pt}%
\pgfpathmoveto{\pgfqpoint{1.718821in}{1.047662in}}%
\pgfusepath{stroke}%
\end{pgfscope}%
\begin{pgfscope}%
\pgfpathrectangle{\pgfqpoint{0.647939in}{0.492442in}}{\pgfqpoint{4.273799in}{2.331163in}}%
\pgfusepath{clip}%
\pgfsetroundcap%
\pgfsetroundjoin%
\definecolor{currentfill}{rgb}{0.500000,0.500000,0.500000}%
\pgfsetfillcolor{currentfill}%
\pgfsetfillopacity{0.300000}%
\pgfsetlinewidth{0.301125pt}%
\definecolor{currentstroke}{rgb}{0.500000,0.500000,0.500000}%
\pgfsetstrokecolor{currentstroke}%
\pgfsetstrokeopacity{0.300000}%
\pgfsetdash{}{0pt}%
\pgfpathmoveto{\pgfqpoint{0.000000in}{0.000000in}}%
\pgfpathlineto{\pgfqpoint{0.000000in}{0.000000in}}%
\pgfpathclose%
\pgfusepath{stroke,fill}%
\end{pgfscope}%
\begin{pgfscope}%
\pgfpathrectangle{\pgfqpoint{0.647939in}{0.492442in}}{\pgfqpoint{4.273799in}{2.331163in}}%
\pgfusepath{clip}%
\pgfsetroundcap%
\pgfsetroundjoin%
\pgfsetlinewidth{0.301125pt}%
\definecolor{currentstroke}{rgb}{0.500000,0.500000,0.500000}%
\pgfsetstrokecolor{currentstroke}%
\pgfsetstrokeopacity{0.300000}%
\pgfsetdash{}{0pt}%
\pgfpathmoveto{\pgfqpoint{2.185161in}{0.661432in}}%
\pgfusepath{stroke}%
\end{pgfscope}%
\begin{pgfscope}%
\pgfpathrectangle{\pgfqpoint{0.647939in}{0.492442in}}{\pgfqpoint{4.273799in}{2.331163in}}%
\pgfusepath{clip}%
\pgfsetroundcap%
\pgfsetroundjoin%
\definecolor{currentfill}{rgb}{0.500000,0.500000,0.500000}%
\pgfsetfillcolor{currentfill}%
\pgfsetfillopacity{0.300000}%
\pgfsetlinewidth{0.301125pt}%
\definecolor{currentstroke}{rgb}{0.500000,0.500000,0.500000}%
\pgfsetstrokecolor{currentstroke}%
\pgfsetstrokeopacity{0.300000}%
\pgfsetdash{}{0pt}%
\pgfpathmoveto{\pgfqpoint{0.000000in}{0.000000in}}%
\pgfpathlineto{\pgfqpoint{0.000000in}{0.000000in}}%
\pgfpathclose%
\pgfusepath{stroke,fill}%
\end{pgfscope}%
\begin{pgfscope}%
\pgfpathrectangle{\pgfqpoint{0.647939in}{0.492442in}}{\pgfqpoint{4.273799in}{2.331163in}}%
\pgfusepath{clip}%
\pgfsetroundcap%
\pgfsetroundjoin%
\pgfsetlinewidth{0.301125pt}%
\definecolor{currentstroke}{rgb}{0.500000,0.500000,0.500000}%
\pgfsetstrokecolor{currentstroke}%
\pgfsetstrokeopacity{0.300000}%
\pgfsetdash{}{0pt}%
\pgfpathmoveto{\pgfqpoint{1.812931in}{1.450441in}}%
\pgfusepath{stroke}%
\end{pgfscope}%
\begin{pgfscope}%
\pgfpathrectangle{\pgfqpoint{0.647939in}{0.492442in}}{\pgfqpoint{4.273799in}{2.331163in}}%
\pgfusepath{clip}%
\pgfsetroundcap%
\pgfsetroundjoin%
\definecolor{currentfill}{rgb}{0.500000,0.500000,0.500000}%
\pgfsetfillcolor{currentfill}%
\pgfsetfillopacity{0.300000}%
\pgfsetlinewidth{0.301125pt}%
\definecolor{currentstroke}{rgb}{0.500000,0.500000,0.500000}%
\pgfsetstrokecolor{currentstroke}%
\pgfsetstrokeopacity{0.300000}%
\pgfsetdash{}{0pt}%
\pgfpathmoveto{\pgfqpoint{0.000000in}{0.000000in}}%
\pgfpathlineto{\pgfqpoint{0.000000in}{0.000000in}}%
\pgfpathclose%
\pgfusepath{stroke,fill}%
\end{pgfscope}%
\begin{pgfscope}%
\pgfpathrectangle{\pgfqpoint{0.647939in}{0.492442in}}{\pgfqpoint{4.273799in}{2.331163in}}%
\pgfusepath{clip}%
\pgfsetroundcap%
\pgfsetroundjoin%
\pgfsetlinewidth{0.301125pt}%
\definecolor{currentstroke}{rgb}{0.500000,0.500000,0.500000}%
\pgfsetstrokecolor{currentstroke}%
\pgfsetstrokeopacity{0.300000}%
\pgfsetdash{}{0pt}%
\pgfpathmoveto{\pgfqpoint{1.951532in}{2.293379in}}%
\pgfusepath{stroke}%
\end{pgfscope}%
\begin{pgfscope}%
\pgfpathrectangle{\pgfqpoint{0.647939in}{0.492442in}}{\pgfqpoint{4.273799in}{2.331163in}}%
\pgfusepath{clip}%
\pgfsetroundcap%
\pgfsetroundjoin%
\definecolor{currentfill}{rgb}{0.500000,0.500000,0.500000}%
\pgfsetfillcolor{currentfill}%
\pgfsetfillopacity{0.300000}%
\pgfsetlinewidth{0.301125pt}%
\definecolor{currentstroke}{rgb}{0.500000,0.500000,0.500000}%
\pgfsetstrokecolor{currentstroke}%
\pgfsetstrokeopacity{0.300000}%
\pgfsetdash{}{0pt}%
\pgfpathmoveto{\pgfqpoint{0.000000in}{0.000000in}}%
\pgfpathlineto{\pgfqpoint{0.000000in}{0.000000in}}%
\pgfpathclose%
\pgfusepath{stroke,fill}%
\end{pgfscope}%
\begin{pgfscope}%
\pgfpathrectangle{\pgfqpoint{0.647939in}{0.492442in}}{\pgfqpoint{4.273799in}{2.331163in}}%
\pgfusepath{clip}%
\pgfsetroundcap%
\pgfsetroundjoin%
\pgfsetlinewidth{0.301125pt}%
\definecolor{currentstroke}{rgb}{0.500000,0.500000,0.500000}%
\pgfsetstrokecolor{currentstroke}%
\pgfsetstrokeopacity{0.300000}%
\pgfsetdash{}{0pt}%
\pgfpathmoveto{\pgfqpoint{2.373180in}{0.806344in}}%
\pgfusepath{stroke}%
\end{pgfscope}%
\begin{pgfscope}%
\pgfpathrectangle{\pgfqpoint{0.647939in}{0.492442in}}{\pgfqpoint{4.273799in}{2.331163in}}%
\pgfusepath{clip}%
\pgfsetroundcap%
\pgfsetroundjoin%
\definecolor{currentfill}{rgb}{0.500000,0.500000,0.500000}%
\pgfsetfillcolor{currentfill}%
\pgfsetfillopacity{0.300000}%
\pgfsetlinewidth{0.301125pt}%
\definecolor{currentstroke}{rgb}{0.500000,0.500000,0.500000}%
\pgfsetstrokecolor{currentstroke}%
\pgfsetstrokeopacity{0.300000}%
\pgfsetdash{}{0pt}%
\pgfpathmoveto{\pgfqpoint{0.000000in}{0.000000in}}%
\pgfpathlineto{\pgfqpoint{0.000000in}{0.000000in}}%
\pgfpathclose%
\pgfusepath{stroke,fill}%
\end{pgfscope}%
\begin{pgfscope}%
\pgfpathrectangle{\pgfqpoint{0.647939in}{0.492442in}}{\pgfqpoint{4.273799in}{2.331163in}}%
\pgfusepath{clip}%
\pgfsetroundcap%
\pgfsetroundjoin%
\pgfsetlinewidth{0.301125pt}%
\definecolor{currentstroke}{rgb}{0.500000,0.500000,0.500000}%
\pgfsetstrokecolor{currentstroke}%
\pgfsetstrokeopacity{0.300000}%
\pgfsetdash{}{0pt}%
\pgfpathmoveto{\pgfqpoint{2.116736in}{1.425113in}}%
\pgfusepath{stroke}%
\end{pgfscope}%
\begin{pgfscope}%
\pgfpathrectangle{\pgfqpoint{0.647939in}{0.492442in}}{\pgfqpoint{4.273799in}{2.331163in}}%
\pgfusepath{clip}%
\pgfsetroundcap%
\pgfsetroundjoin%
\definecolor{currentfill}{rgb}{0.500000,0.500000,0.500000}%
\pgfsetfillcolor{currentfill}%
\pgfsetfillopacity{0.300000}%
\pgfsetlinewidth{0.301125pt}%
\definecolor{currentstroke}{rgb}{0.500000,0.500000,0.500000}%
\pgfsetstrokecolor{currentstroke}%
\pgfsetstrokeopacity{0.300000}%
\pgfsetdash{}{0pt}%
\pgfpathmoveto{\pgfqpoint{0.000000in}{0.000000in}}%
\pgfpathlineto{\pgfqpoint{0.000000in}{0.000000in}}%
\pgfpathclose%
\pgfusepath{stroke,fill}%
\end{pgfscope}%
\begin{pgfscope}%
\pgfpathrectangle{\pgfqpoint{0.647939in}{0.492442in}}{\pgfqpoint{4.273799in}{2.331163in}}%
\pgfusepath{clip}%
\pgfsetroundcap%
\pgfsetroundjoin%
\pgfsetlinewidth{0.301125pt}%
\definecolor{currentstroke}{rgb}{0.500000,0.500000,0.500000}%
\pgfsetstrokecolor{currentstroke}%
\pgfsetstrokeopacity{0.300000}%
\pgfsetdash{}{0pt}%
\pgfpathmoveto{\pgfqpoint{2.217083in}{1.570298in}}%
\pgfusepath{stroke}%
\end{pgfscope}%
\begin{pgfscope}%
\pgfpathrectangle{\pgfqpoint{0.647939in}{0.492442in}}{\pgfqpoint{4.273799in}{2.331163in}}%
\pgfusepath{clip}%
\pgfsetroundcap%
\pgfsetroundjoin%
\definecolor{currentfill}{rgb}{0.500000,0.500000,0.500000}%
\pgfsetfillcolor{currentfill}%
\pgfsetfillopacity{0.300000}%
\pgfsetlinewidth{0.301125pt}%
\definecolor{currentstroke}{rgb}{0.500000,0.500000,0.500000}%
\pgfsetstrokecolor{currentstroke}%
\pgfsetstrokeopacity{0.300000}%
\pgfsetdash{}{0pt}%
\pgfpathmoveto{\pgfqpoint{0.000000in}{0.000000in}}%
\pgfpathlineto{\pgfqpoint{0.000000in}{0.000000in}}%
\pgfpathclose%
\pgfusepath{stroke,fill}%
\end{pgfscope}%
\begin{pgfscope}%
\pgfpathrectangle{\pgfqpoint{0.647939in}{0.492442in}}{\pgfqpoint{4.273799in}{2.331163in}}%
\pgfusepath{clip}%
\pgfsetroundcap%
\pgfsetroundjoin%
\pgfsetlinewidth{0.301125pt}%
\definecolor{currentstroke}{rgb}{0.500000,0.500000,0.500000}%
\pgfsetstrokecolor{currentstroke}%
\pgfsetstrokeopacity{0.300000}%
\pgfsetdash{}{0pt}%
\pgfpathmoveto{\pgfqpoint{2.692315in}{0.846104in}}%
\pgfusepath{stroke}%
\end{pgfscope}%
\begin{pgfscope}%
\pgfpathrectangle{\pgfqpoint{0.647939in}{0.492442in}}{\pgfqpoint{4.273799in}{2.331163in}}%
\pgfusepath{clip}%
\pgfsetroundcap%
\pgfsetroundjoin%
\definecolor{currentfill}{rgb}{0.500000,0.500000,0.500000}%
\pgfsetfillcolor{currentfill}%
\pgfsetfillopacity{0.300000}%
\pgfsetlinewidth{0.301125pt}%
\definecolor{currentstroke}{rgb}{0.500000,0.500000,0.500000}%
\pgfsetstrokecolor{currentstroke}%
\pgfsetstrokeopacity{0.300000}%
\pgfsetdash{}{0pt}%
\pgfpathmoveto{\pgfqpoint{0.000000in}{0.000000in}}%
\pgfpathlineto{\pgfqpoint{0.000000in}{0.000000in}}%
\pgfpathclose%
\pgfusepath{stroke,fill}%
\end{pgfscope}%
\begin{pgfscope}%
\pgfpathrectangle{\pgfqpoint{0.647939in}{0.492442in}}{\pgfqpoint{4.273799in}{2.331163in}}%
\pgfusepath{clip}%
\pgfsetroundcap%
\pgfsetroundjoin%
\pgfsetlinewidth{0.301125pt}%
\definecolor{currentstroke}{rgb}{0.500000,0.500000,0.500000}%
\pgfsetstrokecolor{currentstroke}%
\pgfsetstrokeopacity{0.300000}%
\pgfsetdash{}{0pt}%
\pgfpathmoveto{\pgfqpoint{2.971593in}{0.605492in}}%
\pgfusepath{stroke}%
\end{pgfscope}%
\begin{pgfscope}%
\pgfpathrectangle{\pgfqpoint{0.647939in}{0.492442in}}{\pgfqpoint{4.273799in}{2.331163in}}%
\pgfusepath{clip}%
\pgfsetroundcap%
\pgfsetroundjoin%
\definecolor{currentfill}{rgb}{0.500000,0.500000,0.500000}%
\pgfsetfillcolor{currentfill}%
\pgfsetfillopacity{0.300000}%
\pgfsetlinewidth{0.301125pt}%
\definecolor{currentstroke}{rgb}{0.500000,0.500000,0.500000}%
\pgfsetstrokecolor{currentstroke}%
\pgfsetstrokeopacity{0.300000}%
\pgfsetdash{}{0pt}%
\pgfpathmoveto{\pgfqpoint{0.000000in}{0.000000in}}%
\pgfpathlineto{\pgfqpoint{0.000000in}{0.000000in}}%
\pgfpathclose%
\pgfusepath{stroke,fill}%
\end{pgfscope}%
\begin{pgfscope}%
\pgfpathrectangle{\pgfqpoint{0.647939in}{0.492442in}}{\pgfqpoint{4.273799in}{2.331163in}}%
\pgfusepath{clip}%
\pgfsetroundcap%
\pgfsetroundjoin%
\pgfsetlinewidth{0.301125pt}%
\definecolor{currentstroke}{rgb}{0.500000,0.500000,0.500000}%
\pgfsetstrokecolor{currentstroke}%
\pgfsetstrokeopacity{0.300000}%
\pgfsetdash{}{0pt}%
\pgfpathmoveto{\pgfqpoint{2.486398in}{1.544002in}}%
\pgfusepath{stroke}%
\end{pgfscope}%
\begin{pgfscope}%
\pgfpathrectangle{\pgfqpoint{0.647939in}{0.492442in}}{\pgfqpoint{4.273799in}{2.331163in}}%
\pgfusepath{clip}%
\pgfsetroundcap%
\pgfsetroundjoin%
\definecolor{currentfill}{rgb}{0.500000,0.500000,0.500000}%
\pgfsetfillcolor{currentfill}%
\pgfsetfillopacity{0.300000}%
\pgfsetlinewidth{0.301125pt}%
\definecolor{currentstroke}{rgb}{0.500000,0.500000,0.500000}%
\pgfsetstrokecolor{currentstroke}%
\pgfsetstrokeopacity{0.300000}%
\pgfsetdash{}{0pt}%
\pgfpathmoveto{\pgfqpoint{0.000000in}{0.000000in}}%
\pgfpathlineto{\pgfqpoint{0.000000in}{0.000000in}}%
\pgfpathclose%
\pgfusepath{stroke,fill}%
\end{pgfscope}%
\begin{pgfscope}%
\pgfpathrectangle{\pgfqpoint{0.647939in}{0.492442in}}{\pgfqpoint{4.273799in}{2.331163in}}%
\pgfusepath{clip}%
\pgfsetroundcap%
\pgfsetroundjoin%
\pgfsetlinewidth{0.301125pt}%
\definecolor{currentstroke}{rgb}{0.500000,0.500000,0.500000}%
\pgfsetstrokecolor{currentstroke}%
\pgfsetstrokeopacity{0.300000}%
\pgfsetdash{}{0pt}%
\pgfpathmoveto{\pgfqpoint{2.933711in}{1.009436in}}%
\pgfusepath{stroke}%
\end{pgfscope}%
\begin{pgfscope}%
\pgfpathrectangle{\pgfqpoint{0.647939in}{0.492442in}}{\pgfqpoint{4.273799in}{2.331163in}}%
\pgfusepath{clip}%
\pgfsetroundcap%
\pgfsetroundjoin%
\definecolor{currentfill}{rgb}{0.500000,0.500000,0.500000}%
\pgfsetfillcolor{currentfill}%
\pgfsetfillopacity{0.300000}%
\pgfsetlinewidth{0.301125pt}%
\definecolor{currentstroke}{rgb}{0.500000,0.500000,0.500000}%
\pgfsetstrokecolor{currentstroke}%
\pgfsetstrokeopacity{0.300000}%
\pgfsetdash{}{0pt}%
\pgfpathmoveto{\pgfqpoint{0.000000in}{0.000000in}}%
\pgfpathlineto{\pgfqpoint{0.000000in}{0.000000in}}%
\pgfpathclose%
\pgfusepath{stroke,fill}%
\end{pgfscope}%
\begin{pgfscope}%
\pgfpathrectangle{\pgfqpoint{0.647939in}{0.492442in}}{\pgfqpoint{4.273799in}{2.331163in}}%
\pgfusepath{clip}%
\pgfsetroundcap%
\pgfsetroundjoin%
\pgfsetlinewidth{0.301125pt}%
\definecolor{currentstroke}{rgb}{0.500000,0.500000,0.500000}%
\pgfsetstrokecolor{currentstroke}%
\pgfsetstrokeopacity{0.300000}%
\pgfsetdash{}{0pt}%
\pgfpathmoveto{\pgfqpoint{3.216939in}{0.855958in}}%
\pgfusepath{stroke}%
\end{pgfscope}%
\begin{pgfscope}%
\pgfpathrectangle{\pgfqpoint{0.647939in}{0.492442in}}{\pgfqpoint{4.273799in}{2.331163in}}%
\pgfusepath{clip}%
\pgfsetroundcap%
\pgfsetroundjoin%
\definecolor{currentfill}{rgb}{0.500000,0.500000,0.500000}%
\pgfsetfillcolor{currentfill}%
\pgfsetfillopacity{0.300000}%
\pgfsetlinewidth{0.301125pt}%
\definecolor{currentstroke}{rgb}{0.500000,0.500000,0.500000}%
\pgfsetstrokecolor{currentstroke}%
\pgfsetstrokeopacity{0.300000}%
\pgfsetdash{}{0pt}%
\pgfpathmoveto{\pgfqpoint{0.000000in}{0.000000in}}%
\pgfpathlineto{\pgfqpoint{0.000000in}{0.000000in}}%
\pgfpathclose%
\pgfusepath{stroke,fill}%
\end{pgfscope}%
\begin{pgfscope}%
\pgfpathrectangle{\pgfqpoint{0.647939in}{0.492442in}}{\pgfqpoint{4.273799in}{2.331163in}}%
\pgfusepath{clip}%
\pgfsetroundcap%
\pgfsetroundjoin%
\pgfsetlinewidth{0.301125pt}%
\definecolor{currentstroke}{rgb}{0.500000,0.500000,0.500000}%
\pgfsetstrokecolor{currentstroke}%
\pgfsetstrokeopacity{0.300000}%
\pgfsetdash{}{0pt}%
\pgfpathmoveto{\pgfqpoint{3.439818in}{0.802650in}}%
\pgfusepath{stroke}%
\end{pgfscope}%
\begin{pgfscope}%
\pgfpathrectangle{\pgfqpoint{0.647939in}{0.492442in}}{\pgfqpoint{4.273799in}{2.331163in}}%
\pgfusepath{clip}%
\pgfsetroundcap%
\pgfsetroundjoin%
\definecolor{currentfill}{rgb}{0.500000,0.500000,0.500000}%
\pgfsetfillcolor{currentfill}%
\pgfsetfillopacity{0.300000}%
\pgfsetlinewidth{0.301125pt}%
\definecolor{currentstroke}{rgb}{0.500000,0.500000,0.500000}%
\pgfsetstrokecolor{currentstroke}%
\pgfsetstrokeopacity{0.300000}%
\pgfsetdash{}{0pt}%
\pgfpathmoveto{\pgfqpoint{0.000000in}{0.000000in}}%
\pgfpathlineto{\pgfqpoint{0.000000in}{0.000000in}}%
\pgfpathclose%
\pgfusepath{stroke,fill}%
\end{pgfscope}%
\begin{pgfscope}%
\pgfpathrectangle{\pgfqpoint{0.647939in}{0.492442in}}{\pgfqpoint{4.273799in}{2.331163in}}%
\pgfusepath{clip}%
\pgfsetroundcap%
\pgfsetroundjoin%
\pgfsetlinewidth{0.301125pt}%
\definecolor{currentstroke}{rgb}{0.500000,0.500000,0.500000}%
\pgfsetstrokecolor{currentstroke}%
\pgfsetstrokeopacity{0.300000}%
\pgfsetdash{}{0pt}%
\pgfpathmoveto{\pgfqpoint{3.021070in}{1.327527in}}%
\pgfusepath{stroke}%
\end{pgfscope}%
\begin{pgfscope}%
\pgfpathrectangle{\pgfqpoint{0.647939in}{0.492442in}}{\pgfqpoint{4.273799in}{2.331163in}}%
\pgfusepath{clip}%
\pgfsetroundcap%
\pgfsetroundjoin%
\definecolor{currentfill}{rgb}{0.500000,0.500000,0.500000}%
\pgfsetfillcolor{currentfill}%
\pgfsetfillopacity{0.300000}%
\pgfsetlinewidth{0.301125pt}%
\definecolor{currentstroke}{rgb}{0.500000,0.500000,0.500000}%
\pgfsetstrokecolor{currentstroke}%
\pgfsetstrokeopacity{0.300000}%
\pgfsetdash{}{0pt}%
\pgfpathmoveto{\pgfqpoint{0.000000in}{0.000000in}}%
\pgfpathlineto{\pgfqpoint{0.000000in}{0.000000in}}%
\pgfpathclose%
\pgfusepath{stroke,fill}%
\end{pgfscope}%
\begin{pgfscope}%
\pgfpathrectangle{\pgfqpoint{0.647939in}{0.492442in}}{\pgfqpoint{4.273799in}{2.331163in}}%
\pgfusepath{clip}%
\pgfsetroundcap%
\pgfsetroundjoin%
\pgfsetlinewidth{0.301125pt}%
\definecolor{currentstroke}{rgb}{0.500000,0.500000,0.500000}%
\pgfsetstrokecolor{currentstroke}%
\pgfsetstrokeopacity{0.300000}%
\pgfsetdash{}{0pt}%
\pgfpathmoveto{\pgfqpoint{3.497138in}{1.052922in}}%
\pgfusepath{stroke}%
\end{pgfscope}%
\begin{pgfscope}%
\pgfpathrectangle{\pgfqpoint{0.647939in}{0.492442in}}{\pgfqpoint{4.273799in}{2.331163in}}%
\pgfusepath{clip}%
\pgfsetroundcap%
\pgfsetroundjoin%
\definecolor{currentfill}{rgb}{0.500000,0.500000,0.500000}%
\pgfsetfillcolor{currentfill}%
\pgfsetfillopacity{0.300000}%
\pgfsetlinewidth{0.301125pt}%
\definecolor{currentstroke}{rgb}{0.500000,0.500000,0.500000}%
\pgfsetstrokecolor{currentstroke}%
\pgfsetstrokeopacity{0.300000}%
\pgfsetdash{}{0pt}%
\pgfpathmoveto{\pgfqpoint{0.000000in}{0.000000in}}%
\pgfpathlineto{\pgfqpoint{0.000000in}{0.000000in}}%
\pgfpathclose%
\pgfusepath{stroke,fill}%
\end{pgfscope}%
\begin{pgfscope}%
\pgfpathrectangle{\pgfqpoint{0.647939in}{0.492442in}}{\pgfqpoint{4.273799in}{2.331163in}}%
\pgfusepath{clip}%
\pgfsetroundcap%
\pgfsetroundjoin%
\pgfsetlinewidth{0.301125pt}%
\definecolor{currentstroke}{rgb}{0.500000,0.500000,0.500000}%
\pgfsetstrokecolor{currentstroke}%
\pgfsetstrokeopacity{0.300000}%
\pgfsetdash{}{0pt}%
\pgfpathmoveto{\pgfqpoint{3.442857in}{1.267002in}}%
\pgfusepath{stroke}%
\end{pgfscope}%
\begin{pgfscope}%
\pgfpathrectangle{\pgfqpoint{0.647939in}{0.492442in}}{\pgfqpoint{4.273799in}{2.331163in}}%
\pgfusepath{clip}%
\pgfsetroundcap%
\pgfsetroundjoin%
\definecolor{currentfill}{rgb}{0.500000,0.500000,0.500000}%
\pgfsetfillcolor{currentfill}%
\pgfsetfillopacity{0.300000}%
\pgfsetlinewidth{0.301125pt}%
\definecolor{currentstroke}{rgb}{0.500000,0.500000,0.500000}%
\pgfsetstrokecolor{currentstroke}%
\pgfsetstrokeopacity{0.300000}%
\pgfsetdash{}{0pt}%
\pgfpathmoveto{\pgfqpoint{0.000000in}{0.000000in}}%
\pgfpathlineto{\pgfqpoint{0.000000in}{0.000000in}}%
\pgfpathclose%
\pgfusepath{stroke,fill}%
\end{pgfscope}%
\begin{pgfscope}%
\pgfpathrectangle{\pgfqpoint{0.647939in}{0.492442in}}{\pgfqpoint{4.273799in}{2.331163in}}%
\pgfusepath{clip}%
\pgfsetroundcap%
\pgfsetroundjoin%
\pgfsetlinewidth{0.301125pt}%
\definecolor{currentstroke}{rgb}{0.500000,0.500000,0.500000}%
\pgfsetstrokecolor{currentstroke}%
\pgfsetstrokeopacity{0.300000}%
\pgfsetdash{}{0pt}%
\pgfpathmoveto{\pgfqpoint{3.869106in}{1.136954in}}%
\pgfusepath{stroke}%
\end{pgfscope}%
\begin{pgfscope}%
\pgfpathrectangle{\pgfqpoint{0.647939in}{0.492442in}}{\pgfqpoint{4.273799in}{2.331163in}}%
\pgfusepath{clip}%
\pgfsetroundcap%
\pgfsetroundjoin%
\definecolor{currentfill}{rgb}{0.500000,0.500000,0.500000}%
\pgfsetfillcolor{currentfill}%
\pgfsetfillopacity{0.300000}%
\pgfsetlinewidth{0.301125pt}%
\definecolor{currentstroke}{rgb}{0.500000,0.500000,0.500000}%
\pgfsetstrokecolor{currentstroke}%
\pgfsetstrokeopacity{0.300000}%
\pgfsetdash{}{0pt}%
\pgfpathmoveto{\pgfqpoint{0.000000in}{0.000000in}}%
\pgfpathlineto{\pgfqpoint{0.000000in}{0.000000in}}%
\pgfpathclose%
\pgfusepath{stroke,fill}%
\end{pgfscope}%
\begin{pgfscope}%
\pgfpathrectangle{\pgfqpoint{0.647939in}{0.492442in}}{\pgfqpoint{4.273799in}{2.331163in}}%
\pgfusepath{clip}%
\pgfsetroundcap%
\pgfsetroundjoin%
\pgfsetlinewidth{0.301125pt}%
\definecolor{currentstroke}{rgb}{0.500000,0.500000,0.500000}%
\pgfsetstrokecolor{currentstroke}%
\pgfsetstrokeopacity{0.300000}%
\pgfsetdash{}{0pt}%
\pgfpathmoveto{\pgfqpoint{4.062393in}{1.166983in}}%
\pgfusepath{stroke}%
\end{pgfscope}%
\begin{pgfscope}%
\pgfpathrectangle{\pgfqpoint{0.647939in}{0.492442in}}{\pgfqpoint{4.273799in}{2.331163in}}%
\pgfusepath{clip}%
\pgfsetroundcap%
\pgfsetroundjoin%
\definecolor{currentfill}{rgb}{0.500000,0.500000,0.500000}%
\pgfsetfillcolor{currentfill}%
\pgfsetfillopacity{0.300000}%
\pgfsetlinewidth{0.301125pt}%
\definecolor{currentstroke}{rgb}{0.500000,0.500000,0.500000}%
\pgfsetstrokecolor{currentstroke}%
\pgfsetstrokeopacity{0.300000}%
\pgfsetdash{}{0pt}%
\pgfpathmoveto{\pgfqpoint{0.000000in}{0.000000in}}%
\pgfpathlineto{\pgfqpoint{0.000000in}{0.000000in}}%
\pgfpathclose%
\pgfusepath{stroke,fill}%
\end{pgfscope}%
\begin{pgfscope}%
\pgfpathrectangle{\pgfqpoint{0.647939in}{0.492442in}}{\pgfqpoint{4.273799in}{2.331163in}}%
\pgfusepath{clip}%
\pgfsetroundcap%
\pgfsetroundjoin%
\pgfsetlinewidth{0.301125pt}%
\definecolor{currentstroke}{rgb}{0.500000,0.500000,0.500000}%
\pgfsetstrokecolor{currentstroke}%
\pgfsetstrokeopacity{0.300000}%
\pgfsetdash{}{0pt}%
\pgfpathmoveto{\pgfqpoint{3.992533in}{1.400608in}}%
\pgfusepath{stroke}%
\end{pgfscope}%
\begin{pgfscope}%
\pgfpathrectangle{\pgfqpoint{0.647939in}{0.492442in}}{\pgfqpoint{4.273799in}{2.331163in}}%
\pgfusepath{clip}%
\pgfsetroundcap%
\pgfsetroundjoin%
\definecolor{currentfill}{rgb}{0.500000,0.500000,0.500000}%
\pgfsetfillcolor{currentfill}%
\pgfsetfillopacity{0.300000}%
\pgfsetlinewidth{0.301125pt}%
\definecolor{currentstroke}{rgb}{0.500000,0.500000,0.500000}%
\pgfsetstrokecolor{currentstroke}%
\pgfsetstrokeopacity{0.300000}%
\pgfsetdash{}{0pt}%
\pgfpathmoveto{\pgfqpoint{0.000000in}{0.000000in}}%
\pgfpathlineto{\pgfqpoint{0.000000in}{0.000000in}}%
\pgfpathclose%
\pgfusepath{stroke,fill}%
\end{pgfscope}%
\begin{pgfscope}%
\pgfpathrectangle{\pgfqpoint{0.647939in}{0.492442in}}{\pgfqpoint{4.273799in}{2.331163in}}%
\pgfusepath{clip}%
\pgfsetroundcap%
\pgfsetroundjoin%
\pgfsetlinewidth{0.301125pt}%
\definecolor{currentstroke}{rgb}{0.500000,0.500000,0.500000}%
\pgfsetstrokecolor{currentstroke}%
\pgfsetstrokeopacity{0.300000}%
\pgfsetdash{}{0pt}%
\pgfpathmoveto{\pgfqpoint{4.273363in}{1.452167in}}%
\pgfusepath{stroke}%
\end{pgfscope}%
\begin{pgfscope}%
\pgfpathrectangle{\pgfqpoint{0.647939in}{0.492442in}}{\pgfqpoint{4.273799in}{2.331163in}}%
\pgfusepath{clip}%
\pgfsetroundcap%
\pgfsetroundjoin%
\definecolor{currentfill}{rgb}{0.500000,0.500000,0.500000}%
\pgfsetfillcolor{currentfill}%
\pgfsetfillopacity{0.300000}%
\pgfsetlinewidth{0.301125pt}%
\definecolor{currentstroke}{rgb}{0.500000,0.500000,0.500000}%
\pgfsetstrokecolor{currentstroke}%
\pgfsetstrokeopacity{0.300000}%
\pgfsetdash{}{0pt}%
\pgfpathmoveto{\pgfqpoint{0.000000in}{0.000000in}}%
\pgfpathlineto{\pgfqpoint{0.000000in}{0.000000in}}%
\pgfpathclose%
\pgfusepath{stroke,fill}%
\end{pgfscope}%
\begin{pgfscope}%
\pgfpathrectangle{\pgfqpoint{0.647939in}{0.492442in}}{\pgfqpoint{4.273799in}{2.331163in}}%
\pgfusepath{clip}%
\pgfsetroundcap%
\pgfsetroundjoin%
\pgfsetlinewidth{0.301125pt}%
\definecolor{currentstroke}{rgb}{0.500000,0.500000,0.500000}%
\pgfsetstrokecolor{currentstroke}%
\pgfsetstrokeopacity{0.300000}%
\pgfsetdash{}{0pt}%
\pgfpathmoveto{\pgfqpoint{4.398767in}{1.702436in}}%
\pgfusepath{stroke}%
\end{pgfscope}%
\begin{pgfscope}%
\pgfpathrectangle{\pgfqpoint{0.647939in}{0.492442in}}{\pgfqpoint{4.273799in}{2.331163in}}%
\pgfusepath{clip}%
\pgfsetroundcap%
\pgfsetroundjoin%
\definecolor{currentfill}{rgb}{0.500000,0.500000,0.500000}%
\pgfsetfillcolor{currentfill}%
\pgfsetfillopacity{0.300000}%
\pgfsetlinewidth{0.301125pt}%
\definecolor{currentstroke}{rgb}{0.500000,0.500000,0.500000}%
\pgfsetstrokecolor{currentstroke}%
\pgfsetstrokeopacity{0.300000}%
\pgfsetdash{}{0pt}%
\pgfpathmoveto{\pgfqpoint{0.000000in}{0.000000in}}%
\pgfpathlineto{\pgfqpoint{0.000000in}{0.000000in}}%
\pgfpathclose%
\pgfusepath{stroke,fill}%
\end{pgfscope}%
\begin{pgfscope}%
\pgfpathrectangle{\pgfqpoint{0.647939in}{0.492442in}}{\pgfqpoint{4.273799in}{2.331163in}}%
\pgfusepath{clip}%
\pgfsetroundcap%
\pgfsetroundjoin%
\pgfsetlinewidth{0.301125pt}%
\definecolor{currentstroke}{rgb}{0.500000,0.500000,0.500000}%
\pgfsetstrokecolor{currentstroke}%
\pgfsetstrokeopacity{0.300000}%
\pgfsetdash{}{0pt}%
\pgfpathmoveto{\pgfqpoint{4.654301in}{1.573195in}}%
\pgfusepath{stroke}%
\end{pgfscope}%
\begin{pgfscope}%
\pgfpathrectangle{\pgfqpoint{0.647939in}{0.492442in}}{\pgfqpoint{4.273799in}{2.331163in}}%
\pgfusepath{clip}%
\pgfsetroundcap%
\pgfsetroundjoin%
\definecolor{currentfill}{rgb}{0.500000,0.500000,0.500000}%
\pgfsetfillcolor{currentfill}%
\pgfsetfillopacity{0.300000}%
\pgfsetlinewidth{0.301125pt}%
\definecolor{currentstroke}{rgb}{0.500000,0.500000,0.500000}%
\pgfsetstrokecolor{currentstroke}%
\pgfsetstrokeopacity{0.300000}%
\pgfsetdash{}{0pt}%
\pgfpathmoveto{\pgfqpoint{0.000000in}{0.000000in}}%
\pgfpathlineto{\pgfqpoint{0.000000in}{0.000000in}}%
\pgfpathclose%
\pgfusepath{stroke,fill}%
\end{pgfscope}%
\begin{pgfscope}%
\pgfpathrectangle{\pgfqpoint{0.647939in}{0.492442in}}{\pgfqpoint{4.273799in}{2.331163in}}%
\pgfusepath{clip}%
\pgfsetroundcap%
\pgfsetroundjoin%
\pgfsetlinewidth{0.301125pt}%
\definecolor{currentstroke}{rgb}{0.500000,0.500000,0.500000}%
\pgfsetstrokecolor{currentstroke}%
\pgfsetstrokeopacity{0.300000}%
\pgfsetdash{}{0pt}%
\pgfpathmoveto{\pgfqpoint{4.739475in}{1.844673in}}%
\pgfusepath{stroke}%
\end{pgfscope}%
\begin{pgfscope}%
\pgfpathrectangle{\pgfqpoint{0.647939in}{0.492442in}}{\pgfqpoint{4.273799in}{2.331163in}}%
\pgfusepath{clip}%
\pgfsetroundcap%
\pgfsetroundjoin%
\definecolor{currentfill}{rgb}{0.500000,0.500000,0.500000}%
\pgfsetfillcolor{currentfill}%
\pgfsetfillopacity{0.300000}%
\pgfsetlinewidth{0.301125pt}%
\definecolor{currentstroke}{rgb}{0.500000,0.500000,0.500000}%
\pgfsetstrokecolor{currentstroke}%
\pgfsetstrokeopacity{0.300000}%
\pgfsetdash{}{0pt}%
\pgfpathmoveto{\pgfqpoint{0.000000in}{0.000000in}}%
\pgfpathlineto{\pgfqpoint{0.000000in}{0.000000in}}%
\pgfpathclose%
\pgfusepath{stroke,fill}%
\end{pgfscope}%
\begin{pgfscope}%
\pgfpathrectangle{\pgfqpoint{0.647939in}{0.492442in}}{\pgfqpoint{4.273799in}{2.331163in}}%
\pgfusepath{clip}%
\pgfsetroundcap%
\pgfsetroundjoin%
\pgfsetlinewidth{0.301125pt}%
\definecolor{currentstroke}{rgb}{0.500000,0.500000,0.500000}%
\pgfsetstrokecolor{currentstroke}%
\pgfsetstrokeopacity{0.300000}%
\pgfsetdash{}{0pt}%
\pgfpathmoveto{\pgfqpoint{4.855281in}{1.909687in}}%
\pgfusepath{stroke}%
\end{pgfscope}%
\begin{pgfscope}%
\pgfpathrectangle{\pgfqpoint{0.647939in}{0.492442in}}{\pgfqpoint{4.273799in}{2.331163in}}%
\pgfusepath{clip}%
\pgfsetroundcap%
\pgfsetroundjoin%
\definecolor{currentfill}{rgb}{0.500000,0.500000,0.500000}%
\pgfsetfillcolor{currentfill}%
\pgfsetfillopacity{0.300000}%
\pgfsetlinewidth{0.301125pt}%
\definecolor{currentstroke}{rgb}{0.500000,0.500000,0.500000}%
\pgfsetstrokecolor{currentstroke}%
\pgfsetstrokeopacity{0.300000}%
\pgfsetdash{}{0pt}%
\pgfpathmoveto{\pgfqpoint{0.000000in}{0.000000in}}%
\pgfpathlineto{\pgfqpoint{0.000000in}{0.000000in}}%
\pgfpathclose%
\pgfusepath{stroke,fill}%
\end{pgfscope}%
\begin{pgfscope}%
\pgfpathrectangle{\pgfqpoint{0.647939in}{0.492442in}}{\pgfqpoint{4.273799in}{2.331163in}}%
\pgfusepath{clip}%
\pgfsetroundcap%
\pgfsetroundjoin%
\pgfsetlinewidth{0.301125pt}%
\definecolor{currentstroke}{rgb}{0.500000,0.500000,0.500000}%
\pgfsetstrokecolor{currentstroke}%
\pgfsetstrokeopacity{0.300000}%
\pgfsetdash{}{0pt}%
\pgfpathmoveto{\pgfqpoint{4.908451in}{1.918073in}}%
\pgfusepath{stroke}%
\end{pgfscope}%
\begin{pgfscope}%
\pgfpathrectangle{\pgfqpoint{0.647939in}{0.492442in}}{\pgfqpoint{4.273799in}{2.331163in}}%
\pgfusepath{clip}%
\pgfsetroundcap%
\pgfsetroundjoin%
\definecolor{currentfill}{rgb}{0.500000,0.500000,0.500000}%
\pgfsetfillcolor{currentfill}%
\pgfsetfillopacity{0.300000}%
\pgfsetlinewidth{0.301125pt}%
\definecolor{currentstroke}{rgb}{0.500000,0.500000,0.500000}%
\pgfsetstrokecolor{currentstroke}%
\pgfsetstrokeopacity{0.300000}%
\pgfsetdash{}{0pt}%
\pgfpathmoveto{\pgfqpoint{0.000000in}{0.000000in}}%
\pgfpathlineto{\pgfqpoint{0.000000in}{0.000000in}}%
\pgfpathclose%
\pgfusepath{stroke,fill}%
\end{pgfscope}%
\begin{pgfscope}%
\pgfpathrectangle{\pgfqpoint{0.647939in}{0.492442in}}{\pgfqpoint{4.273799in}{2.331163in}}%
\pgfusepath{clip}%
\pgfsetroundcap%
\pgfsetroundjoin%
\pgfsetlinewidth{0.301125pt}%
\definecolor{currentstroke}{rgb}{0.500000,0.500000,0.500000}%
\pgfsetstrokecolor{currentstroke}%
\pgfsetstrokeopacity{0.300000}%
\pgfsetdash{}{0pt}%
\pgfpathmoveto{\pgfqpoint{4.530555in}{2.655900in}}%
\pgfusepath{stroke}%
\end{pgfscope}%
\begin{pgfscope}%
\pgfpathrectangle{\pgfqpoint{0.647939in}{0.492442in}}{\pgfqpoint{4.273799in}{2.331163in}}%
\pgfusepath{clip}%
\pgfsetroundcap%
\pgfsetroundjoin%
\definecolor{currentfill}{rgb}{0.500000,0.500000,0.500000}%
\pgfsetfillcolor{currentfill}%
\pgfsetfillopacity{0.300000}%
\pgfsetlinewidth{0.301125pt}%
\definecolor{currentstroke}{rgb}{0.500000,0.500000,0.500000}%
\pgfsetstrokecolor{currentstroke}%
\pgfsetstrokeopacity{0.300000}%
\pgfsetdash{}{0pt}%
\pgfpathmoveto{\pgfqpoint{0.000000in}{0.000000in}}%
\pgfpathlineto{\pgfqpoint{0.000000in}{0.000000in}}%
\pgfpathclose%
\pgfusepath{stroke,fill}%
\end{pgfscope}%
\begin{pgfscope}%
\pgfpathrectangle{\pgfqpoint{0.647939in}{0.492442in}}{\pgfqpoint{4.273799in}{2.331163in}}%
\pgfusepath{clip}%
\pgfsetroundcap%
\pgfsetroundjoin%
\pgfsetlinewidth{0.301125pt}%
\definecolor{currentstroke}{rgb}{0.500000,0.500000,0.500000}%
\pgfsetstrokecolor{currentstroke}%
\pgfsetstrokeopacity{0.300000}%
\pgfsetdash{}{0pt}%
\pgfpathmoveto{\pgfqpoint{4.414384in}{2.592208in}}%
\pgfusepath{stroke}%
\end{pgfscope}%
\begin{pgfscope}%
\pgfpathrectangle{\pgfqpoint{0.647939in}{0.492442in}}{\pgfqpoint{4.273799in}{2.331163in}}%
\pgfusepath{clip}%
\pgfsetroundcap%
\pgfsetroundjoin%
\definecolor{currentfill}{rgb}{0.500000,0.500000,0.500000}%
\pgfsetfillcolor{currentfill}%
\pgfsetfillopacity{0.300000}%
\pgfsetlinewidth{0.301125pt}%
\definecolor{currentstroke}{rgb}{0.500000,0.500000,0.500000}%
\pgfsetstrokecolor{currentstroke}%
\pgfsetstrokeopacity{0.300000}%
\pgfsetdash{}{0pt}%
\pgfpathmoveto{\pgfqpoint{0.000000in}{0.000000in}}%
\pgfpathlineto{\pgfqpoint{0.000000in}{0.000000in}}%
\pgfpathclose%
\pgfusepath{stroke,fill}%
\end{pgfscope}%
\begin{pgfscope}%
\pgfpathrectangle{\pgfqpoint{0.647939in}{0.492442in}}{\pgfqpoint{4.273799in}{2.331163in}}%
\pgfusepath{clip}%
\pgfsetroundcap%
\pgfsetroundjoin%
\pgfsetlinewidth{0.301125pt}%
\definecolor{currentstroke}{rgb}{0.500000,0.500000,0.500000}%
\pgfsetstrokecolor{currentstroke}%
\pgfsetstrokeopacity{0.300000}%
\pgfsetdash{}{0pt}%
\pgfpathmoveto{\pgfqpoint{4.270292in}{2.570841in}}%
\pgfusepath{stroke}%
\end{pgfscope}%
\begin{pgfscope}%
\pgfpathrectangle{\pgfqpoint{0.647939in}{0.492442in}}{\pgfqpoint{4.273799in}{2.331163in}}%
\pgfusepath{clip}%
\pgfsetroundcap%
\pgfsetroundjoin%
\definecolor{currentfill}{rgb}{0.500000,0.500000,0.500000}%
\pgfsetfillcolor{currentfill}%
\pgfsetfillopacity{0.300000}%
\pgfsetlinewidth{0.301125pt}%
\definecolor{currentstroke}{rgb}{0.500000,0.500000,0.500000}%
\pgfsetstrokecolor{currentstroke}%
\pgfsetstrokeopacity{0.300000}%
\pgfsetdash{}{0pt}%
\pgfpathmoveto{\pgfqpoint{0.000000in}{0.000000in}}%
\pgfpathlineto{\pgfqpoint{0.000000in}{0.000000in}}%
\pgfpathclose%
\pgfusepath{stroke,fill}%
\end{pgfscope}%
\begin{pgfscope}%
\pgfpathrectangle{\pgfqpoint{0.647939in}{0.492442in}}{\pgfqpoint{4.273799in}{2.331163in}}%
\pgfusepath{clip}%
\pgfsetroundcap%
\pgfsetroundjoin%
\pgfsetlinewidth{0.301125pt}%
\definecolor{currentstroke}{rgb}{0.500000,0.500000,0.500000}%
\pgfsetstrokecolor{currentstroke}%
\pgfsetstrokeopacity{0.300000}%
\pgfsetdash{}{0pt}%
\pgfpathmoveto{\pgfqpoint{4.188623in}{2.516095in}}%
\pgfusepath{stroke}%
\end{pgfscope}%
\begin{pgfscope}%
\pgfpathrectangle{\pgfqpoint{0.647939in}{0.492442in}}{\pgfqpoint{4.273799in}{2.331163in}}%
\pgfusepath{clip}%
\pgfsetroundcap%
\pgfsetroundjoin%
\definecolor{currentfill}{rgb}{0.500000,0.500000,0.500000}%
\pgfsetfillcolor{currentfill}%
\pgfsetfillopacity{0.300000}%
\pgfsetlinewidth{0.301125pt}%
\definecolor{currentstroke}{rgb}{0.500000,0.500000,0.500000}%
\pgfsetstrokecolor{currentstroke}%
\pgfsetstrokeopacity{0.300000}%
\pgfsetdash{}{0pt}%
\pgfpathmoveto{\pgfqpoint{0.000000in}{0.000000in}}%
\pgfpathlineto{\pgfqpoint{0.000000in}{0.000000in}}%
\pgfpathclose%
\pgfusepath{stroke,fill}%
\end{pgfscope}%
\begin{pgfscope}%
\pgfpathrectangle{\pgfqpoint{0.647939in}{0.492442in}}{\pgfqpoint{4.273799in}{2.331163in}}%
\pgfusepath{clip}%
\pgfsetroundcap%
\pgfsetroundjoin%
\pgfsetlinewidth{0.301125pt}%
\definecolor{currentstroke}{rgb}{0.500000,0.500000,0.500000}%
\pgfsetstrokecolor{currentstroke}%
\pgfsetstrokeopacity{0.300000}%
\pgfsetdash{}{0pt}%
\pgfpathmoveto{\pgfqpoint{4.108162in}{2.462737in}}%
\pgfusepath{stroke}%
\end{pgfscope}%
\begin{pgfscope}%
\pgfpathrectangle{\pgfqpoint{0.647939in}{0.492442in}}{\pgfqpoint{4.273799in}{2.331163in}}%
\pgfusepath{clip}%
\pgfsetroundcap%
\pgfsetroundjoin%
\definecolor{currentfill}{rgb}{0.500000,0.500000,0.500000}%
\pgfsetfillcolor{currentfill}%
\pgfsetfillopacity{0.300000}%
\pgfsetlinewidth{0.301125pt}%
\definecolor{currentstroke}{rgb}{0.500000,0.500000,0.500000}%
\pgfsetstrokecolor{currentstroke}%
\pgfsetstrokeopacity{0.300000}%
\pgfsetdash{}{0pt}%
\pgfpathmoveto{\pgfqpoint{0.000000in}{0.000000in}}%
\pgfpathlineto{\pgfqpoint{0.000000in}{0.000000in}}%
\pgfpathclose%
\pgfusepath{stroke,fill}%
\end{pgfscope}%
\begin{pgfscope}%
\pgfpathrectangle{\pgfqpoint{0.647939in}{0.492442in}}{\pgfqpoint{4.273799in}{2.331163in}}%
\pgfusepath{clip}%
\pgfsetroundcap%
\pgfsetroundjoin%
\pgfsetlinewidth{0.301125pt}%
\definecolor{currentstroke}{rgb}{0.500000,0.500000,0.500000}%
\pgfsetstrokecolor{currentstroke}%
\pgfsetstrokeopacity{0.300000}%
\pgfsetdash{}{0pt}%
\pgfpathmoveto{\pgfqpoint{3.998555in}{2.459444in}}%
\pgfusepath{stroke}%
\end{pgfscope}%
\begin{pgfscope}%
\pgfpathrectangle{\pgfqpoint{0.647939in}{0.492442in}}{\pgfqpoint{4.273799in}{2.331163in}}%
\pgfusepath{clip}%
\pgfsetroundcap%
\pgfsetroundjoin%
\definecolor{currentfill}{rgb}{0.500000,0.500000,0.500000}%
\pgfsetfillcolor{currentfill}%
\pgfsetfillopacity{0.300000}%
\pgfsetlinewidth{0.301125pt}%
\definecolor{currentstroke}{rgb}{0.500000,0.500000,0.500000}%
\pgfsetstrokecolor{currentstroke}%
\pgfsetstrokeopacity{0.300000}%
\pgfsetdash{}{0pt}%
\pgfpathmoveto{\pgfqpoint{0.000000in}{0.000000in}}%
\pgfpathlineto{\pgfqpoint{0.000000in}{0.000000in}}%
\pgfpathclose%
\pgfusepath{stroke,fill}%
\end{pgfscope}%
\begin{pgfscope}%
\pgfpathrectangle{\pgfqpoint{0.647939in}{0.492442in}}{\pgfqpoint{4.273799in}{2.331163in}}%
\pgfusepath{clip}%
\pgfsetroundcap%
\pgfsetroundjoin%
\pgfsetlinewidth{0.301125pt}%
\definecolor{currentstroke}{rgb}{0.500000,0.500000,0.500000}%
\pgfsetstrokecolor{currentstroke}%
\pgfsetstrokeopacity{0.300000}%
\pgfsetdash{}{0pt}%
\pgfpathmoveto{\pgfqpoint{4.067571in}{2.133960in}}%
\pgfusepath{stroke}%
\end{pgfscope}%
\begin{pgfscope}%
\pgfpathrectangle{\pgfqpoint{0.647939in}{0.492442in}}{\pgfqpoint{4.273799in}{2.331163in}}%
\pgfusepath{clip}%
\pgfsetroundcap%
\pgfsetroundjoin%
\definecolor{currentfill}{rgb}{0.500000,0.500000,0.500000}%
\pgfsetfillcolor{currentfill}%
\pgfsetfillopacity{0.300000}%
\pgfsetlinewidth{0.301125pt}%
\definecolor{currentstroke}{rgb}{0.500000,0.500000,0.500000}%
\pgfsetstrokecolor{currentstroke}%
\pgfsetstrokeopacity{0.300000}%
\pgfsetdash{}{0pt}%
\pgfpathmoveto{\pgfqpoint{0.000000in}{0.000000in}}%
\pgfpathlineto{\pgfqpoint{0.000000in}{0.000000in}}%
\pgfpathclose%
\pgfusepath{stroke,fill}%
\end{pgfscope}%
\begin{pgfscope}%
\pgfpathrectangle{\pgfqpoint{0.647939in}{0.492442in}}{\pgfqpoint{4.273799in}{2.331163in}}%
\pgfusepath{clip}%
\pgfsetroundcap%
\pgfsetroundjoin%
\pgfsetlinewidth{0.301125pt}%
\definecolor{currentstroke}{rgb}{0.500000,0.500000,0.500000}%
\pgfsetstrokecolor{currentstroke}%
\pgfsetstrokeopacity{0.300000}%
\pgfsetdash{}{0pt}%
\pgfpathmoveto{\pgfqpoint{3.881033in}{2.308707in}}%
\pgfusepath{stroke}%
\end{pgfscope}%
\begin{pgfscope}%
\pgfpathrectangle{\pgfqpoint{0.647939in}{0.492442in}}{\pgfqpoint{4.273799in}{2.331163in}}%
\pgfusepath{clip}%
\pgfsetroundcap%
\pgfsetroundjoin%
\definecolor{currentfill}{rgb}{0.500000,0.500000,0.500000}%
\pgfsetfillcolor{currentfill}%
\pgfsetfillopacity{0.300000}%
\pgfsetlinewidth{0.301125pt}%
\definecolor{currentstroke}{rgb}{0.500000,0.500000,0.500000}%
\pgfsetstrokecolor{currentstroke}%
\pgfsetstrokeopacity{0.300000}%
\pgfsetdash{}{0pt}%
\pgfpathmoveto{\pgfqpoint{0.000000in}{0.000000in}}%
\pgfpathlineto{\pgfqpoint{0.000000in}{0.000000in}}%
\pgfpathclose%
\pgfusepath{stroke,fill}%
\end{pgfscope}%
\begin{pgfscope}%
\pgfpathrectangle{\pgfqpoint{0.647939in}{0.492442in}}{\pgfqpoint{4.273799in}{2.331163in}}%
\pgfusepath{clip}%
\pgfsetroundcap%
\pgfsetroundjoin%
\pgfsetlinewidth{0.301125pt}%
\definecolor{currentstroke}{rgb}{0.500000,0.500000,0.500000}%
\pgfsetstrokecolor{currentstroke}%
\pgfsetstrokeopacity{0.300000}%
\pgfsetdash{}{0pt}%
\pgfpathmoveto{\pgfqpoint{3.763201in}{2.359774in}}%
\pgfusepath{stroke}%
\end{pgfscope}%
\begin{pgfscope}%
\pgfpathrectangle{\pgfqpoint{0.647939in}{0.492442in}}{\pgfqpoint{4.273799in}{2.331163in}}%
\pgfusepath{clip}%
\pgfsetroundcap%
\pgfsetroundjoin%
\definecolor{currentfill}{rgb}{0.500000,0.500000,0.500000}%
\pgfsetfillcolor{currentfill}%
\pgfsetfillopacity{0.300000}%
\pgfsetlinewidth{0.301125pt}%
\definecolor{currentstroke}{rgb}{0.500000,0.500000,0.500000}%
\pgfsetstrokecolor{currentstroke}%
\pgfsetstrokeopacity{0.300000}%
\pgfsetdash{}{0pt}%
\pgfpathmoveto{\pgfqpoint{0.000000in}{0.000000in}}%
\pgfpathlineto{\pgfqpoint{0.000000in}{0.000000in}}%
\pgfpathclose%
\pgfusepath{stroke,fill}%
\end{pgfscope}%
\begin{pgfscope}%
\pgfpathrectangle{\pgfqpoint{0.647939in}{0.492442in}}{\pgfqpoint{4.273799in}{2.331163in}}%
\pgfusepath{clip}%
\pgfsetroundcap%
\pgfsetroundjoin%
\pgfsetlinewidth{0.301125pt}%
\definecolor{currentstroke}{rgb}{0.500000,0.500000,0.500000}%
\pgfsetstrokecolor{currentstroke}%
\pgfsetstrokeopacity{0.300000}%
\pgfsetdash{}{0pt}%
\pgfpathmoveto{\pgfqpoint{3.527142in}{2.608882in}}%
\pgfusepath{stroke}%
\end{pgfscope}%
\begin{pgfscope}%
\pgfpathrectangle{\pgfqpoint{0.647939in}{0.492442in}}{\pgfqpoint{4.273799in}{2.331163in}}%
\pgfusepath{clip}%
\pgfsetroundcap%
\pgfsetroundjoin%
\definecolor{currentfill}{rgb}{0.500000,0.500000,0.500000}%
\pgfsetfillcolor{currentfill}%
\pgfsetfillopacity{0.300000}%
\pgfsetlinewidth{0.301125pt}%
\definecolor{currentstroke}{rgb}{0.500000,0.500000,0.500000}%
\pgfsetstrokecolor{currentstroke}%
\pgfsetstrokeopacity{0.300000}%
\pgfsetdash{}{0pt}%
\pgfpathmoveto{\pgfqpoint{0.000000in}{0.000000in}}%
\pgfpathlineto{\pgfqpoint{0.000000in}{0.000000in}}%
\pgfpathclose%
\pgfusepath{stroke,fill}%
\end{pgfscope}%
\begin{pgfscope}%
\pgfpathrectangle{\pgfqpoint{0.647939in}{0.492442in}}{\pgfqpoint{4.273799in}{2.331163in}}%
\pgfusepath{clip}%
\pgfsetroundcap%
\pgfsetroundjoin%
\pgfsetlinewidth{0.301125pt}%
\definecolor{currentstroke}{rgb}{0.500000,0.500000,0.500000}%
\pgfsetstrokecolor{currentstroke}%
\pgfsetstrokeopacity{0.300000}%
\pgfsetdash{}{0pt}%
\pgfpathmoveto{\pgfqpoint{3.623806in}{2.143878in}}%
\pgfusepath{stroke}%
\end{pgfscope}%
\begin{pgfscope}%
\pgfpathrectangle{\pgfqpoint{0.647939in}{0.492442in}}{\pgfqpoint{4.273799in}{2.331163in}}%
\pgfusepath{clip}%
\pgfsetroundcap%
\pgfsetroundjoin%
\definecolor{currentfill}{rgb}{0.500000,0.500000,0.500000}%
\pgfsetfillcolor{currentfill}%
\pgfsetfillopacity{0.300000}%
\pgfsetlinewidth{0.301125pt}%
\definecolor{currentstroke}{rgb}{0.500000,0.500000,0.500000}%
\pgfsetstrokecolor{currentstroke}%
\pgfsetstrokeopacity{0.300000}%
\pgfsetdash{}{0pt}%
\pgfpathmoveto{\pgfqpoint{0.000000in}{0.000000in}}%
\pgfpathlineto{\pgfqpoint{0.000000in}{0.000000in}}%
\pgfpathclose%
\pgfusepath{stroke,fill}%
\end{pgfscope}%
\begin{pgfscope}%
\pgfpathrectangle{\pgfqpoint{0.647939in}{0.492442in}}{\pgfqpoint{4.273799in}{2.331163in}}%
\pgfusepath{clip}%
\pgfsetroundcap%
\pgfsetroundjoin%
\pgfsetlinewidth{0.301125pt}%
\definecolor{currentstroke}{rgb}{0.500000,0.500000,0.500000}%
\pgfsetstrokecolor{currentstroke}%
\pgfsetstrokeopacity{0.300000}%
\pgfsetdash{}{0pt}%
\pgfpathmoveto{\pgfqpoint{3.411513in}{2.336294in}}%
\pgfusepath{stroke}%
\end{pgfscope}%
\begin{pgfscope}%
\pgfpathrectangle{\pgfqpoint{0.647939in}{0.492442in}}{\pgfqpoint{4.273799in}{2.331163in}}%
\pgfusepath{clip}%
\pgfsetroundcap%
\pgfsetroundjoin%
\definecolor{currentfill}{rgb}{0.500000,0.500000,0.500000}%
\pgfsetfillcolor{currentfill}%
\pgfsetfillopacity{0.300000}%
\pgfsetlinewidth{0.301125pt}%
\definecolor{currentstroke}{rgb}{0.500000,0.500000,0.500000}%
\pgfsetstrokecolor{currentstroke}%
\pgfsetstrokeopacity{0.300000}%
\pgfsetdash{}{0pt}%
\pgfpathmoveto{\pgfqpoint{0.000000in}{0.000000in}}%
\pgfpathlineto{\pgfqpoint{0.000000in}{0.000000in}}%
\pgfpathclose%
\pgfusepath{stroke,fill}%
\end{pgfscope}%
\begin{pgfscope}%
\pgfpathrectangle{\pgfqpoint{0.647939in}{0.492442in}}{\pgfqpoint{4.273799in}{2.331163in}}%
\pgfusepath{clip}%
\pgfsetroundcap%
\pgfsetroundjoin%
\pgfsetlinewidth{0.301125pt}%
\definecolor{currentstroke}{rgb}{0.500000,0.500000,0.500000}%
\pgfsetstrokecolor{currentstroke}%
\pgfsetstrokeopacity{0.300000}%
\pgfsetdash{}{0pt}%
\pgfpathmoveto{\pgfqpoint{3.135206in}{2.551752in}}%
\pgfusepath{stroke}%
\end{pgfscope}%
\begin{pgfscope}%
\pgfpathrectangle{\pgfqpoint{0.647939in}{0.492442in}}{\pgfqpoint{4.273799in}{2.331163in}}%
\pgfusepath{clip}%
\pgfsetroundcap%
\pgfsetroundjoin%
\definecolor{currentfill}{rgb}{0.500000,0.500000,0.500000}%
\pgfsetfillcolor{currentfill}%
\pgfsetfillopacity{0.300000}%
\pgfsetlinewidth{0.301125pt}%
\definecolor{currentstroke}{rgb}{0.500000,0.500000,0.500000}%
\pgfsetstrokecolor{currentstroke}%
\pgfsetstrokeopacity{0.300000}%
\pgfsetdash{}{0pt}%
\pgfpathmoveto{\pgfqpoint{0.000000in}{0.000000in}}%
\pgfpathlineto{\pgfqpoint{0.000000in}{0.000000in}}%
\pgfpathclose%
\pgfusepath{stroke,fill}%
\end{pgfscope}%
\begin{pgfscope}%
\pgfpathrectangle{\pgfqpoint{0.647939in}{0.492442in}}{\pgfqpoint{4.273799in}{2.331163in}}%
\pgfusepath{clip}%
\pgfsetroundcap%
\pgfsetroundjoin%
\pgfsetlinewidth{0.301125pt}%
\definecolor{currentstroke}{rgb}{0.500000,0.500000,0.500000}%
\pgfsetstrokecolor{currentstroke}%
\pgfsetstrokeopacity{0.300000}%
\pgfsetdash{}{0pt}%
\pgfpathmoveto{\pgfqpoint{2.761140in}{2.744414in}}%
\pgfusepath{stroke}%
\end{pgfscope}%
\begin{pgfscope}%
\pgfpathrectangle{\pgfqpoint{0.647939in}{0.492442in}}{\pgfqpoint{4.273799in}{2.331163in}}%
\pgfusepath{clip}%
\pgfsetroundcap%
\pgfsetroundjoin%
\definecolor{currentfill}{rgb}{0.500000,0.500000,0.500000}%
\pgfsetfillcolor{currentfill}%
\pgfsetfillopacity{0.300000}%
\pgfsetlinewidth{0.301125pt}%
\definecolor{currentstroke}{rgb}{0.500000,0.500000,0.500000}%
\pgfsetstrokecolor{currentstroke}%
\pgfsetstrokeopacity{0.300000}%
\pgfsetdash{}{0pt}%
\pgfpathmoveto{\pgfqpoint{0.000000in}{0.000000in}}%
\pgfpathlineto{\pgfqpoint{0.000000in}{0.000000in}}%
\pgfpathclose%
\pgfusepath{stroke,fill}%
\end{pgfscope}%
\begin{pgfscope}%
\pgfpathrectangle{\pgfqpoint{0.647939in}{0.492442in}}{\pgfqpoint{4.273799in}{2.331163in}}%
\pgfusepath{clip}%
\pgfsetroundcap%
\pgfsetroundjoin%
\pgfsetlinewidth{0.301125pt}%
\definecolor{currentstroke}{rgb}{0.500000,0.500000,0.500000}%
\pgfsetstrokecolor{currentstroke}%
\pgfsetstrokeopacity{0.300000}%
\pgfsetdash{}{0pt}%
\pgfpathmoveto{\pgfqpoint{2.995851in}{2.531684in}}%
\pgfusepath{stroke}%
\end{pgfscope}%
\begin{pgfscope}%
\pgfpathrectangle{\pgfqpoint{0.647939in}{0.492442in}}{\pgfqpoint{4.273799in}{2.331163in}}%
\pgfusepath{clip}%
\pgfsetroundcap%
\pgfsetroundjoin%
\definecolor{currentfill}{rgb}{0.500000,0.500000,0.500000}%
\pgfsetfillcolor{currentfill}%
\pgfsetfillopacity{0.300000}%
\pgfsetlinewidth{0.301125pt}%
\definecolor{currentstroke}{rgb}{0.500000,0.500000,0.500000}%
\pgfsetstrokecolor{currentstroke}%
\pgfsetstrokeopacity{0.300000}%
\pgfsetdash{}{0pt}%
\pgfpathmoveto{\pgfqpoint{0.000000in}{0.000000in}}%
\pgfpathlineto{\pgfqpoint{0.000000in}{0.000000in}}%
\pgfpathclose%
\pgfusepath{stroke,fill}%
\end{pgfscope}%
\begin{pgfscope}%
\pgfpathrectangle{\pgfqpoint{0.647939in}{0.492442in}}{\pgfqpoint{4.273799in}{2.331163in}}%
\pgfusepath{clip}%
\pgfsetroundcap%
\pgfsetroundjoin%
\pgfsetlinewidth{0.301125pt}%
\definecolor{currentstroke}{rgb}{0.500000,0.500000,0.500000}%
\pgfsetstrokecolor{currentstroke}%
\pgfsetstrokeopacity{0.300000}%
\pgfsetdash{}{0pt}%
\pgfpathmoveto{\pgfqpoint{2.392154in}{2.748408in}}%
\pgfusepath{stroke}%
\end{pgfscope}%
\begin{pgfscope}%
\pgfpathrectangle{\pgfqpoint{0.647939in}{0.492442in}}{\pgfqpoint{4.273799in}{2.331163in}}%
\pgfusepath{clip}%
\pgfsetroundcap%
\pgfsetroundjoin%
\definecolor{currentfill}{rgb}{0.500000,0.500000,0.500000}%
\pgfsetfillcolor{currentfill}%
\pgfsetfillopacity{0.300000}%
\pgfsetlinewidth{0.301125pt}%
\definecolor{currentstroke}{rgb}{0.500000,0.500000,0.500000}%
\pgfsetstrokecolor{currentstroke}%
\pgfsetstrokeopacity{0.300000}%
\pgfsetdash{}{0pt}%
\pgfpathmoveto{\pgfqpoint{0.000000in}{0.000000in}}%
\pgfpathlineto{\pgfqpoint{0.000000in}{0.000000in}}%
\pgfpathclose%
\pgfusepath{stroke,fill}%
\end{pgfscope}%
\begin{pgfscope}%
\pgfpathrectangle{\pgfqpoint{0.647939in}{0.492442in}}{\pgfqpoint{4.273799in}{2.331163in}}%
\pgfusepath{clip}%
\pgfsetroundcap%
\pgfsetroundjoin%
\pgfsetlinewidth{0.301125pt}%
\definecolor{currentstroke}{rgb}{0.500000,0.500000,0.500000}%
\pgfsetstrokecolor{currentstroke}%
\pgfsetstrokeopacity{0.300000}%
\pgfsetdash{}{0pt}%
\pgfpathmoveto{\pgfqpoint{2.179284in}{2.731936in}}%
\pgfusepath{stroke}%
\end{pgfscope}%
\begin{pgfscope}%
\pgfpathrectangle{\pgfqpoint{0.647939in}{0.492442in}}{\pgfqpoint{4.273799in}{2.331163in}}%
\pgfusepath{clip}%
\pgfsetroundcap%
\pgfsetroundjoin%
\definecolor{currentfill}{rgb}{0.500000,0.500000,0.500000}%
\pgfsetfillcolor{currentfill}%
\pgfsetfillopacity{0.300000}%
\pgfsetlinewidth{0.301125pt}%
\definecolor{currentstroke}{rgb}{0.500000,0.500000,0.500000}%
\pgfsetstrokecolor{currentstroke}%
\pgfsetstrokeopacity{0.300000}%
\pgfsetdash{}{0pt}%
\pgfpathmoveto{\pgfqpoint{0.000000in}{0.000000in}}%
\pgfpathlineto{\pgfqpoint{0.000000in}{0.000000in}}%
\pgfpathclose%
\pgfusepath{stroke,fill}%
\end{pgfscope}%
\begin{pgfscope}%
\pgfpathrectangle{\pgfqpoint{0.647939in}{0.492442in}}{\pgfqpoint{4.273799in}{2.331163in}}%
\pgfusepath{clip}%
\pgfsetroundcap%
\pgfsetroundjoin%
\pgfsetlinewidth{0.301125pt}%
\definecolor{currentstroke}{rgb}{0.500000,0.500000,0.500000}%
\pgfsetstrokecolor{currentstroke}%
\pgfsetstrokeopacity{0.300000}%
\pgfsetdash{}{0pt}%
\pgfpathmoveto{\pgfqpoint{2.283709in}{2.649184in}}%
\pgfusepath{stroke}%
\end{pgfscope}%
\begin{pgfscope}%
\pgfpathrectangle{\pgfqpoint{0.647939in}{0.492442in}}{\pgfqpoint{4.273799in}{2.331163in}}%
\pgfusepath{clip}%
\pgfsetroundcap%
\pgfsetroundjoin%
\definecolor{currentfill}{rgb}{0.500000,0.500000,0.500000}%
\pgfsetfillcolor{currentfill}%
\pgfsetfillopacity{0.300000}%
\pgfsetlinewidth{0.301125pt}%
\definecolor{currentstroke}{rgb}{0.500000,0.500000,0.500000}%
\pgfsetstrokecolor{currentstroke}%
\pgfsetstrokeopacity{0.300000}%
\pgfsetdash{}{0pt}%
\pgfpathmoveto{\pgfqpoint{0.000000in}{0.000000in}}%
\pgfpathlineto{\pgfqpoint{0.000000in}{0.000000in}}%
\pgfpathclose%
\pgfusepath{stroke,fill}%
\end{pgfscope}%
\begin{pgfscope}%
\pgfpathrectangle{\pgfqpoint{0.647939in}{0.492442in}}{\pgfqpoint{4.273799in}{2.331163in}}%
\pgfusepath{clip}%
\pgfsetroundcap%
\pgfsetroundjoin%
\pgfsetlinewidth{0.301125pt}%
\definecolor{currentstroke}{rgb}{0.500000,0.500000,0.500000}%
\pgfsetstrokecolor{currentstroke}%
\pgfsetstrokeopacity{0.300000}%
\pgfsetdash{}{0pt}%
\pgfpathmoveto{\pgfqpoint{1.919787in}{2.598830in}}%
\pgfusepath{stroke}%
\end{pgfscope}%
\begin{pgfscope}%
\pgfpathrectangle{\pgfqpoint{0.647939in}{0.492442in}}{\pgfqpoint{4.273799in}{2.331163in}}%
\pgfusepath{clip}%
\pgfsetroundcap%
\pgfsetroundjoin%
\definecolor{currentfill}{rgb}{0.500000,0.500000,0.500000}%
\pgfsetfillcolor{currentfill}%
\pgfsetfillopacity{0.300000}%
\pgfsetlinewidth{0.301125pt}%
\definecolor{currentstroke}{rgb}{0.500000,0.500000,0.500000}%
\pgfsetstrokecolor{currentstroke}%
\pgfsetstrokeopacity{0.300000}%
\pgfsetdash{}{0pt}%
\pgfpathmoveto{\pgfqpoint{0.000000in}{0.000000in}}%
\pgfpathlineto{\pgfqpoint{0.000000in}{0.000000in}}%
\pgfpathclose%
\pgfusepath{stroke,fill}%
\end{pgfscope}%
\begin{pgfscope}%
\pgfpathrectangle{\pgfqpoint{0.647939in}{0.492442in}}{\pgfqpoint{4.273799in}{2.331163in}}%
\pgfusepath{clip}%
\pgfsetroundcap%
\pgfsetroundjoin%
\pgfsetlinewidth{0.301125pt}%
\definecolor{currentstroke}{rgb}{0.500000,0.500000,0.500000}%
\pgfsetstrokecolor{currentstroke}%
\pgfsetstrokeopacity{0.300000}%
\pgfsetdash{}{0pt}%
\pgfpathmoveto{\pgfqpoint{2.000937in}{2.514303in}}%
\pgfusepath{stroke}%
\end{pgfscope}%
\begin{pgfscope}%
\pgfpathrectangle{\pgfqpoint{0.647939in}{0.492442in}}{\pgfqpoint{4.273799in}{2.331163in}}%
\pgfusepath{clip}%
\pgfsetroundcap%
\pgfsetroundjoin%
\definecolor{currentfill}{rgb}{0.500000,0.500000,0.500000}%
\pgfsetfillcolor{currentfill}%
\pgfsetfillopacity{0.300000}%
\pgfsetlinewidth{0.301125pt}%
\definecolor{currentstroke}{rgb}{0.500000,0.500000,0.500000}%
\pgfsetstrokecolor{currentstroke}%
\pgfsetstrokeopacity{0.300000}%
\pgfsetdash{}{0pt}%
\pgfpathmoveto{\pgfqpoint{0.000000in}{0.000000in}}%
\pgfpathlineto{\pgfqpoint{0.000000in}{0.000000in}}%
\pgfpathclose%
\pgfusepath{stroke,fill}%
\end{pgfscope}%
\begin{pgfscope}%
\pgfpathrectangle{\pgfqpoint{0.647939in}{0.492442in}}{\pgfqpoint{4.273799in}{2.331163in}}%
\pgfusepath{clip}%
\pgfsetroundcap%
\pgfsetroundjoin%
\pgfsetlinewidth{0.301125pt}%
\definecolor{currentstroke}{rgb}{0.500000,0.500000,0.500000}%
\pgfsetstrokecolor{currentstroke}%
\pgfsetstrokeopacity{0.300000}%
\pgfsetdash{}{0pt}%
\pgfpathmoveto{\pgfqpoint{1.637619in}{2.427112in}}%
\pgfusepath{stroke}%
\end{pgfscope}%
\begin{pgfscope}%
\pgfpathrectangle{\pgfqpoint{0.647939in}{0.492442in}}{\pgfqpoint{4.273799in}{2.331163in}}%
\pgfusepath{clip}%
\pgfsetroundcap%
\pgfsetroundjoin%
\definecolor{currentfill}{rgb}{0.500000,0.500000,0.500000}%
\pgfsetfillcolor{currentfill}%
\pgfsetfillopacity{0.300000}%
\pgfsetlinewidth{0.301125pt}%
\definecolor{currentstroke}{rgb}{0.500000,0.500000,0.500000}%
\pgfsetstrokecolor{currentstroke}%
\pgfsetstrokeopacity{0.300000}%
\pgfsetdash{}{0pt}%
\pgfpathmoveto{\pgfqpoint{0.000000in}{0.000000in}}%
\pgfpathlineto{\pgfqpoint{0.000000in}{0.000000in}}%
\pgfpathclose%
\pgfusepath{stroke,fill}%
\end{pgfscope}%
\begin{pgfscope}%
\pgfpathrectangle{\pgfqpoint{0.647939in}{0.492442in}}{\pgfqpoint{4.273799in}{2.331163in}}%
\pgfusepath{clip}%
\pgfsetroundcap%
\pgfsetroundjoin%
\pgfsetlinewidth{0.301125pt}%
\definecolor{currentstroke}{rgb}{0.500000,0.500000,0.500000}%
\pgfsetstrokecolor{currentstroke}%
\pgfsetstrokeopacity{0.300000}%
\pgfsetdash{}{0pt}%
\pgfpathmoveto{\pgfqpoint{1.441367in}{2.399351in}}%
\pgfusepath{stroke}%
\end{pgfscope}%
\begin{pgfscope}%
\pgfpathrectangle{\pgfqpoint{0.647939in}{0.492442in}}{\pgfqpoint{4.273799in}{2.331163in}}%
\pgfusepath{clip}%
\pgfsetroundcap%
\pgfsetroundjoin%
\definecolor{currentfill}{rgb}{0.500000,0.500000,0.500000}%
\pgfsetfillcolor{currentfill}%
\pgfsetfillopacity{0.300000}%
\pgfsetlinewidth{0.301125pt}%
\definecolor{currentstroke}{rgb}{0.500000,0.500000,0.500000}%
\pgfsetstrokecolor{currentstroke}%
\pgfsetstrokeopacity{0.300000}%
\pgfsetdash{}{0pt}%
\pgfpathmoveto{\pgfqpoint{0.000000in}{0.000000in}}%
\pgfpathlineto{\pgfqpoint{0.000000in}{0.000000in}}%
\pgfpathclose%
\pgfusepath{stroke,fill}%
\end{pgfscope}%
\begin{pgfscope}%
\pgfpathrectangle{\pgfqpoint{0.647939in}{0.492442in}}{\pgfqpoint{4.273799in}{2.331163in}}%
\pgfusepath{clip}%
\pgfsetroundcap%
\pgfsetroundjoin%
\pgfsetlinewidth{0.301125pt}%
\definecolor{currentstroke}{rgb}{0.500000,0.500000,0.500000}%
\pgfsetstrokecolor{currentstroke}%
\pgfsetstrokeopacity{0.300000}%
\pgfsetdash{}{0pt}%
\pgfpathmoveto{\pgfqpoint{1.314472in}{2.315082in}}%
\pgfusepath{stroke}%
\end{pgfscope}%
\begin{pgfscope}%
\pgfpathrectangle{\pgfqpoint{0.647939in}{0.492442in}}{\pgfqpoint{4.273799in}{2.331163in}}%
\pgfusepath{clip}%
\pgfsetroundcap%
\pgfsetroundjoin%
\definecolor{currentfill}{rgb}{0.500000,0.500000,0.500000}%
\pgfsetfillcolor{currentfill}%
\pgfsetfillopacity{0.300000}%
\pgfsetlinewidth{0.301125pt}%
\definecolor{currentstroke}{rgb}{0.500000,0.500000,0.500000}%
\pgfsetstrokecolor{currentstroke}%
\pgfsetstrokeopacity{0.300000}%
\pgfsetdash{}{0pt}%
\pgfpathmoveto{\pgfqpoint{0.000000in}{0.000000in}}%
\pgfpathlineto{\pgfqpoint{0.000000in}{0.000000in}}%
\pgfpathclose%
\pgfusepath{stroke,fill}%
\end{pgfscope}%
\begin{pgfscope}%
\pgfpathrectangle{\pgfqpoint{0.647939in}{0.492442in}}{\pgfqpoint{4.273799in}{2.331163in}}%
\pgfusepath{clip}%
\pgfsetroundcap%
\pgfsetroundjoin%
\pgfsetlinewidth{0.301125pt}%
\definecolor{currentstroke}{rgb}{0.500000,0.500000,0.500000}%
\pgfsetstrokecolor{currentstroke}%
\pgfsetstrokeopacity{0.300000}%
\pgfsetdash{}{0pt}%
\pgfpathmoveto{\pgfqpoint{1.215407in}{2.001255in}}%
\pgfusepath{stroke}%
\end{pgfscope}%
\begin{pgfscope}%
\pgfpathrectangle{\pgfqpoint{0.647939in}{0.492442in}}{\pgfqpoint{4.273799in}{2.331163in}}%
\pgfusepath{clip}%
\pgfsetroundcap%
\pgfsetroundjoin%
\definecolor{currentfill}{rgb}{0.500000,0.500000,0.500000}%
\pgfsetfillcolor{currentfill}%
\pgfsetfillopacity{0.300000}%
\pgfsetlinewidth{0.301125pt}%
\definecolor{currentstroke}{rgb}{0.500000,0.500000,0.500000}%
\pgfsetstrokecolor{currentstroke}%
\pgfsetstrokeopacity{0.300000}%
\pgfsetdash{}{0pt}%
\pgfpathmoveto{\pgfqpoint{0.000000in}{0.000000in}}%
\pgfpathlineto{\pgfqpoint{0.000000in}{0.000000in}}%
\pgfpathclose%
\pgfusepath{stroke,fill}%
\end{pgfscope}%
\begin{pgfscope}%
\pgfpathrectangle{\pgfqpoint{0.647939in}{0.492442in}}{\pgfqpoint{4.273799in}{2.331163in}}%
\pgfusepath{clip}%
\pgfsetroundcap%
\pgfsetroundjoin%
\pgfsetlinewidth{0.301125pt}%
\definecolor{currentstroke}{rgb}{0.500000,0.500000,0.500000}%
\pgfsetstrokecolor{currentstroke}%
\pgfsetstrokeopacity{0.300000}%
\pgfsetdash{}{0pt}%
\pgfpathmoveto{\pgfqpoint{1.063092in}{2.205791in}}%
\pgfusepath{stroke}%
\end{pgfscope}%
\begin{pgfscope}%
\pgfpathrectangle{\pgfqpoint{0.647939in}{0.492442in}}{\pgfqpoint{4.273799in}{2.331163in}}%
\pgfusepath{clip}%
\pgfsetroundcap%
\pgfsetroundjoin%
\definecolor{currentfill}{rgb}{0.500000,0.500000,0.500000}%
\pgfsetfillcolor{currentfill}%
\pgfsetfillopacity{0.300000}%
\pgfsetlinewidth{0.301125pt}%
\definecolor{currentstroke}{rgb}{0.500000,0.500000,0.500000}%
\pgfsetstrokecolor{currentstroke}%
\pgfsetstrokeopacity{0.300000}%
\pgfsetdash{}{0pt}%
\pgfpathmoveto{\pgfqpoint{0.000000in}{0.000000in}}%
\pgfpathlineto{\pgfqpoint{0.000000in}{0.000000in}}%
\pgfpathclose%
\pgfusepath{stroke,fill}%
\end{pgfscope}%
\begin{pgfscope}%
\pgfpathrectangle{\pgfqpoint{0.647939in}{0.492442in}}{\pgfqpoint{4.273799in}{2.331163in}}%
\pgfusepath{clip}%
\pgfsetroundcap%
\pgfsetroundjoin%
\pgfsetlinewidth{0.301125pt}%
\definecolor{currentstroke}{rgb}{0.500000,0.500000,0.500000}%
\pgfsetstrokecolor{currentstroke}%
\pgfsetstrokeopacity{0.300000}%
\pgfsetdash{}{0pt}%
\pgfpathmoveto{\pgfqpoint{0.946278in}{1.790231in}}%
\pgfusepath{stroke}%
\end{pgfscope}%
\begin{pgfscope}%
\pgfpathrectangle{\pgfqpoint{0.647939in}{0.492442in}}{\pgfqpoint{4.273799in}{2.331163in}}%
\pgfusepath{clip}%
\pgfsetroundcap%
\pgfsetroundjoin%
\definecolor{currentfill}{rgb}{0.500000,0.500000,0.500000}%
\pgfsetfillcolor{currentfill}%
\pgfsetfillopacity{0.300000}%
\pgfsetlinewidth{0.301125pt}%
\definecolor{currentstroke}{rgb}{0.500000,0.500000,0.500000}%
\pgfsetstrokecolor{currentstroke}%
\pgfsetstrokeopacity{0.300000}%
\pgfsetdash{}{0pt}%
\pgfpathmoveto{\pgfqpoint{0.000000in}{0.000000in}}%
\pgfpathlineto{\pgfqpoint{0.000000in}{0.000000in}}%
\pgfpathclose%
\pgfusepath{stroke,fill}%
\end{pgfscope}%
\begin{pgfscope}%
\pgfpathrectangle{\pgfqpoint{0.647939in}{0.492442in}}{\pgfqpoint{4.273799in}{2.331163in}}%
\pgfusepath{clip}%
\pgfsetroundcap%
\pgfsetroundjoin%
\pgfsetlinewidth{0.301125pt}%
\definecolor{currentstroke}{rgb}{0.500000,0.500000,0.500000}%
\pgfsetstrokecolor{currentstroke}%
\pgfsetstrokeopacity{0.300000}%
\pgfsetdash{}{0pt}%
\pgfpathmoveto{\pgfqpoint{0.836370in}{1.996568in}}%
\pgfusepath{stroke}%
\end{pgfscope}%
\begin{pgfscope}%
\pgfpathrectangle{\pgfqpoint{0.647939in}{0.492442in}}{\pgfqpoint{4.273799in}{2.331163in}}%
\pgfusepath{clip}%
\pgfsetroundcap%
\pgfsetroundjoin%
\definecolor{currentfill}{rgb}{0.500000,0.500000,0.500000}%
\pgfsetfillcolor{currentfill}%
\pgfsetfillopacity{0.300000}%
\pgfsetlinewidth{0.301125pt}%
\definecolor{currentstroke}{rgb}{0.500000,0.500000,0.500000}%
\pgfsetstrokecolor{currentstroke}%
\pgfsetstrokeopacity{0.300000}%
\pgfsetdash{}{0pt}%
\pgfpathmoveto{\pgfqpoint{0.000000in}{0.000000in}}%
\pgfpathlineto{\pgfqpoint{0.000000in}{0.000000in}}%
\pgfpathclose%
\pgfusepath{stroke,fill}%
\end{pgfscope}%
\begin{pgfscope}%
\pgfpathrectangle{\pgfqpoint{0.647939in}{0.492442in}}{\pgfqpoint{4.273799in}{2.331163in}}%
\pgfusepath{clip}%
\pgfsetroundcap%
\pgfsetroundjoin%
\pgfsetlinewidth{0.301125pt}%
\definecolor{currentstroke}{rgb}{0.500000,0.500000,0.500000}%
\pgfsetstrokecolor{currentstroke}%
\pgfsetstrokeopacity{0.300000}%
\pgfsetdash{}{0pt}%
\pgfpathmoveto{\pgfqpoint{0.723841in}{2.047846in}}%
\pgfusepath{stroke}%
\end{pgfscope}%
\begin{pgfscope}%
\pgfpathrectangle{\pgfqpoint{0.647939in}{0.492442in}}{\pgfqpoint{4.273799in}{2.331163in}}%
\pgfusepath{clip}%
\pgfsetroundcap%
\pgfsetroundjoin%
\definecolor{currentfill}{rgb}{0.500000,0.500000,0.500000}%
\pgfsetfillcolor{currentfill}%
\pgfsetfillopacity{0.300000}%
\pgfsetlinewidth{0.301125pt}%
\definecolor{currentstroke}{rgb}{0.500000,0.500000,0.500000}%
\pgfsetstrokecolor{currentstroke}%
\pgfsetstrokeopacity{0.300000}%
\pgfsetdash{}{0pt}%
\pgfpathmoveto{\pgfqpoint{0.000000in}{0.000000in}}%
\pgfpathlineto{\pgfqpoint{0.000000in}{0.000000in}}%
\pgfpathclose%
\pgfusepath{stroke,fill}%
\end{pgfscope}%
\begin{pgfscope}%
\pgfpathrectangle{\pgfqpoint{0.647939in}{0.492442in}}{\pgfqpoint{4.273799in}{2.331163in}}%
\pgfusepath{clip}%
\pgfsetroundcap%
\pgfsetroundjoin%
\pgfsetlinewidth{0.301125pt}%
\definecolor{currentstroke}{rgb}{0.500000,0.500000,0.500000}%
\pgfsetstrokecolor{currentstroke}%
\pgfsetstrokeopacity{0.300000}%
\pgfsetdash{}{0pt}%
\pgfpathmoveto{\pgfqpoint{0.672134in}{2.038555in}}%
\pgfusepath{stroke}%
\end{pgfscope}%
\begin{pgfscope}%
\pgfpathrectangle{\pgfqpoint{0.647939in}{0.492442in}}{\pgfqpoint{4.273799in}{2.331163in}}%
\pgfusepath{clip}%
\pgfsetroundcap%
\pgfsetroundjoin%
\definecolor{currentfill}{rgb}{0.500000,0.500000,0.500000}%
\pgfsetfillcolor{currentfill}%
\pgfsetfillopacity{0.300000}%
\pgfsetlinewidth{0.301125pt}%
\definecolor{currentstroke}{rgb}{0.500000,0.500000,0.500000}%
\pgfsetstrokecolor{currentstroke}%
\pgfsetstrokeopacity{0.300000}%
\pgfsetdash{}{0pt}%
\pgfpathmoveto{\pgfqpoint{0.000000in}{0.000000in}}%
\pgfpathlineto{\pgfqpoint{0.000000in}{0.000000in}}%
\pgfpathclose%
\pgfusepath{stroke,fill}%
\end{pgfscope}%
\begin{pgfscope}%
\pgfpathrectangle{\pgfqpoint{0.647939in}{0.492442in}}{\pgfqpoint{4.273799in}{2.331163in}}%
\pgfusepath{clip}%
\pgfsetroundcap%
\pgfsetroundjoin%
\pgfsetlinewidth{0.301125pt}%
\definecolor{currentstroke}{rgb}{0.500000,0.500000,0.500000}%
\pgfsetstrokecolor{currentstroke}%
\pgfsetstrokeopacity{0.300000}%
\pgfsetdash{}{0pt}%
\pgfpathmoveto{\pgfqpoint{4.523831in}{2.768051in}}%
\pgfusepath{stroke}%
\end{pgfscope}%
\begin{pgfscope}%
\pgfpathrectangle{\pgfqpoint{0.647939in}{0.492442in}}{\pgfqpoint{4.273799in}{2.331163in}}%
\pgfusepath{clip}%
\pgfsetroundcap%
\pgfsetroundjoin%
\definecolor{currentfill}{rgb}{0.500000,0.500000,0.500000}%
\pgfsetfillcolor{currentfill}%
\pgfsetfillopacity{0.300000}%
\pgfsetlinewidth{0.301125pt}%
\definecolor{currentstroke}{rgb}{0.500000,0.500000,0.500000}%
\pgfsetstrokecolor{currentstroke}%
\pgfsetstrokeopacity{0.300000}%
\pgfsetdash{}{0pt}%
\pgfpathmoveto{\pgfqpoint{0.000000in}{0.000000in}}%
\pgfpathlineto{\pgfqpoint{0.000000in}{0.000000in}}%
\pgfpathclose%
\pgfusepath{stroke,fill}%
\end{pgfscope}%
\begin{pgfscope}%
\pgfpathrectangle{\pgfqpoint{0.647939in}{0.492442in}}{\pgfqpoint{4.273799in}{2.331163in}}%
\pgfusepath{clip}%
\pgfsetroundcap%
\pgfsetroundjoin%
\pgfsetlinewidth{0.301125pt}%
\definecolor{currentstroke}{rgb}{0.500000,0.500000,0.500000}%
\pgfsetstrokecolor{currentstroke}%
\pgfsetstrokeopacity{0.300000}%
\pgfsetdash{}{0pt}%
\pgfpathmoveto{\pgfqpoint{3.982870in}{0.626016in}}%
\pgfusepath{stroke}%
\end{pgfscope}%
\begin{pgfscope}%
\pgfpathrectangle{\pgfqpoint{0.647939in}{0.492442in}}{\pgfqpoint{4.273799in}{2.331163in}}%
\pgfusepath{clip}%
\pgfsetroundcap%
\pgfsetroundjoin%
\definecolor{currentfill}{rgb}{0.500000,0.500000,0.500000}%
\pgfsetfillcolor{currentfill}%
\pgfsetfillopacity{0.300000}%
\pgfsetlinewidth{0.301125pt}%
\definecolor{currentstroke}{rgb}{0.500000,0.500000,0.500000}%
\pgfsetstrokecolor{currentstroke}%
\pgfsetstrokeopacity{0.300000}%
\pgfsetdash{}{0pt}%
\pgfpathmoveto{\pgfqpoint{0.000000in}{0.000000in}}%
\pgfpathlineto{\pgfqpoint{0.000000in}{0.000000in}}%
\pgfpathclose%
\pgfusepath{stroke,fill}%
\end{pgfscope}%
\begin{pgfscope}%
\pgfpathrectangle{\pgfqpoint{0.647939in}{0.492442in}}{\pgfqpoint{4.273799in}{2.331163in}}%
\pgfusepath{clip}%
\pgfsetroundcap%
\pgfsetroundjoin%
\pgfsetlinewidth{0.301125pt}%
\definecolor{currentstroke}{rgb}{0.500000,0.500000,0.500000}%
\pgfsetstrokecolor{currentstroke}%
\pgfsetstrokeopacity{0.300000}%
\pgfsetdash{}{0pt}%
\pgfpathmoveto{\pgfqpoint{1.838212in}{2.559558in}}%
\pgfusepath{stroke}%
\end{pgfscope}%
\begin{pgfscope}%
\pgfpathrectangle{\pgfqpoint{0.647939in}{0.492442in}}{\pgfqpoint{4.273799in}{2.331163in}}%
\pgfusepath{clip}%
\pgfsetroundcap%
\pgfsetroundjoin%
\definecolor{currentfill}{rgb}{0.500000,0.500000,0.500000}%
\pgfsetfillcolor{currentfill}%
\pgfsetfillopacity{0.300000}%
\pgfsetlinewidth{0.301125pt}%
\definecolor{currentstroke}{rgb}{0.500000,0.500000,0.500000}%
\pgfsetstrokecolor{currentstroke}%
\pgfsetstrokeopacity{0.300000}%
\pgfsetdash{}{0pt}%
\pgfpathmoveto{\pgfqpoint{0.000000in}{0.000000in}}%
\pgfpathlineto{\pgfqpoint{0.000000in}{0.000000in}}%
\pgfpathclose%
\pgfusepath{stroke,fill}%
\end{pgfscope}%
\begin{pgfscope}%
\pgfpathrectangle{\pgfqpoint{0.647939in}{0.492442in}}{\pgfqpoint{4.273799in}{2.331163in}}%
\pgfusepath{clip}%
\pgfsetroundcap%
\pgfsetroundjoin%
\pgfsetlinewidth{0.301125pt}%
\definecolor{currentstroke}{rgb}{0.500000,0.500000,0.500000}%
\pgfsetstrokecolor{currentstroke}%
\pgfsetstrokeopacity{0.300000}%
\pgfsetdash{}{0pt}%
\pgfpathmoveto{\pgfqpoint{3.786583in}{0.970443in}}%
\pgfusepath{stroke}%
\end{pgfscope}%
\begin{pgfscope}%
\pgfpathrectangle{\pgfqpoint{0.647939in}{0.492442in}}{\pgfqpoint{4.273799in}{2.331163in}}%
\pgfusepath{clip}%
\pgfsetroundcap%
\pgfsetroundjoin%
\definecolor{currentfill}{rgb}{0.500000,0.500000,0.500000}%
\pgfsetfillcolor{currentfill}%
\pgfsetfillopacity{0.300000}%
\pgfsetlinewidth{0.301125pt}%
\definecolor{currentstroke}{rgb}{0.500000,0.500000,0.500000}%
\pgfsetstrokecolor{currentstroke}%
\pgfsetstrokeopacity{0.300000}%
\pgfsetdash{}{0pt}%
\pgfpathmoveto{\pgfqpoint{0.000000in}{0.000000in}}%
\pgfpathlineto{\pgfqpoint{0.000000in}{0.000000in}}%
\pgfpathclose%
\pgfusepath{stroke,fill}%
\end{pgfscope}%
\begin{pgfscope}%
\pgfpathrectangle{\pgfqpoint{0.647939in}{0.492442in}}{\pgfqpoint{4.273799in}{2.331163in}}%
\pgfusepath{clip}%
\pgfsetroundcap%
\pgfsetroundjoin%
\pgfsetlinewidth{0.301125pt}%
\definecolor{currentstroke}{rgb}{0.500000,0.500000,0.500000}%
\pgfsetstrokecolor{currentstroke}%
\pgfsetstrokeopacity{0.300000}%
\pgfsetdash{}{0pt}%
\pgfpathmoveto{\pgfqpoint{4.347967in}{1.553859in}}%
\pgfusepath{stroke}%
\end{pgfscope}%
\begin{pgfscope}%
\pgfpathrectangle{\pgfqpoint{0.647939in}{0.492442in}}{\pgfqpoint{4.273799in}{2.331163in}}%
\pgfusepath{clip}%
\pgfsetroundcap%
\pgfsetroundjoin%
\definecolor{currentfill}{rgb}{0.500000,0.500000,0.500000}%
\pgfsetfillcolor{currentfill}%
\pgfsetfillopacity{0.300000}%
\pgfsetlinewidth{0.301125pt}%
\definecolor{currentstroke}{rgb}{0.500000,0.500000,0.500000}%
\pgfsetstrokecolor{currentstroke}%
\pgfsetstrokeopacity{0.300000}%
\pgfsetdash{}{0pt}%
\pgfpathmoveto{\pgfqpoint{0.000000in}{0.000000in}}%
\pgfpathlineto{\pgfqpoint{0.000000in}{0.000000in}}%
\pgfpathclose%
\pgfusepath{stroke,fill}%
\end{pgfscope}%
\begin{pgfscope}%
\pgfpathrectangle{\pgfqpoint{0.647939in}{0.492442in}}{\pgfqpoint{4.273799in}{2.331163in}}%
\pgfusepath{clip}%
\pgfsetroundcap%
\pgfsetroundjoin%
\pgfsetlinewidth{0.301125pt}%
\definecolor{currentstroke}{rgb}{0.500000,0.500000,0.500000}%
\pgfsetstrokecolor{currentstroke}%
\pgfsetstrokeopacity{0.300000}%
\pgfsetdash{}{0pt}%
\pgfpathmoveto{\pgfqpoint{4.483975in}{1.758097in}}%
\pgfusepath{stroke}%
\end{pgfscope}%
\begin{pgfscope}%
\pgfpathrectangle{\pgfqpoint{0.647939in}{0.492442in}}{\pgfqpoint{4.273799in}{2.331163in}}%
\pgfusepath{clip}%
\pgfsetroundcap%
\pgfsetroundjoin%
\definecolor{currentfill}{rgb}{0.500000,0.500000,0.500000}%
\pgfsetfillcolor{currentfill}%
\pgfsetfillopacity{0.300000}%
\pgfsetlinewidth{0.301125pt}%
\definecolor{currentstroke}{rgb}{0.500000,0.500000,0.500000}%
\pgfsetstrokecolor{currentstroke}%
\pgfsetstrokeopacity{0.300000}%
\pgfsetdash{}{0pt}%
\pgfpathmoveto{\pgfqpoint{0.000000in}{0.000000in}}%
\pgfpathlineto{\pgfqpoint{0.000000in}{0.000000in}}%
\pgfpathclose%
\pgfusepath{stroke,fill}%
\end{pgfscope}%
\begin{pgfscope}%
\pgfpathrectangle{\pgfqpoint{0.647939in}{0.492442in}}{\pgfqpoint{4.273799in}{2.331163in}}%
\pgfusepath{clip}%
\pgfsetroundcap%
\pgfsetroundjoin%
\pgfsetlinewidth{0.301125pt}%
\definecolor{currentstroke}{rgb}{0.500000,0.500000,0.500000}%
\pgfsetstrokecolor{currentstroke}%
\pgfsetstrokeopacity{0.300000}%
\pgfsetdash{}{0pt}%
\pgfpathmoveto{\pgfqpoint{4.505986in}{1.970498in}}%
\pgfusepath{stroke}%
\end{pgfscope}%
\begin{pgfscope}%
\pgfpathrectangle{\pgfqpoint{0.647939in}{0.492442in}}{\pgfqpoint{4.273799in}{2.331163in}}%
\pgfusepath{clip}%
\pgfsetroundcap%
\pgfsetroundjoin%
\definecolor{currentfill}{rgb}{0.500000,0.500000,0.500000}%
\pgfsetfillcolor{currentfill}%
\pgfsetfillopacity{0.300000}%
\pgfsetlinewidth{0.301125pt}%
\definecolor{currentstroke}{rgb}{0.500000,0.500000,0.500000}%
\pgfsetstrokecolor{currentstroke}%
\pgfsetstrokeopacity{0.300000}%
\pgfsetdash{}{0pt}%
\pgfpathmoveto{\pgfqpoint{0.000000in}{0.000000in}}%
\pgfpathlineto{\pgfqpoint{0.000000in}{0.000000in}}%
\pgfpathclose%
\pgfusepath{stroke,fill}%
\end{pgfscope}%
\begin{pgfscope}%
\pgfpathrectangle{\pgfqpoint{0.647939in}{0.492442in}}{\pgfqpoint{4.273799in}{2.331163in}}%
\pgfusepath{clip}%
\pgfsetroundcap%
\pgfsetroundjoin%
\pgfsetlinewidth{0.301125pt}%
\definecolor{currentstroke}{rgb}{0.500000,0.500000,0.500000}%
\pgfsetstrokecolor{currentstroke}%
\pgfsetstrokeopacity{0.300000}%
\pgfsetdash{}{0pt}%
\pgfpathmoveto{\pgfqpoint{2.789116in}{0.874413in}}%
\pgfusepath{stroke}%
\end{pgfscope}%
\begin{pgfscope}%
\pgfpathrectangle{\pgfqpoint{0.647939in}{0.492442in}}{\pgfqpoint{4.273799in}{2.331163in}}%
\pgfusepath{clip}%
\pgfsetroundcap%
\pgfsetroundjoin%
\definecolor{currentfill}{rgb}{0.500000,0.500000,0.500000}%
\pgfsetfillcolor{currentfill}%
\pgfsetfillopacity{0.300000}%
\pgfsetlinewidth{0.301125pt}%
\definecolor{currentstroke}{rgb}{0.500000,0.500000,0.500000}%
\pgfsetstrokecolor{currentstroke}%
\pgfsetstrokeopacity{0.300000}%
\pgfsetdash{}{0pt}%
\pgfpathmoveto{\pgfqpoint{0.000000in}{0.000000in}}%
\pgfpathlineto{\pgfqpoint{0.000000in}{0.000000in}}%
\pgfpathclose%
\pgfusepath{stroke,fill}%
\end{pgfscope}%
\begin{pgfscope}%
\pgfpathrectangle{\pgfqpoint{0.647939in}{0.492442in}}{\pgfqpoint{4.273799in}{2.331163in}}%
\pgfusepath{clip}%
\pgfsetroundcap%
\pgfsetroundjoin%
\pgfsetlinewidth{0.301125pt}%
\definecolor{currentstroke}{rgb}{0.500000,0.500000,0.500000}%
\pgfsetstrokecolor{currentstroke}%
\pgfsetstrokeopacity{0.300000}%
\pgfsetdash{}{0pt}%
\pgfpathmoveto{\pgfqpoint{4.123413in}{1.416246in}}%
\pgfusepath{stroke}%
\end{pgfscope}%
\begin{pgfscope}%
\pgfpathrectangle{\pgfqpoint{0.647939in}{0.492442in}}{\pgfqpoint{4.273799in}{2.331163in}}%
\pgfusepath{clip}%
\pgfsetroundcap%
\pgfsetroundjoin%
\definecolor{currentfill}{rgb}{0.500000,0.500000,0.500000}%
\pgfsetfillcolor{currentfill}%
\pgfsetfillopacity{0.300000}%
\pgfsetlinewidth{0.301125pt}%
\definecolor{currentstroke}{rgb}{0.500000,0.500000,0.500000}%
\pgfsetstrokecolor{currentstroke}%
\pgfsetstrokeopacity{0.300000}%
\pgfsetdash{}{0pt}%
\pgfpathmoveto{\pgfqpoint{0.000000in}{0.000000in}}%
\pgfpathlineto{\pgfqpoint{0.000000in}{0.000000in}}%
\pgfpathclose%
\pgfusepath{stroke,fill}%
\end{pgfscope}%
\begin{pgfscope}%
\pgfpathrectangle{\pgfqpoint{0.647939in}{0.492442in}}{\pgfqpoint{4.273799in}{2.331163in}}%
\pgfusepath{clip}%
\pgfsetroundcap%
\pgfsetroundjoin%
\pgfsetlinewidth{0.301125pt}%
\definecolor{currentstroke}{rgb}{0.500000,0.500000,0.500000}%
\pgfsetstrokecolor{currentstroke}%
\pgfsetstrokeopacity{0.300000}%
\pgfsetdash{}{0pt}%
\pgfpathmoveto{\pgfqpoint{1.704206in}{2.507310in}}%
\pgfusepath{stroke}%
\end{pgfscope}%
\begin{pgfscope}%
\pgfpathrectangle{\pgfqpoint{0.647939in}{0.492442in}}{\pgfqpoint{4.273799in}{2.331163in}}%
\pgfusepath{clip}%
\pgfsetroundcap%
\pgfsetroundjoin%
\definecolor{currentfill}{rgb}{0.500000,0.500000,0.500000}%
\pgfsetfillcolor{currentfill}%
\pgfsetfillopacity{0.300000}%
\pgfsetlinewidth{0.301125pt}%
\definecolor{currentstroke}{rgb}{0.500000,0.500000,0.500000}%
\pgfsetstrokecolor{currentstroke}%
\pgfsetstrokeopacity{0.300000}%
\pgfsetdash{}{0pt}%
\pgfpathmoveto{\pgfqpoint{0.000000in}{0.000000in}}%
\pgfpathlineto{\pgfqpoint{0.000000in}{0.000000in}}%
\pgfpathclose%
\pgfusepath{stroke,fill}%
\end{pgfscope}%
\begin{pgfscope}%
\pgfpathrectangle{\pgfqpoint{0.647939in}{0.492442in}}{\pgfqpoint{4.273799in}{2.331163in}}%
\pgfusepath{clip}%
\pgfsetroundcap%
\pgfsetroundjoin%
\pgfsetlinewidth{0.301125pt}%
\definecolor{currentstroke}{rgb}{0.500000,0.500000,0.500000}%
\pgfsetstrokecolor{currentstroke}%
\pgfsetstrokeopacity{0.300000}%
\pgfsetdash{}{0pt}%
\pgfpathmoveto{\pgfqpoint{1.155255in}{2.033975in}}%
\pgfusepath{stroke}%
\end{pgfscope}%
\begin{pgfscope}%
\pgfpathrectangle{\pgfqpoint{0.647939in}{0.492442in}}{\pgfqpoint{4.273799in}{2.331163in}}%
\pgfusepath{clip}%
\pgfsetroundcap%
\pgfsetroundjoin%
\definecolor{currentfill}{rgb}{0.500000,0.500000,0.500000}%
\pgfsetfillcolor{currentfill}%
\pgfsetfillopacity{0.300000}%
\pgfsetlinewidth{0.301125pt}%
\definecolor{currentstroke}{rgb}{0.500000,0.500000,0.500000}%
\pgfsetstrokecolor{currentstroke}%
\pgfsetstrokeopacity{0.300000}%
\pgfsetdash{}{0pt}%
\pgfpathmoveto{\pgfqpoint{0.000000in}{0.000000in}}%
\pgfpathlineto{\pgfqpoint{0.000000in}{0.000000in}}%
\pgfpathclose%
\pgfusepath{stroke,fill}%
\end{pgfscope}%
\begin{pgfscope}%
\pgfpathrectangle{\pgfqpoint{0.647939in}{0.492442in}}{\pgfqpoint{4.273799in}{2.331163in}}%
\pgfusepath{clip}%
\pgfsetroundcap%
\pgfsetroundjoin%
\pgfsetlinewidth{0.301125pt}%
\definecolor{currentstroke}{rgb}{0.500000,0.500000,0.500000}%
\pgfsetstrokecolor{currentstroke}%
\pgfsetstrokeopacity{0.300000}%
\pgfsetdash{}{0pt}%
\pgfpathmoveto{\pgfqpoint{4.253010in}{1.775330in}}%
\pgfusepath{stroke}%
\end{pgfscope}%
\begin{pgfscope}%
\pgfpathrectangle{\pgfqpoint{0.647939in}{0.492442in}}{\pgfqpoint{4.273799in}{2.331163in}}%
\pgfusepath{clip}%
\pgfsetroundcap%
\pgfsetroundjoin%
\definecolor{currentfill}{rgb}{0.500000,0.500000,0.500000}%
\pgfsetfillcolor{currentfill}%
\pgfsetfillopacity{0.300000}%
\pgfsetlinewidth{0.301125pt}%
\definecolor{currentstroke}{rgb}{0.500000,0.500000,0.500000}%
\pgfsetstrokecolor{currentstroke}%
\pgfsetstrokeopacity{0.300000}%
\pgfsetdash{}{0pt}%
\pgfpathmoveto{\pgfqpoint{0.000000in}{0.000000in}}%
\pgfpathlineto{\pgfqpoint{0.000000in}{0.000000in}}%
\pgfpathclose%
\pgfusepath{stroke,fill}%
\end{pgfscope}%
\begin{pgfscope}%
\pgfpathrectangle{\pgfqpoint{0.647939in}{0.492442in}}{\pgfqpoint{4.273799in}{2.331163in}}%
\pgfusepath{clip}%
\pgfsetroundcap%
\pgfsetroundjoin%
\pgfsetlinewidth{0.301125pt}%
\definecolor{currentstroke}{rgb}{0.500000,0.500000,0.500000}%
\pgfsetstrokecolor{currentstroke}%
\pgfsetstrokeopacity{0.300000}%
\pgfsetdash{}{0pt}%
\pgfpathmoveto{\pgfqpoint{1.456934in}{1.438525in}}%
\pgfusepath{stroke}%
\end{pgfscope}%
\begin{pgfscope}%
\pgfpathrectangle{\pgfqpoint{0.647939in}{0.492442in}}{\pgfqpoint{4.273799in}{2.331163in}}%
\pgfusepath{clip}%
\pgfsetroundcap%
\pgfsetroundjoin%
\definecolor{currentfill}{rgb}{0.500000,0.500000,0.500000}%
\pgfsetfillcolor{currentfill}%
\pgfsetfillopacity{0.300000}%
\pgfsetlinewidth{0.301125pt}%
\definecolor{currentstroke}{rgb}{0.500000,0.500000,0.500000}%
\pgfsetstrokecolor{currentstroke}%
\pgfsetstrokeopacity{0.300000}%
\pgfsetdash{}{0pt}%
\pgfpathmoveto{\pgfqpoint{0.000000in}{0.000000in}}%
\pgfpathlineto{\pgfqpoint{0.000000in}{0.000000in}}%
\pgfpathclose%
\pgfusepath{stroke,fill}%
\end{pgfscope}%
\begin{pgfscope}%
\pgfpathrectangle{\pgfqpoint{0.647939in}{0.492442in}}{\pgfqpoint{4.273799in}{2.331163in}}%
\pgfusepath{clip}%
\pgfsetroundcap%
\pgfsetroundjoin%
\pgfsetlinewidth{0.301125pt}%
\definecolor{currentstroke}{rgb}{0.500000,0.500000,0.500000}%
\pgfsetstrokecolor{currentstroke}%
\pgfsetstrokeopacity{0.300000}%
\pgfsetdash{}{0pt}%
\pgfpathmoveto{\pgfqpoint{4.168818in}{1.663630in}}%
\pgfusepath{stroke}%
\end{pgfscope}%
\begin{pgfscope}%
\pgfpathrectangle{\pgfqpoint{0.647939in}{0.492442in}}{\pgfqpoint{4.273799in}{2.331163in}}%
\pgfusepath{clip}%
\pgfsetroundcap%
\pgfsetroundjoin%
\definecolor{currentfill}{rgb}{0.500000,0.500000,0.500000}%
\pgfsetfillcolor{currentfill}%
\pgfsetfillopacity{0.300000}%
\pgfsetlinewidth{0.301125pt}%
\definecolor{currentstroke}{rgb}{0.500000,0.500000,0.500000}%
\pgfsetstrokecolor{currentstroke}%
\pgfsetstrokeopacity{0.300000}%
\pgfsetdash{}{0pt}%
\pgfpathmoveto{\pgfqpoint{0.000000in}{0.000000in}}%
\pgfpathlineto{\pgfqpoint{0.000000in}{0.000000in}}%
\pgfpathclose%
\pgfusepath{stroke,fill}%
\end{pgfscope}%
\begin{pgfscope}%
\pgfpathrectangle{\pgfqpoint{0.647939in}{0.492442in}}{\pgfqpoint{4.273799in}{2.331163in}}%
\pgfusepath{clip}%
\pgfsetroundcap%
\pgfsetroundjoin%
\pgfsetlinewidth{0.301125pt}%
\definecolor{currentstroke}{rgb}{0.500000,0.500000,0.500000}%
\pgfsetstrokecolor{currentstroke}%
\pgfsetstrokeopacity{0.300000}%
\pgfsetdash{}{0pt}%
\pgfpathmoveto{\pgfqpoint{4.192794in}{2.056621in}}%
\pgfusepath{stroke}%
\end{pgfscope}%
\begin{pgfscope}%
\pgfpathrectangle{\pgfqpoint{0.647939in}{0.492442in}}{\pgfqpoint{4.273799in}{2.331163in}}%
\pgfusepath{clip}%
\pgfsetroundcap%
\pgfsetroundjoin%
\definecolor{currentfill}{rgb}{0.500000,0.500000,0.500000}%
\pgfsetfillcolor{currentfill}%
\pgfsetfillopacity{0.300000}%
\pgfsetlinewidth{0.301125pt}%
\definecolor{currentstroke}{rgb}{0.500000,0.500000,0.500000}%
\pgfsetstrokecolor{currentstroke}%
\pgfsetstrokeopacity{0.300000}%
\pgfsetdash{}{0pt}%
\pgfpathmoveto{\pgfqpoint{0.000000in}{0.000000in}}%
\pgfpathlineto{\pgfqpoint{0.000000in}{0.000000in}}%
\pgfpathclose%
\pgfusepath{stroke,fill}%
\end{pgfscope}%
\begin{pgfscope}%
\pgfpathrectangle{\pgfqpoint{0.647939in}{0.492442in}}{\pgfqpoint{4.273799in}{2.331163in}}%
\pgfusepath{clip}%
\pgfsetroundcap%
\pgfsetroundjoin%
\pgfsetlinewidth{0.301125pt}%
\definecolor{currentstroke}{rgb}{0.500000,0.500000,0.500000}%
\pgfsetstrokecolor{currentstroke}%
\pgfsetstrokeopacity{0.300000}%
\pgfsetdash{}{0pt}%
\pgfpathmoveto{\pgfqpoint{1.665602in}{2.335854in}}%
\pgfusepath{stroke}%
\end{pgfscope}%
\begin{pgfscope}%
\pgfpathrectangle{\pgfqpoint{0.647939in}{0.492442in}}{\pgfqpoint{4.273799in}{2.331163in}}%
\pgfusepath{clip}%
\pgfsetroundcap%
\pgfsetroundjoin%
\definecolor{currentfill}{rgb}{0.500000,0.500000,0.500000}%
\pgfsetfillcolor{currentfill}%
\pgfsetfillopacity{0.300000}%
\pgfsetlinewidth{0.301125pt}%
\definecolor{currentstroke}{rgb}{0.500000,0.500000,0.500000}%
\pgfsetstrokecolor{currentstroke}%
\pgfsetstrokeopacity{0.300000}%
\pgfsetdash{}{0pt}%
\pgfpathmoveto{\pgfqpoint{0.000000in}{0.000000in}}%
\pgfpathlineto{\pgfqpoint{0.000000in}{0.000000in}}%
\pgfpathclose%
\pgfusepath{stroke,fill}%
\end{pgfscope}%
\begin{pgfscope}%
\pgfpathrectangle{\pgfqpoint{0.647939in}{0.492442in}}{\pgfqpoint{4.273799in}{2.331163in}}%
\pgfusepath{clip}%
\pgfsetroundcap%
\pgfsetroundjoin%
\pgfsetlinewidth{0.301125pt}%
\definecolor{currentstroke}{rgb}{0.500000,0.500000,0.500000}%
\pgfsetstrokecolor{currentstroke}%
\pgfsetstrokeopacity{0.300000}%
\pgfsetdash{}{0pt}%
\pgfpathmoveto{\pgfqpoint{1.585887in}{1.574040in}}%
\pgfusepath{stroke}%
\end{pgfscope}%
\begin{pgfscope}%
\pgfpathrectangle{\pgfqpoint{0.647939in}{0.492442in}}{\pgfqpoint{4.273799in}{2.331163in}}%
\pgfusepath{clip}%
\pgfsetroundcap%
\pgfsetroundjoin%
\definecolor{currentfill}{rgb}{0.500000,0.500000,0.500000}%
\pgfsetfillcolor{currentfill}%
\pgfsetfillopacity{0.300000}%
\pgfsetlinewidth{0.301125pt}%
\definecolor{currentstroke}{rgb}{0.500000,0.500000,0.500000}%
\pgfsetstrokecolor{currentstroke}%
\pgfsetstrokeopacity{0.300000}%
\pgfsetdash{}{0pt}%
\pgfpathmoveto{\pgfqpoint{0.000000in}{0.000000in}}%
\pgfpathlineto{\pgfqpoint{0.000000in}{0.000000in}}%
\pgfpathclose%
\pgfusepath{stroke,fill}%
\end{pgfscope}%
\begin{pgfscope}%
\pgfpathrectangle{\pgfqpoint{0.647939in}{0.492442in}}{\pgfqpoint{4.273799in}{2.331163in}}%
\pgfusepath{clip}%
\pgfsetroundcap%
\pgfsetroundjoin%
\pgfsetlinewidth{0.301125pt}%
\definecolor{currentstroke}{rgb}{0.500000,0.500000,0.500000}%
\pgfsetstrokecolor{currentstroke}%
\pgfsetstrokeopacity{0.300000}%
\pgfsetdash{}{0pt}%
\pgfpathmoveto{\pgfqpoint{4.007402in}{1.278397in}}%
\pgfusepath{stroke}%
\end{pgfscope}%
\begin{pgfscope}%
\pgfpathrectangle{\pgfqpoint{0.647939in}{0.492442in}}{\pgfqpoint{4.273799in}{2.331163in}}%
\pgfusepath{clip}%
\pgfsetroundcap%
\pgfsetroundjoin%
\definecolor{currentfill}{rgb}{0.500000,0.500000,0.500000}%
\pgfsetfillcolor{currentfill}%
\pgfsetfillopacity{0.300000}%
\pgfsetlinewidth{0.301125pt}%
\definecolor{currentstroke}{rgb}{0.500000,0.500000,0.500000}%
\pgfsetstrokecolor{currentstroke}%
\pgfsetstrokeopacity{0.300000}%
\pgfsetdash{}{0pt}%
\pgfpathmoveto{\pgfqpoint{0.000000in}{0.000000in}}%
\pgfpathlineto{\pgfqpoint{0.000000in}{0.000000in}}%
\pgfpathclose%
\pgfusepath{stroke,fill}%
\end{pgfscope}%
\begin{pgfscope}%
\pgfpathrectangle{\pgfqpoint{0.647939in}{0.492442in}}{\pgfqpoint{4.273799in}{2.331163in}}%
\pgfusepath{clip}%
\pgfsetroundcap%
\pgfsetroundjoin%
\pgfsetlinewidth{0.301125pt}%
\definecolor{currentstroke}{rgb}{0.500000,0.500000,0.500000}%
\pgfsetstrokecolor{currentstroke}%
\pgfsetstrokeopacity{0.300000}%
\pgfsetdash{}{0pt}%
\pgfpathmoveto{\pgfqpoint{4.121000in}{1.513608in}}%
\pgfusepath{stroke}%
\end{pgfscope}%
\begin{pgfscope}%
\pgfpathrectangle{\pgfqpoint{0.647939in}{0.492442in}}{\pgfqpoint{4.273799in}{2.331163in}}%
\pgfusepath{clip}%
\pgfsetroundcap%
\pgfsetroundjoin%
\definecolor{currentfill}{rgb}{0.500000,0.500000,0.500000}%
\pgfsetfillcolor{currentfill}%
\pgfsetfillopacity{0.300000}%
\pgfsetlinewidth{0.301125pt}%
\definecolor{currentstroke}{rgb}{0.500000,0.500000,0.500000}%
\pgfsetstrokecolor{currentstroke}%
\pgfsetstrokeopacity{0.300000}%
\pgfsetdash{}{0pt}%
\pgfpathmoveto{\pgfqpoint{0.000000in}{0.000000in}}%
\pgfpathlineto{\pgfqpoint{0.000000in}{0.000000in}}%
\pgfpathclose%
\pgfusepath{stroke,fill}%
\end{pgfscope}%
\begin{pgfscope}%
\pgfpathrectangle{\pgfqpoint{0.647939in}{0.492442in}}{\pgfqpoint{4.273799in}{2.331163in}}%
\pgfusepath{clip}%
\pgfsetroundcap%
\pgfsetroundjoin%
\pgfsetlinewidth{0.301125pt}%
\definecolor{currentstroke}{rgb}{0.500000,0.500000,0.500000}%
\pgfsetstrokecolor{currentstroke}%
\pgfsetstrokeopacity{0.300000}%
\pgfsetdash{}{0pt}%
\pgfpathmoveto{\pgfqpoint{2.273663in}{1.924906in}}%
\pgfusepath{stroke}%
\end{pgfscope}%
\begin{pgfscope}%
\pgfpathrectangle{\pgfqpoint{0.647939in}{0.492442in}}{\pgfqpoint{4.273799in}{2.331163in}}%
\pgfusepath{clip}%
\pgfsetroundcap%
\pgfsetroundjoin%
\definecolor{currentfill}{rgb}{0.500000,0.500000,0.500000}%
\pgfsetfillcolor{currentfill}%
\pgfsetfillopacity{0.300000}%
\pgfsetlinewidth{0.301125pt}%
\definecolor{currentstroke}{rgb}{0.500000,0.500000,0.500000}%
\pgfsetstrokecolor{currentstroke}%
\pgfsetstrokeopacity{0.300000}%
\pgfsetdash{}{0pt}%
\pgfpathmoveto{\pgfqpoint{0.000000in}{0.000000in}}%
\pgfpathlineto{\pgfqpoint{0.000000in}{0.000000in}}%
\pgfpathclose%
\pgfusepath{stroke,fill}%
\end{pgfscope}%
\begin{pgfscope}%
\pgfpathrectangle{\pgfqpoint{0.647939in}{0.492442in}}{\pgfqpoint{4.273799in}{2.331163in}}%
\pgfusepath{clip}%
\pgfsetroundcap%
\pgfsetroundjoin%
\pgfsetlinewidth{0.301125pt}%
\definecolor{currentstroke}{rgb}{0.500000,0.500000,0.500000}%
\pgfsetstrokecolor{currentstroke}%
\pgfsetstrokeopacity{0.300000}%
\pgfsetdash{}{0pt}%
\pgfpathmoveto{\pgfqpoint{1.543209in}{2.071889in}}%
\pgfusepath{stroke}%
\end{pgfscope}%
\begin{pgfscope}%
\pgfpathrectangle{\pgfqpoint{0.647939in}{0.492442in}}{\pgfqpoint{4.273799in}{2.331163in}}%
\pgfusepath{clip}%
\pgfsetroundcap%
\pgfsetroundjoin%
\definecolor{currentfill}{rgb}{0.500000,0.500000,0.500000}%
\pgfsetfillcolor{currentfill}%
\pgfsetfillopacity{0.300000}%
\pgfsetlinewidth{0.301125pt}%
\definecolor{currentstroke}{rgb}{0.500000,0.500000,0.500000}%
\pgfsetstrokecolor{currentstroke}%
\pgfsetstrokeopacity{0.300000}%
\pgfsetdash{}{0pt}%
\pgfpathmoveto{\pgfqpoint{0.000000in}{0.000000in}}%
\pgfpathlineto{\pgfqpoint{0.000000in}{0.000000in}}%
\pgfpathclose%
\pgfusepath{stroke,fill}%
\end{pgfscope}%
\begin{pgfscope}%
\pgfpathrectangle{\pgfqpoint{0.647939in}{0.492442in}}{\pgfqpoint{4.273799in}{2.331163in}}%
\pgfusepath{clip}%
\pgfsetroundcap%
\pgfsetroundjoin%
\pgfsetlinewidth{0.301125pt}%
\definecolor{currentstroke}{rgb}{0.500000,0.500000,0.500000}%
\pgfsetstrokecolor{currentstroke}%
\pgfsetstrokeopacity{0.300000}%
\pgfsetdash{}{0pt}%
\pgfpathmoveto{\pgfqpoint{3.865635in}{1.076368in}}%
\pgfusepath{stroke}%
\end{pgfscope}%
\begin{pgfscope}%
\pgfpathrectangle{\pgfqpoint{0.647939in}{0.492442in}}{\pgfqpoint{4.273799in}{2.331163in}}%
\pgfusepath{clip}%
\pgfsetroundcap%
\pgfsetroundjoin%
\definecolor{currentfill}{rgb}{0.500000,0.500000,0.500000}%
\pgfsetfillcolor{currentfill}%
\pgfsetfillopacity{0.300000}%
\pgfsetlinewidth{0.301125pt}%
\definecolor{currentstroke}{rgb}{0.500000,0.500000,0.500000}%
\pgfsetstrokecolor{currentstroke}%
\pgfsetstrokeopacity{0.300000}%
\pgfsetdash{}{0pt}%
\pgfpathmoveto{\pgfqpoint{0.000000in}{0.000000in}}%
\pgfpathlineto{\pgfqpoint{0.000000in}{0.000000in}}%
\pgfpathclose%
\pgfusepath{stroke,fill}%
\end{pgfscope}%
\begin{pgfscope}%
\pgfpathrectangle{\pgfqpoint{0.647939in}{0.492442in}}{\pgfqpoint{4.273799in}{2.331163in}}%
\pgfusepath{clip}%
\pgfsetroundcap%
\pgfsetroundjoin%
\pgfsetlinewidth{0.301125pt}%
\definecolor{currentstroke}{rgb}{0.500000,0.500000,0.500000}%
\pgfsetstrokecolor{currentstroke}%
\pgfsetstrokeopacity{0.300000}%
\pgfsetdash{}{0pt}%
\pgfpathmoveto{\pgfqpoint{3.791589in}{1.266728in}}%
\pgfusepath{stroke}%
\end{pgfscope}%
\begin{pgfscope}%
\pgfpathrectangle{\pgfqpoint{0.647939in}{0.492442in}}{\pgfqpoint{4.273799in}{2.331163in}}%
\pgfusepath{clip}%
\pgfsetroundcap%
\pgfsetroundjoin%
\definecolor{currentfill}{rgb}{0.500000,0.500000,0.500000}%
\pgfsetfillcolor{currentfill}%
\pgfsetfillopacity{0.300000}%
\pgfsetlinewidth{0.301125pt}%
\definecolor{currentstroke}{rgb}{0.500000,0.500000,0.500000}%
\pgfsetstrokecolor{currentstroke}%
\pgfsetstrokeopacity{0.300000}%
\pgfsetdash{}{0pt}%
\pgfpathmoveto{\pgfqpoint{0.000000in}{0.000000in}}%
\pgfpathlineto{\pgfqpoint{0.000000in}{0.000000in}}%
\pgfpathclose%
\pgfusepath{stroke,fill}%
\end{pgfscope}%
\begin{pgfscope}%
\pgfpathrectangle{\pgfqpoint{0.647939in}{0.492442in}}{\pgfqpoint{4.273799in}{2.331163in}}%
\pgfusepath{clip}%
\pgfsetroundcap%
\pgfsetroundjoin%
\pgfsetlinewidth{0.301125pt}%
\definecolor{currentstroke}{rgb}{0.500000,0.500000,0.500000}%
\pgfsetstrokecolor{currentstroke}%
\pgfsetstrokeopacity{0.300000}%
\pgfsetdash{}{0pt}%
\pgfpathmoveto{\pgfqpoint{3.740169in}{1.884679in}}%
\pgfusepath{stroke}%
\end{pgfscope}%
\begin{pgfscope}%
\pgfpathrectangle{\pgfqpoint{0.647939in}{0.492442in}}{\pgfqpoint{4.273799in}{2.331163in}}%
\pgfusepath{clip}%
\pgfsetroundcap%
\pgfsetroundjoin%
\definecolor{currentfill}{rgb}{0.500000,0.500000,0.500000}%
\pgfsetfillcolor{currentfill}%
\pgfsetfillopacity{0.300000}%
\pgfsetlinewidth{0.301125pt}%
\definecolor{currentstroke}{rgb}{0.500000,0.500000,0.500000}%
\pgfsetstrokecolor{currentstroke}%
\pgfsetstrokeopacity{0.300000}%
\pgfsetdash{}{0pt}%
\pgfpathmoveto{\pgfqpoint{0.000000in}{0.000000in}}%
\pgfpathlineto{\pgfqpoint{0.000000in}{0.000000in}}%
\pgfpathclose%
\pgfusepath{stroke,fill}%
\end{pgfscope}%
\begin{pgfscope}%
\pgfpathrectangle{\pgfqpoint{0.647939in}{0.492442in}}{\pgfqpoint{4.273799in}{2.331163in}}%
\pgfusepath{clip}%
\pgfsetroundcap%
\pgfsetroundjoin%
\pgfsetlinewidth{0.301125pt}%
\definecolor{currentstroke}{rgb}{0.500000,0.500000,0.500000}%
\pgfsetstrokecolor{currentstroke}%
\pgfsetstrokeopacity{0.300000}%
\pgfsetdash{}{0pt}%
\pgfpathmoveto{\pgfqpoint{2.502435in}{2.043672in}}%
\pgfusepath{stroke}%
\end{pgfscope}%
\begin{pgfscope}%
\pgfpathrectangle{\pgfqpoint{0.647939in}{0.492442in}}{\pgfqpoint{4.273799in}{2.331163in}}%
\pgfusepath{clip}%
\pgfsetroundcap%
\pgfsetroundjoin%
\definecolor{currentfill}{rgb}{0.500000,0.500000,0.500000}%
\pgfsetfillcolor{currentfill}%
\pgfsetfillopacity{0.300000}%
\pgfsetlinewidth{0.301125pt}%
\definecolor{currentstroke}{rgb}{0.500000,0.500000,0.500000}%
\pgfsetstrokecolor{currentstroke}%
\pgfsetstrokeopacity{0.300000}%
\pgfsetdash{}{0pt}%
\pgfpathmoveto{\pgfqpoint{0.000000in}{0.000000in}}%
\pgfpathlineto{\pgfqpoint{0.000000in}{0.000000in}}%
\pgfpathclose%
\pgfusepath{stroke,fill}%
\end{pgfscope}%
\begin{pgfscope}%
\pgfpathrectangle{\pgfqpoint{0.647939in}{0.492442in}}{\pgfqpoint{4.273799in}{2.331163in}}%
\pgfusepath{clip}%
\pgfsetroundcap%
\pgfsetroundjoin%
\pgfsetlinewidth{0.301125pt}%
\definecolor{currentstroke}{rgb}{0.500000,0.500000,0.500000}%
\pgfsetstrokecolor{currentstroke}%
\pgfsetstrokeopacity{0.300000}%
\pgfsetdash{}{0pt}%
\pgfpathmoveto{\pgfqpoint{2.652797in}{1.784921in}}%
\pgfusepath{stroke}%
\end{pgfscope}%
\begin{pgfscope}%
\pgfpathrectangle{\pgfqpoint{0.647939in}{0.492442in}}{\pgfqpoint{4.273799in}{2.331163in}}%
\pgfusepath{clip}%
\pgfsetroundcap%
\pgfsetroundjoin%
\definecolor{currentfill}{rgb}{0.500000,0.500000,0.500000}%
\pgfsetfillcolor{currentfill}%
\pgfsetfillopacity{0.300000}%
\pgfsetlinewidth{0.301125pt}%
\definecolor{currentstroke}{rgb}{0.500000,0.500000,0.500000}%
\pgfsetstrokecolor{currentstroke}%
\pgfsetstrokeopacity{0.300000}%
\pgfsetdash{}{0pt}%
\pgfpathmoveto{\pgfqpoint{0.000000in}{0.000000in}}%
\pgfpathlineto{\pgfqpoint{0.000000in}{0.000000in}}%
\pgfpathclose%
\pgfusepath{stroke,fill}%
\end{pgfscope}%
\begin{pgfscope}%
\pgfpathrectangle{\pgfqpoint{0.647939in}{0.492442in}}{\pgfqpoint{4.273799in}{2.331163in}}%
\pgfusepath{clip}%
\pgfsetroundcap%
\pgfsetroundjoin%
\pgfsetlinewidth{0.301125pt}%
\definecolor{currentstroke}{rgb}{0.500000,0.500000,0.500000}%
\pgfsetstrokecolor{currentstroke}%
\pgfsetstrokeopacity{0.300000}%
\pgfsetdash{}{0pt}%
\pgfpathmoveto{\pgfqpoint{1.866607in}{1.201233in}}%
\pgfusepath{stroke}%
\end{pgfscope}%
\begin{pgfscope}%
\pgfpathrectangle{\pgfqpoint{0.647939in}{0.492442in}}{\pgfqpoint{4.273799in}{2.331163in}}%
\pgfusepath{clip}%
\pgfsetroundcap%
\pgfsetroundjoin%
\definecolor{currentfill}{rgb}{0.500000,0.500000,0.500000}%
\pgfsetfillcolor{currentfill}%
\pgfsetfillopacity{0.300000}%
\pgfsetlinewidth{0.301125pt}%
\definecolor{currentstroke}{rgb}{0.500000,0.500000,0.500000}%
\pgfsetstrokecolor{currentstroke}%
\pgfsetstrokeopacity{0.300000}%
\pgfsetdash{}{0pt}%
\pgfpathmoveto{\pgfqpoint{0.000000in}{0.000000in}}%
\pgfpathlineto{\pgfqpoint{0.000000in}{0.000000in}}%
\pgfpathclose%
\pgfusepath{stroke,fill}%
\end{pgfscope}%
\begin{pgfscope}%
\pgfpathrectangle{\pgfqpoint{0.647939in}{0.492442in}}{\pgfqpoint{4.273799in}{2.331163in}}%
\pgfusepath{clip}%
\pgfsetroundcap%
\pgfsetroundjoin%
\pgfsetlinewidth{0.301125pt}%
\definecolor{currentstroke}{rgb}{0.500000,0.500000,0.500000}%
\pgfsetstrokecolor{currentstroke}%
\pgfsetstrokeopacity{0.300000}%
\pgfsetdash{}{0pt}%
\pgfpathmoveto{\pgfqpoint{1.850133in}{2.036542in}}%
\pgfusepath{stroke}%
\end{pgfscope}%
\begin{pgfscope}%
\pgfpathrectangle{\pgfqpoint{0.647939in}{0.492442in}}{\pgfqpoint{4.273799in}{2.331163in}}%
\pgfusepath{clip}%
\pgfsetroundcap%
\pgfsetroundjoin%
\definecolor{currentfill}{rgb}{0.500000,0.500000,0.500000}%
\pgfsetfillcolor{currentfill}%
\pgfsetfillopacity{0.300000}%
\pgfsetlinewidth{0.301125pt}%
\definecolor{currentstroke}{rgb}{0.500000,0.500000,0.500000}%
\pgfsetstrokecolor{currentstroke}%
\pgfsetstrokeopacity{0.300000}%
\pgfsetdash{}{0pt}%
\pgfpathmoveto{\pgfqpoint{0.000000in}{0.000000in}}%
\pgfpathlineto{\pgfqpoint{0.000000in}{0.000000in}}%
\pgfpathclose%
\pgfusepath{stroke,fill}%
\end{pgfscope}%
\begin{pgfscope}%
\pgfpathrectangle{\pgfqpoint{0.647939in}{0.492442in}}{\pgfqpoint{4.273799in}{2.331163in}}%
\pgfusepath{clip}%
\pgfsetroundcap%
\pgfsetroundjoin%
\pgfsetlinewidth{0.301125pt}%
\definecolor{currentstroke}{rgb}{0.500000,0.500000,0.500000}%
\pgfsetstrokecolor{currentstroke}%
\pgfsetstrokeopacity{0.300000}%
\pgfsetdash{}{0pt}%
\pgfpathmoveto{\pgfqpoint{1.923436in}{1.822115in}}%
\pgfusepath{stroke}%
\end{pgfscope}%
\begin{pgfscope}%
\pgfpathrectangle{\pgfqpoint{0.647939in}{0.492442in}}{\pgfqpoint{4.273799in}{2.331163in}}%
\pgfusepath{clip}%
\pgfsetroundcap%
\pgfsetroundjoin%
\definecolor{currentfill}{rgb}{0.500000,0.500000,0.500000}%
\pgfsetfillcolor{currentfill}%
\pgfsetfillopacity{0.300000}%
\pgfsetlinewidth{0.301125pt}%
\definecolor{currentstroke}{rgb}{0.500000,0.500000,0.500000}%
\pgfsetstrokecolor{currentstroke}%
\pgfsetstrokeopacity{0.300000}%
\pgfsetdash{}{0pt}%
\pgfpathmoveto{\pgfqpoint{0.000000in}{0.000000in}}%
\pgfpathlineto{\pgfqpoint{0.000000in}{0.000000in}}%
\pgfpathclose%
\pgfusepath{stroke,fill}%
\end{pgfscope}%
\begin{pgfscope}%
\pgfpathrectangle{\pgfqpoint{0.647939in}{0.492442in}}{\pgfqpoint{4.273799in}{2.331163in}}%
\pgfusepath{clip}%
\pgfsetroundcap%
\pgfsetroundjoin%
\pgfsetlinewidth{0.301125pt}%
\definecolor{currentstroke}{rgb}{0.500000,0.500000,0.500000}%
\pgfsetstrokecolor{currentstroke}%
\pgfsetstrokeopacity{0.300000}%
\pgfsetdash{}{0pt}%
\pgfpathmoveto{\pgfqpoint{2.192425in}{1.926715in}}%
\pgfusepath{stroke}%
\end{pgfscope}%
\begin{pgfscope}%
\pgfpathrectangle{\pgfqpoint{0.647939in}{0.492442in}}{\pgfqpoint{4.273799in}{2.331163in}}%
\pgfusepath{clip}%
\pgfsetroundcap%
\pgfsetroundjoin%
\definecolor{currentfill}{rgb}{0.500000,0.500000,0.500000}%
\pgfsetfillcolor{currentfill}%
\pgfsetfillopacity{0.300000}%
\pgfsetlinewidth{0.301125pt}%
\definecolor{currentstroke}{rgb}{0.500000,0.500000,0.500000}%
\pgfsetstrokecolor{currentstroke}%
\pgfsetstrokeopacity{0.300000}%
\pgfsetdash{}{0pt}%
\pgfpathmoveto{\pgfqpoint{0.000000in}{0.000000in}}%
\pgfpathlineto{\pgfqpoint{0.000000in}{0.000000in}}%
\pgfpathclose%
\pgfusepath{stroke,fill}%
\end{pgfscope}%
\begin{pgfscope}%
\pgfpathrectangle{\pgfqpoint{0.647939in}{0.492442in}}{\pgfqpoint{4.273799in}{2.331163in}}%
\pgfusepath{clip}%
\pgfsetroundcap%
\pgfsetroundjoin%
\pgfsetlinewidth{0.301125pt}%
\definecolor{currentstroke}{rgb}{0.500000,0.500000,0.500000}%
\pgfsetstrokecolor{currentstroke}%
\pgfsetstrokeopacity{0.300000}%
\pgfsetdash{}{0pt}%
\pgfpathmoveto{\pgfqpoint{2.065171in}{1.373391in}}%
\pgfusepath{stroke}%
\end{pgfscope}%
\begin{pgfscope}%
\pgfpathrectangle{\pgfqpoint{0.647939in}{0.492442in}}{\pgfqpoint{4.273799in}{2.331163in}}%
\pgfusepath{clip}%
\pgfsetroundcap%
\pgfsetroundjoin%
\definecolor{currentfill}{rgb}{0.500000,0.500000,0.500000}%
\pgfsetfillcolor{currentfill}%
\pgfsetfillopacity{0.300000}%
\pgfsetlinewidth{0.301125pt}%
\definecolor{currentstroke}{rgb}{0.500000,0.500000,0.500000}%
\pgfsetstrokecolor{currentstroke}%
\pgfsetstrokeopacity{0.300000}%
\pgfsetdash{}{0pt}%
\pgfpathmoveto{\pgfqpoint{0.000000in}{0.000000in}}%
\pgfpathlineto{\pgfqpoint{0.000000in}{0.000000in}}%
\pgfpathclose%
\pgfusepath{stroke,fill}%
\end{pgfscope}%
\begin{pgfscope}%
\pgfpathrectangle{\pgfqpoint{0.647939in}{0.492442in}}{\pgfqpoint{4.273799in}{2.331163in}}%
\pgfusepath{clip}%
\pgfsetroundcap%
\pgfsetroundjoin%
\pgfsetlinewidth{0.301125pt}%
\definecolor{currentstroke}{rgb}{0.500000,0.500000,0.500000}%
\pgfsetstrokecolor{currentstroke}%
\pgfsetstrokeopacity{0.300000}%
\pgfsetdash{}{0pt}%
\pgfpathmoveto{\pgfqpoint{3.460262in}{1.924273in}}%
\pgfusepath{stroke}%
\end{pgfscope}%
\begin{pgfscope}%
\pgfpathrectangle{\pgfqpoint{0.647939in}{0.492442in}}{\pgfqpoint{4.273799in}{2.331163in}}%
\pgfusepath{clip}%
\pgfsetroundcap%
\pgfsetroundjoin%
\definecolor{currentfill}{rgb}{0.500000,0.500000,0.500000}%
\pgfsetfillcolor{currentfill}%
\pgfsetfillopacity{0.300000}%
\pgfsetlinewidth{0.301125pt}%
\definecolor{currentstroke}{rgb}{0.500000,0.500000,0.500000}%
\pgfsetstrokecolor{currentstroke}%
\pgfsetstrokeopacity{0.300000}%
\pgfsetdash{}{0pt}%
\pgfpathmoveto{\pgfqpoint{0.000000in}{0.000000in}}%
\pgfpathlineto{\pgfqpoint{0.000000in}{0.000000in}}%
\pgfpathclose%
\pgfusepath{stroke,fill}%
\end{pgfscope}%
\begin{pgfscope}%
\pgfpathrectangle{\pgfqpoint{0.647939in}{0.492442in}}{\pgfqpoint{4.273799in}{2.331163in}}%
\pgfusepath{clip}%
\pgfsetroundcap%
\pgfsetroundjoin%
\pgfsetlinewidth{0.301125pt}%
\definecolor{currentstroke}{rgb}{0.500000,0.500000,0.500000}%
\pgfsetstrokecolor{currentstroke}%
\pgfsetstrokeopacity{0.300000}%
\pgfsetdash{}{0pt}%
\pgfpathmoveto{\pgfqpoint{3.069466in}{1.747185in}}%
\pgfusepath{stroke}%
\end{pgfscope}%
\begin{pgfscope}%
\pgfpathrectangle{\pgfqpoint{0.647939in}{0.492442in}}{\pgfqpoint{4.273799in}{2.331163in}}%
\pgfusepath{clip}%
\pgfsetroundcap%
\pgfsetroundjoin%
\definecolor{currentfill}{rgb}{0.500000,0.500000,0.500000}%
\pgfsetfillcolor{currentfill}%
\pgfsetfillopacity{0.300000}%
\pgfsetlinewidth{0.301125pt}%
\definecolor{currentstroke}{rgb}{0.500000,0.500000,0.500000}%
\pgfsetstrokecolor{currentstroke}%
\pgfsetstrokeopacity{0.300000}%
\pgfsetdash{}{0pt}%
\pgfpathmoveto{\pgfqpoint{0.000000in}{0.000000in}}%
\pgfpathlineto{\pgfqpoint{0.000000in}{0.000000in}}%
\pgfpathclose%
\pgfusepath{stroke,fill}%
\end{pgfscope}%
\begin{pgfscope}%
\pgfpathrectangle{\pgfqpoint{0.647939in}{0.492442in}}{\pgfqpoint{4.273799in}{2.331163in}}%
\pgfusepath{clip}%
\pgfsetrectcap%
\pgfsetroundjoin%
\pgfsetlinewidth{2.007500pt}%
\definecolor{currentstroke}{rgb}{0.000000,0.000000,1.000000}%
\pgfsetstrokecolor{currentstroke}%
\pgfsetdash{}{0pt}%
\pgfpathmoveto{\pgfqpoint{3.756157in}{1.502613in}}%
\pgfpathlineto{\pgfqpoint{4.533211in}{1.502613in}}%
\pgfpathlineto{\pgfqpoint{4.533211in}{2.279667in}}%
\pgfpathlineto{\pgfqpoint{3.756157in}{2.279667in}}%
\pgfpathlineto{\pgfqpoint{3.756157in}{1.502613in}}%
\pgfusepath{stroke}%
\end{pgfscope}%
\begin{pgfscope}%
\pgfpathrectangle{\pgfqpoint{0.647939in}{0.492442in}}{\pgfqpoint{4.273799in}{2.331163in}}%
\pgfusepath{clip}%
\pgfsetrectcap%
\pgfsetroundjoin%
\pgfsetlinewidth{2.007500pt}%
\definecolor{currentstroke}{rgb}{0.000000,0.000000,1.000000}%
\pgfsetstrokecolor{currentstroke}%
\pgfsetdash{}{0pt}%
\pgfpathmoveto{\pgfqpoint{1.580404in}{1.968846in}}%
\pgfpathlineto{\pgfqpoint{1.891226in}{1.968846in}}%
\pgfpathlineto{\pgfqpoint{1.891226in}{2.124256in}}%
\pgfpathlineto{\pgfqpoint{1.580404in}{2.124256in}}%
\pgfpathlineto{\pgfqpoint{1.580404in}{1.968846in}}%
\pgfusepath{stroke}%
\end{pgfscope}%
\begin{pgfscope}%
\pgfpathrectangle{\pgfqpoint{0.647939in}{0.492442in}}{\pgfqpoint{4.273799in}{2.331163in}}%
\pgfusepath{clip}%
\pgfsetrectcap%
\pgfsetroundjoin%
\pgfsetlinewidth{2.007500pt}%
\definecolor{currentstroke}{rgb}{0.000000,0.000000,1.000000}%
\pgfsetstrokecolor{currentstroke}%
\pgfsetdash{}{0pt}%
\pgfpathmoveto{\pgfqpoint{1.891226in}{1.658024in}}%
\pgfpathlineto{\pgfqpoint{2.046637in}{1.658024in}}%
\pgfpathlineto{\pgfqpoint{2.046637in}{2.124256in}}%
\pgfpathlineto{\pgfqpoint{1.891226in}{2.124256in}}%
\pgfpathlineto{\pgfqpoint{1.891226in}{1.658024in}}%
\pgfusepath{stroke}%
\end{pgfscope}%
\begin{pgfscope}%
\pgfpathrectangle{\pgfqpoint{0.647939in}{0.492442in}}{\pgfqpoint{4.273799in}{2.331163in}}%
\pgfusepath{clip}%
\pgfsetrectcap%
\pgfsetroundjoin%
\pgfsetlinewidth{2.007500pt}%
\definecolor{currentstroke}{rgb}{1.000000,0.000000,0.000000}%
\pgfsetstrokecolor{currentstroke}%
\pgfsetdash{}{0pt}%
\pgfpathmoveto{\pgfqpoint{3.289924in}{2.201962in}}%
\pgfpathlineto{\pgfqpoint{3.445335in}{2.201962in}}%
\pgfpathlineto{\pgfqpoint{3.445335in}{2.512784in}}%
\pgfpathlineto{\pgfqpoint{3.289924in}{2.512784in}}%
\pgfpathlineto{\pgfqpoint{3.289924in}{2.201962in}}%
\pgfusepath{stroke}%
\end{pgfscope}%
\begin{pgfscope}%
\pgfpathrectangle{\pgfqpoint{0.647939in}{0.492442in}}{\pgfqpoint{4.273799in}{2.331163in}}%
\pgfusepath{clip}%
\pgfsetrectcap%
\pgfsetroundjoin%
\pgfsetlinewidth{2.007500pt}%
\definecolor{currentstroke}{rgb}{1.000000,0.000000,0.000000}%
\pgfsetstrokecolor{currentstroke}%
\pgfsetdash{}{0pt}%
\pgfpathmoveto{\pgfqpoint{3.445335in}{2.201962in}}%
\pgfpathlineto{\pgfqpoint{3.523040in}{2.201962in}}%
\pgfpathlineto{\pgfqpoint{3.523040in}{2.357373in}}%
\pgfpathlineto{\pgfqpoint{3.445335in}{2.357373in}}%
\pgfpathlineto{\pgfqpoint{3.445335in}{2.201962in}}%
\pgfusepath{stroke}%
\end{pgfscope}%
\begin{pgfscope}%
\pgfsetrectcap%
\pgfsetmiterjoin%
\pgfsetlinewidth{0.803000pt}%
\definecolor{currentstroke}{rgb}{0.000000,0.000000,0.000000}%
\pgfsetstrokecolor{currentstroke}%
\pgfsetdash{}{0pt}%
\pgfpathmoveto{\pgfqpoint{0.647939in}{0.492442in}}%
\pgfpathlineto{\pgfqpoint{0.647939in}{2.823605in}}%
\pgfusepath{stroke}%
\end{pgfscope}%
\begin{pgfscope}%
\pgfsetrectcap%
\pgfsetmiterjoin%
\pgfsetlinewidth{0.803000pt}%
\definecolor{currentstroke}{rgb}{0.000000,0.000000,0.000000}%
\pgfsetstrokecolor{currentstroke}%
\pgfsetdash{}{0pt}%
\pgfpathmoveto{\pgfqpoint{4.921738in}{0.492442in}}%
\pgfpathlineto{\pgfqpoint{4.921738in}{2.823605in}}%
\pgfusepath{stroke}%
\end{pgfscope}%
\begin{pgfscope}%
\pgfsetrectcap%
\pgfsetmiterjoin%
\pgfsetlinewidth{0.803000pt}%
\definecolor{currentstroke}{rgb}{0.000000,0.000000,0.000000}%
\pgfsetstrokecolor{currentstroke}%
\pgfsetdash{}{0pt}%
\pgfpathmoveto{\pgfqpoint{0.647939in}{0.492442in}}%
\pgfpathlineto{\pgfqpoint{4.921738in}{0.492442in}}%
\pgfusepath{stroke}%
\end{pgfscope}%
\begin{pgfscope}%
\pgfsetrectcap%
\pgfsetmiterjoin%
\pgfsetlinewidth{0.803000pt}%
\definecolor{currentstroke}{rgb}{0.000000,0.000000,0.000000}%
\pgfsetstrokecolor{currentstroke}%
\pgfsetdash{}{0pt}%
\pgfpathmoveto{\pgfqpoint{0.647939in}{2.823605in}}%
\pgfpathlineto{\pgfqpoint{4.921738in}{2.823605in}}%
\pgfusepath{stroke}%
\end{pgfscope}%
\begin{pgfscope}%
\pgfsetbuttcap%
\pgfsetmiterjoin%
\definecolor{currentfill}{rgb}{1.000000,1.000000,1.000000}%
\pgfsetfillcolor{currentfill}%
\pgfsetfillopacity{0.800000}%
\pgfsetlinewidth{1.003750pt}%
\definecolor{currentstroke}{rgb}{0.800000,0.800000,0.800000}%
\pgfsetstrokecolor{currentstroke}%
\pgfsetstrokeopacity{0.800000}%
\pgfsetdash{}{0pt}%
\pgfpathmoveto{\pgfqpoint{0.735439in}{2.370139in}}%
\pgfpathlineto{\pgfqpoint{1.327333in}{2.370139in}}%
\pgfpathquadraticcurveto{\pgfqpoint{1.352333in}{2.370139in}}{\pgfqpoint{1.352333in}{2.395139in}}%
\pgfpathlineto{\pgfqpoint{1.352333in}{2.736105in}}%
\pgfpathquadraticcurveto{\pgfqpoint{1.352333in}{2.761105in}}{\pgfqpoint{1.327333in}{2.761105in}}%
\pgfpathlineto{\pgfqpoint{0.735439in}{2.761105in}}%
\pgfpathquadraticcurveto{\pgfqpoint{0.710439in}{2.761105in}}{\pgfqpoint{0.710439in}{2.736105in}}%
\pgfpathlineto{\pgfqpoint{0.710439in}{2.395139in}}%
\pgfpathquadraticcurveto{\pgfqpoint{0.710439in}{2.370139in}}{\pgfqpoint{0.735439in}{2.370139in}}%
\pgfpathlineto{\pgfqpoint{0.735439in}{2.370139in}}%
\pgfpathclose%
\pgfusepath{stroke,fill}%
\end{pgfscope}%
\begin{pgfscope}%
\pgfsetrectcap%
\pgfsetroundjoin%
\pgfsetlinewidth{2.007500pt}%
\definecolor{currentstroke}{rgb}{0.000000,0.000000,1.000000}%
\pgfsetstrokecolor{currentstroke}%
\pgfsetdash{}{0pt}%
\pgfpathmoveto{\pgfqpoint{0.760439in}{2.667355in}}%
\pgfpathlineto{\pgfqpoint{0.885439in}{2.667355in}}%
\pgfpathlineto{\pgfqpoint{1.010439in}{2.667355in}}%
\pgfusepath{stroke}%
\end{pgfscope}%
\begin{pgfscope}%
\definecolor{textcolor}{rgb}{0.000000,0.000000,0.000000}%
\pgfsetstrokecolor{textcolor}%
\pgfsetfillcolor{textcolor}%
\pgftext[x=1.110439in,y=2.623605in,left,base]{\color{textcolor}{\ifdefined\pdftexversion\else\setmainfont{Times New Roman}\rmfamily\fi\fontsize{9.000000}{10.800000}\selectfont\catcode`\^=\active\def^{\ifmmode\sp\else\^{}\fi}\catcode`\%=\active\def%{\%}$\X_0$}}%
\end{pgfscope}%
\begin{pgfscope}%
\pgfsetrectcap%
\pgfsetroundjoin%
\pgfsetlinewidth{2.007500pt}%
\definecolor{currentstroke}{rgb}{1.000000,0.000000,0.000000}%
\pgfsetstrokecolor{currentstroke}%
\pgfsetdash{}{0pt}%
\pgfpathmoveto{\pgfqpoint{0.760439in}{2.490622in}}%
\pgfpathlineto{\pgfqpoint{0.885439in}{2.490622in}}%
\pgfpathlineto{\pgfqpoint{1.010439in}{2.490622in}}%
\pgfusepath{stroke}%
\end{pgfscope}%
\begin{pgfscope}%
\definecolor{textcolor}{rgb}{0.000000,0.000000,0.000000}%
\pgfsetstrokecolor{textcolor}%
\pgfsetfillcolor{textcolor}%
\pgftext[x=1.110439in,y=2.446872in,left,base]{\color{textcolor}{\ifdefined\pdftexversion\else\setmainfont{Times New Roman}\rmfamily\fi\fontsize{9.000000}{10.800000}\selectfont\catcode`\^=\active\def^{\ifmmode\sp\else\^{}\fi}\catcode`\%=\active\def%{\%}$\X_U$}}%
\end{pgfscope}%
\end{pgfpicture}%
\makeatother%
\endgroup%

      \caption{Visualization of the stochastic system behavior (based on 10 random transitions from an $100\times 100$ grid of initial states) and safety specification for the \barrIII benchmark.}
      \label{fig:model-barrier3}
\end{figure}

\begin{table}[tb]
      \centering
      \begin{tabular}{cccccccc}
            \toprule
            \textbf{Freq.} & \textbf{Lattice} & $\eta$ & $\gamma$ & $c$  & \textbf{Runtime} & \textbf{Runtime Mac} & \textbf{Safety} \\
                           & \textbf{Size}    &        &          &      & [mm:ss]          & [mm:ss]              & \textbf{Prob.}  \\ % Units
            \midrule
            15             & $800^2$          & 0.55   & 2        & 0.11 & 66:56            &                      & 42.25\%         \\
            16             & $700^2$          & 0.53   & 2        & 0.12 & 52:51            &                      & 41.98\%         \\
            14             & $800^2$          & 0.58   & 2        & 0.11 & 45:44            &                      & 41.66\%         \\
            13             & $800^2$          & 0.57   & 2        & 0.12 & 35:51            &                      & 40.72\%         \\
            14             & $700^2$          & 0.59   & 2        & 0.11 & 32:42            &                      & 40.57\%         \\
            13             & $700^2$          & 0.60   & 2        & 0.12 & 25:40            &                      & 39.69\%         \\
            15             & $600^2$          & 0.59   & 2        & 0.12 & 25:12            &                      & 39.57\%         \\
            12             & $800^2$          & 0.61   & 2        & 0.11 & 27:38            &                      & 39.40\%         \\
            14             & $600^2$          & 0.60   & 2        & 0.12 & 20:29            &                      & 39.24\%         \\
            13             & $600^2$          & 0.61   & 2        & 0.12 & 16:39            &                      & 38.50\%         \\
            12             & $700^2$          & 0.63   & 2        & 0.12 & 19:10            &                      & 38.43\%         \\
            12             & $600^2$          & 0.64   & 2        & 0.12 & 11:53            &                      & 37.38\%         \\
            15             & $450^2$          & 0.63   & 2        & 0.12 & 12:07            &                      & 36.15\%         \\
            14             & $450^2$          & 0.63   & 2        & 0.12 & 9:58             &                      & 36.07\%         \\
            14             & $400^2$          & 0.65   & 2        & 0.13 & 7:50             &                      & 34.77\%         \\
            14             & $370^2$          & 0.66   & 2        & 0.13 & 6:49             & 149:48               & 33.58\%         \\
            16             & $370^2$          & 0.64   & 2        & 0.13 & 11:35            &                      & 33.55\%         \\
            12             & $370^2$          & 0.71   & 2        & 0.12 & 4:38             &                      & 32.57\%         \\
            10             & $600^2$          & 0.73   & 2        & 0.12 & 46:40            &                      & 32.46\%         \\
            9              & $800^2$          & 0.74   & 2        & 0.12 & 53:07            &                      & 31.45\%         \\
            10             & $500^2$          & 0.76   & 2        & 0.12 & 45:03            &                      & 31.07\%         \\
            9              & $700^2$          & 0.75   & 2        & 0.12 & 56:43            &                      & 30.78\%         \\
            11             & $350^2$          & 0.75   & 2        & 0.12 & 3:14             &                      & 30.07\%         \\
            9              & $600^2$          & 0.77   & 2        & 0.12 & 33:57            &                      & 29.91\%         \\
            10             & $400^2$          & 0.79   & 2        & 0.12 & 34:09            &                      & 29.37\%         \\
            9              & $500^2$          & 0.79   & 2        & 0.12 & 33:40            &                      & 28.74\%         \\
            10             & $350^2$          & 0.81   & 2        & 0.12 & 2:34             &                      & 27.92\%         \\
            9              & $400^2$          & 0.80   & 2        & 0.12 & 13:07            &                      & 27.29\%         \\
            10             & $300^2$          & 0.84   & 2        & 0.12 & 8:18             &                      & 26.26\%         \\
            9              & $350^2$          & 0.83   & 2        & 0.12 & 2:02             &                      & 25.96\%         \\
            9              & $300^2$          & 0.85   & 2        & 0.13 & 6:35             & 1:21                 & 24.54\%         \\
            \bottomrule
      \end{tabular}
      \caption{\barrIII results, sorted by the last column. For each combination of number of frequencies $M$ and lattice size (i.e., number of lattice points per dimension), we report the values of $c$, $\gamma$, $\lambda$, the runtime, and the achieved lower bound on the safety probability.}
      \label{tab:results-barrier3}
\end{table}

\subsection{Overtaking}

We consider a scenario where an autonomous vehicle controlled by a \gls{nn} is overtaking another vehicle.
The dynamics of the ego vehicle are given by Dubin's car model with additive noise $w = \smash{\begin{bmatrix}w_t^1 & w_t^2 & w_t^3\end{bmatrix}\T}$ where each component $w_t^i$ is drawn from a zero-mean Gaussian with standard deviation $0.01$, $0.01$, and $0.001$, respectively.
The steering wheel angle is supplied by the \gls{nn} controller and we travel at a fixed velocity.
Given $N=1000$ sample transitions and the sets
\begin{align*}
       & \X = [ 1, 90 ] \times [ -7, 19 ] \times [ -\pi, \pi ],
       &                                                        & \X_0 = [ 1, 3 ] \times [ -1, 1 ] \times [ -0.5, 0.5 ],
       &                                                        & \X_U = [ 1, 90 ] \times [ -7, -6 ] \times [ -\pi, \pi ]                                                                       \\
       &                                                        &                                                         &  &  & \qquad \cup [ 1, 90 ] \times [ 18, 19 ] \times [ -\pi, \pi ]  \\
       &                                                        &                                                         &  &  & \qquad \cup [ 40, 45 ] \times [ -6, 6 ] \times [ -\pi, \pi ],
\end{align*}
we want to ensure that the system, starting in $\X_0$, does not enter the unsafe regions $\X_U$ within $T=5$ time steps.
The sampled transitions and safety specification are visualized in Figure~\ref{fig:model-overtaking}.
The kernel \hp are set to $\sigma_f=7$, $\sigma_l=[10, 7, 5]$, and $\lambda=10^{-5}$.
The complete configuration for the \overtaking benchmark is shown in Listing~\ref{lst:overtaking}.
A list of experiments using different combinations of frequencies and lattice sizes, showing their impact on performance and final result, is presented in Table~\ref{tab:results-overtaking}.
\lstinputlisting[language=yaml,style={bgnonumbers},caption={Configuration for \overtaking. The transition samples were generated by a separate script and appended to the configuration, abbreviated here for conciseness.},captionpos=b,label={lst:overtaking}]{code/overtaking.yaml}

\begin{figure}[ht]
      \centering
      %% Creator: Matplotlib, PGF backend
%%
%% To include the figure in your LaTeX document, write
%%   \input{<filename>.pgf}
%%
%% Make sure the required packages are loaded in your preamble
%%   \usepackage{pgf}
%%
%% Also ensure that all the required font packages are loaded; for instance,
%% the lmodern package is sometimes necessary when using math font.
%%   \usepackage{lmodern}
%%
%% Figures using additional raster images can only be included by \input if
%% they are in the same directory as the main LaTeX file. For loading figures
%% from other directories you can use the `import` package
%%   \usepackage{import}
%%
%% and then include the figures with
%%   \import{<path to file>}{<filename>.pgf}
%%
%% Matplotlib used the following preamble
%%   \def\mathdefault#1{#1}
%%   \everymath=\expandafter{\the\everymath\displaystyle}
%%   \IfFileExists{scrextend.sty}{
%%     \usepackage[fontsize=10.000000pt]{scrextend}
%%   }{
%%     \renewcommand{\normalsize}{\fontsize{10.000000}{12.000000}\selectfont}
%%     \normalsize
%%   }
%%   
%%   \ifdefined\pdftexversion\else  % non-pdftex case.
%%     \usepackage{fontspec}
%%     \setmainfont{DejaVuSerif.ttf}[Path=\detokenize{/home/campus.ncl.ac.uk/c3054737/miniconda3/envs/pylucid/lib/python3.11/site-packages/matplotlib/mpl-data/fonts/ttf/}]
%%     \setsansfont{DejaVuSans.ttf}[Path=\detokenize{/home/campus.ncl.ac.uk/c3054737/miniconda3/envs/pylucid/lib/python3.11/site-packages/matplotlib/mpl-data/fonts/ttf/}]
%%     \setmonofont{DejaVuSansMono.ttf}[Path=\detokenize{/home/campus.ncl.ac.uk/c3054737/miniconda3/envs/pylucid/lib/python3.11/site-packages/matplotlib/mpl-data/fonts/ttf/}]
%%   \fi
%%   \makeatletter\@ifpackageloaded{underscore}{}{\usepackage[strings]{underscore}}\makeatother
%%
\begingroup%
\makeatletter%
\begin{pgfpicture}%
\pgfpathrectangle{\pgfpointorigin}{\pgfqpoint{5.080264in}{4.789792in}}%
\pgfusepath{use as bounding box}%
\begin{pgfscope}%
\pgfsetbuttcap%
\pgfsetmiterjoin%
\definecolor{currentfill}{rgb}{1.000000,1.000000,1.000000}%
\pgfsetfillcolor{currentfill}%
\pgfsetlinewidth{0.000000pt}%
\definecolor{currentstroke}{rgb}{1.000000,1.000000,1.000000}%
\pgfsetstrokecolor{currentstroke}%
\pgfsetdash{}{0pt}%
\pgfpathmoveto{\pgfqpoint{0.000000in}{0.000000in}}%
\pgfpathlineto{\pgfqpoint{5.080264in}{0.000000in}}%
\pgfpathlineto{\pgfqpoint{5.080264in}{4.789792in}}%
\pgfpathlineto{\pgfqpoint{0.000000in}{4.789792in}}%
\pgfpathlineto{\pgfqpoint{0.000000in}{0.000000in}}%
\pgfpathclose%
\pgfusepath{fill}%
\end{pgfscope}%
\begin{pgfscope}%
\pgfsetbuttcap%
\pgfsetmiterjoin%
\definecolor{currentfill}{rgb}{1.000000,1.000000,1.000000}%
\pgfsetfillcolor{currentfill}%
\pgfsetlinewidth{0.000000pt}%
\definecolor{currentstroke}{rgb}{0.000000,0.000000,0.000000}%
\pgfsetstrokecolor{currentstroke}%
\pgfsetstrokeopacity{0.000000}%
\pgfsetdash{}{0pt}%
\pgfpathmoveto{\pgfqpoint{0.100000in}{0.183744in}}%
\pgfpathlineto{\pgfqpoint{4.606048in}{0.183744in}}%
\pgfpathlineto{\pgfqpoint{4.606048in}{4.689792in}}%
\pgfpathlineto{\pgfqpoint{0.100000in}{4.689792in}}%
\pgfpathlineto{\pgfqpoint{0.100000in}{0.183744in}}%
\pgfpathclose%
\pgfusepath{fill}%
\end{pgfscope}%
\begin{pgfscope}%
\pgfsetbuttcap%
\pgfsetmiterjoin%
\definecolor{currentfill}{rgb}{0.950000,0.950000,0.950000}%
\pgfsetfillcolor{currentfill}%
\pgfsetfillopacity{0.500000}%
\pgfsetlinewidth{1.003750pt}%
\definecolor{currentstroke}{rgb}{0.950000,0.950000,0.950000}%
\pgfsetstrokecolor{currentstroke}%
\pgfsetstrokeopacity{0.500000}%
\pgfsetdash{}{0pt}%
\pgfpathmoveto{\pgfqpoint{3.027307in}{3.016389in}}%
\pgfpathlineto{\pgfqpoint{0.568857in}{2.194556in}}%
\pgfpathlineto{\pgfqpoint{0.453113in}{3.757789in}}%
\pgfpathlineto{\pgfqpoint{3.064378in}{4.572108in}}%
\pgfusepath{stroke,fill}%
\end{pgfscope}%
\begin{pgfscope}%
\pgfsetbuttcap%
\pgfsetmiterjoin%
\definecolor{currentfill}{rgb}{0.900000,0.900000,0.900000}%
\pgfsetfillcolor{currentfill}%
\pgfsetfillopacity{0.500000}%
\pgfsetlinewidth{1.003750pt}%
\definecolor{currentstroke}{rgb}{0.900000,0.900000,0.900000}%
\pgfsetstrokecolor{currentstroke}%
\pgfsetstrokeopacity{0.500000}%
\pgfsetdash{}{0pt}%
\pgfpathmoveto{\pgfqpoint{4.330912in}{1.265686in}}%
\pgfpathlineto{\pgfqpoint{3.027307in}{3.016389in}}%
\pgfpathlineto{\pgfqpoint{3.064378in}{4.572108in}}%
\pgfpathlineto{\pgfqpoint{4.456164in}{2.833159in}}%
\pgfusepath{stroke,fill}%
\end{pgfscope}%
\begin{pgfscope}%
\pgfsetbuttcap%
\pgfsetmiterjoin%
\definecolor{currentfill}{rgb}{0.925000,0.925000,0.925000}%
\pgfsetfillcolor{currentfill}%
\pgfsetfillopacity{0.500000}%
\pgfsetlinewidth{1.003750pt}%
\definecolor{currentstroke}{rgb}{0.925000,0.925000,0.925000}%
\pgfsetstrokecolor{currentstroke}%
\pgfsetstrokeopacity{0.500000}%
\pgfsetdash{}{0pt}%
\pgfpathmoveto{\pgfqpoint{4.330912in}{1.265686in}}%
\pgfpathlineto{\pgfqpoint{3.027307in}{3.016389in}}%
\pgfpathlineto{\pgfqpoint{0.568857in}{2.194556in}}%
\pgfpathlineto{\pgfqpoint{1.728395in}{0.311399in}}%
\pgfusepath{stroke,fill}%
\end{pgfscope}%
\begin{pgfscope}%
\pgfsetbuttcap%
\pgfsetroundjoin%
\pgfsetlinewidth{0.803000pt}%
\definecolor{currentstroke}{rgb}{0.690196,0.690196,0.690196}%
\pgfsetstrokecolor{currentstroke}%
\pgfsetdash{}{0pt}%
\pgfpathmoveto{\pgfqpoint{1.602873in}{0.515254in}}%
\pgfpathlineto{\pgfqpoint{4.190137in}{1.454742in}}%
\pgfpathlineto{\pgfqpoint{4.305247in}{3.021720in}}%
\pgfusepath{stroke}%
\end{pgfscope}%
\begin{pgfscope}%
\pgfsetbuttcap%
\pgfsetroundjoin%
\pgfsetlinewidth{0.803000pt}%
\definecolor{currentstroke}{rgb}{0.690196,0.690196,0.690196}%
\pgfsetstrokecolor{currentstroke}%
\pgfsetdash{}{0pt}%
\pgfpathmoveto{\pgfqpoint{1.465719in}{0.738000in}}%
\pgfpathlineto{\pgfqpoint{4.036221in}{1.661446in}}%
\pgfpathlineto{\pgfqpoint{4.140415in}{3.227667in}}%
\pgfusepath{stroke}%
\end{pgfscope}%
\begin{pgfscope}%
\pgfsetbuttcap%
\pgfsetroundjoin%
\pgfsetlinewidth{0.803000pt}%
\definecolor{currentstroke}{rgb}{0.690196,0.690196,0.690196}%
\pgfsetstrokecolor{currentstroke}%
\pgfsetdash{}{0pt}%
\pgfpathmoveto{\pgfqpoint{1.330940in}{0.956889in}}%
\pgfpathlineto{\pgfqpoint{3.884874in}{1.864701in}}%
\pgfpathlineto{\pgfqpoint{3.978509in}{3.429958in}}%
\pgfusepath{stroke}%
\end{pgfscope}%
\begin{pgfscope}%
\pgfsetbuttcap%
\pgfsetroundjoin%
\pgfsetlinewidth{0.803000pt}%
\definecolor{currentstroke}{rgb}{0.690196,0.690196,0.690196}%
\pgfsetstrokecolor{currentstroke}%
\pgfsetdash{}{0pt}%
\pgfpathmoveto{\pgfqpoint{1.198475in}{1.172020in}}%
\pgfpathlineto{\pgfqpoint{3.736032in}{2.064591in}}%
\pgfpathlineto{\pgfqpoint{3.819452in}{3.628690in}}%
\pgfusepath{stroke}%
\end{pgfscope}%
\begin{pgfscope}%
\pgfsetbuttcap%
\pgfsetroundjoin%
\pgfsetlinewidth{0.803000pt}%
\definecolor{currentstroke}{rgb}{0.690196,0.690196,0.690196}%
\pgfsetstrokecolor{currentstroke}%
\pgfsetdash{}{0pt}%
\pgfpathmoveto{\pgfqpoint{1.068264in}{1.383489in}}%
\pgfpathlineto{\pgfqpoint{3.589634in}{2.261200in}}%
\pgfpathlineto{\pgfqpoint{3.663169in}{3.823956in}}%
\pgfusepath{stroke}%
\end{pgfscope}%
\begin{pgfscope}%
\pgfsetbuttcap%
\pgfsetroundjoin%
\pgfsetlinewidth{0.803000pt}%
\definecolor{currentstroke}{rgb}{0.690196,0.690196,0.690196}%
\pgfsetstrokecolor{currentstroke}%
\pgfsetdash{}{0pt}%
\pgfpathmoveto{\pgfqpoint{0.940251in}{1.591389in}}%
\pgfpathlineto{\pgfqpoint{3.445619in}{2.454608in}}%
\pgfpathlineto{\pgfqpoint{3.509589in}{4.015845in}}%
\pgfusepath{stroke}%
\end{pgfscope}%
\begin{pgfscope}%
\pgfsetbuttcap%
\pgfsetroundjoin%
\pgfsetlinewidth{0.803000pt}%
\definecolor{currentstroke}{rgb}{0.690196,0.690196,0.690196}%
\pgfsetstrokecolor{currentstroke}%
\pgfsetdash{}{0pt}%
\pgfpathmoveto{\pgfqpoint{0.814381in}{1.795810in}}%
\pgfpathlineto{\pgfqpoint{3.303929in}{2.644893in}}%
\pgfpathlineto{\pgfqpoint{3.358641in}{4.204445in}}%
\pgfusepath{stroke}%
\end{pgfscope}%
\begin{pgfscope}%
\pgfsetbuttcap%
\pgfsetroundjoin%
\pgfsetlinewidth{0.803000pt}%
\definecolor{currentstroke}{rgb}{0.690196,0.690196,0.690196}%
\pgfsetstrokecolor{currentstroke}%
\pgfsetdash{}{0pt}%
\pgfpathmoveto{\pgfqpoint{0.690600in}{1.996837in}}%
\pgfpathlineto{\pgfqpoint{3.164510in}{2.832129in}}%
\pgfpathlineto{\pgfqpoint{3.210259in}{4.389838in}}%
\pgfusepath{stroke}%
\end{pgfscope}%
\begin{pgfscope}%
\pgfsetbuttcap%
\pgfsetroundjoin%
\pgfsetlinewidth{0.803000pt}%
\definecolor{currentstroke}{rgb}{0.690196,0.690196,0.690196}%
\pgfsetstrokecolor{currentstroke}%
\pgfsetdash{}{0pt}%
\pgfpathmoveto{\pgfqpoint{0.568857in}{2.194556in}}%
\pgfpathlineto{\pgfqpoint{3.027307in}{3.016389in}}%
\pgfpathlineto{\pgfqpoint{3.064378in}{4.572108in}}%
\pgfusepath{stroke}%
\end{pgfscope}%
\begin{pgfscope}%
\pgfsetbuttcap%
\pgfsetroundjoin%
\pgfsetlinewidth{0.803000pt}%
\definecolor{currentstroke}{rgb}{0.690196,0.690196,0.690196}%
\pgfsetstrokecolor{currentstroke}%
\pgfsetdash{}{0pt}%
\pgfpathmoveto{\pgfqpoint{2.870313in}{4.511589in}}%
\pgfpathlineto{\pgfqpoint{2.844233in}{2.955189in}}%
\pgfpathlineto{\pgfqpoint{4.137598in}{1.194802in}}%
\pgfusepath{stroke}%
\end{pgfscope}%
\begin{pgfscope}%
\pgfsetbuttcap%
\pgfsetroundjoin%
\pgfsetlinewidth{0.803000pt}%
\definecolor{currentstroke}{rgb}{0.690196,0.690196,0.690196}%
\pgfsetstrokecolor{currentstroke}%
\pgfsetdash{}{0pt}%
\pgfpathmoveto{\pgfqpoint{2.380325in}{4.358787in}}%
\pgfpathlineto{\pgfqpoint{2.382258in}{2.800756in}}%
\pgfpathlineto{\pgfqpoint{3.649433in}{1.015802in}}%
\pgfusepath{stroke}%
\end{pgfscope}%
\begin{pgfscope}%
\pgfsetbuttcap%
\pgfsetroundjoin%
\pgfsetlinewidth{0.803000pt}%
\definecolor{currentstroke}{rgb}{0.690196,0.690196,0.690196}%
\pgfsetstrokecolor{currentstroke}%
\pgfsetdash{}{0pt}%
\pgfpathmoveto{\pgfqpoint{1.883325in}{4.203798in}}%
\pgfpathlineto{\pgfqpoint{1.914054in}{2.644241in}}%
\pgfpathlineto{\pgfqpoint{3.154178in}{0.834203in}}%
\pgfusepath{stroke}%
\end{pgfscope}%
\begin{pgfscope}%
\pgfsetbuttcap%
\pgfsetroundjoin%
\pgfsetlinewidth{0.803000pt}%
\definecolor{currentstroke}{rgb}{0.690196,0.690196,0.690196}%
\pgfsetstrokecolor{currentstroke}%
\pgfsetdash{}{0pt}%
\pgfpathmoveto{\pgfqpoint{1.379161in}{4.046576in}}%
\pgfpathlineto{\pgfqpoint{1.439496in}{2.485601in}}%
\pgfpathlineto{\pgfqpoint{2.651676in}{0.649946in}}%
\pgfusepath{stroke}%
\end{pgfscope}%
\begin{pgfscope}%
\pgfsetbuttcap%
\pgfsetroundjoin%
\pgfsetlinewidth{0.803000pt}%
\definecolor{currentstroke}{rgb}{0.690196,0.690196,0.690196}%
\pgfsetstrokecolor{currentstroke}%
\pgfsetdash{}{0pt}%
\pgfpathmoveto{\pgfqpoint{0.867678in}{3.887070in}}%
\pgfpathlineto{\pgfqpoint{0.958451in}{2.324793in}}%
\pgfpathlineto{\pgfqpoint{2.141766in}{0.462973in}}%
\pgfusepath{stroke}%
\end{pgfscope}%
\begin{pgfscope}%
\pgfsetbuttcap%
\pgfsetroundjoin%
\pgfsetlinewidth{0.803000pt}%
\definecolor{currentstroke}{rgb}{0.690196,0.690196,0.690196}%
\pgfsetstrokecolor{currentstroke}%
\pgfsetdash{}{0pt}%
\pgfpathmoveto{\pgfqpoint{4.333565in}{1.298888in}}%
\pgfpathlineto{\pgfqpoint{3.028095in}{3.049492in}}%
\pgfpathlineto{\pgfqpoint{0.566399in}{2.227748in}}%
\pgfusepath{stroke}%
\end{pgfscope}%
\begin{pgfscope}%
\pgfsetbuttcap%
\pgfsetroundjoin%
\pgfsetlinewidth{0.803000pt}%
\definecolor{currentstroke}{rgb}{0.690196,0.690196,0.690196}%
\pgfsetstrokecolor{currentstroke}%
\pgfsetdash{}{0pt}%
\pgfpathmoveto{\pgfqpoint{4.352514in}{1.536026in}}%
\pgfpathlineto{\pgfqpoint{3.033725in}{3.285727in}}%
\pgfpathlineto{\pgfqpoint{0.548854in}{2.464712in}}%
\pgfusepath{stroke}%
\end{pgfscope}%
\begin{pgfscope}%
\pgfsetbuttcap%
\pgfsetroundjoin%
\pgfsetlinewidth{0.803000pt}%
\definecolor{currentstroke}{rgb}{0.690196,0.690196,0.690196}%
\pgfsetstrokecolor{currentstroke}%
\pgfsetdash{}{0pt}%
\pgfpathmoveto{\pgfqpoint{4.371841in}{1.777892in}}%
\pgfpathlineto{\pgfqpoint{3.039458in}{3.526333in}}%
\pgfpathlineto{\pgfqpoint{0.530973in}{2.706219in}}%
\pgfusepath{stroke}%
\end{pgfscope}%
\begin{pgfscope}%
\pgfsetbuttcap%
\pgfsetroundjoin%
\pgfsetlinewidth{0.803000pt}%
\definecolor{currentstroke}{rgb}{0.690196,0.690196,0.690196}%
\pgfsetstrokecolor{currentstroke}%
\pgfsetdash{}{0pt}%
\pgfpathmoveto{\pgfqpoint{4.391557in}{2.024629in}}%
\pgfpathlineto{\pgfqpoint{3.045299in}{3.771432in}}%
\pgfpathlineto{\pgfqpoint{0.512745in}{2.952402in}}%
\pgfusepath{stroke}%
\end{pgfscope}%
\begin{pgfscope}%
\pgfsetbuttcap%
\pgfsetroundjoin%
\pgfsetlinewidth{0.803000pt}%
\definecolor{currentstroke}{rgb}{0.690196,0.690196,0.690196}%
\pgfsetstrokecolor{currentstroke}%
\pgfsetdash{}{0pt}%
\pgfpathmoveto{\pgfqpoint{4.411674in}{2.276385in}}%
\pgfpathlineto{\pgfqpoint{3.051249in}{4.021151in}}%
\pgfpathlineto{\pgfqpoint{0.494161in}{3.203398in}}%
\pgfusepath{stroke}%
\end{pgfscope}%
\begin{pgfscope}%
\pgfsetbuttcap%
\pgfsetroundjoin%
\pgfsetlinewidth{0.803000pt}%
\definecolor{currentstroke}{rgb}{0.690196,0.690196,0.690196}%
\pgfsetstrokecolor{currentstroke}%
\pgfsetdash{}{0pt}%
\pgfpathmoveto{\pgfqpoint{4.432204in}{2.533316in}}%
\pgfpathlineto{\pgfqpoint{3.057313in}{4.275622in}}%
\pgfpathlineto{\pgfqpoint{0.475210in}{3.459349in}}%
\pgfusepath{stroke}%
\end{pgfscope}%
\begin{pgfscope}%
\pgfsetbuttcap%
\pgfsetroundjoin%
\pgfsetlinewidth{0.803000pt}%
\definecolor{currentstroke}{rgb}{0.690196,0.690196,0.690196}%
\pgfsetstrokecolor{currentstroke}%
\pgfsetdash{}{0pt}%
\pgfpathmoveto{\pgfqpoint{4.453161in}{2.795583in}}%
\pgfpathlineto{\pgfqpoint{3.063493in}{4.534981in}}%
\pgfpathlineto{\pgfqpoint{0.455881in}{3.720404in}}%
\pgfusepath{stroke}%
\end{pgfscope}%
\begin{pgfscope}%
\pgfsetrectcap%
\pgfsetroundjoin%
\pgfsetlinewidth{0.803000pt}%
\definecolor{currentstroke}{rgb}{0.000000,0.000000,0.000000}%
\pgfsetstrokecolor{currentstroke}%
\pgfsetdash{}{0pt}%
\pgfpathmoveto{\pgfqpoint{0.568857in}{2.194556in}}%
\pgfpathlineto{\pgfqpoint{1.728395in}{0.311399in}}%
\pgfusepath{stroke}%
\end{pgfscope}%
\begin{pgfscope}%
\pgfsetrectcap%
\pgfsetroundjoin%
\pgfsetlinewidth{0.803000pt}%
\definecolor{currentstroke}{rgb}{0.000000,0.000000,0.000000}%
\pgfsetstrokecolor{currentstroke}%
\pgfsetdash{}{0pt}%
\pgfpathmoveto{\pgfqpoint{1.624356in}{0.523055in}}%
\pgfpathlineto{\pgfqpoint{1.559868in}{0.499637in}}%
\pgfusepath{stroke}%
\end{pgfscope}%
\begin{pgfscope}%
\definecolor{textcolor}{rgb}{0.000000,0.000000,0.000000}%
\pgfsetstrokecolor{textcolor}%
\pgfsetfillcolor{textcolor}%
\pgftext[x=1.423850in,y=0.349209in,,top]{\color{textcolor}{\ifdefined\pdftexversion\else\setmainfont{Times New Roman}\rmfamily\fi\fontsize{10.000000}{12.000000}\selectfont\catcode`\^=\active\def^{\ifmmode\sp\else\^{}\fi}\catcode`\%=\active\def%{\%}10}}%
\end{pgfscope}%
\begin{pgfscope}%
\pgfsetrectcap%
\pgfsetroundjoin%
\pgfsetlinewidth{0.803000pt}%
\definecolor{currentstroke}{rgb}{0.000000,0.000000,0.000000}%
\pgfsetstrokecolor{currentstroke}%
\pgfsetdash{}{0pt}%
\pgfpathmoveto{\pgfqpoint{1.487056in}{0.745665in}}%
\pgfpathlineto{\pgfqpoint{1.423006in}{0.722655in}}%
\pgfusepath{stroke}%
\end{pgfscope}%
\begin{pgfscope}%
\definecolor{textcolor}{rgb}{0.000000,0.000000,0.000000}%
\pgfsetstrokecolor{textcolor}%
\pgfsetfillcolor{textcolor}%
\pgftext[x=1.288487in,y=0.573028in,,top]{\color{textcolor}{\ifdefined\pdftexversion\else\setmainfont{Times New Roman}\rmfamily\fi\fontsize{10.000000}{12.000000}\selectfont\catcode`\^=\active\def^{\ifmmode\sp\else\^{}\fi}\catcode`\%=\active\def%{\%}20}}%
\end{pgfscope}%
\begin{pgfscope}%
\pgfsetrectcap%
\pgfsetroundjoin%
\pgfsetlinewidth{0.803000pt}%
\definecolor{currentstroke}{rgb}{0.000000,0.000000,0.000000}%
\pgfsetstrokecolor{currentstroke}%
\pgfsetdash{}{0pt}%
\pgfpathmoveto{\pgfqpoint{1.352133in}{0.964422in}}%
\pgfpathlineto{\pgfqpoint{1.288516in}{0.941809in}}%
\pgfusepath{stroke}%
\end{pgfscope}%
\begin{pgfscope}%
\definecolor{textcolor}{rgb}{0.000000,0.000000,0.000000}%
\pgfsetstrokecolor{textcolor}%
\pgfsetfillcolor{textcolor}%
\pgftext[x=1.155465in,y=0.792979in,,top]{\color{textcolor}{\ifdefined\pdftexversion\else\setmainfont{Times New Roman}\rmfamily\fi\fontsize{10.000000}{12.000000}\selectfont\catcode`\^=\active\def^{\ifmmode\sp\else\^{}\fi}\catcode`\%=\active\def%{\%}30}}%
\end{pgfscope}%
\begin{pgfscope}%
\pgfsetrectcap%
\pgfsetroundjoin%
\pgfsetlinewidth{0.803000pt}%
\definecolor{currentstroke}{rgb}{0.000000,0.000000,0.000000}%
\pgfsetstrokecolor{currentstroke}%
\pgfsetdash{}{0pt}%
\pgfpathmoveto{\pgfqpoint{1.219525in}{1.179424in}}%
\pgfpathlineto{\pgfqpoint{1.156336in}{1.157198in}}%
\pgfusepath{stroke}%
\end{pgfscope}%
\begin{pgfscope}%
\definecolor{textcolor}{rgb}{0.000000,0.000000,0.000000}%
\pgfsetstrokecolor{textcolor}%
\pgfsetfillcolor{textcolor}%
\pgftext[x=1.024722in,y=1.009160in,,top]{\color{textcolor}{\ifdefined\pdftexversion\else\setmainfont{Times New Roman}\rmfamily\fi\fontsize{10.000000}{12.000000}\selectfont\catcode`\^=\active\def^{\ifmmode\sp\else\^{}\fi}\catcode`\%=\active\def%{\%}40}}%
\end{pgfscope}%
\begin{pgfscope}%
\pgfsetrectcap%
\pgfsetroundjoin%
\pgfsetlinewidth{0.803000pt}%
\definecolor{currentstroke}{rgb}{0.000000,0.000000,0.000000}%
\pgfsetstrokecolor{currentstroke}%
\pgfsetdash{}{0pt}%
\pgfpathmoveto{\pgfqpoint{1.089174in}{1.390768in}}%
\pgfpathlineto{\pgfqpoint{1.026407in}{1.368918in}}%
\pgfusepath{stroke}%
\end{pgfscope}%
\begin{pgfscope}%
\definecolor{textcolor}{rgb}{0.000000,0.000000,0.000000}%
\pgfsetstrokecolor{textcolor}%
\pgfsetfillcolor{textcolor}%
\pgftext[x=0.896201in,y=1.221666in,,top]{\color{textcolor}{\ifdefined\pdftexversion\else\setmainfont{Times New Roman}\rmfamily\fi\fontsize{10.000000}{12.000000}\selectfont\catcode`\^=\active\def^{\ifmmode\sp\else\^{}\fi}\catcode`\%=\active\def%{\%}50}}%
\end{pgfscope}%
\begin{pgfscope}%
\pgfsetrectcap%
\pgfsetroundjoin%
\pgfsetlinewidth{0.803000pt}%
\definecolor{currentstroke}{rgb}{0.000000,0.000000,0.000000}%
\pgfsetstrokecolor{currentstroke}%
\pgfsetdash{}{0pt}%
\pgfpathmoveto{\pgfqpoint{0.961023in}{1.598546in}}%
\pgfpathlineto{\pgfqpoint{0.898673in}{1.577064in}}%
\pgfusepath{stroke}%
\end{pgfscope}%
\begin{pgfscope}%
\definecolor{textcolor}{rgb}{0.000000,0.000000,0.000000}%
\pgfsetstrokecolor{textcolor}%
\pgfsetfillcolor{textcolor}%
\pgftext[x=0.769846in,y=1.430592in,,top]{\color{textcolor}{\ifdefined\pdftexversion\else\setmainfont{Times New Roman}\rmfamily\fi\fontsize{10.000000}{12.000000}\selectfont\catcode`\^=\active\def^{\ifmmode\sp\else\^{}\fi}\catcode`\%=\active\def%{\%}60}}%
\end{pgfscope}%
\begin{pgfscope}%
\pgfsetrectcap%
\pgfsetroundjoin%
\pgfsetlinewidth{0.803000pt}%
\definecolor{currentstroke}{rgb}{0.000000,0.000000,0.000000}%
\pgfsetstrokecolor{currentstroke}%
\pgfsetdash{}{0pt}%
\pgfpathmoveto{\pgfqpoint{0.835015in}{1.802847in}}%
\pgfpathlineto{\pgfqpoint{0.773078in}{1.781723in}}%
\pgfusepath{stroke}%
\end{pgfscope}%
\begin{pgfscope}%
\definecolor{textcolor}{rgb}{0.000000,0.000000,0.000000}%
\pgfsetstrokecolor{textcolor}%
\pgfsetfillcolor{textcolor}%
\pgftext[x=0.645602in,y=1.636027in,,top]{\color{textcolor}{\ifdefined\pdftexversion\else\setmainfont{Times New Roman}\rmfamily\fi\fontsize{10.000000}{12.000000}\selectfont\catcode`\^=\active\def^{\ifmmode\sp\else\^{}\fi}\catcode`\%=\active\def%{\%}70}}%
\end{pgfscope}%
\begin{pgfscope}%
\pgfsetrectcap%
\pgfsetroundjoin%
\pgfsetlinewidth{0.803000pt}%
\definecolor{currentstroke}{rgb}{0.000000,0.000000,0.000000}%
\pgfsetstrokecolor{currentstroke}%
\pgfsetdash{}{0pt}%
\pgfpathmoveto{\pgfqpoint{0.711099in}{2.003758in}}%
\pgfpathlineto{\pgfqpoint{0.649568in}{1.982983in}}%
\pgfusepath{stroke}%
\end{pgfscope}%
\begin{pgfscope}%
\definecolor{textcolor}{rgb}{0.000000,0.000000,0.000000}%
\pgfsetstrokecolor{textcolor}%
\pgfsetfillcolor{textcolor}%
\pgftext[x=0.523417in,y=1.838058in,,top]{\color{textcolor}{\ifdefined\pdftexversion\else\setmainfont{Times New Roman}\rmfamily\fi\fontsize{10.000000}{12.000000}\selectfont\catcode`\^=\active\def^{\ifmmode\sp\else\^{}\fi}\catcode`\%=\active\def%{\%}80}}%
\end{pgfscope}%
\begin{pgfscope}%
\pgfsetrectcap%
\pgfsetroundjoin%
\pgfsetlinewidth{0.803000pt}%
\definecolor{currentstroke}{rgb}{0.000000,0.000000,0.000000}%
\pgfsetstrokecolor{currentstroke}%
\pgfsetdash{}{0pt}%
\pgfpathmoveto{\pgfqpoint{0.589221in}{2.201363in}}%
\pgfpathlineto{\pgfqpoint{0.528093in}{2.180929in}}%
\pgfusepath{stroke}%
\end{pgfscope}%
\begin{pgfscope}%
\definecolor{textcolor}{rgb}{0.000000,0.000000,0.000000}%
\pgfsetstrokecolor{textcolor}%
\pgfsetfillcolor{textcolor}%
\pgftext[x=0.403240in,y=2.036768in,,top]{\color{textcolor}{\ifdefined\pdftexversion\else\setmainfont{Times New Roman}\rmfamily\fi\fontsize{10.000000}{12.000000}\selectfont\catcode`\^=\active\def^{\ifmmode\sp\else\^{}\fi}\catcode`\%=\active\def%{\%}90}}%
\end{pgfscope}%
\begin{pgfscope}%
\definecolor{textcolor}{rgb}{0.000000,0.000000,0.000000}%
\pgfsetstrokecolor{textcolor}%
\pgfsetfillcolor{textcolor}%
\pgftext[x=0.751612in,y=0.936654in,,,rotate=301.622346]{\color{textcolor}{\ifdefined\pdftexversion\else\setmainfont{Times New Roman}\rmfamily\fi\fontsize{11.000000}{13.200000}\selectfont\catcode`\^=\active\def^{\ifmmode\sp\else\^{}\fi}\catcode`\%=\active\def%{\%}$x_1$}}%
\end{pgfscope}%
\begin{pgfscope}%
\pgfsetrectcap%
\pgfsetroundjoin%
\pgfsetlinewidth{0.803000pt}%
\definecolor{currentstroke}{rgb}{0.000000,0.000000,0.000000}%
\pgfsetstrokecolor{currentstroke}%
\pgfsetdash{}{0pt}%
\pgfpathmoveto{\pgfqpoint{4.330912in}{1.265686in}}%
\pgfpathlineto{\pgfqpoint{1.728395in}{0.311399in}}%
\pgfusepath{stroke}%
\end{pgfscope}%
\begin{pgfscope}%
\pgfsetrectcap%
\pgfsetroundjoin%
\pgfsetlinewidth{0.803000pt}%
\definecolor{currentstroke}{rgb}{0.000000,0.000000,0.000000}%
\pgfsetstrokecolor{currentstroke}%
\pgfsetdash{}{0pt}%
\pgfpathmoveto{\pgfqpoint{4.126468in}{1.209950in}}%
\pgfpathlineto{\pgfqpoint{4.159898in}{1.164448in}}%
\pgfusepath{stroke}%
\end{pgfscope}%
\begin{pgfscope}%
\definecolor{textcolor}{rgb}{0.000000,0.000000,0.000000}%
\pgfsetstrokecolor{textcolor}%
\pgfsetfillcolor{textcolor}%
\pgftext[x=4.225212in,y=0.965495in,,top]{\color{textcolor}{\ifdefined\pdftexversion\else\setmainfont{Times New Roman}\rmfamily\fi\fontsize{10.000000}{12.000000}\selectfont\catcode`\^=\active\def^{\ifmmode\sp\else\^{}\fi}\catcode`\%=\active\def%{\%}\ensuremath{-}5}}%
\end{pgfscope}%
\begin{pgfscope}%
\pgfsetrectcap%
\pgfsetroundjoin%
\pgfsetlinewidth{0.803000pt}%
\definecolor{currentstroke}{rgb}{0.000000,0.000000,0.000000}%
\pgfsetstrokecolor{currentstroke}%
\pgfsetdash{}{0pt}%
\pgfpathmoveto{\pgfqpoint{3.638523in}{1.031170in}}%
\pgfpathlineto{\pgfqpoint{3.671294in}{0.985010in}}%
\pgfusepath{stroke}%
\end{pgfscope}%
\begin{pgfscope}%
\definecolor{textcolor}{rgb}{0.000000,0.000000,0.000000}%
\pgfsetstrokecolor{textcolor}%
\pgfsetfillcolor{textcolor}%
\pgftext[x=3.737183in,y=0.784659in,,top]{\color{textcolor}{\ifdefined\pdftexversion\else\setmainfont{Times New Roman}\rmfamily\fi\fontsize{10.000000}{12.000000}\selectfont\catcode`\^=\active\def^{\ifmmode\sp\else\^{}\fi}\catcode`\%=\active\def%{\%}0}}%
\end{pgfscope}%
\begin{pgfscope}%
\pgfsetrectcap%
\pgfsetroundjoin%
\pgfsetlinewidth{0.803000pt}%
\definecolor{currentstroke}{rgb}{0.000000,0.000000,0.000000}%
\pgfsetstrokecolor{currentstroke}%
\pgfsetdash{}{0pt}%
\pgfpathmoveto{\pgfqpoint{3.143495in}{0.849795in}}%
\pgfpathlineto{\pgfqpoint{3.175583in}{0.802961in}}%
\pgfusepath{stroke}%
\end{pgfscope}%
\begin{pgfscope}%
\definecolor{textcolor}{rgb}{0.000000,0.000000,0.000000}%
\pgfsetstrokecolor{textcolor}%
\pgfsetfillcolor{textcolor}%
\pgftext[x=3.242058in,y=0.601193in,,top]{\color{textcolor}{\ifdefined\pdftexversion\else\setmainfont{Times New Roman}\rmfamily\fi\fontsize{10.000000}{12.000000}\selectfont\catcode`\^=\active\def^{\ifmmode\sp\else\^{}\fi}\catcode`\%=\active\def%{\%}5}}%
\end{pgfscope}%
\begin{pgfscope}%
\pgfsetrectcap%
\pgfsetroundjoin%
\pgfsetlinewidth{0.803000pt}%
\definecolor{currentstroke}{rgb}{0.000000,0.000000,0.000000}%
\pgfsetstrokecolor{currentstroke}%
\pgfsetdash{}{0pt}%
\pgfpathmoveto{\pgfqpoint{2.641228in}{0.665767in}}%
\pgfpathlineto{\pgfqpoint{2.672609in}{0.618245in}}%
\pgfusepath{stroke}%
\end{pgfscope}%
\begin{pgfscope}%
\definecolor{textcolor}{rgb}{0.000000,0.000000,0.000000}%
\pgfsetstrokecolor{textcolor}%
\pgfsetfillcolor{textcolor}%
\pgftext[x=2.739681in,y=0.415039in,,top]{\color{textcolor}{\ifdefined\pdftexversion\else\setmainfont{Times New Roman}\rmfamily\fi\fontsize{10.000000}{12.000000}\selectfont\catcode`\^=\active\def^{\ifmmode\sp\else\^{}\fi}\catcode`\%=\active\def%{\%}10}}%
\end{pgfscope}%
\begin{pgfscope}%
\pgfsetrectcap%
\pgfsetroundjoin%
\pgfsetlinewidth{0.803000pt}%
\definecolor{currentstroke}{rgb}{0.000000,0.000000,0.000000}%
\pgfsetstrokecolor{currentstroke}%
\pgfsetdash{}{0pt}%
\pgfpathmoveto{\pgfqpoint{2.131562in}{0.479028in}}%
\pgfpathlineto{\pgfqpoint{2.162213in}{0.430803in}}%
\pgfusepath{stroke}%
\end{pgfscope}%
\begin{pgfscope}%
\definecolor{textcolor}{rgb}{0.000000,0.000000,0.000000}%
\pgfsetstrokecolor{textcolor}%
\pgfsetfillcolor{textcolor}%
\pgftext[x=2.229891in,y=0.226139in,,top]{\color{textcolor}{\ifdefined\pdftexversion\else\setmainfont{Times New Roman}\rmfamily\fi\fontsize{10.000000}{12.000000}\selectfont\catcode`\^=\active\def^{\ifmmode\sp\else\^{}\fi}\catcode`\%=\active\def%{\%}15}}%
\end{pgfscope}%
\begin{pgfscope}%
\definecolor{textcolor}{rgb}{0.000000,0.000000,0.000000}%
\pgfsetstrokecolor{textcolor}%
\pgfsetfillcolor{textcolor}%
\pgftext[x=3.245592in,y=0.289591in,,,rotate=20.136892]{\color{textcolor}{\ifdefined\pdftexversion\else\setmainfont{Times New Roman}\rmfamily\fi\fontsize{11.000000}{13.200000}\selectfont\catcode`\^=\active\def^{\ifmmode\sp\else\^{}\fi}\catcode`\%=\active\def%{\%}$x_2$}}%
\end{pgfscope}%
\begin{pgfscope}%
\pgfsetrectcap%
\pgfsetroundjoin%
\pgfsetlinewidth{0.803000pt}%
\definecolor{currentstroke}{rgb}{0.000000,0.000000,0.000000}%
\pgfsetstrokecolor{currentstroke}%
\pgfsetdash{}{0pt}%
\pgfpathmoveto{\pgfqpoint{4.330912in}{1.265686in}}%
\pgfpathlineto{\pgfqpoint{4.456164in}{2.833159in}}%
\pgfusepath{stroke}%
\end{pgfscope}%
\begin{pgfscope}%
\pgfsetrectcap%
\pgfsetroundjoin%
\pgfsetlinewidth{0.803000pt}%
\definecolor{currentstroke}{rgb}{0.000000,0.000000,0.000000}%
\pgfsetstrokecolor{currentstroke}%
\pgfsetdash{}{0pt}%
\pgfpathmoveto{\pgfqpoint{4.322332in}{1.313951in}}%
\pgfpathlineto{\pgfqpoint{4.356072in}{1.268707in}}%
\pgfusepath{stroke}%
\end{pgfscope}%
\begin{pgfscope}%
\definecolor{textcolor}{rgb}{0.000000,0.000000,0.000000}%
\pgfsetstrokecolor{textcolor}%
\pgfsetfillcolor{textcolor}%
\pgftext[x=4.600665in,y=1.233030in,,top]{\color{textcolor}{\ifdefined\pdftexversion\else\setmainfont{Times New Roman}\rmfamily\fi\fontsize{10.000000}{12.000000}\selectfont\catcode`\^=\active\def^{\ifmmode\sp\else\^{}\fi}\catcode`\%=\active\def%{\%}\ensuremath{-}3}}%
\end{pgfscope}%
\begin{pgfscope}%
\pgfsetrectcap%
\pgfsetroundjoin%
\pgfsetlinewidth{0.803000pt}%
\definecolor{currentstroke}{rgb}{0.000000,0.000000,0.000000}%
\pgfsetstrokecolor{currentstroke}%
\pgfsetdash{}{0pt}%
\pgfpathmoveto{\pgfqpoint{4.341159in}{1.551092in}}%
\pgfpathlineto{\pgfqpoint{4.375267in}{1.505839in}}%
\pgfusepath{stroke}%
\end{pgfscope}%
\begin{pgfscope}%
\definecolor{textcolor}{rgb}{0.000000,0.000000,0.000000}%
\pgfsetstrokecolor{textcolor}%
\pgfsetfillcolor{textcolor}%
\pgftext[x=4.622309in,y=1.470154in,,top]{\color{textcolor}{\ifdefined\pdftexversion\else\setmainfont{Times New Roman}\rmfamily\fi\fontsize{10.000000}{12.000000}\selectfont\catcode`\^=\active\def^{\ifmmode\sp\else\^{}\fi}\catcode`\%=\active\def%{\%}\ensuremath{-}2}}%
\end{pgfscope}%
\begin{pgfscope}%
\pgfsetrectcap%
\pgfsetroundjoin%
\pgfsetlinewidth{0.803000pt}%
\definecolor{currentstroke}{rgb}{0.000000,0.000000,0.000000}%
\pgfsetstrokecolor{currentstroke}%
\pgfsetdash{}{0pt}%
\pgfpathmoveto{\pgfqpoint{4.360360in}{1.792957in}}%
\pgfpathlineto{\pgfqpoint{4.394845in}{1.747705in}}%
\pgfusepath{stroke}%
\end{pgfscope}%
\begin{pgfscope}%
\definecolor{textcolor}{rgb}{0.000000,0.000000,0.000000}%
\pgfsetstrokecolor{textcolor}%
\pgfsetfillcolor{textcolor}%
\pgftext[x=4.644385in,y=1.712019in,,top]{\color{textcolor}{\ifdefined\pdftexversion\else\setmainfont{Times New Roman}\rmfamily\fi\fontsize{10.000000}{12.000000}\selectfont\catcode`\^=\active\def^{\ifmmode\sp\else\^{}\fi}\catcode`\%=\active\def%{\%}\ensuremath{-}1}}%
\end{pgfscope}%
\begin{pgfscope}%
\pgfsetrectcap%
\pgfsetroundjoin%
\pgfsetlinewidth{0.803000pt}%
\definecolor{currentstroke}{rgb}{0.000000,0.000000,0.000000}%
\pgfsetstrokecolor{currentstroke}%
\pgfsetdash{}{0pt}%
\pgfpathmoveto{\pgfqpoint{4.379948in}{2.039691in}}%
\pgfpathlineto{\pgfqpoint{4.414817in}{1.994447in}}%
\pgfusepath{stroke}%
\end{pgfscope}%
\begin{pgfscope}%
\definecolor{textcolor}{rgb}{0.000000,0.000000,0.000000}%
\pgfsetstrokecolor{textcolor}%
\pgfsetfillcolor{textcolor}%
\pgftext[x=4.666907in,y=1.958767in,,top]{\color{textcolor}{\ifdefined\pdftexversion\else\setmainfont{Times New Roman}\rmfamily\fi\fontsize{10.000000}{12.000000}\selectfont\catcode`\^=\active\def^{\ifmmode\sp\else\^{}\fi}\catcode`\%=\active\def%{\%}0}}%
\end{pgfscope}%
\begin{pgfscope}%
\pgfsetrectcap%
\pgfsetroundjoin%
\pgfsetlinewidth{0.803000pt}%
\definecolor{currentstroke}{rgb}{0.000000,0.000000,0.000000}%
\pgfsetstrokecolor{currentstroke}%
\pgfsetdash{}{0pt}%
\pgfpathmoveto{\pgfqpoint{4.399934in}{2.291441in}}%
\pgfpathlineto{\pgfqpoint{4.435197in}{2.246216in}}%
\pgfusepath{stroke}%
\end{pgfscope}%
\begin{pgfscope}%
\definecolor{textcolor}{rgb}{0.000000,0.000000,0.000000}%
\pgfsetstrokecolor{textcolor}%
\pgfsetfillcolor{textcolor}%
\pgftext[x=4.689889in,y=2.210550in,,top]{\color{textcolor}{\ifdefined\pdftexversion\else\setmainfont{Times New Roman}\rmfamily\fi\fontsize{10.000000}{12.000000}\selectfont\catcode`\^=\active\def^{\ifmmode\sp\else\^{}\fi}\catcode`\%=\active\def%{\%}1}}%
\end{pgfscope}%
\begin{pgfscope}%
\pgfsetrectcap%
\pgfsetroundjoin%
\pgfsetlinewidth{0.803000pt}%
\definecolor{currentstroke}{rgb}{0.000000,0.000000,0.000000}%
\pgfsetstrokecolor{currentstroke}%
\pgfsetdash{}{0pt}%
\pgfpathmoveto{\pgfqpoint{4.420331in}{2.548362in}}%
\pgfpathlineto{\pgfqpoint{4.455996in}{2.503167in}}%
\pgfusepath{stroke}%
\end{pgfscope}%
\begin{pgfscope}%
\definecolor{textcolor}{rgb}{0.000000,0.000000,0.000000}%
\pgfsetstrokecolor{textcolor}%
\pgfsetfillcolor{textcolor}%
\pgftext[x=4.713344in,y=2.467523in,,top]{\color{textcolor}{\ifdefined\pdftexversion\else\setmainfont{Times New Roman}\rmfamily\fi\fontsize{10.000000}{12.000000}\selectfont\catcode`\^=\active\def^{\ifmmode\sp\else\^{}\fi}\catcode`\%=\active\def%{\%}2}}%
\end{pgfscope}%
\begin{pgfscope}%
\pgfsetrectcap%
\pgfsetroundjoin%
\pgfsetlinewidth{0.803000pt}%
\definecolor{currentstroke}{rgb}{0.000000,0.000000,0.000000}%
\pgfsetstrokecolor{currentstroke}%
\pgfsetdash{}{0pt}%
\pgfpathmoveto{\pgfqpoint{4.441151in}{2.810615in}}%
\pgfpathlineto{\pgfqpoint{4.477227in}{2.765460in}}%
\pgfusepath{stroke}%
\end{pgfscope}%
\begin{pgfscope}%
\definecolor{textcolor}{rgb}{0.000000,0.000000,0.000000}%
\pgfsetstrokecolor{textcolor}%
\pgfsetfillcolor{textcolor}%
\pgftext[x=4.737287in,y=2.729847in,,top]{\color{textcolor}{\ifdefined\pdftexversion\else\setmainfont{Times New Roman}\rmfamily\fi\fontsize{10.000000}{12.000000}\selectfont\catcode`\^=\active\def^{\ifmmode\sp\else\^{}\fi}\catcode`\%=\active\def%{\%}3}}%
\end{pgfscope}%
\begin{pgfscope}%
\definecolor{textcolor}{rgb}{0.000000,0.000000,0.000000}%
\pgfsetstrokecolor{textcolor}%
\pgfsetfillcolor{textcolor}%
\pgftext[x=4.992080in,y=1.880989in,,,rotate=85.431397]{\color{textcolor}{\ifdefined\pdftexversion\else\setmainfont{Times New Roman}\rmfamily\fi\fontsize{11.000000}{13.200000}\selectfont\catcode`\^=\active\def^{\ifmmode\sp\else\^{}\fi}\catcode`\%=\active\def%{\%}$x_3$}}%
\end{pgfscope}%
\begin{pgfscope}%
\pgfpathrectangle{\pgfqpoint{0.100000in}{0.183744in}}{\pgfqpoint{4.506048in}{4.506048in}}%
\pgfusepath{clip}%
\pgfsetrectcap%
\pgfsetroundjoin%
\pgfsetlinewidth{0.501875pt}%
\definecolor{currentstroke}{rgb}{0.000000,0.000000,1.000000}%
\pgfsetstrokecolor{currentstroke}%
\pgfsetstrokeopacity{0.600000}%
\pgfsetdash{}{0pt}%
\pgfpathmoveto{\pgfqpoint{2.935970in}{2.476502in}}%
\pgfpathlineto{\pgfqpoint{2.811762in}{2.645712in}}%
\pgfusepath{stroke}%
\end{pgfscope}%
\begin{pgfscope}%
\pgfpathrectangle{\pgfqpoint{0.100000in}{0.183744in}}{\pgfqpoint{4.506048in}{4.506048in}}%
\pgfusepath{clip}%
\pgfsetrectcap%
\pgfsetroundjoin%
\pgfsetlinewidth{0.501875pt}%
\definecolor{currentstroke}{rgb}{0.000000,0.000000,1.000000}%
\pgfsetstrokecolor{currentstroke}%
\pgfsetstrokeopacity{0.600000}%
\pgfsetdash{}{0pt}%
\pgfpathmoveto{\pgfqpoint{3.393303in}{2.521538in}}%
\pgfpathlineto{\pgfqpoint{3.419357in}{2.305014in}}%
\pgfusepath{stroke}%
\end{pgfscope}%
\begin{pgfscope}%
\pgfpathrectangle{\pgfqpoint{0.100000in}{0.183744in}}{\pgfqpoint{4.506048in}{4.506048in}}%
\pgfusepath{clip}%
\pgfsetrectcap%
\pgfsetroundjoin%
\pgfsetlinewidth{0.501875pt}%
\definecolor{currentstroke}{rgb}{0.000000,0.000000,1.000000}%
\pgfsetstrokecolor{currentstroke}%
\pgfsetstrokeopacity{0.600000}%
\pgfsetdash{}{0pt}%
\pgfpathmoveto{\pgfqpoint{2.425806in}{3.588653in}}%
\pgfpathlineto{\pgfqpoint{2.354189in}{3.312493in}}%
\pgfusepath{stroke}%
\end{pgfscope}%
\begin{pgfscope}%
\pgfpathrectangle{\pgfqpoint{0.100000in}{0.183744in}}{\pgfqpoint{4.506048in}{4.506048in}}%
\pgfusepath{clip}%
\pgfsetrectcap%
\pgfsetroundjoin%
\pgfsetlinewidth{0.501875pt}%
\definecolor{currentstroke}{rgb}{0.000000,0.000000,1.000000}%
\pgfsetstrokecolor{currentstroke}%
\pgfsetstrokeopacity{0.600000}%
\pgfsetdash{}{0pt}%
\pgfpathmoveto{\pgfqpoint{2.607582in}{3.360038in}}%
\pgfpathlineto{\pgfqpoint{2.622496in}{3.370448in}}%
\pgfusepath{stroke}%
\end{pgfscope}%
\begin{pgfscope}%
\pgfpathrectangle{\pgfqpoint{0.100000in}{0.183744in}}{\pgfqpoint{4.506048in}{4.506048in}}%
\pgfusepath{clip}%
\pgfsetrectcap%
\pgfsetroundjoin%
\pgfsetlinewidth{0.501875pt}%
\definecolor{currentstroke}{rgb}{0.000000,0.000000,1.000000}%
\pgfsetstrokecolor{currentstroke}%
\pgfsetstrokeopacity{0.600000}%
\pgfsetdash{}{0pt}%
\pgfpathmoveto{\pgfqpoint{2.831348in}{3.394941in}}%
\pgfpathlineto{\pgfqpoint{2.787741in}{3.386226in}}%
\pgfusepath{stroke}%
\end{pgfscope}%
\begin{pgfscope}%
\pgfpathrectangle{\pgfqpoint{0.100000in}{0.183744in}}{\pgfqpoint{4.506048in}{4.506048in}}%
\pgfusepath{clip}%
\pgfsetrectcap%
\pgfsetroundjoin%
\pgfsetlinewidth{0.501875pt}%
\definecolor{currentstroke}{rgb}{0.000000,0.000000,1.000000}%
\pgfsetstrokecolor{currentstroke}%
\pgfsetstrokeopacity{0.600000}%
\pgfsetdash{}{0pt}%
\pgfpathmoveto{\pgfqpoint{3.612650in}{2.548262in}}%
\pgfpathlineto{\pgfqpoint{3.519638in}{2.874527in}}%
\pgfusepath{stroke}%
\end{pgfscope}%
\begin{pgfscope}%
\pgfpathrectangle{\pgfqpoint{0.100000in}{0.183744in}}{\pgfqpoint{4.506048in}{4.506048in}}%
\pgfusepath{clip}%
\pgfsetrectcap%
\pgfsetroundjoin%
\pgfsetlinewidth{0.501875pt}%
\definecolor{currentstroke}{rgb}{0.000000,0.000000,1.000000}%
\pgfsetstrokecolor{currentstroke}%
\pgfsetstrokeopacity{0.600000}%
\pgfsetdash{}{0pt}%
\pgfpathmoveto{\pgfqpoint{1.974696in}{1.853090in}}%
\pgfpathlineto{\pgfqpoint{2.034927in}{1.665251in}}%
\pgfusepath{stroke}%
\end{pgfscope}%
\begin{pgfscope}%
\pgfpathrectangle{\pgfqpoint{0.100000in}{0.183744in}}{\pgfqpoint{4.506048in}{4.506048in}}%
\pgfusepath{clip}%
\pgfsetrectcap%
\pgfsetroundjoin%
\pgfsetlinewidth{0.501875pt}%
\definecolor{currentstroke}{rgb}{0.000000,0.000000,1.000000}%
\pgfsetstrokecolor{currentstroke}%
\pgfsetstrokeopacity{0.600000}%
\pgfsetdash{}{0pt}%
\pgfpathmoveto{\pgfqpoint{1.764592in}{2.025285in}}%
\pgfpathlineto{\pgfqpoint{1.683668in}{1.908875in}}%
\pgfusepath{stroke}%
\end{pgfscope}%
\begin{pgfscope}%
\pgfpathrectangle{\pgfqpoint{0.100000in}{0.183744in}}{\pgfqpoint{4.506048in}{4.506048in}}%
\pgfusepath{clip}%
\pgfsetrectcap%
\pgfsetroundjoin%
\pgfsetlinewidth{0.501875pt}%
\definecolor{currentstroke}{rgb}{0.000000,0.000000,1.000000}%
\pgfsetstrokecolor{currentstroke}%
\pgfsetstrokeopacity{0.600000}%
\pgfsetdash{}{0pt}%
\pgfpathmoveto{\pgfqpoint{3.034512in}{1.295547in}}%
\pgfpathlineto{\pgfqpoint{3.159912in}{1.701247in}}%
\pgfusepath{stroke}%
\end{pgfscope}%
\begin{pgfscope}%
\pgfpathrectangle{\pgfqpoint{0.100000in}{0.183744in}}{\pgfqpoint{4.506048in}{4.506048in}}%
\pgfusepath{clip}%
\pgfsetrectcap%
\pgfsetroundjoin%
\pgfsetlinewidth{0.501875pt}%
\definecolor{currentstroke}{rgb}{0.000000,0.000000,1.000000}%
\pgfsetstrokecolor{currentstroke}%
\pgfsetstrokeopacity{0.600000}%
\pgfsetdash{}{0pt}%
\pgfpathmoveto{\pgfqpoint{1.662061in}{2.150426in}}%
\pgfpathlineto{\pgfqpoint{1.618080in}{2.114468in}}%
\pgfusepath{stroke}%
\end{pgfscope}%
\begin{pgfscope}%
\pgfpathrectangle{\pgfqpoint{0.100000in}{0.183744in}}{\pgfqpoint{4.506048in}{4.506048in}}%
\pgfusepath{clip}%
\pgfsetrectcap%
\pgfsetroundjoin%
\pgfsetlinewidth{0.501875pt}%
\definecolor{currentstroke}{rgb}{0.000000,0.000000,1.000000}%
\pgfsetstrokecolor{currentstroke}%
\pgfsetstrokeopacity{0.600000}%
\pgfsetdash{}{0pt}%
\pgfpathmoveto{\pgfqpoint{2.733480in}{2.825063in}}%
\pgfpathlineto{\pgfqpoint{2.890940in}{3.083796in}}%
\pgfusepath{stroke}%
\end{pgfscope}%
\begin{pgfscope}%
\pgfpathrectangle{\pgfqpoint{0.100000in}{0.183744in}}{\pgfqpoint{4.506048in}{4.506048in}}%
\pgfusepath{clip}%
\pgfsetrectcap%
\pgfsetroundjoin%
\pgfsetlinewidth{0.501875pt}%
\definecolor{currentstroke}{rgb}{0.000000,0.000000,1.000000}%
\pgfsetstrokecolor{currentstroke}%
\pgfsetstrokeopacity{0.600000}%
\pgfsetdash{}{0pt}%
\pgfpathmoveto{\pgfqpoint{3.047274in}{2.198060in}}%
\pgfpathlineto{\pgfqpoint{3.048338in}{1.758143in}}%
\pgfusepath{stroke}%
\end{pgfscope}%
\begin{pgfscope}%
\pgfpathrectangle{\pgfqpoint{0.100000in}{0.183744in}}{\pgfqpoint{4.506048in}{4.506048in}}%
\pgfusepath{clip}%
\pgfsetrectcap%
\pgfsetroundjoin%
\pgfsetlinewidth{0.501875pt}%
\definecolor{currentstroke}{rgb}{0.000000,0.000000,1.000000}%
\pgfsetstrokecolor{currentstroke}%
\pgfsetstrokeopacity{0.600000}%
\pgfsetdash{}{0pt}%
\pgfpathmoveto{\pgfqpoint{2.695711in}{1.184306in}}%
\pgfpathlineto{\pgfqpoint{2.813257in}{0.987374in}}%
\pgfusepath{stroke}%
\end{pgfscope}%
\begin{pgfscope}%
\pgfpathrectangle{\pgfqpoint{0.100000in}{0.183744in}}{\pgfqpoint{4.506048in}{4.506048in}}%
\pgfusepath{clip}%
\pgfsetrectcap%
\pgfsetroundjoin%
\pgfsetlinewidth{0.501875pt}%
\definecolor{currentstroke}{rgb}{0.000000,0.000000,1.000000}%
\pgfsetstrokecolor{currentstroke}%
\pgfsetstrokeopacity{0.600000}%
\pgfsetdash{}{0pt}%
\pgfpathmoveto{\pgfqpoint{3.535746in}{2.764244in}}%
\pgfpathlineto{\pgfqpoint{3.500821in}{2.966689in}}%
\pgfusepath{stroke}%
\end{pgfscope}%
\begin{pgfscope}%
\pgfpathrectangle{\pgfqpoint{0.100000in}{0.183744in}}{\pgfqpoint{4.506048in}{4.506048in}}%
\pgfusepath{clip}%
\pgfsetrectcap%
\pgfsetroundjoin%
\pgfsetlinewidth{0.501875pt}%
\definecolor{currentstroke}{rgb}{0.000000,0.000000,1.000000}%
\pgfsetstrokecolor{currentstroke}%
\pgfsetstrokeopacity{0.600000}%
\pgfsetdash{}{0pt}%
\pgfpathmoveto{\pgfqpoint{2.168574in}{3.333200in}}%
\pgfpathlineto{\pgfqpoint{2.048105in}{3.067848in}}%
\pgfusepath{stroke}%
\end{pgfscope}%
\begin{pgfscope}%
\pgfpathrectangle{\pgfqpoint{0.100000in}{0.183744in}}{\pgfqpoint{4.506048in}{4.506048in}}%
\pgfusepath{clip}%
\pgfsetrectcap%
\pgfsetroundjoin%
\pgfsetlinewidth{0.501875pt}%
\definecolor{currentstroke}{rgb}{0.000000,0.000000,1.000000}%
\pgfsetstrokecolor{currentstroke}%
\pgfsetstrokeopacity{0.600000}%
\pgfsetdash{}{0pt}%
\pgfpathmoveto{\pgfqpoint{1.471697in}{2.009673in}}%
\pgfpathlineto{\pgfqpoint{1.510336in}{1.777678in}}%
\pgfusepath{stroke}%
\end{pgfscope}%
\begin{pgfscope}%
\pgfpathrectangle{\pgfqpoint{0.100000in}{0.183744in}}{\pgfqpoint{4.506048in}{4.506048in}}%
\pgfusepath{clip}%
\pgfsetrectcap%
\pgfsetroundjoin%
\pgfsetlinewidth{0.501875pt}%
\definecolor{currentstroke}{rgb}{0.000000,0.000000,1.000000}%
\pgfsetstrokecolor{currentstroke}%
\pgfsetstrokeopacity{0.600000}%
\pgfsetdash{}{0pt}%
\pgfpathmoveto{\pgfqpoint{1.893392in}{1.877140in}}%
\pgfpathlineto{\pgfqpoint{1.870222in}{1.672191in}}%
\pgfusepath{stroke}%
\end{pgfscope}%
\begin{pgfscope}%
\pgfpathrectangle{\pgfqpoint{0.100000in}{0.183744in}}{\pgfqpoint{4.506048in}{4.506048in}}%
\pgfusepath{clip}%
\pgfsetrectcap%
\pgfsetroundjoin%
\pgfsetlinewidth{0.501875pt}%
\definecolor{currentstroke}{rgb}{0.000000,0.000000,1.000000}%
\pgfsetstrokecolor{currentstroke}%
\pgfsetstrokeopacity{0.600000}%
\pgfsetdash{}{0pt}%
\pgfpathmoveto{\pgfqpoint{1.698520in}{1.784829in}}%
\pgfpathlineto{\pgfqpoint{1.738570in}{1.686850in}}%
\pgfusepath{stroke}%
\end{pgfscope}%
\begin{pgfscope}%
\pgfpathrectangle{\pgfqpoint{0.100000in}{0.183744in}}{\pgfqpoint{4.506048in}{4.506048in}}%
\pgfusepath{clip}%
\pgfsetrectcap%
\pgfsetroundjoin%
\pgfsetlinewidth{0.501875pt}%
\definecolor{currentstroke}{rgb}{0.000000,0.000000,1.000000}%
\pgfsetstrokecolor{currentstroke}%
\pgfsetstrokeopacity{0.600000}%
\pgfsetdash{}{0pt}%
\pgfpathmoveto{\pgfqpoint{1.441245in}{3.045469in}}%
\pgfpathlineto{\pgfqpoint{1.451334in}{2.872382in}}%
\pgfusepath{stroke}%
\end{pgfscope}%
\begin{pgfscope}%
\pgfpathrectangle{\pgfqpoint{0.100000in}{0.183744in}}{\pgfqpoint{4.506048in}{4.506048in}}%
\pgfusepath{clip}%
\pgfsetrectcap%
\pgfsetroundjoin%
\pgfsetlinewidth{0.501875pt}%
\definecolor{currentstroke}{rgb}{0.000000,0.000000,1.000000}%
\pgfsetstrokecolor{currentstroke}%
\pgfsetstrokeopacity{0.600000}%
\pgfsetdash{}{0pt}%
\pgfpathmoveto{\pgfqpoint{1.789389in}{2.374133in}}%
\pgfpathlineto{\pgfqpoint{1.810496in}{2.438797in}}%
\pgfusepath{stroke}%
\end{pgfscope}%
\begin{pgfscope}%
\pgfpathrectangle{\pgfqpoint{0.100000in}{0.183744in}}{\pgfqpoint{4.506048in}{4.506048in}}%
\pgfusepath{clip}%
\pgfsetrectcap%
\pgfsetroundjoin%
\pgfsetlinewidth{0.501875pt}%
\definecolor{currentstroke}{rgb}{0.000000,0.000000,1.000000}%
\pgfsetstrokecolor{currentstroke}%
\pgfsetstrokeopacity{0.600000}%
\pgfsetdash{}{0pt}%
\pgfpathmoveto{\pgfqpoint{1.026121in}{2.035976in}}%
\pgfpathlineto{\pgfqpoint{1.146945in}{2.019995in}}%
\pgfusepath{stroke}%
\end{pgfscope}%
\begin{pgfscope}%
\pgfpathrectangle{\pgfqpoint{0.100000in}{0.183744in}}{\pgfqpoint{4.506048in}{4.506048in}}%
\pgfusepath{clip}%
\pgfsetrectcap%
\pgfsetroundjoin%
\pgfsetlinewidth{0.501875pt}%
\definecolor{currentstroke}{rgb}{0.000000,0.000000,1.000000}%
\pgfsetstrokecolor{currentstroke}%
\pgfsetstrokeopacity{0.600000}%
\pgfsetdash{}{0pt}%
\pgfpathmoveto{\pgfqpoint{1.424054in}{1.809599in}}%
\pgfpathlineto{\pgfqpoint{1.361699in}{1.759453in}}%
\pgfusepath{stroke}%
\end{pgfscope}%
\begin{pgfscope}%
\pgfpathrectangle{\pgfqpoint{0.100000in}{0.183744in}}{\pgfqpoint{4.506048in}{4.506048in}}%
\pgfusepath{clip}%
\pgfsetrectcap%
\pgfsetroundjoin%
\pgfsetlinewidth{0.501875pt}%
\definecolor{currentstroke}{rgb}{0.000000,0.000000,1.000000}%
\pgfsetstrokecolor{currentstroke}%
\pgfsetstrokeopacity{0.600000}%
\pgfsetdash{}{0pt}%
\pgfpathmoveto{\pgfqpoint{3.071959in}{3.109322in}}%
\pgfpathlineto{\pgfqpoint{3.060184in}{3.083690in}}%
\pgfusepath{stroke}%
\end{pgfscope}%
\begin{pgfscope}%
\pgfpathrectangle{\pgfqpoint{0.100000in}{0.183744in}}{\pgfqpoint{4.506048in}{4.506048in}}%
\pgfusepath{clip}%
\pgfsetrectcap%
\pgfsetroundjoin%
\pgfsetlinewidth{0.501875pt}%
\definecolor{currentstroke}{rgb}{0.000000,0.000000,1.000000}%
\pgfsetstrokecolor{currentstroke}%
\pgfsetstrokeopacity{0.600000}%
\pgfsetdash{}{0pt}%
\pgfpathmoveto{\pgfqpoint{2.029125in}{2.833397in}}%
\pgfpathlineto{\pgfqpoint{2.018336in}{2.868645in}}%
\pgfusepath{stroke}%
\end{pgfscope}%
\begin{pgfscope}%
\pgfpathrectangle{\pgfqpoint{0.100000in}{0.183744in}}{\pgfqpoint{4.506048in}{4.506048in}}%
\pgfusepath{clip}%
\pgfsetrectcap%
\pgfsetroundjoin%
\pgfsetlinewidth{0.501875pt}%
\definecolor{currentstroke}{rgb}{0.000000,0.000000,1.000000}%
\pgfsetstrokecolor{currentstroke}%
\pgfsetstrokeopacity{0.600000}%
\pgfsetdash{}{0pt}%
\pgfpathmoveto{\pgfqpoint{3.055505in}{4.190872in}}%
\pgfpathlineto{\pgfqpoint{3.048941in}{3.963250in}}%
\pgfusepath{stroke}%
\end{pgfscope}%
\begin{pgfscope}%
\pgfpathrectangle{\pgfqpoint{0.100000in}{0.183744in}}{\pgfqpoint{4.506048in}{4.506048in}}%
\pgfusepath{clip}%
\pgfsetrectcap%
\pgfsetroundjoin%
\pgfsetlinewidth{0.501875pt}%
\definecolor{currentstroke}{rgb}{0.000000,0.000000,1.000000}%
\pgfsetstrokecolor{currentstroke}%
\pgfsetstrokeopacity{0.600000}%
\pgfsetdash{}{0pt}%
\pgfpathmoveto{\pgfqpoint{1.113593in}{3.459724in}}%
\pgfpathlineto{\pgfqpoint{0.981144in}{3.183802in}}%
\pgfusepath{stroke}%
\end{pgfscope}%
\begin{pgfscope}%
\pgfpathrectangle{\pgfqpoint{0.100000in}{0.183744in}}{\pgfqpoint{4.506048in}{4.506048in}}%
\pgfusepath{clip}%
\pgfsetrectcap%
\pgfsetroundjoin%
\pgfsetlinewidth{0.501875pt}%
\definecolor{currentstroke}{rgb}{0.000000,0.000000,1.000000}%
\pgfsetstrokecolor{currentstroke}%
\pgfsetstrokeopacity{0.600000}%
\pgfsetdash{}{0pt}%
\pgfpathmoveto{\pgfqpoint{2.681585in}{2.889403in}}%
\pgfpathlineto{\pgfqpoint{2.623141in}{2.961416in}}%
\pgfusepath{stroke}%
\end{pgfscope}%
\begin{pgfscope}%
\pgfpathrectangle{\pgfqpoint{0.100000in}{0.183744in}}{\pgfqpoint{4.506048in}{4.506048in}}%
\pgfusepath{clip}%
\pgfsetrectcap%
\pgfsetroundjoin%
\pgfsetlinewidth{0.501875pt}%
\definecolor{currentstroke}{rgb}{0.000000,0.000000,1.000000}%
\pgfsetstrokecolor{currentstroke}%
\pgfsetstrokeopacity{0.600000}%
\pgfsetdash{}{0pt}%
\pgfpathmoveto{\pgfqpoint{2.063428in}{3.075280in}}%
\pgfpathlineto{\pgfqpoint{2.165247in}{3.039097in}}%
\pgfusepath{stroke}%
\end{pgfscope}%
\begin{pgfscope}%
\pgfpathrectangle{\pgfqpoint{0.100000in}{0.183744in}}{\pgfqpoint{4.506048in}{4.506048in}}%
\pgfusepath{clip}%
\pgfsetrectcap%
\pgfsetroundjoin%
\pgfsetlinewidth{0.501875pt}%
\definecolor{currentstroke}{rgb}{0.000000,0.000000,1.000000}%
\pgfsetstrokecolor{currentstroke}%
\pgfsetstrokeopacity{0.600000}%
\pgfsetdash{}{0pt}%
\pgfpathmoveto{\pgfqpoint{1.266303in}{3.051720in}}%
\pgfpathlineto{\pgfqpoint{1.190068in}{2.794673in}}%
\pgfusepath{stroke}%
\end{pgfscope}%
\begin{pgfscope}%
\pgfpathrectangle{\pgfqpoint{0.100000in}{0.183744in}}{\pgfqpoint{4.506048in}{4.506048in}}%
\pgfusepath{clip}%
\pgfsetrectcap%
\pgfsetroundjoin%
\pgfsetlinewidth{0.501875pt}%
\definecolor{currentstroke}{rgb}{0.000000,0.000000,1.000000}%
\pgfsetstrokecolor{currentstroke}%
\pgfsetstrokeopacity{0.600000}%
\pgfsetdash{}{0pt}%
\pgfpathmoveto{\pgfqpoint{3.927480in}{2.670312in}}%
\pgfpathlineto{\pgfqpoint{3.968930in}{2.323582in}}%
\pgfusepath{stroke}%
\end{pgfscope}%
\begin{pgfscope}%
\pgfpathrectangle{\pgfqpoint{0.100000in}{0.183744in}}{\pgfqpoint{4.506048in}{4.506048in}}%
\pgfusepath{clip}%
\pgfsetrectcap%
\pgfsetroundjoin%
\pgfsetlinewidth{0.501875pt}%
\definecolor{currentstroke}{rgb}{0.000000,0.000000,1.000000}%
\pgfsetstrokecolor{currentstroke}%
\pgfsetstrokeopacity{0.600000}%
\pgfsetdash{}{0pt}%
\pgfpathmoveto{\pgfqpoint{1.388366in}{1.577088in}}%
\pgfpathlineto{\pgfqpoint{1.203420in}{1.590346in}}%
\pgfusepath{stroke}%
\end{pgfscope}%
\begin{pgfscope}%
\pgfpathrectangle{\pgfqpoint{0.100000in}{0.183744in}}{\pgfqpoint{4.506048in}{4.506048in}}%
\pgfusepath{clip}%
\pgfsetrectcap%
\pgfsetroundjoin%
\pgfsetlinewidth{0.501875pt}%
\definecolor{currentstroke}{rgb}{0.000000,0.000000,1.000000}%
\pgfsetstrokecolor{currentstroke}%
\pgfsetstrokeopacity{0.600000}%
\pgfsetdash{}{0pt}%
\pgfpathmoveto{\pgfqpoint{4.044399in}{3.206014in}}%
\pgfpathlineto{\pgfqpoint{4.111798in}{3.077174in}}%
\pgfusepath{stroke}%
\end{pgfscope}%
\begin{pgfscope}%
\pgfpathrectangle{\pgfqpoint{0.100000in}{0.183744in}}{\pgfqpoint{4.506048in}{4.506048in}}%
\pgfusepath{clip}%
\pgfsetrectcap%
\pgfsetroundjoin%
\pgfsetlinewidth{0.501875pt}%
\definecolor{currentstroke}{rgb}{0.000000,0.000000,1.000000}%
\pgfsetstrokecolor{currentstroke}%
\pgfsetstrokeopacity{0.600000}%
\pgfsetdash{}{0pt}%
\pgfpathmoveto{\pgfqpoint{0.921815in}{2.886385in}}%
\pgfpathlineto{\pgfqpoint{0.614008in}{2.529310in}}%
\pgfusepath{stroke}%
\end{pgfscope}%
\begin{pgfscope}%
\pgfpathrectangle{\pgfqpoint{0.100000in}{0.183744in}}{\pgfqpoint{4.506048in}{4.506048in}}%
\pgfusepath{clip}%
\pgfsetrectcap%
\pgfsetroundjoin%
\pgfsetlinewidth{0.501875pt}%
\definecolor{currentstroke}{rgb}{0.000000,0.000000,1.000000}%
\pgfsetstrokecolor{currentstroke}%
\pgfsetstrokeopacity{0.600000}%
\pgfsetdash{}{0pt}%
\pgfpathmoveto{\pgfqpoint{3.220557in}{3.255236in}}%
\pgfpathlineto{\pgfqpoint{3.203926in}{3.201912in}}%
\pgfusepath{stroke}%
\end{pgfscope}%
\begin{pgfscope}%
\pgfpathrectangle{\pgfqpoint{0.100000in}{0.183744in}}{\pgfqpoint{4.506048in}{4.506048in}}%
\pgfusepath{clip}%
\pgfsetrectcap%
\pgfsetroundjoin%
\pgfsetlinewidth{0.501875pt}%
\definecolor{currentstroke}{rgb}{0.000000,0.000000,1.000000}%
\pgfsetstrokecolor{currentstroke}%
\pgfsetstrokeopacity{0.600000}%
\pgfsetdash{}{0pt}%
\pgfpathmoveto{\pgfqpoint{2.918424in}{3.101572in}}%
\pgfpathlineto{\pgfqpoint{2.799694in}{3.121701in}}%
\pgfusepath{stroke}%
\end{pgfscope}%
\begin{pgfscope}%
\pgfpathrectangle{\pgfqpoint{0.100000in}{0.183744in}}{\pgfqpoint{4.506048in}{4.506048in}}%
\pgfusepath{clip}%
\pgfsetrectcap%
\pgfsetroundjoin%
\pgfsetlinewidth{0.501875pt}%
\definecolor{currentstroke}{rgb}{0.000000,0.000000,1.000000}%
\pgfsetstrokecolor{currentstroke}%
\pgfsetstrokeopacity{0.600000}%
\pgfsetdash{}{0pt}%
\pgfpathmoveto{\pgfqpoint{3.859487in}{2.435686in}}%
\pgfpathlineto{\pgfqpoint{3.677869in}{2.790007in}}%
\pgfusepath{stroke}%
\end{pgfscope}%
\begin{pgfscope}%
\pgfpathrectangle{\pgfqpoint{0.100000in}{0.183744in}}{\pgfqpoint{4.506048in}{4.506048in}}%
\pgfusepath{clip}%
\pgfsetrectcap%
\pgfsetroundjoin%
\pgfsetlinewidth{0.501875pt}%
\definecolor{currentstroke}{rgb}{0.000000,0.000000,1.000000}%
\pgfsetstrokecolor{currentstroke}%
\pgfsetstrokeopacity{0.600000}%
\pgfsetdash{}{0pt}%
\pgfpathmoveto{\pgfqpoint{2.905001in}{2.539891in}}%
\pgfpathlineto{\pgfqpoint{2.708007in}{2.539957in}}%
\pgfusepath{stroke}%
\end{pgfscope}%
\begin{pgfscope}%
\pgfpathrectangle{\pgfqpoint{0.100000in}{0.183744in}}{\pgfqpoint{4.506048in}{4.506048in}}%
\pgfusepath{clip}%
\pgfsetrectcap%
\pgfsetroundjoin%
\pgfsetlinewidth{0.501875pt}%
\definecolor{currentstroke}{rgb}{0.000000,0.000000,1.000000}%
\pgfsetstrokecolor{currentstroke}%
\pgfsetstrokeopacity{0.600000}%
\pgfsetdash{}{0pt}%
\pgfpathmoveto{\pgfqpoint{2.784631in}{1.565421in}}%
\pgfpathlineto{\pgfqpoint{2.832123in}{1.736244in}}%
\pgfusepath{stroke}%
\end{pgfscope}%
\begin{pgfscope}%
\pgfpathrectangle{\pgfqpoint{0.100000in}{0.183744in}}{\pgfqpoint{4.506048in}{4.506048in}}%
\pgfusepath{clip}%
\pgfsetrectcap%
\pgfsetroundjoin%
\pgfsetlinewidth{0.501875pt}%
\definecolor{currentstroke}{rgb}{0.000000,0.000000,1.000000}%
\pgfsetstrokecolor{currentstroke}%
\pgfsetstrokeopacity{0.600000}%
\pgfsetdash{}{0pt}%
\pgfpathmoveto{\pgfqpoint{3.354589in}{2.752584in}}%
\pgfpathlineto{\pgfqpoint{3.366545in}{2.985970in}}%
\pgfusepath{stroke}%
\end{pgfscope}%
\begin{pgfscope}%
\pgfpathrectangle{\pgfqpoint{0.100000in}{0.183744in}}{\pgfqpoint{4.506048in}{4.506048in}}%
\pgfusepath{clip}%
\pgfsetrectcap%
\pgfsetroundjoin%
\pgfsetlinewidth{0.501875pt}%
\definecolor{currentstroke}{rgb}{0.000000,0.000000,1.000000}%
\pgfsetstrokecolor{currentstroke}%
\pgfsetstrokeopacity{0.600000}%
\pgfsetdash{}{0pt}%
\pgfpathmoveto{\pgfqpoint{3.387859in}{3.984731in}}%
\pgfpathlineto{\pgfqpoint{3.339053in}{3.770390in}}%
\pgfusepath{stroke}%
\end{pgfscope}%
\begin{pgfscope}%
\pgfpathrectangle{\pgfqpoint{0.100000in}{0.183744in}}{\pgfqpoint{4.506048in}{4.506048in}}%
\pgfusepath{clip}%
\pgfsetrectcap%
\pgfsetroundjoin%
\pgfsetlinewidth{0.501875pt}%
\definecolor{currentstroke}{rgb}{0.000000,0.000000,1.000000}%
\pgfsetstrokecolor{currentstroke}%
\pgfsetstrokeopacity{0.600000}%
\pgfsetdash{}{0pt}%
\pgfpathmoveto{\pgfqpoint{2.418786in}{3.125140in}}%
\pgfpathlineto{\pgfqpoint{2.396345in}{2.878506in}}%
\pgfusepath{stroke}%
\end{pgfscope}%
\begin{pgfscope}%
\pgfpathrectangle{\pgfqpoint{0.100000in}{0.183744in}}{\pgfqpoint{4.506048in}{4.506048in}}%
\pgfusepath{clip}%
\pgfsetrectcap%
\pgfsetroundjoin%
\pgfsetlinewidth{0.501875pt}%
\definecolor{currentstroke}{rgb}{0.000000,0.000000,1.000000}%
\pgfsetstrokecolor{currentstroke}%
\pgfsetstrokeopacity{0.600000}%
\pgfsetdash{}{0pt}%
\pgfpathmoveto{\pgfqpoint{2.975957in}{1.598204in}}%
\pgfpathlineto{\pgfqpoint{2.898280in}{1.954221in}}%
\pgfusepath{stroke}%
\end{pgfscope}%
\begin{pgfscope}%
\pgfpathrectangle{\pgfqpoint{0.100000in}{0.183744in}}{\pgfqpoint{4.506048in}{4.506048in}}%
\pgfusepath{clip}%
\pgfsetrectcap%
\pgfsetroundjoin%
\pgfsetlinewidth{0.501875pt}%
\definecolor{currentstroke}{rgb}{0.000000,0.000000,1.000000}%
\pgfsetstrokecolor{currentstroke}%
\pgfsetstrokeopacity{0.600000}%
\pgfsetdash{}{0pt}%
\pgfpathmoveto{\pgfqpoint{1.063893in}{3.737299in}}%
\pgfpathlineto{\pgfqpoint{1.127180in}{3.536133in}}%
\pgfusepath{stroke}%
\end{pgfscope}%
\begin{pgfscope}%
\pgfpathrectangle{\pgfqpoint{0.100000in}{0.183744in}}{\pgfqpoint{4.506048in}{4.506048in}}%
\pgfusepath{clip}%
\pgfsetrectcap%
\pgfsetroundjoin%
\pgfsetlinewidth{0.501875pt}%
\definecolor{currentstroke}{rgb}{0.000000,0.000000,1.000000}%
\pgfsetstrokecolor{currentstroke}%
\pgfsetstrokeopacity{0.600000}%
\pgfsetdash{}{0pt}%
\pgfpathmoveto{\pgfqpoint{3.202468in}{2.184915in}}%
\pgfpathlineto{\pgfqpoint{3.164378in}{2.447728in}}%
\pgfusepath{stroke}%
\end{pgfscope}%
\begin{pgfscope}%
\pgfpathrectangle{\pgfqpoint{0.100000in}{0.183744in}}{\pgfqpoint{4.506048in}{4.506048in}}%
\pgfusepath{clip}%
\pgfsetrectcap%
\pgfsetroundjoin%
\pgfsetlinewidth{0.501875pt}%
\definecolor{currentstroke}{rgb}{0.000000,0.000000,1.000000}%
\pgfsetstrokecolor{currentstroke}%
\pgfsetstrokeopacity{0.600000}%
\pgfsetdash{}{0pt}%
\pgfpathmoveto{\pgfqpoint{2.046802in}{1.958148in}}%
\pgfpathlineto{\pgfqpoint{2.077328in}{1.545797in}}%
\pgfusepath{stroke}%
\end{pgfscope}%
\begin{pgfscope}%
\pgfpathrectangle{\pgfqpoint{0.100000in}{0.183744in}}{\pgfqpoint{4.506048in}{4.506048in}}%
\pgfusepath{clip}%
\pgfsetrectcap%
\pgfsetroundjoin%
\pgfsetlinewidth{0.501875pt}%
\definecolor{currentstroke}{rgb}{0.000000,0.000000,1.000000}%
\pgfsetstrokecolor{currentstroke}%
\pgfsetstrokeopacity{0.600000}%
\pgfsetdash{}{0pt}%
\pgfpathmoveto{\pgfqpoint{2.448206in}{1.290844in}}%
\pgfpathlineto{\pgfqpoint{2.429184in}{1.660978in}}%
\pgfusepath{stroke}%
\end{pgfscope}%
\begin{pgfscope}%
\pgfpathrectangle{\pgfqpoint{0.100000in}{0.183744in}}{\pgfqpoint{4.506048in}{4.506048in}}%
\pgfusepath{clip}%
\pgfsetrectcap%
\pgfsetroundjoin%
\pgfsetlinewidth{0.501875pt}%
\definecolor{currentstroke}{rgb}{0.000000,0.000000,1.000000}%
\pgfsetstrokecolor{currentstroke}%
\pgfsetstrokeopacity{0.600000}%
\pgfsetdash{}{0pt}%
\pgfpathmoveto{\pgfqpoint{2.602357in}{2.184097in}}%
\pgfpathlineto{\pgfqpoint{2.550528in}{2.402360in}}%
\pgfusepath{stroke}%
\end{pgfscope}%
\begin{pgfscope}%
\pgfpathrectangle{\pgfqpoint{0.100000in}{0.183744in}}{\pgfqpoint{4.506048in}{4.506048in}}%
\pgfusepath{clip}%
\pgfsetrectcap%
\pgfsetroundjoin%
\pgfsetlinewidth{0.501875pt}%
\definecolor{currentstroke}{rgb}{0.000000,0.000000,1.000000}%
\pgfsetstrokecolor{currentstroke}%
\pgfsetstrokeopacity{0.600000}%
\pgfsetdash{}{0pt}%
\pgfpathmoveto{\pgfqpoint{3.009214in}{2.043248in}}%
\pgfpathlineto{\pgfqpoint{2.962596in}{2.272919in}}%
\pgfusepath{stroke}%
\end{pgfscope}%
\begin{pgfscope}%
\pgfpathrectangle{\pgfqpoint{0.100000in}{0.183744in}}{\pgfqpoint{4.506048in}{4.506048in}}%
\pgfusepath{clip}%
\pgfsetrectcap%
\pgfsetroundjoin%
\pgfsetlinewidth{0.501875pt}%
\definecolor{currentstroke}{rgb}{0.000000,0.000000,1.000000}%
\pgfsetstrokecolor{currentstroke}%
\pgfsetstrokeopacity{0.600000}%
\pgfsetdash{}{0pt}%
\pgfpathmoveto{\pgfqpoint{2.637395in}{3.227359in}}%
\pgfpathlineto{\pgfqpoint{2.671730in}{3.224412in}}%
\pgfusepath{stroke}%
\end{pgfscope}%
\begin{pgfscope}%
\pgfpathrectangle{\pgfqpoint{0.100000in}{0.183744in}}{\pgfqpoint{4.506048in}{4.506048in}}%
\pgfusepath{clip}%
\pgfsetrectcap%
\pgfsetroundjoin%
\pgfsetlinewidth{0.501875pt}%
\definecolor{currentstroke}{rgb}{0.000000,0.000000,1.000000}%
\pgfsetstrokecolor{currentstroke}%
\pgfsetstrokeopacity{0.600000}%
\pgfsetdash{}{0pt}%
\pgfpathmoveto{\pgfqpoint{2.369209in}{2.325086in}}%
\pgfpathlineto{\pgfqpoint{2.291752in}{2.433634in}}%
\pgfusepath{stroke}%
\end{pgfscope}%
\begin{pgfscope}%
\pgfpathrectangle{\pgfqpoint{0.100000in}{0.183744in}}{\pgfqpoint{4.506048in}{4.506048in}}%
\pgfusepath{clip}%
\pgfsetrectcap%
\pgfsetroundjoin%
\pgfsetlinewidth{0.501875pt}%
\definecolor{currentstroke}{rgb}{0.000000,0.000000,1.000000}%
\pgfsetstrokecolor{currentstroke}%
\pgfsetstrokeopacity{0.600000}%
\pgfsetdash{}{0pt}%
\pgfpathmoveto{\pgfqpoint{3.496836in}{2.742682in}}%
\pgfpathlineto{\pgfqpoint{3.467689in}{2.854240in}}%
\pgfusepath{stroke}%
\end{pgfscope}%
\begin{pgfscope}%
\pgfpathrectangle{\pgfqpoint{0.100000in}{0.183744in}}{\pgfqpoint{4.506048in}{4.506048in}}%
\pgfusepath{clip}%
\pgfsetrectcap%
\pgfsetroundjoin%
\pgfsetlinewidth{0.501875pt}%
\definecolor{currentstroke}{rgb}{0.000000,0.000000,1.000000}%
\pgfsetstrokecolor{currentstroke}%
\pgfsetstrokeopacity{0.600000}%
\pgfsetdash{}{0pt}%
\pgfpathmoveto{\pgfqpoint{2.283366in}{2.814283in}}%
\pgfpathlineto{\pgfqpoint{2.444669in}{2.968276in}}%
\pgfusepath{stroke}%
\end{pgfscope}%
\begin{pgfscope}%
\pgfpathrectangle{\pgfqpoint{0.100000in}{0.183744in}}{\pgfqpoint{4.506048in}{4.506048in}}%
\pgfusepath{clip}%
\pgfsetrectcap%
\pgfsetroundjoin%
\pgfsetlinewidth{0.501875pt}%
\definecolor{currentstroke}{rgb}{0.000000,0.000000,1.000000}%
\pgfsetstrokecolor{currentstroke}%
\pgfsetstrokeopacity{0.600000}%
\pgfsetdash{}{0pt}%
\pgfpathmoveto{\pgfqpoint{2.566838in}{3.308596in}}%
\pgfpathlineto{\pgfqpoint{2.587608in}{3.296135in}}%
\pgfusepath{stroke}%
\end{pgfscope}%
\begin{pgfscope}%
\pgfpathrectangle{\pgfqpoint{0.100000in}{0.183744in}}{\pgfqpoint{4.506048in}{4.506048in}}%
\pgfusepath{clip}%
\pgfsetrectcap%
\pgfsetroundjoin%
\pgfsetlinewidth{0.501875pt}%
\definecolor{currentstroke}{rgb}{0.000000,0.000000,1.000000}%
\pgfsetstrokecolor{currentstroke}%
\pgfsetstrokeopacity{0.600000}%
\pgfsetdash{}{0pt}%
\pgfpathmoveto{\pgfqpoint{2.319649in}{2.602525in}}%
\pgfpathlineto{\pgfqpoint{2.398135in}{2.395647in}}%
\pgfusepath{stroke}%
\end{pgfscope}%
\begin{pgfscope}%
\pgfpathrectangle{\pgfqpoint{0.100000in}{0.183744in}}{\pgfqpoint{4.506048in}{4.506048in}}%
\pgfusepath{clip}%
\pgfsetrectcap%
\pgfsetroundjoin%
\pgfsetlinewidth{0.501875pt}%
\definecolor{currentstroke}{rgb}{0.000000,0.000000,1.000000}%
\pgfsetstrokecolor{currentstroke}%
\pgfsetstrokeopacity{0.600000}%
\pgfsetdash{}{0pt}%
\pgfpathmoveto{\pgfqpoint{1.710047in}{2.111733in}}%
\pgfpathlineto{\pgfqpoint{1.850041in}{2.228135in}}%
\pgfusepath{stroke}%
\end{pgfscope}%
\begin{pgfscope}%
\pgfpathrectangle{\pgfqpoint{0.100000in}{0.183744in}}{\pgfqpoint{4.506048in}{4.506048in}}%
\pgfusepath{clip}%
\pgfsetrectcap%
\pgfsetroundjoin%
\pgfsetlinewidth{0.501875pt}%
\definecolor{currentstroke}{rgb}{0.000000,0.000000,1.000000}%
\pgfsetstrokecolor{currentstroke}%
\pgfsetstrokeopacity{0.600000}%
\pgfsetdash{}{0pt}%
\pgfpathmoveto{\pgfqpoint{2.034724in}{2.475340in}}%
\pgfpathlineto{\pgfqpoint{1.920429in}{2.216874in}}%
\pgfusepath{stroke}%
\end{pgfscope}%
\begin{pgfscope}%
\pgfpathrectangle{\pgfqpoint{0.100000in}{0.183744in}}{\pgfqpoint{4.506048in}{4.506048in}}%
\pgfusepath{clip}%
\pgfsetrectcap%
\pgfsetroundjoin%
\pgfsetlinewidth{0.501875pt}%
\definecolor{currentstroke}{rgb}{0.000000,0.000000,1.000000}%
\pgfsetstrokecolor{currentstroke}%
\pgfsetstrokeopacity{0.600000}%
\pgfsetdash{}{0pt}%
\pgfpathmoveto{\pgfqpoint{1.979083in}{1.497561in}}%
\pgfpathlineto{\pgfqpoint{2.017681in}{1.576656in}}%
\pgfusepath{stroke}%
\end{pgfscope}%
\begin{pgfscope}%
\pgfpathrectangle{\pgfqpoint{0.100000in}{0.183744in}}{\pgfqpoint{4.506048in}{4.506048in}}%
\pgfusepath{clip}%
\pgfsetrectcap%
\pgfsetroundjoin%
\pgfsetlinewidth{0.501875pt}%
\definecolor{currentstroke}{rgb}{0.000000,0.000000,1.000000}%
\pgfsetstrokecolor{currentstroke}%
\pgfsetstrokeopacity{0.600000}%
\pgfsetdash{}{0pt}%
\pgfpathmoveto{\pgfqpoint{3.314814in}{1.338473in}}%
\pgfpathlineto{\pgfqpoint{3.308468in}{1.668822in}}%
\pgfusepath{stroke}%
\end{pgfscope}%
\begin{pgfscope}%
\pgfpathrectangle{\pgfqpoint{0.100000in}{0.183744in}}{\pgfqpoint{4.506048in}{4.506048in}}%
\pgfusepath{clip}%
\pgfsetrectcap%
\pgfsetroundjoin%
\pgfsetlinewidth{0.501875pt}%
\definecolor{currentstroke}{rgb}{0.000000,0.000000,1.000000}%
\pgfsetstrokecolor{currentstroke}%
\pgfsetstrokeopacity{0.600000}%
\pgfsetdash{}{0pt}%
\pgfpathmoveto{\pgfqpoint{2.684517in}{1.710212in}}%
\pgfpathlineto{\pgfqpoint{2.798239in}{1.990482in}}%
\pgfusepath{stroke}%
\end{pgfscope}%
\begin{pgfscope}%
\pgfpathrectangle{\pgfqpoint{0.100000in}{0.183744in}}{\pgfqpoint{4.506048in}{4.506048in}}%
\pgfusepath{clip}%
\pgfsetrectcap%
\pgfsetroundjoin%
\pgfsetlinewidth{0.501875pt}%
\definecolor{currentstroke}{rgb}{0.000000,0.000000,1.000000}%
\pgfsetstrokecolor{currentstroke}%
\pgfsetstrokeopacity{0.600000}%
\pgfsetdash{}{0pt}%
\pgfpathmoveto{\pgfqpoint{1.592524in}{2.410611in}}%
\pgfpathlineto{\pgfqpoint{1.608338in}{2.351743in}}%
\pgfusepath{stroke}%
\end{pgfscope}%
\begin{pgfscope}%
\pgfpathrectangle{\pgfqpoint{0.100000in}{0.183744in}}{\pgfqpoint{4.506048in}{4.506048in}}%
\pgfusepath{clip}%
\pgfsetrectcap%
\pgfsetroundjoin%
\pgfsetlinewidth{0.501875pt}%
\definecolor{currentstroke}{rgb}{0.000000,0.000000,1.000000}%
\pgfsetstrokecolor{currentstroke}%
\pgfsetstrokeopacity{0.600000}%
\pgfsetdash{}{0pt}%
\pgfpathmoveto{\pgfqpoint{3.089248in}{2.349653in}}%
\pgfpathlineto{\pgfqpoint{3.069111in}{2.579901in}}%
\pgfusepath{stroke}%
\end{pgfscope}%
\begin{pgfscope}%
\pgfpathrectangle{\pgfqpoint{0.100000in}{0.183744in}}{\pgfqpoint{4.506048in}{4.506048in}}%
\pgfusepath{clip}%
\pgfsetrectcap%
\pgfsetroundjoin%
\pgfsetlinewidth{0.501875pt}%
\definecolor{currentstroke}{rgb}{0.000000,0.000000,1.000000}%
\pgfsetstrokecolor{currentstroke}%
\pgfsetstrokeopacity{0.600000}%
\pgfsetdash{}{0pt}%
\pgfpathmoveto{\pgfqpoint{3.723464in}{3.015199in}}%
\pgfpathlineto{\pgfqpoint{3.648325in}{3.094571in}}%
\pgfusepath{stroke}%
\end{pgfscope}%
\begin{pgfscope}%
\pgfpathrectangle{\pgfqpoint{0.100000in}{0.183744in}}{\pgfqpoint{4.506048in}{4.506048in}}%
\pgfusepath{clip}%
\pgfsetrectcap%
\pgfsetroundjoin%
\pgfsetlinewidth{0.501875pt}%
\definecolor{currentstroke}{rgb}{0.000000,0.000000,1.000000}%
\pgfsetstrokecolor{currentstroke}%
\pgfsetstrokeopacity{0.600000}%
\pgfsetdash{}{0pt}%
\pgfpathmoveto{\pgfqpoint{2.546754in}{2.364749in}}%
\pgfpathlineto{\pgfqpoint{2.588555in}{2.210685in}}%
\pgfusepath{stroke}%
\end{pgfscope}%
\begin{pgfscope}%
\pgfpathrectangle{\pgfqpoint{0.100000in}{0.183744in}}{\pgfqpoint{4.506048in}{4.506048in}}%
\pgfusepath{clip}%
\pgfsetrectcap%
\pgfsetroundjoin%
\pgfsetlinewidth{0.501875pt}%
\definecolor{currentstroke}{rgb}{0.000000,0.000000,1.000000}%
\pgfsetstrokecolor{currentstroke}%
\pgfsetstrokeopacity{0.600000}%
\pgfsetdash{}{0pt}%
\pgfpathmoveto{\pgfqpoint{2.356521in}{1.494659in}}%
\pgfpathlineto{\pgfqpoint{2.358049in}{1.697883in}}%
\pgfusepath{stroke}%
\end{pgfscope}%
\begin{pgfscope}%
\pgfpathrectangle{\pgfqpoint{0.100000in}{0.183744in}}{\pgfqpoint{4.506048in}{4.506048in}}%
\pgfusepath{clip}%
\pgfsetrectcap%
\pgfsetroundjoin%
\pgfsetlinewidth{0.501875pt}%
\definecolor{currentstroke}{rgb}{0.000000,0.000000,1.000000}%
\pgfsetstrokecolor{currentstroke}%
\pgfsetstrokeopacity{0.600000}%
\pgfsetdash{}{0pt}%
\pgfpathmoveto{\pgfqpoint{1.765091in}{1.462528in}}%
\pgfpathlineto{\pgfqpoint{1.631065in}{1.431392in}}%
\pgfusepath{stroke}%
\end{pgfscope}%
\begin{pgfscope}%
\pgfpathrectangle{\pgfqpoint{0.100000in}{0.183744in}}{\pgfqpoint{4.506048in}{4.506048in}}%
\pgfusepath{clip}%
\pgfsetrectcap%
\pgfsetroundjoin%
\pgfsetlinewidth{0.501875pt}%
\definecolor{currentstroke}{rgb}{0.000000,0.000000,1.000000}%
\pgfsetstrokecolor{currentstroke}%
\pgfsetstrokeopacity{0.600000}%
\pgfsetdash{}{0pt}%
\pgfpathmoveto{\pgfqpoint{3.025651in}{1.998203in}}%
\pgfpathlineto{\pgfqpoint{2.880677in}{1.866833in}}%
\pgfusepath{stroke}%
\end{pgfscope}%
\begin{pgfscope}%
\pgfpathrectangle{\pgfqpoint{0.100000in}{0.183744in}}{\pgfqpoint{4.506048in}{4.506048in}}%
\pgfusepath{clip}%
\pgfsetrectcap%
\pgfsetroundjoin%
\pgfsetlinewidth{0.501875pt}%
\definecolor{currentstroke}{rgb}{0.000000,0.000000,1.000000}%
\pgfsetstrokecolor{currentstroke}%
\pgfsetstrokeopacity{0.600000}%
\pgfsetdash{}{0pt}%
\pgfpathmoveto{\pgfqpoint{2.355586in}{3.919439in}}%
\pgfpathlineto{\pgfqpoint{2.163356in}{3.652285in}}%
\pgfusepath{stroke}%
\end{pgfscope}%
\begin{pgfscope}%
\pgfpathrectangle{\pgfqpoint{0.100000in}{0.183744in}}{\pgfqpoint{4.506048in}{4.506048in}}%
\pgfusepath{clip}%
\pgfsetrectcap%
\pgfsetroundjoin%
\pgfsetlinewidth{0.501875pt}%
\definecolor{currentstroke}{rgb}{0.000000,0.000000,1.000000}%
\pgfsetstrokecolor{currentstroke}%
\pgfsetstrokeopacity{0.600000}%
\pgfsetdash{}{0pt}%
\pgfpathmoveto{\pgfqpoint{3.144413in}{1.832274in}}%
\pgfpathlineto{\pgfqpoint{3.153427in}{2.025026in}}%
\pgfusepath{stroke}%
\end{pgfscope}%
\begin{pgfscope}%
\pgfpathrectangle{\pgfqpoint{0.100000in}{0.183744in}}{\pgfqpoint{4.506048in}{4.506048in}}%
\pgfusepath{clip}%
\pgfsetrectcap%
\pgfsetroundjoin%
\pgfsetlinewidth{0.501875pt}%
\definecolor{currentstroke}{rgb}{0.000000,0.000000,1.000000}%
\pgfsetstrokecolor{currentstroke}%
\pgfsetstrokeopacity{0.600000}%
\pgfsetdash{}{0pt}%
\pgfpathmoveto{\pgfqpoint{1.215870in}{3.239356in}}%
\pgfpathlineto{\pgfqpoint{1.197433in}{2.950635in}}%
\pgfusepath{stroke}%
\end{pgfscope}%
\begin{pgfscope}%
\pgfpathrectangle{\pgfqpoint{0.100000in}{0.183744in}}{\pgfqpoint{4.506048in}{4.506048in}}%
\pgfusepath{clip}%
\pgfsetrectcap%
\pgfsetroundjoin%
\pgfsetlinewidth{0.501875pt}%
\definecolor{currentstroke}{rgb}{0.000000,0.000000,1.000000}%
\pgfsetstrokecolor{currentstroke}%
\pgfsetstrokeopacity{0.600000}%
\pgfsetdash{}{0pt}%
\pgfpathmoveto{\pgfqpoint{3.497564in}{3.210938in}}%
\pgfpathlineto{\pgfqpoint{3.335200in}{3.189733in}}%
\pgfusepath{stroke}%
\end{pgfscope}%
\begin{pgfscope}%
\pgfpathrectangle{\pgfqpoint{0.100000in}{0.183744in}}{\pgfqpoint{4.506048in}{4.506048in}}%
\pgfusepath{clip}%
\pgfsetrectcap%
\pgfsetroundjoin%
\pgfsetlinewidth{0.501875pt}%
\definecolor{currentstroke}{rgb}{0.000000,0.000000,1.000000}%
\pgfsetstrokecolor{currentstroke}%
\pgfsetstrokeopacity{0.600000}%
\pgfsetdash{}{0pt}%
\pgfpathmoveto{\pgfqpoint{3.600028in}{1.163008in}}%
\pgfpathlineto{\pgfqpoint{3.583217in}{1.521614in}}%
\pgfusepath{stroke}%
\end{pgfscope}%
\begin{pgfscope}%
\pgfpathrectangle{\pgfqpoint{0.100000in}{0.183744in}}{\pgfqpoint{4.506048in}{4.506048in}}%
\pgfusepath{clip}%
\pgfsetrectcap%
\pgfsetroundjoin%
\pgfsetlinewidth{0.501875pt}%
\definecolor{currentstroke}{rgb}{0.000000,0.000000,1.000000}%
\pgfsetstrokecolor{currentstroke}%
\pgfsetstrokeopacity{0.600000}%
\pgfsetdash{}{0pt}%
\pgfpathmoveto{\pgfqpoint{3.575504in}{2.776603in}}%
\pgfpathlineto{\pgfqpoint{3.384821in}{2.744755in}}%
\pgfusepath{stroke}%
\end{pgfscope}%
\begin{pgfscope}%
\pgfpathrectangle{\pgfqpoint{0.100000in}{0.183744in}}{\pgfqpoint{4.506048in}{4.506048in}}%
\pgfusepath{clip}%
\pgfsetrectcap%
\pgfsetroundjoin%
\pgfsetlinewidth{0.501875pt}%
\definecolor{currentstroke}{rgb}{0.000000,0.000000,1.000000}%
\pgfsetstrokecolor{currentstroke}%
\pgfsetstrokeopacity{0.600000}%
\pgfsetdash{}{0pt}%
\pgfpathmoveto{\pgfqpoint{1.862237in}{2.335760in}}%
\pgfpathlineto{\pgfqpoint{1.860499in}{2.071236in}}%
\pgfusepath{stroke}%
\end{pgfscope}%
\begin{pgfscope}%
\pgfpathrectangle{\pgfqpoint{0.100000in}{0.183744in}}{\pgfqpoint{4.506048in}{4.506048in}}%
\pgfusepath{clip}%
\pgfsetrectcap%
\pgfsetroundjoin%
\pgfsetlinewidth{0.501875pt}%
\definecolor{currentstroke}{rgb}{0.000000,0.000000,1.000000}%
\pgfsetstrokecolor{currentstroke}%
\pgfsetstrokeopacity{0.600000}%
\pgfsetdash{}{0pt}%
\pgfpathmoveto{\pgfqpoint{1.080998in}{2.753279in}}%
\pgfpathlineto{\pgfqpoint{1.224986in}{2.569400in}}%
\pgfusepath{stroke}%
\end{pgfscope}%
\begin{pgfscope}%
\pgfpathrectangle{\pgfqpoint{0.100000in}{0.183744in}}{\pgfqpoint{4.506048in}{4.506048in}}%
\pgfusepath{clip}%
\pgfsetrectcap%
\pgfsetroundjoin%
\pgfsetlinewidth{0.501875pt}%
\definecolor{currentstroke}{rgb}{0.000000,0.000000,1.000000}%
\pgfsetstrokecolor{currentstroke}%
\pgfsetstrokeopacity{0.600000}%
\pgfsetdash{}{0pt}%
\pgfpathmoveto{\pgfqpoint{2.743317in}{3.598096in}}%
\pgfpathlineto{\pgfqpoint{2.563471in}{3.464210in}}%
\pgfusepath{stroke}%
\end{pgfscope}%
\begin{pgfscope}%
\pgfpathrectangle{\pgfqpoint{0.100000in}{0.183744in}}{\pgfqpoint{4.506048in}{4.506048in}}%
\pgfusepath{clip}%
\pgfsetrectcap%
\pgfsetroundjoin%
\pgfsetlinewidth{0.501875pt}%
\definecolor{currentstroke}{rgb}{0.000000,0.000000,1.000000}%
\pgfsetstrokecolor{currentstroke}%
\pgfsetstrokeopacity{0.600000}%
\pgfsetdash{}{0pt}%
\pgfpathmoveto{\pgfqpoint{2.688627in}{1.464254in}}%
\pgfpathlineto{\pgfqpoint{2.608903in}{1.710199in}}%
\pgfusepath{stroke}%
\end{pgfscope}%
\begin{pgfscope}%
\pgfpathrectangle{\pgfqpoint{0.100000in}{0.183744in}}{\pgfqpoint{4.506048in}{4.506048in}}%
\pgfusepath{clip}%
\pgfsetrectcap%
\pgfsetroundjoin%
\pgfsetlinewidth{0.501875pt}%
\definecolor{currentstroke}{rgb}{0.000000,0.000000,1.000000}%
\pgfsetstrokecolor{currentstroke}%
\pgfsetstrokeopacity{0.600000}%
\pgfsetdash{}{0pt}%
\pgfpathmoveto{\pgfqpoint{1.857783in}{3.313120in}}%
\pgfpathlineto{\pgfqpoint{1.736277in}{3.069803in}}%
\pgfusepath{stroke}%
\end{pgfscope}%
\begin{pgfscope}%
\pgfpathrectangle{\pgfqpoint{0.100000in}{0.183744in}}{\pgfqpoint{4.506048in}{4.506048in}}%
\pgfusepath{clip}%
\pgfsetrectcap%
\pgfsetroundjoin%
\pgfsetlinewidth{0.501875pt}%
\definecolor{currentstroke}{rgb}{0.000000,0.000000,1.000000}%
\pgfsetstrokecolor{currentstroke}%
\pgfsetstrokeopacity{0.600000}%
\pgfsetdash{}{0pt}%
\pgfpathmoveto{\pgfqpoint{1.865582in}{0.695321in}}%
\pgfpathlineto{\pgfqpoint{1.916740in}{0.455455in}}%
\pgfusepath{stroke}%
\end{pgfscope}%
\begin{pgfscope}%
\pgfpathrectangle{\pgfqpoint{0.100000in}{0.183744in}}{\pgfqpoint{4.506048in}{4.506048in}}%
\pgfusepath{clip}%
\pgfsetrectcap%
\pgfsetroundjoin%
\pgfsetlinewidth{0.501875pt}%
\definecolor{currentstroke}{rgb}{0.000000,0.000000,1.000000}%
\pgfsetstrokecolor{currentstroke}%
\pgfsetstrokeopacity{0.600000}%
\pgfsetdash{}{0pt}%
\pgfpathmoveto{\pgfqpoint{2.420550in}{2.552221in}}%
\pgfpathlineto{\pgfqpoint{2.591809in}{2.669299in}}%
\pgfusepath{stroke}%
\end{pgfscope}%
\begin{pgfscope}%
\pgfpathrectangle{\pgfqpoint{0.100000in}{0.183744in}}{\pgfqpoint{4.506048in}{4.506048in}}%
\pgfusepath{clip}%
\pgfsetrectcap%
\pgfsetroundjoin%
\pgfsetlinewidth{0.501875pt}%
\definecolor{currentstroke}{rgb}{0.000000,0.000000,1.000000}%
\pgfsetstrokecolor{currentstroke}%
\pgfsetstrokeopacity{0.600000}%
\pgfsetdash{}{0pt}%
\pgfpathmoveto{\pgfqpoint{1.819288in}{3.473179in}}%
\pgfpathlineto{\pgfqpoint{1.769623in}{3.221308in}}%
\pgfusepath{stroke}%
\end{pgfscope}%
\begin{pgfscope}%
\pgfpathrectangle{\pgfqpoint{0.100000in}{0.183744in}}{\pgfqpoint{4.506048in}{4.506048in}}%
\pgfusepath{clip}%
\pgfsetrectcap%
\pgfsetroundjoin%
\pgfsetlinewidth{0.501875pt}%
\definecolor{currentstroke}{rgb}{0.000000,0.000000,1.000000}%
\pgfsetstrokecolor{currentstroke}%
\pgfsetstrokeopacity{0.600000}%
\pgfsetdash{}{0pt}%
\pgfpathmoveto{\pgfqpoint{0.909862in}{2.325240in}}%
\pgfpathlineto{\pgfqpoint{1.069523in}{2.202253in}}%
\pgfusepath{stroke}%
\end{pgfscope}%
\begin{pgfscope}%
\pgfpathrectangle{\pgfqpoint{0.100000in}{0.183744in}}{\pgfqpoint{4.506048in}{4.506048in}}%
\pgfusepath{clip}%
\pgfsetrectcap%
\pgfsetroundjoin%
\pgfsetlinewidth{0.501875pt}%
\definecolor{currentstroke}{rgb}{0.000000,0.000000,1.000000}%
\pgfsetstrokecolor{currentstroke}%
\pgfsetstrokeopacity{0.600000}%
\pgfsetdash{}{0pt}%
\pgfpathmoveto{\pgfqpoint{3.896460in}{2.537935in}}%
\pgfpathlineto{\pgfqpoint{3.872251in}{2.092009in}}%
\pgfusepath{stroke}%
\end{pgfscope}%
\begin{pgfscope}%
\pgfpathrectangle{\pgfqpoint{0.100000in}{0.183744in}}{\pgfqpoint{4.506048in}{4.506048in}}%
\pgfusepath{clip}%
\pgfsetrectcap%
\pgfsetroundjoin%
\pgfsetlinewidth{0.501875pt}%
\definecolor{currentstroke}{rgb}{0.000000,0.000000,1.000000}%
\pgfsetstrokecolor{currentstroke}%
\pgfsetstrokeopacity{0.600000}%
\pgfsetdash{}{0pt}%
\pgfpathmoveto{\pgfqpoint{0.729120in}{2.851069in}}%
\pgfpathlineto{\pgfqpoint{0.897525in}{2.722275in}}%
\pgfusepath{stroke}%
\end{pgfscope}%
\begin{pgfscope}%
\pgfpathrectangle{\pgfqpoint{0.100000in}{0.183744in}}{\pgfqpoint{4.506048in}{4.506048in}}%
\pgfusepath{clip}%
\pgfsetrectcap%
\pgfsetroundjoin%
\pgfsetlinewidth{0.501875pt}%
\definecolor{currentstroke}{rgb}{0.000000,0.000000,1.000000}%
\pgfsetstrokecolor{currentstroke}%
\pgfsetstrokeopacity{0.600000}%
\pgfsetdash{}{0pt}%
\pgfpathmoveto{\pgfqpoint{2.718882in}{2.945892in}}%
\pgfpathlineto{\pgfqpoint{2.828266in}{2.976421in}}%
\pgfusepath{stroke}%
\end{pgfscope}%
\begin{pgfscope}%
\pgfpathrectangle{\pgfqpoint{0.100000in}{0.183744in}}{\pgfqpoint{4.506048in}{4.506048in}}%
\pgfusepath{clip}%
\pgfsetrectcap%
\pgfsetroundjoin%
\pgfsetlinewidth{0.501875pt}%
\definecolor{currentstroke}{rgb}{0.000000,0.000000,1.000000}%
\pgfsetstrokecolor{currentstroke}%
\pgfsetstrokeopacity{0.600000}%
\pgfsetdash{}{0pt}%
\pgfpathmoveto{\pgfqpoint{1.547707in}{2.971423in}}%
\pgfpathlineto{\pgfqpoint{1.597710in}{2.887529in}}%
\pgfusepath{stroke}%
\end{pgfscope}%
\begin{pgfscope}%
\pgfpathrectangle{\pgfqpoint{0.100000in}{0.183744in}}{\pgfqpoint{4.506048in}{4.506048in}}%
\pgfusepath{clip}%
\pgfsetrectcap%
\pgfsetroundjoin%
\pgfsetlinewidth{0.501875pt}%
\definecolor{currentstroke}{rgb}{0.000000,0.000000,1.000000}%
\pgfsetstrokecolor{currentstroke}%
\pgfsetstrokeopacity{0.600000}%
\pgfsetdash{}{0pt}%
\pgfpathmoveto{\pgfqpoint{1.996250in}{2.590251in}}%
\pgfpathlineto{\pgfqpoint{2.203156in}{2.728876in}}%
\pgfusepath{stroke}%
\end{pgfscope}%
\begin{pgfscope}%
\pgfpathrectangle{\pgfqpoint{0.100000in}{0.183744in}}{\pgfqpoint{4.506048in}{4.506048in}}%
\pgfusepath{clip}%
\pgfsetrectcap%
\pgfsetroundjoin%
\pgfsetlinewidth{0.501875pt}%
\definecolor{currentstroke}{rgb}{0.000000,0.000000,1.000000}%
\pgfsetstrokecolor{currentstroke}%
\pgfsetstrokeopacity{0.600000}%
\pgfsetdash{}{0pt}%
\pgfpathmoveto{\pgfqpoint{2.984592in}{1.301709in}}%
\pgfpathlineto{\pgfqpoint{2.892803in}{1.676392in}}%
\pgfusepath{stroke}%
\end{pgfscope}%
\begin{pgfscope}%
\pgfpathrectangle{\pgfqpoint{0.100000in}{0.183744in}}{\pgfqpoint{4.506048in}{4.506048in}}%
\pgfusepath{clip}%
\pgfsetrectcap%
\pgfsetroundjoin%
\pgfsetlinewidth{0.501875pt}%
\definecolor{currentstroke}{rgb}{0.000000,0.000000,1.000000}%
\pgfsetstrokecolor{currentstroke}%
\pgfsetstrokeopacity{0.600000}%
\pgfsetdash{}{0pt}%
\pgfpathmoveto{\pgfqpoint{2.875247in}{3.380560in}}%
\pgfpathlineto{\pgfqpoint{2.798550in}{3.354864in}}%
\pgfusepath{stroke}%
\end{pgfscope}%
\begin{pgfscope}%
\pgfpathrectangle{\pgfqpoint{0.100000in}{0.183744in}}{\pgfqpoint{4.506048in}{4.506048in}}%
\pgfusepath{clip}%
\pgfsetrectcap%
\pgfsetroundjoin%
\pgfsetlinewidth{0.501875pt}%
\definecolor{currentstroke}{rgb}{0.000000,0.000000,1.000000}%
\pgfsetstrokecolor{currentstroke}%
\pgfsetstrokeopacity{0.600000}%
\pgfsetdash{}{0pt}%
\pgfpathmoveto{\pgfqpoint{2.788190in}{2.476357in}}%
\pgfpathlineto{\pgfqpoint{2.751168in}{2.557249in}}%
\pgfusepath{stroke}%
\end{pgfscope}%
\begin{pgfscope}%
\pgfpathrectangle{\pgfqpoint{0.100000in}{0.183744in}}{\pgfqpoint{4.506048in}{4.506048in}}%
\pgfusepath{clip}%
\pgfsetrectcap%
\pgfsetroundjoin%
\pgfsetlinewidth{0.501875pt}%
\definecolor{currentstroke}{rgb}{0.000000,0.000000,1.000000}%
\pgfsetstrokecolor{currentstroke}%
\pgfsetstrokeopacity{0.600000}%
\pgfsetdash{}{0pt}%
\pgfpathmoveto{\pgfqpoint{3.569293in}{2.966261in}}%
\pgfpathlineto{\pgfqpoint{3.814953in}{3.075842in}}%
\pgfusepath{stroke}%
\end{pgfscope}%
\begin{pgfscope}%
\pgfpathrectangle{\pgfqpoint{0.100000in}{0.183744in}}{\pgfqpoint{4.506048in}{4.506048in}}%
\pgfusepath{clip}%
\pgfsetrectcap%
\pgfsetroundjoin%
\pgfsetlinewidth{0.501875pt}%
\definecolor{currentstroke}{rgb}{0.000000,0.000000,1.000000}%
\pgfsetstrokecolor{currentstroke}%
\pgfsetstrokeopacity{0.600000}%
\pgfsetdash{}{0pt}%
\pgfpathmoveto{\pgfqpoint{2.401315in}{1.288591in}}%
\pgfpathlineto{\pgfqpoint{2.311425in}{0.943272in}}%
\pgfusepath{stroke}%
\end{pgfscope}%
\begin{pgfscope}%
\pgfpathrectangle{\pgfqpoint{0.100000in}{0.183744in}}{\pgfqpoint{4.506048in}{4.506048in}}%
\pgfusepath{clip}%
\pgfsetrectcap%
\pgfsetroundjoin%
\pgfsetlinewidth{0.501875pt}%
\definecolor{currentstroke}{rgb}{0.000000,0.000000,1.000000}%
\pgfsetstrokecolor{currentstroke}%
\pgfsetstrokeopacity{0.600000}%
\pgfsetdash{}{0pt}%
\pgfpathmoveto{\pgfqpoint{2.070991in}{2.290743in}}%
\pgfpathlineto{\pgfqpoint{2.177144in}{2.368746in}}%
\pgfusepath{stroke}%
\end{pgfscope}%
\begin{pgfscope}%
\pgfpathrectangle{\pgfqpoint{0.100000in}{0.183744in}}{\pgfqpoint{4.506048in}{4.506048in}}%
\pgfusepath{clip}%
\pgfsetrectcap%
\pgfsetroundjoin%
\pgfsetlinewidth{0.501875pt}%
\definecolor{currentstroke}{rgb}{0.000000,0.000000,1.000000}%
\pgfsetstrokecolor{currentstroke}%
\pgfsetstrokeopacity{0.600000}%
\pgfsetdash{}{0pt}%
\pgfpathmoveto{\pgfqpoint{3.206702in}{1.279105in}}%
\pgfpathlineto{\pgfqpoint{3.327020in}{1.725809in}}%
\pgfusepath{stroke}%
\end{pgfscope}%
\begin{pgfscope}%
\pgfpathrectangle{\pgfqpoint{0.100000in}{0.183744in}}{\pgfqpoint{4.506048in}{4.506048in}}%
\pgfusepath{clip}%
\pgfsetrectcap%
\pgfsetroundjoin%
\pgfsetlinewidth{0.501875pt}%
\definecolor{currentstroke}{rgb}{0.000000,0.000000,1.000000}%
\pgfsetstrokecolor{currentstroke}%
\pgfsetstrokeopacity{0.600000}%
\pgfsetdash{}{0pt}%
\pgfpathmoveto{\pgfqpoint{3.543296in}{2.295968in}}%
\pgfpathlineto{\pgfqpoint{3.661402in}{2.703381in}}%
\pgfusepath{stroke}%
\end{pgfscope}%
\begin{pgfscope}%
\pgfpathrectangle{\pgfqpoint{0.100000in}{0.183744in}}{\pgfqpoint{4.506048in}{4.506048in}}%
\pgfusepath{clip}%
\pgfsetrectcap%
\pgfsetroundjoin%
\pgfsetlinewidth{0.501875pt}%
\definecolor{currentstroke}{rgb}{0.000000,0.000000,1.000000}%
\pgfsetstrokecolor{currentstroke}%
\pgfsetstrokeopacity{0.600000}%
\pgfsetdash{}{0pt}%
\pgfpathmoveto{\pgfqpoint{2.658269in}{2.103897in}}%
\pgfpathlineto{\pgfqpoint{2.519802in}{1.830723in}}%
\pgfusepath{stroke}%
\end{pgfscope}%
\begin{pgfscope}%
\pgfpathrectangle{\pgfqpoint{0.100000in}{0.183744in}}{\pgfqpoint{4.506048in}{4.506048in}}%
\pgfusepath{clip}%
\pgfsetrectcap%
\pgfsetroundjoin%
\pgfsetlinewidth{0.501875pt}%
\definecolor{currentstroke}{rgb}{0.000000,0.000000,1.000000}%
\pgfsetstrokecolor{currentstroke}%
\pgfsetstrokeopacity{0.600000}%
\pgfsetdash{}{0pt}%
\pgfpathmoveto{\pgfqpoint{2.491029in}{3.054423in}}%
\pgfpathlineto{\pgfqpoint{2.574087in}{3.083809in}}%
\pgfusepath{stroke}%
\end{pgfscope}%
\begin{pgfscope}%
\pgfpathrectangle{\pgfqpoint{0.100000in}{0.183744in}}{\pgfqpoint{4.506048in}{4.506048in}}%
\pgfusepath{clip}%
\pgfsetrectcap%
\pgfsetroundjoin%
\pgfsetlinewidth{0.501875pt}%
\definecolor{currentstroke}{rgb}{0.000000,0.000000,1.000000}%
\pgfsetstrokecolor{currentstroke}%
\pgfsetstrokeopacity{0.600000}%
\pgfsetdash{}{0pt}%
\pgfpathmoveto{\pgfqpoint{3.212820in}{2.159101in}}%
\pgfpathlineto{\pgfqpoint{3.257569in}{2.538545in}}%
\pgfusepath{stroke}%
\end{pgfscope}%
\begin{pgfscope}%
\pgfpathrectangle{\pgfqpoint{0.100000in}{0.183744in}}{\pgfqpoint{4.506048in}{4.506048in}}%
\pgfusepath{clip}%
\pgfsetrectcap%
\pgfsetroundjoin%
\pgfsetlinewidth{0.501875pt}%
\definecolor{currentstroke}{rgb}{0.000000,0.000000,1.000000}%
\pgfsetstrokecolor{currentstroke}%
\pgfsetstrokeopacity{0.600000}%
\pgfsetdash{}{0pt}%
\pgfpathmoveto{\pgfqpoint{3.106414in}{2.548365in}}%
\pgfpathlineto{\pgfqpoint{3.438312in}{2.868125in}}%
\pgfusepath{stroke}%
\end{pgfscope}%
\begin{pgfscope}%
\pgfpathrectangle{\pgfqpoint{0.100000in}{0.183744in}}{\pgfqpoint{4.506048in}{4.506048in}}%
\pgfusepath{clip}%
\pgfsetrectcap%
\pgfsetroundjoin%
\pgfsetlinewidth{0.501875pt}%
\definecolor{currentstroke}{rgb}{0.000000,0.000000,1.000000}%
\pgfsetstrokecolor{currentstroke}%
\pgfsetstrokeopacity{0.600000}%
\pgfsetdash{}{0pt}%
\pgfpathmoveto{\pgfqpoint{2.753504in}{1.646013in}}%
\pgfpathlineto{\pgfqpoint{2.730005in}{1.843647in}}%
\pgfusepath{stroke}%
\end{pgfscope}%
\begin{pgfscope}%
\pgfpathrectangle{\pgfqpoint{0.100000in}{0.183744in}}{\pgfqpoint{4.506048in}{4.506048in}}%
\pgfusepath{clip}%
\pgfsetrectcap%
\pgfsetroundjoin%
\pgfsetlinewidth{0.501875pt}%
\definecolor{currentstroke}{rgb}{0.000000,0.000000,1.000000}%
\pgfsetstrokecolor{currentstroke}%
\pgfsetstrokeopacity{0.600000}%
\pgfsetdash{}{0pt}%
\pgfpathmoveto{\pgfqpoint{1.114630in}{2.256585in}}%
\pgfpathlineto{\pgfqpoint{1.111246in}{2.226292in}}%
\pgfusepath{stroke}%
\end{pgfscope}%
\begin{pgfscope}%
\pgfpathrectangle{\pgfqpoint{0.100000in}{0.183744in}}{\pgfqpoint{4.506048in}{4.506048in}}%
\pgfusepath{clip}%
\pgfsetrectcap%
\pgfsetroundjoin%
\pgfsetlinewidth{0.501875pt}%
\definecolor{currentstroke}{rgb}{0.000000,0.000000,1.000000}%
\pgfsetstrokecolor{currentstroke}%
\pgfsetstrokeopacity{0.600000}%
\pgfsetdash{}{0pt}%
\pgfpathmoveto{\pgfqpoint{1.829814in}{3.492603in}}%
\pgfpathlineto{\pgfqpoint{1.890869in}{3.280626in}}%
\pgfusepath{stroke}%
\end{pgfscope}%
\begin{pgfscope}%
\pgfpathrectangle{\pgfqpoint{0.100000in}{0.183744in}}{\pgfqpoint{4.506048in}{4.506048in}}%
\pgfusepath{clip}%
\pgfsetrectcap%
\pgfsetroundjoin%
\pgfsetlinewidth{0.501875pt}%
\definecolor{currentstroke}{rgb}{0.000000,0.000000,1.000000}%
\pgfsetstrokecolor{currentstroke}%
\pgfsetstrokeopacity{0.600000}%
\pgfsetdash{}{0pt}%
\pgfpathmoveto{\pgfqpoint{2.438160in}{2.408027in}}%
\pgfpathlineto{\pgfqpoint{2.299947in}{2.117586in}}%
\pgfusepath{stroke}%
\end{pgfscope}%
\begin{pgfscope}%
\pgfpathrectangle{\pgfqpoint{0.100000in}{0.183744in}}{\pgfqpoint{4.506048in}{4.506048in}}%
\pgfusepath{clip}%
\pgfsetrectcap%
\pgfsetroundjoin%
\pgfsetlinewidth{0.501875pt}%
\definecolor{currentstroke}{rgb}{0.000000,0.000000,1.000000}%
\pgfsetstrokecolor{currentstroke}%
\pgfsetstrokeopacity{0.600000}%
\pgfsetdash{}{0pt}%
\pgfpathmoveto{\pgfqpoint{2.342476in}{2.027581in}}%
\pgfpathlineto{\pgfqpoint{2.302153in}{1.980573in}}%
\pgfusepath{stroke}%
\end{pgfscope}%
\begin{pgfscope}%
\pgfpathrectangle{\pgfqpoint{0.100000in}{0.183744in}}{\pgfqpoint{4.506048in}{4.506048in}}%
\pgfusepath{clip}%
\pgfsetrectcap%
\pgfsetroundjoin%
\pgfsetlinewidth{0.501875pt}%
\definecolor{currentstroke}{rgb}{0.000000,0.000000,1.000000}%
\pgfsetstrokecolor{currentstroke}%
\pgfsetstrokeopacity{0.600000}%
\pgfsetdash{}{0pt}%
\pgfpathmoveto{\pgfqpoint{3.827416in}{1.560859in}}%
\pgfpathlineto{\pgfqpoint{3.866762in}{1.994237in}}%
\pgfusepath{stroke}%
\end{pgfscope}%
\begin{pgfscope}%
\pgfpathrectangle{\pgfqpoint{0.100000in}{0.183744in}}{\pgfqpoint{4.506048in}{4.506048in}}%
\pgfusepath{clip}%
\pgfsetrectcap%
\pgfsetroundjoin%
\pgfsetlinewidth{0.501875pt}%
\definecolor{currentstroke}{rgb}{0.000000,0.000000,1.000000}%
\pgfsetstrokecolor{currentstroke}%
\pgfsetstrokeopacity{0.600000}%
\pgfsetdash{}{0pt}%
\pgfpathmoveto{\pgfqpoint{1.587329in}{3.346542in}}%
\pgfpathlineto{\pgfqpoint{1.452903in}{3.114625in}}%
\pgfusepath{stroke}%
\end{pgfscope}%
\begin{pgfscope}%
\pgfpathrectangle{\pgfqpoint{0.100000in}{0.183744in}}{\pgfqpoint{4.506048in}{4.506048in}}%
\pgfusepath{clip}%
\pgfsetrectcap%
\pgfsetroundjoin%
\pgfsetlinewidth{0.501875pt}%
\definecolor{currentstroke}{rgb}{0.000000,0.000000,1.000000}%
\pgfsetstrokecolor{currentstroke}%
\pgfsetstrokeopacity{0.600000}%
\pgfsetdash{}{0pt}%
\pgfpathmoveto{\pgfqpoint{2.572156in}{3.701984in}}%
\pgfpathlineto{\pgfqpoint{2.411150in}{3.402519in}}%
\pgfusepath{stroke}%
\end{pgfscope}%
\begin{pgfscope}%
\pgfpathrectangle{\pgfqpoint{0.100000in}{0.183744in}}{\pgfqpoint{4.506048in}{4.506048in}}%
\pgfusepath{clip}%
\pgfsetrectcap%
\pgfsetroundjoin%
\pgfsetlinewidth{0.501875pt}%
\definecolor{currentstroke}{rgb}{0.000000,0.000000,1.000000}%
\pgfsetstrokecolor{currentstroke}%
\pgfsetstrokeopacity{0.600000}%
\pgfsetdash{}{0pt}%
\pgfpathmoveto{\pgfqpoint{1.894482in}{2.725630in}}%
\pgfpathlineto{\pgfqpoint{1.943466in}{2.481251in}}%
\pgfusepath{stroke}%
\end{pgfscope}%
\begin{pgfscope}%
\pgfpathrectangle{\pgfqpoint{0.100000in}{0.183744in}}{\pgfqpoint{4.506048in}{4.506048in}}%
\pgfusepath{clip}%
\pgfsetrectcap%
\pgfsetroundjoin%
\pgfsetlinewidth{0.501875pt}%
\definecolor{currentstroke}{rgb}{0.000000,0.000000,1.000000}%
\pgfsetstrokecolor{currentstroke}%
\pgfsetstrokeopacity{0.600000}%
\pgfsetdash{}{0pt}%
\pgfpathmoveto{\pgfqpoint{0.709947in}{3.409618in}}%
\pgfpathlineto{\pgfqpoint{0.721276in}{3.181716in}}%
\pgfusepath{stroke}%
\end{pgfscope}%
\begin{pgfscope}%
\pgfpathrectangle{\pgfqpoint{0.100000in}{0.183744in}}{\pgfqpoint{4.506048in}{4.506048in}}%
\pgfusepath{clip}%
\pgfsetrectcap%
\pgfsetroundjoin%
\pgfsetlinewidth{0.501875pt}%
\definecolor{currentstroke}{rgb}{0.000000,0.000000,1.000000}%
\pgfsetstrokecolor{currentstroke}%
\pgfsetstrokeopacity{0.600000}%
\pgfsetdash{}{0pt}%
\pgfpathmoveto{\pgfqpoint{3.602370in}{2.529869in}}%
\pgfpathlineto{\pgfqpoint{3.387532in}{2.673107in}}%
\pgfusepath{stroke}%
\end{pgfscope}%
\begin{pgfscope}%
\pgfpathrectangle{\pgfqpoint{0.100000in}{0.183744in}}{\pgfqpoint{4.506048in}{4.506048in}}%
\pgfusepath{clip}%
\pgfsetrectcap%
\pgfsetroundjoin%
\pgfsetlinewidth{0.501875pt}%
\definecolor{currentstroke}{rgb}{0.000000,0.000000,1.000000}%
\pgfsetstrokecolor{currentstroke}%
\pgfsetstrokeopacity{0.600000}%
\pgfsetdash{}{0pt}%
\pgfpathmoveto{\pgfqpoint{3.475060in}{3.012523in}}%
\pgfpathlineto{\pgfqpoint{3.476296in}{3.250084in}}%
\pgfusepath{stroke}%
\end{pgfscope}%
\begin{pgfscope}%
\pgfpathrectangle{\pgfqpoint{0.100000in}{0.183744in}}{\pgfqpoint{4.506048in}{4.506048in}}%
\pgfusepath{clip}%
\pgfsetrectcap%
\pgfsetroundjoin%
\pgfsetlinewidth{0.501875pt}%
\definecolor{currentstroke}{rgb}{0.000000,0.000000,1.000000}%
\pgfsetstrokecolor{currentstroke}%
\pgfsetstrokeopacity{0.600000}%
\pgfsetdash{}{0pt}%
\pgfpathmoveto{\pgfqpoint{1.621163in}{2.275355in}}%
\pgfpathlineto{\pgfqpoint{1.625627in}{2.313117in}}%
\pgfusepath{stroke}%
\end{pgfscope}%
\begin{pgfscope}%
\pgfpathrectangle{\pgfqpoint{0.100000in}{0.183744in}}{\pgfqpoint{4.506048in}{4.506048in}}%
\pgfusepath{clip}%
\pgfsetrectcap%
\pgfsetroundjoin%
\pgfsetlinewidth{0.501875pt}%
\definecolor{currentstroke}{rgb}{0.000000,0.000000,1.000000}%
\pgfsetstrokecolor{currentstroke}%
\pgfsetstrokeopacity{0.600000}%
\pgfsetdash{}{0pt}%
\pgfpathmoveto{\pgfqpoint{2.720098in}{3.365855in}}%
\pgfpathlineto{\pgfqpoint{2.663198in}{3.274397in}}%
\pgfusepath{stroke}%
\end{pgfscope}%
\begin{pgfscope}%
\pgfpathrectangle{\pgfqpoint{0.100000in}{0.183744in}}{\pgfqpoint{4.506048in}{4.506048in}}%
\pgfusepath{clip}%
\pgfsetrectcap%
\pgfsetroundjoin%
\pgfsetlinewidth{0.501875pt}%
\definecolor{currentstroke}{rgb}{0.000000,0.000000,1.000000}%
\pgfsetstrokecolor{currentstroke}%
\pgfsetstrokeopacity{0.600000}%
\pgfsetdash{}{0pt}%
\pgfpathmoveto{\pgfqpoint{2.595896in}{2.100383in}}%
\pgfpathlineto{\pgfqpoint{2.388105in}{1.918441in}}%
\pgfusepath{stroke}%
\end{pgfscope}%
\begin{pgfscope}%
\pgfpathrectangle{\pgfqpoint{0.100000in}{0.183744in}}{\pgfqpoint{4.506048in}{4.506048in}}%
\pgfusepath{clip}%
\pgfsetrectcap%
\pgfsetroundjoin%
\pgfsetlinewidth{0.501875pt}%
\definecolor{currentstroke}{rgb}{0.000000,0.000000,1.000000}%
\pgfsetstrokecolor{currentstroke}%
\pgfsetstrokeopacity{0.600000}%
\pgfsetdash{}{0pt}%
\pgfpathmoveto{\pgfqpoint{2.965991in}{2.560492in}}%
\pgfpathlineto{\pgfqpoint{2.772854in}{2.597856in}}%
\pgfusepath{stroke}%
\end{pgfscope}%
\begin{pgfscope}%
\pgfpathrectangle{\pgfqpoint{0.100000in}{0.183744in}}{\pgfqpoint{4.506048in}{4.506048in}}%
\pgfusepath{clip}%
\pgfsetrectcap%
\pgfsetroundjoin%
\pgfsetlinewidth{0.501875pt}%
\definecolor{currentstroke}{rgb}{0.000000,0.000000,1.000000}%
\pgfsetstrokecolor{currentstroke}%
\pgfsetstrokeopacity{0.600000}%
\pgfsetdash{}{0pt}%
\pgfpathmoveto{\pgfqpoint{1.608716in}{2.291373in}}%
\pgfpathlineto{\pgfqpoint{1.745049in}{2.372186in}}%
\pgfusepath{stroke}%
\end{pgfscope}%
\begin{pgfscope}%
\pgfpathrectangle{\pgfqpoint{0.100000in}{0.183744in}}{\pgfqpoint{4.506048in}{4.506048in}}%
\pgfusepath{clip}%
\pgfsetrectcap%
\pgfsetroundjoin%
\pgfsetlinewidth{0.501875pt}%
\definecolor{currentstroke}{rgb}{0.000000,0.000000,1.000000}%
\pgfsetstrokecolor{currentstroke}%
\pgfsetstrokeopacity{0.600000}%
\pgfsetdash{}{0pt}%
\pgfpathmoveto{\pgfqpoint{3.202403in}{2.249695in}}%
\pgfpathlineto{\pgfqpoint{3.328877in}{2.434949in}}%
\pgfusepath{stroke}%
\end{pgfscope}%
\begin{pgfscope}%
\pgfpathrectangle{\pgfqpoint{0.100000in}{0.183744in}}{\pgfqpoint{4.506048in}{4.506048in}}%
\pgfusepath{clip}%
\pgfsetrectcap%
\pgfsetroundjoin%
\pgfsetlinewidth{0.501875pt}%
\definecolor{currentstroke}{rgb}{0.000000,0.000000,1.000000}%
\pgfsetstrokecolor{currentstroke}%
\pgfsetstrokeopacity{0.600000}%
\pgfsetdash{}{0pt}%
\pgfpathmoveto{\pgfqpoint{4.110665in}{2.482121in}}%
\pgfpathlineto{\pgfqpoint{4.160701in}{2.931916in}}%
\pgfusepath{stroke}%
\end{pgfscope}%
\begin{pgfscope}%
\pgfpathrectangle{\pgfqpoint{0.100000in}{0.183744in}}{\pgfqpoint{4.506048in}{4.506048in}}%
\pgfusepath{clip}%
\pgfsetrectcap%
\pgfsetroundjoin%
\pgfsetlinewidth{0.501875pt}%
\definecolor{currentstroke}{rgb}{0.000000,0.000000,1.000000}%
\pgfsetstrokecolor{currentstroke}%
\pgfsetstrokeopacity{0.600000}%
\pgfsetdash{}{0pt}%
\pgfpathmoveto{\pgfqpoint{2.867854in}{2.931823in}}%
\pgfpathlineto{\pgfqpoint{2.909228in}{3.140111in}}%
\pgfusepath{stroke}%
\end{pgfscope}%
\begin{pgfscope}%
\pgfpathrectangle{\pgfqpoint{0.100000in}{0.183744in}}{\pgfqpoint{4.506048in}{4.506048in}}%
\pgfusepath{clip}%
\pgfsetrectcap%
\pgfsetroundjoin%
\pgfsetlinewidth{0.501875pt}%
\definecolor{currentstroke}{rgb}{0.000000,0.000000,1.000000}%
\pgfsetstrokecolor{currentstroke}%
\pgfsetstrokeopacity{0.600000}%
\pgfsetdash{}{0pt}%
\pgfpathmoveto{\pgfqpoint{1.570571in}{2.677734in}}%
\pgfpathlineto{\pgfqpoint{1.283301in}{2.433335in}}%
\pgfusepath{stroke}%
\end{pgfscope}%
\begin{pgfscope}%
\pgfpathrectangle{\pgfqpoint{0.100000in}{0.183744in}}{\pgfqpoint{4.506048in}{4.506048in}}%
\pgfusepath{clip}%
\pgfsetrectcap%
\pgfsetroundjoin%
\pgfsetlinewidth{0.501875pt}%
\definecolor{currentstroke}{rgb}{0.000000,0.000000,1.000000}%
\pgfsetstrokecolor{currentstroke}%
\pgfsetstrokeopacity{0.600000}%
\pgfsetdash{}{0pt}%
\pgfpathmoveto{\pgfqpoint{2.456909in}{1.513587in}}%
\pgfpathlineto{\pgfqpoint{2.492592in}{1.733210in}}%
\pgfusepath{stroke}%
\end{pgfscope}%
\begin{pgfscope}%
\pgfpathrectangle{\pgfqpoint{0.100000in}{0.183744in}}{\pgfqpoint{4.506048in}{4.506048in}}%
\pgfusepath{clip}%
\pgfsetrectcap%
\pgfsetroundjoin%
\pgfsetlinewidth{0.501875pt}%
\definecolor{currentstroke}{rgb}{0.000000,0.000000,1.000000}%
\pgfsetstrokecolor{currentstroke}%
\pgfsetstrokeopacity{0.600000}%
\pgfsetdash{}{0pt}%
\pgfpathmoveto{\pgfqpoint{0.649640in}{3.499463in}}%
\pgfpathlineto{\pgfqpoint{0.465342in}{3.215301in}}%
\pgfusepath{stroke}%
\end{pgfscope}%
\begin{pgfscope}%
\pgfpathrectangle{\pgfqpoint{0.100000in}{0.183744in}}{\pgfqpoint{4.506048in}{4.506048in}}%
\pgfusepath{clip}%
\pgfsetrectcap%
\pgfsetroundjoin%
\pgfsetlinewidth{0.501875pt}%
\definecolor{currentstroke}{rgb}{0.000000,0.000000,1.000000}%
\pgfsetstrokecolor{currentstroke}%
\pgfsetstrokeopacity{0.600000}%
\pgfsetdash{}{0pt}%
\pgfpathmoveto{\pgfqpoint{2.762070in}{2.466887in}}%
\pgfpathlineto{\pgfqpoint{2.727137in}{2.618225in}}%
\pgfusepath{stroke}%
\end{pgfscope}%
\begin{pgfscope}%
\pgfpathrectangle{\pgfqpoint{0.100000in}{0.183744in}}{\pgfqpoint{4.506048in}{4.506048in}}%
\pgfusepath{clip}%
\pgfsetrectcap%
\pgfsetroundjoin%
\pgfsetlinewidth{0.501875pt}%
\definecolor{currentstroke}{rgb}{0.000000,0.000000,1.000000}%
\pgfsetstrokecolor{currentstroke}%
\pgfsetstrokeopacity{0.600000}%
\pgfsetdash{}{0pt}%
\pgfpathmoveto{\pgfqpoint{0.987324in}{1.810284in}}%
\pgfpathlineto{\pgfqpoint{0.963442in}{1.771474in}}%
\pgfusepath{stroke}%
\end{pgfscope}%
\begin{pgfscope}%
\pgfpathrectangle{\pgfqpoint{0.100000in}{0.183744in}}{\pgfqpoint{4.506048in}{4.506048in}}%
\pgfusepath{clip}%
\pgfsetrectcap%
\pgfsetroundjoin%
\pgfsetlinewidth{0.501875pt}%
\definecolor{currentstroke}{rgb}{0.000000,0.000000,1.000000}%
\pgfsetstrokecolor{currentstroke}%
\pgfsetstrokeopacity{0.600000}%
\pgfsetdash{}{0pt}%
\pgfpathmoveto{\pgfqpoint{2.367234in}{3.436139in}}%
\pgfpathlineto{\pgfqpoint{2.332590in}{3.374131in}}%
\pgfusepath{stroke}%
\end{pgfscope}%
\begin{pgfscope}%
\pgfpathrectangle{\pgfqpoint{0.100000in}{0.183744in}}{\pgfqpoint{4.506048in}{4.506048in}}%
\pgfusepath{clip}%
\pgfsetrectcap%
\pgfsetroundjoin%
\pgfsetlinewidth{0.501875pt}%
\definecolor{currentstroke}{rgb}{0.000000,0.000000,1.000000}%
\pgfsetstrokecolor{currentstroke}%
\pgfsetstrokeopacity{0.600000}%
\pgfsetdash{}{0pt}%
\pgfpathmoveto{\pgfqpoint{2.918257in}{1.761597in}}%
\pgfpathlineto{\pgfqpoint{3.018700in}{2.214944in}}%
\pgfusepath{stroke}%
\end{pgfscope}%
\begin{pgfscope}%
\pgfpathrectangle{\pgfqpoint{0.100000in}{0.183744in}}{\pgfqpoint{4.506048in}{4.506048in}}%
\pgfusepath{clip}%
\pgfsetrectcap%
\pgfsetroundjoin%
\pgfsetlinewidth{0.501875pt}%
\definecolor{currentstroke}{rgb}{0.000000,0.000000,1.000000}%
\pgfsetstrokecolor{currentstroke}%
\pgfsetstrokeopacity{0.600000}%
\pgfsetdash{}{0pt}%
\pgfpathmoveto{\pgfqpoint{1.866419in}{2.773181in}}%
\pgfpathlineto{\pgfqpoint{1.953729in}{2.571081in}}%
\pgfusepath{stroke}%
\end{pgfscope}%
\begin{pgfscope}%
\pgfpathrectangle{\pgfqpoint{0.100000in}{0.183744in}}{\pgfqpoint{4.506048in}{4.506048in}}%
\pgfusepath{clip}%
\pgfsetrectcap%
\pgfsetroundjoin%
\pgfsetlinewidth{0.501875pt}%
\definecolor{currentstroke}{rgb}{0.000000,0.000000,1.000000}%
\pgfsetstrokecolor{currentstroke}%
\pgfsetstrokeopacity{0.600000}%
\pgfsetdash{}{0pt}%
\pgfpathmoveto{\pgfqpoint{3.357545in}{2.440922in}}%
\pgfpathlineto{\pgfqpoint{3.314313in}{2.822211in}}%
\pgfusepath{stroke}%
\end{pgfscope}%
\begin{pgfscope}%
\pgfpathrectangle{\pgfqpoint{0.100000in}{0.183744in}}{\pgfqpoint{4.506048in}{4.506048in}}%
\pgfusepath{clip}%
\pgfsetrectcap%
\pgfsetroundjoin%
\pgfsetlinewidth{0.501875pt}%
\definecolor{currentstroke}{rgb}{0.000000,0.000000,1.000000}%
\pgfsetstrokecolor{currentstroke}%
\pgfsetstrokeopacity{0.600000}%
\pgfsetdash{}{0pt}%
\pgfpathmoveto{\pgfqpoint{1.746629in}{2.153440in}}%
\pgfpathlineto{\pgfqpoint{1.698303in}{2.140563in}}%
\pgfusepath{stroke}%
\end{pgfscope}%
\begin{pgfscope}%
\pgfpathrectangle{\pgfqpoint{0.100000in}{0.183744in}}{\pgfqpoint{4.506048in}{4.506048in}}%
\pgfusepath{clip}%
\pgfsetrectcap%
\pgfsetroundjoin%
\pgfsetlinewidth{0.501875pt}%
\definecolor{currentstroke}{rgb}{0.000000,0.000000,1.000000}%
\pgfsetstrokecolor{currentstroke}%
\pgfsetstrokeopacity{0.600000}%
\pgfsetdash{}{0pt}%
\pgfpathmoveto{\pgfqpoint{1.690085in}{2.211822in}}%
\pgfpathlineto{\pgfqpoint{1.748979in}{2.029628in}}%
\pgfusepath{stroke}%
\end{pgfscope}%
\begin{pgfscope}%
\pgfpathrectangle{\pgfqpoint{0.100000in}{0.183744in}}{\pgfqpoint{4.506048in}{4.506048in}}%
\pgfusepath{clip}%
\pgfsetrectcap%
\pgfsetroundjoin%
\pgfsetlinewidth{0.501875pt}%
\definecolor{currentstroke}{rgb}{0.000000,0.000000,1.000000}%
\pgfsetstrokecolor{currentstroke}%
\pgfsetstrokeopacity{0.600000}%
\pgfsetdash{}{0pt}%
\pgfpathmoveto{\pgfqpoint{2.486663in}{2.025726in}}%
\pgfpathlineto{\pgfqpoint{2.442337in}{2.227718in}}%
\pgfusepath{stroke}%
\end{pgfscope}%
\begin{pgfscope}%
\pgfpathrectangle{\pgfqpoint{0.100000in}{0.183744in}}{\pgfqpoint{4.506048in}{4.506048in}}%
\pgfusepath{clip}%
\pgfsetrectcap%
\pgfsetroundjoin%
\pgfsetlinewidth{0.501875pt}%
\definecolor{currentstroke}{rgb}{0.000000,0.000000,1.000000}%
\pgfsetstrokecolor{currentstroke}%
\pgfsetstrokeopacity{0.600000}%
\pgfsetdash{}{0pt}%
\pgfpathmoveto{\pgfqpoint{1.732141in}{2.211647in}}%
\pgfpathlineto{\pgfqpoint{1.813533in}{2.262265in}}%
\pgfusepath{stroke}%
\end{pgfscope}%
\begin{pgfscope}%
\pgfpathrectangle{\pgfqpoint{0.100000in}{0.183744in}}{\pgfqpoint{4.506048in}{4.506048in}}%
\pgfusepath{clip}%
\pgfsetrectcap%
\pgfsetroundjoin%
\pgfsetlinewidth{0.501875pt}%
\definecolor{currentstroke}{rgb}{0.000000,0.000000,1.000000}%
\pgfsetstrokecolor{currentstroke}%
\pgfsetstrokeopacity{0.600000}%
\pgfsetdash{}{0pt}%
\pgfpathmoveto{\pgfqpoint{2.016977in}{2.744962in}}%
\pgfpathlineto{\pgfqpoint{1.941433in}{2.763094in}}%
\pgfusepath{stroke}%
\end{pgfscope}%
\begin{pgfscope}%
\pgfpathrectangle{\pgfqpoint{0.100000in}{0.183744in}}{\pgfqpoint{4.506048in}{4.506048in}}%
\pgfusepath{clip}%
\pgfsetrectcap%
\pgfsetroundjoin%
\pgfsetlinewidth{0.501875pt}%
\definecolor{currentstroke}{rgb}{0.000000,0.000000,1.000000}%
\pgfsetstrokecolor{currentstroke}%
\pgfsetstrokeopacity{0.600000}%
\pgfsetdash{}{0pt}%
\pgfpathmoveto{\pgfqpoint{1.623595in}{1.917412in}}%
\pgfpathlineto{\pgfqpoint{1.491418in}{1.971095in}}%
\pgfusepath{stroke}%
\end{pgfscope}%
\begin{pgfscope}%
\pgfpathrectangle{\pgfqpoint{0.100000in}{0.183744in}}{\pgfqpoint{4.506048in}{4.506048in}}%
\pgfusepath{clip}%
\pgfsetrectcap%
\pgfsetroundjoin%
\pgfsetlinewidth{0.501875pt}%
\definecolor{currentstroke}{rgb}{0.000000,0.000000,1.000000}%
\pgfsetstrokecolor{currentstroke}%
\pgfsetstrokeopacity{0.600000}%
\pgfsetdash{}{0pt}%
\pgfpathmoveto{\pgfqpoint{3.741475in}{1.703462in}}%
\pgfpathlineto{\pgfqpoint{3.833626in}{2.198361in}}%
\pgfusepath{stroke}%
\end{pgfscope}%
\begin{pgfscope}%
\pgfpathrectangle{\pgfqpoint{0.100000in}{0.183744in}}{\pgfqpoint{4.506048in}{4.506048in}}%
\pgfusepath{clip}%
\pgfsetrectcap%
\pgfsetroundjoin%
\pgfsetlinewidth{0.501875pt}%
\definecolor{currentstroke}{rgb}{0.000000,0.000000,1.000000}%
\pgfsetstrokecolor{currentstroke}%
\pgfsetstrokeopacity{0.600000}%
\pgfsetdash{}{0pt}%
\pgfpathmoveto{\pgfqpoint{2.835825in}{2.064082in}}%
\pgfpathlineto{\pgfqpoint{2.719998in}{2.256066in}}%
\pgfusepath{stroke}%
\end{pgfscope}%
\begin{pgfscope}%
\pgfpathrectangle{\pgfqpoint{0.100000in}{0.183744in}}{\pgfqpoint{4.506048in}{4.506048in}}%
\pgfusepath{clip}%
\pgfsetrectcap%
\pgfsetroundjoin%
\pgfsetlinewidth{0.501875pt}%
\definecolor{currentstroke}{rgb}{0.000000,0.000000,1.000000}%
\pgfsetstrokecolor{currentstroke}%
\pgfsetstrokeopacity{0.600000}%
\pgfsetdash{}{0pt}%
\pgfpathmoveto{\pgfqpoint{1.952614in}{2.035558in}}%
\pgfpathlineto{\pgfqpoint{1.865061in}{1.946818in}}%
\pgfusepath{stroke}%
\end{pgfscope}%
\begin{pgfscope}%
\pgfpathrectangle{\pgfqpoint{0.100000in}{0.183744in}}{\pgfqpoint{4.506048in}{4.506048in}}%
\pgfusepath{clip}%
\pgfsetrectcap%
\pgfsetroundjoin%
\pgfsetlinewidth{0.501875pt}%
\definecolor{currentstroke}{rgb}{0.000000,0.000000,1.000000}%
\pgfsetstrokecolor{currentstroke}%
\pgfsetstrokeopacity{0.600000}%
\pgfsetdash{}{0pt}%
\pgfpathmoveto{\pgfqpoint{3.185749in}{1.742261in}}%
\pgfpathlineto{\pgfqpoint{3.259231in}{2.148401in}}%
\pgfusepath{stroke}%
\end{pgfscope}%
\begin{pgfscope}%
\pgfpathrectangle{\pgfqpoint{0.100000in}{0.183744in}}{\pgfqpoint{4.506048in}{4.506048in}}%
\pgfusepath{clip}%
\pgfsetrectcap%
\pgfsetroundjoin%
\pgfsetlinewidth{0.501875pt}%
\definecolor{currentstroke}{rgb}{0.000000,0.000000,1.000000}%
\pgfsetstrokecolor{currentstroke}%
\pgfsetstrokeopacity{0.600000}%
\pgfsetdash{}{0pt}%
\pgfpathmoveto{\pgfqpoint{1.844046in}{2.474123in}}%
\pgfpathlineto{\pgfqpoint{1.750458in}{2.359194in}}%
\pgfusepath{stroke}%
\end{pgfscope}%
\begin{pgfscope}%
\pgfpathrectangle{\pgfqpoint{0.100000in}{0.183744in}}{\pgfqpoint{4.506048in}{4.506048in}}%
\pgfusepath{clip}%
\pgfsetrectcap%
\pgfsetroundjoin%
\pgfsetlinewidth{0.501875pt}%
\definecolor{currentstroke}{rgb}{0.000000,0.000000,1.000000}%
\pgfsetstrokecolor{currentstroke}%
\pgfsetstrokeopacity{0.600000}%
\pgfsetdash{}{0pt}%
\pgfpathmoveto{\pgfqpoint{2.132210in}{1.274832in}}%
\pgfpathlineto{\pgfqpoint{2.350882in}{1.548548in}}%
\pgfusepath{stroke}%
\end{pgfscope}%
\begin{pgfscope}%
\pgfpathrectangle{\pgfqpoint{0.100000in}{0.183744in}}{\pgfqpoint{4.506048in}{4.506048in}}%
\pgfusepath{clip}%
\pgfsetrectcap%
\pgfsetroundjoin%
\pgfsetlinewidth{0.501875pt}%
\definecolor{currentstroke}{rgb}{0.000000,0.000000,1.000000}%
\pgfsetstrokecolor{currentstroke}%
\pgfsetstrokeopacity{0.600000}%
\pgfsetdash{}{0pt}%
\pgfpathmoveto{\pgfqpoint{2.518409in}{1.175589in}}%
\pgfpathlineto{\pgfqpoint{2.626807in}{0.868882in}}%
\pgfusepath{stroke}%
\end{pgfscope}%
\begin{pgfscope}%
\pgfpathrectangle{\pgfqpoint{0.100000in}{0.183744in}}{\pgfqpoint{4.506048in}{4.506048in}}%
\pgfusepath{clip}%
\pgfsetrectcap%
\pgfsetroundjoin%
\pgfsetlinewidth{0.501875pt}%
\definecolor{currentstroke}{rgb}{0.000000,0.000000,1.000000}%
\pgfsetstrokecolor{currentstroke}%
\pgfsetstrokeopacity{0.600000}%
\pgfsetdash{}{0pt}%
\pgfpathmoveto{\pgfqpoint{2.026572in}{2.548779in}}%
\pgfpathlineto{\pgfqpoint{1.897731in}{2.591237in}}%
\pgfusepath{stroke}%
\end{pgfscope}%
\begin{pgfscope}%
\pgfpathrectangle{\pgfqpoint{0.100000in}{0.183744in}}{\pgfqpoint{4.506048in}{4.506048in}}%
\pgfusepath{clip}%
\pgfsetrectcap%
\pgfsetroundjoin%
\pgfsetlinewidth{0.501875pt}%
\definecolor{currentstroke}{rgb}{0.000000,0.000000,1.000000}%
\pgfsetstrokecolor{currentstroke}%
\pgfsetstrokeopacity{0.600000}%
\pgfsetdash{}{0pt}%
\pgfpathmoveto{\pgfqpoint{3.440177in}{2.254777in}}%
\pgfpathlineto{\pgfqpoint{3.433186in}{2.307295in}}%
\pgfusepath{stroke}%
\end{pgfscope}%
\begin{pgfscope}%
\pgfpathrectangle{\pgfqpoint{0.100000in}{0.183744in}}{\pgfqpoint{4.506048in}{4.506048in}}%
\pgfusepath{clip}%
\pgfsetrectcap%
\pgfsetroundjoin%
\pgfsetlinewidth{0.501875pt}%
\definecolor{currentstroke}{rgb}{0.000000,0.000000,1.000000}%
\pgfsetstrokecolor{currentstroke}%
\pgfsetstrokeopacity{0.600000}%
\pgfsetdash{}{0pt}%
\pgfpathmoveto{\pgfqpoint{2.061356in}{3.390787in}}%
\pgfpathlineto{\pgfqpoint{2.270649in}{3.238351in}}%
\pgfusepath{stroke}%
\end{pgfscope}%
\begin{pgfscope}%
\pgfpathrectangle{\pgfqpoint{0.100000in}{0.183744in}}{\pgfqpoint{4.506048in}{4.506048in}}%
\pgfusepath{clip}%
\pgfsetrectcap%
\pgfsetroundjoin%
\pgfsetlinewidth{0.501875pt}%
\definecolor{currentstroke}{rgb}{0.000000,0.000000,1.000000}%
\pgfsetstrokecolor{currentstroke}%
\pgfsetstrokeopacity{0.600000}%
\pgfsetdash{}{0pt}%
\pgfpathmoveto{\pgfqpoint{3.626405in}{2.371179in}}%
\pgfpathlineto{\pgfqpoint{3.562947in}{2.392397in}}%
\pgfusepath{stroke}%
\end{pgfscope}%
\begin{pgfscope}%
\pgfpathrectangle{\pgfqpoint{0.100000in}{0.183744in}}{\pgfqpoint{4.506048in}{4.506048in}}%
\pgfusepath{clip}%
\pgfsetrectcap%
\pgfsetroundjoin%
\pgfsetlinewidth{0.501875pt}%
\definecolor{currentstroke}{rgb}{0.000000,0.000000,1.000000}%
\pgfsetstrokecolor{currentstroke}%
\pgfsetstrokeopacity{0.600000}%
\pgfsetdash{}{0pt}%
\pgfpathmoveto{\pgfqpoint{1.122344in}{1.687462in}}%
\pgfpathlineto{\pgfqpoint{1.315557in}{1.783431in}}%
\pgfusepath{stroke}%
\end{pgfscope}%
\begin{pgfscope}%
\pgfpathrectangle{\pgfqpoint{0.100000in}{0.183744in}}{\pgfqpoint{4.506048in}{4.506048in}}%
\pgfusepath{clip}%
\pgfsetrectcap%
\pgfsetroundjoin%
\pgfsetlinewidth{0.501875pt}%
\definecolor{currentstroke}{rgb}{0.000000,0.000000,1.000000}%
\pgfsetstrokecolor{currentstroke}%
\pgfsetstrokeopacity{0.600000}%
\pgfsetdash{}{0pt}%
\pgfpathmoveto{\pgfqpoint{2.289728in}{3.476991in}}%
\pgfpathlineto{\pgfqpoint{2.368239in}{3.285022in}}%
\pgfusepath{stroke}%
\end{pgfscope}%
\begin{pgfscope}%
\pgfpathrectangle{\pgfqpoint{0.100000in}{0.183744in}}{\pgfqpoint{4.506048in}{4.506048in}}%
\pgfusepath{clip}%
\pgfsetrectcap%
\pgfsetroundjoin%
\pgfsetlinewidth{0.501875pt}%
\definecolor{currentstroke}{rgb}{0.000000,0.000000,1.000000}%
\pgfsetstrokecolor{currentstroke}%
\pgfsetstrokeopacity{0.600000}%
\pgfsetdash{}{0pt}%
\pgfpathmoveto{\pgfqpoint{4.126182in}{2.888701in}}%
\pgfpathlineto{\pgfqpoint{4.001177in}{3.053599in}}%
\pgfusepath{stroke}%
\end{pgfscope}%
\begin{pgfscope}%
\pgfpathrectangle{\pgfqpoint{0.100000in}{0.183744in}}{\pgfqpoint{4.506048in}{4.506048in}}%
\pgfusepath{clip}%
\pgfsetrectcap%
\pgfsetroundjoin%
\pgfsetlinewidth{0.501875pt}%
\definecolor{currentstroke}{rgb}{0.000000,0.000000,1.000000}%
\pgfsetstrokecolor{currentstroke}%
\pgfsetstrokeopacity{0.600000}%
\pgfsetdash{}{0pt}%
\pgfpathmoveto{\pgfqpoint{3.616923in}{2.889966in}}%
\pgfpathlineto{\pgfqpoint{3.566443in}{2.869168in}}%
\pgfusepath{stroke}%
\end{pgfscope}%
\begin{pgfscope}%
\pgfpathrectangle{\pgfqpoint{0.100000in}{0.183744in}}{\pgfqpoint{4.506048in}{4.506048in}}%
\pgfusepath{clip}%
\pgfsetrectcap%
\pgfsetroundjoin%
\pgfsetlinewidth{0.501875pt}%
\definecolor{currentstroke}{rgb}{0.000000,0.000000,1.000000}%
\pgfsetstrokecolor{currentstroke}%
\pgfsetstrokeopacity{0.600000}%
\pgfsetdash{}{0pt}%
\pgfpathmoveto{\pgfqpoint{1.488799in}{2.184380in}}%
\pgfpathlineto{\pgfqpoint{1.561420in}{1.988079in}}%
\pgfusepath{stroke}%
\end{pgfscope}%
\begin{pgfscope}%
\pgfpathrectangle{\pgfqpoint{0.100000in}{0.183744in}}{\pgfqpoint{4.506048in}{4.506048in}}%
\pgfusepath{clip}%
\pgfsetrectcap%
\pgfsetroundjoin%
\pgfsetlinewidth{0.501875pt}%
\definecolor{currentstroke}{rgb}{0.000000,0.000000,1.000000}%
\pgfsetstrokecolor{currentstroke}%
\pgfsetstrokeopacity{0.600000}%
\pgfsetdash{}{0pt}%
\pgfpathmoveto{\pgfqpoint{2.935768in}{2.268308in}}%
\pgfpathlineto{\pgfqpoint{3.085744in}{2.422592in}}%
\pgfusepath{stroke}%
\end{pgfscope}%
\begin{pgfscope}%
\pgfpathrectangle{\pgfqpoint{0.100000in}{0.183744in}}{\pgfqpoint{4.506048in}{4.506048in}}%
\pgfusepath{clip}%
\pgfsetrectcap%
\pgfsetroundjoin%
\pgfsetlinewidth{0.501875pt}%
\definecolor{currentstroke}{rgb}{0.000000,0.000000,1.000000}%
\pgfsetstrokecolor{currentstroke}%
\pgfsetstrokeopacity{0.600000}%
\pgfsetdash{}{0pt}%
\pgfpathmoveto{\pgfqpoint{1.419721in}{2.518880in}}%
\pgfpathlineto{\pgfqpoint{1.340852in}{2.405579in}}%
\pgfusepath{stroke}%
\end{pgfscope}%
\begin{pgfscope}%
\pgfpathrectangle{\pgfqpoint{0.100000in}{0.183744in}}{\pgfqpoint{4.506048in}{4.506048in}}%
\pgfusepath{clip}%
\pgfsetrectcap%
\pgfsetroundjoin%
\pgfsetlinewidth{0.501875pt}%
\definecolor{currentstroke}{rgb}{0.000000,0.000000,1.000000}%
\pgfsetstrokecolor{currentstroke}%
\pgfsetstrokeopacity{0.600000}%
\pgfsetdash{}{0pt}%
\pgfpathmoveto{\pgfqpoint{3.378585in}{2.865423in}}%
\pgfpathlineto{\pgfqpoint{3.382408in}{3.044333in}}%
\pgfusepath{stroke}%
\end{pgfscope}%
\begin{pgfscope}%
\pgfpathrectangle{\pgfqpoint{0.100000in}{0.183744in}}{\pgfqpoint{4.506048in}{4.506048in}}%
\pgfusepath{clip}%
\pgfsetrectcap%
\pgfsetroundjoin%
\pgfsetlinewidth{0.501875pt}%
\definecolor{currentstroke}{rgb}{0.000000,0.000000,1.000000}%
\pgfsetstrokecolor{currentstroke}%
\pgfsetstrokeopacity{0.600000}%
\pgfsetdash{}{0pt}%
\pgfpathmoveto{\pgfqpoint{1.635114in}{2.953674in}}%
\pgfpathlineto{\pgfqpoint{1.648123in}{2.883741in}}%
\pgfusepath{stroke}%
\end{pgfscope}%
\begin{pgfscope}%
\pgfpathrectangle{\pgfqpoint{0.100000in}{0.183744in}}{\pgfqpoint{4.506048in}{4.506048in}}%
\pgfusepath{clip}%
\pgfsetrectcap%
\pgfsetroundjoin%
\pgfsetlinewidth{0.501875pt}%
\definecolor{currentstroke}{rgb}{0.000000,0.000000,1.000000}%
\pgfsetstrokecolor{currentstroke}%
\pgfsetstrokeopacity{0.600000}%
\pgfsetdash{}{0pt}%
\pgfpathmoveto{\pgfqpoint{0.904599in}{3.033458in}}%
\pgfpathlineto{\pgfqpoint{0.774189in}{2.792284in}}%
\pgfusepath{stroke}%
\end{pgfscope}%
\begin{pgfscope}%
\pgfpathrectangle{\pgfqpoint{0.100000in}{0.183744in}}{\pgfqpoint{4.506048in}{4.506048in}}%
\pgfusepath{clip}%
\pgfsetrectcap%
\pgfsetroundjoin%
\pgfsetlinewidth{0.501875pt}%
\definecolor{currentstroke}{rgb}{0.000000,0.000000,1.000000}%
\pgfsetstrokecolor{currentstroke}%
\pgfsetstrokeopacity{0.600000}%
\pgfsetdash{}{0pt}%
\pgfpathmoveto{\pgfqpoint{3.200638in}{2.816905in}}%
\pgfpathlineto{\pgfqpoint{3.154023in}{2.958523in}}%
\pgfusepath{stroke}%
\end{pgfscope}%
\begin{pgfscope}%
\pgfpathrectangle{\pgfqpoint{0.100000in}{0.183744in}}{\pgfqpoint{4.506048in}{4.506048in}}%
\pgfusepath{clip}%
\pgfsetrectcap%
\pgfsetroundjoin%
\pgfsetlinewidth{0.501875pt}%
\definecolor{currentstroke}{rgb}{0.000000,0.000000,1.000000}%
\pgfsetstrokecolor{currentstroke}%
\pgfsetstrokeopacity{0.600000}%
\pgfsetdash{}{0pt}%
\pgfpathmoveto{\pgfqpoint{2.914397in}{3.112031in}}%
\pgfpathlineto{\pgfqpoint{2.827569in}{3.104303in}}%
\pgfusepath{stroke}%
\end{pgfscope}%
\begin{pgfscope}%
\pgfpathrectangle{\pgfqpoint{0.100000in}{0.183744in}}{\pgfqpoint{4.506048in}{4.506048in}}%
\pgfusepath{clip}%
\pgfsetrectcap%
\pgfsetroundjoin%
\pgfsetlinewidth{0.501875pt}%
\definecolor{currentstroke}{rgb}{0.000000,0.000000,1.000000}%
\pgfsetstrokecolor{currentstroke}%
\pgfsetstrokeopacity{0.600000}%
\pgfsetdash{}{0pt}%
\pgfpathmoveto{\pgfqpoint{1.806236in}{1.865726in}}%
\pgfpathlineto{\pgfqpoint{1.754094in}{2.020311in}}%
\pgfusepath{stroke}%
\end{pgfscope}%
\begin{pgfscope}%
\pgfpathrectangle{\pgfqpoint{0.100000in}{0.183744in}}{\pgfqpoint{4.506048in}{4.506048in}}%
\pgfusepath{clip}%
\pgfsetrectcap%
\pgfsetroundjoin%
\pgfsetlinewidth{0.501875pt}%
\definecolor{currentstroke}{rgb}{0.000000,0.000000,1.000000}%
\pgfsetstrokecolor{currentstroke}%
\pgfsetstrokeopacity{0.600000}%
\pgfsetdash{}{0pt}%
\pgfpathmoveto{\pgfqpoint{1.382843in}{2.149759in}}%
\pgfpathlineto{\pgfqpoint{1.279741in}{1.930761in}}%
\pgfusepath{stroke}%
\end{pgfscope}%
\begin{pgfscope}%
\pgfpathrectangle{\pgfqpoint{0.100000in}{0.183744in}}{\pgfqpoint{4.506048in}{4.506048in}}%
\pgfusepath{clip}%
\pgfsetrectcap%
\pgfsetroundjoin%
\pgfsetlinewidth{0.501875pt}%
\definecolor{currentstroke}{rgb}{0.000000,0.000000,1.000000}%
\pgfsetstrokecolor{currentstroke}%
\pgfsetstrokeopacity{0.600000}%
\pgfsetdash{}{0pt}%
\pgfpathmoveto{\pgfqpoint{2.484752in}{3.345127in}}%
\pgfpathlineto{\pgfqpoint{2.498497in}{3.050465in}}%
\pgfusepath{stroke}%
\end{pgfscope}%
\begin{pgfscope}%
\pgfpathrectangle{\pgfqpoint{0.100000in}{0.183744in}}{\pgfqpoint{4.506048in}{4.506048in}}%
\pgfusepath{clip}%
\pgfsetrectcap%
\pgfsetroundjoin%
\pgfsetlinewidth{0.501875pt}%
\definecolor{currentstroke}{rgb}{0.000000,0.000000,1.000000}%
\pgfsetstrokecolor{currentstroke}%
\pgfsetstrokeopacity{0.600000}%
\pgfsetdash{}{0pt}%
\pgfpathmoveto{\pgfqpoint{4.007957in}{2.780785in}}%
\pgfpathlineto{\pgfqpoint{3.955634in}{2.405760in}}%
\pgfusepath{stroke}%
\end{pgfscope}%
\begin{pgfscope}%
\pgfpathrectangle{\pgfqpoint{0.100000in}{0.183744in}}{\pgfqpoint{4.506048in}{4.506048in}}%
\pgfusepath{clip}%
\pgfsetrectcap%
\pgfsetroundjoin%
\pgfsetlinewidth{0.501875pt}%
\definecolor{currentstroke}{rgb}{0.000000,0.000000,1.000000}%
\pgfsetstrokecolor{currentstroke}%
\pgfsetstrokeopacity{0.600000}%
\pgfsetdash{}{0pt}%
\pgfpathmoveto{\pgfqpoint{1.719428in}{1.417670in}}%
\pgfpathlineto{\pgfqpoint{1.602784in}{1.026381in}}%
\pgfusepath{stroke}%
\end{pgfscope}%
\begin{pgfscope}%
\pgfpathrectangle{\pgfqpoint{0.100000in}{0.183744in}}{\pgfqpoint{4.506048in}{4.506048in}}%
\pgfusepath{clip}%
\pgfsetrectcap%
\pgfsetroundjoin%
\pgfsetlinewidth{0.501875pt}%
\definecolor{currentstroke}{rgb}{0.000000,0.000000,1.000000}%
\pgfsetstrokecolor{currentstroke}%
\pgfsetstrokeopacity{0.600000}%
\pgfsetdash{}{0pt}%
\pgfpathmoveto{\pgfqpoint{2.327062in}{3.028218in}}%
\pgfpathlineto{\pgfqpoint{2.303511in}{3.033864in}}%
\pgfusepath{stroke}%
\end{pgfscope}%
\begin{pgfscope}%
\pgfpathrectangle{\pgfqpoint{0.100000in}{0.183744in}}{\pgfqpoint{4.506048in}{4.506048in}}%
\pgfusepath{clip}%
\pgfsetrectcap%
\pgfsetroundjoin%
\pgfsetlinewidth{0.501875pt}%
\definecolor{currentstroke}{rgb}{0.000000,0.000000,1.000000}%
\pgfsetstrokecolor{currentstroke}%
\pgfsetstrokeopacity{0.600000}%
\pgfsetdash{}{0pt}%
\pgfpathmoveto{\pgfqpoint{2.514070in}{2.710777in}}%
\pgfpathlineto{\pgfqpoint{2.326541in}{2.753781in}}%
\pgfusepath{stroke}%
\end{pgfscope}%
\begin{pgfscope}%
\pgfpathrectangle{\pgfqpoint{0.100000in}{0.183744in}}{\pgfqpoint{4.506048in}{4.506048in}}%
\pgfusepath{clip}%
\pgfsetrectcap%
\pgfsetroundjoin%
\pgfsetlinewidth{0.501875pt}%
\definecolor{currentstroke}{rgb}{0.000000,0.000000,1.000000}%
\pgfsetstrokecolor{currentstroke}%
\pgfsetstrokeopacity{0.600000}%
\pgfsetdash{}{0pt}%
\pgfpathmoveto{\pgfqpoint{2.011546in}{2.853353in}}%
\pgfpathlineto{\pgfqpoint{1.841836in}{2.648720in}}%
\pgfusepath{stroke}%
\end{pgfscope}%
\begin{pgfscope}%
\pgfpathrectangle{\pgfqpoint{0.100000in}{0.183744in}}{\pgfqpoint{4.506048in}{4.506048in}}%
\pgfusepath{clip}%
\pgfsetrectcap%
\pgfsetroundjoin%
\pgfsetlinewidth{0.501875pt}%
\definecolor{currentstroke}{rgb}{0.000000,0.000000,1.000000}%
\pgfsetstrokecolor{currentstroke}%
\pgfsetstrokeopacity{0.600000}%
\pgfsetdash{}{0pt}%
\pgfpathmoveto{\pgfqpoint{1.988965in}{1.478269in}}%
\pgfpathlineto{\pgfqpoint{2.038220in}{1.670067in}}%
\pgfusepath{stroke}%
\end{pgfscope}%
\begin{pgfscope}%
\pgfpathrectangle{\pgfqpoint{0.100000in}{0.183744in}}{\pgfqpoint{4.506048in}{4.506048in}}%
\pgfusepath{clip}%
\pgfsetrectcap%
\pgfsetroundjoin%
\pgfsetlinewidth{0.501875pt}%
\definecolor{currentstroke}{rgb}{0.000000,0.000000,1.000000}%
\pgfsetstrokecolor{currentstroke}%
\pgfsetstrokeopacity{0.600000}%
\pgfsetdash{}{0pt}%
\pgfpathmoveto{\pgfqpoint{1.784653in}{3.824669in}}%
\pgfpathlineto{\pgfqpoint{1.789856in}{3.584051in}}%
\pgfusepath{stroke}%
\end{pgfscope}%
\begin{pgfscope}%
\pgfpathrectangle{\pgfqpoint{0.100000in}{0.183744in}}{\pgfqpoint{4.506048in}{4.506048in}}%
\pgfusepath{clip}%
\pgfsetrectcap%
\pgfsetroundjoin%
\pgfsetlinewidth{0.501875pt}%
\definecolor{currentstroke}{rgb}{0.000000,0.000000,1.000000}%
\pgfsetstrokecolor{currentstroke}%
\pgfsetstrokeopacity{0.600000}%
\pgfsetdash{}{0pt}%
\pgfpathmoveto{\pgfqpoint{1.695908in}{1.836961in}}%
\pgfpathlineto{\pgfqpoint{1.656219in}{1.417749in}}%
\pgfusepath{stroke}%
\end{pgfscope}%
\begin{pgfscope}%
\pgfpathrectangle{\pgfqpoint{0.100000in}{0.183744in}}{\pgfqpoint{4.506048in}{4.506048in}}%
\pgfusepath{clip}%
\pgfsetrectcap%
\pgfsetroundjoin%
\pgfsetlinewidth{0.501875pt}%
\definecolor{currentstroke}{rgb}{0.000000,0.000000,1.000000}%
\pgfsetstrokecolor{currentstroke}%
\pgfsetstrokeopacity{0.600000}%
\pgfsetdash{}{0pt}%
\pgfpathmoveto{\pgfqpoint{3.434358in}{1.414328in}}%
\pgfpathlineto{\pgfqpoint{3.386446in}{1.721476in}}%
\pgfusepath{stroke}%
\end{pgfscope}%
\begin{pgfscope}%
\pgfpathrectangle{\pgfqpoint{0.100000in}{0.183744in}}{\pgfqpoint{4.506048in}{4.506048in}}%
\pgfusepath{clip}%
\pgfsetrectcap%
\pgfsetroundjoin%
\pgfsetlinewidth{0.501875pt}%
\definecolor{currentstroke}{rgb}{0.000000,0.000000,1.000000}%
\pgfsetstrokecolor{currentstroke}%
\pgfsetstrokeopacity{0.600000}%
\pgfsetdash{}{0pt}%
\pgfpathmoveto{\pgfqpoint{3.137871in}{2.535715in}}%
\pgfpathlineto{\pgfqpoint{3.244738in}{2.767792in}}%
\pgfusepath{stroke}%
\end{pgfscope}%
\begin{pgfscope}%
\pgfpathrectangle{\pgfqpoint{0.100000in}{0.183744in}}{\pgfqpoint{4.506048in}{4.506048in}}%
\pgfusepath{clip}%
\pgfsetrectcap%
\pgfsetroundjoin%
\pgfsetlinewidth{0.501875pt}%
\definecolor{currentstroke}{rgb}{0.000000,0.000000,1.000000}%
\pgfsetstrokecolor{currentstroke}%
\pgfsetstrokeopacity{0.600000}%
\pgfsetdash{}{0pt}%
\pgfpathmoveto{\pgfqpoint{3.618693in}{2.765119in}}%
\pgfpathlineto{\pgfqpoint{3.491922in}{2.909072in}}%
\pgfusepath{stroke}%
\end{pgfscope}%
\begin{pgfscope}%
\pgfpathrectangle{\pgfqpoint{0.100000in}{0.183744in}}{\pgfqpoint{4.506048in}{4.506048in}}%
\pgfusepath{clip}%
\pgfsetrectcap%
\pgfsetroundjoin%
\pgfsetlinewidth{0.501875pt}%
\definecolor{currentstroke}{rgb}{0.000000,0.000000,1.000000}%
\pgfsetstrokecolor{currentstroke}%
\pgfsetstrokeopacity{0.600000}%
\pgfsetdash{}{0pt}%
\pgfpathmoveto{\pgfqpoint{2.381488in}{2.458600in}}%
\pgfpathlineto{\pgfqpoint{2.347466in}{2.170584in}}%
\pgfusepath{stroke}%
\end{pgfscope}%
\begin{pgfscope}%
\pgfpathrectangle{\pgfqpoint{0.100000in}{0.183744in}}{\pgfqpoint{4.506048in}{4.506048in}}%
\pgfusepath{clip}%
\pgfsetrectcap%
\pgfsetroundjoin%
\pgfsetlinewidth{0.501875pt}%
\definecolor{currentstroke}{rgb}{0.000000,0.000000,1.000000}%
\pgfsetstrokecolor{currentstroke}%
\pgfsetstrokeopacity{0.600000}%
\pgfsetdash{}{0pt}%
\pgfpathmoveto{\pgfqpoint{0.520237in}{3.538425in}}%
\pgfpathlineto{\pgfqpoint{0.502625in}{3.303205in}}%
\pgfusepath{stroke}%
\end{pgfscope}%
\begin{pgfscope}%
\pgfpathrectangle{\pgfqpoint{0.100000in}{0.183744in}}{\pgfqpoint{4.506048in}{4.506048in}}%
\pgfusepath{clip}%
\pgfsetrectcap%
\pgfsetroundjoin%
\pgfsetlinewidth{0.501875pt}%
\definecolor{currentstroke}{rgb}{0.000000,0.000000,1.000000}%
\pgfsetstrokecolor{currentstroke}%
\pgfsetstrokeopacity{0.600000}%
\pgfsetdash{}{0pt}%
\pgfpathmoveto{\pgfqpoint{3.034540in}{3.008258in}}%
\pgfpathlineto{\pgfqpoint{3.014340in}{3.171581in}}%
\pgfusepath{stroke}%
\end{pgfscope}%
\begin{pgfscope}%
\pgfpathrectangle{\pgfqpoint{0.100000in}{0.183744in}}{\pgfqpoint{4.506048in}{4.506048in}}%
\pgfusepath{clip}%
\pgfsetrectcap%
\pgfsetroundjoin%
\pgfsetlinewidth{0.501875pt}%
\definecolor{currentstroke}{rgb}{0.000000,0.000000,1.000000}%
\pgfsetstrokecolor{currentstroke}%
\pgfsetstrokeopacity{0.600000}%
\pgfsetdash{}{0pt}%
\pgfpathmoveto{\pgfqpoint{3.258667in}{1.968097in}}%
\pgfpathlineto{\pgfqpoint{3.291475in}{2.430534in}}%
\pgfusepath{stroke}%
\end{pgfscope}%
\begin{pgfscope}%
\pgfpathrectangle{\pgfqpoint{0.100000in}{0.183744in}}{\pgfqpoint{4.506048in}{4.506048in}}%
\pgfusepath{clip}%
\pgfsetrectcap%
\pgfsetroundjoin%
\pgfsetlinewidth{0.501875pt}%
\definecolor{currentstroke}{rgb}{0.000000,0.000000,1.000000}%
\pgfsetstrokecolor{currentstroke}%
\pgfsetstrokeopacity{0.600000}%
\pgfsetdash{}{0pt}%
\pgfpathmoveto{\pgfqpoint{2.002503in}{1.268556in}}%
\pgfpathlineto{\pgfqpoint{1.881654in}{1.433466in}}%
\pgfusepath{stroke}%
\end{pgfscope}%
\begin{pgfscope}%
\pgfpathrectangle{\pgfqpoint{0.100000in}{0.183744in}}{\pgfqpoint{4.506048in}{4.506048in}}%
\pgfusepath{clip}%
\pgfsetrectcap%
\pgfsetroundjoin%
\pgfsetlinewidth{0.501875pt}%
\definecolor{currentstroke}{rgb}{0.000000,0.000000,1.000000}%
\pgfsetstrokecolor{currentstroke}%
\pgfsetstrokeopacity{0.600000}%
\pgfsetdash{}{0pt}%
\pgfpathmoveto{\pgfqpoint{1.828730in}{3.729567in}}%
\pgfpathlineto{\pgfqpoint{2.002403in}{3.548087in}}%
\pgfusepath{stroke}%
\end{pgfscope}%
\begin{pgfscope}%
\pgfpathrectangle{\pgfqpoint{0.100000in}{0.183744in}}{\pgfqpoint{4.506048in}{4.506048in}}%
\pgfusepath{clip}%
\pgfsetrectcap%
\pgfsetroundjoin%
\pgfsetlinewidth{0.501875pt}%
\definecolor{currentstroke}{rgb}{0.000000,0.000000,1.000000}%
\pgfsetstrokecolor{currentstroke}%
\pgfsetstrokeopacity{0.600000}%
\pgfsetdash{}{0pt}%
\pgfpathmoveto{\pgfqpoint{2.662982in}{1.770595in}}%
\pgfpathlineto{\pgfqpoint{2.769625in}{1.500505in}}%
\pgfusepath{stroke}%
\end{pgfscope}%
\begin{pgfscope}%
\pgfpathrectangle{\pgfqpoint{0.100000in}{0.183744in}}{\pgfqpoint{4.506048in}{4.506048in}}%
\pgfusepath{clip}%
\pgfsetrectcap%
\pgfsetroundjoin%
\pgfsetlinewidth{0.501875pt}%
\definecolor{currentstroke}{rgb}{0.000000,0.000000,1.000000}%
\pgfsetstrokecolor{currentstroke}%
\pgfsetstrokeopacity{0.600000}%
\pgfsetdash{}{0pt}%
\pgfpathmoveto{\pgfqpoint{3.293510in}{2.146527in}}%
\pgfpathlineto{\pgfqpoint{3.190683in}{2.347913in}}%
\pgfusepath{stroke}%
\end{pgfscope}%
\begin{pgfscope}%
\pgfpathrectangle{\pgfqpoint{0.100000in}{0.183744in}}{\pgfqpoint{4.506048in}{4.506048in}}%
\pgfusepath{clip}%
\pgfsetrectcap%
\pgfsetroundjoin%
\pgfsetlinewidth{0.501875pt}%
\definecolor{currentstroke}{rgb}{0.000000,0.000000,1.000000}%
\pgfsetstrokecolor{currentstroke}%
\pgfsetstrokeopacity{0.600000}%
\pgfsetdash{}{0pt}%
\pgfpathmoveto{\pgfqpoint{2.376051in}{2.377553in}}%
\pgfpathlineto{\pgfqpoint{2.201797in}{2.077700in}}%
\pgfusepath{stroke}%
\end{pgfscope}%
\begin{pgfscope}%
\pgfpathrectangle{\pgfqpoint{0.100000in}{0.183744in}}{\pgfqpoint{4.506048in}{4.506048in}}%
\pgfusepath{clip}%
\pgfsetrectcap%
\pgfsetroundjoin%
\pgfsetlinewidth{0.501875pt}%
\definecolor{currentstroke}{rgb}{0.000000,0.000000,1.000000}%
\pgfsetstrokecolor{currentstroke}%
\pgfsetstrokeopacity{0.600000}%
\pgfsetdash{}{0pt}%
\pgfpathmoveto{\pgfqpoint{2.165838in}{3.602371in}}%
\pgfpathlineto{\pgfqpoint{2.207728in}{3.405987in}}%
\pgfusepath{stroke}%
\end{pgfscope}%
\begin{pgfscope}%
\pgfpathrectangle{\pgfqpoint{0.100000in}{0.183744in}}{\pgfqpoint{4.506048in}{4.506048in}}%
\pgfusepath{clip}%
\pgfsetrectcap%
\pgfsetroundjoin%
\pgfsetlinewidth{0.501875pt}%
\definecolor{currentstroke}{rgb}{0.000000,0.000000,1.000000}%
\pgfsetstrokecolor{currentstroke}%
\pgfsetstrokeopacity{0.600000}%
\pgfsetdash{}{0pt}%
\pgfpathmoveto{\pgfqpoint{3.806294in}{2.111896in}}%
\pgfpathlineto{\pgfqpoint{3.734028in}{2.515510in}}%
\pgfusepath{stroke}%
\end{pgfscope}%
\begin{pgfscope}%
\pgfpathrectangle{\pgfqpoint{0.100000in}{0.183744in}}{\pgfqpoint{4.506048in}{4.506048in}}%
\pgfusepath{clip}%
\pgfsetrectcap%
\pgfsetroundjoin%
\pgfsetlinewidth{0.501875pt}%
\definecolor{currentstroke}{rgb}{0.000000,0.000000,1.000000}%
\pgfsetstrokecolor{currentstroke}%
\pgfsetstrokeopacity{0.600000}%
\pgfsetdash{}{0pt}%
\pgfpathmoveto{\pgfqpoint{2.809618in}{1.556324in}}%
\pgfpathlineto{\pgfqpoint{2.900010in}{1.912532in}}%
\pgfusepath{stroke}%
\end{pgfscope}%
\begin{pgfscope}%
\pgfpathrectangle{\pgfqpoint{0.100000in}{0.183744in}}{\pgfqpoint{4.506048in}{4.506048in}}%
\pgfusepath{clip}%
\pgfsetrectcap%
\pgfsetroundjoin%
\pgfsetlinewidth{0.501875pt}%
\definecolor{currentstroke}{rgb}{0.000000,0.000000,1.000000}%
\pgfsetstrokecolor{currentstroke}%
\pgfsetstrokeopacity{0.600000}%
\pgfsetdash{}{0pt}%
\pgfpathmoveto{\pgfqpoint{2.968714in}{2.583374in}}%
\pgfpathlineto{\pgfqpoint{2.951187in}{2.689865in}}%
\pgfusepath{stroke}%
\end{pgfscope}%
\begin{pgfscope}%
\pgfpathrectangle{\pgfqpoint{0.100000in}{0.183744in}}{\pgfqpoint{4.506048in}{4.506048in}}%
\pgfusepath{clip}%
\pgfsetrectcap%
\pgfsetroundjoin%
\pgfsetlinewidth{0.501875pt}%
\definecolor{currentstroke}{rgb}{0.000000,0.000000,1.000000}%
\pgfsetstrokecolor{currentstroke}%
\pgfsetstrokeopacity{0.600000}%
\pgfsetdash{}{0pt}%
\pgfpathmoveto{\pgfqpoint{1.727304in}{2.622384in}}%
\pgfpathlineto{\pgfqpoint{1.830387in}{2.363761in}}%
\pgfusepath{stroke}%
\end{pgfscope}%
\begin{pgfscope}%
\pgfpathrectangle{\pgfqpoint{0.100000in}{0.183744in}}{\pgfqpoint{4.506048in}{4.506048in}}%
\pgfusepath{clip}%
\pgfsetrectcap%
\pgfsetroundjoin%
\pgfsetlinewidth{0.501875pt}%
\definecolor{currentstroke}{rgb}{0.000000,0.000000,1.000000}%
\pgfsetstrokecolor{currentstroke}%
\pgfsetstrokeopacity{0.600000}%
\pgfsetdash{}{0pt}%
\pgfpathmoveto{\pgfqpoint{2.662648in}{2.871875in}}%
\pgfpathlineto{\pgfqpoint{2.553467in}{2.829086in}}%
\pgfusepath{stroke}%
\end{pgfscope}%
\begin{pgfscope}%
\pgfpathrectangle{\pgfqpoint{0.100000in}{0.183744in}}{\pgfqpoint{4.506048in}{4.506048in}}%
\pgfusepath{clip}%
\pgfsetrectcap%
\pgfsetroundjoin%
\pgfsetlinewidth{0.501875pt}%
\definecolor{currentstroke}{rgb}{0.000000,0.000000,1.000000}%
\pgfsetstrokecolor{currentstroke}%
\pgfsetstrokeopacity{0.600000}%
\pgfsetdash{}{0pt}%
\pgfpathmoveto{\pgfqpoint{3.118725in}{1.901416in}}%
\pgfpathlineto{\pgfqpoint{3.096405in}{2.222617in}}%
\pgfusepath{stroke}%
\end{pgfscope}%
\begin{pgfscope}%
\pgfpathrectangle{\pgfqpoint{0.100000in}{0.183744in}}{\pgfqpoint{4.506048in}{4.506048in}}%
\pgfusepath{clip}%
\pgfsetrectcap%
\pgfsetroundjoin%
\pgfsetlinewidth{0.501875pt}%
\definecolor{currentstroke}{rgb}{0.000000,0.000000,1.000000}%
\pgfsetstrokecolor{currentstroke}%
\pgfsetstrokeopacity{0.600000}%
\pgfsetdash{}{0pt}%
\pgfpathmoveto{\pgfqpoint{2.609465in}{3.211702in}}%
\pgfpathlineto{\pgfqpoint{2.547801in}{3.210245in}}%
\pgfusepath{stroke}%
\end{pgfscope}%
\begin{pgfscope}%
\pgfpathrectangle{\pgfqpoint{0.100000in}{0.183744in}}{\pgfqpoint{4.506048in}{4.506048in}}%
\pgfusepath{clip}%
\pgfsetrectcap%
\pgfsetroundjoin%
\pgfsetlinewidth{0.501875pt}%
\definecolor{currentstroke}{rgb}{0.000000,0.000000,1.000000}%
\pgfsetstrokecolor{currentstroke}%
\pgfsetstrokeopacity{0.600000}%
\pgfsetdash{}{0pt}%
\pgfpathmoveto{\pgfqpoint{2.251170in}{1.744225in}}%
\pgfpathlineto{\pgfqpoint{2.178879in}{1.950188in}}%
\pgfusepath{stroke}%
\end{pgfscope}%
\begin{pgfscope}%
\pgfpathrectangle{\pgfqpoint{0.100000in}{0.183744in}}{\pgfqpoint{4.506048in}{4.506048in}}%
\pgfusepath{clip}%
\pgfsetrectcap%
\pgfsetroundjoin%
\pgfsetlinewidth{0.501875pt}%
\definecolor{currentstroke}{rgb}{0.000000,0.000000,1.000000}%
\pgfsetstrokecolor{currentstroke}%
\pgfsetstrokeopacity{0.600000}%
\pgfsetdash{}{0pt}%
\pgfpathmoveto{\pgfqpoint{1.217168in}{2.844019in}}%
\pgfpathlineto{\pgfqpoint{1.076121in}{2.593630in}}%
\pgfusepath{stroke}%
\end{pgfscope}%
\begin{pgfscope}%
\pgfpathrectangle{\pgfqpoint{0.100000in}{0.183744in}}{\pgfqpoint{4.506048in}{4.506048in}}%
\pgfusepath{clip}%
\pgfsetrectcap%
\pgfsetroundjoin%
\pgfsetlinewidth{0.501875pt}%
\definecolor{currentstroke}{rgb}{0.000000,0.000000,1.000000}%
\pgfsetstrokecolor{currentstroke}%
\pgfsetstrokeopacity{0.600000}%
\pgfsetdash{}{0pt}%
\pgfpathmoveto{\pgfqpoint{4.066098in}{1.743541in}}%
\pgfpathlineto{\pgfqpoint{4.006394in}{2.096968in}}%
\pgfusepath{stroke}%
\end{pgfscope}%
\begin{pgfscope}%
\pgfpathrectangle{\pgfqpoint{0.100000in}{0.183744in}}{\pgfqpoint{4.506048in}{4.506048in}}%
\pgfusepath{clip}%
\pgfsetrectcap%
\pgfsetroundjoin%
\pgfsetlinewidth{0.501875pt}%
\definecolor{currentstroke}{rgb}{0.000000,0.000000,1.000000}%
\pgfsetstrokecolor{currentstroke}%
\pgfsetstrokeopacity{0.600000}%
\pgfsetdash{}{0pt}%
\pgfpathmoveto{\pgfqpoint{3.014394in}{1.995204in}}%
\pgfpathlineto{\pgfqpoint{3.084691in}{2.346126in}}%
\pgfusepath{stroke}%
\end{pgfscope}%
\begin{pgfscope}%
\pgfpathrectangle{\pgfqpoint{0.100000in}{0.183744in}}{\pgfqpoint{4.506048in}{4.506048in}}%
\pgfusepath{clip}%
\pgfsetrectcap%
\pgfsetroundjoin%
\pgfsetlinewidth{0.501875pt}%
\definecolor{currentstroke}{rgb}{0.000000,0.000000,1.000000}%
\pgfsetstrokecolor{currentstroke}%
\pgfsetstrokeopacity{0.600000}%
\pgfsetdash{}{0pt}%
\pgfpathmoveto{\pgfqpoint{1.604401in}{2.169903in}}%
\pgfpathlineto{\pgfqpoint{1.414444in}{2.022530in}}%
\pgfusepath{stroke}%
\end{pgfscope}%
\begin{pgfscope}%
\pgfpathrectangle{\pgfqpoint{0.100000in}{0.183744in}}{\pgfqpoint{4.506048in}{4.506048in}}%
\pgfusepath{clip}%
\pgfsetrectcap%
\pgfsetroundjoin%
\pgfsetlinewidth{0.501875pt}%
\definecolor{currentstroke}{rgb}{0.000000,0.000000,1.000000}%
\pgfsetstrokecolor{currentstroke}%
\pgfsetstrokeopacity{0.600000}%
\pgfsetdash{}{0pt}%
\pgfpathmoveto{\pgfqpoint{1.866485in}{1.283549in}}%
\pgfpathlineto{\pgfqpoint{1.958619in}{1.532127in}}%
\pgfusepath{stroke}%
\end{pgfscope}%
\begin{pgfscope}%
\pgfpathrectangle{\pgfqpoint{0.100000in}{0.183744in}}{\pgfqpoint{4.506048in}{4.506048in}}%
\pgfusepath{clip}%
\pgfsetrectcap%
\pgfsetroundjoin%
\pgfsetlinewidth{0.501875pt}%
\definecolor{currentstroke}{rgb}{0.000000,0.000000,1.000000}%
\pgfsetstrokecolor{currentstroke}%
\pgfsetstrokeopacity{0.600000}%
\pgfsetdash{}{0pt}%
\pgfpathmoveto{\pgfqpoint{3.027764in}{2.481156in}}%
\pgfpathlineto{\pgfqpoint{3.070392in}{2.638401in}}%
\pgfusepath{stroke}%
\end{pgfscope}%
\begin{pgfscope}%
\pgfpathrectangle{\pgfqpoint{0.100000in}{0.183744in}}{\pgfqpoint{4.506048in}{4.506048in}}%
\pgfusepath{clip}%
\pgfsetrectcap%
\pgfsetroundjoin%
\pgfsetlinewidth{0.501875pt}%
\definecolor{currentstroke}{rgb}{0.000000,0.000000,1.000000}%
\pgfsetstrokecolor{currentstroke}%
\pgfsetstrokeopacity{0.600000}%
\pgfsetdash{}{0pt}%
\pgfpathmoveto{\pgfqpoint{2.119575in}{1.643162in}}%
\pgfpathlineto{\pgfqpoint{2.229763in}{1.299755in}}%
\pgfusepath{stroke}%
\end{pgfscope}%
\begin{pgfscope}%
\pgfpathrectangle{\pgfqpoint{0.100000in}{0.183744in}}{\pgfqpoint{4.506048in}{4.506048in}}%
\pgfusepath{clip}%
\pgfsetrectcap%
\pgfsetroundjoin%
\pgfsetlinewidth{0.501875pt}%
\definecolor{currentstroke}{rgb}{0.000000,0.000000,1.000000}%
\pgfsetstrokecolor{currentstroke}%
\pgfsetstrokeopacity{0.600000}%
\pgfsetdash{}{0pt}%
\pgfpathmoveto{\pgfqpoint{1.778880in}{1.598294in}}%
\pgfpathlineto{\pgfqpoint{1.799091in}{1.618021in}}%
\pgfusepath{stroke}%
\end{pgfscope}%
\begin{pgfscope}%
\pgfpathrectangle{\pgfqpoint{0.100000in}{0.183744in}}{\pgfqpoint{4.506048in}{4.506048in}}%
\pgfusepath{clip}%
\pgfsetrectcap%
\pgfsetroundjoin%
\pgfsetlinewidth{0.501875pt}%
\definecolor{currentstroke}{rgb}{0.000000,0.000000,1.000000}%
\pgfsetstrokecolor{currentstroke}%
\pgfsetstrokeopacity{0.600000}%
\pgfsetdash{}{0pt}%
\pgfpathmoveto{\pgfqpoint{3.349777in}{2.616305in}}%
\pgfpathlineto{\pgfqpoint{3.332038in}{2.626523in}}%
\pgfusepath{stroke}%
\end{pgfscope}%
\begin{pgfscope}%
\pgfpathrectangle{\pgfqpoint{0.100000in}{0.183744in}}{\pgfqpoint{4.506048in}{4.506048in}}%
\pgfusepath{clip}%
\pgfsetrectcap%
\pgfsetroundjoin%
\pgfsetlinewidth{0.501875pt}%
\definecolor{currentstroke}{rgb}{0.000000,0.000000,1.000000}%
\pgfsetstrokecolor{currentstroke}%
\pgfsetstrokeopacity{0.600000}%
\pgfsetdash{}{0pt}%
\pgfpathmoveto{\pgfqpoint{2.222796in}{2.269566in}}%
\pgfpathlineto{\pgfqpoint{2.300004in}{2.412828in}}%
\pgfusepath{stroke}%
\end{pgfscope}%
\begin{pgfscope}%
\pgfpathrectangle{\pgfqpoint{0.100000in}{0.183744in}}{\pgfqpoint{4.506048in}{4.506048in}}%
\pgfusepath{clip}%
\pgfsetrectcap%
\pgfsetroundjoin%
\pgfsetlinewidth{0.501875pt}%
\definecolor{currentstroke}{rgb}{0.000000,0.000000,1.000000}%
\pgfsetstrokecolor{currentstroke}%
\pgfsetstrokeopacity{0.600000}%
\pgfsetdash{}{0pt}%
\pgfpathmoveto{\pgfqpoint{1.446260in}{1.756610in}}%
\pgfpathlineto{\pgfqpoint{1.490739in}{1.789595in}}%
\pgfusepath{stroke}%
\end{pgfscope}%
\begin{pgfscope}%
\pgfpathrectangle{\pgfqpoint{0.100000in}{0.183744in}}{\pgfqpoint{4.506048in}{4.506048in}}%
\pgfusepath{clip}%
\pgfsetrectcap%
\pgfsetroundjoin%
\pgfsetlinewidth{0.501875pt}%
\definecolor{currentstroke}{rgb}{0.000000,0.000000,1.000000}%
\pgfsetstrokecolor{currentstroke}%
\pgfsetstrokeopacity{0.600000}%
\pgfsetdash{}{0pt}%
\pgfpathmoveto{\pgfqpoint{1.666961in}{2.757566in}}%
\pgfpathlineto{\pgfqpoint{1.615887in}{2.538231in}}%
\pgfusepath{stroke}%
\end{pgfscope}%
\begin{pgfscope}%
\pgfpathrectangle{\pgfqpoint{0.100000in}{0.183744in}}{\pgfqpoint{4.506048in}{4.506048in}}%
\pgfusepath{clip}%
\pgfsetrectcap%
\pgfsetroundjoin%
\pgfsetlinewidth{2.007500pt}%
\definecolor{currentstroke}{rgb}{0.000000,0.000000,1.000000}%
\pgfsetstrokecolor{currentstroke}%
\pgfsetdash{}{0pt}%
\pgfpathmoveto{\pgfqpoint{3.783241in}{1.686687in}}%
\pgfpathlineto{\pgfqpoint{3.751622in}{1.729707in}}%
\pgfusepath{stroke}%
\end{pgfscope}%
\begin{pgfscope}%
\pgfpathrectangle{\pgfqpoint{0.100000in}{0.183744in}}{\pgfqpoint{4.506048in}{4.506048in}}%
\pgfusepath{clip}%
\pgfsetrectcap%
\pgfsetroundjoin%
\pgfsetlinewidth{2.007500pt}%
\definecolor{currentstroke}{rgb}{0.000000,0.000000,1.000000}%
\pgfsetstrokecolor{currentstroke}%
\pgfsetdash{}{0pt}%
\pgfpathmoveto{\pgfqpoint{3.751622in}{1.729707in}}%
\pgfpathlineto{\pgfqpoint{3.550050in}{1.657797in}}%
\pgfusepath{stroke}%
\end{pgfscope}%
\begin{pgfscope}%
\pgfpathrectangle{\pgfqpoint{0.100000in}{0.183744in}}{\pgfqpoint{4.506048in}{4.506048in}}%
\pgfusepath{clip}%
\pgfsetrectcap%
\pgfsetroundjoin%
\pgfsetlinewidth{2.007500pt}%
\definecolor{currentstroke}{rgb}{0.000000,0.000000,1.000000}%
\pgfsetstrokecolor{currentstroke}%
\pgfsetdash{}{0pt}%
\pgfpathmoveto{\pgfqpoint{3.550050in}{1.657797in}}%
\pgfpathlineto{\pgfqpoint{3.581406in}{1.614522in}}%
\pgfusepath{stroke}%
\end{pgfscope}%
\begin{pgfscope}%
\pgfpathrectangle{\pgfqpoint{0.100000in}{0.183744in}}{\pgfqpoint{4.506048in}{4.506048in}}%
\pgfusepath{clip}%
\pgfsetrectcap%
\pgfsetroundjoin%
\pgfsetlinewidth{2.007500pt}%
\definecolor{currentstroke}{rgb}{0.000000,0.000000,1.000000}%
\pgfsetstrokecolor{currentstroke}%
\pgfsetdash{}{0pt}%
\pgfpathmoveto{\pgfqpoint{3.581406in}{1.614522in}}%
\pgfpathlineto{\pgfqpoint{3.783241in}{1.686687in}}%
\pgfusepath{stroke}%
\end{pgfscope}%
\begin{pgfscope}%
\pgfpathrectangle{\pgfqpoint{0.100000in}{0.183744in}}{\pgfqpoint{4.506048in}{4.506048in}}%
\pgfusepath{clip}%
\pgfsetrectcap%
\pgfsetroundjoin%
\pgfsetlinewidth{2.007500pt}%
\definecolor{currentstroke}{rgb}{0.000000,0.000000,1.000000}%
\pgfsetstrokecolor{currentstroke}%
\pgfsetdash{}{0pt}%
\pgfpathmoveto{\pgfqpoint{3.797224in}{1.935963in}}%
\pgfpathlineto{\pgfqpoint{3.765257in}{1.978976in}}%
\pgfusepath{stroke}%
\end{pgfscope}%
\begin{pgfscope}%
\pgfpathrectangle{\pgfqpoint{0.100000in}{0.183744in}}{\pgfqpoint{4.506048in}{4.506048in}}%
\pgfusepath{clip}%
\pgfsetrectcap%
\pgfsetroundjoin%
\pgfsetlinewidth{2.007500pt}%
\definecolor{currentstroke}{rgb}{0.000000,0.000000,1.000000}%
\pgfsetstrokecolor{currentstroke}%
\pgfsetdash{}{0pt}%
\pgfpathmoveto{\pgfqpoint{3.765257in}{1.978976in}}%
\pgfpathlineto{\pgfqpoint{3.561666in}{1.907077in}}%
\pgfusepath{stroke}%
\end{pgfscope}%
\begin{pgfscope}%
\pgfpathrectangle{\pgfqpoint{0.100000in}{0.183744in}}{\pgfqpoint{4.506048in}{4.506048in}}%
\pgfusepath{clip}%
\pgfsetrectcap%
\pgfsetroundjoin%
\pgfsetlinewidth{2.007500pt}%
\definecolor{currentstroke}{rgb}{0.000000,0.000000,1.000000}%
\pgfsetstrokecolor{currentstroke}%
\pgfsetdash{}{0pt}%
\pgfpathmoveto{\pgfqpoint{3.561666in}{1.907077in}}%
\pgfpathlineto{\pgfqpoint{3.593363in}{1.863806in}}%
\pgfusepath{stroke}%
\end{pgfscope}%
\begin{pgfscope}%
\pgfpathrectangle{\pgfqpoint{0.100000in}{0.183744in}}{\pgfqpoint{4.506048in}{4.506048in}}%
\pgfusepath{clip}%
\pgfsetrectcap%
\pgfsetroundjoin%
\pgfsetlinewidth{2.007500pt}%
\definecolor{currentstroke}{rgb}{0.000000,0.000000,1.000000}%
\pgfsetstrokecolor{currentstroke}%
\pgfsetdash{}{0pt}%
\pgfpathmoveto{\pgfqpoint{3.593363in}{1.863806in}}%
\pgfpathlineto{\pgfqpoint{3.797224in}{1.935963in}}%
\pgfusepath{stroke}%
\end{pgfscope}%
\begin{pgfscope}%
\pgfpathrectangle{\pgfqpoint{0.100000in}{0.183744in}}{\pgfqpoint{4.506048in}{4.506048in}}%
\pgfusepath{clip}%
\pgfsetrectcap%
\pgfsetroundjoin%
\pgfsetlinewidth{2.007500pt}%
\definecolor{currentstroke}{rgb}{0.000000,0.000000,1.000000}%
\pgfsetstrokecolor{currentstroke}%
\pgfsetdash{}{0pt}%
\pgfpathmoveto{\pgfqpoint{3.783241in}{1.686687in}}%
\pgfpathlineto{\pgfqpoint{3.797224in}{1.935963in}}%
\pgfusepath{stroke}%
\end{pgfscope}%
\begin{pgfscope}%
\pgfpathrectangle{\pgfqpoint{0.100000in}{0.183744in}}{\pgfqpoint{4.506048in}{4.506048in}}%
\pgfusepath{clip}%
\pgfsetrectcap%
\pgfsetroundjoin%
\pgfsetlinewidth{2.007500pt}%
\definecolor{currentstroke}{rgb}{0.000000,0.000000,1.000000}%
\pgfsetstrokecolor{currentstroke}%
\pgfsetdash{}{0pt}%
\pgfpathmoveto{\pgfqpoint{3.751622in}{1.729707in}}%
\pgfpathlineto{\pgfqpoint{3.765257in}{1.978976in}}%
\pgfusepath{stroke}%
\end{pgfscope}%
\begin{pgfscope}%
\pgfpathrectangle{\pgfqpoint{0.100000in}{0.183744in}}{\pgfqpoint{4.506048in}{4.506048in}}%
\pgfusepath{clip}%
\pgfsetrectcap%
\pgfsetroundjoin%
\pgfsetlinewidth{2.007500pt}%
\definecolor{currentstroke}{rgb}{0.000000,0.000000,1.000000}%
\pgfsetstrokecolor{currentstroke}%
\pgfsetdash{}{0pt}%
\pgfpathmoveto{\pgfqpoint{3.550050in}{1.657797in}}%
\pgfpathlineto{\pgfqpoint{3.561666in}{1.907077in}}%
\pgfusepath{stroke}%
\end{pgfscope}%
\begin{pgfscope}%
\pgfpathrectangle{\pgfqpoint{0.100000in}{0.183744in}}{\pgfqpoint{4.506048in}{4.506048in}}%
\pgfusepath{clip}%
\pgfsetrectcap%
\pgfsetroundjoin%
\pgfsetlinewidth{2.007500pt}%
\definecolor{currentstroke}{rgb}{0.000000,0.000000,1.000000}%
\pgfsetstrokecolor{currentstroke}%
\pgfsetdash{}{0pt}%
\pgfpathmoveto{\pgfqpoint{3.581406in}{1.614522in}}%
\pgfpathlineto{\pgfqpoint{3.593363in}{1.863806in}}%
\pgfusepath{stroke}%
\end{pgfscope}%
\begin{pgfscope}%
\pgfpathrectangle{\pgfqpoint{0.100000in}{0.183744in}}{\pgfqpoint{4.506048in}{4.506048in}}%
\pgfusepath{clip}%
\pgfsetrectcap%
\pgfsetroundjoin%
\pgfsetlinewidth{2.007500pt}%
\definecolor{currentstroke}{rgb}{1.000000,0.000000,0.000000}%
\pgfsetstrokecolor{currentstroke}%
\pgfsetdash{}{0pt}%
\pgfpathmoveto{\pgfqpoint{4.330912in}{1.265686in}}%
\pgfpathlineto{\pgfqpoint{3.027307in}{3.016389in}}%
\pgfusepath{stroke}%
\end{pgfscope}%
\begin{pgfscope}%
\pgfpathrectangle{\pgfqpoint{0.100000in}{0.183744in}}{\pgfqpoint{4.506048in}{4.506048in}}%
\pgfusepath{clip}%
\pgfsetrectcap%
\pgfsetroundjoin%
\pgfsetlinewidth{2.007500pt}%
\definecolor{currentstroke}{rgb}{1.000000,0.000000,0.000000}%
\pgfsetstrokecolor{currentstroke}%
\pgfsetdash{}{0pt}%
\pgfpathmoveto{\pgfqpoint{3.027307in}{3.016389in}}%
\pgfpathlineto{\pgfqpoint{2.935891in}{2.985830in}}%
\pgfusepath{stroke}%
\end{pgfscope}%
\begin{pgfscope}%
\pgfpathrectangle{\pgfqpoint{0.100000in}{0.183744in}}{\pgfqpoint{4.506048in}{4.506048in}}%
\pgfusepath{clip}%
\pgfsetrectcap%
\pgfsetroundjoin%
\pgfsetlinewidth{2.007500pt}%
\definecolor{currentstroke}{rgb}{1.000000,0.000000,0.000000}%
\pgfsetstrokecolor{currentstroke}%
\pgfsetdash{}{0pt}%
\pgfpathmoveto{\pgfqpoint{2.935891in}{2.985830in}}%
\pgfpathlineto{\pgfqpoint{4.234393in}{1.230294in}}%
\pgfusepath{stroke}%
\end{pgfscope}%
\begin{pgfscope}%
\pgfpathrectangle{\pgfqpoint{0.100000in}{0.183744in}}{\pgfqpoint{4.506048in}{4.506048in}}%
\pgfusepath{clip}%
\pgfsetrectcap%
\pgfsetroundjoin%
\pgfsetlinewidth{2.007500pt}%
\definecolor{currentstroke}{rgb}{1.000000,0.000000,0.000000}%
\pgfsetstrokecolor{currentstroke}%
\pgfsetdash{}{0pt}%
\pgfpathmoveto{\pgfqpoint{4.234393in}{1.230294in}}%
\pgfpathlineto{\pgfqpoint{4.330912in}{1.265686in}}%
\pgfusepath{stroke}%
\end{pgfscope}%
\begin{pgfscope}%
\pgfpathrectangle{\pgfqpoint{0.100000in}{0.183744in}}{\pgfqpoint{4.506048in}{4.506048in}}%
\pgfusepath{clip}%
\pgfsetrectcap%
\pgfsetroundjoin%
\pgfsetlinewidth{2.007500pt}%
\definecolor{currentstroke}{rgb}{1.000000,0.000000,0.000000}%
\pgfsetstrokecolor{currentstroke}%
\pgfsetdash{}{0pt}%
\pgfpathmoveto{\pgfqpoint{4.456164in}{2.833159in}}%
\pgfpathlineto{\pgfqpoint{3.064378in}{4.572108in}}%
\pgfusepath{stroke}%
\end{pgfscope}%
\begin{pgfscope}%
\pgfpathrectangle{\pgfqpoint{0.100000in}{0.183744in}}{\pgfqpoint{4.506048in}{4.506048in}}%
\pgfusepath{clip}%
\pgfsetrectcap%
\pgfsetroundjoin%
\pgfsetlinewidth{2.007500pt}%
\definecolor{currentstroke}{rgb}{1.000000,0.000000,0.000000}%
\pgfsetstrokecolor{currentstroke}%
\pgfsetdash{}{0pt}%
\pgfpathmoveto{\pgfqpoint{3.064378in}{4.572108in}}%
\pgfpathlineto{\pgfqpoint{2.967482in}{4.541891in}}%
\pgfusepath{stroke}%
\end{pgfscope}%
\begin{pgfscope}%
\pgfpathrectangle{\pgfqpoint{0.100000in}{0.183744in}}{\pgfqpoint{4.506048in}{4.506048in}}%
\pgfusepath{clip}%
\pgfsetrectcap%
\pgfsetroundjoin%
\pgfsetlinewidth{2.007500pt}%
\definecolor{currentstroke}{rgb}{1.000000,0.000000,0.000000}%
\pgfsetstrokecolor{currentstroke}%
\pgfsetdash{}{0pt}%
\pgfpathmoveto{\pgfqpoint{2.967482in}{4.541891in}}%
\pgfpathlineto{\pgfqpoint{4.353519in}{2.797839in}}%
\pgfusepath{stroke}%
\end{pgfscope}%
\begin{pgfscope}%
\pgfpathrectangle{\pgfqpoint{0.100000in}{0.183744in}}{\pgfqpoint{4.506048in}{4.506048in}}%
\pgfusepath{clip}%
\pgfsetrectcap%
\pgfsetroundjoin%
\pgfsetlinewidth{2.007500pt}%
\definecolor{currentstroke}{rgb}{1.000000,0.000000,0.000000}%
\pgfsetstrokecolor{currentstroke}%
\pgfsetdash{}{0pt}%
\pgfpathmoveto{\pgfqpoint{4.353519in}{2.797839in}}%
\pgfpathlineto{\pgfqpoint{4.456164in}{2.833159in}}%
\pgfusepath{stroke}%
\end{pgfscope}%
\begin{pgfscope}%
\pgfpathrectangle{\pgfqpoint{0.100000in}{0.183744in}}{\pgfqpoint{4.506048in}{4.506048in}}%
\pgfusepath{clip}%
\pgfsetrectcap%
\pgfsetroundjoin%
\pgfsetlinewidth{2.007500pt}%
\definecolor{currentstroke}{rgb}{1.000000,0.000000,0.000000}%
\pgfsetstrokecolor{currentstroke}%
\pgfsetdash{}{0pt}%
\pgfpathmoveto{\pgfqpoint{4.330912in}{1.265686in}}%
\pgfpathlineto{\pgfqpoint{4.456164in}{2.833159in}}%
\pgfusepath{stroke}%
\end{pgfscope}%
\begin{pgfscope}%
\pgfpathrectangle{\pgfqpoint{0.100000in}{0.183744in}}{\pgfqpoint{4.506048in}{4.506048in}}%
\pgfusepath{clip}%
\pgfsetrectcap%
\pgfsetroundjoin%
\pgfsetlinewidth{2.007500pt}%
\definecolor{currentstroke}{rgb}{1.000000,0.000000,0.000000}%
\pgfsetstrokecolor{currentstroke}%
\pgfsetdash{}{0pt}%
\pgfpathmoveto{\pgfqpoint{3.027307in}{3.016389in}}%
\pgfpathlineto{\pgfqpoint{3.064378in}{4.572108in}}%
\pgfusepath{stroke}%
\end{pgfscope}%
\begin{pgfscope}%
\pgfpathrectangle{\pgfqpoint{0.100000in}{0.183744in}}{\pgfqpoint{4.506048in}{4.506048in}}%
\pgfusepath{clip}%
\pgfsetrectcap%
\pgfsetroundjoin%
\pgfsetlinewidth{2.007500pt}%
\definecolor{currentstroke}{rgb}{1.000000,0.000000,0.000000}%
\pgfsetstrokecolor{currentstroke}%
\pgfsetdash{}{0pt}%
\pgfpathmoveto{\pgfqpoint{2.935891in}{2.985830in}}%
\pgfpathlineto{\pgfqpoint{2.967482in}{4.541891in}}%
\pgfusepath{stroke}%
\end{pgfscope}%
\begin{pgfscope}%
\pgfpathrectangle{\pgfqpoint{0.100000in}{0.183744in}}{\pgfqpoint{4.506048in}{4.506048in}}%
\pgfusepath{clip}%
\pgfsetrectcap%
\pgfsetroundjoin%
\pgfsetlinewidth{2.007500pt}%
\definecolor{currentstroke}{rgb}{1.000000,0.000000,0.000000}%
\pgfsetstrokecolor{currentstroke}%
\pgfsetdash{}{0pt}%
\pgfpathmoveto{\pgfqpoint{4.234393in}{1.230294in}}%
\pgfpathlineto{\pgfqpoint{4.353519in}{2.797839in}}%
\pgfusepath{stroke}%
\end{pgfscope}%
\begin{pgfscope}%
\pgfpathrectangle{\pgfqpoint{0.100000in}{0.183744in}}{\pgfqpoint{4.506048in}{4.506048in}}%
\pgfusepath{clip}%
\pgfsetrectcap%
\pgfsetroundjoin%
\pgfsetlinewidth{2.007500pt}%
\definecolor{currentstroke}{rgb}{1.000000,0.000000,0.000000}%
\pgfsetstrokecolor{currentstroke}%
\pgfsetdash{}{0pt}%
\pgfpathmoveto{\pgfqpoint{1.832197in}{0.349461in}}%
\pgfpathlineto{\pgfqpoint{0.666656in}{2.227249in}}%
\pgfusepath{stroke}%
\end{pgfscope}%
\begin{pgfscope}%
\pgfpathrectangle{\pgfqpoint{0.100000in}{0.183744in}}{\pgfqpoint{4.506048in}{4.506048in}}%
\pgfusepath{clip}%
\pgfsetrectcap%
\pgfsetroundjoin%
\pgfsetlinewidth{2.007500pt}%
\definecolor{currentstroke}{rgb}{1.000000,0.000000,0.000000}%
\pgfsetstrokecolor{currentstroke}%
\pgfsetdash{}{0pt}%
\pgfpathmoveto{\pgfqpoint{0.666656in}{2.227249in}}%
\pgfpathlineto{\pgfqpoint{0.568857in}{2.194556in}}%
\pgfusepath{stroke}%
\end{pgfscope}%
\begin{pgfscope}%
\pgfpathrectangle{\pgfqpoint{0.100000in}{0.183744in}}{\pgfqpoint{4.506048in}{4.506048in}}%
\pgfusepath{clip}%
\pgfsetrectcap%
\pgfsetroundjoin%
\pgfsetlinewidth{2.007500pt}%
\definecolor{currentstroke}{rgb}{1.000000,0.000000,0.000000}%
\pgfsetstrokecolor{currentstroke}%
\pgfsetdash{}{0pt}%
\pgfpathmoveto{\pgfqpoint{0.568857in}{2.194556in}}%
\pgfpathlineto{\pgfqpoint{1.728395in}{0.311399in}}%
\pgfusepath{stroke}%
\end{pgfscope}%
\begin{pgfscope}%
\pgfpathrectangle{\pgfqpoint{0.100000in}{0.183744in}}{\pgfqpoint{4.506048in}{4.506048in}}%
\pgfusepath{clip}%
\pgfsetrectcap%
\pgfsetroundjoin%
\pgfsetlinewidth{2.007500pt}%
\definecolor{currentstroke}{rgb}{1.000000,0.000000,0.000000}%
\pgfsetstrokecolor{currentstroke}%
\pgfsetdash{}{0pt}%
\pgfpathmoveto{\pgfqpoint{1.728395in}{0.311399in}}%
\pgfpathlineto{\pgfqpoint{1.832197in}{0.349461in}}%
\pgfusepath{stroke}%
\end{pgfscope}%
\begin{pgfscope}%
\pgfpathrectangle{\pgfqpoint{0.100000in}{0.183744in}}{\pgfqpoint{4.506048in}{4.506048in}}%
\pgfusepath{clip}%
\pgfsetrectcap%
\pgfsetroundjoin%
\pgfsetlinewidth{2.007500pt}%
\definecolor{currentstroke}{rgb}{1.000000,0.000000,0.000000}%
\pgfsetstrokecolor{currentstroke}%
\pgfsetdash{}{0pt}%
\pgfpathmoveto{\pgfqpoint{1.792686in}{1.916664in}}%
\pgfpathlineto{\pgfqpoint{0.557208in}{3.790251in}}%
\pgfusepath{stroke}%
\end{pgfscope}%
\begin{pgfscope}%
\pgfpathrectangle{\pgfqpoint{0.100000in}{0.183744in}}{\pgfqpoint{4.506048in}{4.506048in}}%
\pgfusepath{clip}%
\pgfsetrectcap%
\pgfsetroundjoin%
\pgfsetlinewidth{2.007500pt}%
\definecolor{currentstroke}{rgb}{1.000000,0.000000,0.000000}%
\pgfsetstrokecolor{currentstroke}%
\pgfsetdash{}{0pt}%
\pgfpathmoveto{\pgfqpoint{0.557208in}{3.790251in}}%
\pgfpathlineto{\pgfqpoint{0.453113in}{3.757789in}}%
\pgfusepath{stroke}%
\end{pgfscope}%
\begin{pgfscope}%
\pgfpathrectangle{\pgfqpoint{0.100000in}{0.183744in}}{\pgfqpoint{4.506048in}{4.506048in}}%
\pgfusepath{clip}%
\pgfsetrectcap%
\pgfsetroundjoin%
\pgfsetlinewidth{2.007500pt}%
\definecolor{currentstroke}{rgb}{1.000000,0.000000,0.000000}%
\pgfsetstrokecolor{currentstroke}%
\pgfsetdash{}{0pt}%
\pgfpathmoveto{\pgfqpoint{0.453113in}{3.757789in}}%
\pgfpathlineto{\pgfqpoint{1.681761in}{1.878494in}}%
\pgfusepath{stroke}%
\end{pgfscope}%
\begin{pgfscope}%
\pgfpathrectangle{\pgfqpoint{0.100000in}{0.183744in}}{\pgfqpoint{4.506048in}{4.506048in}}%
\pgfusepath{clip}%
\pgfsetrectcap%
\pgfsetroundjoin%
\pgfsetlinewidth{2.007500pt}%
\definecolor{currentstroke}{rgb}{1.000000,0.000000,0.000000}%
\pgfsetstrokecolor{currentstroke}%
\pgfsetdash{}{0pt}%
\pgfpathmoveto{\pgfqpoint{1.681761in}{1.878494in}}%
\pgfpathlineto{\pgfqpoint{1.792686in}{1.916664in}}%
\pgfusepath{stroke}%
\end{pgfscope}%
\begin{pgfscope}%
\pgfpathrectangle{\pgfqpoint{0.100000in}{0.183744in}}{\pgfqpoint{4.506048in}{4.506048in}}%
\pgfusepath{clip}%
\pgfsetrectcap%
\pgfsetroundjoin%
\pgfsetlinewidth{2.007500pt}%
\definecolor{currentstroke}{rgb}{1.000000,0.000000,0.000000}%
\pgfsetstrokecolor{currentstroke}%
\pgfsetdash{}{0pt}%
\pgfpathmoveto{\pgfqpoint{1.832197in}{0.349461in}}%
\pgfpathlineto{\pgfqpoint{1.792686in}{1.916664in}}%
\pgfusepath{stroke}%
\end{pgfscope}%
\begin{pgfscope}%
\pgfpathrectangle{\pgfqpoint{0.100000in}{0.183744in}}{\pgfqpoint{4.506048in}{4.506048in}}%
\pgfusepath{clip}%
\pgfsetrectcap%
\pgfsetroundjoin%
\pgfsetlinewidth{2.007500pt}%
\definecolor{currentstroke}{rgb}{1.000000,0.000000,0.000000}%
\pgfsetstrokecolor{currentstroke}%
\pgfsetdash{}{0pt}%
\pgfpathmoveto{\pgfqpoint{0.666656in}{2.227249in}}%
\pgfpathlineto{\pgfqpoint{0.557208in}{3.790251in}}%
\pgfusepath{stroke}%
\end{pgfscope}%
\begin{pgfscope}%
\pgfpathrectangle{\pgfqpoint{0.100000in}{0.183744in}}{\pgfqpoint{4.506048in}{4.506048in}}%
\pgfusepath{clip}%
\pgfsetrectcap%
\pgfsetroundjoin%
\pgfsetlinewidth{2.007500pt}%
\definecolor{currentstroke}{rgb}{1.000000,0.000000,0.000000}%
\pgfsetstrokecolor{currentstroke}%
\pgfsetdash{}{0pt}%
\pgfpathmoveto{\pgfqpoint{0.568857in}{2.194556in}}%
\pgfpathlineto{\pgfqpoint{0.453113in}{3.757789in}}%
\pgfusepath{stroke}%
\end{pgfscope}%
\begin{pgfscope}%
\pgfpathrectangle{\pgfqpoint{0.100000in}{0.183744in}}{\pgfqpoint{4.506048in}{4.506048in}}%
\pgfusepath{clip}%
\pgfsetrectcap%
\pgfsetroundjoin%
\pgfsetlinewidth{2.007500pt}%
\definecolor{currentstroke}{rgb}{1.000000,0.000000,0.000000}%
\pgfsetstrokecolor{currentstroke}%
\pgfsetdash{}{0pt}%
\pgfpathmoveto{\pgfqpoint{1.728395in}{0.311399in}}%
\pgfpathlineto{\pgfqpoint{1.681761in}{1.878494in}}%
\pgfusepath{stroke}%
\end{pgfscope}%
\begin{pgfscope}%
\pgfpathrectangle{\pgfqpoint{0.100000in}{0.183744in}}{\pgfqpoint{4.506048in}{4.506048in}}%
\pgfusepath{clip}%
\pgfsetrectcap%
\pgfsetroundjoin%
\pgfsetlinewidth{2.007500pt}%
\definecolor{currentstroke}{rgb}{1.000000,0.000000,0.000000}%
\pgfsetstrokecolor{currentstroke}%
\pgfsetdash{}{0pt}%
\pgfpathmoveto{\pgfqpoint{3.641810in}{2.031449in}}%
\pgfpathlineto{\pgfqpoint{3.568596in}{2.130432in}}%
\pgfusepath{stroke}%
\end{pgfscope}%
\begin{pgfscope}%
\pgfpathrectangle{\pgfqpoint{0.100000in}{0.183744in}}{\pgfqpoint{4.506048in}{4.506048in}}%
\pgfusepath{clip}%
\pgfsetrectcap%
\pgfsetroundjoin%
\pgfsetlinewidth{2.007500pt}%
\definecolor{currentstroke}{rgb}{1.000000,0.000000,0.000000}%
\pgfsetstrokecolor{currentstroke}%
\pgfsetdash{}{0pt}%
\pgfpathmoveto{\pgfqpoint{3.568596in}{2.130432in}}%
\pgfpathlineto{\pgfqpoint{2.420846in}{1.728814in}}%
\pgfusepath{stroke}%
\end{pgfscope}%
\begin{pgfscope}%
\pgfpathrectangle{\pgfqpoint{0.100000in}{0.183744in}}{\pgfqpoint{4.506048in}{4.506048in}}%
\pgfusepath{clip}%
\pgfsetrectcap%
\pgfsetroundjoin%
\pgfsetlinewidth{2.007500pt}%
\definecolor{currentstroke}{rgb}{1.000000,0.000000,0.000000}%
\pgfsetstrokecolor{currentstroke}%
\pgfsetdash{}{0pt}%
\pgfpathmoveto{\pgfqpoint{2.420846in}{1.728814in}}%
\pgfpathlineto{\pgfqpoint{2.490459in}{1.626468in}}%
\pgfusepath{stroke}%
\end{pgfscope}%
\begin{pgfscope}%
\pgfpathrectangle{\pgfqpoint{0.100000in}{0.183744in}}{\pgfqpoint{4.506048in}{4.506048in}}%
\pgfusepath{clip}%
\pgfsetrectcap%
\pgfsetroundjoin%
\pgfsetlinewidth{2.007500pt}%
\definecolor{currentstroke}{rgb}{1.000000,0.000000,0.000000}%
\pgfsetstrokecolor{currentstroke}%
\pgfsetdash{}{0pt}%
\pgfpathmoveto{\pgfqpoint{2.490459in}{1.626468in}}%
\pgfpathlineto{\pgfqpoint{3.641810in}{2.031449in}}%
\pgfusepath{stroke}%
\end{pgfscope}%
\begin{pgfscope}%
\pgfpathrectangle{\pgfqpoint{0.100000in}{0.183744in}}{\pgfqpoint{4.506048in}{4.506048in}}%
\pgfusepath{clip}%
\pgfsetrectcap%
\pgfsetroundjoin%
\pgfsetlinewidth{2.007500pt}%
\definecolor{currentstroke}{rgb}{1.000000,0.000000,0.000000}%
\pgfsetstrokecolor{currentstroke}%
\pgfsetdash{}{0pt}%
\pgfpathmoveto{\pgfqpoint{3.719398in}{3.595754in}}%
\pgfpathlineto{\pgfqpoint{3.641239in}{3.694103in}}%
\pgfusepath{stroke}%
\end{pgfscope}%
\begin{pgfscope}%
\pgfpathrectangle{\pgfqpoint{0.100000in}{0.183744in}}{\pgfqpoint{4.506048in}{4.506048in}}%
\pgfusepath{clip}%
\pgfsetrectcap%
\pgfsetroundjoin%
\pgfsetlinewidth{2.007500pt}%
\definecolor{currentstroke}{rgb}{1.000000,0.000000,0.000000}%
\pgfsetstrokecolor{currentstroke}%
\pgfsetdash{}{0pt}%
\pgfpathmoveto{\pgfqpoint{3.641239in}{3.694103in}}%
\pgfpathlineto{\pgfqpoint{2.421290in}{3.294739in}}%
\pgfusepath{stroke}%
\end{pgfscope}%
\begin{pgfscope}%
\pgfpathrectangle{\pgfqpoint{0.100000in}{0.183744in}}{\pgfqpoint{4.506048in}{4.506048in}}%
\pgfusepath{clip}%
\pgfsetrectcap%
\pgfsetroundjoin%
\pgfsetlinewidth{2.007500pt}%
\definecolor{currentstroke}{rgb}{1.000000,0.000000,0.000000}%
\pgfsetstrokecolor{currentstroke}%
\pgfsetdash{}{0pt}%
\pgfpathmoveto{\pgfqpoint{2.421290in}{3.294739in}}%
\pgfpathlineto{\pgfqpoint{2.495383in}{3.192832in}}%
\pgfusepath{stroke}%
\end{pgfscope}%
\begin{pgfscope}%
\pgfpathrectangle{\pgfqpoint{0.100000in}{0.183744in}}{\pgfqpoint{4.506048in}{4.506048in}}%
\pgfusepath{clip}%
\pgfsetrectcap%
\pgfsetroundjoin%
\pgfsetlinewidth{2.007500pt}%
\definecolor{currentstroke}{rgb}{1.000000,0.000000,0.000000}%
\pgfsetstrokecolor{currentstroke}%
\pgfsetdash{}{0pt}%
\pgfpathmoveto{\pgfqpoint{2.495383in}{3.192832in}}%
\pgfpathlineto{\pgfqpoint{3.719398in}{3.595754in}}%
\pgfusepath{stroke}%
\end{pgfscope}%
\begin{pgfscope}%
\pgfpathrectangle{\pgfqpoint{0.100000in}{0.183744in}}{\pgfqpoint{4.506048in}{4.506048in}}%
\pgfusepath{clip}%
\pgfsetrectcap%
\pgfsetroundjoin%
\pgfsetlinewidth{2.007500pt}%
\definecolor{currentstroke}{rgb}{1.000000,0.000000,0.000000}%
\pgfsetstrokecolor{currentstroke}%
\pgfsetdash{}{0pt}%
\pgfpathmoveto{\pgfqpoint{3.641810in}{2.031449in}}%
\pgfpathlineto{\pgfqpoint{3.719398in}{3.595754in}}%
\pgfusepath{stroke}%
\end{pgfscope}%
\begin{pgfscope}%
\pgfpathrectangle{\pgfqpoint{0.100000in}{0.183744in}}{\pgfqpoint{4.506048in}{4.506048in}}%
\pgfusepath{clip}%
\pgfsetrectcap%
\pgfsetroundjoin%
\pgfsetlinewidth{2.007500pt}%
\definecolor{currentstroke}{rgb}{1.000000,0.000000,0.000000}%
\pgfsetstrokecolor{currentstroke}%
\pgfsetdash{}{0pt}%
\pgfpathmoveto{\pgfqpoint{3.568596in}{2.130432in}}%
\pgfpathlineto{\pgfqpoint{3.641239in}{3.694103in}}%
\pgfusepath{stroke}%
\end{pgfscope}%
\begin{pgfscope}%
\pgfpathrectangle{\pgfqpoint{0.100000in}{0.183744in}}{\pgfqpoint{4.506048in}{4.506048in}}%
\pgfusepath{clip}%
\pgfsetrectcap%
\pgfsetroundjoin%
\pgfsetlinewidth{2.007500pt}%
\definecolor{currentstroke}{rgb}{1.000000,0.000000,0.000000}%
\pgfsetstrokecolor{currentstroke}%
\pgfsetdash{}{0pt}%
\pgfpathmoveto{\pgfqpoint{2.420846in}{1.728814in}}%
\pgfpathlineto{\pgfqpoint{2.421290in}{3.294739in}}%
\pgfusepath{stroke}%
\end{pgfscope}%
\begin{pgfscope}%
\pgfpathrectangle{\pgfqpoint{0.100000in}{0.183744in}}{\pgfqpoint{4.506048in}{4.506048in}}%
\pgfusepath{clip}%
\pgfsetrectcap%
\pgfsetroundjoin%
\pgfsetlinewidth{2.007500pt}%
\definecolor{currentstroke}{rgb}{1.000000,0.000000,0.000000}%
\pgfsetstrokecolor{currentstroke}%
\pgfsetdash{}{0pt}%
\pgfpathmoveto{\pgfqpoint{2.490459in}{1.626468in}}%
\pgfpathlineto{\pgfqpoint{2.495383in}{3.192832in}}%
\pgfusepath{stroke}%
\end{pgfscope}%
\begin{pgfscope}%
\pgfpathrectangle{\pgfqpoint{0.100000in}{0.183744in}}{\pgfqpoint{4.506048in}{4.506048in}}%
\pgfusepath{clip}%
\pgfsetbuttcap%
\pgfsetroundjoin%
\definecolor{currentfill}{rgb}{1.000000,0.647059,0.000000}%
\pgfsetfillcolor{currentfill}%
\pgfsetfillopacity{0.700000}%
\pgfsetlinewidth{1.003750pt}%
\definecolor{currentstroke}{rgb}{1.000000,0.647059,0.000000}%
\pgfsetstrokecolor{currentstroke}%
\pgfsetstrokeopacity{0.700000}%
\pgfsetdash{}{0pt}%
\pgfpathmoveto{\pgfqpoint{2.909228in}{3.118150in}}%
\pgfpathcurveto{\pgfqpoint{2.915052in}{3.118150in}}{\pgfqpoint{2.920639in}{3.120464in}}{\pgfqpoint{2.924757in}{3.124582in}}%
\pgfpathcurveto{\pgfqpoint{2.928875in}{3.128701in}}{\pgfqpoint{2.931189in}{3.134287in}}{\pgfqpoint{2.931189in}{3.140111in}}%
\pgfpathcurveto{\pgfqpoint{2.931189in}{3.145935in}}{\pgfqpoint{2.928875in}{3.151521in}}{\pgfqpoint{2.924757in}{3.155639in}}%
\pgfpathcurveto{\pgfqpoint{2.920639in}{3.159757in}}{\pgfqpoint{2.915052in}{3.162071in}}{\pgfqpoint{2.909228in}{3.162071in}}%
\pgfpathcurveto{\pgfqpoint{2.903405in}{3.162071in}}{\pgfqpoint{2.897818in}{3.159757in}}{\pgfqpoint{2.893700in}{3.155639in}}%
\pgfpathcurveto{\pgfqpoint{2.889582in}{3.151521in}}{\pgfqpoint{2.887268in}{3.145935in}}{\pgfqpoint{2.887268in}{3.140111in}}%
\pgfpathcurveto{\pgfqpoint{2.887268in}{3.134287in}}{\pgfqpoint{2.889582in}{3.128701in}}{\pgfqpoint{2.893700in}{3.124582in}}%
\pgfpathcurveto{\pgfqpoint{2.897818in}{3.120464in}}{\pgfqpoint{2.903405in}{3.118150in}}{\pgfqpoint{2.909228in}{3.118150in}}%
\pgfpathlineto{\pgfqpoint{2.909228in}{3.118150in}}%
\pgfpathclose%
\pgfusepath{stroke,fill}%
\end{pgfscope}%
\begin{pgfscope}%
\pgfpathrectangle{\pgfqpoint{0.100000in}{0.183744in}}{\pgfqpoint{4.506048in}{4.506048in}}%
\pgfusepath{clip}%
\pgfsetbuttcap%
\pgfsetroundjoin%
\definecolor{currentfill}{rgb}{1.000000,0.647059,0.000000}%
\pgfsetfillcolor{currentfill}%
\pgfsetfillopacity{0.700000}%
\pgfsetlinewidth{1.003750pt}%
\definecolor{currentstroke}{rgb}{1.000000,0.647059,0.000000}%
\pgfsetstrokecolor{currentstroke}%
\pgfsetstrokeopacity{0.700000}%
\pgfsetdash{}{0pt}%
\pgfpathmoveto{\pgfqpoint{2.890940in}{3.061836in}}%
\pgfpathcurveto{\pgfqpoint{2.896764in}{3.061836in}}{\pgfqpoint{2.902351in}{3.064150in}}{\pgfqpoint{2.906469in}{3.068268in}}%
\pgfpathcurveto{\pgfqpoint{2.910587in}{3.072386in}}{\pgfqpoint{2.912901in}{3.077972in}}{\pgfqpoint{2.912901in}{3.083796in}}%
\pgfpathcurveto{\pgfqpoint{2.912901in}{3.089620in}}{\pgfqpoint{2.910587in}{3.095206in}}{\pgfqpoint{2.906469in}{3.099325in}}%
\pgfpathcurveto{\pgfqpoint{2.902351in}{3.103443in}}{\pgfqpoint{2.896764in}{3.105757in}}{\pgfqpoint{2.890940in}{3.105757in}}%
\pgfpathcurveto{\pgfqpoint{2.885116in}{3.105757in}}{\pgfqpoint{2.879530in}{3.103443in}}{\pgfqpoint{2.875412in}{3.099325in}}%
\pgfpathcurveto{\pgfqpoint{2.871294in}{3.095206in}}{\pgfqpoint{2.868980in}{3.089620in}}{\pgfqpoint{2.868980in}{3.083796in}}%
\pgfpathcurveto{\pgfqpoint{2.868980in}{3.077972in}}{\pgfqpoint{2.871294in}{3.072386in}}{\pgfqpoint{2.875412in}{3.068268in}}%
\pgfpathcurveto{\pgfqpoint{2.879530in}{3.064150in}}{\pgfqpoint{2.885116in}{3.061836in}}{\pgfqpoint{2.890940in}{3.061836in}}%
\pgfpathlineto{\pgfqpoint{2.890940in}{3.061836in}}%
\pgfpathclose%
\pgfusepath{stroke,fill}%
\end{pgfscope}%
\begin{pgfscope}%
\pgfpathrectangle{\pgfqpoint{0.100000in}{0.183744in}}{\pgfqpoint{4.506048in}{4.506048in}}%
\pgfusepath{clip}%
\pgfsetbuttcap%
\pgfsetroundjoin%
\definecolor{currentfill}{rgb}{1.000000,0.647059,0.000000}%
\pgfsetfillcolor{currentfill}%
\pgfsetfillopacity{0.700000}%
\pgfsetlinewidth{1.003750pt}%
\definecolor{currentstroke}{rgb}{1.000000,0.647059,0.000000}%
\pgfsetstrokecolor{currentstroke}%
\pgfsetstrokeopacity{0.700000}%
\pgfsetdash{}{0pt}%
\pgfpathmoveto{\pgfqpoint{3.014340in}{3.149621in}}%
\pgfpathcurveto{\pgfqpoint{3.020164in}{3.149621in}}{\pgfqpoint{3.025750in}{3.151935in}}{\pgfqpoint{3.029869in}{3.156053in}}%
\pgfpathcurveto{\pgfqpoint{3.033987in}{3.160171in}}{\pgfqpoint{3.036301in}{3.165757in}}{\pgfqpoint{3.036301in}{3.171581in}}%
\pgfpathcurveto{\pgfqpoint{3.036301in}{3.177405in}}{\pgfqpoint{3.033987in}{3.182991in}}{\pgfqpoint{3.029869in}{3.187110in}}%
\pgfpathcurveto{\pgfqpoint{3.025750in}{3.191228in}}{\pgfqpoint{3.020164in}{3.193542in}}{\pgfqpoint{3.014340in}{3.193542in}}%
\pgfpathcurveto{\pgfqpoint{3.008516in}{3.193542in}}{\pgfqpoint{3.002930in}{3.191228in}}{\pgfqpoint{2.998812in}{3.187110in}}%
\pgfpathcurveto{\pgfqpoint{2.994694in}{3.182991in}}{\pgfqpoint{2.992380in}{3.177405in}}{\pgfqpoint{2.992380in}{3.171581in}}%
\pgfpathcurveto{\pgfqpoint{2.992380in}{3.165757in}}{\pgfqpoint{2.994694in}{3.160171in}}{\pgfqpoint{2.998812in}{3.156053in}}%
\pgfpathcurveto{\pgfqpoint{3.002930in}{3.151935in}}{\pgfqpoint{3.008516in}{3.149621in}}{\pgfqpoint{3.014340in}{3.149621in}}%
\pgfpathlineto{\pgfqpoint{3.014340in}{3.149621in}}%
\pgfpathclose%
\pgfusepath{stroke,fill}%
\end{pgfscope}%
\begin{pgfscope}%
\pgfpathrectangle{\pgfqpoint{0.100000in}{0.183744in}}{\pgfqpoint{4.506048in}{4.506048in}}%
\pgfusepath{clip}%
\pgfsetbuttcap%
\pgfsetroundjoin%
\definecolor{currentfill}{rgb}{1.000000,0.647059,0.000000}%
\pgfsetfillcolor{currentfill}%
\pgfsetfillopacity{0.700000}%
\pgfsetlinewidth{1.003750pt}%
\definecolor{currentstroke}{rgb}{1.000000,0.647059,0.000000}%
\pgfsetstrokecolor{currentstroke}%
\pgfsetstrokeopacity{0.700000}%
\pgfsetdash{}{0pt}%
\pgfpathmoveto{\pgfqpoint{2.203156in}{2.706915in}}%
\pgfpathcurveto{\pgfqpoint{2.208980in}{2.706915in}}{\pgfqpoint{2.214566in}{2.709229in}}{\pgfqpoint{2.218684in}{2.713347in}}%
\pgfpathcurveto{\pgfqpoint{2.222802in}{2.717466in}}{\pgfqpoint{2.225116in}{2.723052in}}{\pgfqpoint{2.225116in}{2.728876in}}%
\pgfpathcurveto{\pgfqpoint{2.225116in}{2.734700in}}{\pgfqpoint{2.222802in}{2.740286in}}{\pgfqpoint{2.218684in}{2.744404in}}%
\pgfpathcurveto{\pgfqpoint{2.214566in}{2.748522in}}{\pgfqpoint{2.208980in}{2.750836in}}{\pgfqpoint{2.203156in}{2.750836in}}%
\pgfpathcurveto{\pgfqpoint{2.197332in}{2.750836in}}{\pgfqpoint{2.191746in}{2.748522in}}{\pgfqpoint{2.187628in}{2.744404in}}%
\pgfpathcurveto{\pgfqpoint{2.183510in}{2.740286in}}{\pgfqpoint{2.181196in}{2.734700in}}{\pgfqpoint{2.181196in}{2.728876in}}%
\pgfpathcurveto{\pgfqpoint{2.181196in}{2.723052in}}{\pgfqpoint{2.183510in}{2.717466in}}{\pgfqpoint{2.187628in}{2.713347in}}%
\pgfpathcurveto{\pgfqpoint{2.191746in}{2.709229in}}{\pgfqpoint{2.197332in}{2.706915in}}{\pgfqpoint{2.203156in}{2.706915in}}%
\pgfpathlineto{\pgfqpoint{2.203156in}{2.706915in}}%
\pgfpathclose%
\pgfusepath{stroke,fill}%
\end{pgfscope}%
\begin{pgfscope}%
\pgfpathrectangle{\pgfqpoint{0.100000in}{0.183744in}}{\pgfqpoint{4.506048in}{4.506048in}}%
\pgfusepath{clip}%
\pgfsetbuttcap%
\pgfsetroundjoin%
\definecolor{currentfill}{rgb}{1.000000,0.647059,0.000000}%
\pgfsetfillcolor{currentfill}%
\pgfsetfillopacity{0.700000}%
\pgfsetlinewidth{1.003750pt}%
\definecolor{currentstroke}{rgb}{1.000000,0.647059,0.000000}%
\pgfsetstrokecolor{currentstroke}%
\pgfsetstrokeopacity{0.700000}%
\pgfsetdash{}{0pt}%
\pgfpathmoveto{\pgfqpoint{2.018336in}{2.846685in}}%
\pgfpathcurveto{\pgfqpoint{2.024160in}{2.846685in}}{\pgfqpoint{2.029746in}{2.848999in}}{\pgfqpoint{2.033865in}{2.853117in}}%
\pgfpathcurveto{\pgfqpoint{2.037983in}{2.857235in}}{\pgfqpoint{2.040297in}{2.862821in}}{\pgfqpoint{2.040297in}{2.868645in}}%
\pgfpathcurveto{\pgfqpoint{2.040297in}{2.874469in}}{\pgfqpoint{2.037983in}{2.880055in}}{\pgfqpoint{2.033865in}{2.884173in}}%
\pgfpathcurveto{\pgfqpoint{2.029746in}{2.888291in}}{\pgfqpoint{2.024160in}{2.890605in}}{\pgfqpoint{2.018336in}{2.890605in}}%
\pgfpathcurveto{\pgfqpoint{2.012512in}{2.890605in}}{\pgfqpoint{2.006926in}{2.888291in}}{\pgfqpoint{2.002808in}{2.884173in}}%
\pgfpathcurveto{\pgfqpoint{1.998690in}{2.880055in}}{\pgfqpoint{1.996376in}{2.874469in}}{\pgfqpoint{1.996376in}{2.868645in}}%
\pgfpathcurveto{\pgfqpoint{1.996376in}{2.862821in}}{\pgfqpoint{1.998690in}{2.857235in}}{\pgfqpoint{2.002808in}{2.853117in}}%
\pgfpathcurveto{\pgfqpoint{2.006926in}{2.848999in}}{\pgfqpoint{2.012512in}{2.846685in}}{\pgfqpoint{2.018336in}{2.846685in}}%
\pgfpathlineto{\pgfqpoint{2.018336in}{2.846685in}}%
\pgfpathclose%
\pgfusepath{stroke,fill}%
\end{pgfscope}%
\begin{pgfscope}%
\pgfpathrectangle{\pgfqpoint{0.100000in}{0.183744in}}{\pgfqpoint{4.506048in}{4.506048in}}%
\pgfusepath{clip}%
\pgfsetbuttcap%
\pgfsetroundjoin%
\definecolor{currentfill}{rgb}{1.000000,0.647059,0.000000}%
\pgfsetfillcolor{currentfill}%
\pgfsetfillopacity{0.700000}%
\pgfsetlinewidth{1.003750pt}%
\definecolor{currentstroke}{rgb}{1.000000,0.647059,0.000000}%
\pgfsetstrokecolor{currentstroke}%
\pgfsetstrokeopacity{0.700000}%
\pgfsetdash{}{0pt}%
\pgfpathmoveto{\pgfqpoint{2.444669in}{2.946316in}}%
\pgfpathcurveto{\pgfqpoint{2.450493in}{2.946316in}}{\pgfqpoint{2.456080in}{2.948630in}}{\pgfqpoint{2.460198in}{2.952748in}}%
\pgfpathcurveto{\pgfqpoint{2.464316in}{2.956866in}}{\pgfqpoint{2.466630in}{2.962453in}}{\pgfqpoint{2.466630in}{2.968276in}}%
\pgfpathcurveto{\pgfqpoint{2.466630in}{2.974100in}}{\pgfqpoint{2.464316in}{2.979687in}}{\pgfqpoint{2.460198in}{2.983805in}}%
\pgfpathcurveto{\pgfqpoint{2.456080in}{2.987923in}}{\pgfqpoint{2.450493in}{2.990237in}}{\pgfqpoint{2.444669in}{2.990237in}}%
\pgfpathcurveto{\pgfqpoint{2.438846in}{2.990237in}}{\pgfqpoint{2.433259in}{2.987923in}}{\pgfqpoint{2.429141in}{2.983805in}}%
\pgfpathcurveto{\pgfqpoint{2.425023in}{2.979687in}}{\pgfqpoint{2.422709in}{2.974100in}}{\pgfqpoint{2.422709in}{2.968276in}}%
\pgfpathcurveto{\pgfqpoint{2.422709in}{2.962453in}}{\pgfqpoint{2.425023in}{2.956866in}}{\pgfqpoint{2.429141in}{2.952748in}}%
\pgfpathcurveto{\pgfqpoint{2.433259in}{2.948630in}}{\pgfqpoint{2.438846in}{2.946316in}}{\pgfqpoint{2.444669in}{2.946316in}}%
\pgfpathlineto{\pgfqpoint{2.444669in}{2.946316in}}%
\pgfpathclose%
\pgfusepath{stroke,fill}%
\end{pgfscope}%
\begin{pgfscope}%
\pgfpathrectangle{\pgfqpoint{0.100000in}{0.183744in}}{\pgfqpoint{4.506048in}{4.506048in}}%
\pgfusepath{clip}%
\pgfsetbuttcap%
\pgfsetroundjoin%
\definecolor{currentfill}{rgb}{1.000000,0.647059,0.000000}%
\pgfsetfillcolor{currentfill}%
\pgfsetfillopacity{0.700000}%
\pgfsetlinewidth{1.003750pt}%
\definecolor{currentstroke}{rgb}{1.000000,0.647059,0.000000}%
\pgfsetstrokecolor{currentstroke}%
\pgfsetstrokeopacity{0.700000}%
\pgfsetdash{}{0pt}%
\pgfpathmoveto{\pgfqpoint{2.622496in}{3.348488in}}%
\pgfpathcurveto{\pgfqpoint{2.628320in}{3.348488in}}{\pgfqpoint{2.633906in}{3.350802in}}{\pgfqpoint{2.638024in}{3.354920in}}%
\pgfpathcurveto{\pgfqpoint{2.642142in}{3.359038in}}{\pgfqpoint{2.644456in}{3.364625in}}{\pgfqpoint{2.644456in}{3.370448in}}%
\pgfpathcurveto{\pgfqpoint{2.644456in}{3.376272in}}{\pgfqpoint{2.642142in}{3.381859in}}{\pgfqpoint{2.638024in}{3.385977in}}%
\pgfpathcurveto{\pgfqpoint{2.633906in}{3.390095in}}{\pgfqpoint{2.628320in}{3.392409in}}{\pgfqpoint{2.622496in}{3.392409in}}%
\pgfpathcurveto{\pgfqpoint{2.616672in}{3.392409in}}{\pgfqpoint{2.611086in}{3.390095in}}{\pgfqpoint{2.606968in}{3.385977in}}%
\pgfpathcurveto{\pgfqpoint{2.602850in}{3.381859in}}{\pgfqpoint{2.600536in}{3.376272in}}{\pgfqpoint{2.600536in}{3.370448in}}%
\pgfpathcurveto{\pgfqpoint{2.600536in}{3.364625in}}{\pgfqpoint{2.602850in}{3.359038in}}{\pgfqpoint{2.606968in}{3.354920in}}%
\pgfpathcurveto{\pgfqpoint{2.611086in}{3.350802in}}{\pgfqpoint{2.616672in}{3.348488in}}{\pgfqpoint{2.622496in}{3.348488in}}%
\pgfpathlineto{\pgfqpoint{2.622496in}{3.348488in}}%
\pgfpathclose%
\pgfusepath{stroke,fill}%
\end{pgfscope}%
\begin{pgfscope}%
\pgfpathrectangle{\pgfqpoint{0.100000in}{0.183744in}}{\pgfqpoint{4.506048in}{4.506048in}}%
\pgfusepath{clip}%
\pgfsetbuttcap%
\pgfsetroundjoin%
\definecolor{currentfill}{rgb}{1.000000,0.647059,0.000000}%
\pgfsetfillcolor{currentfill}%
\pgfsetfillopacity{0.700000}%
\pgfsetlinewidth{1.003750pt}%
\definecolor{currentstroke}{rgb}{1.000000,0.647059,0.000000}%
\pgfsetstrokecolor{currentstroke}%
\pgfsetstrokeopacity{0.700000}%
\pgfsetdash{}{0pt}%
\pgfpathmoveto{\pgfqpoint{1.111246in}{2.204332in}}%
\pgfpathcurveto{\pgfqpoint{1.117070in}{2.204332in}}{\pgfqpoint{1.122656in}{2.206645in}}{\pgfqpoint{1.126775in}{2.210764in}}%
\pgfpathcurveto{\pgfqpoint{1.130893in}{2.214882in}}{\pgfqpoint{1.133207in}{2.220468in}}{\pgfqpoint{1.133207in}{2.226292in}}%
\pgfpathcurveto{\pgfqpoint{1.133207in}{2.232116in}}{\pgfqpoint{1.130893in}{2.237702in}}{\pgfqpoint{1.126775in}{2.241820in}}%
\pgfpathcurveto{\pgfqpoint{1.122656in}{2.245938in}}{\pgfqpoint{1.117070in}{2.248252in}}{\pgfqpoint{1.111246in}{2.248252in}}%
\pgfpathcurveto{\pgfqpoint{1.105422in}{2.248252in}}{\pgfqpoint{1.099836in}{2.245938in}}{\pgfqpoint{1.095718in}{2.241820in}}%
\pgfpathcurveto{\pgfqpoint{1.091600in}{2.237702in}}{\pgfqpoint{1.089286in}{2.232116in}}{\pgfqpoint{1.089286in}{2.226292in}}%
\pgfpathcurveto{\pgfqpoint{1.089286in}{2.220468in}}{\pgfqpoint{1.091600in}{2.214882in}}{\pgfqpoint{1.095718in}{2.210764in}}%
\pgfpathcurveto{\pgfqpoint{1.099836in}{2.206645in}}{\pgfqpoint{1.105422in}{2.204332in}}{\pgfqpoint{1.111246in}{2.204332in}}%
\pgfpathlineto{\pgfqpoint{1.111246in}{2.204332in}}%
\pgfpathclose%
\pgfusepath{stroke,fill}%
\end{pgfscope}%
\begin{pgfscope}%
\pgfpathrectangle{\pgfqpoint{0.100000in}{0.183744in}}{\pgfqpoint{4.506048in}{4.506048in}}%
\pgfusepath{clip}%
\pgfsetbuttcap%
\pgfsetroundjoin%
\definecolor{currentfill}{rgb}{1.000000,0.647059,0.000000}%
\pgfsetfillcolor{currentfill}%
\pgfsetfillopacity{0.700000}%
\pgfsetlinewidth{1.003750pt}%
\definecolor{currentstroke}{rgb}{1.000000,0.647059,0.000000}%
\pgfsetstrokecolor{currentstroke}%
\pgfsetstrokeopacity{0.700000}%
\pgfsetdash{}{0pt}%
\pgfpathmoveto{\pgfqpoint{2.547801in}{3.188285in}}%
\pgfpathcurveto{\pgfqpoint{2.553625in}{3.188285in}}{\pgfqpoint{2.559211in}{3.190599in}}{\pgfqpoint{2.563329in}{3.194717in}}%
\pgfpathcurveto{\pgfqpoint{2.567447in}{3.198835in}}{\pgfqpoint{2.569761in}{3.204421in}}{\pgfqpoint{2.569761in}{3.210245in}}%
\pgfpathcurveto{\pgfqpoint{2.569761in}{3.216069in}}{\pgfqpoint{2.567447in}{3.221655in}}{\pgfqpoint{2.563329in}{3.225774in}}%
\pgfpathcurveto{\pgfqpoint{2.559211in}{3.229892in}}{\pgfqpoint{2.553625in}{3.232206in}}{\pgfqpoint{2.547801in}{3.232206in}}%
\pgfpathcurveto{\pgfqpoint{2.541977in}{3.232206in}}{\pgfqpoint{2.536391in}{3.229892in}}{\pgfqpoint{2.532273in}{3.225774in}}%
\pgfpathcurveto{\pgfqpoint{2.528154in}{3.221655in}}{\pgfqpoint{2.525841in}{3.216069in}}{\pgfqpoint{2.525841in}{3.210245in}}%
\pgfpathcurveto{\pgfqpoint{2.525841in}{3.204421in}}{\pgfqpoint{2.528154in}{3.198835in}}{\pgfqpoint{2.532273in}{3.194717in}}%
\pgfpathcurveto{\pgfqpoint{2.536391in}{3.190599in}}{\pgfqpoint{2.541977in}{3.188285in}}{\pgfqpoint{2.547801in}{3.188285in}}%
\pgfpathlineto{\pgfqpoint{2.547801in}{3.188285in}}%
\pgfpathclose%
\pgfusepath{stroke,fill}%
\end{pgfscope}%
\begin{pgfscope}%
\pgfpathrectangle{\pgfqpoint{0.100000in}{0.183744in}}{\pgfqpoint{4.506048in}{4.506048in}}%
\pgfusepath{clip}%
\pgfsetbuttcap%
\pgfsetroundjoin%
\definecolor{currentfill}{rgb}{1.000000,0.647059,0.000000}%
\pgfsetfillcolor{currentfill}%
\pgfsetfillopacity{0.700000}%
\pgfsetlinewidth{1.003750pt}%
\definecolor{currentstroke}{rgb}{1.000000,0.647059,0.000000}%
\pgfsetstrokecolor{currentstroke}%
\pgfsetstrokeopacity{0.700000}%
\pgfsetdash{}{0pt}%
\pgfpathmoveto{\pgfqpoint{1.648123in}{2.861781in}}%
\pgfpathcurveto{\pgfqpoint{1.653947in}{2.861781in}}{\pgfqpoint{1.659533in}{2.864095in}}{\pgfqpoint{1.663652in}{2.868213in}}%
\pgfpathcurveto{\pgfqpoint{1.667770in}{2.872331in}}{\pgfqpoint{1.670084in}{2.877917in}}{\pgfqpoint{1.670084in}{2.883741in}}%
\pgfpathcurveto{\pgfqpoint{1.670084in}{2.889565in}}{\pgfqpoint{1.667770in}{2.895151in}}{\pgfqpoint{1.663652in}{2.899270in}}%
\pgfpathcurveto{\pgfqpoint{1.659533in}{2.903388in}}{\pgfqpoint{1.653947in}{2.905702in}}{\pgfqpoint{1.648123in}{2.905702in}}%
\pgfpathcurveto{\pgfqpoint{1.642299in}{2.905702in}}{\pgfqpoint{1.636713in}{2.903388in}}{\pgfqpoint{1.632595in}{2.899270in}}%
\pgfpathcurveto{\pgfqpoint{1.628477in}{2.895151in}}{\pgfqpoint{1.626163in}{2.889565in}}{\pgfqpoint{1.626163in}{2.883741in}}%
\pgfpathcurveto{\pgfqpoint{1.626163in}{2.877917in}}{\pgfqpoint{1.628477in}{2.872331in}}{\pgfqpoint{1.632595in}{2.868213in}}%
\pgfpathcurveto{\pgfqpoint{1.636713in}{2.864095in}}{\pgfqpoint{1.642299in}{2.861781in}}{\pgfqpoint{1.648123in}{2.861781in}}%
\pgfpathlineto{\pgfqpoint{1.648123in}{2.861781in}}%
\pgfpathclose%
\pgfusepath{stroke,fill}%
\end{pgfscope}%
\begin{pgfscope}%
\pgfpathrectangle{\pgfqpoint{0.100000in}{0.183744in}}{\pgfqpoint{4.506048in}{4.506048in}}%
\pgfusepath{clip}%
\pgfsetbuttcap%
\pgfsetroundjoin%
\definecolor{currentfill}{rgb}{1.000000,0.647059,0.000000}%
\pgfsetfillcolor{currentfill}%
\pgfsetfillopacity{0.700000}%
\pgfsetlinewidth{1.003750pt}%
\definecolor{currentstroke}{rgb}{1.000000,0.647059,0.000000}%
\pgfsetstrokecolor{currentstroke}%
\pgfsetstrokeopacity{0.700000}%
\pgfsetdash{}{0pt}%
\pgfpathmoveto{\pgfqpoint{1.810496in}{2.416836in}}%
\pgfpathcurveto{\pgfqpoint{1.816320in}{2.416836in}}{\pgfqpoint{1.821906in}{2.419150in}}{\pgfqpoint{1.826025in}{2.423268in}}%
\pgfpathcurveto{\pgfqpoint{1.830143in}{2.427387in}}{\pgfqpoint{1.832457in}{2.432973in}}{\pgfqpoint{1.832457in}{2.438797in}}%
\pgfpathcurveto{\pgfqpoint{1.832457in}{2.444621in}}{\pgfqpoint{1.830143in}{2.450207in}}{\pgfqpoint{1.826025in}{2.454325in}}%
\pgfpathcurveto{\pgfqpoint{1.821906in}{2.458443in}}{\pgfqpoint{1.816320in}{2.460757in}}{\pgfqpoint{1.810496in}{2.460757in}}%
\pgfpathcurveto{\pgfqpoint{1.804672in}{2.460757in}}{\pgfqpoint{1.799086in}{2.458443in}}{\pgfqpoint{1.794968in}{2.454325in}}%
\pgfpathcurveto{\pgfqpoint{1.790850in}{2.450207in}}{\pgfqpoint{1.788536in}{2.444621in}}{\pgfqpoint{1.788536in}{2.438797in}}%
\pgfpathcurveto{\pgfqpoint{1.788536in}{2.432973in}}{\pgfqpoint{1.790850in}{2.427387in}}{\pgfqpoint{1.794968in}{2.423268in}}%
\pgfpathcurveto{\pgfqpoint{1.799086in}{2.419150in}}{\pgfqpoint{1.804672in}{2.416836in}}{\pgfqpoint{1.810496in}{2.416836in}}%
\pgfpathlineto{\pgfqpoint{1.810496in}{2.416836in}}%
\pgfpathclose%
\pgfusepath{stroke,fill}%
\end{pgfscope}%
\begin{pgfscope}%
\pgfpathrectangle{\pgfqpoint{0.100000in}{0.183744in}}{\pgfqpoint{4.506048in}{4.506048in}}%
\pgfusepath{clip}%
\pgfsetbuttcap%
\pgfsetroundjoin%
\definecolor{currentfill}{rgb}{1.000000,0.647059,0.000000}%
\pgfsetfillcolor{currentfill}%
\pgfsetfillopacity{0.700000}%
\pgfsetlinewidth{1.003750pt}%
\definecolor{currentstroke}{rgb}{1.000000,0.647059,0.000000}%
\pgfsetstrokecolor{currentstroke}%
\pgfsetstrokeopacity{0.700000}%
\pgfsetdash{}{0pt}%
\pgfpathmoveto{\pgfqpoint{1.597710in}{2.865569in}}%
\pgfpathcurveto{\pgfqpoint{1.603534in}{2.865569in}}{\pgfqpoint{1.609120in}{2.867883in}}{\pgfqpoint{1.613239in}{2.872001in}}%
\pgfpathcurveto{\pgfqpoint{1.617357in}{2.876119in}}{\pgfqpoint{1.619671in}{2.881705in}}{\pgfqpoint{1.619671in}{2.887529in}}%
\pgfpathcurveto{\pgfqpoint{1.619671in}{2.893353in}}{\pgfqpoint{1.617357in}{2.898939in}}{\pgfqpoint{1.613239in}{2.903057in}}%
\pgfpathcurveto{\pgfqpoint{1.609120in}{2.907176in}}{\pgfqpoint{1.603534in}{2.909489in}}{\pgfqpoint{1.597710in}{2.909489in}}%
\pgfpathcurveto{\pgfqpoint{1.591886in}{2.909489in}}{\pgfqpoint{1.586300in}{2.907176in}}{\pgfqpoint{1.582182in}{2.903057in}}%
\pgfpathcurveto{\pgfqpoint{1.578064in}{2.898939in}}{\pgfqpoint{1.575750in}{2.893353in}}{\pgfqpoint{1.575750in}{2.887529in}}%
\pgfpathcurveto{\pgfqpoint{1.575750in}{2.881705in}}{\pgfqpoint{1.578064in}{2.876119in}}{\pgfqpoint{1.582182in}{2.872001in}}%
\pgfpathcurveto{\pgfqpoint{1.586300in}{2.867883in}}{\pgfqpoint{1.591886in}{2.865569in}}{\pgfqpoint{1.597710in}{2.865569in}}%
\pgfpathlineto{\pgfqpoint{1.597710in}{2.865569in}}%
\pgfpathclose%
\pgfusepath{stroke,fill}%
\end{pgfscope}%
\begin{pgfscope}%
\pgfpathrectangle{\pgfqpoint{0.100000in}{0.183744in}}{\pgfqpoint{4.506048in}{4.506048in}}%
\pgfusepath{clip}%
\pgfsetbuttcap%
\pgfsetroundjoin%
\definecolor{currentfill}{rgb}{1.000000,0.647059,0.000000}%
\pgfsetfillcolor{currentfill}%
\pgfsetfillopacity{0.700000}%
\pgfsetlinewidth{1.003750pt}%
\definecolor{currentstroke}{rgb}{1.000000,0.647059,0.000000}%
\pgfsetstrokecolor{currentstroke}%
\pgfsetstrokeopacity{0.700000}%
\pgfsetdash{}{0pt}%
\pgfpathmoveto{\pgfqpoint{3.366545in}{2.964009in}}%
\pgfpathcurveto{\pgfqpoint{3.372369in}{2.964009in}}{\pgfqpoint{3.377955in}{2.966323in}}{\pgfqpoint{3.382074in}{2.970441in}}%
\pgfpathcurveto{\pgfqpoint{3.386192in}{2.974560in}}{\pgfqpoint{3.388506in}{2.980146in}}{\pgfqpoint{3.388506in}{2.985970in}}%
\pgfpathcurveto{\pgfqpoint{3.388506in}{2.991794in}}{\pgfqpoint{3.386192in}{2.997380in}}{\pgfqpoint{3.382074in}{3.001498in}}%
\pgfpathcurveto{\pgfqpoint{3.377955in}{3.005616in}}{\pgfqpoint{3.372369in}{3.007930in}}{\pgfqpoint{3.366545in}{3.007930in}}%
\pgfpathcurveto{\pgfqpoint{3.360721in}{3.007930in}}{\pgfqpoint{3.355135in}{3.005616in}}{\pgfqpoint{3.351017in}{3.001498in}}%
\pgfpathcurveto{\pgfqpoint{3.346899in}{2.997380in}}{\pgfqpoint{3.344585in}{2.991794in}}{\pgfqpoint{3.344585in}{2.985970in}}%
\pgfpathcurveto{\pgfqpoint{3.344585in}{2.980146in}}{\pgfqpoint{3.346899in}{2.974560in}}{\pgfqpoint{3.351017in}{2.970441in}}%
\pgfpathcurveto{\pgfqpoint{3.355135in}{2.966323in}}{\pgfqpoint{3.360721in}{2.964009in}}{\pgfqpoint{3.366545in}{2.964009in}}%
\pgfpathlineto{\pgfqpoint{3.366545in}{2.964009in}}%
\pgfpathclose%
\pgfusepath{stroke,fill}%
\end{pgfscope}%
\begin{pgfscope}%
\pgfpathrectangle{\pgfqpoint{0.100000in}{0.183744in}}{\pgfqpoint{4.506048in}{4.506048in}}%
\pgfusepath{clip}%
\pgfsetbuttcap%
\pgfsetroundjoin%
\definecolor{currentfill}{rgb}{1.000000,0.647059,0.000000}%
\pgfsetfillcolor{currentfill}%
\pgfsetfillopacity{0.700000}%
\pgfsetlinewidth{1.003750pt}%
\definecolor{currentstroke}{rgb}{1.000000,0.647059,0.000000}%
\pgfsetstrokecolor{currentstroke}%
\pgfsetstrokeopacity{0.700000}%
\pgfsetdash{}{0pt}%
\pgfpathmoveto{\pgfqpoint{2.303511in}{3.011904in}}%
\pgfpathcurveto{\pgfqpoint{2.309335in}{3.011904in}}{\pgfqpoint{2.314921in}{3.014218in}}{\pgfqpoint{2.319039in}{3.018336in}}%
\pgfpathcurveto{\pgfqpoint{2.323158in}{3.022454in}}{\pgfqpoint{2.325471in}{3.028040in}}{\pgfqpoint{2.325471in}{3.033864in}}%
\pgfpathcurveto{\pgfqpoint{2.325471in}{3.039688in}}{\pgfqpoint{2.323158in}{3.045274in}}{\pgfqpoint{2.319039in}{3.049393in}}%
\pgfpathcurveto{\pgfqpoint{2.314921in}{3.053511in}}{\pgfqpoint{2.309335in}{3.055825in}}{\pgfqpoint{2.303511in}{3.055825in}}%
\pgfpathcurveto{\pgfqpoint{2.297687in}{3.055825in}}{\pgfqpoint{2.292101in}{3.053511in}}{\pgfqpoint{2.287983in}{3.049393in}}%
\pgfpathcurveto{\pgfqpoint{2.283865in}{3.045274in}}{\pgfqpoint{2.281551in}{3.039688in}}{\pgfqpoint{2.281551in}{3.033864in}}%
\pgfpathcurveto{\pgfqpoint{2.281551in}{3.028040in}}{\pgfqpoint{2.283865in}{3.022454in}}{\pgfqpoint{2.287983in}{3.018336in}}%
\pgfpathcurveto{\pgfqpoint{2.292101in}{3.014218in}}{\pgfqpoint{2.297687in}{3.011904in}}{\pgfqpoint{2.303511in}{3.011904in}}%
\pgfpathlineto{\pgfqpoint{2.303511in}{3.011904in}}%
\pgfpathclose%
\pgfusepath{stroke,fill}%
\end{pgfscope}%
\begin{pgfscope}%
\pgfpathrectangle{\pgfqpoint{0.100000in}{0.183744in}}{\pgfqpoint{4.506048in}{4.506048in}}%
\pgfusepath{clip}%
\pgfsetbuttcap%
\pgfsetroundjoin%
\definecolor{currentfill}{rgb}{1.000000,0.647059,0.000000}%
\pgfsetfillcolor{currentfill}%
\pgfsetfillopacity{0.700000}%
\pgfsetlinewidth{1.003750pt}%
\definecolor{currentstroke}{rgb}{1.000000,0.647059,0.000000}%
\pgfsetstrokecolor{currentstroke}%
\pgfsetstrokeopacity{0.700000}%
\pgfsetdash{}{0pt}%
\pgfpathmoveto{\pgfqpoint{1.451334in}{2.850422in}}%
\pgfpathcurveto{\pgfqpoint{1.457158in}{2.850422in}}{\pgfqpoint{1.462744in}{2.852736in}}{\pgfqpoint{1.466862in}{2.856854in}}%
\pgfpathcurveto{\pgfqpoint{1.470980in}{2.860972in}}{\pgfqpoint{1.473294in}{2.866558in}}{\pgfqpoint{1.473294in}{2.872382in}}%
\pgfpathcurveto{\pgfqpoint{1.473294in}{2.878206in}}{\pgfqpoint{1.470980in}{2.883792in}}{\pgfqpoint{1.466862in}{2.887910in}}%
\pgfpathcurveto{\pgfqpoint{1.462744in}{2.892028in}}{\pgfqpoint{1.457158in}{2.894342in}}{\pgfqpoint{1.451334in}{2.894342in}}%
\pgfpathcurveto{\pgfqpoint{1.445510in}{2.894342in}}{\pgfqpoint{1.439924in}{2.892028in}}{\pgfqpoint{1.435806in}{2.887910in}}%
\pgfpathcurveto{\pgfqpoint{1.431688in}{2.883792in}}{\pgfqpoint{1.429374in}{2.878206in}}{\pgfqpoint{1.429374in}{2.872382in}}%
\pgfpathcurveto{\pgfqpoint{1.429374in}{2.866558in}}{\pgfqpoint{1.431688in}{2.860972in}}{\pgfqpoint{1.435806in}{2.856854in}}%
\pgfpathcurveto{\pgfqpoint{1.439924in}{2.852736in}}{\pgfqpoint{1.445510in}{2.850422in}}{\pgfqpoint{1.451334in}{2.850422in}}%
\pgfpathlineto{\pgfqpoint{1.451334in}{2.850422in}}%
\pgfpathclose%
\pgfusepath{stroke,fill}%
\end{pgfscope}%
\begin{pgfscope}%
\pgfpathrectangle{\pgfqpoint{0.100000in}{0.183744in}}{\pgfqpoint{4.506048in}{4.506048in}}%
\pgfusepath{clip}%
\pgfsetbuttcap%
\pgfsetroundjoin%
\definecolor{currentfill}{rgb}{1.000000,0.647059,0.000000}%
\pgfsetfillcolor{currentfill}%
\pgfsetfillopacity{0.700000}%
\pgfsetlinewidth{1.003750pt}%
\definecolor{currentstroke}{rgb}{1.000000,0.647059,0.000000}%
\pgfsetstrokecolor{currentstroke}%
\pgfsetstrokeopacity{0.700000}%
\pgfsetdash{}{0pt}%
\pgfpathmoveto{\pgfqpoint{2.165247in}{3.017137in}}%
\pgfpathcurveto{\pgfqpoint{2.171071in}{3.017137in}}{\pgfqpoint{2.176657in}{3.019451in}}{\pgfqpoint{2.180775in}{3.023569in}}%
\pgfpathcurveto{\pgfqpoint{2.184893in}{3.027687in}}{\pgfqpoint{2.187207in}{3.033273in}}{\pgfqpoint{2.187207in}{3.039097in}}%
\pgfpathcurveto{\pgfqpoint{2.187207in}{3.044921in}}{\pgfqpoint{2.184893in}{3.050507in}}{\pgfqpoint{2.180775in}{3.054626in}}%
\pgfpathcurveto{\pgfqpoint{2.176657in}{3.058744in}}{\pgfqpoint{2.171071in}{3.061058in}}{\pgfqpoint{2.165247in}{3.061058in}}%
\pgfpathcurveto{\pgfqpoint{2.159423in}{3.061058in}}{\pgfqpoint{2.153837in}{3.058744in}}{\pgfqpoint{2.149719in}{3.054626in}}%
\pgfpathcurveto{\pgfqpoint{2.145601in}{3.050507in}}{\pgfqpoint{2.143287in}{3.044921in}}{\pgfqpoint{2.143287in}{3.039097in}}%
\pgfpathcurveto{\pgfqpoint{2.143287in}{3.033273in}}{\pgfqpoint{2.145601in}{3.027687in}}{\pgfqpoint{2.149719in}{3.023569in}}%
\pgfpathcurveto{\pgfqpoint{2.153837in}{3.019451in}}{\pgfqpoint{2.159423in}{3.017137in}}{\pgfqpoint{2.165247in}{3.017137in}}%
\pgfpathlineto{\pgfqpoint{2.165247in}{3.017137in}}%
\pgfpathclose%
\pgfusepath{stroke,fill}%
\end{pgfscope}%
\begin{pgfscope}%
\pgfpathrectangle{\pgfqpoint{0.100000in}{0.183744in}}{\pgfqpoint{4.506048in}{4.506048in}}%
\pgfusepath{clip}%
\pgfsetbuttcap%
\pgfsetroundjoin%
\definecolor{currentfill}{rgb}{1.000000,0.647059,0.000000}%
\pgfsetfillcolor{currentfill}%
\pgfsetfillopacity{0.700000}%
\pgfsetlinewidth{1.003750pt}%
\definecolor{currentstroke}{rgb}{1.000000,0.647059,0.000000}%
\pgfsetstrokecolor{currentstroke}%
\pgfsetstrokeopacity{0.700000}%
\pgfsetdash{}{0pt}%
\pgfpathmoveto{\pgfqpoint{1.941433in}{2.741134in}}%
\pgfpathcurveto{\pgfqpoint{1.947257in}{2.741134in}}{\pgfqpoint{1.952843in}{2.743448in}}{\pgfqpoint{1.956961in}{2.747566in}}%
\pgfpathcurveto{\pgfqpoint{1.961079in}{2.751684in}}{\pgfqpoint{1.963393in}{2.757270in}}{\pgfqpoint{1.963393in}{2.763094in}}%
\pgfpathcurveto{\pgfqpoint{1.963393in}{2.768918in}}{\pgfqpoint{1.961079in}{2.774504in}}{\pgfqpoint{1.956961in}{2.778622in}}%
\pgfpathcurveto{\pgfqpoint{1.952843in}{2.782741in}}{\pgfqpoint{1.947257in}{2.785054in}}{\pgfqpoint{1.941433in}{2.785054in}}%
\pgfpathcurveto{\pgfqpoint{1.935609in}{2.785054in}}{\pgfqpoint{1.930023in}{2.782741in}}{\pgfqpoint{1.925904in}{2.778622in}}%
\pgfpathcurveto{\pgfqpoint{1.921786in}{2.774504in}}{\pgfqpoint{1.919472in}{2.768918in}}{\pgfqpoint{1.919472in}{2.763094in}}%
\pgfpathcurveto{\pgfqpoint{1.919472in}{2.757270in}}{\pgfqpoint{1.921786in}{2.751684in}}{\pgfqpoint{1.925904in}{2.747566in}}%
\pgfpathcurveto{\pgfqpoint{1.930023in}{2.743448in}}{\pgfqpoint{1.935609in}{2.741134in}}{\pgfqpoint{1.941433in}{2.741134in}}%
\pgfpathlineto{\pgfqpoint{1.941433in}{2.741134in}}%
\pgfpathclose%
\pgfusepath{stroke,fill}%
\end{pgfscope}%
\begin{pgfscope}%
\pgfpathrectangle{\pgfqpoint{0.100000in}{0.183744in}}{\pgfqpoint{4.506048in}{4.506048in}}%
\pgfusepath{clip}%
\pgfsetbuttcap%
\pgfsetroundjoin%
\definecolor{currentfill}{rgb}{1.000000,0.647059,0.000000}%
\pgfsetfillcolor{currentfill}%
\pgfsetfillopacity{0.700000}%
\pgfsetlinewidth{1.003750pt}%
\definecolor{currentstroke}{rgb}{1.000000,0.647059,0.000000}%
\pgfsetstrokecolor{currentstroke}%
\pgfsetstrokeopacity{0.700000}%
\pgfsetdash{}{0pt}%
\pgfpathmoveto{\pgfqpoint{2.587608in}{3.274175in}}%
\pgfpathcurveto{\pgfqpoint{2.593432in}{3.274175in}}{\pgfqpoint{2.599018in}{3.276489in}}{\pgfqpoint{2.603137in}{3.280607in}}%
\pgfpathcurveto{\pgfqpoint{2.607255in}{3.284725in}}{\pgfqpoint{2.609569in}{3.290311in}}{\pgfqpoint{2.609569in}{3.296135in}}%
\pgfpathcurveto{\pgfqpoint{2.609569in}{3.301959in}}{\pgfqpoint{2.607255in}{3.307545in}}{\pgfqpoint{2.603137in}{3.311663in}}%
\pgfpathcurveto{\pgfqpoint{2.599018in}{3.315782in}}{\pgfqpoint{2.593432in}{3.318096in}}{\pgfqpoint{2.587608in}{3.318096in}}%
\pgfpathcurveto{\pgfqpoint{2.581784in}{3.318096in}}{\pgfqpoint{2.576198in}{3.315782in}}{\pgfqpoint{2.572080in}{3.311663in}}%
\pgfpathcurveto{\pgfqpoint{2.567962in}{3.307545in}}{\pgfqpoint{2.565648in}{3.301959in}}{\pgfqpoint{2.565648in}{3.296135in}}%
\pgfpathcurveto{\pgfqpoint{2.565648in}{3.290311in}}{\pgfqpoint{2.567962in}{3.284725in}}{\pgfqpoint{2.572080in}{3.280607in}}%
\pgfpathcurveto{\pgfqpoint{2.576198in}{3.276489in}}{\pgfqpoint{2.581784in}{3.274175in}}{\pgfqpoint{2.587608in}{3.274175in}}%
\pgfpathlineto{\pgfqpoint{2.587608in}{3.274175in}}%
\pgfpathclose%
\pgfusepath{stroke,fill}%
\end{pgfscope}%
\begin{pgfscope}%
\pgfpathrectangle{\pgfqpoint{0.100000in}{0.183744in}}{\pgfqpoint{4.506048in}{4.506048in}}%
\pgfusepath{clip}%
\pgfsetbuttcap%
\pgfsetroundjoin%
\definecolor{currentfill}{rgb}{1.000000,0.647059,0.000000}%
\pgfsetfillcolor{currentfill}%
\pgfsetfillopacity{0.700000}%
\pgfsetlinewidth{1.003750pt}%
\definecolor{currentstroke}{rgb}{1.000000,0.647059,0.000000}%
\pgfsetstrokecolor{currentstroke}%
\pgfsetstrokeopacity{0.700000}%
\pgfsetdash{}{0pt}%
\pgfpathmoveto{\pgfqpoint{2.332590in}{3.352170in}}%
\pgfpathcurveto{\pgfqpoint{2.338413in}{3.352170in}}{\pgfqpoint{2.344000in}{3.354484in}}{\pgfqpoint{2.348118in}{3.358602in}}%
\pgfpathcurveto{\pgfqpoint{2.352236in}{3.362720in}}{\pgfqpoint{2.354550in}{3.368307in}}{\pgfqpoint{2.354550in}{3.374131in}}%
\pgfpathcurveto{\pgfqpoint{2.354550in}{3.379955in}}{\pgfqpoint{2.352236in}{3.385541in}}{\pgfqpoint{2.348118in}{3.389659in}}%
\pgfpathcurveto{\pgfqpoint{2.344000in}{3.393777in}}{\pgfqpoint{2.338413in}{3.396091in}}{\pgfqpoint{2.332590in}{3.396091in}}%
\pgfpathcurveto{\pgfqpoint{2.326766in}{3.396091in}}{\pgfqpoint{2.321179in}{3.393777in}}{\pgfqpoint{2.317061in}{3.389659in}}%
\pgfpathcurveto{\pgfqpoint{2.312943in}{3.385541in}}{\pgfqpoint{2.310629in}{3.379955in}}{\pgfqpoint{2.310629in}{3.374131in}}%
\pgfpathcurveto{\pgfqpoint{2.310629in}{3.368307in}}{\pgfqpoint{2.312943in}{3.362720in}}{\pgfqpoint{2.317061in}{3.358602in}}%
\pgfpathcurveto{\pgfqpoint{2.321179in}{3.354484in}}{\pgfqpoint{2.326766in}{3.352170in}}{\pgfqpoint{2.332590in}{3.352170in}}%
\pgfpathlineto{\pgfqpoint{2.332590in}{3.352170in}}%
\pgfpathclose%
\pgfusepath{stroke,fill}%
\end{pgfscope}%
\begin{pgfscope}%
\pgfpathrectangle{\pgfqpoint{0.100000in}{0.183744in}}{\pgfqpoint{4.506048in}{4.506048in}}%
\pgfusepath{clip}%
\pgfsetbuttcap%
\pgfsetroundjoin%
\definecolor{currentfill}{rgb}{1.000000,0.647059,0.000000}%
\pgfsetfillcolor{currentfill}%
\pgfsetfillopacity{0.700000}%
\pgfsetlinewidth{1.003750pt}%
\definecolor{currentstroke}{rgb}{1.000000,0.647059,0.000000}%
\pgfsetstrokecolor{currentstroke}%
\pgfsetstrokeopacity{0.700000}%
\pgfsetdash{}{0pt}%
\pgfpathmoveto{\pgfqpoint{0.897525in}{2.700315in}}%
\pgfpathcurveto{\pgfqpoint{0.903349in}{2.700315in}}{\pgfqpoint{0.908935in}{2.702629in}}{\pgfqpoint{0.913053in}{2.706747in}}%
\pgfpathcurveto{\pgfqpoint{0.917172in}{2.710865in}}{\pgfqpoint{0.919485in}{2.716451in}}{\pgfqpoint{0.919485in}{2.722275in}}%
\pgfpathcurveto{\pgfqpoint{0.919485in}{2.728099in}}{\pgfqpoint{0.917172in}{2.733685in}}{\pgfqpoint{0.913053in}{2.737803in}}%
\pgfpathcurveto{\pgfqpoint{0.908935in}{2.741921in}}{\pgfqpoint{0.903349in}{2.744235in}}{\pgfqpoint{0.897525in}{2.744235in}}%
\pgfpathcurveto{\pgfqpoint{0.891701in}{2.744235in}}{\pgfqpoint{0.886115in}{2.741921in}}{\pgfqpoint{0.881997in}{2.737803in}}%
\pgfpathcurveto{\pgfqpoint{0.877879in}{2.733685in}}{\pgfqpoint{0.875565in}{2.728099in}}{\pgfqpoint{0.875565in}{2.722275in}}%
\pgfpathcurveto{\pgfqpoint{0.875565in}{2.716451in}}{\pgfqpoint{0.877879in}{2.710865in}}{\pgfqpoint{0.881997in}{2.706747in}}%
\pgfpathcurveto{\pgfqpoint{0.886115in}{2.702629in}}{\pgfqpoint{0.891701in}{2.700315in}}{\pgfqpoint{0.897525in}{2.700315in}}%
\pgfpathlineto{\pgfqpoint{0.897525in}{2.700315in}}%
\pgfpathclose%
\pgfusepath{stroke,fill}%
\end{pgfscope}%
\begin{pgfscope}%
\pgfpathrectangle{\pgfqpoint{0.100000in}{0.183744in}}{\pgfqpoint{4.506048in}{4.506048in}}%
\pgfusepath{clip}%
\pgfsetbuttcap%
\pgfsetroundjoin%
\definecolor{currentfill}{rgb}{1.000000,0.647059,0.000000}%
\pgfsetfillcolor{currentfill}%
\pgfsetfillopacity{0.700000}%
\pgfsetlinewidth{1.003750pt}%
\definecolor{currentstroke}{rgb}{1.000000,0.647059,0.000000}%
\pgfsetstrokecolor{currentstroke}%
\pgfsetstrokeopacity{0.700000}%
\pgfsetdash{}{0pt}%
\pgfpathmoveto{\pgfqpoint{1.897731in}{2.569277in}}%
\pgfpathcurveto{\pgfqpoint{1.903555in}{2.569277in}}{\pgfqpoint{1.909142in}{2.571591in}}{\pgfqpoint{1.913260in}{2.575709in}}%
\pgfpathcurveto{\pgfqpoint{1.917378in}{2.579827in}}{\pgfqpoint{1.919692in}{2.585413in}}{\pgfqpoint{1.919692in}{2.591237in}}%
\pgfpathcurveto{\pgfqpoint{1.919692in}{2.597061in}}{\pgfqpoint{1.917378in}{2.602647in}}{\pgfqpoint{1.913260in}{2.606765in}}%
\pgfpathcurveto{\pgfqpoint{1.909142in}{2.610883in}}{\pgfqpoint{1.903555in}{2.613197in}}{\pgfqpoint{1.897731in}{2.613197in}}%
\pgfpathcurveto{\pgfqpoint{1.891907in}{2.613197in}}{\pgfqpoint{1.886321in}{2.610883in}}{\pgfqpoint{1.882203in}{2.606765in}}%
\pgfpathcurveto{\pgfqpoint{1.878085in}{2.602647in}}{\pgfqpoint{1.875771in}{2.597061in}}{\pgfqpoint{1.875771in}{2.591237in}}%
\pgfpathcurveto{\pgfqpoint{1.875771in}{2.585413in}}{\pgfqpoint{1.878085in}{2.579827in}}{\pgfqpoint{1.882203in}{2.575709in}}%
\pgfpathcurveto{\pgfqpoint{1.886321in}{2.571591in}}{\pgfqpoint{1.891907in}{2.569277in}}{\pgfqpoint{1.897731in}{2.569277in}}%
\pgfpathlineto{\pgfqpoint{1.897731in}{2.569277in}}%
\pgfpathclose%
\pgfusepath{stroke,fill}%
\end{pgfscope}%
\begin{pgfscope}%
\pgfpathrectangle{\pgfqpoint{0.100000in}{0.183744in}}{\pgfqpoint{4.506048in}{4.506048in}}%
\pgfusepath{clip}%
\pgfsetbuttcap%
\pgfsetroundjoin%
\definecolor{currentfill}{rgb}{1.000000,0.647059,0.000000}%
\pgfsetfillcolor{currentfill}%
\pgfsetfillopacity{0.700000}%
\pgfsetlinewidth{1.003750pt}%
\definecolor{currentstroke}{rgb}{1.000000,0.647059,0.000000}%
\pgfsetstrokecolor{currentstroke}%
\pgfsetstrokeopacity{0.700000}%
\pgfsetdash{}{0pt}%
\pgfpathmoveto{\pgfqpoint{2.207728in}{3.384027in}}%
\pgfpathcurveto{\pgfqpoint{2.213552in}{3.384027in}}{\pgfqpoint{2.219138in}{3.386341in}}{\pgfqpoint{2.223256in}{3.390459in}}%
\pgfpathcurveto{\pgfqpoint{2.227374in}{3.394577in}}{\pgfqpoint{2.229688in}{3.400163in}}{\pgfqpoint{2.229688in}{3.405987in}}%
\pgfpathcurveto{\pgfqpoint{2.229688in}{3.411811in}}{\pgfqpoint{2.227374in}{3.417397in}}{\pgfqpoint{2.223256in}{3.421516in}}%
\pgfpathcurveto{\pgfqpoint{2.219138in}{3.425634in}}{\pgfqpoint{2.213552in}{3.427948in}}{\pgfqpoint{2.207728in}{3.427948in}}%
\pgfpathcurveto{\pgfqpoint{2.201904in}{3.427948in}}{\pgfqpoint{2.196318in}{3.425634in}}{\pgfqpoint{2.192200in}{3.421516in}}%
\pgfpathcurveto{\pgfqpoint{2.188081in}{3.417397in}}{\pgfqpoint{2.185768in}{3.411811in}}{\pgfqpoint{2.185768in}{3.405987in}}%
\pgfpathcurveto{\pgfqpoint{2.185768in}{3.400163in}}{\pgfqpoint{2.188081in}{3.394577in}}{\pgfqpoint{2.192200in}{3.390459in}}%
\pgfpathcurveto{\pgfqpoint{2.196318in}{3.386341in}}{\pgfqpoint{2.201904in}{3.384027in}}{\pgfqpoint{2.207728in}{3.384027in}}%
\pgfpathlineto{\pgfqpoint{2.207728in}{3.384027in}}%
\pgfpathclose%
\pgfusepath{stroke,fill}%
\end{pgfscope}%
\begin{pgfscope}%
\pgfpathrectangle{\pgfqpoint{0.100000in}{0.183744in}}{\pgfqpoint{4.506048in}{4.506048in}}%
\pgfusepath{clip}%
\pgfsetbuttcap%
\pgfsetroundjoin%
\definecolor{currentfill}{rgb}{1.000000,0.647059,0.000000}%
\pgfsetfillcolor{currentfill}%
\pgfsetfillopacity{0.700000}%
\pgfsetlinewidth{1.003750pt}%
\definecolor{currentstroke}{rgb}{1.000000,0.647059,0.000000}%
\pgfsetstrokecolor{currentstroke}%
\pgfsetstrokeopacity{0.700000}%
\pgfsetdash{}{0pt}%
\pgfpathmoveto{\pgfqpoint{1.625627in}{2.291156in}}%
\pgfpathcurveto{\pgfqpoint{1.631451in}{2.291156in}}{\pgfqpoint{1.637037in}{2.293470in}}{\pgfqpoint{1.641155in}{2.297588in}}%
\pgfpathcurveto{\pgfqpoint{1.645273in}{2.301707in}}{\pgfqpoint{1.647587in}{2.307293in}}{\pgfqpoint{1.647587in}{2.313117in}}%
\pgfpathcurveto{\pgfqpoint{1.647587in}{2.318941in}}{\pgfqpoint{1.645273in}{2.324527in}}{\pgfqpoint{1.641155in}{2.328645in}}%
\pgfpathcurveto{\pgfqpoint{1.637037in}{2.332763in}}{\pgfqpoint{1.631451in}{2.335077in}}{\pgfqpoint{1.625627in}{2.335077in}}%
\pgfpathcurveto{\pgfqpoint{1.619803in}{2.335077in}}{\pgfqpoint{1.614217in}{2.332763in}}{\pgfqpoint{1.610099in}{2.328645in}}%
\pgfpathcurveto{\pgfqpoint{1.605981in}{2.324527in}}{\pgfqpoint{1.603667in}{2.318941in}}{\pgfqpoint{1.603667in}{2.313117in}}%
\pgfpathcurveto{\pgfqpoint{1.603667in}{2.307293in}}{\pgfqpoint{1.605981in}{2.301707in}}{\pgfqpoint{1.610099in}{2.297588in}}%
\pgfpathcurveto{\pgfqpoint{1.614217in}{2.293470in}}{\pgfqpoint{1.619803in}{2.291156in}}{\pgfqpoint{1.625627in}{2.291156in}}%
\pgfpathlineto{\pgfqpoint{1.625627in}{2.291156in}}%
\pgfpathclose%
\pgfusepath{stroke,fill}%
\end{pgfscope}%
\begin{pgfscope}%
\pgfpathrectangle{\pgfqpoint{0.100000in}{0.183744in}}{\pgfqpoint{4.506048in}{4.506048in}}%
\pgfusepath{clip}%
\pgfsetbuttcap%
\pgfsetroundjoin%
\definecolor{currentfill}{rgb}{1.000000,0.647059,0.000000}%
\pgfsetfillcolor{currentfill}%
\pgfsetfillopacity{0.700000}%
\pgfsetlinewidth{1.003750pt}%
\definecolor{currentstroke}{rgb}{1.000000,0.647059,0.000000}%
\pgfsetstrokecolor{currentstroke}%
\pgfsetstrokeopacity{0.700000}%
\pgfsetdash{}{0pt}%
\pgfpathmoveto{\pgfqpoint{2.671730in}{3.202452in}}%
\pgfpathcurveto{\pgfqpoint{2.677554in}{3.202452in}}{\pgfqpoint{2.683140in}{3.204766in}}{\pgfqpoint{2.687258in}{3.208884in}}%
\pgfpathcurveto{\pgfqpoint{2.691376in}{3.213002in}}{\pgfqpoint{2.693690in}{3.218588in}}{\pgfqpoint{2.693690in}{3.224412in}}%
\pgfpathcurveto{\pgfqpoint{2.693690in}{3.230236in}}{\pgfqpoint{2.691376in}{3.235822in}}{\pgfqpoint{2.687258in}{3.239940in}}%
\pgfpathcurveto{\pgfqpoint{2.683140in}{3.244059in}}{\pgfqpoint{2.677554in}{3.246372in}}{\pgfqpoint{2.671730in}{3.246372in}}%
\pgfpathcurveto{\pgfqpoint{2.665906in}{3.246372in}}{\pgfqpoint{2.660320in}{3.244059in}}{\pgfqpoint{2.656202in}{3.239940in}}%
\pgfpathcurveto{\pgfqpoint{2.652084in}{3.235822in}}{\pgfqpoint{2.649770in}{3.230236in}}{\pgfqpoint{2.649770in}{3.224412in}}%
\pgfpathcurveto{\pgfqpoint{2.649770in}{3.218588in}}{\pgfqpoint{2.652084in}{3.213002in}}{\pgfqpoint{2.656202in}{3.208884in}}%
\pgfpathcurveto{\pgfqpoint{2.660320in}{3.204766in}}{\pgfqpoint{2.665906in}{3.202452in}}{\pgfqpoint{2.671730in}{3.202452in}}%
\pgfpathlineto{\pgfqpoint{2.671730in}{3.202452in}}%
\pgfpathclose%
\pgfusepath{stroke,fill}%
\end{pgfscope}%
\begin{pgfscope}%
\pgfpathrectangle{\pgfqpoint{0.100000in}{0.183744in}}{\pgfqpoint{4.506048in}{4.506048in}}%
\pgfusepath{clip}%
\pgfsetbuttcap%
\pgfsetroundjoin%
\definecolor{currentfill}{rgb}{1.000000,0.647059,0.000000}%
\pgfsetfillcolor{currentfill}%
\pgfsetfillopacity{0.700000}%
\pgfsetlinewidth{1.003750pt}%
\definecolor{currentstroke}{rgb}{1.000000,0.647059,0.000000}%
\pgfsetstrokecolor{currentstroke}%
\pgfsetstrokeopacity{0.700000}%
\pgfsetdash{}{0pt}%
\pgfpathmoveto{\pgfqpoint{2.798550in}{3.332903in}}%
\pgfpathcurveto{\pgfqpoint{2.804374in}{3.332903in}}{\pgfqpoint{2.809960in}{3.335217in}}{\pgfqpoint{2.814078in}{3.339335in}}%
\pgfpathcurveto{\pgfqpoint{2.818196in}{3.343453in}}{\pgfqpoint{2.820510in}{3.349040in}}{\pgfqpoint{2.820510in}{3.354864in}}%
\pgfpathcurveto{\pgfqpoint{2.820510in}{3.360687in}}{\pgfqpoint{2.818196in}{3.366274in}}{\pgfqpoint{2.814078in}{3.370392in}}%
\pgfpathcurveto{\pgfqpoint{2.809960in}{3.374510in}}{\pgfqpoint{2.804374in}{3.376824in}}{\pgfqpoint{2.798550in}{3.376824in}}%
\pgfpathcurveto{\pgfqpoint{2.792726in}{3.376824in}}{\pgfqpoint{2.787140in}{3.374510in}}{\pgfqpoint{2.783022in}{3.370392in}}%
\pgfpathcurveto{\pgfqpoint{2.778903in}{3.366274in}}{\pgfqpoint{2.776590in}{3.360687in}}{\pgfqpoint{2.776590in}{3.354864in}}%
\pgfpathcurveto{\pgfqpoint{2.776590in}{3.349040in}}{\pgfqpoint{2.778903in}{3.343453in}}{\pgfqpoint{2.783022in}{3.339335in}}%
\pgfpathcurveto{\pgfqpoint{2.787140in}{3.335217in}}{\pgfqpoint{2.792726in}{3.332903in}}{\pgfqpoint{2.798550in}{3.332903in}}%
\pgfpathlineto{\pgfqpoint{2.798550in}{3.332903in}}%
\pgfpathclose%
\pgfusepath{stroke,fill}%
\end{pgfscope}%
\begin{pgfscope}%
\pgfpathrectangle{\pgfqpoint{0.100000in}{0.183744in}}{\pgfqpoint{4.506048in}{4.506048in}}%
\pgfusepath{clip}%
\pgfsetbuttcap%
\pgfsetroundjoin%
\definecolor{currentfill}{rgb}{1.000000,0.647059,0.000000}%
\pgfsetfillcolor{currentfill}%
\pgfsetfillopacity{0.700000}%
\pgfsetlinewidth{1.003750pt}%
\definecolor{currentstroke}{rgb}{1.000000,0.647059,0.000000}%
\pgfsetstrokecolor{currentstroke}%
\pgfsetstrokeopacity{0.700000}%
\pgfsetdash{}{0pt}%
\pgfpathmoveto{\pgfqpoint{1.190068in}{2.772713in}}%
\pgfpathcurveto{\pgfqpoint{1.195892in}{2.772713in}}{\pgfqpoint{1.201478in}{2.775027in}}{\pgfqpoint{1.205597in}{2.779145in}}%
\pgfpathcurveto{\pgfqpoint{1.209715in}{2.783263in}}{\pgfqpoint{1.212029in}{2.788849in}}{\pgfqpoint{1.212029in}{2.794673in}}%
\pgfpathcurveto{\pgfqpoint{1.212029in}{2.800497in}}{\pgfqpoint{1.209715in}{2.806084in}}{\pgfqpoint{1.205597in}{2.810202in}}%
\pgfpathcurveto{\pgfqpoint{1.201478in}{2.814320in}}{\pgfqpoint{1.195892in}{2.816634in}}{\pgfqpoint{1.190068in}{2.816634in}}%
\pgfpathcurveto{\pgfqpoint{1.184244in}{2.816634in}}{\pgfqpoint{1.178658in}{2.814320in}}{\pgfqpoint{1.174540in}{2.810202in}}%
\pgfpathcurveto{\pgfqpoint{1.170422in}{2.806084in}}{\pgfqpoint{1.168108in}{2.800497in}}{\pgfqpoint{1.168108in}{2.794673in}}%
\pgfpathcurveto{\pgfqpoint{1.168108in}{2.788849in}}{\pgfqpoint{1.170422in}{2.783263in}}{\pgfqpoint{1.174540in}{2.779145in}}%
\pgfpathcurveto{\pgfqpoint{1.178658in}{2.775027in}}{\pgfqpoint{1.184244in}{2.772713in}}{\pgfqpoint{1.190068in}{2.772713in}}%
\pgfpathlineto{\pgfqpoint{1.190068in}{2.772713in}}%
\pgfpathclose%
\pgfusepath{stroke,fill}%
\end{pgfscope}%
\begin{pgfscope}%
\pgfpathrectangle{\pgfqpoint{0.100000in}{0.183744in}}{\pgfqpoint{4.506048in}{4.506048in}}%
\pgfusepath{clip}%
\pgfsetbuttcap%
\pgfsetroundjoin%
\definecolor{currentfill}{rgb}{1.000000,0.647059,0.000000}%
\pgfsetfillcolor{currentfill}%
\pgfsetfillopacity{0.700000}%
\pgfsetlinewidth{1.003750pt}%
\definecolor{currentstroke}{rgb}{1.000000,0.647059,0.000000}%
\pgfsetstrokecolor{currentstroke}%
\pgfsetstrokeopacity{0.700000}%
\pgfsetdash{}{0pt}%
\pgfpathmoveto{\pgfqpoint{2.811762in}{2.623752in}}%
\pgfpathcurveto{\pgfqpoint{2.817586in}{2.623752in}}{\pgfqpoint{2.823172in}{2.626065in}}{\pgfqpoint{2.827290in}{2.630184in}}%
\pgfpathcurveto{\pgfqpoint{2.831408in}{2.634302in}}{\pgfqpoint{2.833722in}{2.639888in}}{\pgfqpoint{2.833722in}{2.645712in}}%
\pgfpathcurveto{\pgfqpoint{2.833722in}{2.651536in}}{\pgfqpoint{2.831408in}{2.657122in}}{\pgfqpoint{2.827290in}{2.661240in}}%
\pgfpathcurveto{\pgfqpoint{2.823172in}{2.665358in}}{\pgfqpoint{2.817586in}{2.667672in}}{\pgfqpoint{2.811762in}{2.667672in}}%
\pgfpathcurveto{\pgfqpoint{2.805938in}{2.667672in}}{\pgfqpoint{2.800352in}{2.665358in}}{\pgfqpoint{2.796234in}{2.661240in}}%
\pgfpathcurveto{\pgfqpoint{2.792115in}{2.657122in}}{\pgfqpoint{2.789802in}{2.651536in}}{\pgfqpoint{2.789802in}{2.645712in}}%
\pgfpathcurveto{\pgfqpoint{2.789802in}{2.639888in}}{\pgfqpoint{2.792115in}{2.634302in}}{\pgfqpoint{2.796234in}{2.630184in}}%
\pgfpathcurveto{\pgfqpoint{2.800352in}{2.626065in}}{\pgfqpoint{2.805938in}{2.623752in}}{\pgfqpoint{2.811762in}{2.623752in}}%
\pgfpathlineto{\pgfqpoint{2.811762in}{2.623752in}}%
\pgfpathclose%
\pgfusepath{stroke,fill}%
\end{pgfscope}%
\begin{pgfscope}%
\pgfpathrectangle{\pgfqpoint{0.100000in}{0.183744in}}{\pgfqpoint{4.506048in}{4.506048in}}%
\pgfusepath{clip}%
\pgfsetbuttcap%
\pgfsetroundjoin%
\definecolor{currentfill}{rgb}{1.000000,0.647059,0.000000}%
\pgfsetfillcolor{currentfill}%
\pgfsetfillopacity{0.700000}%
\pgfsetlinewidth{1.003750pt}%
\definecolor{currentstroke}{rgb}{1.000000,0.647059,0.000000}%
\pgfsetstrokecolor{currentstroke}%
\pgfsetstrokeopacity{0.700000}%
\pgfsetdash{}{0pt}%
\pgfpathmoveto{\pgfqpoint{3.476296in}{3.228124in}}%
\pgfpathcurveto{\pgfqpoint{3.482120in}{3.228124in}}{\pgfqpoint{3.487706in}{3.230438in}}{\pgfqpoint{3.491825in}{3.234556in}}%
\pgfpathcurveto{\pgfqpoint{3.495943in}{3.238674in}}{\pgfqpoint{3.498257in}{3.244261in}}{\pgfqpoint{3.498257in}{3.250084in}}%
\pgfpathcurveto{\pgfqpoint{3.498257in}{3.255908in}}{\pgfqpoint{3.495943in}{3.261495in}}{\pgfqpoint{3.491825in}{3.265613in}}%
\pgfpathcurveto{\pgfqpoint{3.487706in}{3.269731in}}{\pgfqpoint{3.482120in}{3.272045in}}{\pgfqpoint{3.476296in}{3.272045in}}%
\pgfpathcurveto{\pgfqpoint{3.470472in}{3.272045in}}{\pgfqpoint{3.464886in}{3.269731in}}{\pgfqpoint{3.460768in}{3.265613in}}%
\pgfpathcurveto{\pgfqpoint{3.456650in}{3.261495in}}{\pgfqpoint{3.454336in}{3.255908in}}{\pgfqpoint{3.454336in}{3.250084in}}%
\pgfpathcurveto{\pgfqpoint{3.454336in}{3.244261in}}{\pgfqpoint{3.456650in}{3.238674in}}{\pgfqpoint{3.460768in}{3.234556in}}%
\pgfpathcurveto{\pgfqpoint{3.464886in}{3.230438in}}{\pgfqpoint{3.470472in}{3.228124in}}{\pgfqpoint{3.476296in}{3.228124in}}%
\pgfpathlineto{\pgfqpoint{3.476296in}{3.228124in}}%
\pgfpathclose%
\pgfusepath{stroke,fill}%
\end{pgfscope}%
\begin{pgfscope}%
\pgfpathrectangle{\pgfqpoint{0.100000in}{0.183744in}}{\pgfqpoint{4.506048in}{4.506048in}}%
\pgfusepath{clip}%
\pgfsetbuttcap%
\pgfsetroundjoin%
\definecolor{currentfill}{rgb}{1.000000,0.647059,0.000000}%
\pgfsetfillcolor{currentfill}%
\pgfsetfillopacity{0.700000}%
\pgfsetlinewidth{1.003750pt}%
\definecolor{currentstroke}{rgb}{1.000000,0.647059,0.000000}%
\pgfsetstrokecolor{currentstroke}%
\pgfsetstrokeopacity{0.700000}%
\pgfsetdash{}{0pt}%
\pgfpathmoveto{\pgfqpoint{2.563471in}{3.442250in}}%
\pgfpathcurveto{\pgfqpoint{2.569295in}{3.442250in}}{\pgfqpoint{2.574881in}{3.444564in}}{\pgfqpoint{2.578999in}{3.448682in}}%
\pgfpathcurveto{\pgfqpoint{2.583117in}{3.452800in}}{\pgfqpoint{2.585431in}{3.458386in}}{\pgfqpoint{2.585431in}{3.464210in}}%
\pgfpathcurveto{\pgfqpoint{2.585431in}{3.470034in}}{\pgfqpoint{2.583117in}{3.475620in}}{\pgfqpoint{2.578999in}{3.479738in}}%
\pgfpathcurveto{\pgfqpoint{2.574881in}{3.483856in}}{\pgfqpoint{2.569295in}{3.486170in}}{\pgfqpoint{2.563471in}{3.486170in}}%
\pgfpathcurveto{\pgfqpoint{2.557647in}{3.486170in}}{\pgfqpoint{2.552061in}{3.483856in}}{\pgfqpoint{2.547943in}{3.479738in}}%
\pgfpathcurveto{\pgfqpoint{2.543825in}{3.475620in}}{\pgfqpoint{2.541511in}{3.470034in}}{\pgfqpoint{2.541511in}{3.464210in}}%
\pgfpathcurveto{\pgfqpoint{2.541511in}{3.458386in}}{\pgfqpoint{2.543825in}{3.452800in}}{\pgfqpoint{2.547943in}{3.448682in}}%
\pgfpathcurveto{\pgfqpoint{2.552061in}{3.444564in}}{\pgfqpoint{2.557647in}{3.442250in}}{\pgfqpoint{2.563471in}{3.442250in}}%
\pgfpathlineto{\pgfqpoint{2.563471in}{3.442250in}}%
\pgfpathclose%
\pgfusepath{stroke,fill}%
\end{pgfscope}%
\begin{pgfscope}%
\pgfpathrectangle{\pgfqpoint{0.100000in}{0.183744in}}{\pgfqpoint{4.506048in}{4.506048in}}%
\pgfusepath{clip}%
\pgfsetbuttcap%
\pgfsetroundjoin%
\definecolor{currentfill}{rgb}{1.000000,0.647059,0.000000}%
\pgfsetfillcolor{currentfill}%
\pgfsetfillopacity{0.700000}%
\pgfsetlinewidth{1.003750pt}%
\definecolor{currentstroke}{rgb}{1.000000,0.647059,0.000000}%
\pgfsetstrokecolor{currentstroke}%
\pgfsetstrokeopacity{0.700000}%
\pgfsetdash{}{0pt}%
\pgfpathmoveto{\pgfqpoint{2.787741in}{3.364265in}}%
\pgfpathcurveto{\pgfqpoint{2.793565in}{3.364265in}}{\pgfqpoint{2.799151in}{3.366579in}}{\pgfqpoint{2.803269in}{3.370697in}}%
\pgfpathcurveto{\pgfqpoint{2.807387in}{3.374815in}}{\pgfqpoint{2.809701in}{3.380402in}}{\pgfqpoint{2.809701in}{3.386226in}}%
\pgfpathcurveto{\pgfqpoint{2.809701in}{3.392049in}}{\pgfqpoint{2.807387in}{3.397636in}}{\pgfqpoint{2.803269in}{3.401754in}}%
\pgfpathcurveto{\pgfqpoint{2.799151in}{3.405872in}}{\pgfqpoint{2.793565in}{3.408186in}}{\pgfqpoint{2.787741in}{3.408186in}}%
\pgfpathcurveto{\pgfqpoint{2.781917in}{3.408186in}}{\pgfqpoint{2.776331in}{3.405872in}}{\pgfqpoint{2.772213in}{3.401754in}}%
\pgfpathcurveto{\pgfqpoint{2.768094in}{3.397636in}}{\pgfqpoint{2.765781in}{3.392049in}}{\pgfqpoint{2.765781in}{3.386226in}}%
\pgfpathcurveto{\pgfqpoint{2.765781in}{3.380402in}}{\pgfqpoint{2.768094in}{3.374815in}}{\pgfqpoint{2.772213in}{3.370697in}}%
\pgfpathcurveto{\pgfqpoint{2.776331in}{3.366579in}}{\pgfqpoint{2.781917in}{3.364265in}}{\pgfqpoint{2.787741in}{3.364265in}}%
\pgfpathlineto{\pgfqpoint{2.787741in}{3.364265in}}%
\pgfpathclose%
\pgfusepath{stroke,fill}%
\end{pgfscope}%
\begin{pgfscope}%
\pgfpathrectangle{\pgfqpoint{0.100000in}{0.183744in}}{\pgfqpoint{4.506048in}{4.506048in}}%
\pgfusepath{clip}%
\pgfsetbuttcap%
\pgfsetroundjoin%
\definecolor{currentfill}{rgb}{1.000000,0.647059,0.000000}%
\pgfsetfillcolor{currentfill}%
\pgfsetfillopacity{0.700000}%
\pgfsetlinewidth{1.003750pt}%
\definecolor{currentstroke}{rgb}{1.000000,0.647059,0.000000}%
\pgfsetstrokecolor{currentstroke}%
\pgfsetstrokeopacity{0.700000}%
\pgfsetdash{}{0pt}%
\pgfpathmoveto{\pgfqpoint{2.300004in}{2.390867in}}%
\pgfpathcurveto{\pgfqpoint{2.305828in}{2.390867in}}{\pgfqpoint{2.311414in}{2.393181in}}{\pgfqpoint{2.315532in}{2.397299in}}%
\pgfpathcurveto{\pgfqpoint{2.319650in}{2.401417in}}{\pgfqpoint{2.321964in}{2.407004in}}{\pgfqpoint{2.321964in}{2.412828in}}%
\pgfpathcurveto{\pgfqpoint{2.321964in}{2.418652in}}{\pgfqpoint{2.319650in}{2.424238in}}{\pgfqpoint{2.315532in}{2.428356in}}%
\pgfpathcurveto{\pgfqpoint{2.311414in}{2.432474in}}{\pgfqpoint{2.305828in}{2.434788in}}{\pgfqpoint{2.300004in}{2.434788in}}%
\pgfpathcurveto{\pgfqpoint{2.294180in}{2.434788in}}{\pgfqpoint{2.288594in}{2.432474in}}{\pgfqpoint{2.284476in}{2.428356in}}%
\pgfpathcurveto{\pgfqpoint{2.280358in}{2.424238in}}{\pgfqpoint{2.278044in}{2.418652in}}{\pgfqpoint{2.278044in}{2.412828in}}%
\pgfpathcurveto{\pgfqpoint{2.278044in}{2.407004in}}{\pgfqpoint{2.280358in}{2.401417in}}{\pgfqpoint{2.284476in}{2.397299in}}%
\pgfpathcurveto{\pgfqpoint{2.288594in}{2.393181in}}{\pgfqpoint{2.294180in}{2.390867in}}{\pgfqpoint{2.300004in}{2.390867in}}%
\pgfpathlineto{\pgfqpoint{2.300004in}{2.390867in}}%
\pgfpathclose%
\pgfusepath{stroke,fill}%
\end{pgfscope}%
\begin{pgfscope}%
\pgfpathrectangle{\pgfqpoint{0.100000in}{0.183744in}}{\pgfqpoint{4.506048in}{4.506048in}}%
\pgfusepath{clip}%
\pgfsetbuttcap%
\pgfsetroundjoin%
\definecolor{currentfill}{rgb}{1.000000,0.647059,0.000000}%
\pgfsetfillcolor{currentfill}%
\pgfsetfillopacity{0.700000}%
\pgfsetlinewidth{1.003750pt}%
\definecolor{currentstroke}{rgb}{1.000000,0.647059,0.000000}%
\pgfsetstrokecolor{currentstroke}%
\pgfsetstrokeopacity{0.700000}%
\pgfsetdash{}{0pt}%
\pgfpathmoveto{\pgfqpoint{2.550528in}{2.380400in}}%
\pgfpathcurveto{\pgfqpoint{2.556352in}{2.380400in}}{\pgfqpoint{2.561938in}{2.382713in}}{\pgfqpoint{2.566056in}{2.386832in}}%
\pgfpathcurveto{\pgfqpoint{2.570174in}{2.390950in}}{\pgfqpoint{2.572488in}{2.396536in}}{\pgfqpoint{2.572488in}{2.402360in}}%
\pgfpathcurveto{\pgfqpoint{2.572488in}{2.408184in}}{\pgfqpoint{2.570174in}{2.413770in}}{\pgfqpoint{2.566056in}{2.417888in}}%
\pgfpathcurveto{\pgfqpoint{2.561938in}{2.422006in}}{\pgfqpoint{2.556352in}{2.424320in}}{\pgfqpoint{2.550528in}{2.424320in}}%
\pgfpathcurveto{\pgfqpoint{2.544704in}{2.424320in}}{\pgfqpoint{2.539118in}{2.422006in}}{\pgfqpoint{2.535000in}{2.417888in}}%
\pgfpathcurveto{\pgfqpoint{2.530882in}{2.413770in}}{\pgfqpoint{2.528568in}{2.408184in}}{\pgfqpoint{2.528568in}{2.402360in}}%
\pgfpathcurveto{\pgfqpoint{2.528568in}{2.396536in}}{\pgfqpoint{2.530882in}{2.390950in}}{\pgfqpoint{2.535000in}{2.386832in}}%
\pgfpathcurveto{\pgfqpoint{2.539118in}{2.382713in}}{\pgfqpoint{2.544704in}{2.380400in}}{\pgfqpoint{2.550528in}{2.380400in}}%
\pgfpathlineto{\pgfqpoint{2.550528in}{2.380400in}}%
\pgfpathclose%
\pgfusepath{stroke,fill}%
\end{pgfscope}%
\begin{pgfscope}%
\pgfpathrectangle{\pgfqpoint{0.100000in}{0.183744in}}{\pgfqpoint{4.506048in}{4.506048in}}%
\pgfusepath{clip}%
\pgfsetbuttcap%
\pgfsetroundjoin%
\definecolor{currentfill}{rgb}{1.000000,0.647059,0.000000}%
\pgfsetfillcolor{currentfill}%
\pgfsetfillopacity{0.700000}%
\pgfsetlinewidth{1.003750pt}%
\definecolor{currentstroke}{rgb}{1.000000,0.647059,0.000000}%
\pgfsetstrokecolor{currentstroke}%
\pgfsetstrokeopacity{0.700000}%
\pgfsetdash{}{0pt}%
\pgfpathmoveto{\pgfqpoint{3.048941in}{3.941290in}}%
\pgfpathcurveto{\pgfqpoint{3.054765in}{3.941290in}}{\pgfqpoint{3.060351in}{3.943604in}}{\pgfqpoint{3.064469in}{3.947722in}}%
\pgfpathcurveto{\pgfqpoint{3.068587in}{3.951840in}}{\pgfqpoint{3.070901in}{3.957426in}}{\pgfqpoint{3.070901in}{3.963250in}}%
\pgfpathcurveto{\pgfqpoint{3.070901in}{3.969074in}}{\pgfqpoint{3.068587in}{3.974660in}}{\pgfqpoint{3.064469in}{3.978778in}}%
\pgfpathcurveto{\pgfqpoint{3.060351in}{3.982897in}}{\pgfqpoint{3.054765in}{3.985210in}}{\pgfqpoint{3.048941in}{3.985210in}}%
\pgfpathcurveto{\pgfqpoint{3.043117in}{3.985210in}}{\pgfqpoint{3.037531in}{3.982897in}}{\pgfqpoint{3.033413in}{3.978778in}}%
\pgfpathcurveto{\pgfqpoint{3.029295in}{3.974660in}}{\pgfqpoint{3.026981in}{3.969074in}}{\pgfqpoint{3.026981in}{3.963250in}}%
\pgfpathcurveto{\pgfqpoint{3.026981in}{3.957426in}}{\pgfqpoint{3.029295in}{3.951840in}}{\pgfqpoint{3.033413in}{3.947722in}}%
\pgfpathcurveto{\pgfqpoint{3.037531in}{3.943604in}}{\pgfqpoint{3.043117in}{3.941290in}}{\pgfqpoint{3.048941in}{3.941290in}}%
\pgfpathlineto{\pgfqpoint{3.048941in}{3.941290in}}%
\pgfpathclose%
\pgfusepath{stroke,fill}%
\end{pgfscope}%
\begin{pgfscope}%
\pgfpathrectangle{\pgfqpoint{0.100000in}{0.183744in}}{\pgfqpoint{4.506048in}{4.506048in}}%
\pgfusepath{clip}%
\pgfsetbuttcap%
\pgfsetroundjoin%
\definecolor{currentfill}{rgb}{1.000000,0.647059,0.000000}%
\pgfsetfillcolor{currentfill}%
\pgfsetfillopacity{0.700000}%
\pgfsetlinewidth{1.003750pt}%
\definecolor{currentstroke}{rgb}{1.000000,0.647059,0.000000}%
\pgfsetstrokecolor{currentstroke}%
\pgfsetstrokeopacity{0.700000}%
\pgfsetdash{}{0pt}%
\pgfpathmoveto{\pgfqpoint{1.745049in}{2.350226in}}%
\pgfpathcurveto{\pgfqpoint{1.750873in}{2.350226in}}{\pgfqpoint{1.756459in}{2.352540in}}{\pgfqpoint{1.760577in}{2.356658in}}%
\pgfpathcurveto{\pgfqpoint{1.764695in}{2.360776in}}{\pgfqpoint{1.767009in}{2.366363in}}{\pgfqpoint{1.767009in}{2.372186in}}%
\pgfpathcurveto{\pgfqpoint{1.767009in}{2.378010in}}{\pgfqpoint{1.764695in}{2.383597in}}{\pgfqpoint{1.760577in}{2.387715in}}%
\pgfpathcurveto{\pgfqpoint{1.756459in}{2.391833in}}{\pgfqpoint{1.750873in}{2.394147in}}{\pgfqpoint{1.745049in}{2.394147in}}%
\pgfpathcurveto{\pgfqpoint{1.739225in}{2.394147in}}{\pgfqpoint{1.733639in}{2.391833in}}{\pgfqpoint{1.729521in}{2.387715in}}%
\pgfpathcurveto{\pgfqpoint{1.725403in}{2.383597in}}{\pgfqpoint{1.723089in}{2.378010in}}{\pgfqpoint{1.723089in}{2.372186in}}%
\pgfpathcurveto{\pgfqpoint{1.723089in}{2.366363in}}{\pgfqpoint{1.725403in}{2.360776in}}{\pgfqpoint{1.729521in}{2.356658in}}%
\pgfpathcurveto{\pgfqpoint{1.733639in}{2.352540in}}{\pgfqpoint{1.739225in}{2.350226in}}{\pgfqpoint{1.745049in}{2.350226in}}%
\pgfpathlineto{\pgfqpoint{1.745049in}{2.350226in}}%
\pgfpathclose%
\pgfusepath{stroke,fill}%
\end{pgfscope}%
\begin{pgfscope}%
\pgfpathrectangle{\pgfqpoint{0.100000in}{0.183744in}}{\pgfqpoint{4.506048in}{4.506048in}}%
\pgfusepath{clip}%
\pgfsetbuttcap%
\pgfsetroundjoin%
\definecolor{currentfill}{rgb}{1.000000,0.647059,0.000000}%
\pgfsetfillcolor{currentfill}%
\pgfsetfillopacity{0.700000}%
\pgfsetlinewidth{1.003750pt}%
\definecolor{currentstroke}{rgb}{1.000000,0.647059,0.000000}%
\pgfsetstrokecolor{currentstroke}%
\pgfsetstrokeopacity{0.700000}%
\pgfsetdash{}{0pt}%
\pgfpathmoveto{\pgfqpoint{1.754094in}{1.998351in}}%
\pgfpathcurveto{\pgfqpoint{1.759918in}{1.998351in}}{\pgfqpoint{1.765504in}{2.000665in}}{\pgfqpoint{1.769623in}{2.004783in}}%
\pgfpathcurveto{\pgfqpoint{1.773741in}{2.008901in}}{\pgfqpoint{1.776055in}{2.014487in}}{\pgfqpoint{1.776055in}{2.020311in}}%
\pgfpathcurveto{\pgfqpoint{1.776055in}{2.026135in}}{\pgfqpoint{1.773741in}{2.031721in}}{\pgfqpoint{1.769623in}{2.035839in}}%
\pgfpathcurveto{\pgfqpoint{1.765504in}{2.039957in}}{\pgfqpoint{1.759918in}{2.042271in}}{\pgfqpoint{1.754094in}{2.042271in}}%
\pgfpathcurveto{\pgfqpoint{1.748270in}{2.042271in}}{\pgfqpoint{1.742684in}{2.039957in}}{\pgfqpoint{1.738566in}{2.035839in}}%
\pgfpathcurveto{\pgfqpoint{1.734448in}{2.031721in}}{\pgfqpoint{1.732134in}{2.026135in}}{\pgfqpoint{1.732134in}{2.020311in}}%
\pgfpathcurveto{\pgfqpoint{1.732134in}{2.014487in}}{\pgfqpoint{1.734448in}{2.008901in}}{\pgfqpoint{1.738566in}{2.004783in}}%
\pgfpathcurveto{\pgfqpoint{1.742684in}{2.000665in}}{\pgfqpoint{1.748270in}{1.998351in}}{\pgfqpoint{1.754094in}{1.998351in}}%
\pgfpathlineto{\pgfqpoint{1.754094in}{1.998351in}}%
\pgfpathclose%
\pgfusepath{stroke,fill}%
\end{pgfscope}%
\begin{pgfscope}%
\pgfpathrectangle{\pgfqpoint{0.100000in}{0.183744in}}{\pgfqpoint{4.506048in}{4.506048in}}%
\pgfusepath{clip}%
\pgfsetbuttcap%
\pgfsetroundjoin%
\definecolor{currentfill}{rgb}{1.000000,0.647059,0.000000}%
\pgfsetfillcolor{currentfill}%
\pgfsetfillopacity{0.700000}%
\pgfsetlinewidth{1.003750pt}%
\definecolor{currentstroke}{rgb}{1.000000,0.647059,0.000000}%
\pgfsetstrokecolor{currentstroke}%
\pgfsetstrokeopacity{0.700000}%
\pgfsetdash{}{0pt}%
\pgfpathmoveto{\pgfqpoint{1.146945in}{1.998035in}}%
\pgfpathcurveto{\pgfqpoint{1.152769in}{1.998035in}}{\pgfqpoint{1.158355in}{2.000349in}}{\pgfqpoint{1.162473in}{2.004467in}}%
\pgfpathcurveto{\pgfqpoint{1.166591in}{2.008585in}}{\pgfqpoint{1.168905in}{2.014171in}}{\pgfqpoint{1.168905in}{2.019995in}}%
\pgfpathcurveto{\pgfqpoint{1.168905in}{2.025819in}}{\pgfqpoint{1.166591in}{2.031405in}}{\pgfqpoint{1.162473in}{2.035523in}}%
\pgfpathcurveto{\pgfqpoint{1.158355in}{2.039641in}}{\pgfqpoint{1.152769in}{2.041955in}}{\pgfqpoint{1.146945in}{2.041955in}}%
\pgfpathcurveto{\pgfqpoint{1.141121in}{2.041955in}}{\pgfqpoint{1.135535in}{2.039641in}}{\pgfqpoint{1.131417in}{2.035523in}}%
\pgfpathcurveto{\pgfqpoint{1.127299in}{2.031405in}}{\pgfqpoint{1.124985in}{2.025819in}}{\pgfqpoint{1.124985in}{2.019995in}}%
\pgfpathcurveto{\pgfqpoint{1.124985in}{2.014171in}}{\pgfqpoint{1.127299in}{2.008585in}}{\pgfqpoint{1.131417in}{2.004467in}}%
\pgfpathcurveto{\pgfqpoint{1.135535in}{2.000349in}}{\pgfqpoint{1.141121in}{1.998035in}}{\pgfqpoint{1.146945in}{1.998035in}}%
\pgfpathlineto{\pgfqpoint{1.146945in}{1.998035in}}%
\pgfpathclose%
\pgfusepath{stroke,fill}%
\end{pgfscope}%
\begin{pgfscope}%
\pgfpathrectangle{\pgfqpoint{0.100000in}{0.183744in}}{\pgfqpoint{4.506048in}{4.506048in}}%
\pgfusepath{clip}%
\pgfsetbuttcap%
\pgfsetroundjoin%
\definecolor{currentfill}{rgb}{1.000000,0.647059,0.000000}%
\pgfsetfillcolor{currentfill}%
\pgfsetfillopacity{0.700000}%
\pgfsetlinewidth{1.003750pt}%
\definecolor{currentstroke}{rgb}{1.000000,0.647059,0.000000}%
\pgfsetstrokecolor{currentstroke}%
\pgfsetstrokeopacity{0.700000}%
\pgfsetdash{}{0pt}%
\pgfpathmoveto{\pgfqpoint{1.850041in}{2.206175in}}%
\pgfpathcurveto{\pgfqpoint{1.855865in}{2.206175in}}{\pgfqpoint{1.861451in}{2.208489in}}{\pgfqpoint{1.865569in}{2.212607in}}%
\pgfpathcurveto{\pgfqpoint{1.869687in}{2.216725in}}{\pgfqpoint{1.872001in}{2.222311in}}{\pgfqpoint{1.872001in}{2.228135in}}%
\pgfpathcurveto{\pgfqpoint{1.872001in}{2.233959in}}{\pgfqpoint{1.869687in}{2.239545in}}{\pgfqpoint{1.865569in}{2.243663in}}%
\pgfpathcurveto{\pgfqpoint{1.861451in}{2.247781in}}{\pgfqpoint{1.855865in}{2.250095in}}{\pgfqpoint{1.850041in}{2.250095in}}%
\pgfpathcurveto{\pgfqpoint{1.844217in}{2.250095in}}{\pgfqpoint{1.838631in}{2.247781in}}{\pgfqpoint{1.834513in}{2.243663in}}%
\pgfpathcurveto{\pgfqpoint{1.830395in}{2.239545in}}{\pgfqpoint{1.828081in}{2.233959in}}{\pgfqpoint{1.828081in}{2.228135in}}%
\pgfpathcurveto{\pgfqpoint{1.828081in}{2.222311in}}{\pgfqpoint{1.830395in}{2.216725in}}{\pgfqpoint{1.834513in}{2.212607in}}%
\pgfpathcurveto{\pgfqpoint{1.838631in}{2.208489in}}{\pgfqpoint{1.844217in}{2.206175in}}{\pgfqpoint{1.850041in}{2.206175in}}%
\pgfpathlineto{\pgfqpoint{1.850041in}{2.206175in}}%
\pgfpathclose%
\pgfusepath{stroke,fill}%
\end{pgfscope}%
\begin{pgfscope}%
\pgfpathrectangle{\pgfqpoint{0.100000in}{0.183744in}}{\pgfqpoint{4.506048in}{4.506048in}}%
\pgfusepath{clip}%
\pgfsetbuttcap%
\pgfsetroundjoin%
\definecolor{currentfill}{rgb}{1.000000,0.647059,0.000000}%
\pgfsetfillcolor{currentfill}%
\pgfsetfillopacity{0.700000}%
\pgfsetlinewidth{1.003750pt}%
\definecolor{currentstroke}{rgb}{1.000000,0.647059,0.000000}%
\pgfsetstrokecolor{currentstroke}%
\pgfsetstrokeopacity{0.700000}%
\pgfsetdash{}{0pt}%
\pgfpathmoveto{\pgfqpoint{2.663198in}{3.252437in}}%
\pgfpathcurveto{\pgfqpoint{2.669022in}{3.252437in}}{\pgfqpoint{2.674608in}{3.254751in}}{\pgfqpoint{2.678727in}{3.258869in}}%
\pgfpathcurveto{\pgfqpoint{2.682845in}{3.262987in}}{\pgfqpoint{2.685159in}{3.268573in}}{\pgfqpoint{2.685159in}{3.274397in}}%
\pgfpathcurveto{\pgfqpoint{2.685159in}{3.280221in}}{\pgfqpoint{2.682845in}{3.285807in}}{\pgfqpoint{2.678727in}{3.289925in}}%
\pgfpathcurveto{\pgfqpoint{2.674608in}{3.294043in}}{\pgfqpoint{2.669022in}{3.296357in}}{\pgfqpoint{2.663198in}{3.296357in}}%
\pgfpathcurveto{\pgfqpoint{2.657374in}{3.296357in}}{\pgfqpoint{2.651788in}{3.294043in}}{\pgfqpoint{2.647670in}{3.289925in}}%
\pgfpathcurveto{\pgfqpoint{2.643552in}{3.285807in}}{\pgfqpoint{2.641238in}{3.280221in}}{\pgfqpoint{2.641238in}{3.274397in}}%
\pgfpathcurveto{\pgfqpoint{2.641238in}{3.268573in}}{\pgfqpoint{2.643552in}{3.262987in}}{\pgfqpoint{2.647670in}{3.258869in}}%
\pgfpathcurveto{\pgfqpoint{2.651788in}{3.254751in}}{\pgfqpoint{2.657374in}{3.252437in}}{\pgfqpoint{2.663198in}{3.252437in}}%
\pgfpathlineto{\pgfqpoint{2.663198in}{3.252437in}}%
\pgfpathclose%
\pgfusepath{stroke,fill}%
\end{pgfscope}%
\begin{pgfscope}%
\pgfpathrectangle{\pgfqpoint{0.100000in}{0.183744in}}{\pgfqpoint{4.506048in}{4.506048in}}%
\pgfusepath{clip}%
\pgfsetbuttcap%
\pgfsetroundjoin%
\definecolor{currentfill}{rgb}{1.000000,0.647059,0.000000}%
\pgfsetfillcolor{currentfill}%
\pgfsetfillopacity{0.700000}%
\pgfsetlinewidth{1.003750pt}%
\definecolor{currentstroke}{rgb}{1.000000,0.647059,0.000000}%
\pgfsetstrokecolor{currentstroke}%
\pgfsetstrokeopacity{0.700000}%
\pgfsetdash{}{0pt}%
\pgfpathmoveto{\pgfqpoint{1.452903in}{3.092665in}}%
\pgfpathcurveto{\pgfqpoint{1.458726in}{3.092665in}}{\pgfqpoint{1.464313in}{3.094978in}}{\pgfqpoint{1.468431in}{3.099097in}}%
\pgfpathcurveto{\pgfqpoint{1.472549in}{3.103215in}}{\pgfqpoint{1.474863in}{3.108801in}}{\pgfqpoint{1.474863in}{3.114625in}}%
\pgfpathcurveto{\pgfqpoint{1.474863in}{3.120449in}}{\pgfqpoint{1.472549in}{3.126035in}}{\pgfqpoint{1.468431in}{3.130153in}}%
\pgfpathcurveto{\pgfqpoint{1.464313in}{3.134271in}}{\pgfqpoint{1.458726in}{3.136585in}}{\pgfqpoint{1.452903in}{3.136585in}}%
\pgfpathcurveto{\pgfqpoint{1.447079in}{3.136585in}}{\pgfqpoint{1.441492in}{3.134271in}}{\pgfqpoint{1.437374in}{3.130153in}}%
\pgfpathcurveto{\pgfqpoint{1.433256in}{3.126035in}}{\pgfqpoint{1.430942in}{3.120449in}}{\pgfqpoint{1.430942in}{3.114625in}}%
\pgfpathcurveto{\pgfqpoint{1.430942in}{3.108801in}}{\pgfqpoint{1.433256in}{3.103215in}}{\pgfqpoint{1.437374in}{3.099097in}}%
\pgfpathcurveto{\pgfqpoint{1.441492in}{3.094978in}}{\pgfqpoint{1.447079in}{3.092665in}}{\pgfqpoint{1.452903in}{3.092665in}}%
\pgfpathlineto{\pgfqpoint{1.452903in}{3.092665in}}%
\pgfpathclose%
\pgfusepath{stroke,fill}%
\end{pgfscope}%
\begin{pgfscope}%
\pgfpathrectangle{\pgfqpoint{0.100000in}{0.183744in}}{\pgfqpoint{4.506048in}{4.506048in}}%
\pgfusepath{clip}%
\pgfsetbuttcap%
\pgfsetroundjoin%
\definecolor{currentfill}{rgb}{1.000000,0.647059,0.000000}%
\pgfsetfillcolor{currentfill}%
\pgfsetfillopacity{0.700000}%
\pgfsetlinewidth{1.003750pt}%
\definecolor{currentstroke}{rgb}{1.000000,0.647059,0.000000}%
\pgfsetstrokecolor{currentstroke}%
\pgfsetstrokeopacity{0.700000}%
\pgfsetdash{}{0pt}%
\pgfpathmoveto{\pgfqpoint{2.442337in}{2.205758in}}%
\pgfpathcurveto{\pgfqpoint{2.448161in}{2.205758in}}{\pgfqpoint{2.453747in}{2.208072in}}{\pgfqpoint{2.457865in}{2.212190in}}%
\pgfpathcurveto{\pgfqpoint{2.461983in}{2.216308in}}{\pgfqpoint{2.464297in}{2.221894in}}{\pgfqpoint{2.464297in}{2.227718in}}%
\pgfpathcurveto{\pgfqpoint{2.464297in}{2.233542in}}{\pgfqpoint{2.461983in}{2.239128in}}{\pgfqpoint{2.457865in}{2.243246in}}%
\pgfpathcurveto{\pgfqpoint{2.453747in}{2.247364in}}{\pgfqpoint{2.448161in}{2.249678in}}{\pgfqpoint{2.442337in}{2.249678in}}%
\pgfpathcurveto{\pgfqpoint{2.436513in}{2.249678in}}{\pgfqpoint{2.430927in}{2.247364in}}{\pgfqpoint{2.426809in}{2.243246in}}%
\pgfpathcurveto{\pgfqpoint{2.422690in}{2.239128in}}{\pgfqpoint{2.420377in}{2.233542in}}{\pgfqpoint{2.420377in}{2.227718in}}%
\pgfpathcurveto{\pgfqpoint{2.420377in}{2.221894in}}{\pgfqpoint{2.422690in}{2.216308in}}{\pgfqpoint{2.426809in}{2.212190in}}%
\pgfpathcurveto{\pgfqpoint{2.430927in}{2.208072in}}{\pgfqpoint{2.436513in}{2.205758in}}{\pgfqpoint{2.442337in}{2.205758in}}%
\pgfpathlineto{\pgfqpoint{2.442337in}{2.205758in}}%
\pgfpathclose%
\pgfusepath{stroke,fill}%
\end{pgfscope}%
\begin{pgfscope}%
\pgfpathrectangle{\pgfqpoint{0.100000in}{0.183744in}}{\pgfqpoint{4.506048in}{4.506048in}}%
\pgfusepath{clip}%
\pgfsetbuttcap%
\pgfsetroundjoin%
\definecolor{currentfill}{rgb}{1.000000,0.647059,0.000000}%
\pgfsetfillcolor{currentfill}%
\pgfsetfillopacity{0.700000}%
\pgfsetlinewidth{1.003750pt}%
\definecolor{currentstroke}{rgb}{1.000000,0.647059,0.000000}%
\pgfsetstrokecolor{currentstroke}%
\pgfsetstrokeopacity{0.700000}%
\pgfsetdash{}{0pt}%
\pgfpathmoveto{\pgfqpoint{2.574087in}{3.061849in}}%
\pgfpathcurveto{\pgfqpoint{2.579911in}{3.061849in}}{\pgfqpoint{2.585498in}{3.064163in}}{\pgfqpoint{2.589616in}{3.068281in}}%
\pgfpathcurveto{\pgfqpoint{2.593734in}{3.072399in}}{\pgfqpoint{2.596048in}{3.077985in}}{\pgfqpoint{2.596048in}{3.083809in}}%
\pgfpathcurveto{\pgfqpoint{2.596048in}{3.089633in}}{\pgfqpoint{2.593734in}{3.095219in}}{\pgfqpoint{2.589616in}{3.099337in}}%
\pgfpathcurveto{\pgfqpoint{2.585498in}{3.103456in}}{\pgfqpoint{2.579911in}{3.105769in}}{\pgfqpoint{2.574087in}{3.105769in}}%
\pgfpathcurveto{\pgfqpoint{2.568263in}{3.105769in}}{\pgfqpoint{2.562677in}{3.103456in}}{\pgfqpoint{2.558559in}{3.099337in}}%
\pgfpathcurveto{\pgfqpoint{2.554441in}{3.095219in}}{\pgfqpoint{2.552127in}{3.089633in}}{\pgfqpoint{2.552127in}{3.083809in}}%
\pgfpathcurveto{\pgfqpoint{2.552127in}{3.077985in}}{\pgfqpoint{2.554441in}{3.072399in}}{\pgfqpoint{2.558559in}{3.068281in}}%
\pgfpathcurveto{\pgfqpoint{2.562677in}{3.064163in}}{\pgfqpoint{2.568263in}{3.061849in}}{\pgfqpoint{2.574087in}{3.061849in}}%
\pgfpathlineto{\pgfqpoint{2.574087in}{3.061849in}}%
\pgfpathclose%
\pgfusepath{stroke,fill}%
\end{pgfscope}%
\begin{pgfscope}%
\pgfpathrectangle{\pgfqpoint{0.100000in}{0.183744in}}{\pgfqpoint{4.506048in}{4.506048in}}%
\pgfusepath{clip}%
\pgfsetbuttcap%
\pgfsetroundjoin%
\definecolor{currentfill}{rgb}{1.000000,0.647059,0.000000}%
\pgfsetfillcolor{currentfill}%
\pgfsetfillopacity{0.700000}%
\pgfsetlinewidth{1.003750pt}%
\definecolor{currentstroke}{rgb}{1.000000,0.647059,0.000000}%
\pgfsetstrokecolor{currentstroke}%
\pgfsetstrokeopacity{0.700000}%
\pgfsetdash{}{0pt}%
\pgfpathmoveto{\pgfqpoint{2.591809in}{2.647339in}}%
\pgfpathcurveto{\pgfqpoint{2.597633in}{2.647339in}}{\pgfqpoint{2.603219in}{2.649653in}}{\pgfqpoint{2.607338in}{2.653771in}}%
\pgfpathcurveto{\pgfqpoint{2.611456in}{2.657889in}}{\pgfqpoint{2.613770in}{2.663475in}}{\pgfqpoint{2.613770in}{2.669299in}}%
\pgfpathcurveto{\pgfqpoint{2.613770in}{2.675123in}}{\pgfqpoint{2.611456in}{2.680709in}}{\pgfqpoint{2.607338in}{2.684828in}}%
\pgfpathcurveto{\pgfqpoint{2.603219in}{2.688946in}}{\pgfqpoint{2.597633in}{2.691260in}}{\pgfqpoint{2.591809in}{2.691260in}}%
\pgfpathcurveto{\pgfqpoint{2.585985in}{2.691260in}}{\pgfqpoint{2.580399in}{2.688946in}}{\pgfqpoint{2.576281in}{2.684828in}}%
\pgfpathcurveto{\pgfqpoint{2.572163in}{2.680709in}}{\pgfqpoint{2.569849in}{2.675123in}}{\pgfqpoint{2.569849in}{2.669299in}}%
\pgfpathcurveto{\pgfqpoint{2.569849in}{2.663475in}}{\pgfqpoint{2.572163in}{2.657889in}}{\pgfqpoint{2.576281in}{2.653771in}}%
\pgfpathcurveto{\pgfqpoint{2.580399in}{2.649653in}}{\pgfqpoint{2.585985in}{2.647339in}}{\pgfqpoint{2.591809in}{2.647339in}}%
\pgfpathlineto{\pgfqpoint{2.591809in}{2.647339in}}%
\pgfpathclose%
\pgfusepath{stroke,fill}%
\end{pgfscope}%
\begin{pgfscope}%
\pgfpathrectangle{\pgfqpoint{0.100000in}{0.183744in}}{\pgfqpoint{4.506048in}{4.506048in}}%
\pgfusepath{clip}%
\pgfsetbuttcap%
\pgfsetroundjoin%
\definecolor{currentfill}{rgb}{1.000000,0.647059,0.000000}%
\pgfsetfillcolor{currentfill}%
\pgfsetfillopacity{0.700000}%
\pgfsetlinewidth{1.003750pt}%
\definecolor{currentstroke}{rgb}{1.000000,0.647059,0.000000}%
\pgfsetstrokecolor{currentstroke}%
\pgfsetstrokeopacity{0.700000}%
\pgfsetdash{}{0pt}%
\pgfpathmoveto{\pgfqpoint{3.661402in}{2.681421in}}%
\pgfpathcurveto{\pgfqpoint{3.667226in}{2.681421in}}{\pgfqpoint{3.672812in}{2.683735in}}{\pgfqpoint{3.676930in}{2.687853in}}%
\pgfpathcurveto{\pgfqpoint{3.681049in}{2.691971in}}{\pgfqpoint{3.683362in}{2.697558in}}{\pgfqpoint{3.683362in}{2.703381in}}%
\pgfpathcurveto{\pgfqpoint{3.683362in}{2.709205in}}{\pgfqpoint{3.681049in}{2.714792in}}{\pgfqpoint{3.676930in}{2.718910in}}%
\pgfpathcurveto{\pgfqpoint{3.672812in}{2.723028in}}{\pgfqpoint{3.667226in}{2.725342in}}{\pgfqpoint{3.661402in}{2.725342in}}%
\pgfpathcurveto{\pgfqpoint{3.655578in}{2.725342in}}{\pgfqpoint{3.649992in}{2.723028in}}{\pgfqpoint{3.645874in}{2.718910in}}%
\pgfpathcurveto{\pgfqpoint{3.641756in}{2.714792in}}{\pgfqpoint{3.639442in}{2.709205in}}{\pgfqpoint{3.639442in}{2.703381in}}%
\pgfpathcurveto{\pgfqpoint{3.639442in}{2.697558in}}{\pgfqpoint{3.641756in}{2.691971in}}{\pgfqpoint{3.645874in}{2.687853in}}%
\pgfpathcurveto{\pgfqpoint{3.649992in}{2.683735in}}{\pgfqpoint{3.655578in}{2.681421in}}{\pgfqpoint{3.661402in}{2.681421in}}%
\pgfpathlineto{\pgfqpoint{3.661402in}{2.681421in}}%
\pgfpathclose%
\pgfusepath{stroke,fill}%
\end{pgfscope}%
\begin{pgfscope}%
\pgfpathrectangle{\pgfqpoint{0.100000in}{0.183744in}}{\pgfqpoint{4.506048in}{4.506048in}}%
\pgfusepath{clip}%
\pgfsetbuttcap%
\pgfsetroundjoin%
\definecolor{currentfill}{rgb}{1.000000,0.647059,0.000000}%
\pgfsetfillcolor{currentfill}%
\pgfsetfillopacity{0.700000}%
\pgfsetlinewidth{1.003750pt}%
\definecolor{currentstroke}{rgb}{1.000000,0.647059,0.000000}%
\pgfsetstrokecolor{currentstroke}%
\pgfsetstrokeopacity{0.700000}%
\pgfsetdash{}{0pt}%
\pgfpathmoveto{\pgfqpoint{3.438312in}{2.846165in}}%
\pgfpathcurveto{\pgfqpoint{3.444136in}{2.846165in}}{\pgfqpoint{3.449722in}{2.848478in}}{\pgfqpoint{3.453840in}{2.852597in}}%
\pgfpathcurveto{\pgfqpoint{3.457958in}{2.856715in}}{\pgfqpoint{3.460272in}{2.862301in}}{\pgfqpoint{3.460272in}{2.868125in}}%
\pgfpathcurveto{\pgfqpoint{3.460272in}{2.873949in}}{\pgfqpoint{3.457958in}{2.879535in}}{\pgfqpoint{3.453840in}{2.883653in}}%
\pgfpathcurveto{\pgfqpoint{3.449722in}{2.887771in}}{\pgfqpoint{3.444136in}{2.890085in}}{\pgfqpoint{3.438312in}{2.890085in}}%
\pgfpathcurveto{\pgfqpoint{3.432488in}{2.890085in}}{\pgfqpoint{3.426902in}{2.887771in}}{\pgfqpoint{3.422784in}{2.883653in}}%
\pgfpathcurveto{\pgfqpoint{3.418665in}{2.879535in}}{\pgfqpoint{3.416352in}{2.873949in}}{\pgfqpoint{3.416352in}{2.868125in}}%
\pgfpathcurveto{\pgfqpoint{3.416352in}{2.862301in}}{\pgfqpoint{3.418665in}{2.856715in}}{\pgfqpoint{3.422784in}{2.852597in}}%
\pgfpathcurveto{\pgfqpoint{3.426902in}{2.848478in}}{\pgfqpoint{3.432488in}{2.846165in}}{\pgfqpoint{3.438312in}{2.846165in}}%
\pgfpathlineto{\pgfqpoint{3.438312in}{2.846165in}}%
\pgfpathclose%
\pgfusepath{stroke,fill}%
\end{pgfscope}%
\begin{pgfscope}%
\pgfpathrectangle{\pgfqpoint{0.100000in}{0.183744in}}{\pgfqpoint{4.506048in}{4.506048in}}%
\pgfusepath{clip}%
\pgfsetbuttcap%
\pgfsetroundjoin%
\definecolor{currentfill}{rgb}{1.000000,0.647059,0.000000}%
\pgfsetfillcolor{currentfill}%
\pgfsetfillopacity{0.700000}%
\pgfsetlinewidth{1.003750pt}%
\definecolor{currentstroke}{rgb}{1.000000,0.647059,0.000000}%
\pgfsetstrokecolor{currentstroke}%
\pgfsetstrokeopacity{0.700000}%
\pgfsetdash{}{0pt}%
\pgfpathmoveto{\pgfqpoint{3.314313in}{2.800250in}}%
\pgfpathcurveto{\pgfqpoint{3.320137in}{2.800250in}}{\pgfqpoint{3.325723in}{2.802564in}}{\pgfqpoint{3.329841in}{2.806682in}}%
\pgfpathcurveto{\pgfqpoint{3.333959in}{2.810801in}}{\pgfqpoint{3.336273in}{2.816387in}}{\pgfqpoint{3.336273in}{2.822211in}}%
\pgfpathcurveto{\pgfqpoint{3.336273in}{2.828035in}}{\pgfqpoint{3.333959in}{2.833621in}}{\pgfqpoint{3.329841in}{2.837739in}}%
\pgfpathcurveto{\pgfqpoint{3.325723in}{2.841857in}}{\pgfqpoint{3.320137in}{2.844171in}}{\pgfqpoint{3.314313in}{2.844171in}}%
\pgfpathcurveto{\pgfqpoint{3.308489in}{2.844171in}}{\pgfqpoint{3.302903in}{2.841857in}}{\pgfqpoint{3.298785in}{2.837739in}}%
\pgfpathcurveto{\pgfqpoint{3.294666in}{2.833621in}}{\pgfqpoint{3.292353in}{2.828035in}}{\pgfqpoint{3.292353in}{2.822211in}}%
\pgfpathcurveto{\pgfqpoint{3.292353in}{2.816387in}}{\pgfqpoint{3.294666in}{2.810801in}}{\pgfqpoint{3.298785in}{2.806682in}}%
\pgfpathcurveto{\pgfqpoint{3.302903in}{2.802564in}}{\pgfqpoint{3.308489in}{2.800250in}}{\pgfqpoint{3.314313in}{2.800250in}}%
\pgfpathlineto{\pgfqpoint{3.314313in}{2.800250in}}%
\pgfpathclose%
\pgfusepath{stroke,fill}%
\end{pgfscope}%
\begin{pgfscope}%
\pgfpathrectangle{\pgfqpoint{0.100000in}{0.183744in}}{\pgfqpoint{4.506048in}{4.506048in}}%
\pgfusepath{clip}%
\pgfsetbuttcap%
\pgfsetroundjoin%
\definecolor{currentfill}{rgb}{1.000000,0.647059,0.000000}%
\pgfsetfillcolor{currentfill}%
\pgfsetfillopacity{0.700000}%
\pgfsetlinewidth{1.003750pt}%
\definecolor{currentstroke}{rgb}{1.000000,0.647059,0.000000}%
\pgfsetstrokecolor{currentstroke}%
\pgfsetstrokeopacity{0.700000}%
\pgfsetdash{}{0pt}%
\pgfpathmoveto{\pgfqpoint{1.736277in}{3.047843in}}%
\pgfpathcurveto{\pgfqpoint{1.742101in}{3.047843in}}{\pgfqpoint{1.747687in}{3.050157in}}{\pgfqpoint{1.751805in}{3.054275in}}%
\pgfpathcurveto{\pgfqpoint{1.755924in}{3.058393in}}{\pgfqpoint{1.758237in}{3.063979in}}{\pgfqpoint{1.758237in}{3.069803in}}%
\pgfpathcurveto{\pgfqpoint{1.758237in}{3.075627in}}{\pgfqpoint{1.755924in}{3.081213in}}{\pgfqpoint{1.751805in}{3.085332in}}%
\pgfpathcurveto{\pgfqpoint{1.747687in}{3.089450in}}{\pgfqpoint{1.742101in}{3.091764in}}{\pgfqpoint{1.736277in}{3.091764in}}%
\pgfpathcurveto{\pgfqpoint{1.730453in}{3.091764in}}{\pgfqpoint{1.724867in}{3.089450in}}{\pgfqpoint{1.720749in}{3.085332in}}%
\pgfpathcurveto{\pgfqpoint{1.716631in}{3.081213in}}{\pgfqpoint{1.714317in}{3.075627in}}{\pgfqpoint{1.714317in}{3.069803in}}%
\pgfpathcurveto{\pgfqpoint{1.714317in}{3.063979in}}{\pgfqpoint{1.716631in}{3.058393in}}{\pgfqpoint{1.720749in}{3.054275in}}%
\pgfpathcurveto{\pgfqpoint{1.724867in}{3.050157in}}{\pgfqpoint{1.730453in}{3.047843in}}{\pgfqpoint{1.736277in}{3.047843in}}%
\pgfpathlineto{\pgfqpoint{1.736277in}{3.047843in}}%
\pgfpathclose%
\pgfusepath{stroke,fill}%
\end{pgfscope}%
\begin{pgfscope}%
\pgfpathrectangle{\pgfqpoint{0.100000in}{0.183744in}}{\pgfqpoint{4.506048in}{4.506048in}}%
\pgfusepath{clip}%
\pgfsetbuttcap%
\pgfsetroundjoin%
\definecolor{currentfill}{rgb}{1.000000,0.647059,0.000000}%
\pgfsetfillcolor{currentfill}%
\pgfsetfillopacity{0.700000}%
\pgfsetlinewidth{1.003750pt}%
\definecolor{currentstroke}{rgb}{1.000000,0.647059,0.000000}%
\pgfsetstrokecolor{currentstroke}%
\pgfsetstrokeopacity{0.700000}%
\pgfsetdash{}{0pt}%
\pgfpathmoveto{\pgfqpoint{1.069523in}{2.180292in}}%
\pgfpathcurveto{\pgfqpoint{1.075347in}{2.180292in}}{\pgfqpoint{1.080933in}{2.182606in}}{\pgfqpoint{1.085051in}{2.186725in}}%
\pgfpathcurveto{\pgfqpoint{1.089169in}{2.190843in}}{\pgfqpoint{1.091483in}{2.196429in}}{\pgfqpoint{1.091483in}{2.202253in}}%
\pgfpathcurveto{\pgfqpoint{1.091483in}{2.208077in}}{\pgfqpoint{1.089169in}{2.213663in}}{\pgfqpoint{1.085051in}{2.217781in}}%
\pgfpathcurveto{\pgfqpoint{1.080933in}{2.221899in}}{\pgfqpoint{1.075347in}{2.224213in}}{\pgfqpoint{1.069523in}{2.224213in}}%
\pgfpathcurveto{\pgfqpoint{1.063699in}{2.224213in}}{\pgfqpoint{1.058113in}{2.221899in}}{\pgfqpoint{1.053995in}{2.217781in}}%
\pgfpathcurveto{\pgfqpoint{1.049877in}{2.213663in}}{\pgfqpoint{1.047563in}{2.208077in}}{\pgfqpoint{1.047563in}{2.202253in}}%
\pgfpathcurveto{\pgfqpoint{1.047563in}{2.196429in}}{\pgfqpoint{1.049877in}{2.190843in}}{\pgfqpoint{1.053995in}{2.186725in}}%
\pgfpathcurveto{\pgfqpoint{1.058113in}{2.182606in}}{\pgfqpoint{1.063699in}{2.180292in}}{\pgfqpoint{1.069523in}{2.180292in}}%
\pgfpathlineto{\pgfqpoint{1.069523in}{2.180292in}}%
\pgfpathclose%
\pgfusepath{stroke,fill}%
\end{pgfscope}%
\begin{pgfscope}%
\pgfpathrectangle{\pgfqpoint{0.100000in}{0.183744in}}{\pgfqpoint{4.506048in}{4.506048in}}%
\pgfusepath{clip}%
\pgfsetbuttcap%
\pgfsetroundjoin%
\definecolor{currentfill}{rgb}{1.000000,0.647059,0.000000}%
\pgfsetfillcolor{currentfill}%
\pgfsetfillopacity{0.700000}%
\pgfsetlinewidth{1.003750pt}%
\definecolor{currentstroke}{rgb}{1.000000,0.647059,0.000000}%
\pgfsetstrokecolor{currentstroke}%
\pgfsetstrokeopacity{0.700000}%
\pgfsetdash{}{0pt}%
\pgfpathmoveto{\pgfqpoint{2.326541in}{2.731821in}}%
\pgfpathcurveto{\pgfqpoint{2.332365in}{2.731821in}}{\pgfqpoint{2.337952in}{2.734135in}}{\pgfqpoint{2.342070in}{2.738253in}}%
\pgfpathcurveto{\pgfqpoint{2.346188in}{2.742371in}}{\pgfqpoint{2.348502in}{2.747957in}}{\pgfqpoint{2.348502in}{2.753781in}}%
\pgfpathcurveto{\pgfqpoint{2.348502in}{2.759605in}}{\pgfqpoint{2.346188in}{2.765191in}}{\pgfqpoint{2.342070in}{2.769310in}}%
\pgfpathcurveto{\pgfqpoint{2.337952in}{2.773428in}}{\pgfqpoint{2.332365in}{2.775742in}}{\pgfqpoint{2.326541in}{2.775742in}}%
\pgfpathcurveto{\pgfqpoint{2.320718in}{2.775742in}}{\pgfqpoint{2.315131in}{2.773428in}}{\pgfqpoint{2.311013in}{2.769310in}}%
\pgfpathcurveto{\pgfqpoint{2.306895in}{2.765191in}}{\pgfqpoint{2.304581in}{2.759605in}}{\pgfqpoint{2.304581in}{2.753781in}}%
\pgfpathcurveto{\pgfqpoint{2.304581in}{2.747957in}}{\pgfqpoint{2.306895in}{2.742371in}}{\pgfqpoint{2.311013in}{2.738253in}}%
\pgfpathcurveto{\pgfqpoint{2.315131in}{2.734135in}}{\pgfqpoint{2.320718in}{2.731821in}}{\pgfqpoint{2.326541in}{2.731821in}}%
\pgfpathlineto{\pgfqpoint{2.326541in}{2.731821in}}%
\pgfpathclose%
\pgfusepath{stroke,fill}%
\end{pgfscope}%
\begin{pgfscope}%
\pgfpathrectangle{\pgfqpoint{0.100000in}{0.183744in}}{\pgfqpoint{4.506048in}{4.506048in}}%
\pgfusepath{clip}%
\pgfsetbuttcap%
\pgfsetroundjoin%
\definecolor{currentfill}{rgb}{1.000000,0.647059,0.000000}%
\pgfsetfillcolor{currentfill}%
\pgfsetfillopacity{0.700000}%
\pgfsetlinewidth{1.003750pt}%
\definecolor{currentstroke}{rgb}{1.000000,0.647059,0.000000}%
\pgfsetstrokecolor{currentstroke}%
\pgfsetstrokeopacity{0.700000}%
\pgfsetdash{}{0pt}%
\pgfpathmoveto{\pgfqpoint{1.076121in}{2.571669in}}%
\pgfpathcurveto{\pgfqpoint{1.081945in}{2.571669in}}{\pgfqpoint{1.087531in}{2.573983in}}{\pgfqpoint{1.091649in}{2.578101in}}%
\pgfpathcurveto{\pgfqpoint{1.095767in}{2.582220in}}{\pgfqpoint{1.098081in}{2.587806in}}{\pgfqpoint{1.098081in}{2.593630in}}%
\pgfpathcurveto{\pgfqpoint{1.098081in}{2.599454in}}{\pgfqpoint{1.095767in}{2.605040in}}{\pgfqpoint{1.091649in}{2.609158in}}%
\pgfpathcurveto{\pgfqpoint{1.087531in}{2.613276in}}{\pgfqpoint{1.081945in}{2.615590in}}{\pgfqpoint{1.076121in}{2.615590in}}%
\pgfpathcurveto{\pgfqpoint{1.070297in}{2.615590in}}{\pgfqpoint{1.064711in}{2.613276in}}{\pgfqpoint{1.060592in}{2.609158in}}%
\pgfpathcurveto{\pgfqpoint{1.056474in}{2.605040in}}{\pgfqpoint{1.054160in}{2.599454in}}{\pgfqpoint{1.054160in}{2.593630in}}%
\pgfpathcurveto{\pgfqpoint{1.054160in}{2.587806in}}{\pgfqpoint{1.056474in}{2.582220in}}{\pgfqpoint{1.060592in}{2.578101in}}%
\pgfpathcurveto{\pgfqpoint{1.064711in}{2.573983in}}{\pgfqpoint{1.070297in}{2.571669in}}{\pgfqpoint{1.076121in}{2.571669in}}%
\pgfpathlineto{\pgfqpoint{1.076121in}{2.571669in}}%
\pgfpathclose%
\pgfusepath{stroke,fill}%
\end{pgfscope}%
\begin{pgfscope}%
\pgfpathrectangle{\pgfqpoint{0.100000in}{0.183744in}}{\pgfqpoint{4.506048in}{4.506048in}}%
\pgfusepath{clip}%
\pgfsetbuttcap%
\pgfsetroundjoin%
\definecolor{currentfill}{rgb}{1.000000,0.647059,0.000000}%
\pgfsetfillcolor{currentfill}%
\pgfsetfillopacity{0.700000}%
\pgfsetlinewidth{1.003750pt}%
\definecolor{currentstroke}{rgb}{1.000000,0.647059,0.000000}%
\pgfsetstrokecolor{currentstroke}%
\pgfsetstrokeopacity{0.700000}%
\pgfsetdash{}{0pt}%
\pgfpathmoveto{\pgfqpoint{2.354189in}{3.290533in}}%
\pgfpathcurveto{\pgfqpoint{2.360013in}{3.290533in}}{\pgfqpoint{2.365599in}{3.292847in}}{\pgfqpoint{2.369717in}{3.296965in}}%
\pgfpathcurveto{\pgfqpoint{2.373835in}{3.301083in}}{\pgfqpoint{2.376149in}{3.306669in}}{\pgfqpoint{2.376149in}{3.312493in}}%
\pgfpathcurveto{\pgfqpoint{2.376149in}{3.318317in}}{\pgfqpoint{2.373835in}{3.323903in}}{\pgfqpoint{2.369717in}{3.328021in}}%
\pgfpathcurveto{\pgfqpoint{2.365599in}{3.332140in}}{\pgfqpoint{2.360013in}{3.334454in}}{\pgfqpoint{2.354189in}{3.334454in}}%
\pgfpathcurveto{\pgfqpoint{2.348365in}{3.334454in}}{\pgfqpoint{2.342779in}{3.332140in}}{\pgfqpoint{2.338661in}{3.328021in}}%
\pgfpathcurveto{\pgfqpoint{2.334542in}{3.323903in}}{\pgfqpoint{2.332229in}{3.318317in}}{\pgfqpoint{2.332229in}{3.312493in}}%
\pgfpathcurveto{\pgfqpoint{2.332229in}{3.306669in}}{\pgfqpoint{2.334542in}{3.301083in}}{\pgfqpoint{2.338661in}{3.296965in}}%
\pgfpathcurveto{\pgfqpoint{2.342779in}{3.292847in}}{\pgfqpoint{2.348365in}{3.290533in}}{\pgfqpoint{2.354189in}{3.290533in}}%
\pgfpathlineto{\pgfqpoint{2.354189in}{3.290533in}}%
\pgfpathclose%
\pgfusepath{stroke,fill}%
\end{pgfscope}%
\begin{pgfscope}%
\pgfpathrectangle{\pgfqpoint{0.100000in}{0.183744in}}{\pgfqpoint{4.506048in}{4.506048in}}%
\pgfusepath{clip}%
\pgfsetbuttcap%
\pgfsetroundjoin%
\definecolor{currentfill}{rgb}{1.000000,0.647059,0.000000}%
\pgfsetfillcolor{currentfill}%
\pgfsetfillopacity{0.700000}%
\pgfsetlinewidth{1.003750pt}%
\definecolor{currentstroke}{rgb}{1.000000,0.647059,0.000000}%
\pgfsetstrokecolor{currentstroke}%
\pgfsetstrokeopacity{0.700000}%
\pgfsetdash{}{0pt}%
\pgfpathmoveto{\pgfqpoint{2.623141in}{2.939456in}}%
\pgfpathcurveto{\pgfqpoint{2.628965in}{2.939456in}}{\pgfqpoint{2.634551in}{2.941770in}}{\pgfqpoint{2.638669in}{2.945888in}}%
\pgfpathcurveto{\pgfqpoint{2.642787in}{2.950006in}}{\pgfqpoint{2.645101in}{2.955592in}}{\pgfqpoint{2.645101in}{2.961416in}}%
\pgfpathcurveto{\pgfqpoint{2.645101in}{2.967240in}}{\pgfqpoint{2.642787in}{2.972826in}}{\pgfqpoint{2.638669in}{2.976944in}}%
\pgfpathcurveto{\pgfqpoint{2.634551in}{2.981062in}}{\pgfqpoint{2.628965in}{2.983376in}}{\pgfqpoint{2.623141in}{2.983376in}}%
\pgfpathcurveto{\pgfqpoint{2.617317in}{2.983376in}}{\pgfqpoint{2.611731in}{2.981062in}}{\pgfqpoint{2.607613in}{2.976944in}}%
\pgfpathcurveto{\pgfqpoint{2.603495in}{2.972826in}}{\pgfqpoint{2.601181in}{2.967240in}}{\pgfqpoint{2.601181in}{2.961416in}}%
\pgfpathcurveto{\pgfqpoint{2.601181in}{2.955592in}}{\pgfqpoint{2.603495in}{2.950006in}}{\pgfqpoint{2.607613in}{2.945888in}}%
\pgfpathcurveto{\pgfqpoint{2.611731in}{2.941770in}}{\pgfqpoint{2.617317in}{2.939456in}}{\pgfqpoint{2.623141in}{2.939456in}}%
\pgfpathlineto{\pgfqpoint{2.623141in}{2.939456in}}%
\pgfpathclose%
\pgfusepath{stroke,fill}%
\end{pgfscope}%
\begin{pgfscope}%
\pgfpathrectangle{\pgfqpoint{0.100000in}{0.183744in}}{\pgfqpoint{4.506048in}{4.506048in}}%
\pgfusepath{clip}%
\pgfsetbuttcap%
\pgfsetroundjoin%
\definecolor{currentfill}{rgb}{1.000000,0.647059,0.000000}%
\pgfsetfillcolor{currentfill}%
\pgfsetfillopacity{0.700000}%
\pgfsetlinewidth{1.003750pt}%
\definecolor{currentstroke}{rgb}{1.000000,0.647059,0.000000}%
\pgfsetstrokecolor{currentstroke}%
\pgfsetstrokeopacity{0.700000}%
\pgfsetdash{}{0pt}%
\pgfpathmoveto{\pgfqpoint{3.244738in}{2.745832in}}%
\pgfpathcurveto{\pgfqpoint{3.250562in}{2.745832in}}{\pgfqpoint{3.256148in}{2.748146in}}{\pgfqpoint{3.260266in}{2.752264in}}%
\pgfpathcurveto{\pgfqpoint{3.264384in}{2.756382in}}{\pgfqpoint{3.266698in}{2.761968in}}{\pgfqpoint{3.266698in}{2.767792in}}%
\pgfpathcurveto{\pgfqpoint{3.266698in}{2.773616in}}{\pgfqpoint{3.264384in}{2.779202in}}{\pgfqpoint{3.260266in}{2.783320in}}%
\pgfpathcurveto{\pgfqpoint{3.256148in}{2.787438in}}{\pgfqpoint{3.250562in}{2.789752in}}{\pgfqpoint{3.244738in}{2.789752in}}%
\pgfpathcurveto{\pgfqpoint{3.238914in}{2.789752in}}{\pgfqpoint{3.233328in}{2.787438in}}{\pgfqpoint{3.229210in}{2.783320in}}%
\pgfpathcurveto{\pgfqpoint{3.225092in}{2.779202in}}{\pgfqpoint{3.222778in}{2.773616in}}{\pgfqpoint{3.222778in}{2.767792in}}%
\pgfpathcurveto{\pgfqpoint{3.222778in}{2.761968in}}{\pgfqpoint{3.225092in}{2.756382in}}{\pgfqpoint{3.229210in}{2.752264in}}%
\pgfpathcurveto{\pgfqpoint{3.233328in}{2.748146in}}{\pgfqpoint{3.238914in}{2.745832in}}{\pgfqpoint{3.244738in}{2.745832in}}%
\pgfpathlineto{\pgfqpoint{3.244738in}{2.745832in}}%
\pgfpathclose%
\pgfusepath{stroke,fill}%
\end{pgfscope}%
\begin{pgfscope}%
\pgfpathrectangle{\pgfqpoint{0.100000in}{0.183744in}}{\pgfqpoint{4.506048in}{4.506048in}}%
\pgfusepath{clip}%
\pgfsetbuttcap%
\pgfsetroundjoin%
\definecolor{currentfill}{rgb}{1.000000,0.647059,0.000000}%
\pgfsetfillcolor{currentfill}%
\pgfsetfillopacity{0.700000}%
\pgfsetlinewidth{1.003750pt}%
\definecolor{currentstroke}{rgb}{1.000000,0.647059,0.000000}%
\pgfsetstrokecolor{currentstroke}%
\pgfsetstrokeopacity{0.700000}%
\pgfsetdash{}{0pt}%
\pgfpathmoveto{\pgfqpoint{3.519638in}{2.852567in}}%
\pgfpathcurveto{\pgfqpoint{3.525462in}{2.852567in}}{\pgfqpoint{3.531048in}{2.854881in}}{\pgfqpoint{3.535166in}{2.858999in}}%
\pgfpathcurveto{\pgfqpoint{3.539284in}{2.863117in}}{\pgfqpoint{3.541598in}{2.868703in}}{\pgfqpoint{3.541598in}{2.874527in}}%
\pgfpathcurveto{\pgfqpoint{3.541598in}{2.880351in}}{\pgfqpoint{3.539284in}{2.885937in}}{\pgfqpoint{3.535166in}{2.890055in}}%
\pgfpathcurveto{\pgfqpoint{3.531048in}{2.894173in}}{\pgfqpoint{3.525462in}{2.896487in}}{\pgfqpoint{3.519638in}{2.896487in}}%
\pgfpathcurveto{\pgfqpoint{3.513814in}{2.896487in}}{\pgfqpoint{3.508227in}{2.894173in}}{\pgfqpoint{3.504109in}{2.890055in}}%
\pgfpathcurveto{\pgfqpoint{3.499991in}{2.885937in}}{\pgfqpoint{3.497677in}{2.880351in}}{\pgfqpoint{3.497677in}{2.874527in}}%
\pgfpathcurveto{\pgfqpoint{3.497677in}{2.868703in}}{\pgfqpoint{3.499991in}{2.863117in}}{\pgfqpoint{3.504109in}{2.858999in}}%
\pgfpathcurveto{\pgfqpoint{3.508227in}{2.854881in}}{\pgfqpoint{3.513814in}{2.852567in}}{\pgfqpoint{3.519638in}{2.852567in}}%
\pgfpathlineto{\pgfqpoint{3.519638in}{2.852567in}}%
\pgfpathclose%
\pgfusepath{stroke,fill}%
\end{pgfscope}%
\begin{pgfscope}%
\pgfpathrectangle{\pgfqpoint{0.100000in}{0.183744in}}{\pgfqpoint{4.506048in}{4.506048in}}%
\pgfusepath{clip}%
\pgfsetbuttcap%
\pgfsetroundjoin%
\definecolor{currentfill}{rgb}{1.000000,0.647059,0.000000}%
\pgfsetfillcolor{currentfill}%
\pgfsetfillopacity{0.700000}%
\pgfsetlinewidth{1.003750pt}%
\definecolor{currentstroke}{rgb}{1.000000,0.647059,0.000000}%
\pgfsetstrokecolor{currentstroke}%
\pgfsetstrokeopacity{0.700000}%
\pgfsetdash{}{0pt}%
\pgfpathmoveto{\pgfqpoint{1.491418in}{1.949135in}}%
\pgfpathcurveto{\pgfqpoint{1.497242in}{1.949135in}}{\pgfqpoint{1.502829in}{1.951449in}}{\pgfqpoint{1.506947in}{1.955567in}}%
\pgfpathcurveto{\pgfqpoint{1.511065in}{1.959685in}}{\pgfqpoint{1.513379in}{1.965272in}}{\pgfqpoint{1.513379in}{1.971095in}}%
\pgfpathcurveto{\pgfqpoint{1.513379in}{1.976919in}}{\pgfqpoint{1.511065in}{1.982506in}}{\pgfqpoint{1.506947in}{1.986624in}}%
\pgfpathcurveto{\pgfqpoint{1.502829in}{1.990742in}}{\pgfqpoint{1.497242in}{1.993056in}}{\pgfqpoint{1.491418in}{1.993056in}}%
\pgfpathcurveto{\pgfqpoint{1.485595in}{1.993056in}}{\pgfqpoint{1.480008in}{1.990742in}}{\pgfqpoint{1.475890in}{1.986624in}}%
\pgfpathcurveto{\pgfqpoint{1.471772in}{1.982506in}}{\pgfqpoint{1.469458in}{1.976919in}}{\pgfqpoint{1.469458in}{1.971095in}}%
\pgfpathcurveto{\pgfqpoint{1.469458in}{1.965272in}}{\pgfqpoint{1.471772in}{1.959685in}}{\pgfqpoint{1.475890in}{1.955567in}}%
\pgfpathcurveto{\pgfqpoint{1.480008in}{1.951449in}}{\pgfqpoint{1.485595in}{1.949135in}}{\pgfqpoint{1.491418in}{1.949135in}}%
\pgfpathlineto{\pgfqpoint{1.491418in}{1.949135in}}%
\pgfpathclose%
\pgfusepath{stroke,fill}%
\end{pgfscope}%
\begin{pgfscope}%
\pgfpathrectangle{\pgfqpoint{0.100000in}{0.183744in}}{\pgfqpoint{4.506048in}{4.506048in}}%
\pgfusepath{clip}%
\pgfsetbuttcap%
\pgfsetroundjoin%
\definecolor{currentfill}{rgb}{1.000000,0.647059,0.000000}%
\pgfsetfillcolor{currentfill}%
\pgfsetfillopacity{0.700000}%
\pgfsetlinewidth{1.003750pt}%
\definecolor{currentstroke}{rgb}{1.000000,0.647059,0.000000}%
\pgfsetstrokecolor{currentstroke}%
\pgfsetstrokeopacity{0.700000}%
\pgfsetdash{}{0pt}%
\pgfpathmoveto{\pgfqpoint{1.789856in}{3.562090in}}%
\pgfpathcurveto{\pgfqpoint{1.795680in}{3.562090in}}{\pgfqpoint{1.801266in}{3.564404in}}{\pgfqpoint{1.805385in}{3.568522in}}%
\pgfpathcurveto{\pgfqpoint{1.809503in}{3.572640in}}{\pgfqpoint{1.811817in}{3.578227in}}{\pgfqpoint{1.811817in}{3.584051in}}%
\pgfpathcurveto{\pgfqpoint{1.811817in}{3.589874in}}{\pgfqpoint{1.809503in}{3.595461in}}{\pgfqpoint{1.805385in}{3.599579in}}%
\pgfpathcurveto{\pgfqpoint{1.801266in}{3.603697in}}{\pgfqpoint{1.795680in}{3.606011in}}{\pgfqpoint{1.789856in}{3.606011in}}%
\pgfpathcurveto{\pgfqpoint{1.784032in}{3.606011in}}{\pgfqpoint{1.778446in}{3.603697in}}{\pgfqpoint{1.774328in}{3.599579in}}%
\pgfpathcurveto{\pgfqpoint{1.770210in}{3.595461in}}{\pgfqpoint{1.767896in}{3.589874in}}{\pgfqpoint{1.767896in}{3.584051in}}%
\pgfpathcurveto{\pgfqpoint{1.767896in}{3.578227in}}{\pgfqpoint{1.770210in}{3.572640in}}{\pgfqpoint{1.774328in}{3.568522in}}%
\pgfpathcurveto{\pgfqpoint{1.778446in}{3.564404in}}{\pgfqpoint{1.784032in}{3.562090in}}{\pgfqpoint{1.789856in}{3.562090in}}%
\pgfpathlineto{\pgfqpoint{1.789856in}{3.562090in}}%
\pgfpathclose%
\pgfusepath{stroke,fill}%
\end{pgfscope}%
\begin{pgfscope}%
\pgfpathrectangle{\pgfqpoint{0.100000in}{0.183744in}}{\pgfqpoint{4.506048in}{4.506048in}}%
\pgfusepath{clip}%
\pgfsetbuttcap%
\pgfsetroundjoin%
\definecolor{currentfill}{rgb}{1.000000,0.647059,0.000000}%
\pgfsetfillcolor{currentfill}%
\pgfsetfillopacity{0.700000}%
\pgfsetlinewidth{1.003750pt}%
\definecolor{currentstroke}{rgb}{1.000000,0.647059,0.000000}%
\pgfsetstrokecolor{currentstroke}%
\pgfsetstrokeopacity{0.700000}%
\pgfsetdash{}{0pt}%
\pgfpathmoveto{\pgfqpoint{2.799694in}{3.099741in}}%
\pgfpathcurveto{\pgfqpoint{2.805518in}{3.099741in}}{\pgfqpoint{2.811104in}{3.102054in}}{\pgfqpoint{2.815222in}{3.106173in}}%
\pgfpathcurveto{\pgfqpoint{2.819340in}{3.110291in}}{\pgfqpoint{2.821654in}{3.115877in}}{\pgfqpoint{2.821654in}{3.121701in}}%
\pgfpathcurveto{\pgfqpoint{2.821654in}{3.127525in}}{\pgfqpoint{2.819340in}{3.133111in}}{\pgfqpoint{2.815222in}{3.137229in}}%
\pgfpathcurveto{\pgfqpoint{2.811104in}{3.141347in}}{\pgfqpoint{2.805518in}{3.143661in}}{\pgfqpoint{2.799694in}{3.143661in}}%
\pgfpathcurveto{\pgfqpoint{2.793870in}{3.143661in}}{\pgfqpoint{2.788284in}{3.141347in}}{\pgfqpoint{2.784166in}{3.137229in}}%
\pgfpathcurveto{\pgfqpoint{2.780047in}{3.133111in}}{\pgfqpoint{2.777734in}{3.127525in}}{\pgfqpoint{2.777734in}{3.121701in}}%
\pgfpathcurveto{\pgfqpoint{2.777734in}{3.115877in}}{\pgfqpoint{2.780047in}{3.110291in}}{\pgfqpoint{2.784166in}{3.106173in}}%
\pgfpathcurveto{\pgfqpoint{2.788284in}{3.102054in}}{\pgfqpoint{2.793870in}{3.099741in}}{\pgfqpoint{2.799694in}{3.099741in}}%
\pgfpathlineto{\pgfqpoint{2.799694in}{3.099741in}}%
\pgfpathclose%
\pgfusepath{stroke,fill}%
\end{pgfscope}%
\begin{pgfscope}%
\pgfpathrectangle{\pgfqpoint{0.100000in}{0.183744in}}{\pgfqpoint{4.506048in}{4.506048in}}%
\pgfusepath{clip}%
\pgfsetbuttcap%
\pgfsetroundjoin%
\definecolor{currentfill}{rgb}{1.000000,0.647059,0.000000}%
\pgfsetfillcolor{currentfill}%
\pgfsetfillopacity{0.700000}%
\pgfsetlinewidth{1.003750pt}%
\definecolor{currentstroke}{rgb}{1.000000,0.647059,0.000000}%
\pgfsetstrokecolor{currentstroke}%
\pgfsetstrokeopacity{0.700000}%
\pgfsetdash{}{0pt}%
\pgfpathmoveto{\pgfqpoint{2.828266in}{2.954461in}}%
\pgfpathcurveto{\pgfqpoint{2.834090in}{2.954461in}}{\pgfqpoint{2.839676in}{2.956774in}}{\pgfqpoint{2.843794in}{2.960893in}}%
\pgfpathcurveto{\pgfqpoint{2.847912in}{2.965011in}}{\pgfqpoint{2.850226in}{2.970597in}}{\pgfqpoint{2.850226in}{2.976421in}}%
\pgfpathcurveto{\pgfqpoint{2.850226in}{2.982245in}}{\pgfqpoint{2.847912in}{2.987831in}}{\pgfqpoint{2.843794in}{2.991949in}}%
\pgfpathcurveto{\pgfqpoint{2.839676in}{2.996067in}}{\pgfqpoint{2.834090in}{2.998381in}}{\pgfqpoint{2.828266in}{2.998381in}}%
\pgfpathcurveto{\pgfqpoint{2.822442in}{2.998381in}}{\pgfqpoint{2.816856in}{2.996067in}}{\pgfqpoint{2.812737in}{2.991949in}}%
\pgfpathcurveto{\pgfqpoint{2.808619in}{2.987831in}}{\pgfqpoint{2.806305in}{2.982245in}}{\pgfqpoint{2.806305in}{2.976421in}}%
\pgfpathcurveto{\pgfqpoint{2.806305in}{2.970597in}}{\pgfqpoint{2.808619in}{2.965011in}}{\pgfqpoint{2.812737in}{2.960893in}}%
\pgfpathcurveto{\pgfqpoint{2.816856in}{2.956774in}}{\pgfqpoint{2.822442in}{2.954461in}}{\pgfqpoint{2.828266in}{2.954461in}}%
\pgfpathlineto{\pgfqpoint{2.828266in}{2.954461in}}%
\pgfpathclose%
\pgfusepath{stroke,fill}%
\end{pgfscope}%
\begin{pgfscope}%
\pgfpathrectangle{\pgfqpoint{0.100000in}{0.183744in}}{\pgfqpoint{4.506048in}{4.506048in}}%
\pgfusepath{clip}%
\pgfsetbuttcap%
\pgfsetroundjoin%
\definecolor{currentfill}{rgb}{1.000000,0.647059,0.000000}%
\pgfsetfillcolor{currentfill}%
\pgfsetfillopacity{0.700000}%
\pgfsetlinewidth{1.003750pt}%
\definecolor{currentstroke}{rgb}{1.000000,0.647059,0.000000}%
\pgfsetstrokecolor{currentstroke}%
\pgfsetstrokeopacity{0.700000}%
\pgfsetdash{}{0pt}%
\pgfpathmoveto{\pgfqpoint{1.127180in}{3.514173in}}%
\pgfpathcurveto{\pgfqpoint{1.133004in}{3.514173in}}{\pgfqpoint{1.138590in}{3.516487in}}{\pgfqpoint{1.142708in}{3.520605in}}%
\pgfpathcurveto{\pgfqpoint{1.146827in}{3.524723in}}{\pgfqpoint{1.149140in}{3.530309in}}{\pgfqpoint{1.149140in}{3.536133in}}%
\pgfpathcurveto{\pgfqpoint{1.149140in}{3.541957in}}{\pgfqpoint{1.146827in}{3.547543in}}{\pgfqpoint{1.142708in}{3.551661in}}%
\pgfpathcurveto{\pgfqpoint{1.138590in}{3.555779in}}{\pgfqpoint{1.133004in}{3.558093in}}{\pgfqpoint{1.127180in}{3.558093in}}%
\pgfpathcurveto{\pgfqpoint{1.121356in}{3.558093in}}{\pgfqpoint{1.115770in}{3.555779in}}{\pgfqpoint{1.111652in}{3.551661in}}%
\pgfpathcurveto{\pgfqpoint{1.107534in}{3.547543in}}{\pgfqpoint{1.105220in}{3.541957in}}{\pgfqpoint{1.105220in}{3.536133in}}%
\pgfpathcurveto{\pgfqpoint{1.105220in}{3.530309in}}{\pgfqpoint{1.107534in}{3.524723in}}{\pgfqpoint{1.111652in}{3.520605in}}%
\pgfpathcurveto{\pgfqpoint{1.115770in}{3.516487in}}{\pgfqpoint{1.121356in}{3.514173in}}{\pgfqpoint{1.127180in}{3.514173in}}%
\pgfpathlineto{\pgfqpoint{1.127180in}{3.514173in}}%
\pgfpathclose%
\pgfusepath{stroke,fill}%
\end{pgfscope}%
\begin{pgfscope}%
\pgfpathrectangle{\pgfqpoint{0.100000in}{0.183744in}}{\pgfqpoint{4.506048in}{4.506048in}}%
\pgfusepath{clip}%
\pgfsetbuttcap%
\pgfsetroundjoin%
\definecolor{currentfill}{rgb}{1.000000,0.647059,0.000000}%
\pgfsetfillcolor{currentfill}%
\pgfsetfillopacity{0.700000}%
\pgfsetlinewidth{1.003750pt}%
\definecolor{currentstroke}{rgb}{1.000000,0.647059,0.000000}%
\pgfsetstrokecolor{currentstroke}%
\pgfsetstrokeopacity{0.700000}%
\pgfsetdash{}{0pt}%
\pgfpathmoveto{\pgfqpoint{0.963442in}{1.749514in}}%
\pgfpathcurveto{\pgfqpoint{0.969266in}{1.749514in}}{\pgfqpoint{0.974852in}{1.751828in}}{\pgfqpoint{0.978970in}{1.755946in}}%
\pgfpathcurveto{\pgfqpoint{0.983088in}{1.760064in}}{\pgfqpoint{0.985402in}{1.765650in}}{\pgfqpoint{0.985402in}{1.771474in}}%
\pgfpathcurveto{\pgfqpoint{0.985402in}{1.777298in}}{\pgfqpoint{0.983088in}{1.782884in}}{\pgfqpoint{0.978970in}{1.787002in}}%
\pgfpathcurveto{\pgfqpoint{0.974852in}{1.791121in}}{\pgfqpoint{0.969266in}{1.793434in}}{\pgfqpoint{0.963442in}{1.793434in}}%
\pgfpathcurveto{\pgfqpoint{0.957618in}{1.793434in}}{\pgfqpoint{0.952032in}{1.791121in}}{\pgfqpoint{0.947913in}{1.787002in}}%
\pgfpathcurveto{\pgfqpoint{0.943795in}{1.782884in}}{\pgfqpoint{0.941481in}{1.777298in}}{\pgfqpoint{0.941481in}{1.771474in}}%
\pgfpathcurveto{\pgfqpoint{0.941481in}{1.765650in}}{\pgfqpoint{0.943795in}{1.760064in}}{\pgfqpoint{0.947913in}{1.755946in}}%
\pgfpathcurveto{\pgfqpoint{0.952032in}{1.751828in}}{\pgfqpoint{0.957618in}{1.749514in}}{\pgfqpoint{0.963442in}{1.749514in}}%
\pgfpathlineto{\pgfqpoint{0.963442in}{1.749514in}}%
\pgfpathclose%
\pgfusepath{stroke,fill}%
\end{pgfscope}%
\begin{pgfscope}%
\pgfpathrectangle{\pgfqpoint{0.100000in}{0.183744in}}{\pgfqpoint{4.506048in}{4.506048in}}%
\pgfusepath{clip}%
\pgfsetbuttcap%
\pgfsetroundjoin%
\definecolor{currentfill}{rgb}{1.000000,0.647059,0.000000}%
\pgfsetfillcolor{currentfill}%
\pgfsetfillopacity{0.700000}%
\pgfsetlinewidth{1.003750pt}%
\definecolor{currentstroke}{rgb}{1.000000,0.647059,0.000000}%
\pgfsetstrokecolor{currentstroke}%
\pgfsetstrokeopacity{0.700000}%
\pgfsetdash{}{0pt}%
\pgfpathmoveto{\pgfqpoint{2.177144in}{2.346786in}}%
\pgfpathcurveto{\pgfqpoint{2.182968in}{2.346786in}}{\pgfqpoint{2.188554in}{2.349100in}}{\pgfqpoint{2.192672in}{2.353218in}}%
\pgfpathcurveto{\pgfqpoint{2.196790in}{2.357336in}}{\pgfqpoint{2.199104in}{2.362922in}}{\pgfqpoint{2.199104in}{2.368746in}}%
\pgfpathcurveto{\pgfqpoint{2.199104in}{2.374570in}}{\pgfqpoint{2.196790in}{2.380156in}}{\pgfqpoint{2.192672in}{2.384275in}}%
\pgfpathcurveto{\pgfqpoint{2.188554in}{2.388393in}}{\pgfqpoint{2.182968in}{2.390707in}}{\pgfqpoint{2.177144in}{2.390707in}}%
\pgfpathcurveto{\pgfqpoint{2.171320in}{2.390707in}}{\pgfqpoint{2.165734in}{2.388393in}}{\pgfqpoint{2.161616in}{2.384275in}}%
\pgfpathcurveto{\pgfqpoint{2.157498in}{2.380156in}}{\pgfqpoint{2.155184in}{2.374570in}}{\pgfqpoint{2.155184in}{2.368746in}}%
\pgfpathcurveto{\pgfqpoint{2.155184in}{2.362922in}}{\pgfqpoint{2.157498in}{2.357336in}}{\pgfqpoint{2.161616in}{2.353218in}}%
\pgfpathcurveto{\pgfqpoint{2.165734in}{2.349100in}}{\pgfqpoint{2.171320in}{2.346786in}}{\pgfqpoint{2.177144in}{2.346786in}}%
\pgfpathlineto{\pgfqpoint{2.177144in}{2.346786in}}%
\pgfpathclose%
\pgfusepath{stroke,fill}%
\end{pgfscope}%
\begin{pgfscope}%
\pgfpathrectangle{\pgfqpoint{0.100000in}{0.183744in}}{\pgfqpoint{4.506048in}{4.506048in}}%
\pgfusepath{clip}%
\pgfsetbuttcap%
\pgfsetroundjoin%
\definecolor{currentfill}{rgb}{1.000000,0.647059,0.000000}%
\pgfsetfillcolor{currentfill}%
\pgfsetfillopacity{0.700000}%
\pgfsetlinewidth{1.003750pt}%
\definecolor{currentstroke}{rgb}{1.000000,0.647059,0.000000}%
\pgfsetstrokecolor{currentstroke}%
\pgfsetstrokeopacity{0.700000}%
\pgfsetdash{}{0pt}%
\pgfpathmoveto{\pgfqpoint{1.813533in}{2.240305in}}%
\pgfpathcurveto{\pgfqpoint{1.819357in}{2.240305in}}{\pgfqpoint{1.824943in}{2.242619in}}{\pgfqpoint{1.829061in}{2.246737in}}%
\pgfpathcurveto{\pgfqpoint{1.833179in}{2.250855in}}{\pgfqpoint{1.835493in}{2.256441in}}{\pgfqpoint{1.835493in}{2.262265in}}%
\pgfpathcurveto{\pgfqpoint{1.835493in}{2.268089in}}{\pgfqpoint{1.833179in}{2.273675in}}{\pgfqpoint{1.829061in}{2.277793in}}%
\pgfpathcurveto{\pgfqpoint{1.824943in}{2.281911in}}{\pgfqpoint{1.819357in}{2.284225in}}{\pgfqpoint{1.813533in}{2.284225in}}%
\pgfpathcurveto{\pgfqpoint{1.807709in}{2.284225in}}{\pgfqpoint{1.802123in}{2.281911in}}{\pgfqpoint{1.798005in}{2.277793in}}%
\pgfpathcurveto{\pgfqpoint{1.793887in}{2.273675in}}{\pgfqpoint{1.791573in}{2.268089in}}{\pgfqpoint{1.791573in}{2.262265in}}%
\pgfpathcurveto{\pgfqpoint{1.791573in}{2.256441in}}{\pgfqpoint{1.793887in}{2.250855in}}{\pgfqpoint{1.798005in}{2.246737in}}%
\pgfpathcurveto{\pgfqpoint{1.802123in}{2.242619in}}{\pgfqpoint{1.807709in}{2.240305in}}{\pgfqpoint{1.813533in}{2.240305in}}%
\pgfpathlineto{\pgfqpoint{1.813533in}{2.240305in}}%
\pgfpathclose%
\pgfusepath{stroke,fill}%
\end{pgfscope}%
\begin{pgfscope}%
\pgfpathrectangle{\pgfqpoint{0.100000in}{0.183744in}}{\pgfqpoint{4.506048in}{4.506048in}}%
\pgfusepath{clip}%
\pgfsetbuttcap%
\pgfsetroundjoin%
\definecolor{currentfill}{rgb}{1.000000,0.647059,0.000000}%
\pgfsetfillcolor{currentfill}%
\pgfsetfillopacity{0.700000}%
\pgfsetlinewidth{1.003750pt}%
\definecolor{currentstroke}{rgb}{1.000000,0.647059,0.000000}%
\pgfsetstrokecolor{currentstroke}%
\pgfsetstrokeopacity{0.700000}%
\pgfsetdash{}{0pt}%
\pgfpathmoveto{\pgfqpoint{2.411150in}{3.380559in}}%
\pgfpathcurveto{\pgfqpoint{2.416974in}{3.380559in}}{\pgfqpoint{2.422561in}{3.382873in}}{\pgfqpoint{2.426679in}{3.386991in}}%
\pgfpathcurveto{\pgfqpoint{2.430797in}{3.391109in}}{\pgfqpoint{2.433111in}{3.396695in}}{\pgfqpoint{2.433111in}{3.402519in}}%
\pgfpathcurveto{\pgfqpoint{2.433111in}{3.408343in}}{\pgfqpoint{2.430797in}{3.413929in}}{\pgfqpoint{2.426679in}{3.418048in}}%
\pgfpathcurveto{\pgfqpoint{2.422561in}{3.422166in}}{\pgfqpoint{2.416974in}{3.424480in}}{\pgfqpoint{2.411150in}{3.424480in}}%
\pgfpathcurveto{\pgfqpoint{2.405326in}{3.424480in}}{\pgfqpoint{2.399740in}{3.422166in}}{\pgfqpoint{2.395622in}{3.418048in}}%
\pgfpathcurveto{\pgfqpoint{2.391504in}{3.413929in}}{\pgfqpoint{2.389190in}{3.408343in}}{\pgfqpoint{2.389190in}{3.402519in}}%
\pgfpathcurveto{\pgfqpoint{2.389190in}{3.396695in}}{\pgfqpoint{2.391504in}{3.391109in}}{\pgfqpoint{2.395622in}{3.386991in}}%
\pgfpathcurveto{\pgfqpoint{2.399740in}{3.382873in}}{\pgfqpoint{2.405326in}{3.380559in}}{\pgfqpoint{2.411150in}{3.380559in}}%
\pgfpathlineto{\pgfqpoint{2.411150in}{3.380559in}}%
\pgfpathclose%
\pgfusepath{stroke,fill}%
\end{pgfscope}%
\begin{pgfscope}%
\pgfpathrectangle{\pgfqpoint{0.100000in}{0.183744in}}{\pgfqpoint{4.506048in}{4.506048in}}%
\pgfusepath{clip}%
\pgfsetbuttcap%
\pgfsetroundjoin%
\definecolor{currentfill}{rgb}{1.000000,0.647059,0.000000}%
\pgfsetfillcolor{currentfill}%
\pgfsetfillopacity{0.700000}%
\pgfsetlinewidth{1.003750pt}%
\definecolor{currentstroke}{rgb}{1.000000,0.647059,0.000000}%
\pgfsetstrokecolor{currentstroke}%
\pgfsetstrokeopacity{0.700000}%
\pgfsetdash{}{0pt}%
\pgfpathmoveto{\pgfqpoint{1.340852in}{2.383619in}}%
\pgfpathcurveto{\pgfqpoint{1.346676in}{2.383619in}}{\pgfqpoint{1.352263in}{2.385933in}}{\pgfqpoint{1.356381in}{2.390051in}}%
\pgfpathcurveto{\pgfqpoint{1.360499in}{2.394169in}}{\pgfqpoint{1.362813in}{2.399755in}}{\pgfqpoint{1.362813in}{2.405579in}}%
\pgfpathcurveto{\pgfqpoint{1.362813in}{2.411403in}}{\pgfqpoint{1.360499in}{2.416990in}}{\pgfqpoint{1.356381in}{2.421108in}}%
\pgfpathcurveto{\pgfqpoint{1.352263in}{2.425226in}}{\pgfqpoint{1.346676in}{2.427540in}}{\pgfqpoint{1.340852in}{2.427540in}}%
\pgfpathcurveto{\pgfqpoint{1.335029in}{2.427540in}}{\pgfqpoint{1.329442in}{2.425226in}}{\pgfqpoint{1.325324in}{2.421108in}}%
\pgfpathcurveto{\pgfqpoint{1.321206in}{2.416990in}}{\pgfqpoint{1.318892in}{2.411403in}}{\pgfqpoint{1.318892in}{2.405579in}}%
\pgfpathcurveto{\pgfqpoint{1.318892in}{2.399755in}}{\pgfqpoint{1.321206in}{2.394169in}}{\pgfqpoint{1.325324in}{2.390051in}}%
\pgfpathcurveto{\pgfqpoint{1.329442in}{2.385933in}}{\pgfqpoint{1.335029in}{2.383619in}}{\pgfqpoint{1.340852in}{2.383619in}}%
\pgfpathlineto{\pgfqpoint{1.340852in}{2.383619in}}%
\pgfpathclose%
\pgfusepath{stroke,fill}%
\end{pgfscope}%
\begin{pgfscope}%
\pgfpathrectangle{\pgfqpoint{0.100000in}{0.183744in}}{\pgfqpoint{4.506048in}{4.506048in}}%
\pgfusepath{clip}%
\pgfsetbuttcap%
\pgfsetroundjoin%
\definecolor{currentfill}{rgb}{1.000000,0.647059,0.000000}%
\pgfsetfillcolor{currentfill}%
\pgfsetfillopacity{0.700000}%
\pgfsetlinewidth{1.003750pt}%
\definecolor{currentstroke}{rgb}{1.000000,0.647059,0.000000}%
\pgfsetstrokecolor{currentstroke}%
\pgfsetstrokeopacity{0.700000}%
\pgfsetdash{}{0pt}%
\pgfpathmoveto{\pgfqpoint{2.827569in}{3.082343in}}%
\pgfpathcurveto{\pgfqpoint{2.833393in}{3.082343in}}{\pgfqpoint{2.838979in}{3.084657in}}{\pgfqpoint{2.843097in}{3.088775in}}%
\pgfpathcurveto{\pgfqpoint{2.847216in}{3.092893in}}{\pgfqpoint{2.849529in}{3.098479in}}{\pgfqpoint{2.849529in}{3.104303in}}%
\pgfpathcurveto{\pgfqpoint{2.849529in}{3.110127in}}{\pgfqpoint{2.847216in}{3.115713in}}{\pgfqpoint{2.843097in}{3.119831in}}%
\pgfpathcurveto{\pgfqpoint{2.838979in}{3.123949in}}{\pgfqpoint{2.833393in}{3.126263in}}{\pgfqpoint{2.827569in}{3.126263in}}%
\pgfpathcurveto{\pgfqpoint{2.821745in}{3.126263in}}{\pgfqpoint{2.816159in}{3.123949in}}{\pgfqpoint{2.812041in}{3.119831in}}%
\pgfpathcurveto{\pgfqpoint{2.807923in}{3.115713in}}{\pgfqpoint{2.805609in}{3.110127in}}{\pgfqpoint{2.805609in}{3.104303in}}%
\pgfpathcurveto{\pgfqpoint{2.805609in}{3.098479in}}{\pgfqpoint{2.807923in}{3.092893in}}{\pgfqpoint{2.812041in}{3.088775in}}%
\pgfpathcurveto{\pgfqpoint{2.816159in}{3.084657in}}{\pgfqpoint{2.821745in}{3.082343in}}{\pgfqpoint{2.827569in}{3.082343in}}%
\pgfpathlineto{\pgfqpoint{2.827569in}{3.082343in}}%
\pgfpathclose%
\pgfusepath{stroke,fill}%
\end{pgfscope}%
\begin{pgfscope}%
\pgfpathrectangle{\pgfqpoint{0.100000in}{0.183744in}}{\pgfqpoint{4.506048in}{4.506048in}}%
\pgfusepath{clip}%
\pgfsetbuttcap%
\pgfsetroundjoin%
\definecolor{currentfill}{rgb}{1.000000,0.647059,0.000000}%
\pgfsetfillcolor{currentfill}%
\pgfsetfillopacity{0.700000}%
\pgfsetlinewidth{1.003750pt}%
\definecolor{currentstroke}{rgb}{1.000000,0.647059,0.000000}%
\pgfsetstrokecolor{currentstroke}%
\pgfsetstrokeopacity{0.700000}%
\pgfsetdash{}{0pt}%
\pgfpathmoveto{\pgfqpoint{3.154023in}{2.936563in}}%
\pgfpathcurveto{\pgfqpoint{3.159847in}{2.936563in}}{\pgfqpoint{3.165433in}{2.938877in}}{\pgfqpoint{3.169551in}{2.942995in}}%
\pgfpathcurveto{\pgfqpoint{3.173669in}{2.947113in}}{\pgfqpoint{3.175983in}{2.952699in}}{\pgfqpoint{3.175983in}{2.958523in}}%
\pgfpathcurveto{\pgfqpoint{3.175983in}{2.964347in}}{\pgfqpoint{3.173669in}{2.969933in}}{\pgfqpoint{3.169551in}{2.974051in}}%
\pgfpathcurveto{\pgfqpoint{3.165433in}{2.978170in}}{\pgfqpoint{3.159847in}{2.980483in}}{\pgfqpoint{3.154023in}{2.980483in}}%
\pgfpathcurveto{\pgfqpoint{3.148199in}{2.980483in}}{\pgfqpoint{3.142613in}{2.978170in}}{\pgfqpoint{3.138495in}{2.974051in}}%
\pgfpathcurveto{\pgfqpoint{3.134377in}{2.969933in}}{\pgfqpoint{3.132063in}{2.964347in}}{\pgfqpoint{3.132063in}{2.958523in}}%
\pgfpathcurveto{\pgfqpoint{3.132063in}{2.952699in}}{\pgfqpoint{3.134377in}{2.947113in}}{\pgfqpoint{3.138495in}{2.942995in}}%
\pgfpathcurveto{\pgfqpoint{3.142613in}{2.938877in}}{\pgfqpoint{3.148199in}{2.936563in}}{\pgfqpoint{3.154023in}{2.936563in}}%
\pgfpathlineto{\pgfqpoint{3.154023in}{2.936563in}}%
\pgfpathclose%
\pgfusepath{stroke,fill}%
\end{pgfscope}%
\begin{pgfscope}%
\pgfpathrectangle{\pgfqpoint{0.100000in}{0.183744in}}{\pgfqpoint{4.506048in}{4.506048in}}%
\pgfusepath{clip}%
\pgfsetbuttcap%
\pgfsetroundjoin%
\definecolor{currentfill}{rgb}{1.000000,0.647059,0.000000}%
\pgfsetfillcolor{currentfill}%
\pgfsetfillopacity{0.700000}%
\pgfsetlinewidth{1.003750pt}%
\definecolor{currentstroke}{rgb}{1.000000,0.647059,0.000000}%
\pgfsetstrokecolor{currentstroke}%
\pgfsetstrokeopacity{0.700000}%
\pgfsetdash{}{0pt}%
\pgfpathmoveto{\pgfqpoint{2.727137in}{2.596264in}}%
\pgfpathcurveto{\pgfqpoint{2.732961in}{2.596264in}}{\pgfqpoint{2.738547in}{2.598578in}}{\pgfqpoint{2.742665in}{2.602696in}}%
\pgfpathcurveto{\pgfqpoint{2.746783in}{2.606814in}}{\pgfqpoint{2.749097in}{2.612401in}}{\pgfqpoint{2.749097in}{2.618225in}}%
\pgfpathcurveto{\pgfqpoint{2.749097in}{2.624048in}}{\pgfqpoint{2.746783in}{2.629635in}}{\pgfqpoint{2.742665in}{2.633753in}}%
\pgfpathcurveto{\pgfqpoint{2.738547in}{2.637871in}}{\pgfqpoint{2.732961in}{2.640185in}}{\pgfqpoint{2.727137in}{2.640185in}}%
\pgfpathcurveto{\pgfqpoint{2.721313in}{2.640185in}}{\pgfqpoint{2.715727in}{2.637871in}}{\pgfqpoint{2.711609in}{2.633753in}}%
\pgfpathcurveto{\pgfqpoint{2.707490in}{2.629635in}}{\pgfqpoint{2.705177in}{2.624048in}}{\pgfqpoint{2.705177in}{2.618225in}}%
\pgfpathcurveto{\pgfqpoint{2.705177in}{2.612401in}}{\pgfqpoint{2.707490in}{2.606814in}}{\pgfqpoint{2.711609in}{2.602696in}}%
\pgfpathcurveto{\pgfqpoint{2.715727in}{2.598578in}}{\pgfqpoint{2.721313in}{2.596264in}}{\pgfqpoint{2.727137in}{2.596264in}}%
\pgfpathlineto{\pgfqpoint{2.727137in}{2.596264in}}%
\pgfpathclose%
\pgfusepath{stroke,fill}%
\end{pgfscope}%
\begin{pgfscope}%
\pgfpathrectangle{\pgfqpoint{0.100000in}{0.183744in}}{\pgfqpoint{4.506048in}{4.506048in}}%
\pgfusepath{clip}%
\pgfsetbuttcap%
\pgfsetroundjoin%
\definecolor{currentfill}{rgb}{1.000000,0.647059,0.000000}%
\pgfsetfillcolor{currentfill}%
\pgfsetfillopacity{0.700000}%
\pgfsetlinewidth{1.003750pt}%
\definecolor{currentstroke}{rgb}{1.000000,0.647059,0.000000}%
\pgfsetstrokecolor{currentstroke}%
\pgfsetstrokeopacity{0.700000}%
\pgfsetdash{}{0pt}%
\pgfpathmoveto{\pgfqpoint{3.084691in}{2.324165in}}%
\pgfpathcurveto{\pgfqpoint{3.090515in}{2.324165in}}{\pgfqpoint{3.096101in}{2.326479in}}{\pgfqpoint{3.100219in}{2.330597in}}%
\pgfpathcurveto{\pgfqpoint{3.104337in}{2.334716in}}{\pgfqpoint{3.106651in}{2.340302in}}{\pgfqpoint{3.106651in}{2.346126in}}%
\pgfpathcurveto{\pgfqpoint{3.106651in}{2.351950in}}{\pgfqpoint{3.104337in}{2.357536in}}{\pgfqpoint{3.100219in}{2.361654in}}%
\pgfpathcurveto{\pgfqpoint{3.096101in}{2.365772in}}{\pgfqpoint{3.090515in}{2.368086in}}{\pgfqpoint{3.084691in}{2.368086in}}%
\pgfpathcurveto{\pgfqpoint{3.078867in}{2.368086in}}{\pgfqpoint{3.073281in}{2.365772in}}{\pgfqpoint{3.069163in}{2.361654in}}%
\pgfpathcurveto{\pgfqpoint{3.065044in}{2.357536in}}{\pgfqpoint{3.062731in}{2.351950in}}{\pgfqpoint{3.062731in}{2.346126in}}%
\pgfpathcurveto{\pgfqpoint{3.062731in}{2.340302in}}{\pgfqpoint{3.065044in}{2.334716in}}{\pgfqpoint{3.069163in}{2.330597in}}%
\pgfpathcurveto{\pgfqpoint{3.073281in}{2.326479in}}{\pgfqpoint{3.078867in}{2.324165in}}{\pgfqpoint{3.084691in}{2.324165in}}%
\pgfpathlineto{\pgfqpoint{3.084691in}{2.324165in}}%
\pgfpathclose%
\pgfusepath{stroke,fill}%
\end{pgfscope}%
\begin{pgfscope}%
\pgfpathrectangle{\pgfqpoint{0.100000in}{0.183744in}}{\pgfqpoint{4.506048in}{4.506048in}}%
\pgfusepath{clip}%
\pgfsetbuttcap%
\pgfsetroundjoin%
\definecolor{currentfill}{rgb}{1.000000,0.647059,0.000000}%
\pgfsetfillcolor{currentfill}%
\pgfsetfillopacity{0.700000}%
\pgfsetlinewidth{1.003750pt}%
\definecolor{currentstroke}{rgb}{1.000000,0.647059,0.000000}%
\pgfsetstrokecolor{currentstroke}%
\pgfsetstrokeopacity{0.700000}%
\pgfsetdash{}{0pt}%
\pgfpathmoveto{\pgfqpoint{3.382408in}{3.022373in}}%
\pgfpathcurveto{\pgfqpoint{3.388232in}{3.022373in}}{\pgfqpoint{3.393818in}{3.024687in}}{\pgfqpoint{3.397936in}{3.028805in}}%
\pgfpathcurveto{\pgfqpoint{3.402054in}{3.032923in}}{\pgfqpoint{3.404368in}{3.038509in}}{\pgfqpoint{3.404368in}{3.044333in}}%
\pgfpathcurveto{\pgfqpoint{3.404368in}{3.050157in}}{\pgfqpoint{3.402054in}{3.055743in}}{\pgfqpoint{3.397936in}{3.059862in}}%
\pgfpathcurveto{\pgfqpoint{3.393818in}{3.063980in}}{\pgfqpoint{3.388232in}{3.066294in}}{\pgfqpoint{3.382408in}{3.066294in}}%
\pgfpathcurveto{\pgfqpoint{3.376584in}{3.066294in}}{\pgfqpoint{3.370998in}{3.063980in}}{\pgfqpoint{3.366879in}{3.059862in}}%
\pgfpathcurveto{\pgfqpoint{3.362761in}{3.055743in}}{\pgfqpoint{3.360447in}{3.050157in}}{\pgfqpoint{3.360447in}{3.044333in}}%
\pgfpathcurveto{\pgfqpoint{3.360447in}{3.038509in}}{\pgfqpoint{3.362761in}{3.032923in}}{\pgfqpoint{3.366879in}{3.028805in}}%
\pgfpathcurveto{\pgfqpoint{3.370998in}{3.024687in}}{\pgfqpoint{3.376584in}{3.022373in}}{\pgfqpoint{3.382408in}{3.022373in}}%
\pgfpathlineto{\pgfqpoint{3.382408in}{3.022373in}}%
\pgfpathclose%
\pgfusepath{stroke,fill}%
\end{pgfscope}%
\begin{pgfscope}%
\pgfpathrectangle{\pgfqpoint{0.100000in}{0.183744in}}{\pgfqpoint{4.506048in}{4.506048in}}%
\pgfusepath{clip}%
\pgfsetbuttcap%
\pgfsetroundjoin%
\definecolor{currentfill}{rgb}{1.000000,0.647059,0.000000}%
\pgfsetfillcolor{currentfill}%
\pgfsetfillopacity{0.700000}%
\pgfsetlinewidth{1.003750pt}%
\definecolor{currentstroke}{rgb}{1.000000,0.647059,0.000000}%
\pgfsetstrokecolor{currentstroke}%
\pgfsetstrokeopacity{0.700000}%
\pgfsetdash{}{0pt}%
\pgfpathmoveto{\pgfqpoint{2.270649in}{3.216391in}}%
\pgfpathcurveto{\pgfqpoint{2.276473in}{3.216391in}}{\pgfqpoint{2.282060in}{3.218705in}}{\pgfqpoint{2.286178in}{3.222823in}}%
\pgfpathcurveto{\pgfqpoint{2.290296in}{3.226941in}}{\pgfqpoint{2.292610in}{3.232527in}}{\pgfqpoint{2.292610in}{3.238351in}}%
\pgfpathcurveto{\pgfqpoint{2.292610in}{3.244175in}}{\pgfqpoint{2.290296in}{3.249761in}}{\pgfqpoint{2.286178in}{3.253879in}}%
\pgfpathcurveto{\pgfqpoint{2.282060in}{3.257997in}}{\pgfqpoint{2.276473in}{3.260311in}}{\pgfqpoint{2.270649in}{3.260311in}}%
\pgfpathcurveto{\pgfqpoint{2.264825in}{3.260311in}}{\pgfqpoint{2.259239in}{3.257997in}}{\pgfqpoint{2.255121in}{3.253879in}}%
\pgfpathcurveto{\pgfqpoint{2.251003in}{3.249761in}}{\pgfqpoint{2.248689in}{3.244175in}}{\pgfqpoint{2.248689in}{3.238351in}}%
\pgfpathcurveto{\pgfqpoint{2.248689in}{3.232527in}}{\pgfqpoint{2.251003in}{3.226941in}}{\pgfqpoint{2.255121in}{3.222823in}}%
\pgfpathcurveto{\pgfqpoint{2.259239in}{3.218705in}}{\pgfqpoint{2.264825in}{3.216391in}}{\pgfqpoint{2.270649in}{3.216391in}}%
\pgfpathlineto{\pgfqpoint{2.270649in}{3.216391in}}%
\pgfpathclose%
\pgfusepath{stroke,fill}%
\end{pgfscope}%
\begin{pgfscope}%
\pgfpathrectangle{\pgfqpoint{0.100000in}{0.183744in}}{\pgfqpoint{4.506048in}{4.506048in}}%
\pgfusepath{clip}%
\pgfsetbuttcap%
\pgfsetroundjoin%
\definecolor{currentfill}{rgb}{1.000000,0.647059,0.000000}%
\pgfsetfillcolor{currentfill}%
\pgfsetfillopacity{0.700000}%
\pgfsetlinewidth{1.003750pt}%
\definecolor{currentstroke}{rgb}{1.000000,0.647059,0.000000}%
\pgfsetstrokecolor{currentstroke}%
\pgfsetstrokeopacity{0.700000}%
\pgfsetdash{}{0pt}%
\pgfpathmoveto{\pgfqpoint{0.981144in}{3.161842in}}%
\pgfpathcurveto{\pgfqpoint{0.986968in}{3.161842in}}{\pgfqpoint{0.992555in}{3.164156in}}{\pgfqpoint{0.996673in}{3.168274in}}%
\pgfpathcurveto{\pgfqpoint{1.000791in}{3.172392in}}{\pgfqpoint{1.003105in}{3.177978in}}{\pgfqpoint{1.003105in}{3.183802in}}%
\pgfpathcurveto{\pgfqpoint{1.003105in}{3.189626in}}{\pgfqpoint{1.000791in}{3.195212in}}{\pgfqpoint{0.996673in}{3.199330in}}%
\pgfpathcurveto{\pgfqpoint{0.992555in}{3.203449in}}{\pgfqpoint{0.986968in}{3.205762in}}{\pgfqpoint{0.981144in}{3.205762in}}%
\pgfpathcurveto{\pgfqpoint{0.975321in}{3.205762in}}{\pgfqpoint{0.969734in}{3.203449in}}{\pgfqpoint{0.965616in}{3.199330in}}%
\pgfpathcurveto{\pgfqpoint{0.961498in}{3.195212in}}{\pgfqpoint{0.959184in}{3.189626in}}{\pgfqpoint{0.959184in}{3.183802in}}%
\pgfpathcurveto{\pgfqpoint{0.959184in}{3.177978in}}{\pgfqpoint{0.961498in}{3.172392in}}{\pgfqpoint{0.965616in}{3.168274in}}%
\pgfpathcurveto{\pgfqpoint{0.969734in}{3.164156in}}{\pgfqpoint{0.975321in}{3.161842in}}{\pgfqpoint{0.981144in}{3.161842in}}%
\pgfpathlineto{\pgfqpoint{0.981144in}{3.161842in}}%
\pgfpathclose%
\pgfusepath{stroke,fill}%
\end{pgfscope}%
\begin{pgfscope}%
\pgfpathrectangle{\pgfqpoint{0.100000in}{0.183744in}}{\pgfqpoint{4.506048in}{4.506048in}}%
\pgfusepath{clip}%
\pgfsetbuttcap%
\pgfsetroundjoin%
\definecolor{currentfill}{rgb}{1.000000,0.647059,0.000000}%
\pgfsetfillcolor{currentfill}%
\pgfsetfillopacity{0.700000}%
\pgfsetlinewidth{1.003750pt}%
\definecolor{currentstroke}{rgb}{1.000000,0.647059,0.000000}%
\pgfsetstrokecolor{currentstroke}%
\pgfsetstrokeopacity{0.700000}%
\pgfsetdash{}{0pt}%
\pgfpathmoveto{\pgfqpoint{3.259231in}{2.126441in}}%
\pgfpathcurveto{\pgfqpoint{3.265055in}{2.126441in}}{\pgfqpoint{3.270641in}{2.128755in}}{\pgfqpoint{3.274759in}{2.132873in}}%
\pgfpathcurveto{\pgfqpoint{3.278877in}{2.136991in}}{\pgfqpoint{3.281191in}{2.142577in}}{\pgfqpoint{3.281191in}{2.148401in}}%
\pgfpathcurveto{\pgfqpoint{3.281191in}{2.154225in}}{\pgfqpoint{3.278877in}{2.159811in}}{\pgfqpoint{3.274759in}{2.163929in}}%
\pgfpathcurveto{\pgfqpoint{3.270641in}{2.168047in}}{\pgfqpoint{3.265055in}{2.170361in}}{\pgfqpoint{3.259231in}{2.170361in}}%
\pgfpathcurveto{\pgfqpoint{3.253407in}{2.170361in}}{\pgfqpoint{3.247821in}{2.168047in}}{\pgfqpoint{3.243703in}{2.163929in}}%
\pgfpathcurveto{\pgfqpoint{3.239584in}{2.159811in}}{\pgfqpoint{3.237271in}{2.154225in}}{\pgfqpoint{3.237271in}{2.148401in}}%
\pgfpathcurveto{\pgfqpoint{3.237271in}{2.142577in}}{\pgfqpoint{3.239584in}{2.136991in}}{\pgfqpoint{3.243703in}{2.132873in}}%
\pgfpathcurveto{\pgfqpoint{3.247821in}{2.128755in}}{\pgfqpoint{3.253407in}{2.126441in}}{\pgfqpoint{3.259231in}{2.126441in}}%
\pgfpathlineto{\pgfqpoint{3.259231in}{2.126441in}}%
\pgfpathclose%
\pgfusepath{stroke,fill}%
\end{pgfscope}%
\begin{pgfscope}%
\pgfpathrectangle{\pgfqpoint{0.100000in}{0.183744in}}{\pgfqpoint{4.506048in}{4.506048in}}%
\pgfusepath{clip}%
\pgfsetbuttcap%
\pgfsetroundjoin%
\definecolor{currentfill}{rgb}{1.000000,0.647059,0.000000}%
\pgfsetfillcolor{currentfill}%
\pgfsetfillopacity{0.700000}%
\pgfsetlinewidth{1.003750pt}%
\definecolor{currentstroke}{rgb}{1.000000,0.647059,0.000000}%
\pgfsetstrokecolor{currentstroke}%
\pgfsetstrokeopacity{0.700000}%
\pgfsetdash{}{0pt}%
\pgfpathmoveto{\pgfqpoint{1.315557in}{1.761470in}}%
\pgfpathcurveto{\pgfqpoint{1.321381in}{1.761470in}}{\pgfqpoint{1.326967in}{1.763784in}}{\pgfqpoint{1.331086in}{1.767902in}}%
\pgfpathcurveto{\pgfqpoint{1.335204in}{1.772021in}}{\pgfqpoint{1.337518in}{1.777607in}}{\pgfqpoint{1.337518in}{1.783431in}}%
\pgfpathcurveto{\pgfqpoint{1.337518in}{1.789255in}}{\pgfqpoint{1.335204in}{1.794841in}}{\pgfqpoint{1.331086in}{1.798959in}}%
\pgfpathcurveto{\pgfqpoint{1.326967in}{1.803077in}}{\pgfqpoint{1.321381in}{1.805391in}}{\pgfqpoint{1.315557in}{1.805391in}}%
\pgfpathcurveto{\pgfqpoint{1.309733in}{1.805391in}}{\pgfqpoint{1.304147in}{1.803077in}}{\pgfqpoint{1.300029in}{1.798959in}}%
\pgfpathcurveto{\pgfqpoint{1.295911in}{1.794841in}}{\pgfqpoint{1.293597in}{1.789255in}}{\pgfqpoint{1.293597in}{1.783431in}}%
\pgfpathcurveto{\pgfqpoint{1.293597in}{1.777607in}}{\pgfqpoint{1.295911in}{1.772021in}}{\pgfqpoint{1.300029in}{1.767902in}}%
\pgfpathcurveto{\pgfqpoint{1.304147in}{1.763784in}}{\pgfqpoint{1.309733in}{1.761470in}}{\pgfqpoint{1.315557in}{1.761470in}}%
\pgfpathlineto{\pgfqpoint{1.315557in}{1.761470in}}%
\pgfpathclose%
\pgfusepath{stroke,fill}%
\end{pgfscope}%
\begin{pgfscope}%
\pgfpathrectangle{\pgfqpoint{0.100000in}{0.183744in}}{\pgfqpoint{4.506048in}{4.506048in}}%
\pgfusepath{clip}%
\pgfsetbuttcap%
\pgfsetroundjoin%
\definecolor{currentfill}{rgb}{1.000000,0.647059,0.000000}%
\pgfsetfillcolor{currentfill}%
\pgfsetfillopacity{0.700000}%
\pgfsetlinewidth{1.003750pt}%
\definecolor{currentstroke}{rgb}{1.000000,0.647059,0.000000}%
\pgfsetstrokecolor{currentstroke}%
\pgfsetstrokeopacity{0.700000}%
\pgfsetdash{}{0pt}%
\pgfpathmoveto{\pgfqpoint{1.608338in}{2.329782in}}%
\pgfpathcurveto{\pgfqpoint{1.614162in}{2.329782in}}{\pgfqpoint{1.619748in}{2.332096in}}{\pgfqpoint{1.623866in}{2.336214in}}%
\pgfpathcurveto{\pgfqpoint{1.627984in}{2.340332in}}{\pgfqpoint{1.630298in}{2.345919in}}{\pgfqpoint{1.630298in}{2.351743in}}%
\pgfpathcurveto{\pgfqpoint{1.630298in}{2.357567in}}{\pgfqpoint{1.627984in}{2.363153in}}{\pgfqpoint{1.623866in}{2.367271in}}%
\pgfpathcurveto{\pgfqpoint{1.619748in}{2.371389in}}{\pgfqpoint{1.614162in}{2.373703in}}{\pgfqpoint{1.608338in}{2.373703in}}%
\pgfpathcurveto{\pgfqpoint{1.602514in}{2.373703in}}{\pgfqpoint{1.596928in}{2.371389in}}{\pgfqpoint{1.592810in}{2.367271in}}%
\pgfpathcurveto{\pgfqpoint{1.588692in}{2.363153in}}{\pgfqpoint{1.586378in}{2.357567in}}{\pgfqpoint{1.586378in}{2.351743in}}%
\pgfpathcurveto{\pgfqpoint{1.586378in}{2.345919in}}{\pgfqpoint{1.588692in}{2.340332in}}{\pgfqpoint{1.592810in}{2.336214in}}%
\pgfpathcurveto{\pgfqpoint{1.596928in}{2.332096in}}{\pgfqpoint{1.602514in}{2.329782in}}{\pgfqpoint{1.608338in}{2.329782in}}%
\pgfpathlineto{\pgfqpoint{1.608338in}{2.329782in}}%
\pgfpathclose%
\pgfusepath{stroke,fill}%
\end{pgfscope}%
\begin{pgfscope}%
\pgfpathrectangle{\pgfqpoint{0.100000in}{0.183744in}}{\pgfqpoint{4.506048in}{4.506048in}}%
\pgfusepath{clip}%
\pgfsetbuttcap%
\pgfsetroundjoin%
\definecolor{currentfill}{rgb}{1.000000,0.647059,0.000000}%
\pgfsetfillcolor{currentfill}%
\pgfsetfillopacity{0.700000}%
\pgfsetlinewidth{1.003750pt}%
\definecolor{currentstroke}{rgb}{1.000000,0.647059,0.000000}%
\pgfsetstrokecolor{currentstroke}%
\pgfsetstrokeopacity{0.700000}%
\pgfsetdash{}{0pt}%
\pgfpathmoveto{\pgfqpoint{3.096405in}{2.200657in}}%
\pgfpathcurveto{\pgfqpoint{3.102229in}{2.200657in}}{\pgfqpoint{3.107815in}{2.202970in}}{\pgfqpoint{3.111933in}{2.207089in}}%
\pgfpathcurveto{\pgfqpoint{3.116051in}{2.211207in}}{\pgfqpoint{3.118365in}{2.216793in}}{\pgfqpoint{3.118365in}{2.222617in}}%
\pgfpathcurveto{\pgfqpoint{3.118365in}{2.228441in}}{\pgfqpoint{3.116051in}{2.234027in}}{\pgfqpoint{3.111933in}{2.238145in}}%
\pgfpathcurveto{\pgfqpoint{3.107815in}{2.242263in}}{\pgfqpoint{3.102229in}{2.244577in}}{\pgfqpoint{3.096405in}{2.244577in}}%
\pgfpathcurveto{\pgfqpoint{3.090581in}{2.244577in}}{\pgfqpoint{3.084995in}{2.242263in}}{\pgfqpoint{3.080877in}{2.238145in}}%
\pgfpathcurveto{\pgfqpoint{3.076759in}{2.234027in}}{\pgfqpoint{3.074445in}{2.228441in}}{\pgfqpoint{3.074445in}{2.222617in}}%
\pgfpathcurveto{\pgfqpoint{3.074445in}{2.216793in}}{\pgfqpoint{3.076759in}{2.211207in}}{\pgfqpoint{3.080877in}{2.207089in}}%
\pgfpathcurveto{\pgfqpoint{3.084995in}{2.202970in}}{\pgfqpoint{3.090581in}{2.200657in}}{\pgfqpoint{3.096405in}{2.200657in}}%
\pgfpathlineto{\pgfqpoint{3.096405in}{2.200657in}}%
\pgfpathclose%
\pgfusepath{stroke,fill}%
\end{pgfscope}%
\begin{pgfscope}%
\pgfpathrectangle{\pgfqpoint{0.100000in}{0.183744in}}{\pgfqpoint{4.506048in}{4.506048in}}%
\pgfusepath{clip}%
\pgfsetbuttcap%
\pgfsetroundjoin%
\definecolor{currentfill}{rgb}{1.000000,0.647059,0.000000}%
\pgfsetfillcolor{currentfill}%
\pgfsetfillopacity{0.700000}%
\pgfsetlinewidth{1.003750pt}%
\definecolor{currentstroke}{rgb}{1.000000,0.647059,0.000000}%
\pgfsetstrokecolor{currentstroke}%
\pgfsetstrokeopacity{0.700000}%
\pgfsetdash{}{0pt}%
\pgfpathmoveto{\pgfqpoint{2.038220in}{1.648106in}}%
\pgfpathcurveto{\pgfqpoint{2.044043in}{1.648106in}}{\pgfqpoint{2.049630in}{1.650420in}}{\pgfqpoint{2.053748in}{1.654538in}}%
\pgfpathcurveto{\pgfqpoint{2.057866in}{1.658657in}}{\pgfqpoint{2.060180in}{1.664243in}}{\pgfqpoint{2.060180in}{1.670067in}}%
\pgfpathcurveto{\pgfqpoint{2.060180in}{1.675891in}}{\pgfqpoint{2.057866in}{1.681477in}}{\pgfqpoint{2.053748in}{1.685595in}}%
\pgfpathcurveto{\pgfqpoint{2.049630in}{1.689713in}}{\pgfqpoint{2.044043in}{1.692027in}}{\pgfqpoint{2.038220in}{1.692027in}}%
\pgfpathcurveto{\pgfqpoint{2.032396in}{1.692027in}}{\pgfqpoint{2.026809in}{1.689713in}}{\pgfqpoint{2.022691in}{1.685595in}}%
\pgfpathcurveto{\pgfqpoint{2.018573in}{1.681477in}}{\pgfqpoint{2.016259in}{1.675891in}}{\pgfqpoint{2.016259in}{1.670067in}}%
\pgfpathcurveto{\pgfqpoint{2.016259in}{1.664243in}}{\pgfqpoint{2.018573in}{1.658657in}}{\pgfqpoint{2.022691in}{1.654538in}}%
\pgfpathcurveto{\pgfqpoint{2.026809in}{1.650420in}}{\pgfqpoint{2.032396in}{1.648106in}}{\pgfqpoint{2.038220in}{1.648106in}}%
\pgfpathlineto{\pgfqpoint{2.038220in}{1.648106in}}%
\pgfpathclose%
\pgfusepath{stroke,fill}%
\end{pgfscope}%
\begin{pgfscope}%
\pgfpathrectangle{\pgfqpoint{0.100000in}{0.183744in}}{\pgfqpoint{4.506048in}{4.506048in}}%
\pgfusepath{clip}%
\pgfsetbuttcap%
\pgfsetroundjoin%
\definecolor{currentfill}{rgb}{1.000000,0.647059,0.000000}%
\pgfsetfillcolor{currentfill}%
\pgfsetfillopacity{0.700000}%
\pgfsetlinewidth{1.003750pt}%
\definecolor{currentstroke}{rgb}{1.000000,0.647059,0.000000}%
\pgfsetstrokecolor{currentstroke}%
\pgfsetstrokeopacity{0.700000}%
\pgfsetdash{}{0pt}%
\pgfpathmoveto{\pgfqpoint{3.257569in}{2.516585in}}%
\pgfpathcurveto{\pgfqpoint{3.263393in}{2.516585in}}{\pgfqpoint{3.268979in}{2.518899in}}{\pgfqpoint{3.273097in}{2.523017in}}%
\pgfpathcurveto{\pgfqpoint{3.277215in}{2.527135in}}{\pgfqpoint{3.279529in}{2.532722in}}{\pgfqpoint{3.279529in}{2.538545in}}%
\pgfpathcurveto{\pgfqpoint{3.279529in}{2.544369in}}{\pgfqpoint{3.277215in}{2.549956in}}{\pgfqpoint{3.273097in}{2.554074in}}%
\pgfpathcurveto{\pgfqpoint{3.268979in}{2.558192in}}{\pgfqpoint{3.263393in}{2.560506in}}{\pgfqpoint{3.257569in}{2.560506in}}%
\pgfpathcurveto{\pgfqpoint{3.251745in}{2.560506in}}{\pgfqpoint{3.246158in}{2.558192in}}{\pgfqpoint{3.242040in}{2.554074in}}%
\pgfpathcurveto{\pgfqpoint{3.237922in}{2.549956in}}{\pgfqpoint{3.235608in}{2.544369in}}{\pgfqpoint{3.235608in}{2.538545in}}%
\pgfpathcurveto{\pgfqpoint{3.235608in}{2.532722in}}{\pgfqpoint{3.237922in}{2.527135in}}{\pgfqpoint{3.242040in}{2.523017in}}%
\pgfpathcurveto{\pgfqpoint{3.246158in}{2.518899in}}{\pgfqpoint{3.251745in}{2.516585in}}{\pgfqpoint{3.257569in}{2.516585in}}%
\pgfpathlineto{\pgfqpoint{3.257569in}{2.516585in}}%
\pgfpathclose%
\pgfusepath{stroke,fill}%
\end{pgfscope}%
\begin{pgfscope}%
\pgfpathrectangle{\pgfqpoint{0.100000in}{0.183744in}}{\pgfqpoint{4.506048in}{4.506048in}}%
\pgfusepath{clip}%
\pgfsetbuttcap%
\pgfsetroundjoin%
\definecolor{currentfill}{rgb}{1.000000,0.647059,0.000000}%
\pgfsetfillcolor{currentfill}%
\pgfsetfillopacity{0.700000}%
\pgfsetlinewidth{1.003750pt}%
\definecolor{currentstroke}{rgb}{1.000000,0.647059,0.000000}%
\pgfsetstrokecolor{currentstroke}%
\pgfsetstrokeopacity{0.700000}%
\pgfsetdash{}{0pt}%
\pgfpathmoveto{\pgfqpoint{2.178879in}{1.928228in}}%
\pgfpathcurveto{\pgfqpoint{2.184703in}{1.928228in}}{\pgfqpoint{2.190290in}{1.930542in}}{\pgfqpoint{2.194408in}{1.934660in}}%
\pgfpathcurveto{\pgfqpoint{2.198526in}{1.938778in}}{\pgfqpoint{2.200840in}{1.944364in}}{\pgfqpoint{2.200840in}{1.950188in}}%
\pgfpathcurveto{\pgfqpoint{2.200840in}{1.956012in}}{\pgfqpoint{2.198526in}{1.961599in}}{\pgfqpoint{2.194408in}{1.965717in}}%
\pgfpathcurveto{\pgfqpoint{2.190290in}{1.969835in}}{\pgfqpoint{2.184703in}{1.972149in}}{\pgfqpoint{2.178879in}{1.972149in}}%
\pgfpathcurveto{\pgfqpoint{2.173055in}{1.972149in}}{\pgfqpoint{2.167469in}{1.969835in}}{\pgfqpoint{2.163351in}{1.965717in}}%
\pgfpathcurveto{\pgfqpoint{2.159233in}{1.961599in}}{\pgfqpoint{2.156919in}{1.956012in}}{\pgfqpoint{2.156919in}{1.950188in}}%
\pgfpathcurveto{\pgfqpoint{2.156919in}{1.944364in}}{\pgfqpoint{2.159233in}{1.938778in}}{\pgfqpoint{2.163351in}{1.934660in}}%
\pgfpathcurveto{\pgfqpoint{2.167469in}{1.930542in}}{\pgfqpoint{2.173055in}{1.928228in}}{\pgfqpoint{2.178879in}{1.928228in}}%
\pgfpathlineto{\pgfqpoint{2.178879in}{1.928228in}}%
\pgfpathclose%
\pgfusepath{stroke,fill}%
\end{pgfscope}%
\begin{pgfscope}%
\pgfpathrectangle{\pgfqpoint{0.100000in}{0.183744in}}{\pgfqpoint{4.506048in}{4.506048in}}%
\pgfusepath{clip}%
\pgfsetbuttcap%
\pgfsetroundjoin%
\definecolor{currentfill}{rgb}{1.000000,0.647059,0.000000}%
\pgfsetfillcolor{currentfill}%
\pgfsetfillopacity{0.700000}%
\pgfsetlinewidth{1.003750pt}%
\definecolor{currentstroke}{rgb}{1.000000,0.647059,0.000000}%
\pgfsetstrokecolor{currentstroke}%
\pgfsetstrokeopacity{0.700000}%
\pgfsetdash{}{0pt}%
\pgfpathmoveto{\pgfqpoint{0.502625in}{3.281245in}}%
\pgfpathcurveto{\pgfqpoint{0.508448in}{3.281245in}}{\pgfqpoint{0.514035in}{3.283559in}}{\pgfqpoint{0.518153in}{3.287677in}}%
\pgfpathcurveto{\pgfqpoint{0.522271in}{3.291795in}}{\pgfqpoint{0.524585in}{3.297381in}}{\pgfqpoint{0.524585in}{3.303205in}}%
\pgfpathcurveto{\pgfqpoint{0.524585in}{3.309029in}}{\pgfqpoint{0.522271in}{3.314615in}}{\pgfqpoint{0.518153in}{3.318733in}}%
\pgfpathcurveto{\pgfqpoint{0.514035in}{3.322852in}}{\pgfqpoint{0.508448in}{3.325165in}}{\pgfqpoint{0.502625in}{3.325165in}}%
\pgfpathcurveto{\pgfqpoint{0.496801in}{3.325165in}}{\pgfqpoint{0.491214in}{3.322852in}}{\pgfqpoint{0.487096in}{3.318733in}}%
\pgfpathcurveto{\pgfqpoint{0.482978in}{3.314615in}}{\pgfqpoint{0.480664in}{3.309029in}}{\pgfqpoint{0.480664in}{3.303205in}}%
\pgfpathcurveto{\pgfqpoint{0.480664in}{3.297381in}}{\pgfqpoint{0.482978in}{3.291795in}}{\pgfqpoint{0.487096in}{3.287677in}}%
\pgfpathcurveto{\pgfqpoint{0.491214in}{3.283559in}}{\pgfqpoint{0.496801in}{3.281245in}}{\pgfqpoint{0.502625in}{3.281245in}}%
\pgfpathlineto{\pgfqpoint{0.502625in}{3.281245in}}%
\pgfpathclose%
\pgfusepath{stroke,fill}%
\end{pgfscope}%
\begin{pgfscope}%
\pgfpathrectangle{\pgfqpoint{0.100000in}{0.183744in}}{\pgfqpoint{4.506048in}{4.506048in}}%
\pgfusepath{clip}%
\pgfsetbuttcap%
\pgfsetroundjoin%
\definecolor{currentfill}{rgb}{1.000000,0.647059,0.000000}%
\pgfsetfillcolor{currentfill}%
\pgfsetfillopacity{0.700000}%
\pgfsetlinewidth{1.003750pt}%
\definecolor{currentstroke}{rgb}{1.000000,0.647059,0.000000}%
\pgfsetstrokecolor{currentstroke}%
\pgfsetstrokeopacity{0.700000}%
\pgfsetdash{}{0pt}%
\pgfpathmoveto{\pgfqpoint{3.339053in}{3.748429in}}%
\pgfpathcurveto{\pgfqpoint{3.344877in}{3.748429in}}{\pgfqpoint{3.350463in}{3.750743in}}{\pgfqpoint{3.354581in}{3.754861in}}%
\pgfpathcurveto{\pgfqpoint{3.358699in}{3.758980in}}{\pgfqpoint{3.361013in}{3.764566in}}{\pgfqpoint{3.361013in}{3.770390in}}%
\pgfpathcurveto{\pgfqpoint{3.361013in}{3.776214in}}{\pgfqpoint{3.358699in}{3.781800in}}{\pgfqpoint{3.354581in}{3.785918in}}%
\pgfpathcurveto{\pgfqpoint{3.350463in}{3.790036in}}{\pgfqpoint{3.344877in}{3.792350in}}{\pgfqpoint{3.339053in}{3.792350in}}%
\pgfpathcurveto{\pgfqpoint{3.333229in}{3.792350in}}{\pgfqpoint{3.327642in}{3.790036in}}{\pgfqpoint{3.323524in}{3.785918in}}%
\pgfpathcurveto{\pgfqpoint{3.319406in}{3.781800in}}{\pgfqpoint{3.317092in}{3.776214in}}{\pgfqpoint{3.317092in}{3.770390in}}%
\pgfpathcurveto{\pgfqpoint{3.317092in}{3.764566in}}{\pgfqpoint{3.319406in}{3.758980in}}{\pgfqpoint{3.323524in}{3.754861in}}%
\pgfpathcurveto{\pgfqpoint{3.327642in}{3.750743in}}{\pgfqpoint{3.333229in}{3.748429in}}{\pgfqpoint{3.339053in}{3.748429in}}%
\pgfpathlineto{\pgfqpoint{3.339053in}{3.748429in}}%
\pgfpathclose%
\pgfusepath{stroke,fill}%
\end{pgfscope}%
\begin{pgfscope}%
\pgfpathrectangle{\pgfqpoint{0.100000in}{0.183744in}}{\pgfqpoint{4.506048in}{4.506048in}}%
\pgfusepath{clip}%
\pgfsetbuttcap%
\pgfsetroundjoin%
\definecolor{currentfill}{rgb}{1.000000,0.647059,0.000000}%
\pgfsetfillcolor{currentfill}%
\pgfsetfillopacity{0.700000}%
\pgfsetlinewidth{1.003750pt}%
\definecolor{currentstroke}{rgb}{1.000000,0.647059,0.000000}%
\pgfsetstrokecolor{currentstroke}%
\pgfsetstrokeopacity{0.700000}%
\pgfsetdash{}{0pt}%
\pgfpathmoveto{\pgfqpoint{3.069111in}{2.557941in}}%
\pgfpathcurveto{\pgfqpoint{3.074934in}{2.557941in}}{\pgfqpoint{3.080521in}{2.560255in}}{\pgfqpoint{3.084639in}{2.564373in}}%
\pgfpathcurveto{\pgfqpoint{3.088757in}{2.568491in}}{\pgfqpoint{3.091071in}{2.574077in}}{\pgfqpoint{3.091071in}{2.579901in}}%
\pgfpathcurveto{\pgfqpoint{3.091071in}{2.585725in}}{\pgfqpoint{3.088757in}{2.591311in}}{\pgfqpoint{3.084639in}{2.595430in}}%
\pgfpathcurveto{\pgfqpoint{3.080521in}{2.599548in}}{\pgfqpoint{3.074934in}{2.601862in}}{\pgfqpoint{3.069111in}{2.601862in}}%
\pgfpathcurveto{\pgfqpoint{3.063287in}{2.601862in}}{\pgfqpoint{3.057700in}{2.599548in}}{\pgfqpoint{3.053582in}{2.595430in}}%
\pgfpathcurveto{\pgfqpoint{3.049464in}{2.591311in}}{\pgfqpoint{3.047150in}{2.585725in}}{\pgfqpoint{3.047150in}{2.579901in}}%
\pgfpathcurveto{\pgfqpoint{3.047150in}{2.574077in}}{\pgfqpoint{3.049464in}{2.568491in}}{\pgfqpoint{3.053582in}{2.564373in}}%
\pgfpathcurveto{\pgfqpoint{3.057700in}{2.560255in}}{\pgfqpoint{3.063287in}{2.557941in}}{\pgfqpoint{3.069111in}{2.557941in}}%
\pgfpathlineto{\pgfqpoint{3.069111in}{2.557941in}}%
\pgfpathclose%
\pgfusepath{stroke,fill}%
\end{pgfscope}%
\begin{pgfscope}%
\pgfpathrectangle{\pgfqpoint{0.100000in}{0.183744in}}{\pgfqpoint{4.506048in}{4.506048in}}%
\pgfusepath{clip}%
\pgfsetbuttcap%
\pgfsetroundjoin%
\definecolor{currentfill}{rgb}{1.000000,0.647059,0.000000}%
\pgfsetfillcolor{currentfill}%
\pgfsetfillopacity{0.700000}%
\pgfsetlinewidth{1.003750pt}%
\definecolor{currentstroke}{rgb}{1.000000,0.647059,0.000000}%
\pgfsetstrokecolor{currentstroke}%
\pgfsetstrokeopacity{0.700000}%
\pgfsetdash{}{0pt}%
\pgfpathmoveto{\pgfqpoint{2.163356in}{3.630324in}}%
\pgfpathcurveto{\pgfqpoint{2.169180in}{3.630324in}}{\pgfqpoint{2.174766in}{3.632638in}}{\pgfqpoint{2.178884in}{3.636756in}}%
\pgfpathcurveto{\pgfqpoint{2.183003in}{3.640875in}}{\pgfqpoint{2.185316in}{3.646461in}}{\pgfqpoint{2.185316in}{3.652285in}}%
\pgfpathcurveto{\pgfqpoint{2.185316in}{3.658109in}}{\pgfqpoint{2.183003in}{3.663695in}}{\pgfqpoint{2.178884in}{3.667813in}}%
\pgfpathcurveto{\pgfqpoint{2.174766in}{3.671931in}}{\pgfqpoint{2.169180in}{3.674245in}}{\pgfqpoint{2.163356in}{3.674245in}}%
\pgfpathcurveto{\pgfqpoint{2.157532in}{3.674245in}}{\pgfqpoint{2.151946in}{3.671931in}}{\pgfqpoint{2.147828in}{3.667813in}}%
\pgfpathcurveto{\pgfqpoint{2.143710in}{3.663695in}}{\pgfqpoint{2.141396in}{3.658109in}}{\pgfqpoint{2.141396in}{3.652285in}}%
\pgfpathcurveto{\pgfqpoint{2.141396in}{3.646461in}}{\pgfqpoint{2.143710in}{3.640875in}}{\pgfqpoint{2.147828in}{3.636756in}}%
\pgfpathcurveto{\pgfqpoint{2.151946in}{3.632638in}}{\pgfqpoint{2.157532in}{3.630324in}}{\pgfqpoint{2.163356in}{3.630324in}}%
\pgfpathlineto{\pgfqpoint{2.163356in}{3.630324in}}%
\pgfpathclose%
\pgfusepath{stroke,fill}%
\end{pgfscope}%
\begin{pgfscope}%
\pgfpathrectangle{\pgfqpoint{0.100000in}{0.183744in}}{\pgfqpoint{4.506048in}{4.506048in}}%
\pgfusepath{clip}%
\pgfsetbuttcap%
\pgfsetroundjoin%
\definecolor{currentfill}{rgb}{1.000000,0.647059,0.000000}%
\pgfsetfillcolor{currentfill}%
\pgfsetfillopacity{0.700000}%
\pgfsetlinewidth{1.003750pt}%
\definecolor{currentstroke}{rgb}{1.000000,0.647059,0.000000}%
\pgfsetstrokecolor{currentstroke}%
\pgfsetstrokeopacity{0.700000}%
\pgfsetdash{}{0pt}%
\pgfpathmoveto{\pgfqpoint{3.070392in}{2.616441in}}%
\pgfpathcurveto{\pgfqpoint{3.076216in}{2.616441in}}{\pgfqpoint{3.081802in}{2.618755in}}{\pgfqpoint{3.085920in}{2.622873in}}%
\pgfpathcurveto{\pgfqpoint{3.090038in}{2.626991in}}{\pgfqpoint{3.092352in}{2.632577in}}{\pgfqpoint{3.092352in}{2.638401in}}%
\pgfpathcurveto{\pgfqpoint{3.092352in}{2.644225in}}{\pgfqpoint{3.090038in}{2.649811in}}{\pgfqpoint{3.085920in}{2.653930in}}%
\pgfpathcurveto{\pgfqpoint{3.081802in}{2.658048in}}{\pgfqpoint{3.076216in}{2.660362in}}{\pgfqpoint{3.070392in}{2.660362in}}%
\pgfpathcurveto{\pgfqpoint{3.064568in}{2.660362in}}{\pgfqpoint{3.058982in}{2.658048in}}{\pgfqpoint{3.054864in}{2.653930in}}%
\pgfpathcurveto{\pgfqpoint{3.050746in}{2.649811in}}{\pgfqpoint{3.048432in}{2.644225in}}{\pgfqpoint{3.048432in}{2.638401in}}%
\pgfpathcurveto{\pgfqpoint{3.048432in}{2.632577in}}{\pgfqpoint{3.050746in}{2.626991in}}{\pgfqpoint{3.054864in}{2.622873in}}%
\pgfpathcurveto{\pgfqpoint{3.058982in}{2.618755in}}{\pgfqpoint{3.064568in}{2.616441in}}{\pgfqpoint{3.070392in}{2.616441in}}%
\pgfpathlineto{\pgfqpoint{3.070392in}{2.616441in}}%
\pgfpathclose%
\pgfusepath{stroke,fill}%
\end{pgfscope}%
\begin{pgfscope}%
\pgfpathrectangle{\pgfqpoint{0.100000in}{0.183744in}}{\pgfqpoint{4.506048in}{4.506048in}}%
\pgfusepath{clip}%
\pgfsetbuttcap%
\pgfsetroundjoin%
\definecolor{currentfill}{rgb}{1.000000,0.647059,0.000000}%
\pgfsetfillcolor{currentfill}%
\pgfsetfillopacity{0.700000}%
\pgfsetlinewidth{1.003750pt}%
\definecolor{currentstroke}{rgb}{1.000000,0.647059,0.000000}%
\pgfsetstrokecolor{currentstroke}%
\pgfsetstrokeopacity{0.700000}%
\pgfsetdash{}{0pt}%
\pgfpathmoveto{\pgfqpoint{2.900010in}{1.890572in}}%
\pgfpathcurveto{\pgfqpoint{2.905834in}{1.890572in}}{\pgfqpoint{2.911420in}{1.892886in}}{\pgfqpoint{2.915538in}{1.897004in}}%
\pgfpathcurveto{\pgfqpoint{2.919657in}{1.901122in}}{\pgfqpoint{2.921970in}{1.906708in}}{\pgfqpoint{2.921970in}{1.912532in}}%
\pgfpathcurveto{\pgfqpoint{2.921970in}{1.918356in}}{\pgfqpoint{2.919657in}{1.923942in}}{\pgfqpoint{2.915538in}{1.928061in}}%
\pgfpathcurveto{\pgfqpoint{2.911420in}{1.932179in}}{\pgfqpoint{2.905834in}{1.934493in}}{\pgfqpoint{2.900010in}{1.934493in}}%
\pgfpathcurveto{\pgfqpoint{2.894186in}{1.934493in}}{\pgfqpoint{2.888600in}{1.932179in}}{\pgfqpoint{2.884482in}{1.928061in}}%
\pgfpathcurveto{\pgfqpoint{2.880364in}{1.923942in}}{\pgfqpoint{2.878050in}{1.918356in}}{\pgfqpoint{2.878050in}{1.912532in}}%
\pgfpathcurveto{\pgfqpoint{2.878050in}{1.906708in}}{\pgfqpoint{2.880364in}{1.901122in}}{\pgfqpoint{2.884482in}{1.897004in}}%
\pgfpathcurveto{\pgfqpoint{2.888600in}{1.892886in}}{\pgfqpoint{2.894186in}{1.890572in}}{\pgfqpoint{2.900010in}{1.890572in}}%
\pgfpathlineto{\pgfqpoint{2.900010in}{1.890572in}}%
\pgfpathclose%
\pgfusepath{stroke,fill}%
\end{pgfscope}%
\begin{pgfscope}%
\pgfpathrectangle{\pgfqpoint{0.100000in}{0.183744in}}{\pgfqpoint{4.506048in}{4.506048in}}%
\pgfusepath{clip}%
\pgfsetbuttcap%
\pgfsetroundjoin%
\definecolor{currentfill}{rgb}{1.000000,0.647059,0.000000}%
\pgfsetfillcolor{currentfill}%
\pgfsetfillopacity{0.700000}%
\pgfsetlinewidth{1.003750pt}%
\definecolor{currentstroke}{rgb}{1.000000,0.647059,0.000000}%
\pgfsetstrokecolor{currentstroke}%
\pgfsetstrokeopacity{0.700000}%
\pgfsetdash{}{0pt}%
\pgfpathmoveto{\pgfqpoint{2.951187in}{2.667905in}}%
\pgfpathcurveto{\pgfqpoint{2.957011in}{2.667905in}}{\pgfqpoint{2.962597in}{2.670219in}}{\pgfqpoint{2.966715in}{2.674337in}}%
\pgfpathcurveto{\pgfqpoint{2.970834in}{2.678455in}}{\pgfqpoint{2.973147in}{2.684041in}}{\pgfqpoint{2.973147in}{2.689865in}}%
\pgfpathcurveto{\pgfqpoint{2.973147in}{2.695689in}}{\pgfqpoint{2.970834in}{2.701275in}}{\pgfqpoint{2.966715in}{2.705393in}}%
\pgfpathcurveto{\pgfqpoint{2.962597in}{2.709511in}}{\pgfqpoint{2.957011in}{2.711825in}}{\pgfqpoint{2.951187in}{2.711825in}}%
\pgfpathcurveto{\pgfqpoint{2.945363in}{2.711825in}}{\pgfqpoint{2.939777in}{2.709511in}}{\pgfqpoint{2.935659in}{2.705393in}}%
\pgfpathcurveto{\pgfqpoint{2.931541in}{2.701275in}}{\pgfqpoint{2.929227in}{2.695689in}}{\pgfqpoint{2.929227in}{2.689865in}}%
\pgfpathcurveto{\pgfqpoint{2.929227in}{2.684041in}}{\pgfqpoint{2.931541in}{2.678455in}}{\pgfqpoint{2.935659in}{2.674337in}}%
\pgfpathcurveto{\pgfqpoint{2.939777in}{2.670219in}}{\pgfqpoint{2.945363in}{2.667905in}}{\pgfqpoint{2.951187in}{2.667905in}}%
\pgfpathlineto{\pgfqpoint{2.951187in}{2.667905in}}%
\pgfpathclose%
\pgfusepath{stroke,fill}%
\end{pgfscope}%
\begin{pgfscope}%
\pgfpathrectangle{\pgfqpoint{0.100000in}{0.183744in}}{\pgfqpoint{4.506048in}{4.506048in}}%
\pgfusepath{clip}%
\pgfsetbuttcap%
\pgfsetroundjoin%
\definecolor{currentfill}{rgb}{1.000000,0.647059,0.000000}%
\pgfsetfillcolor{currentfill}%
\pgfsetfillopacity{0.700000}%
\pgfsetlinewidth{1.003750pt}%
\definecolor{currentstroke}{rgb}{1.000000,0.647059,0.000000}%
\pgfsetstrokecolor{currentstroke}%
\pgfsetstrokeopacity{0.700000}%
\pgfsetdash{}{0pt}%
\pgfpathmoveto{\pgfqpoint{3.060184in}{3.061730in}}%
\pgfpathcurveto{\pgfqpoint{3.066008in}{3.061730in}}{\pgfqpoint{3.071594in}{3.064044in}}{\pgfqpoint{3.075712in}{3.068162in}}%
\pgfpathcurveto{\pgfqpoint{3.079831in}{3.072280in}}{\pgfqpoint{3.082144in}{3.077866in}}{\pgfqpoint{3.082144in}{3.083690in}}%
\pgfpathcurveto{\pgfqpoint{3.082144in}{3.089514in}}{\pgfqpoint{3.079831in}{3.095100in}}{\pgfqpoint{3.075712in}{3.099219in}}%
\pgfpathcurveto{\pgfqpoint{3.071594in}{3.103337in}}{\pgfqpoint{3.066008in}{3.105651in}}{\pgfqpoint{3.060184in}{3.105651in}}%
\pgfpathcurveto{\pgfqpoint{3.054360in}{3.105651in}}{\pgfqpoint{3.048774in}{3.103337in}}{\pgfqpoint{3.044656in}{3.099219in}}%
\pgfpathcurveto{\pgfqpoint{3.040538in}{3.095100in}}{\pgfqpoint{3.038224in}{3.089514in}}{\pgfqpoint{3.038224in}{3.083690in}}%
\pgfpathcurveto{\pgfqpoint{3.038224in}{3.077866in}}{\pgfqpoint{3.040538in}{3.072280in}}{\pgfqpoint{3.044656in}{3.068162in}}%
\pgfpathcurveto{\pgfqpoint{3.048774in}{3.064044in}}{\pgfqpoint{3.054360in}{3.061730in}}{\pgfqpoint{3.060184in}{3.061730in}}%
\pgfpathlineto{\pgfqpoint{3.060184in}{3.061730in}}%
\pgfpathclose%
\pgfusepath{stroke,fill}%
\end{pgfscope}%
\begin{pgfscope}%
\pgfpathrectangle{\pgfqpoint{0.100000in}{0.183744in}}{\pgfqpoint{4.506048in}{4.506048in}}%
\pgfusepath{clip}%
\pgfsetbuttcap%
\pgfsetroundjoin%
\definecolor{currentfill}{rgb}{1.000000,0.647059,0.000000}%
\pgfsetfillcolor{currentfill}%
\pgfsetfillopacity{0.700000}%
\pgfsetlinewidth{1.003750pt}%
\definecolor{currentstroke}{rgb}{1.000000,0.647059,0.000000}%
\pgfsetstrokecolor{currentstroke}%
\pgfsetstrokeopacity{0.700000}%
\pgfsetdash{}{0pt}%
\pgfpathmoveto{\pgfqpoint{3.500821in}{2.944729in}}%
\pgfpathcurveto{\pgfqpoint{3.506645in}{2.944729in}}{\pgfqpoint{3.512231in}{2.947043in}}{\pgfqpoint{3.516349in}{2.951161in}}%
\pgfpathcurveto{\pgfqpoint{3.520467in}{2.955279in}}{\pgfqpoint{3.522781in}{2.960865in}}{\pgfqpoint{3.522781in}{2.966689in}}%
\pgfpathcurveto{\pgfqpoint{3.522781in}{2.972513in}}{\pgfqpoint{3.520467in}{2.978099in}}{\pgfqpoint{3.516349in}{2.982218in}}%
\pgfpathcurveto{\pgfqpoint{3.512231in}{2.986336in}}{\pgfqpoint{3.506645in}{2.988650in}}{\pgfqpoint{3.500821in}{2.988650in}}%
\pgfpathcurveto{\pgfqpoint{3.494997in}{2.988650in}}{\pgfqpoint{3.489411in}{2.986336in}}{\pgfqpoint{3.485292in}{2.982218in}}%
\pgfpathcurveto{\pgfqpoint{3.481174in}{2.978099in}}{\pgfqpoint{3.478860in}{2.972513in}}{\pgfqpoint{3.478860in}{2.966689in}}%
\pgfpathcurveto{\pgfqpoint{3.478860in}{2.960865in}}{\pgfqpoint{3.481174in}{2.955279in}}{\pgfqpoint{3.485292in}{2.951161in}}%
\pgfpathcurveto{\pgfqpoint{3.489411in}{2.947043in}}{\pgfqpoint{3.494997in}{2.944729in}}{\pgfqpoint{3.500821in}{2.944729in}}%
\pgfpathlineto{\pgfqpoint{3.500821in}{2.944729in}}%
\pgfpathclose%
\pgfusepath{stroke,fill}%
\end{pgfscope}%
\begin{pgfscope}%
\pgfpathrectangle{\pgfqpoint{0.100000in}{0.183744in}}{\pgfqpoint{4.506048in}{4.506048in}}%
\pgfusepath{clip}%
\pgfsetbuttcap%
\pgfsetroundjoin%
\definecolor{currentfill}{rgb}{1.000000,0.647059,0.000000}%
\pgfsetfillcolor{currentfill}%
\pgfsetfillopacity{0.700000}%
\pgfsetlinewidth{1.003750pt}%
\definecolor{currentstroke}{rgb}{1.000000,0.647059,0.000000}%
\pgfsetstrokecolor{currentstroke}%
\pgfsetstrokeopacity{0.700000}%
\pgfsetdash{}{0pt}%
\pgfpathmoveto{\pgfqpoint{4.006394in}{2.075007in}}%
\pgfpathcurveto{\pgfqpoint{4.012218in}{2.075007in}}{\pgfqpoint{4.017805in}{2.077321in}}{\pgfqpoint{4.021923in}{2.081439in}}%
\pgfpathcurveto{\pgfqpoint{4.026041in}{2.085557in}}{\pgfqpoint{4.028355in}{2.091144in}}{\pgfqpoint{4.028355in}{2.096968in}}%
\pgfpathcurveto{\pgfqpoint{4.028355in}{2.102792in}}{\pgfqpoint{4.026041in}{2.108378in}}{\pgfqpoint{4.021923in}{2.112496in}}%
\pgfpathcurveto{\pgfqpoint{4.017805in}{2.116614in}}{\pgfqpoint{4.012218in}{2.118928in}}{\pgfqpoint{4.006394in}{2.118928in}}%
\pgfpathcurveto{\pgfqpoint{4.000571in}{2.118928in}}{\pgfqpoint{3.994984in}{2.116614in}}{\pgfqpoint{3.990866in}{2.112496in}}%
\pgfpathcurveto{\pgfqpoint{3.986748in}{2.108378in}}{\pgfqpoint{3.984434in}{2.102792in}}{\pgfqpoint{3.984434in}{2.096968in}}%
\pgfpathcurveto{\pgfqpoint{3.984434in}{2.091144in}}{\pgfqpoint{3.986748in}{2.085557in}}{\pgfqpoint{3.990866in}{2.081439in}}%
\pgfpathcurveto{\pgfqpoint{3.994984in}{2.077321in}}{\pgfqpoint{4.000571in}{2.075007in}}{\pgfqpoint{4.006394in}{2.075007in}}%
\pgfpathlineto{\pgfqpoint{4.006394in}{2.075007in}}%
\pgfpathclose%
\pgfusepath{stroke,fill}%
\end{pgfscope}%
\begin{pgfscope}%
\pgfpathrectangle{\pgfqpoint{0.100000in}{0.183744in}}{\pgfqpoint{4.506048in}{4.506048in}}%
\pgfusepath{clip}%
\pgfsetbuttcap%
\pgfsetroundjoin%
\definecolor{currentfill}{rgb}{1.000000,0.647059,0.000000}%
\pgfsetfillcolor{currentfill}%
\pgfsetfillopacity{0.700000}%
\pgfsetlinewidth{1.003750pt}%
\definecolor{currentstroke}{rgb}{1.000000,0.647059,0.000000}%
\pgfsetstrokecolor{currentstroke}%
\pgfsetstrokeopacity{0.700000}%
\pgfsetdash{}{0pt}%
\pgfpathmoveto{\pgfqpoint{1.841836in}{2.626760in}}%
\pgfpathcurveto{\pgfqpoint{1.847660in}{2.626760in}}{\pgfqpoint{1.853246in}{2.629074in}}{\pgfqpoint{1.857364in}{2.633192in}}%
\pgfpathcurveto{\pgfqpoint{1.861482in}{2.637310in}}{\pgfqpoint{1.863796in}{2.642896in}}{\pgfqpoint{1.863796in}{2.648720in}}%
\pgfpathcurveto{\pgfqpoint{1.863796in}{2.654544in}}{\pgfqpoint{1.861482in}{2.660130in}}{\pgfqpoint{1.857364in}{2.664249in}}%
\pgfpathcurveto{\pgfqpoint{1.853246in}{2.668367in}}{\pgfqpoint{1.847660in}{2.670681in}}{\pgfqpoint{1.841836in}{2.670681in}}%
\pgfpathcurveto{\pgfqpoint{1.836012in}{2.670681in}}{\pgfqpoint{1.830426in}{2.668367in}}{\pgfqpoint{1.826308in}{2.664249in}}%
\pgfpathcurveto{\pgfqpoint{1.822189in}{2.660130in}}{\pgfqpoint{1.819876in}{2.654544in}}{\pgfqpoint{1.819876in}{2.648720in}}%
\pgfpathcurveto{\pgfqpoint{1.819876in}{2.642896in}}{\pgfqpoint{1.822189in}{2.637310in}}{\pgfqpoint{1.826308in}{2.633192in}}%
\pgfpathcurveto{\pgfqpoint{1.830426in}{2.629074in}}{\pgfqpoint{1.836012in}{2.626760in}}{\pgfqpoint{1.841836in}{2.626760in}}%
\pgfpathlineto{\pgfqpoint{1.841836in}{2.626760in}}%
\pgfpathclose%
\pgfusepath{stroke,fill}%
\end{pgfscope}%
\begin{pgfscope}%
\pgfpathrectangle{\pgfqpoint{0.100000in}{0.183744in}}{\pgfqpoint{4.506048in}{4.506048in}}%
\pgfusepath{clip}%
\pgfsetbuttcap%
\pgfsetroundjoin%
\definecolor{currentfill}{rgb}{1.000000,0.647059,0.000000}%
\pgfsetfillcolor{currentfill}%
\pgfsetfillopacity{0.700000}%
\pgfsetlinewidth{1.003750pt}%
\definecolor{currentstroke}{rgb}{1.000000,0.647059,0.000000}%
\pgfsetstrokecolor{currentstroke}%
\pgfsetstrokeopacity{0.700000}%
\pgfsetdash{}{0pt}%
\pgfpathmoveto{\pgfqpoint{1.750458in}{2.337234in}}%
\pgfpathcurveto{\pgfqpoint{1.756282in}{2.337234in}}{\pgfqpoint{1.761868in}{2.339548in}}{\pgfqpoint{1.765986in}{2.343666in}}%
\pgfpathcurveto{\pgfqpoint{1.770104in}{2.347784in}}{\pgfqpoint{1.772418in}{2.353370in}}{\pgfqpoint{1.772418in}{2.359194in}}%
\pgfpathcurveto{\pgfqpoint{1.772418in}{2.365018in}}{\pgfqpoint{1.770104in}{2.370604in}}{\pgfqpoint{1.765986in}{2.374723in}}%
\pgfpathcurveto{\pgfqpoint{1.761868in}{2.378841in}}{\pgfqpoint{1.756282in}{2.381155in}}{\pgfqpoint{1.750458in}{2.381155in}}%
\pgfpathcurveto{\pgfqpoint{1.744634in}{2.381155in}}{\pgfqpoint{1.739048in}{2.378841in}}{\pgfqpoint{1.734930in}{2.374723in}}%
\pgfpathcurveto{\pgfqpoint{1.730812in}{2.370604in}}{\pgfqpoint{1.728498in}{2.365018in}}{\pgfqpoint{1.728498in}{2.359194in}}%
\pgfpathcurveto{\pgfqpoint{1.728498in}{2.353370in}}{\pgfqpoint{1.730812in}{2.347784in}}{\pgfqpoint{1.734930in}{2.343666in}}%
\pgfpathcurveto{\pgfqpoint{1.739048in}{2.339548in}}{\pgfqpoint{1.744634in}{2.337234in}}{\pgfqpoint{1.750458in}{2.337234in}}%
\pgfpathlineto{\pgfqpoint{1.750458in}{2.337234in}}%
\pgfpathclose%
\pgfusepath{stroke,fill}%
\end{pgfscope}%
\begin{pgfscope}%
\pgfpathrectangle{\pgfqpoint{0.100000in}{0.183744in}}{\pgfqpoint{4.506048in}{4.506048in}}%
\pgfusepath{clip}%
\pgfsetbuttcap%
\pgfsetroundjoin%
\definecolor{currentfill}{rgb}{1.000000,0.647059,0.000000}%
\pgfsetfillcolor{currentfill}%
\pgfsetfillopacity{0.700000}%
\pgfsetlinewidth{1.003750pt}%
\definecolor{currentstroke}{rgb}{1.000000,0.647059,0.000000}%
\pgfsetstrokecolor{currentstroke}%
\pgfsetstrokeopacity{0.700000}%
\pgfsetdash{}{0pt}%
\pgfpathmoveto{\pgfqpoint{1.618080in}{2.092508in}}%
\pgfpathcurveto{\pgfqpoint{1.623904in}{2.092508in}}{\pgfqpoint{1.629491in}{2.094821in}}{\pgfqpoint{1.633609in}{2.098940in}}%
\pgfpathcurveto{\pgfqpoint{1.637727in}{2.103058in}}{\pgfqpoint{1.640041in}{2.108644in}}{\pgfqpoint{1.640041in}{2.114468in}}%
\pgfpathcurveto{\pgfqpoint{1.640041in}{2.120292in}}{\pgfqpoint{1.637727in}{2.125878in}}{\pgfqpoint{1.633609in}{2.129996in}}%
\pgfpathcurveto{\pgfqpoint{1.629491in}{2.134114in}}{\pgfqpoint{1.623904in}{2.136428in}}{\pgfqpoint{1.618080in}{2.136428in}}%
\pgfpathcurveto{\pgfqpoint{1.612257in}{2.136428in}}{\pgfqpoint{1.606670in}{2.134114in}}{\pgfqpoint{1.602552in}{2.129996in}}%
\pgfpathcurveto{\pgfqpoint{1.598434in}{2.125878in}}{\pgfqpoint{1.596120in}{2.120292in}}{\pgfqpoint{1.596120in}{2.114468in}}%
\pgfpathcurveto{\pgfqpoint{1.596120in}{2.108644in}}{\pgfqpoint{1.598434in}{2.103058in}}{\pgfqpoint{1.602552in}{2.098940in}}%
\pgfpathcurveto{\pgfqpoint{1.606670in}{2.094821in}}{\pgfqpoint{1.612257in}{2.092508in}}{\pgfqpoint{1.618080in}{2.092508in}}%
\pgfpathlineto{\pgfqpoint{1.618080in}{2.092508in}}%
\pgfpathclose%
\pgfusepath{stroke,fill}%
\end{pgfscope}%
\begin{pgfscope}%
\pgfpathrectangle{\pgfqpoint{0.100000in}{0.183744in}}{\pgfqpoint{4.506048in}{4.506048in}}%
\pgfusepath{clip}%
\pgfsetbuttcap%
\pgfsetroundjoin%
\definecolor{currentfill}{rgb}{1.000000,0.647059,0.000000}%
\pgfsetfillcolor{currentfill}%
\pgfsetfillopacity{0.700000}%
\pgfsetlinewidth{1.003750pt}%
\definecolor{currentstroke}{rgb}{1.000000,0.647059,0.000000}%
\pgfsetstrokecolor{currentstroke}%
\pgfsetstrokeopacity{0.700000}%
\pgfsetdash{}{0pt}%
\pgfpathmoveto{\pgfqpoint{1.490739in}{1.767635in}}%
\pgfpathcurveto{\pgfqpoint{1.496563in}{1.767635in}}{\pgfqpoint{1.502149in}{1.769948in}}{\pgfqpoint{1.506267in}{1.774067in}}%
\pgfpathcurveto{\pgfqpoint{1.510385in}{1.778185in}}{\pgfqpoint{1.512699in}{1.783771in}}{\pgfqpoint{1.512699in}{1.789595in}}%
\pgfpathcurveto{\pgfqpoint{1.512699in}{1.795419in}}{\pgfqpoint{1.510385in}{1.801005in}}{\pgfqpoint{1.506267in}{1.805123in}}%
\pgfpathcurveto{\pgfqpoint{1.502149in}{1.809241in}}{\pgfqpoint{1.496563in}{1.811555in}}{\pgfqpoint{1.490739in}{1.811555in}}%
\pgfpathcurveto{\pgfqpoint{1.484915in}{1.811555in}}{\pgfqpoint{1.479329in}{1.809241in}}{\pgfqpoint{1.475211in}{1.805123in}}%
\pgfpathcurveto{\pgfqpoint{1.471093in}{1.801005in}}{\pgfqpoint{1.468779in}{1.795419in}}{\pgfqpoint{1.468779in}{1.789595in}}%
\pgfpathcurveto{\pgfqpoint{1.468779in}{1.783771in}}{\pgfqpoint{1.471093in}{1.778185in}}{\pgfqpoint{1.475211in}{1.774067in}}%
\pgfpathcurveto{\pgfqpoint{1.479329in}{1.769948in}}{\pgfqpoint{1.484915in}{1.767635in}}{\pgfqpoint{1.490739in}{1.767635in}}%
\pgfpathlineto{\pgfqpoint{1.490739in}{1.767635in}}%
\pgfpathclose%
\pgfusepath{stroke,fill}%
\end{pgfscope}%
\begin{pgfscope}%
\pgfpathrectangle{\pgfqpoint{0.100000in}{0.183744in}}{\pgfqpoint{4.506048in}{4.506048in}}%
\pgfusepath{clip}%
\pgfsetbuttcap%
\pgfsetroundjoin%
\definecolor{currentfill}{rgb}{1.000000,0.647059,0.000000}%
\pgfsetfillcolor{currentfill}%
\pgfsetfillopacity{0.700000}%
\pgfsetlinewidth{1.003750pt}%
\definecolor{currentstroke}{rgb}{1.000000,0.647059,0.000000}%
\pgfsetstrokecolor{currentstroke}%
\pgfsetstrokeopacity{0.700000}%
\pgfsetdash{}{0pt}%
\pgfpathmoveto{\pgfqpoint{2.002403in}{3.526126in}}%
\pgfpathcurveto{\pgfqpoint{2.008227in}{3.526126in}}{\pgfqpoint{2.013814in}{3.528440in}}{\pgfqpoint{2.017932in}{3.532558in}}%
\pgfpathcurveto{\pgfqpoint{2.022050in}{3.536676in}}{\pgfqpoint{2.024364in}{3.542263in}}{\pgfqpoint{2.024364in}{3.548087in}}%
\pgfpathcurveto{\pgfqpoint{2.024364in}{3.553911in}}{\pgfqpoint{2.022050in}{3.559497in}}{\pgfqpoint{2.017932in}{3.563615in}}%
\pgfpathcurveto{\pgfqpoint{2.013814in}{3.567733in}}{\pgfqpoint{2.008227in}{3.570047in}}{\pgfqpoint{2.002403in}{3.570047in}}%
\pgfpathcurveto{\pgfqpoint{1.996579in}{3.570047in}}{\pgfqpoint{1.990993in}{3.567733in}}{\pgfqpoint{1.986875in}{3.563615in}}%
\pgfpathcurveto{\pgfqpoint{1.982757in}{3.559497in}}{\pgfqpoint{1.980443in}{3.553911in}}{\pgfqpoint{1.980443in}{3.548087in}}%
\pgfpathcurveto{\pgfqpoint{1.980443in}{3.542263in}}{\pgfqpoint{1.982757in}{3.536676in}}{\pgfqpoint{1.986875in}{3.532558in}}%
\pgfpathcurveto{\pgfqpoint{1.990993in}{3.528440in}}{\pgfqpoint{1.996579in}{3.526126in}}{\pgfqpoint{2.002403in}{3.526126in}}%
\pgfpathlineto{\pgfqpoint{2.002403in}{3.526126in}}%
\pgfpathclose%
\pgfusepath{stroke,fill}%
\end{pgfscope}%
\begin{pgfscope}%
\pgfpathrectangle{\pgfqpoint{0.100000in}{0.183744in}}{\pgfqpoint{4.506048in}{4.506048in}}%
\pgfusepath{clip}%
\pgfsetbuttcap%
\pgfsetroundjoin%
\definecolor{currentfill}{rgb}{1.000000,0.647059,0.000000}%
\pgfsetfillcolor{currentfill}%
\pgfsetfillopacity{0.700000}%
\pgfsetlinewidth{1.003750pt}%
\definecolor{currentstroke}{rgb}{1.000000,0.647059,0.000000}%
\pgfsetstrokecolor{currentstroke}%
\pgfsetstrokeopacity{0.700000}%
\pgfsetdash{}{0pt}%
\pgfpathmoveto{\pgfqpoint{1.283301in}{2.411374in}}%
\pgfpathcurveto{\pgfqpoint{1.289125in}{2.411374in}}{\pgfqpoint{1.294712in}{2.413688in}}{\pgfqpoint{1.298830in}{2.417806in}}%
\pgfpathcurveto{\pgfqpoint{1.302948in}{2.421925in}}{\pgfqpoint{1.305262in}{2.427511in}}{\pgfqpoint{1.305262in}{2.433335in}}%
\pgfpathcurveto{\pgfqpoint{1.305262in}{2.439159in}}{\pgfqpoint{1.302948in}{2.444745in}}{\pgfqpoint{1.298830in}{2.448863in}}%
\pgfpathcurveto{\pgfqpoint{1.294712in}{2.452981in}}{\pgfqpoint{1.289125in}{2.455295in}}{\pgfqpoint{1.283301in}{2.455295in}}%
\pgfpathcurveto{\pgfqpoint{1.277478in}{2.455295in}}{\pgfqpoint{1.271891in}{2.452981in}}{\pgfqpoint{1.267773in}{2.448863in}}%
\pgfpathcurveto{\pgfqpoint{1.263655in}{2.444745in}}{\pgfqpoint{1.261341in}{2.439159in}}{\pgfqpoint{1.261341in}{2.433335in}}%
\pgfpathcurveto{\pgfqpoint{1.261341in}{2.427511in}}{\pgfqpoint{1.263655in}{2.421925in}}{\pgfqpoint{1.267773in}{2.417806in}}%
\pgfpathcurveto{\pgfqpoint{1.271891in}{2.413688in}}{\pgfqpoint{1.277478in}{2.411374in}}{\pgfqpoint{1.283301in}{2.411374in}}%
\pgfpathlineto{\pgfqpoint{1.283301in}{2.411374in}}%
\pgfpathclose%
\pgfusepath{stroke,fill}%
\end{pgfscope}%
\begin{pgfscope}%
\pgfpathrectangle{\pgfqpoint{0.100000in}{0.183744in}}{\pgfqpoint{4.506048in}{4.506048in}}%
\pgfusepath{clip}%
\pgfsetbuttcap%
\pgfsetroundjoin%
\definecolor{currentfill}{rgb}{1.000000,0.647059,0.000000}%
\pgfsetfillcolor{currentfill}%
\pgfsetfillopacity{0.700000}%
\pgfsetlinewidth{1.003750pt}%
\definecolor{currentstroke}{rgb}{1.000000,0.647059,0.000000}%
\pgfsetstrokecolor{currentstroke}%
\pgfsetstrokeopacity{0.700000}%
\pgfsetdash{}{0pt}%
\pgfpathmoveto{\pgfqpoint{2.291752in}{2.411674in}}%
\pgfpathcurveto{\pgfqpoint{2.297576in}{2.411674in}}{\pgfqpoint{2.303162in}{2.413988in}}{\pgfqpoint{2.307280in}{2.418106in}}%
\pgfpathcurveto{\pgfqpoint{2.311398in}{2.422224in}}{\pgfqpoint{2.313712in}{2.427810in}}{\pgfqpoint{2.313712in}{2.433634in}}%
\pgfpathcurveto{\pgfqpoint{2.313712in}{2.439458in}}{\pgfqpoint{2.311398in}{2.445044in}}{\pgfqpoint{2.307280in}{2.449162in}}%
\pgfpathcurveto{\pgfqpoint{2.303162in}{2.453280in}}{\pgfqpoint{2.297576in}{2.455594in}}{\pgfqpoint{2.291752in}{2.455594in}}%
\pgfpathcurveto{\pgfqpoint{2.285928in}{2.455594in}}{\pgfqpoint{2.280342in}{2.453280in}}{\pgfqpoint{2.276224in}{2.449162in}}%
\pgfpathcurveto{\pgfqpoint{2.272106in}{2.445044in}}{\pgfqpoint{2.269792in}{2.439458in}}{\pgfqpoint{2.269792in}{2.433634in}}%
\pgfpathcurveto{\pgfqpoint{2.269792in}{2.427810in}}{\pgfqpoint{2.272106in}{2.422224in}}{\pgfqpoint{2.276224in}{2.418106in}}%
\pgfpathcurveto{\pgfqpoint{2.280342in}{2.413988in}}{\pgfqpoint{2.285928in}{2.411674in}}{\pgfqpoint{2.291752in}{2.411674in}}%
\pgfpathlineto{\pgfqpoint{2.291752in}{2.411674in}}%
\pgfpathclose%
\pgfusepath{stroke,fill}%
\end{pgfscope}%
\begin{pgfscope}%
\pgfpathrectangle{\pgfqpoint{0.100000in}{0.183744in}}{\pgfqpoint{4.506048in}{4.506048in}}%
\pgfusepath{clip}%
\pgfsetbuttcap%
\pgfsetroundjoin%
\definecolor{currentfill}{rgb}{1.000000,0.647059,0.000000}%
\pgfsetfillcolor{currentfill}%
\pgfsetfillopacity{0.700000}%
\pgfsetlinewidth{1.003750pt}%
\definecolor{currentstroke}{rgb}{1.000000,0.647059,0.000000}%
\pgfsetstrokecolor{currentstroke}%
\pgfsetstrokeopacity{0.700000}%
\pgfsetdash{}{0pt}%
\pgfpathmoveto{\pgfqpoint{1.881654in}{1.411506in}}%
\pgfpathcurveto{\pgfqpoint{1.887478in}{1.411506in}}{\pgfqpoint{1.893064in}{1.413820in}}{\pgfqpoint{1.897182in}{1.417938in}}%
\pgfpathcurveto{\pgfqpoint{1.901300in}{1.422056in}}{\pgfqpoint{1.903614in}{1.427642in}}{\pgfqpoint{1.903614in}{1.433466in}}%
\pgfpathcurveto{\pgfqpoint{1.903614in}{1.439290in}}{\pgfqpoint{1.901300in}{1.444876in}}{\pgfqpoint{1.897182in}{1.448994in}}%
\pgfpathcurveto{\pgfqpoint{1.893064in}{1.453113in}}{\pgfqpoint{1.887478in}{1.455426in}}{\pgfqpoint{1.881654in}{1.455426in}}%
\pgfpathcurveto{\pgfqpoint{1.875830in}{1.455426in}}{\pgfqpoint{1.870244in}{1.453113in}}{\pgfqpoint{1.866126in}{1.448994in}}%
\pgfpathcurveto{\pgfqpoint{1.862007in}{1.444876in}}{\pgfqpoint{1.859694in}{1.439290in}}{\pgfqpoint{1.859694in}{1.433466in}}%
\pgfpathcurveto{\pgfqpoint{1.859694in}{1.427642in}}{\pgfqpoint{1.862007in}{1.422056in}}{\pgfqpoint{1.866126in}{1.417938in}}%
\pgfpathcurveto{\pgfqpoint{1.870244in}{1.413820in}}{\pgfqpoint{1.875830in}{1.411506in}}{\pgfqpoint{1.881654in}{1.411506in}}%
\pgfpathlineto{\pgfqpoint{1.881654in}{1.411506in}}%
\pgfpathclose%
\pgfusepath{stroke,fill}%
\end{pgfscope}%
\begin{pgfscope}%
\pgfpathrectangle{\pgfqpoint{0.100000in}{0.183744in}}{\pgfqpoint{4.506048in}{4.506048in}}%
\pgfusepath{clip}%
\pgfsetbuttcap%
\pgfsetroundjoin%
\definecolor{currentfill}{rgb}{1.000000,0.647059,0.000000}%
\pgfsetfillcolor{currentfill}%
\pgfsetfillopacity{0.700000}%
\pgfsetlinewidth{1.003750pt}%
\definecolor{currentstroke}{rgb}{1.000000,0.647059,0.000000}%
\pgfsetstrokecolor{currentstroke}%
\pgfsetstrokeopacity{0.700000}%
\pgfsetdash{}{0pt}%
\pgfpathmoveto{\pgfqpoint{2.962596in}{2.250959in}}%
\pgfpathcurveto{\pgfqpoint{2.968420in}{2.250959in}}{\pgfqpoint{2.974006in}{2.253273in}}{\pgfqpoint{2.978124in}{2.257391in}}%
\pgfpathcurveto{\pgfqpoint{2.982242in}{2.261509in}}{\pgfqpoint{2.984556in}{2.267095in}}{\pgfqpoint{2.984556in}{2.272919in}}%
\pgfpathcurveto{\pgfqpoint{2.984556in}{2.278743in}}{\pgfqpoint{2.982242in}{2.284329in}}{\pgfqpoint{2.978124in}{2.288447in}}%
\pgfpathcurveto{\pgfqpoint{2.974006in}{2.292565in}}{\pgfqpoint{2.968420in}{2.294879in}}{\pgfqpoint{2.962596in}{2.294879in}}%
\pgfpathcurveto{\pgfqpoint{2.956772in}{2.294879in}}{\pgfqpoint{2.951186in}{2.292565in}}{\pgfqpoint{2.947068in}{2.288447in}}%
\pgfpathcurveto{\pgfqpoint{2.942949in}{2.284329in}}{\pgfqpoint{2.940636in}{2.278743in}}{\pgfqpoint{2.940636in}{2.272919in}}%
\pgfpathcurveto{\pgfqpoint{2.940636in}{2.267095in}}{\pgfqpoint{2.942949in}{2.261509in}}{\pgfqpoint{2.947068in}{2.257391in}}%
\pgfpathcurveto{\pgfqpoint{2.951186in}{2.253273in}}{\pgfqpoint{2.956772in}{2.250959in}}{\pgfqpoint{2.962596in}{2.250959in}}%
\pgfpathlineto{\pgfqpoint{2.962596in}{2.250959in}}%
\pgfpathclose%
\pgfusepath{stroke,fill}%
\end{pgfscope}%
\begin{pgfscope}%
\pgfpathrectangle{\pgfqpoint{0.100000in}{0.183744in}}{\pgfqpoint{4.506048in}{4.506048in}}%
\pgfusepath{clip}%
\pgfsetbuttcap%
\pgfsetroundjoin%
\definecolor{currentfill}{rgb}{1.000000,0.647059,0.000000}%
\pgfsetfillcolor{currentfill}%
\pgfsetfillopacity{0.700000}%
\pgfsetlinewidth{1.003750pt}%
\definecolor{currentstroke}{rgb}{1.000000,0.647059,0.000000}%
\pgfsetstrokecolor{currentstroke}%
\pgfsetstrokeopacity{0.700000}%
\pgfsetdash{}{0pt}%
\pgfpathmoveto{\pgfqpoint{3.159912in}{1.679287in}}%
\pgfpathcurveto{\pgfqpoint{3.165736in}{1.679287in}}{\pgfqpoint{3.171322in}{1.681601in}}{\pgfqpoint{3.175440in}{1.685719in}}%
\pgfpathcurveto{\pgfqpoint{3.179558in}{1.689837in}}{\pgfqpoint{3.181872in}{1.695423in}}{\pgfqpoint{3.181872in}{1.701247in}}%
\pgfpathcurveto{\pgfqpoint{3.181872in}{1.707071in}}{\pgfqpoint{3.179558in}{1.712657in}}{\pgfqpoint{3.175440in}{1.716775in}}%
\pgfpathcurveto{\pgfqpoint{3.171322in}{1.720893in}}{\pgfqpoint{3.165736in}{1.723207in}}{\pgfqpoint{3.159912in}{1.723207in}}%
\pgfpathcurveto{\pgfqpoint{3.154088in}{1.723207in}}{\pgfqpoint{3.148502in}{1.720893in}}{\pgfqpoint{3.144383in}{1.716775in}}%
\pgfpathcurveto{\pgfqpoint{3.140265in}{1.712657in}}{\pgfqpoint{3.137951in}{1.707071in}}{\pgfqpoint{3.137951in}{1.701247in}}%
\pgfpathcurveto{\pgfqpoint{3.137951in}{1.695423in}}{\pgfqpoint{3.140265in}{1.689837in}}{\pgfqpoint{3.144383in}{1.685719in}}%
\pgfpathcurveto{\pgfqpoint{3.148502in}{1.681601in}}{\pgfqpoint{3.154088in}{1.679287in}}{\pgfqpoint{3.159912in}{1.679287in}}%
\pgfpathlineto{\pgfqpoint{3.159912in}{1.679287in}}%
\pgfpathclose%
\pgfusepath{stroke,fill}%
\end{pgfscope}%
\begin{pgfscope}%
\pgfpathrectangle{\pgfqpoint{0.100000in}{0.183744in}}{\pgfqpoint{4.506048in}{4.506048in}}%
\pgfusepath{clip}%
\pgfsetbuttcap%
\pgfsetroundjoin%
\definecolor{currentfill}{rgb}{1.000000,0.647059,0.000000}%
\pgfsetfillcolor{currentfill}%
\pgfsetfillopacity{0.700000}%
\pgfsetlinewidth{1.003750pt}%
\definecolor{currentstroke}{rgb}{1.000000,0.647059,0.000000}%
\pgfsetstrokecolor{currentstroke}%
\pgfsetstrokeopacity{0.700000}%
\pgfsetdash{}{0pt}%
\pgfpathmoveto{\pgfqpoint{2.553467in}{2.807125in}}%
\pgfpathcurveto{\pgfqpoint{2.559291in}{2.807125in}}{\pgfqpoint{2.564877in}{2.809439in}}{\pgfqpoint{2.568996in}{2.813557in}}%
\pgfpathcurveto{\pgfqpoint{2.573114in}{2.817676in}}{\pgfqpoint{2.575428in}{2.823262in}}{\pgfqpoint{2.575428in}{2.829086in}}%
\pgfpathcurveto{\pgfqpoint{2.575428in}{2.834910in}}{\pgfqpoint{2.573114in}{2.840496in}}{\pgfqpoint{2.568996in}{2.844614in}}%
\pgfpathcurveto{\pgfqpoint{2.564877in}{2.848732in}}{\pgfqpoint{2.559291in}{2.851046in}}{\pgfqpoint{2.553467in}{2.851046in}}%
\pgfpathcurveto{\pgfqpoint{2.547643in}{2.851046in}}{\pgfqpoint{2.542057in}{2.848732in}}{\pgfqpoint{2.537939in}{2.844614in}}%
\pgfpathcurveto{\pgfqpoint{2.533821in}{2.840496in}}{\pgfqpoint{2.531507in}{2.834910in}}{\pgfqpoint{2.531507in}{2.829086in}}%
\pgfpathcurveto{\pgfqpoint{2.531507in}{2.823262in}}{\pgfqpoint{2.533821in}{2.817676in}}{\pgfqpoint{2.537939in}{2.813557in}}%
\pgfpathcurveto{\pgfqpoint{2.542057in}{2.809439in}}{\pgfqpoint{2.547643in}{2.807125in}}{\pgfqpoint{2.553467in}{2.807125in}}%
\pgfpathlineto{\pgfqpoint{2.553467in}{2.807125in}}%
\pgfpathclose%
\pgfusepath{stroke,fill}%
\end{pgfscope}%
\begin{pgfscope}%
\pgfpathrectangle{\pgfqpoint{0.100000in}{0.183744in}}{\pgfqpoint{4.506048in}{4.506048in}}%
\pgfusepath{clip}%
\pgfsetbuttcap%
\pgfsetroundjoin%
\definecolor{currentfill}{rgb}{1.000000,0.647059,0.000000}%
\pgfsetfillcolor{currentfill}%
\pgfsetfillopacity{0.700000}%
\pgfsetlinewidth{1.003750pt}%
\definecolor{currentstroke}{rgb}{1.000000,0.647059,0.000000}%
\pgfsetstrokecolor{currentstroke}%
\pgfsetstrokeopacity{0.700000}%
\pgfsetdash{}{0pt}%
\pgfpathmoveto{\pgfqpoint{2.358049in}{1.675923in}}%
\pgfpathcurveto{\pgfqpoint{2.363873in}{1.675923in}}{\pgfqpoint{2.369459in}{1.678237in}}{\pgfqpoint{2.373577in}{1.682355in}}%
\pgfpathcurveto{\pgfqpoint{2.377696in}{1.686473in}}{\pgfqpoint{2.380009in}{1.692059in}}{\pgfqpoint{2.380009in}{1.697883in}}%
\pgfpathcurveto{\pgfqpoint{2.380009in}{1.703707in}}{\pgfqpoint{2.377696in}{1.709293in}}{\pgfqpoint{2.373577in}{1.713412in}}%
\pgfpathcurveto{\pgfqpoint{2.369459in}{1.717530in}}{\pgfqpoint{2.363873in}{1.719844in}}{\pgfqpoint{2.358049in}{1.719844in}}%
\pgfpathcurveto{\pgfqpoint{2.352225in}{1.719844in}}{\pgfqpoint{2.346639in}{1.717530in}}{\pgfqpoint{2.342521in}{1.713412in}}%
\pgfpathcurveto{\pgfqpoint{2.338403in}{1.709293in}}{\pgfqpoint{2.336089in}{1.703707in}}{\pgfqpoint{2.336089in}{1.697883in}}%
\pgfpathcurveto{\pgfqpoint{2.336089in}{1.692059in}}{\pgfqpoint{2.338403in}{1.686473in}}{\pgfqpoint{2.342521in}{1.682355in}}%
\pgfpathcurveto{\pgfqpoint{2.346639in}{1.678237in}}{\pgfqpoint{2.352225in}{1.675923in}}{\pgfqpoint{2.358049in}{1.675923in}}%
\pgfpathlineto{\pgfqpoint{2.358049in}{1.675923in}}%
\pgfpathclose%
\pgfusepath{stroke,fill}%
\end{pgfscope}%
\begin{pgfscope}%
\pgfpathrectangle{\pgfqpoint{0.100000in}{0.183744in}}{\pgfqpoint{4.506048in}{4.506048in}}%
\pgfusepath{clip}%
\pgfsetbuttcap%
\pgfsetroundjoin%
\definecolor{currentfill}{rgb}{1.000000,0.647059,0.000000}%
\pgfsetfillcolor{currentfill}%
\pgfsetfillopacity{0.700000}%
\pgfsetlinewidth{1.003750pt}%
\definecolor{currentstroke}{rgb}{1.000000,0.647059,0.000000}%
\pgfsetstrokecolor{currentstroke}%
\pgfsetstrokeopacity{0.700000}%
\pgfsetdash{}{0pt}%
\pgfpathmoveto{\pgfqpoint{2.719998in}{2.234106in}}%
\pgfpathcurveto{\pgfqpoint{2.725822in}{2.234106in}}{\pgfqpoint{2.731408in}{2.236420in}}{\pgfqpoint{2.735526in}{2.240538in}}%
\pgfpathcurveto{\pgfqpoint{2.739644in}{2.244656in}}{\pgfqpoint{2.741958in}{2.250243in}}{\pgfqpoint{2.741958in}{2.256066in}}%
\pgfpathcurveto{\pgfqpoint{2.741958in}{2.261890in}}{\pgfqpoint{2.739644in}{2.267477in}}{\pgfqpoint{2.735526in}{2.271595in}}%
\pgfpathcurveto{\pgfqpoint{2.731408in}{2.275713in}}{\pgfqpoint{2.725822in}{2.278027in}}{\pgfqpoint{2.719998in}{2.278027in}}%
\pgfpathcurveto{\pgfqpoint{2.714174in}{2.278027in}}{\pgfqpoint{2.708588in}{2.275713in}}{\pgfqpoint{2.704470in}{2.271595in}}%
\pgfpathcurveto{\pgfqpoint{2.700351in}{2.267477in}}{\pgfqpoint{2.698038in}{2.261890in}}{\pgfqpoint{2.698038in}{2.256066in}}%
\pgfpathcurveto{\pgfqpoint{2.698038in}{2.250243in}}{\pgfqpoint{2.700351in}{2.244656in}}{\pgfqpoint{2.704470in}{2.240538in}}%
\pgfpathcurveto{\pgfqpoint{2.708588in}{2.236420in}}{\pgfqpoint{2.714174in}{2.234106in}}{\pgfqpoint{2.719998in}{2.234106in}}%
\pgfpathlineto{\pgfqpoint{2.719998in}{2.234106in}}%
\pgfpathclose%
\pgfusepath{stroke,fill}%
\end{pgfscope}%
\begin{pgfscope}%
\pgfpathrectangle{\pgfqpoint{0.100000in}{0.183744in}}{\pgfqpoint{4.506048in}{4.506048in}}%
\pgfusepath{clip}%
\pgfsetbuttcap%
\pgfsetroundjoin%
\definecolor{currentfill}{rgb}{1.000000,0.647059,0.000000}%
\pgfsetfillcolor{currentfill}%
\pgfsetfillopacity{0.700000}%
\pgfsetlinewidth{1.003750pt}%
\definecolor{currentstroke}{rgb}{1.000000,0.647059,0.000000}%
\pgfsetstrokecolor{currentstroke}%
\pgfsetstrokeopacity{0.700000}%
\pgfsetdash{}{0pt}%
\pgfpathmoveto{\pgfqpoint{1.890869in}{3.258666in}}%
\pgfpathcurveto{\pgfqpoint{1.896693in}{3.258666in}}{\pgfqpoint{1.902279in}{3.260980in}}{\pgfqpoint{1.906397in}{3.265098in}}%
\pgfpathcurveto{\pgfqpoint{1.910515in}{3.269216in}}{\pgfqpoint{1.912829in}{3.274802in}}{\pgfqpoint{1.912829in}{3.280626in}}%
\pgfpathcurveto{\pgfqpoint{1.912829in}{3.286450in}}{\pgfqpoint{1.910515in}{3.292036in}}{\pgfqpoint{1.906397in}{3.296155in}}%
\pgfpathcurveto{\pgfqpoint{1.902279in}{3.300273in}}{\pgfqpoint{1.896693in}{3.302587in}}{\pgfqpoint{1.890869in}{3.302587in}}%
\pgfpathcurveto{\pgfqpoint{1.885045in}{3.302587in}}{\pgfqpoint{1.879459in}{3.300273in}}{\pgfqpoint{1.875340in}{3.296155in}}%
\pgfpathcurveto{\pgfqpoint{1.871222in}{3.292036in}}{\pgfqpoint{1.868908in}{3.286450in}}{\pgfqpoint{1.868908in}{3.280626in}}%
\pgfpathcurveto{\pgfqpoint{1.868908in}{3.274802in}}{\pgfqpoint{1.871222in}{3.269216in}}{\pgfqpoint{1.875340in}{3.265098in}}%
\pgfpathcurveto{\pgfqpoint{1.879459in}{3.260980in}}{\pgfqpoint{1.885045in}{3.258666in}}{\pgfqpoint{1.890869in}{3.258666in}}%
\pgfpathlineto{\pgfqpoint{1.890869in}{3.258666in}}%
\pgfpathclose%
\pgfusepath{stroke,fill}%
\end{pgfscope}%
\begin{pgfscope}%
\pgfpathrectangle{\pgfqpoint{0.100000in}{0.183744in}}{\pgfqpoint{4.506048in}{4.506048in}}%
\pgfusepath{clip}%
\pgfsetbuttcap%
\pgfsetroundjoin%
\definecolor{currentfill}{rgb}{1.000000,0.647059,0.000000}%
\pgfsetfillcolor{currentfill}%
\pgfsetfillopacity{0.700000}%
\pgfsetlinewidth{1.003750pt}%
\definecolor{currentstroke}{rgb}{1.000000,0.647059,0.000000}%
\pgfsetstrokecolor{currentstroke}%
\pgfsetstrokeopacity{0.700000}%
\pgfsetdash{}{0pt}%
\pgfpathmoveto{\pgfqpoint{1.203420in}{1.568385in}}%
\pgfpathcurveto{\pgfqpoint{1.209244in}{1.568385in}}{\pgfqpoint{1.214831in}{1.570699in}}{\pgfqpoint{1.218949in}{1.574817in}}%
\pgfpathcurveto{\pgfqpoint{1.223067in}{1.578935in}}{\pgfqpoint{1.225381in}{1.584522in}}{\pgfqpoint{1.225381in}{1.590346in}}%
\pgfpathcurveto{\pgfqpoint{1.225381in}{1.596170in}}{\pgfqpoint{1.223067in}{1.601756in}}{\pgfqpoint{1.218949in}{1.605874in}}%
\pgfpathcurveto{\pgfqpoint{1.214831in}{1.609992in}}{\pgfqpoint{1.209244in}{1.612306in}}{\pgfqpoint{1.203420in}{1.612306in}}%
\pgfpathcurveto{\pgfqpoint{1.197596in}{1.612306in}}{\pgfqpoint{1.192010in}{1.609992in}}{\pgfqpoint{1.187892in}{1.605874in}}%
\pgfpathcurveto{\pgfqpoint{1.183774in}{1.601756in}}{\pgfqpoint{1.181460in}{1.596170in}}{\pgfqpoint{1.181460in}{1.590346in}}%
\pgfpathcurveto{\pgfqpoint{1.181460in}{1.584522in}}{\pgfqpoint{1.183774in}{1.578935in}}{\pgfqpoint{1.187892in}{1.574817in}}%
\pgfpathcurveto{\pgfqpoint{1.192010in}{1.570699in}}{\pgfqpoint{1.197596in}{1.568385in}}{\pgfqpoint{1.203420in}{1.568385in}}%
\pgfpathlineto{\pgfqpoint{1.203420in}{1.568385in}}%
\pgfpathclose%
\pgfusepath{stroke,fill}%
\end{pgfscope}%
\begin{pgfscope}%
\pgfpathrectangle{\pgfqpoint{0.100000in}{0.183744in}}{\pgfqpoint{4.506048in}{4.506048in}}%
\pgfusepath{clip}%
\pgfsetbuttcap%
\pgfsetroundjoin%
\definecolor{currentfill}{rgb}{1.000000,0.647059,0.000000}%
\pgfsetfillcolor{currentfill}%
\pgfsetfillopacity{0.700000}%
\pgfsetlinewidth{1.003750pt}%
\definecolor{currentstroke}{rgb}{1.000000,0.647059,0.000000}%
\pgfsetstrokecolor{currentstroke}%
\pgfsetstrokeopacity{0.700000}%
\pgfsetdash{}{0pt}%
\pgfpathmoveto{\pgfqpoint{1.698303in}{2.118603in}}%
\pgfpathcurveto{\pgfqpoint{1.704127in}{2.118603in}}{\pgfqpoint{1.709713in}{2.120917in}}{\pgfqpoint{1.713832in}{2.125035in}}%
\pgfpathcurveto{\pgfqpoint{1.717950in}{2.129153in}}{\pgfqpoint{1.720264in}{2.134739in}}{\pgfqpoint{1.720264in}{2.140563in}}%
\pgfpathcurveto{\pgfqpoint{1.720264in}{2.146387in}}{\pgfqpoint{1.717950in}{2.151973in}}{\pgfqpoint{1.713832in}{2.156091in}}%
\pgfpathcurveto{\pgfqpoint{1.709713in}{2.160209in}}{\pgfqpoint{1.704127in}{2.162523in}}{\pgfqpoint{1.698303in}{2.162523in}}%
\pgfpathcurveto{\pgfqpoint{1.692479in}{2.162523in}}{\pgfqpoint{1.686893in}{2.160209in}}{\pgfqpoint{1.682775in}{2.156091in}}%
\pgfpathcurveto{\pgfqpoint{1.678657in}{2.151973in}}{\pgfqpoint{1.676343in}{2.146387in}}{\pgfqpoint{1.676343in}{2.140563in}}%
\pgfpathcurveto{\pgfqpoint{1.676343in}{2.134739in}}{\pgfqpoint{1.678657in}{2.129153in}}{\pgfqpoint{1.682775in}{2.125035in}}%
\pgfpathcurveto{\pgfqpoint{1.686893in}{2.120917in}}{\pgfqpoint{1.692479in}{2.118603in}}{\pgfqpoint{1.698303in}{2.118603in}}%
\pgfpathlineto{\pgfqpoint{1.698303in}{2.118603in}}%
\pgfpathclose%
\pgfusepath{stroke,fill}%
\end{pgfscope}%
\begin{pgfscope}%
\pgfpathrectangle{\pgfqpoint{0.100000in}{0.183744in}}{\pgfqpoint{4.506048in}{4.506048in}}%
\pgfusepath{clip}%
\pgfsetbuttcap%
\pgfsetroundjoin%
\definecolor{currentfill}{rgb}{1.000000,0.647059,0.000000}%
\pgfsetfillcolor{currentfill}%
\pgfsetfillopacity{0.700000}%
\pgfsetlinewidth{1.003750pt}%
\definecolor{currentstroke}{rgb}{1.000000,0.647059,0.000000}%
\pgfsetstrokecolor{currentstroke}%
\pgfsetstrokeopacity{0.700000}%
\pgfsetdash{}{0pt}%
\pgfpathmoveto{\pgfqpoint{2.350882in}{1.526588in}}%
\pgfpathcurveto{\pgfqpoint{2.356706in}{1.526588in}}{\pgfqpoint{2.362292in}{1.528902in}}{\pgfqpoint{2.366410in}{1.533020in}}%
\pgfpathcurveto{\pgfqpoint{2.370528in}{1.537138in}}{\pgfqpoint{2.372842in}{1.542724in}}{\pgfqpoint{2.372842in}{1.548548in}}%
\pgfpathcurveto{\pgfqpoint{2.372842in}{1.554372in}}{\pgfqpoint{2.370528in}{1.559958in}}{\pgfqpoint{2.366410in}{1.564077in}}%
\pgfpathcurveto{\pgfqpoint{2.362292in}{1.568195in}}{\pgfqpoint{2.356706in}{1.570509in}}{\pgfqpoint{2.350882in}{1.570509in}}%
\pgfpathcurveto{\pgfqpoint{2.345058in}{1.570509in}}{\pgfqpoint{2.339472in}{1.568195in}}{\pgfqpoint{2.335353in}{1.564077in}}%
\pgfpathcurveto{\pgfqpoint{2.331235in}{1.559958in}}{\pgfqpoint{2.328921in}{1.554372in}}{\pgfqpoint{2.328921in}{1.548548in}}%
\pgfpathcurveto{\pgfqpoint{2.328921in}{1.542724in}}{\pgfqpoint{2.331235in}{1.537138in}}{\pgfqpoint{2.335353in}{1.533020in}}%
\pgfpathcurveto{\pgfqpoint{2.339472in}{1.528902in}}{\pgfqpoint{2.345058in}{1.526588in}}{\pgfqpoint{2.350882in}{1.526588in}}%
\pgfpathlineto{\pgfqpoint{2.350882in}{1.526588in}}%
\pgfpathclose%
\pgfusepath{stroke,fill}%
\end{pgfscope}%
\begin{pgfscope}%
\pgfpathrectangle{\pgfqpoint{0.100000in}{0.183744in}}{\pgfqpoint{4.506048in}{4.506048in}}%
\pgfusepath{clip}%
\pgfsetbuttcap%
\pgfsetroundjoin%
\definecolor{currentfill}{rgb}{1.000000,0.647059,0.000000}%
\pgfsetfillcolor{currentfill}%
\pgfsetfillopacity{0.700000}%
\pgfsetlinewidth{1.003750pt}%
\definecolor{currentstroke}{rgb}{1.000000,0.647059,0.000000}%
\pgfsetstrokecolor{currentstroke}%
\pgfsetstrokeopacity{0.700000}%
\pgfsetdash{}{0pt}%
\pgfpathmoveto{\pgfqpoint{0.721276in}{3.159756in}}%
\pgfpathcurveto{\pgfqpoint{0.727100in}{3.159756in}}{\pgfqpoint{0.732686in}{3.162069in}}{\pgfqpoint{0.736804in}{3.166188in}}%
\pgfpathcurveto{\pgfqpoint{0.740922in}{3.170306in}}{\pgfqpoint{0.743236in}{3.175892in}}{\pgfqpoint{0.743236in}{3.181716in}}%
\pgfpathcurveto{\pgfqpoint{0.743236in}{3.187540in}}{\pgfqpoint{0.740922in}{3.193126in}}{\pgfqpoint{0.736804in}{3.197244in}}%
\pgfpathcurveto{\pgfqpoint{0.732686in}{3.201362in}}{\pgfqpoint{0.727100in}{3.203676in}}{\pgfqpoint{0.721276in}{3.203676in}}%
\pgfpathcurveto{\pgfqpoint{0.715452in}{3.203676in}}{\pgfqpoint{0.709866in}{3.201362in}}{\pgfqpoint{0.705747in}{3.197244in}}%
\pgfpathcurveto{\pgfqpoint{0.701629in}{3.193126in}}{\pgfqpoint{0.699315in}{3.187540in}}{\pgfqpoint{0.699315in}{3.181716in}}%
\pgfpathcurveto{\pgfqpoint{0.699315in}{3.175892in}}{\pgfqpoint{0.701629in}{3.170306in}}{\pgfqpoint{0.705747in}{3.166188in}}%
\pgfpathcurveto{\pgfqpoint{0.709866in}{3.162069in}}{\pgfqpoint{0.715452in}{3.159756in}}{\pgfqpoint{0.721276in}{3.159756in}}%
\pgfpathlineto{\pgfqpoint{0.721276in}{3.159756in}}%
\pgfpathclose%
\pgfusepath{stroke,fill}%
\end{pgfscope}%
\begin{pgfscope}%
\pgfpathrectangle{\pgfqpoint{0.100000in}{0.183744in}}{\pgfqpoint{4.506048in}{4.506048in}}%
\pgfusepath{clip}%
\pgfsetbuttcap%
\pgfsetroundjoin%
\definecolor{currentfill}{rgb}{1.000000,0.647059,0.000000}%
\pgfsetfillcolor{currentfill}%
\pgfsetfillopacity{0.700000}%
\pgfsetlinewidth{1.003750pt}%
\definecolor{currentstroke}{rgb}{1.000000,0.647059,0.000000}%
\pgfsetstrokecolor{currentstroke}%
\pgfsetstrokeopacity{0.700000}%
\pgfsetdash{}{0pt}%
\pgfpathmoveto{\pgfqpoint{1.197433in}{2.928675in}}%
\pgfpathcurveto{\pgfqpoint{1.203257in}{2.928675in}}{\pgfqpoint{1.208844in}{2.930989in}}{\pgfqpoint{1.212962in}{2.935107in}}%
\pgfpathcurveto{\pgfqpoint{1.217080in}{2.939225in}}{\pgfqpoint{1.219394in}{2.944811in}}{\pgfqpoint{1.219394in}{2.950635in}}%
\pgfpathcurveto{\pgfqpoint{1.219394in}{2.956459in}}{\pgfqpoint{1.217080in}{2.962045in}}{\pgfqpoint{1.212962in}{2.966163in}}%
\pgfpathcurveto{\pgfqpoint{1.208844in}{2.970281in}}{\pgfqpoint{1.203257in}{2.972595in}}{\pgfqpoint{1.197433in}{2.972595in}}%
\pgfpathcurveto{\pgfqpoint{1.191610in}{2.972595in}}{\pgfqpoint{1.186023in}{2.970281in}}{\pgfqpoint{1.181905in}{2.966163in}}%
\pgfpathcurveto{\pgfqpoint{1.177787in}{2.962045in}}{\pgfqpoint{1.175473in}{2.956459in}}{\pgfqpoint{1.175473in}{2.950635in}}%
\pgfpathcurveto{\pgfqpoint{1.175473in}{2.944811in}}{\pgfqpoint{1.177787in}{2.939225in}}{\pgfqpoint{1.181905in}{2.935107in}}%
\pgfpathcurveto{\pgfqpoint{1.186023in}{2.930989in}}{\pgfqpoint{1.191610in}{2.928675in}}{\pgfqpoint{1.197433in}{2.928675in}}%
\pgfpathlineto{\pgfqpoint{1.197433in}{2.928675in}}%
\pgfpathclose%
\pgfusepath{stroke,fill}%
\end{pgfscope}%
\begin{pgfscope}%
\pgfpathrectangle{\pgfqpoint{0.100000in}{0.183744in}}{\pgfqpoint{4.506048in}{4.506048in}}%
\pgfusepath{clip}%
\pgfsetbuttcap%
\pgfsetroundjoin%
\definecolor{currentfill}{rgb}{1.000000,0.647059,0.000000}%
\pgfsetfillcolor{currentfill}%
\pgfsetfillopacity{0.700000}%
\pgfsetlinewidth{1.003750pt}%
\definecolor{currentstroke}{rgb}{1.000000,0.647059,0.000000}%
\pgfsetstrokecolor{currentstroke}%
\pgfsetstrokeopacity{0.700000}%
\pgfsetdash{}{0pt}%
\pgfpathmoveto{\pgfqpoint{0.774189in}{2.770323in}}%
\pgfpathcurveto{\pgfqpoint{0.780013in}{2.770323in}}{\pgfqpoint{0.785600in}{2.772637in}}{\pgfqpoint{0.789718in}{2.776755in}}%
\pgfpathcurveto{\pgfqpoint{0.793836in}{2.780873in}}{\pgfqpoint{0.796150in}{2.786460in}}{\pgfqpoint{0.796150in}{2.792284in}}%
\pgfpathcurveto{\pgfqpoint{0.796150in}{2.798107in}}{\pgfqpoint{0.793836in}{2.803694in}}{\pgfqpoint{0.789718in}{2.807812in}}%
\pgfpathcurveto{\pgfqpoint{0.785600in}{2.811930in}}{\pgfqpoint{0.780013in}{2.814244in}}{\pgfqpoint{0.774189in}{2.814244in}}%
\pgfpathcurveto{\pgfqpoint{0.768366in}{2.814244in}}{\pgfqpoint{0.762779in}{2.811930in}}{\pgfqpoint{0.758661in}{2.807812in}}%
\pgfpathcurveto{\pgfqpoint{0.754543in}{2.803694in}}{\pgfqpoint{0.752229in}{2.798107in}}{\pgfqpoint{0.752229in}{2.792284in}}%
\pgfpathcurveto{\pgfqpoint{0.752229in}{2.786460in}}{\pgfqpoint{0.754543in}{2.780873in}}{\pgfqpoint{0.758661in}{2.776755in}}%
\pgfpathcurveto{\pgfqpoint{0.762779in}{2.772637in}}{\pgfqpoint{0.768366in}{2.770323in}}{\pgfqpoint{0.774189in}{2.770323in}}%
\pgfpathlineto{\pgfqpoint{0.774189in}{2.770323in}}%
\pgfpathclose%
\pgfusepath{stroke,fill}%
\end{pgfscope}%
\begin{pgfscope}%
\pgfpathrectangle{\pgfqpoint{0.100000in}{0.183744in}}{\pgfqpoint{4.506048in}{4.506048in}}%
\pgfusepath{clip}%
\pgfsetbuttcap%
\pgfsetroundjoin%
\definecolor{currentfill}{rgb}{1.000000,0.647059,0.000000}%
\pgfsetfillcolor{currentfill}%
\pgfsetfillopacity{0.700000}%
\pgfsetlinewidth{1.003750pt}%
\definecolor{currentstroke}{rgb}{1.000000,0.647059,0.000000}%
\pgfsetstrokecolor{currentstroke}%
\pgfsetstrokeopacity{0.700000}%
\pgfsetdash{}{0pt}%
\pgfpathmoveto{\pgfqpoint{3.491922in}{2.887112in}}%
\pgfpathcurveto{\pgfqpoint{3.497746in}{2.887112in}}{\pgfqpoint{3.503332in}{2.889426in}}{\pgfqpoint{3.507450in}{2.893544in}}%
\pgfpathcurveto{\pgfqpoint{3.511568in}{2.897662in}}{\pgfqpoint{3.513882in}{2.903248in}}{\pgfqpoint{3.513882in}{2.909072in}}%
\pgfpathcurveto{\pgfqpoint{3.513882in}{2.914896in}}{\pgfqpoint{3.511568in}{2.920482in}}{\pgfqpoint{3.507450in}{2.924600in}}%
\pgfpathcurveto{\pgfqpoint{3.503332in}{2.928718in}}{\pgfqpoint{3.497746in}{2.931032in}}{\pgfqpoint{3.491922in}{2.931032in}}%
\pgfpathcurveto{\pgfqpoint{3.486098in}{2.931032in}}{\pgfqpoint{3.480512in}{2.928718in}}{\pgfqpoint{3.476393in}{2.924600in}}%
\pgfpathcurveto{\pgfqpoint{3.472275in}{2.920482in}}{\pgfqpoint{3.469961in}{2.914896in}}{\pgfqpoint{3.469961in}{2.909072in}}%
\pgfpathcurveto{\pgfqpoint{3.469961in}{2.903248in}}{\pgfqpoint{3.472275in}{2.897662in}}{\pgfqpoint{3.476393in}{2.893544in}}%
\pgfpathcurveto{\pgfqpoint{3.480512in}{2.889426in}}{\pgfqpoint{3.486098in}{2.887112in}}{\pgfqpoint{3.491922in}{2.887112in}}%
\pgfpathlineto{\pgfqpoint{3.491922in}{2.887112in}}%
\pgfpathclose%
\pgfusepath{stroke,fill}%
\end{pgfscope}%
\begin{pgfscope}%
\pgfpathrectangle{\pgfqpoint{0.100000in}{0.183744in}}{\pgfqpoint{4.506048in}{4.506048in}}%
\pgfusepath{clip}%
\pgfsetbuttcap%
\pgfsetroundjoin%
\definecolor{currentfill}{rgb}{1.000000,0.647059,0.000000}%
\pgfsetfillcolor{currentfill}%
\pgfsetfillopacity{0.700000}%
\pgfsetlinewidth{1.003750pt}%
\definecolor{currentstroke}{rgb}{1.000000,0.647059,0.000000}%
\pgfsetstrokecolor{currentstroke}%
\pgfsetstrokeopacity{0.700000}%
\pgfsetdash{}{0pt}%
\pgfpathmoveto{\pgfqpoint{3.386446in}{1.699516in}}%
\pgfpathcurveto{\pgfqpoint{3.392269in}{1.699516in}}{\pgfqpoint{3.397856in}{1.701830in}}{\pgfqpoint{3.401974in}{1.705948in}}%
\pgfpathcurveto{\pgfqpoint{3.406092in}{1.710066in}}{\pgfqpoint{3.408406in}{1.715652in}}{\pgfqpoint{3.408406in}{1.721476in}}%
\pgfpathcurveto{\pgfqpoint{3.408406in}{1.727300in}}{\pgfqpoint{3.406092in}{1.732886in}}{\pgfqpoint{3.401974in}{1.737004in}}%
\pgfpathcurveto{\pgfqpoint{3.397856in}{1.741123in}}{\pgfqpoint{3.392269in}{1.743436in}}{\pgfqpoint{3.386446in}{1.743436in}}%
\pgfpathcurveto{\pgfqpoint{3.380622in}{1.743436in}}{\pgfqpoint{3.375035in}{1.741123in}}{\pgfqpoint{3.370917in}{1.737004in}}%
\pgfpathcurveto{\pgfqpoint{3.366799in}{1.732886in}}{\pgfqpoint{3.364485in}{1.727300in}}{\pgfqpoint{3.364485in}{1.721476in}}%
\pgfpathcurveto{\pgfqpoint{3.364485in}{1.715652in}}{\pgfqpoint{3.366799in}{1.710066in}}{\pgfqpoint{3.370917in}{1.705948in}}%
\pgfpathcurveto{\pgfqpoint{3.375035in}{1.701830in}}{\pgfqpoint{3.380622in}{1.699516in}}{\pgfqpoint{3.386446in}{1.699516in}}%
\pgfpathlineto{\pgfqpoint{3.386446in}{1.699516in}}%
\pgfpathclose%
\pgfusepath{stroke,fill}%
\end{pgfscope}%
\begin{pgfscope}%
\pgfpathrectangle{\pgfqpoint{0.100000in}{0.183744in}}{\pgfqpoint{4.506048in}{4.506048in}}%
\pgfusepath{clip}%
\pgfsetbuttcap%
\pgfsetroundjoin%
\definecolor{currentfill}{rgb}{1.000000,0.647059,0.000000}%
\pgfsetfillcolor{currentfill}%
\pgfsetfillopacity{0.700000}%
\pgfsetlinewidth{1.003750pt}%
\definecolor{currentstroke}{rgb}{1.000000,0.647059,0.000000}%
\pgfsetstrokecolor{currentstroke}%
\pgfsetstrokeopacity{0.700000}%
\pgfsetdash{}{0pt}%
\pgfpathmoveto{\pgfqpoint{2.368239in}{3.263062in}}%
\pgfpathcurveto{\pgfqpoint{2.374063in}{3.263062in}}{\pgfqpoint{2.379649in}{3.265376in}}{\pgfqpoint{2.383767in}{3.269494in}}%
\pgfpathcurveto{\pgfqpoint{2.387886in}{3.273612in}}{\pgfqpoint{2.390199in}{3.279199in}}{\pgfqpoint{2.390199in}{3.285022in}}%
\pgfpathcurveto{\pgfqpoint{2.390199in}{3.290846in}}{\pgfqpoint{2.387886in}{3.296433in}}{\pgfqpoint{2.383767in}{3.300551in}}%
\pgfpathcurveto{\pgfqpoint{2.379649in}{3.304669in}}{\pgfqpoint{2.374063in}{3.306983in}}{\pgfqpoint{2.368239in}{3.306983in}}%
\pgfpathcurveto{\pgfqpoint{2.362415in}{3.306983in}}{\pgfqpoint{2.356829in}{3.304669in}}{\pgfqpoint{2.352711in}{3.300551in}}%
\pgfpathcurveto{\pgfqpoint{2.348593in}{3.296433in}}{\pgfqpoint{2.346279in}{3.290846in}}{\pgfqpoint{2.346279in}{3.285022in}}%
\pgfpathcurveto{\pgfqpoint{2.346279in}{3.279199in}}{\pgfqpoint{2.348593in}{3.273612in}}{\pgfqpoint{2.352711in}{3.269494in}}%
\pgfpathcurveto{\pgfqpoint{2.356829in}{3.265376in}}{\pgfqpoint{2.362415in}{3.263062in}}{\pgfqpoint{2.368239in}{3.263062in}}%
\pgfpathlineto{\pgfqpoint{2.368239in}{3.263062in}}%
\pgfpathclose%
\pgfusepath{stroke,fill}%
\end{pgfscope}%
\begin{pgfscope}%
\pgfpathrectangle{\pgfqpoint{0.100000in}{0.183744in}}{\pgfqpoint{4.506048in}{4.506048in}}%
\pgfusepath{clip}%
\pgfsetbuttcap%
\pgfsetroundjoin%
\definecolor{currentfill}{rgb}{1.000000,0.647059,0.000000}%
\pgfsetfillcolor{currentfill}%
\pgfsetfillopacity{0.700000}%
\pgfsetlinewidth{1.003750pt}%
\definecolor{currentstroke}{rgb}{1.000000,0.647059,0.000000}%
\pgfsetstrokecolor{currentstroke}%
\pgfsetstrokeopacity{0.700000}%
\pgfsetdash{}{0pt}%
\pgfpathmoveto{\pgfqpoint{3.308468in}{1.646862in}}%
\pgfpathcurveto{\pgfqpoint{3.314292in}{1.646862in}}{\pgfqpoint{3.319878in}{1.649176in}}{\pgfqpoint{3.323996in}{1.653294in}}%
\pgfpathcurveto{\pgfqpoint{3.328114in}{1.657412in}}{\pgfqpoint{3.330428in}{1.662999in}}{\pgfqpoint{3.330428in}{1.668822in}}%
\pgfpathcurveto{\pgfqpoint{3.330428in}{1.674646in}}{\pgfqpoint{3.328114in}{1.680233in}}{\pgfqpoint{3.323996in}{1.684351in}}%
\pgfpathcurveto{\pgfqpoint{3.319878in}{1.688469in}}{\pgfqpoint{3.314292in}{1.690783in}}{\pgfqpoint{3.308468in}{1.690783in}}%
\pgfpathcurveto{\pgfqpoint{3.302644in}{1.690783in}}{\pgfqpoint{3.297058in}{1.688469in}}{\pgfqpoint{3.292940in}{1.684351in}}%
\pgfpathcurveto{\pgfqpoint{3.288822in}{1.680233in}}{\pgfqpoint{3.286508in}{1.674646in}}{\pgfqpoint{3.286508in}{1.668822in}}%
\pgfpathcurveto{\pgfqpoint{3.286508in}{1.662999in}}{\pgfqpoint{3.288822in}{1.657412in}}{\pgfqpoint{3.292940in}{1.653294in}}%
\pgfpathcurveto{\pgfqpoint{3.297058in}{1.649176in}}{\pgfqpoint{3.302644in}{1.646862in}}{\pgfqpoint{3.308468in}{1.646862in}}%
\pgfpathlineto{\pgfqpoint{3.308468in}{1.646862in}}%
\pgfpathclose%
\pgfusepath{stroke,fill}%
\end{pgfscope}%
\begin{pgfscope}%
\pgfpathrectangle{\pgfqpoint{0.100000in}{0.183744in}}{\pgfqpoint{4.506048in}{4.506048in}}%
\pgfusepath{clip}%
\pgfsetbuttcap%
\pgfsetroundjoin%
\definecolor{currentfill}{rgb}{1.000000,0.647059,0.000000}%
\pgfsetfillcolor{currentfill}%
\pgfsetfillopacity{0.700000}%
\pgfsetlinewidth{1.003750pt}%
\definecolor{currentstroke}{rgb}{1.000000,0.647059,0.000000}%
\pgfsetstrokecolor{currentstroke}%
\pgfsetstrokeopacity{0.700000}%
\pgfsetdash{}{0pt}%
\pgfpathmoveto{\pgfqpoint{3.335200in}{3.167773in}}%
\pgfpathcurveto{\pgfqpoint{3.341024in}{3.167773in}}{\pgfqpoint{3.346610in}{3.170087in}}{\pgfqpoint{3.350728in}{3.174205in}}%
\pgfpathcurveto{\pgfqpoint{3.354846in}{3.178323in}}{\pgfqpoint{3.357160in}{3.183909in}}{\pgfqpoint{3.357160in}{3.189733in}}%
\pgfpathcurveto{\pgfqpoint{3.357160in}{3.195557in}}{\pgfqpoint{3.354846in}{3.201143in}}{\pgfqpoint{3.350728in}{3.205261in}}%
\pgfpathcurveto{\pgfqpoint{3.346610in}{3.209379in}}{\pgfqpoint{3.341024in}{3.211693in}}{\pgfqpoint{3.335200in}{3.211693in}}%
\pgfpathcurveto{\pgfqpoint{3.329376in}{3.211693in}}{\pgfqpoint{3.323790in}{3.209379in}}{\pgfqpoint{3.319672in}{3.205261in}}%
\pgfpathcurveto{\pgfqpoint{3.315553in}{3.201143in}}{\pgfqpoint{3.313240in}{3.195557in}}{\pgfqpoint{3.313240in}{3.189733in}}%
\pgfpathcurveto{\pgfqpoint{3.313240in}{3.183909in}}{\pgfqpoint{3.315553in}{3.178323in}}{\pgfqpoint{3.319672in}{3.174205in}}%
\pgfpathcurveto{\pgfqpoint{3.323790in}{3.170087in}}{\pgfqpoint{3.329376in}{3.167773in}}{\pgfqpoint{3.335200in}{3.167773in}}%
\pgfpathlineto{\pgfqpoint{3.335200in}{3.167773in}}%
\pgfpathclose%
\pgfusepath{stroke,fill}%
\end{pgfscope}%
\begin{pgfscope}%
\pgfpathrectangle{\pgfqpoint{0.100000in}{0.183744in}}{\pgfqpoint{4.506048in}{4.506048in}}%
\pgfusepath{clip}%
\pgfsetbuttcap%
\pgfsetroundjoin%
\definecolor{currentfill}{rgb}{1.000000,0.647059,0.000000}%
\pgfsetfillcolor{currentfill}%
\pgfsetfillopacity{0.700000}%
\pgfsetlinewidth{1.003750pt}%
\definecolor{currentstroke}{rgb}{1.000000,0.647059,0.000000}%
\pgfsetstrokecolor{currentstroke}%
\pgfsetstrokeopacity{0.700000}%
\pgfsetdash{}{0pt}%
\pgfpathmoveto{\pgfqpoint{2.798239in}{1.968522in}}%
\pgfpathcurveto{\pgfqpoint{2.804063in}{1.968522in}}{\pgfqpoint{2.809649in}{1.970836in}}{\pgfqpoint{2.813767in}{1.974954in}}%
\pgfpathcurveto{\pgfqpoint{2.817886in}{1.979072in}}{\pgfqpoint{2.820199in}{1.984658in}}{\pgfqpoint{2.820199in}{1.990482in}}%
\pgfpathcurveto{\pgfqpoint{2.820199in}{1.996306in}}{\pgfqpoint{2.817886in}{2.001893in}}{\pgfqpoint{2.813767in}{2.006011in}}%
\pgfpathcurveto{\pgfqpoint{2.809649in}{2.010129in}}{\pgfqpoint{2.804063in}{2.012443in}}{\pgfqpoint{2.798239in}{2.012443in}}%
\pgfpathcurveto{\pgfqpoint{2.792415in}{2.012443in}}{\pgfqpoint{2.786829in}{2.010129in}}{\pgfqpoint{2.782711in}{2.006011in}}%
\pgfpathcurveto{\pgfqpoint{2.778593in}{2.001893in}}{\pgfqpoint{2.776279in}{1.996306in}}{\pgfqpoint{2.776279in}{1.990482in}}%
\pgfpathcurveto{\pgfqpoint{2.776279in}{1.984658in}}{\pgfqpoint{2.778593in}{1.979072in}}{\pgfqpoint{2.782711in}{1.974954in}}%
\pgfpathcurveto{\pgfqpoint{2.786829in}{1.970836in}}{\pgfqpoint{2.792415in}{1.968522in}}{\pgfqpoint{2.798239in}{1.968522in}}%
\pgfpathlineto{\pgfqpoint{2.798239in}{1.968522in}}%
\pgfpathclose%
\pgfusepath{stroke,fill}%
\end{pgfscope}%
\begin{pgfscope}%
\pgfpathrectangle{\pgfqpoint{0.100000in}{0.183744in}}{\pgfqpoint{4.506048in}{4.506048in}}%
\pgfusepath{clip}%
\pgfsetbuttcap%
\pgfsetroundjoin%
\definecolor{currentfill}{rgb}{1.000000,0.647059,0.000000}%
\pgfsetfillcolor{currentfill}%
\pgfsetfillopacity{0.700000}%
\pgfsetlinewidth{1.003750pt}%
\definecolor{currentstroke}{rgb}{1.000000,0.647059,0.000000}%
\pgfsetstrokecolor{currentstroke}%
\pgfsetstrokeopacity{0.700000}%
\pgfsetdash{}{0pt}%
\pgfpathmoveto{\pgfqpoint{3.583217in}{1.499654in}}%
\pgfpathcurveto{\pgfqpoint{3.589041in}{1.499654in}}{\pgfqpoint{3.594627in}{1.501968in}}{\pgfqpoint{3.598745in}{1.506086in}}%
\pgfpathcurveto{\pgfqpoint{3.602864in}{1.510204in}}{\pgfqpoint{3.605177in}{1.515790in}}{\pgfqpoint{3.605177in}{1.521614in}}%
\pgfpathcurveto{\pgfqpoint{3.605177in}{1.527438in}}{\pgfqpoint{3.602864in}{1.533024in}}{\pgfqpoint{3.598745in}{1.537142in}}%
\pgfpathcurveto{\pgfqpoint{3.594627in}{1.541261in}}{\pgfqpoint{3.589041in}{1.543574in}}{\pgfqpoint{3.583217in}{1.543574in}}%
\pgfpathcurveto{\pgfqpoint{3.577393in}{1.543574in}}{\pgfqpoint{3.571807in}{1.541261in}}{\pgfqpoint{3.567689in}{1.537142in}}%
\pgfpathcurveto{\pgfqpoint{3.563571in}{1.533024in}}{\pgfqpoint{3.561257in}{1.527438in}}{\pgfqpoint{3.561257in}{1.521614in}}%
\pgfpathcurveto{\pgfqpoint{3.561257in}{1.515790in}}{\pgfqpoint{3.563571in}{1.510204in}}{\pgfqpoint{3.567689in}{1.506086in}}%
\pgfpathcurveto{\pgfqpoint{3.571807in}{1.501968in}}{\pgfqpoint{3.577393in}{1.499654in}}{\pgfqpoint{3.583217in}{1.499654in}}%
\pgfpathlineto{\pgfqpoint{3.583217in}{1.499654in}}%
\pgfpathclose%
\pgfusepath{stroke,fill}%
\end{pgfscope}%
\begin{pgfscope}%
\pgfpathrectangle{\pgfqpoint{0.100000in}{0.183744in}}{\pgfqpoint{4.506048in}{4.506048in}}%
\pgfusepath{clip}%
\pgfsetbuttcap%
\pgfsetroundjoin%
\definecolor{currentfill}{rgb}{1.000000,0.647059,0.000000}%
\pgfsetfillcolor{currentfill}%
\pgfsetfillopacity{0.700000}%
\pgfsetlinewidth{1.003750pt}%
\definecolor{currentstroke}{rgb}{1.000000,0.647059,0.000000}%
\pgfsetstrokecolor{currentstroke}%
\pgfsetstrokeopacity{0.700000}%
\pgfsetdash{}{0pt}%
\pgfpathmoveto{\pgfqpoint{3.164378in}{2.425768in}}%
\pgfpathcurveto{\pgfqpoint{3.170202in}{2.425768in}}{\pgfqpoint{3.175788in}{2.428082in}}{\pgfqpoint{3.179907in}{2.432200in}}%
\pgfpathcurveto{\pgfqpoint{3.184025in}{2.436318in}}{\pgfqpoint{3.186339in}{2.441904in}}{\pgfqpoint{3.186339in}{2.447728in}}%
\pgfpathcurveto{\pgfqpoint{3.186339in}{2.453552in}}{\pgfqpoint{3.184025in}{2.459138in}}{\pgfqpoint{3.179907in}{2.463256in}}%
\pgfpathcurveto{\pgfqpoint{3.175788in}{2.467374in}}{\pgfqpoint{3.170202in}{2.469688in}}{\pgfqpoint{3.164378in}{2.469688in}}%
\pgfpathcurveto{\pgfqpoint{3.158554in}{2.469688in}}{\pgfqpoint{3.152968in}{2.467374in}}{\pgfqpoint{3.148850in}{2.463256in}}%
\pgfpathcurveto{\pgfqpoint{3.144732in}{2.459138in}}{\pgfqpoint{3.142418in}{2.453552in}}{\pgfqpoint{3.142418in}{2.447728in}}%
\pgfpathcurveto{\pgfqpoint{3.142418in}{2.441904in}}{\pgfqpoint{3.144732in}{2.436318in}}{\pgfqpoint{3.148850in}{2.432200in}}%
\pgfpathcurveto{\pgfqpoint{3.152968in}{2.428082in}}{\pgfqpoint{3.158554in}{2.425768in}}{\pgfqpoint{3.164378in}{2.425768in}}%
\pgfpathlineto{\pgfqpoint{3.164378in}{2.425768in}}%
\pgfpathclose%
\pgfusepath{stroke,fill}%
\end{pgfscope}%
\begin{pgfscope}%
\pgfpathrectangle{\pgfqpoint{0.100000in}{0.183744in}}{\pgfqpoint{4.506048in}{4.506048in}}%
\pgfusepath{clip}%
\pgfsetbuttcap%
\pgfsetroundjoin%
\definecolor{currentfill}{rgb}{1.000000,0.647059,0.000000}%
\pgfsetfillcolor{currentfill}%
\pgfsetfillopacity{0.700000}%
\pgfsetlinewidth{1.003750pt}%
\definecolor{currentstroke}{rgb}{1.000000,0.647059,0.000000}%
\pgfsetstrokecolor{currentstroke}%
\pgfsetstrokeopacity{0.700000}%
\pgfsetdash{}{0pt}%
\pgfpathmoveto{\pgfqpoint{3.203926in}{3.179951in}}%
\pgfpathcurveto{\pgfqpoint{3.209750in}{3.179951in}}{\pgfqpoint{3.215336in}{3.182265in}}{\pgfqpoint{3.219454in}{3.186383in}}%
\pgfpathcurveto{\pgfqpoint{3.223572in}{3.190501in}}{\pgfqpoint{3.225886in}{3.196088in}}{\pgfqpoint{3.225886in}{3.201912in}}%
\pgfpathcurveto{\pgfqpoint{3.225886in}{3.207736in}}{\pgfqpoint{3.223572in}{3.213322in}}{\pgfqpoint{3.219454in}{3.217440in}}%
\pgfpathcurveto{\pgfqpoint{3.215336in}{3.221558in}}{\pgfqpoint{3.209750in}{3.223872in}}{\pgfqpoint{3.203926in}{3.223872in}}%
\pgfpathcurveto{\pgfqpoint{3.198102in}{3.223872in}}{\pgfqpoint{3.192516in}{3.221558in}}{\pgfqpoint{3.188398in}{3.217440in}}%
\pgfpathcurveto{\pgfqpoint{3.184280in}{3.213322in}}{\pgfqpoint{3.181966in}{3.207736in}}{\pgfqpoint{3.181966in}{3.201912in}}%
\pgfpathcurveto{\pgfqpoint{3.181966in}{3.196088in}}{\pgfqpoint{3.184280in}{3.190501in}}{\pgfqpoint{3.188398in}{3.186383in}}%
\pgfpathcurveto{\pgfqpoint{3.192516in}{3.182265in}}{\pgfqpoint{3.198102in}{3.179951in}}{\pgfqpoint{3.203926in}{3.179951in}}%
\pgfpathlineto{\pgfqpoint{3.203926in}{3.179951in}}%
\pgfpathclose%
\pgfusepath{stroke,fill}%
\end{pgfscope}%
\begin{pgfscope}%
\pgfpathrectangle{\pgfqpoint{0.100000in}{0.183744in}}{\pgfqpoint{4.506048in}{4.506048in}}%
\pgfusepath{clip}%
\pgfsetbuttcap%
\pgfsetroundjoin%
\definecolor{currentfill}{rgb}{1.000000,0.647059,0.000000}%
\pgfsetfillcolor{currentfill}%
\pgfsetfillopacity{0.700000}%
\pgfsetlinewidth{1.003750pt}%
\definecolor{currentstroke}{rgb}{1.000000,0.647059,0.000000}%
\pgfsetstrokecolor{currentstroke}%
\pgfsetstrokeopacity{0.700000}%
\pgfsetdash{}{0pt}%
\pgfpathmoveto{\pgfqpoint{1.769623in}{3.199348in}}%
\pgfpathcurveto{\pgfqpoint{1.775447in}{3.199348in}}{\pgfqpoint{1.781033in}{3.201662in}}{\pgfqpoint{1.785151in}{3.205780in}}%
\pgfpathcurveto{\pgfqpoint{1.789269in}{3.209898in}}{\pgfqpoint{1.791583in}{3.215485in}}{\pgfqpoint{1.791583in}{3.221308in}}%
\pgfpathcurveto{\pgfqpoint{1.791583in}{3.227132in}}{\pgfqpoint{1.789269in}{3.232719in}}{\pgfqpoint{1.785151in}{3.236837in}}%
\pgfpathcurveto{\pgfqpoint{1.781033in}{3.240955in}}{\pgfqpoint{1.775447in}{3.243269in}}{\pgfqpoint{1.769623in}{3.243269in}}%
\pgfpathcurveto{\pgfqpoint{1.763799in}{3.243269in}}{\pgfqpoint{1.758213in}{3.240955in}}{\pgfqpoint{1.754095in}{3.236837in}}%
\pgfpathcurveto{\pgfqpoint{1.749976in}{3.232719in}}{\pgfqpoint{1.747663in}{3.227132in}}{\pgfqpoint{1.747663in}{3.221308in}}%
\pgfpathcurveto{\pgfqpoint{1.747663in}{3.215485in}}{\pgfqpoint{1.749976in}{3.209898in}}{\pgfqpoint{1.754095in}{3.205780in}}%
\pgfpathcurveto{\pgfqpoint{1.758213in}{3.201662in}}{\pgfqpoint{1.763799in}{3.199348in}}{\pgfqpoint{1.769623in}{3.199348in}}%
\pgfpathlineto{\pgfqpoint{1.769623in}{3.199348in}}%
\pgfpathclose%
\pgfusepath{stroke,fill}%
\end{pgfscope}%
\begin{pgfscope}%
\pgfpathrectangle{\pgfqpoint{0.100000in}{0.183744in}}{\pgfqpoint{4.506048in}{4.506048in}}%
\pgfusepath{clip}%
\pgfsetbuttcap%
\pgfsetroundjoin%
\definecolor{currentfill}{rgb}{1.000000,0.647059,0.000000}%
\pgfsetfillcolor{currentfill}%
\pgfsetfillopacity{0.700000}%
\pgfsetlinewidth{1.003750pt}%
\definecolor{currentstroke}{rgb}{1.000000,0.647059,0.000000}%
\pgfsetstrokecolor{currentstroke}%
\pgfsetstrokeopacity{0.700000}%
\pgfsetdash{}{0pt}%
\pgfpathmoveto{\pgfqpoint{1.916740in}{0.433494in}}%
\pgfpathcurveto{\pgfqpoint{1.922564in}{0.433494in}}{\pgfqpoint{1.928150in}{0.435808in}}{\pgfqpoint{1.932268in}{0.439926in}}%
\pgfpathcurveto{\pgfqpoint{1.936386in}{0.444044in}}{\pgfqpoint{1.938700in}{0.449631in}}{\pgfqpoint{1.938700in}{0.455455in}}%
\pgfpathcurveto{\pgfqpoint{1.938700in}{0.461278in}}{\pgfqpoint{1.936386in}{0.466865in}}{\pgfqpoint{1.932268in}{0.470983in}}%
\pgfpathcurveto{\pgfqpoint{1.928150in}{0.475101in}}{\pgfqpoint{1.922564in}{0.477415in}}{\pgfqpoint{1.916740in}{0.477415in}}%
\pgfpathcurveto{\pgfqpoint{1.910916in}{0.477415in}}{\pgfqpoint{1.905330in}{0.475101in}}{\pgfqpoint{1.901212in}{0.470983in}}%
\pgfpathcurveto{\pgfqpoint{1.897094in}{0.466865in}}{\pgfqpoint{1.894780in}{0.461278in}}{\pgfqpoint{1.894780in}{0.455455in}}%
\pgfpathcurveto{\pgfqpoint{1.894780in}{0.449631in}}{\pgfqpoint{1.897094in}{0.444044in}}{\pgfqpoint{1.901212in}{0.439926in}}%
\pgfpathcurveto{\pgfqpoint{1.905330in}{0.435808in}}{\pgfqpoint{1.910916in}{0.433494in}}{\pgfqpoint{1.916740in}{0.433494in}}%
\pgfpathlineto{\pgfqpoint{1.916740in}{0.433494in}}%
\pgfpathclose%
\pgfusepath{stroke,fill}%
\end{pgfscope}%
\begin{pgfscope}%
\pgfpathrectangle{\pgfqpoint{0.100000in}{0.183744in}}{\pgfqpoint{4.506048in}{4.506048in}}%
\pgfusepath{clip}%
\pgfsetbuttcap%
\pgfsetroundjoin%
\definecolor{currentfill}{rgb}{1.000000,0.647059,0.000000}%
\pgfsetfillcolor{currentfill}%
\pgfsetfillopacity{0.700000}%
\pgfsetlinewidth{1.003750pt}%
\definecolor{currentstroke}{rgb}{1.000000,0.647059,0.000000}%
\pgfsetstrokecolor{currentstroke}%
\pgfsetstrokeopacity{0.700000}%
\pgfsetdash{}{0pt}%
\pgfpathmoveto{\pgfqpoint{0.465342in}{3.193340in}}%
\pgfpathcurveto{\pgfqpoint{0.471166in}{3.193340in}}{\pgfqpoint{0.476752in}{3.195654in}}{\pgfqpoint{0.480870in}{3.199772in}}%
\pgfpathcurveto{\pgfqpoint{0.484988in}{3.203891in}}{\pgfqpoint{0.487302in}{3.209477in}}{\pgfqpoint{0.487302in}{3.215301in}}%
\pgfpathcurveto{\pgfqpoint{0.487302in}{3.221125in}}{\pgfqpoint{0.484988in}{3.226711in}}{\pgfqpoint{0.480870in}{3.230829in}}%
\pgfpathcurveto{\pgfqpoint{0.476752in}{3.234947in}}{\pgfqpoint{0.471166in}{3.237261in}}{\pgfqpoint{0.465342in}{3.237261in}}%
\pgfpathcurveto{\pgfqpoint{0.459518in}{3.237261in}}{\pgfqpoint{0.453932in}{3.234947in}}{\pgfqpoint{0.449814in}{3.230829in}}%
\pgfpathcurveto{\pgfqpoint{0.445695in}{3.226711in}}{\pgfqpoint{0.443382in}{3.221125in}}{\pgfqpoint{0.443382in}{3.215301in}}%
\pgfpathcurveto{\pgfqpoint{0.443382in}{3.209477in}}{\pgfqpoint{0.445695in}{3.203891in}}{\pgfqpoint{0.449814in}{3.199772in}}%
\pgfpathcurveto{\pgfqpoint{0.453932in}{3.195654in}}{\pgfqpoint{0.459518in}{3.193340in}}{\pgfqpoint{0.465342in}{3.193340in}}%
\pgfpathlineto{\pgfqpoint{0.465342in}{3.193340in}}%
\pgfpathclose%
\pgfusepath{stroke,fill}%
\end{pgfscope}%
\begin{pgfscope}%
\pgfpathrectangle{\pgfqpoint{0.100000in}{0.183744in}}{\pgfqpoint{4.506048in}{4.506048in}}%
\pgfusepath{clip}%
\pgfsetbuttcap%
\pgfsetroundjoin%
\definecolor{currentfill}{rgb}{1.000000,0.647059,0.000000}%
\pgfsetfillcolor{currentfill}%
\pgfsetfillopacity{0.700000}%
\pgfsetlinewidth{1.003750pt}%
\definecolor{currentstroke}{rgb}{1.000000,0.647059,0.000000}%
\pgfsetstrokecolor{currentstroke}%
\pgfsetstrokeopacity{0.700000}%
\pgfsetdash{}{0pt}%
\pgfpathmoveto{\pgfqpoint{3.387532in}{2.651147in}}%
\pgfpathcurveto{\pgfqpoint{3.393355in}{2.651147in}}{\pgfqpoint{3.398942in}{2.653460in}}{\pgfqpoint{3.403060in}{2.657579in}}%
\pgfpathcurveto{\pgfqpoint{3.407178in}{2.661697in}}{\pgfqpoint{3.409492in}{2.667283in}}{\pgfqpoint{3.409492in}{2.673107in}}%
\pgfpathcurveto{\pgfqpoint{3.409492in}{2.678931in}}{\pgfqpoint{3.407178in}{2.684517in}}{\pgfqpoint{3.403060in}{2.688635in}}%
\pgfpathcurveto{\pgfqpoint{3.398942in}{2.692753in}}{\pgfqpoint{3.393355in}{2.695067in}}{\pgfqpoint{3.387532in}{2.695067in}}%
\pgfpathcurveto{\pgfqpoint{3.381708in}{2.695067in}}{\pgfqpoint{3.376121in}{2.692753in}}{\pgfqpoint{3.372003in}{2.688635in}}%
\pgfpathcurveto{\pgfqpoint{3.367885in}{2.684517in}}{\pgfqpoint{3.365571in}{2.678931in}}{\pgfqpoint{3.365571in}{2.673107in}}%
\pgfpathcurveto{\pgfqpoint{3.365571in}{2.667283in}}{\pgfqpoint{3.367885in}{2.661697in}}{\pgfqpoint{3.372003in}{2.657579in}}%
\pgfpathcurveto{\pgfqpoint{3.376121in}{2.653460in}}{\pgfqpoint{3.381708in}{2.651147in}}{\pgfqpoint{3.387532in}{2.651147in}}%
\pgfpathlineto{\pgfqpoint{3.387532in}{2.651147in}}%
\pgfpathclose%
\pgfusepath{stroke,fill}%
\end{pgfscope}%
\begin{pgfscope}%
\pgfpathrectangle{\pgfqpoint{0.100000in}{0.183744in}}{\pgfqpoint{4.506048in}{4.506048in}}%
\pgfusepath{clip}%
\pgfsetbuttcap%
\pgfsetroundjoin%
\definecolor{currentfill}{rgb}{1.000000,0.647059,0.000000}%
\pgfsetfillcolor{currentfill}%
\pgfsetfillopacity{0.700000}%
\pgfsetlinewidth{1.003750pt}%
\definecolor{currentstroke}{rgb}{1.000000,0.647059,0.000000}%
\pgfsetstrokecolor{currentstroke}%
\pgfsetstrokeopacity{0.700000}%
\pgfsetdash{}{0pt}%
\pgfpathmoveto{\pgfqpoint{2.772854in}{2.575895in}}%
\pgfpathcurveto{\pgfqpoint{2.778678in}{2.575895in}}{\pgfqpoint{2.784264in}{2.578209in}}{\pgfqpoint{2.788382in}{2.582327in}}%
\pgfpathcurveto{\pgfqpoint{2.792501in}{2.586445in}}{\pgfqpoint{2.794814in}{2.592032in}}{\pgfqpoint{2.794814in}{2.597856in}}%
\pgfpathcurveto{\pgfqpoint{2.794814in}{2.603680in}}{\pgfqpoint{2.792501in}{2.609266in}}{\pgfqpoint{2.788382in}{2.613384in}}%
\pgfpathcurveto{\pgfqpoint{2.784264in}{2.617502in}}{\pgfqpoint{2.778678in}{2.619816in}}{\pgfqpoint{2.772854in}{2.619816in}}%
\pgfpathcurveto{\pgfqpoint{2.767030in}{2.619816in}}{\pgfqpoint{2.761444in}{2.617502in}}{\pgfqpoint{2.757326in}{2.613384in}}%
\pgfpathcurveto{\pgfqpoint{2.753208in}{2.609266in}}{\pgfqpoint{2.750894in}{2.603680in}}{\pgfqpoint{2.750894in}{2.597856in}}%
\pgfpathcurveto{\pgfqpoint{2.750894in}{2.592032in}}{\pgfqpoint{2.753208in}{2.586445in}}{\pgfqpoint{2.757326in}{2.582327in}}%
\pgfpathcurveto{\pgfqpoint{2.761444in}{2.578209in}}{\pgfqpoint{2.767030in}{2.575895in}}{\pgfqpoint{2.772854in}{2.575895in}}%
\pgfpathlineto{\pgfqpoint{2.772854in}{2.575895in}}%
\pgfpathclose%
\pgfusepath{stroke,fill}%
\end{pgfscope}%
\begin{pgfscope}%
\pgfpathrectangle{\pgfqpoint{0.100000in}{0.183744in}}{\pgfqpoint{4.506048in}{4.506048in}}%
\pgfusepath{clip}%
\pgfsetbuttcap%
\pgfsetroundjoin%
\definecolor{currentfill}{rgb}{1.000000,0.647059,0.000000}%
\pgfsetfillcolor{currentfill}%
\pgfsetfillopacity{0.700000}%
\pgfsetlinewidth{1.003750pt}%
\definecolor{currentstroke}{rgb}{1.000000,0.647059,0.000000}%
\pgfsetstrokecolor{currentstroke}%
\pgfsetstrokeopacity{0.700000}%
\pgfsetdash{}{0pt}%
\pgfpathmoveto{\pgfqpoint{1.958619in}{1.510166in}}%
\pgfpathcurveto{\pgfqpoint{1.964443in}{1.510166in}}{\pgfqpoint{1.970029in}{1.512480in}}{\pgfqpoint{1.974147in}{1.516598in}}%
\pgfpathcurveto{\pgfqpoint{1.978265in}{1.520717in}}{\pgfqpoint{1.980579in}{1.526303in}}{\pgfqpoint{1.980579in}{1.532127in}}%
\pgfpathcurveto{\pgfqpoint{1.980579in}{1.537951in}}{\pgfqpoint{1.978265in}{1.543537in}}{\pgfqpoint{1.974147in}{1.547655in}}%
\pgfpathcurveto{\pgfqpoint{1.970029in}{1.551773in}}{\pgfqpoint{1.964443in}{1.554087in}}{\pgfqpoint{1.958619in}{1.554087in}}%
\pgfpathcurveto{\pgfqpoint{1.952795in}{1.554087in}}{\pgfqpoint{1.947209in}{1.551773in}}{\pgfqpoint{1.943091in}{1.547655in}}%
\pgfpathcurveto{\pgfqpoint{1.938973in}{1.543537in}}{\pgfqpoint{1.936659in}{1.537951in}}{\pgfqpoint{1.936659in}{1.532127in}}%
\pgfpathcurveto{\pgfqpoint{1.936659in}{1.526303in}}{\pgfqpoint{1.938973in}{1.520717in}}{\pgfqpoint{1.943091in}{1.516598in}}%
\pgfpathcurveto{\pgfqpoint{1.947209in}{1.512480in}}{\pgfqpoint{1.952795in}{1.510166in}}{\pgfqpoint{1.958619in}{1.510166in}}%
\pgfpathlineto{\pgfqpoint{1.958619in}{1.510166in}}%
\pgfpathclose%
\pgfusepath{stroke,fill}%
\end{pgfscope}%
\begin{pgfscope}%
\pgfpathrectangle{\pgfqpoint{0.100000in}{0.183744in}}{\pgfqpoint{4.506048in}{4.506048in}}%
\pgfusepath{clip}%
\pgfsetbuttcap%
\pgfsetroundjoin%
\definecolor{currentfill}{rgb}{1.000000,0.647059,0.000000}%
\pgfsetfillcolor{currentfill}%
\pgfsetfillopacity{0.700000}%
\pgfsetlinewidth{1.003750pt}%
\definecolor{currentstroke}{rgb}{1.000000,0.647059,0.000000}%
\pgfsetstrokecolor{currentstroke}%
\pgfsetstrokeopacity{0.700000}%
\pgfsetdash{}{0pt}%
\pgfpathmoveto{\pgfqpoint{1.361699in}{1.737493in}}%
\pgfpathcurveto{\pgfqpoint{1.367523in}{1.737493in}}{\pgfqpoint{1.373109in}{1.739807in}}{\pgfqpoint{1.377227in}{1.743925in}}%
\pgfpathcurveto{\pgfqpoint{1.381345in}{1.748043in}}{\pgfqpoint{1.383659in}{1.753629in}}{\pgfqpoint{1.383659in}{1.759453in}}%
\pgfpathcurveto{\pgfqpoint{1.383659in}{1.765277in}}{\pgfqpoint{1.381345in}{1.770863in}}{\pgfqpoint{1.377227in}{1.774982in}}%
\pgfpathcurveto{\pgfqpoint{1.373109in}{1.779100in}}{\pgfqpoint{1.367523in}{1.781414in}}{\pgfqpoint{1.361699in}{1.781414in}}%
\pgfpathcurveto{\pgfqpoint{1.355875in}{1.781414in}}{\pgfqpoint{1.350289in}{1.779100in}}{\pgfqpoint{1.346171in}{1.774982in}}%
\pgfpathcurveto{\pgfqpoint{1.342053in}{1.770863in}}{\pgfqpoint{1.339739in}{1.765277in}}{\pgfqpoint{1.339739in}{1.759453in}}%
\pgfpathcurveto{\pgfqpoint{1.339739in}{1.753629in}}{\pgfqpoint{1.342053in}{1.748043in}}{\pgfqpoint{1.346171in}{1.743925in}}%
\pgfpathcurveto{\pgfqpoint{1.350289in}{1.739807in}}{\pgfqpoint{1.355875in}{1.737493in}}{\pgfqpoint{1.361699in}{1.737493in}}%
\pgfpathlineto{\pgfqpoint{1.361699in}{1.737493in}}%
\pgfpathclose%
\pgfusepath{stroke,fill}%
\end{pgfscope}%
\begin{pgfscope}%
\pgfpathrectangle{\pgfqpoint{0.100000in}{0.183744in}}{\pgfqpoint{4.506048in}{4.506048in}}%
\pgfusepath{clip}%
\pgfsetbuttcap%
\pgfsetroundjoin%
\definecolor{currentfill}{rgb}{1.000000,0.647059,0.000000}%
\pgfsetfillcolor{currentfill}%
\pgfsetfillopacity{0.700000}%
\pgfsetlinewidth{1.003750pt}%
\definecolor{currentstroke}{rgb}{1.000000,0.647059,0.000000}%
\pgfsetstrokecolor{currentstroke}%
\pgfsetstrokeopacity{0.700000}%
\pgfsetdash{}{0pt}%
\pgfpathmoveto{\pgfqpoint{1.414444in}{2.000570in}}%
\pgfpathcurveto{\pgfqpoint{1.420268in}{2.000570in}}{\pgfqpoint{1.425854in}{2.002884in}}{\pgfqpoint{1.429972in}{2.007002in}}%
\pgfpathcurveto{\pgfqpoint{1.434090in}{2.011120in}}{\pgfqpoint{1.436404in}{2.016706in}}{\pgfqpoint{1.436404in}{2.022530in}}%
\pgfpathcurveto{\pgfqpoint{1.436404in}{2.028354in}}{\pgfqpoint{1.434090in}{2.033940in}}{\pgfqpoint{1.429972in}{2.038058in}}%
\pgfpathcurveto{\pgfqpoint{1.425854in}{2.042176in}}{\pgfqpoint{1.420268in}{2.044490in}}{\pgfqpoint{1.414444in}{2.044490in}}%
\pgfpathcurveto{\pgfqpoint{1.408620in}{2.044490in}}{\pgfqpoint{1.403034in}{2.042176in}}{\pgfqpoint{1.398916in}{2.038058in}}%
\pgfpathcurveto{\pgfqpoint{1.394798in}{2.033940in}}{\pgfqpoint{1.392484in}{2.028354in}}{\pgfqpoint{1.392484in}{2.022530in}}%
\pgfpathcurveto{\pgfqpoint{1.392484in}{2.016706in}}{\pgfqpoint{1.394798in}{2.011120in}}{\pgfqpoint{1.398916in}{2.007002in}}%
\pgfpathcurveto{\pgfqpoint{1.403034in}{2.002884in}}{\pgfqpoint{1.408620in}{2.000570in}}{\pgfqpoint{1.414444in}{2.000570in}}%
\pgfpathlineto{\pgfqpoint{1.414444in}{2.000570in}}%
\pgfpathclose%
\pgfusepath{stroke,fill}%
\end{pgfscope}%
\begin{pgfscope}%
\pgfpathrectangle{\pgfqpoint{0.100000in}{0.183744in}}{\pgfqpoint{4.506048in}{4.506048in}}%
\pgfusepath{clip}%
\pgfsetbuttcap%
\pgfsetroundjoin%
\definecolor{currentfill}{rgb}{1.000000,0.647059,0.000000}%
\pgfsetfillcolor{currentfill}%
\pgfsetfillopacity{0.700000}%
\pgfsetlinewidth{1.003750pt}%
\definecolor{currentstroke}{rgb}{1.000000,0.647059,0.000000}%
\pgfsetstrokecolor{currentstroke}%
\pgfsetstrokeopacity{0.700000}%
\pgfsetdash{}{0pt}%
\pgfpathmoveto{\pgfqpoint{3.327020in}{1.703849in}}%
\pgfpathcurveto{\pgfqpoint{3.332844in}{1.703849in}}{\pgfqpoint{3.338431in}{1.706163in}}{\pgfqpoint{3.342549in}{1.710281in}}%
\pgfpathcurveto{\pgfqpoint{3.346667in}{1.714399in}}{\pgfqpoint{3.348981in}{1.719985in}}{\pgfqpoint{3.348981in}{1.725809in}}%
\pgfpathcurveto{\pgfqpoint{3.348981in}{1.731633in}}{\pgfqpoint{3.346667in}{1.737219in}}{\pgfqpoint{3.342549in}{1.741338in}}%
\pgfpathcurveto{\pgfqpoint{3.338431in}{1.745456in}}{\pgfqpoint{3.332844in}{1.747770in}}{\pgfqpoint{3.327020in}{1.747770in}}%
\pgfpathcurveto{\pgfqpoint{3.321197in}{1.747770in}}{\pgfqpoint{3.315610in}{1.745456in}}{\pgfqpoint{3.311492in}{1.741338in}}%
\pgfpathcurveto{\pgfqpoint{3.307374in}{1.737219in}}{\pgfqpoint{3.305060in}{1.731633in}}{\pgfqpoint{3.305060in}{1.725809in}}%
\pgfpathcurveto{\pgfqpoint{3.305060in}{1.719985in}}{\pgfqpoint{3.307374in}{1.714399in}}{\pgfqpoint{3.311492in}{1.710281in}}%
\pgfpathcurveto{\pgfqpoint{3.315610in}{1.706163in}}{\pgfqpoint{3.321197in}{1.703849in}}{\pgfqpoint{3.327020in}{1.703849in}}%
\pgfpathlineto{\pgfqpoint{3.327020in}{1.703849in}}%
\pgfpathclose%
\pgfusepath{stroke,fill}%
\end{pgfscope}%
\begin{pgfscope}%
\pgfpathrectangle{\pgfqpoint{0.100000in}{0.183744in}}{\pgfqpoint{4.506048in}{4.506048in}}%
\pgfusepath{clip}%
\pgfsetbuttcap%
\pgfsetroundjoin%
\definecolor{currentfill}{rgb}{1.000000,0.647059,0.000000}%
\pgfsetfillcolor{currentfill}%
\pgfsetfillopacity{0.700000}%
\pgfsetlinewidth{1.003750pt}%
\definecolor{currentstroke}{rgb}{1.000000,0.647059,0.000000}%
\pgfsetstrokecolor{currentstroke}%
\pgfsetstrokeopacity{0.700000}%
\pgfsetdash{}{0pt}%
\pgfpathmoveto{\pgfqpoint{2.892803in}{1.654432in}}%
\pgfpathcurveto{\pgfqpoint{2.898627in}{1.654432in}}{\pgfqpoint{2.904213in}{1.656746in}}{\pgfqpoint{2.908331in}{1.660864in}}%
\pgfpathcurveto{\pgfqpoint{2.912449in}{1.664982in}}{\pgfqpoint{2.914763in}{1.670568in}}{\pgfqpoint{2.914763in}{1.676392in}}%
\pgfpathcurveto{\pgfqpoint{2.914763in}{1.682216in}}{\pgfqpoint{2.912449in}{1.687803in}}{\pgfqpoint{2.908331in}{1.691921in}}%
\pgfpathcurveto{\pgfqpoint{2.904213in}{1.696039in}}{\pgfqpoint{2.898627in}{1.698353in}}{\pgfqpoint{2.892803in}{1.698353in}}%
\pgfpathcurveto{\pgfqpoint{2.886979in}{1.698353in}}{\pgfqpoint{2.881393in}{1.696039in}}{\pgfqpoint{2.877275in}{1.691921in}}%
\pgfpathcurveto{\pgfqpoint{2.873157in}{1.687803in}}{\pgfqpoint{2.870843in}{1.682216in}}{\pgfqpoint{2.870843in}{1.676392in}}%
\pgfpathcurveto{\pgfqpoint{2.870843in}{1.670568in}}{\pgfqpoint{2.873157in}{1.664982in}}{\pgfqpoint{2.877275in}{1.660864in}}%
\pgfpathcurveto{\pgfqpoint{2.881393in}{1.656746in}}{\pgfqpoint{2.886979in}{1.654432in}}{\pgfqpoint{2.892803in}{1.654432in}}%
\pgfpathlineto{\pgfqpoint{2.892803in}{1.654432in}}%
\pgfpathclose%
\pgfusepath{stroke,fill}%
\end{pgfscope}%
\begin{pgfscope}%
\pgfpathrectangle{\pgfqpoint{0.100000in}{0.183744in}}{\pgfqpoint{4.506048in}{4.506048in}}%
\pgfusepath{clip}%
\pgfsetbuttcap%
\pgfsetroundjoin%
\definecolor{currentfill}{rgb}{1.000000,0.647059,0.000000}%
\pgfsetfillcolor{currentfill}%
\pgfsetfillopacity{0.700000}%
\pgfsetlinewidth{1.003750pt}%
\definecolor{currentstroke}{rgb}{1.000000,0.647059,0.000000}%
\pgfsetstrokecolor{currentstroke}%
\pgfsetstrokeopacity{0.700000}%
\pgfsetdash{}{0pt}%
\pgfpathmoveto{\pgfqpoint{2.498497in}{3.028505in}}%
\pgfpathcurveto{\pgfqpoint{2.504321in}{3.028505in}}{\pgfqpoint{2.509907in}{3.030818in}}{\pgfqpoint{2.514025in}{3.034937in}}%
\pgfpathcurveto{\pgfqpoint{2.518143in}{3.039055in}}{\pgfqpoint{2.520457in}{3.044641in}}{\pgfqpoint{2.520457in}{3.050465in}}%
\pgfpathcurveto{\pgfqpoint{2.520457in}{3.056289in}}{\pgfqpoint{2.518143in}{3.061875in}}{\pgfqpoint{2.514025in}{3.065993in}}%
\pgfpathcurveto{\pgfqpoint{2.509907in}{3.070111in}}{\pgfqpoint{2.504321in}{3.072425in}}{\pgfqpoint{2.498497in}{3.072425in}}%
\pgfpathcurveto{\pgfqpoint{2.492673in}{3.072425in}}{\pgfqpoint{2.487087in}{3.070111in}}{\pgfqpoint{2.482968in}{3.065993in}}%
\pgfpathcurveto{\pgfqpoint{2.478850in}{3.061875in}}{\pgfqpoint{2.476536in}{3.056289in}}{\pgfqpoint{2.476536in}{3.050465in}}%
\pgfpathcurveto{\pgfqpoint{2.476536in}{3.044641in}}{\pgfqpoint{2.478850in}{3.039055in}}{\pgfqpoint{2.482968in}{3.034937in}}%
\pgfpathcurveto{\pgfqpoint{2.487087in}{3.030818in}}{\pgfqpoint{2.492673in}{3.028505in}}{\pgfqpoint{2.498497in}{3.028505in}}%
\pgfpathlineto{\pgfqpoint{2.498497in}{3.028505in}}%
\pgfpathclose%
\pgfusepath{stroke,fill}%
\end{pgfscope}%
\begin{pgfscope}%
\pgfpathrectangle{\pgfqpoint{0.100000in}{0.183744in}}{\pgfqpoint{4.506048in}{4.506048in}}%
\pgfusepath{clip}%
\pgfsetbuttcap%
\pgfsetroundjoin%
\definecolor{currentfill}{rgb}{1.000000,0.647059,0.000000}%
\pgfsetfillcolor{currentfill}%
\pgfsetfillopacity{0.700000}%
\pgfsetlinewidth{1.003750pt}%
\definecolor{currentstroke}{rgb}{1.000000,0.647059,0.000000}%
\pgfsetstrokecolor{currentstroke}%
\pgfsetstrokeopacity{0.700000}%
\pgfsetdash{}{0pt}%
\pgfpathmoveto{\pgfqpoint{3.018700in}{2.192983in}}%
\pgfpathcurveto{\pgfqpoint{3.024524in}{2.192983in}}{\pgfqpoint{3.030110in}{2.195297in}}{\pgfqpoint{3.034228in}{2.199415in}}%
\pgfpathcurveto{\pgfqpoint{3.038346in}{2.203534in}}{\pgfqpoint{3.040660in}{2.209120in}}{\pgfqpoint{3.040660in}{2.214944in}}%
\pgfpathcurveto{\pgfqpoint{3.040660in}{2.220768in}}{\pgfqpoint{3.038346in}{2.226354in}}{\pgfqpoint{3.034228in}{2.230472in}}%
\pgfpathcurveto{\pgfqpoint{3.030110in}{2.234590in}}{\pgfqpoint{3.024524in}{2.236904in}}{\pgfqpoint{3.018700in}{2.236904in}}%
\pgfpathcurveto{\pgfqpoint{3.012876in}{2.236904in}}{\pgfqpoint{3.007290in}{2.234590in}}{\pgfqpoint{3.003171in}{2.230472in}}%
\pgfpathcurveto{\pgfqpoint{2.999053in}{2.226354in}}{\pgfqpoint{2.996739in}{2.220768in}}{\pgfqpoint{2.996739in}{2.214944in}}%
\pgfpathcurveto{\pgfqpoint{2.996739in}{2.209120in}}{\pgfqpoint{2.999053in}{2.203534in}}{\pgfqpoint{3.003171in}{2.199415in}}%
\pgfpathcurveto{\pgfqpoint{3.007290in}{2.195297in}}{\pgfqpoint{3.012876in}{2.192983in}}{\pgfqpoint{3.018700in}{2.192983in}}%
\pgfpathlineto{\pgfqpoint{3.018700in}{2.192983in}}%
\pgfpathclose%
\pgfusepath{stroke,fill}%
\end{pgfscope}%
\begin{pgfscope}%
\pgfpathrectangle{\pgfqpoint{0.100000in}{0.183744in}}{\pgfqpoint{4.506048in}{4.506048in}}%
\pgfusepath{clip}%
\pgfsetbuttcap%
\pgfsetroundjoin%
\definecolor{currentfill}{rgb}{1.000000,0.647059,0.000000}%
\pgfsetfillcolor{currentfill}%
\pgfsetfillopacity{0.700000}%
\pgfsetlinewidth{1.003750pt}%
\definecolor{currentstroke}{rgb}{1.000000,0.647059,0.000000}%
\pgfsetstrokecolor{currentstroke}%
\pgfsetstrokeopacity{0.700000}%
\pgfsetdash{}{0pt}%
\pgfpathmoveto{\pgfqpoint{2.048105in}{3.045888in}}%
\pgfpathcurveto{\pgfqpoint{2.053929in}{3.045888in}}{\pgfqpoint{2.059515in}{3.048202in}}{\pgfqpoint{2.063633in}{3.052320in}}%
\pgfpathcurveto{\pgfqpoint{2.067752in}{3.056438in}}{\pgfqpoint{2.070065in}{3.062024in}}{\pgfqpoint{2.070065in}{3.067848in}}%
\pgfpathcurveto{\pgfqpoint{2.070065in}{3.073672in}}{\pgfqpoint{2.067752in}{3.079258in}}{\pgfqpoint{2.063633in}{3.083376in}}%
\pgfpathcurveto{\pgfqpoint{2.059515in}{3.087494in}}{\pgfqpoint{2.053929in}{3.089808in}}{\pgfqpoint{2.048105in}{3.089808in}}%
\pgfpathcurveto{\pgfqpoint{2.042281in}{3.089808in}}{\pgfqpoint{2.036695in}{3.087494in}}{\pgfqpoint{2.032577in}{3.083376in}}%
\pgfpathcurveto{\pgfqpoint{2.028459in}{3.079258in}}{\pgfqpoint{2.026145in}{3.073672in}}{\pgfqpoint{2.026145in}{3.067848in}}%
\pgfpathcurveto{\pgfqpoint{2.026145in}{3.062024in}}{\pgfqpoint{2.028459in}{3.056438in}}{\pgfqpoint{2.032577in}{3.052320in}}%
\pgfpathcurveto{\pgfqpoint{2.036695in}{3.048202in}}{\pgfqpoint{2.042281in}{3.045888in}}{\pgfqpoint{2.048105in}{3.045888in}}%
\pgfpathlineto{\pgfqpoint{2.048105in}{3.045888in}}%
\pgfpathclose%
\pgfusepath{stroke,fill}%
\end{pgfscope}%
\begin{pgfscope}%
\pgfpathrectangle{\pgfqpoint{0.100000in}{0.183744in}}{\pgfqpoint{4.506048in}{4.506048in}}%
\pgfusepath{clip}%
\pgfsetbuttcap%
\pgfsetroundjoin%
\definecolor{currentfill}{rgb}{1.000000,0.647059,0.000000}%
\pgfsetfillcolor{currentfill}%
\pgfsetfillopacity{0.700000}%
\pgfsetlinewidth{1.003750pt}%
\definecolor{currentstroke}{rgb}{1.000000,0.647059,0.000000}%
\pgfsetstrokecolor{currentstroke}%
\pgfsetstrokeopacity{0.700000}%
\pgfsetdash{}{0pt}%
\pgfpathmoveto{\pgfqpoint{1.943466in}{2.459290in}}%
\pgfpathcurveto{\pgfqpoint{1.949290in}{2.459290in}}{\pgfqpoint{1.954876in}{2.461604in}}{\pgfqpoint{1.958994in}{2.465722in}}%
\pgfpathcurveto{\pgfqpoint{1.963113in}{2.469840in}}{\pgfqpoint{1.965426in}{2.475427in}}{\pgfqpoint{1.965426in}{2.481251in}}%
\pgfpathcurveto{\pgfqpoint{1.965426in}{2.487074in}}{\pgfqpoint{1.963113in}{2.492661in}}{\pgfqpoint{1.958994in}{2.496779in}}%
\pgfpathcurveto{\pgfqpoint{1.954876in}{2.500897in}}{\pgfqpoint{1.949290in}{2.503211in}}{\pgfqpoint{1.943466in}{2.503211in}}%
\pgfpathcurveto{\pgfqpoint{1.937642in}{2.503211in}}{\pgfqpoint{1.932056in}{2.500897in}}{\pgfqpoint{1.927938in}{2.496779in}}%
\pgfpathcurveto{\pgfqpoint{1.923820in}{2.492661in}}{\pgfqpoint{1.921506in}{2.487074in}}{\pgfqpoint{1.921506in}{2.481251in}}%
\pgfpathcurveto{\pgfqpoint{1.921506in}{2.475427in}}{\pgfqpoint{1.923820in}{2.469840in}}{\pgfqpoint{1.927938in}{2.465722in}}%
\pgfpathcurveto{\pgfqpoint{1.932056in}{2.461604in}}{\pgfqpoint{1.937642in}{2.459290in}}{\pgfqpoint{1.943466in}{2.459290in}}%
\pgfpathlineto{\pgfqpoint{1.943466in}{2.459290in}}%
\pgfpathclose%
\pgfusepath{stroke,fill}%
\end{pgfscope}%
\begin{pgfscope}%
\pgfpathrectangle{\pgfqpoint{0.100000in}{0.183744in}}{\pgfqpoint{4.506048in}{4.506048in}}%
\pgfusepath{clip}%
\pgfsetbuttcap%
\pgfsetroundjoin%
\definecolor{currentfill}{rgb}{1.000000,0.647059,0.000000}%
\pgfsetfillcolor{currentfill}%
\pgfsetfillopacity{0.700000}%
\pgfsetlinewidth{1.003750pt}%
\definecolor{currentstroke}{rgb}{1.000000,0.647059,0.000000}%
\pgfsetstrokecolor{currentstroke}%
\pgfsetstrokeopacity{0.700000}%
\pgfsetdash{}{0pt}%
\pgfpathmoveto{\pgfqpoint{1.279741in}{1.908801in}}%
\pgfpathcurveto{\pgfqpoint{1.285565in}{1.908801in}}{\pgfqpoint{1.291151in}{1.911115in}}{\pgfqpoint{1.295269in}{1.915233in}}%
\pgfpathcurveto{\pgfqpoint{1.299388in}{1.919351in}}{\pgfqpoint{1.301701in}{1.924937in}}{\pgfqpoint{1.301701in}{1.930761in}}%
\pgfpathcurveto{\pgfqpoint{1.301701in}{1.936585in}}{\pgfqpoint{1.299388in}{1.942171in}}{\pgfqpoint{1.295269in}{1.946290in}}%
\pgfpathcurveto{\pgfqpoint{1.291151in}{1.950408in}}{\pgfqpoint{1.285565in}{1.952722in}}{\pgfqpoint{1.279741in}{1.952722in}}%
\pgfpathcurveto{\pgfqpoint{1.273917in}{1.952722in}}{\pgfqpoint{1.268331in}{1.950408in}}{\pgfqpoint{1.264213in}{1.946290in}}%
\pgfpathcurveto{\pgfqpoint{1.260095in}{1.942171in}}{\pgfqpoint{1.257781in}{1.936585in}}{\pgfqpoint{1.257781in}{1.930761in}}%
\pgfpathcurveto{\pgfqpoint{1.257781in}{1.924937in}}{\pgfqpoint{1.260095in}{1.919351in}}{\pgfqpoint{1.264213in}{1.915233in}}%
\pgfpathcurveto{\pgfqpoint{1.268331in}{1.911115in}}{\pgfqpoint{1.273917in}{1.908801in}}{\pgfqpoint{1.279741in}{1.908801in}}%
\pgfpathlineto{\pgfqpoint{1.279741in}{1.908801in}}%
\pgfpathclose%
\pgfusepath{stroke,fill}%
\end{pgfscope}%
\begin{pgfscope}%
\pgfpathrectangle{\pgfqpoint{0.100000in}{0.183744in}}{\pgfqpoint{4.506048in}{4.506048in}}%
\pgfusepath{clip}%
\pgfsetbuttcap%
\pgfsetroundjoin%
\definecolor{currentfill}{rgb}{1.000000,0.647059,0.000000}%
\pgfsetfillcolor{currentfill}%
\pgfsetfillopacity{0.700000}%
\pgfsetlinewidth{1.003750pt}%
\definecolor{currentstroke}{rgb}{1.000000,0.647059,0.000000}%
\pgfsetstrokecolor{currentstroke}%
\pgfsetstrokeopacity{0.700000}%
\pgfsetdash{}{0pt}%
\pgfpathmoveto{\pgfqpoint{2.492592in}{1.711250in}}%
\pgfpathcurveto{\pgfqpoint{2.498416in}{1.711250in}}{\pgfqpoint{2.504002in}{1.713564in}}{\pgfqpoint{2.508120in}{1.717682in}}%
\pgfpathcurveto{\pgfqpoint{2.512238in}{1.721800in}}{\pgfqpoint{2.514552in}{1.727386in}}{\pgfqpoint{2.514552in}{1.733210in}}%
\pgfpathcurveto{\pgfqpoint{2.514552in}{1.739034in}}{\pgfqpoint{2.512238in}{1.744620in}}{\pgfqpoint{2.508120in}{1.748738in}}%
\pgfpathcurveto{\pgfqpoint{2.504002in}{1.752856in}}{\pgfqpoint{2.498416in}{1.755170in}}{\pgfqpoint{2.492592in}{1.755170in}}%
\pgfpathcurveto{\pgfqpoint{2.486768in}{1.755170in}}{\pgfqpoint{2.481182in}{1.752856in}}{\pgfqpoint{2.477064in}{1.748738in}}%
\pgfpathcurveto{\pgfqpoint{2.472946in}{1.744620in}}{\pgfqpoint{2.470632in}{1.739034in}}{\pgfqpoint{2.470632in}{1.733210in}}%
\pgfpathcurveto{\pgfqpoint{2.470632in}{1.727386in}}{\pgfqpoint{2.472946in}{1.721800in}}{\pgfqpoint{2.477064in}{1.717682in}}%
\pgfpathcurveto{\pgfqpoint{2.481182in}{1.713564in}}{\pgfqpoint{2.486768in}{1.711250in}}{\pgfqpoint{2.492592in}{1.711250in}}%
\pgfpathlineto{\pgfqpoint{2.492592in}{1.711250in}}%
\pgfpathclose%
\pgfusepath{stroke,fill}%
\end{pgfscope}%
\begin{pgfscope}%
\pgfpathrectangle{\pgfqpoint{0.100000in}{0.183744in}}{\pgfqpoint{4.506048in}{4.506048in}}%
\pgfusepath{clip}%
\pgfsetbuttcap%
\pgfsetroundjoin%
\definecolor{currentfill}{rgb}{1.000000,0.647059,0.000000}%
\pgfsetfillcolor{currentfill}%
\pgfsetfillopacity{0.700000}%
\pgfsetlinewidth{1.003750pt}%
\definecolor{currentstroke}{rgb}{1.000000,0.647059,0.000000}%
\pgfsetstrokecolor{currentstroke}%
\pgfsetstrokeopacity{0.700000}%
\pgfsetdash{}{0pt}%
\pgfpathmoveto{\pgfqpoint{3.648325in}{3.072611in}}%
\pgfpathcurveto{\pgfqpoint{3.654149in}{3.072611in}}{\pgfqpoint{3.659736in}{3.074925in}}{\pgfqpoint{3.663854in}{3.079043in}}%
\pgfpathcurveto{\pgfqpoint{3.667972in}{3.083161in}}{\pgfqpoint{3.670286in}{3.088747in}}{\pgfqpoint{3.670286in}{3.094571in}}%
\pgfpathcurveto{\pgfqpoint{3.670286in}{3.100395in}}{\pgfqpoint{3.667972in}{3.105981in}}{\pgfqpoint{3.663854in}{3.110099in}}%
\pgfpathcurveto{\pgfqpoint{3.659736in}{3.114217in}}{\pgfqpoint{3.654149in}{3.116531in}}{\pgfqpoint{3.648325in}{3.116531in}}%
\pgfpathcurveto{\pgfqpoint{3.642502in}{3.116531in}}{\pgfqpoint{3.636915in}{3.114217in}}{\pgfqpoint{3.632797in}{3.110099in}}%
\pgfpathcurveto{\pgfqpoint{3.628679in}{3.105981in}}{\pgfqpoint{3.626365in}{3.100395in}}{\pgfqpoint{3.626365in}{3.094571in}}%
\pgfpathcurveto{\pgfqpoint{3.626365in}{3.088747in}}{\pgfqpoint{3.628679in}{3.083161in}}{\pgfqpoint{3.632797in}{3.079043in}}%
\pgfpathcurveto{\pgfqpoint{3.636915in}{3.074925in}}{\pgfqpoint{3.642502in}{3.072611in}}{\pgfqpoint{3.648325in}{3.072611in}}%
\pgfpathlineto{\pgfqpoint{3.648325in}{3.072611in}}%
\pgfpathclose%
\pgfusepath{stroke,fill}%
\end{pgfscope}%
\begin{pgfscope}%
\pgfpathrectangle{\pgfqpoint{0.100000in}{0.183744in}}{\pgfqpoint{4.506048in}{4.506048in}}%
\pgfusepath{clip}%
\pgfsetbuttcap%
\pgfsetroundjoin%
\definecolor{currentfill}{rgb}{1.000000,0.647059,0.000000}%
\pgfsetfillcolor{currentfill}%
\pgfsetfillopacity{0.700000}%
\pgfsetlinewidth{1.003750pt}%
\definecolor{currentstroke}{rgb}{1.000000,0.647059,0.000000}%
\pgfsetstrokecolor{currentstroke}%
\pgfsetstrokeopacity{0.700000}%
\pgfsetdash{}{0pt}%
\pgfpathmoveto{\pgfqpoint{3.467689in}{2.832280in}}%
\pgfpathcurveto{\pgfqpoint{3.473513in}{2.832280in}}{\pgfqpoint{3.479099in}{2.834594in}}{\pgfqpoint{3.483217in}{2.838712in}}%
\pgfpathcurveto{\pgfqpoint{3.487335in}{2.842830in}}{\pgfqpoint{3.489649in}{2.848416in}}{\pgfqpoint{3.489649in}{2.854240in}}%
\pgfpathcurveto{\pgfqpoint{3.489649in}{2.860064in}}{\pgfqpoint{3.487335in}{2.865650in}}{\pgfqpoint{3.483217in}{2.869769in}}%
\pgfpathcurveto{\pgfqpoint{3.479099in}{2.873887in}}{\pgfqpoint{3.473513in}{2.876201in}}{\pgfqpoint{3.467689in}{2.876201in}}%
\pgfpathcurveto{\pgfqpoint{3.461865in}{2.876201in}}{\pgfqpoint{3.456279in}{2.873887in}}{\pgfqpoint{3.452160in}{2.869769in}}%
\pgfpathcurveto{\pgfqpoint{3.448042in}{2.865650in}}{\pgfqpoint{3.445728in}{2.860064in}}{\pgfqpoint{3.445728in}{2.854240in}}%
\pgfpathcurveto{\pgfqpoint{3.445728in}{2.848416in}}{\pgfqpoint{3.448042in}{2.842830in}}{\pgfqpoint{3.452160in}{2.838712in}}%
\pgfpathcurveto{\pgfqpoint{3.456279in}{2.834594in}}{\pgfqpoint{3.461865in}{2.832280in}}{\pgfqpoint{3.467689in}{2.832280in}}%
\pgfpathlineto{\pgfqpoint{3.467689in}{2.832280in}}%
\pgfpathclose%
\pgfusepath{stroke,fill}%
\end{pgfscope}%
\begin{pgfscope}%
\pgfpathrectangle{\pgfqpoint{0.100000in}{0.183744in}}{\pgfqpoint{4.506048in}{4.506048in}}%
\pgfusepath{clip}%
\pgfsetbuttcap%
\pgfsetroundjoin%
\definecolor{currentfill}{rgb}{1.000000,0.647059,0.000000}%
\pgfsetfillcolor{currentfill}%
\pgfsetfillopacity{0.700000}%
\pgfsetlinewidth{1.003750pt}%
\definecolor{currentstroke}{rgb}{1.000000,0.647059,0.000000}%
\pgfsetstrokecolor{currentstroke}%
\pgfsetstrokeopacity{0.700000}%
\pgfsetdash{}{0pt}%
\pgfpathmoveto{\pgfqpoint{1.683668in}{1.886914in}}%
\pgfpathcurveto{\pgfqpoint{1.689492in}{1.886914in}}{\pgfqpoint{1.695078in}{1.889228in}}{\pgfqpoint{1.699197in}{1.893346in}}%
\pgfpathcurveto{\pgfqpoint{1.703315in}{1.897465in}}{\pgfqpoint{1.705629in}{1.903051in}}{\pgfqpoint{1.705629in}{1.908875in}}%
\pgfpathcurveto{\pgfqpoint{1.705629in}{1.914699in}}{\pgfqpoint{1.703315in}{1.920285in}}{\pgfqpoint{1.699197in}{1.924403in}}%
\pgfpathcurveto{\pgfqpoint{1.695078in}{1.928521in}}{\pgfqpoint{1.689492in}{1.930835in}}{\pgfqpoint{1.683668in}{1.930835in}}%
\pgfpathcurveto{\pgfqpoint{1.677844in}{1.930835in}}{\pgfqpoint{1.672258in}{1.928521in}}{\pgfqpoint{1.668140in}{1.924403in}}%
\pgfpathcurveto{\pgfqpoint{1.664022in}{1.920285in}}{\pgfqpoint{1.661708in}{1.914699in}}{\pgfqpoint{1.661708in}{1.908875in}}%
\pgfpathcurveto{\pgfqpoint{1.661708in}{1.903051in}}{\pgfqpoint{1.664022in}{1.897465in}}{\pgfqpoint{1.668140in}{1.893346in}}%
\pgfpathcurveto{\pgfqpoint{1.672258in}{1.889228in}}{\pgfqpoint{1.677844in}{1.886914in}}{\pgfqpoint{1.683668in}{1.886914in}}%
\pgfpathlineto{\pgfqpoint{1.683668in}{1.886914in}}%
\pgfpathclose%
\pgfusepath{stroke,fill}%
\end{pgfscope}%
\begin{pgfscope}%
\pgfpathrectangle{\pgfqpoint{0.100000in}{0.183744in}}{\pgfqpoint{4.506048in}{4.506048in}}%
\pgfusepath{clip}%
\pgfsetbuttcap%
\pgfsetroundjoin%
\definecolor{currentfill}{rgb}{1.000000,0.647059,0.000000}%
\pgfsetfillcolor{currentfill}%
\pgfsetfillopacity{0.700000}%
\pgfsetlinewidth{1.003750pt}%
\definecolor{currentstroke}{rgb}{1.000000,0.647059,0.000000}%
\pgfsetstrokecolor{currentstroke}%
\pgfsetstrokeopacity{0.700000}%
\pgfsetdash{}{0pt}%
\pgfpathmoveto{\pgfqpoint{0.614008in}{2.507350in}}%
\pgfpathcurveto{\pgfqpoint{0.619832in}{2.507350in}}{\pgfqpoint{0.625418in}{2.509664in}}{\pgfqpoint{0.629536in}{2.513782in}}%
\pgfpathcurveto{\pgfqpoint{0.633654in}{2.517900in}}{\pgfqpoint{0.635968in}{2.523486in}}{\pgfqpoint{0.635968in}{2.529310in}}%
\pgfpathcurveto{\pgfqpoint{0.635968in}{2.535134in}}{\pgfqpoint{0.633654in}{2.540720in}}{\pgfqpoint{0.629536in}{2.544839in}}%
\pgfpathcurveto{\pgfqpoint{0.625418in}{2.548957in}}{\pgfqpoint{0.619832in}{2.551271in}}{\pgfqpoint{0.614008in}{2.551271in}}%
\pgfpathcurveto{\pgfqpoint{0.608184in}{2.551271in}}{\pgfqpoint{0.602598in}{2.548957in}}{\pgfqpoint{0.598479in}{2.544839in}}%
\pgfpathcurveto{\pgfqpoint{0.594361in}{2.540720in}}{\pgfqpoint{0.592047in}{2.535134in}}{\pgfqpoint{0.592047in}{2.529310in}}%
\pgfpathcurveto{\pgfqpoint{0.592047in}{2.523486in}}{\pgfqpoint{0.594361in}{2.517900in}}{\pgfqpoint{0.598479in}{2.513782in}}%
\pgfpathcurveto{\pgfqpoint{0.602598in}{2.509664in}}{\pgfqpoint{0.608184in}{2.507350in}}{\pgfqpoint{0.614008in}{2.507350in}}%
\pgfpathlineto{\pgfqpoint{0.614008in}{2.507350in}}%
\pgfpathclose%
\pgfusepath{stroke,fill}%
\end{pgfscope}%
\begin{pgfscope}%
\pgfpathrectangle{\pgfqpoint{0.100000in}{0.183744in}}{\pgfqpoint{4.506048in}{4.506048in}}%
\pgfusepath{clip}%
\pgfsetbuttcap%
\pgfsetroundjoin%
\definecolor{currentfill}{rgb}{1.000000,0.647059,0.000000}%
\pgfsetfillcolor{currentfill}%
\pgfsetfillopacity{0.700000}%
\pgfsetlinewidth{1.003750pt}%
\definecolor{currentstroke}{rgb}{1.000000,0.647059,0.000000}%
\pgfsetstrokecolor{currentstroke}%
\pgfsetstrokeopacity{0.700000}%
\pgfsetdash{}{0pt}%
\pgfpathmoveto{\pgfqpoint{2.626807in}{0.846922in}}%
\pgfpathcurveto{\pgfqpoint{2.632631in}{0.846922in}}{\pgfqpoint{2.638217in}{0.849236in}}{\pgfqpoint{2.642335in}{0.853354in}}%
\pgfpathcurveto{\pgfqpoint{2.646453in}{0.857472in}}{\pgfqpoint{2.648767in}{0.863058in}}{\pgfqpoint{2.648767in}{0.868882in}}%
\pgfpathcurveto{\pgfqpoint{2.648767in}{0.874706in}}{\pgfqpoint{2.646453in}{0.880292in}}{\pgfqpoint{2.642335in}{0.884411in}}%
\pgfpathcurveto{\pgfqpoint{2.638217in}{0.888529in}}{\pgfqpoint{2.632631in}{0.890843in}}{\pgfqpoint{2.626807in}{0.890843in}}%
\pgfpathcurveto{\pgfqpoint{2.620983in}{0.890843in}}{\pgfqpoint{2.615396in}{0.888529in}}{\pgfqpoint{2.611278in}{0.884411in}}%
\pgfpathcurveto{\pgfqpoint{2.607160in}{0.880292in}}{\pgfqpoint{2.604846in}{0.874706in}}{\pgfqpoint{2.604846in}{0.868882in}}%
\pgfpathcurveto{\pgfqpoint{2.604846in}{0.863058in}}{\pgfqpoint{2.607160in}{0.857472in}}{\pgfqpoint{2.611278in}{0.853354in}}%
\pgfpathcurveto{\pgfqpoint{2.615396in}{0.849236in}}{\pgfqpoint{2.620983in}{0.846922in}}{\pgfqpoint{2.626807in}{0.846922in}}%
\pgfpathlineto{\pgfqpoint{2.626807in}{0.846922in}}%
\pgfpathclose%
\pgfusepath{stroke,fill}%
\end{pgfscope}%
\begin{pgfscope}%
\pgfpathrectangle{\pgfqpoint{0.100000in}{0.183744in}}{\pgfqpoint{4.506048in}{4.506048in}}%
\pgfusepath{clip}%
\pgfsetbuttcap%
\pgfsetroundjoin%
\definecolor{currentfill}{rgb}{1.000000,0.647059,0.000000}%
\pgfsetfillcolor{currentfill}%
\pgfsetfillopacity{0.700000}%
\pgfsetlinewidth{1.003750pt}%
\definecolor{currentstroke}{rgb}{1.000000,0.647059,0.000000}%
\pgfsetstrokecolor{currentstroke}%
\pgfsetstrokeopacity{0.700000}%
\pgfsetdash{}{0pt}%
\pgfpathmoveto{\pgfqpoint{3.814953in}{3.053882in}}%
\pgfpathcurveto{\pgfqpoint{3.820777in}{3.053882in}}{\pgfqpoint{3.826364in}{3.056195in}}{\pgfqpoint{3.830482in}{3.060314in}}%
\pgfpathcurveto{\pgfqpoint{3.834600in}{3.064432in}}{\pgfqpoint{3.836914in}{3.070018in}}{\pgfqpoint{3.836914in}{3.075842in}}%
\pgfpathcurveto{\pgfqpoint{3.836914in}{3.081666in}}{\pgfqpoint{3.834600in}{3.087252in}}{\pgfqpoint{3.830482in}{3.091370in}}%
\pgfpathcurveto{\pgfqpoint{3.826364in}{3.095488in}}{\pgfqpoint{3.820777in}{3.097802in}}{\pgfqpoint{3.814953in}{3.097802in}}%
\pgfpathcurveto{\pgfqpoint{3.809130in}{3.097802in}}{\pgfqpoint{3.803543in}{3.095488in}}{\pgfqpoint{3.799425in}{3.091370in}}%
\pgfpathcurveto{\pgfqpoint{3.795307in}{3.087252in}}{\pgfqpoint{3.792993in}{3.081666in}}{\pgfqpoint{3.792993in}{3.075842in}}%
\pgfpathcurveto{\pgfqpoint{3.792993in}{3.070018in}}{\pgfqpoint{3.795307in}{3.064432in}}{\pgfqpoint{3.799425in}{3.060314in}}%
\pgfpathcurveto{\pgfqpoint{3.803543in}{3.056195in}}{\pgfqpoint{3.809130in}{3.053882in}}{\pgfqpoint{3.814953in}{3.053882in}}%
\pgfpathlineto{\pgfqpoint{3.814953in}{3.053882in}}%
\pgfpathclose%
\pgfusepath{stroke,fill}%
\end{pgfscope}%
\begin{pgfscope}%
\pgfpathrectangle{\pgfqpoint{0.100000in}{0.183744in}}{\pgfqpoint{4.506048in}{4.506048in}}%
\pgfusepath{clip}%
\pgfsetbuttcap%
\pgfsetroundjoin%
\definecolor{currentfill}{rgb}{1.000000,0.647059,0.000000}%
\pgfsetfillcolor{currentfill}%
\pgfsetfillopacity{0.700000}%
\pgfsetlinewidth{1.003750pt}%
\definecolor{currentstroke}{rgb}{1.000000,0.647059,0.000000}%
\pgfsetstrokecolor{currentstroke}%
\pgfsetstrokeopacity{0.700000}%
\pgfsetdash{}{0pt}%
\pgfpathmoveto{\pgfqpoint{2.751168in}{2.535289in}}%
\pgfpathcurveto{\pgfqpoint{2.756992in}{2.535289in}}{\pgfqpoint{2.762578in}{2.537603in}}{\pgfqpoint{2.766697in}{2.541721in}}%
\pgfpathcurveto{\pgfqpoint{2.770815in}{2.545839in}}{\pgfqpoint{2.773129in}{2.551426in}}{\pgfqpoint{2.773129in}{2.557249in}}%
\pgfpathcurveto{\pgfqpoint{2.773129in}{2.563073in}}{\pgfqpoint{2.770815in}{2.568660in}}{\pgfqpoint{2.766697in}{2.572778in}}%
\pgfpathcurveto{\pgfqpoint{2.762578in}{2.576896in}}{\pgfqpoint{2.756992in}{2.579210in}}{\pgfqpoint{2.751168in}{2.579210in}}%
\pgfpathcurveto{\pgfqpoint{2.745344in}{2.579210in}}{\pgfqpoint{2.739758in}{2.576896in}}{\pgfqpoint{2.735640in}{2.572778in}}%
\pgfpathcurveto{\pgfqpoint{2.731522in}{2.568660in}}{\pgfqpoint{2.729208in}{2.563073in}}{\pgfqpoint{2.729208in}{2.557249in}}%
\pgfpathcurveto{\pgfqpoint{2.729208in}{2.551426in}}{\pgfqpoint{2.731522in}{2.545839in}}{\pgfqpoint{2.735640in}{2.541721in}}%
\pgfpathcurveto{\pgfqpoint{2.739758in}{2.537603in}}{\pgfqpoint{2.745344in}{2.535289in}}{\pgfqpoint{2.751168in}{2.535289in}}%
\pgfpathlineto{\pgfqpoint{2.751168in}{2.535289in}}%
\pgfpathclose%
\pgfusepath{stroke,fill}%
\end{pgfscope}%
\begin{pgfscope}%
\pgfpathrectangle{\pgfqpoint{0.100000in}{0.183744in}}{\pgfqpoint{4.506048in}{4.506048in}}%
\pgfusepath{clip}%
\pgfsetbuttcap%
\pgfsetroundjoin%
\definecolor{currentfill}{rgb}{1.000000,0.647059,0.000000}%
\pgfsetfillcolor{currentfill}%
\pgfsetfillopacity{0.700000}%
\pgfsetlinewidth{1.003750pt}%
\definecolor{currentstroke}{rgb}{1.000000,0.647059,0.000000}%
\pgfsetstrokecolor{currentstroke}%
\pgfsetstrokeopacity{0.700000}%
\pgfsetdash{}{0pt}%
\pgfpathmoveto{\pgfqpoint{2.396345in}{2.856546in}}%
\pgfpathcurveto{\pgfqpoint{2.402169in}{2.856546in}}{\pgfqpoint{2.407755in}{2.858859in}}{\pgfqpoint{2.411873in}{2.862978in}}%
\pgfpathcurveto{\pgfqpoint{2.415991in}{2.867096in}}{\pgfqpoint{2.418305in}{2.872682in}}{\pgfqpoint{2.418305in}{2.878506in}}%
\pgfpathcurveto{\pgfqpoint{2.418305in}{2.884330in}}{\pgfqpoint{2.415991in}{2.889916in}}{\pgfqpoint{2.411873in}{2.894034in}}%
\pgfpathcurveto{\pgfqpoint{2.407755in}{2.898152in}}{\pgfqpoint{2.402169in}{2.900466in}}{\pgfqpoint{2.396345in}{2.900466in}}%
\pgfpathcurveto{\pgfqpoint{2.390521in}{2.900466in}}{\pgfqpoint{2.384935in}{2.898152in}}{\pgfqpoint{2.380817in}{2.894034in}}%
\pgfpathcurveto{\pgfqpoint{2.376699in}{2.889916in}}{\pgfqpoint{2.374385in}{2.884330in}}{\pgfqpoint{2.374385in}{2.878506in}}%
\pgfpathcurveto{\pgfqpoint{2.374385in}{2.872682in}}{\pgfqpoint{2.376699in}{2.867096in}}{\pgfqpoint{2.380817in}{2.862978in}}%
\pgfpathcurveto{\pgfqpoint{2.384935in}{2.858859in}}{\pgfqpoint{2.390521in}{2.856546in}}{\pgfqpoint{2.396345in}{2.856546in}}%
\pgfpathlineto{\pgfqpoint{2.396345in}{2.856546in}}%
\pgfpathclose%
\pgfusepath{stroke,fill}%
\end{pgfscope}%
\begin{pgfscope}%
\pgfpathrectangle{\pgfqpoint{0.100000in}{0.183744in}}{\pgfqpoint{4.506048in}{4.506048in}}%
\pgfusepath{clip}%
\pgfsetbuttcap%
\pgfsetroundjoin%
\definecolor{currentfill}{rgb}{1.000000,0.647059,0.000000}%
\pgfsetfillcolor{currentfill}%
\pgfsetfillopacity{0.700000}%
\pgfsetlinewidth{1.003750pt}%
\definecolor{currentstroke}{rgb}{1.000000,0.647059,0.000000}%
\pgfsetstrokecolor{currentstroke}%
\pgfsetstrokeopacity{0.700000}%
\pgfsetdash{}{0pt}%
\pgfpathmoveto{\pgfqpoint{2.813257in}{0.965413in}}%
\pgfpathcurveto{\pgfqpoint{2.819081in}{0.965413in}}{\pgfqpoint{2.824667in}{0.967727in}}{\pgfqpoint{2.828785in}{0.971845in}}%
\pgfpathcurveto{\pgfqpoint{2.832903in}{0.975964in}}{\pgfqpoint{2.835217in}{0.981550in}}{\pgfqpoint{2.835217in}{0.987374in}}%
\pgfpathcurveto{\pgfqpoint{2.835217in}{0.993198in}}{\pgfqpoint{2.832903in}{0.998784in}}{\pgfqpoint{2.828785in}{1.002902in}}%
\pgfpathcurveto{\pgfqpoint{2.824667in}{1.007020in}}{\pgfqpoint{2.819081in}{1.009334in}}{\pgfqpoint{2.813257in}{1.009334in}}%
\pgfpathcurveto{\pgfqpoint{2.807433in}{1.009334in}}{\pgfqpoint{2.801847in}{1.007020in}}{\pgfqpoint{2.797728in}{1.002902in}}%
\pgfpathcurveto{\pgfqpoint{2.793610in}{0.998784in}}{\pgfqpoint{2.791296in}{0.993198in}}{\pgfqpoint{2.791296in}{0.987374in}}%
\pgfpathcurveto{\pgfqpoint{2.791296in}{0.981550in}}{\pgfqpoint{2.793610in}{0.975964in}}{\pgfqpoint{2.797728in}{0.971845in}}%
\pgfpathcurveto{\pgfqpoint{2.801847in}{0.967727in}}{\pgfqpoint{2.807433in}{0.965413in}}{\pgfqpoint{2.813257in}{0.965413in}}%
\pgfpathlineto{\pgfqpoint{2.813257in}{0.965413in}}%
\pgfpathclose%
\pgfusepath{stroke,fill}%
\end{pgfscope}%
\begin{pgfscope}%
\pgfpathrectangle{\pgfqpoint{0.100000in}{0.183744in}}{\pgfqpoint{4.506048in}{4.506048in}}%
\pgfusepath{clip}%
\pgfsetbuttcap%
\pgfsetroundjoin%
\definecolor{currentfill}{rgb}{1.000000,0.647059,0.000000}%
\pgfsetfillcolor{currentfill}%
\pgfsetfillopacity{0.700000}%
\pgfsetlinewidth{1.003750pt}%
\definecolor{currentstroke}{rgb}{1.000000,0.647059,0.000000}%
\pgfsetstrokecolor{currentstroke}%
\pgfsetstrokeopacity{0.700000}%
\pgfsetdash{}{0pt}%
\pgfpathmoveto{\pgfqpoint{2.898280in}{1.932261in}}%
\pgfpathcurveto{\pgfqpoint{2.904104in}{1.932261in}}{\pgfqpoint{2.909690in}{1.934574in}}{\pgfqpoint{2.913808in}{1.938693in}}%
\pgfpathcurveto{\pgfqpoint{2.917926in}{1.942811in}}{\pgfqpoint{2.920240in}{1.948397in}}{\pgfqpoint{2.920240in}{1.954221in}}%
\pgfpathcurveto{\pgfqpoint{2.920240in}{1.960045in}}{\pgfqpoint{2.917926in}{1.965631in}}{\pgfqpoint{2.913808in}{1.969749in}}%
\pgfpathcurveto{\pgfqpoint{2.909690in}{1.973867in}}{\pgfqpoint{2.904104in}{1.976181in}}{\pgfqpoint{2.898280in}{1.976181in}}%
\pgfpathcurveto{\pgfqpoint{2.892456in}{1.976181in}}{\pgfqpoint{2.886870in}{1.973867in}}{\pgfqpoint{2.882752in}{1.969749in}}%
\pgfpathcurveto{\pgfqpoint{2.878633in}{1.965631in}}{\pgfqpoint{2.876320in}{1.960045in}}{\pgfqpoint{2.876320in}{1.954221in}}%
\pgfpathcurveto{\pgfqpoint{2.876320in}{1.948397in}}{\pgfqpoint{2.878633in}{1.942811in}}{\pgfqpoint{2.882752in}{1.938693in}}%
\pgfpathcurveto{\pgfqpoint{2.886870in}{1.934574in}}{\pgfqpoint{2.892456in}{1.932261in}}{\pgfqpoint{2.898280in}{1.932261in}}%
\pgfpathlineto{\pgfqpoint{2.898280in}{1.932261in}}%
\pgfpathclose%
\pgfusepath{stroke,fill}%
\end{pgfscope}%
\begin{pgfscope}%
\pgfpathrectangle{\pgfqpoint{0.100000in}{0.183744in}}{\pgfqpoint{4.506048in}{4.506048in}}%
\pgfusepath{clip}%
\pgfsetbuttcap%
\pgfsetroundjoin%
\definecolor{currentfill}{rgb}{1.000000,0.647059,0.000000}%
\pgfsetfillcolor{currentfill}%
\pgfsetfillopacity{0.700000}%
\pgfsetlinewidth{1.003750pt}%
\definecolor{currentstroke}{rgb}{1.000000,0.647059,0.000000}%
\pgfsetstrokecolor{currentstroke}%
\pgfsetstrokeopacity{0.700000}%
\pgfsetdash{}{0pt}%
\pgfpathmoveto{\pgfqpoint{3.190683in}{2.325953in}}%
\pgfpathcurveto{\pgfqpoint{3.196507in}{2.325953in}}{\pgfqpoint{3.202093in}{2.328267in}}{\pgfqpoint{3.206211in}{2.332385in}}%
\pgfpathcurveto{\pgfqpoint{3.210329in}{2.336503in}}{\pgfqpoint{3.212643in}{2.342090in}}{\pgfqpoint{3.212643in}{2.347913in}}%
\pgfpathcurveto{\pgfqpoint{3.212643in}{2.353737in}}{\pgfqpoint{3.210329in}{2.359324in}}{\pgfqpoint{3.206211in}{2.363442in}}%
\pgfpathcurveto{\pgfqpoint{3.202093in}{2.367560in}}{\pgfqpoint{3.196507in}{2.369874in}}{\pgfqpoint{3.190683in}{2.369874in}}%
\pgfpathcurveto{\pgfqpoint{3.184859in}{2.369874in}}{\pgfqpoint{3.179273in}{2.367560in}}{\pgfqpoint{3.175154in}{2.363442in}}%
\pgfpathcurveto{\pgfqpoint{3.171036in}{2.359324in}}{\pgfqpoint{3.168722in}{2.353737in}}{\pgfqpoint{3.168722in}{2.347913in}}%
\pgfpathcurveto{\pgfqpoint{3.168722in}{2.342090in}}{\pgfqpoint{3.171036in}{2.336503in}}{\pgfqpoint{3.175154in}{2.332385in}}%
\pgfpathcurveto{\pgfqpoint{3.179273in}{2.328267in}}{\pgfqpoint{3.184859in}{2.325953in}}{\pgfqpoint{3.190683in}{2.325953in}}%
\pgfpathlineto{\pgfqpoint{3.190683in}{2.325953in}}%
\pgfpathclose%
\pgfusepath{stroke,fill}%
\end{pgfscope}%
\begin{pgfscope}%
\pgfpathrectangle{\pgfqpoint{0.100000in}{0.183744in}}{\pgfqpoint{4.506048in}{4.506048in}}%
\pgfusepath{clip}%
\pgfsetbuttcap%
\pgfsetroundjoin%
\definecolor{currentfill}{rgb}{1.000000,0.647059,0.000000}%
\pgfsetfillcolor{currentfill}%
\pgfsetfillopacity{0.700000}%
\pgfsetlinewidth{1.003750pt}%
\definecolor{currentstroke}{rgb}{1.000000,0.647059,0.000000}%
\pgfsetstrokecolor{currentstroke}%
\pgfsetstrokeopacity{0.700000}%
\pgfsetdash{}{0pt}%
\pgfpathmoveto{\pgfqpoint{2.708007in}{2.517997in}}%
\pgfpathcurveto{\pgfqpoint{2.713831in}{2.517997in}}{\pgfqpoint{2.719417in}{2.520311in}}{\pgfqpoint{2.723535in}{2.524429in}}%
\pgfpathcurveto{\pgfqpoint{2.727653in}{2.528547in}}{\pgfqpoint{2.729967in}{2.534134in}}{\pgfqpoint{2.729967in}{2.539957in}}%
\pgfpathcurveto{\pgfqpoint{2.729967in}{2.545781in}}{\pgfqpoint{2.727653in}{2.551368in}}{\pgfqpoint{2.723535in}{2.555486in}}%
\pgfpathcurveto{\pgfqpoint{2.719417in}{2.559604in}}{\pgfqpoint{2.713831in}{2.561918in}}{\pgfqpoint{2.708007in}{2.561918in}}%
\pgfpathcurveto{\pgfqpoint{2.702183in}{2.561918in}}{\pgfqpoint{2.696597in}{2.559604in}}{\pgfqpoint{2.692479in}{2.555486in}}%
\pgfpathcurveto{\pgfqpoint{2.688361in}{2.551368in}}{\pgfqpoint{2.686047in}{2.545781in}}{\pgfqpoint{2.686047in}{2.539957in}}%
\pgfpathcurveto{\pgfqpoint{2.686047in}{2.534134in}}{\pgfqpoint{2.688361in}{2.528547in}}{\pgfqpoint{2.692479in}{2.524429in}}%
\pgfpathcurveto{\pgfqpoint{2.696597in}{2.520311in}}{\pgfqpoint{2.702183in}{2.517997in}}{\pgfqpoint{2.708007in}{2.517997in}}%
\pgfpathlineto{\pgfqpoint{2.708007in}{2.517997in}}%
\pgfpathclose%
\pgfusepath{stroke,fill}%
\end{pgfscope}%
\begin{pgfscope}%
\pgfpathrectangle{\pgfqpoint{0.100000in}{0.183744in}}{\pgfqpoint{4.506048in}{4.506048in}}%
\pgfusepath{clip}%
\pgfsetbuttcap%
\pgfsetroundjoin%
\definecolor{currentfill}{rgb}{1.000000,0.647059,0.000000}%
\pgfsetfillcolor{currentfill}%
\pgfsetfillopacity{0.700000}%
\pgfsetlinewidth{1.003750pt}%
\definecolor{currentstroke}{rgb}{1.000000,0.647059,0.000000}%
\pgfsetstrokecolor{currentstroke}%
\pgfsetstrokeopacity{0.700000}%
\pgfsetdash{}{0pt}%
\pgfpathmoveto{\pgfqpoint{1.865061in}{1.924857in}}%
\pgfpathcurveto{\pgfqpoint{1.870885in}{1.924857in}}{\pgfqpoint{1.876471in}{1.927171in}}{\pgfqpoint{1.880589in}{1.931289in}}%
\pgfpathcurveto{\pgfqpoint{1.884708in}{1.935408in}}{\pgfqpoint{1.887021in}{1.940994in}}{\pgfqpoint{1.887021in}{1.946818in}}%
\pgfpathcurveto{\pgfqpoint{1.887021in}{1.952642in}}{\pgfqpoint{1.884708in}{1.958228in}}{\pgfqpoint{1.880589in}{1.962346in}}%
\pgfpathcurveto{\pgfqpoint{1.876471in}{1.966464in}}{\pgfqpoint{1.870885in}{1.968778in}}{\pgfqpoint{1.865061in}{1.968778in}}%
\pgfpathcurveto{\pgfqpoint{1.859237in}{1.968778in}}{\pgfqpoint{1.853651in}{1.966464in}}{\pgfqpoint{1.849533in}{1.962346in}}%
\pgfpathcurveto{\pgfqpoint{1.845415in}{1.958228in}}{\pgfqpoint{1.843101in}{1.952642in}}{\pgfqpoint{1.843101in}{1.946818in}}%
\pgfpathcurveto{\pgfqpoint{1.843101in}{1.940994in}}{\pgfqpoint{1.845415in}{1.935408in}}{\pgfqpoint{1.849533in}{1.931289in}}%
\pgfpathcurveto{\pgfqpoint{1.853651in}{1.927171in}}{\pgfqpoint{1.859237in}{1.924857in}}{\pgfqpoint{1.865061in}{1.924857in}}%
\pgfpathlineto{\pgfqpoint{1.865061in}{1.924857in}}%
\pgfpathclose%
\pgfusepath{stroke,fill}%
\end{pgfscope}%
\begin{pgfscope}%
\pgfpathrectangle{\pgfqpoint{0.100000in}{0.183744in}}{\pgfqpoint{4.506048in}{4.506048in}}%
\pgfusepath{clip}%
\pgfsetbuttcap%
\pgfsetroundjoin%
\definecolor{currentfill}{rgb}{1.000000,0.647059,0.000000}%
\pgfsetfillcolor{currentfill}%
\pgfsetfillopacity{0.700000}%
\pgfsetlinewidth{1.003750pt}%
\definecolor{currentstroke}{rgb}{1.000000,0.647059,0.000000}%
\pgfsetstrokecolor{currentstroke}%
\pgfsetstrokeopacity{0.700000}%
\pgfsetdash{}{0pt}%
\pgfpathmoveto{\pgfqpoint{1.953729in}{2.549121in}}%
\pgfpathcurveto{\pgfqpoint{1.959553in}{2.549121in}}{\pgfqpoint{1.965139in}{2.551435in}}{\pgfqpoint{1.969258in}{2.555553in}}%
\pgfpathcurveto{\pgfqpoint{1.973376in}{2.559671in}}{\pgfqpoint{1.975690in}{2.565258in}}{\pgfqpoint{1.975690in}{2.571081in}}%
\pgfpathcurveto{\pgfqpoint{1.975690in}{2.576905in}}{\pgfqpoint{1.973376in}{2.582492in}}{\pgfqpoint{1.969258in}{2.586610in}}%
\pgfpathcurveto{\pgfqpoint{1.965139in}{2.590728in}}{\pgfqpoint{1.959553in}{2.593042in}}{\pgfqpoint{1.953729in}{2.593042in}}%
\pgfpathcurveto{\pgfqpoint{1.947905in}{2.593042in}}{\pgfqpoint{1.942319in}{2.590728in}}{\pgfqpoint{1.938201in}{2.586610in}}%
\pgfpathcurveto{\pgfqpoint{1.934083in}{2.582492in}}{\pgfqpoint{1.931769in}{2.576905in}}{\pgfqpoint{1.931769in}{2.571081in}}%
\pgfpathcurveto{\pgfqpoint{1.931769in}{2.565258in}}{\pgfqpoint{1.934083in}{2.559671in}}{\pgfqpoint{1.938201in}{2.555553in}}%
\pgfpathcurveto{\pgfqpoint{1.942319in}{2.551435in}}{\pgfqpoint{1.947905in}{2.549121in}}{\pgfqpoint{1.953729in}{2.549121in}}%
\pgfpathlineto{\pgfqpoint{1.953729in}{2.549121in}}%
\pgfpathclose%
\pgfusepath{stroke,fill}%
\end{pgfscope}%
\begin{pgfscope}%
\pgfpathrectangle{\pgfqpoint{0.100000in}{0.183744in}}{\pgfqpoint{4.506048in}{4.506048in}}%
\pgfusepath{clip}%
\pgfsetbuttcap%
\pgfsetroundjoin%
\definecolor{currentfill}{rgb}{1.000000,0.647059,0.000000}%
\pgfsetfillcolor{currentfill}%
\pgfsetfillopacity{0.700000}%
\pgfsetlinewidth{1.003750pt}%
\definecolor{currentstroke}{rgb}{1.000000,0.647059,0.000000}%
\pgfsetstrokecolor{currentstroke}%
\pgfsetstrokeopacity{0.700000}%
\pgfsetdash{}{0pt}%
\pgfpathmoveto{\pgfqpoint{1.799091in}{1.596061in}}%
\pgfpathcurveto{\pgfqpoint{1.804915in}{1.596061in}}{\pgfqpoint{1.810501in}{1.598375in}}{\pgfqpoint{1.814619in}{1.602493in}}%
\pgfpathcurveto{\pgfqpoint{1.818737in}{1.606611in}}{\pgfqpoint{1.821051in}{1.612197in}}{\pgfqpoint{1.821051in}{1.618021in}}%
\pgfpathcurveto{\pgfqpoint{1.821051in}{1.623845in}}{\pgfqpoint{1.818737in}{1.629431in}}{\pgfqpoint{1.814619in}{1.633549in}}%
\pgfpathcurveto{\pgfqpoint{1.810501in}{1.637668in}}{\pgfqpoint{1.804915in}{1.639981in}}{\pgfqpoint{1.799091in}{1.639981in}}%
\pgfpathcurveto{\pgfqpoint{1.793267in}{1.639981in}}{\pgfqpoint{1.787681in}{1.637668in}}{\pgfqpoint{1.783563in}{1.633549in}}%
\pgfpathcurveto{\pgfqpoint{1.779445in}{1.629431in}}{\pgfqpoint{1.777131in}{1.623845in}}{\pgfqpoint{1.777131in}{1.618021in}}%
\pgfpathcurveto{\pgfqpoint{1.777131in}{1.612197in}}{\pgfqpoint{1.779445in}{1.606611in}}{\pgfqpoint{1.783563in}{1.602493in}}%
\pgfpathcurveto{\pgfqpoint{1.787681in}{1.598375in}}{\pgfqpoint{1.793267in}{1.596061in}}{\pgfqpoint{1.799091in}{1.596061in}}%
\pgfpathlineto{\pgfqpoint{1.799091in}{1.596061in}}%
\pgfpathclose%
\pgfusepath{stroke,fill}%
\end{pgfscope}%
\begin{pgfscope}%
\pgfpathrectangle{\pgfqpoint{0.100000in}{0.183744in}}{\pgfqpoint{4.506048in}{4.506048in}}%
\pgfusepath{clip}%
\pgfsetbuttcap%
\pgfsetroundjoin%
\definecolor{currentfill}{rgb}{1.000000,0.647059,0.000000}%
\pgfsetfillcolor{currentfill}%
\pgfsetfillopacity{0.700000}%
\pgfsetlinewidth{1.003750pt}%
\definecolor{currentstroke}{rgb}{1.000000,0.647059,0.000000}%
\pgfsetstrokecolor{currentstroke}%
\pgfsetstrokeopacity{0.700000}%
\pgfsetdash{}{0pt}%
\pgfpathmoveto{\pgfqpoint{2.730005in}{1.821686in}}%
\pgfpathcurveto{\pgfqpoint{2.735829in}{1.821686in}}{\pgfqpoint{2.741416in}{1.824000in}}{\pgfqpoint{2.745534in}{1.828119in}}%
\pgfpathcurveto{\pgfqpoint{2.749652in}{1.832237in}}{\pgfqpoint{2.751966in}{1.837823in}}{\pgfqpoint{2.751966in}{1.843647in}}%
\pgfpathcurveto{\pgfqpoint{2.751966in}{1.849471in}}{\pgfqpoint{2.749652in}{1.855057in}}{\pgfqpoint{2.745534in}{1.859175in}}%
\pgfpathcurveto{\pgfqpoint{2.741416in}{1.863293in}}{\pgfqpoint{2.735829in}{1.865607in}}{\pgfqpoint{2.730005in}{1.865607in}}%
\pgfpathcurveto{\pgfqpoint{2.724182in}{1.865607in}}{\pgfqpoint{2.718595in}{1.863293in}}{\pgfqpoint{2.714477in}{1.859175in}}%
\pgfpathcurveto{\pgfqpoint{2.710359in}{1.855057in}}{\pgfqpoint{2.708045in}{1.849471in}}{\pgfqpoint{2.708045in}{1.843647in}}%
\pgfpathcurveto{\pgfqpoint{2.708045in}{1.837823in}}{\pgfqpoint{2.710359in}{1.832237in}}{\pgfqpoint{2.714477in}{1.828119in}}%
\pgfpathcurveto{\pgfqpoint{2.718595in}{1.824000in}}{\pgfqpoint{2.724182in}{1.821686in}}{\pgfqpoint{2.730005in}{1.821686in}}%
\pgfpathlineto{\pgfqpoint{2.730005in}{1.821686in}}%
\pgfpathclose%
\pgfusepath{stroke,fill}%
\end{pgfscope}%
\begin{pgfscope}%
\pgfpathrectangle{\pgfqpoint{0.100000in}{0.183744in}}{\pgfqpoint{4.506048in}{4.506048in}}%
\pgfusepath{clip}%
\pgfsetbuttcap%
\pgfsetroundjoin%
\definecolor{currentfill}{rgb}{1.000000,0.647059,0.000000}%
\pgfsetfillcolor{currentfill}%
\pgfsetfillopacity{0.700000}%
\pgfsetlinewidth{1.003750pt}%
\definecolor{currentstroke}{rgb}{1.000000,0.647059,0.000000}%
\pgfsetstrokecolor{currentstroke}%
\pgfsetstrokeopacity{0.700000}%
\pgfsetdash{}{0pt}%
\pgfpathmoveto{\pgfqpoint{1.738570in}{1.664890in}}%
\pgfpathcurveto{\pgfqpoint{1.744394in}{1.664890in}}{\pgfqpoint{1.749980in}{1.667204in}}{\pgfqpoint{1.754098in}{1.671322in}}%
\pgfpathcurveto{\pgfqpoint{1.758216in}{1.675440in}}{\pgfqpoint{1.760530in}{1.681026in}}{\pgfqpoint{1.760530in}{1.686850in}}%
\pgfpathcurveto{\pgfqpoint{1.760530in}{1.692674in}}{\pgfqpoint{1.758216in}{1.698260in}}{\pgfqpoint{1.754098in}{1.702378in}}%
\pgfpathcurveto{\pgfqpoint{1.749980in}{1.706496in}}{\pgfqpoint{1.744394in}{1.708810in}}{\pgfqpoint{1.738570in}{1.708810in}}%
\pgfpathcurveto{\pgfqpoint{1.732746in}{1.708810in}}{\pgfqpoint{1.727160in}{1.706496in}}{\pgfqpoint{1.723042in}{1.702378in}}%
\pgfpathcurveto{\pgfqpoint{1.718923in}{1.698260in}}{\pgfqpoint{1.716610in}{1.692674in}}{\pgfqpoint{1.716610in}{1.686850in}}%
\pgfpathcurveto{\pgfqpoint{1.716610in}{1.681026in}}{\pgfqpoint{1.718923in}{1.675440in}}{\pgfqpoint{1.723042in}{1.671322in}}%
\pgfpathcurveto{\pgfqpoint{1.727160in}{1.667204in}}{\pgfqpoint{1.732746in}{1.664890in}}{\pgfqpoint{1.738570in}{1.664890in}}%
\pgfpathlineto{\pgfqpoint{1.738570in}{1.664890in}}%
\pgfpathclose%
\pgfusepath{stroke,fill}%
\end{pgfscope}%
\begin{pgfscope}%
\pgfpathrectangle{\pgfqpoint{0.100000in}{0.183744in}}{\pgfqpoint{4.506048in}{4.506048in}}%
\pgfusepath{clip}%
\pgfsetbuttcap%
\pgfsetroundjoin%
\definecolor{currentfill}{rgb}{1.000000,0.647059,0.000000}%
\pgfsetfillcolor{currentfill}%
\pgfsetfillopacity{0.700000}%
\pgfsetlinewidth{1.003750pt}%
\definecolor{currentstroke}{rgb}{1.000000,0.647059,0.000000}%
\pgfsetstrokecolor{currentstroke}%
\pgfsetstrokeopacity{0.700000}%
\pgfsetdash{}{0pt}%
\pgfpathmoveto{\pgfqpoint{1.920429in}{2.194914in}}%
\pgfpathcurveto{\pgfqpoint{1.926253in}{2.194914in}}{\pgfqpoint{1.931839in}{2.197227in}}{\pgfqpoint{1.935957in}{2.201346in}}%
\pgfpathcurveto{\pgfqpoint{1.940075in}{2.205464in}}{\pgfqpoint{1.942389in}{2.211050in}}{\pgfqpoint{1.942389in}{2.216874in}}%
\pgfpathcurveto{\pgfqpoint{1.942389in}{2.222698in}}{\pgfqpoint{1.940075in}{2.228284in}}{\pgfqpoint{1.935957in}{2.232402in}}%
\pgfpathcurveto{\pgfqpoint{1.931839in}{2.236520in}}{\pgfqpoint{1.926253in}{2.238834in}}{\pgfqpoint{1.920429in}{2.238834in}}%
\pgfpathcurveto{\pgfqpoint{1.914605in}{2.238834in}}{\pgfqpoint{1.909019in}{2.236520in}}{\pgfqpoint{1.904900in}{2.232402in}}%
\pgfpathcurveto{\pgfqpoint{1.900782in}{2.228284in}}{\pgfqpoint{1.898468in}{2.222698in}}{\pgfqpoint{1.898468in}{2.216874in}}%
\pgfpathcurveto{\pgfqpoint{1.898468in}{2.211050in}}{\pgfqpoint{1.900782in}{2.205464in}}{\pgfqpoint{1.904900in}{2.201346in}}%
\pgfpathcurveto{\pgfqpoint{1.909019in}{2.197227in}}{\pgfqpoint{1.914605in}{2.194914in}}{\pgfqpoint{1.920429in}{2.194914in}}%
\pgfpathlineto{\pgfqpoint{1.920429in}{2.194914in}}%
\pgfpathclose%
\pgfusepath{stroke,fill}%
\end{pgfscope}%
\begin{pgfscope}%
\pgfpathrectangle{\pgfqpoint{0.100000in}{0.183744in}}{\pgfqpoint{4.506048in}{4.506048in}}%
\pgfusepath{clip}%
\pgfsetbuttcap%
\pgfsetroundjoin%
\definecolor{currentfill}{rgb}{1.000000,0.647059,0.000000}%
\pgfsetfillcolor{currentfill}%
\pgfsetfillopacity{0.700000}%
\pgfsetlinewidth{1.003750pt}%
\definecolor{currentstroke}{rgb}{1.000000,0.647059,0.000000}%
\pgfsetstrokecolor{currentstroke}%
\pgfsetstrokeopacity{0.700000}%
\pgfsetdash{}{0pt}%
\pgfpathmoveto{\pgfqpoint{3.085744in}{2.400632in}}%
\pgfpathcurveto{\pgfqpoint{3.091568in}{2.400632in}}{\pgfqpoint{3.097154in}{2.402945in}}{\pgfqpoint{3.101272in}{2.407064in}}%
\pgfpathcurveto{\pgfqpoint{3.105390in}{2.411182in}}{\pgfqpoint{3.107704in}{2.416768in}}{\pgfqpoint{3.107704in}{2.422592in}}%
\pgfpathcurveto{\pgfqpoint{3.107704in}{2.428416in}}{\pgfqpoint{3.105390in}{2.434002in}}{\pgfqpoint{3.101272in}{2.438120in}}%
\pgfpathcurveto{\pgfqpoint{3.097154in}{2.442238in}}{\pgfqpoint{3.091568in}{2.444552in}}{\pgfqpoint{3.085744in}{2.444552in}}%
\pgfpathcurveto{\pgfqpoint{3.079920in}{2.444552in}}{\pgfqpoint{3.074334in}{2.442238in}}{\pgfqpoint{3.070216in}{2.438120in}}%
\pgfpathcurveto{\pgfqpoint{3.066098in}{2.434002in}}{\pgfqpoint{3.063784in}{2.428416in}}{\pgfqpoint{3.063784in}{2.422592in}}%
\pgfpathcurveto{\pgfqpoint{3.063784in}{2.416768in}}{\pgfqpoint{3.066098in}{2.411182in}}{\pgfqpoint{3.070216in}{2.407064in}}%
\pgfpathcurveto{\pgfqpoint{3.074334in}{2.402945in}}{\pgfqpoint{3.079920in}{2.400632in}}{\pgfqpoint{3.085744in}{2.400632in}}%
\pgfpathlineto{\pgfqpoint{3.085744in}{2.400632in}}%
\pgfpathclose%
\pgfusepath{stroke,fill}%
\end{pgfscope}%
\begin{pgfscope}%
\pgfpathrectangle{\pgfqpoint{0.100000in}{0.183744in}}{\pgfqpoint{4.506048in}{4.506048in}}%
\pgfusepath{clip}%
\pgfsetbuttcap%
\pgfsetroundjoin%
\definecolor{currentfill}{rgb}{1.000000,0.647059,0.000000}%
\pgfsetfillcolor{currentfill}%
\pgfsetfillopacity{0.700000}%
\pgfsetlinewidth{1.003750pt}%
\definecolor{currentstroke}{rgb}{1.000000,0.647059,0.000000}%
\pgfsetstrokecolor{currentstroke}%
\pgfsetstrokeopacity{0.700000}%
\pgfsetdash{}{0pt}%
\pgfpathmoveto{\pgfqpoint{2.017681in}{1.554696in}}%
\pgfpathcurveto{\pgfqpoint{2.023505in}{1.554696in}}{\pgfqpoint{2.029091in}{1.557010in}}{\pgfqpoint{2.033209in}{1.561128in}}%
\pgfpathcurveto{\pgfqpoint{2.037328in}{1.565246in}}{\pgfqpoint{2.039641in}{1.570832in}}{\pgfqpoint{2.039641in}{1.576656in}}%
\pgfpathcurveto{\pgfqpoint{2.039641in}{1.582480in}}{\pgfqpoint{2.037328in}{1.588066in}}{\pgfqpoint{2.033209in}{1.592184in}}%
\pgfpathcurveto{\pgfqpoint{2.029091in}{1.596303in}}{\pgfqpoint{2.023505in}{1.598616in}}{\pgfqpoint{2.017681in}{1.598616in}}%
\pgfpathcurveto{\pgfqpoint{2.011857in}{1.598616in}}{\pgfqpoint{2.006271in}{1.596303in}}{\pgfqpoint{2.002153in}{1.592184in}}%
\pgfpathcurveto{\pgfqpoint{1.998035in}{1.588066in}}{\pgfqpoint{1.995721in}{1.582480in}}{\pgfqpoint{1.995721in}{1.576656in}}%
\pgfpathcurveto{\pgfqpoint{1.995721in}{1.570832in}}{\pgfqpoint{1.998035in}{1.565246in}}{\pgfqpoint{2.002153in}{1.561128in}}%
\pgfpathcurveto{\pgfqpoint{2.006271in}{1.557010in}}{\pgfqpoint{2.011857in}{1.554696in}}{\pgfqpoint{2.017681in}{1.554696in}}%
\pgfpathlineto{\pgfqpoint{2.017681in}{1.554696in}}%
\pgfpathclose%
\pgfusepath{stroke,fill}%
\end{pgfscope}%
\begin{pgfscope}%
\pgfpathrectangle{\pgfqpoint{0.100000in}{0.183744in}}{\pgfqpoint{4.506048in}{4.506048in}}%
\pgfusepath{clip}%
\pgfsetbuttcap%
\pgfsetroundjoin%
\definecolor{currentfill}{rgb}{1.000000,0.647059,0.000000}%
\pgfsetfillcolor{currentfill}%
\pgfsetfillopacity{0.700000}%
\pgfsetlinewidth{1.003750pt}%
\definecolor{currentstroke}{rgb}{1.000000,0.647059,0.000000}%
\pgfsetstrokecolor{currentstroke}%
\pgfsetstrokeopacity{0.700000}%
\pgfsetdash{}{0pt}%
\pgfpathmoveto{\pgfqpoint{2.311425in}{0.921312in}}%
\pgfpathcurveto{\pgfqpoint{2.317249in}{0.921312in}}{\pgfqpoint{2.322835in}{0.923626in}}{\pgfqpoint{2.326953in}{0.927744in}}%
\pgfpathcurveto{\pgfqpoint{2.331071in}{0.931862in}}{\pgfqpoint{2.333385in}{0.937448in}}{\pgfqpoint{2.333385in}{0.943272in}}%
\pgfpathcurveto{\pgfqpoint{2.333385in}{0.949096in}}{\pgfqpoint{2.331071in}{0.954682in}}{\pgfqpoint{2.326953in}{0.958800in}}%
\pgfpathcurveto{\pgfqpoint{2.322835in}{0.962918in}}{\pgfqpoint{2.317249in}{0.965232in}}{\pgfqpoint{2.311425in}{0.965232in}}%
\pgfpathcurveto{\pgfqpoint{2.305601in}{0.965232in}}{\pgfqpoint{2.300015in}{0.962918in}}{\pgfqpoint{2.295897in}{0.958800in}}%
\pgfpathcurveto{\pgfqpoint{2.291778in}{0.954682in}}{\pgfqpoint{2.289464in}{0.949096in}}{\pgfqpoint{2.289464in}{0.943272in}}%
\pgfpathcurveto{\pgfqpoint{2.289464in}{0.937448in}}{\pgfqpoint{2.291778in}{0.931862in}}{\pgfqpoint{2.295897in}{0.927744in}}%
\pgfpathcurveto{\pgfqpoint{2.300015in}{0.923626in}}{\pgfqpoint{2.305601in}{0.921312in}}{\pgfqpoint{2.311425in}{0.921312in}}%
\pgfpathlineto{\pgfqpoint{2.311425in}{0.921312in}}%
\pgfpathclose%
\pgfusepath{stroke,fill}%
\end{pgfscope}%
\begin{pgfscope}%
\pgfpathrectangle{\pgfqpoint{0.100000in}{0.183744in}}{\pgfqpoint{4.506048in}{4.506048in}}%
\pgfusepath{clip}%
\pgfsetbuttcap%
\pgfsetroundjoin%
\definecolor{currentfill}{rgb}{1.000000,0.647059,0.000000}%
\pgfsetfillcolor{currentfill}%
\pgfsetfillopacity{0.700000}%
\pgfsetlinewidth{1.003750pt}%
\definecolor{currentstroke}{rgb}{1.000000,0.647059,0.000000}%
\pgfsetstrokecolor{currentstroke}%
\pgfsetstrokeopacity{0.700000}%
\pgfsetdash{}{0pt}%
\pgfpathmoveto{\pgfqpoint{3.866762in}{1.972276in}}%
\pgfpathcurveto{\pgfqpoint{3.872586in}{1.972276in}}{\pgfqpoint{3.878172in}{1.974590in}}{\pgfqpoint{3.882290in}{1.978708in}}%
\pgfpathcurveto{\pgfqpoint{3.886408in}{1.982827in}}{\pgfqpoint{3.888722in}{1.988413in}}{\pgfqpoint{3.888722in}{1.994237in}}%
\pgfpathcurveto{\pgfqpoint{3.888722in}{2.000061in}}{\pgfqpoint{3.886408in}{2.005647in}}{\pgfqpoint{3.882290in}{2.009765in}}%
\pgfpathcurveto{\pgfqpoint{3.878172in}{2.013883in}}{\pgfqpoint{3.872586in}{2.016197in}}{\pgfqpoint{3.866762in}{2.016197in}}%
\pgfpathcurveto{\pgfqpoint{3.860938in}{2.016197in}}{\pgfqpoint{3.855352in}{2.013883in}}{\pgfqpoint{3.851233in}{2.009765in}}%
\pgfpathcurveto{\pgfqpoint{3.847115in}{2.005647in}}{\pgfqpoint{3.844801in}{2.000061in}}{\pgfqpoint{3.844801in}{1.994237in}}%
\pgfpathcurveto{\pgfqpoint{3.844801in}{1.988413in}}{\pgfqpoint{3.847115in}{1.982827in}}{\pgfqpoint{3.851233in}{1.978708in}}%
\pgfpathcurveto{\pgfqpoint{3.855352in}{1.974590in}}{\pgfqpoint{3.860938in}{1.972276in}}{\pgfqpoint{3.866762in}{1.972276in}}%
\pgfpathlineto{\pgfqpoint{3.866762in}{1.972276in}}%
\pgfpathclose%
\pgfusepath{stroke,fill}%
\end{pgfscope}%
\begin{pgfscope}%
\pgfpathrectangle{\pgfqpoint{0.100000in}{0.183744in}}{\pgfqpoint{4.506048in}{4.506048in}}%
\pgfusepath{clip}%
\pgfsetbuttcap%
\pgfsetroundjoin%
\definecolor{currentfill}{rgb}{1.000000,0.647059,0.000000}%
\pgfsetfillcolor{currentfill}%
\pgfsetfillopacity{0.700000}%
\pgfsetlinewidth{1.003750pt}%
\definecolor{currentstroke}{rgb}{1.000000,0.647059,0.000000}%
\pgfsetstrokecolor{currentstroke}%
\pgfsetstrokeopacity{0.700000}%
\pgfsetdash{}{0pt}%
\pgfpathmoveto{\pgfqpoint{2.398135in}{2.373687in}}%
\pgfpathcurveto{\pgfqpoint{2.403959in}{2.373687in}}{\pgfqpoint{2.409546in}{2.376001in}}{\pgfqpoint{2.413664in}{2.380119in}}%
\pgfpathcurveto{\pgfqpoint{2.417782in}{2.384237in}}{\pgfqpoint{2.420096in}{2.389823in}}{\pgfqpoint{2.420096in}{2.395647in}}%
\pgfpathcurveto{\pgfqpoint{2.420096in}{2.401471in}}{\pgfqpoint{2.417782in}{2.407057in}}{\pgfqpoint{2.413664in}{2.411175in}}%
\pgfpathcurveto{\pgfqpoint{2.409546in}{2.415293in}}{\pgfqpoint{2.403959in}{2.417607in}}{\pgfqpoint{2.398135in}{2.417607in}}%
\pgfpathcurveto{\pgfqpoint{2.392312in}{2.417607in}}{\pgfqpoint{2.386725in}{2.415293in}}{\pgfqpoint{2.382607in}{2.411175in}}%
\pgfpathcurveto{\pgfqpoint{2.378489in}{2.407057in}}{\pgfqpoint{2.376175in}{2.401471in}}{\pgfqpoint{2.376175in}{2.395647in}}%
\pgfpathcurveto{\pgfqpoint{2.376175in}{2.389823in}}{\pgfqpoint{2.378489in}{2.384237in}}{\pgfqpoint{2.382607in}{2.380119in}}%
\pgfpathcurveto{\pgfqpoint{2.386725in}{2.376001in}}{\pgfqpoint{2.392312in}{2.373687in}}{\pgfqpoint{2.398135in}{2.373687in}}%
\pgfpathlineto{\pgfqpoint{2.398135in}{2.373687in}}%
\pgfpathclose%
\pgfusepath{stroke,fill}%
\end{pgfscope}%
\begin{pgfscope}%
\pgfpathrectangle{\pgfqpoint{0.100000in}{0.183744in}}{\pgfqpoint{4.506048in}{4.506048in}}%
\pgfusepath{clip}%
\pgfsetbuttcap%
\pgfsetroundjoin%
\definecolor{currentfill}{rgb}{1.000000,0.647059,0.000000}%
\pgfsetfillcolor{currentfill}%
\pgfsetfillopacity{0.700000}%
\pgfsetlinewidth{1.003750pt}%
\definecolor{currentstroke}{rgb}{1.000000,0.647059,0.000000}%
\pgfsetstrokecolor{currentstroke}%
\pgfsetstrokeopacity{0.700000}%
\pgfsetdash{}{0pt}%
\pgfpathmoveto{\pgfqpoint{4.111798in}{3.055214in}}%
\pgfpathcurveto{\pgfqpoint{4.117622in}{3.055214in}}{\pgfqpoint{4.123208in}{3.057528in}}{\pgfqpoint{4.127326in}{3.061646in}}%
\pgfpathcurveto{\pgfqpoint{4.131444in}{3.065764in}}{\pgfqpoint{4.133758in}{3.071350in}}{\pgfqpoint{4.133758in}{3.077174in}}%
\pgfpathcurveto{\pgfqpoint{4.133758in}{3.082998in}}{\pgfqpoint{4.131444in}{3.088584in}}{\pgfqpoint{4.127326in}{3.092702in}}%
\pgfpathcurveto{\pgfqpoint{4.123208in}{3.096820in}}{\pgfqpoint{4.117622in}{3.099134in}}{\pgfqpoint{4.111798in}{3.099134in}}%
\pgfpathcurveto{\pgfqpoint{4.105974in}{3.099134in}}{\pgfqpoint{4.100388in}{3.096820in}}{\pgfqpoint{4.096270in}{3.092702in}}%
\pgfpathcurveto{\pgfqpoint{4.092151in}{3.088584in}}{\pgfqpoint{4.089837in}{3.082998in}}{\pgfqpoint{4.089837in}{3.077174in}}%
\pgfpathcurveto{\pgfqpoint{4.089837in}{3.071350in}}{\pgfqpoint{4.092151in}{3.065764in}}{\pgfqpoint{4.096270in}{3.061646in}}%
\pgfpathcurveto{\pgfqpoint{4.100388in}{3.057528in}}{\pgfqpoint{4.105974in}{3.055214in}}{\pgfqpoint{4.111798in}{3.055214in}}%
\pgfpathlineto{\pgfqpoint{4.111798in}{3.055214in}}%
\pgfpathclose%
\pgfusepath{stroke,fill}%
\end{pgfscope}%
\begin{pgfscope}%
\pgfpathrectangle{\pgfqpoint{0.100000in}{0.183744in}}{\pgfqpoint{4.506048in}{4.506048in}}%
\pgfusepath{clip}%
\pgfsetbuttcap%
\pgfsetroundjoin%
\definecolor{currentfill}{rgb}{1.000000,0.647059,0.000000}%
\pgfsetfillcolor{currentfill}%
\pgfsetfillopacity{0.700000}%
\pgfsetlinewidth{1.003750pt}%
\definecolor{currentstroke}{rgb}{1.000000,0.647059,0.000000}%
\pgfsetstrokecolor{currentstroke}%
\pgfsetstrokeopacity{0.700000}%
\pgfsetdash{}{0pt}%
\pgfpathmoveto{\pgfqpoint{2.302153in}{1.958613in}}%
\pgfpathcurveto{\pgfqpoint{2.307977in}{1.958613in}}{\pgfqpoint{2.313563in}{1.960927in}}{\pgfqpoint{2.317681in}{1.965045in}}%
\pgfpathcurveto{\pgfqpoint{2.321799in}{1.969163in}}{\pgfqpoint{2.324113in}{1.974749in}}{\pgfqpoint{2.324113in}{1.980573in}}%
\pgfpathcurveto{\pgfqpoint{2.324113in}{1.986397in}}{\pgfqpoint{2.321799in}{1.991983in}}{\pgfqpoint{2.317681in}{1.996101in}}%
\pgfpathcurveto{\pgfqpoint{2.313563in}{2.000220in}}{\pgfqpoint{2.307977in}{2.002533in}}{\pgfqpoint{2.302153in}{2.002533in}}%
\pgfpathcurveto{\pgfqpoint{2.296329in}{2.002533in}}{\pgfqpoint{2.290743in}{2.000220in}}{\pgfqpoint{2.286624in}{1.996101in}}%
\pgfpathcurveto{\pgfqpoint{2.282506in}{1.991983in}}{\pgfqpoint{2.280192in}{1.986397in}}{\pgfqpoint{2.280192in}{1.980573in}}%
\pgfpathcurveto{\pgfqpoint{2.280192in}{1.974749in}}{\pgfqpoint{2.282506in}{1.969163in}}{\pgfqpoint{2.286624in}{1.965045in}}%
\pgfpathcurveto{\pgfqpoint{2.290743in}{1.960927in}}{\pgfqpoint{2.296329in}{1.958613in}}{\pgfqpoint{2.302153in}{1.958613in}}%
\pgfpathlineto{\pgfqpoint{2.302153in}{1.958613in}}%
\pgfpathclose%
\pgfusepath{stroke,fill}%
\end{pgfscope}%
\begin{pgfscope}%
\pgfpathrectangle{\pgfqpoint{0.100000in}{0.183744in}}{\pgfqpoint{4.506048in}{4.506048in}}%
\pgfusepath{clip}%
\pgfsetbuttcap%
\pgfsetroundjoin%
\definecolor{currentfill}{rgb}{1.000000,0.647059,0.000000}%
\pgfsetfillcolor{currentfill}%
\pgfsetfillopacity{0.700000}%
\pgfsetlinewidth{1.003750pt}%
\definecolor{currentstroke}{rgb}{1.000000,0.647059,0.000000}%
\pgfsetstrokecolor{currentstroke}%
\pgfsetstrokeopacity{0.700000}%
\pgfsetdash{}{0pt}%
\pgfpathmoveto{\pgfqpoint{1.631065in}{1.409432in}}%
\pgfpathcurveto{\pgfqpoint{1.636889in}{1.409432in}}{\pgfqpoint{1.642475in}{1.411746in}}{\pgfqpoint{1.646593in}{1.415864in}}%
\pgfpathcurveto{\pgfqpoint{1.650712in}{1.419982in}}{\pgfqpoint{1.653025in}{1.425569in}}{\pgfqpoint{1.653025in}{1.431392in}}%
\pgfpathcurveto{\pgfqpoint{1.653025in}{1.437216in}}{\pgfqpoint{1.650712in}{1.442803in}}{\pgfqpoint{1.646593in}{1.446921in}}%
\pgfpathcurveto{\pgfqpoint{1.642475in}{1.451039in}}{\pgfqpoint{1.636889in}{1.453353in}}{\pgfqpoint{1.631065in}{1.453353in}}%
\pgfpathcurveto{\pgfqpoint{1.625241in}{1.453353in}}{\pgfqpoint{1.619655in}{1.451039in}}{\pgfqpoint{1.615537in}{1.446921in}}%
\pgfpathcurveto{\pgfqpoint{1.611419in}{1.442803in}}{\pgfqpoint{1.609105in}{1.437216in}}{\pgfqpoint{1.609105in}{1.431392in}}%
\pgfpathcurveto{\pgfqpoint{1.609105in}{1.425569in}}{\pgfqpoint{1.611419in}{1.419982in}}{\pgfqpoint{1.615537in}{1.415864in}}%
\pgfpathcurveto{\pgfqpoint{1.619655in}{1.411746in}}{\pgfqpoint{1.625241in}{1.409432in}}{\pgfqpoint{1.631065in}{1.409432in}}%
\pgfpathlineto{\pgfqpoint{1.631065in}{1.409432in}}%
\pgfpathclose%
\pgfusepath{stroke,fill}%
\end{pgfscope}%
\begin{pgfscope}%
\pgfpathrectangle{\pgfqpoint{0.100000in}{0.183744in}}{\pgfqpoint{4.506048in}{4.506048in}}%
\pgfusepath{clip}%
\pgfsetbuttcap%
\pgfsetroundjoin%
\definecolor{currentfill}{rgb}{1.000000,0.647059,0.000000}%
\pgfsetfillcolor{currentfill}%
\pgfsetfillopacity{0.700000}%
\pgfsetlinewidth{1.003750pt}%
\definecolor{currentstroke}{rgb}{1.000000,0.647059,0.000000}%
\pgfsetstrokecolor{currentstroke}%
\pgfsetstrokeopacity{0.700000}%
\pgfsetdash{}{0pt}%
\pgfpathmoveto{\pgfqpoint{4.160701in}{2.909956in}}%
\pgfpathcurveto{\pgfqpoint{4.166525in}{2.909956in}}{\pgfqpoint{4.172111in}{2.912270in}}{\pgfqpoint{4.176230in}{2.916388in}}%
\pgfpathcurveto{\pgfqpoint{4.180348in}{2.920506in}}{\pgfqpoint{4.182662in}{2.926093in}}{\pgfqpoint{4.182662in}{2.931916in}}%
\pgfpathcurveto{\pgfqpoint{4.182662in}{2.937740in}}{\pgfqpoint{4.180348in}{2.943327in}}{\pgfqpoint{4.176230in}{2.947445in}}%
\pgfpathcurveto{\pgfqpoint{4.172111in}{2.951563in}}{\pgfqpoint{4.166525in}{2.953877in}}{\pgfqpoint{4.160701in}{2.953877in}}%
\pgfpathcurveto{\pgfqpoint{4.154877in}{2.953877in}}{\pgfqpoint{4.149291in}{2.951563in}}{\pgfqpoint{4.145173in}{2.947445in}}%
\pgfpathcurveto{\pgfqpoint{4.141055in}{2.943327in}}{\pgfqpoint{4.138741in}{2.937740in}}{\pgfqpoint{4.138741in}{2.931916in}}%
\pgfpathcurveto{\pgfqpoint{4.138741in}{2.926093in}}{\pgfqpoint{4.141055in}{2.920506in}}{\pgfqpoint{4.145173in}{2.916388in}}%
\pgfpathcurveto{\pgfqpoint{4.149291in}{2.912270in}}{\pgfqpoint{4.154877in}{2.909956in}}{\pgfqpoint{4.160701in}{2.909956in}}%
\pgfpathlineto{\pgfqpoint{4.160701in}{2.909956in}}%
\pgfpathclose%
\pgfusepath{stroke,fill}%
\end{pgfscope}%
\begin{pgfscope}%
\pgfpathrectangle{\pgfqpoint{0.100000in}{0.183744in}}{\pgfqpoint{4.506048in}{4.506048in}}%
\pgfusepath{clip}%
\pgfsetbuttcap%
\pgfsetroundjoin%
\definecolor{currentfill}{rgb}{1.000000,0.647059,0.000000}%
\pgfsetfillcolor{currentfill}%
\pgfsetfillopacity{0.700000}%
\pgfsetlinewidth{1.003750pt}%
\definecolor{currentstroke}{rgb}{1.000000,0.647059,0.000000}%
\pgfsetstrokecolor{currentstroke}%
\pgfsetstrokeopacity{0.700000}%
\pgfsetdash{}{0pt}%
\pgfpathmoveto{\pgfqpoint{2.608903in}{1.688239in}}%
\pgfpathcurveto{\pgfqpoint{2.614727in}{1.688239in}}{\pgfqpoint{2.620313in}{1.690553in}}{\pgfqpoint{2.624431in}{1.694671in}}%
\pgfpathcurveto{\pgfqpoint{2.628549in}{1.698789in}}{\pgfqpoint{2.630863in}{1.704375in}}{\pgfqpoint{2.630863in}{1.710199in}}%
\pgfpathcurveto{\pgfqpoint{2.630863in}{1.716023in}}{\pgfqpoint{2.628549in}{1.721609in}}{\pgfqpoint{2.624431in}{1.725727in}}%
\pgfpathcurveto{\pgfqpoint{2.620313in}{1.729846in}}{\pgfqpoint{2.614727in}{1.732159in}}{\pgfqpoint{2.608903in}{1.732159in}}%
\pgfpathcurveto{\pgfqpoint{2.603079in}{1.732159in}}{\pgfqpoint{2.597493in}{1.729846in}}{\pgfqpoint{2.593375in}{1.725727in}}%
\pgfpathcurveto{\pgfqpoint{2.589257in}{1.721609in}}{\pgfqpoint{2.586943in}{1.716023in}}{\pgfqpoint{2.586943in}{1.710199in}}%
\pgfpathcurveto{\pgfqpoint{2.586943in}{1.704375in}}{\pgfqpoint{2.589257in}{1.698789in}}{\pgfqpoint{2.593375in}{1.694671in}}%
\pgfpathcurveto{\pgfqpoint{2.597493in}{1.690553in}}{\pgfqpoint{2.603079in}{1.688239in}}{\pgfqpoint{2.608903in}{1.688239in}}%
\pgfpathlineto{\pgfqpoint{2.608903in}{1.688239in}}%
\pgfpathclose%
\pgfusepath{stroke,fill}%
\end{pgfscope}%
\begin{pgfscope}%
\pgfpathrectangle{\pgfqpoint{0.100000in}{0.183744in}}{\pgfqpoint{4.506048in}{4.506048in}}%
\pgfusepath{clip}%
\pgfsetbuttcap%
\pgfsetroundjoin%
\definecolor{currentfill}{rgb}{1.000000,0.647059,0.000000}%
\pgfsetfillcolor{currentfill}%
\pgfsetfillopacity{0.700000}%
\pgfsetlinewidth{1.003750pt}%
\definecolor{currentstroke}{rgb}{1.000000,0.647059,0.000000}%
\pgfsetstrokecolor{currentstroke}%
\pgfsetstrokeopacity{0.700000}%
\pgfsetdash{}{0pt}%
\pgfpathmoveto{\pgfqpoint{3.734028in}{2.493550in}}%
\pgfpathcurveto{\pgfqpoint{3.739852in}{2.493550in}}{\pgfqpoint{3.745438in}{2.495864in}}{\pgfqpoint{3.749556in}{2.499982in}}%
\pgfpathcurveto{\pgfqpoint{3.753675in}{2.504100in}}{\pgfqpoint{3.755988in}{2.509687in}}{\pgfqpoint{3.755988in}{2.515510in}}%
\pgfpathcurveto{\pgfqpoint{3.755988in}{2.521334in}}{\pgfqpoint{3.753675in}{2.526921in}}{\pgfqpoint{3.749556in}{2.531039in}}%
\pgfpathcurveto{\pgfqpoint{3.745438in}{2.535157in}}{\pgfqpoint{3.739852in}{2.537471in}}{\pgfqpoint{3.734028in}{2.537471in}}%
\pgfpathcurveto{\pgfqpoint{3.728204in}{2.537471in}}{\pgfqpoint{3.722618in}{2.535157in}}{\pgfqpoint{3.718500in}{2.531039in}}%
\pgfpathcurveto{\pgfqpoint{3.714382in}{2.526921in}}{\pgfqpoint{3.712068in}{2.521334in}}{\pgfqpoint{3.712068in}{2.515510in}}%
\pgfpathcurveto{\pgfqpoint{3.712068in}{2.509687in}}{\pgfqpoint{3.714382in}{2.504100in}}{\pgfqpoint{3.718500in}{2.499982in}}%
\pgfpathcurveto{\pgfqpoint{3.722618in}{2.495864in}}{\pgfqpoint{3.728204in}{2.493550in}}{\pgfqpoint{3.734028in}{2.493550in}}%
\pgfpathlineto{\pgfqpoint{3.734028in}{2.493550in}}%
\pgfpathclose%
\pgfusepath{stroke,fill}%
\end{pgfscope}%
\begin{pgfscope}%
\pgfpathrectangle{\pgfqpoint{0.100000in}{0.183744in}}{\pgfqpoint{4.506048in}{4.506048in}}%
\pgfusepath{clip}%
\pgfsetbuttcap%
\pgfsetroundjoin%
\definecolor{currentfill}{rgb}{1.000000,0.647059,0.000000}%
\pgfsetfillcolor{currentfill}%
\pgfsetfillopacity{0.700000}%
\pgfsetlinewidth{1.003750pt}%
\definecolor{currentstroke}{rgb}{1.000000,0.647059,0.000000}%
\pgfsetstrokecolor{currentstroke}%
\pgfsetstrokeopacity{0.700000}%
\pgfsetdash{}{0pt}%
\pgfpathmoveto{\pgfqpoint{3.328877in}{2.412989in}}%
\pgfpathcurveto{\pgfqpoint{3.334701in}{2.412989in}}{\pgfqpoint{3.340287in}{2.415303in}}{\pgfqpoint{3.344406in}{2.419421in}}%
\pgfpathcurveto{\pgfqpoint{3.348524in}{2.423539in}}{\pgfqpoint{3.350838in}{2.429125in}}{\pgfqpoint{3.350838in}{2.434949in}}%
\pgfpathcurveto{\pgfqpoint{3.350838in}{2.440773in}}{\pgfqpoint{3.348524in}{2.446359in}}{\pgfqpoint{3.344406in}{2.450478in}}%
\pgfpathcurveto{\pgfqpoint{3.340287in}{2.454596in}}{\pgfqpoint{3.334701in}{2.456910in}}{\pgfqpoint{3.328877in}{2.456910in}}%
\pgfpathcurveto{\pgfqpoint{3.323053in}{2.456910in}}{\pgfqpoint{3.317467in}{2.454596in}}{\pgfqpoint{3.313349in}{2.450478in}}%
\pgfpathcurveto{\pgfqpoint{3.309231in}{2.446359in}}{\pgfqpoint{3.306917in}{2.440773in}}{\pgfqpoint{3.306917in}{2.434949in}}%
\pgfpathcurveto{\pgfqpoint{3.306917in}{2.429125in}}{\pgfqpoint{3.309231in}{2.423539in}}{\pgfqpoint{3.313349in}{2.419421in}}%
\pgfpathcurveto{\pgfqpoint{3.317467in}{2.415303in}}{\pgfqpoint{3.323053in}{2.412989in}}{\pgfqpoint{3.328877in}{2.412989in}}%
\pgfpathlineto{\pgfqpoint{3.328877in}{2.412989in}}%
\pgfpathclose%
\pgfusepath{stroke,fill}%
\end{pgfscope}%
\begin{pgfscope}%
\pgfpathrectangle{\pgfqpoint{0.100000in}{0.183744in}}{\pgfqpoint{4.506048in}{4.506048in}}%
\pgfusepath{clip}%
\pgfsetbuttcap%
\pgfsetroundjoin%
\definecolor{currentfill}{rgb}{1.000000,0.647059,0.000000}%
\pgfsetfillcolor{currentfill}%
\pgfsetfillopacity{0.700000}%
\pgfsetlinewidth{1.003750pt}%
\definecolor{currentstroke}{rgb}{1.000000,0.647059,0.000000}%
\pgfsetstrokecolor{currentstroke}%
\pgfsetstrokeopacity{0.700000}%
\pgfsetdash{}{0pt}%
\pgfpathmoveto{\pgfqpoint{2.429184in}{1.639018in}}%
\pgfpathcurveto{\pgfqpoint{2.435008in}{1.639018in}}{\pgfqpoint{2.440594in}{1.641332in}}{\pgfqpoint{2.444712in}{1.645450in}}%
\pgfpathcurveto{\pgfqpoint{2.448831in}{1.649568in}}{\pgfqpoint{2.451144in}{1.655154in}}{\pgfqpoint{2.451144in}{1.660978in}}%
\pgfpathcurveto{\pgfqpoint{2.451144in}{1.666802in}}{\pgfqpoint{2.448831in}{1.672388in}}{\pgfqpoint{2.444712in}{1.676507in}}%
\pgfpathcurveto{\pgfqpoint{2.440594in}{1.680625in}}{\pgfqpoint{2.435008in}{1.682939in}}{\pgfqpoint{2.429184in}{1.682939in}}%
\pgfpathcurveto{\pgfqpoint{2.423360in}{1.682939in}}{\pgfqpoint{2.417774in}{1.680625in}}{\pgfqpoint{2.413656in}{1.676507in}}%
\pgfpathcurveto{\pgfqpoint{2.409538in}{1.672388in}}{\pgfqpoint{2.407224in}{1.666802in}}{\pgfqpoint{2.407224in}{1.660978in}}%
\pgfpathcurveto{\pgfqpoint{2.407224in}{1.655154in}}{\pgfqpoint{2.409538in}{1.649568in}}{\pgfqpoint{2.413656in}{1.645450in}}%
\pgfpathcurveto{\pgfqpoint{2.417774in}{1.641332in}}{\pgfqpoint{2.423360in}{1.639018in}}{\pgfqpoint{2.429184in}{1.639018in}}%
\pgfpathlineto{\pgfqpoint{2.429184in}{1.639018in}}%
\pgfpathclose%
\pgfusepath{stroke,fill}%
\end{pgfscope}%
\begin{pgfscope}%
\pgfpathrectangle{\pgfqpoint{0.100000in}{0.183744in}}{\pgfqpoint{4.506048in}{4.506048in}}%
\pgfusepath{clip}%
\pgfsetbuttcap%
\pgfsetroundjoin%
\definecolor{currentfill}{rgb}{1.000000,0.647059,0.000000}%
\pgfsetfillcolor{currentfill}%
\pgfsetfillopacity{0.700000}%
\pgfsetlinewidth{1.003750pt}%
\definecolor{currentstroke}{rgb}{1.000000,0.647059,0.000000}%
\pgfsetstrokecolor{currentstroke}%
\pgfsetstrokeopacity{0.700000}%
\pgfsetdash{}{0pt}%
\pgfpathmoveto{\pgfqpoint{3.677869in}{2.768047in}}%
\pgfpathcurveto{\pgfqpoint{3.683693in}{2.768047in}}{\pgfqpoint{3.689279in}{2.770361in}}{\pgfqpoint{3.693398in}{2.774479in}}%
\pgfpathcurveto{\pgfqpoint{3.697516in}{2.778597in}}{\pgfqpoint{3.699830in}{2.784183in}}{\pgfqpoint{3.699830in}{2.790007in}}%
\pgfpathcurveto{\pgfqpoint{3.699830in}{2.795831in}}{\pgfqpoint{3.697516in}{2.801417in}}{\pgfqpoint{3.693398in}{2.805535in}}%
\pgfpathcurveto{\pgfqpoint{3.689279in}{2.809653in}}{\pgfqpoint{3.683693in}{2.811967in}}{\pgfqpoint{3.677869in}{2.811967in}}%
\pgfpathcurveto{\pgfqpoint{3.672045in}{2.811967in}}{\pgfqpoint{3.666459in}{2.809653in}}{\pgfqpoint{3.662341in}{2.805535in}}%
\pgfpathcurveto{\pgfqpoint{3.658223in}{2.801417in}}{\pgfqpoint{3.655909in}{2.795831in}}{\pgfqpoint{3.655909in}{2.790007in}}%
\pgfpathcurveto{\pgfqpoint{3.655909in}{2.784183in}}{\pgfqpoint{3.658223in}{2.778597in}}{\pgfqpoint{3.662341in}{2.774479in}}%
\pgfpathcurveto{\pgfqpoint{3.666459in}{2.770361in}}{\pgfqpoint{3.672045in}{2.768047in}}{\pgfqpoint{3.677869in}{2.768047in}}%
\pgfpathlineto{\pgfqpoint{3.677869in}{2.768047in}}%
\pgfpathclose%
\pgfusepath{stroke,fill}%
\end{pgfscope}%
\begin{pgfscope}%
\pgfpathrectangle{\pgfqpoint{0.100000in}{0.183744in}}{\pgfqpoint{4.506048in}{4.506048in}}%
\pgfusepath{clip}%
\pgfsetbuttcap%
\pgfsetroundjoin%
\definecolor{currentfill}{rgb}{1.000000,0.647059,0.000000}%
\pgfsetfillcolor{currentfill}%
\pgfsetfillopacity{0.700000}%
\pgfsetlinewidth{1.003750pt}%
\definecolor{currentstroke}{rgb}{1.000000,0.647059,0.000000}%
\pgfsetstrokecolor{currentstroke}%
\pgfsetstrokeopacity{0.700000}%
\pgfsetdash{}{0pt}%
\pgfpathmoveto{\pgfqpoint{3.291475in}{2.408574in}}%
\pgfpathcurveto{\pgfqpoint{3.297299in}{2.408574in}}{\pgfqpoint{3.302885in}{2.410888in}}{\pgfqpoint{3.307004in}{2.415006in}}%
\pgfpathcurveto{\pgfqpoint{3.311122in}{2.419124in}}{\pgfqpoint{3.313436in}{2.424710in}}{\pgfqpoint{3.313436in}{2.430534in}}%
\pgfpathcurveto{\pgfqpoint{3.313436in}{2.436358in}}{\pgfqpoint{3.311122in}{2.441944in}}{\pgfqpoint{3.307004in}{2.446062in}}%
\pgfpathcurveto{\pgfqpoint{3.302885in}{2.450180in}}{\pgfqpoint{3.297299in}{2.452494in}}{\pgfqpoint{3.291475in}{2.452494in}}%
\pgfpathcurveto{\pgfqpoint{3.285651in}{2.452494in}}{\pgfqpoint{3.280065in}{2.450180in}}{\pgfqpoint{3.275947in}{2.446062in}}%
\pgfpathcurveto{\pgfqpoint{3.271829in}{2.441944in}}{\pgfqpoint{3.269515in}{2.436358in}}{\pgfqpoint{3.269515in}{2.430534in}}%
\pgfpathcurveto{\pgfqpoint{3.269515in}{2.424710in}}{\pgfqpoint{3.271829in}{2.419124in}}{\pgfqpoint{3.275947in}{2.415006in}}%
\pgfpathcurveto{\pgfqpoint{3.280065in}{2.410888in}}{\pgfqpoint{3.285651in}{2.408574in}}{\pgfqpoint{3.291475in}{2.408574in}}%
\pgfpathlineto{\pgfqpoint{3.291475in}{2.408574in}}%
\pgfpathclose%
\pgfusepath{stroke,fill}%
\end{pgfscope}%
\begin{pgfscope}%
\pgfpathrectangle{\pgfqpoint{0.100000in}{0.183744in}}{\pgfqpoint{4.506048in}{4.506048in}}%
\pgfusepath{clip}%
\pgfsetbuttcap%
\pgfsetroundjoin%
\definecolor{currentfill}{rgb}{1.000000,0.647059,0.000000}%
\pgfsetfillcolor{currentfill}%
\pgfsetfillopacity{0.700000}%
\pgfsetlinewidth{1.003750pt}%
\definecolor{currentstroke}{rgb}{1.000000,0.647059,0.000000}%
\pgfsetstrokecolor{currentstroke}%
\pgfsetstrokeopacity{0.700000}%
\pgfsetdash{}{0pt}%
\pgfpathmoveto{\pgfqpoint{1.224986in}{2.547440in}}%
\pgfpathcurveto{\pgfqpoint{1.230810in}{2.547440in}}{\pgfqpoint{1.236396in}{2.549754in}}{\pgfqpoint{1.240514in}{2.553872in}}%
\pgfpathcurveto{\pgfqpoint{1.244632in}{2.557990in}}{\pgfqpoint{1.246946in}{2.563576in}}{\pgfqpoint{1.246946in}{2.569400in}}%
\pgfpathcurveto{\pgfqpoint{1.246946in}{2.575224in}}{\pgfqpoint{1.244632in}{2.580810in}}{\pgfqpoint{1.240514in}{2.584928in}}%
\pgfpathcurveto{\pgfqpoint{1.236396in}{2.589047in}}{\pgfqpoint{1.230810in}{2.591360in}}{\pgfqpoint{1.224986in}{2.591360in}}%
\pgfpathcurveto{\pgfqpoint{1.219162in}{2.591360in}}{\pgfqpoint{1.213576in}{2.589047in}}{\pgfqpoint{1.209458in}{2.584928in}}%
\pgfpathcurveto{\pgfqpoint{1.205339in}{2.580810in}}{\pgfqpoint{1.203026in}{2.575224in}}{\pgfqpoint{1.203026in}{2.569400in}}%
\pgfpathcurveto{\pgfqpoint{1.203026in}{2.563576in}}{\pgfqpoint{1.205339in}{2.557990in}}{\pgfqpoint{1.209458in}{2.553872in}}%
\pgfpathcurveto{\pgfqpoint{1.213576in}{2.549754in}}{\pgfqpoint{1.219162in}{2.547440in}}{\pgfqpoint{1.224986in}{2.547440in}}%
\pgfpathlineto{\pgfqpoint{1.224986in}{2.547440in}}%
\pgfpathclose%
\pgfusepath{stroke,fill}%
\end{pgfscope}%
\begin{pgfscope}%
\pgfpathrectangle{\pgfqpoint{0.100000in}{0.183744in}}{\pgfqpoint{4.506048in}{4.506048in}}%
\pgfusepath{clip}%
\pgfsetbuttcap%
\pgfsetroundjoin%
\definecolor{currentfill}{rgb}{1.000000,0.647059,0.000000}%
\pgfsetfillcolor{currentfill}%
\pgfsetfillopacity{0.700000}%
\pgfsetlinewidth{1.003750pt}%
\definecolor{currentstroke}{rgb}{1.000000,0.647059,0.000000}%
\pgfsetstrokecolor{currentstroke}%
\pgfsetstrokeopacity{0.700000}%
\pgfsetdash{}{0pt}%
\pgfpathmoveto{\pgfqpoint{3.833626in}{2.176401in}}%
\pgfpathcurveto{\pgfqpoint{3.839450in}{2.176401in}}{\pgfqpoint{3.845036in}{2.178715in}}{\pgfqpoint{3.849154in}{2.182833in}}%
\pgfpathcurveto{\pgfqpoint{3.853272in}{2.186951in}}{\pgfqpoint{3.855586in}{2.192537in}}{\pgfqpoint{3.855586in}{2.198361in}}%
\pgfpathcurveto{\pgfqpoint{3.855586in}{2.204185in}}{\pgfqpoint{3.853272in}{2.209771in}}{\pgfqpoint{3.849154in}{2.213889in}}%
\pgfpathcurveto{\pgfqpoint{3.845036in}{2.218008in}}{\pgfqpoint{3.839450in}{2.220321in}}{\pgfqpoint{3.833626in}{2.220321in}}%
\pgfpathcurveto{\pgfqpoint{3.827802in}{2.220321in}}{\pgfqpoint{3.822216in}{2.218008in}}{\pgfqpoint{3.818098in}{2.213889in}}%
\pgfpathcurveto{\pgfqpoint{3.813979in}{2.209771in}}{\pgfqpoint{3.811666in}{2.204185in}}{\pgfqpoint{3.811666in}{2.198361in}}%
\pgfpathcurveto{\pgfqpoint{3.811666in}{2.192537in}}{\pgfqpoint{3.813979in}{2.186951in}}{\pgfqpoint{3.818098in}{2.182833in}}%
\pgfpathcurveto{\pgfqpoint{3.822216in}{2.178715in}}{\pgfqpoint{3.827802in}{2.176401in}}{\pgfqpoint{3.833626in}{2.176401in}}%
\pgfpathlineto{\pgfqpoint{3.833626in}{2.176401in}}%
\pgfpathclose%
\pgfusepath{stroke,fill}%
\end{pgfscope}%
\begin{pgfscope}%
\pgfpathrectangle{\pgfqpoint{0.100000in}{0.183744in}}{\pgfqpoint{4.506048in}{4.506048in}}%
\pgfusepath{clip}%
\pgfsetbuttcap%
\pgfsetroundjoin%
\definecolor{currentfill}{rgb}{1.000000,0.647059,0.000000}%
\pgfsetfillcolor{currentfill}%
\pgfsetfillopacity{0.700000}%
\pgfsetlinewidth{1.003750pt}%
\definecolor{currentstroke}{rgb}{1.000000,0.647059,0.000000}%
\pgfsetstrokecolor{currentstroke}%
\pgfsetstrokeopacity{0.700000}%
\pgfsetdash{}{0pt}%
\pgfpathmoveto{\pgfqpoint{2.588555in}{2.188725in}}%
\pgfpathcurveto{\pgfqpoint{2.594378in}{2.188725in}}{\pgfqpoint{2.599965in}{2.191039in}}{\pgfqpoint{2.604083in}{2.195157in}}%
\pgfpathcurveto{\pgfqpoint{2.608201in}{2.199275in}}{\pgfqpoint{2.610515in}{2.204862in}}{\pgfqpoint{2.610515in}{2.210685in}}%
\pgfpathcurveto{\pgfqpoint{2.610515in}{2.216509in}}{\pgfqpoint{2.608201in}{2.222096in}}{\pgfqpoint{2.604083in}{2.226214in}}%
\pgfpathcurveto{\pgfqpoint{2.599965in}{2.230332in}}{\pgfqpoint{2.594378in}{2.232646in}}{\pgfqpoint{2.588555in}{2.232646in}}%
\pgfpathcurveto{\pgfqpoint{2.582731in}{2.232646in}}{\pgfqpoint{2.577144in}{2.230332in}}{\pgfqpoint{2.573026in}{2.226214in}}%
\pgfpathcurveto{\pgfqpoint{2.568908in}{2.222096in}}{\pgfqpoint{2.566594in}{2.216509in}}{\pgfqpoint{2.566594in}{2.210685in}}%
\pgfpathcurveto{\pgfqpoint{2.566594in}{2.204862in}}{\pgfqpoint{2.568908in}{2.199275in}}{\pgfqpoint{2.573026in}{2.195157in}}%
\pgfpathcurveto{\pgfqpoint{2.577144in}{2.191039in}}{\pgfqpoint{2.582731in}{2.188725in}}{\pgfqpoint{2.588555in}{2.188725in}}%
\pgfpathlineto{\pgfqpoint{2.588555in}{2.188725in}}%
\pgfpathclose%
\pgfusepath{stroke,fill}%
\end{pgfscope}%
\begin{pgfscope}%
\pgfpathrectangle{\pgfqpoint{0.100000in}{0.183744in}}{\pgfqpoint{4.506048in}{4.506048in}}%
\pgfusepath{clip}%
\pgfsetbuttcap%
\pgfsetroundjoin%
\definecolor{currentfill}{rgb}{1.000000,0.647059,0.000000}%
\pgfsetfillcolor{currentfill}%
\pgfsetfillopacity{0.700000}%
\pgfsetlinewidth{1.003750pt}%
\definecolor{currentstroke}{rgb}{1.000000,0.647059,0.000000}%
\pgfsetstrokecolor{currentstroke}%
\pgfsetstrokeopacity{0.700000}%
\pgfsetdash{}{0pt}%
\pgfpathmoveto{\pgfqpoint{3.332038in}{2.604563in}}%
\pgfpathcurveto{\pgfqpoint{3.337862in}{2.604563in}}{\pgfqpoint{3.343448in}{2.606877in}}{\pgfqpoint{3.347566in}{2.610995in}}%
\pgfpathcurveto{\pgfqpoint{3.351684in}{2.615113in}}{\pgfqpoint{3.353998in}{2.620699in}}{\pgfqpoint{3.353998in}{2.626523in}}%
\pgfpathcurveto{\pgfqpoint{3.353998in}{2.632347in}}{\pgfqpoint{3.351684in}{2.637933in}}{\pgfqpoint{3.347566in}{2.642052in}}%
\pgfpathcurveto{\pgfqpoint{3.343448in}{2.646170in}}{\pgfqpoint{3.337862in}{2.648484in}}{\pgfqpoint{3.332038in}{2.648484in}}%
\pgfpathcurveto{\pgfqpoint{3.326214in}{2.648484in}}{\pgfqpoint{3.320628in}{2.646170in}}{\pgfqpoint{3.316510in}{2.642052in}}%
\pgfpathcurveto{\pgfqpoint{3.312392in}{2.637933in}}{\pgfqpoint{3.310078in}{2.632347in}}{\pgfqpoint{3.310078in}{2.626523in}}%
\pgfpathcurveto{\pgfqpoint{3.310078in}{2.620699in}}{\pgfqpoint{3.312392in}{2.615113in}}{\pgfqpoint{3.316510in}{2.610995in}}%
\pgfpathcurveto{\pgfqpoint{3.320628in}{2.606877in}}{\pgfqpoint{3.326214in}{2.604563in}}{\pgfqpoint{3.332038in}{2.604563in}}%
\pgfpathlineto{\pgfqpoint{3.332038in}{2.604563in}}%
\pgfpathclose%
\pgfusepath{stroke,fill}%
\end{pgfscope}%
\begin{pgfscope}%
\pgfpathrectangle{\pgfqpoint{0.100000in}{0.183744in}}{\pgfqpoint{4.506048in}{4.506048in}}%
\pgfusepath{clip}%
\pgfsetbuttcap%
\pgfsetroundjoin%
\definecolor{currentfill}{rgb}{1.000000,0.647059,0.000000}%
\pgfsetfillcolor{currentfill}%
\pgfsetfillopacity{0.700000}%
\pgfsetlinewidth{1.003750pt}%
\definecolor{currentstroke}{rgb}{1.000000,0.647059,0.000000}%
\pgfsetstrokecolor{currentstroke}%
\pgfsetstrokeopacity{0.700000}%
\pgfsetdash{}{0pt}%
\pgfpathmoveto{\pgfqpoint{3.566443in}{2.847207in}}%
\pgfpathcurveto{\pgfqpoint{3.572267in}{2.847207in}}{\pgfqpoint{3.577853in}{2.849521in}}{\pgfqpoint{3.581971in}{2.853639in}}%
\pgfpathcurveto{\pgfqpoint{3.586089in}{2.857757in}}{\pgfqpoint{3.588403in}{2.863344in}}{\pgfqpoint{3.588403in}{2.869168in}}%
\pgfpathcurveto{\pgfqpoint{3.588403in}{2.874991in}}{\pgfqpoint{3.586089in}{2.880578in}}{\pgfqpoint{3.581971in}{2.884696in}}%
\pgfpathcurveto{\pgfqpoint{3.577853in}{2.888814in}}{\pgfqpoint{3.572267in}{2.891128in}}{\pgfqpoint{3.566443in}{2.891128in}}%
\pgfpathcurveto{\pgfqpoint{3.560619in}{2.891128in}}{\pgfqpoint{3.555033in}{2.888814in}}{\pgfqpoint{3.550915in}{2.884696in}}%
\pgfpathcurveto{\pgfqpoint{3.546797in}{2.880578in}}{\pgfqpoint{3.544483in}{2.874991in}}{\pgfqpoint{3.544483in}{2.869168in}}%
\pgfpathcurveto{\pgfqpoint{3.544483in}{2.863344in}}{\pgfqpoint{3.546797in}{2.857757in}}{\pgfqpoint{3.550915in}{2.853639in}}%
\pgfpathcurveto{\pgfqpoint{3.555033in}{2.849521in}}{\pgfqpoint{3.560619in}{2.847207in}}{\pgfqpoint{3.566443in}{2.847207in}}%
\pgfpathlineto{\pgfqpoint{3.566443in}{2.847207in}}%
\pgfpathclose%
\pgfusepath{stroke,fill}%
\end{pgfscope}%
\begin{pgfscope}%
\pgfpathrectangle{\pgfqpoint{0.100000in}{0.183744in}}{\pgfqpoint{4.506048in}{4.506048in}}%
\pgfusepath{clip}%
\pgfsetbuttcap%
\pgfsetroundjoin%
\definecolor{currentfill}{rgb}{1.000000,0.647059,0.000000}%
\pgfsetfillcolor{currentfill}%
\pgfsetfillopacity{0.700000}%
\pgfsetlinewidth{1.003750pt}%
\definecolor{currentstroke}{rgb}{1.000000,0.647059,0.000000}%
\pgfsetstrokecolor{currentstroke}%
\pgfsetstrokeopacity{0.700000}%
\pgfsetdash{}{0pt}%
\pgfpathmoveto{\pgfqpoint{2.832123in}{1.714284in}}%
\pgfpathcurveto{\pgfqpoint{2.837947in}{1.714284in}}{\pgfqpoint{2.843533in}{1.716598in}}{\pgfqpoint{2.847651in}{1.720716in}}%
\pgfpathcurveto{\pgfqpoint{2.851769in}{1.724834in}}{\pgfqpoint{2.854083in}{1.730420in}}{\pgfqpoint{2.854083in}{1.736244in}}%
\pgfpathcurveto{\pgfqpoint{2.854083in}{1.742068in}}{\pgfqpoint{2.851769in}{1.747654in}}{\pgfqpoint{2.847651in}{1.751773in}}%
\pgfpathcurveto{\pgfqpoint{2.843533in}{1.755891in}}{\pgfqpoint{2.837947in}{1.758205in}}{\pgfqpoint{2.832123in}{1.758205in}}%
\pgfpathcurveto{\pgfqpoint{2.826299in}{1.758205in}}{\pgfqpoint{2.820713in}{1.755891in}}{\pgfqpoint{2.816594in}{1.751773in}}%
\pgfpathcurveto{\pgfqpoint{2.812476in}{1.747654in}}{\pgfqpoint{2.810162in}{1.742068in}}{\pgfqpoint{2.810162in}{1.736244in}}%
\pgfpathcurveto{\pgfqpoint{2.810162in}{1.730420in}}{\pgfqpoint{2.812476in}{1.724834in}}{\pgfqpoint{2.816594in}{1.720716in}}%
\pgfpathcurveto{\pgfqpoint{2.820713in}{1.716598in}}{\pgfqpoint{2.826299in}{1.714284in}}{\pgfqpoint{2.832123in}{1.714284in}}%
\pgfpathlineto{\pgfqpoint{2.832123in}{1.714284in}}%
\pgfpathclose%
\pgfusepath{stroke,fill}%
\end{pgfscope}%
\begin{pgfscope}%
\pgfpathrectangle{\pgfqpoint{0.100000in}{0.183744in}}{\pgfqpoint{4.506048in}{4.506048in}}%
\pgfusepath{clip}%
\pgfsetbuttcap%
\pgfsetroundjoin%
\definecolor{currentfill}{rgb}{1.000000,0.647059,0.000000}%
\pgfsetfillcolor{currentfill}%
\pgfsetfillopacity{0.700000}%
\pgfsetlinewidth{1.003750pt}%
\definecolor{currentstroke}{rgb}{1.000000,0.647059,0.000000}%
\pgfsetstrokecolor{currentstroke}%
\pgfsetstrokeopacity{0.700000}%
\pgfsetdash{}{0pt}%
\pgfpathmoveto{\pgfqpoint{1.561420in}{1.966119in}}%
\pgfpathcurveto{\pgfqpoint{1.567244in}{1.966119in}}{\pgfqpoint{1.572830in}{1.968432in}}{\pgfqpoint{1.576948in}{1.972551in}}%
\pgfpathcurveto{\pgfqpoint{1.581066in}{1.976669in}}{\pgfqpoint{1.583380in}{1.982255in}}{\pgfqpoint{1.583380in}{1.988079in}}%
\pgfpathcurveto{\pgfqpoint{1.583380in}{1.993903in}}{\pgfqpoint{1.581066in}{1.999489in}}{\pgfqpoint{1.576948in}{2.003607in}}%
\pgfpathcurveto{\pgfqpoint{1.572830in}{2.007725in}}{\pgfqpoint{1.567244in}{2.010039in}}{\pgfqpoint{1.561420in}{2.010039in}}%
\pgfpathcurveto{\pgfqpoint{1.555596in}{2.010039in}}{\pgfqpoint{1.550010in}{2.007725in}}{\pgfqpoint{1.545892in}{2.003607in}}%
\pgfpathcurveto{\pgfqpoint{1.541773in}{1.999489in}}{\pgfqpoint{1.539460in}{1.993903in}}{\pgfqpoint{1.539460in}{1.988079in}}%
\pgfpathcurveto{\pgfqpoint{1.539460in}{1.982255in}}{\pgfqpoint{1.541773in}{1.976669in}}{\pgfqpoint{1.545892in}{1.972551in}}%
\pgfpathcurveto{\pgfqpoint{1.550010in}{1.968432in}}{\pgfqpoint{1.555596in}{1.966119in}}{\pgfqpoint{1.561420in}{1.966119in}}%
\pgfpathlineto{\pgfqpoint{1.561420in}{1.966119in}}%
\pgfpathclose%
\pgfusepath{stroke,fill}%
\end{pgfscope}%
\begin{pgfscope}%
\pgfpathrectangle{\pgfqpoint{0.100000in}{0.183744in}}{\pgfqpoint{4.506048in}{4.506048in}}%
\pgfusepath{clip}%
\pgfsetbuttcap%
\pgfsetroundjoin%
\definecolor{currentfill}{rgb}{1.000000,0.647059,0.000000}%
\pgfsetfillcolor{currentfill}%
\pgfsetfillopacity{0.700000}%
\pgfsetlinewidth{1.003750pt}%
\definecolor{currentstroke}{rgb}{1.000000,0.647059,0.000000}%
\pgfsetstrokecolor{currentstroke}%
\pgfsetstrokeopacity{0.700000}%
\pgfsetdash{}{0pt}%
\pgfpathmoveto{\pgfqpoint{3.968930in}{2.301622in}}%
\pgfpathcurveto{\pgfqpoint{3.974754in}{2.301622in}}{\pgfqpoint{3.980340in}{2.303936in}}{\pgfqpoint{3.984458in}{2.308054in}}%
\pgfpathcurveto{\pgfqpoint{3.988576in}{2.312172in}}{\pgfqpoint{3.990890in}{2.317758in}}{\pgfqpoint{3.990890in}{2.323582in}}%
\pgfpathcurveto{\pgfqpoint{3.990890in}{2.329406in}}{\pgfqpoint{3.988576in}{2.334992in}}{\pgfqpoint{3.984458in}{2.339110in}}%
\pgfpathcurveto{\pgfqpoint{3.980340in}{2.343229in}}{\pgfqpoint{3.974754in}{2.345542in}}{\pgfqpoint{3.968930in}{2.345542in}}%
\pgfpathcurveto{\pgfqpoint{3.963106in}{2.345542in}}{\pgfqpoint{3.957519in}{2.343229in}}{\pgfqpoint{3.953401in}{2.339110in}}%
\pgfpathcurveto{\pgfqpoint{3.949283in}{2.334992in}}{\pgfqpoint{3.946969in}{2.329406in}}{\pgfqpoint{3.946969in}{2.323582in}}%
\pgfpathcurveto{\pgfqpoint{3.946969in}{2.317758in}}{\pgfqpoint{3.949283in}{2.312172in}}{\pgfqpoint{3.953401in}{2.308054in}}%
\pgfpathcurveto{\pgfqpoint{3.957519in}{2.303936in}}{\pgfqpoint{3.963106in}{2.301622in}}{\pgfqpoint{3.968930in}{2.301622in}}%
\pgfpathlineto{\pgfqpoint{3.968930in}{2.301622in}}%
\pgfpathclose%
\pgfusepath{stroke,fill}%
\end{pgfscope}%
\begin{pgfscope}%
\pgfpathrectangle{\pgfqpoint{0.100000in}{0.183744in}}{\pgfqpoint{4.506048in}{4.506048in}}%
\pgfusepath{clip}%
\pgfsetbuttcap%
\pgfsetroundjoin%
\definecolor{currentfill}{rgb}{1.000000,0.647059,0.000000}%
\pgfsetfillcolor{currentfill}%
\pgfsetfillopacity{0.700000}%
\pgfsetlinewidth{1.003750pt}%
\definecolor{currentstroke}{rgb}{1.000000,0.647059,0.000000}%
\pgfsetstrokecolor{currentstroke}%
\pgfsetstrokeopacity{0.700000}%
\pgfsetdash{}{0pt}%
\pgfpathmoveto{\pgfqpoint{3.384821in}{2.722794in}}%
\pgfpathcurveto{\pgfqpoint{3.390645in}{2.722794in}}{\pgfqpoint{3.396231in}{2.725108in}}{\pgfqpoint{3.400350in}{2.729226in}}%
\pgfpathcurveto{\pgfqpoint{3.404468in}{2.733344in}}{\pgfqpoint{3.406782in}{2.738931in}}{\pgfqpoint{3.406782in}{2.744755in}}%
\pgfpathcurveto{\pgfqpoint{3.406782in}{2.750579in}}{\pgfqpoint{3.404468in}{2.756165in}}{\pgfqpoint{3.400350in}{2.760283in}}%
\pgfpathcurveto{\pgfqpoint{3.396231in}{2.764401in}}{\pgfqpoint{3.390645in}{2.766715in}}{\pgfqpoint{3.384821in}{2.766715in}}%
\pgfpathcurveto{\pgfqpoint{3.378997in}{2.766715in}}{\pgfqpoint{3.373411in}{2.764401in}}{\pgfqpoint{3.369293in}{2.760283in}}%
\pgfpathcurveto{\pgfqpoint{3.365175in}{2.756165in}}{\pgfqpoint{3.362861in}{2.750579in}}{\pgfqpoint{3.362861in}{2.744755in}}%
\pgfpathcurveto{\pgfqpoint{3.362861in}{2.738931in}}{\pgfqpoint{3.365175in}{2.733344in}}{\pgfqpoint{3.369293in}{2.729226in}}%
\pgfpathcurveto{\pgfqpoint{3.373411in}{2.725108in}}{\pgfqpoint{3.378997in}{2.722794in}}{\pgfqpoint{3.384821in}{2.722794in}}%
\pgfpathlineto{\pgfqpoint{3.384821in}{2.722794in}}%
\pgfpathclose%
\pgfusepath{stroke,fill}%
\end{pgfscope}%
\begin{pgfscope}%
\pgfpathrectangle{\pgfqpoint{0.100000in}{0.183744in}}{\pgfqpoint{4.506048in}{4.506048in}}%
\pgfusepath{clip}%
\pgfsetbuttcap%
\pgfsetroundjoin%
\definecolor{currentfill}{rgb}{1.000000,0.647059,0.000000}%
\pgfsetfillcolor{currentfill}%
\pgfsetfillopacity{0.700000}%
\pgfsetlinewidth{1.003750pt}%
\definecolor{currentstroke}{rgb}{1.000000,0.647059,0.000000}%
\pgfsetstrokecolor{currentstroke}%
\pgfsetstrokeopacity{0.700000}%
\pgfsetdash{}{0pt}%
\pgfpathmoveto{\pgfqpoint{3.955634in}{2.383800in}}%
\pgfpathcurveto{\pgfqpoint{3.961458in}{2.383800in}}{\pgfqpoint{3.967044in}{2.386114in}}{\pgfqpoint{3.971162in}{2.390232in}}%
\pgfpathcurveto{\pgfqpoint{3.975281in}{2.394350in}}{\pgfqpoint{3.977594in}{2.399936in}}{\pgfqpoint{3.977594in}{2.405760in}}%
\pgfpathcurveto{\pgfqpoint{3.977594in}{2.411584in}}{\pgfqpoint{3.975281in}{2.417170in}}{\pgfqpoint{3.971162in}{2.421288in}}%
\pgfpathcurveto{\pgfqpoint{3.967044in}{2.425406in}}{\pgfqpoint{3.961458in}{2.427720in}}{\pgfqpoint{3.955634in}{2.427720in}}%
\pgfpathcurveto{\pgfqpoint{3.949810in}{2.427720in}}{\pgfqpoint{3.944224in}{2.425406in}}{\pgfqpoint{3.940106in}{2.421288in}}%
\pgfpathcurveto{\pgfqpoint{3.935988in}{2.417170in}}{\pgfqpoint{3.933674in}{2.411584in}}{\pgfqpoint{3.933674in}{2.405760in}}%
\pgfpathcurveto{\pgfqpoint{3.933674in}{2.399936in}}{\pgfqpoint{3.935988in}{2.394350in}}{\pgfqpoint{3.940106in}{2.390232in}}%
\pgfpathcurveto{\pgfqpoint{3.944224in}{2.386114in}}{\pgfqpoint{3.949810in}{2.383800in}}{\pgfqpoint{3.955634in}{2.383800in}}%
\pgfpathlineto{\pgfqpoint{3.955634in}{2.383800in}}%
\pgfpathclose%
\pgfusepath{stroke,fill}%
\end{pgfscope}%
\begin{pgfscope}%
\pgfpathrectangle{\pgfqpoint{0.100000in}{0.183744in}}{\pgfqpoint{4.506048in}{4.506048in}}%
\pgfusepath{clip}%
\pgfsetbuttcap%
\pgfsetroundjoin%
\definecolor{currentfill}{rgb}{1.000000,0.647059,0.000000}%
\pgfsetfillcolor{currentfill}%
\pgfsetfillopacity{0.700000}%
\pgfsetlinewidth{1.003750pt}%
\definecolor{currentstroke}{rgb}{1.000000,0.647059,0.000000}%
\pgfsetstrokecolor{currentstroke}%
\pgfsetstrokeopacity{0.700000}%
\pgfsetdash{}{0pt}%
\pgfpathmoveto{\pgfqpoint{2.769625in}{1.478545in}}%
\pgfpathcurveto{\pgfqpoint{2.775449in}{1.478545in}}{\pgfqpoint{2.781035in}{1.480859in}}{\pgfqpoint{2.785154in}{1.484977in}}%
\pgfpathcurveto{\pgfqpoint{2.789272in}{1.489095in}}{\pgfqpoint{2.791586in}{1.494681in}}{\pgfqpoint{2.791586in}{1.500505in}}%
\pgfpathcurveto{\pgfqpoint{2.791586in}{1.506329in}}{\pgfqpoint{2.789272in}{1.511915in}}{\pgfqpoint{2.785154in}{1.516033in}}%
\pgfpathcurveto{\pgfqpoint{2.781035in}{1.520151in}}{\pgfqpoint{2.775449in}{1.522465in}}{\pgfqpoint{2.769625in}{1.522465in}}%
\pgfpathcurveto{\pgfqpoint{2.763801in}{1.522465in}}{\pgfqpoint{2.758215in}{1.520151in}}{\pgfqpoint{2.754097in}{1.516033in}}%
\pgfpathcurveto{\pgfqpoint{2.749979in}{1.511915in}}{\pgfqpoint{2.747665in}{1.506329in}}{\pgfqpoint{2.747665in}{1.500505in}}%
\pgfpathcurveto{\pgfqpoint{2.747665in}{1.494681in}}{\pgfqpoint{2.749979in}{1.489095in}}{\pgfqpoint{2.754097in}{1.484977in}}%
\pgfpathcurveto{\pgfqpoint{2.758215in}{1.480859in}}{\pgfqpoint{2.763801in}{1.478545in}}{\pgfqpoint{2.769625in}{1.478545in}}%
\pgfpathlineto{\pgfqpoint{2.769625in}{1.478545in}}%
\pgfpathclose%
\pgfusepath{stroke,fill}%
\end{pgfscope}%
\begin{pgfscope}%
\pgfpathrectangle{\pgfqpoint{0.100000in}{0.183744in}}{\pgfqpoint{4.506048in}{4.506048in}}%
\pgfusepath{clip}%
\pgfsetbuttcap%
\pgfsetroundjoin%
\definecolor{currentfill}{rgb}{1.000000,0.647059,0.000000}%
\pgfsetfillcolor{currentfill}%
\pgfsetfillopacity{0.700000}%
\pgfsetlinewidth{1.003750pt}%
\definecolor{currentstroke}{rgb}{1.000000,0.647059,0.000000}%
\pgfsetstrokecolor{currentstroke}%
\pgfsetstrokeopacity{0.700000}%
\pgfsetdash{}{0pt}%
\pgfpathmoveto{\pgfqpoint{2.034927in}{1.643291in}}%
\pgfpathcurveto{\pgfqpoint{2.040751in}{1.643291in}}{\pgfqpoint{2.046337in}{1.645605in}}{\pgfqpoint{2.050455in}{1.649723in}}%
\pgfpathcurveto{\pgfqpoint{2.054573in}{1.653841in}}{\pgfqpoint{2.056887in}{1.659427in}}{\pgfqpoint{2.056887in}{1.665251in}}%
\pgfpathcurveto{\pgfqpoint{2.056887in}{1.671075in}}{\pgfqpoint{2.054573in}{1.676661in}}{\pgfqpoint{2.050455in}{1.680779in}}%
\pgfpathcurveto{\pgfqpoint{2.046337in}{1.684897in}}{\pgfqpoint{2.040751in}{1.687211in}}{\pgfqpoint{2.034927in}{1.687211in}}%
\pgfpathcurveto{\pgfqpoint{2.029103in}{1.687211in}}{\pgfqpoint{2.023517in}{1.684897in}}{\pgfqpoint{2.019399in}{1.680779in}}%
\pgfpathcurveto{\pgfqpoint{2.015281in}{1.676661in}}{\pgfqpoint{2.012967in}{1.671075in}}{\pgfqpoint{2.012967in}{1.665251in}}%
\pgfpathcurveto{\pgfqpoint{2.012967in}{1.659427in}}{\pgfqpoint{2.015281in}{1.653841in}}{\pgfqpoint{2.019399in}{1.649723in}}%
\pgfpathcurveto{\pgfqpoint{2.023517in}{1.645605in}}{\pgfqpoint{2.029103in}{1.643291in}}{\pgfqpoint{2.034927in}{1.643291in}}%
\pgfpathlineto{\pgfqpoint{2.034927in}{1.643291in}}%
\pgfpathclose%
\pgfusepath{stroke,fill}%
\end{pgfscope}%
\begin{pgfscope}%
\pgfpathrectangle{\pgfqpoint{0.100000in}{0.183744in}}{\pgfqpoint{4.506048in}{4.506048in}}%
\pgfusepath{clip}%
\pgfsetbuttcap%
\pgfsetroundjoin%
\definecolor{currentfill}{rgb}{1.000000,0.647059,0.000000}%
\pgfsetfillcolor{currentfill}%
\pgfsetfillopacity{0.700000}%
\pgfsetlinewidth{1.003750pt}%
\definecolor{currentstroke}{rgb}{1.000000,0.647059,0.000000}%
\pgfsetstrokecolor{currentstroke}%
\pgfsetstrokeopacity{0.700000}%
\pgfsetdash{}{0pt}%
\pgfpathmoveto{\pgfqpoint{3.153427in}{2.003066in}}%
\pgfpathcurveto{\pgfqpoint{3.159251in}{2.003066in}}{\pgfqpoint{3.164838in}{2.005379in}}{\pgfqpoint{3.168956in}{2.009498in}}%
\pgfpathcurveto{\pgfqpoint{3.173074in}{2.013616in}}{\pgfqpoint{3.175388in}{2.019202in}}{\pgfqpoint{3.175388in}{2.025026in}}%
\pgfpathcurveto{\pgfqpoint{3.175388in}{2.030850in}}{\pgfqpoint{3.173074in}{2.036436in}}{\pgfqpoint{3.168956in}{2.040554in}}%
\pgfpathcurveto{\pgfqpoint{3.164838in}{2.044672in}}{\pgfqpoint{3.159251in}{2.046986in}}{\pgfqpoint{3.153427in}{2.046986in}}%
\pgfpathcurveto{\pgfqpoint{3.147604in}{2.046986in}}{\pgfqpoint{3.142017in}{2.044672in}}{\pgfqpoint{3.137899in}{2.040554in}}%
\pgfpathcurveto{\pgfqpoint{3.133781in}{2.036436in}}{\pgfqpoint{3.131467in}{2.030850in}}{\pgfqpoint{3.131467in}{2.025026in}}%
\pgfpathcurveto{\pgfqpoint{3.131467in}{2.019202in}}{\pgfqpoint{3.133781in}{2.013616in}}{\pgfqpoint{3.137899in}{2.009498in}}%
\pgfpathcurveto{\pgfqpoint{3.142017in}{2.005379in}}{\pgfqpoint{3.147604in}{2.003066in}}{\pgfqpoint{3.153427in}{2.003066in}}%
\pgfpathlineto{\pgfqpoint{3.153427in}{2.003066in}}%
\pgfpathclose%
\pgfusepath{stroke,fill}%
\end{pgfscope}%
\begin{pgfscope}%
\pgfpathrectangle{\pgfqpoint{0.100000in}{0.183744in}}{\pgfqpoint{4.506048in}{4.506048in}}%
\pgfusepath{clip}%
\pgfsetbuttcap%
\pgfsetroundjoin%
\definecolor{currentfill}{rgb}{1.000000,0.647059,0.000000}%
\pgfsetfillcolor{currentfill}%
\pgfsetfillopacity{0.700000}%
\pgfsetlinewidth{1.003750pt}%
\definecolor{currentstroke}{rgb}{1.000000,0.647059,0.000000}%
\pgfsetstrokecolor{currentstroke}%
\pgfsetstrokeopacity{0.700000}%
\pgfsetdash{}{0pt}%
\pgfpathmoveto{\pgfqpoint{1.615887in}{2.516271in}}%
\pgfpathcurveto{\pgfqpoint{1.621711in}{2.516271in}}{\pgfqpoint{1.627297in}{2.518584in}}{\pgfqpoint{1.631415in}{2.522703in}}%
\pgfpathcurveto{\pgfqpoint{1.635533in}{2.526821in}}{\pgfqpoint{1.637847in}{2.532407in}}{\pgfqpoint{1.637847in}{2.538231in}}%
\pgfpathcurveto{\pgfqpoint{1.637847in}{2.544055in}}{\pgfqpoint{1.635533in}{2.549641in}}{\pgfqpoint{1.631415in}{2.553759in}}%
\pgfpathcurveto{\pgfqpoint{1.627297in}{2.557877in}}{\pgfqpoint{1.621711in}{2.560191in}}{\pgfqpoint{1.615887in}{2.560191in}}%
\pgfpathcurveto{\pgfqpoint{1.610063in}{2.560191in}}{\pgfqpoint{1.604477in}{2.557877in}}{\pgfqpoint{1.600358in}{2.553759in}}%
\pgfpathcurveto{\pgfqpoint{1.596240in}{2.549641in}}{\pgfqpoint{1.593926in}{2.544055in}}{\pgfqpoint{1.593926in}{2.538231in}}%
\pgfpathcurveto{\pgfqpoint{1.593926in}{2.532407in}}{\pgfqpoint{1.596240in}{2.526821in}}{\pgfqpoint{1.600358in}{2.522703in}}%
\pgfpathcurveto{\pgfqpoint{1.604477in}{2.518584in}}{\pgfqpoint{1.610063in}{2.516271in}}{\pgfqpoint{1.615887in}{2.516271in}}%
\pgfpathlineto{\pgfqpoint{1.615887in}{2.516271in}}%
\pgfpathclose%
\pgfusepath{stroke,fill}%
\end{pgfscope}%
\begin{pgfscope}%
\pgfpathrectangle{\pgfqpoint{0.100000in}{0.183744in}}{\pgfqpoint{4.506048in}{4.506048in}}%
\pgfusepath{clip}%
\pgfsetbuttcap%
\pgfsetroundjoin%
\definecolor{currentfill}{rgb}{1.000000,0.647059,0.000000}%
\pgfsetfillcolor{currentfill}%
\pgfsetfillopacity{0.700000}%
\pgfsetlinewidth{1.003750pt}%
\definecolor{currentstroke}{rgb}{1.000000,0.647059,0.000000}%
\pgfsetstrokecolor{currentstroke}%
\pgfsetstrokeopacity{0.700000}%
\pgfsetdash{}{0pt}%
\pgfpathmoveto{\pgfqpoint{3.433186in}{2.285335in}}%
\pgfpathcurveto{\pgfqpoint{3.439010in}{2.285335in}}{\pgfqpoint{3.444596in}{2.287649in}}{\pgfqpoint{3.448714in}{2.291767in}}%
\pgfpathcurveto{\pgfqpoint{3.452832in}{2.295885in}}{\pgfqpoint{3.455146in}{2.301471in}}{\pgfqpoint{3.455146in}{2.307295in}}%
\pgfpathcurveto{\pgfqpoint{3.455146in}{2.313119in}}{\pgfqpoint{3.452832in}{2.318706in}}{\pgfqpoint{3.448714in}{2.322824in}}%
\pgfpathcurveto{\pgfqpoint{3.444596in}{2.326942in}}{\pgfqpoint{3.439010in}{2.329256in}}{\pgfqpoint{3.433186in}{2.329256in}}%
\pgfpathcurveto{\pgfqpoint{3.427362in}{2.329256in}}{\pgfqpoint{3.421776in}{2.326942in}}{\pgfqpoint{3.417658in}{2.322824in}}%
\pgfpathcurveto{\pgfqpoint{3.413540in}{2.318706in}}{\pgfqpoint{3.411226in}{2.313119in}}{\pgfqpoint{3.411226in}{2.307295in}}%
\pgfpathcurveto{\pgfqpoint{3.411226in}{2.301471in}}{\pgfqpoint{3.413540in}{2.295885in}}{\pgfqpoint{3.417658in}{2.291767in}}%
\pgfpathcurveto{\pgfqpoint{3.421776in}{2.287649in}}{\pgfqpoint{3.427362in}{2.285335in}}{\pgfqpoint{3.433186in}{2.285335in}}%
\pgfpathlineto{\pgfqpoint{3.433186in}{2.285335in}}%
\pgfpathclose%
\pgfusepath{stroke,fill}%
\end{pgfscope}%
\begin{pgfscope}%
\pgfpathrectangle{\pgfqpoint{0.100000in}{0.183744in}}{\pgfqpoint{4.506048in}{4.506048in}}%
\pgfusepath{clip}%
\pgfsetbuttcap%
\pgfsetroundjoin%
\definecolor{currentfill}{rgb}{1.000000,0.647059,0.000000}%
\pgfsetfillcolor{currentfill}%
\pgfsetfillopacity{0.700000}%
\pgfsetlinewidth{1.003750pt}%
\definecolor{currentstroke}{rgb}{1.000000,0.647059,0.000000}%
\pgfsetstrokecolor{currentstroke}%
\pgfsetstrokeopacity{0.700000}%
\pgfsetdash{}{0pt}%
\pgfpathmoveto{\pgfqpoint{2.388105in}{1.896481in}}%
\pgfpathcurveto{\pgfqpoint{2.393928in}{1.896481in}}{\pgfqpoint{2.399515in}{1.898794in}}{\pgfqpoint{2.403633in}{1.902913in}}%
\pgfpathcurveto{\pgfqpoint{2.407751in}{1.907031in}}{\pgfqpoint{2.410065in}{1.912617in}}{\pgfqpoint{2.410065in}{1.918441in}}%
\pgfpathcurveto{\pgfqpoint{2.410065in}{1.924265in}}{\pgfqpoint{2.407751in}{1.929851in}}{\pgfqpoint{2.403633in}{1.933969in}}%
\pgfpathcurveto{\pgfqpoint{2.399515in}{1.938087in}}{\pgfqpoint{2.393928in}{1.940401in}}{\pgfqpoint{2.388105in}{1.940401in}}%
\pgfpathcurveto{\pgfqpoint{2.382281in}{1.940401in}}{\pgfqpoint{2.376694in}{1.938087in}}{\pgfqpoint{2.372576in}{1.933969in}}%
\pgfpathcurveto{\pgfqpoint{2.368458in}{1.929851in}}{\pgfqpoint{2.366144in}{1.924265in}}{\pgfqpoint{2.366144in}{1.918441in}}%
\pgfpathcurveto{\pgfqpoint{2.366144in}{1.912617in}}{\pgfqpoint{2.368458in}{1.907031in}}{\pgfqpoint{2.372576in}{1.902913in}}%
\pgfpathcurveto{\pgfqpoint{2.376694in}{1.898794in}}{\pgfqpoint{2.382281in}{1.896481in}}{\pgfqpoint{2.388105in}{1.896481in}}%
\pgfpathlineto{\pgfqpoint{2.388105in}{1.896481in}}%
\pgfpathclose%
\pgfusepath{stroke,fill}%
\end{pgfscope}%
\begin{pgfscope}%
\pgfpathrectangle{\pgfqpoint{0.100000in}{0.183744in}}{\pgfqpoint{4.506048in}{4.506048in}}%
\pgfusepath{clip}%
\pgfsetbuttcap%
\pgfsetroundjoin%
\definecolor{currentfill}{rgb}{1.000000,0.647059,0.000000}%
\pgfsetfillcolor{currentfill}%
\pgfsetfillopacity{0.700000}%
\pgfsetlinewidth{1.003750pt}%
\definecolor{currentstroke}{rgb}{1.000000,0.647059,0.000000}%
\pgfsetstrokecolor{currentstroke}%
\pgfsetstrokeopacity{0.700000}%
\pgfsetdash{}{0pt}%
\pgfpathmoveto{\pgfqpoint{3.872251in}{2.070048in}}%
\pgfpathcurveto{\pgfqpoint{3.878075in}{2.070048in}}{\pgfqpoint{3.883661in}{2.072362in}}{\pgfqpoint{3.887779in}{2.076480in}}%
\pgfpathcurveto{\pgfqpoint{3.891897in}{2.080599in}}{\pgfqpoint{3.894211in}{2.086185in}}{\pgfqpoint{3.894211in}{2.092009in}}%
\pgfpathcurveto{\pgfqpoint{3.894211in}{2.097833in}}{\pgfqpoint{3.891897in}{2.103419in}}{\pgfqpoint{3.887779in}{2.107537in}}%
\pgfpathcurveto{\pgfqpoint{3.883661in}{2.111655in}}{\pgfqpoint{3.878075in}{2.113969in}}{\pgfqpoint{3.872251in}{2.113969in}}%
\pgfpathcurveto{\pgfqpoint{3.866427in}{2.113969in}}{\pgfqpoint{3.860841in}{2.111655in}}{\pgfqpoint{3.856722in}{2.107537in}}%
\pgfpathcurveto{\pgfqpoint{3.852604in}{2.103419in}}{\pgfqpoint{3.850290in}{2.097833in}}{\pgfqpoint{3.850290in}{2.092009in}}%
\pgfpathcurveto{\pgfqpoint{3.850290in}{2.086185in}}{\pgfqpoint{3.852604in}{2.080599in}}{\pgfqpoint{3.856722in}{2.076480in}}%
\pgfpathcurveto{\pgfqpoint{3.860841in}{2.072362in}}{\pgfqpoint{3.866427in}{2.070048in}}{\pgfqpoint{3.872251in}{2.070048in}}%
\pgfpathlineto{\pgfqpoint{3.872251in}{2.070048in}}%
\pgfpathclose%
\pgfusepath{stroke,fill}%
\end{pgfscope}%
\begin{pgfscope}%
\pgfpathrectangle{\pgfqpoint{0.100000in}{0.183744in}}{\pgfqpoint{4.506048in}{4.506048in}}%
\pgfusepath{clip}%
\pgfsetbuttcap%
\pgfsetroundjoin%
\definecolor{currentfill}{rgb}{1.000000,0.647059,0.000000}%
\pgfsetfillcolor{currentfill}%
\pgfsetfillopacity{0.700000}%
\pgfsetlinewidth{1.003750pt}%
\definecolor{currentstroke}{rgb}{1.000000,0.647059,0.000000}%
\pgfsetstrokecolor{currentstroke}%
\pgfsetstrokeopacity{0.700000}%
\pgfsetdash{}{0pt}%
\pgfpathmoveto{\pgfqpoint{3.562947in}{2.370436in}}%
\pgfpathcurveto{\pgfqpoint{3.568771in}{2.370436in}}{\pgfqpoint{3.574358in}{2.372750in}}{\pgfqpoint{3.578476in}{2.376868in}}%
\pgfpathcurveto{\pgfqpoint{3.582594in}{2.380987in}}{\pgfqpoint{3.584908in}{2.386573in}}{\pgfqpoint{3.584908in}{2.392397in}}%
\pgfpathcurveto{\pgfqpoint{3.584908in}{2.398221in}}{\pgfqpoint{3.582594in}{2.403807in}}{\pgfqpoint{3.578476in}{2.407925in}}%
\pgfpathcurveto{\pgfqpoint{3.574358in}{2.412043in}}{\pgfqpoint{3.568771in}{2.414357in}}{\pgfqpoint{3.562947in}{2.414357in}}%
\pgfpathcurveto{\pgfqpoint{3.557123in}{2.414357in}}{\pgfqpoint{3.551537in}{2.412043in}}{\pgfqpoint{3.547419in}{2.407925in}}%
\pgfpathcurveto{\pgfqpoint{3.543301in}{2.403807in}}{\pgfqpoint{3.540987in}{2.398221in}}{\pgfqpoint{3.540987in}{2.392397in}}%
\pgfpathcurveto{\pgfqpoint{3.540987in}{2.386573in}}{\pgfqpoint{3.543301in}{2.380987in}}{\pgfqpoint{3.547419in}{2.376868in}}%
\pgfpathcurveto{\pgfqpoint{3.551537in}{2.372750in}}{\pgfqpoint{3.557123in}{2.370436in}}{\pgfqpoint{3.562947in}{2.370436in}}%
\pgfpathlineto{\pgfqpoint{3.562947in}{2.370436in}}%
\pgfpathclose%
\pgfusepath{stroke,fill}%
\end{pgfscope}%
\begin{pgfscope}%
\pgfpathrectangle{\pgfqpoint{0.100000in}{0.183744in}}{\pgfqpoint{4.506048in}{4.506048in}}%
\pgfusepath{clip}%
\pgfsetbuttcap%
\pgfsetroundjoin%
\definecolor{currentfill}{rgb}{1.000000,0.647059,0.000000}%
\pgfsetfillcolor{currentfill}%
\pgfsetfillopacity{0.700000}%
\pgfsetlinewidth{1.003750pt}%
\definecolor{currentstroke}{rgb}{1.000000,0.647059,0.000000}%
\pgfsetstrokecolor{currentstroke}%
\pgfsetstrokeopacity{0.700000}%
\pgfsetdash{}{0pt}%
\pgfpathmoveto{\pgfqpoint{1.510336in}{1.755718in}}%
\pgfpathcurveto{\pgfqpoint{1.516160in}{1.755718in}}{\pgfqpoint{1.521746in}{1.758031in}}{\pgfqpoint{1.525864in}{1.762150in}}%
\pgfpathcurveto{\pgfqpoint{1.529982in}{1.766268in}}{\pgfqpoint{1.532296in}{1.771854in}}{\pgfqpoint{1.532296in}{1.777678in}}%
\pgfpathcurveto{\pgfqpoint{1.532296in}{1.783502in}}{\pgfqpoint{1.529982in}{1.789088in}}{\pgfqpoint{1.525864in}{1.793206in}}%
\pgfpathcurveto{\pgfqpoint{1.521746in}{1.797324in}}{\pgfqpoint{1.516160in}{1.799638in}}{\pgfqpoint{1.510336in}{1.799638in}}%
\pgfpathcurveto{\pgfqpoint{1.504512in}{1.799638in}}{\pgfqpoint{1.498926in}{1.797324in}}{\pgfqpoint{1.494808in}{1.793206in}}%
\pgfpathcurveto{\pgfqpoint{1.490690in}{1.789088in}}{\pgfqpoint{1.488376in}{1.783502in}}{\pgfqpoint{1.488376in}{1.777678in}}%
\pgfpathcurveto{\pgfqpoint{1.488376in}{1.771854in}}{\pgfqpoint{1.490690in}{1.766268in}}{\pgfqpoint{1.494808in}{1.762150in}}%
\pgfpathcurveto{\pgfqpoint{1.498926in}{1.758031in}}{\pgfqpoint{1.504512in}{1.755718in}}{\pgfqpoint{1.510336in}{1.755718in}}%
\pgfpathlineto{\pgfqpoint{1.510336in}{1.755718in}}%
\pgfpathclose%
\pgfusepath{stroke,fill}%
\end{pgfscope}%
\begin{pgfscope}%
\pgfpathrectangle{\pgfqpoint{0.100000in}{0.183744in}}{\pgfqpoint{4.506048in}{4.506048in}}%
\pgfusepath{clip}%
\pgfsetbuttcap%
\pgfsetroundjoin%
\definecolor{currentfill}{rgb}{1.000000,0.647059,0.000000}%
\pgfsetfillcolor{currentfill}%
\pgfsetfillopacity{0.700000}%
\pgfsetlinewidth{1.003750pt}%
\definecolor{currentstroke}{rgb}{1.000000,0.647059,0.000000}%
\pgfsetstrokecolor{currentstroke}%
\pgfsetstrokeopacity{0.700000}%
\pgfsetdash{}{0pt}%
\pgfpathmoveto{\pgfqpoint{3.048338in}{1.736182in}}%
\pgfpathcurveto{\pgfqpoint{3.054162in}{1.736182in}}{\pgfqpoint{3.059748in}{1.738496in}}{\pgfqpoint{3.063866in}{1.742614in}}%
\pgfpathcurveto{\pgfqpoint{3.067985in}{1.746732in}}{\pgfqpoint{3.070298in}{1.752319in}}{\pgfqpoint{3.070298in}{1.758143in}}%
\pgfpathcurveto{\pgfqpoint{3.070298in}{1.763966in}}{\pgfqpoint{3.067985in}{1.769553in}}{\pgfqpoint{3.063866in}{1.773671in}}%
\pgfpathcurveto{\pgfqpoint{3.059748in}{1.777789in}}{\pgfqpoint{3.054162in}{1.780103in}}{\pgfqpoint{3.048338in}{1.780103in}}%
\pgfpathcurveto{\pgfqpoint{3.042514in}{1.780103in}}{\pgfqpoint{3.036928in}{1.777789in}}{\pgfqpoint{3.032810in}{1.773671in}}%
\pgfpathcurveto{\pgfqpoint{3.028692in}{1.769553in}}{\pgfqpoint{3.026378in}{1.763966in}}{\pgfqpoint{3.026378in}{1.758143in}}%
\pgfpathcurveto{\pgfqpoint{3.026378in}{1.752319in}}{\pgfqpoint{3.028692in}{1.746732in}}{\pgfqpoint{3.032810in}{1.742614in}}%
\pgfpathcurveto{\pgfqpoint{3.036928in}{1.738496in}}{\pgfqpoint{3.042514in}{1.736182in}}{\pgfqpoint{3.048338in}{1.736182in}}%
\pgfpathlineto{\pgfqpoint{3.048338in}{1.736182in}}%
\pgfpathclose%
\pgfusepath{stroke,fill}%
\end{pgfscope}%
\begin{pgfscope}%
\pgfpathrectangle{\pgfqpoint{0.100000in}{0.183744in}}{\pgfqpoint{4.506048in}{4.506048in}}%
\pgfusepath{clip}%
\pgfsetbuttcap%
\pgfsetroundjoin%
\definecolor{currentfill}{rgb}{1.000000,0.647059,0.000000}%
\pgfsetfillcolor{currentfill}%
\pgfsetfillopacity{0.700000}%
\pgfsetlinewidth{1.003750pt}%
\definecolor{currentstroke}{rgb}{1.000000,0.647059,0.000000}%
\pgfsetstrokecolor{currentstroke}%
\pgfsetstrokeopacity{0.700000}%
\pgfsetdash{}{0pt}%
\pgfpathmoveto{\pgfqpoint{1.870222in}{1.650231in}}%
\pgfpathcurveto{\pgfqpoint{1.876046in}{1.650231in}}{\pgfqpoint{1.881633in}{1.652545in}}{\pgfqpoint{1.885751in}{1.656663in}}%
\pgfpathcurveto{\pgfqpoint{1.889869in}{1.660781in}}{\pgfqpoint{1.892183in}{1.666367in}}{\pgfqpoint{1.892183in}{1.672191in}}%
\pgfpathcurveto{\pgfqpoint{1.892183in}{1.678015in}}{\pgfqpoint{1.889869in}{1.683601in}}{\pgfqpoint{1.885751in}{1.687719in}}%
\pgfpathcurveto{\pgfqpoint{1.881633in}{1.691837in}}{\pgfqpoint{1.876046in}{1.694151in}}{\pgfqpoint{1.870222in}{1.694151in}}%
\pgfpathcurveto{\pgfqpoint{1.864399in}{1.694151in}}{\pgfqpoint{1.858812in}{1.691837in}}{\pgfqpoint{1.854694in}{1.687719in}}%
\pgfpathcurveto{\pgfqpoint{1.850576in}{1.683601in}}{\pgfqpoint{1.848262in}{1.678015in}}{\pgfqpoint{1.848262in}{1.672191in}}%
\pgfpathcurveto{\pgfqpoint{1.848262in}{1.666367in}}{\pgfqpoint{1.850576in}{1.660781in}}{\pgfqpoint{1.854694in}{1.656663in}}%
\pgfpathcurveto{\pgfqpoint{1.858812in}{1.652545in}}{\pgfqpoint{1.864399in}{1.650231in}}{\pgfqpoint{1.870222in}{1.650231in}}%
\pgfpathlineto{\pgfqpoint{1.870222in}{1.650231in}}%
\pgfpathclose%
\pgfusepath{stroke,fill}%
\end{pgfscope}%
\begin{pgfscope}%
\pgfpathrectangle{\pgfqpoint{0.100000in}{0.183744in}}{\pgfqpoint{4.506048in}{4.506048in}}%
\pgfusepath{clip}%
\pgfsetbuttcap%
\pgfsetroundjoin%
\definecolor{currentfill}{rgb}{1.000000,0.647059,0.000000}%
\pgfsetfillcolor{currentfill}%
\pgfsetfillopacity{0.700000}%
\pgfsetlinewidth{1.003750pt}%
\definecolor{currentstroke}{rgb}{1.000000,0.647059,0.000000}%
\pgfsetstrokecolor{currentstroke}%
\pgfsetstrokeopacity{0.700000}%
\pgfsetdash{}{0pt}%
\pgfpathmoveto{\pgfqpoint{2.880677in}{1.844873in}}%
\pgfpathcurveto{\pgfqpoint{2.886501in}{1.844873in}}{\pgfqpoint{2.892087in}{1.847187in}}{\pgfqpoint{2.896205in}{1.851305in}}%
\pgfpathcurveto{\pgfqpoint{2.900323in}{1.855423in}}{\pgfqpoint{2.902637in}{1.861009in}}{\pgfqpoint{2.902637in}{1.866833in}}%
\pgfpathcurveto{\pgfqpoint{2.902637in}{1.872657in}}{\pgfqpoint{2.900323in}{1.878243in}}{\pgfqpoint{2.896205in}{1.882362in}}%
\pgfpathcurveto{\pgfqpoint{2.892087in}{1.886480in}}{\pgfqpoint{2.886501in}{1.888794in}}{\pgfqpoint{2.880677in}{1.888794in}}%
\pgfpathcurveto{\pgfqpoint{2.874853in}{1.888794in}}{\pgfqpoint{2.869267in}{1.886480in}}{\pgfqpoint{2.865148in}{1.882362in}}%
\pgfpathcurveto{\pgfqpoint{2.861030in}{1.878243in}}{\pgfqpoint{2.858716in}{1.872657in}}{\pgfqpoint{2.858716in}{1.866833in}}%
\pgfpathcurveto{\pgfqpoint{2.858716in}{1.861009in}}{\pgfqpoint{2.861030in}{1.855423in}}{\pgfqpoint{2.865148in}{1.851305in}}%
\pgfpathcurveto{\pgfqpoint{2.869267in}{1.847187in}}{\pgfqpoint{2.874853in}{1.844873in}}{\pgfqpoint{2.880677in}{1.844873in}}%
\pgfpathlineto{\pgfqpoint{2.880677in}{1.844873in}}%
\pgfpathclose%
\pgfusepath{stroke,fill}%
\end{pgfscope}%
\begin{pgfscope}%
\pgfpathrectangle{\pgfqpoint{0.100000in}{0.183744in}}{\pgfqpoint{4.506048in}{4.506048in}}%
\pgfusepath{clip}%
\pgfsetbuttcap%
\pgfsetroundjoin%
\definecolor{currentfill}{rgb}{1.000000,0.647059,0.000000}%
\pgfsetfillcolor{currentfill}%
\pgfsetfillopacity{0.700000}%
\pgfsetlinewidth{1.003750pt}%
\definecolor{currentstroke}{rgb}{1.000000,0.647059,0.000000}%
\pgfsetstrokecolor{currentstroke}%
\pgfsetstrokeopacity{0.700000}%
\pgfsetdash{}{0pt}%
\pgfpathmoveto{\pgfqpoint{1.748979in}{2.007667in}}%
\pgfpathcurveto{\pgfqpoint{1.754803in}{2.007667in}}{\pgfqpoint{1.760389in}{2.009981in}}{\pgfqpoint{1.764507in}{2.014099in}}%
\pgfpathcurveto{\pgfqpoint{1.768625in}{2.018217in}}{\pgfqpoint{1.770939in}{2.023804in}}{\pgfqpoint{1.770939in}{2.029628in}}%
\pgfpathcurveto{\pgfqpoint{1.770939in}{2.035451in}}{\pgfqpoint{1.768625in}{2.041038in}}{\pgfqpoint{1.764507in}{2.045156in}}%
\pgfpathcurveto{\pgfqpoint{1.760389in}{2.049274in}}{\pgfqpoint{1.754803in}{2.051588in}}{\pgfqpoint{1.748979in}{2.051588in}}%
\pgfpathcurveto{\pgfqpoint{1.743155in}{2.051588in}}{\pgfqpoint{1.737569in}{2.049274in}}{\pgfqpoint{1.733451in}{2.045156in}}%
\pgfpathcurveto{\pgfqpoint{1.729333in}{2.041038in}}{\pgfqpoint{1.727019in}{2.035451in}}{\pgfqpoint{1.727019in}{2.029628in}}%
\pgfpathcurveto{\pgfqpoint{1.727019in}{2.023804in}}{\pgfqpoint{1.729333in}{2.018217in}}{\pgfqpoint{1.733451in}{2.014099in}}%
\pgfpathcurveto{\pgfqpoint{1.737569in}{2.009981in}}{\pgfqpoint{1.743155in}{2.007667in}}{\pgfqpoint{1.748979in}{2.007667in}}%
\pgfpathlineto{\pgfqpoint{1.748979in}{2.007667in}}%
\pgfpathclose%
\pgfusepath{stroke,fill}%
\end{pgfscope}%
\begin{pgfscope}%
\pgfpathrectangle{\pgfqpoint{0.100000in}{0.183744in}}{\pgfqpoint{4.506048in}{4.506048in}}%
\pgfusepath{clip}%
\pgfsetbuttcap%
\pgfsetroundjoin%
\definecolor{currentfill}{rgb}{1.000000,0.647059,0.000000}%
\pgfsetfillcolor{currentfill}%
\pgfsetfillopacity{0.700000}%
\pgfsetlinewidth{1.003750pt}%
\definecolor{currentstroke}{rgb}{1.000000,0.647059,0.000000}%
\pgfsetstrokecolor{currentstroke}%
\pgfsetstrokeopacity{0.700000}%
\pgfsetdash{}{0pt}%
\pgfpathmoveto{\pgfqpoint{4.001177in}{3.031639in}}%
\pgfpathcurveto{\pgfqpoint{4.007001in}{3.031639in}}{\pgfqpoint{4.012587in}{3.033953in}}{\pgfqpoint{4.016705in}{3.038071in}}%
\pgfpathcurveto{\pgfqpoint{4.020823in}{3.042189in}}{\pgfqpoint{4.023137in}{3.047776in}}{\pgfqpoint{4.023137in}{3.053599in}}%
\pgfpathcurveto{\pgfqpoint{4.023137in}{3.059423in}}{\pgfqpoint{4.020823in}{3.065010in}}{\pgfqpoint{4.016705in}{3.069128in}}%
\pgfpathcurveto{\pgfqpoint{4.012587in}{3.073246in}}{\pgfqpoint{4.007001in}{3.075560in}}{\pgfqpoint{4.001177in}{3.075560in}}%
\pgfpathcurveto{\pgfqpoint{3.995353in}{3.075560in}}{\pgfqpoint{3.989767in}{3.073246in}}{\pgfqpoint{3.985649in}{3.069128in}}%
\pgfpathcurveto{\pgfqpoint{3.981530in}{3.065010in}}{\pgfqpoint{3.979217in}{3.059423in}}{\pgfqpoint{3.979217in}{3.053599in}}%
\pgfpathcurveto{\pgfqpoint{3.979217in}{3.047776in}}{\pgfqpoint{3.981530in}{3.042189in}}{\pgfqpoint{3.985649in}{3.038071in}}%
\pgfpathcurveto{\pgfqpoint{3.989767in}{3.033953in}}{\pgfqpoint{3.995353in}{3.031639in}}{\pgfqpoint{4.001177in}{3.031639in}}%
\pgfpathlineto{\pgfqpoint{4.001177in}{3.031639in}}%
\pgfpathclose%
\pgfusepath{stroke,fill}%
\end{pgfscope}%
\begin{pgfscope}%
\pgfpathrectangle{\pgfqpoint{0.100000in}{0.183744in}}{\pgfqpoint{4.506048in}{4.506048in}}%
\pgfusepath{clip}%
\pgfsetbuttcap%
\pgfsetroundjoin%
\definecolor{currentfill}{rgb}{1.000000,0.647059,0.000000}%
\pgfsetfillcolor{currentfill}%
\pgfsetfillopacity{0.700000}%
\pgfsetlinewidth{1.003750pt}%
\definecolor{currentstroke}{rgb}{1.000000,0.647059,0.000000}%
\pgfsetstrokecolor{currentstroke}%
\pgfsetstrokeopacity{0.700000}%
\pgfsetdash{}{0pt}%
\pgfpathmoveto{\pgfqpoint{1.830387in}{2.341800in}}%
\pgfpathcurveto{\pgfqpoint{1.836211in}{2.341800in}}{\pgfqpoint{1.841797in}{2.344114in}}{\pgfqpoint{1.845916in}{2.348232in}}%
\pgfpathcurveto{\pgfqpoint{1.850034in}{2.352350in}}{\pgfqpoint{1.852348in}{2.357937in}}{\pgfqpoint{1.852348in}{2.363761in}}%
\pgfpathcurveto{\pgfqpoint{1.852348in}{2.369585in}}{\pgfqpoint{1.850034in}{2.375171in}}{\pgfqpoint{1.845916in}{2.379289in}}%
\pgfpathcurveto{\pgfqpoint{1.841797in}{2.383407in}}{\pgfqpoint{1.836211in}{2.385721in}}{\pgfqpoint{1.830387in}{2.385721in}}%
\pgfpathcurveto{\pgfqpoint{1.824563in}{2.385721in}}{\pgfqpoint{1.818977in}{2.383407in}}{\pgfqpoint{1.814859in}{2.379289in}}%
\pgfpathcurveto{\pgfqpoint{1.810741in}{2.375171in}}{\pgfqpoint{1.808427in}{2.369585in}}{\pgfqpoint{1.808427in}{2.363761in}}%
\pgfpathcurveto{\pgfqpoint{1.808427in}{2.357937in}}{\pgfqpoint{1.810741in}{2.352350in}}{\pgfqpoint{1.814859in}{2.348232in}}%
\pgfpathcurveto{\pgfqpoint{1.818977in}{2.344114in}}{\pgfqpoint{1.824563in}{2.341800in}}{\pgfqpoint{1.830387in}{2.341800in}}%
\pgfpathlineto{\pgfqpoint{1.830387in}{2.341800in}}%
\pgfpathclose%
\pgfusepath{stroke,fill}%
\end{pgfscope}%
\begin{pgfscope}%
\pgfpathrectangle{\pgfqpoint{0.100000in}{0.183744in}}{\pgfqpoint{4.506048in}{4.506048in}}%
\pgfusepath{clip}%
\pgfsetbuttcap%
\pgfsetroundjoin%
\definecolor{currentfill}{rgb}{1.000000,0.647059,0.000000}%
\pgfsetfillcolor{currentfill}%
\pgfsetfillopacity{0.700000}%
\pgfsetlinewidth{1.003750pt}%
\definecolor{currentstroke}{rgb}{1.000000,0.647059,0.000000}%
\pgfsetstrokecolor{currentstroke}%
\pgfsetstrokeopacity{0.700000}%
\pgfsetdash{}{0pt}%
\pgfpathmoveto{\pgfqpoint{3.419357in}{2.283053in}}%
\pgfpathcurveto{\pgfqpoint{3.425181in}{2.283053in}}{\pgfqpoint{3.430767in}{2.285367in}}{\pgfqpoint{3.434885in}{2.289485in}}%
\pgfpathcurveto{\pgfqpoint{3.439003in}{2.293604in}}{\pgfqpoint{3.441317in}{2.299190in}}{\pgfqpoint{3.441317in}{2.305014in}}%
\pgfpathcurveto{\pgfqpoint{3.441317in}{2.310838in}}{\pgfqpoint{3.439003in}{2.316424in}}{\pgfqpoint{3.434885in}{2.320542in}}%
\pgfpathcurveto{\pgfqpoint{3.430767in}{2.324660in}}{\pgfqpoint{3.425181in}{2.326974in}}{\pgfqpoint{3.419357in}{2.326974in}}%
\pgfpathcurveto{\pgfqpoint{3.413533in}{2.326974in}}{\pgfqpoint{3.407947in}{2.324660in}}{\pgfqpoint{3.403829in}{2.320542in}}%
\pgfpathcurveto{\pgfqpoint{3.399711in}{2.316424in}}{\pgfqpoint{3.397397in}{2.310838in}}{\pgfqpoint{3.397397in}{2.305014in}}%
\pgfpathcurveto{\pgfqpoint{3.397397in}{2.299190in}}{\pgfqpoint{3.399711in}{2.293604in}}{\pgfqpoint{3.403829in}{2.289485in}}%
\pgfpathcurveto{\pgfqpoint{3.407947in}{2.285367in}}{\pgfqpoint{3.413533in}{2.283053in}}{\pgfqpoint{3.419357in}{2.283053in}}%
\pgfpathlineto{\pgfqpoint{3.419357in}{2.283053in}}%
\pgfpathclose%
\pgfusepath{stroke,fill}%
\end{pgfscope}%
\begin{pgfscope}%
\pgfpathrectangle{\pgfqpoint{0.100000in}{0.183744in}}{\pgfqpoint{4.506048in}{4.506048in}}%
\pgfusepath{clip}%
\pgfsetbuttcap%
\pgfsetroundjoin%
\definecolor{currentfill}{rgb}{1.000000,0.647059,0.000000}%
\pgfsetfillcolor{currentfill}%
\pgfsetfillopacity{0.700000}%
\pgfsetlinewidth{1.003750pt}%
\definecolor{currentstroke}{rgb}{1.000000,0.647059,0.000000}%
\pgfsetstrokecolor{currentstroke}%
\pgfsetstrokeopacity{0.700000}%
\pgfsetdash{}{0pt}%
\pgfpathmoveto{\pgfqpoint{1.602784in}{1.004421in}}%
\pgfpathcurveto{\pgfqpoint{1.608608in}{1.004421in}}{\pgfqpoint{1.614194in}{1.006735in}}{\pgfqpoint{1.618312in}{1.010853in}}%
\pgfpathcurveto{\pgfqpoint{1.622431in}{1.014971in}}{\pgfqpoint{1.624744in}{1.020557in}}{\pgfqpoint{1.624744in}{1.026381in}}%
\pgfpathcurveto{\pgfqpoint{1.624744in}{1.032205in}}{\pgfqpoint{1.622431in}{1.037791in}}{\pgfqpoint{1.618312in}{1.041909in}}%
\pgfpathcurveto{\pgfqpoint{1.614194in}{1.046027in}}{\pgfqpoint{1.608608in}{1.048341in}}{\pgfqpoint{1.602784in}{1.048341in}}%
\pgfpathcurveto{\pgfqpoint{1.596960in}{1.048341in}}{\pgfqpoint{1.591374in}{1.046027in}}{\pgfqpoint{1.587256in}{1.041909in}}%
\pgfpathcurveto{\pgfqpoint{1.583138in}{1.037791in}}{\pgfqpoint{1.580824in}{1.032205in}}{\pgfqpoint{1.580824in}{1.026381in}}%
\pgfpathcurveto{\pgfqpoint{1.580824in}{1.020557in}}{\pgfqpoint{1.583138in}{1.014971in}}{\pgfqpoint{1.587256in}{1.010853in}}%
\pgfpathcurveto{\pgfqpoint{1.591374in}{1.006735in}}{\pgfqpoint{1.596960in}{1.004421in}}{\pgfqpoint{1.602784in}{1.004421in}}%
\pgfpathlineto{\pgfqpoint{1.602784in}{1.004421in}}%
\pgfpathclose%
\pgfusepath{stroke,fill}%
\end{pgfscope}%
\begin{pgfscope}%
\pgfpathrectangle{\pgfqpoint{0.100000in}{0.183744in}}{\pgfqpoint{4.506048in}{4.506048in}}%
\pgfusepath{clip}%
\pgfsetbuttcap%
\pgfsetroundjoin%
\definecolor{currentfill}{rgb}{1.000000,0.647059,0.000000}%
\pgfsetfillcolor{currentfill}%
\pgfsetfillopacity{0.700000}%
\pgfsetlinewidth{1.003750pt}%
\definecolor{currentstroke}{rgb}{1.000000,0.647059,0.000000}%
\pgfsetstrokecolor{currentstroke}%
\pgfsetstrokeopacity{0.700000}%
\pgfsetdash{}{0pt}%
\pgfpathmoveto{\pgfqpoint{2.519802in}{1.808762in}}%
\pgfpathcurveto{\pgfqpoint{2.525626in}{1.808762in}}{\pgfqpoint{2.531212in}{1.811076in}}{\pgfqpoint{2.535330in}{1.815194in}}%
\pgfpathcurveto{\pgfqpoint{2.539448in}{1.819313in}}{\pgfqpoint{2.541762in}{1.824899in}}{\pgfqpoint{2.541762in}{1.830723in}}%
\pgfpathcurveto{\pgfqpoint{2.541762in}{1.836547in}}{\pgfqpoint{2.539448in}{1.842133in}}{\pgfqpoint{2.535330in}{1.846251in}}%
\pgfpathcurveto{\pgfqpoint{2.531212in}{1.850369in}}{\pgfqpoint{2.525626in}{1.852683in}}{\pgfqpoint{2.519802in}{1.852683in}}%
\pgfpathcurveto{\pgfqpoint{2.513978in}{1.852683in}}{\pgfqpoint{2.508392in}{1.850369in}}{\pgfqpoint{2.504274in}{1.846251in}}%
\pgfpathcurveto{\pgfqpoint{2.500155in}{1.842133in}}{\pgfqpoint{2.497842in}{1.836547in}}{\pgfqpoint{2.497842in}{1.830723in}}%
\pgfpathcurveto{\pgfqpoint{2.497842in}{1.824899in}}{\pgfqpoint{2.500155in}{1.819313in}}{\pgfqpoint{2.504274in}{1.815194in}}%
\pgfpathcurveto{\pgfqpoint{2.508392in}{1.811076in}}{\pgfqpoint{2.513978in}{1.808762in}}{\pgfqpoint{2.519802in}{1.808762in}}%
\pgfpathlineto{\pgfqpoint{2.519802in}{1.808762in}}%
\pgfpathclose%
\pgfusepath{stroke,fill}%
\end{pgfscope}%
\begin{pgfscope}%
\pgfpathrectangle{\pgfqpoint{0.100000in}{0.183744in}}{\pgfqpoint{4.506048in}{4.506048in}}%
\pgfusepath{clip}%
\pgfsetbuttcap%
\pgfsetroundjoin%
\definecolor{currentfill}{rgb}{1.000000,0.647059,0.000000}%
\pgfsetfillcolor{currentfill}%
\pgfsetfillopacity{0.700000}%
\pgfsetlinewidth{1.003750pt}%
\definecolor{currentstroke}{rgb}{1.000000,0.647059,0.000000}%
\pgfsetstrokecolor{currentstroke}%
\pgfsetstrokeopacity{0.700000}%
\pgfsetdash{}{0pt}%
\pgfpathmoveto{\pgfqpoint{2.229763in}{1.277795in}}%
\pgfpathcurveto{\pgfqpoint{2.235587in}{1.277795in}}{\pgfqpoint{2.241173in}{1.280109in}}{\pgfqpoint{2.245291in}{1.284227in}}%
\pgfpathcurveto{\pgfqpoint{2.249409in}{1.288345in}}{\pgfqpoint{2.251723in}{1.293931in}}{\pgfqpoint{2.251723in}{1.299755in}}%
\pgfpathcurveto{\pgfqpoint{2.251723in}{1.305579in}}{\pgfqpoint{2.249409in}{1.311165in}}{\pgfqpoint{2.245291in}{1.315283in}}%
\pgfpathcurveto{\pgfqpoint{2.241173in}{1.319402in}}{\pgfqpoint{2.235587in}{1.321715in}}{\pgfqpoint{2.229763in}{1.321715in}}%
\pgfpathcurveto{\pgfqpoint{2.223939in}{1.321715in}}{\pgfqpoint{2.218353in}{1.319402in}}{\pgfqpoint{2.214235in}{1.315283in}}%
\pgfpathcurveto{\pgfqpoint{2.210117in}{1.311165in}}{\pgfqpoint{2.207803in}{1.305579in}}{\pgfqpoint{2.207803in}{1.299755in}}%
\pgfpathcurveto{\pgfqpoint{2.207803in}{1.293931in}}{\pgfqpoint{2.210117in}{1.288345in}}{\pgfqpoint{2.214235in}{1.284227in}}%
\pgfpathcurveto{\pgfqpoint{2.218353in}{1.280109in}}{\pgfqpoint{2.223939in}{1.277795in}}{\pgfqpoint{2.229763in}{1.277795in}}%
\pgfpathlineto{\pgfqpoint{2.229763in}{1.277795in}}%
\pgfpathclose%
\pgfusepath{stroke,fill}%
\end{pgfscope}%
\begin{pgfscope}%
\pgfpathrectangle{\pgfqpoint{0.100000in}{0.183744in}}{\pgfqpoint{4.506048in}{4.506048in}}%
\pgfusepath{clip}%
\pgfsetbuttcap%
\pgfsetroundjoin%
\definecolor{currentfill}{rgb}{1.000000,0.647059,0.000000}%
\pgfsetfillcolor{currentfill}%
\pgfsetfillopacity{0.700000}%
\pgfsetlinewidth{1.003750pt}%
\definecolor{currentstroke}{rgb}{1.000000,0.647059,0.000000}%
\pgfsetstrokecolor{currentstroke}%
\pgfsetstrokeopacity{0.700000}%
\pgfsetdash{}{0pt}%
\pgfpathmoveto{\pgfqpoint{2.201797in}{2.055739in}}%
\pgfpathcurveto{\pgfqpoint{2.207620in}{2.055739in}}{\pgfqpoint{2.213207in}{2.058053in}}{\pgfqpoint{2.217325in}{2.062171in}}%
\pgfpathcurveto{\pgfqpoint{2.221443in}{2.066289in}}{\pgfqpoint{2.223757in}{2.071876in}}{\pgfqpoint{2.223757in}{2.077700in}}%
\pgfpathcurveto{\pgfqpoint{2.223757in}{2.083523in}}{\pgfqpoint{2.221443in}{2.089110in}}{\pgfqpoint{2.217325in}{2.093228in}}%
\pgfpathcurveto{\pgfqpoint{2.213207in}{2.097346in}}{\pgfqpoint{2.207620in}{2.099660in}}{\pgfqpoint{2.201797in}{2.099660in}}%
\pgfpathcurveto{\pgfqpoint{2.195973in}{2.099660in}}{\pgfqpoint{2.190386in}{2.097346in}}{\pgfqpoint{2.186268in}{2.093228in}}%
\pgfpathcurveto{\pgfqpoint{2.182150in}{2.089110in}}{\pgfqpoint{2.179836in}{2.083523in}}{\pgfqpoint{2.179836in}{2.077700in}}%
\pgfpathcurveto{\pgfqpoint{2.179836in}{2.071876in}}{\pgfqpoint{2.182150in}{2.066289in}}{\pgfqpoint{2.186268in}{2.062171in}}%
\pgfpathcurveto{\pgfqpoint{2.190386in}{2.058053in}}{\pgfqpoint{2.195973in}{2.055739in}}{\pgfqpoint{2.201797in}{2.055739in}}%
\pgfpathlineto{\pgfqpoint{2.201797in}{2.055739in}}%
\pgfpathclose%
\pgfusepath{stroke,fill}%
\end{pgfscope}%
\begin{pgfscope}%
\pgfpathrectangle{\pgfqpoint{0.100000in}{0.183744in}}{\pgfqpoint{4.506048in}{4.506048in}}%
\pgfusepath{clip}%
\pgfsetbuttcap%
\pgfsetroundjoin%
\definecolor{currentfill}{rgb}{1.000000,0.647059,0.000000}%
\pgfsetfillcolor{currentfill}%
\pgfsetfillopacity{0.700000}%
\pgfsetlinewidth{1.003750pt}%
\definecolor{currentstroke}{rgb}{1.000000,0.647059,0.000000}%
\pgfsetstrokecolor{currentstroke}%
\pgfsetstrokeopacity{0.700000}%
\pgfsetdash{}{0pt}%
\pgfpathmoveto{\pgfqpoint{2.347466in}{2.148624in}}%
\pgfpathcurveto{\pgfqpoint{2.353290in}{2.148624in}}{\pgfqpoint{2.358876in}{2.150938in}}{\pgfqpoint{2.362994in}{2.155056in}}%
\pgfpathcurveto{\pgfqpoint{2.367112in}{2.159174in}}{\pgfqpoint{2.369426in}{2.164761in}}{\pgfqpoint{2.369426in}{2.170584in}}%
\pgfpathcurveto{\pgfqpoint{2.369426in}{2.176408in}}{\pgfqpoint{2.367112in}{2.181995in}}{\pgfqpoint{2.362994in}{2.186113in}}%
\pgfpathcurveto{\pgfqpoint{2.358876in}{2.190231in}}{\pgfqpoint{2.353290in}{2.192545in}}{\pgfqpoint{2.347466in}{2.192545in}}%
\pgfpathcurveto{\pgfqpoint{2.341642in}{2.192545in}}{\pgfqpoint{2.336056in}{2.190231in}}{\pgfqpoint{2.331937in}{2.186113in}}%
\pgfpathcurveto{\pgfqpoint{2.327819in}{2.181995in}}{\pgfqpoint{2.325505in}{2.176408in}}{\pgfqpoint{2.325505in}{2.170584in}}%
\pgfpathcurveto{\pgfqpoint{2.325505in}{2.164761in}}{\pgfqpoint{2.327819in}{2.159174in}}{\pgfqpoint{2.331937in}{2.155056in}}%
\pgfpathcurveto{\pgfqpoint{2.336056in}{2.150938in}}{\pgfqpoint{2.341642in}{2.148624in}}{\pgfqpoint{2.347466in}{2.148624in}}%
\pgfpathlineto{\pgfqpoint{2.347466in}{2.148624in}}%
\pgfpathclose%
\pgfusepath{stroke,fill}%
\end{pgfscope}%
\begin{pgfscope}%
\pgfpathrectangle{\pgfqpoint{0.100000in}{0.183744in}}{\pgfqpoint{4.506048in}{4.506048in}}%
\pgfusepath{clip}%
\pgfsetbuttcap%
\pgfsetroundjoin%
\definecolor{currentfill}{rgb}{1.000000,0.647059,0.000000}%
\pgfsetfillcolor{currentfill}%
\pgfsetfillopacity{0.700000}%
\pgfsetlinewidth{1.003750pt}%
\definecolor{currentstroke}{rgb}{1.000000,0.647059,0.000000}%
\pgfsetstrokecolor{currentstroke}%
\pgfsetstrokeopacity{0.700000}%
\pgfsetdash{}{0pt}%
\pgfpathmoveto{\pgfqpoint{2.299947in}{2.095625in}}%
\pgfpathcurveto{\pgfqpoint{2.305771in}{2.095625in}}{\pgfqpoint{2.311357in}{2.097939in}}{\pgfqpoint{2.315475in}{2.102057in}}%
\pgfpathcurveto{\pgfqpoint{2.319593in}{2.106175in}}{\pgfqpoint{2.321907in}{2.111762in}}{\pgfqpoint{2.321907in}{2.117586in}}%
\pgfpathcurveto{\pgfqpoint{2.321907in}{2.123409in}}{\pgfqpoint{2.319593in}{2.128996in}}{\pgfqpoint{2.315475in}{2.133114in}}%
\pgfpathcurveto{\pgfqpoint{2.311357in}{2.137232in}}{\pgfqpoint{2.305771in}{2.139546in}}{\pgfqpoint{2.299947in}{2.139546in}}%
\pgfpathcurveto{\pgfqpoint{2.294123in}{2.139546in}}{\pgfqpoint{2.288537in}{2.137232in}}{\pgfqpoint{2.284418in}{2.133114in}}%
\pgfpathcurveto{\pgfqpoint{2.280300in}{2.128996in}}{\pgfqpoint{2.277986in}{2.123409in}}{\pgfqpoint{2.277986in}{2.117586in}}%
\pgfpathcurveto{\pgfqpoint{2.277986in}{2.111762in}}{\pgfqpoint{2.280300in}{2.106175in}}{\pgfqpoint{2.284418in}{2.102057in}}%
\pgfpathcurveto{\pgfqpoint{2.288537in}{2.097939in}}{\pgfqpoint{2.294123in}{2.095625in}}{\pgfqpoint{2.299947in}{2.095625in}}%
\pgfpathlineto{\pgfqpoint{2.299947in}{2.095625in}}%
\pgfpathclose%
\pgfusepath{stroke,fill}%
\end{pgfscope}%
\begin{pgfscope}%
\pgfpathrectangle{\pgfqpoint{0.100000in}{0.183744in}}{\pgfqpoint{4.506048in}{4.506048in}}%
\pgfusepath{clip}%
\pgfsetbuttcap%
\pgfsetroundjoin%
\definecolor{currentfill}{rgb}{1.000000,0.647059,0.000000}%
\pgfsetfillcolor{currentfill}%
\pgfsetfillopacity{0.700000}%
\pgfsetlinewidth{1.003750pt}%
\definecolor{currentstroke}{rgb}{1.000000,0.647059,0.000000}%
\pgfsetstrokecolor{currentstroke}%
\pgfsetstrokeopacity{0.700000}%
\pgfsetdash{}{0pt}%
\pgfpathmoveto{\pgfqpoint{1.860499in}{2.049276in}}%
\pgfpathcurveto{\pgfqpoint{1.866323in}{2.049276in}}{\pgfqpoint{1.871909in}{2.051590in}}{\pgfqpoint{1.876027in}{2.055708in}}%
\pgfpathcurveto{\pgfqpoint{1.880145in}{2.059826in}}{\pgfqpoint{1.882459in}{2.065412in}}{\pgfqpoint{1.882459in}{2.071236in}}%
\pgfpathcurveto{\pgfqpoint{1.882459in}{2.077060in}}{\pgfqpoint{1.880145in}{2.082646in}}{\pgfqpoint{1.876027in}{2.086765in}}%
\pgfpathcurveto{\pgfqpoint{1.871909in}{2.090883in}}{\pgfqpoint{1.866323in}{2.093197in}}{\pgfqpoint{1.860499in}{2.093197in}}%
\pgfpathcurveto{\pgfqpoint{1.854675in}{2.093197in}}{\pgfqpoint{1.849089in}{2.090883in}}{\pgfqpoint{1.844971in}{2.086765in}}%
\pgfpathcurveto{\pgfqpoint{1.840853in}{2.082646in}}{\pgfqpoint{1.838539in}{2.077060in}}{\pgfqpoint{1.838539in}{2.071236in}}%
\pgfpathcurveto{\pgfqpoint{1.838539in}{2.065412in}}{\pgfqpoint{1.840853in}{2.059826in}}{\pgfqpoint{1.844971in}{2.055708in}}%
\pgfpathcurveto{\pgfqpoint{1.849089in}{2.051590in}}{\pgfqpoint{1.854675in}{2.049276in}}{\pgfqpoint{1.860499in}{2.049276in}}%
\pgfpathlineto{\pgfqpoint{1.860499in}{2.049276in}}%
\pgfpathclose%
\pgfusepath{stroke,fill}%
\end{pgfscope}%
\begin{pgfscope}%
\pgfpathrectangle{\pgfqpoint{0.100000in}{0.183744in}}{\pgfqpoint{4.506048in}{4.506048in}}%
\pgfusepath{clip}%
\pgfsetbuttcap%
\pgfsetroundjoin%
\definecolor{currentfill}{rgb}{1.000000,0.647059,0.000000}%
\pgfsetfillcolor{currentfill}%
\pgfsetfillopacity{0.700000}%
\pgfsetlinewidth{1.003750pt}%
\definecolor{currentstroke}{rgb}{1.000000,0.647059,0.000000}%
\pgfsetstrokecolor{currentstroke}%
\pgfsetstrokeopacity{0.700000}%
\pgfsetdash{}{0pt}%
\pgfpathmoveto{\pgfqpoint{2.077328in}{1.523837in}}%
\pgfpathcurveto{\pgfqpoint{2.083151in}{1.523837in}}{\pgfqpoint{2.088738in}{1.526151in}}{\pgfqpoint{2.092856in}{1.530269in}}%
\pgfpathcurveto{\pgfqpoint{2.096974in}{1.534387in}}{\pgfqpoint{2.099288in}{1.539973in}}{\pgfqpoint{2.099288in}{1.545797in}}%
\pgfpathcurveto{\pgfqpoint{2.099288in}{1.551621in}}{\pgfqpoint{2.096974in}{1.557207in}}{\pgfqpoint{2.092856in}{1.561325in}}%
\pgfpathcurveto{\pgfqpoint{2.088738in}{1.565444in}}{\pgfqpoint{2.083151in}{1.567757in}}{\pgfqpoint{2.077328in}{1.567757in}}%
\pgfpathcurveto{\pgfqpoint{2.071504in}{1.567757in}}{\pgfqpoint{2.065917in}{1.565444in}}{\pgfqpoint{2.061799in}{1.561325in}}%
\pgfpathcurveto{\pgfqpoint{2.057681in}{1.557207in}}{\pgfqpoint{2.055367in}{1.551621in}}{\pgfqpoint{2.055367in}{1.545797in}}%
\pgfpathcurveto{\pgfqpoint{2.055367in}{1.539973in}}{\pgfqpoint{2.057681in}{1.534387in}}{\pgfqpoint{2.061799in}{1.530269in}}%
\pgfpathcurveto{\pgfqpoint{2.065917in}{1.526151in}}{\pgfqpoint{2.071504in}{1.523837in}}{\pgfqpoint{2.077328in}{1.523837in}}%
\pgfpathlineto{\pgfqpoint{2.077328in}{1.523837in}}%
\pgfpathclose%
\pgfusepath{stroke,fill}%
\end{pgfscope}%
\begin{pgfscope}%
\pgfpathrectangle{\pgfqpoint{0.100000in}{0.183744in}}{\pgfqpoint{4.506048in}{4.506048in}}%
\pgfusepath{clip}%
\pgfsetbuttcap%
\pgfsetroundjoin%
\definecolor{currentfill}{rgb}{1.000000,0.647059,0.000000}%
\pgfsetfillcolor{currentfill}%
\pgfsetfillopacity{0.700000}%
\pgfsetlinewidth{1.003750pt}%
\definecolor{currentstroke}{rgb}{1.000000,0.647059,0.000000}%
\pgfsetstrokecolor{currentstroke}%
\pgfsetstrokeopacity{0.700000}%
\pgfsetdash{}{0pt}%
\pgfpathmoveto{\pgfqpoint{1.656219in}{1.395789in}}%
\pgfpathcurveto{\pgfqpoint{1.662043in}{1.395789in}}{\pgfqpoint{1.667629in}{1.398103in}}{\pgfqpoint{1.671747in}{1.402221in}}%
\pgfpathcurveto{\pgfqpoint{1.675865in}{1.406339in}}{\pgfqpoint{1.678179in}{1.411926in}}{\pgfqpoint{1.678179in}{1.417749in}}%
\pgfpathcurveto{\pgfqpoint{1.678179in}{1.423573in}}{\pgfqpoint{1.675865in}{1.429160in}}{\pgfqpoint{1.671747in}{1.433278in}}%
\pgfpathcurveto{\pgfqpoint{1.667629in}{1.437396in}}{\pgfqpoint{1.662043in}{1.439710in}}{\pgfqpoint{1.656219in}{1.439710in}}%
\pgfpathcurveto{\pgfqpoint{1.650395in}{1.439710in}}{\pgfqpoint{1.644809in}{1.437396in}}{\pgfqpoint{1.640690in}{1.433278in}}%
\pgfpathcurveto{\pgfqpoint{1.636572in}{1.429160in}}{\pgfqpoint{1.634258in}{1.423573in}}{\pgfqpoint{1.634258in}{1.417749in}}%
\pgfpathcurveto{\pgfqpoint{1.634258in}{1.411926in}}{\pgfqpoint{1.636572in}{1.406339in}}{\pgfqpoint{1.640690in}{1.402221in}}%
\pgfpathcurveto{\pgfqpoint{1.644809in}{1.398103in}}{\pgfqpoint{1.650395in}{1.395789in}}{\pgfqpoint{1.656219in}{1.395789in}}%
\pgfpathlineto{\pgfqpoint{1.656219in}{1.395789in}}%
\pgfpathclose%
\pgfusepath{stroke,fill}%
\end{pgfscope}%
\begin{pgfscope}%
\pgfpathrectangle{\pgfqpoint{0.100000in}{0.183744in}}{\pgfqpoint{4.506048in}{4.506048in}}%
\pgfusepath{clip}%
\pgfsetbuttcap%
\pgfsetroundjoin%
\definecolor{currentfill}{rgb}{0.000000,0.000000,1.000000}%
\pgfsetfillcolor{currentfill}%
\pgfsetfillopacity{0.700000}%
\pgfsetlinewidth{1.003750pt}%
\definecolor{currentstroke}{rgb}{0.000000,0.000000,1.000000}%
\pgfsetstrokecolor{currentstroke}%
\pgfsetstrokeopacity{0.700000}%
\pgfsetdash{}{0pt}%
\pgfpathmoveto{\pgfqpoint{2.867854in}{2.909863in}}%
\pgfpathcurveto{\pgfqpoint{2.873678in}{2.909863in}}{\pgfqpoint{2.879264in}{2.912176in}}{\pgfqpoint{2.883382in}{2.916295in}}%
\pgfpathcurveto{\pgfqpoint{2.887500in}{2.920413in}}{\pgfqpoint{2.889814in}{2.925999in}}{\pgfqpoint{2.889814in}{2.931823in}}%
\pgfpathcurveto{\pgfqpoint{2.889814in}{2.937647in}}{\pgfqpoint{2.887500in}{2.943233in}}{\pgfqpoint{2.883382in}{2.947351in}}%
\pgfpathcurveto{\pgfqpoint{2.879264in}{2.951469in}}{\pgfqpoint{2.873678in}{2.953783in}}{\pgfqpoint{2.867854in}{2.953783in}}%
\pgfpathcurveto{\pgfqpoint{2.862030in}{2.953783in}}{\pgfqpoint{2.856444in}{2.951469in}}{\pgfqpoint{2.852326in}{2.947351in}}%
\pgfpathcurveto{\pgfqpoint{2.848207in}{2.943233in}}{\pgfqpoint{2.845894in}{2.937647in}}{\pgfqpoint{2.845894in}{2.931823in}}%
\pgfpathcurveto{\pgfqpoint{2.845894in}{2.925999in}}{\pgfqpoint{2.848207in}{2.920413in}}{\pgfqpoint{2.852326in}{2.916295in}}%
\pgfpathcurveto{\pgfqpoint{2.856444in}{2.912176in}}{\pgfqpoint{2.862030in}{2.909863in}}{\pgfqpoint{2.867854in}{2.909863in}}%
\pgfpathlineto{\pgfqpoint{2.867854in}{2.909863in}}%
\pgfpathclose%
\pgfusepath{stroke,fill}%
\end{pgfscope}%
\begin{pgfscope}%
\pgfpathrectangle{\pgfqpoint{0.100000in}{0.183744in}}{\pgfqpoint{4.506048in}{4.506048in}}%
\pgfusepath{clip}%
\pgfsetbuttcap%
\pgfsetroundjoin%
\definecolor{currentfill}{rgb}{0.000000,0.000000,1.000000}%
\pgfsetfillcolor{currentfill}%
\pgfsetfillopacity{0.700000}%
\pgfsetlinewidth{1.003750pt}%
\definecolor{currentstroke}{rgb}{0.000000,0.000000,1.000000}%
\pgfsetstrokecolor{currentstroke}%
\pgfsetstrokeopacity{0.700000}%
\pgfsetdash{}{0pt}%
\pgfpathmoveto{\pgfqpoint{2.733480in}{2.803103in}}%
\pgfpathcurveto{\pgfqpoint{2.739304in}{2.803103in}}{\pgfqpoint{2.744890in}{2.805417in}}{\pgfqpoint{2.749008in}{2.809535in}}%
\pgfpathcurveto{\pgfqpoint{2.753126in}{2.813653in}}{\pgfqpoint{2.755440in}{2.819239in}}{\pgfqpoint{2.755440in}{2.825063in}}%
\pgfpathcurveto{\pgfqpoint{2.755440in}{2.830887in}}{\pgfqpoint{2.753126in}{2.836474in}}{\pgfqpoint{2.749008in}{2.840592in}}%
\pgfpathcurveto{\pgfqpoint{2.744890in}{2.844710in}}{\pgfqpoint{2.739304in}{2.847024in}}{\pgfqpoint{2.733480in}{2.847024in}}%
\pgfpathcurveto{\pgfqpoint{2.727656in}{2.847024in}}{\pgfqpoint{2.722069in}{2.844710in}}{\pgfqpoint{2.717951in}{2.840592in}}%
\pgfpathcurveto{\pgfqpoint{2.713833in}{2.836474in}}{\pgfqpoint{2.711519in}{2.830887in}}{\pgfqpoint{2.711519in}{2.825063in}}%
\pgfpathcurveto{\pgfqpoint{2.711519in}{2.819239in}}{\pgfqpoint{2.713833in}{2.813653in}}{\pgfqpoint{2.717951in}{2.809535in}}%
\pgfpathcurveto{\pgfqpoint{2.722069in}{2.805417in}}{\pgfqpoint{2.727656in}{2.803103in}}{\pgfqpoint{2.733480in}{2.803103in}}%
\pgfpathlineto{\pgfqpoint{2.733480in}{2.803103in}}%
\pgfpathclose%
\pgfusepath{stroke,fill}%
\end{pgfscope}%
\begin{pgfscope}%
\pgfpathrectangle{\pgfqpoint{0.100000in}{0.183744in}}{\pgfqpoint{4.506048in}{4.506048in}}%
\pgfusepath{clip}%
\pgfsetbuttcap%
\pgfsetroundjoin%
\definecolor{currentfill}{rgb}{0.000000,0.000000,1.000000}%
\pgfsetfillcolor{currentfill}%
\pgfsetfillopacity{0.700000}%
\pgfsetlinewidth{1.003750pt}%
\definecolor{currentstroke}{rgb}{0.000000,0.000000,1.000000}%
\pgfsetstrokecolor{currentstroke}%
\pgfsetstrokeopacity{0.700000}%
\pgfsetdash{}{0pt}%
\pgfpathmoveto{\pgfqpoint{3.034540in}{2.986298in}}%
\pgfpathcurveto{\pgfqpoint{3.040364in}{2.986298in}}{\pgfqpoint{3.045951in}{2.988612in}}{\pgfqpoint{3.050069in}{2.992730in}}%
\pgfpathcurveto{\pgfqpoint{3.054187in}{2.996848in}}{\pgfqpoint{3.056501in}{3.002434in}}{\pgfqpoint{3.056501in}{3.008258in}}%
\pgfpathcurveto{\pgfqpoint{3.056501in}{3.014082in}}{\pgfqpoint{3.054187in}{3.019668in}}{\pgfqpoint{3.050069in}{3.023786in}}%
\pgfpathcurveto{\pgfqpoint{3.045951in}{3.027904in}}{\pgfqpoint{3.040364in}{3.030218in}}{\pgfqpoint{3.034540in}{3.030218in}}%
\pgfpathcurveto{\pgfqpoint{3.028716in}{3.030218in}}{\pgfqpoint{3.023130in}{3.027904in}}{\pgfqpoint{3.019012in}{3.023786in}}%
\pgfpathcurveto{\pgfqpoint{3.014894in}{3.019668in}}{\pgfqpoint{3.012580in}{3.014082in}}{\pgfqpoint{3.012580in}{3.008258in}}%
\pgfpathcurveto{\pgfqpoint{3.012580in}{3.002434in}}{\pgfqpoint{3.014894in}{2.996848in}}{\pgfqpoint{3.019012in}{2.992730in}}%
\pgfpathcurveto{\pgfqpoint{3.023130in}{2.988612in}}{\pgfqpoint{3.028716in}{2.986298in}}{\pgfqpoint{3.034540in}{2.986298in}}%
\pgfpathlineto{\pgfqpoint{3.034540in}{2.986298in}}%
\pgfpathclose%
\pgfusepath{stroke,fill}%
\end{pgfscope}%
\begin{pgfscope}%
\pgfpathrectangle{\pgfqpoint{0.100000in}{0.183744in}}{\pgfqpoint{4.506048in}{4.506048in}}%
\pgfusepath{clip}%
\pgfsetbuttcap%
\pgfsetroundjoin%
\definecolor{currentfill}{rgb}{0.000000,0.000000,1.000000}%
\pgfsetfillcolor{currentfill}%
\pgfsetfillopacity{0.700000}%
\pgfsetlinewidth{1.003750pt}%
\definecolor{currentstroke}{rgb}{0.000000,0.000000,1.000000}%
\pgfsetstrokecolor{currentstroke}%
\pgfsetstrokeopacity{0.700000}%
\pgfsetdash{}{0pt}%
\pgfpathmoveto{\pgfqpoint{1.996250in}{2.568290in}}%
\pgfpathcurveto{\pgfqpoint{2.002074in}{2.568290in}}{\pgfqpoint{2.007660in}{2.570604in}}{\pgfqpoint{2.011778in}{2.574722in}}%
\pgfpathcurveto{\pgfqpoint{2.015896in}{2.578841in}}{\pgfqpoint{2.018210in}{2.584427in}}{\pgfqpoint{2.018210in}{2.590251in}}%
\pgfpathcurveto{\pgfqpoint{2.018210in}{2.596075in}}{\pgfqpoint{2.015896in}{2.601661in}}{\pgfqpoint{2.011778in}{2.605779in}}%
\pgfpathcurveto{\pgfqpoint{2.007660in}{2.609897in}}{\pgfqpoint{2.002074in}{2.612211in}}{\pgfqpoint{1.996250in}{2.612211in}}%
\pgfpathcurveto{\pgfqpoint{1.990426in}{2.612211in}}{\pgfqpoint{1.984840in}{2.609897in}}{\pgfqpoint{1.980722in}{2.605779in}}%
\pgfpathcurveto{\pgfqpoint{1.976604in}{2.601661in}}{\pgfqpoint{1.974290in}{2.596075in}}{\pgfqpoint{1.974290in}{2.590251in}}%
\pgfpathcurveto{\pgfqpoint{1.974290in}{2.584427in}}{\pgfqpoint{1.976604in}{2.578841in}}{\pgfqpoint{1.980722in}{2.574722in}}%
\pgfpathcurveto{\pgfqpoint{1.984840in}{2.570604in}}{\pgfqpoint{1.990426in}{2.568290in}}{\pgfqpoint{1.996250in}{2.568290in}}%
\pgfpathlineto{\pgfqpoint{1.996250in}{2.568290in}}%
\pgfpathclose%
\pgfusepath{stroke,fill}%
\end{pgfscope}%
\begin{pgfscope}%
\pgfpathrectangle{\pgfqpoint{0.100000in}{0.183744in}}{\pgfqpoint{4.506048in}{4.506048in}}%
\pgfusepath{clip}%
\pgfsetbuttcap%
\pgfsetroundjoin%
\definecolor{currentfill}{rgb}{0.000000,0.000000,1.000000}%
\pgfsetfillcolor{currentfill}%
\pgfsetfillopacity{0.700000}%
\pgfsetlinewidth{1.003750pt}%
\definecolor{currentstroke}{rgb}{0.000000,0.000000,1.000000}%
\pgfsetstrokecolor{currentstroke}%
\pgfsetstrokeopacity{0.700000}%
\pgfsetdash{}{0pt}%
\pgfpathmoveto{\pgfqpoint{2.029125in}{2.811437in}}%
\pgfpathcurveto{\pgfqpoint{2.034949in}{2.811437in}}{\pgfqpoint{2.040535in}{2.813751in}}{\pgfqpoint{2.044653in}{2.817869in}}%
\pgfpathcurveto{\pgfqpoint{2.048771in}{2.821987in}}{\pgfqpoint{2.051085in}{2.827573in}}{\pgfqpoint{2.051085in}{2.833397in}}%
\pgfpathcurveto{\pgfqpoint{2.051085in}{2.839221in}}{\pgfqpoint{2.048771in}{2.844807in}}{\pgfqpoint{2.044653in}{2.848926in}}%
\pgfpathcurveto{\pgfqpoint{2.040535in}{2.853044in}}{\pgfqpoint{2.034949in}{2.855358in}}{\pgfqpoint{2.029125in}{2.855358in}}%
\pgfpathcurveto{\pgfqpoint{2.023301in}{2.855358in}}{\pgfqpoint{2.017715in}{2.853044in}}{\pgfqpoint{2.013596in}{2.848926in}}%
\pgfpathcurveto{\pgfqpoint{2.009478in}{2.844807in}}{\pgfqpoint{2.007164in}{2.839221in}}{\pgfqpoint{2.007164in}{2.833397in}}%
\pgfpathcurveto{\pgfqpoint{2.007164in}{2.827573in}}{\pgfqpoint{2.009478in}{2.821987in}}{\pgfqpoint{2.013596in}{2.817869in}}%
\pgfpathcurveto{\pgfqpoint{2.017715in}{2.813751in}}{\pgfqpoint{2.023301in}{2.811437in}}{\pgfqpoint{2.029125in}{2.811437in}}%
\pgfpathlineto{\pgfqpoint{2.029125in}{2.811437in}}%
\pgfpathclose%
\pgfusepath{stroke,fill}%
\end{pgfscope}%
\begin{pgfscope}%
\pgfpathrectangle{\pgfqpoint{0.100000in}{0.183744in}}{\pgfqpoint{4.506048in}{4.506048in}}%
\pgfusepath{clip}%
\pgfsetbuttcap%
\pgfsetroundjoin%
\definecolor{currentfill}{rgb}{0.000000,0.000000,1.000000}%
\pgfsetfillcolor{currentfill}%
\pgfsetfillopacity{0.700000}%
\pgfsetlinewidth{1.003750pt}%
\definecolor{currentstroke}{rgb}{0.000000,0.000000,1.000000}%
\pgfsetstrokecolor{currentstroke}%
\pgfsetstrokeopacity{0.700000}%
\pgfsetdash{}{0pt}%
\pgfpathmoveto{\pgfqpoint{2.283366in}{2.792323in}}%
\pgfpathcurveto{\pgfqpoint{2.289190in}{2.792323in}}{\pgfqpoint{2.294776in}{2.794637in}}{\pgfqpoint{2.298895in}{2.798755in}}%
\pgfpathcurveto{\pgfqpoint{2.303013in}{2.802873in}}{\pgfqpoint{2.305327in}{2.808459in}}{\pgfqpoint{2.305327in}{2.814283in}}%
\pgfpathcurveto{\pgfqpoint{2.305327in}{2.820107in}}{\pgfqpoint{2.303013in}{2.825693in}}{\pgfqpoint{2.298895in}{2.829812in}}%
\pgfpathcurveto{\pgfqpoint{2.294776in}{2.833930in}}{\pgfqpoint{2.289190in}{2.836244in}}{\pgfqpoint{2.283366in}{2.836244in}}%
\pgfpathcurveto{\pgfqpoint{2.277542in}{2.836244in}}{\pgfqpoint{2.271956in}{2.833930in}}{\pgfqpoint{2.267838in}{2.829812in}}%
\pgfpathcurveto{\pgfqpoint{2.263720in}{2.825693in}}{\pgfqpoint{2.261406in}{2.820107in}}{\pgfqpoint{2.261406in}{2.814283in}}%
\pgfpathcurveto{\pgfqpoint{2.261406in}{2.808459in}}{\pgfqpoint{2.263720in}{2.802873in}}{\pgfqpoint{2.267838in}{2.798755in}}%
\pgfpathcurveto{\pgfqpoint{2.271956in}{2.794637in}}{\pgfqpoint{2.277542in}{2.792323in}}{\pgfqpoint{2.283366in}{2.792323in}}%
\pgfpathlineto{\pgfqpoint{2.283366in}{2.792323in}}%
\pgfpathclose%
\pgfusepath{stroke,fill}%
\end{pgfscope}%
\begin{pgfscope}%
\pgfpathrectangle{\pgfqpoint{0.100000in}{0.183744in}}{\pgfqpoint{4.506048in}{4.506048in}}%
\pgfusepath{clip}%
\pgfsetbuttcap%
\pgfsetroundjoin%
\definecolor{currentfill}{rgb}{0.000000,0.000000,1.000000}%
\pgfsetfillcolor{currentfill}%
\pgfsetfillopacity{0.700000}%
\pgfsetlinewidth{1.003750pt}%
\definecolor{currentstroke}{rgb}{0.000000,0.000000,1.000000}%
\pgfsetstrokecolor{currentstroke}%
\pgfsetstrokeopacity{0.700000}%
\pgfsetdash{}{0pt}%
\pgfpathmoveto{\pgfqpoint{3.354589in}{2.730624in}}%
\pgfpathcurveto{\pgfqpoint{3.360413in}{2.730624in}}{\pgfqpoint{3.366000in}{2.732938in}}{\pgfqpoint{3.370118in}{2.737056in}}%
\pgfpathcurveto{\pgfqpoint{3.374236in}{2.741174in}}{\pgfqpoint{3.376550in}{2.746760in}}{\pgfqpoint{3.376550in}{2.752584in}}%
\pgfpathcurveto{\pgfqpoint{3.376550in}{2.758408in}}{\pgfqpoint{3.374236in}{2.763994in}}{\pgfqpoint{3.370118in}{2.768112in}}%
\pgfpathcurveto{\pgfqpoint{3.366000in}{2.772230in}}{\pgfqpoint{3.360413in}{2.774544in}}{\pgfqpoint{3.354589in}{2.774544in}}%
\pgfpathcurveto{\pgfqpoint{3.348766in}{2.774544in}}{\pgfqpoint{3.343179in}{2.772230in}}{\pgfqpoint{3.339061in}{2.768112in}}%
\pgfpathcurveto{\pgfqpoint{3.334943in}{2.763994in}}{\pgfqpoint{3.332629in}{2.758408in}}{\pgfqpoint{3.332629in}{2.752584in}}%
\pgfpathcurveto{\pgfqpoint{3.332629in}{2.746760in}}{\pgfqpoint{3.334943in}{2.741174in}}{\pgfqpoint{3.339061in}{2.737056in}}%
\pgfpathcurveto{\pgfqpoint{3.343179in}{2.732938in}}{\pgfqpoint{3.348766in}{2.730624in}}{\pgfqpoint{3.354589in}{2.730624in}}%
\pgfpathlineto{\pgfqpoint{3.354589in}{2.730624in}}%
\pgfpathclose%
\pgfusepath{stroke,fill}%
\end{pgfscope}%
\begin{pgfscope}%
\pgfpathrectangle{\pgfqpoint{0.100000in}{0.183744in}}{\pgfqpoint{4.506048in}{4.506048in}}%
\pgfusepath{clip}%
\pgfsetbuttcap%
\pgfsetroundjoin%
\definecolor{currentfill}{rgb}{0.000000,0.000000,1.000000}%
\pgfsetfillcolor{currentfill}%
\pgfsetfillopacity{0.700000}%
\pgfsetlinewidth{1.003750pt}%
\definecolor{currentstroke}{rgb}{0.000000,0.000000,1.000000}%
\pgfsetstrokecolor{currentstroke}%
\pgfsetstrokeopacity{0.700000}%
\pgfsetdash{}{0pt}%
\pgfpathmoveto{\pgfqpoint{1.114630in}{2.234625in}}%
\pgfpathcurveto{\pgfqpoint{1.120454in}{2.234625in}}{\pgfqpoint{1.126040in}{2.236939in}}{\pgfqpoint{1.130158in}{2.241057in}}%
\pgfpathcurveto{\pgfqpoint{1.134277in}{2.245175in}}{\pgfqpoint{1.136590in}{2.250762in}}{\pgfqpoint{1.136590in}{2.256585in}}%
\pgfpathcurveto{\pgfqpoint{1.136590in}{2.262409in}}{\pgfqpoint{1.134277in}{2.267996in}}{\pgfqpoint{1.130158in}{2.272114in}}%
\pgfpathcurveto{\pgfqpoint{1.126040in}{2.276232in}}{\pgfqpoint{1.120454in}{2.278546in}}{\pgfqpoint{1.114630in}{2.278546in}}%
\pgfpathcurveto{\pgfqpoint{1.108806in}{2.278546in}}{\pgfqpoint{1.103220in}{2.276232in}}{\pgfqpoint{1.099102in}{2.272114in}}%
\pgfpathcurveto{\pgfqpoint{1.094984in}{2.267996in}}{\pgfqpoint{1.092670in}{2.262409in}}{\pgfqpoint{1.092670in}{2.256585in}}%
\pgfpathcurveto{\pgfqpoint{1.092670in}{2.250762in}}{\pgfqpoint{1.094984in}{2.245175in}}{\pgfqpoint{1.099102in}{2.241057in}}%
\pgfpathcurveto{\pgfqpoint{1.103220in}{2.236939in}}{\pgfqpoint{1.108806in}{2.234625in}}{\pgfqpoint{1.114630in}{2.234625in}}%
\pgfpathlineto{\pgfqpoint{1.114630in}{2.234625in}}%
\pgfpathclose%
\pgfusepath{stroke,fill}%
\end{pgfscope}%
\begin{pgfscope}%
\pgfpathrectangle{\pgfqpoint{0.100000in}{0.183744in}}{\pgfqpoint{4.506048in}{4.506048in}}%
\pgfusepath{clip}%
\pgfsetbuttcap%
\pgfsetroundjoin%
\definecolor{currentfill}{rgb}{0.000000,0.000000,1.000000}%
\pgfsetfillcolor{currentfill}%
\pgfsetfillopacity{0.700000}%
\pgfsetlinewidth{1.003750pt}%
\definecolor{currentstroke}{rgb}{0.000000,0.000000,1.000000}%
\pgfsetstrokecolor{currentstroke}%
\pgfsetstrokeopacity{0.700000}%
\pgfsetdash{}{0pt}%
\pgfpathmoveto{\pgfqpoint{2.609465in}{3.189742in}}%
\pgfpathcurveto{\pgfqpoint{2.615289in}{3.189742in}}{\pgfqpoint{2.620875in}{3.192056in}}{\pgfqpoint{2.624994in}{3.196174in}}%
\pgfpathcurveto{\pgfqpoint{2.629112in}{3.200292in}}{\pgfqpoint{2.631426in}{3.205878in}}{\pgfqpoint{2.631426in}{3.211702in}}%
\pgfpathcurveto{\pgfqpoint{2.631426in}{3.217526in}}{\pgfqpoint{2.629112in}{3.223112in}}{\pgfqpoint{2.624994in}{3.227230in}}%
\pgfpathcurveto{\pgfqpoint{2.620875in}{3.231349in}}{\pgfqpoint{2.615289in}{3.233662in}}{\pgfqpoint{2.609465in}{3.233662in}}%
\pgfpathcurveto{\pgfqpoint{2.603641in}{3.233662in}}{\pgfqpoint{2.598055in}{3.231349in}}{\pgfqpoint{2.593937in}{3.227230in}}%
\pgfpathcurveto{\pgfqpoint{2.589819in}{3.223112in}}{\pgfqpoint{2.587505in}{3.217526in}}{\pgfqpoint{2.587505in}{3.211702in}}%
\pgfpathcurveto{\pgfqpoint{2.587505in}{3.205878in}}{\pgfqpoint{2.589819in}{3.200292in}}{\pgfqpoint{2.593937in}{3.196174in}}%
\pgfpathcurveto{\pgfqpoint{2.598055in}{3.192056in}}{\pgfqpoint{2.603641in}{3.189742in}}{\pgfqpoint{2.609465in}{3.189742in}}%
\pgfpathlineto{\pgfqpoint{2.609465in}{3.189742in}}%
\pgfpathclose%
\pgfusepath{stroke,fill}%
\end{pgfscope}%
\begin{pgfscope}%
\pgfpathrectangle{\pgfqpoint{0.100000in}{0.183744in}}{\pgfqpoint{4.506048in}{4.506048in}}%
\pgfusepath{clip}%
\pgfsetbuttcap%
\pgfsetroundjoin%
\definecolor{currentfill}{rgb}{0.000000,0.000000,1.000000}%
\pgfsetfillcolor{currentfill}%
\pgfsetfillopacity{0.700000}%
\pgfsetlinewidth{1.003750pt}%
\definecolor{currentstroke}{rgb}{0.000000,0.000000,1.000000}%
\pgfsetstrokecolor{currentstroke}%
\pgfsetstrokeopacity{0.700000}%
\pgfsetdash{}{0pt}%
\pgfpathmoveto{\pgfqpoint{2.607582in}{3.338077in}}%
\pgfpathcurveto{\pgfqpoint{2.613406in}{3.338077in}}{\pgfqpoint{2.618992in}{3.340391in}}{\pgfqpoint{2.623110in}{3.344509in}}%
\pgfpathcurveto{\pgfqpoint{2.627228in}{3.348628in}}{\pgfqpoint{2.629542in}{3.354214in}}{\pgfqpoint{2.629542in}{3.360038in}}%
\pgfpathcurveto{\pgfqpoint{2.629542in}{3.365862in}}{\pgfqpoint{2.627228in}{3.371448in}}{\pgfqpoint{2.623110in}{3.375566in}}%
\pgfpathcurveto{\pgfqpoint{2.618992in}{3.379684in}}{\pgfqpoint{2.613406in}{3.381998in}}{\pgfqpoint{2.607582in}{3.381998in}}%
\pgfpathcurveto{\pgfqpoint{2.601758in}{3.381998in}}{\pgfqpoint{2.596172in}{3.379684in}}{\pgfqpoint{2.592053in}{3.375566in}}%
\pgfpathcurveto{\pgfqpoint{2.587935in}{3.371448in}}{\pgfqpoint{2.585621in}{3.365862in}}{\pgfqpoint{2.585621in}{3.360038in}}%
\pgfpathcurveto{\pgfqpoint{2.585621in}{3.354214in}}{\pgfqpoint{2.587935in}{3.348628in}}{\pgfqpoint{2.592053in}{3.344509in}}%
\pgfpathcurveto{\pgfqpoint{2.596172in}{3.340391in}}{\pgfqpoint{2.601758in}{3.338077in}}{\pgfqpoint{2.607582in}{3.338077in}}%
\pgfpathlineto{\pgfqpoint{2.607582in}{3.338077in}}%
\pgfpathclose%
\pgfusepath{stroke,fill}%
\end{pgfscope}%
\begin{pgfscope}%
\pgfpathrectangle{\pgfqpoint{0.100000in}{0.183744in}}{\pgfqpoint{4.506048in}{4.506048in}}%
\pgfusepath{clip}%
\pgfsetbuttcap%
\pgfsetroundjoin%
\definecolor{currentfill}{rgb}{0.000000,0.000000,1.000000}%
\pgfsetfillcolor{currentfill}%
\pgfsetfillopacity{0.700000}%
\pgfsetlinewidth{1.003750pt}%
\definecolor{currentstroke}{rgb}{0.000000,0.000000,1.000000}%
\pgfsetstrokecolor{currentstroke}%
\pgfsetstrokeopacity{0.700000}%
\pgfsetdash{}{0pt}%
\pgfpathmoveto{\pgfqpoint{1.789389in}{2.352173in}}%
\pgfpathcurveto{\pgfqpoint{1.795213in}{2.352173in}}{\pgfqpoint{1.800799in}{2.354486in}}{\pgfqpoint{1.804917in}{2.358605in}}%
\pgfpathcurveto{\pgfqpoint{1.809035in}{2.362723in}}{\pgfqpoint{1.811349in}{2.368309in}}{\pgfqpoint{1.811349in}{2.374133in}}%
\pgfpathcurveto{\pgfqpoint{1.811349in}{2.379957in}}{\pgfqpoint{1.809035in}{2.385543in}}{\pgfqpoint{1.804917in}{2.389661in}}%
\pgfpathcurveto{\pgfqpoint{1.800799in}{2.393779in}}{\pgfqpoint{1.795213in}{2.396093in}}{\pgfqpoint{1.789389in}{2.396093in}}%
\pgfpathcurveto{\pgfqpoint{1.783565in}{2.396093in}}{\pgfqpoint{1.777979in}{2.393779in}}{\pgfqpoint{1.773861in}{2.389661in}}%
\pgfpathcurveto{\pgfqpoint{1.769742in}{2.385543in}}{\pgfqpoint{1.767429in}{2.379957in}}{\pgfqpoint{1.767429in}{2.374133in}}%
\pgfpathcurveto{\pgfqpoint{1.767429in}{2.368309in}}{\pgfqpoint{1.769742in}{2.362723in}}{\pgfqpoint{1.773861in}{2.358605in}}%
\pgfpathcurveto{\pgfqpoint{1.777979in}{2.354486in}}{\pgfqpoint{1.783565in}{2.352173in}}{\pgfqpoint{1.789389in}{2.352173in}}%
\pgfpathlineto{\pgfqpoint{1.789389in}{2.352173in}}%
\pgfpathclose%
\pgfusepath{stroke,fill}%
\end{pgfscope}%
\begin{pgfscope}%
\pgfpathrectangle{\pgfqpoint{0.100000in}{0.183744in}}{\pgfqpoint{4.506048in}{4.506048in}}%
\pgfusepath{clip}%
\pgfsetbuttcap%
\pgfsetroundjoin%
\definecolor{currentfill}{rgb}{0.000000,0.000000,1.000000}%
\pgfsetfillcolor{currentfill}%
\pgfsetfillopacity{0.700000}%
\pgfsetlinewidth{1.003750pt}%
\definecolor{currentstroke}{rgb}{0.000000,0.000000,1.000000}%
\pgfsetstrokecolor{currentstroke}%
\pgfsetstrokeopacity{0.700000}%
\pgfsetdash{}{0pt}%
\pgfpathmoveto{\pgfqpoint{2.026572in}{2.526818in}}%
\pgfpathcurveto{\pgfqpoint{2.032396in}{2.526818in}}{\pgfqpoint{2.037982in}{2.529132in}}{\pgfqpoint{2.042100in}{2.533250in}}%
\pgfpathcurveto{\pgfqpoint{2.046218in}{2.537369in}}{\pgfqpoint{2.048532in}{2.542955in}}{\pgfqpoint{2.048532in}{2.548779in}}%
\pgfpathcurveto{\pgfqpoint{2.048532in}{2.554603in}}{\pgfqpoint{2.046218in}{2.560189in}}{\pgfqpoint{2.042100in}{2.564307in}}%
\pgfpathcurveto{\pgfqpoint{2.037982in}{2.568425in}}{\pgfqpoint{2.032396in}{2.570739in}}{\pgfqpoint{2.026572in}{2.570739in}}%
\pgfpathcurveto{\pgfqpoint{2.020748in}{2.570739in}}{\pgfqpoint{2.015162in}{2.568425in}}{\pgfqpoint{2.011044in}{2.564307in}}%
\pgfpathcurveto{\pgfqpoint{2.006926in}{2.560189in}}{\pgfqpoint{2.004612in}{2.554603in}}{\pgfqpoint{2.004612in}{2.548779in}}%
\pgfpathcurveto{\pgfqpoint{2.004612in}{2.542955in}}{\pgfqpoint{2.006926in}{2.537369in}}{\pgfqpoint{2.011044in}{2.533250in}}%
\pgfpathcurveto{\pgfqpoint{2.015162in}{2.529132in}}{\pgfqpoint{2.020748in}{2.526818in}}{\pgfqpoint{2.026572in}{2.526818in}}%
\pgfpathlineto{\pgfqpoint{2.026572in}{2.526818in}}%
\pgfpathclose%
\pgfusepath{stroke,fill}%
\end{pgfscope}%
\begin{pgfscope}%
\pgfpathrectangle{\pgfqpoint{0.100000in}{0.183744in}}{\pgfqpoint{4.506048in}{4.506048in}}%
\pgfusepath{clip}%
\pgfsetbuttcap%
\pgfsetroundjoin%
\definecolor{currentfill}{rgb}{0.000000,0.000000,1.000000}%
\pgfsetfillcolor{currentfill}%
\pgfsetfillopacity{0.700000}%
\pgfsetlinewidth{1.003750pt}%
\definecolor{currentstroke}{rgb}{0.000000,0.000000,1.000000}%
\pgfsetstrokecolor{currentstroke}%
\pgfsetstrokeopacity{0.700000}%
\pgfsetdash{}{0pt}%
\pgfpathmoveto{\pgfqpoint{2.327062in}{3.006257in}}%
\pgfpathcurveto{\pgfqpoint{2.332886in}{3.006257in}}{\pgfqpoint{2.338472in}{3.008571in}}{\pgfqpoint{2.342590in}{3.012689in}}%
\pgfpathcurveto{\pgfqpoint{2.346708in}{3.016808in}}{\pgfqpoint{2.349022in}{3.022394in}}{\pgfqpoint{2.349022in}{3.028218in}}%
\pgfpathcurveto{\pgfqpoint{2.349022in}{3.034042in}}{\pgfqpoint{2.346708in}{3.039628in}}{\pgfqpoint{2.342590in}{3.043746in}}%
\pgfpathcurveto{\pgfqpoint{2.338472in}{3.047864in}}{\pgfqpoint{2.332886in}{3.050178in}}{\pgfqpoint{2.327062in}{3.050178in}}%
\pgfpathcurveto{\pgfqpoint{2.321238in}{3.050178in}}{\pgfqpoint{2.315651in}{3.047864in}}{\pgfqpoint{2.311533in}{3.043746in}}%
\pgfpathcurveto{\pgfqpoint{2.307415in}{3.039628in}}{\pgfqpoint{2.305101in}{3.034042in}}{\pgfqpoint{2.305101in}{3.028218in}}%
\pgfpathcurveto{\pgfqpoint{2.305101in}{3.022394in}}{\pgfqpoint{2.307415in}{3.016808in}}{\pgfqpoint{2.311533in}{3.012689in}}%
\pgfpathcurveto{\pgfqpoint{2.315651in}{3.008571in}}{\pgfqpoint{2.321238in}{3.006257in}}{\pgfqpoint{2.327062in}{3.006257in}}%
\pgfpathlineto{\pgfqpoint{2.327062in}{3.006257in}}%
\pgfpathclose%
\pgfusepath{stroke,fill}%
\end{pgfscope}%
\begin{pgfscope}%
\pgfpathrectangle{\pgfqpoint{0.100000in}{0.183744in}}{\pgfqpoint{4.506048in}{4.506048in}}%
\pgfusepath{clip}%
\pgfsetbuttcap%
\pgfsetroundjoin%
\definecolor{currentfill}{rgb}{0.000000,0.000000,1.000000}%
\pgfsetfillcolor{currentfill}%
\pgfsetfillopacity{0.700000}%
\pgfsetlinewidth{1.003750pt}%
\definecolor{currentstroke}{rgb}{0.000000,0.000000,1.000000}%
\pgfsetstrokecolor{currentstroke}%
\pgfsetstrokeopacity{0.700000}%
\pgfsetdash{}{0pt}%
\pgfpathmoveto{\pgfqpoint{2.016977in}{2.723001in}}%
\pgfpathcurveto{\pgfqpoint{2.022801in}{2.723001in}}{\pgfqpoint{2.028387in}{2.725315in}}{\pgfqpoint{2.032505in}{2.729434in}}%
\pgfpathcurveto{\pgfqpoint{2.036623in}{2.733552in}}{\pgfqpoint{2.038937in}{2.739138in}}{\pgfqpoint{2.038937in}{2.744962in}}%
\pgfpathcurveto{\pgfqpoint{2.038937in}{2.750786in}}{\pgfqpoint{2.036623in}{2.756372in}}{\pgfqpoint{2.032505in}{2.760490in}}%
\pgfpathcurveto{\pgfqpoint{2.028387in}{2.764608in}}{\pgfqpoint{2.022801in}{2.766922in}}{\pgfqpoint{2.016977in}{2.766922in}}%
\pgfpathcurveto{\pgfqpoint{2.011153in}{2.766922in}}{\pgfqpoint{2.005566in}{2.764608in}}{\pgfqpoint{2.001448in}{2.760490in}}%
\pgfpathcurveto{\pgfqpoint{1.997330in}{2.756372in}}{\pgfqpoint{1.995016in}{2.750786in}}{\pgfqpoint{1.995016in}{2.744962in}}%
\pgfpathcurveto{\pgfqpoint{1.995016in}{2.739138in}}{\pgfqpoint{1.997330in}{2.733552in}}{\pgfqpoint{2.001448in}{2.729434in}}%
\pgfpathcurveto{\pgfqpoint{2.005566in}{2.725315in}}{\pgfqpoint{2.011153in}{2.723001in}}{\pgfqpoint{2.016977in}{2.723001in}}%
\pgfpathlineto{\pgfqpoint{2.016977in}{2.723001in}}%
\pgfpathclose%
\pgfusepath{stroke,fill}%
\end{pgfscope}%
\begin{pgfscope}%
\pgfpathrectangle{\pgfqpoint{0.100000in}{0.183744in}}{\pgfqpoint{4.506048in}{4.506048in}}%
\pgfusepath{clip}%
\pgfsetbuttcap%
\pgfsetroundjoin%
\definecolor{currentfill}{rgb}{0.000000,0.000000,1.000000}%
\pgfsetfillcolor{currentfill}%
\pgfsetfillopacity{0.700000}%
\pgfsetlinewidth{1.003750pt}%
\definecolor{currentstroke}{rgb}{0.000000,0.000000,1.000000}%
\pgfsetstrokecolor{currentstroke}%
\pgfsetstrokeopacity{0.700000}%
\pgfsetdash{}{0pt}%
\pgfpathmoveto{\pgfqpoint{2.935970in}{2.454542in}}%
\pgfpathcurveto{\pgfqpoint{2.941794in}{2.454542in}}{\pgfqpoint{2.947380in}{2.456856in}}{\pgfqpoint{2.951498in}{2.460974in}}%
\pgfpathcurveto{\pgfqpoint{2.955617in}{2.465092in}}{\pgfqpoint{2.957930in}{2.470678in}}{\pgfqpoint{2.957930in}{2.476502in}}%
\pgfpathcurveto{\pgfqpoint{2.957930in}{2.482326in}}{\pgfqpoint{2.955617in}{2.487912in}}{\pgfqpoint{2.951498in}{2.492031in}}%
\pgfpathcurveto{\pgfqpoint{2.947380in}{2.496149in}}{\pgfqpoint{2.941794in}{2.498463in}}{\pgfqpoint{2.935970in}{2.498463in}}%
\pgfpathcurveto{\pgfqpoint{2.930146in}{2.498463in}}{\pgfqpoint{2.924560in}{2.496149in}}{\pgfqpoint{2.920442in}{2.492031in}}%
\pgfpathcurveto{\pgfqpoint{2.916324in}{2.487912in}}{\pgfqpoint{2.914010in}{2.482326in}}{\pgfqpoint{2.914010in}{2.476502in}}%
\pgfpathcurveto{\pgfqpoint{2.914010in}{2.470678in}}{\pgfqpoint{2.916324in}{2.465092in}}{\pgfqpoint{2.920442in}{2.460974in}}%
\pgfpathcurveto{\pgfqpoint{2.924560in}{2.456856in}}{\pgfqpoint{2.930146in}{2.454542in}}{\pgfqpoint{2.935970in}{2.454542in}}%
\pgfpathlineto{\pgfqpoint{2.935970in}{2.454542in}}%
\pgfpathclose%
\pgfusepath{stroke,fill}%
\end{pgfscope}%
\begin{pgfscope}%
\pgfpathrectangle{\pgfqpoint{0.100000in}{0.183744in}}{\pgfqpoint{4.506048in}{4.506048in}}%
\pgfusepath{clip}%
\pgfsetbuttcap%
\pgfsetroundjoin%
\definecolor{currentfill}{rgb}{0.000000,0.000000,1.000000}%
\pgfsetfillcolor{currentfill}%
\pgfsetfillopacity{0.700000}%
\pgfsetlinewidth{1.003750pt}%
\definecolor{currentstroke}{rgb}{0.000000,0.000000,1.000000}%
\pgfsetstrokecolor{currentstroke}%
\pgfsetstrokeopacity{0.700000}%
\pgfsetdash{}{0pt}%
\pgfpathmoveto{\pgfqpoint{1.635114in}{2.931714in}}%
\pgfpathcurveto{\pgfqpoint{1.640937in}{2.931714in}}{\pgfqpoint{1.646524in}{2.934028in}}{\pgfqpoint{1.650642in}{2.938146in}}%
\pgfpathcurveto{\pgfqpoint{1.654760in}{2.942264in}}{\pgfqpoint{1.657074in}{2.947850in}}{\pgfqpoint{1.657074in}{2.953674in}}%
\pgfpathcurveto{\pgfqpoint{1.657074in}{2.959498in}}{\pgfqpoint{1.654760in}{2.965084in}}{\pgfqpoint{1.650642in}{2.969202in}}%
\pgfpathcurveto{\pgfqpoint{1.646524in}{2.973320in}}{\pgfqpoint{1.640937in}{2.975634in}}{\pgfqpoint{1.635114in}{2.975634in}}%
\pgfpathcurveto{\pgfqpoint{1.629290in}{2.975634in}}{\pgfqpoint{1.623703in}{2.973320in}}{\pgfqpoint{1.619585in}{2.969202in}}%
\pgfpathcurveto{\pgfqpoint{1.615467in}{2.965084in}}{\pgfqpoint{1.613153in}{2.959498in}}{\pgfqpoint{1.613153in}{2.953674in}}%
\pgfpathcurveto{\pgfqpoint{1.613153in}{2.947850in}}{\pgfqpoint{1.615467in}{2.942264in}}{\pgfqpoint{1.619585in}{2.938146in}}%
\pgfpathcurveto{\pgfqpoint{1.623703in}{2.934028in}}{\pgfqpoint{1.629290in}{2.931714in}}{\pgfqpoint{1.635114in}{2.931714in}}%
\pgfpathlineto{\pgfqpoint{1.635114in}{2.931714in}}%
\pgfpathclose%
\pgfusepath{stroke,fill}%
\end{pgfscope}%
\begin{pgfscope}%
\pgfpathrectangle{\pgfqpoint{0.100000in}{0.183744in}}{\pgfqpoint{4.506048in}{4.506048in}}%
\pgfusepath{clip}%
\pgfsetbuttcap%
\pgfsetroundjoin%
\definecolor{currentfill}{rgb}{0.000000,0.000000,1.000000}%
\pgfsetfillcolor{currentfill}%
\pgfsetfillopacity{0.700000}%
\pgfsetlinewidth{1.003750pt}%
\definecolor{currentstroke}{rgb}{0.000000,0.000000,1.000000}%
\pgfsetstrokecolor{currentstroke}%
\pgfsetstrokeopacity{0.700000}%
\pgfsetdash{}{0pt}%
\pgfpathmoveto{\pgfqpoint{2.602357in}{2.162137in}}%
\pgfpathcurveto{\pgfqpoint{2.608181in}{2.162137in}}{\pgfqpoint{2.613767in}{2.164450in}}{\pgfqpoint{2.617885in}{2.168569in}}%
\pgfpathcurveto{\pgfqpoint{2.622004in}{2.172687in}}{\pgfqpoint{2.624317in}{2.178273in}}{\pgfqpoint{2.624317in}{2.184097in}}%
\pgfpathcurveto{\pgfqpoint{2.624317in}{2.189921in}}{\pgfqpoint{2.622004in}{2.195507in}}{\pgfqpoint{2.617885in}{2.199625in}}%
\pgfpathcurveto{\pgfqpoint{2.613767in}{2.203743in}}{\pgfqpoint{2.608181in}{2.206057in}}{\pgfqpoint{2.602357in}{2.206057in}}%
\pgfpathcurveto{\pgfqpoint{2.596533in}{2.206057in}}{\pgfqpoint{2.590947in}{2.203743in}}{\pgfqpoint{2.586829in}{2.199625in}}%
\pgfpathcurveto{\pgfqpoint{2.582711in}{2.195507in}}{\pgfqpoint{2.580397in}{2.189921in}}{\pgfqpoint{2.580397in}{2.184097in}}%
\pgfpathcurveto{\pgfqpoint{2.580397in}{2.178273in}}{\pgfqpoint{2.582711in}{2.172687in}}{\pgfqpoint{2.586829in}{2.168569in}}%
\pgfpathcurveto{\pgfqpoint{2.590947in}{2.164450in}}{\pgfqpoint{2.596533in}{2.162137in}}{\pgfqpoint{2.602357in}{2.162137in}}%
\pgfpathlineto{\pgfqpoint{2.602357in}{2.162137in}}%
\pgfpathclose%
\pgfusepath{stroke,fill}%
\end{pgfscope}%
\begin{pgfscope}%
\pgfpathrectangle{\pgfqpoint{0.100000in}{0.183744in}}{\pgfqpoint{4.506048in}{4.506048in}}%
\pgfusepath{clip}%
\pgfsetbuttcap%
\pgfsetroundjoin%
\definecolor{currentfill}{rgb}{0.000000,0.000000,1.000000}%
\pgfsetfillcolor{currentfill}%
\pgfsetfillopacity{0.700000}%
\pgfsetlinewidth{1.003750pt}%
\definecolor{currentstroke}{rgb}{0.000000,0.000000,1.000000}%
\pgfsetstrokecolor{currentstroke}%
\pgfsetstrokeopacity{0.700000}%
\pgfsetdash{}{0pt}%
\pgfpathmoveto{\pgfqpoint{3.543296in}{2.274008in}}%
\pgfpathcurveto{\pgfqpoint{3.549120in}{2.274008in}}{\pgfqpoint{3.554707in}{2.276322in}}{\pgfqpoint{3.558825in}{2.280440in}}%
\pgfpathcurveto{\pgfqpoint{3.562943in}{2.284558in}}{\pgfqpoint{3.565257in}{2.290144in}}{\pgfqpoint{3.565257in}{2.295968in}}%
\pgfpathcurveto{\pgfqpoint{3.565257in}{2.301792in}}{\pgfqpoint{3.562943in}{2.307378in}}{\pgfqpoint{3.558825in}{2.311496in}}%
\pgfpathcurveto{\pgfqpoint{3.554707in}{2.315615in}}{\pgfqpoint{3.549120in}{2.317928in}}{\pgfqpoint{3.543296in}{2.317928in}}%
\pgfpathcurveto{\pgfqpoint{3.537473in}{2.317928in}}{\pgfqpoint{3.531886in}{2.315615in}}{\pgfqpoint{3.527768in}{2.311496in}}%
\pgfpathcurveto{\pgfqpoint{3.523650in}{2.307378in}}{\pgfqpoint{3.521336in}{2.301792in}}{\pgfqpoint{3.521336in}{2.295968in}}%
\pgfpathcurveto{\pgfqpoint{3.521336in}{2.290144in}}{\pgfqpoint{3.523650in}{2.284558in}}{\pgfqpoint{3.527768in}{2.280440in}}%
\pgfpathcurveto{\pgfqpoint{3.531886in}{2.276322in}}{\pgfqpoint{3.537473in}{2.274008in}}{\pgfqpoint{3.543296in}{2.274008in}}%
\pgfpathlineto{\pgfqpoint{3.543296in}{2.274008in}}%
\pgfpathclose%
\pgfusepath{stroke,fill}%
\end{pgfscope}%
\begin{pgfscope}%
\pgfpathrectangle{\pgfqpoint{0.100000in}{0.183744in}}{\pgfqpoint{4.506048in}{4.506048in}}%
\pgfusepath{clip}%
\pgfsetbuttcap%
\pgfsetroundjoin%
\definecolor{currentfill}{rgb}{0.000000,0.000000,1.000000}%
\pgfsetfillcolor{currentfill}%
\pgfsetfillopacity{0.700000}%
\pgfsetlinewidth{1.003750pt}%
\definecolor{currentstroke}{rgb}{0.000000,0.000000,1.000000}%
\pgfsetstrokecolor{currentstroke}%
\pgfsetstrokeopacity{0.700000}%
\pgfsetdash{}{0pt}%
\pgfpathmoveto{\pgfqpoint{2.566838in}{3.286636in}}%
\pgfpathcurveto{\pgfqpoint{2.572662in}{3.286636in}}{\pgfqpoint{2.578248in}{3.288950in}}{\pgfqpoint{2.582366in}{3.293068in}}%
\pgfpathcurveto{\pgfqpoint{2.586484in}{3.297186in}}{\pgfqpoint{2.588798in}{3.302773in}}{\pgfqpoint{2.588798in}{3.308596in}}%
\pgfpathcurveto{\pgfqpoint{2.588798in}{3.314420in}}{\pgfqpoint{2.586484in}{3.320007in}}{\pgfqpoint{2.582366in}{3.324125in}}%
\pgfpathcurveto{\pgfqpoint{2.578248in}{3.328243in}}{\pgfqpoint{2.572662in}{3.330557in}}{\pgfqpoint{2.566838in}{3.330557in}}%
\pgfpathcurveto{\pgfqpoint{2.561014in}{3.330557in}}{\pgfqpoint{2.555428in}{3.328243in}}{\pgfqpoint{2.551310in}{3.324125in}}%
\pgfpathcurveto{\pgfqpoint{2.547192in}{3.320007in}}{\pgfqpoint{2.544878in}{3.314420in}}{\pgfqpoint{2.544878in}{3.308596in}}%
\pgfpathcurveto{\pgfqpoint{2.544878in}{3.302773in}}{\pgfqpoint{2.547192in}{3.297186in}}{\pgfqpoint{2.551310in}{3.293068in}}%
\pgfpathcurveto{\pgfqpoint{2.555428in}{3.288950in}}{\pgfqpoint{2.561014in}{3.286636in}}{\pgfqpoint{2.566838in}{3.286636in}}%
\pgfpathlineto{\pgfqpoint{2.566838in}{3.286636in}}%
\pgfpathclose%
\pgfusepath{stroke,fill}%
\end{pgfscope}%
\begin{pgfscope}%
\pgfpathrectangle{\pgfqpoint{0.100000in}{0.183744in}}{\pgfqpoint{4.506048in}{4.506048in}}%
\pgfusepath{clip}%
\pgfsetbuttcap%
\pgfsetroundjoin%
\definecolor{currentfill}{rgb}{0.000000,0.000000,1.000000}%
\pgfsetfillcolor{currentfill}%
\pgfsetfillopacity{0.700000}%
\pgfsetlinewidth{1.003750pt}%
\definecolor{currentstroke}{rgb}{0.000000,0.000000,1.000000}%
\pgfsetstrokecolor{currentstroke}%
\pgfsetstrokeopacity{0.700000}%
\pgfsetdash{}{0pt}%
\pgfpathmoveto{\pgfqpoint{1.547707in}{2.949463in}}%
\pgfpathcurveto{\pgfqpoint{1.553531in}{2.949463in}}{\pgfqpoint{1.559117in}{2.951777in}}{\pgfqpoint{1.563235in}{2.955895in}}%
\pgfpathcurveto{\pgfqpoint{1.567353in}{2.960013in}}{\pgfqpoint{1.569667in}{2.965599in}}{\pgfqpoint{1.569667in}{2.971423in}}%
\pgfpathcurveto{\pgfqpoint{1.569667in}{2.977247in}}{\pgfqpoint{1.567353in}{2.982833in}}{\pgfqpoint{1.563235in}{2.986952in}}%
\pgfpathcurveto{\pgfqpoint{1.559117in}{2.991070in}}{\pgfqpoint{1.553531in}{2.993384in}}{\pgfqpoint{1.547707in}{2.993384in}}%
\pgfpathcurveto{\pgfqpoint{1.541883in}{2.993384in}}{\pgfqpoint{1.536297in}{2.991070in}}{\pgfqpoint{1.532178in}{2.986952in}}%
\pgfpathcurveto{\pgfqpoint{1.528060in}{2.982833in}}{\pgfqpoint{1.525746in}{2.977247in}}{\pgfqpoint{1.525746in}{2.971423in}}%
\pgfpathcurveto{\pgfqpoint{1.525746in}{2.965599in}}{\pgfqpoint{1.528060in}{2.960013in}}{\pgfqpoint{1.532178in}{2.955895in}}%
\pgfpathcurveto{\pgfqpoint{1.536297in}{2.951777in}}{\pgfqpoint{1.541883in}{2.949463in}}{\pgfqpoint{1.547707in}{2.949463in}}%
\pgfpathlineto{\pgfqpoint{1.547707in}{2.949463in}}%
\pgfpathclose%
\pgfusepath{stroke,fill}%
\end{pgfscope}%
\begin{pgfscope}%
\pgfpathrectangle{\pgfqpoint{0.100000in}{0.183744in}}{\pgfqpoint{4.506048in}{4.506048in}}%
\pgfusepath{clip}%
\pgfsetbuttcap%
\pgfsetroundjoin%
\definecolor{currentfill}{rgb}{0.000000,0.000000,1.000000}%
\pgfsetfillcolor{currentfill}%
\pgfsetfillopacity{0.700000}%
\pgfsetlinewidth{1.003750pt}%
\definecolor{currentstroke}{rgb}{0.000000,0.000000,1.000000}%
\pgfsetstrokecolor{currentstroke}%
\pgfsetstrokeopacity{0.700000}%
\pgfsetdash{}{0pt}%
\pgfpathmoveto{\pgfqpoint{1.621163in}{2.253395in}}%
\pgfpathcurveto{\pgfqpoint{1.626987in}{2.253395in}}{\pgfqpoint{1.632573in}{2.255709in}}{\pgfqpoint{1.636691in}{2.259827in}}%
\pgfpathcurveto{\pgfqpoint{1.640809in}{2.263945in}}{\pgfqpoint{1.643123in}{2.269531in}}{\pgfqpoint{1.643123in}{2.275355in}}%
\pgfpathcurveto{\pgfqpoint{1.643123in}{2.281179in}}{\pgfqpoint{1.640809in}{2.286765in}}{\pgfqpoint{1.636691in}{2.290884in}}%
\pgfpathcurveto{\pgfqpoint{1.632573in}{2.295002in}}{\pgfqpoint{1.626987in}{2.297316in}}{\pgfqpoint{1.621163in}{2.297316in}}%
\pgfpathcurveto{\pgfqpoint{1.615339in}{2.297316in}}{\pgfqpoint{1.609753in}{2.295002in}}{\pgfqpoint{1.605635in}{2.290884in}}%
\pgfpathcurveto{\pgfqpoint{1.601517in}{2.286765in}}{\pgfqpoint{1.599203in}{2.281179in}}{\pgfqpoint{1.599203in}{2.275355in}}%
\pgfpathcurveto{\pgfqpoint{1.599203in}{2.269531in}}{\pgfqpoint{1.601517in}{2.263945in}}{\pgfqpoint{1.605635in}{2.259827in}}%
\pgfpathcurveto{\pgfqpoint{1.609753in}{2.255709in}}{\pgfqpoint{1.615339in}{2.253395in}}{\pgfqpoint{1.621163in}{2.253395in}}%
\pgfpathlineto{\pgfqpoint{1.621163in}{2.253395in}}%
\pgfpathclose%
\pgfusepath{stroke,fill}%
\end{pgfscope}%
\begin{pgfscope}%
\pgfpathrectangle{\pgfqpoint{0.100000in}{0.183744in}}{\pgfqpoint{4.506048in}{4.506048in}}%
\pgfusepath{clip}%
\pgfsetbuttcap%
\pgfsetroundjoin%
\definecolor{currentfill}{rgb}{0.000000,0.000000,1.000000}%
\pgfsetfillcolor{currentfill}%
\pgfsetfillopacity{0.700000}%
\pgfsetlinewidth{1.003750pt}%
\definecolor{currentstroke}{rgb}{0.000000,0.000000,1.000000}%
\pgfsetstrokecolor{currentstroke}%
\pgfsetstrokeopacity{0.700000}%
\pgfsetdash{}{0pt}%
\pgfpathmoveto{\pgfqpoint{2.486663in}{2.003765in}}%
\pgfpathcurveto{\pgfqpoint{2.492487in}{2.003765in}}{\pgfqpoint{2.498073in}{2.006079in}}{\pgfqpoint{2.502191in}{2.010197in}}%
\pgfpathcurveto{\pgfqpoint{2.506309in}{2.014315in}}{\pgfqpoint{2.508623in}{2.019902in}}{\pgfqpoint{2.508623in}{2.025726in}}%
\pgfpathcurveto{\pgfqpoint{2.508623in}{2.031549in}}{\pgfqpoint{2.506309in}{2.037136in}}{\pgfqpoint{2.502191in}{2.041254in}}%
\pgfpathcurveto{\pgfqpoint{2.498073in}{2.045372in}}{\pgfqpoint{2.492487in}{2.047686in}}{\pgfqpoint{2.486663in}{2.047686in}}%
\pgfpathcurveto{\pgfqpoint{2.480839in}{2.047686in}}{\pgfqpoint{2.475253in}{2.045372in}}{\pgfqpoint{2.471134in}{2.041254in}}%
\pgfpathcurveto{\pgfqpoint{2.467016in}{2.037136in}}{\pgfqpoint{2.464702in}{2.031549in}}{\pgfqpoint{2.464702in}{2.025726in}}%
\pgfpathcurveto{\pgfqpoint{2.464702in}{2.019902in}}{\pgfqpoint{2.467016in}{2.014315in}}{\pgfqpoint{2.471134in}{2.010197in}}%
\pgfpathcurveto{\pgfqpoint{2.475253in}{2.006079in}}{\pgfqpoint{2.480839in}{2.003765in}}{\pgfqpoint{2.486663in}{2.003765in}}%
\pgfpathlineto{\pgfqpoint{2.486663in}{2.003765in}}%
\pgfpathclose%
\pgfusepath{stroke,fill}%
\end{pgfscope}%
\begin{pgfscope}%
\pgfpathrectangle{\pgfqpoint{0.100000in}{0.183744in}}{\pgfqpoint{4.506048in}{4.506048in}}%
\pgfusepath{clip}%
\pgfsetbuttcap%
\pgfsetroundjoin%
\definecolor{currentfill}{rgb}{0.000000,0.000000,1.000000}%
\pgfsetfillcolor{currentfill}%
\pgfsetfillopacity{0.700000}%
\pgfsetlinewidth{1.003750pt}%
\definecolor{currentstroke}{rgb}{0.000000,0.000000,1.000000}%
\pgfsetstrokecolor{currentstroke}%
\pgfsetstrokeopacity{0.700000}%
\pgfsetdash{}{0pt}%
\pgfpathmoveto{\pgfqpoint{2.063428in}{3.053320in}}%
\pgfpathcurveto{\pgfqpoint{2.069252in}{3.053320in}}{\pgfqpoint{2.074838in}{3.055633in}}{\pgfqpoint{2.078956in}{3.059752in}}%
\pgfpathcurveto{\pgfqpoint{2.083074in}{3.063870in}}{\pgfqpoint{2.085388in}{3.069456in}}{\pgfqpoint{2.085388in}{3.075280in}}%
\pgfpathcurveto{\pgfqpoint{2.085388in}{3.081104in}}{\pgfqpoint{2.083074in}{3.086690in}}{\pgfqpoint{2.078956in}{3.090808in}}%
\pgfpathcurveto{\pgfqpoint{2.074838in}{3.094926in}}{\pgfqpoint{2.069252in}{3.097240in}}{\pgfqpoint{2.063428in}{3.097240in}}%
\pgfpathcurveto{\pgfqpoint{2.057604in}{3.097240in}}{\pgfqpoint{2.052018in}{3.094926in}}{\pgfqpoint{2.047900in}{3.090808in}}%
\pgfpathcurveto{\pgfqpoint{2.043781in}{3.086690in}}{\pgfqpoint{2.041468in}{3.081104in}}{\pgfqpoint{2.041468in}{3.075280in}}%
\pgfpathcurveto{\pgfqpoint{2.041468in}{3.069456in}}{\pgfqpoint{2.043781in}{3.063870in}}{\pgfqpoint{2.047900in}{3.059752in}}%
\pgfpathcurveto{\pgfqpoint{2.052018in}{3.055633in}}{\pgfqpoint{2.057604in}{3.053320in}}{\pgfqpoint{2.063428in}{3.053320in}}%
\pgfpathlineto{\pgfqpoint{2.063428in}{3.053320in}}%
\pgfpathclose%
\pgfusepath{stroke,fill}%
\end{pgfscope}%
\begin{pgfscope}%
\pgfpathrectangle{\pgfqpoint{0.100000in}{0.183744in}}{\pgfqpoint{4.506048in}{4.506048in}}%
\pgfusepath{clip}%
\pgfsetbuttcap%
\pgfsetroundjoin%
\definecolor{currentfill}{rgb}{0.000000,0.000000,1.000000}%
\pgfsetfillcolor{currentfill}%
\pgfsetfillopacity{0.700000}%
\pgfsetlinewidth{1.003750pt}%
\definecolor{currentstroke}{rgb}{0.000000,0.000000,1.000000}%
\pgfsetstrokecolor{currentstroke}%
\pgfsetstrokeopacity{0.700000}%
\pgfsetdash{}{0pt}%
\pgfpathmoveto{\pgfqpoint{2.222796in}{2.247606in}}%
\pgfpathcurveto{\pgfqpoint{2.228620in}{2.247606in}}{\pgfqpoint{2.234206in}{2.249920in}}{\pgfqpoint{2.238324in}{2.254038in}}%
\pgfpathcurveto{\pgfqpoint{2.242442in}{2.258156in}}{\pgfqpoint{2.244756in}{2.263742in}}{\pgfqpoint{2.244756in}{2.269566in}}%
\pgfpathcurveto{\pgfqpoint{2.244756in}{2.275390in}}{\pgfqpoint{2.242442in}{2.280976in}}{\pgfqpoint{2.238324in}{2.285094in}}%
\pgfpathcurveto{\pgfqpoint{2.234206in}{2.289213in}}{\pgfqpoint{2.228620in}{2.291526in}}{\pgfqpoint{2.222796in}{2.291526in}}%
\pgfpathcurveto{\pgfqpoint{2.216972in}{2.291526in}}{\pgfqpoint{2.211386in}{2.289213in}}{\pgfqpoint{2.207267in}{2.285094in}}%
\pgfpathcurveto{\pgfqpoint{2.203149in}{2.280976in}}{\pgfqpoint{2.200835in}{2.275390in}}{\pgfqpoint{2.200835in}{2.269566in}}%
\pgfpathcurveto{\pgfqpoint{2.200835in}{2.263742in}}{\pgfqpoint{2.203149in}{2.258156in}}{\pgfqpoint{2.207267in}{2.254038in}}%
\pgfpathcurveto{\pgfqpoint{2.211386in}{2.249920in}}{\pgfqpoint{2.216972in}{2.247606in}}{\pgfqpoint{2.222796in}{2.247606in}}%
\pgfpathlineto{\pgfqpoint{2.222796in}{2.247606in}}%
\pgfpathclose%
\pgfusepath{stroke,fill}%
\end{pgfscope}%
\begin{pgfscope}%
\pgfpathrectangle{\pgfqpoint{0.100000in}{0.183744in}}{\pgfqpoint{4.506048in}{4.506048in}}%
\pgfusepath{clip}%
\pgfsetbuttcap%
\pgfsetroundjoin%
\definecolor{currentfill}{rgb}{0.000000,0.000000,1.000000}%
\pgfsetfillcolor{currentfill}%
\pgfsetfillopacity{0.700000}%
\pgfsetlinewidth{1.003750pt}%
\definecolor{currentstroke}{rgb}{0.000000,0.000000,1.000000}%
\pgfsetstrokecolor{currentstroke}%
\pgfsetstrokeopacity{0.700000}%
\pgfsetdash{}{0pt}%
\pgfpathmoveto{\pgfqpoint{3.475060in}{2.990563in}}%
\pgfpathcurveto{\pgfqpoint{3.480884in}{2.990563in}}{\pgfqpoint{3.486470in}{2.992877in}}{\pgfqpoint{3.490589in}{2.996995in}}%
\pgfpathcurveto{\pgfqpoint{3.494707in}{3.001113in}}{\pgfqpoint{3.497021in}{3.006699in}}{\pgfqpoint{3.497021in}{3.012523in}}%
\pgfpathcurveto{\pgfqpoint{3.497021in}{3.018347in}}{\pgfqpoint{3.494707in}{3.023933in}}{\pgfqpoint{3.490589in}{3.028051in}}%
\pgfpathcurveto{\pgfqpoint{3.486470in}{3.032170in}}{\pgfqpoint{3.480884in}{3.034483in}}{\pgfqpoint{3.475060in}{3.034483in}}%
\pgfpathcurveto{\pgfqpoint{3.469236in}{3.034483in}}{\pgfqpoint{3.463650in}{3.032170in}}{\pgfqpoint{3.459532in}{3.028051in}}%
\pgfpathcurveto{\pgfqpoint{3.455414in}{3.023933in}}{\pgfqpoint{3.453100in}{3.018347in}}{\pgfqpoint{3.453100in}{3.012523in}}%
\pgfpathcurveto{\pgfqpoint{3.453100in}{3.006699in}}{\pgfqpoint{3.455414in}{3.001113in}}{\pgfqpoint{3.459532in}{2.996995in}}%
\pgfpathcurveto{\pgfqpoint{3.463650in}{2.992877in}}{\pgfqpoint{3.469236in}{2.990563in}}{\pgfqpoint{3.475060in}{2.990563in}}%
\pgfpathlineto{\pgfqpoint{3.475060in}{2.990563in}}%
\pgfpathclose%
\pgfusepath{stroke,fill}%
\end{pgfscope}%
\begin{pgfscope}%
\pgfpathrectangle{\pgfqpoint{0.100000in}{0.183744in}}{\pgfqpoint{4.506048in}{4.506048in}}%
\pgfusepath{clip}%
\pgfsetbuttcap%
\pgfsetroundjoin%
\definecolor{currentfill}{rgb}{0.000000,0.000000,1.000000}%
\pgfsetfillcolor{currentfill}%
\pgfsetfillopacity{0.700000}%
\pgfsetlinewidth{1.003750pt}%
\definecolor{currentstroke}{rgb}{0.000000,0.000000,1.000000}%
\pgfsetstrokecolor{currentstroke}%
\pgfsetstrokeopacity{0.700000}%
\pgfsetdash{}{0pt}%
\pgfpathmoveto{\pgfqpoint{1.441245in}{3.023508in}}%
\pgfpathcurveto{\pgfqpoint{1.447069in}{3.023508in}}{\pgfqpoint{1.452655in}{3.025822in}}{\pgfqpoint{1.456774in}{3.029940in}}%
\pgfpathcurveto{\pgfqpoint{1.460892in}{3.034058in}}{\pgfqpoint{1.463206in}{3.039645in}}{\pgfqpoint{1.463206in}{3.045469in}}%
\pgfpathcurveto{\pgfqpoint{1.463206in}{3.051292in}}{\pgfqpoint{1.460892in}{3.056879in}}{\pgfqpoint{1.456774in}{3.060997in}}%
\pgfpathcurveto{\pgfqpoint{1.452655in}{3.065115in}}{\pgfqpoint{1.447069in}{3.067429in}}{\pgfqpoint{1.441245in}{3.067429in}}%
\pgfpathcurveto{\pgfqpoint{1.435421in}{3.067429in}}{\pgfqpoint{1.429835in}{3.065115in}}{\pgfqpoint{1.425717in}{3.060997in}}%
\pgfpathcurveto{\pgfqpoint{1.421599in}{3.056879in}}{\pgfqpoint{1.419285in}{3.051292in}}{\pgfqpoint{1.419285in}{3.045469in}}%
\pgfpathcurveto{\pgfqpoint{1.419285in}{3.039645in}}{\pgfqpoint{1.421599in}{3.034058in}}{\pgfqpoint{1.425717in}{3.029940in}}%
\pgfpathcurveto{\pgfqpoint{1.429835in}{3.025822in}}{\pgfqpoint{1.435421in}{3.023508in}}{\pgfqpoint{1.441245in}{3.023508in}}%
\pgfpathlineto{\pgfqpoint{1.441245in}{3.023508in}}%
\pgfpathclose%
\pgfusepath{stroke,fill}%
\end{pgfscope}%
\begin{pgfscope}%
\pgfpathrectangle{\pgfqpoint{0.100000in}{0.183744in}}{\pgfqpoint{4.506048in}{4.506048in}}%
\pgfusepath{clip}%
\pgfsetbuttcap%
\pgfsetroundjoin%
\definecolor{currentfill}{rgb}{0.000000,0.000000,1.000000}%
\pgfsetfillcolor{currentfill}%
\pgfsetfillopacity{0.700000}%
\pgfsetlinewidth{1.003750pt}%
\definecolor{currentstroke}{rgb}{0.000000,0.000000,1.000000}%
\pgfsetstrokecolor{currentstroke}%
\pgfsetstrokeopacity{0.700000}%
\pgfsetdash{}{0pt}%
\pgfpathmoveto{\pgfqpoint{1.806236in}{1.843766in}}%
\pgfpathcurveto{\pgfqpoint{1.812060in}{1.843766in}}{\pgfqpoint{1.817647in}{1.846080in}}{\pgfqpoint{1.821765in}{1.850198in}}%
\pgfpathcurveto{\pgfqpoint{1.825883in}{1.854316in}}{\pgfqpoint{1.828197in}{1.859902in}}{\pgfqpoint{1.828197in}{1.865726in}}%
\pgfpathcurveto{\pgfqpoint{1.828197in}{1.871550in}}{\pgfqpoint{1.825883in}{1.877136in}}{\pgfqpoint{1.821765in}{1.881254in}}%
\pgfpathcurveto{\pgfqpoint{1.817647in}{1.885372in}}{\pgfqpoint{1.812060in}{1.887686in}}{\pgfqpoint{1.806236in}{1.887686in}}%
\pgfpathcurveto{\pgfqpoint{1.800412in}{1.887686in}}{\pgfqpoint{1.794826in}{1.885372in}}{\pgfqpoint{1.790708in}{1.881254in}}%
\pgfpathcurveto{\pgfqpoint{1.786590in}{1.877136in}}{\pgfqpoint{1.784276in}{1.871550in}}{\pgfqpoint{1.784276in}{1.865726in}}%
\pgfpathcurveto{\pgfqpoint{1.784276in}{1.859902in}}{\pgfqpoint{1.786590in}{1.854316in}}{\pgfqpoint{1.790708in}{1.850198in}}%
\pgfpathcurveto{\pgfqpoint{1.794826in}{1.846080in}}{\pgfqpoint{1.800412in}{1.843766in}}{\pgfqpoint{1.806236in}{1.843766in}}%
\pgfpathlineto{\pgfqpoint{1.806236in}{1.843766in}}%
\pgfpathclose%
\pgfusepath{stroke,fill}%
\end{pgfscope}%
\begin{pgfscope}%
\pgfpathrectangle{\pgfqpoint{0.100000in}{0.183744in}}{\pgfqpoint{4.506048in}{4.506048in}}%
\pgfusepath{clip}%
\pgfsetbuttcap%
\pgfsetroundjoin%
\definecolor{currentfill}{rgb}{0.000000,0.000000,1.000000}%
\pgfsetfillcolor{currentfill}%
\pgfsetfillopacity{0.700000}%
\pgfsetlinewidth{1.003750pt}%
\definecolor{currentstroke}{rgb}{0.000000,0.000000,1.000000}%
\pgfsetstrokecolor{currentstroke}%
\pgfsetstrokeopacity{0.700000}%
\pgfsetdash{}{0pt}%
\pgfpathmoveto{\pgfqpoint{3.357545in}{2.418962in}}%
\pgfpathcurveto{\pgfqpoint{3.363369in}{2.418962in}}{\pgfqpoint{3.368955in}{2.421275in}}{\pgfqpoint{3.373073in}{2.425394in}}%
\pgfpathcurveto{\pgfqpoint{3.377191in}{2.429512in}}{\pgfqpoint{3.379505in}{2.435098in}}{\pgfqpoint{3.379505in}{2.440922in}}%
\pgfpathcurveto{\pgfqpoint{3.379505in}{2.446746in}}{\pgfqpoint{3.377191in}{2.452332in}}{\pgfqpoint{3.373073in}{2.456450in}}%
\pgfpathcurveto{\pgfqpoint{3.368955in}{2.460568in}}{\pgfqpoint{3.363369in}{2.462882in}}{\pgfqpoint{3.357545in}{2.462882in}}%
\pgfpathcurveto{\pgfqpoint{3.351721in}{2.462882in}}{\pgfqpoint{3.346135in}{2.460568in}}{\pgfqpoint{3.342017in}{2.456450in}}%
\pgfpathcurveto{\pgfqpoint{3.337899in}{2.452332in}}{\pgfqpoint{3.335585in}{2.446746in}}{\pgfqpoint{3.335585in}{2.440922in}}%
\pgfpathcurveto{\pgfqpoint{3.335585in}{2.435098in}}{\pgfqpoint{3.337899in}{2.429512in}}{\pgfqpoint{3.342017in}{2.425394in}}%
\pgfpathcurveto{\pgfqpoint{3.346135in}{2.421275in}}{\pgfqpoint{3.351721in}{2.418962in}}{\pgfqpoint{3.357545in}{2.418962in}}%
\pgfpathlineto{\pgfqpoint{3.357545in}{2.418962in}}%
\pgfpathclose%
\pgfusepath{stroke,fill}%
\end{pgfscope}%
\begin{pgfscope}%
\pgfpathrectangle{\pgfqpoint{0.100000in}{0.183744in}}{\pgfqpoint{4.506048in}{4.506048in}}%
\pgfusepath{clip}%
\pgfsetbuttcap%
\pgfsetroundjoin%
\definecolor{currentfill}{rgb}{0.000000,0.000000,1.000000}%
\pgfsetfillcolor{currentfill}%
\pgfsetfillopacity{0.700000}%
\pgfsetlinewidth{1.003750pt}%
\definecolor{currentstroke}{rgb}{0.000000,0.000000,1.000000}%
\pgfsetstrokecolor{currentstroke}%
\pgfsetstrokeopacity{0.700000}%
\pgfsetdash{}{0pt}%
\pgfpathmoveto{\pgfqpoint{2.367234in}{3.414178in}}%
\pgfpathcurveto{\pgfqpoint{2.373058in}{3.414178in}}{\pgfqpoint{2.378644in}{3.416492in}}{\pgfqpoint{2.382763in}{3.420610in}}%
\pgfpathcurveto{\pgfqpoint{2.386881in}{3.424729in}}{\pgfqpoint{2.389195in}{3.430315in}}{\pgfqpoint{2.389195in}{3.436139in}}%
\pgfpathcurveto{\pgfqpoint{2.389195in}{3.441963in}}{\pgfqpoint{2.386881in}{3.447549in}}{\pgfqpoint{2.382763in}{3.451667in}}%
\pgfpathcurveto{\pgfqpoint{2.378644in}{3.455785in}}{\pgfqpoint{2.373058in}{3.458099in}}{\pgfqpoint{2.367234in}{3.458099in}}%
\pgfpathcurveto{\pgfqpoint{2.361410in}{3.458099in}}{\pgfqpoint{2.355824in}{3.455785in}}{\pgfqpoint{2.351706in}{3.451667in}}%
\pgfpathcurveto{\pgfqpoint{2.347588in}{3.447549in}}{\pgfqpoint{2.345274in}{3.441963in}}{\pgfqpoint{2.345274in}{3.436139in}}%
\pgfpathcurveto{\pgfqpoint{2.345274in}{3.430315in}}{\pgfqpoint{2.347588in}{3.424729in}}{\pgfqpoint{2.351706in}{3.420610in}}%
\pgfpathcurveto{\pgfqpoint{2.355824in}{3.416492in}}{\pgfqpoint{2.361410in}{3.414178in}}{\pgfqpoint{2.367234in}{3.414178in}}%
\pgfpathlineto{\pgfqpoint{2.367234in}{3.414178in}}%
\pgfpathclose%
\pgfusepath{stroke,fill}%
\end{pgfscope}%
\begin{pgfscope}%
\pgfpathrectangle{\pgfqpoint{0.100000in}{0.183744in}}{\pgfqpoint{4.506048in}{4.506048in}}%
\pgfusepath{clip}%
\pgfsetbuttcap%
\pgfsetroundjoin%
\definecolor{currentfill}{rgb}{0.000000,0.000000,1.000000}%
\pgfsetfillcolor{currentfill}%
\pgfsetfillopacity{0.700000}%
\pgfsetlinewidth{1.003750pt}%
\definecolor{currentstroke}{rgb}{0.000000,0.000000,1.000000}%
\pgfsetstrokecolor{currentstroke}%
\pgfsetstrokeopacity{0.700000}%
\pgfsetdash{}{0pt}%
\pgfpathmoveto{\pgfqpoint{3.612650in}{2.526302in}}%
\pgfpathcurveto{\pgfqpoint{3.618474in}{2.526302in}}{\pgfqpoint{3.624060in}{2.528616in}}{\pgfqpoint{3.628179in}{2.532734in}}%
\pgfpathcurveto{\pgfqpoint{3.632297in}{2.536852in}}{\pgfqpoint{3.634611in}{2.542439in}}{\pgfqpoint{3.634611in}{2.548262in}}%
\pgfpathcurveto{\pgfqpoint{3.634611in}{2.554086in}}{\pgfqpoint{3.632297in}{2.559673in}}{\pgfqpoint{3.628179in}{2.563791in}}%
\pgfpathcurveto{\pgfqpoint{3.624060in}{2.567909in}}{\pgfqpoint{3.618474in}{2.570223in}}{\pgfqpoint{3.612650in}{2.570223in}}%
\pgfpathcurveto{\pgfqpoint{3.606826in}{2.570223in}}{\pgfqpoint{3.601240in}{2.567909in}}{\pgfqpoint{3.597122in}{2.563791in}}%
\pgfpathcurveto{\pgfqpoint{3.593004in}{2.559673in}}{\pgfqpoint{3.590690in}{2.554086in}}{\pgfqpoint{3.590690in}{2.548262in}}%
\pgfpathcurveto{\pgfqpoint{3.590690in}{2.542439in}}{\pgfqpoint{3.593004in}{2.536852in}}{\pgfqpoint{3.597122in}{2.532734in}}%
\pgfpathcurveto{\pgfqpoint{3.601240in}{2.528616in}}{\pgfqpoint{3.606826in}{2.526302in}}{\pgfqpoint{3.612650in}{2.526302in}}%
\pgfpathlineto{\pgfqpoint{3.612650in}{2.526302in}}%
\pgfpathclose%
\pgfusepath{stroke,fill}%
\end{pgfscope}%
\begin{pgfscope}%
\pgfpathrectangle{\pgfqpoint{0.100000in}{0.183744in}}{\pgfqpoint{4.506048in}{4.506048in}}%
\pgfusepath{clip}%
\pgfsetbuttcap%
\pgfsetroundjoin%
\definecolor{currentfill}{rgb}{0.000000,0.000000,1.000000}%
\pgfsetfillcolor{currentfill}%
\pgfsetfillopacity{0.700000}%
\pgfsetlinewidth{1.003750pt}%
\definecolor{currentstroke}{rgb}{0.000000,0.000000,1.000000}%
\pgfsetstrokecolor{currentstroke}%
\pgfsetstrokeopacity{0.700000}%
\pgfsetdash{}{0pt}%
\pgfpathmoveto{\pgfqpoint{2.875247in}{3.358600in}}%
\pgfpathcurveto{\pgfqpoint{2.881071in}{3.358600in}}{\pgfqpoint{2.886657in}{3.360914in}}{\pgfqpoint{2.890776in}{3.365032in}}%
\pgfpathcurveto{\pgfqpoint{2.894894in}{3.369150in}}{\pgfqpoint{2.897208in}{3.374736in}}{\pgfqpoint{2.897208in}{3.380560in}}%
\pgfpathcurveto{\pgfqpoint{2.897208in}{3.386384in}}{\pgfqpoint{2.894894in}{3.391970in}}{\pgfqpoint{2.890776in}{3.396088in}}%
\pgfpathcurveto{\pgfqpoint{2.886657in}{3.400207in}}{\pgfqpoint{2.881071in}{3.402520in}}{\pgfqpoint{2.875247in}{3.402520in}}%
\pgfpathcurveto{\pgfqpoint{2.869423in}{3.402520in}}{\pgfqpoint{2.863837in}{3.400207in}}{\pgfqpoint{2.859719in}{3.396088in}}%
\pgfpathcurveto{\pgfqpoint{2.855601in}{3.391970in}}{\pgfqpoint{2.853287in}{3.386384in}}{\pgfqpoint{2.853287in}{3.380560in}}%
\pgfpathcurveto{\pgfqpoint{2.853287in}{3.374736in}}{\pgfqpoint{2.855601in}{3.369150in}}{\pgfqpoint{2.859719in}{3.365032in}}%
\pgfpathcurveto{\pgfqpoint{2.863837in}{3.360914in}}{\pgfqpoint{2.869423in}{3.358600in}}{\pgfqpoint{2.875247in}{3.358600in}}%
\pgfpathlineto{\pgfqpoint{2.875247in}{3.358600in}}%
\pgfpathclose%
\pgfusepath{stroke,fill}%
\end{pgfscope}%
\begin{pgfscope}%
\pgfpathrectangle{\pgfqpoint{0.100000in}{0.183744in}}{\pgfqpoint{4.506048in}{4.506048in}}%
\pgfusepath{clip}%
\pgfsetbuttcap%
\pgfsetroundjoin%
\definecolor{currentfill}{rgb}{0.000000,0.000000,1.000000}%
\pgfsetfillcolor{currentfill}%
\pgfsetfillopacity{0.700000}%
\pgfsetlinewidth{1.003750pt}%
\definecolor{currentstroke}{rgb}{0.000000,0.000000,1.000000}%
\pgfsetstrokecolor{currentstroke}%
\pgfsetstrokeopacity{0.700000}%
\pgfsetdash{}{0pt}%
\pgfpathmoveto{\pgfqpoint{2.637395in}{3.205399in}}%
\pgfpathcurveto{\pgfqpoint{2.643219in}{3.205399in}}{\pgfqpoint{2.648805in}{3.207713in}}{\pgfqpoint{2.652923in}{3.211831in}}%
\pgfpathcurveto{\pgfqpoint{2.657042in}{3.215949in}}{\pgfqpoint{2.659355in}{3.221535in}}{\pgfqpoint{2.659355in}{3.227359in}}%
\pgfpathcurveto{\pgfqpoint{2.659355in}{3.233183in}}{\pgfqpoint{2.657042in}{3.238769in}}{\pgfqpoint{2.652923in}{3.242887in}}%
\pgfpathcurveto{\pgfqpoint{2.648805in}{3.247005in}}{\pgfqpoint{2.643219in}{3.249319in}}{\pgfqpoint{2.637395in}{3.249319in}}%
\pgfpathcurveto{\pgfqpoint{2.631571in}{3.249319in}}{\pgfqpoint{2.625985in}{3.247005in}}{\pgfqpoint{2.621867in}{3.242887in}}%
\pgfpathcurveto{\pgfqpoint{2.617749in}{3.238769in}}{\pgfqpoint{2.615435in}{3.233183in}}{\pgfqpoint{2.615435in}{3.227359in}}%
\pgfpathcurveto{\pgfqpoint{2.615435in}{3.221535in}}{\pgfqpoint{2.617749in}{3.215949in}}{\pgfqpoint{2.621867in}{3.211831in}}%
\pgfpathcurveto{\pgfqpoint{2.625985in}{3.207713in}}{\pgfqpoint{2.631571in}{3.205399in}}{\pgfqpoint{2.637395in}{3.205399in}}%
\pgfpathlineto{\pgfqpoint{2.637395in}{3.205399in}}%
\pgfpathclose%
\pgfusepath{stroke,fill}%
\end{pgfscope}%
\begin{pgfscope}%
\pgfpathrectangle{\pgfqpoint{0.100000in}{0.183744in}}{\pgfqpoint{4.506048in}{4.506048in}}%
\pgfusepath{clip}%
\pgfsetbuttcap%
\pgfsetroundjoin%
\definecolor{currentfill}{rgb}{0.000000,0.000000,1.000000}%
\pgfsetfillcolor{currentfill}%
\pgfsetfillopacity{0.700000}%
\pgfsetlinewidth{1.003750pt}%
\definecolor{currentstroke}{rgb}{0.000000,0.000000,1.000000}%
\pgfsetstrokecolor{currentstroke}%
\pgfsetstrokeopacity{0.700000}%
\pgfsetdash{}{0pt}%
\pgfpathmoveto{\pgfqpoint{2.165838in}{3.580410in}}%
\pgfpathcurveto{\pgfqpoint{2.171662in}{3.580410in}}{\pgfqpoint{2.177248in}{3.582724in}}{\pgfqpoint{2.181366in}{3.586842in}}%
\pgfpathcurveto{\pgfqpoint{2.185484in}{3.590961in}}{\pgfqpoint{2.187798in}{3.596547in}}{\pgfqpoint{2.187798in}{3.602371in}}%
\pgfpathcurveto{\pgfqpoint{2.187798in}{3.608195in}}{\pgfqpoint{2.185484in}{3.613781in}}{\pgfqpoint{2.181366in}{3.617899in}}%
\pgfpathcurveto{\pgfqpoint{2.177248in}{3.622017in}}{\pgfqpoint{2.171662in}{3.624331in}}{\pgfqpoint{2.165838in}{3.624331in}}%
\pgfpathcurveto{\pgfqpoint{2.160014in}{3.624331in}}{\pgfqpoint{2.154428in}{3.622017in}}{\pgfqpoint{2.150310in}{3.617899in}}%
\pgfpathcurveto{\pgfqpoint{2.146191in}{3.613781in}}{\pgfqpoint{2.143878in}{3.608195in}}{\pgfqpoint{2.143878in}{3.602371in}}%
\pgfpathcurveto{\pgfqpoint{2.143878in}{3.596547in}}{\pgfqpoint{2.146191in}{3.590961in}}{\pgfqpoint{2.150310in}{3.586842in}}%
\pgfpathcurveto{\pgfqpoint{2.154428in}{3.582724in}}{\pgfqpoint{2.160014in}{3.580410in}}{\pgfqpoint{2.165838in}{3.580410in}}%
\pgfpathlineto{\pgfqpoint{2.165838in}{3.580410in}}%
\pgfpathclose%
\pgfusepath{stroke,fill}%
\end{pgfscope}%
\begin{pgfscope}%
\pgfpathrectangle{\pgfqpoint{0.100000in}{0.183744in}}{\pgfqpoint{4.506048in}{4.506048in}}%
\pgfusepath{clip}%
\pgfsetbuttcap%
\pgfsetroundjoin%
\definecolor{currentfill}{rgb}{0.000000,0.000000,1.000000}%
\pgfsetfillcolor{currentfill}%
\pgfsetfillopacity{0.700000}%
\pgfsetlinewidth{1.003750pt}%
\definecolor{currentstroke}{rgb}{0.000000,0.000000,1.000000}%
\pgfsetstrokecolor{currentstroke}%
\pgfsetstrokeopacity{0.700000}%
\pgfsetdash{}{0pt}%
\pgfpathmoveto{\pgfqpoint{0.729120in}{2.829109in}}%
\pgfpathcurveto{\pgfqpoint{0.734944in}{2.829109in}}{\pgfqpoint{0.740530in}{2.831423in}}{\pgfqpoint{0.744649in}{2.835541in}}%
\pgfpathcurveto{\pgfqpoint{0.748767in}{2.839659in}}{\pgfqpoint{0.751081in}{2.845245in}}{\pgfqpoint{0.751081in}{2.851069in}}%
\pgfpathcurveto{\pgfqpoint{0.751081in}{2.856893in}}{\pgfqpoint{0.748767in}{2.862479in}}{\pgfqpoint{0.744649in}{2.866598in}}%
\pgfpathcurveto{\pgfqpoint{0.740530in}{2.870716in}}{\pgfqpoint{0.734944in}{2.873030in}}{\pgfqpoint{0.729120in}{2.873030in}}%
\pgfpathcurveto{\pgfqpoint{0.723296in}{2.873030in}}{\pgfqpoint{0.717710in}{2.870716in}}{\pgfqpoint{0.713592in}{2.866598in}}%
\pgfpathcurveto{\pgfqpoint{0.709474in}{2.862479in}}{\pgfqpoint{0.707160in}{2.856893in}}{\pgfqpoint{0.707160in}{2.851069in}}%
\pgfpathcurveto{\pgfqpoint{0.707160in}{2.845245in}}{\pgfqpoint{0.709474in}{2.839659in}}{\pgfqpoint{0.713592in}{2.835541in}}%
\pgfpathcurveto{\pgfqpoint{0.717710in}{2.831423in}}{\pgfqpoint{0.723296in}{2.829109in}}{\pgfqpoint{0.729120in}{2.829109in}}%
\pgfpathlineto{\pgfqpoint{0.729120in}{2.829109in}}%
\pgfpathclose%
\pgfusepath{stroke,fill}%
\end{pgfscope}%
\begin{pgfscope}%
\pgfpathrectangle{\pgfqpoint{0.100000in}{0.183744in}}{\pgfqpoint{4.506048in}{4.506048in}}%
\pgfusepath{clip}%
\pgfsetbuttcap%
\pgfsetroundjoin%
\definecolor{currentfill}{rgb}{0.000000,0.000000,1.000000}%
\pgfsetfillcolor{currentfill}%
\pgfsetfillopacity{0.700000}%
\pgfsetlinewidth{1.003750pt}%
\definecolor{currentstroke}{rgb}{0.000000,0.000000,1.000000}%
\pgfsetstrokecolor{currentstroke}%
\pgfsetstrokeopacity{0.700000}%
\pgfsetdash{}{0pt}%
\pgfpathmoveto{\pgfqpoint{1.710047in}{2.089773in}}%
\pgfpathcurveto{\pgfqpoint{1.715871in}{2.089773in}}{\pgfqpoint{1.721457in}{2.092087in}}{\pgfqpoint{1.725575in}{2.096205in}}%
\pgfpathcurveto{\pgfqpoint{1.729693in}{2.100323in}}{\pgfqpoint{1.732007in}{2.105909in}}{\pgfqpoint{1.732007in}{2.111733in}}%
\pgfpathcurveto{\pgfqpoint{1.732007in}{2.117557in}}{\pgfqpoint{1.729693in}{2.123143in}}{\pgfqpoint{1.725575in}{2.127261in}}%
\pgfpathcurveto{\pgfqpoint{1.721457in}{2.131380in}}{\pgfqpoint{1.715871in}{2.133693in}}{\pgfqpoint{1.710047in}{2.133693in}}%
\pgfpathcurveto{\pgfqpoint{1.704223in}{2.133693in}}{\pgfqpoint{1.698637in}{2.131380in}}{\pgfqpoint{1.694518in}{2.127261in}}%
\pgfpathcurveto{\pgfqpoint{1.690400in}{2.123143in}}{\pgfqpoint{1.688086in}{2.117557in}}{\pgfqpoint{1.688086in}{2.111733in}}%
\pgfpathcurveto{\pgfqpoint{1.688086in}{2.105909in}}{\pgfqpoint{1.690400in}{2.100323in}}{\pgfqpoint{1.694518in}{2.096205in}}%
\pgfpathcurveto{\pgfqpoint{1.698637in}{2.092087in}}{\pgfqpoint{1.704223in}{2.089773in}}{\pgfqpoint{1.710047in}{2.089773in}}%
\pgfpathlineto{\pgfqpoint{1.710047in}{2.089773in}}%
\pgfpathclose%
\pgfusepath{stroke,fill}%
\end{pgfscope}%
\begin{pgfscope}%
\pgfpathrectangle{\pgfqpoint{0.100000in}{0.183744in}}{\pgfqpoint{4.506048in}{4.506048in}}%
\pgfusepath{clip}%
\pgfsetbuttcap%
\pgfsetroundjoin%
\definecolor{currentfill}{rgb}{0.000000,0.000000,1.000000}%
\pgfsetfillcolor{currentfill}%
\pgfsetfillopacity{0.700000}%
\pgfsetlinewidth{1.003750pt}%
\definecolor{currentstroke}{rgb}{0.000000,0.000000,1.000000}%
\pgfsetstrokecolor{currentstroke}%
\pgfsetstrokeopacity{0.700000}%
\pgfsetdash{}{0pt}%
\pgfpathmoveto{\pgfqpoint{3.014394in}{1.973244in}}%
\pgfpathcurveto{\pgfqpoint{3.020218in}{1.973244in}}{\pgfqpoint{3.025804in}{1.975557in}}{\pgfqpoint{3.029922in}{1.979676in}}%
\pgfpathcurveto{\pgfqpoint{3.034040in}{1.983794in}}{\pgfqpoint{3.036354in}{1.989380in}}{\pgfqpoint{3.036354in}{1.995204in}}%
\pgfpathcurveto{\pgfqpoint{3.036354in}{2.001028in}}{\pgfqpoint{3.034040in}{2.006614in}}{\pgfqpoint{3.029922in}{2.010732in}}%
\pgfpathcurveto{\pgfqpoint{3.025804in}{2.014850in}}{\pgfqpoint{3.020218in}{2.017164in}}{\pgfqpoint{3.014394in}{2.017164in}}%
\pgfpathcurveto{\pgfqpoint{3.008570in}{2.017164in}}{\pgfqpoint{3.002984in}{2.014850in}}{\pgfqpoint{2.998866in}{2.010732in}}%
\pgfpathcurveto{\pgfqpoint{2.994748in}{2.006614in}}{\pgfqpoint{2.992434in}{2.001028in}}{\pgfqpoint{2.992434in}{1.995204in}}%
\pgfpathcurveto{\pgfqpoint{2.992434in}{1.989380in}}{\pgfqpoint{2.994748in}{1.983794in}}{\pgfqpoint{2.998866in}{1.979676in}}%
\pgfpathcurveto{\pgfqpoint{3.002984in}{1.975557in}}{\pgfqpoint{3.008570in}{1.973244in}}{\pgfqpoint{3.014394in}{1.973244in}}%
\pgfpathlineto{\pgfqpoint{3.014394in}{1.973244in}}%
\pgfpathclose%
\pgfusepath{stroke,fill}%
\end{pgfscope}%
\begin{pgfscope}%
\pgfpathrectangle{\pgfqpoint{0.100000in}{0.183744in}}{\pgfqpoint{4.506048in}{4.506048in}}%
\pgfusepath{clip}%
\pgfsetbuttcap%
\pgfsetroundjoin%
\definecolor{currentfill}{rgb}{0.000000,0.000000,1.000000}%
\pgfsetfillcolor{currentfill}%
\pgfsetfillopacity{0.700000}%
\pgfsetlinewidth{1.003750pt}%
\definecolor{currentstroke}{rgb}{0.000000,0.000000,1.000000}%
\pgfsetstrokecolor{currentstroke}%
\pgfsetstrokeopacity{0.700000}%
\pgfsetdash{}{0pt}%
\pgfpathmoveto{\pgfqpoint{2.420550in}{2.530261in}}%
\pgfpathcurveto{\pgfqpoint{2.426374in}{2.530261in}}{\pgfqpoint{2.431961in}{2.532575in}}{\pgfqpoint{2.436079in}{2.536693in}}%
\pgfpathcurveto{\pgfqpoint{2.440197in}{2.540811in}}{\pgfqpoint{2.442511in}{2.546397in}}{\pgfqpoint{2.442511in}{2.552221in}}%
\pgfpathcurveto{\pgfqpoint{2.442511in}{2.558045in}}{\pgfqpoint{2.440197in}{2.563631in}}{\pgfqpoint{2.436079in}{2.567750in}}%
\pgfpathcurveto{\pgfqpoint{2.431961in}{2.571868in}}{\pgfqpoint{2.426374in}{2.574182in}}{\pgfqpoint{2.420550in}{2.574182in}}%
\pgfpathcurveto{\pgfqpoint{2.414726in}{2.574182in}}{\pgfqpoint{2.409140in}{2.571868in}}{\pgfqpoint{2.405022in}{2.567750in}}%
\pgfpathcurveto{\pgfqpoint{2.400904in}{2.563631in}}{\pgfqpoint{2.398590in}{2.558045in}}{\pgfqpoint{2.398590in}{2.552221in}}%
\pgfpathcurveto{\pgfqpoint{2.398590in}{2.546397in}}{\pgfqpoint{2.400904in}{2.540811in}}{\pgfqpoint{2.405022in}{2.536693in}}%
\pgfpathcurveto{\pgfqpoint{2.409140in}{2.532575in}}{\pgfqpoint{2.414726in}{2.530261in}}{\pgfqpoint{2.420550in}{2.530261in}}%
\pgfpathlineto{\pgfqpoint{2.420550in}{2.530261in}}%
\pgfpathclose%
\pgfusepath{stroke,fill}%
\end{pgfscope}%
\begin{pgfscope}%
\pgfpathrectangle{\pgfqpoint{0.100000in}{0.183744in}}{\pgfqpoint{4.506048in}{4.506048in}}%
\pgfusepath{clip}%
\pgfsetbuttcap%
\pgfsetroundjoin%
\definecolor{currentfill}{rgb}{0.000000,0.000000,1.000000}%
\pgfsetfillcolor{currentfill}%
\pgfsetfillopacity{0.700000}%
\pgfsetlinewidth{1.003750pt}%
\definecolor{currentstroke}{rgb}{0.000000,0.000000,1.000000}%
\pgfsetstrokecolor{currentstroke}%
\pgfsetstrokeopacity{0.700000}%
\pgfsetdash{}{0pt}%
\pgfpathmoveto{\pgfqpoint{2.743317in}{3.576136in}}%
\pgfpathcurveto{\pgfqpoint{2.749141in}{3.576136in}}{\pgfqpoint{2.754727in}{3.578450in}}{\pgfqpoint{2.758845in}{3.582568in}}%
\pgfpathcurveto{\pgfqpoint{2.762963in}{3.586686in}}{\pgfqpoint{2.765277in}{3.592272in}}{\pgfqpoint{2.765277in}{3.598096in}}%
\pgfpathcurveto{\pgfqpoint{2.765277in}{3.603920in}}{\pgfqpoint{2.762963in}{3.609506in}}{\pgfqpoint{2.758845in}{3.613624in}}%
\pgfpathcurveto{\pgfqpoint{2.754727in}{3.617742in}}{\pgfqpoint{2.749141in}{3.620056in}}{\pgfqpoint{2.743317in}{3.620056in}}%
\pgfpathcurveto{\pgfqpoint{2.737493in}{3.620056in}}{\pgfqpoint{2.731907in}{3.617742in}}{\pgfqpoint{2.727789in}{3.613624in}}%
\pgfpathcurveto{\pgfqpoint{2.723671in}{3.609506in}}{\pgfqpoint{2.721357in}{3.603920in}}{\pgfqpoint{2.721357in}{3.598096in}}%
\pgfpathcurveto{\pgfqpoint{2.721357in}{3.592272in}}{\pgfqpoint{2.723671in}{3.586686in}}{\pgfqpoint{2.727789in}{3.582568in}}%
\pgfpathcurveto{\pgfqpoint{2.731907in}{3.578450in}}{\pgfqpoint{2.737493in}{3.576136in}}{\pgfqpoint{2.743317in}{3.576136in}}%
\pgfpathlineto{\pgfqpoint{2.743317in}{3.576136in}}%
\pgfpathclose%
\pgfusepath{stroke,fill}%
\end{pgfscope}%
\begin{pgfscope}%
\pgfpathrectangle{\pgfqpoint{0.100000in}{0.183744in}}{\pgfqpoint{4.506048in}{4.506048in}}%
\pgfusepath{clip}%
\pgfsetbuttcap%
\pgfsetroundjoin%
\definecolor{currentfill}{rgb}{0.000000,0.000000,1.000000}%
\pgfsetfillcolor{currentfill}%
\pgfsetfillopacity{0.700000}%
\pgfsetlinewidth{1.003750pt}%
\definecolor{currentstroke}{rgb}{0.000000,0.000000,1.000000}%
\pgfsetstrokecolor{currentstroke}%
\pgfsetstrokeopacity{0.700000}%
\pgfsetdash{}{0pt}%
\pgfpathmoveto{\pgfqpoint{3.118725in}{1.879456in}}%
\pgfpathcurveto{\pgfqpoint{3.124549in}{1.879456in}}{\pgfqpoint{3.130135in}{1.881770in}}{\pgfqpoint{3.134253in}{1.885888in}}%
\pgfpathcurveto{\pgfqpoint{3.138371in}{1.890006in}}{\pgfqpoint{3.140685in}{1.895592in}}{\pgfqpoint{3.140685in}{1.901416in}}%
\pgfpathcurveto{\pgfqpoint{3.140685in}{1.907240in}}{\pgfqpoint{3.138371in}{1.912826in}}{\pgfqpoint{3.134253in}{1.916944in}}%
\pgfpathcurveto{\pgfqpoint{3.130135in}{1.921063in}}{\pgfqpoint{3.124549in}{1.923376in}}{\pgfqpoint{3.118725in}{1.923376in}}%
\pgfpathcurveto{\pgfqpoint{3.112901in}{1.923376in}}{\pgfqpoint{3.107315in}{1.921063in}}{\pgfqpoint{3.103196in}{1.916944in}}%
\pgfpathcurveto{\pgfqpoint{3.099078in}{1.912826in}}{\pgfqpoint{3.096764in}{1.907240in}}{\pgfqpoint{3.096764in}{1.901416in}}%
\pgfpathcurveto{\pgfqpoint{3.096764in}{1.895592in}}{\pgfqpoint{3.099078in}{1.890006in}}{\pgfqpoint{3.103196in}{1.885888in}}%
\pgfpathcurveto{\pgfqpoint{3.107315in}{1.881770in}}{\pgfqpoint{3.112901in}{1.879456in}}{\pgfqpoint{3.118725in}{1.879456in}}%
\pgfpathlineto{\pgfqpoint{3.118725in}{1.879456in}}%
\pgfpathclose%
\pgfusepath{stroke,fill}%
\end{pgfscope}%
\begin{pgfscope}%
\pgfpathrectangle{\pgfqpoint{0.100000in}{0.183744in}}{\pgfqpoint{4.506048in}{4.506048in}}%
\pgfusepath{clip}%
\pgfsetbuttcap%
\pgfsetroundjoin%
\definecolor{currentfill}{rgb}{0.000000,0.000000,1.000000}%
\pgfsetfillcolor{currentfill}%
\pgfsetfillopacity{0.700000}%
\pgfsetlinewidth{1.003750pt}%
\definecolor{currentstroke}{rgb}{0.000000,0.000000,1.000000}%
\pgfsetstrokecolor{currentstroke}%
\pgfsetstrokeopacity{0.700000}%
\pgfsetdash{}{0pt}%
\pgfpathmoveto{\pgfqpoint{1.608716in}{2.269412in}}%
\pgfpathcurveto{\pgfqpoint{1.614540in}{2.269412in}}{\pgfqpoint{1.620127in}{2.271726in}}{\pgfqpoint{1.624245in}{2.275844in}}%
\pgfpathcurveto{\pgfqpoint{1.628363in}{2.279963in}}{\pgfqpoint{1.630677in}{2.285549in}}{\pgfqpoint{1.630677in}{2.291373in}}%
\pgfpathcurveto{\pgfqpoint{1.630677in}{2.297197in}}{\pgfqpoint{1.628363in}{2.302783in}}{\pgfqpoint{1.624245in}{2.306901in}}%
\pgfpathcurveto{\pgfqpoint{1.620127in}{2.311019in}}{\pgfqpoint{1.614540in}{2.313333in}}{\pgfqpoint{1.608716in}{2.313333in}}%
\pgfpathcurveto{\pgfqpoint{1.602893in}{2.313333in}}{\pgfqpoint{1.597306in}{2.311019in}}{\pgfqpoint{1.593188in}{2.306901in}}%
\pgfpathcurveto{\pgfqpoint{1.589070in}{2.302783in}}{\pgfqpoint{1.586756in}{2.297197in}}{\pgfqpoint{1.586756in}{2.291373in}}%
\pgfpathcurveto{\pgfqpoint{1.586756in}{2.285549in}}{\pgfqpoint{1.589070in}{2.279963in}}{\pgfqpoint{1.593188in}{2.275844in}}%
\pgfpathcurveto{\pgfqpoint{1.597306in}{2.271726in}}{\pgfqpoint{1.602893in}{2.269412in}}{\pgfqpoint{1.608716in}{2.269412in}}%
\pgfpathlineto{\pgfqpoint{1.608716in}{2.269412in}}%
\pgfpathclose%
\pgfusepath{stroke,fill}%
\end{pgfscope}%
\begin{pgfscope}%
\pgfpathrectangle{\pgfqpoint{0.100000in}{0.183744in}}{\pgfqpoint{4.506048in}{4.506048in}}%
\pgfusepath{clip}%
\pgfsetbuttcap%
\pgfsetroundjoin%
\definecolor{currentfill}{rgb}{0.000000,0.000000,1.000000}%
\pgfsetfillcolor{currentfill}%
\pgfsetfillopacity{0.700000}%
\pgfsetlinewidth{1.003750pt}%
\definecolor{currentstroke}{rgb}{0.000000,0.000000,1.000000}%
\pgfsetstrokecolor{currentstroke}%
\pgfsetstrokeopacity{0.700000}%
\pgfsetdash{}{0pt}%
\pgfpathmoveto{\pgfqpoint{2.720098in}{3.343895in}}%
\pgfpathcurveto{\pgfqpoint{2.725922in}{3.343895in}}{\pgfqpoint{2.731508in}{3.346209in}}{\pgfqpoint{2.735626in}{3.350327in}}%
\pgfpathcurveto{\pgfqpoint{2.739744in}{3.354445in}}{\pgfqpoint{2.742058in}{3.360031in}}{\pgfqpoint{2.742058in}{3.365855in}}%
\pgfpathcurveto{\pgfqpoint{2.742058in}{3.371679in}}{\pgfqpoint{2.739744in}{3.377265in}}{\pgfqpoint{2.735626in}{3.381383in}}%
\pgfpathcurveto{\pgfqpoint{2.731508in}{3.385502in}}{\pgfqpoint{2.725922in}{3.387815in}}{\pgfqpoint{2.720098in}{3.387815in}}%
\pgfpathcurveto{\pgfqpoint{2.714274in}{3.387815in}}{\pgfqpoint{2.708688in}{3.385502in}}{\pgfqpoint{2.704569in}{3.381383in}}%
\pgfpathcurveto{\pgfqpoint{2.700451in}{3.377265in}}{\pgfqpoint{2.698137in}{3.371679in}}{\pgfqpoint{2.698137in}{3.365855in}}%
\pgfpathcurveto{\pgfqpoint{2.698137in}{3.360031in}}{\pgfqpoint{2.700451in}{3.354445in}}{\pgfqpoint{2.704569in}{3.350327in}}%
\pgfpathcurveto{\pgfqpoint{2.708688in}{3.346209in}}{\pgfqpoint{2.714274in}{3.343895in}}{\pgfqpoint{2.720098in}{3.343895in}}%
\pgfpathlineto{\pgfqpoint{2.720098in}{3.343895in}}%
\pgfpathclose%
\pgfusepath{stroke,fill}%
\end{pgfscope}%
\begin{pgfscope}%
\pgfpathrectangle{\pgfqpoint{0.100000in}{0.183744in}}{\pgfqpoint{4.506048in}{4.506048in}}%
\pgfusepath{clip}%
\pgfsetbuttcap%
\pgfsetroundjoin%
\definecolor{currentfill}{rgb}{0.000000,0.000000,1.000000}%
\pgfsetfillcolor{currentfill}%
\pgfsetfillopacity{0.700000}%
\pgfsetlinewidth{1.003750pt}%
\definecolor{currentstroke}{rgb}{0.000000,0.000000,1.000000}%
\pgfsetstrokecolor{currentstroke}%
\pgfsetstrokeopacity{0.700000}%
\pgfsetdash{}{0pt}%
\pgfpathmoveto{\pgfqpoint{1.026121in}{2.014016in}}%
\pgfpathcurveto{\pgfqpoint{1.031945in}{2.014016in}}{\pgfqpoint{1.037531in}{2.016330in}}{\pgfqpoint{1.041650in}{2.020448in}}%
\pgfpathcurveto{\pgfqpoint{1.045768in}{2.024566in}}{\pgfqpoint{1.048082in}{2.030152in}}{\pgfqpoint{1.048082in}{2.035976in}}%
\pgfpathcurveto{\pgfqpoint{1.048082in}{2.041800in}}{\pgfqpoint{1.045768in}{2.047387in}}{\pgfqpoint{1.041650in}{2.051505in}}%
\pgfpathcurveto{\pgfqpoint{1.037531in}{2.055623in}}{\pgfqpoint{1.031945in}{2.057937in}}{\pgfqpoint{1.026121in}{2.057937in}}%
\pgfpathcurveto{\pgfqpoint{1.020297in}{2.057937in}}{\pgfqpoint{1.014711in}{2.055623in}}{\pgfqpoint{1.010593in}{2.051505in}}%
\pgfpathcurveto{\pgfqpoint{1.006475in}{2.047387in}}{\pgfqpoint{1.004161in}{2.041800in}}{\pgfqpoint{1.004161in}{2.035976in}}%
\pgfpathcurveto{\pgfqpoint{1.004161in}{2.030152in}}{\pgfqpoint{1.006475in}{2.024566in}}{\pgfqpoint{1.010593in}{2.020448in}}%
\pgfpathcurveto{\pgfqpoint{1.014711in}{2.016330in}}{\pgfqpoint{1.020297in}{2.014016in}}{\pgfqpoint{1.026121in}{2.014016in}}%
\pgfpathlineto{\pgfqpoint{1.026121in}{2.014016in}}%
\pgfpathclose%
\pgfusepath{stroke,fill}%
\end{pgfscope}%
\begin{pgfscope}%
\pgfpathrectangle{\pgfqpoint{0.100000in}{0.183744in}}{\pgfqpoint{4.506048in}{4.506048in}}%
\pgfusepath{clip}%
\pgfsetbuttcap%
\pgfsetroundjoin%
\definecolor{currentfill}{rgb}{0.000000,0.000000,1.000000}%
\pgfsetfillcolor{currentfill}%
\pgfsetfillopacity{0.700000}%
\pgfsetlinewidth{1.003750pt}%
\definecolor{currentstroke}{rgb}{0.000000,0.000000,1.000000}%
\pgfsetstrokecolor{currentstroke}%
\pgfsetstrokeopacity{0.700000}%
\pgfsetdash{}{0pt}%
\pgfpathmoveto{\pgfqpoint{2.831348in}{3.372980in}}%
\pgfpathcurveto{\pgfqpoint{2.837172in}{3.372980in}}{\pgfqpoint{2.842758in}{3.375294in}}{\pgfqpoint{2.846876in}{3.379412in}}%
\pgfpathcurveto{\pgfqpoint{2.850994in}{3.383531in}}{\pgfqpoint{2.853308in}{3.389117in}}{\pgfqpoint{2.853308in}{3.394941in}}%
\pgfpathcurveto{\pgfqpoint{2.853308in}{3.400765in}}{\pgfqpoint{2.850994in}{3.406351in}}{\pgfqpoint{2.846876in}{3.410469in}}%
\pgfpathcurveto{\pgfqpoint{2.842758in}{3.414587in}}{\pgfqpoint{2.837172in}{3.416901in}}{\pgfqpoint{2.831348in}{3.416901in}}%
\pgfpathcurveto{\pgfqpoint{2.825524in}{3.416901in}}{\pgfqpoint{2.819938in}{3.414587in}}{\pgfqpoint{2.815820in}{3.410469in}}%
\pgfpathcurveto{\pgfqpoint{2.811701in}{3.406351in}}{\pgfqpoint{2.809388in}{3.400765in}}{\pgfqpoint{2.809388in}{3.394941in}}%
\pgfpathcurveto{\pgfqpoint{2.809388in}{3.389117in}}{\pgfqpoint{2.811701in}{3.383531in}}{\pgfqpoint{2.815820in}{3.379412in}}%
\pgfpathcurveto{\pgfqpoint{2.819938in}{3.375294in}}{\pgfqpoint{2.825524in}{3.372980in}}{\pgfqpoint{2.831348in}{3.372980in}}%
\pgfpathlineto{\pgfqpoint{2.831348in}{3.372980in}}%
\pgfpathclose%
\pgfusepath{stroke,fill}%
\end{pgfscope}%
\begin{pgfscope}%
\pgfpathrectangle{\pgfqpoint{0.100000in}{0.183744in}}{\pgfqpoint{4.506048in}{4.506048in}}%
\pgfusepath{clip}%
\pgfsetbuttcap%
\pgfsetroundjoin%
\definecolor{currentfill}{rgb}{0.000000,0.000000,1.000000}%
\pgfsetfillcolor{currentfill}%
\pgfsetfillopacity{0.700000}%
\pgfsetlinewidth{1.003750pt}%
\definecolor{currentstroke}{rgb}{0.000000,0.000000,1.000000}%
\pgfsetstrokecolor{currentstroke}%
\pgfsetstrokeopacity{0.700000}%
\pgfsetdash{}{0pt}%
\pgfpathmoveto{\pgfqpoint{1.266303in}{3.029760in}}%
\pgfpathcurveto{\pgfqpoint{1.272127in}{3.029760in}}{\pgfqpoint{1.277713in}{3.032073in}}{\pgfqpoint{1.281831in}{3.036192in}}%
\pgfpathcurveto{\pgfqpoint{1.285949in}{3.040310in}}{\pgfqpoint{1.288263in}{3.045896in}}{\pgfqpoint{1.288263in}{3.051720in}}%
\pgfpathcurveto{\pgfqpoint{1.288263in}{3.057544in}}{\pgfqpoint{1.285949in}{3.063130in}}{\pgfqpoint{1.281831in}{3.067248in}}%
\pgfpathcurveto{\pgfqpoint{1.277713in}{3.071366in}}{\pgfqpoint{1.272127in}{3.073680in}}{\pgfqpoint{1.266303in}{3.073680in}}%
\pgfpathcurveto{\pgfqpoint{1.260479in}{3.073680in}}{\pgfqpoint{1.254893in}{3.071366in}}{\pgfqpoint{1.250775in}{3.067248in}}%
\pgfpathcurveto{\pgfqpoint{1.246656in}{3.063130in}}{\pgfqpoint{1.244343in}{3.057544in}}{\pgfqpoint{1.244343in}{3.051720in}}%
\pgfpathcurveto{\pgfqpoint{1.244343in}{3.045896in}}{\pgfqpoint{1.246656in}{3.040310in}}{\pgfqpoint{1.250775in}{3.036192in}}%
\pgfpathcurveto{\pgfqpoint{1.254893in}{3.032073in}}{\pgfqpoint{1.260479in}{3.029760in}}{\pgfqpoint{1.266303in}{3.029760in}}%
\pgfpathlineto{\pgfqpoint{1.266303in}{3.029760in}}%
\pgfpathclose%
\pgfusepath{stroke,fill}%
\end{pgfscope}%
\begin{pgfscope}%
\pgfpathrectangle{\pgfqpoint{0.100000in}{0.183744in}}{\pgfqpoint{4.506048in}{4.506048in}}%
\pgfusepath{clip}%
\pgfsetbuttcap%
\pgfsetroundjoin%
\definecolor{currentfill}{rgb}{0.000000,0.000000,1.000000}%
\pgfsetfillcolor{currentfill}%
\pgfsetfillopacity{0.700000}%
\pgfsetlinewidth{1.003750pt}%
\definecolor{currentstroke}{rgb}{0.000000,0.000000,1.000000}%
\pgfsetstrokecolor{currentstroke}%
\pgfsetstrokeopacity{0.700000}%
\pgfsetdash{}{0pt}%
\pgfpathmoveto{\pgfqpoint{3.185749in}{1.720301in}}%
\pgfpathcurveto{\pgfqpoint{3.191573in}{1.720301in}}{\pgfqpoint{3.197159in}{1.722614in}}{\pgfqpoint{3.201278in}{1.726733in}}%
\pgfpathcurveto{\pgfqpoint{3.205396in}{1.730851in}}{\pgfqpoint{3.207710in}{1.736437in}}{\pgfqpoint{3.207710in}{1.742261in}}%
\pgfpathcurveto{\pgfqpoint{3.207710in}{1.748085in}}{\pgfqpoint{3.205396in}{1.753671in}}{\pgfqpoint{3.201278in}{1.757789in}}%
\pgfpathcurveto{\pgfqpoint{3.197159in}{1.761907in}}{\pgfqpoint{3.191573in}{1.764221in}}{\pgfqpoint{3.185749in}{1.764221in}}%
\pgfpathcurveto{\pgfqpoint{3.179925in}{1.764221in}}{\pgfqpoint{3.174339in}{1.761907in}}{\pgfqpoint{3.170221in}{1.757789in}}%
\pgfpathcurveto{\pgfqpoint{3.166103in}{1.753671in}}{\pgfqpoint{3.163789in}{1.748085in}}{\pgfqpoint{3.163789in}{1.742261in}}%
\pgfpathcurveto{\pgfqpoint{3.163789in}{1.736437in}}{\pgfqpoint{3.166103in}{1.730851in}}{\pgfqpoint{3.170221in}{1.726733in}}%
\pgfpathcurveto{\pgfqpoint{3.174339in}{1.722614in}}{\pgfqpoint{3.179925in}{1.720301in}}{\pgfqpoint{3.185749in}{1.720301in}}%
\pgfpathlineto{\pgfqpoint{3.185749in}{1.720301in}}%
\pgfpathclose%
\pgfusepath{stroke,fill}%
\end{pgfscope}%
\begin{pgfscope}%
\pgfpathrectangle{\pgfqpoint{0.100000in}{0.183744in}}{\pgfqpoint{4.506048in}{4.506048in}}%
\pgfusepath{clip}%
\pgfsetbuttcap%
\pgfsetroundjoin%
\definecolor{currentfill}{rgb}{0.000000,0.000000,1.000000}%
\pgfsetfillcolor{currentfill}%
\pgfsetfillopacity{0.700000}%
\pgfsetlinewidth{1.003750pt}%
\definecolor{currentstroke}{rgb}{0.000000,0.000000,1.000000}%
\pgfsetstrokecolor{currentstroke}%
\pgfsetstrokeopacity{0.700000}%
\pgfsetdash{}{0pt}%
\pgfpathmoveto{\pgfqpoint{3.106414in}{2.526405in}}%
\pgfpathcurveto{\pgfqpoint{3.112238in}{2.526405in}}{\pgfqpoint{3.117824in}{2.528719in}}{\pgfqpoint{3.121943in}{2.532837in}}%
\pgfpathcurveto{\pgfqpoint{3.126061in}{2.536955in}}{\pgfqpoint{3.128375in}{2.542541in}}{\pgfqpoint{3.128375in}{2.548365in}}%
\pgfpathcurveto{\pgfqpoint{3.128375in}{2.554189in}}{\pgfqpoint{3.126061in}{2.559775in}}{\pgfqpoint{3.121943in}{2.563893in}}%
\pgfpathcurveto{\pgfqpoint{3.117824in}{2.568012in}}{\pgfqpoint{3.112238in}{2.570325in}}{\pgfqpoint{3.106414in}{2.570325in}}%
\pgfpathcurveto{\pgfqpoint{3.100590in}{2.570325in}}{\pgfqpoint{3.095004in}{2.568012in}}{\pgfqpoint{3.090886in}{2.563893in}}%
\pgfpathcurveto{\pgfqpoint{3.086768in}{2.559775in}}{\pgfqpoint{3.084454in}{2.554189in}}{\pgfqpoint{3.084454in}{2.548365in}}%
\pgfpathcurveto{\pgfqpoint{3.084454in}{2.542541in}}{\pgfqpoint{3.086768in}{2.536955in}}{\pgfqpoint{3.090886in}{2.532837in}}%
\pgfpathcurveto{\pgfqpoint{3.095004in}{2.528719in}}{\pgfqpoint{3.100590in}{2.526405in}}{\pgfqpoint{3.106414in}{2.526405in}}%
\pgfpathlineto{\pgfqpoint{3.106414in}{2.526405in}}%
\pgfpathclose%
\pgfusepath{stroke,fill}%
\end{pgfscope}%
\begin{pgfscope}%
\pgfpathrectangle{\pgfqpoint{0.100000in}{0.183744in}}{\pgfqpoint{4.506048in}{4.506048in}}%
\pgfusepath{clip}%
\pgfsetbuttcap%
\pgfsetroundjoin%
\definecolor{currentfill}{rgb}{0.000000,0.000000,1.000000}%
\pgfsetfillcolor{currentfill}%
\pgfsetfillopacity{0.700000}%
\pgfsetlinewidth{1.003750pt}%
\definecolor{currentstroke}{rgb}{0.000000,0.000000,1.000000}%
\pgfsetstrokecolor{currentstroke}%
\pgfsetstrokeopacity{0.700000}%
\pgfsetdash{}{0pt}%
\pgfpathmoveto{\pgfqpoint{3.212820in}{2.137141in}}%
\pgfpathcurveto{\pgfqpoint{3.218644in}{2.137141in}}{\pgfqpoint{3.224230in}{2.139455in}}{\pgfqpoint{3.228348in}{2.143573in}}%
\pgfpathcurveto{\pgfqpoint{3.232466in}{2.147691in}}{\pgfqpoint{3.234780in}{2.153277in}}{\pgfqpoint{3.234780in}{2.159101in}}%
\pgfpathcurveto{\pgfqpoint{3.234780in}{2.164925in}}{\pgfqpoint{3.232466in}{2.170511in}}{\pgfqpoint{3.228348in}{2.174629in}}%
\pgfpathcurveto{\pgfqpoint{3.224230in}{2.178748in}}{\pgfqpoint{3.218644in}{2.181062in}}{\pgfqpoint{3.212820in}{2.181062in}}%
\pgfpathcurveto{\pgfqpoint{3.206996in}{2.181062in}}{\pgfqpoint{3.201410in}{2.178748in}}{\pgfqpoint{3.197292in}{2.174629in}}%
\pgfpathcurveto{\pgfqpoint{3.193173in}{2.170511in}}{\pgfqpoint{3.190859in}{2.164925in}}{\pgfqpoint{3.190859in}{2.159101in}}%
\pgfpathcurveto{\pgfqpoint{3.190859in}{2.153277in}}{\pgfqpoint{3.193173in}{2.147691in}}{\pgfqpoint{3.197292in}{2.143573in}}%
\pgfpathcurveto{\pgfqpoint{3.201410in}{2.139455in}}{\pgfqpoint{3.206996in}{2.137141in}}{\pgfqpoint{3.212820in}{2.137141in}}%
\pgfpathlineto{\pgfqpoint{3.212820in}{2.137141in}}%
\pgfpathclose%
\pgfusepath{stroke,fill}%
\end{pgfscope}%
\begin{pgfscope}%
\pgfpathrectangle{\pgfqpoint{0.100000in}{0.183744in}}{\pgfqpoint{4.506048in}{4.506048in}}%
\pgfusepath{clip}%
\pgfsetbuttcap%
\pgfsetroundjoin%
\definecolor{currentfill}{rgb}{0.000000,0.000000,1.000000}%
\pgfsetfillcolor{currentfill}%
\pgfsetfillopacity{0.700000}%
\pgfsetlinewidth{1.003750pt}%
\definecolor{currentstroke}{rgb}{0.000000,0.000000,1.000000}%
\pgfsetstrokecolor{currentstroke}%
\pgfsetstrokeopacity{0.700000}%
\pgfsetdash{}{0pt}%
\pgfpathmoveto{\pgfqpoint{3.137871in}{2.513755in}}%
\pgfpathcurveto{\pgfqpoint{3.143695in}{2.513755in}}{\pgfqpoint{3.149281in}{2.516069in}}{\pgfqpoint{3.153399in}{2.520187in}}%
\pgfpathcurveto{\pgfqpoint{3.157517in}{2.524305in}}{\pgfqpoint{3.159831in}{2.529891in}}{\pgfqpoint{3.159831in}{2.535715in}}%
\pgfpathcurveto{\pgfqpoint{3.159831in}{2.541539in}}{\pgfqpoint{3.157517in}{2.547125in}}{\pgfqpoint{3.153399in}{2.551243in}}%
\pgfpathcurveto{\pgfqpoint{3.149281in}{2.555361in}}{\pgfqpoint{3.143695in}{2.557675in}}{\pgfqpoint{3.137871in}{2.557675in}}%
\pgfpathcurveto{\pgfqpoint{3.132047in}{2.557675in}}{\pgfqpoint{3.126461in}{2.555361in}}{\pgfqpoint{3.122343in}{2.551243in}}%
\pgfpathcurveto{\pgfqpoint{3.118224in}{2.547125in}}{\pgfqpoint{3.115911in}{2.541539in}}{\pgfqpoint{3.115911in}{2.535715in}}%
\pgfpathcurveto{\pgfqpoint{3.115911in}{2.529891in}}{\pgfqpoint{3.118224in}{2.524305in}}{\pgfqpoint{3.122343in}{2.520187in}}%
\pgfpathcurveto{\pgfqpoint{3.126461in}{2.516069in}}{\pgfqpoint{3.132047in}{2.513755in}}{\pgfqpoint{3.137871in}{2.513755in}}%
\pgfpathlineto{\pgfqpoint{3.137871in}{2.513755in}}%
\pgfpathclose%
\pgfusepath{stroke,fill}%
\end{pgfscope}%
\begin{pgfscope}%
\pgfpathrectangle{\pgfqpoint{0.100000in}{0.183744in}}{\pgfqpoint{4.506048in}{4.506048in}}%
\pgfusepath{clip}%
\pgfsetbuttcap%
\pgfsetroundjoin%
\definecolor{currentfill}{rgb}{0.000000,0.000000,1.000000}%
\pgfsetfillcolor{currentfill}%
\pgfsetfillopacity{0.700000}%
\pgfsetlinewidth{1.003750pt}%
\definecolor{currentstroke}{rgb}{0.000000,0.000000,1.000000}%
\pgfsetstrokecolor{currentstroke}%
\pgfsetstrokeopacity{0.700000}%
\pgfsetdash{}{0pt}%
\pgfpathmoveto{\pgfqpoint{3.055505in}{4.168912in}}%
\pgfpathcurveto{\pgfqpoint{3.061329in}{4.168912in}}{\pgfqpoint{3.066915in}{4.171226in}}{\pgfqpoint{3.071033in}{4.175344in}}%
\pgfpathcurveto{\pgfqpoint{3.075152in}{4.179462in}}{\pgfqpoint{3.077465in}{4.185048in}}{\pgfqpoint{3.077465in}{4.190872in}}%
\pgfpathcurveto{\pgfqpoint{3.077465in}{4.196696in}}{\pgfqpoint{3.075152in}{4.202282in}}{\pgfqpoint{3.071033in}{4.206401in}}%
\pgfpathcurveto{\pgfqpoint{3.066915in}{4.210519in}}{\pgfqpoint{3.061329in}{4.212833in}}{\pgfqpoint{3.055505in}{4.212833in}}%
\pgfpathcurveto{\pgfqpoint{3.049681in}{4.212833in}}{\pgfqpoint{3.044095in}{4.210519in}}{\pgfqpoint{3.039977in}{4.206401in}}%
\pgfpathcurveto{\pgfqpoint{3.035859in}{4.202282in}}{\pgfqpoint{3.033545in}{4.196696in}}{\pgfqpoint{3.033545in}{4.190872in}}%
\pgfpathcurveto{\pgfqpoint{3.033545in}{4.185048in}}{\pgfqpoint{3.035859in}{4.179462in}}{\pgfqpoint{3.039977in}{4.175344in}}%
\pgfpathcurveto{\pgfqpoint{3.044095in}{4.171226in}}{\pgfqpoint{3.049681in}{4.168912in}}{\pgfqpoint{3.055505in}{4.168912in}}%
\pgfpathlineto{\pgfqpoint{3.055505in}{4.168912in}}%
\pgfpathclose%
\pgfusepath{stroke,fill}%
\end{pgfscope}%
\begin{pgfscope}%
\pgfpathrectangle{\pgfqpoint{0.100000in}{0.183744in}}{\pgfqpoint{4.506048in}{4.506048in}}%
\pgfusepath{clip}%
\pgfsetbuttcap%
\pgfsetroundjoin%
\definecolor{currentfill}{rgb}{0.000000,0.000000,1.000000}%
\pgfsetfillcolor{currentfill}%
\pgfsetfillopacity{0.700000}%
\pgfsetlinewidth{1.003750pt}%
\definecolor{currentstroke}{rgb}{0.000000,0.000000,1.000000}%
\pgfsetstrokecolor{currentstroke}%
\pgfsetstrokeopacity{0.700000}%
\pgfsetdash{}{0pt}%
\pgfpathmoveto{\pgfqpoint{2.491029in}{3.032463in}}%
\pgfpathcurveto{\pgfqpoint{2.496853in}{3.032463in}}{\pgfqpoint{2.502439in}{3.034777in}}{\pgfqpoint{2.506557in}{3.038895in}}%
\pgfpathcurveto{\pgfqpoint{2.510675in}{3.043013in}}{\pgfqpoint{2.512989in}{3.048599in}}{\pgfqpoint{2.512989in}{3.054423in}}%
\pgfpathcurveto{\pgfqpoint{2.512989in}{3.060247in}}{\pgfqpoint{2.510675in}{3.065833in}}{\pgfqpoint{2.506557in}{3.069952in}}%
\pgfpathcurveto{\pgfqpoint{2.502439in}{3.074070in}}{\pgfqpoint{2.496853in}{3.076384in}}{\pgfqpoint{2.491029in}{3.076384in}}%
\pgfpathcurveto{\pgfqpoint{2.485205in}{3.076384in}}{\pgfqpoint{2.479619in}{3.074070in}}{\pgfqpoint{2.475501in}{3.069952in}}%
\pgfpathcurveto{\pgfqpoint{2.471382in}{3.065833in}}{\pgfqpoint{2.469069in}{3.060247in}}{\pgfqpoint{2.469069in}{3.054423in}}%
\pgfpathcurveto{\pgfqpoint{2.469069in}{3.048599in}}{\pgfqpoint{2.471382in}{3.043013in}}{\pgfqpoint{2.475501in}{3.038895in}}%
\pgfpathcurveto{\pgfqpoint{2.479619in}{3.034777in}}{\pgfqpoint{2.485205in}{3.032463in}}{\pgfqpoint{2.491029in}{3.032463in}}%
\pgfpathlineto{\pgfqpoint{2.491029in}{3.032463in}}%
\pgfpathclose%
\pgfusepath{stroke,fill}%
\end{pgfscope}%
\begin{pgfscope}%
\pgfpathrectangle{\pgfqpoint{0.100000in}{0.183744in}}{\pgfqpoint{4.506048in}{4.506048in}}%
\pgfusepath{clip}%
\pgfsetbuttcap%
\pgfsetroundjoin%
\definecolor{currentfill}{rgb}{0.000000,0.000000,1.000000}%
\pgfsetfillcolor{currentfill}%
\pgfsetfillopacity{0.700000}%
\pgfsetlinewidth{1.003750pt}%
\definecolor{currentstroke}{rgb}{0.000000,0.000000,1.000000}%
\pgfsetstrokecolor{currentstroke}%
\pgfsetstrokeopacity{0.700000}%
\pgfsetdash{}{0pt}%
\pgfpathmoveto{\pgfqpoint{1.623595in}{1.895452in}}%
\pgfpathcurveto{\pgfqpoint{1.629419in}{1.895452in}}{\pgfqpoint{1.635005in}{1.897766in}}{\pgfqpoint{1.639123in}{1.901884in}}%
\pgfpathcurveto{\pgfqpoint{1.643241in}{1.906002in}}{\pgfqpoint{1.645555in}{1.911588in}}{\pgfqpoint{1.645555in}{1.917412in}}%
\pgfpathcurveto{\pgfqpoint{1.645555in}{1.923236in}}{\pgfqpoint{1.643241in}{1.928822in}}{\pgfqpoint{1.639123in}{1.932940in}}%
\pgfpathcurveto{\pgfqpoint{1.635005in}{1.937058in}}{\pgfqpoint{1.629419in}{1.939372in}}{\pgfqpoint{1.623595in}{1.939372in}}%
\pgfpathcurveto{\pgfqpoint{1.617771in}{1.939372in}}{\pgfqpoint{1.612185in}{1.937058in}}{\pgfqpoint{1.608067in}{1.932940in}}%
\pgfpathcurveto{\pgfqpoint{1.603949in}{1.928822in}}{\pgfqpoint{1.601635in}{1.923236in}}{\pgfqpoint{1.601635in}{1.917412in}}%
\pgfpathcurveto{\pgfqpoint{1.601635in}{1.911588in}}{\pgfqpoint{1.603949in}{1.906002in}}{\pgfqpoint{1.608067in}{1.901884in}}%
\pgfpathcurveto{\pgfqpoint{1.612185in}{1.897766in}}{\pgfqpoint{1.617771in}{1.895452in}}{\pgfqpoint{1.623595in}{1.895452in}}%
\pgfpathlineto{\pgfqpoint{1.623595in}{1.895452in}}%
\pgfpathclose%
\pgfusepath{stroke,fill}%
\end{pgfscope}%
\begin{pgfscope}%
\pgfpathrectangle{\pgfqpoint{0.100000in}{0.183744in}}{\pgfqpoint{4.506048in}{4.506048in}}%
\pgfusepath{clip}%
\pgfsetbuttcap%
\pgfsetroundjoin%
\definecolor{currentfill}{rgb}{0.000000,0.000000,1.000000}%
\pgfsetfillcolor{currentfill}%
\pgfsetfillopacity{0.700000}%
\pgfsetlinewidth{1.003750pt}%
\definecolor{currentstroke}{rgb}{0.000000,0.000000,1.000000}%
\pgfsetstrokecolor{currentstroke}%
\pgfsetstrokeopacity{0.700000}%
\pgfsetdash{}{0pt}%
\pgfpathmoveto{\pgfqpoint{2.514070in}{2.688816in}}%
\pgfpathcurveto{\pgfqpoint{2.519894in}{2.688816in}}{\pgfqpoint{2.525480in}{2.691130in}}{\pgfqpoint{2.529598in}{2.695248in}}%
\pgfpathcurveto{\pgfqpoint{2.533716in}{2.699367in}}{\pgfqpoint{2.536030in}{2.704953in}}{\pgfqpoint{2.536030in}{2.710777in}}%
\pgfpathcurveto{\pgfqpoint{2.536030in}{2.716601in}}{\pgfqpoint{2.533716in}{2.722187in}}{\pgfqpoint{2.529598in}{2.726305in}}%
\pgfpathcurveto{\pgfqpoint{2.525480in}{2.730423in}}{\pgfqpoint{2.519894in}{2.732737in}}{\pgfqpoint{2.514070in}{2.732737in}}%
\pgfpathcurveto{\pgfqpoint{2.508246in}{2.732737in}}{\pgfqpoint{2.502660in}{2.730423in}}{\pgfqpoint{2.498542in}{2.726305in}}%
\pgfpathcurveto{\pgfqpoint{2.494424in}{2.722187in}}{\pgfqpoint{2.492110in}{2.716601in}}{\pgfqpoint{2.492110in}{2.710777in}}%
\pgfpathcurveto{\pgfqpoint{2.492110in}{2.704953in}}{\pgfqpoint{2.494424in}{2.699367in}}{\pgfqpoint{2.498542in}{2.695248in}}%
\pgfpathcurveto{\pgfqpoint{2.502660in}{2.691130in}}{\pgfqpoint{2.508246in}{2.688816in}}{\pgfqpoint{2.514070in}{2.688816in}}%
\pgfpathlineto{\pgfqpoint{2.514070in}{2.688816in}}%
\pgfpathclose%
\pgfusepath{stroke,fill}%
\end{pgfscope}%
\begin{pgfscope}%
\pgfpathrectangle{\pgfqpoint{0.100000in}{0.183744in}}{\pgfqpoint{4.506048in}{4.506048in}}%
\pgfusepath{clip}%
\pgfsetbuttcap%
\pgfsetroundjoin%
\definecolor{currentfill}{rgb}{0.000000,0.000000,1.000000}%
\pgfsetfillcolor{currentfill}%
\pgfsetfillopacity{0.700000}%
\pgfsetlinewidth{1.003750pt}%
\definecolor{currentstroke}{rgb}{0.000000,0.000000,1.000000}%
\pgfsetstrokecolor{currentstroke}%
\pgfsetstrokeopacity{0.700000}%
\pgfsetdash{}{0pt}%
\pgfpathmoveto{\pgfqpoint{1.587329in}{3.324581in}}%
\pgfpathcurveto{\pgfqpoint{1.593153in}{3.324581in}}{\pgfqpoint{1.598739in}{3.326895in}}{\pgfqpoint{1.602857in}{3.331013in}}%
\pgfpathcurveto{\pgfqpoint{1.606976in}{3.335131in}}{\pgfqpoint{1.609289in}{3.340718in}}{\pgfqpoint{1.609289in}{3.346542in}}%
\pgfpathcurveto{\pgfqpoint{1.609289in}{3.352366in}}{\pgfqpoint{1.606976in}{3.357952in}}{\pgfqpoint{1.602857in}{3.362070in}}%
\pgfpathcurveto{\pgfqpoint{1.598739in}{3.366188in}}{\pgfqpoint{1.593153in}{3.368502in}}{\pgfqpoint{1.587329in}{3.368502in}}%
\pgfpathcurveto{\pgfqpoint{1.581505in}{3.368502in}}{\pgfqpoint{1.575919in}{3.366188in}}{\pgfqpoint{1.571801in}{3.362070in}}%
\pgfpathcurveto{\pgfqpoint{1.567683in}{3.357952in}}{\pgfqpoint{1.565369in}{3.352366in}}{\pgfqpoint{1.565369in}{3.346542in}}%
\pgfpathcurveto{\pgfqpoint{1.565369in}{3.340718in}}{\pgfqpoint{1.567683in}{3.335131in}}{\pgfqpoint{1.571801in}{3.331013in}}%
\pgfpathcurveto{\pgfqpoint{1.575919in}{3.326895in}}{\pgfqpoint{1.581505in}{3.324581in}}{\pgfqpoint{1.587329in}{3.324581in}}%
\pgfpathlineto{\pgfqpoint{1.587329in}{3.324581in}}%
\pgfpathclose%
\pgfusepath{stroke,fill}%
\end{pgfscope}%
\begin{pgfscope}%
\pgfpathrectangle{\pgfqpoint{0.100000in}{0.183744in}}{\pgfqpoint{4.506048in}{4.506048in}}%
\pgfusepath{clip}%
\pgfsetbuttcap%
\pgfsetroundjoin%
\definecolor{currentfill}{rgb}{0.000000,0.000000,1.000000}%
\pgfsetfillcolor{currentfill}%
\pgfsetfillopacity{0.700000}%
\pgfsetlinewidth{1.003750pt}%
\definecolor{currentstroke}{rgb}{0.000000,0.000000,1.000000}%
\pgfsetstrokecolor{currentstroke}%
\pgfsetstrokeopacity{0.700000}%
\pgfsetdash{}{0pt}%
\pgfpathmoveto{\pgfqpoint{2.809618in}{1.534364in}}%
\pgfpathcurveto{\pgfqpoint{2.815442in}{1.534364in}}{\pgfqpoint{2.821028in}{1.536678in}}{\pgfqpoint{2.825146in}{1.540796in}}%
\pgfpathcurveto{\pgfqpoint{2.829264in}{1.544914in}}{\pgfqpoint{2.831578in}{1.550501in}}{\pgfqpoint{2.831578in}{1.556324in}}%
\pgfpathcurveto{\pgfqpoint{2.831578in}{1.562148in}}{\pgfqpoint{2.829264in}{1.567735in}}{\pgfqpoint{2.825146in}{1.571853in}}%
\pgfpathcurveto{\pgfqpoint{2.821028in}{1.575971in}}{\pgfqpoint{2.815442in}{1.578285in}}{\pgfqpoint{2.809618in}{1.578285in}}%
\pgfpathcurveto{\pgfqpoint{2.803794in}{1.578285in}}{\pgfqpoint{2.798208in}{1.575971in}}{\pgfqpoint{2.794090in}{1.571853in}}%
\pgfpathcurveto{\pgfqpoint{2.789971in}{1.567735in}}{\pgfqpoint{2.787658in}{1.562148in}}{\pgfqpoint{2.787658in}{1.556324in}}%
\pgfpathcurveto{\pgfqpoint{2.787658in}{1.550501in}}{\pgfqpoint{2.789971in}{1.544914in}}{\pgfqpoint{2.794090in}{1.540796in}}%
\pgfpathcurveto{\pgfqpoint{2.798208in}{1.536678in}}{\pgfqpoint{2.803794in}{1.534364in}}{\pgfqpoint{2.809618in}{1.534364in}}%
\pgfpathlineto{\pgfqpoint{2.809618in}{1.534364in}}%
\pgfpathclose%
\pgfusepath{stroke,fill}%
\end{pgfscope}%
\begin{pgfscope}%
\pgfpathrectangle{\pgfqpoint{0.100000in}{0.183744in}}{\pgfqpoint{4.506048in}{4.506048in}}%
\pgfusepath{clip}%
\pgfsetbuttcap%
\pgfsetroundjoin%
\definecolor{currentfill}{rgb}{0.000000,0.000000,1.000000}%
\pgfsetfillcolor{currentfill}%
\pgfsetfillopacity{0.700000}%
\pgfsetlinewidth{1.003750pt}%
\definecolor{currentstroke}{rgb}{0.000000,0.000000,1.000000}%
\pgfsetstrokecolor{currentstroke}%
\pgfsetstrokeopacity{0.700000}%
\pgfsetdash{}{0pt}%
\pgfpathmoveto{\pgfqpoint{4.066098in}{1.721581in}}%
\pgfpathcurveto{\pgfqpoint{4.071922in}{1.721581in}}{\pgfqpoint{4.077509in}{1.723895in}}{\pgfqpoint{4.081627in}{1.728013in}}%
\pgfpathcurveto{\pgfqpoint{4.085745in}{1.732131in}}{\pgfqpoint{4.088059in}{1.737717in}}{\pgfqpoint{4.088059in}{1.743541in}}%
\pgfpathcurveto{\pgfqpoint{4.088059in}{1.749365in}}{\pgfqpoint{4.085745in}{1.754951in}}{\pgfqpoint{4.081627in}{1.759069in}}%
\pgfpathcurveto{\pgfqpoint{4.077509in}{1.763188in}}{\pgfqpoint{4.071922in}{1.765502in}}{\pgfqpoint{4.066098in}{1.765502in}}%
\pgfpathcurveto{\pgfqpoint{4.060275in}{1.765502in}}{\pgfqpoint{4.054688in}{1.763188in}}{\pgfqpoint{4.050570in}{1.759069in}}%
\pgfpathcurveto{\pgfqpoint{4.046452in}{1.754951in}}{\pgfqpoint{4.044138in}{1.749365in}}{\pgfqpoint{4.044138in}{1.743541in}}%
\pgfpathcurveto{\pgfqpoint{4.044138in}{1.737717in}}{\pgfqpoint{4.046452in}{1.732131in}}{\pgfqpoint{4.050570in}{1.728013in}}%
\pgfpathcurveto{\pgfqpoint{4.054688in}{1.723895in}}{\pgfqpoint{4.060275in}{1.721581in}}{\pgfqpoint{4.066098in}{1.721581in}}%
\pgfpathlineto{\pgfqpoint{4.066098in}{1.721581in}}%
\pgfpathclose%
\pgfusepath{stroke,fill}%
\end{pgfscope}%
\begin{pgfscope}%
\pgfpathrectangle{\pgfqpoint{0.100000in}{0.183744in}}{\pgfqpoint{4.506048in}{4.506048in}}%
\pgfusepath{clip}%
\pgfsetbuttcap%
\pgfsetroundjoin%
\definecolor{currentfill}{rgb}{0.000000,0.000000,1.000000}%
\pgfsetfillcolor{currentfill}%
\pgfsetfillopacity{0.700000}%
\pgfsetlinewidth{1.003750pt}%
\definecolor{currentstroke}{rgb}{0.000000,0.000000,1.000000}%
\pgfsetstrokecolor{currentstroke}%
\pgfsetstrokeopacity{0.700000}%
\pgfsetdash{}{0pt}%
\pgfpathmoveto{\pgfqpoint{1.988965in}{1.456308in}}%
\pgfpathcurveto{\pgfqpoint{1.994789in}{1.456308in}}{\pgfqpoint{2.000375in}{1.458622in}}{\pgfqpoint{2.004494in}{1.462740in}}%
\pgfpathcurveto{\pgfqpoint{2.008612in}{1.466858in}}{\pgfqpoint{2.010926in}{1.472445in}}{\pgfqpoint{2.010926in}{1.478269in}}%
\pgfpathcurveto{\pgfqpoint{2.010926in}{1.484093in}}{\pgfqpoint{2.008612in}{1.489679in}}{\pgfqpoint{2.004494in}{1.493797in}}%
\pgfpathcurveto{\pgfqpoint{2.000375in}{1.497915in}}{\pgfqpoint{1.994789in}{1.500229in}}{\pgfqpoint{1.988965in}{1.500229in}}%
\pgfpathcurveto{\pgfqpoint{1.983141in}{1.500229in}}{\pgfqpoint{1.977555in}{1.497915in}}{\pgfqpoint{1.973437in}{1.493797in}}%
\pgfpathcurveto{\pgfqpoint{1.969319in}{1.489679in}}{\pgfqpoint{1.967005in}{1.484093in}}{\pgfqpoint{1.967005in}{1.478269in}}%
\pgfpathcurveto{\pgfqpoint{1.967005in}{1.472445in}}{\pgfqpoint{1.969319in}{1.466858in}}{\pgfqpoint{1.973437in}{1.462740in}}%
\pgfpathcurveto{\pgfqpoint{1.977555in}{1.458622in}}{\pgfqpoint{1.983141in}{1.456308in}}{\pgfqpoint{1.988965in}{1.456308in}}%
\pgfpathlineto{\pgfqpoint{1.988965in}{1.456308in}}%
\pgfpathclose%
\pgfusepath{stroke,fill}%
\end{pgfscope}%
\begin{pgfscope}%
\pgfpathrectangle{\pgfqpoint{0.100000in}{0.183744in}}{\pgfqpoint{4.506048in}{4.506048in}}%
\pgfusepath{clip}%
\pgfsetbuttcap%
\pgfsetroundjoin%
\definecolor{currentfill}{rgb}{0.000000,0.000000,1.000000}%
\pgfsetfillcolor{currentfill}%
\pgfsetfillopacity{0.700000}%
\pgfsetlinewidth{1.003750pt}%
\definecolor{currentstroke}{rgb}{0.000000,0.000000,1.000000}%
\pgfsetstrokecolor{currentstroke}%
\pgfsetstrokeopacity{0.700000}%
\pgfsetdash{}{0pt}%
\pgfpathmoveto{\pgfqpoint{2.762070in}{2.444926in}}%
\pgfpathcurveto{\pgfqpoint{2.767894in}{2.444926in}}{\pgfqpoint{2.773480in}{2.447240in}}{\pgfqpoint{2.777598in}{2.451358in}}%
\pgfpathcurveto{\pgfqpoint{2.781716in}{2.455476in}}{\pgfqpoint{2.784030in}{2.461063in}}{\pgfqpoint{2.784030in}{2.466887in}}%
\pgfpathcurveto{\pgfqpoint{2.784030in}{2.472710in}}{\pgfqpoint{2.781716in}{2.478297in}}{\pgfqpoint{2.777598in}{2.482415in}}%
\pgfpathcurveto{\pgfqpoint{2.773480in}{2.486533in}}{\pgfqpoint{2.767894in}{2.488847in}}{\pgfqpoint{2.762070in}{2.488847in}}%
\pgfpathcurveto{\pgfqpoint{2.756246in}{2.488847in}}{\pgfqpoint{2.750660in}{2.486533in}}{\pgfqpoint{2.746542in}{2.482415in}}%
\pgfpathcurveto{\pgfqpoint{2.742424in}{2.478297in}}{\pgfqpoint{2.740110in}{2.472710in}}{\pgfqpoint{2.740110in}{2.466887in}}%
\pgfpathcurveto{\pgfqpoint{2.740110in}{2.461063in}}{\pgfqpoint{2.742424in}{2.455476in}}{\pgfqpoint{2.746542in}{2.451358in}}%
\pgfpathcurveto{\pgfqpoint{2.750660in}{2.447240in}}{\pgfqpoint{2.756246in}{2.444926in}}{\pgfqpoint{2.762070in}{2.444926in}}%
\pgfpathlineto{\pgfqpoint{2.762070in}{2.444926in}}%
\pgfpathclose%
\pgfusepath{stroke,fill}%
\end{pgfscope}%
\begin{pgfscope}%
\pgfpathrectangle{\pgfqpoint{0.100000in}{0.183744in}}{\pgfqpoint{4.506048in}{4.506048in}}%
\pgfusepath{clip}%
\pgfsetbuttcap%
\pgfsetroundjoin%
\definecolor{currentfill}{rgb}{0.000000,0.000000,1.000000}%
\pgfsetfillcolor{currentfill}%
\pgfsetfillopacity{0.700000}%
\pgfsetlinewidth{1.003750pt}%
\definecolor{currentstroke}{rgb}{0.000000,0.000000,1.000000}%
\pgfsetstrokecolor{currentstroke}%
\pgfsetstrokeopacity{0.700000}%
\pgfsetdash{}{0pt}%
\pgfpathmoveto{\pgfqpoint{3.378585in}{2.843462in}}%
\pgfpathcurveto{\pgfqpoint{3.384409in}{2.843462in}}{\pgfqpoint{3.389995in}{2.845776in}}{\pgfqpoint{3.394114in}{2.849894in}}%
\pgfpathcurveto{\pgfqpoint{3.398232in}{2.854013in}}{\pgfqpoint{3.400546in}{2.859599in}}{\pgfqpoint{3.400546in}{2.865423in}}%
\pgfpathcurveto{\pgfqpoint{3.400546in}{2.871247in}}{\pgfqpoint{3.398232in}{2.876833in}}{\pgfqpoint{3.394114in}{2.880951in}}%
\pgfpathcurveto{\pgfqpoint{3.389995in}{2.885069in}}{\pgfqpoint{3.384409in}{2.887383in}}{\pgfqpoint{3.378585in}{2.887383in}}%
\pgfpathcurveto{\pgfqpoint{3.372761in}{2.887383in}}{\pgfqpoint{3.367175in}{2.885069in}}{\pgfqpoint{3.363057in}{2.880951in}}%
\pgfpathcurveto{\pgfqpoint{3.358939in}{2.876833in}}{\pgfqpoint{3.356625in}{2.871247in}}{\pgfqpoint{3.356625in}{2.865423in}}%
\pgfpathcurveto{\pgfqpoint{3.356625in}{2.859599in}}{\pgfqpoint{3.358939in}{2.854013in}}{\pgfqpoint{3.363057in}{2.849894in}}%
\pgfpathcurveto{\pgfqpoint{3.367175in}{2.845776in}}{\pgfqpoint{3.372761in}{2.843462in}}{\pgfqpoint{3.378585in}{2.843462in}}%
\pgfpathlineto{\pgfqpoint{3.378585in}{2.843462in}}%
\pgfpathclose%
\pgfusepath{stroke,fill}%
\end{pgfscope}%
\begin{pgfscope}%
\pgfpathrectangle{\pgfqpoint{0.100000in}{0.183744in}}{\pgfqpoint{4.506048in}{4.506048in}}%
\pgfusepath{clip}%
\pgfsetbuttcap%
\pgfsetroundjoin%
\definecolor{currentfill}{rgb}{0.000000,0.000000,1.000000}%
\pgfsetfillcolor{currentfill}%
\pgfsetfillopacity{0.700000}%
\pgfsetlinewidth{1.003750pt}%
\definecolor{currentstroke}{rgb}{0.000000,0.000000,1.000000}%
\pgfsetstrokecolor{currentstroke}%
\pgfsetstrokeopacity{0.700000}%
\pgfsetdash{}{0pt}%
\pgfpathmoveto{\pgfqpoint{3.200638in}{2.794944in}}%
\pgfpathcurveto{\pgfqpoint{3.206461in}{2.794944in}}{\pgfqpoint{3.212048in}{2.797258in}}{\pgfqpoint{3.216166in}{2.801376in}}%
\pgfpathcurveto{\pgfqpoint{3.220284in}{2.805494in}}{\pgfqpoint{3.222598in}{2.811081in}}{\pgfqpoint{3.222598in}{2.816905in}}%
\pgfpathcurveto{\pgfqpoint{3.222598in}{2.822728in}}{\pgfqpoint{3.220284in}{2.828315in}}{\pgfqpoint{3.216166in}{2.832433in}}%
\pgfpathcurveto{\pgfqpoint{3.212048in}{2.836551in}}{\pgfqpoint{3.206461in}{2.838865in}}{\pgfqpoint{3.200638in}{2.838865in}}%
\pgfpathcurveto{\pgfqpoint{3.194814in}{2.838865in}}{\pgfqpoint{3.189227in}{2.836551in}}{\pgfqpoint{3.185109in}{2.832433in}}%
\pgfpathcurveto{\pgfqpoint{3.180991in}{2.828315in}}{\pgfqpoint{3.178677in}{2.822728in}}{\pgfqpoint{3.178677in}{2.816905in}}%
\pgfpathcurveto{\pgfqpoint{3.178677in}{2.811081in}}{\pgfqpoint{3.180991in}{2.805494in}}{\pgfqpoint{3.185109in}{2.801376in}}%
\pgfpathcurveto{\pgfqpoint{3.189227in}{2.797258in}}{\pgfqpoint{3.194814in}{2.794944in}}{\pgfqpoint{3.200638in}{2.794944in}}%
\pgfpathlineto{\pgfqpoint{3.200638in}{2.794944in}}%
\pgfpathclose%
\pgfusepath{stroke,fill}%
\end{pgfscope}%
\begin{pgfscope}%
\pgfpathrectangle{\pgfqpoint{0.100000in}{0.183744in}}{\pgfqpoint{4.506048in}{4.506048in}}%
\pgfusepath{clip}%
\pgfsetbuttcap%
\pgfsetroundjoin%
\definecolor{currentfill}{rgb}{0.000000,0.000000,1.000000}%
\pgfsetfillcolor{currentfill}%
\pgfsetfillopacity{0.700000}%
\pgfsetlinewidth{1.003750pt}%
\definecolor{currentstroke}{rgb}{0.000000,0.000000,1.000000}%
\pgfsetstrokecolor{currentstroke}%
\pgfsetstrokeopacity{0.700000}%
\pgfsetdash{}{0pt}%
\pgfpathmoveto{\pgfqpoint{2.681585in}{2.867443in}}%
\pgfpathcurveto{\pgfqpoint{2.687409in}{2.867443in}}{\pgfqpoint{2.692995in}{2.869756in}}{\pgfqpoint{2.697113in}{2.873875in}}%
\pgfpathcurveto{\pgfqpoint{2.701231in}{2.877993in}}{\pgfqpoint{2.703545in}{2.883579in}}{\pgfqpoint{2.703545in}{2.889403in}}%
\pgfpathcurveto{\pgfqpoint{2.703545in}{2.895227in}}{\pgfqpoint{2.701231in}{2.900813in}}{\pgfqpoint{2.697113in}{2.904931in}}%
\pgfpathcurveto{\pgfqpoint{2.692995in}{2.909049in}}{\pgfqpoint{2.687409in}{2.911363in}}{\pgfqpoint{2.681585in}{2.911363in}}%
\pgfpathcurveto{\pgfqpoint{2.675761in}{2.911363in}}{\pgfqpoint{2.670175in}{2.909049in}}{\pgfqpoint{2.666057in}{2.904931in}}%
\pgfpathcurveto{\pgfqpoint{2.661939in}{2.900813in}}{\pgfqpoint{2.659625in}{2.895227in}}{\pgfqpoint{2.659625in}{2.889403in}}%
\pgfpathcurveto{\pgfqpoint{2.659625in}{2.883579in}}{\pgfqpoint{2.661939in}{2.877993in}}{\pgfqpoint{2.666057in}{2.873875in}}%
\pgfpathcurveto{\pgfqpoint{2.670175in}{2.869756in}}{\pgfqpoint{2.675761in}{2.867443in}}{\pgfqpoint{2.681585in}{2.867443in}}%
\pgfpathlineto{\pgfqpoint{2.681585in}{2.867443in}}%
\pgfpathclose%
\pgfusepath{stroke,fill}%
\end{pgfscope}%
\begin{pgfscope}%
\pgfpathrectangle{\pgfqpoint{0.100000in}{0.183744in}}{\pgfqpoint{4.506048in}{4.506048in}}%
\pgfusepath{clip}%
\pgfsetbuttcap%
\pgfsetroundjoin%
\definecolor{currentfill}{rgb}{0.000000,0.000000,1.000000}%
\pgfsetfillcolor{currentfill}%
\pgfsetfillopacity{0.700000}%
\pgfsetlinewidth{1.003750pt}%
\definecolor{currentstroke}{rgb}{0.000000,0.000000,1.000000}%
\pgfsetstrokecolor{currentstroke}%
\pgfsetstrokeopacity{0.700000}%
\pgfsetdash{}{0pt}%
\pgfpathmoveto{\pgfqpoint{2.070991in}{2.268783in}}%
\pgfpathcurveto{\pgfqpoint{2.076815in}{2.268783in}}{\pgfqpoint{2.082401in}{2.271097in}}{\pgfqpoint{2.086519in}{2.275215in}}%
\pgfpathcurveto{\pgfqpoint{2.090637in}{2.279333in}}{\pgfqpoint{2.092951in}{2.284920in}}{\pgfqpoint{2.092951in}{2.290743in}}%
\pgfpathcurveto{\pgfqpoint{2.092951in}{2.296567in}}{\pgfqpoint{2.090637in}{2.302154in}}{\pgfqpoint{2.086519in}{2.306272in}}%
\pgfpathcurveto{\pgfqpoint{2.082401in}{2.310390in}}{\pgfqpoint{2.076815in}{2.312704in}}{\pgfqpoint{2.070991in}{2.312704in}}%
\pgfpathcurveto{\pgfqpoint{2.065167in}{2.312704in}}{\pgfqpoint{2.059581in}{2.310390in}}{\pgfqpoint{2.055463in}{2.306272in}}%
\pgfpathcurveto{\pgfqpoint{2.051345in}{2.302154in}}{\pgfqpoint{2.049031in}{2.296567in}}{\pgfqpoint{2.049031in}{2.290743in}}%
\pgfpathcurveto{\pgfqpoint{2.049031in}{2.284920in}}{\pgfqpoint{2.051345in}{2.279333in}}{\pgfqpoint{2.055463in}{2.275215in}}%
\pgfpathcurveto{\pgfqpoint{2.059581in}{2.271097in}}{\pgfqpoint{2.065167in}{2.268783in}}{\pgfqpoint{2.070991in}{2.268783in}}%
\pgfpathlineto{\pgfqpoint{2.070991in}{2.268783in}}%
\pgfpathclose%
\pgfusepath{stroke,fill}%
\end{pgfscope}%
\begin{pgfscope}%
\pgfpathrectangle{\pgfqpoint{0.100000in}{0.183744in}}{\pgfqpoint{4.506048in}{4.506048in}}%
\pgfusepath{clip}%
\pgfsetbuttcap%
\pgfsetroundjoin%
\definecolor{currentfill}{rgb}{0.000000,0.000000,1.000000}%
\pgfsetfillcolor{currentfill}%
\pgfsetfillopacity{0.700000}%
\pgfsetlinewidth{1.003750pt}%
\definecolor{currentstroke}{rgb}{0.000000,0.000000,1.000000}%
\pgfsetstrokecolor{currentstroke}%
\pgfsetstrokeopacity{0.700000}%
\pgfsetdash{}{0pt}%
\pgfpathmoveto{\pgfqpoint{2.251170in}{1.722264in}}%
\pgfpathcurveto{\pgfqpoint{2.256994in}{1.722264in}}{\pgfqpoint{2.262580in}{1.724578in}}{\pgfqpoint{2.266698in}{1.728696in}}%
\pgfpathcurveto{\pgfqpoint{2.270816in}{1.732815in}}{\pgfqpoint{2.273130in}{1.738401in}}{\pgfqpoint{2.273130in}{1.744225in}}%
\pgfpathcurveto{\pgfqpoint{2.273130in}{1.750049in}}{\pgfqpoint{2.270816in}{1.755635in}}{\pgfqpoint{2.266698in}{1.759753in}}%
\pgfpathcurveto{\pgfqpoint{2.262580in}{1.763871in}}{\pgfqpoint{2.256994in}{1.766185in}}{\pgfqpoint{2.251170in}{1.766185in}}%
\pgfpathcurveto{\pgfqpoint{2.245346in}{1.766185in}}{\pgfqpoint{2.239760in}{1.763871in}}{\pgfqpoint{2.235642in}{1.759753in}}%
\pgfpathcurveto{\pgfqpoint{2.231523in}{1.755635in}}{\pgfqpoint{2.229210in}{1.750049in}}{\pgfqpoint{2.229210in}{1.744225in}}%
\pgfpathcurveto{\pgfqpoint{2.229210in}{1.738401in}}{\pgfqpoint{2.231523in}{1.732815in}}{\pgfqpoint{2.235642in}{1.728696in}}%
\pgfpathcurveto{\pgfqpoint{2.239760in}{1.724578in}}{\pgfqpoint{2.245346in}{1.722264in}}{\pgfqpoint{2.251170in}{1.722264in}}%
\pgfpathlineto{\pgfqpoint{2.251170in}{1.722264in}}%
\pgfpathclose%
\pgfusepath{stroke,fill}%
\end{pgfscope}%
\begin{pgfscope}%
\pgfpathrectangle{\pgfqpoint{0.100000in}{0.183744in}}{\pgfqpoint{4.506048in}{4.506048in}}%
\pgfusepath{clip}%
\pgfsetbuttcap%
\pgfsetroundjoin%
\definecolor{currentfill}{rgb}{0.000000,0.000000,1.000000}%
\pgfsetfillcolor{currentfill}%
\pgfsetfillopacity{0.700000}%
\pgfsetlinewidth{1.003750pt}%
\definecolor{currentstroke}{rgb}{0.000000,0.000000,1.000000}%
\pgfsetstrokecolor{currentstroke}%
\pgfsetstrokeopacity{0.700000}%
\pgfsetdash{}{0pt}%
\pgfpathmoveto{\pgfqpoint{2.918424in}{3.079611in}}%
\pgfpathcurveto{\pgfqpoint{2.924248in}{3.079611in}}{\pgfqpoint{2.929834in}{3.081925in}}{\pgfqpoint{2.933952in}{3.086043in}}%
\pgfpathcurveto{\pgfqpoint{2.938071in}{3.090161in}}{\pgfqpoint{2.940384in}{3.095748in}}{\pgfqpoint{2.940384in}{3.101572in}}%
\pgfpathcurveto{\pgfqpoint{2.940384in}{3.107396in}}{\pgfqpoint{2.938071in}{3.112982in}}{\pgfqpoint{2.933952in}{3.117100in}}%
\pgfpathcurveto{\pgfqpoint{2.929834in}{3.121218in}}{\pgfqpoint{2.924248in}{3.123532in}}{\pgfqpoint{2.918424in}{3.123532in}}%
\pgfpathcurveto{\pgfqpoint{2.912600in}{3.123532in}}{\pgfqpoint{2.907014in}{3.121218in}}{\pgfqpoint{2.902896in}{3.117100in}}%
\pgfpathcurveto{\pgfqpoint{2.898778in}{3.112982in}}{\pgfqpoint{2.896464in}{3.107396in}}{\pgfqpoint{2.896464in}{3.101572in}}%
\pgfpathcurveto{\pgfqpoint{2.896464in}{3.095748in}}{\pgfqpoint{2.898778in}{3.090161in}}{\pgfqpoint{2.902896in}{3.086043in}}%
\pgfpathcurveto{\pgfqpoint{2.907014in}{3.081925in}}{\pgfqpoint{2.912600in}{3.079611in}}{\pgfqpoint{2.918424in}{3.079611in}}%
\pgfpathlineto{\pgfqpoint{2.918424in}{3.079611in}}%
\pgfpathclose%
\pgfusepath{stroke,fill}%
\end{pgfscope}%
\begin{pgfscope}%
\pgfpathrectangle{\pgfqpoint{0.100000in}{0.183744in}}{\pgfqpoint{4.506048in}{4.506048in}}%
\pgfusepath{clip}%
\pgfsetbuttcap%
\pgfsetroundjoin%
\definecolor{currentfill}{rgb}{0.000000,0.000000,1.000000}%
\pgfsetfillcolor{currentfill}%
\pgfsetfillopacity{0.700000}%
\pgfsetlinewidth{1.003750pt}%
\definecolor{currentstroke}{rgb}{0.000000,0.000000,1.000000}%
\pgfsetstrokecolor{currentstroke}%
\pgfsetstrokeopacity{0.700000}%
\pgfsetdash{}{0pt}%
\pgfpathmoveto{\pgfqpoint{3.034512in}{1.273587in}}%
\pgfpathcurveto{\pgfqpoint{3.040335in}{1.273587in}}{\pgfqpoint{3.045922in}{1.275901in}}{\pgfqpoint{3.050040in}{1.280019in}}%
\pgfpathcurveto{\pgfqpoint{3.054158in}{1.284137in}}{\pgfqpoint{3.056472in}{1.289723in}}{\pgfqpoint{3.056472in}{1.295547in}}%
\pgfpathcurveto{\pgfqpoint{3.056472in}{1.301371in}}{\pgfqpoint{3.054158in}{1.306957in}}{\pgfqpoint{3.050040in}{1.311075in}}%
\pgfpathcurveto{\pgfqpoint{3.045922in}{1.315194in}}{\pgfqpoint{3.040335in}{1.317508in}}{\pgfqpoint{3.034512in}{1.317508in}}%
\pgfpathcurveto{\pgfqpoint{3.028688in}{1.317508in}}{\pgfqpoint{3.023101in}{1.315194in}}{\pgfqpoint{3.018983in}{1.311075in}}%
\pgfpathcurveto{\pgfqpoint{3.014865in}{1.306957in}}{\pgfqpoint{3.012551in}{1.301371in}}{\pgfqpoint{3.012551in}{1.295547in}}%
\pgfpathcurveto{\pgfqpoint{3.012551in}{1.289723in}}{\pgfqpoint{3.014865in}{1.284137in}}{\pgfqpoint{3.018983in}{1.280019in}}%
\pgfpathcurveto{\pgfqpoint{3.023101in}{1.275901in}}{\pgfqpoint{3.028688in}{1.273587in}}{\pgfqpoint{3.034512in}{1.273587in}}%
\pgfpathlineto{\pgfqpoint{3.034512in}{1.273587in}}%
\pgfpathclose%
\pgfusepath{stroke,fill}%
\end{pgfscope}%
\begin{pgfscope}%
\pgfpathrectangle{\pgfqpoint{0.100000in}{0.183744in}}{\pgfqpoint{4.506048in}{4.506048in}}%
\pgfusepath{clip}%
\pgfsetbuttcap%
\pgfsetroundjoin%
\definecolor{currentfill}{rgb}{0.000000,0.000000,1.000000}%
\pgfsetfillcolor{currentfill}%
\pgfsetfillopacity{0.700000}%
\pgfsetlinewidth{1.003750pt}%
\definecolor{currentstroke}{rgb}{0.000000,0.000000,1.000000}%
\pgfsetstrokecolor{currentstroke}%
\pgfsetstrokeopacity{0.700000}%
\pgfsetdash{}{0pt}%
\pgfpathmoveto{\pgfqpoint{2.718882in}{2.923931in}}%
\pgfpathcurveto{\pgfqpoint{2.724706in}{2.923931in}}{\pgfqpoint{2.730292in}{2.926245in}}{\pgfqpoint{2.734411in}{2.930363in}}%
\pgfpathcurveto{\pgfqpoint{2.738529in}{2.934482in}}{\pgfqpoint{2.740843in}{2.940068in}}{\pgfqpoint{2.740843in}{2.945892in}}%
\pgfpathcurveto{\pgfqpoint{2.740843in}{2.951716in}}{\pgfqpoint{2.738529in}{2.957302in}}{\pgfqpoint{2.734411in}{2.961420in}}%
\pgfpathcurveto{\pgfqpoint{2.730292in}{2.965538in}}{\pgfqpoint{2.724706in}{2.967852in}}{\pgfqpoint{2.718882in}{2.967852in}}%
\pgfpathcurveto{\pgfqpoint{2.713058in}{2.967852in}}{\pgfqpoint{2.707472in}{2.965538in}}{\pgfqpoint{2.703354in}{2.961420in}}%
\pgfpathcurveto{\pgfqpoint{2.699236in}{2.957302in}}{\pgfqpoint{2.696922in}{2.951716in}}{\pgfqpoint{2.696922in}{2.945892in}}%
\pgfpathcurveto{\pgfqpoint{2.696922in}{2.940068in}}{\pgfqpoint{2.699236in}{2.934482in}}{\pgfqpoint{2.703354in}{2.930363in}}%
\pgfpathcurveto{\pgfqpoint{2.707472in}{2.926245in}}{\pgfqpoint{2.713058in}{2.923931in}}{\pgfqpoint{2.718882in}{2.923931in}}%
\pgfpathlineto{\pgfqpoint{2.718882in}{2.923931in}}%
\pgfpathclose%
\pgfusepath{stroke,fill}%
\end{pgfscope}%
\begin{pgfscope}%
\pgfpathrectangle{\pgfqpoint{0.100000in}{0.183744in}}{\pgfqpoint{4.506048in}{4.506048in}}%
\pgfusepath{clip}%
\pgfsetbuttcap%
\pgfsetroundjoin%
\definecolor{currentfill}{rgb}{0.000000,0.000000,1.000000}%
\pgfsetfillcolor{currentfill}%
\pgfsetfillopacity{0.700000}%
\pgfsetlinewidth{1.003750pt}%
\definecolor{currentstroke}{rgb}{0.000000,0.000000,1.000000}%
\pgfsetstrokecolor{currentstroke}%
\pgfsetstrokeopacity{0.700000}%
\pgfsetdash{}{0pt}%
\pgfpathmoveto{\pgfqpoint{2.425806in}{3.566693in}}%
\pgfpathcurveto{\pgfqpoint{2.431630in}{3.566693in}}{\pgfqpoint{2.437216in}{3.569007in}}{\pgfqpoint{2.441335in}{3.573125in}}%
\pgfpathcurveto{\pgfqpoint{2.445453in}{3.577243in}}{\pgfqpoint{2.447767in}{3.582829in}}{\pgfqpoint{2.447767in}{3.588653in}}%
\pgfpathcurveto{\pgfqpoint{2.447767in}{3.594477in}}{\pgfqpoint{2.445453in}{3.600063in}}{\pgfqpoint{2.441335in}{3.604181in}}%
\pgfpathcurveto{\pgfqpoint{2.437216in}{3.608299in}}{\pgfqpoint{2.431630in}{3.610613in}}{\pgfqpoint{2.425806in}{3.610613in}}%
\pgfpathcurveto{\pgfqpoint{2.419982in}{3.610613in}}{\pgfqpoint{2.414396in}{3.608299in}}{\pgfqpoint{2.410278in}{3.604181in}}%
\pgfpathcurveto{\pgfqpoint{2.406160in}{3.600063in}}{\pgfqpoint{2.403846in}{3.594477in}}{\pgfqpoint{2.403846in}{3.588653in}}%
\pgfpathcurveto{\pgfqpoint{2.403846in}{3.582829in}}{\pgfqpoint{2.406160in}{3.577243in}}{\pgfqpoint{2.410278in}{3.573125in}}%
\pgfpathcurveto{\pgfqpoint{2.414396in}{3.569007in}}{\pgfqpoint{2.419982in}{3.566693in}}{\pgfqpoint{2.425806in}{3.566693in}}%
\pgfpathlineto{\pgfqpoint{2.425806in}{3.566693in}}%
\pgfpathclose%
\pgfusepath{stroke,fill}%
\end{pgfscope}%
\begin{pgfscope}%
\pgfpathrectangle{\pgfqpoint{0.100000in}{0.183744in}}{\pgfqpoint{4.506048in}{4.506048in}}%
\pgfusepath{clip}%
\pgfsetbuttcap%
\pgfsetroundjoin%
\definecolor{currentfill}{rgb}{0.000000,0.000000,1.000000}%
\pgfsetfillcolor{currentfill}%
\pgfsetfillopacity{0.700000}%
\pgfsetlinewidth{1.003750pt}%
\definecolor{currentstroke}{rgb}{0.000000,0.000000,1.000000}%
\pgfsetstrokecolor{currentstroke}%
\pgfsetstrokeopacity{0.700000}%
\pgfsetdash{}{0pt}%
\pgfpathmoveto{\pgfqpoint{3.089248in}{2.327693in}}%
\pgfpathcurveto{\pgfqpoint{3.095072in}{2.327693in}}{\pgfqpoint{3.100658in}{2.330006in}}{\pgfqpoint{3.104776in}{2.334125in}}%
\pgfpathcurveto{\pgfqpoint{3.108894in}{2.338243in}}{\pgfqpoint{3.111208in}{2.343829in}}{\pgfqpoint{3.111208in}{2.349653in}}%
\pgfpathcurveto{\pgfqpoint{3.111208in}{2.355477in}}{\pgfqpoint{3.108894in}{2.361063in}}{\pgfqpoint{3.104776in}{2.365181in}}%
\pgfpathcurveto{\pgfqpoint{3.100658in}{2.369299in}}{\pgfqpoint{3.095072in}{2.371613in}}{\pgfqpoint{3.089248in}{2.371613in}}%
\pgfpathcurveto{\pgfqpoint{3.083424in}{2.371613in}}{\pgfqpoint{3.077837in}{2.369299in}}{\pgfqpoint{3.073719in}{2.365181in}}%
\pgfpathcurveto{\pgfqpoint{3.069601in}{2.361063in}}{\pgfqpoint{3.067287in}{2.355477in}}{\pgfqpoint{3.067287in}{2.349653in}}%
\pgfpathcurveto{\pgfqpoint{3.067287in}{2.343829in}}{\pgfqpoint{3.069601in}{2.338243in}}{\pgfqpoint{3.073719in}{2.334125in}}%
\pgfpathcurveto{\pgfqpoint{3.077837in}{2.330006in}}{\pgfqpoint{3.083424in}{2.327693in}}{\pgfqpoint{3.089248in}{2.327693in}}%
\pgfpathlineto{\pgfqpoint{3.089248in}{2.327693in}}%
\pgfpathclose%
\pgfusepath{stroke,fill}%
\end{pgfscope}%
\begin{pgfscope}%
\pgfpathrectangle{\pgfqpoint{0.100000in}{0.183744in}}{\pgfqpoint{4.506048in}{4.506048in}}%
\pgfusepath{clip}%
\pgfsetbuttcap%
\pgfsetroundjoin%
\definecolor{currentfill}{rgb}{0.000000,0.000000,1.000000}%
\pgfsetfillcolor{currentfill}%
\pgfsetfillopacity{0.700000}%
\pgfsetlinewidth{1.003750pt}%
\definecolor{currentstroke}{rgb}{0.000000,0.000000,1.000000}%
\pgfsetstrokecolor{currentstroke}%
\pgfsetstrokeopacity{0.700000}%
\pgfsetdash{}{0pt}%
\pgfpathmoveto{\pgfqpoint{0.987324in}{1.788324in}}%
\pgfpathcurveto{\pgfqpoint{0.993147in}{1.788324in}}{\pgfqpoint{0.998734in}{1.790638in}}{\pgfqpoint{1.002852in}{1.794756in}}%
\pgfpathcurveto{\pgfqpoint{1.006970in}{1.798874in}}{\pgfqpoint{1.009284in}{1.804460in}}{\pgfqpoint{1.009284in}{1.810284in}}%
\pgfpathcurveto{\pgfqpoint{1.009284in}{1.816108in}}{\pgfqpoint{1.006970in}{1.821694in}}{\pgfqpoint{1.002852in}{1.825812in}}%
\pgfpathcurveto{\pgfqpoint{0.998734in}{1.829930in}}{\pgfqpoint{0.993147in}{1.832244in}}{\pgfqpoint{0.987324in}{1.832244in}}%
\pgfpathcurveto{\pgfqpoint{0.981500in}{1.832244in}}{\pgfqpoint{0.975913in}{1.829930in}}{\pgfqpoint{0.971795in}{1.825812in}}%
\pgfpathcurveto{\pgfqpoint{0.967677in}{1.821694in}}{\pgfqpoint{0.965363in}{1.816108in}}{\pgfqpoint{0.965363in}{1.810284in}}%
\pgfpathcurveto{\pgfqpoint{0.965363in}{1.804460in}}{\pgfqpoint{0.967677in}{1.798874in}}{\pgfqpoint{0.971795in}{1.794756in}}%
\pgfpathcurveto{\pgfqpoint{0.975913in}{1.790638in}}{\pgfqpoint{0.981500in}{1.788324in}}{\pgfqpoint{0.987324in}{1.788324in}}%
\pgfpathlineto{\pgfqpoint{0.987324in}{1.788324in}}%
\pgfpathclose%
\pgfusepath{stroke,fill}%
\end{pgfscope}%
\begin{pgfscope}%
\pgfpathrectangle{\pgfqpoint{0.100000in}{0.183744in}}{\pgfqpoint{4.506048in}{4.506048in}}%
\pgfusepath{clip}%
\pgfsetbuttcap%
\pgfsetroundjoin%
\definecolor{currentfill}{rgb}{0.000000,0.000000,1.000000}%
\pgfsetfillcolor{currentfill}%
\pgfsetfillopacity{0.700000}%
\pgfsetlinewidth{1.003750pt}%
\definecolor{currentstroke}{rgb}{0.000000,0.000000,1.000000}%
\pgfsetstrokecolor{currentstroke}%
\pgfsetstrokeopacity{0.700000}%
\pgfsetdash{}{0pt}%
\pgfpathmoveto{\pgfqpoint{1.857783in}{3.291160in}}%
\pgfpathcurveto{\pgfqpoint{1.863607in}{3.291160in}}{\pgfqpoint{1.869193in}{3.293474in}}{\pgfqpoint{1.873311in}{3.297592in}}%
\pgfpathcurveto{\pgfqpoint{1.877429in}{3.301710in}}{\pgfqpoint{1.879743in}{3.307296in}}{\pgfqpoint{1.879743in}{3.313120in}}%
\pgfpathcurveto{\pgfqpoint{1.879743in}{3.318944in}}{\pgfqpoint{1.877429in}{3.324530in}}{\pgfqpoint{1.873311in}{3.328649in}}%
\pgfpathcurveto{\pgfqpoint{1.869193in}{3.332767in}}{\pgfqpoint{1.863607in}{3.335081in}}{\pgfqpoint{1.857783in}{3.335081in}}%
\pgfpathcurveto{\pgfqpoint{1.851959in}{3.335081in}}{\pgfqpoint{1.846373in}{3.332767in}}{\pgfqpoint{1.842255in}{3.328649in}}%
\pgfpathcurveto{\pgfqpoint{1.838137in}{3.324530in}}{\pgfqpoint{1.835823in}{3.318944in}}{\pgfqpoint{1.835823in}{3.313120in}}%
\pgfpathcurveto{\pgfqpoint{1.835823in}{3.307296in}}{\pgfqpoint{1.838137in}{3.301710in}}{\pgfqpoint{1.842255in}{3.297592in}}%
\pgfpathcurveto{\pgfqpoint{1.846373in}{3.293474in}}{\pgfqpoint{1.851959in}{3.291160in}}{\pgfqpoint{1.857783in}{3.291160in}}%
\pgfpathlineto{\pgfqpoint{1.857783in}{3.291160in}}%
\pgfpathclose%
\pgfusepath{stroke,fill}%
\end{pgfscope}%
\begin{pgfscope}%
\pgfpathrectangle{\pgfqpoint{0.100000in}{0.183744in}}{\pgfqpoint{4.506048in}{4.506048in}}%
\pgfusepath{clip}%
\pgfsetbuttcap%
\pgfsetroundjoin%
\definecolor{currentfill}{rgb}{0.000000,0.000000,1.000000}%
\pgfsetfillcolor{currentfill}%
\pgfsetfillopacity{0.700000}%
\pgfsetlinewidth{1.003750pt}%
\definecolor{currentstroke}{rgb}{0.000000,0.000000,1.000000}%
\pgfsetstrokecolor{currentstroke}%
\pgfsetstrokeopacity{0.700000}%
\pgfsetdash{}{0pt}%
\pgfpathmoveto{\pgfqpoint{1.732141in}{2.189687in}}%
\pgfpathcurveto{\pgfqpoint{1.737965in}{2.189687in}}{\pgfqpoint{1.743551in}{2.192001in}}{\pgfqpoint{1.747669in}{2.196119in}}%
\pgfpathcurveto{\pgfqpoint{1.751787in}{2.200237in}}{\pgfqpoint{1.754101in}{2.205823in}}{\pgfqpoint{1.754101in}{2.211647in}}%
\pgfpathcurveto{\pgfqpoint{1.754101in}{2.217471in}}{\pgfqpoint{1.751787in}{2.223057in}}{\pgfqpoint{1.747669in}{2.227175in}}%
\pgfpathcurveto{\pgfqpoint{1.743551in}{2.231293in}}{\pgfqpoint{1.737965in}{2.233607in}}{\pgfqpoint{1.732141in}{2.233607in}}%
\pgfpathcurveto{\pgfqpoint{1.726317in}{2.233607in}}{\pgfqpoint{1.720731in}{2.231293in}}{\pgfqpoint{1.716613in}{2.227175in}}%
\pgfpathcurveto{\pgfqpoint{1.712495in}{2.223057in}}{\pgfqpoint{1.710181in}{2.217471in}}{\pgfqpoint{1.710181in}{2.211647in}}%
\pgfpathcurveto{\pgfqpoint{1.710181in}{2.205823in}}{\pgfqpoint{1.712495in}{2.200237in}}{\pgfqpoint{1.716613in}{2.196119in}}%
\pgfpathcurveto{\pgfqpoint{1.720731in}{2.192001in}}{\pgfqpoint{1.726317in}{2.189687in}}{\pgfqpoint{1.732141in}{2.189687in}}%
\pgfpathlineto{\pgfqpoint{1.732141in}{2.189687in}}%
\pgfpathclose%
\pgfusepath{stroke,fill}%
\end{pgfscope}%
\begin{pgfscope}%
\pgfpathrectangle{\pgfqpoint{0.100000in}{0.183744in}}{\pgfqpoint{4.506048in}{4.506048in}}%
\pgfusepath{clip}%
\pgfsetbuttcap%
\pgfsetroundjoin%
\definecolor{currentfill}{rgb}{0.000000,0.000000,1.000000}%
\pgfsetfillcolor{currentfill}%
\pgfsetfillopacity{0.700000}%
\pgfsetlinewidth{1.003750pt}%
\definecolor{currentstroke}{rgb}{0.000000,0.000000,1.000000}%
\pgfsetstrokecolor{currentstroke}%
\pgfsetstrokeopacity{0.700000}%
\pgfsetdash{}{0pt}%
\pgfpathmoveto{\pgfqpoint{2.914397in}{3.090071in}}%
\pgfpathcurveto{\pgfqpoint{2.920221in}{3.090071in}}{\pgfqpoint{2.925807in}{3.092385in}}{\pgfqpoint{2.929925in}{3.096503in}}%
\pgfpathcurveto{\pgfqpoint{2.934043in}{3.100621in}}{\pgfqpoint{2.936357in}{3.106207in}}{\pgfqpoint{2.936357in}{3.112031in}}%
\pgfpathcurveto{\pgfqpoint{2.936357in}{3.117855in}}{\pgfqpoint{2.934043in}{3.123441in}}{\pgfqpoint{2.929925in}{3.127560in}}%
\pgfpathcurveto{\pgfqpoint{2.925807in}{3.131678in}}{\pgfqpoint{2.920221in}{3.133992in}}{\pgfqpoint{2.914397in}{3.133992in}}%
\pgfpathcurveto{\pgfqpoint{2.908573in}{3.133992in}}{\pgfqpoint{2.902987in}{3.131678in}}{\pgfqpoint{2.898869in}{3.127560in}}%
\pgfpathcurveto{\pgfqpoint{2.894751in}{3.123441in}}{\pgfqpoint{2.892437in}{3.117855in}}{\pgfqpoint{2.892437in}{3.112031in}}%
\pgfpathcurveto{\pgfqpoint{2.892437in}{3.106207in}}{\pgfqpoint{2.894751in}{3.100621in}}{\pgfqpoint{2.898869in}{3.096503in}}%
\pgfpathcurveto{\pgfqpoint{2.902987in}{3.092385in}}{\pgfqpoint{2.908573in}{3.090071in}}{\pgfqpoint{2.914397in}{3.090071in}}%
\pgfpathlineto{\pgfqpoint{2.914397in}{3.090071in}}%
\pgfpathclose%
\pgfusepath{stroke,fill}%
\end{pgfscope}%
\begin{pgfscope}%
\pgfpathrectangle{\pgfqpoint{0.100000in}{0.183744in}}{\pgfqpoint{4.506048in}{4.506048in}}%
\pgfusepath{clip}%
\pgfsetbuttcap%
\pgfsetroundjoin%
\definecolor{currentfill}{rgb}{0.000000,0.000000,1.000000}%
\pgfsetfillcolor{currentfill}%
\pgfsetfillopacity{0.700000}%
\pgfsetlinewidth{1.003750pt}%
\definecolor{currentstroke}{rgb}{0.000000,0.000000,1.000000}%
\pgfsetstrokecolor{currentstroke}%
\pgfsetstrokeopacity{0.700000}%
\pgfsetdash{}{0pt}%
\pgfpathmoveto{\pgfqpoint{2.002503in}{1.246595in}}%
\pgfpathcurveto{\pgfqpoint{2.008327in}{1.246595in}}{\pgfqpoint{2.013913in}{1.248909in}}{\pgfqpoint{2.018031in}{1.253027in}}%
\pgfpathcurveto{\pgfqpoint{2.022150in}{1.257146in}}{\pgfqpoint{2.024463in}{1.262732in}}{\pgfqpoint{2.024463in}{1.268556in}}%
\pgfpathcurveto{\pgfqpoint{2.024463in}{1.274380in}}{\pgfqpoint{2.022150in}{1.279966in}}{\pgfqpoint{2.018031in}{1.284084in}}%
\pgfpathcurveto{\pgfqpoint{2.013913in}{1.288202in}}{\pgfqpoint{2.008327in}{1.290516in}}{\pgfqpoint{2.002503in}{1.290516in}}%
\pgfpathcurveto{\pgfqpoint{1.996679in}{1.290516in}}{\pgfqpoint{1.991093in}{1.288202in}}{\pgfqpoint{1.986975in}{1.284084in}}%
\pgfpathcurveto{\pgfqpoint{1.982857in}{1.279966in}}{\pgfqpoint{1.980543in}{1.274380in}}{\pgfqpoint{1.980543in}{1.268556in}}%
\pgfpathcurveto{\pgfqpoint{1.980543in}{1.262732in}}{\pgfqpoint{1.982857in}{1.257146in}}{\pgfqpoint{1.986975in}{1.253027in}}%
\pgfpathcurveto{\pgfqpoint{1.991093in}{1.248909in}}{\pgfqpoint{1.996679in}{1.246595in}}{\pgfqpoint{2.002503in}{1.246595in}}%
\pgfpathlineto{\pgfqpoint{2.002503in}{1.246595in}}%
\pgfpathclose%
\pgfusepath{stroke,fill}%
\end{pgfscope}%
\begin{pgfscope}%
\pgfpathrectangle{\pgfqpoint{0.100000in}{0.183744in}}{\pgfqpoint{4.506048in}{4.506048in}}%
\pgfusepath{clip}%
\pgfsetbuttcap%
\pgfsetroundjoin%
\definecolor{currentfill}{rgb}{0.000000,0.000000,1.000000}%
\pgfsetfillcolor{currentfill}%
\pgfsetfillopacity{0.700000}%
\pgfsetlinewidth{1.003750pt}%
\definecolor{currentstroke}{rgb}{0.000000,0.000000,1.000000}%
\pgfsetstrokecolor{currentstroke}%
\pgfsetstrokeopacity{0.700000}%
\pgfsetdash{}{0pt}%
\pgfpathmoveto{\pgfqpoint{1.217168in}{2.822059in}}%
\pgfpathcurveto{\pgfqpoint{1.222992in}{2.822059in}}{\pgfqpoint{1.228578in}{2.824373in}}{\pgfqpoint{1.232697in}{2.828491in}}%
\pgfpathcurveto{\pgfqpoint{1.236815in}{2.832609in}}{\pgfqpoint{1.239129in}{2.838195in}}{\pgfqpoint{1.239129in}{2.844019in}}%
\pgfpathcurveto{\pgfqpoint{1.239129in}{2.849843in}}{\pgfqpoint{1.236815in}{2.855429in}}{\pgfqpoint{1.232697in}{2.859547in}}%
\pgfpathcurveto{\pgfqpoint{1.228578in}{2.863666in}}{\pgfqpoint{1.222992in}{2.865979in}}{\pgfqpoint{1.217168in}{2.865979in}}%
\pgfpathcurveto{\pgfqpoint{1.211344in}{2.865979in}}{\pgfqpoint{1.205758in}{2.863666in}}{\pgfqpoint{1.201640in}{2.859547in}}%
\pgfpathcurveto{\pgfqpoint{1.197522in}{2.855429in}}{\pgfqpoint{1.195208in}{2.849843in}}{\pgfqpoint{1.195208in}{2.844019in}}%
\pgfpathcurveto{\pgfqpoint{1.195208in}{2.838195in}}{\pgfqpoint{1.197522in}{2.832609in}}{\pgfqpoint{1.201640in}{2.828491in}}%
\pgfpathcurveto{\pgfqpoint{1.205758in}{2.824373in}}{\pgfqpoint{1.211344in}{2.822059in}}{\pgfqpoint{1.217168in}{2.822059in}}%
\pgfpathlineto{\pgfqpoint{1.217168in}{2.822059in}}%
\pgfpathclose%
\pgfusepath{stroke,fill}%
\end{pgfscope}%
\begin{pgfscope}%
\pgfpathrectangle{\pgfqpoint{0.100000in}{0.183744in}}{\pgfqpoint{4.506048in}{4.506048in}}%
\pgfusepath{clip}%
\pgfsetbuttcap%
\pgfsetroundjoin%
\definecolor{currentfill}{rgb}{0.000000,0.000000,1.000000}%
\pgfsetfillcolor{currentfill}%
\pgfsetfillopacity{0.700000}%
\pgfsetlinewidth{1.003750pt}%
\definecolor{currentstroke}{rgb}{0.000000,0.000000,1.000000}%
\pgfsetstrokecolor{currentstroke}%
\pgfsetstrokeopacity{0.700000}%
\pgfsetdash{}{0pt}%
\pgfpathmoveto{\pgfqpoint{3.027764in}{2.459196in}}%
\pgfpathcurveto{\pgfqpoint{3.033588in}{2.459196in}}{\pgfqpoint{3.039175in}{2.461510in}}{\pgfqpoint{3.043293in}{2.465628in}}%
\pgfpathcurveto{\pgfqpoint{3.047411in}{2.469746in}}{\pgfqpoint{3.049725in}{2.475332in}}{\pgfqpoint{3.049725in}{2.481156in}}%
\pgfpathcurveto{\pgfqpoint{3.049725in}{2.486980in}}{\pgfqpoint{3.047411in}{2.492566in}}{\pgfqpoint{3.043293in}{2.496684in}}%
\pgfpathcurveto{\pgfqpoint{3.039175in}{2.500803in}}{\pgfqpoint{3.033588in}{2.503116in}}{\pgfqpoint{3.027764in}{2.503116in}}%
\pgfpathcurveto{\pgfqpoint{3.021941in}{2.503116in}}{\pgfqpoint{3.016354in}{2.500803in}}{\pgfqpoint{3.012236in}{2.496684in}}%
\pgfpathcurveto{\pgfqpoint{3.008118in}{2.492566in}}{\pgfqpoint{3.005804in}{2.486980in}}{\pgfqpoint{3.005804in}{2.481156in}}%
\pgfpathcurveto{\pgfqpoint{3.005804in}{2.475332in}}{\pgfqpoint{3.008118in}{2.469746in}}{\pgfqpoint{3.012236in}{2.465628in}}%
\pgfpathcurveto{\pgfqpoint{3.016354in}{2.461510in}}{\pgfqpoint{3.021941in}{2.459196in}}{\pgfqpoint{3.027764in}{2.459196in}}%
\pgfpathlineto{\pgfqpoint{3.027764in}{2.459196in}}%
\pgfpathclose%
\pgfusepath{stroke,fill}%
\end{pgfscope}%
\begin{pgfscope}%
\pgfpathrectangle{\pgfqpoint{0.100000in}{0.183744in}}{\pgfqpoint{4.506048in}{4.506048in}}%
\pgfusepath{clip}%
\pgfsetbuttcap%
\pgfsetroundjoin%
\definecolor{currentfill}{rgb}{0.000000,0.000000,1.000000}%
\pgfsetfillcolor{currentfill}%
\pgfsetfillopacity{0.700000}%
\pgfsetlinewidth{1.003750pt}%
\definecolor{currentstroke}{rgb}{0.000000,0.000000,1.000000}%
\pgfsetstrokecolor{currentstroke}%
\pgfsetstrokeopacity{0.700000}%
\pgfsetdash{}{0pt}%
\pgfpathmoveto{\pgfqpoint{0.909862in}{2.303280in}}%
\pgfpathcurveto{\pgfqpoint{0.915686in}{2.303280in}}{\pgfqpoint{0.921272in}{2.305593in}}{\pgfqpoint{0.925390in}{2.309712in}}%
\pgfpathcurveto{\pgfqpoint{0.929509in}{2.313830in}}{\pgfqpoint{0.931822in}{2.319416in}}{\pgfqpoint{0.931822in}{2.325240in}}%
\pgfpathcurveto{\pgfqpoint{0.931822in}{2.331064in}}{\pgfqpoint{0.929509in}{2.336650in}}{\pgfqpoint{0.925390in}{2.340768in}}%
\pgfpathcurveto{\pgfqpoint{0.921272in}{2.344886in}}{\pgfqpoint{0.915686in}{2.347200in}}{\pgfqpoint{0.909862in}{2.347200in}}%
\pgfpathcurveto{\pgfqpoint{0.904038in}{2.347200in}}{\pgfqpoint{0.898452in}{2.344886in}}{\pgfqpoint{0.894334in}{2.340768in}}%
\pgfpathcurveto{\pgfqpoint{0.890216in}{2.336650in}}{\pgfqpoint{0.887902in}{2.331064in}}{\pgfqpoint{0.887902in}{2.325240in}}%
\pgfpathcurveto{\pgfqpoint{0.887902in}{2.319416in}}{\pgfqpoint{0.890216in}{2.313830in}}{\pgfqpoint{0.894334in}{2.309712in}}%
\pgfpathcurveto{\pgfqpoint{0.898452in}{2.305593in}}{\pgfqpoint{0.904038in}{2.303280in}}{\pgfqpoint{0.909862in}{2.303280in}}%
\pgfpathlineto{\pgfqpoint{0.909862in}{2.303280in}}%
\pgfpathclose%
\pgfusepath{stroke,fill}%
\end{pgfscope}%
\begin{pgfscope}%
\pgfpathrectangle{\pgfqpoint{0.100000in}{0.183744in}}{\pgfqpoint{4.506048in}{4.506048in}}%
\pgfusepath{clip}%
\pgfsetbuttcap%
\pgfsetroundjoin%
\definecolor{currentfill}{rgb}{0.000000,0.000000,1.000000}%
\pgfsetfillcolor{currentfill}%
\pgfsetfillopacity{0.700000}%
\pgfsetlinewidth{1.003750pt}%
\definecolor{currentstroke}{rgb}{0.000000,0.000000,1.000000}%
\pgfsetstrokecolor{currentstroke}%
\pgfsetstrokeopacity{0.700000}%
\pgfsetdash{}{0pt}%
\pgfpathmoveto{\pgfqpoint{3.434358in}{1.392368in}}%
\pgfpathcurveto{\pgfqpoint{3.440182in}{1.392368in}}{\pgfqpoint{3.445768in}{1.394681in}}{\pgfqpoint{3.449886in}{1.398800in}}%
\pgfpathcurveto{\pgfqpoint{3.454004in}{1.402918in}}{\pgfqpoint{3.456318in}{1.408504in}}{\pgfqpoint{3.456318in}{1.414328in}}%
\pgfpathcurveto{\pgfqpoint{3.456318in}{1.420152in}}{\pgfqpoint{3.454004in}{1.425738in}}{\pgfqpoint{3.449886in}{1.429856in}}%
\pgfpathcurveto{\pgfqpoint{3.445768in}{1.433974in}}{\pgfqpoint{3.440182in}{1.436288in}}{\pgfqpoint{3.434358in}{1.436288in}}%
\pgfpathcurveto{\pgfqpoint{3.428534in}{1.436288in}}{\pgfqpoint{3.422947in}{1.433974in}}{\pgfqpoint{3.418829in}{1.429856in}}%
\pgfpathcurveto{\pgfqpoint{3.414711in}{1.425738in}}{\pgfqpoint{3.412397in}{1.420152in}}{\pgfqpoint{3.412397in}{1.414328in}}%
\pgfpathcurveto{\pgfqpoint{3.412397in}{1.408504in}}{\pgfqpoint{3.414711in}{1.402918in}}{\pgfqpoint{3.418829in}{1.398800in}}%
\pgfpathcurveto{\pgfqpoint{3.422947in}{1.394681in}}{\pgfqpoint{3.428534in}{1.392368in}}{\pgfqpoint{3.434358in}{1.392368in}}%
\pgfpathlineto{\pgfqpoint{3.434358in}{1.392368in}}%
\pgfpathclose%
\pgfusepath{stroke,fill}%
\end{pgfscope}%
\begin{pgfscope}%
\pgfpathrectangle{\pgfqpoint{0.100000in}{0.183744in}}{\pgfqpoint{4.506048in}{4.506048in}}%
\pgfusepath{clip}%
\pgfsetbuttcap%
\pgfsetroundjoin%
\definecolor{currentfill}{rgb}{0.000000,0.000000,1.000000}%
\pgfsetfillcolor{currentfill}%
\pgfsetfillopacity{0.700000}%
\pgfsetlinewidth{1.003750pt}%
\definecolor{currentstroke}{rgb}{0.000000,0.000000,1.000000}%
\pgfsetstrokecolor{currentstroke}%
\pgfsetstrokeopacity{0.700000}%
\pgfsetdash{}{0pt}%
\pgfpathmoveto{\pgfqpoint{3.314814in}{1.316513in}}%
\pgfpathcurveto{\pgfqpoint{3.320638in}{1.316513in}}{\pgfqpoint{3.326224in}{1.318827in}}{\pgfqpoint{3.330342in}{1.322945in}}%
\pgfpathcurveto{\pgfqpoint{3.334460in}{1.327063in}}{\pgfqpoint{3.336774in}{1.332649in}}{\pgfqpoint{3.336774in}{1.338473in}}%
\pgfpathcurveto{\pgfqpoint{3.336774in}{1.344297in}}{\pgfqpoint{3.334460in}{1.349883in}}{\pgfqpoint{3.330342in}{1.354001in}}%
\pgfpathcurveto{\pgfqpoint{3.326224in}{1.358120in}}{\pgfqpoint{3.320638in}{1.360433in}}{\pgfqpoint{3.314814in}{1.360433in}}%
\pgfpathcurveto{\pgfqpoint{3.308990in}{1.360433in}}{\pgfqpoint{3.303404in}{1.358120in}}{\pgfqpoint{3.299286in}{1.354001in}}%
\pgfpathcurveto{\pgfqpoint{3.295167in}{1.349883in}}{\pgfqpoint{3.292854in}{1.344297in}}{\pgfqpoint{3.292854in}{1.338473in}}%
\pgfpathcurveto{\pgfqpoint{3.292854in}{1.332649in}}{\pgfqpoint{3.295167in}{1.327063in}}{\pgfqpoint{3.299286in}{1.322945in}}%
\pgfpathcurveto{\pgfqpoint{3.303404in}{1.318827in}}{\pgfqpoint{3.308990in}{1.316513in}}{\pgfqpoint{3.314814in}{1.316513in}}%
\pgfpathlineto{\pgfqpoint{3.314814in}{1.316513in}}%
\pgfpathclose%
\pgfusepath{stroke,fill}%
\end{pgfscope}%
\begin{pgfscope}%
\pgfpathrectangle{\pgfqpoint{0.100000in}{0.183744in}}{\pgfqpoint{4.506048in}{4.506048in}}%
\pgfusepath{clip}%
\pgfsetbuttcap%
\pgfsetroundjoin%
\definecolor{currentfill}{rgb}{0.000000,0.000000,1.000000}%
\pgfsetfillcolor{currentfill}%
\pgfsetfillopacity{0.700000}%
\pgfsetlinewidth{1.003750pt}%
\definecolor{currentstroke}{rgb}{0.000000,0.000000,1.000000}%
\pgfsetstrokecolor{currentstroke}%
\pgfsetstrokeopacity{0.700000}%
\pgfsetdash{}{0pt}%
\pgfpathmoveto{\pgfqpoint{3.535746in}{2.742284in}}%
\pgfpathcurveto{\pgfqpoint{3.541570in}{2.742284in}}{\pgfqpoint{3.547157in}{2.744598in}}{\pgfqpoint{3.551275in}{2.748716in}}%
\pgfpathcurveto{\pgfqpoint{3.555393in}{2.752834in}}{\pgfqpoint{3.557707in}{2.758420in}}{\pgfqpoint{3.557707in}{2.764244in}}%
\pgfpathcurveto{\pgfqpoint{3.557707in}{2.770068in}}{\pgfqpoint{3.555393in}{2.775654in}}{\pgfqpoint{3.551275in}{2.779773in}}%
\pgfpathcurveto{\pgfqpoint{3.547157in}{2.783891in}}{\pgfqpoint{3.541570in}{2.786205in}}{\pgfqpoint{3.535746in}{2.786205in}}%
\pgfpathcurveto{\pgfqpoint{3.529923in}{2.786205in}}{\pgfqpoint{3.524336in}{2.783891in}}{\pgfqpoint{3.520218in}{2.779773in}}%
\pgfpathcurveto{\pgfqpoint{3.516100in}{2.775654in}}{\pgfqpoint{3.513786in}{2.770068in}}{\pgfqpoint{3.513786in}{2.764244in}}%
\pgfpathcurveto{\pgfqpoint{3.513786in}{2.758420in}}{\pgfqpoint{3.516100in}{2.752834in}}{\pgfqpoint{3.520218in}{2.748716in}}%
\pgfpathcurveto{\pgfqpoint{3.524336in}{2.744598in}}{\pgfqpoint{3.529923in}{2.742284in}}{\pgfqpoint{3.535746in}{2.742284in}}%
\pgfpathlineto{\pgfqpoint{3.535746in}{2.742284in}}%
\pgfpathclose%
\pgfusepath{stroke,fill}%
\end{pgfscope}%
\begin{pgfscope}%
\pgfpathrectangle{\pgfqpoint{0.100000in}{0.183744in}}{\pgfqpoint{4.506048in}{4.506048in}}%
\pgfusepath{clip}%
\pgfsetbuttcap%
\pgfsetroundjoin%
\definecolor{currentfill}{rgb}{0.000000,0.000000,1.000000}%
\pgfsetfillcolor{currentfill}%
\pgfsetfillopacity{0.700000}%
\pgfsetlinewidth{1.003750pt}%
\definecolor{currentstroke}{rgb}{0.000000,0.000000,1.000000}%
\pgfsetstrokecolor{currentstroke}%
\pgfsetstrokeopacity{0.700000}%
\pgfsetdash{}{0pt}%
\pgfpathmoveto{\pgfqpoint{2.968714in}{2.561414in}}%
\pgfpathcurveto{\pgfqpoint{2.974538in}{2.561414in}}{\pgfqpoint{2.980125in}{2.563728in}}{\pgfqpoint{2.984243in}{2.567846in}}%
\pgfpathcurveto{\pgfqpoint{2.988361in}{2.571964in}}{\pgfqpoint{2.990675in}{2.577550in}}{\pgfqpoint{2.990675in}{2.583374in}}%
\pgfpathcurveto{\pgfqpoint{2.990675in}{2.589198in}}{\pgfqpoint{2.988361in}{2.594784in}}{\pgfqpoint{2.984243in}{2.598902in}}%
\pgfpathcurveto{\pgfqpoint{2.980125in}{2.603021in}}{\pgfqpoint{2.974538in}{2.605334in}}{\pgfqpoint{2.968714in}{2.605334in}}%
\pgfpathcurveto{\pgfqpoint{2.962891in}{2.605334in}}{\pgfqpoint{2.957304in}{2.603021in}}{\pgfqpoint{2.953186in}{2.598902in}}%
\pgfpathcurveto{\pgfqpoint{2.949068in}{2.594784in}}{\pgfqpoint{2.946754in}{2.589198in}}{\pgfqpoint{2.946754in}{2.583374in}}%
\pgfpathcurveto{\pgfqpoint{2.946754in}{2.577550in}}{\pgfqpoint{2.949068in}{2.571964in}}{\pgfqpoint{2.953186in}{2.567846in}}%
\pgfpathcurveto{\pgfqpoint{2.957304in}{2.563728in}}{\pgfqpoint{2.962891in}{2.561414in}}{\pgfqpoint{2.968714in}{2.561414in}}%
\pgfpathlineto{\pgfqpoint{2.968714in}{2.561414in}}%
\pgfpathclose%
\pgfusepath{stroke,fill}%
\end{pgfscope}%
\begin{pgfscope}%
\pgfpathrectangle{\pgfqpoint{0.100000in}{0.183744in}}{\pgfqpoint{4.506048in}{4.506048in}}%
\pgfusepath{clip}%
\pgfsetbuttcap%
\pgfsetroundjoin%
\definecolor{currentfill}{rgb}{0.000000,0.000000,1.000000}%
\pgfsetfillcolor{currentfill}%
\pgfsetfillopacity{0.700000}%
\pgfsetlinewidth{1.003750pt}%
\definecolor{currentstroke}{rgb}{0.000000,0.000000,1.000000}%
\pgfsetstrokecolor{currentstroke}%
\pgfsetstrokeopacity{0.700000}%
\pgfsetdash{}{0pt}%
\pgfpathmoveto{\pgfqpoint{2.572156in}{3.680024in}}%
\pgfpathcurveto{\pgfqpoint{2.577980in}{3.680024in}}{\pgfqpoint{2.583566in}{3.682338in}}{\pgfqpoint{2.587684in}{3.686456in}}%
\pgfpathcurveto{\pgfqpoint{2.591802in}{3.690574in}}{\pgfqpoint{2.594116in}{3.696160in}}{\pgfqpoint{2.594116in}{3.701984in}}%
\pgfpathcurveto{\pgfqpoint{2.594116in}{3.707808in}}{\pgfqpoint{2.591802in}{3.713394in}}{\pgfqpoint{2.587684in}{3.717512in}}%
\pgfpathcurveto{\pgfqpoint{2.583566in}{3.721631in}}{\pgfqpoint{2.577980in}{3.723944in}}{\pgfqpoint{2.572156in}{3.723944in}}%
\pgfpathcurveto{\pgfqpoint{2.566332in}{3.723944in}}{\pgfqpoint{2.560746in}{3.721631in}}{\pgfqpoint{2.556628in}{3.717512in}}%
\pgfpathcurveto{\pgfqpoint{2.552510in}{3.713394in}}{\pgfqpoint{2.550196in}{3.707808in}}{\pgfqpoint{2.550196in}{3.701984in}}%
\pgfpathcurveto{\pgfqpoint{2.550196in}{3.696160in}}{\pgfqpoint{2.552510in}{3.690574in}}{\pgfqpoint{2.556628in}{3.686456in}}%
\pgfpathcurveto{\pgfqpoint{2.560746in}{3.682338in}}{\pgfqpoint{2.566332in}{3.680024in}}{\pgfqpoint{2.572156in}{3.680024in}}%
\pgfpathlineto{\pgfqpoint{2.572156in}{3.680024in}}%
\pgfpathclose%
\pgfusepath{stroke,fill}%
\end{pgfscope}%
\begin{pgfscope}%
\pgfpathrectangle{\pgfqpoint{0.100000in}{0.183744in}}{\pgfqpoint{4.506048in}{4.506048in}}%
\pgfusepath{clip}%
\pgfsetbuttcap%
\pgfsetroundjoin%
\definecolor{currentfill}{rgb}{0.000000,0.000000,1.000000}%
\pgfsetfillcolor{currentfill}%
\pgfsetfillopacity{0.700000}%
\pgfsetlinewidth{1.003750pt}%
\definecolor{currentstroke}{rgb}{0.000000,0.000000,1.000000}%
\pgfsetstrokecolor{currentstroke}%
\pgfsetstrokeopacity{0.700000}%
\pgfsetdash{}{0pt}%
\pgfpathmoveto{\pgfqpoint{3.600028in}{1.141048in}}%
\pgfpathcurveto{\pgfqpoint{3.605851in}{1.141048in}}{\pgfqpoint{3.611438in}{1.143362in}}{\pgfqpoint{3.615556in}{1.147480in}}%
\pgfpathcurveto{\pgfqpoint{3.619674in}{1.151598in}}{\pgfqpoint{3.621988in}{1.157184in}}{\pgfqpoint{3.621988in}{1.163008in}}%
\pgfpathcurveto{\pgfqpoint{3.621988in}{1.168832in}}{\pgfqpoint{3.619674in}{1.174418in}}{\pgfqpoint{3.615556in}{1.178536in}}%
\pgfpathcurveto{\pgfqpoint{3.611438in}{1.182654in}}{\pgfqpoint{3.605851in}{1.184968in}}{\pgfqpoint{3.600028in}{1.184968in}}%
\pgfpathcurveto{\pgfqpoint{3.594204in}{1.184968in}}{\pgfqpoint{3.588617in}{1.182654in}}{\pgfqpoint{3.584499in}{1.178536in}}%
\pgfpathcurveto{\pgfqpoint{3.580381in}{1.174418in}}{\pgfqpoint{3.578067in}{1.168832in}}{\pgfqpoint{3.578067in}{1.163008in}}%
\pgfpathcurveto{\pgfqpoint{3.578067in}{1.157184in}}{\pgfqpoint{3.580381in}{1.151598in}}{\pgfqpoint{3.584499in}{1.147480in}}%
\pgfpathcurveto{\pgfqpoint{3.588617in}{1.143362in}}{\pgfqpoint{3.594204in}{1.141048in}}{\pgfqpoint{3.600028in}{1.141048in}}%
\pgfpathlineto{\pgfqpoint{3.600028in}{1.141048in}}%
\pgfpathclose%
\pgfusepath{stroke,fill}%
\end{pgfscope}%
\begin{pgfscope}%
\pgfpathrectangle{\pgfqpoint{0.100000in}{0.183744in}}{\pgfqpoint{4.506048in}{4.506048in}}%
\pgfusepath{clip}%
\pgfsetbuttcap%
\pgfsetroundjoin%
\definecolor{currentfill}{rgb}{0.000000,0.000000,1.000000}%
\pgfsetfillcolor{currentfill}%
\pgfsetfillopacity{0.700000}%
\pgfsetlinewidth{1.003750pt}%
\definecolor{currentstroke}{rgb}{0.000000,0.000000,1.000000}%
\pgfsetstrokecolor{currentstroke}%
\pgfsetstrokeopacity{0.700000}%
\pgfsetdash{}{0pt}%
\pgfpathmoveto{\pgfqpoint{2.356521in}{1.472699in}}%
\pgfpathcurveto{\pgfqpoint{2.362345in}{1.472699in}}{\pgfqpoint{2.367931in}{1.475013in}}{\pgfqpoint{2.372049in}{1.479131in}}%
\pgfpathcurveto{\pgfqpoint{2.376167in}{1.483249in}}{\pgfqpoint{2.378481in}{1.488835in}}{\pgfqpoint{2.378481in}{1.494659in}}%
\pgfpathcurveto{\pgfqpoint{2.378481in}{1.500483in}}{\pgfqpoint{2.376167in}{1.506069in}}{\pgfqpoint{2.372049in}{1.510188in}}%
\pgfpathcurveto{\pgfqpoint{2.367931in}{1.514306in}}{\pgfqpoint{2.362345in}{1.516620in}}{\pgfqpoint{2.356521in}{1.516620in}}%
\pgfpathcurveto{\pgfqpoint{2.350697in}{1.516620in}}{\pgfqpoint{2.345111in}{1.514306in}}{\pgfqpoint{2.340993in}{1.510188in}}%
\pgfpathcurveto{\pgfqpoint{2.336875in}{1.506069in}}{\pgfqpoint{2.334561in}{1.500483in}}{\pgfqpoint{2.334561in}{1.494659in}}%
\pgfpathcurveto{\pgfqpoint{2.334561in}{1.488835in}}{\pgfqpoint{2.336875in}{1.483249in}}{\pgfqpoint{2.340993in}{1.479131in}}%
\pgfpathcurveto{\pgfqpoint{2.345111in}{1.475013in}}{\pgfqpoint{2.350697in}{1.472699in}}{\pgfqpoint{2.356521in}{1.472699in}}%
\pgfpathlineto{\pgfqpoint{2.356521in}{1.472699in}}%
\pgfpathclose%
\pgfusepath{stroke,fill}%
\end{pgfscope}%
\begin{pgfscope}%
\pgfpathrectangle{\pgfqpoint{0.100000in}{0.183744in}}{\pgfqpoint{4.506048in}{4.506048in}}%
\pgfusepath{clip}%
\pgfsetbuttcap%
\pgfsetroundjoin%
\definecolor{currentfill}{rgb}{0.000000,0.000000,1.000000}%
\pgfsetfillcolor{currentfill}%
\pgfsetfillopacity{0.700000}%
\pgfsetlinewidth{1.003750pt}%
\definecolor{currentstroke}{rgb}{0.000000,0.000000,1.000000}%
\pgfsetstrokecolor{currentstroke}%
\pgfsetstrokeopacity{0.700000}%
\pgfsetdash{}{0pt}%
\pgfpathmoveto{\pgfqpoint{1.784653in}{3.802709in}}%
\pgfpathcurveto{\pgfqpoint{1.790477in}{3.802709in}}{\pgfqpoint{1.796063in}{3.805022in}}{\pgfqpoint{1.800181in}{3.809141in}}%
\pgfpathcurveto{\pgfqpoint{1.804299in}{3.813259in}}{\pgfqpoint{1.806613in}{3.818845in}}{\pgfqpoint{1.806613in}{3.824669in}}%
\pgfpathcurveto{\pgfqpoint{1.806613in}{3.830493in}}{\pgfqpoint{1.804299in}{3.836079in}}{\pgfqpoint{1.800181in}{3.840197in}}%
\pgfpathcurveto{\pgfqpoint{1.796063in}{3.844315in}}{\pgfqpoint{1.790477in}{3.846629in}}{\pgfqpoint{1.784653in}{3.846629in}}%
\pgfpathcurveto{\pgfqpoint{1.778829in}{3.846629in}}{\pgfqpoint{1.773243in}{3.844315in}}{\pgfqpoint{1.769125in}{3.840197in}}%
\pgfpathcurveto{\pgfqpoint{1.765007in}{3.836079in}}{\pgfqpoint{1.762693in}{3.830493in}}{\pgfqpoint{1.762693in}{3.824669in}}%
\pgfpathcurveto{\pgfqpoint{1.762693in}{3.818845in}}{\pgfqpoint{1.765007in}{3.813259in}}{\pgfqpoint{1.769125in}{3.809141in}}%
\pgfpathcurveto{\pgfqpoint{1.773243in}{3.805022in}}{\pgfqpoint{1.778829in}{3.802709in}}{\pgfqpoint{1.784653in}{3.802709in}}%
\pgfpathlineto{\pgfqpoint{1.784653in}{3.802709in}}%
\pgfpathclose%
\pgfusepath{stroke,fill}%
\end{pgfscope}%
\begin{pgfscope}%
\pgfpathrectangle{\pgfqpoint{0.100000in}{0.183744in}}{\pgfqpoint{4.506048in}{4.506048in}}%
\pgfusepath{clip}%
\pgfsetbuttcap%
\pgfsetroundjoin%
\definecolor{currentfill}{rgb}{0.000000,0.000000,1.000000}%
\pgfsetfillcolor{currentfill}%
\pgfsetfillopacity{0.700000}%
\pgfsetlinewidth{1.003750pt}%
\definecolor{currentstroke}{rgb}{0.000000,0.000000,1.000000}%
\pgfsetstrokecolor{currentstroke}%
\pgfsetstrokeopacity{0.700000}%
\pgfsetdash{}{0pt}%
\pgfpathmoveto{\pgfqpoint{3.009214in}{2.021288in}}%
\pgfpathcurveto{\pgfqpoint{3.015038in}{2.021288in}}{\pgfqpoint{3.020624in}{2.023602in}}{\pgfqpoint{3.024742in}{2.027720in}}%
\pgfpathcurveto{\pgfqpoint{3.028861in}{2.031838in}}{\pgfqpoint{3.031174in}{2.037424in}}{\pgfqpoint{3.031174in}{2.043248in}}%
\pgfpathcurveto{\pgfqpoint{3.031174in}{2.049072in}}{\pgfqpoint{3.028861in}{2.054658in}}{\pgfqpoint{3.024742in}{2.058777in}}%
\pgfpathcurveto{\pgfqpoint{3.020624in}{2.062895in}}{\pgfqpoint{3.015038in}{2.065209in}}{\pgfqpoint{3.009214in}{2.065209in}}%
\pgfpathcurveto{\pgfqpoint{3.003390in}{2.065209in}}{\pgfqpoint{2.997804in}{2.062895in}}{\pgfqpoint{2.993686in}{2.058777in}}%
\pgfpathcurveto{\pgfqpoint{2.989568in}{2.054658in}}{\pgfqpoint{2.987254in}{2.049072in}}{\pgfqpoint{2.987254in}{2.043248in}}%
\pgfpathcurveto{\pgfqpoint{2.987254in}{2.037424in}}{\pgfqpoint{2.989568in}{2.031838in}}{\pgfqpoint{2.993686in}{2.027720in}}%
\pgfpathcurveto{\pgfqpoint{2.997804in}{2.023602in}}{\pgfqpoint{3.003390in}{2.021288in}}{\pgfqpoint{3.009214in}{2.021288in}}%
\pgfpathlineto{\pgfqpoint{3.009214in}{2.021288in}}%
\pgfpathclose%
\pgfusepath{stroke,fill}%
\end{pgfscope}%
\begin{pgfscope}%
\pgfpathrectangle{\pgfqpoint{0.100000in}{0.183744in}}{\pgfqpoint{4.506048in}{4.506048in}}%
\pgfusepath{clip}%
\pgfsetbuttcap%
\pgfsetroundjoin%
\definecolor{currentfill}{rgb}{0.000000,0.000000,1.000000}%
\pgfsetfillcolor{currentfill}%
\pgfsetfillopacity{0.700000}%
\pgfsetlinewidth{1.003750pt}%
\definecolor{currentstroke}{rgb}{0.000000,0.000000,1.000000}%
\pgfsetstrokecolor{currentstroke}%
\pgfsetstrokeopacity{0.700000}%
\pgfsetdash{}{0pt}%
\pgfpathmoveto{\pgfqpoint{2.835825in}{2.042122in}}%
\pgfpathcurveto{\pgfqpoint{2.841649in}{2.042122in}}{\pgfqpoint{2.847235in}{2.044435in}}{\pgfqpoint{2.851353in}{2.048554in}}%
\pgfpathcurveto{\pgfqpoint{2.855471in}{2.052672in}}{\pgfqpoint{2.857785in}{2.058258in}}{\pgfqpoint{2.857785in}{2.064082in}}%
\pgfpathcurveto{\pgfqpoint{2.857785in}{2.069906in}}{\pgfqpoint{2.855471in}{2.075492in}}{\pgfqpoint{2.851353in}{2.079610in}}%
\pgfpathcurveto{\pgfqpoint{2.847235in}{2.083728in}}{\pgfqpoint{2.841649in}{2.086042in}}{\pgfqpoint{2.835825in}{2.086042in}}%
\pgfpathcurveto{\pgfqpoint{2.830001in}{2.086042in}}{\pgfqpoint{2.824415in}{2.083728in}}{\pgfqpoint{2.820297in}{2.079610in}}%
\pgfpathcurveto{\pgfqpoint{2.816179in}{2.075492in}}{\pgfqpoint{2.813865in}{2.069906in}}{\pgfqpoint{2.813865in}{2.064082in}}%
\pgfpathcurveto{\pgfqpoint{2.813865in}{2.058258in}}{\pgfqpoint{2.816179in}{2.052672in}}{\pgfqpoint{2.820297in}{2.048554in}}%
\pgfpathcurveto{\pgfqpoint{2.824415in}{2.044435in}}{\pgfqpoint{2.830001in}{2.042122in}}{\pgfqpoint{2.835825in}{2.042122in}}%
\pgfpathlineto{\pgfqpoint{2.835825in}{2.042122in}}%
\pgfpathclose%
\pgfusepath{stroke,fill}%
\end{pgfscope}%
\begin{pgfscope}%
\pgfpathrectangle{\pgfqpoint{0.100000in}{0.183744in}}{\pgfqpoint{4.506048in}{4.506048in}}%
\pgfusepath{clip}%
\pgfsetbuttcap%
\pgfsetroundjoin%
\definecolor{currentfill}{rgb}{0.000000,0.000000,1.000000}%
\pgfsetfillcolor{currentfill}%
\pgfsetfillopacity{0.700000}%
\pgfsetlinewidth{1.003750pt}%
\definecolor{currentstroke}{rgb}{0.000000,0.000000,1.000000}%
\pgfsetstrokecolor{currentstroke}%
\pgfsetstrokeopacity{0.700000}%
\pgfsetdash{}{0pt}%
\pgfpathmoveto{\pgfqpoint{1.063893in}{3.715338in}}%
\pgfpathcurveto{\pgfqpoint{1.069717in}{3.715338in}}{\pgfqpoint{1.075303in}{3.717652in}}{\pgfqpoint{1.079421in}{3.721770in}}%
\pgfpathcurveto{\pgfqpoint{1.083540in}{3.725889in}}{\pgfqpoint{1.085853in}{3.731475in}}{\pgfqpoint{1.085853in}{3.737299in}}%
\pgfpathcurveto{\pgfqpoint{1.085853in}{3.743123in}}{\pgfqpoint{1.083540in}{3.748709in}}{\pgfqpoint{1.079421in}{3.752827in}}%
\pgfpathcurveto{\pgfqpoint{1.075303in}{3.756945in}}{\pgfqpoint{1.069717in}{3.759259in}}{\pgfqpoint{1.063893in}{3.759259in}}%
\pgfpathcurveto{\pgfqpoint{1.058069in}{3.759259in}}{\pgfqpoint{1.052483in}{3.756945in}}{\pgfqpoint{1.048365in}{3.752827in}}%
\pgfpathcurveto{\pgfqpoint{1.044247in}{3.748709in}}{\pgfqpoint{1.041933in}{3.743123in}}{\pgfqpoint{1.041933in}{3.737299in}}%
\pgfpathcurveto{\pgfqpoint{1.041933in}{3.731475in}}{\pgfqpoint{1.044247in}{3.725889in}}{\pgfqpoint{1.048365in}{3.721770in}}%
\pgfpathcurveto{\pgfqpoint{1.052483in}{3.717652in}}{\pgfqpoint{1.058069in}{3.715338in}}{\pgfqpoint{1.063893in}{3.715338in}}%
\pgfpathlineto{\pgfqpoint{1.063893in}{3.715338in}}%
\pgfpathclose%
\pgfusepath{stroke,fill}%
\end{pgfscope}%
\begin{pgfscope}%
\pgfpathrectangle{\pgfqpoint{0.100000in}{0.183744in}}{\pgfqpoint{4.506048in}{4.506048in}}%
\pgfusepath{clip}%
\pgfsetbuttcap%
\pgfsetroundjoin%
\definecolor{currentfill}{rgb}{0.000000,0.000000,1.000000}%
\pgfsetfillcolor{currentfill}%
\pgfsetfillopacity{0.700000}%
\pgfsetlinewidth{1.003750pt}%
\definecolor{currentstroke}{rgb}{0.000000,0.000000,1.000000}%
\pgfsetstrokecolor{currentstroke}%
\pgfsetstrokeopacity{0.700000}%
\pgfsetdash{}{0pt}%
\pgfpathmoveto{\pgfqpoint{1.419721in}{2.496920in}}%
\pgfpathcurveto{\pgfqpoint{1.425545in}{2.496920in}}{\pgfqpoint{1.431131in}{2.499234in}}{\pgfqpoint{1.435250in}{2.503352in}}%
\pgfpathcurveto{\pgfqpoint{1.439368in}{2.507470in}}{\pgfqpoint{1.441682in}{2.513056in}}{\pgfqpoint{1.441682in}{2.518880in}}%
\pgfpathcurveto{\pgfqpoint{1.441682in}{2.524704in}}{\pgfqpoint{1.439368in}{2.530290in}}{\pgfqpoint{1.435250in}{2.534409in}}%
\pgfpathcurveto{\pgfqpoint{1.431131in}{2.538527in}}{\pgfqpoint{1.425545in}{2.540841in}}{\pgfqpoint{1.419721in}{2.540841in}}%
\pgfpathcurveto{\pgfqpoint{1.413897in}{2.540841in}}{\pgfqpoint{1.408311in}{2.538527in}}{\pgfqpoint{1.404193in}{2.534409in}}%
\pgfpathcurveto{\pgfqpoint{1.400075in}{2.530290in}}{\pgfqpoint{1.397761in}{2.524704in}}{\pgfqpoint{1.397761in}{2.518880in}}%
\pgfpathcurveto{\pgfqpoint{1.397761in}{2.513056in}}{\pgfqpoint{1.400075in}{2.507470in}}{\pgfqpoint{1.404193in}{2.503352in}}%
\pgfpathcurveto{\pgfqpoint{1.408311in}{2.499234in}}{\pgfqpoint{1.413897in}{2.496920in}}{\pgfqpoint{1.419721in}{2.496920in}}%
\pgfpathlineto{\pgfqpoint{1.419721in}{2.496920in}}%
\pgfpathclose%
\pgfusepath{stroke,fill}%
\end{pgfscope}%
\begin{pgfscope}%
\pgfpathrectangle{\pgfqpoint{0.100000in}{0.183744in}}{\pgfqpoint{4.506048in}{4.506048in}}%
\pgfusepath{clip}%
\pgfsetbuttcap%
\pgfsetroundjoin%
\definecolor{currentfill}{rgb}{0.000000,0.000000,1.000000}%
\pgfsetfillcolor{currentfill}%
\pgfsetfillopacity{0.700000}%
\pgfsetlinewidth{1.003750pt}%
\definecolor{currentstroke}{rgb}{0.000000,0.000000,1.000000}%
\pgfsetstrokecolor{currentstroke}%
\pgfsetstrokeopacity{0.700000}%
\pgfsetdash{}{0pt}%
\pgfpathmoveto{\pgfqpoint{1.122344in}{1.665501in}}%
\pgfpathcurveto{\pgfqpoint{1.128168in}{1.665501in}}{\pgfqpoint{1.133754in}{1.667815in}}{\pgfqpoint{1.137872in}{1.671933in}}%
\pgfpathcurveto{\pgfqpoint{1.141990in}{1.676052in}}{\pgfqpoint{1.144304in}{1.681638in}}{\pgfqpoint{1.144304in}{1.687462in}}%
\pgfpathcurveto{\pgfqpoint{1.144304in}{1.693286in}}{\pgfqpoint{1.141990in}{1.698872in}}{\pgfqpoint{1.137872in}{1.702990in}}%
\pgfpathcurveto{\pgfqpoint{1.133754in}{1.707108in}}{\pgfqpoint{1.128168in}{1.709422in}}{\pgfqpoint{1.122344in}{1.709422in}}%
\pgfpathcurveto{\pgfqpoint{1.116520in}{1.709422in}}{\pgfqpoint{1.110934in}{1.707108in}}{\pgfqpoint{1.106815in}{1.702990in}}%
\pgfpathcurveto{\pgfqpoint{1.102697in}{1.698872in}}{\pgfqpoint{1.100383in}{1.693286in}}{\pgfqpoint{1.100383in}{1.687462in}}%
\pgfpathcurveto{\pgfqpoint{1.100383in}{1.681638in}}{\pgfqpoint{1.102697in}{1.676052in}}{\pgfqpoint{1.106815in}{1.671933in}}%
\pgfpathcurveto{\pgfqpoint{1.110934in}{1.667815in}}{\pgfqpoint{1.116520in}{1.665501in}}{\pgfqpoint{1.122344in}{1.665501in}}%
\pgfpathlineto{\pgfqpoint{1.122344in}{1.665501in}}%
\pgfpathclose%
\pgfusepath{stroke,fill}%
\end{pgfscope}%
\begin{pgfscope}%
\pgfpathrectangle{\pgfqpoint{0.100000in}{0.183744in}}{\pgfqpoint{4.506048in}{4.506048in}}%
\pgfusepath{clip}%
\pgfsetbuttcap%
\pgfsetroundjoin%
\definecolor{currentfill}{rgb}{0.000000,0.000000,1.000000}%
\pgfsetfillcolor{currentfill}%
\pgfsetfillopacity{0.700000}%
\pgfsetlinewidth{1.003750pt}%
\definecolor{currentstroke}{rgb}{0.000000,0.000000,1.000000}%
\pgfsetstrokecolor{currentstroke}%
\pgfsetstrokeopacity{0.700000}%
\pgfsetdash{}{0pt}%
\pgfpathmoveto{\pgfqpoint{1.113593in}{3.437764in}}%
\pgfpathcurveto{\pgfqpoint{1.119417in}{3.437764in}}{\pgfqpoint{1.125003in}{3.440078in}}{\pgfqpoint{1.129121in}{3.444196in}}%
\pgfpathcurveto{\pgfqpoint{1.133239in}{3.448314in}}{\pgfqpoint{1.135553in}{3.453900in}}{\pgfqpoint{1.135553in}{3.459724in}}%
\pgfpathcurveto{\pgfqpoint{1.135553in}{3.465548in}}{\pgfqpoint{1.133239in}{3.471134in}}{\pgfqpoint{1.129121in}{3.475252in}}%
\pgfpathcurveto{\pgfqpoint{1.125003in}{3.479370in}}{\pgfqpoint{1.119417in}{3.481684in}}{\pgfqpoint{1.113593in}{3.481684in}}%
\pgfpathcurveto{\pgfqpoint{1.107769in}{3.481684in}}{\pgfqpoint{1.102183in}{3.479370in}}{\pgfqpoint{1.098064in}{3.475252in}}%
\pgfpathcurveto{\pgfqpoint{1.093946in}{3.471134in}}{\pgfqpoint{1.091632in}{3.465548in}}{\pgfqpoint{1.091632in}{3.459724in}}%
\pgfpathcurveto{\pgfqpoint{1.091632in}{3.453900in}}{\pgfqpoint{1.093946in}{3.448314in}}{\pgfqpoint{1.098064in}{3.444196in}}%
\pgfpathcurveto{\pgfqpoint{1.102183in}{3.440078in}}{\pgfqpoint{1.107769in}{3.437764in}}{\pgfqpoint{1.113593in}{3.437764in}}%
\pgfpathlineto{\pgfqpoint{1.113593in}{3.437764in}}%
\pgfpathclose%
\pgfusepath{stroke,fill}%
\end{pgfscope}%
\begin{pgfscope}%
\pgfpathrectangle{\pgfqpoint{0.100000in}{0.183744in}}{\pgfqpoint{4.506048in}{4.506048in}}%
\pgfusepath{clip}%
\pgfsetbuttcap%
\pgfsetroundjoin%
\definecolor{currentfill}{rgb}{0.000000,0.000000,1.000000}%
\pgfsetfillcolor{currentfill}%
\pgfsetfillopacity{0.700000}%
\pgfsetlinewidth{1.003750pt}%
\definecolor{currentstroke}{rgb}{0.000000,0.000000,1.000000}%
\pgfsetstrokecolor{currentstroke}%
\pgfsetstrokeopacity{0.700000}%
\pgfsetdash{}{0pt}%
\pgfpathmoveto{\pgfqpoint{1.592524in}{2.388651in}}%
\pgfpathcurveto{\pgfqpoint{1.598348in}{2.388651in}}{\pgfqpoint{1.603934in}{2.390965in}}{\pgfqpoint{1.608052in}{2.395083in}}%
\pgfpathcurveto{\pgfqpoint{1.612170in}{2.399201in}}{\pgfqpoint{1.614484in}{2.404787in}}{\pgfqpoint{1.614484in}{2.410611in}}%
\pgfpathcurveto{\pgfqpoint{1.614484in}{2.416435in}}{\pgfqpoint{1.612170in}{2.422021in}}{\pgfqpoint{1.608052in}{2.426139in}}%
\pgfpathcurveto{\pgfqpoint{1.603934in}{2.430258in}}{\pgfqpoint{1.598348in}{2.432571in}}{\pgfqpoint{1.592524in}{2.432571in}}%
\pgfpathcurveto{\pgfqpoint{1.586700in}{2.432571in}}{\pgfqpoint{1.581114in}{2.430258in}}{\pgfqpoint{1.576996in}{2.426139in}}%
\pgfpathcurveto{\pgfqpoint{1.572878in}{2.422021in}}{\pgfqpoint{1.570564in}{2.416435in}}{\pgfqpoint{1.570564in}{2.410611in}}%
\pgfpathcurveto{\pgfqpoint{1.570564in}{2.404787in}}{\pgfqpoint{1.572878in}{2.399201in}}{\pgfqpoint{1.576996in}{2.395083in}}%
\pgfpathcurveto{\pgfqpoint{1.581114in}{2.390965in}}{\pgfqpoint{1.586700in}{2.388651in}}{\pgfqpoint{1.592524in}{2.388651in}}%
\pgfpathlineto{\pgfqpoint{1.592524in}{2.388651in}}%
\pgfpathclose%
\pgfusepath{stroke,fill}%
\end{pgfscope}%
\begin{pgfscope}%
\pgfpathrectangle{\pgfqpoint{0.100000in}{0.183744in}}{\pgfqpoint{4.506048in}{4.506048in}}%
\pgfusepath{clip}%
\pgfsetbuttcap%
\pgfsetroundjoin%
\definecolor{currentfill}{rgb}{0.000000,0.000000,1.000000}%
\pgfsetfillcolor{currentfill}%
\pgfsetfillopacity{0.700000}%
\pgfsetlinewidth{1.003750pt}%
\definecolor{currentstroke}{rgb}{0.000000,0.000000,1.000000}%
\pgfsetstrokecolor{currentstroke}%
\pgfsetstrokeopacity{0.700000}%
\pgfsetdash{}{0pt}%
\pgfpathmoveto{\pgfqpoint{3.071959in}{3.087362in}}%
\pgfpathcurveto{\pgfqpoint{3.077783in}{3.087362in}}{\pgfqpoint{3.083369in}{3.089676in}}{\pgfqpoint{3.087487in}{3.093794in}}%
\pgfpathcurveto{\pgfqpoint{3.091605in}{3.097912in}}{\pgfqpoint{3.093919in}{3.103498in}}{\pgfqpoint{3.093919in}{3.109322in}}%
\pgfpathcurveto{\pgfqpoint{3.093919in}{3.115146in}}{\pgfqpoint{3.091605in}{3.120732in}}{\pgfqpoint{3.087487in}{3.124850in}}%
\pgfpathcurveto{\pgfqpoint{3.083369in}{3.128968in}}{\pgfqpoint{3.077783in}{3.131282in}}{\pgfqpoint{3.071959in}{3.131282in}}%
\pgfpathcurveto{\pgfqpoint{3.066135in}{3.131282in}}{\pgfqpoint{3.060549in}{3.128968in}}{\pgfqpoint{3.056431in}{3.124850in}}%
\pgfpathcurveto{\pgfqpoint{3.052312in}{3.120732in}}{\pgfqpoint{3.049999in}{3.115146in}}{\pgfqpoint{3.049999in}{3.109322in}}%
\pgfpathcurveto{\pgfqpoint{3.049999in}{3.103498in}}{\pgfqpoint{3.052312in}{3.097912in}}{\pgfqpoint{3.056431in}{3.093794in}}%
\pgfpathcurveto{\pgfqpoint{3.060549in}{3.089676in}}{\pgfqpoint{3.066135in}{3.087362in}}{\pgfqpoint{3.071959in}{3.087362in}}%
\pgfpathlineto{\pgfqpoint{3.071959in}{3.087362in}}%
\pgfpathclose%
\pgfusepath{stroke,fill}%
\end{pgfscope}%
\begin{pgfscope}%
\pgfpathrectangle{\pgfqpoint{0.100000in}{0.183744in}}{\pgfqpoint{4.506048in}{4.506048in}}%
\pgfusepath{clip}%
\pgfsetbuttcap%
\pgfsetroundjoin%
\definecolor{currentfill}{rgb}{0.000000,0.000000,1.000000}%
\pgfsetfillcolor{currentfill}%
\pgfsetfillopacity{0.700000}%
\pgfsetlinewidth{1.003750pt}%
\definecolor{currentstroke}{rgb}{0.000000,0.000000,1.000000}%
\pgfsetstrokecolor{currentstroke}%
\pgfsetstrokeopacity{0.700000}%
\pgfsetdash{}{0pt}%
\pgfpathmoveto{\pgfqpoint{2.132210in}{1.252871in}}%
\pgfpathcurveto{\pgfqpoint{2.138034in}{1.252871in}}{\pgfqpoint{2.143620in}{1.255185in}}{\pgfqpoint{2.147738in}{1.259303in}}%
\pgfpathcurveto{\pgfqpoint{2.151856in}{1.263422in}}{\pgfqpoint{2.154170in}{1.269008in}}{\pgfqpoint{2.154170in}{1.274832in}}%
\pgfpathcurveto{\pgfqpoint{2.154170in}{1.280656in}}{\pgfqpoint{2.151856in}{1.286242in}}{\pgfqpoint{2.147738in}{1.290360in}}%
\pgfpathcurveto{\pgfqpoint{2.143620in}{1.294478in}}{\pgfqpoint{2.138034in}{1.296792in}}{\pgfqpoint{2.132210in}{1.296792in}}%
\pgfpathcurveto{\pgfqpoint{2.126386in}{1.296792in}}{\pgfqpoint{2.120800in}{1.294478in}}{\pgfqpoint{2.116682in}{1.290360in}}%
\pgfpathcurveto{\pgfqpoint{2.112563in}{1.286242in}}{\pgfqpoint{2.110250in}{1.280656in}}{\pgfqpoint{2.110250in}{1.274832in}}%
\pgfpathcurveto{\pgfqpoint{2.110250in}{1.269008in}}{\pgfqpoint{2.112563in}{1.263422in}}{\pgfqpoint{2.116682in}{1.259303in}}%
\pgfpathcurveto{\pgfqpoint{2.120800in}{1.255185in}}{\pgfqpoint{2.126386in}{1.252871in}}{\pgfqpoint{2.132210in}{1.252871in}}%
\pgfpathlineto{\pgfqpoint{2.132210in}{1.252871in}}%
\pgfpathclose%
\pgfusepath{stroke,fill}%
\end{pgfscope}%
\begin{pgfscope}%
\pgfpathrectangle{\pgfqpoint{0.100000in}{0.183744in}}{\pgfqpoint{4.506048in}{4.506048in}}%
\pgfusepath{clip}%
\pgfsetbuttcap%
\pgfsetroundjoin%
\definecolor{currentfill}{rgb}{0.000000,0.000000,1.000000}%
\pgfsetfillcolor{currentfill}%
\pgfsetfillopacity{0.700000}%
\pgfsetlinewidth{1.003750pt}%
\definecolor{currentstroke}{rgb}{0.000000,0.000000,1.000000}%
\pgfsetstrokecolor{currentstroke}%
\pgfsetstrokeopacity{0.700000}%
\pgfsetdash{}{0pt}%
\pgfpathmoveto{\pgfqpoint{1.446260in}{1.734650in}}%
\pgfpathcurveto{\pgfqpoint{1.452084in}{1.734650in}}{\pgfqpoint{1.457670in}{1.736964in}}{\pgfqpoint{1.461788in}{1.741082in}}%
\pgfpathcurveto{\pgfqpoint{1.465906in}{1.745200in}}{\pgfqpoint{1.468220in}{1.750786in}}{\pgfqpoint{1.468220in}{1.756610in}}%
\pgfpathcurveto{\pgfqpoint{1.468220in}{1.762434in}}{\pgfqpoint{1.465906in}{1.768020in}}{\pgfqpoint{1.461788in}{1.772138in}}%
\pgfpathcurveto{\pgfqpoint{1.457670in}{1.776256in}}{\pgfqpoint{1.452084in}{1.778570in}}{\pgfqpoint{1.446260in}{1.778570in}}%
\pgfpathcurveto{\pgfqpoint{1.440436in}{1.778570in}}{\pgfqpoint{1.434850in}{1.776256in}}{\pgfqpoint{1.430732in}{1.772138in}}%
\pgfpathcurveto{\pgfqpoint{1.426613in}{1.768020in}}{\pgfqpoint{1.424300in}{1.762434in}}{\pgfqpoint{1.424300in}{1.756610in}}%
\pgfpathcurveto{\pgfqpoint{1.424300in}{1.750786in}}{\pgfqpoint{1.426613in}{1.745200in}}{\pgfqpoint{1.430732in}{1.741082in}}%
\pgfpathcurveto{\pgfqpoint{1.434850in}{1.736964in}}{\pgfqpoint{1.440436in}{1.734650in}}{\pgfqpoint{1.446260in}{1.734650in}}%
\pgfpathlineto{\pgfqpoint{1.446260in}{1.734650in}}%
\pgfpathclose%
\pgfusepath{stroke,fill}%
\end{pgfscope}%
\begin{pgfscope}%
\pgfpathrectangle{\pgfqpoint{0.100000in}{0.183744in}}{\pgfqpoint{4.506048in}{4.506048in}}%
\pgfusepath{clip}%
\pgfsetbuttcap%
\pgfsetroundjoin%
\definecolor{currentfill}{rgb}{0.000000,0.000000,1.000000}%
\pgfsetfillcolor{currentfill}%
\pgfsetfillopacity{0.700000}%
\pgfsetlinewidth{1.003750pt}%
\definecolor{currentstroke}{rgb}{0.000000,0.000000,1.000000}%
\pgfsetstrokecolor{currentstroke}%
\pgfsetstrokeopacity{0.700000}%
\pgfsetdash{}{0pt}%
\pgfpathmoveto{\pgfqpoint{3.618693in}{2.743159in}}%
\pgfpathcurveto{\pgfqpoint{3.624516in}{2.743159in}}{\pgfqpoint{3.630103in}{2.745473in}}{\pgfqpoint{3.634221in}{2.749591in}}%
\pgfpathcurveto{\pgfqpoint{3.638339in}{2.753709in}}{\pgfqpoint{3.640653in}{2.759295in}}{\pgfqpoint{3.640653in}{2.765119in}}%
\pgfpathcurveto{\pgfqpoint{3.640653in}{2.770943in}}{\pgfqpoint{3.638339in}{2.776529in}}{\pgfqpoint{3.634221in}{2.780647in}}%
\pgfpathcurveto{\pgfqpoint{3.630103in}{2.784765in}}{\pgfqpoint{3.624516in}{2.787079in}}{\pgfqpoint{3.618693in}{2.787079in}}%
\pgfpathcurveto{\pgfqpoint{3.612869in}{2.787079in}}{\pgfqpoint{3.607282in}{2.784765in}}{\pgfqpoint{3.603164in}{2.780647in}}%
\pgfpathcurveto{\pgfqpoint{3.599046in}{2.776529in}}{\pgfqpoint{3.596732in}{2.770943in}}{\pgfqpoint{3.596732in}{2.765119in}}%
\pgfpathcurveto{\pgfqpoint{3.596732in}{2.759295in}}{\pgfqpoint{3.599046in}{2.753709in}}{\pgfqpoint{3.603164in}{2.749591in}}%
\pgfpathcurveto{\pgfqpoint{3.607282in}{2.745473in}}{\pgfqpoint{3.612869in}{2.743159in}}{\pgfqpoint{3.618693in}{2.743159in}}%
\pgfpathlineto{\pgfqpoint{3.618693in}{2.743159in}}%
\pgfpathclose%
\pgfusepath{stroke,fill}%
\end{pgfscope}%
\begin{pgfscope}%
\pgfpathrectangle{\pgfqpoint{0.100000in}{0.183744in}}{\pgfqpoint{4.506048in}{4.506048in}}%
\pgfusepath{clip}%
\pgfsetbuttcap%
\pgfsetroundjoin%
\definecolor{currentfill}{rgb}{0.000000,0.000000,1.000000}%
\pgfsetfillcolor{currentfill}%
\pgfsetfillopacity{0.700000}%
\pgfsetlinewidth{1.003750pt}%
\definecolor{currentstroke}{rgb}{0.000000,0.000000,1.000000}%
\pgfsetstrokecolor{currentstroke}%
\pgfsetstrokeopacity{0.700000}%
\pgfsetdash{}{0pt}%
\pgfpathmoveto{\pgfqpoint{2.984592in}{1.279749in}}%
\pgfpathcurveto{\pgfqpoint{2.990416in}{1.279749in}}{\pgfqpoint{2.996002in}{1.282063in}}{\pgfqpoint{3.000120in}{1.286181in}}%
\pgfpathcurveto{\pgfqpoint{3.004239in}{1.290299in}}{\pgfqpoint{3.006552in}{1.295885in}}{\pgfqpoint{3.006552in}{1.301709in}}%
\pgfpathcurveto{\pgfqpoint{3.006552in}{1.307533in}}{\pgfqpoint{3.004239in}{1.313119in}}{\pgfqpoint{3.000120in}{1.317237in}}%
\pgfpathcurveto{\pgfqpoint{2.996002in}{1.321355in}}{\pgfqpoint{2.990416in}{1.323669in}}{\pgfqpoint{2.984592in}{1.323669in}}%
\pgfpathcurveto{\pgfqpoint{2.978768in}{1.323669in}}{\pgfqpoint{2.973182in}{1.321355in}}{\pgfqpoint{2.969064in}{1.317237in}}%
\pgfpathcurveto{\pgfqpoint{2.964946in}{1.313119in}}{\pgfqpoint{2.962632in}{1.307533in}}{\pgfqpoint{2.962632in}{1.301709in}}%
\pgfpathcurveto{\pgfqpoint{2.962632in}{1.295885in}}{\pgfqpoint{2.964946in}{1.290299in}}{\pgfqpoint{2.969064in}{1.286181in}}%
\pgfpathcurveto{\pgfqpoint{2.973182in}{1.282063in}}{\pgfqpoint{2.978768in}{1.279749in}}{\pgfqpoint{2.984592in}{1.279749in}}%
\pgfpathlineto{\pgfqpoint{2.984592in}{1.279749in}}%
\pgfpathclose%
\pgfusepath{stroke,fill}%
\end{pgfscope}%
\begin{pgfscope}%
\pgfpathrectangle{\pgfqpoint{0.100000in}{0.183744in}}{\pgfqpoint{4.506048in}{4.506048in}}%
\pgfusepath{clip}%
\pgfsetbuttcap%
\pgfsetroundjoin%
\definecolor{currentfill}{rgb}{0.000000,0.000000,1.000000}%
\pgfsetfillcolor{currentfill}%
\pgfsetfillopacity{0.700000}%
\pgfsetlinewidth{1.003750pt}%
\definecolor{currentstroke}{rgb}{0.000000,0.000000,1.000000}%
\pgfsetstrokecolor{currentstroke}%
\pgfsetstrokeopacity{0.700000}%
\pgfsetdash{}{0pt}%
\pgfpathmoveto{\pgfqpoint{2.369209in}{2.303126in}}%
\pgfpathcurveto{\pgfqpoint{2.375033in}{2.303126in}}{\pgfqpoint{2.380619in}{2.305440in}}{\pgfqpoint{2.384738in}{2.309558in}}%
\pgfpathcurveto{\pgfqpoint{2.388856in}{2.313676in}}{\pgfqpoint{2.391170in}{2.319262in}}{\pgfqpoint{2.391170in}{2.325086in}}%
\pgfpathcurveto{\pgfqpoint{2.391170in}{2.330910in}}{\pgfqpoint{2.388856in}{2.336496in}}{\pgfqpoint{2.384738in}{2.340614in}}%
\pgfpathcurveto{\pgfqpoint{2.380619in}{2.344732in}}{\pgfqpoint{2.375033in}{2.347046in}}{\pgfqpoint{2.369209in}{2.347046in}}%
\pgfpathcurveto{\pgfqpoint{2.363385in}{2.347046in}}{\pgfqpoint{2.357799in}{2.344732in}}{\pgfqpoint{2.353681in}{2.340614in}}%
\pgfpathcurveto{\pgfqpoint{2.349563in}{2.336496in}}{\pgfqpoint{2.347249in}{2.330910in}}{\pgfqpoint{2.347249in}{2.325086in}}%
\pgfpathcurveto{\pgfqpoint{2.347249in}{2.319262in}}{\pgfqpoint{2.349563in}{2.313676in}}{\pgfqpoint{2.353681in}{2.309558in}}%
\pgfpathcurveto{\pgfqpoint{2.357799in}{2.305440in}}{\pgfqpoint{2.363385in}{2.303126in}}{\pgfqpoint{2.369209in}{2.303126in}}%
\pgfpathlineto{\pgfqpoint{2.369209in}{2.303126in}}%
\pgfpathclose%
\pgfusepath{stroke,fill}%
\end{pgfscope}%
\begin{pgfscope}%
\pgfpathrectangle{\pgfqpoint{0.100000in}{0.183744in}}{\pgfqpoint{4.506048in}{4.506048in}}%
\pgfusepath{clip}%
\pgfsetbuttcap%
\pgfsetroundjoin%
\definecolor{currentfill}{rgb}{0.000000,0.000000,1.000000}%
\pgfsetfillcolor{currentfill}%
\pgfsetfillopacity{0.700000}%
\pgfsetlinewidth{1.003750pt}%
\definecolor{currentstroke}{rgb}{0.000000,0.000000,1.000000}%
\pgfsetstrokecolor{currentstroke}%
\pgfsetstrokeopacity{0.700000}%
\pgfsetdash{}{0pt}%
\pgfpathmoveto{\pgfqpoint{3.387859in}{3.962771in}}%
\pgfpathcurveto{\pgfqpoint{3.393683in}{3.962771in}}{\pgfqpoint{3.399269in}{3.965085in}}{\pgfqpoint{3.403387in}{3.969203in}}%
\pgfpathcurveto{\pgfqpoint{3.407505in}{3.973321in}}{\pgfqpoint{3.409819in}{3.978907in}}{\pgfqpoint{3.409819in}{3.984731in}}%
\pgfpathcurveto{\pgfqpoint{3.409819in}{3.990555in}}{\pgfqpoint{3.407505in}{3.996141in}}{\pgfqpoint{3.403387in}{4.000260in}}%
\pgfpathcurveto{\pgfqpoint{3.399269in}{4.004378in}}{\pgfqpoint{3.393683in}{4.006692in}}{\pgfqpoint{3.387859in}{4.006692in}}%
\pgfpathcurveto{\pgfqpoint{3.382035in}{4.006692in}}{\pgfqpoint{3.376449in}{4.004378in}}{\pgfqpoint{3.372330in}{4.000260in}}%
\pgfpathcurveto{\pgfqpoint{3.368212in}{3.996141in}}{\pgfqpoint{3.365898in}{3.990555in}}{\pgfqpoint{3.365898in}{3.984731in}}%
\pgfpathcurveto{\pgfqpoint{3.365898in}{3.978907in}}{\pgfqpoint{3.368212in}{3.973321in}}{\pgfqpoint{3.372330in}{3.969203in}}%
\pgfpathcurveto{\pgfqpoint{3.376449in}{3.965085in}}{\pgfqpoint{3.382035in}{3.962771in}}{\pgfqpoint{3.387859in}{3.962771in}}%
\pgfpathlineto{\pgfqpoint{3.387859in}{3.962771in}}%
\pgfpathclose%
\pgfusepath{stroke,fill}%
\end{pgfscope}%
\begin{pgfscope}%
\pgfpathrectangle{\pgfqpoint{0.100000in}{0.183744in}}{\pgfqpoint{4.506048in}{4.506048in}}%
\pgfusepath{clip}%
\pgfsetbuttcap%
\pgfsetroundjoin%
\definecolor{currentfill}{rgb}{0.000000,0.000000,1.000000}%
\pgfsetfillcolor{currentfill}%
\pgfsetfillopacity{0.700000}%
\pgfsetlinewidth{1.003750pt}%
\definecolor{currentstroke}{rgb}{0.000000,0.000000,1.000000}%
\pgfsetstrokecolor{currentstroke}%
\pgfsetstrokeopacity{0.700000}%
\pgfsetdash{}{0pt}%
\pgfpathmoveto{\pgfqpoint{1.388366in}{1.555127in}}%
\pgfpathcurveto{\pgfqpoint{1.394190in}{1.555127in}}{\pgfqpoint{1.399776in}{1.557441in}}{\pgfqpoint{1.403894in}{1.561559in}}%
\pgfpathcurveto{\pgfqpoint{1.408012in}{1.565677in}}{\pgfqpoint{1.410326in}{1.571264in}}{\pgfqpoint{1.410326in}{1.577088in}}%
\pgfpathcurveto{\pgfqpoint{1.410326in}{1.582912in}}{\pgfqpoint{1.408012in}{1.588498in}}{\pgfqpoint{1.403894in}{1.592616in}}%
\pgfpathcurveto{\pgfqpoint{1.399776in}{1.596734in}}{\pgfqpoint{1.394190in}{1.599048in}}{\pgfqpoint{1.388366in}{1.599048in}}%
\pgfpathcurveto{\pgfqpoint{1.382542in}{1.599048in}}{\pgfqpoint{1.376956in}{1.596734in}}{\pgfqpoint{1.372837in}{1.592616in}}%
\pgfpathcurveto{\pgfqpoint{1.368719in}{1.588498in}}{\pgfqpoint{1.366405in}{1.582912in}}{\pgfqpoint{1.366405in}{1.577088in}}%
\pgfpathcurveto{\pgfqpoint{1.366405in}{1.571264in}}{\pgfqpoint{1.368719in}{1.565677in}}{\pgfqpoint{1.372837in}{1.561559in}}%
\pgfpathcurveto{\pgfqpoint{1.376956in}{1.557441in}}{\pgfqpoint{1.382542in}{1.555127in}}{\pgfqpoint{1.388366in}{1.555127in}}%
\pgfpathlineto{\pgfqpoint{1.388366in}{1.555127in}}%
\pgfpathclose%
\pgfusepath{stroke,fill}%
\end{pgfscope}%
\begin{pgfscope}%
\pgfpathrectangle{\pgfqpoint{0.100000in}{0.183744in}}{\pgfqpoint{4.506048in}{4.506048in}}%
\pgfusepath{clip}%
\pgfsetbuttcap%
\pgfsetroundjoin%
\definecolor{currentfill}{rgb}{0.000000,0.000000,1.000000}%
\pgfsetfillcolor{currentfill}%
\pgfsetfillopacity{0.700000}%
\pgfsetlinewidth{1.003750pt}%
\definecolor{currentstroke}{rgb}{0.000000,0.000000,1.000000}%
\pgfsetstrokecolor{currentstroke}%
\pgfsetstrokeopacity{0.700000}%
\pgfsetdash{}{0pt}%
\pgfpathmoveto{\pgfqpoint{1.662061in}{2.128466in}}%
\pgfpathcurveto{\pgfqpoint{1.667885in}{2.128466in}}{\pgfqpoint{1.673471in}{2.130780in}}{\pgfqpoint{1.677590in}{2.134898in}}%
\pgfpathcurveto{\pgfqpoint{1.681708in}{2.139016in}}{\pgfqpoint{1.684022in}{2.144602in}}{\pgfqpoint{1.684022in}{2.150426in}}%
\pgfpathcurveto{\pgfqpoint{1.684022in}{2.156250in}}{\pgfqpoint{1.681708in}{2.161836in}}{\pgfqpoint{1.677590in}{2.165955in}}%
\pgfpathcurveto{\pgfqpoint{1.673471in}{2.170073in}}{\pgfqpoint{1.667885in}{2.172387in}}{\pgfqpoint{1.662061in}{2.172387in}}%
\pgfpathcurveto{\pgfqpoint{1.656237in}{2.172387in}}{\pgfqpoint{1.650651in}{2.170073in}}{\pgfqpoint{1.646533in}{2.165955in}}%
\pgfpathcurveto{\pgfqpoint{1.642415in}{2.161836in}}{\pgfqpoint{1.640101in}{2.156250in}}{\pgfqpoint{1.640101in}{2.150426in}}%
\pgfpathcurveto{\pgfqpoint{1.640101in}{2.144602in}}{\pgfqpoint{1.642415in}{2.139016in}}{\pgfqpoint{1.646533in}{2.134898in}}%
\pgfpathcurveto{\pgfqpoint{1.650651in}{2.130780in}}{\pgfqpoint{1.656237in}{2.128466in}}{\pgfqpoint{1.662061in}{2.128466in}}%
\pgfpathlineto{\pgfqpoint{1.662061in}{2.128466in}}%
\pgfpathclose%
\pgfusepath{stroke,fill}%
\end{pgfscope}%
\begin{pgfscope}%
\pgfpathrectangle{\pgfqpoint{0.100000in}{0.183744in}}{\pgfqpoint{4.506048in}{4.506048in}}%
\pgfusepath{clip}%
\pgfsetbuttcap%
\pgfsetroundjoin%
\definecolor{currentfill}{rgb}{0.000000,0.000000,1.000000}%
\pgfsetfillcolor{currentfill}%
\pgfsetfillopacity{0.700000}%
\pgfsetlinewidth{1.003750pt}%
\definecolor{currentstroke}{rgb}{0.000000,0.000000,1.000000}%
\pgfsetstrokecolor{currentstroke}%
\pgfsetstrokeopacity{0.700000}%
\pgfsetdash{}{0pt}%
\pgfpathmoveto{\pgfqpoint{1.844046in}{2.452163in}}%
\pgfpathcurveto{\pgfqpoint{1.849870in}{2.452163in}}{\pgfqpoint{1.855456in}{2.454477in}}{\pgfqpoint{1.859575in}{2.458595in}}%
\pgfpathcurveto{\pgfqpoint{1.863693in}{2.462713in}}{\pgfqpoint{1.866007in}{2.468300in}}{\pgfqpoint{1.866007in}{2.474123in}}%
\pgfpathcurveto{\pgfqpoint{1.866007in}{2.479947in}}{\pgfqpoint{1.863693in}{2.485534in}}{\pgfqpoint{1.859575in}{2.489652in}}%
\pgfpathcurveto{\pgfqpoint{1.855456in}{2.493770in}}{\pgfqpoint{1.849870in}{2.496084in}}{\pgfqpoint{1.844046in}{2.496084in}}%
\pgfpathcurveto{\pgfqpoint{1.838222in}{2.496084in}}{\pgfqpoint{1.832636in}{2.493770in}}{\pgfqpoint{1.828518in}{2.489652in}}%
\pgfpathcurveto{\pgfqpoint{1.824400in}{2.485534in}}{\pgfqpoint{1.822086in}{2.479947in}}{\pgfqpoint{1.822086in}{2.474123in}}%
\pgfpathcurveto{\pgfqpoint{1.822086in}{2.468300in}}{\pgfqpoint{1.824400in}{2.462713in}}{\pgfqpoint{1.828518in}{2.458595in}}%
\pgfpathcurveto{\pgfqpoint{1.832636in}{2.454477in}}{\pgfqpoint{1.838222in}{2.452163in}}{\pgfqpoint{1.844046in}{2.452163in}}%
\pgfpathlineto{\pgfqpoint{1.844046in}{2.452163in}}%
\pgfpathclose%
\pgfusepath{stroke,fill}%
\end{pgfscope}%
\begin{pgfscope}%
\pgfpathrectangle{\pgfqpoint{0.100000in}{0.183744in}}{\pgfqpoint{4.506048in}{4.506048in}}%
\pgfusepath{clip}%
\pgfsetbuttcap%
\pgfsetroundjoin%
\definecolor{currentfill}{rgb}{0.000000,0.000000,1.000000}%
\pgfsetfillcolor{currentfill}%
\pgfsetfillopacity{0.700000}%
\pgfsetlinewidth{1.003750pt}%
\definecolor{currentstroke}{rgb}{0.000000,0.000000,1.000000}%
\pgfsetstrokecolor{currentstroke}%
\pgfsetstrokeopacity{0.700000}%
\pgfsetdash{}{0pt}%
\pgfpathmoveto{\pgfqpoint{2.061356in}{3.368827in}}%
\pgfpathcurveto{\pgfqpoint{2.067180in}{3.368827in}}{\pgfqpoint{2.072766in}{3.371141in}}{\pgfqpoint{2.076885in}{3.375259in}}%
\pgfpathcurveto{\pgfqpoint{2.081003in}{3.379377in}}{\pgfqpoint{2.083317in}{3.384963in}}{\pgfqpoint{2.083317in}{3.390787in}}%
\pgfpathcurveto{\pgfqpoint{2.083317in}{3.396611in}}{\pgfqpoint{2.081003in}{3.402197in}}{\pgfqpoint{2.076885in}{3.406315in}}%
\pgfpathcurveto{\pgfqpoint{2.072766in}{3.410433in}}{\pgfqpoint{2.067180in}{3.412747in}}{\pgfqpoint{2.061356in}{3.412747in}}%
\pgfpathcurveto{\pgfqpoint{2.055532in}{3.412747in}}{\pgfqpoint{2.049946in}{3.410433in}}{\pgfqpoint{2.045828in}{3.406315in}}%
\pgfpathcurveto{\pgfqpoint{2.041710in}{3.402197in}}{\pgfqpoint{2.039396in}{3.396611in}}{\pgfqpoint{2.039396in}{3.390787in}}%
\pgfpathcurveto{\pgfqpoint{2.039396in}{3.384963in}}{\pgfqpoint{2.041710in}{3.379377in}}{\pgfqpoint{2.045828in}{3.375259in}}%
\pgfpathcurveto{\pgfqpoint{2.049946in}{3.371141in}}{\pgfqpoint{2.055532in}{3.368827in}}{\pgfqpoint{2.061356in}{3.368827in}}%
\pgfpathlineto{\pgfqpoint{2.061356in}{3.368827in}}%
\pgfpathclose%
\pgfusepath{stroke,fill}%
\end{pgfscope}%
\begin{pgfscope}%
\pgfpathrectangle{\pgfqpoint{0.100000in}{0.183744in}}{\pgfqpoint{4.506048in}{4.506048in}}%
\pgfusepath{clip}%
\pgfsetbuttcap%
\pgfsetroundjoin%
\definecolor{currentfill}{rgb}{0.000000,0.000000,1.000000}%
\pgfsetfillcolor{currentfill}%
\pgfsetfillopacity{0.700000}%
\pgfsetlinewidth{1.003750pt}%
\definecolor{currentstroke}{rgb}{0.000000,0.000000,1.000000}%
\pgfsetstrokecolor{currentstroke}%
\pgfsetstrokeopacity{0.700000}%
\pgfsetdash{}{0pt}%
\pgfpathmoveto{\pgfqpoint{0.520237in}{3.516465in}}%
\pgfpathcurveto{\pgfqpoint{0.526061in}{3.516465in}}{\pgfqpoint{0.531647in}{3.518778in}}{\pgfqpoint{0.535766in}{3.522897in}}%
\pgfpathcurveto{\pgfqpoint{0.539884in}{3.527015in}}{\pgfqpoint{0.542198in}{3.532601in}}{\pgfqpoint{0.542198in}{3.538425in}}%
\pgfpathcurveto{\pgfqpoint{0.542198in}{3.544249in}}{\pgfqpoint{0.539884in}{3.549835in}}{\pgfqpoint{0.535766in}{3.553953in}}%
\pgfpathcurveto{\pgfqpoint{0.531647in}{3.558071in}}{\pgfqpoint{0.526061in}{3.560385in}}{\pgfqpoint{0.520237in}{3.560385in}}%
\pgfpathcurveto{\pgfqpoint{0.514413in}{3.560385in}}{\pgfqpoint{0.508827in}{3.558071in}}{\pgfqpoint{0.504709in}{3.553953in}}%
\pgfpathcurveto{\pgfqpoint{0.500591in}{3.549835in}}{\pgfqpoint{0.498277in}{3.544249in}}{\pgfqpoint{0.498277in}{3.538425in}}%
\pgfpathcurveto{\pgfqpoint{0.498277in}{3.532601in}}{\pgfqpoint{0.500591in}{3.527015in}}{\pgfqpoint{0.504709in}{3.522897in}}%
\pgfpathcurveto{\pgfqpoint{0.508827in}{3.518778in}}{\pgfqpoint{0.514413in}{3.516465in}}{\pgfqpoint{0.520237in}{3.516465in}}%
\pgfpathlineto{\pgfqpoint{0.520237in}{3.516465in}}%
\pgfpathclose%
\pgfusepath{stroke,fill}%
\end{pgfscope}%
\begin{pgfscope}%
\pgfpathrectangle{\pgfqpoint{0.100000in}{0.183744in}}{\pgfqpoint{4.506048in}{4.506048in}}%
\pgfusepath{clip}%
\pgfsetbuttcap%
\pgfsetroundjoin%
\definecolor{currentfill}{rgb}{0.000000,0.000000,1.000000}%
\pgfsetfillcolor{currentfill}%
\pgfsetfillopacity{0.700000}%
\pgfsetlinewidth{1.003750pt}%
\definecolor{currentstroke}{rgb}{0.000000,0.000000,1.000000}%
\pgfsetstrokecolor{currentstroke}%
\pgfsetstrokeopacity{0.700000}%
\pgfsetdash{}{0pt}%
\pgfpathmoveto{\pgfqpoint{2.355586in}{3.897479in}}%
\pgfpathcurveto{\pgfqpoint{2.361410in}{3.897479in}}{\pgfqpoint{2.366996in}{3.899793in}}{\pgfqpoint{2.371115in}{3.903911in}}%
\pgfpathcurveto{\pgfqpoint{2.375233in}{3.908029in}}{\pgfqpoint{2.377547in}{3.913615in}}{\pgfqpoint{2.377547in}{3.919439in}}%
\pgfpathcurveto{\pgfqpoint{2.377547in}{3.925263in}}{\pgfqpoint{2.375233in}{3.930849in}}{\pgfqpoint{2.371115in}{3.934967in}}%
\pgfpathcurveto{\pgfqpoint{2.366996in}{3.939085in}}{\pgfqpoint{2.361410in}{3.941399in}}{\pgfqpoint{2.355586in}{3.941399in}}%
\pgfpathcurveto{\pgfqpoint{2.349762in}{3.941399in}}{\pgfqpoint{2.344176in}{3.939085in}}{\pgfqpoint{2.340058in}{3.934967in}}%
\pgfpathcurveto{\pgfqpoint{2.335940in}{3.930849in}}{\pgfqpoint{2.333626in}{3.925263in}}{\pgfqpoint{2.333626in}{3.919439in}}%
\pgfpathcurveto{\pgfqpoint{2.333626in}{3.913615in}}{\pgfqpoint{2.335940in}{3.908029in}}{\pgfqpoint{2.340058in}{3.903911in}}%
\pgfpathcurveto{\pgfqpoint{2.344176in}{3.899793in}}{\pgfqpoint{2.349762in}{3.897479in}}{\pgfqpoint{2.355586in}{3.897479in}}%
\pgfpathlineto{\pgfqpoint{2.355586in}{3.897479in}}%
\pgfpathclose%
\pgfusepath{stroke,fill}%
\end{pgfscope}%
\begin{pgfscope}%
\pgfpathrectangle{\pgfqpoint{0.100000in}{0.183744in}}{\pgfqpoint{4.506048in}{4.506048in}}%
\pgfusepath{clip}%
\pgfsetbuttcap%
\pgfsetroundjoin%
\definecolor{currentfill}{rgb}{0.000000,0.000000,1.000000}%
\pgfsetfillcolor{currentfill}%
\pgfsetfillopacity{0.700000}%
\pgfsetlinewidth{1.003750pt}%
\definecolor{currentstroke}{rgb}{0.000000,0.000000,1.000000}%
\pgfsetstrokecolor{currentstroke}%
\pgfsetstrokeopacity{0.700000}%
\pgfsetdash{}{0pt}%
\pgfpathmoveto{\pgfqpoint{2.011546in}{2.831393in}}%
\pgfpathcurveto{\pgfqpoint{2.017370in}{2.831393in}}{\pgfqpoint{2.022956in}{2.833706in}}{\pgfqpoint{2.027074in}{2.837825in}}%
\pgfpathcurveto{\pgfqpoint{2.031192in}{2.841943in}}{\pgfqpoint{2.033506in}{2.847529in}}{\pgfqpoint{2.033506in}{2.853353in}}%
\pgfpathcurveto{\pgfqpoint{2.033506in}{2.859177in}}{\pgfqpoint{2.031192in}{2.864763in}}{\pgfqpoint{2.027074in}{2.868881in}}%
\pgfpathcurveto{\pgfqpoint{2.022956in}{2.872999in}}{\pgfqpoint{2.017370in}{2.875313in}}{\pgfqpoint{2.011546in}{2.875313in}}%
\pgfpathcurveto{\pgfqpoint{2.005722in}{2.875313in}}{\pgfqpoint{2.000136in}{2.872999in}}{\pgfqpoint{1.996018in}{2.868881in}}%
\pgfpathcurveto{\pgfqpoint{1.991899in}{2.864763in}}{\pgfqpoint{1.989585in}{2.859177in}}{\pgfqpoint{1.989585in}{2.853353in}}%
\pgfpathcurveto{\pgfqpoint{1.989585in}{2.847529in}}{\pgfqpoint{1.991899in}{2.841943in}}{\pgfqpoint{1.996018in}{2.837825in}}%
\pgfpathcurveto{\pgfqpoint{2.000136in}{2.833706in}}{\pgfqpoint{2.005722in}{2.831393in}}{\pgfqpoint{2.011546in}{2.831393in}}%
\pgfpathlineto{\pgfqpoint{2.011546in}{2.831393in}}%
\pgfpathclose%
\pgfusepath{stroke,fill}%
\end{pgfscope}%
\begin{pgfscope}%
\pgfpathrectangle{\pgfqpoint{0.100000in}{0.183744in}}{\pgfqpoint{4.506048in}{4.506048in}}%
\pgfusepath{clip}%
\pgfsetbuttcap%
\pgfsetroundjoin%
\definecolor{currentfill}{rgb}{0.000000,0.000000,1.000000}%
\pgfsetfillcolor{currentfill}%
\pgfsetfillopacity{0.700000}%
\pgfsetlinewidth{1.003750pt}%
\definecolor{currentstroke}{rgb}{0.000000,0.000000,1.000000}%
\pgfsetstrokecolor{currentstroke}%
\pgfsetstrokeopacity{0.700000}%
\pgfsetdash{}{0pt}%
\pgfpathmoveto{\pgfqpoint{3.602370in}{2.507909in}}%
\pgfpathcurveto{\pgfqpoint{3.608194in}{2.507909in}}{\pgfqpoint{3.613780in}{2.510223in}}{\pgfqpoint{3.617898in}{2.514341in}}%
\pgfpathcurveto{\pgfqpoint{3.622017in}{2.518459in}}{\pgfqpoint{3.624330in}{2.524045in}}{\pgfqpoint{3.624330in}{2.529869in}}%
\pgfpathcurveto{\pgfqpoint{3.624330in}{2.535693in}}{\pgfqpoint{3.622017in}{2.541279in}}{\pgfqpoint{3.617898in}{2.545397in}}%
\pgfpathcurveto{\pgfqpoint{3.613780in}{2.549515in}}{\pgfqpoint{3.608194in}{2.551829in}}{\pgfqpoint{3.602370in}{2.551829in}}%
\pgfpathcurveto{\pgfqpoint{3.596546in}{2.551829in}}{\pgfqpoint{3.590960in}{2.549515in}}{\pgfqpoint{3.586842in}{2.545397in}}%
\pgfpathcurveto{\pgfqpoint{3.582724in}{2.541279in}}{\pgfqpoint{3.580410in}{2.535693in}}{\pgfqpoint{3.580410in}{2.529869in}}%
\pgfpathcurveto{\pgfqpoint{3.580410in}{2.524045in}}{\pgfqpoint{3.582724in}{2.518459in}}{\pgfqpoint{3.586842in}{2.514341in}}%
\pgfpathcurveto{\pgfqpoint{3.590960in}{2.510223in}}{\pgfqpoint{3.596546in}{2.507909in}}{\pgfqpoint{3.602370in}{2.507909in}}%
\pgfpathlineto{\pgfqpoint{3.602370in}{2.507909in}}%
\pgfpathclose%
\pgfusepath{stroke,fill}%
\end{pgfscope}%
\begin{pgfscope}%
\pgfpathrectangle{\pgfqpoint{0.100000in}{0.183744in}}{\pgfqpoint{4.506048in}{4.506048in}}%
\pgfusepath{clip}%
\pgfsetbuttcap%
\pgfsetroundjoin%
\definecolor{currentfill}{rgb}{0.000000,0.000000,1.000000}%
\pgfsetfillcolor{currentfill}%
\pgfsetfillopacity{0.700000}%
\pgfsetlinewidth{1.003750pt}%
\definecolor{currentstroke}{rgb}{0.000000,0.000000,1.000000}%
\pgfsetstrokecolor{currentstroke}%
\pgfsetstrokeopacity{0.700000}%
\pgfsetdash{}{0pt}%
\pgfpathmoveto{\pgfqpoint{3.206702in}{1.257144in}}%
\pgfpathcurveto{\pgfqpoint{3.212526in}{1.257144in}}{\pgfqpoint{3.218113in}{1.259458in}}{\pgfqpoint{3.222231in}{1.263576in}}%
\pgfpathcurveto{\pgfqpoint{3.226349in}{1.267694in}}{\pgfqpoint{3.228663in}{1.273281in}}{\pgfqpoint{3.228663in}{1.279105in}}%
\pgfpathcurveto{\pgfqpoint{3.228663in}{1.284928in}}{\pgfqpoint{3.226349in}{1.290515in}}{\pgfqpoint{3.222231in}{1.294633in}}%
\pgfpathcurveto{\pgfqpoint{3.218113in}{1.298751in}}{\pgfqpoint{3.212526in}{1.301065in}}{\pgfqpoint{3.206702in}{1.301065in}}%
\pgfpathcurveto{\pgfqpoint{3.200878in}{1.301065in}}{\pgfqpoint{3.195292in}{1.298751in}}{\pgfqpoint{3.191174in}{1.294633in}}%
\pgfpathcurveto{\pgfqpoint{3.187056in}{1.290515in}}{\pgfqpoint{3.184742in}{1.284928in}}{\pgfqpoint{3.184742in}{1.279105in}}%
\pgfpathcurveto{\pgfqpoint{3.184742in}{1.273281in}}{\pgfqpoint{3.187056in}{1.267694in}}{\pgfqpoint{3.191174in}{1.263576in}}%
\pgfpathcurveto{\pgfqpoint{3.195292in}{1.259458in}}{\pgfqpoint{3.200878in}{1.257144in}}{\pgfqpoint{3.206702in}{1.257144in}}%
\pgfpathlineto{\pgfqpoint{3.206702in}{1.257144in}}%
\pgfpathclose%
\pgfusepath{stroke,fill}%
\end{pgfscope}%
\begin{pgfscope}%
\pgfpathrectangle{\pgfqpoint{0.100000in}{0.183744in}}{\pgfqpoint{4.506048in}{4.506048in}}%
\pgfusepath{clip}%
\pgfsetbuttcap%
\pgfsetroundjoin%
\definecolor{currentfill}{rgb}{0.000000,0.000000,1.000000}%
\pgfsetfillcolor{currentfill}%
\pgfsetfillopacity{0.700000}%
\pgfsetlinewidth{1.003750pt}%
\definecolor{currentstroke}{rgb}{0.000000,0.000000,1.000000}%
\pgfsetstrokecolor{currentstroke}%
\pgfsetstrokeopacity{0.700000}%
\pgfsetdash{}{0pt}%
\pgfpathmoveto{\pgfqpoint{2.684517in}{1.688252in}}%
\pgfpathcurveto{\pgfqpoint{2.690340in}{1.688252in}}{\pgfqpoint{2.695927in}{1.690566in}}{\pgfqpoint{2.700045in}{1.694684in}}%
\pgfpathcurveto{\pgfqpoint{2.704163in}{1.698802in}}{\pgfqpoint{2.706477in}{1.704388in}}{\pgfqpoint{2.706477in}{1.710212in}}%
\pgfpathcurveto{\pgfqpoint{2.706477in}{1.716036in}}{\pgfqpoint{2.704163in}{1.721622in}}{\pgfqpoint{2.700045in}{1.725740in}}%
\pgfpathcurveto{\pgfqpoint{2.695927in}{1.729859in}}{\pgfqpoint{2.690340in}{1.732172in}}{\pgfqpoint{2.684517in}{1.732172in}}%
\pgfpathcurveto{\pgfqpoint{2.678693in}{1.732172in}}{\pgfqpoint{2.673106in}{1.729859in}}{\pgfqpoint{2.668988in}{1.725740in}}%
\pgfpathcurveto{\pgfqpoint{2.664870in}{1.721622in}}{\pgfqpoint{2.662556in}{1.716036in}}{\pgfqpoint{2.662556in}{1.710212in}}%
\pgfpathcurveto{\pgfqpoint{2.662556in}{1.704388in}}{\pgfqpoint{2.664870in}{1.698802in}}{\pgfqpoint{2.668988in}{1.694684in}}%
\pgfpathcurveto{\pgfqpoint{2.673106in}{1.690566in}}{\pgfqpoint{2.678693in}{1.688252in}}{\pgfqpoint{2.684517in}{1.688252in}}%
\pgfpathlineto{\pgfqpoint{2.684517in}{1.688252in}}%
\pgfpathclose%
\pgfusepath{stroke,fill}%
\end{pgfscope}%
\begin{pgfscope}%
\pgfpathrectangle{\pgfqpoint{0.100000in}{0.183744in}}{\pgfqpoint{4.506048in}{4.506048in}}%
\pgfusepath{clip}%
\pgfsetbuttcap%
\pgfsetroundjoin%
\definecolor{currentfill}{rgb}{0.000000,0.000000,1.000000}%
\pgfsetfillcolor{currentfill}%
\pgfsetfillopacity{0.700000}%
\pgfsetlinewidth{1.003750pt}%
\definecolor{currentstroke}{rgb}{0.000000,0.000000,1.000000}%
\pgfsetstrokecolor{currentstroke}%
\pgfsetstrokeopacity{0.700000}%
\pgfsetdash{}{0pt}%
\pgfpathmoveto{\pgfqpoint{2.662648in}{2.849915in}}%
\pgfpathcurveto{\pgfqpoint{2.668472in}{2.849915in}}{\pgfqpoint{2.674058in}{2.852229in}}{\pgfqpoint{2.678176in}{2.856347in}}%
\pgfpathcurveto{\pgfqpoint{2.682294in}{2.860465in}}{\pgfqpoint{2.684608in}{2.866051in}}{\pgfqpoint{2.684608in}{2.871875in}}%
\pgfpathcurveto{\pgfqpoint{2.684608in}{2.877699in}}{\pgfqpoint{2.682294in}{2.883285in}}{\pgfqpoint{2.678176in}{2.887403in}}%
\pgfpathcurveto{\pgfqpoint{2.674058in}{2.891522in}}{\pgfqpoint{2.668472in}{2.893835in}}{\pgfqpoint{2.662648in}{2.893835in}}%
\pgfpathcurveto{\pgfqpoint{2.656824in}{2.893835in}}{\pgfqpoint{2.651237in}{2.891522in}}{\pgfqpoint{2.647119in}{2.887403in}}%
\pgfpathcurveto{\pgfqpoint{2.643001in}{2.883285in}}{\pgfqpoint{2.640687in}{2.877699in}}{\pgfqpoint{2.640687in}{2.871875in}}%
\pgfpathcurveto{\pgfqpoint{2.640687in}{2.866051in}}{\pgfqpoint{2.643001in}{2.860465in}}{\pgfqpoint{2.647119in}{2.856347in}}%
\pgfpathcurveto{\pgfqpoint{2.651237in}{2.852229in}}{\pgfqpoint{2.656824in}{2.849915in}}{\pgfqpoint{2.662648in}{2.849915in}}%
\pgfpathlineto{\pgfqpoint{2.662648in}{2.849915in}}%
\pgfpathclose%
\pgfusepath{stroke,fill}%
\end{pgfscope}%
\begin{pgfscope}%
\pgfpathrectangle{\pgfqpoint{0.100000in}{0.183744in}}{\pgfqpoint{4.506048in}{4.506048in}}%
\pgfusepath{clip}%
\pgfsetbuttcap%
\pgfsetroundjoin%
\definecolor{currentfill}{rgb}{0.000000,0.000000,1.000000}%
\pgfsetfillcolor{currentfill}%
\pgfsetfillopacity{0.700000}%
\pgfsetlinewidth{1.003750pt}%
\definecolor{currentstroke}{rgb}{0.000000,0.000000,1.000000}%
\pgfsetstrokecolor{currentstroke}%
\pgfsetstrokeopacity{0.700000}%
\pgfsetdash{}{0pt}%
\pgfpathmoveto{\pgfqpoint{3.202468in}{2.162955in}}%
\pgfpathcurveto{\pgfqpoint{3.208292in}{2.162955in}}{\pgfqpoint{3.213878in}{2.165269in}}{\pgfqpoint{3.217996in}{2.169387in}}%
\pgfpathcurveto{\pgfqpoint{3.222114in}{2.173505in}}{\pgfqpoint{3.224428in}{2.179092in}}{\pgfqpoint{3.224428in}{2.184915in}}%
\pgfpathcurveto{\pgfqpoint{3.224428in}{2.190739in}}{\pgfqpoint{3.222114in}{2.196326in}}{\pgfqpoint{3.217996in}{2.200444in}}%
\pgfpathcurveto{\pgfqpoint{3.213878in}{2.204562in}}{\pgfqpoint{3.208292in}{2.206876in}}{\pgfqpoint{3.202468in}{2.206876in}}%
\pgfpathcurveto{\pgfqpoint{3.196644in}{2.206876in}}{\pgfqpoint{3.191057in}{2.204562in}}{\pgfqpoint{3.186939in}{2.200444in}}%
\pgfpathcurveto{\pgfqpoint{3.182821in}{2.196326in}}{\pgfqpoint{3.180507in}{2.190739in}}{\pgfqpoint{3.180507in}{2.184915in}}%
\pgfpathcurveto{\pgfqpoint{3.180507in}{2.179092in}}{\pgfqpoint{3.182821in}{2.173505in}}{\pgfqpoint{3.186939in}{2.169387in}}%
\pgfpathcurveto{\pgfqpoint{3.191057in}{2.165269in}}{\pgfqpoint{3.196644in}{2.162955in}}{\pgfqpoint{3.202468in}{2.162955in}}%
\pgfpathlineto{\pgfqpoint{3.202468in}{2.162955in}}%
\pgfpathclose%
\pgfusepath{stroke,fill}%
\end{pgfscope}%
\begin{pgfscope}%
\pgfpathrectangle{\pgfqpoint{0.100000in}{0.183744in}}{\pgfqpoint{4.506048in}{4.506048in}}%
\pgfusepath{clip}%
\pgfsetbuttcap%
\pgfsetroundjoin%
\definecolor{currentfill}{rgb}{0.000000,0.000000,1.000000}%
\pgfsetfillcolor{currentfill}%
\pgfsetfillopacity{0.700000}%
\pgfsetlinewidth{1.003750pt}%
\definecolor{currentstroke}{rgb}{0.000000,0.000000,1.000000}%
\pgfsetstrokecolor{currentstroke}%
\pgfsetstrokeopacity{0.700000}%
\pgfsetdash{}{0pt}%
\pgfpathmoveto{\pgfqpoint{2.918257in}{1.739637in}}%
\pgfpathcurveto{\pgfqpoint{2.924081in}{1.739637in}}{\pgfqpoint{2.929667in}{1.741951in}}{\pgfqpoint{2.933785in}{1.746069in}}%
\pgfpathcurveto{\pgfqpoint{2.937903in}{1.750187in}}{\pgfqpoint{2.940217in}{1.755773in}}{\pgfqpoint{2.940217in}{1.761597in}}%
\pgfpathcurveto{\pgfqpoint{2.940217in}{1.767421in}}{\pgfqpoint{2.937903in}{1.773007in}}{\pgfqpoint{2.933785in}{1.777125in}}%
\pgfpathcurveto{\pgfqpoint{2.929667in}{1.781243in}}{\pgfqpoint{2.924081in}{1.783557in}}{\pgfqpoint{2.918257in}{1.783557in}}%
\pgfpathcurveto{\pgfqpoint{2.912433in}{1.783557in}}{\pgfqpoint{2.906847in}{1.781243in}}{\pgfqpoint{2.902728in}{1.777125in}}%
\pgfpathcurveto{\pgfqpoint{2.898610in}{1.773007in}}{\pgfqpoint{2.896296in}{1.767421in}}{\pgfqpoint{2.896296in}{1.761597in}}%
\pgfpathcurveto{\pgfqpoint{2.896296in}{1.755773in}}{\pgfqpoint{2.898610in}{1.750187in}}{\pgfqpoint{2.902728in}{1.746069in}}%
\pgfpathcurveto{\pgfqpoint{2.906847in}{1.741951in}}{\pgfqpoint{2.912433in}{1.739637in}}{\pgfqpoint{2.918257in}{1.739637in}}%
\pgfpathlineto{\pgfqpoint{2.918257in}{1.739637in}}%
\pgfpathclose%
\pgfusepath{stroke,fill}%
\end{pgfscope}%
\begin{pgfscope}%
\pgfpathrectangle{\pgfqpoint{0.100000in}{0.183744in}}{\pgfqpoint{4.506048in}{4.506048in}}%
\pgfusepath{clip}%
\pgfsetbuttcap%
\pgfsetroundjoin%
\definecolor{currentfill}{rgb}{0.000000,0.000000,1.000000}%
\pgfsetfillcolor{currentfill}%
\pgfsetfillopacity{0.700000}%
\pgfsetlinewidth{1.003750pt}%
\definecolor{currentstroke}{rgb}{0.000000,0.000000,1.000000}%
\pgfsetstrokecolor{currentstroke}%
\pgfsetstrokeopacity{0.700000}%
\pgfsetdash{}{0pt}%
\pgfpathmoveto{\pgfqpoint{2.975957in}{1.576244in}}%
\pgfpathcurveto{\pgfqpoint{2.981781in}{1.576244in}}{\pgfqpoint{2.987367in}{1.578557in}}{\pgfqpoint{2.991486in}{1.582676in}}%
\pgfpathcurveto{\pgfqpoint{2.995604in}{1.586794in}}{\pgfqpoint{2.997918in}{1.592380in}}{\pgfqpoint{2.997918in}{1.598204in}}%
\pgfpathcurveto{\pgfqpoint{2.997918in}{1.604028in}}{\pgfqpoint{2.995604in}{1.609614in}}{\pgfqpoint{2.991486in}{1.613732in}}%
\pgfpathcurveto{\pgfqpoint{2.987367in}{1.617850in}}{\pgfqpoint{2.981781in}{1.620164in}}{\pgfqpoint{2.975957in}{1.620164in}}%
\pgfpathcurveto{\pgfqpoint{2.970133in}{1.620164in}}{\pgfqpoint{2.964547in}{1.617850in}}{\pgfqpoint{2.960429in}{1.613732in}}%
\pgfpathcurveto{\pgfqpoint{2.956311in}{1.609614in}}{\pgfqpoint{2.953997in}{1.604028in}}{\pgfqpoint{2.953997in}{1.598204in}}%
\pgfpathcurveto{\pgfqpoint{2.953997in}{1.592380in}}{\pgfqpoint{2.956311in}{1.586794in}}{\pgfqpoint{2.960429in}{1.582676in}}%
\pgfpathcurveto{\pgfqpoint{2.964547in}{1.578557in}}{\pgfqpoint{2.970133in}{1.576244in}}{\pgfqpoint{2.975957in}{1.576244in}}%
\pgfpathlineto{\pgfqpoint{2.975957in}{1.576244in}}%
\pgfpathclose%
\pgfusepath{stroke,fill}%
\end{pgfscope}%
\begin{pgfscope}%
\pgfpathrectangle{\pgfqpoint{0.100000in}{0.183744in}}{\pgfqpoint{4.506048in}{4.506048in}}%
\pgfusepath{clip}%
\pgfsetbuttcap%
\pgfsetroundjoin%
\definecolor{currentfill}{rgb}{0.000000,0.000000,1.000000}%
\pgfsetfillcolor{currentfill}%
\pgfsetfillopacity{0.700000}%
\pgfsetlinewidth{1.003750pt}%
\definecolor{currentstroke}{rgb}{0.000000,0.000000,1.000000}%
\pgfsetstrokecolor{currentstroke}%
\pgfsetstrokeopacity{0.700000}%
\pgfsetdash{}{0pt}%
\pgfpathmoveto{\pgfqpoint{2.965991in}{2.538532in}}%
\pgfpathcurveto{\pgfqpoint{2.971815in}{2.538532in}}{\pgfqpoint{2.977401in}{2.540846in}}{\pgfqpoint{2.981520in}{2.544964in}}%
\pgfpathcurveto{\pgfqpoint{2.985638in}{2.549082in}}{\pgfqpoint{2.987952in}{2.554668in}}{\pgfqpoint{2.987952in}{2.560492in}}%
\pgfpathcurveto{\pgfqpoint{2.987952in}{2.566316in}}{\pgfqpoint{2.985638in}{2.571902in}}{\pgfqpoint{2.981520in}{2.576020in}}%
\pgfpathcurveto{\pgfqpoint{2.977401in}{2.580138in}}{\pgfqpoint{2.971815in}{2.582452in}}{\pgfqpoint{2.965991in}{2.582452in}}%
\pgfpathcurveto{\pgfqpoint{2.960167in}{2.582452in}}{\pgfqpoint{2.954581in}{2.580138in}}{\pgfqpoint{2.950463in}{2.576020in}}%
\pgfpathcurveto{\pgfqpoint{2.946345in}{2.571902in}}{\pgfqpoint{2.944031in}{2.566316in}}{\pgfqpoint{2.944031in}{2.560492in}}%
\pgfpathcurveto{\pgfqpoint{2.944031in}{2.554668in}}{\pgfqpoint{2.946345in}{2.549082in}}{\pgfqpoint{2.950463in}{2.544964in}}%
\pgfpathcurveto{\pgfqpoint{2.954581in}{2.540846in}}{\pgfqpoint{2.960167in}{2.538532in}}{\pgfqpoint{2.965991in}{2.538532in}}%
\pgfpathlineto{\pgfqpoint{2.965991in}{2.538532in}}%
\pgfpathclose%
\pgfusepath{stroke,fill}%
\end{pgfscope}%
\begin{pgfscope}%
\pgfpathrectangle{\pgfqpoint{0.100000in}{0.183744in}}{\pgfqpoint{4.506048in}{4.506048in}}%
\pgfusepath{clip}%
\pgfsetbuttcap%
\pgfsetroundjoin%
\definecolor{currentfill}{rgb}{0.000000,0.000000,1.000000}%
\pgfsetfillcolor{currentfill}%
\pgfsetfillopacity{0.700000}%
\pgfsetlinewidth{1.003750pt}%
\definecolor{currentstroke}{rgb}{0.000000,0.000000,1.000000}%
\pgfsetstrokecolor{currentstroke}%
\pgfsetstrokeopacity{0.700000}%
\pgfsetdash{}{0pt}%
\pgfpathmoveto{\pgfqpoint{1.570571in}{2.655774in}}%
\pgfpathcurveto{\pgfqpoint{1.576395in}{2.655774in}}{\pgfqpoint{1.581981in}{2.658088in}}{\pgfqpoint{1.586099in}{2.662206in}}%
\pgfpathcurveto{\pgfqpoint{1.590217in}{2.666324in}}{\pgfqpoint{1.592531in}{2.671910in}}{\pgfqpoint{1.592531in}{2.677734in}}%
\pgfpathcurveto{\pgfqpoint{1.592531in}{2.683558in}}{\pgfqpoint{1.590217in}{2.689145in}}{\pgfqpoint{1.586099in}{2.693263in}}%
\pgfpathcurveto{\pgfqpoint{1.581981in}{2.697381in}}{\pgfqpoint{1.576395in}{2.699695in}}{\pgfqpoint{1.570571in}{2.699695in}}%
\pgfpathcurveto{\pgfqpoint{1.564747in}{2.699695in}}{\pgfqpoint{1.559161in}{2.697381in}}{\pgfqpoint{1.555042in}{2.693263in}}%
\pgfpathcurveto{\pgfqpoint{1.550924in}{2.689145in}}{\pgfqpoint{1.548610in}{2.683558in}}{\pgfqpoint{1.548610in}{2.677734in}}%
\pgfpathcurveto{\pgfqpoint{1.548610in}{2.671910in}}{\pgfqpoint{1.550924in}{2.666324in}}{\pgfqpoint{1.555042in}{2.662206in}}%
\pgfpathcurveto{\pgfqpoint{1.559161in}{2.658088in}}{\pgfqpoint{1.564747in}{2.655774in}}{\pgfqpoint{1.570571in}{2.655774in}}%
\pgfpathlineto{\pgfqpoint{1.570571in}{2.655774in}}%
\pgfpathclose%
\pgfusepath{stroke,fill}%
\end{pgfscope}%
\begin{pgfscope}%
\pgfpathrectangle{\pgfqpoint{0.100000in}{0.183744in}}{\pgfqpoint{4.506048in}{4.506048in}}%
\pgfusepath{clip}%
\pgfsetbuttcap%
\pgfsetroundjoin%
\definecolor{currentfill}{rgb}{0.000000,0.000000,1.000000}%
\pgfsetfillcolor{currentfill}%
\pgfsetfillopacity{0.700000}%
\pgfsetlinewidth{1.003750pt}%
\definecolor{currentstroke}{rgb}{0.000000,0.000000,1.000000}%
\pgfsetstrokecolor{currentstroke}%
\pgfsetstrokeopacity{0.700000}%
\pgfsetdash{}{0pt}%
\pgfpathmoveto{\pgfqpoint{3.497564in}{3.188977in}}%
\pgfpathcurveto{\pgfqpoint{3.503388in}{3.188977in}}{\pgfqpoint{3.508974in}{3.191291in}}{\pgfqpoint{3.513092in}{3.195409in}}%
\pgfpathcurveto{\pgfqpoint{3.517211in}{3.199527in}}{\pgfqpoint{3.519524in}{3.205114in}}{\pgfqpoint{3.519524in}{3.210938in}}%
\pgfpathcurveto{\pgfqpoint{3.519524in}{3.216761in}}{\pgfqpoint{3.517211in}{3.222348in}}{\pgfqpoint{3.513092in}{3.226466in}}%
\pgfpathcurveto{\pgfqpoint{3.508974in}{3.230584in}}{\pgfqpoint{3.503388in}{3.232898in}}{\pgfqpoint{3.497564in}{3.232898in}}%
\pgfpathcurveto{\pgfqpoint{3.491740in}{3.232898in}}{\pgfqpoint{3.486154in}{3.230584in}}{\pgfqpoint{3.482036in}{3.226466in}}%
\pgfpathcurveto{\pgfqpoint{3.477918in}{3.222348in}}{\pgfqpoint{3.475604in}{3.216761in}}{\pgfqpoint{3.475604in}{3.210938in}}%
\pgfpathcurveto{\pgfqpoint{3.475604in}{3.205114in}}{\pgfqpoint{3.477918in}{3.199527in}}{\pgfqpoint{3.482036in}{3.195409in}}%
\pgfpathcurveto{\pgfqpoint{3.486154in}{3.191291in}}{\pgfqpoint{3.491740in}{3.188977in}}{\pgfqpoint{3.497564in}{3.188977in}}%
\pgfpathlineto{\pgfqpoint{3.497564in}{3.188977in}}%
\pgfpathclose%
\pgfusepath{stroke,fill}%
\end{pgfscope}%
\begin{pgfscope}%
\pgfpathrectangle{\pgfqpoint{0.100000in}{0.183744in}}{\pgfqpoint{4.506048in}{4.506048in}}%
\pgfusepath{clip}%
\pgfsetbuttcap%
\pgfsetroundjoin%
\definecolor{currentfill}{rgb}{0.000000,0.000000,1.000000}%
\pgfsetfillcolor{currentfill}%
\pgfsetfillopacity{0.700000}%
\pgfsetlinewidth{1.003750pt}%
\definecolor{currentstroke}{rgb}{0.000000,0.000000,1.000000}%
\pgfsetstrokecolor{currentstroke}%
\pgfsetstrokeopacity{0.700000}%
\pgfsetdash{}{0pt}%
\pgfpathmoveto{\pgfqpoint{1.746629in}{2.131480in}}%
\pgfpathcurveto{\pgfqpoint{1.752453in}{2.131480in}}{\pgfqpoint{1.758039in}{2.133794in}}{\pgfqpoint{1.762157in}{2.137912in}}%
\pgfpathcurveto{\pgfqpoint{1.766275in}{2.142030in}}{\pgfqpoint{1.768589in}{2.147616in}}{\pgfqpoint{1.768589in}{2.153440in}}%
\pgfpathcurveto{\pgfqpoint{1.768589in}{2.159264in}}{\pgfqpoint{1.766275in}{2.164850in}}{\pgfqpoint{1.762157in}{2.168969in}}%
\pgfpathcurveto{\pgfqpoint{1.758039in}{2.173087in}}{\pgfqpoint{1.752453in}{2.175401in}}{\pgfqpoint{1.746629in}{2.175401in}}%
\pgfpathcurveto{\pgfqpoint{1.740805in}{2.175401in}}{\pgfqpoint{1.735219in}{2.173087in}}{\pgfqpoint{1.731100in}{2.168969in}}%
\pgfpathcurveto{\pgfqpoint{1.726982in}{2.164850in}}{\pgfqpoint{1.724668in}{2.159264in}}{\pgfqpoint{1.724668in}{2.153440in}}%
\pgfpathcurveto{\pgfqpoint{1.724668in}{2.147616in}}{\pgfqpoint{1.726982in}{2.142030in}}{\pgfqpoint{1.731100in}{2.137912in}}%
\pgfpathcurveto{\pgfqpoint{1.735219in}{2.133794in}}{\pgfqpoint{1.740805in}{2.131480in}}{\pgfqpoint{1.746629in}{2.131480in}}%
\pgfpathlineto{\pgfqpoint{1.746629in}{2.131480in}}%
\pgfpathclose%
\pgfusepath{stroke,fill}%
\end{pgfscope}%
\begin{pgfscope}%
\pgfpathrectangle{\pgfqpoint{0.100000in}{0.183744in}}{\pgfqpoint{4.506048in}{4.506048in}}%
\pgfusepath{clip}%
\pgfsetbuttcap%
\pgfsetroundjoin%
\definecolor{currentfill}{rgb}{0.000000,0.000000,1.000000}%
\pgfsetfillcolor{currentfill}%
\pgfsetfillopacity{0.700000}%
\pgfsetlinewidth{1.003750pt}%
\definecolor{currentstroke}{rgb}{0.000000,0.000000,1.000000}%
\pgfsetstrokecolor{currentstroke}%
\pgfsetstrokeopacity{0.700000}%
\pgfsetdash{}{0pt}%
\pgfpathmoveto{\pgfqpoint{1.866485in}{1.261589in}}%
\pgfpathcurveto{\pgfqpoint{1.872309in}{1.261589in}}{\pgfqpoint{1.877895in}{1.263903in}}{\pgfqpoint{1.882014in}{1.268021in}}%
\pgfpathcurveto{\pgfqpoint{1.886132in}{1.272139in}}{\pgfqpoint{1.888446in}{1.277725in}}{\pgfqpoint{1.888446in}{1.283549in}}%
\pgfpathcurveto{\pgfqpoint{1.888446in}{1.289373in}}{\pgfqpoint{1.886132in}{1.294959in}}{\pgfqpoint{1.882014in}{1.299077in}}%
\pgfpathcurveto{\pgfqpoint{1.877895in}{1.303196in}}{\pgfqpoint{1.872309in}{1.305509in}}{\pgfqpoint{1.866485in}{1.305509in}}%
\pgfpathcurveto{\pgfqpoint{1.860661in}{1.305509in}}{\pgfqpoint{1.855075in}{1.303196in}}{\pgfqpoint{1.850957in}{1.299077in}}%
\pgfpathcurveto{\pgfqpoint{1.846839in}{1.294959in}}{\pgfqpoint{1.844525in}{1.289373in}}{\pgfqpoint{1.844525in}{1.283549in}}%
\pgfpathcurveto{\pgfqpoint{1.844525in}{1.277725in}}{\pgfqpoint{1.846839in}{1.272139in}}{\pgfqpoint{1.850957in}{1.268021in}}%
\pgfpathcurveto{\pgfqpoint{1.855075in}{1.263903in}}{\pgfqpoint{1.860661in}{1.261589in}}{\pgfqpoint{1.866485in}{1.261589in}}%
\pgfpathlineto{\pgfqpoint{1.866485in}{1.261589in}}%
\pgfpathclose%
\pgfusepath{stroke,fill}%
\end{pgfscope}%
\begin{pgfscope}%
\pgfpathrectangle{\pgfqpoint{0.100000in}{0.183744in}}{\pgfqpoint{4.506048in}{4.506048in}}%
\pgfusepath{clip}%
\pgfsetbuttcap%
\pgfsetroundjoin%
\definecolor{currentfill}{rgb}{0.000000,0.000000,1.000000}%
\pgfsetfillcolor{currentfill}%
\pgfsetfillopacity{0.700000}%
\pgfsetlinewidth{1.003750pt}%
\definecolor{currentstroke}{rgb}{0.000000,0.000000,1.000000}%
\pgfsetstrokecolor{currentstroke}%
\pgfsetstrokeopacity{0.700000}%
\pgfsetdash{}{0pt}%
\pgfpathmoveto{\pgfqpoint{1.828730in}{3.707606in}}%
\pgfpathcurveto{\pgfqpoint{1.834554in}{3.707606in}}{\pgfqpoint{1.840140in}{3.709920in}}{\pgfqpoint{1.844258in}{3.714038in}}%
\pgfpathcurveto{\pgfqpoint{1.848376in}{3.718157in}}{\pgfqpoint{1.850690in}{3.723743in}}{\pgfqpoint{1.850690in}{3.729567in}}%
\pgfpathcurveto{\pgfqpoint{1.850690in}{3.735391in}}{\pgfqpoint{1.848376in}{3.740977in}}{\pgfqpoint{1.844258in}{3.745095in}}%
\pgfpathcurveto{\pgfqpoint{1.840140in}{3.749213in}}{\pgfqpoint{1.834554in}{3.751527in}}{\pgfqpoint{1.828730in}{3.751527in}}%
\pgfpathcurveto{\pgfqpoint{1.822906in}{3.751527in}}{\pgfqpoint{1.817320in}{3.749213in}}{\pgfqpoint{1.813202in}{3.745095in}}%
\pgfpathcurveto{\pgfqpoint{1.809084in}{3.740977in}}{\pgfqpoint{1.806770in}{3.735391in}}{\pgfqpoint{1.806770in}{3.729567in}}%
\pgfpathcurveto{\pgfqpoint{1.806770in}{3.723743in}}{\pgfqpoint{1.809084in}{3.718157in}}{\pgfqpoint{1.813202in}{3.714038in}}%
\pgfpathcurveto{\pgfqpoint{1.817320in}{3.709920in}}{\pgfqpoint{1.822906in}{3.707606in}}{\pgfqpoint{1.828730in}{3.707606in}}%
\pgfpathlineto{\pgfqpoint{1.828730in}{3.707606in}}%
\pgfpathclose%
\pgfusepath{stroke,fill}%
\end{pgfscope}%
\begin{pgfscope}%
\pgfpathrectangle{\pgfqpoint{0.100000in}{0.183744in}}{\pgfqpoint{4.506048in}{4.506048in}}%
\pgfusepath{clip}%
\pgfsetbuttcap%
\pgfsetroundjoin%
\definecolor{currentfill}{rgb}{0.000000,0.000000,1.000000}%
\pgfsetfillcolor{currentfill}%
\pgfsetfillopacity{0.700000}%
\pgfsetlinewidth{1.003750pt}%
\definecolor{currentstroke}{rgb}{0.000000,0.000000,1.000000}%
\pgfsetstrokecolor{currentstroke}%
\pgfsetstrokeopacity{0.700000}%
\pgfsetdash{}{0pt}%
\pgfpathmoveto{\pgfqpoint{2.456909in}{1.491627in}}%
\pgfpathcurveto{\pgfqpoint{2.462733in}{1.491627in}}{\pgfqpoint{2.468319in}{1.493941in}}{\pgfqpoint{2.472437in}{1.498059in}}%
\pgfpathcurveto{\pgfqpoint{2.476555in}{1.502177in}}{\pgfqpoint{2.478869in}{1.507763in}}{\pgfqpoint{2.478869in}{1.513587in}}%
\pgfpathcurveto{\pgfqpoint{2.478869in}{1.519411in}}{\pgfqpoint{2.476555in}{1.524997in}}{\pgfqpoint{2.472437in}{1.529115in}}%
\pgfpathcurveto{\pgfqpoint{2.468319in}{1.533234in}}{\pgfqpoint{2.462733in}{1.535547in}}{\pgfqpoint{2.456909in}{1.535547in}}%
\pgfpathcurveto{\pgfqpoint{2.451085in}{1.535547in}}{\pgfqpoint{2.445499in}{1.533234in}}{\pgfqpoint{2.441381in}{1.529115in}}%
\pgfpathcurveto{\pgfqpoint{2.437263in}{1.524997in}}{\pgfqpoint{2.434949in}{1.519411in}}{\pgfqpoint{2.434949in}{1.513587in}}%
\pgfpathcurveto{\pgfqpoint{2.434949in}{1.507763in}}{\pgfqpoint{2.437263in}{1.502177in}}{\pgfqpoint{2.441381in}{1.498059in}}%
\pgfpathcurveto{\pgfqpoint{2.445499in}{1.493941in}}{\pgfqpoint{2.451085in}{1.491627in}}{\pgfqpoint{2.456909in}{1.491627in}}%
\pgfpathlineto{\pgfqpoint{2.456909in}{1.491627in}}%
\pgfpathclose%
\pgfusepath{stroke,fill}%
\end{pgfscope}%
\begin{pgfscope}%
\pgfpathrectangle{\pgfqpoint{0.100000in}{0.183744in}}{\pgfqpoint{4.506048in}{4.506048in}}%
\pgfusepath{clip}%
\pgfsetbuttcap%
\pgfsetroundjoin%
\definecolor{currentfill}{rgb}{0.000000,0.000000,1.000000}%
\pgfsetfillcolor{currentfill}%
\pgfsetfillopacity{0.700000}%
\pgfsetlinewidth{1.003750pt}%
\definecolor{currentstroke}{rgb}{0.000000,0.000000,1.000000}%
\pgfsetstrokecolor{currentstroke}%
\pgfsetstrokeopacity{0.700000}%
\pgfsetdash{}{0pt}%
\pgfpathmoveto{\pgfqpoint{3.859487in}{2.413726in}}%
\pgfpathcurveto{\pgfqpoint{3.865311in}{2.413726in}}{\pgfqpoint{3.870897in}{2.416040in}}{\pgfqpoint{3.875015in}{2.420158in}}%
\pgfpathcurveto{\pgfqpoint{3.879133in}{2.424276in}}{\pgfqpoint{3.881447in}{2.429863in}}{\pgfqpoint{3.881447in}{2.435686in}}%
\pgfpathcurveto{\pgfqpoint{3.881447in}{2.441510in}}{\pgfqpoint{3.879133in}{2.447097in}}{\pgfqpoint{3.875015in}{2.451215in}}%
\pgfpathcurveto{\pgfqpoint{3.870897in}{2.455333in}}{\pgfqpoint{3.865311in}{2.457647in}}{\pgfqpoint{3.859487in}{2.457647in}}%
\pgfpathcurveto{\pgfqpoint{3.853663in}{2.457647in}}{\pgfqpoint{3.848077in}{2.455333in}}{\pgfqpoint{3.843958in}{2.451215in}}%
\pgfpathcurveto{\pgfqpoint{3.839840in}{2.447097in}}{\pgfqpoint{3.837526in}{2.441510in}}{\pgfqpoint{3.837526in}{2.435686in}}%
\pgfpathcurveto{\pgfqpoint{3.837526in}{2.429863in}}{\pgfqpoint{3.839840in}{2.424276in}}{\pgfqpoint{3.843958in}{2.420158in}}%
\pgfpathcurveto{\pgfqpoint{3.848077in}{2.416040in}}{\pgfqpoint{3.853663in}{2.413726in}}{\pgfqpoint{3.859487in}{2.413726in}}%
\pgfpathlineto{\pgfqpoint{3.859487in}{2.413726in}}%
\pgfpathclose%
\pgfusepath{stroke,fill}%
\end{pgfscope}%
\begin{pgfscope}%
\pgfpathrectangle{\pgfqpoint{0.100000in}{0.183744in}}{\pgfqpoint{4.506048in}{4.506048in}}%
\pgfusepath{clip}%
\pgfsetbuttcap%
\pgfsetroundjoin%
\definecolor{currentfill}{rgb}{0.000000,0.000000,1.000000}%
\pgfsetfillcolor{currentfill}%
\pgfsetfillopacity{0.700000}%
\pgfsetlinewidth{1.003750pt}%
\definecolor{currentstroke}{rgb}{0.000000,0.000000,1.000000}%
\pgfsetstrokecolor{currentstroke}%
\pgfsetstrokeopacity{0.700000}%
\pgfsetdash{}{0pt}%
\pgfpathmoveto{\pgfqpoint{3.220557in}{3.233276in}}%
\pgfpathcurveto{\pgfqpoint{3.226381in}{3.233276in}}{\pgfqpoint{3.231967in}{3.235590in}}{\pgfqpoint{3.236086in}{3.239708in}}%
\pgfpathcurveto{\pgfqpoint{3.240204in}{3.243826in}}{\pgfqpoint{3.242518in}{3.249412in}}{\pgfqpoint{3.242518in}{3.255236in}}%
\pgfpathcurveto{\pgfqpoint{3.242518in}{3.261060in}}{\pgfqpoint{3.240204in}{3.266646in}}{\pgfqpoint{3.236086in}{3.270765in}}%
\pgfpathcurveto{\pgfqpoint{3.231967in}{3.274883in}}{\pgfqpoint{3.226381in}{3.277197in}}{\pgfqpoint{3.220557in}{3.277197in}}%
\pgfpathcurveto{\pgfqpoint{3.214733in}{3.277197in}}{\pgfqpoint{3.209147in}{3.274883in}}{\pgfqpoint{3.205029in}{3.270765in}}%
\pgfpathcurveto{\pgfqpoint{3.200911in}{3.266646in}}{\pgfqpoint{3.198597in}{3.261060in}}{\pgfqpoint{3.198597in}{3.255236in}}%
\pgfpathcurveto{\pgfqpoint{3.198597in}{3.249412in}}{\pgfqpoint{3.200911in}{3.243826in}}{\pgfqpoint{3.205029in}{3.239708in}}%
\pgfpathcurveto{\pgfqpoint{3.209147in}{3.235590in}}{\pgfqpoint{3.214733in}{3.233276in}}{\pgfqpoint{3.220557in}{3.233276in}}%
\pgfpathlineto{\pgfqpoint{3.220557in}{3.233276in}}%
\pgfpathclose%
\pgfusepath{stroke,fill}%
\end{pgfscope}%
\begin{pgfscope}%
\pgfpathrectangle{\pgfqpoint{0.100000in}{0.183744in}}{\pgfqpoint{4.506048in}{4.506048in}}%
\pgfusepath{clip}%
\pgfsetbuttcap%
\pgfsetroundjoin%
\definecolor{currentfill}{rgb}{0.000000,0.000000,1.000000}%
\pgfsetfillcolor{currentfill}%
\pgfsetfillopacity{0.700000}%
\pgfsetlinewidth{1.003750pt}%
\definecolor{currentstroke}{rgb}{0.000000,0.000000,1.000000}%
\pgfsetstrokecolor{currentstroke}%
\pgfsetstrokeopacity{0.700000}%
\pgfsetdash{}{0pt}%
\pgfpathmoveto{\pgfqpoint{3.827416in}{1.538899in}}%
\pgfpathcurveto{\pgfqpoint{3.833240in}{1.538899in}}{\pgfqpoint{3.838827in}{1.541213in}}{\pgfqpoint{3.842945in}{1.545331in}}%
\pgfpathcurveto{\pgfqpoint{3.847063in}{1.549449in}}{\pgfqpoint{3.849377in}{1.555035in}}{\pgfqpoint{3.849377in}{1.560859in}}%
\pgfpathcurveto{\pgfqpoint{3.849377in}{1.566683in}}{\pgfqpoint{3.847063in}{1.572269in}}{\pgfqpoint{3.842945in}{1.576387in}}%
\pgfpathcurveto{\pgfqpoint{3.838827in}{1.580506in}}{\pgfqpoint{3.833240in}{1.582819in}}{\pgfqpoint{3.827416in}{1.582819in}}%
\pgfpathcurveto{\pgfqpoint{3.821593in}{1.582819in}}{\pgfqpoint{3.816006in}{1.580506in}}{\pgfqpoint{3.811888in}{1.576387in}}%
\pgfpathcurveto{\pgfqpoint{3.807770in}{1.572269in}}{\pgfqpoint{3.805456in}{1.566683in}}{\pgfqpoint{3.805456in}{1.560859in}}%
\pgfpathcurveto{\pgfqpoint{3.805456in}{1.555035in}}{\pgfqpoint{3.807770in}{1.549449in}}{\pgfqpoint{3.811888in}{1.545331in}}%
\pgfpathcurveto{\pgfqpoint{3.816006in}{1.541213in}}{\pgfqpoint{3.821593in}{1.538899in}}{\pgfqpoint{3.827416in}{1.538899in}}%
\pgfpathlineto{\pgfqpoint{3.827416in}{1.538899in}}%
\pgfpathclose%
\pgfusepath{stroke,fill}%
\end{pgfscope}%
\begin{pgfscope}%
\pgfpathrectangle{\pgfqpoint{0.100000in}{0.183744in}}{\pgfqpoint{4.506048in}{4.506048in}}%
\pgfusepath{clip}%
\pgfsetbuttcap%
\pgfsetroundjoin%
\definecolor{currentfill}{rgb}{0.000000,0.000000,1.000000}%
\pgfsetfillcolor{currentfill}%
\pgfsetfillopacity{0.700000}%
\pgfsetlinewidth{1.003750pt}%
\definecolor{currentstroke}{rgb}{0.000000,0.000000,1.000000}%
\pgfsetstrokecolor{currentstroke}%
\pgfsetstrokeopacity{0.700000}%
\pgfsetdash{}{0pt}%
\pgfpathmoveto{\pgfqpoint{1.215870in}{3.217396in}}%
\pgfpathcurveto{\pgfqpoint{1.221694in}{3.217396in}}{\pgfqpoint{1.227280in}{3.219710in}}{\pgfqpoint{1.231398in}{3.223828in}}%
\pgfpathcurveto{\pgfqpoint{1.235516in}{3.227946in}}{\pgfqpoint{1.237830in}{3.233532in}}{\pgfqpoint{1.237830in}{3.239356in}}%
\pgfpathcurveto{\pgfqpoint{1.237830in}{3.245180in}}{\pgfqpoint{1.235516in}{3.250766in}}{\pgfqpoint{1.231398in}{3.254884in}}%
\pgfpathcurveto{\pgfqpoint{1.227280in}{3.259002in}}{\pgfqpoint{1.221694in}{3.261316in}}{\pgfqpoint{1.215870in}{3.261316in}}%
\pgfpathcurveto{\pgfqpoint{1.210046in}{3.261316in}}{\pgfqpoint{1.204460in}{3.259002in}}{\pgfqpoint{1.200342in}{3.254884in}}%
\pgfpathcurveto{\pgfqpoint{1.196224in}{3.250766in}}{\pgfqpoint{1.193910in}{3.245180in}}{\pgfqpoint{1.193910in}{3.239356in}}%
\pgfpathcurveto{\pgfqpoint{1.193910in}{3.233532in}}{\pgfqpoint{1.196224in}{3.227946in}}{\pgfqpoint{1.200342in}{3.223828in}}%
\pgfpathcurveto{\pgfqpoint{1.204460in}{3.219710in}}{\pgfqpoint{1.210046in}{3.217396in}}{\pgfqpoint{1.215870in}{3.217396in}}%
\pgfpathlineto{\pgfqpoint{1.215870in}{3.217396in}}%
\pgfpathclose%
\pgfusepath{stroke,fill}%
\end{pgfscope}%
\begin{pgfscope}%
\pgfpathrectangle{\pgfqpoint{0.100000in}{0.183744in}}{\pgfqpoint{4.506048in}{4.506048in}}%
\pgfusepath{clip}%
\pgfsetbuttcap%
\pgfsetroundjoin%
\definecolor{currentfill}{rgb}{0.000000,0.000000,1.000000}%
\pgfsetfillcolor{currentfill}%
\pgfsetfillopacity{0.700000}%
\pgfsetlinewidth{1.003750pt}%
\definecolor{currentstroke}{rgb}{0.000000,0.000000,1.000000}%
\pgfsetstrokecolor{currentstroke}%
\pgfsetstrokeopacity{0.700000}%
\pgfsetdash{}{0pt}%
\pgfpathmoveto{\pgfqpoint{2.448206in}{1.268884in}}%
\pgfpathcurveto{\pgfqpoint{2.454030in}{1.268884in}}{\pgfqpoint{2.459616in}{1.271198in}}{\pgfqpoint{2.463735in}{1.275316in}}%
\pgfpathcurveto{\pgfqpoint{2.467853in}{1.279434in}}{\pgfqpoint{2.470167in}{1.285020in}}{\pgfqpoint{2.470167in}{1.290844in}}%
\pgfpathcurveto{\pgfqpoint{2.470167in}{1.296668in}}{\pgfqpoint{2.467853in}{1.302254in}}{\pgfqpoint{2.463735in}{1.306372in}}%
\pgfpathcurveto{\pgfqpoint{2.459616in}{1.310491in}}{\pgfqpoint{2.454030in}{1.312805in}}{\pgfqpoint{2.448206in}{1.312805in}}%
\pgfpathcurveto{\pgfqpoint{2.442382in}{1.312805in}}{\pgfqpoint{2.436796in}{1.310491in}}{\pgfqpoint{2.432678in}{1.306372in}}%
\pgfpathcurveto{\pgfqpoint{2.428560in}{1.302254in}}{\pgfqpoint{2.426246in}{1.296668in}}{\pgfqpoint{2.426246in}{1.290844in}}%
\pgfpathcurveto{\pgfqpoint{2.426246in}{1.285020in}}{\pgfqpoint{2.428560in}{1.279434in}}{\pgfqpoint{2.432678in}{1.275316in}}%
\pgfpathcurveto{\pgfqpoint{2.436796in}{1.271198in}}{\pgfqpoint{2.442382in}{1.268884in}}{\pgfqpoint{2.448206in}{1.268884in}}%
\pgfpathlineto{\pgfqpoint{2.448206in}{1.268884in}}%
\pgfpathclose%
\pgfusepath{stroke,fill}%
\end{pgfscope}%
\begin{pgfscope}%
\pgfpathrectangle{\pgfqpoint{0.100000in}{0.183744in}}{\pgfqpoint{4.506048in}{4.506048in}}%
\pgfusepath{clip}%
\pgfsetbuttcap%
\pgfsetroundjoin%
\definecolor{currentfill}{rgb}{0.000000,0.000000,1.000000}%
\pgfsetfillcolor{currentfill}%
\pgfsetfillopacity{0.700000}%
\pgfsetlinewidth{1.003750pt}%
\definecolor{currentstroke}{rgb}{0.000000,0.000000,1.000000}%
\pgfsetstrokecolor{currentstroke}%
\pgfsetstrokeopacity{0.700000}%
\pgfsetdash{}{0pt}%
\pgfpathmoveto{\pgfqpoint{3.806294in}{2.089935in}}%
\pgfpathcurveto{\pgfqpoint{3.812118in}{2.089935in}}{\pgfqpoint{3.817704in}{2.092249in}}{\pgfqpoint{3.821822in}{2.096367in}}%
\pgfpathcurveto{\pgfqpoint{3.825940in}{2.100485in}}{\pgfqpoint{3.828254in}{2.106072in}}{\pgfqpoint{3.828254in}{2.111896in}}%
\pgfpathcurveto{\pgfqpoint{3.828254in}{2.117719in}}{\pgfqpoint{3.825940in}{2.123306in}}{\pgfqpoint{3.821822in}{2.127424in}}%
\pgfpathcurveto{\pgfqpoint{3.817704in}{2.131542in}}{\pgfqpoint{3.812118in}{2.133856in}}{\pgfqpoint{3.806294in}{2.133856in}}%
\pgfpathcurveto{\pgfqpoint{3.800470in}{2.133856in}}{\pgfqpoint{3.794884in}{2.131542in}}{\pgfqpoint{3.790766in}{2.127424in}}%
\pgfpathcurveto{\pgfqpoint{3.786647in}{2.123306in}}{\pgfqpoint{3.784334in}{2.117719in}}{\pgfqpoint{3.784334in}{2.111896in}}%
\pgfpathcurveto{\pgfqpoint{3.784334in}{2.106072in}}{\pgfqpoint{3.786647in}{2.100485in}}{\pgfqpoint{3.790766in}{2.096367in}}%
\pgfpathcurveto{\pgfqpoint{3.794884in}{2.092249in}}{\pgfqpoint{3.800470in}{2.089935in}}{\pgfqpoint{3.806294in}{2.089935in}}%
\pgfpathlineto{\pgfqpoint{3.806294in}{2.089935in}}%
\pgfpathclose%
\pgfusepath{stroke,fill}%
\end{pgfscope}%
\begin{pgfscope}%
\pgfpathrectangle{\pgfqpoint{0.100000in}{0.183744in}}{\pgfqpoint{4.506048in}{4.506048in}}%
\pgfusepath{clip}%
\pgfsetbuttcap%
\pgfsetroundjoin%
\definecolor{currentfill}{rgb}{0.000000,0.000000,1.000000}%
\pgfsetfillcolor{currentfill}%
\pgfsetfillopacity{0.700000}%
\pgfsetlinewidth{1.003750pt}%
\definecolor{currentstroke}{rgb}{0.000000,0.000000,1.000000}%
\pgfsetstrokecolor{currentstroke}%
\pgfsetstrokeopacity{0.700000}%
\pgfsetdash{}{0pt}%
\pgfpathmoveto{\pgfqpoint{1.829814in}{3.470643in}}%
\pgfpathcurveto{\pgfqpoint{1.835638in}{3.470643in}}{\pgfqpoint{1.841224in}{3.472957in}}{\pgfqpoint{1.845342in}{3.477075in}}%
\pgfpathcurveto{\pgfqpoint{1.849460in}{3.481193in}}{\pgfqpoint{1.851774in}{3.486779in}}{\pgfqpoint{1.851774in}{3.492603in}}%
\pgfpathcurveto{\pgfqpoint{1.851774in}{3.498427in}}{\pgfqpoint{1.849460in}{3.504013in}}{\pgfqpoint{1.845342in}{3.508131in}}%
\pgfpathcurveto{\pgfqpoint{1.841224in}{3.512250in}}{\pgfqpoint{1.835638in}{3.514563in}}{\pgfqpoint{1.829814in}{3.514563in}}%
\pgfpathcurveto{\pgfqpoint{1.823990in}{3.514563in}}{\pgfqpoint{1.818404in}{3.512250in}}{\pgfqpoint{1.814286in}{3.508131in}}%
\pgfpathcurveto{\pgfqpoint{1.810168in}{3.504013in}}{\pgfqpoint{1.807854in}{3.498427in}}{\pgfqpoint{1.807854in}{3.492603in}}%
\pgfpathcurveto{\pgfqpoint{1.807854in}{3.486779in}}{\pgfqpoint{1.810168in}{3.481193in}}{\pgfqpoint{1.814286in}{3.477075in}}%
\pgfpathcurveto{\pgfqpoint{1.818404in}{3.472957in}}{\pgfqpoint{1.823990in}{3.470643in}}{\pgfqpoint{1.829814in}{3.470643in}}%
\pgfpathlineto{\pgfqpoint{1.829814in}{3.470643in}}%
\pgfpathclose%
\pgfusepath{stroke,fill}%
\end{pgfscope}%
\begin{pgfscope}%
\pgfpathrectangle{\pgfqpoint{0.100000in}{0.183744in}}{\pgfqpoint{4.506048in}{4.506048in}}%
\pgfusepath{clip}%
\pgfsetbuttcap%
\pgfsetroundjoin%
\definecolor{currentfill}{rgb}{0.000000,0.000000,1.000000}%
\pgfsetfillcolor{currentfill}%
\pgfsetfillopacity{0.700000}%
\pgfsetlinewidth{1.003750pt}%
\definecolor{currentstroke}{rgb}{0.000000,0.000000,1.000000}%
\pgfsetstrokecolor{currentstroke}%
\pgfsetstrokeopacity{0.700000}%
\pgfsetdash{}{0pt}%
\pgfpathmoveto{\pgfqpoint{3.496836in}{2.720722in}}%
\pgfpathcurveto{\pgfqpoint{3.502659in}{2.720722in}}{\pgfqpoint{3.508246in}{2.723036in}}{\pgfqpoint{3.512364in}{2.727154in}}%
\pgfpathcurveto{\pgfqpoint{3.516482in}{2.731272in}}{\pgfqpoint{3.518796in}{2.736858in}}{\pgfqpoint{3.518796in}{2.742682in}}%
\pgfpathcurveto{\pgfqpoint{3.518796in}{2.748506in}}{\pgfqpoint{3.516482in}{2.754092in}}{\pgfqpoint{3.512364in}{2.758211in}}%
\pgfpathcurveto{\pgfqpoint{3.508246in}{2.762329in}}{\pgfqpoint{3.502659in}{2.764643in}}{\pgfqpoint{3.496836in}{2.764643in}}%
\pgfpathcurveto{\pgfqpoint{3.491012in}{2.764643in}}{\pgfqpoint{3.485425in}{2.762329in}}{\pgfqpoint{3.481307in}{2.758211in}}%
\pgfpathcurveto{\pgfqpoint{3.477189in}{2.754092in}}{\pgfqpoint{3.474875in}{2.748506in}}{\pgfqpoint{3.474875in}{2.742682in}}%
\pgfpathcurveto{\pgfqpoint{3.474875in}{2.736858in}}{\pgfqpoint{3.477189in}{2.731272in}}{\pgfqpoint{3.481307in}{2.727154in}}%
\pgfpathcurveto{\pgfqpoint{3.485425in}{2.723036in}}{\pgfqpoint{3.491012in}{2.720722in}}{\pgfqpoint{3.496836in}{2.720722in}}%
\pgfpathlineto{\pgfqpoint{3.496836in}{2.720722in}}%
\pgfpathclose%
\pgfusepath{stroke,fill}%
\end{pgfscope}%
\begin{pgfscope}%
\pgfpathrectangle{\pgfqpoint{0.100000in}{0.183744in}}{\pgfqpoint{4.506048in}{4.506048in}}%
\pgfusepath{clip}%
\pgfsetbuttcap%
\pgfsetroundjoin%
\definecolor{currentfill}{rgb}{0.000000,0.000000,1.000000}%
\pgfsetfillcolor{currentfill}%
\pgfsetfillopacity{0.700000}%
\pgfsetlinewidth{1.003750pt}%
\definecolor{currentstroke}{rgb}{0.000000,0.000000,1.000000}%
\pgfsetstrokecolor{currentstroke}%
\pgfsetstrokeopacity{0.700000}%
\pgfsetdash{}{0pt}%
\pgfpathmoveto{\pgfqpoint{3.723464in}{2.993239in}}%
\pgfpathcurveto{\pgfqpoint{3.729288in}{2.993239in}}{\pgfqpoint{3.734874in}{2.995553in}}{\pgfqpoint{3.738992in}{2.999671in}}%
\pgfpathcurveto{\pgfqpoint{3.743111in}{3.003789in}}{\pgfqpoint{3.745424in}{3.009375in}}{\pgfqpoint{3.745424in}{3.015199in}}%
\pgfpathcurveto{\pgfqpoint{3.745424in}{3.021023in}}{\pgfqpoint{3.743111in}{3.026609in}}{\pgfqpoint{3.738992in}{3.030727in}}%
\pgfpathcurveto{\pgfqpoint{3.734874in}{3.034845in}}{\pgfqpoint{3.729288in}{3.037159in}}{\pgfqpoint{3.723464in}{3.037159in}}%
\pgfpathcurveto{\pgfqpoint{3.717640in}{3.037159in}}{\pgfqpoint{3.712054in}{3.034845in}}{\pgfqpoint{3.707936in}{3.030727in}}%
\pgfpathcurveto{\pgfqpoint{3.703818in}{3.026609in}}{\pgfqpoint{3.701504in}{3.021023in}}{\pgfqpoint{3.701504in}{3.015199in}}%
\pgfpathcurveto{\pgfqpoint{3.701504in}{3.009375in}}{\pgfqpoint{3.703818in}{3.003789in}}{\pgfqpoint{3.707936in}{2.999671in}}%
\pgfpathcurveto{\pgfqpoint{3.712054in}{2.995553in}}{\pgfqpoint{3.717640in}{2.993239in}}{\pgfqpoint{3.723464in}{2.993239in}}%
\pgfpathlineto{\pgfqpoint{3.723464in}{2.993239in}}%
\pgfpathclose%
\pgfusepath{stroke,fill}%
\end{pgfscope}%
\begin{pgfscope}%
\pgfpathrectangle{\pgfqpoint{0.100000in}{0.183744in}}{\pgfqpoint{4.506048in}{4.506048in}}%
\pgfusepath{clip}%
\pgfsetbuttcap%
\pgfsetroundjoin%
\definecolor{currentfill}{rgb}{0.000000,0.000000,1.000000}%
\pgfsetfillcolor{currentfill}%
\pgfsetfillopacity{0.700000}%
\pgfsetlinewidth{1.003750pt}%
\definecolor{currentstroke}{rgb}{0.000000,0.000000,1.000000}%
\pgfsetstrokecolor{currentstroke}%
\pgfsetstrokeopacity{0.700000}%
\pgfsetdash{}{0pt}%
\pgfpathmoveto{\pgfqpoint{4.110665in}{2.460161in}}%
\pgfpathcurveto{\pgfqpoint{4.116489in}{2.460161in}}{\pgfqpoint{4.122075in}{2.462475in}}{\pgfqpoint{4.126193in}{2.466593in}}%
\pgfpathcurveto{\pgfqpoint{4.130311in}{2.470711in}}{\pgfqpoint{4.132625in}{2.476297in}}{\pgfqpoint{4.132625in}{2.482121in}}%
\pgfpathcurveto{\pgfqpoint{4.132625in}{2.487945in}}{\pgfqpoint{4.130311in}{2.493531in}}{\pgfqpoint{4.126193in}{2.497649in}}%
\pgfpathcurveto{\pgfqpoint{4.122075in}{2.501767in}}{\pgfqpoint{4.116489in}{2.504081in}}{\pgfqpoint{4.110665in}{2.504081in}}%
\pgfpathcurveto{\pgfqpoint{4.104841in}{2.504081in}}{\pgfqpoint{4.099255in}{2.501767in}}{\pgfqpoint{4.095136in}{2.497649in}}%
\pgfpathcurveto{\pgfqpoint{4.091018in}{2.493531in}}{\pgfqpoint{4.088704in}{2.487945in}}{\pgfqpoint{4.088704in}{2.482121in}}%
\pgfpathcurveto{\pgfqpoint{4.088704in}{2.476297in}}{\pgfqpoint{4.091018in}{2.470711in}}{\pgfqpoint{4.095136in}{2.466593in}}%
\pgfpathcurveto{\pgfqpoint{4.099255in}{2.462475in}}{\pgfqpoint{4.104841in}{2.460161in}}{\pgfqpoint{4.110665in}{2.460161in}}%
\pgfpathlineto{\pgfqpoint{4.110665in}{2.460161in}}%
\pgfpathclose%
\pgfusepath{stroke,fill}%
\end{pgfscope}%
\begin{pgfscope}%
\pgfpathrectangle{\pgfqpoint{0.100000in}{0.183744in}}{\pgfqpoint{4.506048in}{4.506048in}}%
\pgfusepath{clip}%
\pgfsetbuttcap%
\pgfsetroundjoin%
\definecolor{currentfill}{rgb}{0.000000,0.000000,1.000000}%
\pgfsetfillcolor{currentfill}%
\pgfsetfillopacity{0.700000}%
\pgfsetlinewidth{1.003750pt}%
\definecolor{currentstroke}{rgb}{0.000000,0.000000,1.000000}%
\pgfsetstrokecolor{currentstroke}%
\pgfsetstrokeopacity{0.700000}%
\pgfsetdash{}{0pt}%
\pgfpathmoveto{\pgfqpoint{3.293510in}{2.124567in}}%
\pgfpathcurveto{\pgfqpoint{3.299334in}{2.124567in}}{\pgfqpoint{3.304920in}{2.126881in}}{\pgfqpoint{3.309039in}{2.130999in}}%
\pgfpathcurveto{\pgfqpoint{3.313157in}{2.135117in}}{\pgfqpoint{3.315471in}{2.140704in}}{\pgfqpoint{3.315471in}{2.146527in}}%
\pgfpathcurveto{\pgfqpoint{3.315471in}{2.152351in}}{\pgfqpoint{3.313157in}{2.157938in}}{\pgfqpoint{3.309039in}{2.162056in}}%
\pgfpathcurveto{\pgfqpoint{3.304920in}{2.166174in}}{\pgfqpoint{3.299334in}{2.168488in}}{\pgfqpoint{3.293510in}{2.168488in}}%
\pgfpathcurveto{\pgfqpoint{3.287686in}{2.168488in}}{\pgfqpoint{3.282100in}{2.166174in}}{\pgfqpoint{3.277982in}{2.162056in}}%
\pgfpathcurveto{\pgfqpoint{3.273864in}{2.157938in}}{\pgfqpoint{3.271550in}{2.152351in}}{\pgfqpoint{3.271550in}{2.146527in}}%
\pgfpathcurveto{\pgfqpoint{3.271550in}{2.140704in}}{\pgfqpoint{3.273864in}{2.135117in}}{\pgfqpoint{3.277982in}{2.130999in}}%
\pgfpathcurveto{\pgfqpoint{3.282100in}{2.126881in}}{\pgfqpoint{3.287686in}{2.124567in}}{\pgfqpoint{3.293510in}{2.124567in}}%
\pgfpathlineto{\pgfqpoint{3.293510in}{2.124567in}}%
\pgfpathclose%
\pgfusepath{stroke,fill}%
\end{pgfscope}%
\begin{pgfscope}%
\pgfpathrectangle{\pgfqpoint{0.100000in}{0.183744in}}{\pgfqpoint{4.506048in}{4.506048in}}%
\pgfusepath{clip}%
\pgfsetbuttcap%
\pgfsetroundjoin%
\definecolor{currentfill}{rgb}{0.000000,0.000000,1.000000}%
\pgfsetfillcolor{currentfill}%
\pgfsetfillopacity{0.700000}%
\pgfsetlinewidth{1.003750pt}%
\definecolor{currentstroke}{rgb}{0.000000,0.000000,1.000000}%
\pgfsetstrokecolor{currentstroke}%
\pgfsetstrokeopacity{0.700000}%
\pgfsetdash{}{0pt}%
\pgfpathmoveto{\pgfqpoint{0.904599in}{3.011498in}}%
\pgfpathcurveto{\pgfqpoint{0.910423in}{3.011498in}}{\pgfqpoint{0.916009in}{3.013812in}}{\pgfqpoint{0.920127in}{3.017930in}}%
\pgfpathcurveto{\pgfqpoint{0.924245in}{3.022048in}}{\pgfqpoint{0.926559in}{3.027634in}}{\pgfqpoint{0.926559in}{3.033458in}}%
\pgfpathcurveto{\pgfqpoint{0.926559in}{3.039282in}}{\pgfqpoint{0.924245in}{3.044868in}}{\pgfqpoint{0.920127in}{3.048987in}}%
\pgfpathcurveto{\pgfqpoint{0.916009in}{3.053105in}}{\pgfqpoint{0.910423in}{3.055419in}}{\pgfqpoint{0.904599in}{3.055419in}}%
\pgfpathcurveto{\pgfqpoint{0.898775in}{3.055419in}}{\pgfqpoint{0.893189in}{3.053105in}}{\pgfqpoint{0.889071in}{3.048987in}}%
\pgfpathcurveto{\pgfqpoint{0.884953in}{3.044868in}}{\pgfqpoint{0.882639in}{3.039282in}}{\pgfqpoint{0.882639in}{3.033458in}}%
\pgfpathcurveto{\pgfqpoint{0.882639in}{3.027634in}}{\pgfqpoint{0.884953in}{3.022048in}}{\pgfqpoint{0.889071in}{3.017930in}}%
\pgfpathcurveto{\pgfqpoint{0.893189in}{3.013812in}}{\pgfqpoint{0.898775in}{3.011498in}}{\pgfqpoint{0.904599in}{3.011498in}}%
\pgfpathlineto{\pgfqpoint{0.904599in}{3.011498in}}%
\pgfpathclose%
\pgfusepath{stroke,fill}%
\end{pgfscope}%
\begin{pgfscope}%
\pgfpathrectangle{\pgfqpoint{0.100000in}{0.183744in}}{\pgfqpoint{4.506048in}{4.506048in}}%
\pgfusepath{clip}%
\pgfsetbuttcap%
\pgfsetroundjoin%
\definecolor{currentfill}{rgb}{0.000000,0.000000,1.000000}%
\pgfsetfillcolor{currentfill}%
\pgfsetfillopacity{0.700000}%
\pgfsetlinewidth{1.003750pt}%
\definecolor{currentstroke}{rgb}{0.000000,0.000000,1.000000}%
\pgfsetstrokecolor{currentstroke}%
\pgfsetstrokeopacity{0.700000}%
\pgfsetdash{}{0pt}%
\pgfpathmoveto{\pgfqpoint{2.753504in}{1.624052in}}%
\pgfpathcurveto{\pgfqpoint{2.759328in}{1.624052in}}{\pgfqpoint{2.764914in}{1.626366in}}{\pgfqpoint{2.769032in}{1.630484in}}%
\pgfpathcurveto{\pgfqpoint{2.773150in}{1.634603in}}{\pgfqpoint{2.775464in}{1.640189in}}{\pgfqpoint{2.775464in}{1.646013in}}%
\pgfpathcurveto{\pgfqpoint{2.775464in}{1.651837in}}{\pgfqpoint{2.773150in}{1.657423in}}{\pgfqpoint{2.769032in}{1.661541in}}%
\pgfpathcurveto{\pgfqpoint{2.764914in}{1.665659in}}{\pgfqpoint{2.759328in}{1.667973in}}{\pgfqpoint{2.753504in}{1.667973in}}%
\pgfpathcurveto{\pgfqpoint{2.747680in}{1.667973in}}{\pgfqpoint{2.742094in}{1.665659in}}{\pgfqpoint{2.737975in}{1.661541in}}%
\pgfpathcurveto{\pgfqpoint{2.733857in}{1.657423in}}{\pgfqpoint{2.731543in}{1.651837in}}{\pgfqpoint{2.731543in}{1.646013in}}%
\pgfpathcurveto{\pgfqpoint{2.731543in}{1.640189in}}{\pgfqpoint{2.733857in}{1.634603in}}{\pgfqpoint{2.737975in}{1.630484in}}%
\pgfpathcurveto{\pgfqpoint{2.742094in}{1.626366in}}{\pgfqpoint{2.747680in}{1.624052in}}{\pgfqpoint{2.753504in}{1.624052in}}%
\pgfpathlineto{\pgfqpoint{2.753504in}{1.624052in}}%
\pgfpathclose%
\pgfusepath{stroke,fill}%
\end{pgfscope}%
\begin{pgfscope}%
\pgfpathrectangle{\pgfqpoint{0.100000in}{0.183744in}}{\pgfqpoint{4.506048in}{4.506048in}}%
\pgfusepath{clip}%
\pgfsetbuttcap%
\pgfsetroundjoin%
\definecolor{currentfill}{rgb}{0.000000,0.000000,1.000000}%
\pgfsetfillcolor{currentfill}%
\pgfsetfillopacity{0.700000}%
\pgfsetlinewidth{1.003750pt}%
\definecolor{currentstroke}{rgb}{0.000000,0.000000,1.000000}%
\pgfsetstrokecolor{currentstroke}%
\pgfsetstrokeopacity{0.700000}%
\pgfsetdash{}{0pt}%
\pgfpathmoveto{\pgfqpoint{0.709947in}{3.387658in}}%
\pgfpathcurveto{\pgfqpoint{0.715771in}{3.387658in}}{\pgfqpoint{0.721358in}{3.389972in}}{\pgfqpoint{0.725476in}{3.394090in}}%
\pgfpathcurveto{\pgfqpoint{0.729594in}{3.398208in}}{\pgfqpoint{0.731908in}{3.403794in}}{\pgfqpoint{0.731908in}{3.409618in}}%
\pgfpathcurveto{\pgfqpoint{0.731908in}{3.415442in}}{\pgfqpoint{0.729594in}{3.421029in}}{\pgfqpoint{0.725476in}{3.425147in}}%
\pgfpathcurveto{\pgfqpoint{0.721358in}{3.429265in}}{\pgfqpoint{0.715771in}{3.431579in}}{\pgfqpoint{0.709947in}{3.431579in}}%
\pgfpathcurveto{\pgfqpoint{0.704124in}{3.431579in}}{\pgfqpoint{0.698537in}{3.429265in}}{\pgfqpoint{0.694419in}{3.425147in}}%
\pgfpathcurveto{\pgfqpoint{0.690301in}{3.421029in}}{\pgfqpoint{0.687987in}{3.415442in}}{\pgfqpoint{0.687987in}{3.409618in}}%
\pgfpathcurveto{\pgfqpoint{0.687987in}{3.403794in}}{\pgfqpoint{0.690301in}{3.398208in}}{\pgfqpoint{0.694419in}{3.394090in}}%
\pgfpathcurveto{\pgfqpoint{0.698537in}{3.389972in}}{\pgfqpoint{0.704124in}{3.387658in}}{\pgfqpoint{0.709947in}{3.387658in}}%
\pgfpathlineto{\pgfqpoint{0.709947in}{3.387658in}}%
\pgfpathclose%
\pgfusepath{stroke,fill}%
\end{pgfscope}%
\begin{pgfscope}%
\pgfpathrectangle{\pgfqpoint{0.100000in}{0.183744in}}{\pgfqpoint{4.506048in}{4.506048in}}%
\pgfusepath{clip}%
\pgfsetbuttcap%
\pgfsetroundjoin%
\definecolor{currentfill}{rgb}{0.000000,0.000000,1.000000}%
\pgfsetfillcolor{currentfill}%
\pgfsetfillopacity{0.700000}%
\pgfsetlinewidth{1.003750pt}%
\definecolor{currentstroke}{rgb}{0.000000,0.000000,1.000000}%
\pgfsetstrokecolor{currentstroke}%
\pgfsetstrokeopacity{0.700000}%
\pgfsetdash{}{0pt}%
\pgfpathmoveto{\pgfqpoint{3.258667in}{1.946137in}}%
\pgfpathcurveto{\pgfqpoint{3.264491in}{1.946137in}}{\pgfqpoint{3.270077in}{1.948451in}}{\pgfqpoint{3.274195in}{1.952569in}}%
\pgfpathcurveto{\pgfqpoint{3.278314in}{1.956687in}}{\pgfqpoint{3.280627in}{1.962273in}}{\pgfqpoint{3.280627in}{1.968097in}}%
\pgfpathcurveto{\pgfqpoint{3.280627in}{1.973921in}}{\pgfqpoint{3.278314in}{1.979507in}}{\pgfqpoint{3.274195in}{1.983626in}}%
\pgfpathcurveto{\pgfqpoint{3.270077in}{1.987744in}}{\pgfqpoint{3.264491in}{1.990058in}}{\pgfqpoint{3.258667in}{1.990058in}}%
\pgfpathcurveto{\pgfqpoint{3.252843in}{1.990058in}}{\pgfqpoint{3.247257in}{1.987744in}}{\pgfqpoint{3.243139in}{1.983626in}}%
\pgfpathcurveto{\pgfqpoint{3.239021in}{1.979507in}}{\pgfqpoint{3.236707in}{1.973921in}}{\pgfqpoint{3.236707in}{1.968097in}}%
\pgfpathcurveto{\pgfqpoint{3.236707in}{1.962273in}}{\pgfqpoint{3.239021in}{1.956687in}}{\pgfqpoint{3.243139in}{1.952569in}}%
\pgfpathcurveto{\pgfqpoint{3.247257in}{1.948451in}}{\pgfqpoint{3.252843in}{1.946137in}}{\pgfqpoint{3.258667in}{1.946137in}}%
\pgfpathlineto{\pgfqpoint{3.258667in}{1.946137in}}%
\pgfpathclose%
\pgfusepath{stroke,fill}%
\end{pgfscope}%
\begin{pgfscope}%
\pgfpathrectangle{\pgfqpoint{0.100000in}{0.183744in}}{\pgfqpoint{4.506048in}{4.506048in}}%
\pgfusepath{clip}%
\pgfsetbuttcap%
\pgfsetroundjoin%
\definecolor{currentfill}{rgb}{0.000000,0.000000,1.000000}%
\pgfsetfillcolor{currentfill}%
\pgfsetfillopacity{0.700000}%
\pgfsetlinewidth{1.003750pt}%
\definecolor{currentstroke}{rgb}{0.000000,0.000000,1.000000}%
\pgfsetstrokecolor{currentstroke}%
\pgfsetstrokeopacity{0.700000}%
\pgfsetdash{}{0pt}%
\pgfpathmoveto{\pgfqpoint{1.424054in}{1.787639in}}%
\pgfpathcurveto{\pgfqpoint{1.429878in}{1.787639in}}{\pgfqpoint{1.435464in}{1.789953in}}{\pgfqpoint{1.439582in}{1.794071in}}%
\pgfpathcurveto{\pgfqpoint{1.443700in}{1.798189in}}{\pgfqpoint{1.446014in}{1.803776in}}{\pgfqpoint{1.446014in}{1.809599in}}%
\pgfpathcurveto{\pgfqpoint{1.446014in}{1.815423in}}{\pgfqpoint{1.443700in}{1.821010in}}{\pgfqpoint{1.439582in}{1.825128in}}%
\pgfpathcurveto{\pgfqpoint{1.435464in}{1.829246in}}{\pgfqpoint{1.429878in}{1.831560in}}{\pgfqpoint{1.424054in}{1.831560in}}%
\pgfpathcurveto{\pgfqpoint{1.418230in}{1.831560in}}{\pgfqpoint{1.412644in}{1.829246in}}{\pgfqpoint{1.408526in}{1.825128in}}%
\pgfpathcurveto{\pgfqpoint{1.404408in}{1.821010in}}{\pgfqpoint{1.402094in}{1.815423in}}{\pgfqpoint{1.402094in}{1.809599in}}%
\pgfpathcurveto{\pgfqpoint{1.402094in}{1.803776in}}{\pgfqpoint{1.404408in}{1.798189in}}{\pgfqpoint{1.408526in}{1.794071in}}%
\pgfpathcurveto{\pgfqpoint{1.412644in}{1.789953in}}{\pgfqpoint{1.418230in}{1.787639in}}{\pgfqpoint{1.424054in}{1.787639in}}%
\pgfpathlineto{\pgfqpoint{1.424054in}{1.787639in}}%
\pgfpathclose%
\pgfusepath{stroke,fill}%
\end{pgfscope}%
\begin{pgfscope}%
\pgfpathrectangle{\pgfqpoint{0.100000in}{0.183744in}}{\pgfqpoint{4.506048in}{4.506048in}}%
\pgfusepath{clip}%
\pgfsetbuttcap%
\pgfsetroundjoin%
\definecolor{currentfill}{rgb}{0.000000,0.000000,1.000000}%
\pgfsetfillcolor{currentfill}%
\pgfsetfillopacity{0.700000}%
\pgfsetlinewidth{1.003750pt}%
\definecolor{currentstroke}{rgb}{0.000000,0.000000,1.000000}%
\pgfsetstrokecolor{currentstroke}%
\pgfsetstrokeopacity{0.700000}%
\pgfsetdash{}{0pt}%
\pgfpathmoveto{\pgfqpoint{2.905001in}{2.517931in}}%
\pgfpathcurveto{\pgfqpoint{2.910825in}{2.517931in}}{\pgfqpoint{2.916411in}{2.520245in}}{\pgfqpoint{2.920529in}{2.524363in}}%
\pgfpathcurveto{\pgfqpoint{2.924648in}{2.528481in}}{\pgfqpoint{2.926961in}{2.534067in}}{\pgfqpoint{2.926961in}{2.539891in}}%
\pgfpathcurveto{\pgfqpoint{2.926961in}{2.545715in}}{\pgfqpoint{2.924648in}{2.551301in}}{\pgfqpoint{2.920529in}{2.555419in}}%
\pgfpathcurveto{\pgfqpoint{2.916411in}{2.559537in}}{\pgfqpoint{2.910825in}{2.561851in}}{\pgfqpoint{2.905001in}{2.561851in}}%
\pgfpathcurveto{\pgfqpoint{2.899177in}{2.561851in}}{\pgfqpoint{2.893591in}{2.559537in}}{\pgfqpoint{2.889473in}{2.555419in}}%
\pgfpathcurveto{\pgfqpoint{2.885355in}{2.551301in}}{\pgfqpoint{2.883041in}{2.545715in}}{\pgfqpoint{2.883041in}{2.539891in}}%
\pgfpathcurveto{\pgfqpoint{2.883041in}{2.534067in}}{\pgfqpoint{2.885355in}{2.528481in}}{\pgfqpoint{2.889473in}{2.524363in}}%
\pgfpathcurveto{\pgfqpoint{2.893591in}{2.520245in}}{\pgfqpoint{2.899177in}{2.517931in}}{\pgfqpoint{2.905001in}{2.517931in}}%
\pgfpathlineto{\pgfqpoint{2.905001in}{2.517931in}}%
\pgfpathclose%
\pgfusepath{stroke,fill}%
\end{pgfscope}%
\begin{pgfscope}%
\pgfpathrectangle{\pgfqpoint{0.100000in}{0.183744in}}{\pgfqpoint{4.506048in}{4.506048in}}%
\pgfusepath{clip}%
\pgfsetbuttcap%
\pgfsetroundjoin%
\definecolor{currentfill}{rgb}{0.000000,0.000000,1.000000}%
\pgfsetfillcolor{currentfill}%
\pgfsetfillopacity{0.700000}%
\pgfsetlinewidth{1.003750pt}%
\definecolor{currentstroke}{rgb}{0.000000,0.000000,1.000000}%
\pgfsetstrokecolor{currentstroke}%
\pgfsetstrokeopacity{0.700000}%
\pgfsetdash{}{0pt}%
\pgfpathmoveto{\pgfqpoint{0.649640in}{3.477503in}}%
\pgfpathcurveto{\pgfqpoint{0.655464in}{3.477503in}}{\pgfqpoint{0.661050in}{3.479817in}}{\pgfqpoint{0.665168in}{3.483935in}}%
\pgfpathcurveto{\pgfqpoint{0.669286in}{3.488053in}}{\pgfqpoint{0.671600in}{3.493639in}}{\pgfqpoint{0.671600in}{3.499463in}}%
\pgfpathcurveto{\pgfqpoint{0.671600in}{3.505287in}}{\pgfqpoint{0.669286in}{3.510873in}}{\pgfqpoint{0.665168in}{3.514991in}}%
\pgfpathcurveto{\pgfqpoint{0.661050in}{3.519110in}}{\pgfqpoint{0.655464in}{3.521423in}}{\pgfqpoint{0.649640in}{3.521423in}}%
\pgfpathcurveto{\pgfqpoint{0.643816in}{3.521423in}}{\pgfqpoint{0.638230in}{3.519110in}}{\pgfqpoint{0.634112in}{3.514991in}}%
\pgfpathcurveto{\pgfqpoint{0.629993in}{3.510873in}}{\pgfqpoint{0.627680in}{3.505287in}}{\pgfqpoint{0.627680in}{3.499463in}}%
\pgfpathcurveto{\pgfqpoint{0.627680in}{3.493639in}}{\pgfqpoint{0.629993in}{3.488053in}}{\pgfqpoint{0.634112in}{3.483935in}}%
\pgfpathcurveto{\pgfqpoint{0.638230in}{3.479817in}}{\pgfqpoint{0.643816in}{3.477503in}}{\pgfqpoint{0.649640in}{3.477503in}}%
\pgfpathlineto{\pgfqpoint{0.649640in}{3.477503in}}%
\pgfpathclose%
\pgfusepath{stroke,fill}%
\end{pgfscope}%
\begin{pgfscope}%
\pgfpathrectangle{\pgfqpoint{0.100000in}{0.183744in}}{\pgfqpoint{4.506048in}{4.506048in}}%
\pgfusepath{clip}%
\pgfsetbuttcap%
\pgfsetroundjoin%
\definecolor{currentfill}{rgb}{0.000000,0.000000,1.000000}%
\pgfsetfillcolor{currentfill}%
\pgfsetfillopacity{0.700000}%
\pgfsetlinewidth{1.003750pt}%
\definecolor{currentstroke}{rgb}{0.000000,0.000000,1.000000}%
\pgfsetstrokecolor{currentstroke}%
\pgfsetstrokeopacity{0.700000}%
\pgfsetdash{}{0pt}%
\pgfpathmoveto{\pgfqpoint{1.819288in}{3.451218in}}%
\pgfpathcurveto{\pgfqpoint{1.825112in}{3.451218in}}{\pgfqpoint{1.830698in}{3.453532in}}{\pgfqpoint{1.834816in}{3.457650in}}%
\pgfpathcurveto{\pgfqpoint{1.838934in}{3.461769in}}{\pgfqpoint{1.841248in}{3.467355in}}{\pgfqpoint{1.841248in}{3.473179in}}%
\pgfpathcurveto{\pgfqpoint{1.841248in}{3.479003in}}{\pgfqpoint{1.838934in}{3.484589in}}{\pgfqpoint{1.834816in}{3.488707in}}%
\pgfpathcurveto{\pgfqpoint{1.830698in}{3.492825in}}{\pgfqpoint{1.825112in}{3.495139in}}{\pgfqpoint{1.819288in}{3.495139in}}%
\pgfpathcurveto{\pgfqpoint{1.813464in}{3.495139in}}{\pgfqpoint{1.807878in}{3.492825in}}{\pgfqpoint{1.803760in}{3.488707in}}%
\pgfpathcurveto{\pgfqpoint{1.799642in}{3.484589in}}{\pgfqpoint{1.797328in}{3.479003in}}{\pgfqpoint{1.797328in}{3.473179in}}%
\pgfpathcurveto{\pgfqpoint{1.797328in}{3.467355in}}{\pgfqpoint{1.799642in}{3.461769in}}{\pgfqpoint{1.803760in}{3.457650in}}%
\pgfpathcurveto{\pgfqpoint{1.807878in}{3.453532in}}{\pgfqpoint{1.813464in}{3.451218in}}{\pgfqpoint{1.819288in}{3.451218in}}%
\pgfpathlineto{\pgfqpoint{1.819288in}{3.451218in}}%
\pgfpathclose%
\pgfusepath{stroke,fill}%
\end{pgfscope}%
\begin{pgfscope}%
\pgfpathrectangle{\pgfqpoint{0.100000in}{0.183744in}}{\pgfqpoint{4.506048in}{4.506048in}}%
\pgfusepath{clip}%
\pgfsetbuttcap%
\pgfsetroundjoin%
\definecolor{currentfill}{rgb}{0.000000,0.000000,1.000000}%
\pgfsetfillcolor{currentfill}%
\pgfsetfillopacity{0.700000}%
\pgfsetlinewidth{1.003750pt}%
\definecolor{currentstroke}{rgb}{0.000000,0.000000,1.000000}%
\pgfsetstrokecolor{currentstroke}%
\pgfsetstrokeopacity{0.700000}%
\pgfsetdash{}{0pt}%
\pgfpathmoveto{\pgfqpoint{2.289728in}{3.455030in}}%
\pgfpathcurveto{\pgfqpoint{2.295552in}{3.455030in}}{\pgfqpoint{2.301138in}{3.457344in}}{\pgfqpoint{2.305256in}{3.461462in}}%
\pgfpathcurveto{\pgfqpoint{2.309374in}{3.465581in}}{\pgfqpoint{2.311688in}{3.471167in}}{\pgfqpoint{2.311688in}{3.476991in}}%
\pgfpathcurveto{\pgfqpoint{2.311688in}{3.482815in}}{\pgfqpoint{2.309374in}{3.488401in}}{\pgfqpoint{2.305256in}{3.492519in}}%
\pgfpathcurveto{\pgfqpoint{2.301138in}{3.496637in}}{\pgfqpoint{2.295552in}{3.498951in}}{\pgfqpoint{2.289728in}{3.498951in}}%
\pgfpathcurveto{\pgfqpoint{2.283904in}{3.498951in}}{\pgfqpoint{2.278318in}{3.496637in}}{\pgfqpoint{2.274199in}{3.492519in}}%
\pgfpathcurveto{\pgfqpoint{2.270081in}{3.488401in}}{\pgfqpoint{2.267767in}{3.482815in}}{\pgfqpoint{2.267767in}{3.476991in}}%
\pgfpathcurveto{\pgfqpoint{2.267767in}{3.471167in}}{\pgfqpoint{2.270081in}{3.465581in}}{\pgfqpoint{2.274199in}{3.461462in}}%
\pgfpathcurveto{\pgfqpoint{2.278318in}{3.457344in}}{\pgfqpoint{2.283904in}{3.455030in}}{\pgfqpoint{2.289728in}{3.455030in}}%
\pgfpathlineto{\pgfqpoint{2.289728in}{3.455030in}}%
\pgfpathclose%
\pgfusepath{stroke,fill}%
\end{pgfscope}%
\begin{pgfscope}%
\pgfpathrectangle{\pgfqpoint{0.100000in}{0.183744in}}{\pgfqpoint{4.506048in}{4.506048in}}%
\pgfusepath{clip}%
\pgfsetbuttcap%
\pgfsetroundjoin%
\definecolor{currentfill}{rgb}{0.000000,0.000000,1.000000}%
\pgfsetfillcolor{currentfill}%
\pgfsetfillopacity{0.700000}%
\pgfsetlinewidth{1.003750pt}%
\definecolor{currentstroke}{rgb}{0.000000,0.000000,1.000000}%
\pgfsetstrokecolor{currentstroke}%
\pgfsetstrokeopacity{0.700000}%
\pgfsetdash{}{0pt}%
\pgfpathmoveto{\pgfqpoint{3.741475in}{1.681501in}}%
\pgfpathcurveto{\pgfqpoint{3.747299in}{1.681501in}}{\pgfqpoint{3.752885in}{1.683815in}}{\pgfqpoint{3.757003in}{1.687933in}}%
\pgfpathcurveto{\pgfqpoint{3.761121in}{1.692052in}}{\pgfqpoint{3.763435in}{1.697638in}}{\pgfqpoint{3.763435in}{1.703462in}}%
\pgfpathcurveto{\pgfqpoint{3.763435in}{1.709286in}}{\pgfqpoint{3.761121in}{1.714872in}}{\pgfqpoint{3.757003in}{1.718990in}}%
\pgfpathcurveto{\pgfqpoint{3.752885in}{1.723108in}}{\pgfqpoint{3.747299in}{1.725422in}}{\pgfqpoint{3.741475in}{1.725422in}}%
\pgfpathcurveto{\pgfqpoint{3.735651in}{1.725422in}}{\pgfqpoint{3.730065in}{1.723108in}}{\pgfqpoint{3.725947in}{1.718990in}}%
\pgfpathcurveto{\pgfqpoint{3.721829in}{1.714872in}}{\pgfqpoint{3.719515in}{1.709286in}}{\pgfqpoint{3.719515in}{1.703462in}}%
\pgfpathcurveto{\pgfqpoint{3.719515in}{1.697638in}}{\pgfqpoint{3.721829in}{1.692052in}}{\pgfqpoint{3.725947in}{1.687933in}}%
\pgfpathcurveto{\pgfqpoint{3.730065in}{1.683815in}}{\pgfqpoint{3.735651in}{1.681501in}}{\pgfqpoint{3.741475in}{1.681501in}}%
\pgfpathlineto{\pgfqpoint{3.741475in}{1.681501in}}%
\pgfpathclose%
\pgfusepath{stroke,fill}%
\end{pgfscope}%
\begin{pgfscope}%
\pgfpathrectangle{\pgfqpoint{0.100000in}{0.183744in}}{\pgfqpoint{4.506048in}{4.506048in}}%
\pgfusepath{clip}%
\pgfsetbuttcap%
\pgfsetroundjoin%
\definecolor{currentfill}{rgb}{0.000000,0.000000,1.000000}%
\pgfsetfillcolor{currentfill}%
\pgfsetfillopacity{0.700000}%
\pgfsetlinewidth{1.003750pt}%
\definecolor{currentstroke}{rgb}{0.000000,0.000000,1.000000}%
\pgfsetstrokecolor{currentstroke}%
\pgfsetstrokeopacity{0.700000}%
\pgfsetdash{}{0pt}%
\pgfpathmoveto{\pgfqpoint{1.604401in}{2.147943in}}%
\pgfpathcurveto{\pgfqpoint{1.610225in}{2.147943in}}{\pgfqpoint{1.615812in}{2.150256in}}{\pgfqpoint{1.619930in}{2.154375in}}%
\pgfpathcurveto{\pgfqpoint{1.624048in}{2.158493in}}{\pgfqpoint{1.626362in}{2.164079in}}{\pgfqpoint{1.626362in}{2.169903in}}%
\pgfpathcurveto{\pgfqpoint{1.626362in}{2.175727in}}{\pgfqpoint{1.624048in}{2.181313in}}{\pgfqpoint{1.619930in}{2.185431in}}%
\pgfpathcurveto{\pgfqpoint{1.615812in}{2.189549in}}{\pgfqpoint{1.610225in}{2.191863in}}{\pgfqpoint{1.604401in}{2.191863in}}%
\pgfpathcurveto{\pgfqpoint{1.598578in}{2.191863in}}{\pgfqpoint{1.592991in}{2.189549in}}{\pgfqpoint{1.588873in}{2.185431in}}%
\pgfpathcurveto{\pgfqpoint{1.584755in}{2.181313in}}{\pgfqpoint{1.582441in}{2.175727in}}{\pgfqpoint{1.582441in}{2.169903in}}%
\pgfpathcurveto{\pgfqpoint{1.582441in}{2.164079in}}{\pgfqpoint{1.584755in}{2.158493in}}{\pgfqpoint{1.588873in}{2.154375in}}%
\pgfpathcurveto{\pgfqpoint{1.592991in}{2.150256in}}{\pgfqpoint{1.598578in}{2.147943in}}{\pgfqpoint{1.604401in}{2.147943in}}%
\pgfpathlineto{\pgfqpoint{1.604401in}{2.147943in}}%
\pgfpathclose%
\pgfusepath{stroke,fill}%
\end{pgfscope}%
\begin{pgfscope}%
\pgfpathrectangle{\pgfqpoint{0.100000in}{0.183744in}}{\pgfqpoint{4.506048in}{4.506048in}}%
\pgfusepath{clip}%
\pgfsetbuttcap%
\pgfsetroundjoin%
\definecolor{currentfill}{rgb}{0.000000,0.000000,1.000000}%
\pgfsetfillcolor{currentfill}%
\pgfsetfillopacity{0.700000}%
\pgfsetlinewidth{1.003750pt}%
\definecolor{currentstroke}{rgb}{0.000000,0.000000,1.000000}%
\pgfsetstrokecolor{currentstroke}%
\pgfsetstrokeopacity{0.700000}%
\pgfsetdash{}{0pt}%
\pgfpathmoveto{\pgfqpoint{3.569293in}{2.944301in}}%
\pgfpathcurveto{\pgfqpoint{3.575117in}{2.944301in}}{\pgfqpoint{3.580703in}{2.946615in}}{\pgfqpoint{3.584821in}{2.950733in}}%
\pgfpathcurveto{\pgfqpoint{3.588939in}{2.954851in}}{\pgfqpoint{3.591253in}{2.960437in}}{\pgfqpoint{3.591253in}{2.966261in}}%
\pgfpathcurveto{\pgfqpoint{3.591253in}{2.972085in}}{\pgfqpoint{3.588939in}{2.977671in}}{\pgfqpoint{3.584821in}{2.981789in}}%
\pgfpathcurveto{\pgfqpoint{3.580703in}{2.985908in}}{\pgfqpoint{3.575117in}{2.988221in}}{\pgfqpoint{3.569293in}{2.988221in}}%
\pgfpathcurveto{\pgfqpoint{3.563469in}{2.988221in}}{\pgfqpoint{3.557883in}{2.985908in}}{\pgfqpoint{3.553765in}{2.981789in}}%
\pgfpathcurveto{\pgfqpoint{3.549646in}{2.977671in}}{\pgfqpoint{3.547333in}{2.972085in}}{\pgfqpoint{3.547333in}{2.966261in}}%
\pgfpathcurveto{\pgfqpoint{3.547333in}{2.960437in}}{\pgfqpoint{3.549646in}{2.954851in}}{\pgfqpoint{3.553765in}{2.950733in}}%
\pgfpathcurveto{\pgfqpoint{3.557883in}{2.946615in}}{\pgfqpoint{3.563469in}{2.944301in}}{\pgfqpoint{3.569293in}{2.944301in}}%
\pgfpathlineto{\pgfqpoint{3.569293in}{2.944301in}}%
\pgfpathclose%
\pgfusepath{stroke,fill}%
\end{pgfscope}%
\begin{pgfscope}%
\pgfpathrectangle{\pgfqpoint{0.100000in}{0.183744in}}{\pgfqpoint{4.506048in}{4.506048in}}%
\pgfusepath{clip}%
\pgfsetbuttcap%
\pgfsetroundjoin%
\definecolor{currentfill}{rgb}{0.000000,0.000000,1.000000}%
\pgfsetfillcolor{currentfill}%
\pgfsetfillopacity{0.700000}%
\pgfsetlinewidth{1.003750pt}%
\definecolor{currentstroke}{rgb}{0.000000,0.000000,1.000000}%
\pgfsetstrokecolor{currentstroke}%
\pgfsetstrokeopacity{0.700000}%
\pgfsetdash{}{0pt}%
\pgfpathmoveto{\pgfqpoint{2.688627in}{1.442293in}}%
\pgfpathcurveto{\pgfqpoint{2.694451in}{1.442293in}}{\pgfqpoint{2.700037in}{1.444607in}}{\pgfqpoint{2.704155in}{1.448725in}}%
\pgfpathcurveto{\pgfqpoint{2.708274in}{1.452844in}}{\pgfqpoint{2.710587in}{1.458430in}}{\pgfqpoint{2.710587in}{1.464254in}}%
\pgfpathcurveto{\pgfqpoint{2.710587in}{1.470078in}}{\pgfqpoint{2.708274in}{1.475664in}}{\pgfqpoint{2.704155in}{1.479782in}}%
\pgfpathcurveto{\pgfqpoint{2.700037in}{1.483900in}}{\pgfqpoint{2.694451in}{1.486214in}}{\pgfqpoint{2.688627in}{1.486214in}}%
\pgfpathcurveto{\pgfqpoint{2.682803in}{1.486214in}}{\pgfqpoint{2.677217in}{1.483900in}}{\pgfqpoint{2.673099in}{1.479782in}}%
\pgfpathcurveto{\pgfqpoint{2.668981in}{1.475664in}}{\pgfqpoint{2.666667in}{1.470078in}}{\pgfqpoint{2.666667in}{1.464254in}}%
\pgfpathcurveto{\pgfqpoint{2.666667in}{1.458430in}}{\pgfqpoint{2.668981in}{1.452844in}}{\pgfqpoint{2.673099in}{1.448725in}}%
\pgfpathcurveto{\pgfqpoint{2.677217in}{1.444607in}}{\pgfqpoint{2.682803in}{1.442293in}}{\pgfqpoint{2.688627in}{1.442293in}}%
\pgfpathlineto{\pgfqpoint{2.688627in}{1.442293in}}%
\pgfpathclose%
\pgfusepath{stroke,fill}%
\end{pgfscope}%
\begin{pgfscope}%
\pgfpathrectangle{\pgfqpoint{0.100000in}{0.183744in}}{\pgfqpoint{4.506048in}{4.506048in}}%
\pgfusepath{clip}%
\pgfsetbuttcap%
\pgfsetroundjoin%
\definecolor{currentfill}{rgb}{0.000000,0.000000,1.000000}%
\pgfsetfillcolor{currentfill}%
\pgfsetfillopacity{0.700000}%
\pgfsetlinewidth{1.003750pt}%
\definecolor{currentstroke}{rgb}{0.000000,0.000000,1.000000}%
\pgfsetstrokecolor{currentstroke}%
\pgfsetstrokeopacity{0.700000}%
\pgfsetdash{}{0pt}%
\pgfpathmoveto{\pgfqpoint{2.788190in}{2.454397in}}%
\pgfpathcurveto{\pgfqpoint{2.794014in}{2.454397in}}{\pgfqpoint{2.799600in}{2.456711in}}{\pgfqpoint{2.803718in}{2.460829in}}%
\pgfpathcurveto{\pgfqpoint{2.807836in}{2.464947in}}{\pgfqpoint{2.810150in}{2.470533in}}{\pgfqpoint{2.810150in}{2.476357in}}%
\pgfpathcurveto{\pgfqpoint{2.810150in}{2.482181in}}{\pgfqpoint{2.807836in}{2.487767in}}{\pgfqpoint{2.803718in}{2.491886in}}%
\pgfpathcurveto{\pgfqpoint{2.799600in}{2.496004in}}{\pgfqpoint{2.794014in}{2.498318in}}{\pgfqpoint{2.788190in}{2.498318in}}%
\pgfpathcurveto{\pgfqpoint{2.782366in}{2.498318in}}{\pgfqpoint{2.776780in}{2.496004in}}{\pgfqpoint{2.772662in}{2.491886in}}%
\pgfpathcurveto{\pgfqpoint{2.768543in}{2.487767in}}{\pgfqpoint{2.766230in}{2.482181in}}{\pgfqpoint{2.766230in}{2.476357in}}%
\pgfpathcurveto{\pgfqpoint{2.766230in}{2.470533in}}{\pgfqpoint{2.768543in}{2.464947in}}{\pgfqpoint{2.772662in}{2.460829in}}%
\pgfpathcurveto{\pgfqpoint{2.776780in}{2.456711in}}{\pgfqpoint{2.782366in}{2.454397in}}{\pgfqpoint{2.788190in}{2.454397in}}%
\pgfpathlineto{\pgfqpoint{2.788190in}{2.454397in}}%
\pgfpathclose%
\pgfusepath{stroke,fill}%
\end{pgfscope}%
\begin{pgfscope}%
\pgfpathrectangle{\pgfqpoint{0.100000in}{0.183744in}}{\pgfqpoint{4.506048in}{4.506048in}}%
\pgfusepath{clip}%
\pgfsetbuttcap%
\pgfsetroundjoin%
\definecolor{currentfill}{rgb}{0.000000,0.000000,1.000000}%
\pgfsetfillcolor{currentfill}%
\pgfsetfillopacity{0.700000}%
\pgfsetlinewidth{1.003750pt}%
\definecolor{currentstroke}{rgb}{0.000000,0.000000,1.000000}%
\pgfsetstrokecolor{currentstroke}%
\pgfsetstrokeopacity{0.700000}%
\pgfsetdash{}{0pt}%
\pgfpathmoveto{\pgfqpoint{2.484752in}{3.323166in}}%
\pgfpathcurveto{\pgfqpoint{2.490576in}{3.323166in}}{\pgfqpoint{2.496162in}{3.325480in}}{\pgfqpoint{2.500280in}{3.329598in}}%
\pgfpathcurveto{\pgfqpoint{2.504398in}{3.333717in}}{\pgfqpoint{2.506712in}{3.339303in}}{\pgfqpoint{2.506712in}{3.345127in}}%
\pgfpathcurveto{\pgfqpoint{2.506712in}{3.350951in}}{\pgfqpoint{2.504398in}{3.356537in}}{\pgfqpoint{2.500280in}{3.360655in}}%
\pgfpathcurveto{\pgfqpoint{2.496162in}{3.364773in}}{\pgfqpoint{2.490576in}{3.367087in}}{\pgfqpoint{2.484752in}{3.367087in}}%
\pgfpathcurveto{\pgfqpoint{2.478928in}{3.367087in}}{\pgfqpoint{2.473342in}{3.364773in}}{\pgfqpoint{2.469224in}{3.360655in}}%
\pgfpathcurveto{\pgfqpoint{2.465106in}{3.356537in}}{\pgfqpoint{2.462792in}{3.350951in}}{\pgfqpoint{2.462792in}{3.345127in}}%
\pgfpathcurveto{\pgfqpoint{2.462792in}{3.339303in}}{\pgfqpoint{2.465106in}{3.333717in}}{\pgfqpoint{2.469224in}{3.329598in}}%
\pgfpathcurveto{\pgfqpoint{2.473342in}{3.325480in}}{\pgfqpoint{2.478928in}{3.323166in}}{\pgfqpoint{2.484752in}{3.323166in}}%
\pgfpathlineto{\pgfqpoint{2.484752in}{3.323166in}}%
\pgfpathclose%
\pgfusepath{stroke,fill}%
\end{pgfscope}%
\begin{pgfscope}%
\pgfpathrectangle{\pgfqpoint{0.100000in}{0.183744in}}{\pgfqpoint{4.506048in}{4.506048in}}%
\pgfusepath{clip}%
\pgfsetbuttcap%
\pgfsetroundjoin%
\definecolor{currentfill}{rgb}{0.000000,0.000000,1.000000}%
\pgfsetfillcolor{currentfill}%
\pgfsetfillopacity{0.700000}%
\pgfsetlinewidth{1.003750pt}%
\definecolor{currentstroke}{rgb}{0.000000,0.000000,1.000000}%
\pgfsetstrokecolor{currentstroke}%
\pgfsetstrokeopacity{0.700000}%
\pgfsetdash{}{0pt}%
\pgfpathmoveto{\pgfqpoint{0.921815in}{2.864424in}}%
\pgfpathcurveto{\pgfqpoint{0.927639in}{2.864424in}}{\pgfqpoint{0.933225in}{2.866738in}}{\pgfqpoint{0.937343in}{2.870856in}}%
\pgfpathcurveto{\pgfqpoint{0.941461in}{2.874974in}}{\pgfqpoint{0.943775in}{2.880561in}}{\pgfqpoint{0.943775in}{2.886385in}}%
\pgfpathcurveto{\pgfqpoint{0.943775in}{2.892209in}}{\pgfqpoint{0.941461in}{2.897795in}}{\pgfqpoint{0.937343in}{2.901913in}}%
\pgfpathcurveto{\pgfqpoint{0.933225in}{2.906031in}}{\pgfqpoint{0.927639in}{2.908345in}}{\pgfqpoint{0.921815in}{2.908345in}}%
\pgfpathcurveto{\pgfqpoint{0.915991in}{2.908345in}}{\pgfqpoint{0.910405in}{2.906031in}}{\pgfqpoint{0.906287in}{2.901913in}}%
\pgfpathcurveto{\pgfqpoint{0.902168in}{2.897795in}}{\pgfqpoint{0.899855in}{2.892209in}}{\pgfqpoint{0.899855in}{2.886385in}}%
\pgfpathcurveto{\pgfqpoint{0.899855in}{2.880561in}}{\pgfqpoint{0.902168in}{2.874974in}}{\pgfqpoint{0.906287in}{2.870856in}}%
\pgfpathcurveto{\pgfqpoint{0.910405in}{2.866738in}}{\pgfqpoint{0.915991in}{2.864424in}}{\pgfqpoint{0.921815in}{2.864424in}}%
\pgfpathlineto{\pgfqpoint{0.921815in}{2.864424in}}%
\pgfpathclose%
\pgfusepath{stroke,fill}%
\end{pgfscope}%
\begin{pgfscope}%
\pgfpathrectangle{\pgfqpoint{0.100000in}{0.183744in}}{\pgfqpoint{4.506048in}{4.506048in}}%
\pgfusepath{clip}%
\pgfsetbuttcap%
\pgfsetroundjoin%
\definecolor{currentfill}{rgb}{0.000000,0.000000,1.000000}%
\pgfsetfillcolor{currentfill}%
\pgfsetfillopacity{0.700000}%
\pgfsetlinewidth{1.003750pt}%
\definecolor{currentstroke}{rgb}{0.000000,0.000000,1.000000}%
\pgfsetstrokecolor{currentstroke}%
\pgfsetstrokeopacity{0.700000}%
\pgfsetdash{}{0pt}%
\pgfpathmoveto{\pgfqpoint{1.764592in}{2.003324in}}%
\pgfpathcurveto{\pgfqpoint{1.770416in}{2.003324in}}{\pgfqpoint{1.776003in}{2.005638in}}{\pgfqpoint{1.780121in}{2.009756in}}%
\pgfpathcurveto{\pgfqpoint{1.784239in}{2.013874in}}{\pgfqpoint{1.786553in}{2.019461in}}{\pgfqpoint{1.786553in}{2.025285in}}%
\pgfpathcurveto{\pgfqpoint{1.786553in}{2.031109in}}{\pgfqpoint{1.784239in}{2.036695in}}{\pgfqpoint{1.780121in}{2.040813in}}%
\pgfpathcurveto{\pgfqpoint{1.776003in}{2.044931in}}{\pgfqpoint{1.770416in}{2.047245in}}{\pgfqpoint{1.764592in}{2.047245in}}%
\pgfpathcurveto{\pgfqpoint{1.758769in}{2.047245in}}{\pgfqpoint{1.753182in}{2.044931in}}{\pgfqpoint{1.749064in}{2.040813in}}%
\pgfpathcurveto{\pgfqpoint{1.744946in}{2.036695in}}{\pgfqpoint{1.742632in}{2.031109in}}{\pgfqpoint{1.742632in}{2.025285in}}%
\pgfpathcurveto{\pgfqpoint{1.742632in}{2.019461in}}{\pgfqpoint{1.744946in}{2.013874in}}{\pgfqpoint{1.749064in}{2.009756in}}%
\pgfpathcurveto{\pgfqpoint{1.753182in}{2.005638in}}{\pgfqpoint{1.758769in}{2.003324in}}{\pgfqpoint{1.764592in}{2.003324in}}%
\pgfpathlineto{\pgfqpoint{1.764592in}{2.003324in}}%
\pgfpathclose%
\pgfusepath{stroke,fill}%
\end{pgfscope}%
\begin{pgfscope}%
\pgfpathrectangle{\pgfqpoint{0.100000in}{0.183744in}}{\pgfqpoint{4.506048in}{4.506048in}}%
\pgfusepath{clip}%
\pgfsetbuttcap%
\pgfsetroundjoin%
\definecolor{currentfill}{rgb}{0.000000,0.000000,1.000000}%
\pgfsetfillcolor{currentfill}%
\pgfsetfillopacity{0.700000}%
\pgfsetlinewidth{1.003750pt}%
\definecolor{currentstroke}{rgb}{0.000000,0.000000,1.000000}%
\pgfsetstrokecolor{currentstroke}%
\pgfsetstrokeopacity{0.700000}%
\pgfsetdash{}{0pt}%
\pgfpathmoveto{\pgfqpoint{1.979083in}{1.475601in}}%
\pgfpathcurveto{\pgfqpoint{1.984907in}{1.475601in}}{\pgfqpoint{1.990493in}{1.477915in}}{\pgfqpoint{1.994611in}{1.482033in}}%
\pgfpathcurveto{\pgfqpoint{1.998729in}{1.486151in}}{\pgfqpoint{2.001043in}{1.491737in}}{\pgfqpoint{2.001043in}{1.497561in}}%
\pgfpathcurveto{\pgfqpoint{2.001043in}{1.503385in}}{\pgfqpoint{1.998729in}{1.508971in}}{\pgfqpoint{1.994611in}{1.513089in}}%
\pgfpathcurveto{\pgfqpoint{1.990493in}{1.517207in}}{\pgfqpoint{1.984907in}{1.519521in}}{\pgfqpoint{1.979083in}{1.519521in}}%
\pgfpathcurveto{\pgfqpoint{1.973259in}{1.519521in}}{\pgfqpoint{1.967673in}{1.517207in}}{\pgfqpoint{1.963555in}{1.513089in}}%
\pgfpathcurveto{\pgfqpoint{1.959436in}{1.508971in}}{\pgfqpoint{1.957123in}{1.503385in}}{\pgfqpoint{1.957123in}{1.497561in}}%
\pgfpathcurveto{\pgfqpoint{1.957123in}{1.491737in}}{\pgfqpoint{1.959436in}{1.486151in}}{\pgfqpoint{1.963555in}{1.482033in}}%
\pgfpathcurveto{\pgfqpoint{1.967673in}{1.477915in}}{\pgfqpoint{1.973259in}{1.475601in}}{\pgfqpoint{1.979083in}{1.475601in}}%
\pgfpathlineto{\pgfqpoint{1.979083in}{1.475601in}}%
\pgfpathclose%
\pgfusepath{stroke,fill}%
\end{pgfscope}%
\begin{pgfscope}%
\pgfpathrectangle{\pgfqpoint{0.100000in}{0.183744in}}{\pgfqpoint{4.506048in}{4.506048in}}%
\pgfusepath{clip}%
\pgfsetbuttcap%
\pgfsetroundjoin%
\definecolor{currentfill}{rgb}{0.000000,0.000000,1.000000}%
\pgfsetfillcolor{currentfill}%
\pgfsetfillopacity{0.700000}%
\pgfsetlinewidth{1.003750pt}%
\definecolor{currentstroke}{rgb}{0.000000,0.000000,1.000000}%
\pgfsetstrokecolor{currentstroke}%
\pgfsetstrokeopacity{0.700000}%
\pgfsetdash{}{0pt}%
\pgfpathmoveto{\pgfqpoint{1.865582in}{0.673361in}}%
\pgfpathcurveto{\pgfqpoint{1.871406in}{0.673361in}}{\pgfqpoint{1.876993in}{0.675675in}}{\pgfqpoint{1.881111in}{0.679793in}}%
\pgfpathcurveto{\pgfqpoint{1.885229in}{0.683911in}}{\pgfqpoint{1.887543in}{0.689497in}}{\pgfqpoint{1.887543in}{0.695321in}}%
\pgfpathcurveto{\pgfqpoint{1.887543in}{0.701145in}}{\pgfqpoint{1.885229in}{0.706731in}}{\pgfqpoint{1.881111in}{0.710849in}}%
\pgfpathcurveto{\pgfqpoint{1.876993in}{0.714968in}}{\pgfqpoint{1.871406in}{0.717281in}}{\pgfqpoint{1.865582in}{0.717281in}}%
\pgfpathcurveto{\pgfqpoint{1.859759in}{0.717281in}}{\pgfqpoint{1.854172in}{0.714968in}}{\pgfqpoint{1.850054in}{0.710849in}}%
\pgfpathcurveto{\pgfqpoint{1.845936in}{0.706731in}}{\pgfqpoint{1.843622in}{0.701145in}}{\pgfqpoint{1.843622in}{0.695321in}}%
\pgfpathcurveto{\pgfqpoint{1.843622in}{0.689497in}}{\pgfqpoint{1.845936in}{0.683911in}}{\pgfqpoint{1.850054in}{0.679793in}}%
\pgfpathcurveto{\pgfqpoint{1.854172in}{0.675675in}}{\pgfqpoint{1.859759in}{0.673361in}}{\pgfqpoint{1.865582in}{0.673361in}}%
\pgfpathlineto{\pgfqpoint{1.865582in}{0.673361in}}%
\pgfpathclose%
\pgfusepath{stroke,fill}%
\end{pgfscope}%
\begin{pgfscope}%
\pgfpathrectangle{\pgfqpoint{0.100000in}{0.183744in}}{\pgfqpoint{4.506048in}{4.506048in}}%
\pgfusepath{clip}%
\pgfsetbuttcap%
\pgfsetroundjoin%
\definecolor{currentfill}{rgb}{0.000000,0.000000,1.000000}%
\pgfsetfillcolor{currentfill}%
\pgfsetfillopacity{0.700000}%
\pgfsetlinewidth{1.003750pt}%
\definecolor{currentstroke}{rgb}{0.000000,0.000000,1.000000}%
\pgfsetstrokecolor{currentstroke}%
\pgfsetstrokeopacity{0.700000}%
\pgfsetdash{}{0pt}%
\pgfpathmoveto{\pgfqpoint{2.168574in}{3.311239in}}%
\pgfpathcurveto{\pgfqpoint{2.174398in}{3.311239in}}{\pgfqpoint{2.179984in}{3.313553in}}{\pgfqpoint{2.184102in}{3.317671in}}%
\pgfpathcurveto{\pgfqpoint{2.188220in}{3.321789in}}{\pgfqpoint{2.190534in}{3.327376in}}{\pgfqpoint{2.190534in}{3.333200in}}%
\pgfpathcurveto{\pgfqpoint{2.190534in}{3.339024in}}{\pgfqpoint{2.188220in}{3.344610in}}{\pgfqpoint{2.184102in}{3.348728in}}%
\pgfpathcurveto{\pgfqpoint{2.179984in}{3.352846in}}{\pgfqpoint{2.174398in}{3.355160in}}{\pgfqpoint{2.168574in}{3.355160in}}%
\pgfpathcurveto{\pgfqpoint{2.162750in}{3.355160in}}{\pgfqpoint{2.157163in}{3.352846in}}{\pgfqpoint{2.153045in}{3.348728in}}%
\pgfpathcurveto{\pgfqpoint{2.148927in}{3.344610in}}{\pgfqpoint{2.146613in}{3.339024in}}{\pgfqpoint{2.146613in}{3.333200in}}%
\pgfpathcurveto{\pgfqpoint{2.146613in}{3.327376in}}{\pgfqpoint{2.148927in}{3.321789in}}{\pgfqpoint{2.153045in}{3.317671in}}%
\pgfpathcurveto{\pgfqpoint{2.157163in}{3.313553in}}{\pgfqpoint{2.162750in}{3.311239in}}{\pgfqpoint{2.168574in}{3.311239in}}%
\pgfpathlineto{\pgfqpoint{2.168574in}{3.311239in}}%
\pgfpathclose%
\pgfusepath{stroke,fill}%
\end{pgfscope}%
\begin{pgfscope}%
\pgfpathrectangle{\pgfqpoint{0.100000in}{0.183744in}}{\pgfqpoint{4.506048in}{4.506048in}}%
\pgfusepath{clip}%
\pgfsetbuttcap%
\pgfsetroundjoin%
\definecolor{currentfill}{rgb}{0.000000,0.000000,1.000000}%
\pgfsetfillcolor{currentfill}%
\pgfsetfillopacity{0.700000}%
\pgfsetlinewidth{1.003750pt}%
\definecolor{currentstroke}{rgb}{0.000000,0.000000,1.000000}%
\pgfsetstrokecolor{currentstroke}%
\pgfsetstrokeopacity{0.700000}%
\pgfsetdash{}{0pt}%
\pgfpathmoveto{\pgfqpoint{1.952614in}{2.013598in}}%
\pgfpathcurveto{\pgfqpoint{1.958438in}{2.013598in}}{\pgfqpoint{1.964024in}{2.015911in}}{\pgfqpoint{1.968142in}{2.020030in}}%
\pgfpathcurveto{\pgfqpoint{1.972260in}{2.024148in}}{\pgfqpoint{1.974574in}{2.029734in}}{\pgfqpoint{1.974574in}{2.035558in}}%
\pgfpathcurveto{\pgfqpoint{1.974574in}{2.041382in}}{\pgfqpoint{1.972260in}{2.046968in}}{\pgfqpoint{1.968142in}{2.051086in}}%
\pgfpathcurveto{\pgfqpoint{1.964024in}{2.055204in}}{\pgfqpoint{1.958438in}{2.057518in}}{\pgfqpoint{1.952614in}{2.057518in}}%
\pgfpathcurveto{\pgfqpoint{1.946790in}{2.057518in}}{\pgfqpoint{1.941204in}{2.055204in}}{\pgfqpoint{1.937085in}{2.051086in}}%
\pgfpathcurveto{\pgfqpoint{1.932967in}{2.046968in}}{\pgfqpoint{1.930653in}{2.041382in}}{\pgfqpoint{1.930653in}{2.035558in}}%
\pgfpathcurveto{\pgfqpoint{1.930653in}{2.029734in}}{\pgfqpoint{1.932967in}{2.024148in}}{\pgfqpoint{1.937085in}{2.020030in}}%
\pgfpathcurveto{\pgfqpoint{1.941204in}{2.015911in}}{\pgfqpoint{1.946790in}{2.013598in}}{\pgfqpoint{1.952614in}{2.013598in}}%
\pgfpathlineto{\pgfqpoint{1.952614in}{2.013598in}}%
\pgfpathclose%
\pgfusepath{stroke,fill}%
\end{pgfscope}%
\begin{pgfscope}%
\pgfpathrectangle{\pgfqpoint{0.100000in}{0.183744in}}{\pgfqpoint{4.506048in}{4.506048in}}%
\pgfusepath{clip}%
\pgfsetbuttcap%
\pgfsetroundjoin%
\definecolor{currentfill}{rgb}{0.000000,0.000000,1.000000}%
\pgfsetfillcolor{currentfill}%
\pgfsetfillopacity{0.700000}%
\pgfsetlinewidth{1.003750pt}%
\definecolor{currentstroke}{rgb}{0.000000,0.000000,1.000000}%
\pgfsetstrokecolor{currentstroke}%
\pgfsetstrokeopacity{0.700000}%
\pgfsetdash{}{0pt}%
\pgfpathmoveto{\pgfqpoint{1.894482in}{2.703670in}}%
\pgfpathcurveto{\pgfqpoint{1.900306in}{2.703670in}}{\pgfqpoint{1.905892in}{2.705984in}}{\pgfqpoint{1.910010in}{2.710102in}}%
\pgfpathcurveto{\pgfqpoint{1.914129in}{2.714220in}}{\pgfqpoint{1.916442in}{2.719806in}}{\pgfqpoint{1.916442in}{2.725630in}}%
\pgfpathcurveto{\pgfqpoint{1.916442in}{2.731454in}}{\pgfqpoint{1.914129in}{2.737040in}}{\pgfqpoint{1.910010in}{2.741158in}}%
\pgfpathcurveto{\pgfqpoint{1.905892in}{2.745277in}}{\pgfqpoint{1.900306in}{2.747590in}}{\pgfqpoint{1.894482in}{2.747590in}}%
\pgfpathcurveto{\pgfqpoint{1.888658in}{2.747590in}}{\pgfqpoint{1.883072in}{2.745277in}}{\pgfqpoint{1.878954in}{2.741158in}}%
\pgfpathcurveto{\pgfqpoint{1.874836in}{2.737040in}}{\pgfqpoint{1.872522in}{2.731454in}}{\pgfqpoint{1.872522in}{2.725630in}}%
\pgfpathcurveto{\pgfqpoint{1.872522in}{2.719806in}}{\pgfqpoint{1.874836in}{2.714220in}}{\pgfqpoint{1.878954in}{2.710102in}}%
\pgfpathcurveto{\pgfqpoint{1.883072in}{2.705984in}}{\pgfqpoint{1.888658in}{2.703670in}}{\pgfqpoint{1.894482in}{2.703670in}}%
\pgfpathlineto{\pgfqpoint{1.894482in}{2.703670in}}%
\pgfpathclose%
\pgfusepath{stroke,fill}%
\end{pgfscope}%
\begin{pgfscope}%
\pgfpathrectangle{\pgfqpoint{0.100000in}{0.183744in}}{\pgfqpoint{4.506048in}{4.506048in}}%
\pgfusepath{clip}%
\pgfsetbuttcap%
\pgfsetroundjoin%
\definecolor{currentfill}{rgb}{0.000000,0.000000,1.000000}%
\pgfsetfillcolor{currentfill}%
\pgfsetfillopacity{0.700000}%
\pgfsetlinewidth{1.003750pt}%
\definecolor{currentstroke}{rgb}{0.000000,0.000000,1.000000}%
\pgfsetstrokecolor{currentstroke}%
\pgfsetstrokeopacity{0.700000}%
\pgfsetdash{}{0pt}%
\pgfpathmoveto{\pgfqpoint{1.382843in}{2.127799in}}%
\pgfpathcurveto{\pgfqpoint{1.388666in}{2.127799in}}{\pgfqpoint{1.394253in}{2.130112in}}{\pgfqpoint{1.398371in}{2.134231in}}%
\pgfpathcurveto{\pgfqpoint{1.402489in}{2.138349in}}{\pgfqpoint{1.404803in}{2.143935in}}{\pgfqpoint{1.404803in}{2.149759in}}%
\pgfpathcurveto{\pgfqpoint{1.404803in}{2.155583in}}{\pgfqpoint{1.402489in}{2.161169in}}{\pgfqpoint{1.398371in}{2.165287in}}%
\pgfpathcurveto{\pgfqpoint{1.394253in}{2.169405in}}{\pgfqpoint{1.388666in}{2.171719in}}{\pgfqpoint{1.382843in}{2.171719in}}%
\pgfpathcurveto{\pgfqpoint{1.377019in}{2.171719in}}{\pgfqpoint{1.371432in}{2.169405in}}{\pgfqpoint{1.367314in}{2.165287in}}%
\pgfpathcurveto{\pgfqpoint{1.363196in}{2.161169in}}{\pgfqpoint{1.360882in}{2.155583in}}{\pgfqpoint{1.360882in}{2.149759in}}%
\pgfpathcurveto{\pgfqpoint{1.360882in}{2.143935in}}{\pgfqpoint{1.363196in}{2.138349in}}{\pgfqpoint{1.367314in}{2.134231in}}%
\pgfpathcurveto{\pgfqpoint{1.371432in}{2.130112in}}{\pgfqpoint{1.377019in}{2.127799in}}{\pgfqpoint{1.382843in}{2.127799in}}%
\pgfpathlineto{\pgfqpoint{1.382843in}{2.127799in}}%
\pgfpathclose%
\pgfusepath{stroke,fill}%
\end{pgfscope}%
\begin{pgfscope}%
\pgfpathrectangle{\pgfqpoint{0.100000in}{0.183744in}}{\pgfqpoint{4.506048in}{4.506048in}}%
\pgfusepath{clip}%
\pgfsetbuttcap%
\pgfsetroundjoin%
\definecolor{currentfill}{rgb}{0.000000,0.000000,1.000000}%
\pgfsetfillcolor{currentfill}%
\pgfsetfillopacity{0.700000}%
\pgfsetlinewidth{1.003750pt}%
\definecolor{currentstroke}{rgb}{0.000000,0.000000,1.000000}%
\pgfsetstrokecolor{currentstroke}%
\pgfsetstrokeopacity{0.700000}%
\pgfsetdash{}{0pt}%
\pgfpathmoveto{\pgfqpoint{1.765091in}{1.440568in}}%
\pgfpathcurveto{\pgfqpoint{1.770915in}{1.440568in}}{\pgfqpoint{1.776501in}{1.442882in}}{\pgfqpoint{1.780619in}{1.447000in}}%
\pgfpathcurveto{\pgfqpoint{1.784738in}{1.451118in}}{\pgfqpoint{1.787051in}{1.456704in}}{\pgfqpoint{1.787051in}{1.462528in}}%
\pgfpathcurveto{\pgfqpoint{1.787051in}{1.468352in}}{\pgfqpoint{1.784738in}{1.473938in}}{\pgfqpoint{1.780619in}{1.478056in}}%
\pgfpathcurveto{\pgfqpoint{1.776501in}{1.482174in}}{\pgfqpoint{1.770915in}{1.484488in}}{\pgfqpoint{1.765091in}{1.484488in}}%
\pgfpathcurveto{\pgfqpoint{1.759267in}{1.484488in}}{\pgfqpoint{1.753681in}{1.482174in}}{\pgfqpoint{1.749563in}{1.478056in}}%
\pgfpathcurveto{\pgfqpoint{1.745445in}{1.473938in}}{\pgfqpoint{1.743131in}{1.468352in}}{\pgfqpoint{1.743131in}{1.462528in}}%
\pgfpathcurveto{\pgfqpoint{1.743131in}{1.456704in}}{\pgfqpoint{1.745445in}{1.451118in}}{\pgfqpoint{1.749563in}{1.447000in}}%
\pgfpathcurveto{\pgfqpoint{1.753681in}{1.442882in}}{\pgfqpoint{1.759267in}{1.440568in}}{\pgfqpoint{1.765091in}{1.440568in}}%
\pgfpathlineto{\pgfqpoint{1.765091in}{1.440568in}}%
\pgfpathclose%
\pgfusepath{stroke,fill}%
\end{pgfscope}%
\begin{pgfscope}%
\pgfpathrectangle{\pgfqpoint{0.100000in}{0.183744in}}{\pgfqpoint{4.506048in}{4.506048in}}%
\pgfusepath{clip}%
\pgfsetbuttcap%
\pgfsetroundjoin%
\definecolor{currentfill}{rgb}{0.000000,0.000000,1.000000}%
\pgfsetfillcolor{currentfill}%
\pgfsetfillopacity{0.700000}%
\pgfsetlinewidth{1.003750pt}%
\definecolor{currentstroke}{rgb}{0.000000,0.000000,1.000000}%
\pgfsetstrokecolor{currentstroke}%
\pgfsetstrokeopacity{0.700000}%
\pgfsetdash{}{0pt}%
\pgfpathmoveto{\pgfqpoint{3.575504in}{2.754643in}}%
\pgfpathcurveto{\pgfqpoint{3.581327in}{2.754643in}}{\pgfqpoint{3.586914in}{2.756957in}}{\pgfqpoint{3.591032in}{2.761075in}}%
\pgfpathcurveto{\pgfqpoint{3.595150in}{2.765193in}}{\pgfqpoint{3.597464in}{2.770779in}}{\pgfqpoint{3.597464in}{2.776603in}}%
\pgfpathcurveto{\pgfqpoint{3.597464in}{2.782427in}}{\pgfqpoint{3.595150in}{2.788013in}}{\pgfqpoint{3.591032in}{2.792132in}}%
\pgfpathcurveto{\pgfqpoint{3.586914in}{2.796250in}}{\pgfqpoint{3.581327in}{2.798564in}}{\pgfqpoint{3.575504in}{2.798564in}}%
\pgfpathcurveto{\pgfqpoint{3.569680in}{2.798564in}}{\pgfqpoint{3.564093in}{2.796250in}}{\pgfqpoint{3.559975in}{2.792132in}}%
\pgfpathcurveto{\pgfqpoint{3.555857in}{2.788013in}}{\pgfqpoint{3.553543in}{2.782427in}}{\pgfqpoint{3.553543in}{2.776603in}}%
\pgfpathcurveto{\pgfqpoint{3.553543in}{2.770779in}}{\pgfqpoint{3.555857in}{2.765193in}}{\pgfqpoint{3.559975in}{2.761075in}}%
\pgfpathcurveto{\pgfqpoint{3.564093in}{2.756957in}}{\pgfqpoint{3.569680in}{2.754643in}}{\pgfqpoint{3.575504in}{2.754643in}}%
\pgfpathlineto{\pgfqpoint{3.575504in}{2.754643in}}%
\pgfpathclose%
\pgfusepath{stroke,fill}%
\end{pgfscope}%
\begin{pgfscope}%
\pgfpathrectangle{\pgfqpoint{0.100000in}{0.183744in}}{\pgfqpoint{4.506048in}{4.506048in}}%
\pgfusepath{clip}%
\pgfsetbuttcap%
\pgfsetroundjoin%
\definecolor{currentfill}{rgb}{0.000000,0.000000,1.000000}%
\pgfsetfillcolor{currentfill}%
\pgfsetfillopacity{0.700000}%
\pgfsetlinewidth{1.003750pt}%
\definecolor{currentstroke}{rgb}{0.000000,0.000000,1.000000}%
\pgfsetstrokecolor{currentstroke}%
\pgfsetstrokeopacity{0.700000}%
\pgfsetdash{}{0pt}%
\pgfpathmoveto{\pgfqpoint{2.784631in}{1.543460in}}%
\pgfpathcurveto{\pgfqpoint{2.790455in}{1.543460in}}{\pgfqpoint{2.796041in}{1.545774in}}{\pgfqpoint{2.800159in}{1.549892in}}%
\pgfpathcurveto{\pgfqpoint{2.804277in}{1.554010in}}{\pgfqpoint{2.806591in}{1.559597in}}{\pgfqpoint{2.806591in}{1.565421in}}%
\pgfpathcurveto{\pgfqpoint{2.806591in}{1.571245in}}{\pgfqpoint{2.804277in}{1.576831in}}{\pgfqpoint{2.800159in}{1.580949in}}%
\pgfpathcurveto{\pgfqpoint{2.796041in}{1.585067in}}{\pgfqpoint{2.790455in}{1.587381in}}{\pgfqpoint{2.784631in}{1.587381in}}%
\pgfpathcurveto{\pgfqpoint{2.778807in}{1.587381in}}{\pgfqpoint{2.773221in}{1.585067in}}{\pgfqpoint{2.769102in}{1.580949in}}%
\pgfpathcurveto{\pgfqpoint{2.764984in}{1.576831in}}{\pgfqpoint{2.762670in}{1.571245in}}{\pgfqpoint{2.762670in}{1.565421in}}%
\pgfpathcurveto{\pgfqpoint{2.762670in}{1.559597in}}{\pgfqpoint{2.764984in}{1.554010in}}{\pgfqpoint{2.769102in}{1.549892in}}%
\pgfpathcurveto{\pgfqpoint{2.773221in}{1.545774in}}{\pgfqpoint{2.778807in}{1.543460in}}{\pgfqpoint{2.784631in}{1.543460in}}%
\pgfpathlineto{\pgfqpoint{2.784631in}{1.543460in}}%
\pgfpathclose%
\pgfusepath{stroke,fill}%
\end{pgfscope}%
\begin{pgfscope}%
\pgfpathrectangle{\pgfqpoint{0.100000in}{0.183744in}}{\pgfqpoint{4.506048in}{4.506048in}}%
\pgfusepath{clip}%
\pgfsetbuttcap%
\pgfsetroundjoin%
\definecolor{currentfill}{rgb}{0.000000,0.000000,1.000000}%
\pgfsetfillcolor{currentfill}%
\pgfsetfillopacity{0.700000}%
\pgfsetlinewidth{1.003750pt}%
\definecolor{currentstroke}{rgb}{0.000000,0.000000,1.000000}%
\pgfsetstrokecolor{currentstroke}%
\pgfsetstrokeopacity{0.700000}%
\pgfsetdash{}{0pt}%
\pgfpathmoveto{\pgfqpoint{3.349777in}{2.594344in}}%
\pgfpathcurveto{\pgfqpoint{3.355601in}{2.594344in}}{\pgfqpoint{3.361187in}{2.596658in}}{\pgfqpoint{3.365305in}{2.600776in}}%
\pgfpathcurveto{\pgfqpoint{3.369423in}{2.604894in}}{\pgfqpoint{3.371737in}{2.610481in}}{\pgfqpoint{3.371737in}{2.616305in}}%
\pgfpathcurveto{\pgfqpoint{3.371737in}{2.622129in}}{\pgfqpoint{3.369423in}{2.627715in}}{\pgfqpoint{3.365305in}{2.631833in}}%
\pgfpathcurveto{\pgfqpoint{3.361187in}{2.635951in}}{\pgfqpoint{3.355601in}{2.638265in}}{\pgfqpoint{3.349777in}{2.638265in}}%
\pgfpathcurveto{\pgfqpoint{3.343953in}{2.638265in}}{\pgfqpoint{3.338367in}{2.635951in}}{\pgfqpoint{3.334249in}{2.631833in}}%
\pgfpathcurveto{\pgfqpoint{3.330130in}{2.627715in}}{\pgfqpoint{3.327817in}{2.622129in}}{\pgfqpoint{3.327817in}{2.616305in}}%
\pgfpathcurveto{\pgfqpoint{3.327817in}{2.610481in}}{\pgfqpoint{3.330130in}{2.604894in}}{\pgfqpoint{3.334249in}{2.600776in}}%
\pgfpathcurveto{\pgfqpoint{3.338367in}{2.596658in}}{\pgfqpoint{3.343953in}{2.594344in}}{\pgfqpoint{3.349777in}{2.594344in}}%
\pgfpathlineto{\pgfqpoint{3.349777in}{2.594344in}}%
\pgfpathclose%
\pgfusepath{stroke,fill}%
\end{pgfscope}%
\begin{pgfscope}%
\pgfpathrectangle{\pgfqpoint{0.100000in}{0.183744in}}{\pgfqpoint{4.506048in}{4.506048in}}%
\pgfusepath{clip}%
\pgfsetbuttcap%
\pgfsetroundjoin%
\definecolor{currentfill}{rgb}{0.000000,0.000000,1.000000}%
\pgfsetfillcolor{currentfill}%
\pgfsetfillopacity{0.700000}%
\pgfsetlinewidth{1.003750pt}%
\definecolor{currentstroke}{rgb}{0.000000,0.000000,1.000000}%
\pgfsetstrokecolor{currentstroke}%
\pgfsetstrokeopacity{0.700000}%
\pgfsetdash{}{0pt}%
\pgfpathmoveto{\pgfqpoint{1.778880in}{1.576334in}}%
\pgfpathcurveto{\pgfqpoint{1.784704in}{1.576334in}}{\pgfqpoint{1.790290in}{1.578648in}}{\pgfqpoint{1.794408in}{1.582766in}}%
\pgfpathcurveto{\pgfqpoint{1.798526in}{1.586884in}}{\pgfqpoint{1.800840in}{1.592470in}}{\pgfqpoint{1.800840in}{1.598294in}}%
\pgfpathcurveto{\pgfqpoint{1.800840in}{1.604118in}}{\pgfqpoint{1.798526in}{1.609705in}}{\pgfqpoint{1.794408in}{1.613823in}}%
\pgfpathcurveto{\pgfqpoint{1.790290in}{1.617941in}}{\pgfqpoint{1.784704in}{1.620255in}}{\pgfqpoint{1.778880in}{1.620255in}}%
\pgfpathcurveto{\pgfqpoint{1.773056in}{1.620255in}}{\pgfqpoint{1.767470in}{1.617941in}}{\pgfqpoint{1.763351in}{1.613823in}}%
\pgfpathcurveto{\pgfqpoint{1.759233in}{1.609705in}}{\pgfqpoint{1.756919in}{1.604118in}}{\pgfqpoint{1.756919in}{1.598294in}}%
\pgfpathcurveto{\pgfqpoint{1.756919in}{1.592470in}}{\pgfqpoint{1.759233in}{1.586884in}}{\pgfqpoint{1.763351in}{1.582766in}}%
\pgfpathcurveto{\pgfqpoint{1.767470in}{1.578648in}}{\pgfqpoint{1.773056in}{1.576334in}}{\pgfqpoint{1.778880in}{1.576334in}}%
\pgfpathlineto{\pgfqpoint{1.778880in}{1.576334in}}%
\pgfpathclose%
\pgfusepath{stroke,fill}%
\end{pgfscope}%
\begin{pgfscope}%
\pgfpathrectangle{\pgfqpoint{0.100000in}{0.183744in}}{\pgfqpoint{4.506048in}{4.506048in}}%
\pgfusepath{clip}%
\pgfsetbuttcap%
\pgfsetroundjoin%
\definecolor{currentfill}{rgb}{0.000000,0.000000,1.000000}%
\pgfsetfillcolor{currentfill}%
\pgfsetfillopacity{0.700000}%
\pgfsetlinewidth{1.003750pt}%
\definecolor{currentstroke}{rgb}{0.000000,0.000000,1.000000}%
\pgfsetstrokecolor{currentstroke}%
\pgfsetstrokeopacity{0.700000}%
\pgfsetdash{}{0pt}%
\pgfpathmoveto{\pgfqpoint{2.418786in}{3.103179in}}%
\pgfpathcurveto{\pgfqpoint{2.424610in}{3.103179in}}{\pgfqpoint{2.430196in}{3.105493in}}{\pgfqpoint{2.434315in}{3.109611in}}%
\pgfpathcurveto{\pgfqpoint{2.438433in}{3.113730in}}{\pgfqpoint{2.440747in}{3.119316in}}{\pgfqpoint{2.440747in}{3.125140in}}%
\pgfpathcurveto{\pgfqpoint{2.440747in}{3.130964in}}{\pgfqpoint{2.438433in}{3.136550in}}{\pgfqpoint{2.434315in}{3.140668in}}%
\pgfpathcurveto{\pgfqpoint{2.430196in}{3.144786in}}{\pgfqpoint{2.424610in}{3.147100in}}{\pgfqpoint{2.418786in}{3.147100in}}%
\pgfpathcurveto{\pgfqpoint{2.412962in}{3.147100in}}{\pgfqpoint{2.407376in}{3.144786in}}{\pgfqpoint{2.403258in}{3.140668in}}%
\pgfpathcurveto{\pgfqpoint{2.399140in}{3.136550in}}{\pgfqpoint{2.396826in}{3.130964in}}{\pgfqpoint{2.396826in}{3.125140in}}%
\pgfpathcurveto{\pgfqpoint{2.396826in}{3.119316in}}{\pgfqpoint{2.399140in}{3.113730in}}{\pgfqpoint{2.403258in}{3.109611in}}%
\pgfpathcurveto{\pgfqpoint{2.407376in}{3.105493in}}{\pgfqpoint{2.412962in}{3.103179in}}{\pgfqpoint{2.418786in}{3.103179in}}%
\pgfpathlineto{\pgfqpoint{2.418786in}{3.103179in}}%
\pgfpathclose%
\pgfusepath{stroke,fill}%
\end{pgfscope}%
\begin{pgfscope}%
\pgfpathrectangle{\pgfqpoint{0.100000in}{0.183744in}}{\pgfqpoint{4.506048in}{4.506048in}}%
\pgfusepath{clip}%
\pgfsetbuttcap%
\pgfsetroundjoin%
\definecolor{currentfill}{rgb}{0.000000,0.000000,1.000000}%
\pgfsetfillcolor{currentfill}%
\pgfsetfillopacity{0.700000}%
\pgfsetlinewidth{1.003750pt}%
\definecolor{currentstroke}{rgb}{0.000000,0.000000,1.000000}%
\pgfsetstrokecolor{currentstroke}%
\pgfsetstrokeopacity{0.700000}%
\pgfsetdash{}{0pt}%
\pgfpathmoveto{\pgfqpoint{3.202403in}{2.227735in}}%
\pgfpathcurveto{\pgfqpoint{3.208227in}{2.227735in}}{\pgfqpoint{3.213813in}{2.230049in}}{\pgfqpoint{3.217931in}{2.234167in}}%
\pgfpathcurveto{\pgfqpoint{3.222049in}{2.238285in}}{\pgfqpoint{3.224363in}{2.243871in}}{\pgfqpoint{3.224363in}{2.249695in}}%
\pgfpathcurveto{\pgfqpoint{3.224363in}{2.255519in}}{\pgfqpoint{3.222049in}{2.261105in}}{\pgfqpoint{3.217931in}{2.265223in}}%
\pgfpathcurveto{\pgfqpoint{3.213813in}{2.269341in}}{\pgfqpoint{3.208227in}{2.271655in}}{\pgfqpoint{3.202403in}{2.271655in}}%
\pgfpathcurveto{\pgfqpoint{3.196579in}{2.271655in}}{\pgfqpoint{3.190993in}{2.269341in}}{\pgfqpoint{3.186875in}{2.265223in}}%
\pgfpathcurveto{\pgfqpoint{3.182757in}{2.261105in}}{\pgfqpoint{3.180443in}{2.255519in}}{\pgfqpoint{3.180443in}{2.249695in}}%
\pgfpathcurveto{\pgfqpoint{3.180443in}{2.243871in}}{\pgfqpoint{3.182757in}{2.238285in}}{\pgfqpoint{3.186875in}{2.234167in}}%
\pgfpathcurveto{\pgfqpoint{3.190993in}{2.230049in}}{\pgfqpoint{3.196579in}{2.227735in}}{\pgfqpoint{3.202403in}{2.227735in}}%
\pgfpathlineto{\pgfqpoint{3.202403in}{2.227735in}}%
\pgfpathclose%
\pgfusepath{stroke,fill}%
\end{pgfscope}%
\begin{pgfscope}%
\pgfpathrectangle{\pgfqpoint{0.100000in}{0.183744in}}{\pgfqpoint{4.506048in}{4.506048in}}%
\pgfusepath{clip}%
\pgfsetbuttcap%
\pgfsetroundjoin%
\definecolor{currentfill}{rgb}{0.000000,0.000000,1.000000}%
\pgfsetfillcolor{currentfill}%
\pgfsetfillopacity{0.700000}%
\pgfsetlinewidth{1.003750pt}%
\definecolor{currentstroke}{rgb}{0.000000,0.000000,1.000000}%
\pgfsetstrokecolor{currentstroke}%
\pgfsetstrokeopacity{0.700000}%
\pgfsetdash{}{0pt}%
\pgfpathmoveto{\pgfqpoint{3.144413in}{1.810314in}}%
\pgfpathcurveto{\pgfqpoint{3.150237in}{1.810314in}}{\pgfqpoint{3.155823in}{1.812628in}}{\pgfqpoint{3.159941in}{1.816746in}}%
\pgfpathcurveto{\pgfqpoint{3.164060in}{1.820864in}}{\pgfqpoint{3.166373in}{1.826450in}}{\pgfqpoint{3.166373in}{1.832274in}}%
\pgfpathcurveto{\pgfqpoint{3.166373in}{1.838098in}}{\pgfqpoint{3.164060in}{1.843684in}}{\pgfqpoint{3.159941in}{1.847802in}}%
\pgfpathcurveto{\pgfqpoint{3.155823in}{1.851920in}}{\pgfqpoint{3.150237in}{1.854234in}}{\pgfqpoint{3.144413in}{1.854234in}}%
\pgfpathcurveto{\pgfqpoint{3.138589in}{1.854234in}}{\pgfqpoint{3.133003in}{1.851920in}}{\pgfqpoint{3.128885in}{1.847802in}}%
\pgfpathcurveto{\pgfqpoint{3.124767in}{1.843684in}}{\pgfqpoint{3.122453in}{1.838098in}}{\pgfqpoint{3.122453in}{1.832274in}}%
\pgfpathcurveto{\pgfqpoint{3.122453in}{1.826450in}}{\pgfqpoint{3.124767in}{1.820864in}}{\pgfqpoint{3.128885in}{1.816746in}}%
\pgfpathcurveto{\pgfqpoint{3.133003in}{1.812628in}}{\pgfqpoint{3.138589in}{1.810314in}}{\pgfqpoint{3.144413in}{1.810314in}}%
\pgfpathlineto{\pgfqpoint{3.144413in}{1.810314in}}%
\pgfpathclose%
\pgfusepath{stroke,fill}%
\end{pgfscope}%
\begin{pgfscope}%
\pgfpathrectangle{\pgfqpoint{0.100000in}{0.183744in}}{\pgfqpoint{4.506048in}{4.506048in}}%
\pgfusepath{clip}%
\pgfsetbuttcap%
\pgfsetroundjoin%
\definecolor{currentfill}{rgb}{0.000000,0.000000,1.000000}%
\pgfsetfillcolor{currentfill}%
\pgfsetfillopacity{0.700000}%
\pgfsetlinewidth{1.003750pt}%
\definecolor{currentstroke}{rgb}{0.000000,0.000000,1.000000}%
\pgfsetstrokecolor{currentstroke}%
\pgfsetstrokeopacity{0.700000}%
\pgfsetdash{}{0pt}%
\pgfpathmoveto{\pgfqpoint{3.616923in}{2.868006in}}%
\pgfpathcurveto{\pgfqpoint{3.622747in}{2.868006in}}{\pgfqpoint{3.628333in}{2.870320in}}{\pgfqpoint{3.632451in}{2.874438in}}%
\pgfpathcurveto{\pgfqpoint{3.636569in}{2.878556in}}{\pgfqpoint{3.638883in}{2.884142in}}{\pgfqpoint{3.638883in}{2.889966in}}%
\pgfpathcurveto{\pgfqpoint{3.638883in}{2.895790in}}{\pgfqpoint{3.636569in}{2.901376in}}{\pgfqpoint{3.632451in}{2.905495in}}%
\pgfpathcurveto{\pgfqpoint{3.628333in}{2.909613in}}{\pgfqpoint{3.622747in}{2.911927in}}{\pgfqpoint{3.616923in}{2.911927in}}%
\pgfpathcurveto{\pgfqpoint{3.611099in}{2.911927in}}{\pgfqpoint{3.605512in}{2.909613in}}{\pgfqpoint{3.601394in}{2.905495in}}%
\pgfpathcurveto{\pgfqpoint{3.597276in}{2.901376in}}{\pgfqpoint{3.594962in}{2.895790in}}{\pgfqpoint{3.594962in}{2.889966in}}%
\pgfpathcurveto{\pgfqpoint{3.594962in}{2.884142in}}{\pgfqpoint{3.597276in}{2.878556in}}{\pgfqpoint{3.601394in}{2.874438in}}%
\pgfpathcurveto{\pgfqpoint{3.605512in}{2.870320in}}{\pgfqpoint{3.611099in}{2.868006in}}{\pgfqpoint{3.616923in}{2.868006in}}%
\pgfpathlineto{\pgfqpoint{3.616923in}{2.868006in}}%
\pgfpathclose%
\pgfusepath{stroke,fill}%
\end{pgfscope}%
\begin{pgfscope}%
\pgfpathrectangle{\pgfqpoint{0.100000in}{0.183744in}}{\pgfqpoint{4.506048in}{4.506048in}}%
\pgfusepath{clip}%
\pgfsetbuttcap%
\pgfsetroundjoin%
\definecolor{currentfill}{rgb}{0.000000,0.000000,1.000000}%
\pgfsetfillcolor{currentfill}%
\pgfsetfillopacity{0.700000}%
\pgfsetlinewidth{1.003750pt}%
\definecolor{currentstroke}{rgb}{0.000000,0.000000,1.000000}%
\pgfsetstrokecolor{currentstroke}%
\pgfsetstrokeopacity{0.700000}%
\pgfsetdash{}{0pt}%
\pgfpathmoveto{\pgfqpoint{2.935768in}{2.246348in}}%
\pgfpathcurveto{\pgfqpoint{2.941592in}{2.246348in}}{\pgfqpoint{2.947178in}{2.248662in}}{\pgfqpoint{2.951296in}{2.252780in}}%
\pgfpathcurveto{\pgfqpoint{2.955415in}{2.256898in}}{\pgfqpoint{2.957728in}{2.262484in}}{\pgfqpoint{2.957728in}{2.268308in}}%
\pgfpathcurveto{\pgfqpoint{2.957728in}{2.274132in}}{\pgfqpoint{2.955415in}{2.279718in}}{\pgfqpoint{2.951296in}{2.283836in}}%
\pgfpathcurveto{\pgfqpoint{2.947178in}{2.287955in}}{\pgfqpoint{2.941592in}{2.290269in}}{\pgfqpoint{2.935768in}{2.290269in}}%
\pgfpathcurveto{\pgfqpoint{2.929944in}{2.290269in}}{\pgfqpoint{2.924358in}{2.287955in}}{\pgfqpoint{2.920240in}{2.283836in}}%
\pgfpathcurveto{\pgfqpoint{2.916122in}{2.279718in}}{\pgfqpoint{2.913808in}{2.274132in}}{\pgfqpoint{2.913808in}{2.268308in}}%
\pgfpathcurveto{\pgfqpoint{2.913808in}{2.262484in}}{\pgfqpoint{2.916122in}{2.256898in}}{\pgfqpoint{2.920240in}{2.252780in}}%
\pgfpathcurveto{\pgfqpoint{2.924358in}{2.248662in}}{\pgfqpoint{2.929944in}{2.246348in}}{\pgfqpoint{2.935768in}{2.246348in}}%
\pgfpathlineto{\pgfqpoint{2.935768in}{2.246348in}}%
\pgfpathclose%
\pgfusepath{stroke,fill}%
\end{pgfscope}%
\begin{pgfscope}%
\pgfpathrectangle{\pgfqpoint{0.100000in}{0.183744in}}{\pgfqpoint{4.506048in}{4.506048in}}%
\pgfusepath{clip}%
\pgfsetbuttcap%
\pgfsetroundjoin%
\definecolor{currentfill}{rgb}{0.000000,0.000000,1.000000}%
\pgfsetfillcolor{currentfill}%
\pgfsetfillopacity{0.700000}%
\pgfsetlinewidth{1.003750pt}%
\definecolor{currentstroke}{rgb}{0.000000,0.000000,1.000000}%
\pgfsetstrokecolor{currentstroke}%
\pgfsetstrokeopacity{0.700000}%
\pgfsetdash{}{0pt}%
\pgfpathmoveto{\pgfqpoint{1.698520in}{1.762869in}}%
\pgfpathcurveto{\pgfqpoint{1.704343in}{1.762869in}}{\pgfqpoint{1.709930in}{1.765183in}}{\pgfqpoint{1.714048in}{1.769301in}}%
\pgfpathcurveto{\pgfqpoint{1.718166in}{1.773419in}}{\pgfqpoint{1.720480in}{1.779005in}}{\pgfqpoint{1.720480in}{1.784829in}}%
\pgfpathcurveto{\pgfqpoint{1.720480in}{1.790653in}}{\pgfqpoint{1.718166in}{1.796239in}}{\pgfqpoint{1.714048in}{1.800358in}}%
\pgfpathcurveto{\pgfqpoint{1.709930in}{1.804476in}}{\pgfqpoint{1.704343in}{1.806790in}}{\pgfqpoint{1.698520in}{1.806790in}}%
\pgfpathcurveto{\pgfqpoint{1.692696in}{1.806790in}}{\pgfqpoint{1.687109in}{1.804476in}}{\pgfqpoint{1.682991in}{1.800358in}}%
\pgfpathcurveto{\pgfqpoint{1.678873in}{1.796239in}}{\pgfqpoint{1.676559in}{1.790653in}}{\pgfqpoint{1.676559in}{1.784829in}}%
\pgfpathcurveto{\pgfqpoint{1.676559in}{1.779005in}}{\pgfqpoint{1.678873in}{1.773419in}}{\pgfqpoint{1.682991in}{1.769301in}}%
\pgfpathcurveto{\pgfqpoint{1.687109in}{1.765183in}}{\pgfqpoint{1.692696in}{1.762869in}}{\pgfqpoint{1.698520in}{1.762869in}}%
\pgfpathlineto{\pgfqpoint{1.698520in}{1.762869in}}%
\pgfpathclose%
\pgfusepath{stroke,fill}%
\end{pgfscope}%
\begin{pgfscope}%
\pgfpathrectangle{\pgfqpoint{0.100000in}{0.183744in}}{\pgfqpoint{4.506048in}{4.506048in}}%
\pgfusepath{clip}%
\pgfsetbuttcap%
\pgfsetroundjoin%
\definecolor{currentfill}{rgb}{0.000000,0.000000,1.000000}%
\pgfsetfillcolor{currentfill}%
\pgfsetfillopacity{0.700000}%
\pgfsetlinewidth{1.003750pt}%
\definecolor{currentstroke}{rgb}{0.000000,0.000000,1.000000}%
\pgfsetstrokecolor{currentstroke}%
\pgfsetstrokeopacity{0.700000}%
\pgfsetdash{}{0pt}%
\pgfpathmoveto{\pgfqpoint{4.126182in}{2.866741in}}%
\pgfpathcurveto{\pgfqpoint{4.132006in}{2.866741in}}{\pgfqpoint{4.137592in}{2.869055in}}{\pgfqpoint{4.141710in}{2.873173in}}%
\pgfpathcurveto{\pgfqpoint{4.145828in}{2.877291in}}{\pgfqpoint{4.148142in}{2.882877in}}{\pgfqpoint{4.148142in}{2.888701in}}%
\pgfpathcurveto{\pgfqpoint{4.148142in}{2.894525in}}{\pgfqpoint{4.145828in}{2.900111in}}{\pgfqpoint{4.141710in}{2.904229in}}%
\pgfpathcurveto{\pgfqpoint{4.137592in}{2.908347in}}{\pgfqpoint{4.132006in}{2.910661in}}{\pgfqpoint{4.126182in}{2.910661in}}%
\pgfpathcurveto{\pgfqpoint{4.120358in}{2.910661in}}{\pgfqpoint{4.114772in}{2.908347in}}{\pgfqpoint{4.110654in}{2.904229in}}%
\pgfpathcurveto{\pgfqpoint{4.106536in}{2.900111in}}{\pgfqpoint{4.104222in}{2.894525in}}{\pgfqpoint{4.104222in}{2.888701in}}%
\pgfpathcurveto{\pgfqpoint{4.104222in}{2.882877in}}{\pgfqpoint{4.106536in}{2.877291in}}{\pgfqpoint{4.110654in}{2.873173in}}%
\pgfpathcurveto{\pgfqpoint{4.114772in}{2.869055in}}{\pgfqpoint{4.120358in}{2.866741in}}{\pgfqpoint{4.126182in}{2.866741in}}%
\pgfpathlineto{\pgfqpoint{4.126182in}{2.866741in}}%
\pgfpathclose%
\pgfusepath{stroke,fill}%
\end{pgfscope}%
\begin{pgfscope}%
\pgfpathrectangle{\pgfqpoint{0.100000in}{0.183744in}}{\pgfqpoint{4.506048in}{4.506048in}}%
\pgfusepath{clip}%
\pgfsetbuttcap%
\pgfsetroundjoin%
\definecolor{currentfill}{rgb}{0.000000,0.000000,1.000000}%
\pgfsetfillcolor{currentfill}%
\pgfsetfillopacity{0.700000}%
\pgfsetlinewidth{1.003750pt}%
\definecolor{currentstroke}{rgb}{0.000000,0.000000,1.000000}%
\pgfsetstrokecolor{currentstroke}%
\pgfsetstrokeopacity{0.700000}%
\pgfsetdash{}{0pt}%
\pgfpathmoveto{\pgfqpoint{2.034724in}{2.453379in}}%
\pgfpathcurveto{\pgfqpoint{2.040548in}{2.453379in}}{\pgfqpoint{2.046135in}{2.455693in}}{\pgfqpoint{2.050253in}{2.459811in}}%
\pgfpathcurveto{\pgfqpoint{2.054371in}{2.463929in}}{\pgfqpoint{2.056685in}{2.469516in}}{\pgfqpoint{2.056685in}{2.475340in}}%
\pgfpathcurveto{\pgfqpoint{2.056685in}{2.481163in}}{\pgfqpoint{2.054371in}{2.486750in}}{\pgfqpoint{2.050253in}{2.490868in}}%
\pgfpathcurveto{\pgfqpoint{2.046135in}{2.494986in}}{\pgfqpoint{2.040548in}{2.497300in}}{\pgfqpoint{2.034724in}{2.497300in}}%
\pgfpathcurveto{\pgfqpoint{2.028901in}{2.497300in}}{\pgfqpoint{2.023314in}{2.494986in}}{\pgfqpoint{2.019196in}{2.490868in}}%
\pgfpathcurveto{\pgfqpoint{2.015078in}{2.486750in}}{\pgfqpoint{2.012764in}{2.481163in}}{\pgfqpoint{2.012764in}{2.475340in}}%
\pgfpathcurveto{\pgfqpoint{2.012764in}{2.469516in}}{\pgfqpoint{2.015078in}{2.463929in}}{\pgfqpoint{2.019196in}{2.459811in}}%
\pgfpathcurveto{\pgfqpoint{2.023314in}{2.455693in}}{\pgfqpoint{2.028901in}{2.453379in}}{\pgfqpoint{2.034724in}{2.453379in}}%
\pgfpathlineto{\pgfqpoint{2.034724in}{2.453379in}}%
\pgfpathclose%
\pgfusepath{stroke,fill}%
\end{pgfscope}%
\begin{pgfscope}%
\pgfpathrectangle{\pgfqpoint{0.100000in}{0.183744in}}{\pgfqpoint{4.506048in}{4.506048in}}%
\pgfusepath{clip}%
\pgfsetbuttcap%
\pgfsetroundjoin%
\definecolor{currentfill}{rgb}{0.000000,0.000000,1.000000}%
\pgfsetfillcolor{currentfill}%
\pgfsetfillopacity{0.700000}%
\pgfsetlinewidth{1.003750pt}%
\definecolor{currentstroke}{rgb}{0.000000,0.000000,1.000000}%
\pgfsetstrokecolor{currentstroke}%
\pgfsetstrokeopacity{0.700000}%
\pgfsetdash{}{0pt}%
\pgfpathmoveto{\pgfqpoint{2.695711in}{1.162346in}}%
\pgfpathcurveto{\pgfqpoint{2.701535in}{1.162346in}}{\pgfqpoint{2.707121in}{1.164660in}}{\pgfqpoint{2.711239in}{1.168778in}}%
\pgfpathcurveto{\pgfqpoint{2.715358in}{1.172896in}}{\pgfqpoint{2.717671in}{1.178482in}}{\pgfqpoint{2.717671in}{1.184306in}}%
\pgfpathcurveto{\pgfqpoint{2.717671in}{1.190130in}}{\pgfqpoint{2.715358in}{1.195716in}}{\pgfqpoint{2.711239in}{1.199834in}}%
\pgfpathcurveto{\pgfqpoint{2.707121in}{1.203952in}}{\pgfqpoint{2.701535in}{1.206266in}}{\pgfqpoint{2.695711in}{1.206266in}}%
\pgfpathcurveto{\pgfqpoint{2.689887in}{1.206266in}}{\pgfqpoint{2.684301in}{1.203952in}}{\pgfqpoint{2.680183in}{1.199834in}}%
\pgfpathcurveto{\pgfqpoint{2.676065in}{1.195716in}}{\pgfqpoint{2.673751in}{1.190130in}}{\pgfqpoint{2.673751in}{1.184306in}}%
\pgfpathcurveto{\pgfqpoint{2.673751in}{1.178482in}}{\pgfqpoint{2.676065in}{1.172896in}}{\pgfqpoint{2.680183in}{1.168778in}}%
\pgfpathcurveto{\pgfqpoint{2.684301in}{1.164660in}}{\pgfqpoint{2.689887in}{1.162346in}}{\pgfqpoint{2.695711in}{1.162346in}}%
\pgfpathlineto{\pgfqpoint{2.695711in}{1.162346in}}%
\pgfpathclose%
\pgfusepath{stroke,fill}%
\end{pgfscope}%
\begin{pgfscope}%
\pgfpathrectangle{\pgfqpoint{0.100000in}{0.183744in}}{\pgfqpoint{4.506048in}{4.506048in}}%
\pgfusepath{clip}%
\pgfsetbuttcap%
\pgfsetroundjoin%
\definecolor{currentfill}{rgb}{0.000000,0.000000,1.000000}%
\pgfsetfillcolor{currentfill}%
\pgfsetfillopacity{0.700000}%
\pgfsetlinewidth{1.003750pt}%
\definecolor{currentstroke}{rgb}{0.000000,0.000000,1.000000}%
\pgfsetstrokecolor{currentstroke}%
\pgfsetstrokeopacity{0.700000}%
\pgfsetdash{}{0pt}%
\pgfpathmoveto{\pgfqpoint{2.342476in}{2.005621in}}%
\pgfpathcurveto{\pgfqpoint{2.348300in}{2.005621in}}{\pgfqpoint{2.353886in}{2.007934in}}{\pgfqpoint{2.358004in}{2.012053in}}%
\pgfpathcurveto{\pgfqpoint{2.362122in}{2.016171in}}{\pgfqpoint{2.364436in}{2.021757in}}{\pgfqpoint{2.364436in}{2.027581in}}%
\pgfpathcurveto{\pgfqpoint{2.364436in}{2.033405in}}{\pgfqpoint{2.362122in}{2.038991in}}{\pgfqpoint{2.358004in}{2.043109in}}%
\pgfpathcurveto{\pgfqpoint{2.353886in}{2.047227in}}{\pgfqpoint{2.348300in}{2.049541in}}{\pgfqpoint{2.342476in}{2.049541in}}%
\pgfpathcurveto{\pgfqpoint{2.336652in}{2.049541in}}{\pgfqpoint{2.331066in}{2.047227in}}{\pgfqpoint{2.326947in}{2.043109in}}%
\pgfpathcurveto{\pgfqpoint{2.322829in}{2.038991in}}{\pgfqpoint{2.320515in}{2.033405in}}{\pgfqpoint{2.320515in}{2.027581in}}%
\pgfpathcurveto{\pgfqpoint{2.320515in}{2.021757in}}{\pgfqpoint{2.322829in}{2.016171in}}{\pgfqpoint{2.326947in}{2.012053in}}%
\pgfpathcurveto{\pgfqpoint{2.331066in}{2.007934in}}{\pgfqpoint{2.336652in}{2.005621in}}{\pgfqpoint{2.342476in}{2.005621in}}%
\pgfpathlineto{\pgfqpoint{2.342476in}{2.005621in}}%
\pgfpathclose%
\pgfusepath{stroke,fill}%
\end{pgfscope}%
\begin{pgfscope}%
\pgfpathrectangle{\pgfqpoint{0.100000in}{0.183744in}}{\pgfqpoint{4.506048in}{4.506048in}}%
\pgfusepath{clip}%
\pgfsetbuttcap%
\pgfsetroundjoin%
\definecolor{currentfill}{rgb}{0.000000,0.000000,1.000000}%
\pgfsetfillcolor{currentfill}%
\pgfsetfillopacity{0.700000}%
\pgfsetlinewidth{1.003750pt}%
\definecolor{currentstroke}{rgb}{0.000000,0.000000,1.000000}%
\pgfsetstrokecolor{currentstroke}%
\pgfsetstrokeopacity{0.700000}%
\pgfsetdash{}{0pt}%
\pgfpathmoveto{\pgfqpoint{1.866419in}{2.751221in}}%
\pgfpathcurveto{\pgfqpoint{1.872243in}{2.751221in}}{\pgfqpoint{1.877829in}{2.753535in}}{\pgfqpoint{1.881947in}{2.757653in}}%
\pgfpathcurveto{\pgfqpoint{1.886065in}{2.761771in}}{\pgfqpoint{1.888379in}{2.767357in}}{\pgfqpoint{1.888379in}{2.773181in}}%
\pgfpathcurveto{\pgfqpoint{1.888379in}{2.779005in}}{\pgfqpoint{1.886065in}{2.784591in}}{\pgfqpoint{1.881947in}{2.788709in}}%
\pgfpathcurveto{\pgfqpoint{1.877829in}{2.792827in}}{\pgfqpoint{1.872243in}{2.795141in}}{\pgfqpoint{1.866419in}{2.795141in}}%
\pgfpathcurveto{\pgfqpoint{1.860595in}{2.795141in}}{\pgfqpoint{1.855009in}{2.792827in}}{\pgfqpoint{1.850891in}{2.788709in}}%
\pgfpathcurveto{\pgfqpoint{1.846773in}{2.784591in}}{\pgfqpoint{1.844459in}{2.779005in}}{\pgfqpoint{1.844459in}{2.773181in}}%
\pgfpathcurveto{\pgfqpoint{1.844459in}{2.767357in}}{\pgfqpoint{1.846773in}{2.761771in}}{\pgfqpoint{1.850891in}{2.757653in}}%
\pgfpathcurveto{\pgfqpoint{1.855009in}{2.753535in}}{\pgfqpoint{1.860595in}{2.751221in}}{\pgfqpoint{1.866419in}{2.751221in}}%
\pgfpathlineto{\pgfqpoint{1.866419in}{2.751221in}}%
\pgfpathclose%
\pgfusepath{stroke,fill}%
\end{pgfscope}%
\begin{pgfscope}%
\pgfpathrectangle{\pgfqpoint{0.100000in}{0.183744in}}{\pgfqpoint{4.506048in}{4.506048in}}%
\pgfusepath{clip}%
\pgfsetbuttcap%
\pgfsetroundjoin%
\definecolor{currentfill}{rgb}{0.000000,0.000000,1.000000}%
\pgfsetfillcolor{currentfill}%
\pgfsetfillopacity{0.700000}%
\pgfsetlinewidth{1.003750pt}%
\definecolor{currentstroke}{rgb}{0.000000,0.000000,1.000000}%
\pgfsetstrokecolor{currentstroke}%
\pgfsetstrokeopacity{0.700000}%
\pgfsetdash{}{0pt}%
\pgfpathmoveto{\pgfqpoint{4.044399in}{3.184053in}}%
\pgfpathcurveto{\pgfqpoint{4.050223in}{3.184053in}}{\pgfqpoint{4.055810in}{3.186367in}}{\pgfqpoint{4.059928in}{3.190485in}}%
\pgfpathcurveto{\pgfqpoint{4.064046in}{3.194603in}}{\pgfqpoint{4.066360in}{3.200190in}}{\pgfqpoint{4.066360in}{3.206014in}}%
\pgfpathcurveto{\pgfqpoint{4.066360in}{3.211838in}}{\pgfqpoint{4.064046in}{3.217424in}}{\pgfqpoint{4.059928in}{3.221542in}}%
\pgfpathcurveto{\pgfqpoint{4.055810in}{3.225660in}}{\pgfqpoint{4.050223in}{3.227974in}}{\pgfqpoint{4.044399in}{3.227974in}}%
\pgfpathcurveto{\pgfqpoint{4.038575in}{3.227974in}}{\pgfqpoint{4.032989in}{3.225660in}}{\pgfqpoint{4.028871in}{3.221542in}}%
\pgfpathcurveto{\pgfqpoint{4.024753in}{3.217424in}}{\pgfqpoint{4.022439in}{3.211838in}}{\pgfqpoint{4.022439in}{3.206014in}}%
\pgfpathcurveto{\pgfqpoint{4.022439in}{3.200190in}}{\pgfqpoint{4.024753in}{3.194603in}}{\pgfqpoint{4.028871in}{3.190485in}}%
\pgfpathcurveto{\pgfqpoint{4.032989in}{3.186367in}}{\pgfqpoint{4.038575in}{3.184053in}}{\pgfqpoint{4.044399in}{3.184053in}}%
\pgfpathlineto{\pgfqpoint{4.044399in}{3.184053in}}%
\pgfpathclose%
\pgfusepath{stroke,fill}%
\end{pgfscope}%
\begin{pgfscope}%
\pgfpathrectangle{\pgfqpoint{0.100000in}{0.183744in}}{\pgfqpoint{4.506048in}{4.506048in}}%
\pgfusepath{clip}%
\pgfsetbuttcap%
\pgfsetroundjoin%
\definecolor{currentfill}{rgb}{0.000000,0.000000,1.000000}%
\pgfsetfillcolor{currentfill}%
\pgfsetfillopacity{0.700000}%
\pgfsetlinewidth{1.003750pt}%
\definecolor{currentstroke}{rgb}{0.000000,0.000000,1.000000}%
\pgfsetstrokecolor{currentstroke}%
\pgfsetstrokeopacity{0.700000}%
\pgfsetdash{}{0pt}%
\pgfpathmoveto{\pgfqpoint{2.319649in}{2.580565in}}%
\pgfpathcurveto{\pgfqpoint{2.325473in}{2.580565in}}{\pgfqpoint{2.331059in}{2.582879in}}{\pgfqpoint{2.335177in}{2.586997in}}%
\pgfpathcurveto{\pgfqpoint{2.339295in}{2.591115in}}{\pgfqpoint{2.341609in}{2.596701in}}{\pgfqpoint{2.341609in}{2.602525in}}%
\pgfpathcurveto{\pgfqpoint{2.341609in}{2.608349in}}{\pgfqpoint{2.339295in}{2.613935in}}{\pgfqpoint{2.335177in}{2.618054in}}%
\pgfpathcurveto{\pgfqpoint{2.331059in}{2.622172in}}{\pgfqpoint{2.325473in}{2.624486in}}{\pgfqpoint{2.319649in}{2.624486in}}%
\pgfpathcurveto{\pgfqpoint{2.313825in}{2.624486in}}{\pgfqpoint{2.308239in}{2.622172in}}{\pgfqpoint{2.304120in}{2.618054in}}%
\pgfpathcurveto{\pgfqpoint{2.300002in}{2.613935in}}{\pgfqpoint{2.297688in}{2.608349in}}{\pgfqpoint{2.297688in}{2.602525in}}%
\pgfpathcurveto{\pgfqpoint{2.297688in}{2.596701in}}{\pgfqpoint{2.300002in}{2.591115in}}{\pgfqpoint{2.304120in}{2.586997in}}%
\pgfpathcurveto{\pgfqpoint{2.308239in}{2.582879in}}{\pgfqpoint{2.313825in}{2.580565in}}{\pgfqpoint{2.319649in}{2.580565in}}%
\pgfpathlineto{\pgfqpoint{2.319649in}{2.580565in}}%
\pgfpathclose%
\pgfusepath{stroke,fill}%
\end{pgfscope}%
\begin{pgfscope}%
\pgfpathrectangle{\pgfqpoint{0.100000in}{0.183744in}}{\pgfqpoint{4.506048in}{4.506048in}}%
\pgfusepath{clip}%
\pgfsetbuttcap%
\pgfsetroundjoin%
\definecolor{currentfill}{rgb}{0.000000,0.000000,1.000000}%
\pgfsetfillcolor{currentfill}%
\pgfsetfillopacity{0.700000}%
\pgfsetlinewidth{1.003750pt}%
\definecolor{currentstroke}{rgb}{0.000000,0.000000,1.000000}%
\pgfsetstrokecolor{currentstroke}%
\pgfsetstrokeopacity{0.700000}%
\pgfsetdash{}{0pt}%
\pgfpathmoveto{\pgfqpoint{3.626405in}{2.349219in}}%
\pgfpathcurveto{\pgfqpoint{3.632229in}{2.349219in}}{\pgfqpoint{3.637815in}{2.351533in}}{\pgfqpoint{3.641933in}{2.355651in}}%
\pgfpathcurveto{\pgfqpoint{3.646051in}{2.359769in}}{\pgfqpoint{3.648365in}{2.365355in}}{\pgfqpoint{3.648365in}{2.371179in}}%
\pgfpathcurveto{\pgfqpoint{3.648365in}{2.377003in}}{\pgfqpoint{3.646051in}{2.382589in}}{\pgfqpoint{3.641933in}{2.386708in}}%
\pgfpathcurveto{\pgfqpoint{3.637815in}{2.390826in}}{\pgfqpoint{3.632229in}{2.393140in}}{\pgfqpoint{3.626405in}{2.393140in}}%
\pgfpathcurveto{\pgfqpoint{3.620581in}{2.393140in}}{\pgfqpoint{3.614995in}{2.390826in}}{\pgfqpoint{3.610877in}{2.386708in}}%
\pgfpathcurveto{\pgfqpoint{3.606759in}{2.382589in}}{\pgfqpoint{3.604445in}{2.377003in}}{\pgfqpoint{3.604445in}{2.371179in}}%
\pgfpathcurveto{\pgfqpoint{3.604445in}{2.365355in}}{\pgfqpoint{3.606759in}{2.359769in}}{\pgfqpoint{3.610877in}{2.355651in}}%
\pgfpathcurveto{\pgfqpoint{3.614995in}{2.351533in}}{\pgfqpoint{3.620581in}{2.349219in}}{\pgfqpoint{3.626405in}{2.349219in}}%
\pgfpathlineto{\pgfqpoint{3.626405in}{2.349219in}}%
\pgfpathclose%
\pgfusepath{stroke,fill}%
\end{pgfscope}%
\begin{pgfscope}%
\pgfpathrectangle{\pgfqpoint{0.100000in}{0.183744in}}{\pgfqpoint{4.506048in}{4.506048in}}%
\pgfusepath{clip}%
\pgfsetbuttcap%
\pgfsetroundjoin%
\definecolor{currentfill}{rgb}{0.000000,0.000000,1.000000}%
\pgfsetfillcolor{currentfill}%
\pgfsetfillopacity{0.700000}%
\pgfsetlinewidth{1.003750pt}%
\definecolor{currentstroke}{rgb}{0.000000,0.000000,1.000000}%
\pgfsetstrokecolor{currentstroke}%
\pgfsetstrokeopacity{0.700000}%
\pgfsetdash{}{0pt}%
\pgfpathmoveto{\pgfqpoint{3.440177in}{2.232817in}}%
\pgfpathcurveto{\pgfqpoint{3.446001in}{2.232817in}}{\pgfqpoint{3.451587in}{2.235131in}}{\pgfqpoint{3.455705in}{2.239249in}}%
\pgfpathcurveto{\pgfqpoint{3.459823in}{2.243367in}}{\pgfqpoint{3.462137in}{2.248953in}}{\pgfqpoint{3.462137in}{2.254777in}}%
\pgfpathcurveto{\pgfqpoint{3.462137in}{2.260601in}}{\pgfqpoint{3.459823in}{2.266187in}}{\pgfqpoint{3.455705in}{2.270305in}}%
\pgfpathcurveto{\pgfqpoint{3.451587in}{2.274423in}}{\pgfqpoint{3.446001in}{2.276737in}}{\pgfqpoint{3.440177in}{2.276737in}}%
\pgfpathcurveto{\pgfqpoint{3.434353in}{2.276737in}}{\pgfqpoint{3.428767in}{2.274423in}}{\pgfqpoint{3.424649in}{2.270305in}}%
\pgfpathcurveto{\pgfqpoint{3.420531in}{2.266187in}}{\pgfqpoint{3.418217in}{2.260601in}}{\pgfqpoint{3.418217in}{2.254777in}}%
\pgfpathcurveto{\pgfqpoint{3.418217in}{2.248953in}}{\pgfqpoint{3.420531in}{2.243367in}}{\pgfqpoint{3.424649in}{2.239249in}}%
\pgfpathcurveto{\pgfqpoint{3.428767in}{2.235131in}}{\pgfqpoint{3.434353in}{2.232817in}}{\pgfqpoint{3.440177in}{2.232817in}}%
\pgfpathlineto{\pgfqpoint{3.440177in}{2.232817in}}%
\pgfpathclose%
\pgfusepath{stroke,fill}%
\end{pgfscope}%
\begin{pgfscope}%
\pgfpathrectangle{\pgfqpoint{0.100000in}{0.183744in}}{\pgfqpoint{4.506048in}{4.506048in}}%
\pgfusepath{clip}%
\pgfsetbuttcap%
\pgfsetroundjoin%
\definecolor{currentfill}{rgb}{0.000000,0.000000,1.000000}%
\pgfsetfillcolor{currentfill}%
\pgfsetfillopacity{0.700000}%
\pgfsetlinewidth{1.003750pt}%
\definecolor{currentstroke}{rgb}{0.000000,0.000000,1.000000}%
\pgfsetstrokecolor{currentstroke}%
\pgfsetstrokeopacity{0.700000}%
\pgfsetdash{}{0pt}%
\pgfpathmoveto{\pgfqpoint{2.518409in}{1.153629in}}%
\pgfpathcurveto{\pgfqpoint{2.524233in}{1.153629in}}{\pgfqpoint{2.529819in}{1.155942in}}{\pgfqpoint{2.533937in}{1.160061in}}%
\pgfpathcurveto{\pgfqpoint{2.538055in}{1.164179in}}{\pgfqpoint{2.540369in}{1.169765in}}{\pgfqpoint{2.540369in}{1.175589in}}%
\pgfpathcurveto{\pgfqpoint{2.540369in}{1.181413in}}{\pgfqpoint{2.538055in}{1.186999in}}{\pgfqpoint{2.533937in}{1.191117in}}%
\pgfpathcurveto{\pgfqpoint{2.529819in}{1.195235in}}{\pgfqpoint{2.524233in}{1.197549in}}{\pgfqpoint{2.518409in}{1.197549in}}%
\pgfpathcurveto{\pgfqpoint{2.512585in}{1.197549in}}{\pgfqpoint{2.506999in}{1.195235in}}{\pgfqpoint{2.502881in}{1.191117in}}%
\pgfpathcurveto{\pgfqpoint{2.498762in}{1.186999in}}{\pgfqpoint{2.496449in}{1.181413in}}{\pgfqpoint{2.496449in}{1.175589in}}%
\pgfpathcurveto{\pgfqpoint{2.496449in}{1.169765in}}{\pgfqpoint{2.498762in}{1.164179in}}{\pgfqpoint{2.502881in}{1.160061in}}%
\pgfpathcurveto{\pgfqpoint{2.506999in}{1.155942in}}{\pgfqpoint{2.512585in}{1.153629in}}{\pgfqpoint{2.518409in}{1.153629in}}%
\pgfpathlineto{\pgfqpoint{2.518409in}{1.153629in}}%
\pgfpathclose%
\pgfusepath{stroke,fill}%
\end{pgfscope}%
\begin{pgfscope}%
\pgfpathrectangle{\pgfqpoint{0.100000in}{0.183744in}}{\pgfqpoint{4.506048in}{4.506048in}}%
\pgfusepath{clip}%
\pgfsetbuttcap%
\pgfsetroundjoin%
\definecolor{currentfill}{rgb}{0.000000,0.000000,1.000000}%
\pgfsetfillcolor{currentfill}%
\pgfsetfillopacity{0.700000}%
\pgfsetlinewidth{1.003750pt}%
\definecolor{currentstroke}{rgb}{0.000000,0.000000,1.000000}%
\pgfsetstrokecolor{currentstroke}%
\pgfsetstrokeopacity{0.700000}%
\pgfsetdash{}{0pt}%
\pgfpathmoveto{\pgfqpoint{2.546754in}{2.342789in}}%
\pgfpathcurveto{\pgfqpoint{2.552578in}{2.342789in}}{\pgfqpoint{2.558164in}{2.345102in}}{\pgfqpoint{2.562282in}{2.349221in}}%
\pgfpathcurveto{\pgfqpoint{2.566400in}{2.353339in}}{\pgfqpoint{2.568714in}{2.358925in}}{\pgfqpoint{2.568714in}{2.364749in}}%
\pgfpathcurveto{\pgfqpoint{2.568714in}{2.370573in}}{\pgfqpoint{2.566400in}{2.376159in}}{\pgfqpoint{2.562282in}{2.380277in}}%
\pgfpathcurveto{\pgfqpoint{2.558164in}{2.384395in}}{\pgfqpoint{2.552578in}{2.386709in}}{\pgfqpoint{2.546754in}{2.386709in}}%
\pgfpathcurveto{\pgfqpoint{2.540930in}{2.386709in}}{\pgfqpoint{2.535343in}{2.384395in}}{\pgfqpoint{2.531225in}{2.380277in}}%
\pgfpathcurveto{\pgfqpoint{2.527107in}{2.376159in}}{\pgfqpoint{2.524793in}{2.370573in}}{\pgfqpoint{2.524793in}{2.364749in}}%
\pgfpathcurveto{\pgfqpoint{2.524793in}{2.358925in}}{\pgfqpoint{2.527107in}{2.353339in}}{\pgfqpoint{2.531225in}{2.349221in}}%
\pgfpathcurveto{\pgfqpoint{2.535343in}{2.345102in}}{\pgfqpoint{2.540930in}{2.342789in}}{\pgfqpoint{2.546754in}{2.342789in}}%
\pgfpathlineto{\pgfqpoint{2.546754in}{2.342789in}}%
\pgfpathclose%
\pgfusepath{stroke,fill}%
\end{pgfscope}%
\begin{pgfscope}%
\pgfpathrectangle{\pgfqpoint{0.100000in}{0.183744in}}{\pgfqpoint{4.506048in}{4.506048in}}%
\pgfusepath{clip}%
\pgfsetbuttcap%
\pgfsetroundjoin%
\definecolor{currentfill}{rgb}{0.000000,0.000000,1.000000}%
\pgfsetfillcolor{currentfill}%
\pgfsetfillopacity{0.700000}%
\pgfsetlinewidth{1.003750pt}%
\definecolor{currentstroke}{rgb}{0.000000,0.000000,1.000000}%
\pgfsetstrokecolor{currentstroke}%
\pgfsetstrokeopacity{0.700000}%
\pgfsetdash{}{0pt}%
\pgfpathmoveto{\pgfqpoint{1.080998in}{2.731319in}}%
\pgfpathcurveto{\pgfqpoint{1.086822in}{2.731319in}}{\pgfqpoint{1.092408in}{2.733633in}}{\pgfqpoint{1.096526in}{2.737751in}}%
\pgfpathcurveto{\pgfqpoint{1.100644in}{2.741869in}}{\pgfqpoint{1.102958in}{2.747455in}}{\pgfqpoint{1.102958in}{2.753279in}}%
\pgfpathcurveto{\pgfqpoint{1.102958in}{2.759103in}}{\pgfqpoint{1.100644in}{2.764689in}}{\pgfqpoint{1.096526in}{2.768807in}}%
\pgfpathcurveto{\pgfqpoint{1.092408in}{2.772925in}}{\pgfqpoint{1.086822in}{2.775239in}}{\pgfqpoint{1.080998in}{2.775239in}}%
\pgfpathcurveto{\pgfqpoint{1.075174in}{2.775239in}}{\pgfqpoint{1.069588in}{2.772925in}}{\pgfqpoint{1.065470in}{2.768807in}}%
\pgfpathcurveto{\pgfqpoint{1.061351in}{2.764689in}}{\pgfqpoint{1.059038in}{2.759103in}}{\pgfqpoint{1.059038in}{2.753279in}}%
\pgfpathcurveto{\pgfqpoint{1.059038in}{2.747455in}}{\pgfqpoint{1.061351in}{2.741869in}}{\pgfqpoint{1.065470in}{2.737751in}}%
\pgfpathcurveto{\pgfqpoint{1.069588in}{2.733633in}}{\pgfqpoint{1.075174in}{2.731319in}}{\pgfqpoint{1.080998in}{2.731319in}}%
\pgfpathlineto{\pgfqpoint{1.080998in}{2.731319in}}%
\pgfpathclose%
\pgfusepath{stroke,fill}%
\end{pgfscope}%
\begin{pgfscope}%
\pgfpathrectangle{\pgfqpoint{0.100000in}{0.183744in}}{\pgfqpoint{4.506048in}{4.506048in}}%
\pgfusepath{clip}%
\pgfsetbuttcap%
\pgfsetroundjoin%
\definecolor{currentfill}{rgb}{0.000000,0.000000,1.000000}%
\pgfsetfillcolor{currentfill}%
\pgfsetfillopacity{0.700000}%
\pgfsetlinewidth{1.003750pt}%
\definecolor{currentstroke}{rgb}{0.000000,0.000000,1.000000}%
\pgfsetstrokecolor{currentstroke}%
\pgfsetstrokeopacity{0.700000}%
\pgfsetdash{}{0pt}%
\pgfpathmoveto{\pgfqpoint{2.401315in}{1.266631in}}%
\pgfpathcurveto{\pgfqpoint{2.407139in}{1.266631in}}{\pgfqpoint{2.412726in}{1.268945in}}{\pgfqpoint{2.416844in}{1.273063in}}%
\pgfpathcurveto{\pgfqpoint{2.420962in}{1.277181in}}{\pgfqpoint{2.423276in}{1.282767in}}{\pgfqpoint{2.423276in}{1.288591in}}%
\pgfpathcurveto{\pgfqpoint{2.423276in}{1.294415in}}{\pgfqpoint{2.420962in}{1.300001in}}{\pgfqpoint{2.416844in}{1.304120in}}%
\pgfpathcurveto{\pgfqpoint{2.412726in}{1.308238in}}{\pgfqpoint{2.407139in}{1.310552in}}{\pgfqpoint{2.401315in}{1.310552in}}%
\pgfpathcurveto{\pgfqpoint{2.395492in}{1.310552in}}{\pgfqpoint{2.389905in}{1.308238in}}{\pgfqpoint{2.385787in}{1.304120in}}%
\pgfpathcurveto{\pgfqpoint{2.381669in}{1.300001in}}{\pgfqpoint{2.379355in}{1.294415in}}{\pgfqpoint{2.379355in}{1.288591in}}%
\pgfpathcurveto{\pgfqpoint{2.379355in}{1.282767in}}{\pgfqpoint{2.381669in}{1.277181in}}{\pgfqpoint{2.385787in}{1.273063in}}%
\pgfpathcurveto{\pgfqpoint{2.389905in}{1.268945in}}{\pgfqpoint{2.395492in}{1.266631in}}{\pgfqpoint{2.401315in}{1.266631in}}%
\pgfpathlineto{\pgfqpoint{2.401315in}{1.266631in}}%
\pgfpathclose%
\pgfusepath{stroke,fill}%
\end{pgfscope}%
\begin{pgfscope}%
\pgfpathrectangle{\pgfqpoint{0.100000in}{0.183744in}}{\pgfqpoint{4.506048in}{4.506048in}}%
\pgfusepath{clip}%
\pgfsetbuttcap%
\pgfsetroundjoin%
\definecolor{currentfill}{rgb}{0.000000,0.000000,1.000000}%
\pgfsetfillcolor{currentfill}%
\pgfsetfillopacity{0.700000}%
\pgfsetlinewidth{1.003750pt}%
\definecolor{currentstroke}{rgb}{0.000000,0.000000,1.000000}%
\pgfsetstrokecolor{currentstroke}%
\pgfsetstrokeopacity{0.700000}%
\pgfsetdash{}{0pt}%
\pgfpathmoveto{\pgfqpoint{1.974696in}{1.831130in}}%
\pgfpathcurveto{\pgfqpoint{1.980520in}{1.831130in}}{\pgfqpoint{1.986106in}{1.833444in}}{\pgfqpoint{1.990224in}{1.837562in}}%
\pgfpathcurveto{\pgfqpoint{1.994342in}{1.841680in}}{\pgfqpoint{1.996656in}{1.847266in}}{\pgfqpoint{1.996656in}{1.853090in}}%
\pgfpathcurveto{\pgfqpoint{1.996656in}{1.858914in}}{\pgfqpoint{1.994342in}{1.864501in}}{\pgfqpoint{1.990224in}{1.868619in}}%
\pgfpathcurveto{\pgfqpoint{1.986106in}{1.872737in}}{\pgfqpoint{1.980520in}{1.875051in}}{\pgfqpoint{1.974696in}{1.875051in}}%
\pgfpathcurveto{\pgfqpoint{1.968872in}{1.875051in}}{\pgfqpoint{1.963286in}{1.872737in}}{\pgfqpoint{1.959168in}{1.868619in}}%
\pgfpathcurveto{\pgfqpoint{1.955049in}{1.864501in}}{\pgfqpoint{1.952736in}{1.858914in}}{\pgfqpoint{1.952736in}{1.853090in}}%
\pgfpathcurveto{\pgfqpoint{1.952736in}{1.847266in}}{\pgfqpoint{1.955049in}{1.841680in}}{\pgfqpoint{1.959168in}{1.837562in}}%
\pgfpathcurveto{\pgfqpoint{1.963286in}{1.833444in}}{\pgfqpoint{1.968872in}{1.831130in}}{\pgfqpoint{1.974696in}{1.831130in}}%
\pgfpathlineto{\pgfqpoint{1.974696in}{1.831130in}}%
\pgfpathclose%
\pgfusepath{stroke,fill}%
\end{pgfscope}%
\begin{pgfscope}%
\pgfpathrectangle{\pgfqpoint{0.100000in}{0.183744in}}{\pgfqpoint{4.506048in}{4.506048in}}%
\pgfusepath{clip}%
\pgfsetbuttcap%
\pgfsetroundjoin%
\definecolor{currentfill}{rgb}{0.000000,0.000000,1.000000}%
\pgfsetfillcolor{currentfill}%
\pgfsetfillopacity{0.700000}%
\pgfsetlinewidth{1.003750pt}%
\definecolor{currentstroke}{rgb}{0.000000,0.000000,1.000000}%
\pgfsetstrokecolor{currentstroke}%
\pgfsetstrokeopacity{0.700000}%
\pgfsetdash{}{0pt}%
\pgfpathmoveto{\pgfqpoint{4.007957in}{2.758825in}}%
\pgfpathcurveto{\pgfqpoint{4.013781in}{2.758825in}}{\pgfqpoint{4.019367in}{2.761139in}}{\pgfqpoint{4.023485in}{2.765257in}}%
\pgfpathcurveto{\pgfqpoint{4.027604in}{2.769375in}}{\pgfqpoint{4.029917in}{2.774961in}}{\pgfqpoint{4.029917in}{2.780785in}}%
\pgfpathcurveto{\pgfqpoint{4.029917in}{2.786609in}}{\pgfqpoint{4.027604in}{2.792195in}}{\pgfqpoint{4.023485in}{2.796313in}}%
\pgfpathcurveto{\pgfqpoint{4.019367in}{2.800431in}}{\pgfqpoint{4.013781in}{2.802745in}}{\pgfqpoint{4.007957in}{2.802745in}}%
\pgfpathcurveto{\pgfqpoint{4.002133in}{2.802745in}}{\pgfqpoint{3.996547in}{2.800431in}}{\pgfqpoint{3.992429in}{2.796313in}}%
\pgfpathcurveto{\pgfqpoint{3.988311in}{2.792195in}}{\pgfqpoint{3.985997in}{2.786609in}}{\pgfqpoint{3.985997in}{2.780785in}}%
\pgfpathcurveto{\pgfqpoint{3.985997in}{2.774961in}}{\pgfqpoint{3.988311in}{2.769375in}}{\pgfqpoint{3.992429in}{2.765257in}}%
\pgfpathcurveto{\pgfqpoint{3.996547in}{2.761139in}}{\pgfqpoint{4.002133in}{2.758825in}}{\pgfqpoint{4.007957in}{2.758825in}}%
\pgfpathlineto{\pgfqpoint{4.007957in}{2.758825in}}%
\pgfpathclose%
\pgfusepath{stroke,fill}%
\end{pgfscope}%
\begin{pgfscope}%
\pgfpathrectangle{\pgfqpoint{0.100000in}{0.183744in}}{\pgfqpoint{4.506048in}{4.506048in}}%
\pgfusepath{clip}%
\pgfsetbuttcap%
\pgfsetroundjoin%
\definecolor{currentfill}{rgb}{0.000000,0.000000,1.000000}%
\pgfsetfillcolor{currentfill}%
\pgfsetfillopacity{0.700000}%
\pgfsetlinewidth{1.003750pt}%
\definecolor{currentstroke}{rgb}{0.000000,0.000000,1.000000}%
\pgfsetstrokecolor{currentstroke}%
\pgfsetstrokeopacity{0.700000}%
\pgfsetdash{}{0pt}%
\pgfpathmoveto{\pgfqpoint{1.488799in}{2.162420in}}%
\pgfpathcurveto{\pgfqpoint{1.494623in}{2.162420in}}{\pgfqpoint{1.500209in}{2.164733in}}{\pgfqpoint{1.504327in}{2.168852in}}%
\pgfpathcurveto{\pgfqpoint{1.508445in}{2.172970in}}{\pgfqpoint{1.510759in}{2.178556in}}{\pgfqpoint{1.510759in}{2.184380in}}%
\pgfpathcurveto{\pgfqpoint{1.510759in}{2.190204in}}{\pgfqpoint{1.508445in}{2.195790in}}{\pgfqpoint{1.504327in}{2.199908in}}%
\pgfpathcurveto{\pgfqpoint{1.500209in}{2.204026in}}{\pgfqpoint{1.494623in}{2.206340in}}{\pgfqpoint{1.488799in}{2.206340in}}%
\pgfpathcurveto{\pgfqpoint{1.482975in}{2.206340in}}{\pgfqpoint{1.477388in}{2.204026in}}{\pgfqpoint{1.473270in}{2.199908in}}%
\pgfpathcurveto{\pgfqpoint{1.469152in}{2.195790in}}{\pgfqpoint{1.466838in}{2.190204in}}{\pgfqpoint{1.466838in}{2.184380in}}%
\pgfpathcurveto{\pgfqpoint{1.466838in}{2.178556in}}{\pgfqpoint{1.469152in}{2.172970in}}{\pgfqpoint{1.473270in}{2.168852in}}%
\pgfpathcurveto{\pgfqpoint{1.477388in}{2.164733in}}{\pgfqpoint{1.482975in}{2.162420in}}{\pgfqpoint{1.488799in}{2.162420in}}%
\pgfpathlineto{\pgfqpoint{1.488799in}{2.162420in}}%
\pgfpathclose%
\pgfusepath{stroke,fill}%
\end{pgfscope}%
\begin{pgfscope}%
\pgfpathrectangle{\pgfqpoint{0.100000in}{0.183744in}}{\pgfqpoint{4.506048in}{4.506048in}}%
\pgfusepath{clip}%
\pgfsetbuttcap%
\pgfsetroundjoin%
\definecolor{currentfill}{rgb}{0.000000,0.000000,1.000000}%
\pgfsetfillcolor{currentfill}%
\pgfsetfillopacity{0.700000}%
\pgfsetlinewidth{1.003750pt}%
\definecolor{currentstroke}{rgb}{0.000000,0.000000,1.000000}%
\pgfsetstrokecolor{currentstroke}%
\pgfsetstrokeopacity{0.700000}%
\pgfsetdash{}{0pt}%
\pgfpathmoveto{\pgfqpoint{1.666961in}{2.735606in}}%
\pgfpathcurveto{\pgfqpoint{1.672785in}{2.735606in}}{\pgfqpoint{1.678371in}{2.737920in}}{\pgfqpoint{1.682489in}{2.742038in}}%
\pgfpathcurveto{\pgfqpoint{1.686607in}{2.746156in}}{\pgfqpoint{1.688921in}{2.751742in}}{\pgfqpoint{1.688921in}{2.757566in}}%
\pgfpathcurveto{\pgfqpoint{1.688921in}{2.763390in}}{\pgfqpoint{1.686607in}{2.768976in}}{\pgfqpoint{1.682489in}{2.773094in}}%
\pgfpathcurveto{\pgfqpoint{1.678371in}{2.777212in}}{\pgfqpoint{1.672785in}{2.779526in}}{\pgfqpoint{1.666961in}{2.779526in}}%
\pgfpathcurveto{\pgfqpoint{1.661137in}{2.779526in}}{\pgfqpoint{1.655550in}{2.777212in}}{\pgfqpoint{1.651432in}{2.773094in}}%
\pgfpathcurveto{\pgfqpoint{1.647314in}{2.768976in}}{\pgfqpoint{1.645000in}{2.763390in}}{\pgfqpoint{1.645000in}{2.757566in}}%
\pgfpathcurveto{\pgfqpoint{1.645000in}{2.751742in}}{\pgfqpoint{1.647314in}{2.746156in}}{\pgfqpoint{1.651432in}{2.742038in}}%
\pgfpathcurveto{\pgfqpoint{1.655550in}{2.737920in}}{\pgfqpoint{1.661137in}{2.735606in}}{\pgfqpoint{1.666961in}{2.735606in}}%
\pgfpathlineto{\pgfqpoint{1.666961in}{2.735606in}}%
\pgfpathclose%
\pgfusepath{stroke,fill}%
\end{pgfscope}%
\begin{pgfscope}%
\pgfpathrectangle{\pgfqpoint{0.100000in}{0.183744in}}{\pgfqpoint{4.506048in}{4.506048in}}%
\pgfusepath{clip}%
\pgfsetbuttcap%
\pgfsetroundjoin%
\definecolor{currentfill}{rgb}{0.000000,0.000000,1.000000}%
\pgfsetfillcolor{currentfill}%
\pgfsetfillopacity{0.700000}%
\pgfsetlinewidth{1.003750pt}%
\definecolor{currentstroke}{rgb}{0.000000,0.000000,1.000000}%
\pgfsetstrokecolor{currentstroke}%
\pgfsetstrokeopacity{0.700000}%
\pgfsetdash{}{0pt}%
\pgfpathmoveto{\pgfqpoint{2.595896in}{2.078423in}}%
\pgfpathcurveto{\pgfqpoint{2.601720in}{2.078423in}}{\pgfqpoint{2.607306in}{2.080737in}}{\pgfqpoint{2.611424in}{2.084855in}}%
\pgfpathcurveto{\pgfqpoint{2.615542in}{2.088973in}}{\pgfqpoint{2.617856in}{2.094559in}}{\pgfqpoint{2.617856in}{2.100383in}}%
\pgfpathcurveto{\pgfqpoint{2.617856in}{2.106207in}}{\pgfqpoint{2.615542in}{2.111793in}}{\pgfqpoint{2.611424in}{2.115912in}}%
\pgfpathcurveto{\pgfqpoint{2.607306in}{2.120030in}}{\pgfqpoint{2.601720in}{2.122344in}}{\pgfqpoint{2.595896in}{2.122344in}}%
\pgfpathcurveto{\pgfqpoint{2.590072in}{2.122344in}}{\pgfqpoint{2.584486in}{2.120030in}}{\pgfqpoint{2.580368in}{2.115912in}}%
\pgfpathcurveto{\pgfqpoint{2.576249in}{2.111793in}}{\pgfqpoint{2.573936in}{2.106207in}}{\pgfqpoint{2.573936in}{2.100383in}}%
\pgfpathcurveto{\pgfqpoint{2.573936in}{2.094559in}}{\pgfqpoint{2.576249in}{2.088973in}}{\pgfqpoint{2.580368in}{2.084855in}}%
\pgfpathcurveto{\pgfqpoint{2.584486in}{2.080737in}}{\pgfqpoint{2.590072in}{2.078423in}}{\pgfqpoint{2.595896in}{2.078423in}}%
\pgfpathlineto{\pgfqpoint{2.595896in}{2.078423in}}%
\pgfpathclose%
\pgfusepath{stroke,fill}%
\end{pgfscope}%
\begin{pgfscope}%
\pgfpathrectangle{\pgfqpoint{0.100000in}{0.183744in}}{\pgfqpoint{4.506048in}{4.506048in}}%
\pgfusepath{clip}%
\pgfsetbuttcap%
\pgfsetroundjoin%
\definecolor{currentfill}{rgb}{0.000000,0.000000,1.000000}%
\pgfsetfillcolor{currentfill}%
\pgfsetfillopacity{0.700000}%
\pgfsetlinewidth{1.003750pt}%
\definecolor{currentstroke}{rgb}{0.000000,0.000000,1.000000}%
\pgfsetstrokecolor{currentstroke}%
\pgfsetstrokeopacity{0.700000}%
\pgfsetdash{}{0pt}%
\pgfpathmoveto{\pgfqpoint{3.927480in}{2.648352in}}%
\pgfpathcurveto{\pgfqpoint{3.933304in}{2.648352in}}{\pgfqpoint{3.938890in}{2.650666in}}{\pgfqpoint{3.943008in}{2.654784in}}%
\pgfpathcurveto{\pgfqpoint{3.947126in}{2.658902in}}{\pgfqpoint{3.949440in}{2.664488in}}{\pgfqpoint{3.949440in}{2.670312in}}%
\pgfpathcurveto{\pgfqpoint{3.949440in}{2.676136in}}{\pgfqpoint{3.947126in}{2.681723in}}{\pgfqpoint{3.943008in}{2.685841in}}%
\pgfpathcurveto{\pgfqpoint{3.938890in}{2.689959in}}{\pgfqpoint{3.933304in}{2.692273in}}{\pgfqpoint{3.927480in}{2.692273in}}%
\pgfpathcurveto{\pgfqpoint{3.921656in}{2.692273in}}{\pgfqpoint{3.916070in}{2.689959in}}{\pgfqpoint{3.911952in}{2.685841in}}%
\pgfpathcurveto{\pgfqpoint{3.907834in}{2.681723in}}{\pgfqpoint{3.905520in}{2.676136in}}{\pgfqpoint{3.905520in}{2.670312in}}%
\pgfpathcurveto{\pgfqpoint{3.905520in}{2.664488in}}{\pgfqpoint{3.907834in}{2.658902in}}{\pgfqpoint{3.911952in}{2.654784in}}%
\pgfpathcurveto{\pgfqpoint{3.916070in}{2.650666in}}{\pgfqpoint{3.921656in}{2.648352in}}{\pgfqpoint{3.927480in}{2.648352in}}%
\pgfpathlineto{\pgfqpoint{3.927480in}{2.648352in}}%
\pgfpathclose%
\pgfusepath{stroke,fill}%
\end{pgfscope}%
\begin{pgfscope}%
\pgfpathrectangle{\pgfqpoint{0.100000in}{0.183744in}}{\pgfqpoint{4.506048in}{4.506048in}}%
\pgfusepath{clip}%
\pgfsetbuttcap%
\pgfsetroundjoin%
\definecolor{currentfill}{rgb}{0.000000,0.000000,1.000000}%
\pgfsetfillcolor{currentfill}%
\pgfsetfillopacity{0.700000}%
\pgfsetlinewidth{1.003750pt}%
\definecolor{currentstroke}{rgb}{0.000000,0.000000,1.000000}%
\pgfsetstrokecolor{currentstroke}%
\pgfsetstrokeopacity{0.700000}%
\pgfsetdash{}{0pt}%
\pgfpathmoveto{\pgfqpoint{3.025651in}{1.976243in}}%
\pgfpathcurveto{\pgfqpoint{3.031475in}{1.976243in}}{\pgfqpoint{3.037061in}{1.978556in}}{\pgfqpoint{3.041179in}{1.982675in}}%
\pgfpathcurveto{\pgfqpoint{3.045298in}{1.986793in}}{\pgfqpoint{3.047611in}{1.992379in}}{\pgfqpoint{3.047611in}{1.998203in}}%
\pgfpathcurveto{\pgfqpoint{3.047611in}{2.004027in}}{\pgfqpoint{3.045298in}{2.009613in}}{\pgfqpoint{3.041179in}{2.013731in}}%
\pgfpathcurveto{\pgfqpoint{3.037061in}{2.017849in}}{\pgfqpoint{3.031475in}{2.020163in}}{\pgfqpoint{3.025651in}{2.020163in}}%
\pgfpathcurveto{\pgfqpoint{3.019827in}{2.020163in}}{\pgfqpoint{3.014241in}{2.017849in}}{\pgfqpoint{3.010123in}{2.013731in}}%
\pgfpathcurveto{\pgfqpoint{3.006005in}{2.009613in}}{\pgfqpoint{3.003691in}{2.004027in}}{\pgfqpoint{3.003691in}{1.998203in}}%
\pgfpathcurveto{\pgfqpoint{3.003691in}{1.992379in}}{\pgfqpoint{3.006005in}{1.986793in}}{\pgfqpoint{3.010123in}{1.982675in}}%
\pgfpathcurveto{\pgfqpoint{3.014241in}{1.978556in}}{\pgfqpoint{3.019827in}{1.976243in}}{\pgfqpoint{3.025651in}{1.976243in}}%
\pgfpathlineto{\pgfqpoint{3.025651in}{1.976243in}}%
\pgfpathclose%
\pgfusepath{stroke,fill}%
\end{pgfscope}%
\begin{pgfscope}%
\pgfpathrectangle{\pgfqpoint{0.100000in}{0.183744in}}{\pgfqpoint{4.506048in}{4.506048in}}%
\pgfusepath{clip}%
\pgfsetbuttcap%
\pgfsetroundjoin%
\definecolor{currentfill}{rgb}{0.000000,0.000000,1.000000}%
\pgfsetfillcolor{currentfill}%
\pgfsetfillopacity{0.700000}%
\pgfsetlinewidth{1.003750pt}%
\definecolor{currentstroke}{rgb}{0.000000,0.000000,1.000000}%
\pgfsetstrokecolor{currentstroke}%
\pgfsetstrokeopacity{0.700000}%
\pgfsetdash{}{0pt}%
\pgfpathmoveto{\pgfqpoint{2.662982in}{1.748635in}}%
\pgfpathcurveto{\pgfqpoint{2.668806in}{1.748635in}}{\pgfqpoint{2.674393in}{1.750949in}}{\pgfqpoint{2.678511in}{1.755067in}}%
\pgfpathcurveto{\pgfqpoint{2.682629in}{1.759185in}}{\pgfqpoint{2.684943in}{1.764771in}}{\pgfqpoint{2.684943in}{1.770595in}}%
\pgfpathcurveto{\pgfqpoint{2.684943in}{1.776419in}}{\pgfqpoint{2.682629in}{1.782005in}}{\pgfqpoint{2.678511in}{1.786123in}}%
\pgfpathcurveto{\pgfqpoint{2.674393in}{1.790242in}}{\pgfqpoint{2.668806in}{1.792555in}}{\pgfqpoint{2.662982in}{1.792555in}}%
\pgfpathcurveto{\pgfqpoint{2.657159in}{1.792555in}}{\pgfqpoint{2.651572in}{1.790242in}}{\pgfqpoint{2.647454in}{1.786123in}}%
\pgfpathcurveto{\pgfqpoint{2.643336in}{1.782005in}}{\pgfqpoint{2.641022in}{1.776419in}}{\pgfqpoint{2.641022in}{1.770595in}}%
\pgfpathcurveto{\pgfqpoint{2.641022in}{1.764771in}}{\pgfqpoint{2.643336in}{1.759185in}}{\pgfqpoint{2.647454in}{1.755067in}}%
\pgfpathcurveto{\pgfqpoint{2.651572in}{1.750949in}}{\pgfqpoint{2.657159in}{1.748635in}}{\pgfqpoint{2.662982in}{1.748635in}}%
\pgfpathlineto{\pgfqpoint{2.662982in}{1.748635in}}%
\pgfpathclose%
\pgfusepath{stroke,fill}%
\end{pgfscope}%
\begin{pgfscope}%
\pgfpathrectangle{\pgfqpoint{0.100000in}{0.183744in}}{\pgfqpoint{4.506048in}{4.506048in}}%
\pgfusepath{clip}%
\pgfsetbuttcap%
\pgfsetroundjoin%
\definecolor{currentfill}{rgb}{0.000000,0.000000,1.000000}%
\pgfsetfillcolor{currentfill}%
\pgfsetfillopacity{0.700000}%
\pgfsetlinewidth{1.003750pt}%
\definecolor{currentstroke}{rgb}{0.000000,0.000000,1.000000}%
\pgfsetstrokecolor{currentstroke}%
\pgfsetstrokeopacity{0.700000}%
\pgfsetdash{}{0pt}%
\pgfpathmoveto{\pgfqpoint{1.471697in}{1.987713in}}%
\pgfpathcurveto{\pgfqpoint{1.477521in}{1.987713in}}{\pgfqpoint{1.483107in}{1.990027in}}{\pgfqpoint{1.487225in}{1.994145in}}%
\pgfpathcurveto{\pgfqpoint{1.491343in}{1.998263in}}{\pgfqpoint{1.493657in}{2.003849in}}{\pgfqpoint{1.493657in}{2.009673in}}%
\pgfpathcurveto{\pgfqpoint{1.493657in}{2.015497in}}{\pgfqpoint{1.491343in}{2.021083in}}{\pgfqpoint{1.487225in}{2.025202in}}%
\pgfpathcurveto{\pgfqpoint{1.483107in}{2.029320in}}{\pgfqpoint{1.477521in}{2.031634in}}{\pgfqpoint{1.471697in}{2.031634in}}%
\pgfpathcurveto{\pgfqpoint{1.465873in}{2.031634in}}{\pgfqpoint{1.460287in}{2.029320in}}{\pgfqpoint{1.456169in}{2.025202in}}%
\pgfpathcurveto{\pgfqpoint{1.452051in}{2.021083in}}{\pgfqpoint{1.449737in}{2.015497in}}{\pgfqpoint{1.449737in}{2.009673in}}%
\pgfpathcurveto{\pgfqpoint{1.449737in}{2.003849in}}{\pgfqpoint{1.452051in}{1.998263in}}{\pgfqpoint{1.456169in}{1.994145in}}%
\pgfpathcurveto{\pgfqpoint{1.460287in}{1.990027in}}{\pgfqpoint{1.465873in}{1.987713in}}{\pgfqpoint{1.471697in}{1.987713in}}%
\pgfpathlineto{\pgfqpoint{1.471697in}{1.987713in}}%
\pgfpathclose%
\pgfusepath{stroke,fill}%
\end{pgfscope}%
\begin{pgfscope}%
\pgfpathrectangle{\pgfqpoint{0.100000in}{0.183744in}}{\pgfqpoint{4.506048in}{4.506048in}}%
\pgfusepath{clip}%
\pgfsetbuttcap%
\pgfsetroundjoin%
\definecolor{currentfill}{rgb}{0.000000,0.000000,1.000000}%
\pgfsetfillcolor{currentfill}%
\pgfsetfillopacity{0.700000}%
\pgfsetlinewidth{1.003750pt}%
\definecolor{currentstroke}{rgb}{0.000000,0.000000,1.000000}%
\pgfsetstrokecolor{currentstroke}%
\pgfsetstrokeopacity{0.700000}%
\pgfsetdash{}{0pt}%
\pgfpathmoveto{\pgfqpoint{3.896460in}{2.515974in}}%
\pgfpathcurveto{\pgfqpoint{3.902284in}{2.515974in}}{\pgfqpoint{3.907870in}{2.518288in}}{\pgfqpoint{3.911988in}{2.522406in}}%
\pgfpathcurveto{\pgfqpoint{3.916106in}{2.526524in}}{\pgfqpoint{3.918420in}{2.532111in}}{\pgfqpoint{3.918420in}{2.537935in}}%
\pgfpathcurveto{\pgfqpoint{3.918420in}{2.543758in}}{\pgfqpoint{3.916106in}{2.549345in}}{\pgfqpoint{3.911988in}{2.553463in}}%
\pgfpathcurveto{\pgfqpoint{3.907870in}{2.557581in}}{\pgfqpoint{3.902284in}{2.559895in}}{\pgfqpoint{3.896460in}{2.559895in}}%
\pgfpathcurveto{\pgfqpoint{3.890636in}{2.559895in}}{\pgfqpoint{3.885050in}{2.557581in}}{\pgfqpoint{3.880932in}{2.553463in}}%
\pgfpathcurveto{\pgfqpoint{3.876814in}{2.549345in}}{\pgfqpoint{3.874500in}{2.543758in}}{\pgfqpoint{3.874500in}{2.537935in}}%
\pgfpathcurveto{\pgfqpoint{3.874500in}{2.532111in}}{\pgfqpoint{3.876814in}{2.526524in}}{\pgfqpoint{3.880932in}{2.522406in}}%
\pgfpathcurveto{\pgfqpoint{3.885050in}{2.518288in}}{\pgfqpoint{3.890636in}{2.515974in}}{\pgfqpoint{3.896460in}{2.515974in}}%
\pgfpathlineto{\pgfqpoint{3.896460in}{2.515974in}}%
\pgfpathclose%
\pgfusepath{stroke,fill}%
\end{pgfscope}%
\begin{pgfscope}%
\pgfpathrectangle{\pgfqpoint{0.100000in}{0.183744in}}{\pgfqpoint{4.506048in}{4.506048in}}%
\pgfusepath{clip}%
\pgfsetbuttcap%
\pgfsetroundjoin%
\definecolor{currentfill}{rgb}{0.000000,0.000000,1.000000}%
\pgfsetfillcolor{currentfill}%
\pgfsetfillopacity{0.700000}%
\pgfsetlinewidth{1.003750pt}%
\definecolor{currentstroke}{rgb}{0.000000,0.000000,1.000000}%
\pgfsetstrokecolor{currentstroke}%
\pgfsetstrokeopacity{0.700000}%
\pgfsetdash{}{0pt}%
\pgfpathmoveto{\pgfqpoint{1.893392in}{1.855180in}}%
\pgfpathcurveto{\pgfqpoint{1.899216in}{1.855180in}}{\pgfqpoint{1.904802in}{1.857494in}}{\pgfqpoint{1.908920in}{1.861612in}}%
\pgfpathcurveto{\pgfqpoint{1.913038in}{1.865730in}}{\pgfqpoint{1.915352in}{1.871316in}}{\pgfqpoint{1.915352in}{1.877140in}}%
\pgfpathcurveto{\pgfqpoint{1.915352in}{1.882964in}}{\pgfqpoint{1.913038in}{1.888550in}}{\pgfqpoint{1.908920in}{1.892668in}}%
\pgfpathcurveto{\pgfqpoint{1.904802in}{1.896786in}}{\pgfqpoint{1.899216in}{1.899100in}}{\pgfqpoint{1.893392in}{1.899100in}}%
\pgfpathcurveto{\pgfqpoint{1.887568in}{1.899100in}}{\pgfqpoint{1.881982in}{1.896786in}}{\pgfqpoint{1.877864in}{1.892668in}}%
\pgfpathcurveto{\pgfqpoint{1.873746in}{1.888550in}}{\pgfqpoint{1.871432in}{1.882964in}}{\pgfqpoint{1.871432in}{1.877140in}}%
\pgfpathcurveto{\pgfqpoint{1.871432in}{1.871316in}}{\pgfqpoint{1.873746in}{1.865730in}}{\pgfqpoint{1.877864in}{1.861612in}}%
\pgfpathcurveto{\pgfqpoint{1.881982in}{1.857494in}}{\pgfqpoint{1.887568in}{1.855180in}}{\pgfqpoint{1.893392in}{1.855180in}}%
\pgfpathlineto{\pgfqpoint{1.893392in}{1.855180in}}%
\pgfpathclose%
\pgfusepath{stroke,fill}%
\end{pgfscope}%
\begin{pgfscope}%
\pgfpathrectangle{\pgfqpoint{0.100000in}{0.183744in}}{\pgfqpoint{4.506048in}{4.506048in}}%
\pgfusepath{clip}%
\pgfsetbuttcap%
\pgfsetroundjoin%
\definecolor{currentfill}{rgb}{0.000000,0.000000,1.000000}%
\pgfsetfillcolor{currentfill}%
\pgfsetfillopacity{0.700000}%
\pgfsetlinewidth{1.003750pt}%
\definecolor{currentstroke}{rgb}{0.000000,0.000000,1.000000}%
\pgfsetstrokecolor{currentstroke}%
\pgfsetstrokeopacity{0.700000}%
\pgfsetdash{}{0pt}%
\pgfpathmoveto{\pgfqpoint{3.393303in}{2.499577in}}%
\pgfpathcurveto{\pgfqpoint{3.399127in}{2.499577in}}{\pgfqpoint{3.404713in}{2.501891in}}{\pgfqpoint{3.408831in}{2.506009in}}%
\pgfpathcurveto{\pgfqpoint{3.412950in}{2.510128in}}{\pgfqpoint{3.415263in}{2.515714in}}{\pgfqpoint{3.415263in}{2.521538in}}%
\pgfpathcurveto{\pgfqpoint{3.415263in}{2.527362in}}{\pgfqpoint{3.412950in}{2.532948in}}{\pgfqpoint{3.408831in}{2.537066in}}%
\pgfpathcurveto{\pgfqpoint{3.404713in}{2.541184in}}{\pgfqpoint{3.399127in}{2.543498in}}{\pgfqpoint{3.393303in}{2.543498in}}%
\pgfpathcurveto{\pgfqpoint{3.387479in}{2.543498in}}{\pgfqpoint{3.381893in}{2.541184in}}{\pgfqpoint{3.377775in}{2.537066in}}%
\pgfpathcurveto{\pgfqpoint{3.373657in}{2.532948in}}{\pgfqpoint{3.371343in}{2.527362in}}{\pgfqpoint{3.371343in}{2.521538in}}%
\pgfpathcurveto{\pgfqpoint{3.371343in}{2.515714in}}{\pgfqpoint{3.373657in}{2.510128in}}{\pgfqpoint{3.377775in}{2.506009in}}%
\pgfpathcurveto{\pgfqpoint{3.381893in}{2.501891in}}{\pgfqpoint{3.387479in}{2.499577in}}{\pgfqpoint{3.393303in}{2.499577in}}%
\pgfpathlineto{\pgfqpoint{3.393303in}{2.499577in}}%
\pgfpathclose%
\pgfusepath{stroke,fill}%
\end{pgfscope}%
\begin{pgfscope}%
\pgfpathrectangle{\pgfqpoint{0.100000in}{0.183744in}}{\pgfqpoint{4.506048in}{4.506048in}}%
\pgfusepath{clip}%
\pgfsetbuttcap%
\pgfsetroundjoin%
\definecolor{currentfill}{rgb}{0.000000,0.000000,1.000000}%
\pgfsetfillcolor{currentfill}%
\pgfsetfillopacity{0.700000}%
\pgfsetlinewidth{1.003750pt}%
\definecolor{currentstroke}{rgb}{0.000000,0.000000,1.000000}%
\pgfsetstrokecolor{currentstroke}%
\pgfsetstrokeopacity{0.700000}%
\pgfsetdash{}{0pt}%
\pgfpathmoveto{\pgfqpoint{1.727304in}{2.600423in}}%
\pgfpathcurveto{\pgfqpoint{1.733128in}{2.600423in}}{\pgfqpoint{1.738715in}{2.602737in}}{\pgfqpoint{1.742833in}{2.606855in}}%
\pgfpathcurveto{\pgfqpoint{1.746951in}{2.610974in}}{\pgfqpoint{1.749265in}{2.616560in}}{\pgfqpoint{1.749265in}{2.622384in}}%
\pgfpathcurveto{\pgfqpoint{1.749265in}{2.628208in}}{\pgfqpoint{1.746951in}{2.633794in}}{\pgfqpoint{1.742833in}{2.637912in}}%
\pgfpathcurveto{\pgfqpoint{1.738715in}{2.642030in}}{\pgfqpoint{1.733128in}{2.644344in}}{\pgfqpoint{1.727304in}{2.644344in}}%
\pgfpathcurveto{\pgfqpoint{1.721480in}{2.644344in}}{\pgfqpoint{1.715894in}{2.642030in}}{\pgfqpoint{1.711776in}{2.637912in}}%
\pgfpathcurveto{\pgfqpoint{1.707658in}{2.633794in}}{\pgfqpoint{1.705344in}{2.628208in}}{\pgfqpoint{1.705344in}{2.622384in}}%
\pgfpathcurveto{\pgfqpoint{1.705344in}{2.616560in}}{\pgfqpoint{1.707658in}{2.610974in}}{\pgfqpoint{1.711776in}{2.606855in}}%
\pgfpathcurveto{\pgfqpoint{1.715894in}{2.602737in}}{\pgfqpoint{1.721480in}{2.600423in}}{\pgfqpoint{1.727304in}{2.600423in}}%
\pgfpathlineto{\pgfqpoint{1.727304in}{2.600423in}}%
\pgfpathclose%
\pgfusepath{stroke,fill}%
\end{pgfscope}%
\begin{pgfscope}%
\pgfpathrectangle{\pgfqpoint{0.100000in}{0.183744in}}{\pgfqpoint{4.506048in}{4.506048in}}%
\pgfusepath{clip}%
\pgfsetbuttcap%
\pgfsetroundjoin%
\definecolor{currentfill}{rgb}{0.000000,0.000000,1.000000}%
\pgfsetfillcolor{currentfill}%
\pgfsetfillopacity{0.700000}%
\pgfsetlinewidth{1.003750pt}%
\definecolor{currentstroke}{rgb}{0.000000,0.000000,1.000000}%
\pgfsetstrokecolor{currentstroke}%
\pgfsetstrokeopacity{0.700000}%
\pgfsetdash{}{0pt}%
\pgfpathmoveto{\pgfqpoint{2.658269in}{2.081937in}}%
\pgfpathcurveto{\pgfqpoint{2.664093in}{2.081937in}}{\pgfqpoint{2.669679in}{2.084251in}}{\pgfqpoint{2.673797in}{2.088369in}}%
\pgfpathcurveto{\pgfqpoint{2.677916in}{2.092487in}}{\pgfqpoint{2.680229in}{2.098073in}}{\pgfqpoint{2.680229in}{2.103897in}}%
\pgfpathcurveto{\pgfqpoint{2.680229in}{2.109721in}}{\pgfqpoint{2.677916in}{2.115307in}}{\pgfqpoint{2.673797in}{2.119425in}}%
\pgfpathcurveto{\pgfqpoint{2.669679in}{2.123543in}}{\pgfqpoint{2.664093in}{2.125857in}}{\pgfqpoint{2.658269in}{2.125857in}}%
\pgfpathcurveto{\pgfqpoint{2.652445in}{2.125857in}}{\pgfqpoint{2.646859in}{2.123543in}}{\pgfqpoint{2.642741in}{2.119425in}}%
\pgfpathcurveto{\pgfqpoint{2.638623in}{2.115307in}}{\pgfqpoint{2.636309in}{2.109721in}}{\pgfqpoint{2.636309in}{2.103897in}}%
\pgfpathcurveto{\pgfqpoint{2.636309in}{2.098073in}}{\pgfqpoint{2.638623in}{2.092487in}}{\pgfqpoint{2.642741in}{2.088369in}}%
\pgfpathcurveto{\pgfqpoint{2.646859in}{2.084251in}}{\pgfqpoint{2.652445in}{2.081937in}}{\pgfqpoint{2.658269in}{2.081937in}}%
\pgfpathlineto{\pgfqpoint{2.658269in}{2.081937in}}%
\pgfpathclose%
\pgfusepath{stroke,fill}%
\end{pgfscope}%
\begin{pgfscope}%
\pgfpathrectangle{\pgfqpoint{0.100000in}{0.183744in}}{\pgfqpoint{4.506048in}{4.506048in}}%
\pgfusepath{clip}%
\pgfsetbuttcap%
\pgfsetroundjoin%
\definecolor{currentfill}{rgb}{0.000000,0.000000,1.000000}%
\pgfsetfillcolor{currentfill}%
\pgfsetfillopacity{0.700000}%
\pgfsetlinewidth{1.003750pt}%
\definecolor{currentstroke}{rgb}{0.000000,0.000000,1.000000}%
\pgfsetstrokecolor{currentstroke}%
\pgfsetstrokeopacity{0.700000}%
\pgfsetdash{}{0pt}%
\pgfpathmoveto{\pgfqpoint{1.690085in}{2.189862in}}%
\pgfpathcurveto{\pgfqpoint{1.695909in}{2.189862in}}{\pgfqpoint{1.701495in}{2.192176in}}{\pgfqpoint{1.705613in}{2.196294in}}%
\pgfpathcurveto{\pgfqpoint{1.709732in}{2.200412in}}{\pgfqpoint{1.712045in}{2.205998in}}{\pgfqpoint{1.712045in}{2.211822in}}%
\pgfpathcurveto{\pgfqpoint{1.712045in}{2.217646in}}{\pgfqpoint{1.709732in}{2.223232in}}{\pgfqpoint{1.705613in}{2.227350in}}%
\pgfpathcurveto{\pgfqpoint{1.701495in}{2.231468in}}{\pgfqpoint{1.695909in}{2.233782in}}{\pgfqpoint{1.690085in}{2.233782in}}%
\pgfpathcurveto{\pgfqpoint{1.684261in}{2.233782in}}{\pgfqpoint{1.678675in}{2.231468in}}{\pgfqpoint{1.674557in}{2.227350in}}%
\pgfpathcurveto{\pgfqpoint{1.670439in}{2.223232in}}{\pgfqpoint{1.668125in}{2.217646in}}{\pgfqpoint{1.668125in}{2.211822in}}%
\pgfpathcurveto{\pgfqpoint{1.668125in}{2.205998in}}{\pgfqpoint{1.670439in}{2.200412in}}{\pgfqpoint{1.674557in}{2.196294in}}%
\pgfpathcurveto{\pgfqpoint{1.678675in}{2.192176in}}{\pgfqpoint{1.684261in}{2.189862in}}{\pgfqpoint{1.690085in}{2.189862in}}%
\pgfpathlineto{\pgfqpoint{1.690085in}{2.189862in}}%
\pgfpathclose%
\pgfusepath{stroke,fill}%
\end{pgfscope}%
\begin{pgfscope}%
\pgfpathrectangle{\pgfqpoint{0.100000in}{0.183744in}}{\pgfqpoint{4.506048in}{4.506048in}}%
\pgfusepath{clip}%
\pgfsetbuttcap%
\pgfsetroundjoin%
\definecolor{currentfill}{rgb}{0.000000,0.000000,1.000000}%
\pgfsetfillcolor{currentfill}%
\pgfsetfillopacity{0.700000}%
\pgfsetlinewidth{1.003750pt}%
\definecolor{currentstroke}{rgb}{0.000000,0.000000,1.000000}%
\pgfsetstrokecolor{currentstroke}%
\pgfsetstrokeopacity{0.700000}%
\pgfsetdash{}{0pt}%
\pgfpathmoveto{\pgfqpoint{2.376051in}{2.355593in}}%
\pgfpathcurveto{\pgfqpoint{2.381875in}{2.355593in}}{\pgfqpoint{2.387461in}{2.357907in}}{\pgfqpoint{2.391579in}{2.362025in}}%
\pgfpathcurveto{\pgfqpoint{2.395697in}{2.366143in}}{\pgfqpoint{2.398011in}{2.371729in}}{\pgfqpoint{2.398011in}{2.377553in}}%
\pgfpathcurveto{\pgfqpoint{2.398011in}{2.383377in}}{\pgfqpoint{2.395697in}{2.388963in}}{\pgfqpoint{2.391579in}{2.393082in}}%
\pgfpathcurveto{\pgfqpoint{2.387461in}{2.397200in}}{\pgfqpoint{2.381875in}{2.399514in}}{\pgfqpoint{2.376051in}{2.399514in}}%
\pgfpathcurveto{\pgfqpoint{2.370227in}{2.399514in}}{\pgfqpoint{2.364641in}{2.397200in}}{\pgfqpoint{2.360522in}{2.393082in}}%
\pgfpathcurveto{\pgfqpoint{2.356404in}{2.388963in}}{\pgfqpoint{2.354090in}{2.383377in}}{\pgfqpoint{2.354090in}{2.377553in}}%
\pgfpathcurveto{\pgfqpoint{2.354090in}{2.371729in}}{\pgfqpoint{2.356404in}{2.366143in}}{\pgfqpoint{2.360522in}{2.362025in}}%
\pgfpathcurveto{\pgfqpoint{2.364641in}{2.357907in}}{\pgfqpoint{2.370227in}{2.355593in}}{\pgfqpoint{2.376051in}{2.355593in}}%
\pgfpathlineto{\pgfqpoint{2.376051in}{2.355593in}}%
\pgfpathclose%
\pgfusepath{stroke,fill}%
\end{pgfscope}%
\begin{pgfscope}%
\pgfpathrectangle{\pgfqpoint{0.100000in}{0.183744in}}{\pgfqpoint{4.506048in}{4.506048in}}%
\pgfusepath{clip}%
\pgfsetbuttcap%
\pgfsetroundjoin%
\definecolor{currentfill}{rgb}{0.000000,0.000000,1.000000}%
\pgfsetfillcolor{currentfill}%
\pgfsetfillopacity{0.700000}%
\pgfsetlinewidth{1.003750pt}%
\definecolor{currentstroke}{rgb}{0.000000,0.000000,1.000000}%
\pgfsetstrokecolor{currentstroke}%
\pgfsetstrokeopacity{0.700000}%
\pgfsetdash{}{0pt}%
\pgfpathmoveto{\pgfqpoint{3.047274in}{2.176100in}}%
\pgfpathcurveto{\pgfqpoint{3.053097in}{2.176100in}}{\pgfqpoint{3.058684in}{2.178414in}}{\pgfqpoint{3.062802in}{2.182532in}}%
\pgfpathcurveto{\pgfqpoint{3.066920in}{2.186650in}}{\pgfqpoint{3.069234in}{2.192236in}}{\pgfqpoint{3.069234in}{2.198060in}}%
\pgfpathcurveto{\pgfqpoint{3.069234in}{2.203884in}}{\pgfqpoint{3.066920in}{2.209470in}}{\pgfqpoint{3.062802in}{2.213589in}}%
\pgfpathcurveto{\pgfqpoint{3.058684in}{2.217707in}}{\pgfqpoint{3.053097in}{2.220021in}}{\pgfqpoint{3.047274in}{2.220021in}}%
\pgfpathcurveto{\pgfqpoint{3.041450in}{2.220021in}}{\pgfqpoint{3.035863in}{2.217707in}}{\pgfqpoint{3.031745in}{2.213589in}}%
\pgfpathcurveto{\pgfqpoint{3.027627in}{2.209470in}}{\pgfqpoint{3.025313in}{2.203884in}}{\pgfqpoint{3.025313in}{2.198060in}}%
\pgfpathcurveto{\pgfqpoint{3.025313in}{2.192236in}}{\pgfqpoint{3.027627in}{2.186650in}}{\pgfqpoint{3.031745in}{2.182532in}}%
\pgfpathcurveto{\pgfqpoint{3.035863in}{2.178414in}}{\pgfqpoint{3.041450in}{2.176100in}}{\pgfqpoint{3.047274in}{2.176100in}}%
\pgfpathlineto{\pgfqpoint{3.047274in}{2.176100in}}%
\pgfpathclose%
\pgfusepath{stroke,fill}%
\end{pgfscope}%
\begin{pgfscope}%
\pgfpathrectangle{\pgfqpoint{0.100000in}{0.183744in}}{\pgfqpoint{4.506048in}{4.506048in}}%
\pgfusepath{clip}%
\pgfsetbuttcap%
\pgfsetroundjoin%
\definecolor{currentfill}{rgb}{0.000000,0.000000,1.000000}%
\pgfsetfillcolor{currentfill}%
\pgfsetfillopacity{0.700000}%
\pgfsetlinewidth{1.003750pt}%
\definecolor{currentstroke}{rgb}{0.000000,0.000000,1.000000}%
\pgfsetstrokecolor{currentstroke}%
\pgfsetstrokeopacity{0.700000}%
\pgfsetdash{}{0pt}%
\pgfpathmoveto{\pgfqpoint{2.438160in}{2.386066in}}%
\pgfpathcurveto{\pgfqpoint{2.443984in}{2.386066in}}{\pgfqpoint{2.449570in}{2.388380in}}{\pgfqpoint{2.453688in}{2.392498in}}%
\pgfpathcurveto{\pgfqpoint{2.457806in}{2.396617in}}{\pgfqpoint{2.460120in}{2.402203in}}{\pgfqpoint{2.460120in}{2.408027in}}%
\pgfpathcurveto{\pgfqpoint{2.460120in}{2.413851in}}{\pgfqpoint{2.457806in}{2.419437in}}{\pgfqpoint{2.453688in}{2.423555in}}%
\pgfpathcurveto{\pgfqpoint{2.449570in}{2.427673in}}{\pgfqpoint{2.443984in}{2.429987in}}{\pgfqpoint{2.438160in}{2.429987in}}%
\pgfpathcurveto{\pgfqpoint{2.432336in}{2.429987in}}{\pgfqpoint{2.426750in}{2.427673in}}{\pgfqpoint{2.422632in}{2.423555in}}%
\pgfpathcurveto{\pgfqpoint{2.418514in}{2.419437in}}{\pgfqpoint{2.416200in}{2.413851in}}{\pgfqpoint{2.416200in}{2.408027in}}%
\pgfpathcurveto{\pgfqpoint{2.416200in}{2.402203in}}{\pgfqpoint{2.418514in}{2.396617in}}{\pgfqpoint{2.422632in}{2.392498in}}%
\pgfpathcurveto{\pgfqpoint{2.426750in}{2.388380in}}{\pgfqpoint{2.432336in}{2.386066in}}{\pgfqpoint{2.438160in}{2.386066in}}%
\pgfpathlineto{\pgfqpoint{2.438160in}{2.386066in}}%
\pgfpathclose%
\pgfusepath{stroke,fill}%
\end{pgfscope}%
\begin{pgfscope}%
\pgfpathrectangle{\pgfqpoint{0.100000in}{0.183744in}}{\pgfqpoint{4.506048in}{4.506048in}}%
\pgfusepath{clip}%
\pgfsetbuttcap%
\pgfsetroundjoin%
\definecolor{currentfill}{rgb}{0.000000,0.000000,1.000000}%
\pgfsetfillcolor{currentfill}%
\pgfsetfillopacity{0.700000}%
\pgfsetlinewidth{1.003750pt}%
\definecolor{currentstroke}{rgb}{0.000000,0.000000,1.000000}%
\pgfsetstrokecolor{currentstroke}%
\pgfsetstrokeopacity{0.700000}%
\pgfsetdash{}{0pt}%
\pgfpathmoveto{\pgfqpoint{2.381488in}{2.436640in}}%
\pgfpathcurveto{\pgfqpoint{2.387312in}{2.436640in}}{\pgfqpoint{2.392898in}{2.438954in}}{\pgfqpoint{2.397016in}{2.443072in}}%
\pgfpathcurveto{\pgfqpoint{2.401134in}{2.447190in}}{\pgfqpoint{2.403448in}{2.452776in}}{\pgfqpoint{2.403448in}{2.458600in}}%
\pgfpathcurveto{\pgfqpoint{2.403448in}{2.464424in}}{\pgfqpoint{2.401134in}{2.470010in}}{\pgfqpoint{2.397016in}{2.474128in}}%
\pgfpathcurveto{\pgfqpoint{2.392898in}{2.478247in}}{\pgfqpoint{2.387312in}{2.480560in}}{\pgfqpoint{2.381488in}{2.480560in}}%
\pgfpathcurveto{\pgfqpoint{2.375664in}{2.480560in}}{\pgfqpoint{2.370078in}{2.478247in}}{\pgfqpoint{2.365960in}{2.474128in}}%
\pgfpathcurveto{\pgfqpoint{2.361841in}{2.470010in}}{\pgfqpoint{2.359528in}{2.464424in}}{\pgfqpoint{2.359528in}{2.458600in}}%
\pgfpathcurveto{\pgfqpoint{2.359528in}{2.452776in}}{\pgfqpoint{2.361841in}{2.447190in}}{\pgfqpoint{2.365960in}{2.443072in}}%
\pgfpathcurveto{\pgfqpoint{2.370078in}{2.438954in}}{\pgfqpoint{2.375664in}{2.436640in}}{\pgfqpoint{2.381488in}{2.436640in}}%
\pgfpathlineto{\pgfqpoint{2.381488in}{2.436640in}}%
\pgfpathclose%
\pgfusepath{stroke,fill}%
\end{pgfscope}%
\begin{pgfscope}%
\pgfpathrectangle{\pgfqpoint{0.100000in}{0.183744in}}{\pgfqpoint{4.506048in}{4.506048in}}%
\pgfusepath{clip}%
\pgfsetbuttcap%
\pgfsetroundjoin%
\definecolor{currentfill}{rgb}{0.000000,0.000000,1.000000}%
\pgfsetfillcolor{currentfill}%
\pgfsetfillopacity{0.700000}%
\pgfsetlinewidth{1.003750pt}%
\definecolor{currentstroke}{rgb}{0.000000,0.000000,1.000000}%
\pgfsetstrokecolor{currentstroke}%
\pgfsetstrokeopacity{0.700000}%
\pgfsetdash{}{0pt}%
\pgfpathmoveto{\pgfqpoint{1.862237in}{2.313799in}}%
\pgfpathcurveto{\pgfqpoint{1.868061in}{2.313799in}}{\pgfqpoint{1.873647in}{2.316113in}}{\pgfqpoint{1.877766in}{2.320231in}}%
\pgfpathcurveto{\pgfqpoint{1.881884in}{2.324350in}}{\pgfqpoint{1.884198in}{2.329936in}}{\pgfqpoint{1.884198in}{2.335760in}}%
\pgfpathcurveto{\pgfqpoint{1.884198in}{2.341584in}}{\pgfqpoint{1.881884in}{2.347170in}}{\pgfqpoint{1.877766in}{2.351288in}}%
\pgfpathcurveto{\pgfqpoint{1.873647in}{2.355406in}}{\pgfqpoint{1.868061in}{2.357720in}}{\pgfqpoint{1.862237in}{2.357720in}}%
\pgfpathcurveto{\pgfqpoint{1.856413in}{2.357720in}}{\pgfqpoint{1.850827in}{2.355406in}}{\pgfqpoint{1.846709in}{2.351288in}}%
\pgfpathcurveto{\pgfqpoint{1.842591in}{2.347170in}}{\pgfqpoint{1.840277in}{2.341584in}}{\pgfqpoint{1.840277in}{2.335760in}}%
\pgfpathcurveto{\pgfqpoint{1.840277in}{2.329936in}}{\pgfqpoint{1.842591in}{2.324350in}}{\pgfqpoint{1.846709in}{2.320231in}}%
\pgfpathcurveto{\pgfqpoint{1.850827in}{2.316113in}}{\pgfqpoint{1.856413in}{2.313799in}}{\pgfqpoint{1.862237in}{2.313799in}}%
\pgfpathlineto{\pgfqpoint{1.862237in}{2.313799in}}%
\pgfpathclose%
\pgfusepath{stroke,fill}%
\end{pgfscope}%
\begin{pgfscope}%
\pgfpathrectangle{\pgfqpoint{0.100000in}{0.183744in}}{\pgfqpoint{4.506048in}{4.506048in}}%
\pgfusepath{clip}%
\pgfsetbuttcap%
\pgfsetroundjoin%
\definecolor{currentfill}{rgb}{0.000000,0.000000,1.000000}%
\pgfsetfillcolor{currentfill}%
\pgfsetfillopacity{0.700000}%
\pgfsetlinewidth{1.003750pt}%
\definecolor{currentstroke}{rgb}{0.000000,0.000000,1.000000}%
\pgfsetstrokecolor{currentstroke}%
\pgfsetstrokeopacity{0.700000}%
\pgfsetdash{}{0pt}%
\pgfpathmoveto{\pgfqpoint{1.719428in}{1.395710in}}%
\pgfpathcurveto{\pgfqpoint{1.725251in}{1.395710in}}{\pgfqpoint{1.730838in}{1.398024in}}{\pgfqpoint{1.734956in}{1.402142in}}%
\pgfpathcurveto{\pgfqpoint{1.739074in}{1.406260in}}{\pgfqpoint{1.741388in}{1.411846in}}{\pgfqpoint{1.741388in}{1.417670in}}%
\pgfpathcurveto{\pgfqpoint{1.741388in}{1.423494in}}{\pgfqpoint{1.739074in}{1.429080in}}{\pgfqpoint{1.734956in}{1.433199in}}%
\pgfpathcurveto{\pgfqpoint{1.730838in}{1.437317in}}{\pgfqpoint{1.725251in}{1.439631in}}{\pgfqpoint{1.719428in}{1.439631in}}%
\pgfpathcurveto{\pgfqpoint{1.713604in}{1.439631in}}{\pgfqpoint{1.708017in}{1.437317in}}{\pgfqpoint{1.703899in}{1.433199in}}%
\pgfpathcurveto{\pgfqpoint{1.699781in}{1.429080in}}{\pgfqpoint{1.697467in}{1.423494in}}{\pgfqpoint{1.697467in}{1.417670in}}%
\pgfpathcurveto{\pgfqpoint{1.697467in}{1.411846in}}{\pgfqpoint{1.699781in}{1.406260in}}{\pgfqpoint{1.703899in}{1.402142in}}%
\pgfpathcurveto{\pgfqpoint{1.708017in}{1.398024in}}{\pgfqpoint{1.713604in}{1.395710in}}{\pgfqpoint{1.719428in}{1.395710in}}%
\pgfpathlineto{\pgfqpoint{1.719428in}{1.395710in}}%
\pgfpathclose%
\pgfusepath{stroke,fill}%
\end{pgfscope}%
\begin{pgfscope}%
\pgfpathrectangle{\pgfqpoint{0.100000in}{0.183744in}}{\pgfqpoint{4.506048in}{4.506048in}}%
\pgfusepath{clip}%
\pgfsetbuttcap%
\pgfsetroundjoin%
\definecolor{currentfill}{rgb}{0.000000,0.000000,1.000000}%
\pgfsetfillcolor{currentfill}%
\pgfsetfillopacity{0.700000}%
\pgfsetlinewidth{1.003750pt}%
\definecolor{currentstroke}{rgb}{0.000000,0.000000,1.000000}%
\pgfsetstrokecolor{currentstroke}%
\pgfsetstrokeopacity{0.700000}%
\pgfsetdash{}{0pt}%
\pgfpathmoveto{\pgfqpoint{2.119575in}{1.621202in}}%
\pgfpathcurveto{\pgfqpoint{2.125399in}{1.621202in}}{\pgfqpoint{2.130985in}{1.623516in}}{\pgfqpoint{2.135103in}{1.627634in}}%
\pgfpathcurveto{\pgfqpoint{2.139221in}{1.631752in}}{\pgfqpoint{2.141535in}{1.637338in}}{\pgfqpoint{2.141535in}{1.643162in}}%
\pgfpathcurveto{\pgfqpoint{2.141535in}{1.648986in}}{\pgfqpoint{2.139221in}{1.654572in}}{\pgfqpoint{2.135103in}{1.658691in}}%
\pgfpathcurveto{\pgfqpoint{2.130985in}{1.662809in}}{\pgfqpoint{2.125399in}{1.665123in}}{\pgfqpoint{2.119575in}{1.665123in}}%
\pgfpathcurveto{\pgfqpoint{2.113751in}{1.665123in}}{\pgfqpoint{2.108165in}{1.662809in}}{\pgfqpoint{2.104047in}{1.658691in}}%
\pgfpathcurveto{\pgfqpoint{2.099928in}{1.654572in}}{\pgfqpoint{2.097615in}{1.648986in}}{\pgfqpoint{2.097615in}{1.643162in}}%
\pgfpathcurveto{\pgfqpoint{2.097615in}{1.637338in}}{\pgfqpoint{2.099928in}{1.631752in}}{\pgfqpoint{2.104047in}{1.627634in}}%
\pgfpathcurveto{\pgfqpoint{2.108165in}{1.623516in}}{\pgfqpoint{2.113751in}{1.621202in}}{\pgfqpoint{2.119575in}{1.621202in}}%
\pgfpathlineto{\pgfqpoint{2.119575in}{1.621202in}}%
\pgfpathclose%
\pgfusepath{stroke,fill}%
\end{pgfscope}%
\begin{pgfscope}%
\pgfpathrectangle{\pgfqpoint{0.100000in}{0.183744in}}{\pgfqpoint{4.506048in}{4.506048in}}%
\pgfusepath{clip}%
\pgfsetbuttcap%
\pgfsetroundjoin%
\definecolor{currentfill}{rgb}{0.000000,0.000000,1.000000}%
\pgfsetfillcolor{currentfill}%
\pgfsetfillopacity{0.700000}%
\pgfsetlinewidth{1.003750pt}%
\definecolor{currentstroke}{rgb}{0.000000,0.000000,1.000000}%
\pgfsetstrokecolor{currentstroke}%
\pgfsetstrokeopacity{0.700000}%
\pgfsetdash{}{0pt}%
\pgfpathmoveto{\pgfqpoint{2.046802in}{1.936188in}}%
\pgfpathcurveto{\pgfqpoint{2.052626in}{1.936188in}}{\pgfqpoint{2.058212in}{1.938501in}}{\pgfqpoint{2.062330in}{1.942620in}}%
\pgfpathcurveto{\pgfqpoint{2.066448in}{1.946738in}}{\pgfqpoint{2.068762in}{1.952324in}}{\pgfqpoint{2.068762in}{1.958148in}}%
\pgfpathcurveto{\pgfqpoint{2.068762in}{1.963972in}}{\pgfqpoint{2.066448in}{1.969558in}}{\pgfqpoint{2.062330in}{1.973676in}}%
\pgfpathcurveto{\pgfqpoint{2.058212in}{1.977794in}}{\pgfqpoint{2.052626in}{1.980108in}}{\pgfqpoint{2.046802in}{1.980108in}}%
\pgfpathcurveto{\pgfqpoint{2.040978in}{1.980108in}}{\pgfqpoint{2.035392in}{1.977794in}}{\pgfqpoint{2.031274in}{1.973676in}}%
\pgfpathcurveto{\pgfqpoint{2.027155in}{1.969558in}}{\pgfqpoint{2.024841in}{1.963972in}}{\pgfqpoint{2.024841in}{1.958148in}}%
\pgfpathcurveto{\pgfqpoint{2.024841in}{1.952324in}}{\pgfqpoint{2.027155in}{1.946738in}}{\pgfqpoint{2.031274in}{1.942620in}}%
\pgfpathcurveto{\pgfqpoint{2.035392in}{1.938501in}}{\pgfqpoint{2.040978in}{1.936188in}}{\pgfqpoint{2.046802in}{1.936188in}}%
\pgfpathlineto{\pgfqpoint{2.046802in}{1.936188in}}%
\pgfpathclose%
\pgfusepath{stroke,fill}%
\end{pgfscope}%
\begin{pgfscope}%
\pgfpathrectangle{\pgfqpoint{0.100000in}{0.183744in}}{\pgfqpoint{4.506048in}{4.506048in}}%
\pgfusepath{clip}%
\pgfsetbuttcap%
\pgfsetroundjoin%
\definecolor{currentfill}{rgb}{0.000000,0.000000,1.000000}%
\pgfsetfillcolor{currentfill}%
\pgfsetfillopacity{0.700000}%
\pgfsetlinewidth{1.003750pt}%
\definecolor{currentstroke}{rgb}{0.000000,0.000000,1.000000}%
\pgfsetstrokecolor{currentstroke}%
\pgfsetstrokeopacity{0.700000}%
\pgfsetdash{}{0pt}%
\pgfpathmoveto{\pgfqpoint{1.695908in}{1.815000in}}%
\pgfpathcurveto{\pgfqpoint{1.701732in}{1.815000in}}{\pgfqpoint{1.707318in}{1.817314in}}{\pgfqpoint{1.711436in}{1.821432in}}%
\pgfpathcurveto{\pgfqpoint{1.715554in}{1.825551in}}{\pgfqpoint{1.717868in}{1.831137in}}{\pgfqpoint{1.717868in}{1.836961in}}%
\pgfpathcurveto{\pgfqpoint{1.717868in}{1.842785in}}{\pgfqpoint{1.715554in}{1.848371in}}{\pgfqpoint{1.711436in}{1.852489in}}%
\pgfpathcurveto{\pgfqpoint{1.707318in}{1.856607in}}{\pgfqpoint{1.701732in}{1.858921in}}{\pgfqpoint{1.695908in}{1.858921in}}%
\pgfpathcurveto{\pgfqpoint{1.690084in}{1.858921in}}{\pgfqpoint{1.684498in}{1.856607in}}{\pgfqpoint{1.680380in}{1.852489in}}%
\pgfpathcurveto{\pgfqpoint{1.676261in}{1.848371in}}{\pgfqpoint{1.673948in}{1.842785in}}{\pgfqpoint{1.673948in}{1.836961in}}%
\pgfpathcurveto{\pgfqpoint{1.673948in}{1.831137in}}{\pgfqpoint{1.676261in}{1.825551in}}{\pgfqpoint{1.680380in}{1.821432in}}%
\pgfpathcurveto{\pgfqpoint{1.684498in}{1.817314in}}{\pgfqpoint{1.690084in}{1.815000in}}{\pgfqpoint{1.695908in}{1.815000in}}%
\pgfpathlineto{\pgfqpoint{1.695908in}{1.815000in}}%
\pgfpathclose%
\pgfusepath{stroke,fill}%
\end{pgfscope}%
\begin{pgfscope}%
\pgfsetbuttcap%
\pgfsetmiterjoin%
\definecolor{currentfill}{rgb}{1.000000,1.000000,1.000000}%
\pgfsetfillcolor{currentfill}%
\pgfsetfillopacity{0.800000}%
\pgfsetlinewidth{1.003750pt}%
\definecolor{currentstroke}{rgb}{0.800000,0.800000,0.800000}%
\pgfsetstrokecolor{currentstroke}%
\pgfsetstrokeopacity{0.800000}%
\pgfsetdash{}{0pt}%
\pgfpathmoveto{\pgfqpoint{4.236176in}{3.701828in}}%
\pgfpathlineto{\pgfqpoint{4.949709in}{3.701828in}}%
\pgfpathquadraticcurveto{\pgfqpoint{4.980264in}{3.701828in}}{\pgfqpoint{4.980264in}{3.732383in}}%
\pgfpathlineto{\pgfqpoint{4.980264in}{4.582847in}}%
\pgfpathquadraticcurveto{\pgfqpoint{4.980264in}{4.613403in}}{\pgfqpoint{4.949709in}{4.613403in}}%
\pgfpathlineto{\pgfqpoint{4.236176in}{4.613403in}}%
\pgfpathquadraticcurveto{\pgfqpoint{4.205620in}{4.613403in}}{\pgfqpoint{4.205620in}{4.582847in}}%
\pgfpathlineto{\pgfqpoint{4.205620in}{3.732383in}}%
\pgfpathquadraticcurveto{\pgfqpoint{4.205620in}{3.701828in}}{\pgfqpoint{4.236176in}{3.701828in}}%
\pgfpathlineto{\pgfqpoint{4.236176in}{3.701828in}}%
\pgfpathclose%
\pgfusepath{stroke,fill}%
\end{pgfscope}%
\begin{pgfscope}%
\pgfsetbuttcap%
\pgfsetroundjoin%
\definecolor{currentfill}{rgb}{0.000000,0.000000,1.000000}%
\pgfsetfillcolor{currentfill}%
\pgfsetfillopacity{0.700000}%
\pgfsetlinewidth{1.003750pt}%
\definecolor{currentstroke}{rgb}{0.000000,0.000000,1.000000}%
\pgfsetstrokecolor{currentstroke}%
\pgfsetstrokeopacity{0.700000}%
\pgfsetdash{}{0pt}%
\pgfsys@defobject{currentmarker}{\pgfqpoint{-0.021960in}{-0.021960in}}{\pgfqpoint{0.021960in}{0.021960in}}{%
\pgfpathmoveto{\pgfqpoint{0.000000in}{-0.021960in}}%
\pgfpathcurveto{\pgfqpoint{0.005824in}{-0.021960in}}{\pgfqpoint{0.011410in}{-0.019646in}}{\pgfqpoint{0.015528in}{-0.015528in}}%
\pgfpathcurveto{\pgfqpoint{0.019646in}{-0.011410in}}{\pgfqpoint{0.021960in}{-0.005824in}}{\pgfqpoint{0.021960in}{0.000000in}}%
\pgfpathcurveto{\pgfqpoint{0.021960in}{0.005824in}}{\pgfqpoint{0.019646in}{0.011410in}}{\pgfqpoint{0.015528in}{0.015528in}}%
\pgfpathcurveto{\pgfqpoint{0.011410in}{0.019646in}}{\pgfqpoint{0.005824in}{0.021960in}}{\pgfqpoint{0.000000in}{0.021960in}}%
\pgfpathcurveto{\pgfqpoint{-0.005824in}{0.021960in}}{\pgfqpoint{-0.011410in}{0.019646in}}{\pgfqpoint{-0.015528in}{0.015528in}}%
\pgfpathcurveto{\pgfqpoint{-0.019646in}{0.011410in}}{\pgfqpoint{-0.021960in}{0.005824in}}{\pgfqpoint{-0.021960in}{0.000000in}}%
\pgfpathcurveto{\pgfqpoint{-0.021960in}{-0.005824in}}{\pgfqpoint{-0.019646in}{-0.011410in}}{\pgfqpoint{-0.015528in}{-0.015528in}}%
\pgfpathcurveto{\pgfqpoint{-0.011410in}{-0.019646in}}{\pgfqpoint{-0.005824in}{-0.021960in}}{\pgfqpoint{0.000000in}{-0.021960in}}%
\pgfpathlineto{\pgfqpoint{0.000000in}{-0.021960in}}%
\pgfpathclose%
\pgfusepath{stroke,fill}%
}%
\begin{pgfscope}%
\pgfsys@transformshift{4.419509in}{4.485452in}%
\pgfsys@useobject{currentmarker}{}%
\end{pgfscope}%
\end{pgfscope}%
\begin{pgfscope}%
\definecolor{textcolor}{rgb}{0.000000,0.000000,0.000000}%
\pgfsetstrokecolor{textcolor}%
\pgfsetfillcolor{textcolor}%
\pgftext[x=4.694509in,y=4.445347in,left,base]{\color{textcolor}{\ifdefined\pdftexversion\else\setmainfont{Times New Roman}\rmfamily\fi\fontsize{11.000000}{13.200000}\selectfont\catcode`\^=\active\def^{\ifmmode\sp\else\^{}\fi}\catcode`\%=\active\def%{\%}$x$}}%
\end{pgfscope}%
\begin{pgfscope}%
\pgfsetbuttcap%
\pgfsetroundjoin%
\definecolor{currentfill}{rgb}{1.000000,0.647059,0.000000}%
\pgfsetfillcolor{currentfill}%
\pgfsetfillopacity{0.700000}%
\pgfsetlinewidth{1.003750pt}%
\definecolor{currentstroke}{rgb}{1.000000,0.647059,0.000000}%
\pgfsetstrokecolor{currentstroke}%
\pgfsetstrokeopacity{0.700000}%
\pgfsetdash{}{0pt}%
\pgfsys@defobject{currentmarker}{\pgfqpoint{-0.021960in}{-0.021960in}}{\pgfqpoint{0.021960in}{0.021960in}}{%
\pgfpathmoveto{\pgfqpoint{0.000000in}{-0.021960in}}%
\pgfpathcurveto{\pgfqpoint{0.005824in}{-0.021960in}}{\pgfqpoint{0.011410in}{-0.019646in}}{\pgfqpoint{0.015528in}{-0.015528in}}%
\pgfpathcurveto{\pgfqpoint{0.019646in}{-0.011410in}}{\pgfqpoint{0.021960in}{-0.005824in}}{\pgfqpoint{0.021960in}{0.000000in}}%
\pgfpathcurveto{\pgfqpoint{0.021960in}{0.005824in}}{\pgfqpoint{0.019646in}{0.011410in}}{\pgfqpoint{0.015528in}{0.015528in}}%
\pgfpathcurveto{\pgfqpoint{0.011410in}{0.019646in}}{\pgfqpoint{0.005824in}{0.021960in}}{\pgfqpoint{0.000000in}{0.021960in}}%
\pgfpathcurveto{\pgfqpoint{-0.005824in}{0.021960in}}{\pgfqpoint{-0.011410in}{0.019646in}}{\pgfqpoint{-0.015528in}{0.015528in}}%
\pgfpathcurveto{\pgfqpoint{-0.019646in}{0.011410in}}{\pgfqpoint{-0.021960in}{0.005824in}}{\pgfqpoint{-0.021960in}{0.000000in}}%
\pgfpathcurveto{\pgfqpoint{-0.021960in}{-0.005824in}}{\pgfqpoint{-0.019646in}{-0.011410in}}{\pgfqpoint{-0.015528in}{-0.015528in}}%
\pgfpathcurveto{\pgfqpoint{-0.011410in}{-0.019646in}}{\pgfqpoint{-0.005824in}{-0.021960in}}{\pgfqpoint{0.000000in}{-0.021960in}}%
\pgfpathlineto{\pgfqpoint{0.000000in}{-0.021960in}}%
\pgfpathclose%
\pgfusepath{stroke,fill}%
}%
\begin{pgfscope}%
\pgfsys@transformshift{4.419509in}{4.269444in}%
\pgfsys@useobject{currentmarker}{}%
\end{pgfscope}%
\end{pgfscope}%
\begin{pgfscope}%
\definecolor{textcolor}{rgb}{0.000000,0.000000,0.000000}%
\pgfsetstrokecolor{textcolor}%
\pgfsetfillcolor{textcolor}%
\pgftext[x=4.694509in,y=4.229340in,left,base]{\color{textcolor}{\ifdefined\pdftexversion\else\setmainfont{Times New Roman}\rmfamily\fi\fontsize{11.000000}{13.200000}\selectfont\catcode`\^=\active\def^{\ifmmode\sp\else\^{}\fi}\catcode`\%=\active\def%{\%}$x_+$}}%
\end{pgfscope}%
\begin{pgfscope}%
\pgfsetrectcap%
\pgfsetroundjoin%
\pgfsetlinewidth{2.007500pt}%
\definecolor{currentstroke}{rgb}{0.000000,0.000000,1.000000}%
\pgfsetstrokecolor{currentstroke}%
\pgfsetdash{}{0pt}%
\pgfpathmoveto{\pgfqpoint{4.266732in}{4.065092in}}%
\pgfpathlineto{\pgfqpoint{4.419509in}{4.065092in}}%
\pgfpathlineto{\pgfqpoint{4.572287in}{4.065092in}}%
\pgfusepath{stroke}%
\end{pgfscope}%
\begin{pgfscope}%
\definecolor{textcolor}{rgb}{0.000000,0.000000,0.000000}%
\pgfsetstrokecolor{textcolor}%
\pgfsetfillcolor{textcolor}%
\pgftext[x=4.694509in,y=4.011620in,left,base]{\color{textcolor}{\ifdefined\pdftexversion\else\setmainfont{Times New Roman}\rmfamily\fi\fontsize{11.000000}{13.200000}\selectfont\catcode`\^=\active\def^{\ifmmode\sp\else\^{}\fi}\catcode`\%=\active\def%{\%}$\X_0$}}%
\end{pgfscope}%
\begin{pgfscope}%
\pgfsetrectcap%
\pgfsetroundjoin%
\pgfsetlinewidth{2.007500pt}%
\definecolor{currentstroke}{rgb}{1.000000,0.000000,0.000000}%
\pgfsetstrokecolor{currentstroke}%
\pgfsetdash{}{0pt}%
\pgfpathmoveto{\pgfqpoint{4.266732in}{3.849085in}}%
\pgfpathlineto{\pgfqpoint{4.419509in}{3.849085in}}%
\pgfpathlineto{\pgfqpoint{4.572287in}{3.849085in}}%
\pgfusepath{stroke}%
\end{pgfscope}%
\begin{pgfscope}%
\definecolor{textcolor}{rgb}{0.000000,0.000000,0.000000}%
\pgfsetstrokecolor{textcolor}%
\pgfsetfillcolor{textcolor}%
\pgftext[x=4.694509in,y=3.795613in,left,base]{\color{textcolor}{\ifdefined\pdftexversion\else\setmainfont{Times New Roman}\rmfamily\fi\fontsize{11.000000}{13.200000}\selectfont\catcode`\^=\active\def^{\ifmmode\sp\else\^{}\fi}\catcode`\%=\active\def%{\%}$\X_U$}}%
\end{pgfscope}%
\end{pgfpicture}%
\makeatother%
\endgroup%

      \caption{Visualization of $N=200$ sample transitions and safety specification for the \overtaking benchmark.}
      \label{fig:model-overtaking}
\end{figure}

\begin{table}[tb]
      \centering
      \begin{tabular}{ccccccc}
            \toprule
            \textbf{Freq.} & \textbf{Lattice} & $\eta$ & $\gamma$ & $c$  & \textbf{Runtime} & \textbf{Safety} \\
                           & \textbf{Size}    &        &          &      & [mm:ss]          & \textbf{Prob.}  \\ % Units
            \midrule
            5              & $99^3$           & 0.09   & 0.5      & 0.05 & 46:33            & 30.31\%         \\
            5              & $100^3$          & 0.10   & 0.5      & 0.05 & 49:21            & 29.39\%         \\
            5              & $88^3$           & 0.09   & 0.5      & 0.05 & 25:52            & 28.73\%         \\
            5              & $77^3$           & 0.10   & 0.5      & 0.05 & 13:49            & 24.94\%         \\
            5              & $66^3$           & 0.12   & 0.5      & 0.05 & 8:16             & 16.61\%         \\
            4              & $90^3$           & 0.10   & 0.5      & 0.06 & 11:21            & 11.79\%         \\
            4              & $81^3$           & 0.11   & 0.5      & 0.06 & 8:08             & 9.52\%          \\
            5              & $55^3$           & 0.13   & 0.5      & 0.06 & 4:15             & 9.21\%          \\
            4              & $72^3$           & 0.10   & 0.5      & 0.06 & 5:10             & 8.46\%          \\
            5              & $44^3$           & 0.16   & 0.5      & 0.06 & 2:06             & 4.84\%          \\
            4              & $63^3$           & 0.12   & 0.5      & 0.07 & 3:25             & 4.19\%          \\
            \bottomrule
      \end{tabular}
      \caption{\overtaking results, sorted by the last column. For each combination of number of frequencies $M$ and lattice size (i.e., number of lattice points per dimension), we report the values of $c$, $\gamma$, $\lambda$, the runtime, and the achieved lower bound on the safety probability.}
      \label{tab:results-overtaking}
\end{table}

\begin{figure}[ht]
      \centering
      %% Creator: Matplotlib, PGF backend
%%
%% To include the figure in your LaTeX document, write
%%   \input{<filename>.pgf}
%%
%% Make sure the required packages are loaded in your preamble
%%   \usepackage{pgf}
%%
%% Also ensure that all the required font packages are loaded; for instance,
%% the lmodern package is sometimes necessary when using math font.
%%   \usepackage{lmodern}
%%
%% Figures using additional raster images can only be included by \input if
%% they are in the same directory as the main LaTeX file. For loading figures
%% from other directories you can use the `import` package
%%   \usepackage{import}
%%
%% and then include the figures with
%%   \import{<path to file>}{<filename>.pgf}
%%
%% Matplotlib used the following preamble
%%   \def\mathdefault#1{#1}
%%   \everymath=\expandafter{\the\everymath\displaystyle}
%%   \IfFileExists{scrextend.sty}{
%%     \usepackage[fontsize=10.000000pt]{scrextend}
%%   }{
%%     \renewcommand{\normalsize}{\fontsize{10.000000}{12.000000}\selectfont}
%%     \normalsize
%%   }
%%   
%%   \ifdefined\pdftexversion\else  % non-pdftex case.
%%     \usepackage{fontspec}
%%     \setmainfont{DejaVuSerif.ttf}[Path=\detokenize{/home/campus.ncl.ac.uk/c3054737/miniconda3/envs/pylucid/lib/python3.11/site-packages/matplotlib/mpl-data/fonts/ttf/}]
%%     \setsansfont{DejaVuSans.ttf}[Path=\detokenize{/home/campus.ncl.ac.uk/c3054737/miniconda3/envs/pylucid/lib/python3.11/site-packages/matplotlib/mpl-data/fonts/ttf/}]
%%     \setmonofont{DejaVuSansMono.ttf}[Path=\detokenize{/home/campus.ncl.ac.uk/c3054737/miniconda3/envs/pylucid/lib/python3.11/site-packages/matplotlib/mpl-data/fonts/ttf/}]
%%   \fi
%%   \makeatletter\@ifpackageloaded{underscore}{}{\usepackage[strings]{underscore}}\makeatother
%%
\begingroup%
\makeatletter%
\begin{pgfpicture}%
\pgfpathrectangle{\pgfpointorigin}{\pgfqpoint{4.214534in}{4.208819in}}%
\pgfusepath{use as bounding box, clip}%
\begin{pgfscope}%
\pgfsetbuttcap%
\pgfsetmiterjoin%
\definecolor{currentfill}{rgb}{1.000000,1.000000,1.000000}%
\pgfsetfillcolor{currentfill}%
\pgfsetlinewidth{0.000000pt}%
\definecolor{currentstroke}{rgb}{1.000000,1.000000,1.000000}%
\pgfsetstrokecolor{currentstroke}%
\pgfsetdash{}{0pt}%
\pgfpathmoveto{\pgfqpoint{0.000000in}{0.000000in}}%
\pgfpathlineto{\pgfqpoint{4.214534in}{0.000000in}}%
\pgfpathlineto{\pgfqpoint{4.214534in}{4.208819in}}%
\pgfpathlineto{\pgfqpoint{0.000000in}{4.208819in}}%
\pgfpathlineto{\pgfqpoint{0.000000in}{0.000000in}}%
\pgfpathclose%
\pgfusepath{fill}%
\end{pgfscope}%
\begin{pgfscope}%
\pgfsetbuttcap%
\pgfsetmiterjoin%
\definecolor{currentfill}{rgb}{1.000000,1.000000,1.000000}%
\pgfsetfillcolor{currentfill}%
\pgfsetlinewidth{0.000000pt}%
\definecolor{currentstroke}{rgb}{0.000000,0.000000,0.000000}%
\pgfsetstrokecolor{currentstroke}%
\pgfsetstrokeopacity{0.000000}%
\pgfsetdash{}{0pt}%
\pgfpathmoveto{\pgfqpoint{0.599864in}{0.517670in}}%
\pgfpathlineto{\pgfqpoint{3.420801in}{0.517670in}}%
\pgfpathlineto{\pgfqpoint{3.420801in}{4.060601in}}%
\pgfpathlineto{\pgfqpoint{0.599864in}{4.060601in}}%
\pgfpathlineto{\pgfqpoint{0.599864in}{0.517670in}}%
\pgfpathclose%
\pgfusepath{fill}%
\end{pgfscope}%
\begin{pgfscope}%
\pgfpathrectangle{\pgfqpoint{0.599864in}{0.517670in}}{\pgfqpoint{2.820937in}{3.542931in}}%
\pgfusepath{clip}%
\pgfsetbuttcap%
\pgfsetroundjoin%
\definecolor{currentfill}{rgb}{0.611534,0.066436,0.154787}%
\pgfsetfillcolor{currentfill}%
\pgfsetlinewidth{0.000000pt}%
\definecolor{currentstroke}{rgb}{0.000000,0.000000,0.000000}%
\pgfsetstrokecolor{currentstroke}%
\pgfsetdash{}{0pt}%
\pgfpathmoveto{\pgfqpoint{3.420801in}{0.944545in}}%
\pgfpathlineto{\pgfqpoint{2.739538in}{1.211871in}}%
\pgfpathlineto{\pgfqpoint{1.463393in}{1.586902in}}%
\pgfpathlineto{\pgfqpoint{0.599864in}{1.967988in}}%
\pgfpathlineto{\pgfqpoint{0.599864in}{1.537605in}}%
\pgfpathlineto{\pgfqpoint{3.420801in}{0.517670in}}%
\pgfpathlineto{\pgfqpoint{3.420801in}{0.944545in}}%
\pgfpathclose%
\pgfusepath{fill}%
\end{pgfscope}%
\begin{pgfscope}%
\pgfpathrectangle{\pgfqpoint{0.599864in}{0.517670in}}{\pgfqpoint{2.820937in}{3.542931in}}%
\pgfusepath{clip}%
\pgfsetbuttcap%
\pgfsetroundjoin%
\definecolor{currentfill}{rgb}{0.853057,0.408304,0.326413}%
\pgfsetfillcolor{currentfill}%
\pgfsetlinewidth{0.000000pt}%
\definecolor{currentstroke}{rgb}{0.000000,0.000000,0.000000}%
\pgfsetstrokecolor{currentstroke}%
\pgfsetdash{}{0pt}%
\pgfpathmoveto{\pgfqpoint{0.599864in}{1.967988in}}%
\pgfpathlineto{\pgfqpoint{1.463393in}{1.586902in}}%
\pgfpathlineto{\pgfqpoint{2.739538in}{1.211871in}}%
\pgfpathlineto{\pgfqpoint{3.420801in}{0.944545in}}%
\pgfpathlineto{\pgfqpoint{3.420801in}{1.108158in}}%
\pgfpathlineto{\pgfqpoint{3.420801in}{1.330591in}}%
\pgfpathlineto{\pgfqpoint{2.372108in}{1.638779in}}%
\pgfpathlineto{\pgfqpoint{1.823090in}{1.881068in}}%
\pgfpathlineto{\pgfqpoint{1.204711in}{2.078281in}}%
\pgfpathlineto{\pgfqpoint{1.052191in}{2.469428in}}%
\pgfpathlineto{\pgfqpoint{0.962978in}{2.552227in}}%
\pgfpathlineto{\pgfqpoint{0.634777in}{2.985657in}}%
\pgfpathlineto{\pgfqpoint{0.631797in}{2.992454in}}%
\pgfpathlineto{\pgfqpoint{0.599864in}{3.029091in}}%
\pgfpathlineto{\pgfqpoint{0.599864in}{2.986985in}}%
\pgfpathlineto{\pgfqpoint{0.599864in}{2.503859in}}%
\pgfpathlineto{\pgfqpoint{0.599864in}{2.020732in}}%
\pgfpathlineto{\pgfqpoint{0.599864in}{1.967988in}}%
\pgfpathclose%
\pgfusepath{fill}%
\end{pgfscope}%
\begin{pgfscope}%
\pgfpathrectangle{\pgfqpoint{0.599864in}{0.517670in}}{\pgfqpoint{2.820937in}{3.542931in}}%
\pgfusepath{clip}%
\pgfsetbuttcap%
\pgfsetroundjoin%
\definecolor{currentfill}{rgb}{0.976932,0.767474,0.663668}%
\pgfsetfillcolor{currentfill}%
\pgfsetlinewidth{0.000000pt}%
\definecolor{currentstroke}{rgb}{0.000000,0.000000,0.000000}%
\pgfsetstrokecolor{currentstroke}%
\pgfsetdash{}{0pt}%
\pgfpathmoveto{\pgfqpoint{0.599864in}{3.029091in}}%
\pgfpathlineto{\pgfqpoint{0.631797in}{2.992454in}}%
\pgfpathlineto{\pgfqpoint{0.634777in}{2.985657in}}%
\pgfpathlineto{\pgfqpoint{0.962978in}{2.552227in}}%
\pgfpathlineto{\pgfqpoint{1.052191in}{2.469428in}}%
\pgfpathlineto{\pgfqpoint{1.204711in}{2.078281in}}%
\pgfpathlineto{\pgfqpoint{1.823090in}{1.881068in}}%
\pgfpathlineto{\pgfqpoint{2.372108in}{1.638779in}}%
\pgfpathlineto{\pgfqpoint{3.420801in}{1.330591in}}%
\pgfpathlineto{\pgfqpoint{3.420801in}{1.649520in}}%
\pgfpathlineto{\pgfqpoint{3.280824in}{1.690656in}}%
\pgfpathlineto{\pgfqpoint{3.207542in}{1.722996in}}%
\pgfpathlineto{\pgfqpoint{1.889279in}{2.143416in}}%
\pgfpathlineto{\pgfqpoint{1.783874in}{2.413734in}}%
\pgfpathlineto{\pgfqpoint{1.550350in}{2.630469in}}%
\pgfpathlineto{\pgfqpoint{1.300583in}{2.960317in}}%
\pgfpathlineto{\pgfqpoint{1.240766in}{3.096749in}}%
\pgfpathlineto{\pgfqpoint{0.873087in}{3.518594in}}%
\pgfpathlineto{\pgfqpoint{0.889967in}{3.523793in}}%
\pgfpathlineto{\pgfqpoint{0.889790in}{4.005430in}}%
\pgfpathlineto{\pgfqpoint{0.599864in}{4.060601in}}%
\pgfpathlineto{\pgfqpoint{0.599864in}{3.523793in}}%
\pgfpathlineto{\pgfqpoint{0.599864in}{3.029091in}}%
\pgfpathclose%
\pgfusepath{fill}%
\end{pgfscope}%
\begin{pgfscope}%
\pgfpathrectangle{\pgfqpoint{0.599864in}{0.517670in}}{\pgfqpoint{2.820937in}{3.542931in}}%
\pgfusepath{clip}%
\pgfsetbuttcap%
\pgfsetroundjoin%
\definecolor{currentfill}{rgb}{0.969089,0.966474,0.964937}%
\pgfsetfillcolor{currentfill}%
\pgfsetlinewidth{0.000000pt}%
\definecolor{currentstroke}{rgb}{0.000000,0.000000,0.000000}%
\pgfsetstrokecolor{currentstroke}%
\pgfsetdash{}{0pt}%
\pgfpathmoveto{\pgfqpoint{0.889790in}{4.005430in}}%
\pgfpathlineto{\pgfqpoint{0.889967in}{3.523793in}}%
\pgfpathlineto{\pgfqpoint{0.873087in}{3.518594in}}%
\pgfpathlineto{\pgfqpoint{1.240766in}{3.096749in}}%
\pgfpathlineto{\pgfqpoint{1.300583in}{2.960317in}}%
\pgfpathlineto{\pgfqpoint{1.550350in}{2.630469in}}%
\pgfpathlineto{\pgfqpoint{1.783874in}{2.413734in}}%
\pgfpathlineto{\pgfqpoint{1.889279in}{2.143416in}}%
\pgfpathlineto{\pgfqpoint{3.207542in}{1.722996in}}%
\pgfpathlineto{\pgfqpoint{3.280824in}{1.690656in}}%
\pgfpathlineto{\pgfqpoint{3.420801in}{1.649520in}}%
\pgfpathlineto{\pgfqpoint{3.420801in}{1.698647in}}%
\pgfpathlineto{\pgfqpoint{3.420801in}{1.938441in}}%
\pgfpathlineto{\pgfqpoint{2.573848in}{2.208550in}}%
\pgfpathlineto{\pgfqpoint{2.515557in}{2.358040in}}%
\pgfpathlineto{\pgfqpoint{2.137722in}{2.708710in}}%
\pgfpathlineto{\pgfqpoint{1.966388in}{2.934977in}}%
\pgfpathlineto{\pgfqpoint{1.849735in}{3.201044in}}%
\pgfpathlineto{\pgfqpoint{1.584763in}{3.505051in}}%
\pgfpathlineto{\pgfqpoint{1.645612in}{3.523793in}}%
\pgfpathlineto{\pgfqpoint{1.645488in}{3.861625in}}%
\pgfpathlineto{\pgfqpoint{0.889790in}{4.005430in}}%
\pgfpathclose%
\pgfusepath{fill}%
\end{pgfscope}%
\begin{pgfscope}%
\pgfpathrectangle{\pgfqpoint{0.599864in}{0.517670in}}{\pgfqpoint{2.820937in}{3.542931in}}%
\pgfusepath{clip}%
\pgfsetbuttcap%
\pgfsetroundjoin%
\definecolor{currentfill}{rgb}{0.713033,0.843906,0.910727}%
\pgfsetfillcolor{currentfill}%
\pgfsetlinewidth{0.000000pt}%
\definecolor{currentstroke}{rgb}{0.000000,0.000000,0.000000}%
\pgfsetstrokecolor{currentstroke}%
\pgfsetdash{}{0pt}%
\pgfpathmoveto{\pgfqpoint{1.645488in}{3.861625in}}%
\pgfpathlineto{\pgfqpoint{1.645612in}{3.523793in}}%
\pgfpathlineto{\pgfqpoint{1.584763in}{3.505051in}}%
\pgfpathlineto{\pgfqpoint{1.849735in}{3.201044in}}%
\pgfpathlineto{\pgfqpoint{1.966388in}{2.934977in}}%
\pgfpathlineto{\pgfqpoint{2.137722in}{2.708710in}}%
\pgfpathlineto{\pgfqpoint{2.515557in}{2.358040in}}%
\pgfpathlineto{\pgfqpoint{2.573848in}{2.208550in}}%
\pgfpathlineto{\pgfqpoint{3.420801in}{1.938441in}}%
\pgfpathlineto{\pgfqpoint{3.420801in}{2.221897in}}%
\pgfpathlineto{\pgfqpoint{3.258417in}{2.273685in}}%
\pgfpathlineto{\pgfqpoint{3.247241in}{2.302346in}}%
\pgfpathlineto{\pgfqpoint{2.725094in}{2.786952in}}%
\pgfpathlineto{\pgfqpoint{2.632194in}{2.909637in}}%
\pgfpathlineto{\pgfqpoint{2.458704in}{3.305339in}}%
\pgfpathlineto{\pgfqpoint{2.296439in}{3.491508in}}%
\pgfpathlineto{\pgfqpoint{2.401258in}{3.523793in}}%
\pgfpathlineto{\pgfqpoint{2.401186in}{3.717820in}}%
\pgfpathlineto{\pgfqpoint{1.645488in}{3.861625in}}%
\pgfpathclose%
\pgfusepath{fill}%
\end{pgfscope}%
\begin{pgfscope}%
\pgfpathrectangle{\pgfqpoint{0.599864in}{0.517670in}}{\pgfqpoint{2.820937in}{3.542931in}}%
\pgfusepath{clip}%
\pgfsetbuttcap%
\pgfsetroundjoin%
\definecolor{currentfill}{rgb}{0.299193,0.599539,0.777163}%
\pgfsetfillcolor{currentfill}%
\pgfsetlinewidth{0.000000pt}%
\definecolor{currentstroke}{rgb}{0.000000,0.000000,0.000000}%
\pgfsetstrokecolor{currentstroke}%
\pgfsetdash{}{0pt}%
\pgfpathmoveto{\pgfqpoint{3.420801in}{2.764647in}}%
\pgfpathlineto{\pgfqpoint{3.312466in}{2.865193in}}%
\pgfpathlineto{\pgfqpoint{3.297999in}{2.884298in}}%
\pgfpathlineto{\pgfqpoint{3.067673in}{3.409634in}}%
\pgfpathlineto{\pgfqpoint{3.008115in}{3.477966in}}%
\pgfpathlineto{\pgfqpoint{3.156903in}{3.523793in}}%
\pgfpathlineto{\pgfqpoint{3.156884in}{3.574015in}}%
\pgfpathlineto{\pgfqpoint{2.401186in}{3.717820in}}%
\pgfpathlineto{\pgfqpoint{2.401258in}{3.523793in}}%
\pgfpathlineto{\pgfqpoint{2.296439in}{3.491508in}}%
\pgfpathlineto{\pgfqpoint{2.458704in}{3.305339in}}%
\pgfpathlineto{\pgfqpoint{2.632194in}{2.909637in}}%
\pgfpathlineto{\pgfqpoint{2.725094in}{2.786952in}}%
\pgfpathlineto{\pgfqpoint{3.247241in}{2.302346in}}%
\pgfpathlineto{\pgfqpoint{3.258417in}{2.273685in}}%
\pgfpathlineto{\pgfqpoint{3.420801in}{2.221897in}}%
\pgfpathlineto{\pgfqpoint{3.420801in}{2.289135in}}%
\pgfpathlineto{\pgfqpoint{3.420801in}{2.764647in}}%
\pgfpathclose%
\pgfusepath{fill}%
\end{pgfscope}%
\begin{pgfscope}%
\pgfpathrectangle{\pgfqpoint{0.599864in}{0.517670in}}{\pgfqpoint{2.820937in}{3.542931in}}%
\pgfusepath{clip}%
\pgfsetbuttcap%
\pgfsetroundjoin%
\definecolor{currentfill}{rgb}{0.097116,0.337716,0.588005}%
\pgfsetfillcolor{currentfill}%
\pgfsetlinewidth{0.000000pt}%
\definecolor{currentstroke}{rgb}{0.000000,0.000000,0.000000}%
\pgfsetstrokecolor{currentstroke}%
\pgfsetdash{}{0pt}%
\pgfpathmoveto{\pgfqpoint{3.156884in}{3.574015in}}%
\pgfpathlineto{\pgfqpoint{3.156903in}{3.523793in}}%
\pgfpathlineto{\pgfqpoint{3.008115in}{3.477966in}}%
\pgfpathlineto{\pgfqpoint{3.067673in}{3.409634in}}%
\pgfpathlineto{\pgfqpoint{3.297999in}{2.884298in}}%
\pgfpathlineto{\pgfqpoint{3.312466in}{2.865193in}}%
\pgfpathlineto{\pgfqpoint{3.420801in}{2.764647in}}%
\pgfpathlineto{\pgfqpoint{3.420801in}{2.879624in}}%
\pgfpathlineto{\pgfqpoint{3.420801in}{3.470112in}}%
\pgfpathlineto{\pgfqpoint{3.420801in}{3.523793in}}%
\pgfpathlineto{\pgfqpoint{3.156884in}{3.574015in}}%
\pgfpathclose%
\pgfusepath{fill}%
\end{pgfscope}%
\begin{pgfscope}%
\pgfsetbuttcap%
\pgfsetroundjoin%
\definecolor{currentfill}{rgb}{0.000000,0.000000,0.000000}%
\pgfsetfillcolor{currentfill}%
\pgfsetlinewidth{0.803000pt}%
\definecolor{currentstroke}{rgb}{0.000000,0.000000,0.000000}%
\pgfsetstrokecolor{currentstroke}%
\pgfsetdash{}{0pt}%
\pgfsys@defobject{currentmarker}{\pgfqpoint{0.000000in}{-0.048611in}}{\pgfqpoint{0.000000in}{0.000000in}}{%
\pgfpathmoveto{\pgfqpoint{0.000000in}{0.000000in}}%
\pgfpathlineto{\pgfqpoint{0.000000in}{-0.048611in}}%
\pgfusepath{stroke,fill}%
}%
\begin{pgfscope}%
\pgfsys@transformshift{0.599864in}{0.517670in}%
\pgfsys@useobject{currentmarker}{}%
\end{pgfscope}%
\end{pgfscope}%
\begin{pgfscope}%
\definecolor{textcolor}{rgb}{0.000000,0.000000,0.000000}%
\pgfsetstrokecolor{textcolor}%
\pgfsetfillcolor{textcolor}%
\pgftext[x=0.599864in,y=0.420448in,,top]{\color{textcolor}{\ifdefined\pdftexversion\else\setmainfont{Times New Roman}\rmfamily\fi\fontsize{10.000000}{12.000000}\selectfont\catcode`\^=\active\def^{\ifmmode\sp\else\^{}\fi}\catcode`\%=\active\def%{\%}4}}%
\end{pgfscope}%
\begin{pgfscope}%
\pgfsetbuttcap%
\pgfsetroundjoin%
\definecolor{currentfill}{rgb}{0.000000,0.000000,0.000000}%
\pgfsetfillcolor{currentfill}%
\pgfsetlinewidth{0.803000pt}%
\definecolor{currentstroke}{rgb}{0.000000,0.000000,0.000000}%
\pgfsetstrokecolor{currentstroke}%
\pgfsetdash{}{0pt}%
\pgfsys@defobject{currentmarker}{\pgfqpoint{0.000000in}{-0.048611in}}{\pgfqpoint{0.000000in}{0.000000in}}{%
\pgfpathmoveto{\pgfqpoint{0.000000in}{0.000000in}}%
\pgfpathlineto{\pgfqpoint{0.000000in}{-0.048611in}}%
\pgfusepath{stroke,fill}%
}%
\begin{pgfscope}%
\pgfsys@transformshift{3.420801in}{0.517670in}%
\pgfsys@useobject{currentmarker}{}%
\end{pgfscope}%
\end{pgfscope}%
\begin{pgfscope}%
\definecolor{textcolor}{rgb}{0.000000,0.000000,0.000000}%
\pgfsetstrokecolor{textcolor}%
\pgfsetfillcolor{textcolor}%
\pgftext[x=3.420801in,y=0.420448in,,top]{\color{textcolor}{\ifdefined\pdftexversion\else\setmainfont{Times New Roman}\rmfamily\fi\fontsize{10.000000}{12.000000}\selectfont\catcode`\^=\active\def^{\ifmmode\sp\else\^{}\fi}\catcode`\%=\active\def%{\%}5}}%
\end{pgfscope}%
\begin{pgfscope}%
\definecolor{textcolor}{rgb}{0.000000,0.000000,0.000000}%
\pgfsetstrokecolor{textcolor}%
\pgfsetfillcolor{textcolor}%
\pgftext[x=2.010333in,y=0.238753in,,top]{\color{textcolor}{\ifdefined\pdftexversion\else\setmainfont{Times New Roman}\rmfamily\fi\fontsize{11.000000}{13.200000}\selectfont\catcode`\^=\active\def^{\ifmmode\sp\else\^{}\fi}\catcode`\%=\active\def%{\%}Number of frequencies}}%
\end{pgfscope}%
\begin{pgfscope}%
\pgfsetbuttcap%
\pgfsetroundjoin%
\definecolor{currentfill}{rgb}{0.000000,0.000000,0.000000}%
\pgfsetfillcolor{currentfill}%
\pgfsetlinewidth{0.803000pt}%
\definecolor{currentstroke}{rgb}{0.000000,0.000000,0.000000}%
\pgfsetstrokecolor{currentstroke}%
\pgfsetdash{}{0pt}%
\pgfsys@defobject{currentmarker}{\pgfqpoint{-0.048611in}{0.000000in}}{\pgfqpoint{-0.000000in}{0.000000in}}{%
\pgfpathmoveto{\pgfqpoint{-0.000000in}{0.000000in}}%
\pgfpathlineto{\pgfqpoint{-0.048611in}{0.000000in}}%
\pgfusepath{stroke,fill}%
}%
\begin{pgfscope}%
\pgfsys@transformshift{0.599864in}{0.839755in}%
\pgfsys@useobject{currentmarker}{}%
\end{pgfscope}%
\end{pgfscope}%
\begin{pgfscope}%
\definecolor{textcolor}{rgb}{0.000000,0.000000,0.000000}%
\pgfsetstrokecolor{textcolor}%
\pgfsetfillcolor{textcolor}%
\pgftext[x=0.363753in, y=0.791537in, left, base]{\color{textcolor}{\ifdefined\pdftexversion\else\setmainfont{Times New Roman}\rmfamily\fi\fontsize{10.000000}{12.000000}\selectfont\catcode`\^=\active\def^{\ifmmode\sp\else\^{}\fi}\catcode`\%=\active\def%{\%}50}}%
\end{pgfscope}%
\begin{pgfscope}%
\pgfsetbuttcap%
\pgfsetroundjoin%
\definecolor{currentfill}{rgb}{0.000000,0.000000,0.000000}%
\pgfsetfillcolor{currentfill}%
\pgfsetlinewidth{0.803000pt}%
\definecolor{currentstroke}{rgb}{0.000000,0.000000,0.000000}%
\pgfsetstrokecolor{currentstroke}%
\pgfsetdash{}{0pt}%
\pgfsys@defobject{currentmarker}{\pgfqpoint{-0.048611in}{0.000000in}}{\pgfqpoint{-0.000000in}{0.000000in}}{%
\pgfpathmoveto{\pgfqpoint{-0.000000in}{0.000000in}}%
\pgfpathlineto{\pgfqpoint{-0.048611in}{0.000000in}}%
\pgfusepath{stroke,fill}%
}%
\begin{pgfscope}%
\pgfsys@transformshift{0.599864in}{1.376562in}%
\pgfsys@useobject{currentmarker}{}%
\end{pgfscope}%
\end{pgfscope}%
\begin{pgfscope}%
\definecolor{textcolor}{rgb}{0.000000,0.000000,0.000000}%
\pgfsetstrokecolor{textcolor}%
\pgfsetfillcolor{textcolor}%
\pgftext[x=0.363753in, y=1.328345in, left, base]{\color{textcolor}{\ifdefined\pdftexversion\else\setmainfont{Times New Roman}\rmfamily\fi\fontsize{10.000000}{12.000000}\selectfont\catcode`\^=\active\def^{\ifmmode\sp\else\^{}\fi}\catcode`\%=\active\def%{\%}60}}%
\end{pgfscope}%
\begin{pgfscope}%
\pgfsetbuttcap%
\pgfsetroundjoin%
\definecolor{currentfill}{rgb}{0.000000,0.000000,0.000000}%
\pgfsetfillcolor{currentfill}%
\pgfsetlinewidth{0.803000pt}%
\definecolor{currentstroke}{rgb}{0.000000,0.000000,0.000000}%
\pgfsetstrokecolor{currentstroke}%
\pgfsetdash{}{0pt}%
\pgfsys@defobject{currentmarker}{\pgfqpoint{-0.048611in}{0.000000in}}{\pgfqpoint{-0.000000in}{0.000000in}}{%
\pgfpathmoveto{\pgfqpoint{-0.000000in}{0.000000in}}%
\pgfpathlineto{\pgfqpoint{-0.048611in}{0.000000in}}%
\pgfusepath{stroke,fill}%
}%
\begin{pgfscope}%
\pgfsys@transformshift{0.599864in}{1.913370in}%
\pgfsys@useobject{currentmarker}{}%
\end{pgfscope}%
\end{pgfscope}%
\begin{pgfscope}%
\definecolor{textcolor}{rgb}{0.000000,0.000000,0.000000}%
\pgfsetstrokecolor{textcolor}%
\pgfsetfillcolor{textcolor}%
\pgftext[x=0.363753in, y=1.865152in, left, base]{\color{textcolor}{\ifdefined\pdftexversion\else\setmainfont{Times New Roman}\rmfamily\fi\fontsize{10.000000}{12.000000}\selectfont\catcode`\^=\active\def^{\ifmmode\sp\else\^{}\fi}\catcode`\%=\active\def%{\%}70}}%
\end{pgfscope}%
\begin{pgfscope}%
\pgfsetbuttcap%
\pgfsetroundjoin%
\definecolor{currentfill}{rgb}{0.000000,0.000000,0.000000}%
\pgfsetfillcolor{currentfill}%
\pgfsetlinewidth{0.803000pt}%
\definecolor{currentstroke}{rgb}{0.000000,0.000000,0.000000}%
\pgfsetstrokecolor{currentstroke}%
\pgfsetdash{}{0pt}%
\pgfsys@defobject{currentmarker}{\pgfqpoint{-0.048611in}{0.000000in}}{\pgfqpoint{-0.000000in}{0.000000in}}{%
\pgfpathmoveto{\pgfqpoint{-0.000000in}{0.000000in}}%
\pgfpathlineto{\pgfqpoint{-0.048611in}{0.000000in}}%
\pgfusepath{stroke,fill}%
}%
\begin{pgfscope}%
\pgfsys@transformshift{0.599864in}{2.450178in}%
\pgfsys@useobject{currentmarker}{}%
\end{pgfscope}%
\end{pgfscope}%
\begin{pgfscope}%
\definecolor{textcolor}{rgb}{0.000000,0.000000,0.000000}%
\pgfsetstrokecolor{textcolor}%
\pgfsetfillcolor{textcolor}%
\pgftext[x=0.363753in, y=2.401960in, left, base]{\color{textcolor}{\ifdefined\pdftexversion\else\setmainfont{Times New Roman}\rmfamily\fi\fontsize{10.000000}{12.000000}\selectfont\catcode`\^=\active\def^{\ifmmode\sp\else\^{}\fi}\catcode`\%=\active\def%{\%}80}}%
\end{pgfscope}%
\begin{pgfscope}%
\pgfsetbuttcap%
\pgfsetroundjoin%
\definecolor{currentfill}{rgb}{0.000000,0.000000,0.000000}%
\pgfsetfillcolor{currentfill}%
\pgfsetlinewidth{0.803000pt}%
\definecolor{currentstroke}{rgb}{0.000000,0.000000,0.000000}%
\pgfsetstrokecolor{currentstroke}%
\pgfsetdash{}{0pt}%
\pgfsys@defobject{currentmarker}{\pgfqpoint{-0.048611in}{0.000000in}}{\pgfqpoint{-0.000000in}{0.000000in}}{%
\pgfpathmoveto{\pgfqpoint{-0.000000in}{0.000000in}}%
\pgfpathlineto{\pgfqpoint{-0.048611in}{0.000000in}}%
\pgfusepath{stroke,fill}%
}%
\begin{pgfscope}%
\pgfsys@transformshift{0.599864in}{2.986985in}%
\pgfsys@useobject{currentmarker}{}%
\end{pgfscope}%
\end{pgfscope}%
\begin{pgfscope}%
\definecolor{textcolor}{rgb}{0.000000,0.000000,0.000000}%
\pgfsetstrokecolor{textcolor}%
\pgfsetfillcolor{textcolor}%
\pgftext[x=0.363753in, y=2.938768in, left, base]{\color{textcolor}{\ifdefined\pdftexversion\else\setmainfont{Times New Roman}\rmfamily\fi\fontsize{10.000000}{12.000000}\selectfont\catcode`\^=\active\def^{\ifmmode\sp\else\^{}\fi}\catcode`\%=\active\def%{\%}90}}%
\end{pgfscope}%
\begin{pgfscope}%
\pgfsetbuttcap%
\pgfsetroundjoin%
\definecolor{currentfill}{rgb}{0.000000,0.000000,0.000000}%
\pgfsetfillcolor{currentfill}%
\pgfsetlinewidth{0.803000pt}%
\definecolor{currentstroke}{rgb}{0.000000,0.000000,0.000000}%
\pgfsetstrokecolor{currentstroke}%
\pgfsetdash{}{0pt}%
\pgfsys@defobject{currentmarker}{\pgfqpoint{-0.048611in}{0.000000in}}{\pgfqpoint{-0.000000in}{0.000000in}}{%
\pgfpathmoveto{\pgfqpoint{-0.000000in}{0.000000in}}%
\pgfpathlineto{\pgfqpoint{-0.048611in}{0.000000in}}%
\pgfusepath{stroke,fill}%
}%
\begin{pgfscope}%
\pgfsys@transformshift{0.599864in}{3.523793in}%
\pgfsys@useobject{currentmarker}{}%
\end{pgfscope}%
\end{pgfscope}%
\begin{pgfscope}%
\definecolor{textcolor}{rgb}{0.000000,0.000000,0.000000}%
\pgfsetstrokecolor{textcolor}%
\pgfsetfillcolor{textcolor}%
\pgftext[x=0.294309in, y=3.475575in, left, base]{\color{textcolor}{\ifdefined\pdftexversion\else\setmainfont{Times New Roman}\rmfamily\fi\fontsize{10.000000}{12.000000}\selectfont\catcode`\^=\active\def^{\ifmmode\sp\else\^{}\fi}\catcode`\%=\active\def%{\%}100}}%
\end{pgfscope}%
\begin{pgfscope}%
\pgfsetbuttcap%
\pgfsetroundjoin%
\definecolor{currentfill}{rgb}{0.000000,0.000000,0.000000}%
\pgfsetfillcolor{currentfill}%
\pgfsetlinewidth{0.803000pt}%
\definecolor{currentstroke}{rgb}{0.000000,0.000000,0.000000}%
\pgfsetstrokecolor{currentstroke}%
\pgfsetdash{}{0pt}%
\pgfsys@defobject{currentmarker}{\pgfqpoint{-0.048611in}{0.000000in}}{\pgfqpoint{-0.000000in}{0.000000in}}{%
\pgfpathmoveto{\pgfqpoint{-0.000000in}{0.000000in}}%
\pgfpathlineto{\pgfqpoint{-0.048611in}{0.000000in}}%
\pgfusepath{stroke,fill}%
}%
\begin{pgfscope}%
\pgfsys@transformshift{0.599864in}{4.060601in}%
\pgfsys@useobject{currentmarker}{}%
\end{pgfscope}%
\end{pgfscope}%
\begin{pgfscope}%
\definecolor{textcolor}{rgb}{0.000000,0.000000,0.000000}%
\pgfsetstrokecolor{textcolor}%
\pgfsetfillcolor{textcolor}%
\pgftext[x=0.299463in, y=4.012383in, left, base]{\color{textcolor}{\ifdefined\pdftexversion\else\setmainfont{Times New Roman}\rmfamily\fi\fontsize{10.000000}{12.000000}\selectfont\catcode`\^=\active\def^{\ifmmode\sp\else\^{}\fi}\catcode`\%=\active\def%{\%}110}}%
\end{pgfscope}%
\begin{pgfscope}%
\definecolor{textcolor}{rgb}{0.000000,0.000000,0.000000}%
\pgfsetstrokecolor{textcolor}%
\pgfsetfillcolor{textcolor}%
\pgftext[x=0.238753in,y=2.289135in,,bottom,rotate=90.000000]{\color{textcolor}{\ifdefined\pdftexversion\else\setmainfont{Times New Roman}\rmfamily\fi\fontsize{11.000000}{13.200000}\selectfont\catcode`\^=\active\def^{\ifmmode\sp\else\^{}\fi}\catcode`\%=\active\def%{\%}Lattice size per dimension}}%
\end{pgfscope}%
\begin{pgfscope}%
\pgfpathrectangle{\pgfqpoint{0.599864in}{0.517670in}}{\pgfqpoint{2.820937in}{3.542931in}}%
\pgfusepath{clip}%
\pgfsetbuttcap%
\pgfsetroundjoin%
\pgfsetlinewidth{0.501875pt}%
\definecolor{currentstroke}{rgb}{0.000000,0.000000,0.000000}%
\pgfsetstrokecolor{currentstroke}%
\pgfsetdash{}{0pt}%
\pgfusepath{stroke}%
\end{pgfscope}%
\begin{pgfscope}%
\pgfpathrectangle{\pgfqpoint{0.599864in}{0.517670in}}{\pgfqpoint{2.820937in}{3.542931in}}%
\pgfusepath{clip}%
\pgfsetbuttcap%
\pgfsetroundjoin%
\pgfsetlinewidth{0.501875pt}%
\definecolor{currentstroke}{rgb}{0.000000,0.000000,0.000000}%
\pgfsetstrokecolor{currentstroke}%
\pgfsetdash{}{0pt}%
\pgfpathmoveto{\pgfqpoint{0.599864in}{1.967988in}}%
\pgfpathlineto{\pgfqpoint{1.463393in}{1.586902in}}%
\pgfpathlineto{\pgfqpoint{2.739538in}{1.211871in}}%
\pgfpathlineto{\pgfqpoint{3.420801in}{0.944545in}}%
\pgfusepath{stroke}%
\end{pgfscope}%
\begin{pgfscope}%
\pgfpathrectangle{\pgfqpoint{0.599864in}{0.517670in}}{\pgfqpoint{2.820937in}{3.542931in}}%
\pgfusepath{clip}%
\pgfsetbuttcap%
\pgfsetroundjoin%
\pgfsetlinewidth{0.501875pt}%
\definecolor{currentstroke}{rgb}{0.000000,0.000000,0.000000}%
\pgfsetstrokecolor{currentstroke}%
\pgfsetdash{}{0pt}%
\pgfpathmoveto{\pgfqpoint{0.599864in}{3.029091in}}%
\pgfpathlineto{\pgfqpoint{0.631797in}{2.992454in}}%
\pgfpathlineto{\pgfqpoint{0.634777in}{2.985657in}}%
\pgfpathlineto{\pgfqpoint{0.962978in}{2.552227in}}%
\pgfpathlineto{\pgfqpoint{1.052191in}{2.469428in}}%
\pgfpathlineto{\pgfqpoint{1.204711in}{2.078281in}}%
\pgfpathlineto{\pgfqpoint{1.823090in}{1.881068in}}%
\pgfpathlineto{\pgfqpoint{2.372108in}{1.638779in}}%
\pgfpathlineto{\pgfqpoint{3.420801in}{1.330591in}}%
\pgfusepath{stroke}%
\end{pgfscope}%
\begin{pgfscope}%
\pgfpathrectangle{\pgfqpoint{0.599864in}{0.517670in}}{\pgfqpoint{2.820937in}{3.542931in}}%
\pgfusepath{clip}%
\pgfsetbuttcap%
\pgfsetroundjoin%
\pgfsetlinewidth{0.501875pt}%
\definecolor{currentstroke}{rgb}{0.000000,0.000000,0.000000}%
\pgfsetstrokecolor{currentstroke}%
\pgfsetdash{}{0pt}%
\pgfpathmoveto{\pgfqpoint{0.889790in}{4.005430in}}%
\pgfpathlineto{\pgfqpoint{0.889967in}{3.523793in}}%
\pgfpathlineto{\pgfqpoint{0.873087in}{3.518594in}}%
\pgfpathlineto{\pgfqpoint{1.240766in}{3.096749in}}%
\pgfpathlineto{\pgfqpoint{1.300583in}{2.960317in}}%
\pgfpathlineto{\pgfqpoint{1.550350in}{2.630469in}}%
\pgfpathlineto{\pgfqpoint{1.783874in}{2.413734in}}%
\pgfpathlineto{\pgfqpoint{1.889279in}{2.143416in}}%
\pgfpathlineto{\pgfqpoint{3.207542in}{1.722996in}}%
\pgfpathlineto{\pgfqpoint{3.280824in}{1.690656in}}%
\pgfpathlineto{\pgfqpoint{3.420801in}{1.649520in}}%
\pgfusepath{stroke}%
\end{pgfscope}%
\begin{pgfscope}%
\pgfpathrectangle{\pgfqpoint{0.599864in}{0.517670in}}{\pgfqpoint{2.820937in}{3.542931in}}%
\pgfusepath{clip}%
\pgfsetbuttcap%
\pgfsetroundjoin%
\pgfsetlinewidth{0.501875pt}%
\definecolor{currentstroke}{rgb}{0.000000,0.000000,0.000000}%
\pgfsetstrokecolor{currentstroke}%
\pgfsetdash{}{0pt}%
\pgfpathmoveto{\pgfqpoint{1.645488in}{3.861625in}}%
\pgfpathlineto{\pgfqpoint{1.645612in}{3.523793in}}%
\pgfpathlineto{\pgfqpoint{1.584763in}{3.505051in}}%
\pgfpathlineto{\pgfqpoint{1.849735in}{3.201044in}}%
\pgfpathlineto{\pgfqpoint{1.966388in}{2.934977in}}%
\pgfpathlineto{\pgfqpoint{2.137722in}{2.708710in}}%
\pgfpathlineto{\pgfqpoint{2.515557in}{2.358040in}}%
\pgfpathlineto{\pgfqpoint{2.573848in}{2.208550in}}%
\pgfpathlineto{\pgfqpoint{3.420801in}{1.938441in}}%
\pgfusepath{stroke}%
\end{pgfscope}%
\begin{pgfscope}%
\pgfpathrectangle{\pgfqpoint{0.599864in}{0.517670in}}{\pgfqpoint{2.820937in}{3.542931in}}%
\pgfusepath{clip}%
\pgfsetbuttcap%
\pgfsetroundjoin%
\pgfsetlinewidth{0.501875pt}%
\definecolor{currentstroke}{rgb}{0.000000,0.000000,0.000000}%
\pgfsetstrokecolor{currentstroke}%
\pgfsetdash{}{0pt}%
\pgfpathmoveto{\pgfqpoint{2.401186in}{3.717820in}}%
\pgfpathlineto{\pgfqpoint{2.401258in}{3.523793in}}%
\pgfpathlineto{\pgfqpoint{2.296439in}{3.491508in}}%
\pgfpathlineto{\pgfqpoint{2.458704in}{3.305339in}}%
\pgfpathlineto{\pgfqpoint{2.632194in}{2.909637in}}%
\pgfpathlineto{\pgfqpoint{2.725094in}{2.786952in}}%
\pgfpathlineto{\pgfqpoint{3.247241in}{2.302346in}}%
\pgfpathlineto{\pgfqpoint{3.258417in}{2.273685in}}%
\pgfpathlineto{\pgfqpoint{3.420801in}{2.221897in}}%
\pgfusepath{stroke}%
\end{pgfscope}%
\begin{pgfscope}%
\pgfpathrectangle{\pgfqpoint{0.599864in}{0.517670in}}{\pgfqpoint{2.820937in}{3.542931in}}%
\pgfusepath{clip}%
\pgfsetbuttcap%
\pgfsetroundjoin%
\pgfsetlinewidth{0.501875pt}%
\definecolor{currentstroke}{rgb}{0.000000,0.000000,0.000000}%
\pgfsetstrokecolor{currentstroke}%
\pgfsetdash{}{0pt}%
\pgfpathmoveto{\pgfqpoint{3.156884in}{3.574015in}}%
\pgfpathlineto{\pgfqpoint{3.156903in}{3.523793in}}%
\pgfpathlineto{\pgfqpoint{3.008115in}{3.477966in}}%
\pgfpathlineto{\pgfqpoint{3.067673in}{3.409634in}}%
\pgfpathlineto{\pgfqpoint{3.297999in}{2.884298in}}%
\pgfpathlineto{\pgfqpoint{3.312466in}{2.865193in}}%
\pgfpathlineto{\pgfqpoint{3.420801in}{2.764647in}}%
\pgfusepath{stroke}%
\end{pgfscope}%
\begin{pgfscope}%
\pgfpathrectangle{\pgfqpoint{0.599864in}{0.517670in}}{\pgfqpoint{2.820937in}{3.542931in}}%
\pgfusepath{clip}%
\pgfsetbuttcap%
\pgfsetroundjoin%
\pgfsetlinewidth{0.501875pt}%
\definecolor{currentstroke}{rgb}{0.000000,0.000000,0.000000}%
\pgfsetstrokecolor{currentstroke}%
\pgfsetdash{}{0pt}%
\pgfusepath{stroke}%
\end{pgfscope}%
\begin{pgfscope}%
\pgfpathrectangle{\pgfqpoint{0.599864in}{0.517670in}}{\pgfqpoint{2.820937in}{3.542931in}}%
\pgfusepath{clip}%
\pgfsetbuttcap%
\pgfsetroundjoin%
\definecolor{currentfill}{rgb}{0.000000,0.000000,0.000000}%
\pgfsetfillcolor{currentfill}%
\pgfsetlinewidth{1.003750pt}%
\definecolor{currentstroke}{rgb}{0.000000,0.000000,0.000000}%
\pgfsetstrokecolor{currentstroke}%
\pgfsetdash{}{0pt}%
\pgfsys@defobject{currentmarker}{\pgfqpoint{-0.020833in}{-0.020833in}}{\pgfqpoint{0.020833in}{0.020833in}}{%
\pgfpathmoveto{\pgfqpoint{0.000000in}{-0.020833in}}%
\pgfpathcurveto{\pgfqpoint{0.005525in}{-0.020833in}}{\pgfqpoint{0.010825in}{-0.018638in}}{\pgfqpoint{0.014731in}{-0.014731in}}%
\pgfpathcurveto{\pgfqpoint{0.018638in}{-0.010825in}}{\pgfqpoint{0.020833in}{-0.005525in}}{\pgfqpoint{0.020833in}{0.000000in}}%
\pgfpathcurveto{\pgfqpoint{0.020833in}{0.005525in}}{\pgfqpoint{0.018638in}{0.010825in}}{\pgfqpoint{0.014731in}{0.014731in}}%
\pgfpathcurveto{\pgfqpoint{0.010825in}{0.018638in}}{\pgfqpoint{0.005525in}{0.020833in}}{\pgfqpoint{0.000000in}{0.020833in}}%
\pgfpathcurveto{\pgfqpoint{-0.005525in}{0.020833in}}{\pgfqpoint{-0.010825in}{0.018638in}}{\pgfqpoint{-0.014731in}{0.014731in}}%
\pgfpathcurveto{\pgfqpoint{-0.018638in}{0.010825in}}{\pgfqpoint{-0.020833in}{0.005525in}}{\pgfqpoint{-0.020833in}{0.000000in}}%
\pgfpathcurveto{\pgfqpoint{-0.020833in}{-0.005525in}}{\pgfqpoint{-0.018638in}{-0.010825in}}{\pgfqpoint{-0.014731in}{-0.014731in}}%
\pgfpathcurveto{\pgfqpoint{-0.010825in}{-0.018638in}}{\pgfqpoint{-0.005525in}{-0.020833in}}{\pgfqpoint{0.000000in}{-0.020833in}}%
\pgfpathlineto{\pgfqpoint{0.000000in}{-0.020833in}}%
\pgfpathclose%
\pgfusepath{stroke,fill}%
}%
\begin{pgfscope}%
\pgfsys@transformshift{3.420801in}{3.470112in}%
\pgfsys@useobject{currentmarker}{}%
\end{pgfscope}%
\begin{pgfscope}%
\pgfsys@transformshift{3.420801in}{3.523793in}%
\pgfsys@useobject{currentmarker}{}%
\end{pgfscope}%
\begin{pgfscope}%
\pgfsys@transformshift{3.420801in}{2.879624in}%
\pgfsys@useobject{currentmarker}{}%
\end{pgfscope}%
\begin{pgfscope}%
\pgfsys@transformshift{3.420801in}{2.289135in}%
\pgfsys@useobject{currentmarker}{}%
\end{pgfscope}%
\begin{pgfscope}%
\pgfsys@transformshift{3.420801in}{1.698647in}%
\pgfsys@useobject{currentmarker}{}%
\end{pgfscope}%
\begin{pgfscope}%
\pgfsys@transformshift{0.599864in}{4.060601in}%
\pgfsys@useobject{currentmarker}{}%
\end{pgfscope}%
\begin{pgfscope}%
\pgfsys@transformshift{0.599864in}{3.523793in}%
\pgfsys@useobject{currentmarker}{}%
\end{pgfscope}%
\begin{pgfscope}%
\pgfsys@transformshift{0.599864in}{2.986985in}%
\pgfsys@useobject{currentmarker}{}%
\end{pgfscope}%
\begin{pgfscope}%
\pgfsys@transformshift{0.599864in}{2.503859in}%
\pgfsys@useobject{currentmarker}{}%
\end{pgfscope}%
\begin{pgfscope}%
\pgfsys@transformshift{3.420801in}{1.108158in}%
\pgfsys@useobject{currentmarker}{}%
\end{pgfscope}%
\begin{pgfscope}%
\pgfsys@transformshift{0.599864in}{2.020732in}%
\pgfsys@useobject{currentmarker}{}%
\end{pgfscope}%
\begin{pgfscope}%
\pgfsys@transformshift{3.420801in}{0.517670in}%
\pgfsys@useobject{currentmarker}{}%
\end{pgfscope}%
\begin{pgfscope}%
\pgfsys@transformshift{0.599864in}{1.537605in}%
\pgfsys@useobject{currentmarker}{}%
\end{pgfscope}%
\end{pgfscope}%
\begin{pgfscope}%
\pgfsetrectcap%
\pgfsetmiterjoin%
\pgfsetlinewidth{0.803000pt}%
\definecolor{currentstroke}{rgb}{0.000000,0.000000,0.000000}%
\pgfsetstrokecolor{currentstroke}%
\pgfsetdash{}{0pt}%
\pgfpathmoveto{\pgfqpoint{0.599864in}{0.517670in}}%
\pgfpathlineto{\pgfqpoint{0.599864in}{4.060601in}}%
\pgfusepath{stroke}%
\end{pgfscope}%
\begin{pgfscope}%
\pgfsetrectcap%
\pgfsetmiterjoin%
\pgfsetlinewidth{0.803000pt}%
\definecolor{currentstroke}{rgb}{0.000000,0.000000,0.000000}%
\pgfsetstrokecolor{currentstroke}%
\pgfsetdash{}{0pt}%
\pgfpathmoveto{\pgfqpoint{3.420801in}{0.517670in}}%
\pgfpathlineto{\pgfqpoint{3.420801in}{4.060601in}}%
\pgfusepath{stroke}%
\end{pgfscope}%
\begin{pgfscope}%
\pgfsetrectcap%
\pgfsetmiterjoin%
\pgfsetlinewidth{0.803000pt}%
\definecolor{currentstroke}{rgb}{0.000000,0.000000,0.000000}%
\pgfsetstrokecolor{currentstroke}%
\pgfsetdash{}{0pt}%
\pgfpathmoveto{\pgfqpoint{0.599864in}{0.517670in}}%
\pgfpathlineto{\pgfqpoint{3.420801in}{0.517670in}}%
\pgfusepath{stroke}%
\end{pgfscope}%
\begin{pgfscope}%
\pgfsetrectcap%
\pgfsetmiterjoin%
\pgfsetlinewidth{0.803000pt}%
\definecolor{currentstroke}{rgb}{0.000000,0.000000,0.000000}%
\pgfsetstrokecolor{currentstroke}%
\pgfsetdash{}{0pt}%
\pgfpathmoveto{\pgfqpoint{0.599864in}{4.060601in}}%
\pgfpathlineto{\pgfqpoint{3.420801in}{4.060601in}}%
\pgfusepath{stroke}%
\end{pgfscope}%
\begin{pgfscope}%
\pgfsetbuttcap%
\pgfsetmiterjoin%
\definecolor{currentfill}{rgb}{1.000000,1.000000,1.000000}%
\pgfsetfillcolor{currentfill}%
\pgfsetlinewidth{0.000000pt}%
\definecolor{currentstroke}{rgb}{0.000000,0.000000,0.000000}%
\pgfsetstrokecolor{currentstroke}%
\pgfsetstrokeopacity{0.000000}%
\pgfsetdash{}{0pt}%
\pgfpathmoveto{\pgfqpoint{3.597110in}{0.517670in}}%
\pgfpathlineto{\pgfqpoint{3.774256in}{0.517670in}}%
\pgfpathlineto{\pgfqpoint{3.774256in}{4.060601in}}%
\pgfpathlineto{\pgfqpoint{3.597110in}{4.060601in}}%
\pgfpathlineto{\pgfqpoint{3.597110in}{0.517670in}}%
\pgfpathclose%
\pgfusepath{fill}%
\end{pgfscope}%
\begin{pgfscope}%
\pgfpathrectangle{\pgfqpoint{3.597110in}{0.517670in}}{\pgfqpoint{0.177147in}{3.542931in}}%
\pgfusepath{clip}%
\pgfsetbuttcap%
\pgfsetroundjoin%
\definecolor{currentfill}{rgb}{0.611534,0.066436,0.154787}%
\pgfsetfillcolor{currentfill}%
\pgfsetlinewidth{0.000000pt}%
\definecolor{currentstroke}{rgb}{0.000000,0.000000,0.000000}%
\pgfsetstrokecolor{currentstroke}%
\pgfsetdash{}{0pt}%
\pgfpathmoveto{\pgfqpoint{3.597110in}{0.517670in}}%
\pgfpathlineto{\pgfqpoint{3.774256in}{0.517670in}}%
\pgfpathlineto{\pgfqpoint{3.774256in}{1.023803in}}%
\pgfpathlineto{\pgfqpoint{3.597110in}{1.023803in}}%
\pgfpathlineto{\pgfqpoint{3.597110in}{0.517670in}}%
\pgfusepath{fill}%
\end{pgfscope}%
\begin{pgfscope}%
\pgfpathrectangle{\pgfqpoint{3.597110in}{0.517670in}}{\pgfqpoint{0.177147in}{3.542931in}}%
\pgfusepath{clip}%
\pgfsetbuttcap%
\pgfsetroundjoin%
\definecolor{currentfill}{rgb}{0.853057,0.408304,0.326413}%
\pgfsetfillcolor{currentfill}%
\pgfsetlinewidth{0.000000pt}%
\definecolor{currentstroke}{rgb}{0.000000,0.000000,0.000000}%
\pgfsetstrokecolor{currentstroke}%
\pgfsetdash{}{0pt}%
\pgfpathmoveto{\pgfqpoint{3.597110in}{1.023803in}}%
\pgfpathlineto{\pgfqpoint{3.774256in}{1.023803in}}%
\pgfpathlineto{\pgfqpoint{3.774256in}{1.529936in}}%
\pgfpathlineto{\pgfqpoint{3.597110in}{1.529936in}}%
\pgfpathlineto{\pgfqpoint{3.597110in}{1.023803in}}%
\pgfusepath{fill}%
\end{pgfscope}%
\begin{pgfscope}%
\pgfpathrectangle{\pgfqpoint{3.597110in}{0.517670in}}{\pgfqpoint{0.177147in}{3.542931in}}%
\pgfusepath{clip}%
\pgfsetbuttcap%
\pgfsetroundjoin%
\definecolor{currentfill}{rgb}{0.976932,0.767474,0.663668}%
\pgfsetfillcolor{currentfill}%
\pgfsetlinewidth{0.000000pt}%
\definecolor{currentstroke}{rgb}{0.000000,0.000000,0.000000}%
\pgfsetstrokecolor{currentstroke}%
\pgfsetdash{}{0pt}%
\pgfpathmoveto{\pgfqpoint{3.597110in}{1.529936in}}%
\pgfpathlineto{\pgfqpoint{3.774256in}{1.529936in}}%
\pgfpathlineto{\pgfqpoint{3.774256in}{2.036069in}}%
\pgfpathlineto{\pgfqpoint{3.597110in}{2.036069in}}%
\pgfpathlineto{\pgfqpoint{3.597110in}{1.529936in}}%
\pgfusepath{fill}%
\end{pgfscope}%
\begin{pgfscope}%
\pgfpathrectangle{\pgfqpoint{3.597110in}{0.517670in}}{\pgfqpoint{0.177147in}{3.542931in}}%
\pgfusepath{clip}%
\pgfsetbuttcap%
\pgfsetroundjoin%
\definecolor{currentfill}{rgb}{0.969089,0.966474,0.964937}%
\pgfsetfillcolor{currentfill}%
\pgfsetlinewidth{0.000000pt}%
\definecolor{currentstroke}{rgb}{0.000000,0.000000,0.000000}%
\pgfsetstrokecolor{currentstroke}%
\pgfsetdash{}{0pt}%
\pgfpathmoveto{\pgfqpoint{3.597110in}{2.036069in}}%
\pgfpathlineto{\pgfqpoint{3.774256in}{2.036069in}}%
\pgfpathlineto{\pgfqpoint{3.774256in}{2.542202in}}%
\pgfpathlineto{\pgfqpoint{3.597110in}{2.542202in}}%
\pgfpathlineto{\pgfqpoint{3.597110in}{2.036069in}}%
\pgfusepath{fill}%
\end{pgfscope}%
\begin{pgfscope}%
\pgfpathrectangle{\pgfqpoint{3.597110in}{0.517670in}}{\pgfqpoint{0.177147in}{3.542931in}}%
\pgfusepath{clip}%
\pgfsetbuttcap%
\pgfsetroundjoin%
\definecolor{currentfill}{rgb}{0.713033,0.843906,0.910727}%
\pgfsetfillcolor{currentfill}%
\pgfsetlinewidth{0.000000pt}%
\definecolor{currentstroke}{rgb}{0.000000,0.000000,0.000000}%
\pgfsetstrokecolor{currentstroke}%
\pgfsetdash{}{0pt}%
\pgfpathmoveto{\pgfqpoint{3.597110in}{2.542202in}}%
\pgfpathlineto{\pgfqpoint{3.774256in}{2.542202in}}%
\pgfpathlineto{\pgfqpoint{3.774256in}{3.048335in}}%
\pgfpathlineto{\pgfqpoint{3.597110in}{3.048335in}}%
\pgfpathlineto{\pgfqpoint{3.597110in}{2.542202in}}%
\pgfusepath{fill}%
\end{pgfscope}%
\begin{pgfscope}%
\pgfpathrectangle{\pgfqpoint{3.597110in}{0.517670in}}{\pgfqpoint{0.177147in}{3.542931in}}%
\pgfusepath{clip}%
\pgfsetbuttcap%
\pgfsetroundjoin%
\definecolor{currentfill}{rgb}{0.299193,0.599539,0.777163}%
\pgfsetfillcolor{currentfill}%
\pgfsetlinewidth{0.000000pt}%
\definecolor{currentstroke}{rgb}{0.000000,0.000000,0.000000}%
\pgfsetstrokecolor{currentstroke}%
\pgfsetdash{}{0pt}%
\pgfpathmoveto{\pgfqpoint{3.597110in}{3.048335in}}%
\pgfpathlineto{\pgfqpoint{3.774256in}{3.048335in}}%
\pgfpathlineto{\pgfqpoint{3.774256in}{3.554468in}}%
\pgfpathlineto{\pgfqpoint{3.597110in}{3.554468in}}%
\pgfpathlineto{\pgfqpoint{3.597110in}{3.048335in}}%
\pgfusepath{fill}%
\end{pgfscope}%
\begin{pgfscope}%
\pgfpathrectangle{\pgfqpoint{3.597110in}{0.517670in}}{\pgfqpoint{0.177147in}{3.542931in}}%
\pgfusepath{clip}%
\pgfsetbuttcap%
\pgfsetroundjoin%
\definecolor{currentfill}{rgb}{0.097116,0.337716,0.588005}%
\pgfsetfillcolor{currentfill}%
\pgfsetlinewidth{0.000000pt}%
\definecolor{currentstroke}{rgb}{0.000000,0.000000,0.000000}%
\pgfsetstrokecolor{currentstroke}%
\pgfsetdash{}{0pt}%
\pgfpathmoveto{\pgfqpoint{3.597110in}{3.554468in}}%
\pgfpathlineto{\pgfqpoint{3.774256in}{3.554468in}}%
\pgfpathlineto{\pgfqpoint{3.774256in}{4.060601in}}%
\pgfpathlineto{\pgfqpoint{3.597110in}{4.060601in}}%
\pgfpathlineto{\pgfqpoint{3.597110in}{3.554468in}}%
\pgfusepath{fill}%
\end{pgfscope}%
\begin{pgfscope}%
\pgfsetbuttcap%
\pgfsetroundjoin%
\definecolor{currentfill}{rgb}{0.000000,0.000000,0.000000}%
\pgfsetfillcolor{currentfill}%
\pgfsetlinewidth{0.803000pt}%
\definecolor{currentstroke}{rgb}{0.000000,0.000000,0.000000}%
\pgfsetstrokecolor{currentstroke}%
\pgfsetdash{}{0pt}%
\pgfsys@defobject{currentmarker}{\pgfqpoint{0.000000in}{0.000000in}}{\pgfqpoint{0.048611in}{0.000000in}}{%
\pgfpathmoveto{\pgfqpoint{0.000000in}{0.000000in}}%
\pgfpathlineto{\pgfqpoint{0.048611in}{0.000000in}}%
\pgfusepath{stroke,fill}%
}%
\begin{pgfscope}%
\pgfsys@transformshift{3.774256in}{0.517670in}%
\pgfsys@useobject{currentmarker}{}%
\end{pgfscope}%
\end{pgfscope}%
\begin{pgfscope}%
\definecolor{textcolor}{rgb}{0.000000,0.000000,0.000000}%
\pgfsetstrokecolor{textcolor}%
\pgfsetfillcolor{textcolor}%
\pgftext[x=3.871479in, y=0.469452in, left, base]{\color{textcolor}{\ifdefined\pdftexversion\else\setmainfont{Times New Roman}\rmfamily\fi\fontsize{10.000000}{12.000000}\selectfont\catcode`\^=\active\def^{\ifmmode\sp\else\^{}\fi}\catcode`\%=\active\def%{\%}0.04}}%
\end{pgfscope}%
\begin{pgfscope}%
\pgfsetbuttcap%
\pgfsetroundjoin%
\definecolor{currentfill}{rgb}{0.000000,0.000000,0.000000}%
\pgfsetfillcolor{currentfill}%
\pgfsetlinewidth{0.803000pt}%
\definecolor{currentstroke}{rgb}{0.000000,0.000000,0.000000}%
\pgfsetstrokecolor{currentstroke}%
\pgfsetdash{}{0pt}%
\pgfsys@defobject{currentmarker}{\pgfqpoint{0.000000in}{0.000000in}}{\pgfqpoint{0.048611in}{0.000000in}}{%
\pgfpathmoveto{\pgfqpoint{0.000000in}{0.000000in}}%
\pgfpathlineto{\pgfqpoint{0.048611in}{0.000000in}}%
\pgfusepath{stroke,fill}%
}%
\begin{pgfscope}%
\pgfsys@transformshift{3.774256in}{1.023803in}%
\pgfsys@useobject{currentmarker}{}%
\end{pgfscope}%
\end{pgfscope}%
\begin{pgfscope}%
\definecolor{textcolor}{rgb}{0.000000,0.000000,0.000000}%
\pgfsetstrokecolor{textcolor}%
\pgfsetfillcolor{textcolor}%
\pgftext[x=3.871479in, y=0.975585in, left, base]{\color{textcolor}{\ifdefined\pdftexversion\else\setmainfont{Times New Roman}\rmfamily\fi\fontsize{10.000000}{12.000000}\selectfont\catcode`\^=\active\def^{\ifmmode\sp\else\^{}\fi}\catcode`\%=\active\def%{\%}0.08}}%
\end{pgfscope}%
\begin{pgfscope}%
\pgfsetbuttcap%
\pgfsetroundjoin%
\definecolor{currentfill}{rgb}{0.000000,0.000000,0.000000}%
\pgfsetfillcolor{currentfill}%
\pgfsetlinewidth{0.803000pt}%
\definecolor{currentstroke}{rgb}{0.000000,0.000000,0.000000}%
\pgfsetstrokecolor{currentstroke}%
\pgfsetdash{}{0pt}%
\pgfsys@defobject{currentmarker}{\pgfqpoint{0.000000in}{0.000000in}}{\pgfqpoint{0.048611in}{0.000000in}}{%
\pgfpathmoveto{\pgfqpoint{0.000000in}{0.000000in}}%
\pgfpathlineto{\pgfqpoint{0.048611in}{0.000000in}}%
\pgfusepath{stroke,fill}%
}%
\begin{pgfscope}%
\pgfsys@transformshift{3.774256in}{1.529936in}%
\pgfsys@useobject{currentmarker}{}%
\end{pgfscope}%
\end{pgfscope}%
\begin{pgfscope}%
\definecolor{textcolor}{rgb}{0.000000,0.000000,0.000000}%
\pgfsetstrokecolor{textcolor}%
\pgfsetfillcolor{textcolor}%
\pgftext[x=3.871479in, y=1.481718in, left, base]{\color{textcolor}{\ifdefined\pdftexversion\else\setmainfont{Times New Roman}\rmfamily\fi\fontsize{10.000000}{12.000000}\selectfont\catcode`\^=\active\def^{\ifmmode\sp\else\^{}\fi}\catcode`\%=\active\def%{\%}0.12}}%
\end{pgfscope}%
\begin{pgfscope}%
\pgfsetbuttcap%
\pgfsetroundjoin%
\definecolor{currentfill}{rgb}{0.000000,0.000000,0.000000}%
\pgfsetfillcolor{currentfill}%
\pgfsetlinewidth{0.803000pt}%
\definecolor{currentstroke}{rgb}{0.000000,0.000000,0.000000}%
\pgfsetstrokecolor{currentstroke}%
\pgfsetdash{}{0pt}%
\pgfsys@defobject{currentmarker}{\pgfqpoint{0.000000in}{0.000000in}}{\pgfqpoint{0.048611in}{0.000000in}}{%
\pgfpathmoveto{\pgfqpoint{0.000000in}{0.000000in}}%
\pgfpathlineto{\pgfqpoint{0.048611in}{0.000000in}}%
\pgfusepath{stroke,fill}%
}%
\begin{pgfscope}%
\pgfsys@transformshift{3.774256in}{2.036069in}%
\pgfsys@useobject{currentmarker}{}%
\end{pgfscope}%
\end{pgfscope}%
\begin{pgfscope}%
\definecolor{textcolor}{rgb}{0.000000,0.000000,0.000000}%
\pgfsetstrokecolor{textcolor}%
\pgfsetfillcolor{textcolor}%
\pgftext[x=3.871479in, y=1.987851in, left, base]{\color{textcolor}{\ifdefined\pdftexversion\else\setmainfont{Times New Roman}\rmfamily\fi\fontsize{10.000000}{12.000000}\selectfont\catcode`\^=\active\def^{\ifmmode\sp\else\^{}\fi}\catcode`\%=\active\def%{\%}0.16}}%
\end{pgfscope}%
\begin{pgfscope}%
\pgfsetbuttcap%
\pgfsetroundjoin%
\definecolor{currentfill}{rgb}{0.000000,0.000000,0.000000}%
\pgfsetfillcolor{currentfill}%
\pgfsetlinewidth{0.803000pt}%
\definecolor{currentstroke}{rgb}{0.000000,0.000000,0.000000}%
\pgfsetstrokecolor{currentstroke}%
\pgfsetdash{}{0pt}%
\pgfsys@defobject{currentmarker}{\pgfqpoint{0.000000in}{0.000000in}}{\pgfqpoint{0.048611in}{0.000000in}}{%
\pgfpathmoveto{\pgfqpoint{0.000000in}{0.000000in}}%
\pgfpathlineto{\pgfqpoint{0.048611in}{0.000000in}}%
\pgfusepath{stroke,fill}%
}%
\begin{pgfscope}%
\pgfsys@transformshift{3.774256in}{2.542202in}%
\pgfsys@useobject{currentmarker}{}%
\end{pgfscope}%
\end{pgfscope}%
\begin{pgfscope}%
\definecolor{textcolor}{rgb}{0.000000,0.000000,0.000000}%
\pgfsetstrokecolor{textcolor}%
\pgfsetfillcolor{textcolor}%
\pgftext[x=3.871479in, y=2.493984in, left, base]{\color{textcolor}{\ifdefined\pdftexversion\else\setmainfont{Times New Roman}\rmfamily\fi\fontsize{10.000000}{12.000000}\selectfont\catcode`\^=\active\def^{\ifmmode\sp\else\^{}\fi}\catcode`\%=\active\def%{\%}0.20}}%
\end{pgfscope}%
\begin{pgfscope}%
\pgfsetbuttcap%
\pgfsetroundjoin%
\definecolor{currentfill}{rgb}{0.000000,0.000000,0.000000}%
\pgfsetfillcolor{currentfill}%
\pgfsetlinewidth{0.803000pt}%
\definecolor{currentstroke}{rgb}{0.000000,0.000000,0.000000}%
\pgfsetstrokecolor{currentstroke}%
\pgfsetdash{}{0pt}%
\pgfsys@defobject{currentmarker}{\pgfqpoint{0.000000in}{0.000000in}}{\pgfqpoint{0.048611in}{0.000000in}}{%
\pgfpathmoveto{\pgfqpoint{0.000000in}{0.000000in}}%
\pgfpathlineto{\pgfqpoint{0.048611in}{0.000000in}}%
\pgfusepath{stroke,fill}%
}%
\begin{pgfscope}%
\pgfsys@transformshift{3.774256in}{3.048335in}%
\pgfsys@useobject{currentmarker}{}%
\end{pgfscope}%
\end{pgfscope}%
\begin{pgfscope}%
\definecolor{textcolor}{rgb}{0.000000,0.000000,0.000000}%
\pgfsetstrokecolor{textcolor}%
\pgfsetfillcolor{textcolor}%
\pgftext[x=3.871479in, y=3.000117in, left, base]{\color{textcolor}{\ifdefined\pdftexversion\else\setmainfont{Times New Roman}\rmfamily\fi\fontsize{10.000000}{12.000000}\selectfont\catcode`\^=\active\def^{\ifmmode\sp\else\^{}\fi}\catcode`\%=\active\def%{\%}0.24}}%
\end{pgfscope}%
\begin{pgfscope}%
\pgfsetbuttcap%
\pgfsetroundjoin%
\definecolor{currentfill}{rgb}{0.000000,0.000000,0.000000}%
\pgfsetfillcolor{currentfill}%
\pgfsetlinewidth{0.803000pt}%
\definecolor{currentstroke}{rgb}{0.000000,0.000000,0.000000}%
\pgfsetstrokecolor{currentstroke}%
\pgfsetdash{}{0pt}%
\pgfsys@defobject{currentmarker}{\pgfqpoint{0.000000in}{0.000000in}}{\pgfqpoint{0.048611in}{0.000000in}}{%
\pgfpathmoveto{\pgfqpoint{0.000000in}{0.000000in}}%
\pgfpathlineto{\pgfqpoint{0.048611in}{0.000000in}}%
\pgfusepath{stroke,fill}%
}%
\begin{pgfscope}%
\pgfsys@transformshift{3.774256in}{3.554468in}%
\pgfsys@useobject{currentmarker}{}%
\end{pgfscope}%
\end{pgfscope}%
\begin{pgfscope}%
\definecolor{textcolor}{rgb}{0.000000,0.000000,0.000000}%
\pgfsetstrokecolor{textcolor}%
\pgfsetfillcolor{textcolor}%
\pgftext[x=3.871479in, y=3.506250in, left, base]{\color{textcolor}{\ifdefined\pdftexversion\else\setmainfont{Times New Roman}\rmfamily\fi\fontsize{10.000000}{12.000000}\selectfont\catcode`\^=\active\def^{\ifmmode\sp\else\^{}\fi}\catcode`\%=\active\def%{\%}0.28}}%
\end{pgfscope}%
\begin{pgfscope}%
\pgfsetbuttcap%
\pgfsetroundjoin%
\definecolor{currentfill}{rgb}{0.000000,0.000000,0.000000}%
\pgfsetfillcolor{currentfill}%
\pgfsetlinewidth{0.803000pt}%
\definecolor{currentstroke}{rgb}{0.000000,0.000000,0.000000}%
\pgfsetstrokecolor{currentstroke}%
\pgfsetdash{}{0pt}%
\pgfsys@defobject{currentmarker}{\pgfqpoint{0.000000in}{0.000000in}}{\pgfqpoint{0.048611in}{0.000000in}}{%
\pgfpathmoveto{\pgfqpoint{0.000000in}{0.000000in}}%
\pgfpathlineto{\pgfqpoint{0.048611in}{0.000000in}}%
\pgfusepath{stroke,fill}%
}%
\begin{pgfscope}%
\pgfsys@transformshift{3.774256in}{4.060601in}%
\pgfsys@useobject{currentmarker}{}%
\end{pgfscope}%
\end{pgfscope}%
\begin{pgfscope}%
\definecolor{textcolor}{rgb}{0.000000,0.000000,0.000000}%
\pgfsetstrokecolor{textcolor}%
\pgfsetfillcolor{textcolor}%
\pgftext[x=3.871479in, y=4.012383in, left, base]{\color{textcolor}{\ifdefined\pdftexversion\else\setmainfont{Times New Roman}\rmfamily\fi\fontsize{10.000000}{12.000000}\selectfont\catcode`\^=\active\def^{\ifmmode\sp\else\^{}\fi}\catcode`\%=\active\def%{\%}0.32}}%
\end{pgfscope}%
\begin{pgfscope}%
\pgfsetrectcap%
\pgfsetmiterjoin%
\pgfsetlinewidth{0.803000pt}%
\definecolor{currentstroke}{rgb}{0.000000,0.000000,0.000000}%
\pgfsetstrokecolor{currentstroke}%
\pgfsetdash{}{0pt}%
\pgfpathmoveto{\pgfqpoint{3.597110in}{0.517670in}}%
\pgfpathlineto{\pgfqpoint{3.685683in}{0.517670in}}%
\pgfpathlineto{\pgfqpoint{3.774256in}{0.517670in}}%
\pgfpathlineto{\pgfqpoint{3.774256in}{4.060601in}}%
\pgfpathlineto{\pgfqpoint{3.685683in}{4.060601in}}%
\pgfpathlineto{\pgfqpoint{3.597110in}{4.060601in}}%
\pgfpathlineto{\pgfqpoint{3.597110in}{0.517670in}}%
\pgfpathclose%
\pgfusepath{stroke}%
\end{pgfscope}%
\end{pgfpicture}%
\makeatother%
\endgroup%

      \caption{Contour plot of the lower bound on the safety probability with respect to the number of frequencies $M$ and the lattice size per dimension for the \overtaking benchmark.}
      \label{fig:contour-overtaking}
\end{figure}
